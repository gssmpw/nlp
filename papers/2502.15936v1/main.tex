\documentclass[journal]{IEEEtran}
\IEEEoverridecommandlockouts

\usepackage{cite}
\usepackage{amsmath,amssymb,amsfonts}
\usepackage{algorithmic}
\usepackage{graphicx}
\usepackage{textcomp}
\usepackage{xcolor}
\usepackage{multirow}
\usepackage{graphicx}
\usepackage{array}
\usepackage{caption}
\usepackage{booktabs}
\usepackage{multirow}
%\usepackage{glossaries}
\usepackage{soul}
\usepackage{subcaption}
\usepackage{balance}
\usepackage[capitalise]{cleveref}
\crefname{section}{Sec.}{Secs.}
\crefname{figure}{Fig.}{Figs.}
\usepackage{tikz}
\usetikzlibrary{fit}
\makeatletter
\tikzset{
  fitting node/.style={
    inner sep=0pt,
    fill=none,
    draw=none,
    reset transform,
    fit={(\pgf@pathminx,\pgf@pathminy) (\pgf@pathmaxx,\pgf@pathmaxy)}
  },
  reset transform/.code={\pgftransformreset}
}
\makeatother

\usepackage{pgfplots}
\pgfplotsset{compat=newest} 
\pgfplotsset{plot coordinates/math parser=false} 

\usepackage[acronyms,nonumberlist,nopostdot,nomain,nogroupskip,acronymlists={hidden}]{glossaries}
\newlength\fheight
\setlength{\fheight}{0.5\columnwidth}
\newlength\fwidth
\setlength{\fwidth}{0.8\columnwidth}

\def\BibTeX{{\rm B\kern-.05em{\sc i\kern-.025em b}\kern-.08em
    T\kern-.1667em\lower.7ex\hbox{E}\kern-.125emX}}
\begin{document}


\begin{acronym}
\acro{gan}[GANs]{Generative Adversarial Networks}
\acro{rl}[RL]{Reinforcement Learning}
\acro{pae}[PAE]{Periodic Autoencoder}
\acro{fld}[FLD]{Fourier Latent Dynamics}
\acro{ppo}[PPO]{Proximal Policy Optimization}
\acro{fft}[FFT]{Fast Fourier Transform}
\acro{pca}[PCA]{Principal Component Analysis}
\acro{dfm}[DFM]{Deep Fourier Mimic}
\acro{dof}[DoF]{Degrees of Freedom}
\acro{mlp}[MLPs]{Multi-Layer Perceptrons}
\end{acronym}




%\title{Integrating O-RAN with leo Satellite Networks: A Path to Unified Non-Terrestrial and Terrestrial Communications in 6G

%\title{ASTRO-RAN: A Path to Unified Space and Terrestrial Communications in 6G

\title{Space-O-RAN: Enabling Intelligent, Open, and Interoperable Non Terrestrial Networks in 6G

%\title{Integrating O-RAN with Non-Terrestrial Networks: A Path to Unified Space and Terrestrial Communications in 6G


%\thanks{Identify applicable funding agency here. If none, delete this.}
}



\author{\IEEEauthorblockN{
Eduardo Baena, %\IEEEauthorrefmark{1},
Paolo Testolina, %\IEEEauthorrefmark{1}, 
Michele Polese, %\IEEEauthorrefmark{1}, 
Dimitrios Koutsonikolas, %\IEEEauthorrefmark{1}, 
Josep Jornet, %\IEEEauthorrefmark{1}, 
Tommaso Melodia%\IEEEauthorrefmark{1}
}

\IEEEauthorblockN{Institute for the Wireless Internet of Things, Northeastern University, Boston, MA, U.S.A.}
}

\maketitle



\glsresetall
\glsunset{rapp}
\glsunset{xapp}
\glsunset{dapp}


\begin{abstract}

\glspl{ntn} are essential for ubiquitous connectivity, providing coverage in remote and underserved areas. However, since \glspl{ntn} are currently operated independently, they face challenges such as isolation, limited scalability, and high operational costs. Integrating satellite constellations with terrestrial networks offers a way to address these limitations while enabling adaptive and cost-efficient connectivity through the application of \gls{ai} models.  

This paper introduces {\em Space-O-RAN}, a framework that extends Open \gls{ran} principles to \glspl{ntn}. It employs hierarchical closed-loop control with distributed \glspl{spaceric} to dynamically manage and optimize operations across both domains. To enable adaptive resource allocation and network orchestration, the proposed architecture integrates real-time satellite optimization and control with \gls{ai}-driven management and \gls{dt} modeling. It incorporates distributed \glspl{sapp} and distributed applications (\glspl{dapp}) to ensure robust performance in highly dynamic orbital environments. A core feature is dynamic link-interface mapping, which allows network functions to adapt to specific application requirements and changing link conditions using all physical links on the satellite.

Simulation results evaluate its feasibility by analyzing latency constraints across different \gls{ntn} link types, demonstrating that intra-cluster coordination operates within viable signaling delay bounds, while offloading non-real-time tasks to ground infrastructure enhances scalability toward \gls{6g} networks.  

\end{abstract}


\begin{IEEEkeywords}
NTN, O-RAN, Integration, 6G, Management
\end{IEEEkeywords}

\begin{picture}(0,0)(40,-460)
\put(0,0){
\put(3,0){\footnotesize This work has been submitted to the IEEE for possible publication.}
\put(0,-10){
\footnotesize Copyright may be transferred without notice, after which this version may no longer be accessible.}}
\end{picture}

\glsresetall
\glsunset{rapp}
\glsunset{xapp}
\glsunset{dapp}

\section{Introduction}

%The growing demand for reliable, high-capacity, and globally accessible connectivity has positioned \glspl{ntn} as a critical component of next-generation communication systems. NTNs - including \glspl{leo}, \glspl{meo}, \glspl{geo}, \glspl{hap}, and deep-space platforms, extend connectivity toremote and underserved regions,  overcoming the inherent limitations of terrestrial infrastructure. Recent advancements in satellite constellations, exemplified by initiatives like SpaceX's Starlink and Amazon's Project Kuiper highlight the transformative potential of \glspl{ntn} in achieving global connectivity~\cite{abdelsadek2023future, cheng2022service}. These developments underscore the growing role of integrated terrestrial and non-terrestrial architectures in future global communication systems.

The demand for high-capacity, globally available connectivity has positioned \glspl{ntn} as a key element of next-generation communication systems. These networks—including \glspl{leo}, \glspl{meo}, \glspl{geo}, \glspl{hap}, and deep-space platforms—extend coverage to remote regions, addressing the limitations of terrestrial infrastructure. Recent advancements in satellite constellations, such as Starlink by SpaceX and Project Kuiper by Amazon, illustrate their potential to improve connectivity worldwide~\cite{abdelsadek2023future, cheng2022service}. While terrestrial and  \glspl{ntn} are still managed separately, there is a growing trend toward their complementary use \cite{abdelsadek2023future}.

However, standalone and closed NTN systems face fundamental limitations due to their isolated operation, reliance on dedicated feeder link infrastructure, high operational costs, and constrained optimization capabilities. Integrating  NTNs with terrestrial networks operating in the same spectrum and interference domains can transform these systems from siloed architectures into unified, efficient, and resilient networks. This convergence enables adaptive resource management and enhanced scalability, making \glspl{ntn} better suited to address the complex demands of emerging 6G applications \cite{abdelsadek2023future, cheng2022service}.

Moreover, the integration introduces significant improvements in network resiliency and adaptability. Jointly managed networks could autonomously deploy \gls{ntn} backhaul cells in disaster recovery and ad hoc communication scenarios, ensuring uninterrupted service when terrestrial infrastructure is compromised. Dynamic traffic monitoring within terrestrial networks enables real-time load balancing, optimizing resource allocation across \gls{ntn} and terrestrial domains based on contextual demands. Hierarchical closed-loop control mechanisms can facilitate unified network management, improving spectrum efficiency, minimizing interference, and enabling seamless traffic offloading between orbital and terrestrial segments. Furthermore, integrated architectures can unlock advanced use cases, such as distributed computing frameworks spanning terrestrial and space domains. The unified management of all on-board radio links and user-plane resources enables the deployment of virtualized network functions and other computational workloads, dynamically adapting to traffic fluctuations and processing demands. 

Achieving cohesive coordination presents significant technical challenges because of differences in operational characteristics and resource constraints \cite{cheng2022service}. Unlike terrestrial systems, \glspl{ntn} operate in highly dynamic and resource-constrained environments. For example,  the continuous movement of \glspl{leo} satellites necessitates frequent handovers and real-time connectivity updates. These dynamics introduce time-varying latency, intermittent connectivity, and frequent \gls{los} disruption, impacting service continuity and network stability \cite{iqbal2023ai}. Additionally, onboard limitations in power, computational resources, and thermal management constrain the deployment of advanced optimization algorithms, emphasizing the need for efficient task offloading strategies \cite{ren2023review}. The absence of standardized coordination protocols between the orbital and terrestrial layers further exacerbates these challenges, leading to fragmented operations, inefficient resource allocation, and increased interference risks in shared or adjacent spectrum bands \cite{jia2020virtual}.

To address these challenges, this paper proposes {\bf Space-O-RAN}, a novel architectural framework for the integrated control and optimization of terrestrial and non-terrestrial networks. Built on Open RAN principles, Space-O-RAN incorporates (i) programmability through software-based protocol stacks, (ii) open interfaces for seamless cross-domain coordination, and (iii) multi-scale hierarchical closed-loop control for cross-domain optimization and orchestration.

The Space-O-RAN architecture extends O-RAN controllers to satellite systems through distributed \glspl{spaceric}, addressing the unique characteristics of \glspl{ntn}. By leveraging standardized interfaces such as E2, O1, A1, and O2, it enables efficient coordination across heterogeneous environments. Customizable rApps, sApps, and dApps enable real-time network control and optimization, while flexible functional splits enhance computational and communication workload distribution. Additionally, a terrestrial \gls{dt} framework supports predictive optimization, resource management, and performance evaluation. The architecture's capabilities are validated through numerical analysis of latency domains, demonstrating its potential for adaptive, scalable, and efficient network integration.

The article is organized as follows. Section~\ref{sec:background} discusses related work on \glspl{ntn} integration. Section~\ref{sec:challenges} highlights the challenges for efficient integrated control and optimization and discusses specific use cases that underline the need for such integration. Section~\ref{sec:architecture} provides details of the proposed framework, emphasizing its hierarchical structure and adaptive control mechanisms. It also presents an evaluation of feasible control latency. Finally, \cref{sec:future,sec:conclusions} presents the future outlook and summarizes the conclusions.

\section{Background and Related Work}
\label{sec:background}

Existing integrated terrestrial-\gls{ntn} architectures, such as \gls{sagin}~\cite{cui2022architecture}, focus primarily on user-plane solutions and lack closed-loop control mechanisms. While \gls{sagin} envisions seamless \gls{ntn}-terrestrial integration, its practical implementation remains limited by dynamic link variability, intermittent connectivity, and latency constraints. These factors make efficient real-time control loops difficult to implement, particularly when relying on existing \gls{ai}-driven approaches, which struggle with inconsistent data availability and computational constraints.

The integration of \gls{ai} into \glspl{ntn} faces significant challenges. Limited onboard resources restrict the execution of complex learning algorithms, while intermittent satellite connectivity and long propagation delays disrupt data collection, model training, and inference processes. Current solutions often fail to address real-time resource allocation, adaptive control, and precise timing synchronization, leaving interoperability between orbital and terrestrial layers largely conceptual, with no standardized implementations for protocol or interface alignment \cite{10716597}.

Recent research on \gls{ntn}-terrestrial integration has identified key trade-offs in \gls{ran} architectures. Lower-layer functional splits reduce latency but impose strict timing constraints that are difficult to meet in high-latency \gls{ntn} environments. Centralized architectures, while easier to implement, introduce bandwidth inefficiencies over feeder links \cite{Muro2024}. Modular architectures that leverage standardized interfaces, such as near-RT and non-RT \glspl{ric}, show potential for adaptive resource management and real-time control. However, they remain constrained by scalability issues, high latency, and the complexity of ensuring seamless handovers between satellite clusters \cite{oranntn2025}. Addressing these limitations requires innovations that balance distributed decision-making, adaptive resource allocation, and efficient multi-domain integration.

On the industry side, \gls{3gpp} has made significant progress in interoperability with \gls{ntn}. Release 17 introduced extensions to \gls{5g} \gls{nr} for direct-to-device satellite connectivity, while Release 18 builds on these enhancements with dynamic handovers and spectrum sharing.  ITU-T Y.3207 (2024) defines INCA for Fixed, Mobile, and Satellite Convergence (FMSC), proposing a centralized control framework, but its feasibility in dynamic \gls{ntn} environments remains uncertain due to latency, overhead, and computational constraints. These developments provide a foundation for integration, but remain insufficient without additional architectural enhancements to address specific constraints in \gls{ntn}, such as real-time coordination, cross-layer control, and resource-aware \gls{ai} processing.
%The lack of closed-loop control procedures remains a major limitation in state-of-the-art approaches like \gls{sagin}~\cite{cui2022architecture}, which provide only conceptual models for integrated architectures and focus primarily on user plane solutions. While \gls{sagin} frameworks envision seamless integration of terrestrial and \glspl{ntn}, their practical implementation is hindered by the dynamic nature of \gls{ntn} links, intermittent connectivity, and significant latency. These challenges complicate the design of stable and efficient control loops and are exacerbated by the limitations of existing \gls{ai}-driven approaches.

%The incorporation of \gls{ai} technologies into \glspl{ntn} faces significant hurdles, as limited satellite resources constrain the deployment of complex algorithms, while intermittent connectivity and long propagation delays disrupt data collection and training cycles. Moreover, current solutions frequently do not succeed in address critical operational requirements, including real-time resource allocation, adaptive control, and precise timing synchronization. Furthermore, interoperability between orbital and terrestrial layers remains largely conceptual, with no concrete implementations for protocol or interface alignment \cite{10716597}.

%Recent efforts to integrate \glspl{ntn} with terrestrial networks have highlighted critical trade-offs in \gls{ran} architectures. Lower-layer splits reduce latency, but face strict delay constraints that are difficult to meet in high-latency \gls{ntn} environments, while centralized approaches suffer from bandwidth inefficiencies over feeder links \cite{seeram2024feasibilitystudyfunctionsplits}. Modular architectures that leverage standardized interfaces like near-RT and non-RT \glspl{ric} show potential for adaptive resource management and real-time control, but are challenged by scalability, high latency, and seamless handovers \cite{oranntn2025}. Addressing these limitations requires innovations that balance distributed decision-making, adaptive resource allocation, and efficient integration across network segments.

%On the industry side, \gls{3gpp} standardization has significantly advanced \gls{ntn} interoperability. Release 17 introduced extensions to \gls{5g} \gls{nr} for direct-to-cell connectivity, and Release 18 builds on these developments by incorporating dynamic handovers and spectrum sharing. These provide a solid foundation for the bridge between terrestrial and satellite networks, but remain insufficient without additional architectural enhancements to address NTN-specific constraints.

 \begin{figure*}[h]
    \centering
    % Combinar width, height, trim y clip
    \includegraphics[width=0.77\textwidth]{figures/arch.pdf}
    \captionsetup{singlelinecheck=false}
    \caption{The proposed framework coordinates on board Space-RIC (with s-Apps) with terrestrial SMO via flexible link mapping}
    \label{fig:oran_space_arch}
\end{figure*}
\vskip -2\baselineskip plus -1fil

\section{Challenges and Use Cases}
\label{sec:challenges}

This section identifies the key challenges in integrating optimization and control for terrestrial and \glspl{ntn}, highlighting the technical and operational barriers to seamless coordination.

\textbf{Dynamic Topologies and Fragmented Control.} The constant mobility of \glspl{leo} and \glspl{meo} satellites necessitates frequent handovers and dynamic network reconfiguration. These conditions introduce time-varying latency, intermittent connectivity, and \gls{los} disruptions due to environmental factors and orbital motion. For example, \glspl{geo} experience round-trip delays exceeding 240 ms, while \glspl{isl} exhibit variable latencies depending on orbital alignment. Such dynamics demand mechanisms for localized adaptation to ensure service continuity while maintaining coherence with global directives \cite{ren2023review}. Section~\ref{sec:feasibility} further examines latency boundaries and their implications for real-time operations.

\textbf{Interoperability and Standardization.} The absence of unified protocols for coordinating terrestrial and satellite systems presents a significant barrier to seamless integration. While theoretical frameworks have proposed conceptual models, practical mechanisms for signaling, closed-loop control, and cross-domain communication are still in their early stages of development. The lack of standardized interfaces and synchronization protocols leads to operational fragmentation, complicating interoperability across a diverse supply chain \cite{mahboob2023revolutionizing, jia2020virtual}.

\textbf{Spectrum Management and Coexistence.} The shared use of spectrum across terrestrial and \glspl{ntn} poses significant challenges for coexistence and interference mitigation. Feasibility studies for Ka, Ku, and S bands have demonstrated potential, but the real-time allocation of spectrum and mitigation of interference between constellations require sophisticated coordination strategies. The increasing number of constellations and the growing demand for bandwidth underscore the urgency of developing mechanisms to maximize spectral efficiency while avoiding service disruption~\cite{jamshed2024non}.

\textbf{Security and Operational Reliability.} The expanded attack surface introduced by the integration of terrestrial and satellite domains amplifies risks such as unauthorized access, data breaches, and disruption. The intermittent connectivity of satellite links, combined with the multi-hop nature of \glspl{isl}, creates vulnerabilities that traditional security protocols cannot fully address. Additionally, the disparate operational dynamics of terrestrial and satellite systems increase the complexity of ensuring reliable performance under constantly changing conditions \cite{mahboob2023revolutionizing}.

\textbf{Hardware and Computational Constraints.} Satellites operate under strict constraints in power, processing capacity, and thermal management. CubeSats, for instance, are equipped with processors like the ARM Cortex-M, delivering tens to hundreds of megaflops, while larger geostationary platforms employ more powerful processors like the RAD750, capable of up to 400 megaflops. Memory capacities vary from 512 MB to 16 GB, depending on the platform. These limitations restrict the execution of complex algorithms and necessitate offloading of computationally intensive tasks to terrestrial systems. However, intermittent and bandwidth-limited satellite-terrestrial links complicate this approach, particularly for latency-sensitive applications \cite{ren2023review, wang2023satellite}. Power availability further exacerbates these challenges, with budgets ranging from $5–20\:\mathrm{W}$ for CubeSats and up to 20 kW for larger satellites, fluctuating due to factors like solar orientation and eclipse phases. Advanced power management technologies, such as \gls{mppt} and scalable energy storage, alleviate some of these issues but require careful optimization to sustain operations efficiently.


\begin{figure*}[h]
    \begin{subfigure}[t]{\columnwidth}
        \centering
        % \includegraphics[width=0.81\linewidth]{figures/latency.pdf}
        \setlength{\fheight}{0.4\columnwidth}
        \setlength{\fwidth}{0.8\columnwidth}
       \input{figures/latency_g2s.tikz}
       \captionsetup{justification=centering}
        \caption{Propagation delay between the ground stations and the satellites of the Starlink constellation at the minimum (solid orange) and maximum (dashed orange) \gls{leo} altitude. The histogram (blue) reports the frequency of observation of the satellites at the corresponding elevation angle.}
        \label{fig:g2s_latency}
        % \vspace{-10pt}
    \end{subfigure}
    \begin{subfigure}[t]{\columnwidth}
        \centering
        % \includegraphics[width=0.81\linewidth]{figures/latency.pdf}
        \setlength{\fheight}{0.4\columnwidth}
        \setlength{\fwidth}{0.8\columnwidth}
        % This file was created by matlab2tikz.
%
%The latest updates can be retrieved from
%  http://www.mathworks.com/matlabcentral/fileexchange/22022-matlab2tikz-matlab2tikz
%where you can also make suggestions and rate matlab2tikz.
%
\definecolor{mycolor1}{rgb}{0.00000,0.44700,0.74100}%
\definecolor{mycolor2}{rgb}{0.85000,0.32500,0.09800}%
%
\begin{tikzpicture}

\begin{axis}[%
width=0.951\fwidth,
height=\fheight,
at={(0\fwidth,0\fheight)},
scale only axis,
xmin=0,
xmax=12,
xlabel style={font=\color{white!15!black}},
xlabel={Time [h]},
ymin=250,
ymax=500,
ylabel style={font=\color{white!15!black}},
ylabel={Number of satellite},
axis background/.style={fill=white},
xmajorgrids,
ymajorgrids,
legend style={legend cell align=left, align=left, draw=white!15!black}
]

\addplot[area legend, draw=black, fill=mycolor1, fill opacity=0.5, forget plot]
table[row sep=crcr] {%
x	y\\
0	272.987849915995\\
0.0166666666666667	273.159673104126\\
0.0333333333333333	273.162162153718\\
0.05	273.594249734173\\
0.0666666666666667	273.629514937711\\
0.0833333333333333	273.798994540854\\
0.1	273.551094045905\\
0.116666666666667	273.726078865339\\
0.133333333333333	273.809122377657\\
0.15	273.968230415941\\
0.166666666666667	273.820169426721\\
0.183333333333333	273.756543832711\\
0.2	273.633050670655\\
0.216666666666667	273.485362441969\\
0.233333333333333	273.203621404862\\
0.25	273.411713254829\\
0.266666666666667	273.08651902894\\
0.283333333333333	272.985280346444\\
0.3	272.860055110288\\
0.316666666666667	272.99429256976\\
0.333333333333333	273.181950555475\\
0.35	273.359834716079\\
0.366666666666667	273.300340802927\\
0.383333333333333	273.225055137292\\
0.4	273.46518351964\\
0.416666666666667	273.565273051015\\
0.433333333333333	273.558699969326\\
0.45	273.493603743686\\
0.466666666666667	273.843519137967\\
0.483333333333333	274.126372727548\\
0.5	274.223831516727\\
0.516666666666667	274.20142925731\\
0.533333333333333	274.20070780901\\
0.55	274.423960040626\\
0.566666666666667	274.318011937715\\
0.583333333333333	274.324692403668\\
0.6	273.989845493805\\
0.616666666666667	273.934784691746\\
0.633333333333333	273.510580702134\\
0.65	273.29534275661\\
0.666666666666667	273.018182119283\\
0.683333333333333	272.923694798258\\
0.7	272.777724965867\\
0.716666666666667	272.649316802021\\
0.733333333333333	272.620820391691\\
0.75	272.818086877298\\
0.766666666666667	272.758957430528\\
0.783333333333333	272.949901478648\\
0.8	273.041143547761\\
0.816666666666667	273.284038897855\\
0.833333333333333	273.631808712306\\
0.85	274.086708182817\\
0.866666666666667	274.027052657143\\
0.883333333333333	274.072906287741\\
0.9	274.335693174045\\
0.916666666666667	274.302793120221\\
0.933333333333333	274.292644561854\\
0.95	274.262108462739\\
0.966666666666667	274.281317083796\\
0.983333333333333	274.15464897265\\
1	274.063721611852\\
1.01666666666667	274.017269014624\\
1.03333333333333	273.712790669631\\
1.05	273.706164224657\\
1.06666666666667	273.402017954328\\
1.08333333333333	273.346711305754\\
1.1	273.150892323959\\
1.11666666666667	273.290588938007\\
1.13333333333333	273.546953632293\\
1.15	273.463394721782\\
1.16666666666667	273.449697600676\\
1.18333333333333	273.600428215251\\
1.2	273.702347605256\\
1.21666666666667	273.689651454473\\
1.23333333333333	273.821999892186\\
1.25	273.952645337196\\
1.26666666666667	274.039029900019\\
1.28333333333333	274.174527675703\\
1.3	274.186677467586\\
1.31666666666667	274.124099402286\\
1.33333333333333	274.229637249507\\
1.35	274.426620707047\\
1.36666666666667	274.32583738355\\
1.38333333333333	274.318848850828\\
1.4	273.940417290878\\
1.41666666666667	273.584259384625\\
1.43333333333333	273.259264979152\\
1.45	273.152962493274\\
1.46666666666667	272.814792660729\\
1.48333333333333	272.490958457233\\
1.5	272.513563755914\\
1.51666666666667	272.575285567614\\
1.53333333333333	272.591277737999\\
1.55	272.723574013243\\
1.56666666666667	272.901819265711\\
1.58333333333333	273.007085415962\\
1.6	273.248235633817\\
1.61666666666667	273.511742702986\\
1.63333333333333	273.896753448643\\
1.65	274.032791759071\\
1.66666666666667	274.15620825046\\
1.68333333333333	274.255078576834\\
1.7	274.386309598876\\
1.71666666666667	274.287345526037\\
1.73333333333333	274.34907467139\\
1.75	274.302286571832\\
1.76666666666667	274.284871956638\\
1.78333333333333	274.126255568467\\
1.8	274.168171132838\\
1.81666666666667	273.883135972318\\
1.83333333333333	273.672332062953\\
1.85	273.587514204756\\
1.86666666666667	273.391583675241\\
1.88333333333333	273.143955378584\\
1.9	273.240976424866\\
1.91666666666667	273.437783658423\\
1.93333333333333	273.598914024268\\
1.95	273.541518209224\\
1.96666666666667	273.727780931002\\
1.98333333333333	273.823370481804\\
2	273.834122668234\\
2.01666666666667	274.013793038126\\
2.03333333333333	274.030358050815\\
2.05	274.113836803975\\
2.06666666666667	274.173856872046\\
2.08333333333333	274.146320148492\\
2.1	274.123961430808\\
2.11666666666667	274.216532457549\\
2.13333333333333	274.305550211985\\
2.15	274.360294929362\\
2.16666666666667	274.11402880066\\
2.18333333333333	273.831319907651\\
2.2	273.49188504599\\
2.21666666666667	273.008504712416\\
2.23333333333333	272.863367672584\\
2.25	272.744415127318\\
2.26666666666667	272.411622725306\\
2.28333333333333	272.359544660725\\
2.3	272.253008532619\\
2.31666666666667	272.526621924182\\
2.33333333333333	272.768153983572\\
2.35	272.900430678412\\
2.36666666666667	273.149347401624\\
2.38333333333333	273.509553665476\\
2.4	273.826336029258\\
2.41666666666667	274.181492194594\\
2.43333333333333	274.406299584144\\
2.45	274.403608845076\\
2.46666666666667	274.632360559352\\
2.48333333333333	274.565867007477\\
2.5	274.553665959697\\
2.51666666666667	274.793365714146\\
2.53333333333333	274.777160085082\\
2.55	274.761683841549\\
2.56666666666667	274.482946919567\\
2.58333333333333	274.443080447026\\
2.6	274.443731564938\\
2.61666666666667	274.06367293825\\
2.63333333333333	273.934726945224\\
2.65	273.685478189807\\
2.66666666666667	273.57372086554\\
2.68333333333333	273.47807731649\\
2.7	273.474034384421\\
2.71666666666667	273.720952557875\\
2.73333333333333	273.827363040348\\
2.75	273.895655784474\\
2.76666666666667	273.946918575431\\
2.78333333333333	273.992663407936\\
2.8	274.156231563976\\
2.81666666666667	274.384110950004\\
2.83333333333333	274.135166388947\\
2.85	274.029806694462\\
2.86666666666667	274.138449126975\\
2.88333333333333	274.169832209687\\
2.9	274.267403892786\\
2.91666666666667	274.239383753696\\
2.93333333333333	274.053559172076\\
2.95	274.01849682444\\
2.96666666666667	273.741520350595\\
2.98333333333333	273.472444658005\\
3	273.109327857718\\
3.01666666666667	272.788339235257\\
3.03333333333333	272.510780068998\\
3.05	272.353176521787\\
3.06666666666667	272.296877309272\\
3.08333333333333	272.132122706762\\
3.1	272.242243104557\\
3.11666666666667	272.580161275383\\
3.13333333333333	272.927193102691\\
3.15	273.182622286731\\
3.16666666666667	273.460717212281\\
3.18333333333333	273.908239850138\\
3.2	274.088603826495\\
3.21666666666667	274.305135888235\\
3.23333333333333	274.567387000955\\
3.25	274.857561879934\\
3.26666666666667	274.711715841206\\
3.28333333333333	274.751073076335\\
3.3	274.779234161674\\
3.31666666666667	274.849375166128\\
3.33333333333333	274.911950271013\\
3.35	274.854249026154\\
3.36666666666667	274.610493327281\\
3.38333333333333	274.350821882534\\
3.4	274.220231990512\\
3.41666666666667	273.839609876113\\
3.43333333333333	273.80083861031\\
3.45	273.470641118096\\
3.46666666666667	273.470663762815\\
3.48333333333333	273.296701804405\\
3.5	273.376704677526\\
3.51666666666667	273.512487589531\\
3.53333333333333	273.570695677613\\
3.55	273.827169847133\\
3.56666666666667	273.866872576586\\
3.58333333333333	273.913380736964\\
3.6	273.991810999755\\
3.61666666666667	274.084130059135\\
3.63333333333333	274.144050186585\\
3.65	273.923713265086\\
3.66666666666667	273.973132865625\\
3.68333333333333	273.983460374097\\
3.7	273.959812651872\\
3.71666666666667	273.536907810942\\
3.73333333333333	273.55560415829\\
3.75	273.487217397724\\
3.76666666666667	273.241493493706\\
3.78333333333333	273.022082502464\\
3.8	272.63230378427\\
3.81666666666667	272.434630245697\\
3.83333333333333	272.2573767982\\
3.85	272.136088934401\\
3.86666666666667	272.238556448794\\
3.88333333333333	272.310100979564\\
3.9	272.465521796146\\
3.91666666666667	272.854362743827\\
3.93333333333333	273.256583677321\\
3.95	273.562128267981\\
3.96666666666667	274.143558293908\\
3.98333333333333	274.461684485911\\
4	274.679214189855\\
4.01666666666667	275.048162340654\\
4.03333333333333	275.185079640901\\
4.05	275.137353170181\\
4.06666666666667	275.186693089535\\
4.08333333333333	275.224527079286\\
4.1	275.329816947177\\
4.11666666666667	275.305275473594\\
4.13333333333333	275.229828259845\\
4.15	275.055228719373\\
4.16666666666667	274.651500057101\\
4.18333333333333	274.370937754097\\
4.2	274.225790715138\\
4.21666666666667	273.884612125482\\
4.23333333333333	273.591310565125\\
4.25	273.433045648152\\
4.26666666666667	273.416325750876\\
4.28333333333333	273.378848705634\\
4.3	273.412849219044\\
4.31666666666667	273.806470547697\\
4.33333333333333	273.665772673898\\
4.35	273.853060074594\\
4.36666666666667	274.077711973902\\
4.38333333333333	274.129690547319\\
4.4	274.19738653047\\
4.41666666666667	274.151853939336\\
4.43333333333333	273.88460760733\\
4.45	273.774806519454\\
4.46666666666667	273.878948396846\\
4.48333333333333	273.759054510279\\
4.5	273.514449698002\\
4.51666666666667	273.276726758781\\
4.53333333333333	273.310721206072\\
4.55	273.233035089811\\
4.56666666666667	272.933165311917\\
4.58333333333333	272.631960853091\\
4.6	272.300518940811\\
4.61666666666667	272.067537768437\\
4.63333333333333	272.014618057008\\
4.65	271.999501115459\\
4.66666666666667	272.261800667787\\
4.68333333333333	272.502160954456\\
4.7	272.820781360964\\
4.71666666666667	273.254175682901\\
4.73333333333333	273.705956659372\\
4.75	274.174507197759\\
4.76666666666667	274.531076166683\\
4.78333333333333	274.750153118644\\
4.8	274.832543279759\\
4.81666666666667	275.106422572308\\
4.83333333333333	275.143554859179\\
4.85	275.251492522499\\
4.86666666666667	275.109444149026\\
4.88333333333333	275.273790284449\\
4.9	275.177427001244\\
4.91666666666667	275.244563096597\\
4.93333333333333	275.076132075421\\
4.95	274.819654920221\\
4.96666666666667	274.598112156671\\
4.98333333333333	274.216637359336\\
5	273.902301795233\\
5.01666666666667	273.633787876631\\
5.03333333333333	273.428063037473\\
5.05	273.374722516719\\
5.06666666666667	273.237657195796\\
5.08333333333333	273.487101983276\\
5.1	273.570453249659\\
5.11666666666667	273.839155478897\\
5.13333333333333	273.944178900444\\
5.15	274.203562459566\\
5.16666666666667	274.190934709567\\
5.18333333333333	274.191392099565\\
5.2	274.188868688491\\
5.21666666666667	274.115804518982\\
5.23333333333333	273.66804060653\\
5.25	273.69845758703\\
5.26666666666667	273.621607171794\\
5.28333333333333	273.280832137106\\
5.3	273.240654764776\\
5.31666666666667	273.140743730313\\
5.33333333333333	273.022521669737\\
5.35	272.792293904473\\
5.36666666666667	272.647913510349\\
5.38333333333333	272.564513030068\\
5.4	272.386556527773\\
5.41666666666667	272.347099254557\\
5.43333333333333	272.451590193255\\
5.45	272.513471588361\\
5.46666666666667	272.717840776102\\
5.48333333333333	273.240719495249\\
5.5	273.550779474076\\
5.51666666666667	274.281451431488\\
5.53333333333333	274.670394200924\\
5.55	275.017629626548\\
5.56666666666667	275.225109560273\\
5.58333333333333	275.381864646342\\
5.6	275.535030719762\\
5.61666666666667	275.491597880615\\
5.63333333333333	275.564625204801\\
5.65	275.430733542299\\
5.66666666666667	275.556031161756\\
5.68333333333333	275.248199120948\\
5.7	275.357420820653\\
5.71666666666667	275.252003563764\\
5.73333333333333	275.043375689943\\
5.75	274.767567601602\\
5.76666666666667	274.380231012847\\
5.78333333333333	274.11748032697\\
5.8	273.768732930812\\
5.81666666666667	273.756516563273\\
5.83333333333333	273.493716310424\\
5.85	273.503034023995\\
5.86666666666667	273.532757168403\\
5.88333333333333	273.743981791631\\
5.9	273.876690710122\\
5.91666666666667	274.161857450149\\
5.93333333333333	274.254786382547\\
5.95	274.14134014145\\
5.96666666666667	274.151938303584\\
5.98333333333333	274.17904548997\\
6	273.926958013186\\
6.01666666666667	273.750786621565\\
6.03333333333333	273.649255022827\\
6.05	273.57662897304\\
6.06666666666667	273.422643109957\\
6.08333333333333	273.169631645961\\
6.1	272.888002646111\\
6.11666666666667	272.566741691666\\
6.13333333333333	272.68873938214\\
6.15	272.647541383037\\
6.16666666666667	272.625764314079\\
6.18333333333333	272.581442865581\\
6.2	272.722109571033\\
6.21666666666667	272.648562254897\\
6.23333333333333	272.664420248437\\
6.25	273.024681817459\\
6.26666666666667	273.401567728387\\
6.28333333333333	273.770300660439\\
6.3	274.266278739898\\
6.31666666666667	274.69306143715\\
6.33333333333333	274.999925865878\\
6.35	275.248717029702\\
6.36666666666667	275.571965806616\\
6.38333333333333	275.416343968215\\
6.4	275.427787794842\\
6.41666666666667	275.389627802861\\
6.43333333333333	275.65034098724\\
6.45	275.355150634226\\
6.46666666666667	275.413452264254\\
6.48333333333333	275.343281803873\\
6.5	275.271122382144\\
6.51666666666667	274.890725396931\\
6.53333333333333	274.80887877787\\
6.55	274.606674758786\\
6.56666666666667	274.276354495475\\
6.58333333333333	274.027368025196\\
6.6	273.984694438927\\
6.61666666666667	273.707129617979\\
6.63333333333333	273.733334970043\\
6.65	273.531620256449\\
6.66666666666667	273.651473437306\\
6.68333333333333	273.667604813648\\
6.7	273.876530512568\\
6.71666666666667	274.011003566768\\
6.73333333333333	273.966059423311\\
6.75	273.75151825723\\
6.76666666666667	273.735771317724\\
6.78333333333333	273.629613962187\\
6.8	273.442283536902\\
6.81666666666667	273.30001141017\\
6.83333333333333	273.076603246125\\
6.85	273.065041035899\\
6.86666666666667	272.854203151386\\
6.88333333333333	272.723935572075\\
6.9	272.445270832001\\
6.91666666666667	272.41688712691\\
6.93333333333333	272.744132648333\\
6.95	272.741753474648\\
6.96666666666667	272.855849344272\\
6.98333333333333	273.069728756297\\
7	273.056606443058\\
7.01666666666667	273.199712519424\\
7.03333333333333	273.602588247701\\
7.05	273.97260551945\\
7.06666666666667	274.267609385764\\
7.08333333333333	274.653349588478\\
7.1	274.928525505746\\
7.11666666666667	275.353052605109\\
7.13333333333333	275.595529557959\\
7.15	275.706498632989\\
7.16666666666667	275.652733058635\\
7.18333333333333	275.61212863505\\
7.2	275.678144096663\\
7.21666666666667	275.560738405853\\
7.23333333333333	275.416425842501\\
7.25	275.556751205647\\
7.26666666666667	275.367955504531\\
7.28333333333333	275.090307065765\\
7.3	275.001528974942\\
7.31666666666667	275.024981187536\\
7.33333333333333	274.783546091511\\
7.35	274.605577165134\\
7.36666666666667	274.284954568668\\
7.38333333333333	274.2068776499\\
7.4	274.001363708211\\
7.41666666666667	273.971740715671\\
7.43333333333333	273.898748012846\\
7.45	273.686330904307\\
7.46666666666667	273.688062252375\\
7.48333333333333	273.810775468873\\
7.5	273.885308408945\\
7.51666666666667	273.656900131594\\
7.53333333333333	273.669342711888\\
7.55	273.517460106386\\
7.56666666666667	273.517609110369\\
7.58333333333333	273.341512630807\\
7.6	273.181705469798\\
7.61666666666667	273.073809686677\\
7.63333333333333	272.802319545772\\
7.65	272.858447162182\\
7.66666666666667	272.650631893826\\
7.68333333333333	272.550495770054\\
7.7	272.585201079065\\
7.71666666666667	272.680708876695\\
7.73333333333333	272.885648936424\\
7.75	273.076267039771\\
7.76666666666667	273.25078496256\\
7.78333333333333	273.346318249349\\
7.8	273.548149889303\\
7.81666666666667	273.980130806908\\
7.83333333333333	274.338833747479\\
7.85	274.505864491896\\
7.86666666666667	274.58568455694\\
7.88333333333333	274.773580920027\\
7.9	275.156672497099\\
7.91666666666667	275.427205677403\\
7.93333333333333	275.635180791775\\
7.95	275.665136749818\\
7.96666666666667	275.626902959755\\
7.98333333333333	275.410838109342\\
8	275.287981895814\\
8.01666666666667	275.333297670186\\
8.03333333333333	275.459858383541\\
8.05	275.21884019025\\
8.06666666666667	274.926375898751\\
8.08333333333333	274.814906502951\\
8.1	274.909078787263\\
8.11666666666667	274.778666633115\\
8.13333333333333	274.903916772165\\
8.15	274.655288561753\\
8.16666666666667	274.410197401299\\
8.18333333333333	274.367031530113\\
8.2	274.344234247245\\
8.21666666666667	274.122946845023\\
8.23333333333333	273.873753114811\\
8.25	273.760873975876\\
8.26666666666667	273.758268819972\\
8.28333333333333	273.565224626894\\
8.3	273.4703754038\\
8.31666666666667	273.679457793559\\
8.33333333333333	273.628040705278\\
8.35	273.286071456145\\
8.36666666666667	273.232328875295\\
8.38333333333333	273.150689406199\\
8.4	273.029994252036\\
8.41666666666667	272.862550854204\\
8.43333333333333	272.729628309328\\
8.45	272.606122261488\\
8.46666666666667	272.600085991199\\
8.48333333333333	272.716453482745\\
8.5	272.819517884167\\
8.51666666666667	273.011000767797\\
8.53333333333333	273.470804870882\\
8.55	273.673753983686\\
8.56666666666667	273.947856247445\\
8.58333333333333	274.243546683969\\
8.6	274.566184427791\\
8.61666666666667	274.578102644101\\
8.63333333333333	274.668822594452\\
8.65	274.749106229594\\
8.66666666666667	275.198631616433\\
8.68333333333333	275.144141144471\\
8.7	275.479913116348\\
8.71666666666667	275.70922100361\\
8.73333333333333	275.666521072594\\
8.75	275.569545445716\\
8.76666666666667	275.320302417048\\
8.78333333333333	275.23446595564\\
8.8	275.015087375938\\
8.81666666666667	275.231776417582\\
8.83333333333333	275.073615493072\\
8.85	274.915516523026\\
8.86666666666667	274.993947669142\\
8.88333333333333	274.81991805747\\
8.9	274.805238630097\\
8.91666666666667	274.858399700773\\
8.93333333333333	274.868376991151\\
8.95	274.653187163567\\
8.96666666666667	274.550848078069\\
8.98333333333333	274.436156561054\\
9	274.108550589189\\
9.01666666666667	273.974853860381\\
9.03333333333333	273.57565813035\\
9.05	273.300031044688\\
9.06666666666667	273.402967929459\\
9.08333333333333	273.289019112796\\
9.1	273.271778310959\\
9.11666666666667	273.295181423951\\
9.13333333333333	273.270528689391\\
9.15	273.069452552767\\
9.16666666666667	273.039038305581\\
9.18333333333333	273.003495964996\\
9.2	272.720792581478\\
9.21666666666667	272.493382826712\\
9.23333333333333	272.459164130532\\
9.25	272.644673784325\\
9.26666666666667	272.795517386462\\
9.28333333333333	272.862310240783\\
9.3	273.237798748385\\
9.31666666666667	273.439937216972\\
9.33333333333333	273.934970577955\\
9.35	274.143895538552\\
9.36666666666667	274.4419201352\\
9.38333333333333	274.742500962889\\
9.4	274.687638626605\\
9.41666666666667	274.702015292086\\
9.43333333333333	274.817857688404\\
9.45	274.784015333992\\
9.46666666666667	275.01811365345\\
9.48333333333333	275.037719002419\\
9.5	275.26916539568\\
9.51666666666667	275.356451697324\\
9.53333333333333	275.594106162272\\
9.55	275.45500309256\\
9.56666666666667	275.137070171789\\
9.58333333333333	274.994246606655\\
9.6	274.999558841703\\
9.61666666666667	274.92424513696\\
9.63333333333333	274.913477901758\\
9.65	274.800972693695\\
9.66666666666667	275.045754265914\\
9.68333333333333	274.941826261519\\
9.7	274.887429667462\\
9.71666666666667	274.818107792341\\
9.73333333333333	274.741526851451\\
9.75	274.573245483097\\
9.76666666666667	274.224678530883\\
9.78333333333333	273.949130165498\\
9.8	273.532650547677\\
9.81666666666667	273.279829625875\\
9.83333333333333	273.015349858033\\
9.85	272.854145846124\\
9.86666666666667	273.003258365139\\
9.88333333333333	272.975662099371\\
9.9	272.889212201637\\
9.91666666666667	272.85402193326\\
9.93333333333333	272.716638266905\\
9.95	272.820551780356\\
9.96666666666667	272.842225847395\\
9.98333333333333	272.752104130332\\
10	272.727950865716\\
10.0166666666667	272.70126631746\\
10.0333333333333	272.94582813064\\
10.05	273.221481840305\\
10.0666666666667	273.448445428531\\
10.0833333333333	273.911829555016\\
10.1	274.091391436577\\
10.1166666666667	274.351020269699\\
10.1333333333333	274.656738559519\\
10.15	274.848787110061\\
10.1666666666667	275.023489939556\\
10.1833333333333	275.111592303504\\
10.2	275.069581924318\\
10.2166666666667	275.012206408964\\
10.2333333333333	274.985240122711\\
10.25	275.153447637711\\
10.2666666666667	275.180809038222\\
10.2833333333333	275.067494846661\\
10.3	275.259966243009\\
10.3166666666667	275.351680039716\\
10.3333333333333	275.343745636639\\
10.35	275.075662820796\\
10.3666666666667	275.066629428688\\
10.3833333333333	275.028556077709\\
10.4	274.907518593635\\
10.4166666666667	274.998945414185\\
10.4333333333333	275.120210325683\\
10.45	275.074639594808\\
10.4666666666667	274.946554420225\\
10.4833333333333	275.047472535135\\
10.5	275.060962382714\\
10.5166666666667	274.783722930988\\
10.5333333333333	274.618984643455\\
10.55	274.215314948787\\
10.5666666666667	273.865293722243\\
10.5833333333333	273.465393688124\\
10.6	273.140981465731\\
10.6166666666667	272.882841045571\\
10.6333333333333	272.761631406005\\
10.65	272.739181925234\\
10.6666666666667	272.653552846484\\
10.6833333333333	272.598525790424\\
10.7	272.584261234057\\
10.7166666666667	272.525824126648\\
10.7333333333333	272.524160815192\\
10.75	272.675698605747\\
10.7666666666667	272.956591655332\\
10.7833333333333	272.794285096651\\
10.8	273.11069971391\\
10.8166666666667	273.383020050111\\
10.8333333333333	273.673442942475\\
10.85	273.972152216217\\
10.8666666666667	274.163782436016\\
10.8833333333333	274.44356412612\\
10.9	274.573272977756\\
10.9166666666667	274.733346928686\\
10.9333333333333	275.023614559511\\
10.95	275.168202849567\\
10.9666666666667	275.30030020717\\
10.9833333333333	275.010838318874\\
11	275.086715362317\\
11.0166666666667	274.896854133297\\
11.0333333333333	274.901980673781\\
11.05	274.882497376428\\
11.0666666666667	275.083041626723\\
11.0833333333333	274.980754829392\\
11.1	274.818342304374\\
11.1166666666667	274.939858443215\\
11.1333333333333	274.909340572201\\
11.15	274.973015658459\\
11.1666666666667	275.020325814083\\
11.1833333333333	274.83921041477\\
11.2	274.758631808055\\
11.2166666666667	274.957123987227\\
11.2333333333333	274.875424606099\\
11.25	274.71179122572\\
11.2666666666667	274.802229845771\\
11.2833333333333	274.600617292883\\
11.3	274.410233685639\\
11.3166666666667	274.404292360004\\
11.3333333333333	274.029631372317\\
11.35	273.589207857495\\
11.3666666666667	273.299754998227\\
11.3833333333333	272.895312444812\\
11.4	272.65455773002\\
11.4166666666667	272.570197633918\\
11.4333333333333	272.405913935987\\
11.45	272.446880929586\\
11.4666666666667	272.499007656428\\
11.4833333333333	272.461070984798\\
11.5	272.634489009887\\
11.5166666666667	272.758671083744\\
11.5333333333333	272.928021152549\\
11.55	273.132334750773\\
11.5666666666667	273.32719930335\\
11.5833333333333	273.668714622363\\
11.6	274.090645024489\\
11.6166666666667	274.318181812257\\
11.6333333333333	274.612026160953\\
11.65	274.763313815871\\
11.6666666666667	274.719337342699\\
11.6833333333333	274.857892041071\\
11.7	275.049889370325\\
11.7166666666667	275.243315342495\\
11.7333333333333	275.301470722141\\
11.75	275.372000627326\\
11.7666666666667	275.26884020943\\
11.7833333333333	275.177795502424\\
11.8	274.97269626082\\
11.8166666666667	274.873843154613\\
11.8333333333333	274.959219645293\\
11.85	274.909779981321\\
11.8666666666667	274.732675992097\\
11.8833333333333	274.762110036073\\
11.9	274.950700525321\\
11.9166666666667	275.132430616261\\
11.9333333333333	275.096290369669\\
11.95	274.981912186406\\
11.9666666666667	275.069155669326\\
11.9833333333333	274.902725602723\\
12	274.715526475393\\
12	375.301891400849\\
11.9833333333333	375.619505107743\\
11.9666666666667	375.728705293241\\
11.95	376.021907523296\\
11.9333333333333	376.258942632623\\
11.9166666666667	376.264360827589\\
11.9	376.12890222487\\
11.8833333333333	376.010082477296\\
11.8666666666667	375.891617361608\\
11.85	375.825743318908\\
11.8333333333333	375.63818295211\\
11.8166666666667	375.532879534463\\
11.8	375.410802593267\\
11.7833333333333	375.371937117897\\
11.7666666666667	375.276614336025\\
11.75	375.377579204607\\
11.7333333333333	375.37171491575\\
11.7166666666667	375.621772510828\\
11.7	375.320164105611\\
11.6833333333333	375.402153795445\\
11.6666666666667	375.464314299776\\
11.65	375.774806886956\\
11.6333333333333	375.863756879536\\
11.6166666666667	375.768296415398\\
11.6	375.961608604235\\
11.5833333333333	376.003401496812\\
11.5666666666667	375.940333164182\\
11.55	375.935350505148\\
11.5333333333333	376.149442560208\\
11.5166666666667	376.037661994942\\
11.5	375.900270348402\\
11.4833333333333	375.547485164935\\
11.4666666666667	375.328035888263\\
11.45	375.052584311056\\
11.4333333333333	374.685300731699\\
11.4166666666667	374.452567835906\\
11.4	373.939177946068\\
11.3833333333333	373.672754782079\\
11.3666666666667	373.54516478787\\
11.35	373.314382669625\\
11.3333333333333	373.502225006598\\
11.3166666666667	373.663545684305\\
11.3	373.678994733001\\
11.2833333333333	373.74101754287\\
11.2666666666667	374.271566945673\\
11.25	374.608911600865\\
11.2333333333333	374.838249954635\\
11.2166666666667	375.23752841919\\
11.2	375.494080185833\\
11.1833333333333	375.907924802954\\
11.1666666666667	376.166228807766\\
11.15	376.239207412588\\
11.1333333333333	376.178818327723\\
11.1166666666667	376.063961266487\\
11.1	376.137196274694\\
11.0833333333333	375.989451434931\\
11.0666666666667	375.744765859908\\
11.05	375.557838758025\\
11.0333333333333	375.509020090161\\
11.0166666666667	375.375414774266\\
11	375.262558892839\\
10.9833333333333	375.04385992406\\
10.9666666666667	375.265322405512\\
10.95	375.286800970143\\
10.9333333333333	375.446668099007\\
10.9166666666667	375.417302421964\\
10.9	375.409309146003\\
10.8833333333333	375.537795690534\\
10.8666666666667	375.695346670171\\
10.85	375.883309968657\\
10.8333333333333	376.029230854317\\
10.8166666666667	376.051357337207\\
10.8	376.14827660389\\
10.7833333333333	376.236730946129\\
10.7666666666667	376.251047764072\\
10.75	376.311925534818\\
10.7333333333333	376.284395334541\\
10.7166666666667	376.084107118578\\
10.7	375.88693815479\\
10.6833333333333	375.430809580088\\
10.6666666666667	375.17960223373\\
10.65	374.824454438402\\
10.6333333333333	374.484816263973\\
10.6166666666667	374.182705172917\\
10.6	373.647101039999\\
10.5833333333333	373.438196838996\\
10.5666666666667	373.303537446588\\
10.55	373.553669008432\\
10.5333333333333	373.494078763726\\
10.5166666666667	373.734535281388\\
10.5	373.94988559284\\
10.4833333333333	374.105163064559\\
10.4666666666667	374.305699208499\\
10.45	374.576697303588\\
10.4333333333333	375.197283944753\\
10.4166666666667	375.68035175923\\
10.4	375.953291184822\\
10.3833333333333	376.15005354796\\
10.3666666666667	376.281880884528\\
10.35	376.239234047042\\
10.3333333333333	376.51446673922\\
10.3166666666667	376.265737836526\\
10.3	376.14171442926\\
10.2833333333333	375.966271387106\\
10.2666666666667	375.890237562236\\
10.25	375.583297969622\\
10.2333333333333	375.461818700819\\
10.2166666666667	375.37312590578\\
10.2	375.205742751007\\
10.1833333333333	375.1845115926\\
10.1666666666667	375.265891267472\\
10.15	375.310418390321\\
10.1333333333333	375.416905443537\\
10.1166666666667	375.645007232211\\
10.1	375.789739197495\\
10.0833333333333	375.932631865916\\
10.0666666666667	375.993265801415\\
10.05	376.091887143652\\
10.0333333333333	376.429878515654\\
10.0166666666667	376.362752017147\\
10	376.705204214498\\
9.98333333333333	376.589989070585\\
9.96666666666667	376.587873465057\\
9.95	376.617492528277\\
9.93333333333333	376.500168455784\\
9.91666666666667	376.093877226404\\
9.9	375.715218661618\\
9.88333333333333	375.384307342951\\
9.86666666666667	374.991852406442\\
9.85	374.503531770377\\
9.83333333333333	374.296491242044\\
9.81666666666667	373.802828892078\\
9.8	373.424721492048\\
9.78333333333333	373.445216664143\\
9.76666666666667	373.466077771638\\
9.75	373.739665135696\\
9.73333333333333	373.703392934645\\
9.71666666666667	373.959890679776\\
9.7	374.231592486854\\
9.68333333333333	374.413101163997\\
9.66666666666667	374.699089125988\\
9.65	374.910868406382\\
9.63333333333333	375.476743641405\\
9.61666666666667	375.775220103682\\
9.6	376.172939248442\\
9.58333333333333	376.248534142008\\
9.56666666666667	376.411287353039\\
9.55	376.421543889869\\
9.53333333333333	376.468842653618\\
9.51666666666667	376.215740816045\\
9.5	376.129459508827\\
9.48333333333333	375.812701165648\\
9.46666666666667	375.612902389331\\
9.45	375.514533176322\\
9.43333333333333	375.408574702734\\
9.41666666666667	375.350238336638\\
9.4	375.439175735504\\
9.38333333333333	375.348255339632\\
9.36666666666667	375.408652821255\\
9.35	375.522720198652\\
9.33333333333333	375.834930109593\\
9.31666666666667	375.836762553846\\
9.3	375.961284521286\\
9.28333333333333	376.085588918881\\
9.26666666666667	376.324115921406\\
9.25	376.288557690083\\
9.23333333333333	376.537627313318\\
9.21666666666667	376.615860870767\\
9.2	376.779589389492\\
9.18333333333333	376.704067060214\\
9.16666666666667	376.598853214664\\
9.15	376.497545155407\\
9.13333333333333	376.35086168494\\
9.11666666666667	375.875788782314\\
9.1	375.431659427773\\
9.08333333333333	375.069728022422\\
9.06666666666667	374.550584400564\\
9.05	374.157264600843\\
9.03333333333333	373.656121854371\\
9.01666666666667	373.492220241987\\
9	373.353023131209\\
8.98333333333333	373.485004630695\\
8.96666666666667	373.400259637745\\
8.95	373.668890758511\\
8.93333333333333	373.833227286924\\
8.91666666666667	374.132280207554\\
8.9	374.233722408864\\
8.88333333333333	374.569233966977\\
8.86666666666667	374.935464095564\\
8.85	375.249189359327\\
8.83333333333333	375.699799327401\\
8.81666666666667	376.07884237539\\
8.8	376.137853800533\\
8.78333333333333	376.342615786148\\
8.76666666666667	376.371676192577\\
8.75	376.420676097447\\
8.73333333333333	376.354869301738\\
8.71666666666667	376.303002067437\\
8.7	376.024441352712\\
8.68333333333333	375.758227075544\\
8.66666666666667	375.661567008471\\
8.65	375.382597360933\\
8.63333333333333	375.28320185169\\
8.61666666666667	375.26085839486\\
8.6	375.456122676869\\
8.58333333333333	375.520242468056\\
8.56666666666667	375.592861857979\\
8.55	375.697674587743\\
8.53333333333333	375.834007963343\\
8.51666666666667	375.95339953778\\
8.5	376.132659350363\\
8.48333333333333	376.291338725047\\
8.46666666666667	376.500601556547\\
8.45	376.658965591835\\
8.43333333333333	376.834924784637\\
8.41666666666667	376.868847159547\\
8.4	376.782687184175\\
8.38333333333333	376.699272396704\\
8.36666666666667	376.579741407364\\
8.35	376.352431217652\\
8.33333333333333	376.155305360421\\
8.31666666666667	375.632077729741\\
8.3	375.267898087415\\
8.28333333333333	374.907808222609\\
8.26666666666667	374.272900011196\\
8.25	373.798484312894\\
8.23333333333333	373.512648718649\\
8.21666666666667	373.517694866207\\
8.2	373.641098067499\\
8.18333333333333	373.449622404188\\
8.16666666666667	373.568412224369\\
8.15	373.867858879042\\
8.13333333333333	374.135044266796\\
8.11666666666667	374.317131686212\\
8.1	374.704366590889\\
8.08333333333333	374.810761946247\\
8.06666666666667	375.197993849148\\
8.05	375.455567754746\\
8.03333333333333	375.924709989263\\
8.01666666666667	376.136221046392\\
8	376.278099082032\\
7.98333333333333	376.46632002664\\
7.96666666666667	376.522218507014\\
7.95	376.46809472459\\
7.93333333333333	376.361763440464\\
7.91666666666667	375.988531526569\\
7.9	375.937597938348\\
7.88333333333333	375.678367131921\\
7.86666666666667	375.580243632518\\
7.85	375.386266906118\\
7.83333333333333	375.513572669634\\
7.81666666666667	375.463413883695\\
7.8	375.447419247443\\
7.78333333333333	375.599747449658\\
7.76666666666667	375.771674930488\\
7.75	375.891189033567\\
7.73333333333333	375.948575662507\\
7.71666666666667	376.166349946835\\
7.7	376.253301594732\\
7.68333333333333	376.423071838808\\
7.66666666666667	376.830804317022\\
7.65	376.980513876779\\
7.63333333333333	376.811736985931\\
7.61666666666667	376.847657081848\\
7.6	376.707370160454\\
7.58333333333333	376.669182556359\\
7.56666666666667	376.56459104242\\
7.55	376.195603300795\\
7.53333333333333	375.819427341588\\
7.51666666666667	375.359448225931\\
7.5	375.129817641475\\
7.48333333333333	374.491134385978\\
7.46666666666667	373.900631406907\\
7.45	373.790521654898\\
7.43333333333333	373.616607220157\\
7.41666666666667	373.509389918401\\
7.4	373.507879989268\\
7.38333333333333	373.582885528099\\
7.36666666666667	373.770049251042\\
7.35	374.006798694301\\
7.33333333333333	374.300640310323\\
7.31666666666667	374.759587185267\\
7.3	374.813597075478\\
7.28333333333333	375.056064210018\\
7.26666666666667	375.302174365599\\
7.25	375.668917243551\\
7.23333333333333	376.008478664757\\
7.21666666666667	376.192966712558\\
7.2	376.42758546789\\
7.18333333333333	376.44287518466\\
7.16666666666667	376.372935390563\\
7.15	376.182882574039\\
7.13333333333333	375.961995270154\\
7.11666666666667	376.040071917427\\
7.1	375.912727359036\\
7.08333333333333	375.573693956213\\
7.06666666666667	375.489457077185\\
7.05	375.503330309427\\
7.03333333333333	375.214524051764\\
7.01666666666667	375.464000238406\\
7	375.615662464505\\
6.98333333333333	375.588483619562\\
6.96666666666667	375.692966545721\\
6.95	375.83211971099\\
6.93333333333333	376.181764983447\\
6.91666666666667	376.110538388751\\
6.9	376.367563392598\\
6.88333333333333	376.810059844274\\
6.86666666666667	376.988424808889\\
6.85	376.863148421702\\
6.83333333333333	376.894978113692\\
6.81666666666667	376.991814411068\\
6.8	376.773148090294\\
6.78333333333333	376.626153799464\\
6.76666666666667	376.372097284262\\
6.75	375.997603209538\\
6.73333333333333	375.565491378828\\
6.71666666666667	375.13979857227\\
6.7	374.746846110809\\
6.68333333333333	374.261501374282\\
6.66666666666667	374.095508992029\\
6.65	373.818723517425\\
6.63333333333333	373.684846848139\\
6.61666666666667	373.507538067277\\
6.6	373.781844904083\\
6.58333333333333	373.690279033627\\
6.56666666666667	373.881323121026\\
6.55	374.089429137318\\
6.53333333333333	374.374925652993\\
6.51666666666667	374.488648170678\\
6.5	374.811688924197\\
6.48333333333333	375.252898486424\\
6.46666666666667	375.49136056997\\
6.45	375.548439892894\\
6.43333333333333	376.158367950881\\
6.41666666666667	376.247346987055\\
6.4	376.292303878191\\
6.38333333333333	376.211463518416\\
6.36666666666667	376.293580411871\\
6.35	376.0435671567\\
6.33333333333333	376.185406448866\\
6.31666666666667	376.017404567434\\
6.3	375.877647921675\\
6.28333333333333	375.387835321226\\
6.26666666666667	375.506606450376\\
6.25	375.293729183305\\
6.23333333333333	375.114342165619\\
6.21666666666667	375.29276700408\\
6.2	375.565743752114\\
6.18333333333333	375.532842848705\\
6.16666666666667	375.951164639321\\
6.15	376.102344025672\\
6.13333333333333	376.196822115187\\
6.11666666666667	376.334709798021\\
6.1	376.600767407365\\
6.08333333333333	376.601491348692\\
6.06666666666667	376.908907692183\\
6.05	376.962102883338\\
6.03333333333333	376.839362242261\\
6.01666666666667	376.73966410418\\
6	376.605662307669\\
5.98333333333333	376.350519063124\\
5.96666666666667	375.877702643704\\
5.95	375.316261080857\\
5.93333333333333	375.249262509737\\
5.91666666666667	374.832489379492\\
5.9	374.447831826165\\
5.88333333333333	374.084895213716\\
5.86666666666667	373.919649248709\\
5.85	373.730579421383\\
5.83333333333333	373.634778723953\\
5.81666666666667	373.819954024962\\
5.8	373.763734601655\\
5.78333333333333	373.748065891518\\
5.76666666666667	373.921678842004\\
5.75	374.217611924754\\
5.73333333333333	374.419114760783\\
5.71666666666667	374.793527375884\\
5.7	375.106139148789\\
5.68333333333333	375.317423491734\\
5.66666666666667	375.489346989504\\
5.65	375.821214509649\\
5.63333333333333	376.246528347529\\
5.61666666666667	376.13193153115\\
5.6	376.295679746243\\
5.58333333333333	376.344338562214\\
5.56666666666667	376.192461104356\\
5.55	376.135922703475\\
5.53333333333333	376.120744072567\\
5.51666666666667	375.679129164387\\
5.5	375.61285688956\\
5.48333333333333	375.468829779006\\
5.46666666666667	375.314550362171\\
5.45	375.259790443724\\
5.43333333333333	375.104712327754\\
5.41666666666667	375.258095550638\\
5.4	375.419096642586\\
5.38333333333333	375.823569475662\\
5.36666666666667	375.83978702441\\
5.35	375.989371488957\\
5.33333333333333	376.256011561737\\
5.31666666666667	376.535650463728\\
5.3	376.693035074796\\
5.28333333333333	376.713973057699\\
5.26666666666667	376.994741185731\\
5.25	376.888555399983\\
5.23333333333333	376.816298583691\\
5.21666666666667	376.79878677208\\
5.2	376.54726576529\\
5.18333333333333	376.207996746883\\
5.16666666666667	375.717697834359\\
5.15	375.466720198952\\
5.13333333333333	375.118617127058\\
5.11666666666667	374.715771946618\\
5.1	374.449714817568\\
5.08333333333333	374.145594731773\\
5.06666666666667	373.852946318337\\
5.05	373.900449370218\\
5.03333333333333	373.789660415544\\
5.01666666666667	373.887678891895\\
5	373.795177196364\\
4.98333333333333	373.945929485584\\
4.96666666666667	374.041765612618\\
4.95	374.37392796748\\
4.93333333333333	374.720048214877\\
4.91666666666667	375.098599623037\\
4.9	375.18162571075\\
4.88333333333333	375.420785727774\\
4.86666666666667	375.550296110714\\
4.85	375.972953619594\\
4.83333333333333	376.261792734404\\
4.81666666666667	376.274478879182\\
4.8	376.198778339798\\
4.78333333333333	376.329449631547\\
4.76666666666667	376.257006339046\\
4.75	376.084927485205\\
4.73333333333333	375.678306136655\\
4.71666666666667	375.718780772408\\
4.7	375.72314530061\\
4.68333333333333	375.420833697951\\
4.66666666666667	375.174104603413\\
4.65	375.070323177894\\
4.63333333333333	375.066207000288\\
4.61666666666667	375.266916013075\\
4.6	375.693369523665\\
4.58333333333333	375.735953585412\\
4.56666666666667	375.896170058595\\
4.55	376.280181105758\\
4.53333333333333	376.365672987969\\
4.51666666666667	376.509675074679\\
4.5	376.801822265329\\
4.48333333333333	376.948355726543\\
4.46666666666667	376.846185292993\\
4.45	376.594635802929\\
4.43333333333333	376.709128068758\\
4.41666666666667	376.738902363185\\
4.4	376.295508809484\\
4.38333333333333	376.080240697907\\
4.36666666666667	375.82175326674\\
4.35	375.218292102641\\
4.33333333333333	374.687474079349\\
4.31666666666667	374.568166579882\\
4.3	374.277295929925\\
4.28333333333333	374.225123792456\\
4.26666666666667	374.044025662416\\
4.25	374.012026926332\\
4.23333333333333	373.889208915395\\
4.21666666666667	374.090941732425\\
4.2	374.1775706294\\
4.18333333333333	374.027534362022\\
4.16666666666667	374.328790240836\\
4.15	374.602830868099\\
4.13333333333333	374.948323000659\\
4.11666666666667	375.193731401884\\
4.1	375.485767468408\\
4.08333333333333	375.622226167468\\
4.06666666666667	375.723620126661\\
4.05	375.933693430277\\
4.03333333333333	376.167097593629\\
4.01666666666667	376.238010310224\\
4	376.177317513736\\
3.98333333333333	376.124106193997\\
3.96666666666667	376.037648734358\\
3.95	375.66567921865\\
3.93333333333333	375.762820447966\\
3.91666666666667	375.617447798572\\
3.9	375.557702038843\\
3.88333333333333	375.311594971544\\
3.86666666666667	375.293299930122\\
3.85	375.078578750855\\
3.83333333333333	375.124135806842\\
3.81666666666667	375.422359823058\\
3.8	375.710553358587\\
3.78333333333333	375.821309399751\\
3.76666666666667	376.234442335171\\
3.75	376.497809340244\\
3.73333333333333	376.438589882963\\
3.71666666666667	376.69319150151\\
3.7	376.802601404659\\
3.68333333333333	376.797899442557\\
3.66666666666667	376.810060411686\\
3.65	376.811810035143\\
3.63333333333333	376.788875711047\\
3.61666666666667	376.356206075319\\
3.6	376.185729107197\\
3.58333333333333	375.745901157612\\
3.56666666666667	375.410438347784\\
3.55	374.956329006954\\
3.53333333333333	374.556424261272\\
3.51666666666667	374.47467818587\\
3.5	374.287160868692\\
3.48333333333333	374.136147699033\\
3.46666666666667	373.933461523663\\
3.45	373.776417705433\\
3.43333333333333	373.94736612613\\
3.41666666666667	374.079106701427\\
3.4	374.250356244782\\
3.38333333333333	374.070262915021\\
3.36666666666667	374.574533410687\\
3.35	374.94620933901\\
3.33333333333333	375.175750263747\\
3.31666666666667	375.55245829453\\
3.3	375.66110197278\\
3.28333333333333	375.697819207851\\
3.26666666666667	375.937634808145\\
3.25	376.129756683855\\
3.23333333333333	376.026959828685\\
3.21666666666667	375.961021483805\\
3.2	375.952649038287\\
3.18333333333333	375.794433946653\\
3.16666666666667	375.694974155175\\
3.15	375.805001853834\\
3.13333333333333	375.64240200808\\
3.11666666666667	375.526179442723\\
3.1	375.287779813701\\
3.08333333333333	375.055043068639\\
3.06666666666667	374.926957679269\\
3.05	374.947663814347\\
3.03333333333333	375.168975469581\\
3.01666666666667	375.350392621122\\
3	375.630779094154\\
2.98333333333333	375.878204692644\\
2.96666666666667	376.252673690658\\
2.95	376.39983778213\\
2.93333333333333	376.51328574771\\
2.91666666666667	376.864665138588\\
2.9	376.854979606068\\
2.88333333333333	376.95071782851\\
2.86666666666667	376.814033684331\\
2.85	376.633600486592\\
2.83333333333333	376.659638805859\\
2.81666666666667	376.444918843732\\
2.8	376.075701209134\\
2.78333333333333	375.675785789925\\
2.76666666666667	375.267443533048\\
2.75	374.924512282753\\
2.73333333333333	374.725730924511\\
2.71666666666667	374.422668526923\\
2.7	374.169663094571\\
2.68333333333333	374.070280208338\\
2.66666666666667	373.871962862496\\
2.65	373.966164285365\\
2.63333333333333	373.93265273392\\
2.61666666666667	374.037778551437\\
2.6	374.403938412144\\
2.58333333333333	374.548821768406\\
2.56666666666667	374.865868970425\\
2.55	375.230065585495\\
2.53333333333333	375.458439609341\\
2.51666666666667	375.775312666297\\
2.5	375.880558639233\\
2.48333333333333	375.98019869153\\
2.46666666666667	375.902704375713\\
2.45	375.92977541772\\
2.43333333333333	375.946641592327\\
2.41666666666667	375.928821021602\\
2.4	375.799943573493\\
2.38333333333333	375.834372996097\\
2.36666666666667	375.819330978819\\
2.35	375.755337083239\\
2.33333333333333	375.778828445763\\
2.31666666666667	375.525937281318\\
2.3	375.269069389458\\
2.28333333333333	375.018301023003\\
2.26666666666667	375.130012110447\\
2.25	375.333201373828\\
2.23333333333333	375.282698026422\\
2.21666666666667	375.646651895682\\
2.2	376.013997306951\\
2.18333333333333	376.295494454457\\
2.16666666666667	376.383755767713\\
2.15	376.614189409828\\
2.13333333333333	376.774052538206\\
2.11666666666667	377.031137519532\\
2.1	376.947696323202\\
2.08333333333333	377.110517131874\\
2.06666666666667	376.831796298313\\
2.05	376.684635312145\\
2.03333333333333	376.474149206633\\
2.01666666666667	376.219362042088\\
2	375.930277637342\\
1.98333333333333	375.53889078634\\
1.96666666666667	375.100332132405\\
1.95	374.689497833557\\
1.93333333333333	374.586265502088\\
1.91666666666667	374.32936683814\\
1.9	374.095310817304\\
1.88333333333333	373.829917807054\\
1.86666666666667	374.123771557762\\
1.85	374.100797483556\\
1.83333333333333	374.128431878987\\
1.81666666666667	374.360714295062\\
1.8	374.870789906123\\
1.78333333333333	374.982988129012\\
1.76666666666667	375.252179227472\\
1.75	375.489615643599\\
1.73333333333333	375.606005542514\\
1.71666666666667	375.880416124077\\
1.7	375.995966948106\\
1.68333333333333	375.695723562204\\
1.66666666666667	375.591079755651\\
1.65	375.667437423511\\
1.63333333333333	375.772918056322\\
1.61666666666667	375.597500994493\\
1.6	375.730221203463\\
1.58333333333333	375.762968059973\\
1.56666666666667	375.656163927566\\
1.55	375.419435918002\\
1.53333333333333	375.33232806796\\
1.51666666666667	375.206532614204\\
1.5	375.037849536675\\
1.48333333333333	375.042731382339\\
1.46666666666667	375.105146223916\\
1.45	375.217396559438\\
1.43333333333333	375.463729673254\\
1.41666666666667	375.921622968316\\
1.4	376.244456658702\\
1.38333333333333	376.381227543366\\
1.36666666666667	376.615644663815\\
1.35	376.6150905229\\
1.33333333333333	376.823074744382\\
1.31666666666667	377.042287152335\\
1.3	377.126385939595\\
1.28333333333333	376.996748107337\\
1.26666666666667	376.766164905176\\
1.25	376.612366121933\\
1.23333333333333	376.250421803765\\
1.21666666666667	375.927460844992\\
1.2	375.811479743866\\
1.18333333333333	375.357860554802\\
1.16666666666667	374.902937999019\\
1.15	374.730340954306\\
1.13333333333333	374.583833227906\\
1.11666666666667	374.240656287356\\
1.1	374.120612641663\\
1.08333333333333	374.120973950166\\
1.06666666666667	374.013108096092\\
1.05	374.176494293295\\
1.03333333333333	374.320211622195\\
1.01666666666667	374.896405546109\\
1	375.142084346895\\
0.983333333333333	375.326176084646\\
0.966666666666667	375.498973214141\\
0.95	375.435676105634\\
0.933333333333333	375.623321824701\\
0.916666666666667	375.758016658236\\
0.9	375.647500103266\\
0.883333333333333	375.409752230211\\
0.866666666666667	375.318096311535\\
0.85	375.438731083799\\
0.833333333333333	375.346036971422\\
0.816666666666667	375.50679379886\\
0.8	375.531507330772\\
0.783333333333333	375.652084770397\\
0.766666666666667	375.448834777264\\
0.75	375.36220342064\\
0.733333333333333	375.231891602197\\
0.716666666666667	375.042661807605\\
0.7	375.126935079969\\
0.683333333333333	375.246817042842\\
0.666666666666667	375.202444313108\\
0.65	375.504046089837\\
0.633333333333333	375.791176364329\\
0.616666666666667	376.122358165396\\
0.6	376.267755728503\\
0.583333333333333	376.683405380901\\
0.566666666666667	376.880765755181\\
0.55	376.875657988404\\
0.533333333333333	376.909452618798\\
0.516666666666667	377.169693737343\\
0.5	377.187627612379\\
0.483333333333333	377.051014591016\\
0.466666666666667	376.697657332621\\
0.45	376.320299999629\\
0.433333333333333	376.242980702943\\
0.416666666666667	375.974222747304\\
0.4	375.563387908932\\
0.383333333333333	375.174639285932\\
0.366666666666667	374.925174857883\\
0.35	374.775230218986\\
0.333333333333333	374.34990582344\\
0.316666666666667	374.330993908468\\
0.3	374.317332208276\\
0.283333333333333	374.307003839958\\
0.266666666666667	374.368332002381\\
0.25	374.704405920114\\
0.233333333333333	374.562306784595\\
0.216666666666667	374.980947718459\\
0.2	375.18650624302\\
0.183333333333333	375.209078779971\\
0.166666666666667	375.285407349444\\
0.15	375.329859729208\\
0.133333333333333	375.267730181548\\
0.116666666666667	375.136258796998\\
0.1	374.999555304744\\
0.0833333333333333	375.013839683745\\
0.0666666666666667	374.841684451136\\
0.05	374.938065009906\\
0.0333333333333333	375.029128908162\\
0.0166666666666667	375.137347522307\\
0	375.019483926633\\
}--cycle;
\addplot [color=mycolor1]
  table[row sep=crcr]{%
0	324.003666921314\\
0.0166666666666515	324.148510313216\\
0.0333333333333599	324.09564553094\\
0.0500000000000114	324.26615737204\\
0.0666666666666629	324.235599694423\\
0.0833333333333144	324.406417112299\\
0.100000000000023	324.275324675325\\
0.116666666666674	324.431168831169\\
0.149999999999977	324.649045072574\\
0.166666666666686	324.552788388082\\
0.199999999999989	324.409778456837\\
0.21666666666664	324.233155080214\\
0.233333333333348	323.882964094729\\
0.25	324.058059587471\\
0.266666666666652	323.727425515661\\
0.28333333333336	323.646142093201\\
0.300000000000011	323.588693659282\\
0.316666666666663	323.662643239114\\
0.333333333333314	323.765928189458\\
0.350000000000023	324.067532467532\\
0.366666666666674	324.112757830405\\
0.383333333333326	324.199847211612\\
0.399999999999977	324.514285714286\\
0.416666666666686	324.76974789916\\
0.433333333333337	324.900840336134\\
0.449999999999989	324.906951871658\\
0.483333333333348	325.588693659282\\
0.5	325.705729564553\\
0.516666666666652	325.685561497326\\
0.53333333333336	325.555080213904\\
0.550000000000011	325.649809014515\\
0.566666666666663	325.599388846448\\
0.583333333333314	325.504048892284\\
0.600000000000023	325.128800611154\\
0.616666666666674	325.028571428571\\
0.633333333333326	324.650878533231\\
0.666666666666686	324.110313216196\\
0.683333333333337	324.08525592055\\
0.699999999999989	323.952330022918\\
0.71666666666664	323.845989304813\\
0.733333333333348	323.926355996944\\
0.75	324.090145148969\\
0.766666666666652	324.103896103896\\
0.78333333333336	324.300993124523\\
0.800000000000011	324.286325439267\\
0.833333333333314	324.488922841864\\
0.850000000000023	324.762719633308\\
0.866666666666674	324.672574484339\\
0.883333333333326	324.741329258976\\
0.899999999999977	324.991596638655\\
0.916666666666686	325.030404889228\\
0.933333333333337	324.957983193277\\
0.949999999999989	324.848892284186\\
0.96666666666664	324.890145148969\\
1.01666666666665	324.456837280367\\
1.03333333333336	324.016501145913\\
1.05000000000001	323.941329258976\\
1.06666666666666	323.70756302521\\
1.08333333333331	323.73384262796\\
1.10000000000002	323.635752482811\\
1.11666666666667	323.765622612681\\
1.13333333333333	324.065393430099\\
1.14999999999998	324.096867838044\\
1.16666666666669	324.176317799847\\
1.19999999999999	324.756913674561\\
1.21666666666664	324.808556149733\\
1.25	325.282505729565\\
1.26666666666665	325.402597402597\\
1.28333333333336	325.58563789152\\
1.30000000000001	325.656531703591\\
1.31666666666666	325.583193277311\\
1.33333333333331	325.526355996944\\
1.35000000000002	325.520855614973\\
1.36666666666667	325.470741023682\\
1.38333333333333	325.350038197097\\
1.39999999999998	325.09243697479\\
1.43333333333334	324.361497326203\\
1.44999999999999	324.185179526356\\
1.48333333333335	323.766844919786\\
1.5	323.775706646295\\
1.51666666666665	323.890909090909\\
1.53333333333336	323.961802902979\\
1.55000000000001	324.071504965623\\
1.56666666666666	324.278991596639\\
1.60000000000002	324.48922841864\\
1.61666666666667	324.554621848739\\
1.63333333333333	324.834835752483\\
1.64999999999998	324.850114591291\\
1.66666666666669	324.873644003056\\
1.68333333333334	324.975401069519\\
1.69999999999999	325.191138273491\\
1.73333333333335	324.977540106952\\
1.75	324.895951107716\\
1.76666666666665	324.768525592055\\
1.78333333333336	324.554621848739\\
1.80000000000001	324.519480519481\\
1.81666666666666	324.12192513369\\
1.83333333333331	323.90038197097\\
1.85000000000002	323.844155844156\\
1.86666666666667	323.757677616501\\
1.88333333333333	323.486936592819\\
1.89999999999998	323.668143621085\\
1.93333333333334	324.092589763178\\
1.94999999999999	324.11550802139\\
1.98333333333335	324.681130634072\\
2	324.882200152788\\
2.01666666666665	325.116577540107\\
2.06666666666666	325.50282658518\\
2.08333333333331	325.628418640183\\
2.10000000000002	325.535828877005\\
2.11666666666667	325.623834988541\\
2.13333333333333	325.539801375095\\
2.14999999999998	325.487242169595\\
2.16666666666669	325.248892284186\\
2.18333333333334	325.063407181054\\
2.19999999999999	324.752941176471\\
2.21666666666664	324.327578304049\\
2.23333333333335	324.073032849503\\
2.25	324.038808250573\\
2.26666666666665	323.770817417876\\
2.28333333333336	323.688922841864\\
2.30000000000001	323.761038961039\\
2.33333333333331	324.273491214668\\
2.35000000000002	324.327883880825\\
2.36666666666667	324.484339190222\\
2.38333333333333	324.671963330787\\
2.39999999999998	324.813139801375\\
2.41666666666669	325.055156608098\\
2.43333333333334	325.176470588235\\
2.44999999999999	325.166692131398\\
2.46666666666664	325.267532467532\\
2.48333333333335	325.273032849503\\
2.5	325.217112299465\\
2.51666666666665	325.284339190222\\
2.53333333333336	325.117799847212\\
2.55000000000001	324.995874713522\\
2.56666666666666	324.674407944996\\
2.58333333333331	324.495951107716\\
2.60000000000002	324.423834988541\\
2.61666666666667	324.050725744843\\
2.64999999999998	323.825821237586\\
2.66666666666669	323.722841864018\\
2.69999999999999	323.821848739496\\
2.71666666666664	324.071810542399\\
2.73333333333335	324.276546982429\\
2.75	324.410084033613\\
2.78333333333336	324.83422459893\\
2.81666666666666	325.414514896868\\
2.83333333333331	325.397402597403\\
2.85000000000002	325.331703590527\\
2.86666666666667	325.476241405653\\
2.88333333333333	325.560275019099\\
2.89999999999998	325.561191749427\\
2.91666666666669	325.552024446142\\
2.93333333333334	325.283422459893\\
2.94999999999999	325.209167303285\\
2.96666666666664	324.997097020626\\
3.01666666666665	324.069365928189\\
3.03333333333336	323.83987776929\\
3.05000000000001	323.650420168067\\
3.06666666666666	323.61191749427\\
3.08333333333331	323.593582887701\\
3.10000000000002	323.765011459129\\
3.11666666666667	324.053170359053\\
3.14999999999998	324.493812070283\\
3.16666666666669	324.577845683728\\
3.18333333333334	324.851336898396\\
3.19999999999999	325.020626432391\\
3.21666666666664	325.13307868602\\
3.23333333333335	325.29717341482\\
3.25	325.493659281895\\
3.26666666666665	325.324675324675\\
3.28333333333336	325.224446142093\\
3.30000000000001	325.220168067227\\
3.31666666666666	325.200916730329\\
3.35000000000002	324.900229182582\\
3.36666666666667	324.592513368984\\
3.38333333333333	324.210542398778\\
3.39999999999998	324.235294117647\\
3.41666666666669	323.95935828877\\
3.43333333333334	323.87410236822\\
3.44999999999999	323.623529411765\\
3.46666666666664	323.702062643239\\
3.48333333333335	323.716424751719\\
3.5	323.831932773109\\
3.51666666666665	323.993582887701\\
3.53333333333336	324.063559969442\\
3.55000000000001	324.391749427044\\
3.56666666666666	324.638655462185\\
3.58333333333331	324.829640947288\\
3.60000000000002	325.088770053476\\
3.61666666666667	325.220168067227\\
3.63333333333333	325.466462948816\\
3.64999999999998	325.367761650115\\
3.66666666666669	325.391596638655\\
3.68333333333334	325.390679908327\\
3.69999999999999	325.381207028266\\
3.71666666666664	325.115049656226\\
3.73333333333335	324.997097020626\\
3.75	324.992513368984\\
3.76666666666665	324.737967914439\\
3.78333333333336	324.421695951108\\
3.83333333333331	323.690756302521\\
3.85000000000002	323.607333842628\\
3.86666666666667	323.765928189458\\
3.88333333333333	323.810847975554\\
3.91666666666669	324.235905271199\\
3.93333333333334	324.509702062643\\
3.94999999999999	324.613903743316\\
3.96666666666664	325.090603514133\\
3.98333333333335	325.292895339954\\
4	325.428265851795\\
4.01666666666665	325.643086325439\\
4.03333333333336	325.676088617265\\
4.05000000000001	325.535523300229\\
4.06666666666666	325.455156608098\\
4.08333333333331	325.423376623377\\
4.10000000000002	325.407792207792\\
4.13333333333333	325.089075630252\\
4.14999999999998	324.829029793736\\
4.18333333333334	324.19923605806\\
4.19999999999999	324.201680672269\\
4.21666666666664	323.987776928953\\
4.23333333333335	323.74025974026\\
4.25	323.722536287242\\
4.26666666666665	323.730175706646\\
4.28333333333336	323.801986249045\\
4.30000000000001	323.845072574484\\
4.31666666666666	324.187318563789\\
4.33333333333331	324.176623376623\\
4.36666666666667	324.949732620321\\
4.39999999999998	325.246447669977\\
4.41666666666669	325.445378151261\\
4.43333333333334	325.296867838044\\
4.44999999999999	325.184721161192\\
4.46666666666664	325.36256684492\\
4.48333333333335	325.353705118411\\
4.5	325.158135981665\\
4.51666666666665	324.89320091673\\
4.53333333333336	324.838197097021\\
4.55000000000001	324.756608097785\\
4.56666666666666	324.414667685256\\
4.58333333333331	324.183957219251\\
4.60000000000002	323.996944232238\\
4.61666666666667	323.667226890756\\
4.63333333333333	323.540412528648\\
4.64999999999998	323.534912146677\\
4.66666666666669	323.7179526356\\
4.68333333333334	323.961497326203\\
4.69999999999999	324.271963330787\\
4.73333333333335	324.692131398014\\
4.75	325.129717341482\\
4.76666666666665	325.394041252865\\
4.78333333333336	325.539801375095\\
4.80000000000001	325.515660809778\\
4.81666666666666	325.690450725745\\
4.83333333333331	325.702673796791\\
4.85000000000002	325.612223071047\\
4.86666666666667	325.32987012987\\
4.88333333333333	325.347288006112\\
4.89999999999998	325.179526355997\\
4.91666666666669	325.171581359817\\
4.98333333333335	324.08128342246\\
5	323.848739495798\\
5.01666666666665	323.760733384263\\
5.03333333333336	323.608861726509\\
5.05000000000001	323.637585943468\\
5.06666666666666	323.545301757066\\
5.08333333333331	323.816348357525\\
5.10000000000002	324.010084033613\\
5.14999999999998	324.835141329259\\
5.16666666666669	324.954316271963\\
5.18333333333334	325.199694423224\\
5.19999999999999	325.368067226891\\
5.21666666666664	325.457295645531\\
5.23333333333335	325.242169595111\\
5.25	325.293506493506\\
5.26666666666665	325.308174178762\\
5.28333333333336	324.997402597403\\
5.30000000000001	324.966844919786\\
5.31666666666666	324.838197097021\\
5.33333333333331	324.639266615737\\
5.35000000000002	324.390832696715\\
5.36666666666667	324.24385026738\\
5.38333333333333	324.194041252865\\
5.39999999999998	323.90282658518\\
5.41666666666669	323.802597402597\\
5.43333333333334	323.778151260504\\
5.44999999999999	323.886631016043\\
5.46666666666664	324.016195569137\\
5.48333333333335	324.354774637128\\
5.5	324.581818181818\\
5.53333333333336	325.395569136746\\
5.55000000000001	325.576776165011\\
5.56666666666666	325.708785332315\\
5.58333333333331	325.863101604278\\
5.60000000000002	325.915355233002\\
5.61666666666667	325.811764705882\\
5.63333333333333	325.905576776165\\
5.64999999999998	325.625974025974\\
5.66666666666669	325.52268907563\\
5.68333333333334	325.282811306341\\
5.69999999999999	325.231779984721\\
5.71666666666664	325.022765469824\\
5.73333333333335	324.731245225363\\
5.75	324.492589763178\\
5.76666666666665	324.150954927426\\
5.78333333333336	323.932773109244\\
5.80000000000001	323.766233766234\\
5.81666666666666	323.788235294118\\
5.83333333333331	323.564247517189\\
5.85000000000002	323.616806722689\\
5.86666666666667	323.726203208556\\
5.88333333333333	323.914438502674\\
5.89999999999998	324.162261268144\\
5.91666666666669	324.49717341482\\
5.93333333333334	324.752024446142\\
5.94999999999999	324.728800611154\\
5.98333333333335	325.264782276547\\
6	325.266310160428\\
6.01666666666665	325.245225362872\\
6.03333333333336	325.244308632544\\
6.05000000000001	325.269365928189\\
6.06666666666666	325.16577540107\\
6.08333333333331	324.885561497326\\
6.10000000000002	324.744385026738\\
6.11666666666667	324.450725744843\\
6.13333333333333	324.442780748663\\
6.14999999999998	324.374942704354\\
6.16666666666669	324.2884644767\\
6.18333333333334	324.057142857143\\
6.19999999999999	324.143926661574\\
6.21666666666664	323.970664629488\\
6.23333333333335	323.889381207028\\
6.26666666666665	324.454087089381\\
6.28333333333336	324.579067990833\\
6.30000000000001	325.071963330787\\
6.31666666666666	325.355233002292\\
6.33333333333331	325.592666157372\\
6.35000000000002	325.646142093201\\
6.36666666666667	325.932773109244\\
6.38333333333333	325.813903743316\\
6.39999999999998	325.860045836516\\
6.41666666666669	325.818487394958\\
6.43333333333334	325.90435446906\\
6.44999999999999	325.45179526356\\
6.46666666666664	325.452406417112\\
6.48333333333335	325.298090145149\\
6.5	325.04140565317\\
6.51666666666665	324.689686783804\\
6.53333333333336	324.591902215432\\
6.58333333333331	323.858823529412\\
6.60000000000002	323.883269671505\\
6.61666666666667	323.607333842628\\
6.63333333333333	323.709090909091\\
6.64999999999998	323.675171886937\\
6.66666666666669	323.873491214668\\
6.68333333333334	323.964553093965\\
6.69999999999999	324.311688311688\\
6.71666666666664	324.575401069519\\
6.73333333333335	324.76577540107\\
6.75	324.874560733384\\
6.76666666666665	325.053934300993\\
6.78333333333336	325.127883880825\\
6.80000000000001	325.107715813598\\
6.81666666666666	325.145912910619\\
6.83333333333331	324.985790679908\\
6.85000000000002	324.964094728801\\
6.86666666666667	324.921313980137\\
6.88333333333333	324.766997708174\\
6.89999999999998	324.406417112299\\
6.91666666666669	324.26371275783\\
6.93333333333334	324.46294881589\\
6.94999999999999	324.286936592819\\
6.96666666666664	324.274407944996\\
6.98333333333335	324.32910618793\\
7	324.336134453782\\
7.01666666666665	324.331856378915\\
7.03333333333336	324.408556149733\\
7.05000000000001	324.737967914439\\
7.06666666666666	324.878533231474\\
7.08333333333331	325.113521772345\\
7.11666666666667	325.696562261268\\
7.13333333333333	325.778762414057\\
7.14999999999998	325.944690603514\\
7.16666666666669	326.012834224599\\
7.18333333333334	326.027501909855\\
7.19999999999999	326.052864782277\\
7.23333333333335	325.712452253629\\
7.25	325.612834224599\\
7.28333333333336	325.073185637892\\
7.30000000000001	324.90756302521\\
7.31666666666666	324.892284186402\\
7.33333333333331	324.542093200917\\
7.35000000000002	324.306187929717\\
7.36666666666667	324.027501909855\\
7.39999999999998	323.754621848739\\
7.41666666666669	323.740565317036\\
7.43333333333334	323.757677616501\\
7.44999999999999	323.738426279603\\
7.46666666666664	323.794346829641\\
7.5	324.50756302521\\
7.51666666666665	324.508174178762\\
7.53333333333336	324.744385026738\\
7.55000000000001	324.856531703591\\
7.56666666666666	325.041100076394\\
7.58333333333331	325.005347593583\\
7.60000000000002	324.944537815126\\
7.61666666666667	324.960733384263\\
7.63333333333333	324.807028265852\\
7.64999999999998	324.919480519481\\
7.66666666666669	324.740718105424\\
7.68333333333334	324.486783804431\\
7.69999999999999	324.419251336898\\
7.71666666666664	324.423529411765\\
7.73333333333335	324.417112299465\\
7.75	324.483728036669\\
7.76666666666665	324.511229946524\\
7.78333333333336	324.473032849503\\
7.80000000000001	324.497784568373\\
7.83333333333331	324.926203208556\\
7.85000000000002	324.946065699007\\
7.88333333333333	325.225974025974\\
7.89999999999998	325.547135217723\\
7.91666666666669	325.707868601986\\
7.93333333333334	325.998472116119\\
7.94999999999999	326.066615737204\\
7.96666666666664	326.074560733384\\
8	325.783040488923\\
8.03333333333336	325.692284186402\\
8.05000000000001	325.337203972498\\
8.08333333333331	324.812834224599\\
8.10000000000002	324.806722689076\\
8.11666666666667	324.547899159664\\
8.13333333333333	324.519480519481\\
8.16666666666669	323.989304812834\\
8.18333333333334	323.908326967151\\
8.19999999999999	323.992666157372\\
8.21666666666664	323.820320855615\\
8.23333333333335	323.69320091673\\
8.25	323.779679144385\\
8.28333333333336	324.236516424752\\
8.30000000000001	324.369136745607\\
8.31666666666666	324.65576776165\\
8.33333333333331	324.891673032849\\
8.35000000000002	324.819251336898\\
8.36666666666667	324.906035141329\\
8.38333333333333	324.924980901451\\
8.39999999999998	324.906340718105\\
8.41666666666669	324.865699006876\\
8.43333333333334	324.782276546982\\
8.44999999999999	324.632543926662\\
8.46666666666664	324.550343773873\\
8.48333333333335	324.503896103896\\
8.5	324.476088617265\\
8.51666666666665	324.482200152788\\
8.53333333333336	324.652406417112\\
8.55000000000001	324.685714285714\\
8.56666666666666	324.770359052712\\
8.58333333333331	324.881894576012\\
8.60000000000002	325.01115355233\\
8.61666666666667	324.919480519481\\
8.63333333333333	324.976012223071\\
8.64999999999998	325.065851795264\\
8.66666666666669	325.430099312452\\
8.68333333333334	325.451184110008\\
8.69999999999999	325.75217723453\\
8.71666666666664	326.006111535523\\
8.73333333333335	326.010695187166\\
8.75	325.995110771581\\
8.76666666666665	325.845989304813\\
8.78333333333336	325.788540870894\\
8.80000000000001	325.576470588235\\
8.81666666666666	325.655309396486\\
8.85000000000002	325.082352941176\\
8.86666666666667	324.964705882353\\
8.88333333333333	324.694576012223\\
8.89999999999998	324.519480519481\\
8.91666666666669	324.495339954163\\
8.93333333333334	324.350802139037\\
8.96666666666664	323.975553857907\\
8.98333333333335	323.960580595875\\
9	323.730786860199\\
9.01666666666665	323.733537051184\\
9.03333333333336	323.615889992361\\
9.05000000000001	323.728647822765\\
9.06666666666666	323.976776165011\\
9.13333333333333	324.810695187166\\
9.14999999999998	324.783498854087\\
9.18333333333334	324.853781512605\\
9.19999999999999	324.750190985485\\
9.21666666666664	324.554621848739\\
9.23333333333335	324.498395721925\\
9.25	324.466615737204\\
9.26666666666665	324.559816653934\\
9.28333333333336	324.473949579832\\
9.30000000000001	324.599541634836\\
9.31666666666666	324.638349885409\\
9.33333333333331	324.884950343774\\
9.35000000000002	324.833307868602\\
9.36666666666667	324.925286478228\\
9.38333333333333	325.04537815126\\
9.39999999999998	325.063407181054\\
9.41666666666669	325.026126814362\\
9.43333333333334	325.113216195569\\
9.44999999999999	325.149274255157\\
9.46666666666664	325.31550802139\\
9.48333333333335	325.425210084034\\
9.5	325.699312452254\\
9.51666666666665	325.786096256685\\
9.53333333333336	326.031474407945\\
9.55000000000001	325.938273491215\\
9.58333333333331	325.621390374332\\
9.60000000000002	325.586249045073\\
9.61666666666667	325.349732620321\\
9.63333333333333	325.195110771581\\
9.64999999999998	324.855920550038\\
9.66666666666669	324.872421695951\\
9.68333333333334	324.677463712758\\
9.69999999999999	324.559511077158\\
9.73333333333335	324.222459893048\\
9.75	324.156455309397\\
9.76666666666665	323.84537815126\\
9.78333333333336	323.69717341482\\
9.80000000000001	323.478686019862\\
9.81666666666666	323.541329258976\\
9.83333333333331	323.655920550038\\
9.85000000000002	323.678838808251\\
9.86666666666667	323.997555385791\\
9.88333333333333	324.179984721161\\
9.89999999999998	324.302215431627\\
9.91666666666669	324.473949579832\\
9.93333333333334	324.608403361345\\
9.94999999999999	324.719022154316\\
9.96666666666664	324.715049656226\\
9.98333333333335	324.671046600458\\
10	324.716577540107\\
10.0166666666667	324.532009167303\\
10.0333333333334	324.687853323147\\
10.05	324.656684491979\\
10.0666666666667	324.720855614973\\
10.0833333333333	324.922230710466\\
10.1	324.940565317036\\
10.1166666666667	324.998013750955\\
10.15	325.079602750191\\
10.1666666666667	325.144690603514\\
10.1833333333333	325.148051948052\\
10.2	325.137662337662\\
10.2166666666666	325.192666157372\\
10.2333333333333	325.223529411765\\
10.2666666666667	325.535523300229\\
10.2833333333334	325.516883116883\\
10.3	325.700840336134\\
10.3333333333333	325.92910618793\\
10.35	325.657448433919\\
10.3666666666667	325.674255156608\\
10.3833333333333	325.589304812834\\
10.4	325.430404889228\\
10.4166666666667	325.339648586707\\
10.4333333333333	325.158747135218\\
10.45	324.825668449198\\
10.4666666666666	324.626126814362\\
10.4833333333333	324.576317799847\\
10.5	324.505423987777\\
10.5166666666667	324.259129106188\\
10.55	323.88449197861\\
10.5666666666667	323.584415584416\\
10.5833333333333	323.45179526356\\
10.6	323.394041252865\\
10.6166666666667	323.532773109244\\
10.6333333333333	323.623223834989\\
10.6666666666667	323.916577540107\\
10.6833333333333	324.014667685256\\
10.7	324.235599694423\\
10.7166666666666	324.304965622613\\
10.7666666666667	324.603819709702\\
10.7833333333334	324.51550802139\\
10.8	324.6294881589\\
10.8166666666667	324.717188693659\\
10.8333333333333	324.851336898396\\
10.85	324.927731092437\\
10.8666666666667	324.929564553094\\
10.8833333333333	324.990679908327\\
10.9	324.991291061879\\
10.9166666666667	325.075324675325\\
10.9333333333333	325.235141329259\\
10.95	325.227501909855\\
10.9666666666666	325.282811306341\\
10.9833333333333	325.027349121467\\
11	325.174637127578\\
11.0166666666667	325.136134453781\\
11.0333333333334	325.205500381971\\
11.05	325.220168067227\\
11.0666666666667	325.413903743316\\
11.0833333333333	325.485103132162\\
11.1	325.477769289534\\
11.1166666666667	325.501909854851\\
11.1333333333333	325.544079449962\\
11.15	325.606111535523\\
11.1666666666667	325.593277310924\\
11.2	325.126355996944\\
11.2166666666666	325.097326203209\\
11.2333333333333	324.856837280367\\
11.25	324.660351413293\\
11.2666666666667	324.536898395722\\
11.2833333333334	324.170817417876\\
11.3	324.04461420932\\
11.3166666666667	324.033919022154\\
11.3333333333333	323.765928189458\\
11.35	323.45179526356\\
11.3666666666667	323.422459893048\\
11.3833333333333	323.284033613445\\
11.4	323.296867838044\\
11.4166666666667	323.511382734912\\
11.4333333333333	323.545607333843\\
11.45	323.749732620321\\
11.4666666666666	323.913521772345\\
11.4833333333333	324.004278074866\\
11.5	324.267379679144\\
11.5333333333334	324.538731856379\\
11.55	324.53384262796\\
11.5666666666667	324.633766233766\\
11.6	325.026126814362\\
11.6166666666667	325.043239113827\\
11.6333333333333	325.237891520244\\
11.65	325.269060351413\\
11.6666666666667	325.091825821238\\
11.6833333333333	325.130022918258\\
11.7	325.185026737968\\
11.7166666666666	325.432543926662\\
11.7333333333333	325.336592818946\\
11.75	325.374789915966\\
11.7666666666667	325.272727272727\\
11.7833333333334	325.27486631016\\
11.8	325.191749427044\\
11.8166666666667	325.203361344538\\
11.8333333333333	325.298701298701\\
11.85	325.367761650115\\
11.8666666666667	325.312146676853\\
11.8833333333333	325.386096256684\\
11.9166666666667	325.698395721925\\
11.9333333333333	325.677616501146\\
11.95	325.501909854851\\
11.9666666666666	325.398930481283\\
11.9833333333333	325.261115355233\\
12	325.008708938121\\
};
\addlegendentry{$\text{\textless{}10 ms}$}


\addplot[area legend, draw=black, fill=mycolor2, fill opacity=0.5, forget plot]
table[row sep=crcr] {%
x	y\\
0	351.305941387644\\
0.0166666666666667	351.651296761405\\
0.0333333333333333	351.841068223791\\
0.05	352.08289924825\\
0.0666666666666667	352.300364372208\\
0.0833333333333333	352.528634029118\\
0.1	352.631617836143\\
0.116666666666667	352.996014400924\\
0.133333333333333	352.941592344684\\
0.15	353.040828549938\\
0.166666666666667	353.120679899343\\
0.183333333333333	353.080674588671\\
0.2	352.837590932671\\
0.216666666666667	352.383127893698\\
0.233333333333333	352.418543466881\\
0.25	352.255519659216\\
0.266666666666667	351.69368048046\\
0.283333333333333	351.433442907888\\
0.3	351.297735233955\\
0.316666666666667	351.340415398256\\
0.333333333333333	351.463597797304\\
0.35	351.605686216483\\
0.366666666666667	351.596992159547\\
0.383333333333333	351.718778380988\\
0.4	351.820605609744\\
0.416666666666667	352.007396769258\\
0.433333333333333	352.156988878667\\
0.45	352.367484380167\\
0.466666666666667	352.83034560702\\
0.483333333333333	353.003940197811\\
0.5	353.149068810536\\
0.516666666666667	353.260278351005\\
0.533333333333333	353.209579660922\\
0.55	353.288454102034\\
0.566666666666667	353.244121362767\\
0.583333333333333	353.03131617211\\
0.6	352.911879034916\\
0.616666666666667	352.433285061228\\
0.633333333333333	352.278091928449\\
0.65	351.960027336653\\
0.666666666666667	351.49242925905\\
0.683333333333333	351.384413475824\\
0.7	351.22991062553\\
0.716666666666667	351.027320224232\\
0.733333333333333	351.053365955023\\
0.75	350.953921288104\\
0.766666666666667	351.340255088425\\
0.783333333333333	351.459629706253\\
0.8	351.610015059925\\
0.816666666666667	351.968973191139\\
0.833333333333333	352.189597141916\\
0.85	352.549930280226\\
0.866666666666667	352.915343170101\\
0.883333333333333	353.133549470276\\
0.9	353.421957388694\\
0.916666666666667	353.425636268798\\
0.933333333333333	353.440827509212\\
0.95	353.716081067536\\
0.966666666666667	353.468412752088\\
0.983333333333333	353.336794209811\\
1	353.105631353951\\
1.01666666666667	352.918621365467\\
1.03333333333333	352.701539136294\\
1.05	352.415674351737\\
1.06666666666667	352.046126833996\\
1.08333333333333	351.883913467298\\
1.1	351.518046704088\\
1.11666666666667	351.725393196676\\
1.13333333333333	351.943252767059\\
1.15	351.810174292306\\
1.16666666666667	351.919355617848\\
1.18333333333333	352.151480323203\\
1.2	352.13870990724\\
1.21666666666667	352.349634821591\\
1.23333333333333	352.638681830517\\
1.25	352.963242094076\\
1.26666666666667	352.96669528018\\
1.28333333333333	353.07892135225\\
1.3	353.115115456362\\
1.31666666666667	353.281411353795\\
1.33333333333333	353.16282309605\\
1.35	353.434391985082\\
1.36666666666667	353.297296975586\\
1.38333333333333	352.966839027515\\
1.4	352.619135525084\\
1.41666666666667	352.149826185163\\
1.43333333333333	351.84647039247\\
1.45	351.465606975595\\
1.46666666666667	351.303514281082\\
1.48333333333333	351.017868440664\\
1.5	350.847700147412\\
1.51666666666667	350.888939392723\\
1.53333333333333	350.936872457715\\
1.55	351.126549915705\\
1.56666666666667	351.453341466079\\
1.58333333333333	351.607478914192\\
1.6	351.894187922374\\
1.61666666666667	352.190949783384\\
1.63333333333333	352.430764692959\\
1.65	352.832103267142\\
1.66666666666667	353.270989222373\\
1.68333333333333	353.415696128666\\
1.7	353.646906478434\\
1.71666666666667	353.511766893814\\
1.73333333333333	353.599173613533\\
1.75	353.516924382458\\
1.76666666666667	353.341505489068\\
1.78333333333333	353.170094500384\\
1.8	352.960362724213\\
1.81666666666667	352.677059900649\\
1.83333333333333	352.39531102019\\
1.85	352.20195825584\\
1.86666666666667	351.851133015897\\
1.88333333333333	351.669469937529\\
1.9	351.773274480858\\
1.91666666666667	351.764566538793\\
1.93333333333333	351.910221472169\\
1.95	352.034499121303\\
1.96666666666667	352.327451278835\\
1.98333333333333	352.517809339052\\
2	352.406551243807\\
2.01666666666667	352.651280247669\\
2.03333333333333	352.775893344057\\
2.05	352.966057550831\\
2.06666666666667	352.866925377839\\
2.08333333333333	352.989142212722\\
2.1	353.242018753353\\
2.11666666666667	353.287631844035\\
2.13333333333333	353.396252532677\\
2.15	353.07576112612\\
2.16666666666667	352.835640721918\\
2.18333333333333	352.485472921524\\
2.2	352.241995071582\\
2.21666666666667	351.692722198944\\
2.23333333333333	351.575064414764\\
2.25	351.278387444325\\
2.26666666666667	350.901968900088\\
2.28333333333333	351.006464881946\\
2.3	350.89033268442\\
2.31666666666667	351.110253128089\\
2.33333333333333	351.177738419272\\
2.35	351.451008916014\\
2.36666666666667	352.007481462207\\
2.38333333333333	352.331631006066\\
2.4	352.657520418236\\
2.41666666666667	353.22234788088\\
2.43333333333333	353.477784789427\\
2.45	353.699219872696\\
2.46666666666667	353.899217971468\\
2.48333333333333	353.889804221991\\
2.5	354.198264492468\\
2.51666666666667	353.980543087758\\
2.53333333333333	353.966547433127\\
2.55	353.793786499166\\
2.56666666666667	353.568799883195\\
2.58333333333333	353.448831505936\\
2.6	353.163599468186\\
2.61666666666667	352.758324452599\\
2.63333333333333	352.528491711932\\
2.65	352.357246215589\\
2.66666666666667	352.076412263564\\
2.68333333333333	351.915288367956\\
2.7	351.992113437423\\
2.71666666666667	352.078225255206\\
2.73333333333333	352.248417388451\\
2.75	352.405495195713\\
2.76666666666667	352.599598486682\\
2.78333333333333	352.779686484248\\
2.8	352.771274062186\\
2.81666666666667	353.096706541069\\
2.83333333333333	352.924181187355\\
2.85	352.976303500342\\
2.86666666666667	353.086409892117\\
2.88333333333333	353.022949563022\\
2.9	353.229759746388\\
2.91666666666667	353.146715923437\\
2.93333333333333	352.938341390104\\
2.95	352.624617587234\\
2.96666666666667	352.356216762522\\
2.98333333333333	352.143462995232\\
3	351.761700819118\\
3.01666666666667	351.327650537571\\
3.03333333333333	350.954518901463\\
3.05	350.816480552565\\
3.06666666666667	350.739886179943\\
3.08333333333333	350.806944505508\\
3.1	350.836204341317\\
3.11666666666667	351.158931159722\\
3.13333333333333	351.363745434169\\
3.15	351.814431589551\\
3.16666666666667	352.389562558017\\
3.18333333333333	352.897700430338\\
3.2	353.210801470955\\
3.21666666666667	353.602111248723\\
3.23333333333333	353.87819482026\\
3.25	354.051891052273\\
3.26666666666667	353.96484898591\\
3.28333333333333	354.014212258324\\
3.3	354.235987696721\\
3.31666666666667	353.944706809774\\
3.33333333333333	353.672521756157\\
3.35	353.662815813156\\
3.36666666666667	353.484044497656\\
3.38333333333333	353.288590391746\\
3.4	352.907047621088\\
3.41666666666667	352.472767761328\\
3.43333333333333	352.316790741297\\
3.45	352.019818672844\\
3.46666666666667	351.89560696947\\
3.48333333333333	351.9514378991\\
3.5	351.850563757228\\
3.51666666666667	351.949131205603\\
3.53333333333333	352.121668500908\\
3.55	352.509687181896\\
3.56666666666667	352.582590781078\\
3.58333333333333	352.597836179137\\
3.6	352.524962988065\\
3.61666666666667	352.776167684668\\
3.63333333333333	352.89789722778\\
3.65	352.911309447068\\
3.66666666666667	352.835657372277\\
3.68333333333333	352.853875049033\\
3.7	352.908623277968\\
3.71666666666667	352.543204282813\\
3.73333333333333	352.362597983895\\
3.75	352.118903373187\\
3.76666666666667	351.954725934583\\
3.78333333333333	351.678290683739\\
3.8	351.28093655024\\
3.81666666666667	351.010956530297\\
3.83333333333333	350.716459264156\\
3.85	350.638263101719\\
3.86666666666667	350.638054886692\\
3.88333333333333	350.82025820009\\
3.9	351.069299297617\\
3.91666666666667	351.610700697432\\
3.93333333333333	352.267947979294\\
3.95	352.824381068525\\
3.96666666666667	353.388552658949\\
3.98333333333333	353.773300554979\\
4	354.029677938954\\
4.01666666666667	354.415123314607\\
4.03333333333333	354.526782792537\\
4.05	354.519148333353\\
4.06666666666667	354.577723040435\\
4.08333333333333	354.436080644277\\
4.1	354.494172515876\\
4.11666666666667	354.288316656621\\
4.13333333333333	353.987709848613\\
4.15	353.973175571702\\
4.16666666666667	353.783919379776\\
4.18333333333333	353.355511842494\\
4.2	352.947802392281\\
4.21666666666667	352.527180818988\\
4.23333333333333	352.22783609247\\
4.25	352.114627840658\\
4.26666666666667	352.021171170507\\
4.28333333333333	351.889302991035\\
4.3	352.041529720098\\
4.31666666666667	352.242093749085\\
4.33333333333333	352.291810169567\\
4.35	352.661941317093\\
4.36666666666667	352.860183486705\\
4.38333333333333	352.935273832499\\
4.4	353.019689204143\\
4.41666666666667	352.929688176498\\
4.43333333333333	353.012771361983\\
4.45	352.779664042396\\
4.46666666666667	352.672141529487\\
4.48333333333333	352.606106295764\\
4.5	352.36396248649\\
4.51666666666667	352.139189778179\\
4.53333333333333	351.7884131089\\
4.55	351.652944414486\\
4.56666666666667	351.477356435365\\
4.58333333333333	351.219140287722\\
4.6	350.903778488306\\
4.61666666666667	350.73679653497\\
4.63333333333333	350.668690200468\\
4.65	350.67074037694\\
4.66666666666667	350.888322193123\\
4.68333333333333	351.229086789518\\
4.7	351.688614902039\\
4.71666666666667	352.121088098049\\
4.73333333333333	352.679301843532\\
4.75	353.322473225832\\
4.76666666666667	353.59291178773\\
4.78333333333333	354.140872492389\\
4.8	354.412240905868\\
4.81666666666667	354.770384544024\\
4.83333333333333	354.60443546029\\
4.85	354.57898238467\\
4.86666666666667	354.502461827769\\
4.88333333333333	354.353257866525\\
4.9	354.27489922568\\
4.91666666666667	354.322985373818\\
4.93333333333333	354.049875207577\\
4.95	353.756787521914\\
4.96666666666667	353.550135555575\\
4.98333333333333	353.006822239098\\
5	352.652914106836\\
5.01666666666667	352.233870905737\\
5.03333333333333	352.273184163518\\
5.05	352.111015294753\\
5.06666666666667	351.898038186745\\
5.08333333333333	351.879244699776\\
5.1	352.136752042467\\
5.11666666666667	352.336694090493\\
5.13333333333333	352.622326524814\\
5.15	352.830150634089\\
5.16666666666667	352.994634666049\\
5.18333333333333	352.983735909054\\
5.2	353.014844759947\\
5.21666666666667	353.147020046399\\
5.23333333333333	352.787283765548\\
5.25	352.623832772171\\
5.26666666666667	352.501786281614\\
5.28333333333333	352.301340760974\\
5.3	351.962699121281\\
5.31666666666667	351.803643149776\\
5.33333333333333	351.544683784004\\
5.35	351.455032598055\\
5.36666666666667	351.314401383475\\
5.38333333333333	351.25906690988\\
5.4	351.172028498909\\
5.41666666666667	351.050034736125\\
5.43333333333333	351.121249216014\\
5.45	351.266796196435\\
5.46666666666667	351.74627763942\\
5.48333333333333	352.263335254869\\
5.5	352.772985769296\\
5.51666666666667	353.379481574087\\
5.53333333333333	353.93247341533\\
5.55	354.385680020606\\
5.56666666666667	354.587929256301\\
5.58333333333333	354.892767040086\\
5.6	355.016339779716\\
5.61666666666667	354.990529931508\\
5.63333333333333	354.853142190769\\
5.65	354.530720768626\\
5.66666666666667	354.625692192938\\
5.68333333333333	354.43144496206\\
5.7	354.427413902847\\
5.71666666666667	354.21403621656\\
5.73333333333333	354.006800645567\\
5.75	353.583022285161\\
5.76666666666667	353.310197488718\\
5.78333333333333	352.863396177992\\
5.8	352.569296601673\\
5.81666666666667	352.550667864466\\
5.83333333333333	352.286057380711\\
5.85	352.334358633045\\
5.86666666666667	352.090324841124\\
5.88333333333333	352.11694218468\\
5.9	352.434760598163\\
5.91666666666667	352.660695812371\\
5.93333333333333	352.756827920032\\
5.95	352.799715260465\\
5.96666666666667	353.019284489439\\
5.98333333333333	352.913987532415\\
6	352.749419304698\\
6.01666666666667	352.575433036222\\
6.03333333333333	352.568698161519\\
6.05	352.448636190711\\
6.06666666666667	352.209980439015\\
6.08333333333333	351.856349893968\\
6.1	351.416884445379\\
6.11666666666667	351.187528868764\\
6.13333333333333	351.235013484425\\
6.15	351.150994872365\\
6.16666666666667	351.018405742043\\
6.18333333333333	350.915386896892\\
6.2	351.25170164992\\
6.21666666666667	351.239333367794\\
6.23333333333333	351.337068964188\\
6.25	351.908394181011\\
6.26666666666667	352.265835803797\\
6.28333333333333	353.020161578335\\
6.3	353.552058186471\\
6.31666666666667	353.881683315751\\
6.33333333333333	354.224797555667\\
6.35	354.595287262575\\
6.36666666666667	354.933515418938\\
6.38333333333333	355.020875013579\\
6.4	354.829631390757\\
6.41666666666667	354.592501092079\\
6.43333333333333	354.663381405092\\
6.45	354.432137237672\\
6.46666666666667	354.436013731581\\
6.48333333333333	354.217493424286\\
6.5	354.120857145837\\
6.51666666666667	353.906122714089\\
6.53333333333333	353.689995185867\\
6.55	353.524201491007\\
6.56666666666667	353.143238499514\\
6.58333333333333	352.844438884146\\
6.6	352.555870534404\\
6.61666666666667	352.500618484643\\
6.63333333333333	352.362276103499\\
6.65	352.187943654864\\
6.66666666666667	352.070825609111\\
6.68333333333333	352.225995700713\\
6.7	352.48593185414\\
6.71666666666667	352.675373950512\\
6.73333333333333	352.720942652906\\
6.75	352.74712348218\\
6.76666666666667	352.410229390326\\
6.78333333333333	352.429966610715\\
6.8	352.140527090507\\
6.81666666666667	352.006267777322\\
6.83333333333333	352.00236914441\\
6.85	351.651961209175\\
6.86666666666667	351.497947450769\\
6.88333333333333	351.17323750185\\
6.9	351.049307389303\\
6.91666666666667	351.136070268001\\
6.93333333333333	351.15120171672\\
6.95	351.228270281432\\
6.96666666666667	351.355739248327\\
6.98333333333333	351.576170887457\\
7	351.634201389845\\
7.01666666666667	351.937617063111\\
7.03333333333333	352.698718203067\\
7.05	352.911688069457\\
7.06666666666667	353.293561240376\\
7.08333333333333	353.971462640179\\
7.1	354.234611882691\\
7.11666666666667	354.653033096366\\
7.13333333333333	354.929498669844\\
7.15	355.146990462086\\
7.16666666666667	355.185593570342\\
7.18333333333333	355.098455894444\\
7.2	354.95424080377\\
7.21666666666667	354.728685801108\\
7.23333333333333	354.439070079155\\
7.25	354.321792300101\\
7.26666666666667	354.208500107206\\
7.28333333333333	354.089359124836\\
7.3	353.955143204004\\
7.31666666666667	353.913370010429\\
7.33333333333333	353.670161184435\\
7.35	353.310859822241\\
7.36666666666667	353.172445965557\\
7.38333333333333	352.816784150746\\
7.4	352.782540362936\\
7.41666666666667	352.527012775325\\
7.43333333333333	352.584761898817\\
7.45	352.374542038213\\
7.46666666666667	352.376353644395\\
7.48333333333333	352.394982252651\\
7.5	352.424062304561\\
7.51666666666667	352.206637479567\\
7.53333333333333	352.376524613503\\
7.55	352.337068306757\\
7.56666666666667	352.141609088552\\
7.58333333333333	352.052177829886\\
7.6	352.006732820232\\
7.61666666666667	351.911839672516\\
7.63333333333333	351.642725692578\\
7.65	351.432828690488\\
7.66666666666667	351.18666423862\\
7.68333333333333	350.973964478801\\
7.7	351.124133667054\\
7.71666666666667	351.245385405558\\
7.73333333333333	351.43803564113\\
7.75	351.60633768589\\
7.76666666666667	351.851514792084\\
7.78333333333333	352.059826061322\\
7.8	352.342140849888\\
7.81666666666667	352.999192452373\\
7.83333333333333	353.346310224079\\
7.85	353.611593689682\\
7.86666666666667	353.916429357078\\
7.88333333333333	354.193602025203\\
7.9	354.421888714524\\
7.91666666666667	354.809309400089\\
7.93333333333333	355.162386257108\\
7.95	355.128249344905\\
7.96666666666667	355.050840611995\\
7.98333333333333	354.88204912574\\
8	354.558589275797\\
8.01666666666667	354.432142951592\\
8.03333333333333	354.291541890987\\
8.05	353.99807512751\\
8.06666666666667	353.869796145736\\
8.08333333333333	353.739943641992\\
8.1	353.734228206138\\
8.11666666666667	353.636483885292\\
8.13333333333333	353.40280369956\\
8.15	353.301840019685\\
8.16666666666667	353.113617762994\\
8.18333333333333	353.119186697243\\
8.2	352.728100599393\\
8.21666666666667	352.62731955273\\
8.23333333333333	352.517961409784\\
8.25	352.382988335759\\
8.26666666666667	352.130517339935\\
8.28333333333333	351.96846595368\\
8.3	351.942029705225\\
8.31666666666667	352.110903643513\\
8.33333333333333	352.205802992965\\
8.35	352.089877263718\\
8.36666666666667	352.04947851036\\
8.38333333333333	351.94067745791\\
8.4	351.59829276052\\
8.41666666666667	351.593940690699\\
8.43333333333333	351.429150382299\\
8.45	351.22730106057\\
8.46666666666667	351.196509039471\\
8.48333333333333	351.371844781647\\
8.5	351.522304593185\\
8.51666666666667	351.681691695857\\
8.53333333333333	352.061548065526\\
8.55	352.379639481972\\
8.56666666666667	352.660089374213\\
8.58333333333333	353.119907913107\\
8.6	353.490084727018\\
8.61666666666667	353.511941428797\\
8.63333333333333	353.871115988597\\
8.65	354.045041228549\\
8.66666666666667	354.302865571361\\
8.68333333333333	354.519449641734\\
8.7	354.842970982473\\
8.71666666666667	355.18799205086\\
8.73333333333333	355.05586712406\\
8.75	354.875236282737\\
8.76666666666667	354.646262710627\\
8.78333333333333	354.46581785835\\
8.8	354.187343401988\\
8.81666666666667	354.12888926511\\
8.83333333333333	353.805735098791\\
8.85	353.730782977318\\
8.86666666666667	353.771878031136\\
8.88333333333333	353.544857557623\\
8.9	353.534251520756\\
8.91666666666667	353.39480106317\\
8.93333333333333	353.375035468687\\
8.95	353.123995339441\\
8.96666666666667	353.130199879668\\
8.98333333333333	352.763763587711\\
9	352.60637334056\\
9.01666666666667	352.396433891144\\
9.03333333333333	352.069647768089\\
9.05	351.817149235489\\
9.06666666666667	351.730493076497\\
9.08333333333333	351.704030322456\\
9.1	351.722783819491\\
9.11666666666667	351.784418181188\\
9.13333333333333	351.768878536759\\
9.15	351.643302795904\\
9.16666666666667	351.627888747434\\
9.18333333333333	351.55209342003\\
9.2	351.198008506486\\
9.21666666666667	351.236855109815\\
9.23333333333333	351.128687503553\\
9.25	351.219460361568\\
9.26666666666667	351.551599298105\\
9.28333333333333	351.575214278588\\
9.3	352.084057527624\\
9.31666666666667	352.343675544145\\
9.33333333333333	352.643690633285\\
9.35	352.930880400653\\
9.36666666666667	353.229485194489\\
9.38333333333333	353.560767225659\\
9.4	353.527883887166\\
9.41666666666667	353.734095740107\\
9.43333333333333	354.067960692611\\
9.45	353.961445437899\\
9.46666666666667	354.176839904923\\
9.48333333333333	354.480362436818\\
9.5	354.550741245122\\
9.51666666666667	354.62725137596\\
9.53333333333333	354.64537677508\\
9.55	354.433719841337\\
9.56666666666667	354.2612570452\\
9.58333333333333	354.140691307054\\
9.6	353.983113110979\\
9.61666666666667	353.883419968617\\
9.63333333333333	353.621982670233\\
9.65	353.521545371716\\
9.66666666666667	353.565571573497\\
9.68333333333333	353.398320494902\\
9.7	353.361251835298\\
9.71666666666667	353.256486090885\\
9.73333333333333	353.246774352881\\
9.75	352.966762888629\\
9.76666666666667	352.904340496578\\
9.78333333333333	352.434748494047\\
9.8	352.021187816054\\
9.81666666666667	351.770089483742\\
9.83333333333333	351.564240557849\\
9.85	351.361684064688\\
9.86666666666667	351.470463843154\\
9.88333333333333	351.420510454625\\
9.9	351.393449858409\\
9.91666666666667	351.183170730658\\
9.93333333333333	351.20313446667\\
9.95	351.396490048423\\
9.96666666666667	351.510573011436\\
9.98333333333333	351.243720667608\\
10	351.437203780617\\
10.0166666666667	351.487127421953\\
10.0333333333333	351.524242456863\\
10.05	352.120442632956\\
10.0666666666667	352.452886502941\\
10.0833333333333	352.551559583438\\
10.1	353.089970667294\\
10.1166666666667	353.230449794427\\
10.1333333333333	353.603987781264\\
10.15	353.799917527134\\
10.1666666666667	353.92724561552\\
10.1833333333333	354.204642145439\\
10.2	354.21046181634\\
10.2166666666667	354.203605693933\\
10.2333333333333	354.106239034465\\
10.25	354.409844336031\\
10.2666666666667	354.417388277707\\
10.2833333333333	354.513213602623\\
10.3	354.537669403129\\
10.3166666666667	354.552443375313\\
10.3333333333333	354.26868828799\\
10.35	354.079067839034\\
10.3666666666667	354.192927727133\\
10.3833333333333	354.166617937872\\
10.4	353.948407367247\\
10.4166666666667	353.969180396546\\
10.4333333333333	353.684568287904\\
10.45	353.6506386682\\
10.4666666666667	353.543454192204\\
10.4833333333333	353.452917998582\\
10.5	353.549920110647\\
10.5166666666667	353.181019217294\\
10.5333333333333	352.911483233567\\
10.55	352.439531146944\\
10.5666666666667	351.980463963021\\
10.5833333333333	351.753077531459\\
10.6	351.422209314587\\
10.6166666666667	351.31814196488\\
10.6333333333333	351.202845786447\\
10.65	351.244807854371\\
10.6666666666667	351.134202821465\\
10.6833333333333	350.961616500243\\
10.7	350.875550004237\\
10.7166666666667	351.015153743972\\
10.7333333333333	351.13474943212\\
10.75	351.182538165424\\
10.7666666666667	351.410925395469\\
10.7833333333333	351.429497271596\\
10.8	351.659510679597\\
10.8166666666667	352.294262310394\\
10.8333333333333	352.534047298987\\
10.85	352.968401308029\\
10.8666666666667	353.147837228267\\
10.8833333333333	353.119586625896\\
10.9	353.596236860685\\
10.9166666666667	353.696548067483\\
10.9333333333333	354.021367066402\\
10.95	354.180800424306\\
10.9666666666667	354.154879331524\\
10.9833333333333	354.170854098432\\
11	354.171738761224\\
11.0166666666667	354.132896761202\\
11.0333333333333	354.121361384982\\
11.05	354.170871111848\\
11.0666666666667	354.229721458892\\
11.0833333333333	354.153375219954\\
11.1	354.107591803607\\
11.1166666666667	353.915347342956\\
11.1333333333333	353.823917275011\\
11.15	353.896367264608\\
11.1666666666667	354.046648018817\\
11.1833333333333	353.829370380924\\
11.2	353.599156266333\\
11.2166666666667	353.763129922003\\
11.2333333333333	353.326000231927\\
11.25	353.191296194334\\
11.2666666666667	353.15731083024\\
11.2833333333333	352.911675774687\\
11.3	352.837699044501\\
11.3166666666667	352.763115711592\\
11.3333333333333	352.117860918031\\
11.35	351.824866599839\\
11.3666666666667	351.402331686639\\
11.3833333333333	351.283081523311\\
11.4	351.10666393302\\
11.4166666666667	351.077055615723\\
11.4333333333333	350.971115242491\\
11.45	350.84245036452\\
11.4666666666667	350.866976496353\\
11.4833333333333	350.826288454686\\
11.5	351.005021088136\\
11.5166666666667	351.315720987015\\
11.5333333333333	351.750751952538\\
11.55	351.920809494445\\
11.5666666666667	351.958396864055\\
11.5833333333333	352.391900078963\\
11.6	352.942011006491\\
11.6166666666667	353.352339278186\\
11.6333333333333	353.601595351094\\
11.65	353.621178389034\\
11.6666666666667	353.772494582833\\
11.6833333333333	353.90887989369\\
11.7	354.123864660822\\
11.7166666666667	354.228755893401\\
11.7333333333333	354.42105163343\\
11.75	354.582924226331\\
11.7666666666667	354.566439896534\\
11.7833333333333	354.445808995482\\
11.8	354.218511278597\\
11.8166666666667	354.402791383637\\
11.8333333333333	354.166396649334\\
11.85	354.17635457406\\
11.8666666666667	353.892106452106\\
11.8833333333333	354.022279283958\\
11.9	353.873159622257\\
11.9166666666667	353.788265621059\\
11.9333333333333	353.995843350451\\
11.95	354.097201459576\\
11.9666666666667	354.050298463225\\
11.9833333333333	353.831650323863\\
12	353.603111008811\\
12	494.018584942297\\
11.9833333333333	494.297914229231\\
11.9666666666667	494.516852033337\\
11.95	494.668573941493\\
11.9333333333333	494.705149774071\\
11.9166666666667	494.536256915228\\
11.9	494.385969483931\\
11.8833333333333	494.604764260732\\
11.8666666666667	494.246625404273\\
11.85	494.21967292785\\
11.8333333333333	493.868439103149\\
11.8166666666667	493.85083734058\\
11.8	493.566973824535\\
11.7833333333333	493.641585962501\\
11.7666666666667	493.558846581693\\
11.75	493.665815269467\\
11.7333333333333	493.541667999878\\
11.7166666666667	493.594620729976\\
11.7	493.533889349873\\
11.6833333333333	493.564916897754\\
11.6666666666667	493.805809466059\\
11.65	494.043298310737\\
11.6333333333333	494.461964618349\\
11.6166666666667	494.813894488048\\
11.6	494.860586396106\\
11.5833333333333	495.016503282382\\
11.5666666666667	495.100426665356\\
11.55	495.228464760711\\
11.5333333333333	495.272624670838\\
11.5166666666667	495.256624314742\\
11.5	495.117515199106\\
11.4833333333333	494.933222622472\\
11.4666666666667	494.365109065145\\
11.45	494.003997305457\\
11.4333333333333	493.593590639862\\
11.4166666666667	493.433257600473\\
11.4	493.007010627714\\
11.3833333333333	492.83242649808\\
11.3666666666667	492.404543790825\\
11.35	492.534491688931\\
11.3333333333333	492.195049700762\\
11.3166666666667	492.687151668088\\
11.3	492.651071008975\\
11.2833333333333	492.7372165095\\
11.2666666666667	492.841772439431\\
11.25	493.26737454669\\
11.2333333333333	493.468804962878\\
11.2166666666667	494.010132110083\\
11.2	494.204663443369\\
11.1833333333333	494.512875608381\\
11.1666666666667	494.741740063689\\
11.15	494.758789343489\\
11.1333333333333	494.613057514905\\
11.1166666666667	494.394507508075\\
11.1	494.413111022978\\
11.0833333333333	494.199871533293\\
11.0666666666667	493.795335836753\\
11.05	493.86274233353\\
11.0333333333333	493.742351372848\\
11.0166666666667	493.557248387766\\
11	493.376618763604\\
10.9833333333333	493.416006100193\\
10.9666666666667	493.292637857169\\
10.95	493.516678567291\\
10.9333333333333	493.580771971031\\
10.9166666666667	493.583513047873\\
10.9	493.696047325717\\
10.8833333333333	493.92456921826\\
10.8666666666667	494.248954215583\\
10.85	494.612958508625\\
10.8333333333333	494.870689141044\\
10.8166666666667	495.186257170126\\
10.8	495.153629121778\\
10.7833333333333	495.37325291939\\
10.7666666666667	495.480442060604\\
10.75	495.322580245577\\
10.7333333333333	495.490155075137\\
10.7166666666667	495.302951680016\\
10.7	494.937742585526\\
10.6833333333333	494.539987777832\\
10.6666666666667	494.041656613218\\
10.65	493.673144781228\\
10.6333333333333	493.331302418289\\
10.6166666666667	492.951376751698\\
10.6	492.690854092594\\
10.5833333333333	492.4461585266\\
10.5666666666667	492.238329008713\\
10.55	492.32991117544\\
10.5333333333333	492.444055345501\\
10.5166666666667	492.511264969108\\
10.5	492.820744518842\\
10.4833333333333	492.906745866964\\
10.4666666666667	493.120564142403\\
10.45	493.456465991846\\
10.4333333333333	493.847288091011\\
10.4166666666667	494.274669870833\\
10.4	494.421340531912\\
10.3833333333333	494.680440885657\\
10.3666666666667	494.772694885549\\
10.35	494.599312604053\\
10.3333333333333	494.737270459146\\
10.3166666666667	494.591024921097\\
10.3	494.504958557146\\
10.2833333333333	494.374792508913\\
10.2666666666667	494.015308437342\\
10.25	493.924456657094\\
10.2333333333333	493.648688391051\\
10.2166666666667	493.49570675832\\
10.2	493.397177603064\\
10.1833333333333	493.432332644477\\
10.1666666666667	493.234862864235\\
10.15	493.40374939418\\
10.1333333333333	493.677295641196\\
10.1166666666667	493.807594514206\\
10.1	494.15036546716\\
10.0833333333333	494.279914824507\\
10.0666666666667	494.76055887521\\
10.05	495.053888917846\\
10.0333333333333	495.145887413267\\
10.0166666666667	495.334568529155\\
10	495.695569328627\\
9.98333333333333	495.719151754087\\
9.96666666666667	495.740152733408\\
9.95	495.481432029499\\
9.93333333333333	495.502442309495\\
9.91666666666667	495.042650506928\\
9.9	494.877596742049\\
9.88333333333333	494.425784426964\\
9.86666666666667	493.741453651117\\
9.85	493.345420595358\\
9.83333333333333	493.213146760714\\
9.81666666666667	492.813562158733\\
9.8	492.429232352013\\
9.78333333333333	492.316359221767\\
9.76666666666667	492.388860420153\\
9.75	492.535452542998\\
9.73333333333333	492.696693943528\\
9.71666666666667	492.74076371813\\
9.7	493.01735779037\\
9.68333333333333	493.08357408111\\
9.66666666666667	493.32396242192\\
9.65	493.822686866634\\
9.63333333333333	494.039438261776\\
9.61666666666667	494.280369183407\\
9.6	494.463028982222\\
9.58333333333333	494.710645591342\\
9.56666666666667	494.893823168704\\
9.55	494.836868393957\\
9.53333333333333	494.79450099421\\
9.51666666666667	494.620724177898\\
9.5	494.502887479095\\
9.48333333333333	494.228575683885\\
9.46666666666667	493.995505396834\\
9.45	493.820372743919\\
9.43333333333333	493.836699353225\\
9.41666666666667	493.529464229336\\
9.4	493.430710459664\\
9.38333333333333	493.563449733852\\
9.36666666666667	493.504204645083\\
9.35	493.754528308285\\
9.33333333333333	494.110167273514\\
9.31666666666667	494.445018115137\\
9.3	494.749250340978\\
9.28333333333333	494.934182207279\\
9.26666666666667	495.300043177067\\
9.25	495.327216490991\\
9.23333333333333	495.449158180175\\
9.21666666666667	495.777965363829\\
9.2	495.751418537059\\
9.18333333333333	495.87097762657\\
9.16666666666667	495.773791924835\\
9.15	495.380073827473\\
9.13333333333333	495.219051180583\\
9.11666666666667	494.83804171186\\
9.1	494.424809763396\\
9.08333333333333	493.974808485794\\
9.06666666666667	493.343303714947\\
9.05	492.95657116176\\
9.03333333333333	492.570841154752\\
9.01666666666667	492.240999263936\\
9	492.276743542557\\
8.98333333333333	492.31736704636\\
8.96666666666667	492.335346338819\\
8.95	492.538953476449\\
8.93333333333333	492.755445814735\\
8.91666666666667	492.981363948289\\
8.9	493.176214483828\\
8.88333333333333	493.394179875532\\
8.86666666666667	493.492293091859\\
8.85	493.860966449725\\
8.83333333333333	494.068978422981\\
8.81666666666667	494.567214630993\\
8.8	494.629768897477\\
8.78333333333333	494.873830728357\\
8.76666666666667	494.89506654835\\
8.75	494.877552105345\\
8.73333333333333	494.876600408407\\
8.71666666666667	494.804979683289\\
8.7	494.392781500338\\
8.68333333333333	494.231046920528\\
8.66666666666667	493.911649325507\\
8.65	493.737999260374\\
8.63333333333333	493.701687678324\\
8.61666666666667	493.414872933311\\
8.6	493.513123829132\\
8.58333333333333	493.546555035709\\
8.56666666666667	493.645792978728\\
8.55	494.094921251412\\
8.53333333333333	494.394678061288\\
8.51666666666667	494.727475607428\\
8.5	494.8868627101\\
8.48333333333333	495.297062781378\\
8.46666666666667	495.348945505983\\
8.45	495.625563721707\\
8.43333333333333	495.64984121434\\
8.41666666666667	495.840131119843\\
8.4	495.893838637494\\
8.38333333333333	495.992706804886\\
8.36666666666667	495.673668930434\\
8.35	495.272689581202\\
8.33333333333333	494.950193951267\\
8.31666666666667	494.605673896594\\
8.3	494.142003908221\\
8.28333333333333	493.732985536007\\
8.26666666666667	492.997060964114\\
8.25	492.677974231086\\
8.23333333333333	492.431771210537\\
8.21666666666667	492.268020401433\\
8.2	492.321250049958\\
8.18333333333333	492.48936945249\\
8.16666666666667	492.497383000948\\
8.15	492.748580148382\\
8.13333333333333	492.748456804642\\
8.11666666666667	493.106984411118\\
8.1	493.262868814489\\
8.08333333333333	493.444166365647\\
8.06666666666667	493.482992242346\\
8.05	493.917585682269\\
8.03333333333333	494.276219759127\\
8.01666666666667	494.607429240921\\
8	494.752335093951\\
7.98333333333333	494.960578834535\\
7.96666666666667	495.023414544613\\
7.95	494.957006575645\\
7.93333333333333	494.833030091249\\
7.91666666666667	494.384884641164\\
7.9	494.250991346592\\
7.88333333333333	493.983632504972\\
7.86666666666667	493.915503416032\\
7.85	493.653188586865\\
7.83333333333333	493.6513979501\\
7.81666666666667	493.53281671493\\
7.8	493.636163199005\\
7.78333333333333	493.89219838482\\
7.76666666666667	494.258798424111\\
7.75	494.524143597533\\
7.73333333333333	494.803064435264\\
7.71666666666667	495.067219636459\\
7.7	495.258142879928\\
7.68333333333333	495.342918638082\\
7.66666666666667	495.697980528377\\
7.65	495.714000950459\\
7.63333333333333	495.906854139354\\
7.61666666666667	495.896258112053\\
7.6	495.872258776407\\
7.58333333333333	495.651412697234\\
7.56666666666667	495.333410009996\\
7.55	494.985162403709\\
7.53333333333333	494.734246967857\\
7.51666666666667	494.289924781701\\
7.5	493.920322722177\\
7.48333333333333	493.209295822216\\
7.46666666666667	492.63204971695\\
7.45	492.418735272711\\
7.43333333333333	492.360081493085\\
7.41666666666667	492.216149944308\\
7.4	492.380484847148\\
7.38333333333333	492.419579485618\\
7.36666666666667	492.765521948882\\
7.35	492.860568749188\\
7.33333333333333	493.158257455749\\
7.31666666666667	493.300686521274\\
7.3	493.503985902184\\
7.28333333333333	493.59467448861\\
7.26666666666667	493.765220290044\\
7.25	494.115182489815\\
7.23333333333333	494.502106391433\\
7.21666666666667	494.678189676356\\
7.2	494.774101442219\\
7.18333333333333	495.011246168199\\
7.16666666666667	495.104398790239\\
7.15	494.958433525691\\
7.13333333333333	494.507934485237\\
7.11666666666667	494.381649867729\\
7.1	494.210307903405\\
7.08333333333333	494.027926206269\\
7.06666666666667	493.919884137775\\
7.05	493.710771823592\\
7.03333333333333	493.638332981043\\
7.01666666666667	493.544430301289\\
7	494.012857433684\\
6.98333333333333	494.081277546462\\
6.96666666666667	494.453886420122\\
6.95	494.757367610088\\
6.93333333333333	494.942610353563\\
6.91666666666667	495.216870908469\\
6.9	495.34885915004\\
6.88333333333333	495.595288090205\\
6.86666666666667	495.735665994609\\
6.85	495.774776758739\\
6.83333333333333	495.954238953375\\
6.81666666666667	495.861111901822\\
6.8	495.655958776567\\
6.78333333333333	495.517474183785\\
6.76666666666667	495.042177026786\\
6.75	494.891073614841\\
6.73333333333333	494.413663916994\\
6.71666666666667	494.080317416944\\
6.7	493.433701453728\\
6.68333333333333	493.026869081564\\
6.66666666666667	492.361259952386\\
6.65	492.429321432989\\
6.63333333333333	492.338105867471\\
6.61666666666667	492.469893356457\\
6.6	492.376138632899\\
6.58333333333333	492.572826203707\\
6.56666666666667	492.824064785436\\
6.55	493.04814381075\\
6.53333333333333	493.187926892055\\
6.51666666666667	493.251096537248\\
6.5	493.421388843468\\
6.48333333333333	493.755004665859\\
6.46666666666667	493.867729583927\\
6.45	494.201017842542\\
6.43333333333333	494.423249610951\\
6.41666666666667	494.636528701657\\
6.4	494.74592246715\\
6.38333333333333	495.031989768698\\
6.36666666666667	494.932947529877\\
6.35	494.694857886393\\
6.33333333333333	494.36190985457\\
6.31666666666667	494.354221955448\\
6.3	494.211425388777\\
6.28333333333333	494.099166152758\\
6.26666666666667	493.898258925003\\
6.25	493.733164260548\\
6.23333333333333	493.614497116789\\
6.21666666666667	493.883355707836\\
6.2	494.230956868032\\
6.18333333333333	494.34420057446\\
6.16666666666667	494.632931156353\\
6.15	495.003321399599\\
6.13333333333333	495.16376420847\\
6.11666666666667	495.267474950945\\
6.1	495.406033812834\\
6.08333333333333	495.735399533075\\
6.06666666666667	495.752739194293\\
6.05	495.800714458639\\
6.03333333333333	495.75888014253\\
6.01666666666667	495.588356115803\\
6	495.288472215546\\
5.98333333333333	495.033453262085\\
5.96666666666667	494.695917955175\\
5.95	494.394173204011\\
5.93333333333333	494.048978038715\\
5.91666666666667	493.728303423687\\
5.9	493.234910906803\\
5.88333333333333	492.742339709896\\
5.86666666666667	492.47743680899\\
5.85	492.365412184372\\
5.83333333333333	492.409435361841\\
5.81666666666667	492.600134274572\\
5.8	492.469664437288\\
5.78333333333333	492.501920857913\\
5.76666666666667	492.889344146118\\
5.75	493.154334475726\\
5.73333333333333	493.272649316236\\
5.71666666666667	493.333099001164\\
5.7	493.666703744212\\
5.68333333333333	493.827072990575\\
5.66666666666667	494.013574422799\\
5.65	494.212441951007\\
5.63333333333333	494.556177900904\\
5.61666666666667	494.720700015016\\
5.6	495.009023092705\\
5.58333333333333	494.84779827695\\
5.56666666666667	494.742552027122\\
5.55	494.535328382756\\
5.53333333333333	494.533225591545\\
5.51666666666667	494.429227364034\\
5.5	494.273003535517\\
5.48333333333333	494.137887052236\\
5.46666666666667	493.938672704354\\
5.45	493.703104491113\\
5.43333333333333	493.888376452435\\
5.41666666666667	494.083043949895\\
5.4	494.354021921259\\
5.38333333333333	494.450329575987\\
5.36666666666667	494.849082191773\\
5.35	495.012805446254\\
5.33333333333333	495.13950261783\\
5.31666666666667	495.245401922798\\
5.3	495.490318449384\\
5.28333333333333	495.833724174091\\
5.26666666666667	495.697449776446\\
5.25	495.689077846622\\
5.23333333333333	495.486513025896\\
5.21666666666667	495.355348173616\\
5.2	494.970487554797\\
5.18333333333333	494.864086856415\\
5.16666666666667	494.494440964203\\
5.15	494.050827211595\\
5.13333333333333	493.822898838059\\
5.11666666666667	493.314642807903\\
5.1	493.008549714599\\
5.08333333333333	492.572397775395\\
5.06666666666667	492.354979383919\\
5.05	492.397464460786\\
5.03333333333333	492.556762360546\\
5.01666666666667	492.489123746669\\
5	492.390936160544\\
4.98333333333333	492.571787386571\\
4.96666666666667	492.930536713332\\
4.95	493.117467634694\\
4.93333333333333	493.261965892499\\
4.91666666666667	493.408718216709\\
4.9	493.540532401516\\
4.88333333333333	493.709690949365\\
4.86666666666667	494.081189814707\\
4.85	494.193515705475\\
4.83333333333333	494.557367442689\\
4.81666666666667	494.728163966289\\
4.8	494.70189202003\\
4.78333333333333	494.827805888054\\
4.76666666666667	494.699830763836\\
4.75	494.567519134749\\
4.73333333333333	494.421080127438\\
4.71666666666667	494.346138792708\\
4.7	494.10267615984\\
4.68333333333333	493.994442622247\\
4.66666666666667	493.801059013906\\
4.65	493.896410119623\\
4.63333333333333	493.96568718685\\
4.61666666666667	494.141431119729\\
4.6	494.339918990686\\
4.58333333333333	494.539759635884\\
4.56666666666667	494.697127903826\\
4.55	494.791058641282\\
4.53333333333333	495.011281314324\\
4.51666666666667	495.301604721439\\
4.5	495.483249125428\\
4.48333333333333	495.712304705\\
4.46666666666667	495.364527683653\\
4.45	495.447073925518\\
4.43333333333333	495.222522755664\\
4.41666666666667	495.088646430072\\
4.4	494.8079654941\\
4.38333333333333	494.59474908576\\
4.36666666666667	494.34898381658\\
4.35	493.841954786804\\
4.33333333333333	493.622322756331\\
4.31666666666667	493.212451705461\\
4.3	492.864199844455\\
4.28333333333333	492.50550220377\\
4.26666666666667	492.530700487247\\
4.25	492.476968797998\\
4.23333333333333	492.397985145116\\
4.21666666666667	492.347991067948\\
4.2	492.588026484725\\
4.18333333333333	492.850752481417\\
4.16666666666667	493.087891162622\\
4.15	493.070980272454\\
4.13333333333333	493.189066316399\\
4.11666666666667	493.363325818551\\
4.1	493.74601083019\\
4.08333333333333	493.883552663592\\
4.06666666666667	494.047487043598\\
4.05	494.263739367182\\
4.03333333333333	494.606295893483\\
4.01666666666667	494.652561941314\\
4	494.644424429266\\
3.98333333333333	494.719289208199\\
3.96666666666667	494.685396920882\\
3.95	494.676917632773\\
3.93333333333333	494.645421004664\\
3.91666666666667	494.381048729612\\
3.9	494.101059755095\\
3.88333333333333	494.114806734975\\
3.86666666666667	493.898537932254\\
3.85	493.936832391023\\
3.83333333333333	494.067039589931\\
3.81666666666667	494.171014439909\\
3.8	494.38201226565\\
3.78333333333333	494.403909469049\\
3.76666666666667	494.745350459611\\
3.75	494.914251707027\\
3.73333333333333	495.225484521834\\
3.71666666666667	495.542968368065\\
3.7	495.506961137617\\
3.68333333333333	495.565376287865\\
3.66666666666667	495.353189075393\\
3.65	495.367987726347\\
3.63333333333333	495.224944636238\\
3.61666666666667	494.9268116889\\
3.6	494.571140908039\\
3.58333333333333	494.361369321245\\
3.56666666666667	494.058814872092\\
3.55	493.622627562183\\
3.53333333333333	493.524779169069\\
3.51666666666667	493.002434875375\\
3.5	492.831483607172\\
3.48333333333333	492.723886776224\\
3.46666666666667	492.422804031294\\
3.45	492.30103694213\\
3.43333333333333	492.273583590254\\
3.41666666666667	492.304619557236\\
3.4	492.541997451486\\
3.38333333333333	492.807055139194\\
3.36666666666667	493.055909665827\\
3.35	493.189437815568\\
3.33333333333333	493.191954943614\\
3.31666666666667	493.524200753251\\
3.3	493.946747215426\\
3.28333333333333	493.917949697368\\
3.26666666666667	494.100544444189\\
3.25	494.26101956652\\
3.23333333333333	494.538611902429\\
3.21666666666667	494.472449484661\\
3.2	494.486066367089\\
3.18333333333333	494.521092541396\\
3.16666666666667	494.59714485222\\
3.15	494.56478918967\\
3.13333333333333	494.458714458879\\
3.11666666666667	494.19782972645\\
3.1	494.041412159829\\
3.08333333333333	494.006500872643\\
3.06666666666667	493.859655454893\\
3.05	493.868011426045\\
3.03333333333333	494.068552145137\\
3.01666666666667	494.073724557922\\
3	494.277107431455\\
2.98333333333333	494.489997661758\\
2.96666666666667	494.715440991489\\
2.95	494.946963772583\\
2.93333333333333	495.19122316299\\
2.91666666666667	495.346484993293\\
2.9	495.413173790663\\
2.88333333333333	495.458639436214\\
2.86666666666667	495.395179107119\\
2.85	495.263879845724\\
2.83333333333333	495.02998229622\\
2.81666666666667	494.875791549076\\
2.8	494.368221736133\\
2.78333333333333	494.276081277402\\
2.76666666666667	494.036306784517\\
2.75	493.670593421552\\
2.73333333333333	493.388251824689\\
2.71666666666667	493.146526463663\\
2.7	492.824998862042\\
2.68333333333333	492.621304450989\\
2.66666666666667	492.251777194037\\
2.65	492.219071584258\\
2.63333333333333	492.203059090207\\
2.61666666666667	492.231133148623\\
2.6	492.622802365274\\
2.58333333333333	493.025118073896\\
2.56666666666667	493.115386518638\\
2.55	493.181308993577\\
2.53333333333333	493.427188242962\\
2.51666666666667	493.685155919118\\
2.5	494.063309227929\\
2.48333333333333	494.104695396038\\
2.46666666666667	494.135617781015\\
2.45	494.416288148694\\
2.43333333333333	494.312589542123\\
2.41666666666667	494.458629964803\\
2.4	494.409248107356\\
2.38333333333333	494.651409482857\\
2.36666666666667	494.747598751696\\
2.35	494.431344025162\\
2.33333333333333	494.499878081874\\
2.31666666666667	494.368127314997\\
2.3	494.192020256756\\
2.28333333333333	494.150448792614\\
2.26666666666667	493.887641489522\\
2.25	494.036815000289\\
2.23333333333333	494.00782328577\\
2.21666666666667	494.239745333523\\
2.2	494.405522117112\\
2.18333333333333	494.690080936383\\
2.16666666666667	494.824557902987\\
2.15	495.11308532155\\
2.13333333333333	495.412303617055\\
2.11666666666667	495.430473579952\\
2.1	495.385024791338\\
2.08333333333333	495.371438383153\\
2.06666666666667	495.304197616813\\
2.05	495.167173923577\\
2.03333333333333	495.049927893528\\
2.01666666666667	494.612890875325\\
2	494.203685578194\\
1.98333333333333	494.216491654073\\
1.96666666666667	493.960707621089\\
1.95	493.54777742568\\
1.93333333333333	493.382368291009\\
1.91666666666667	493.066602292376\\
1.9	492.752928727698\\
1.88333333333333	492.429536937948\\
1.86666666666667	492.29493268311\\
1.85	492.315077649431\\
1.83333333333333	492.430357429008\\
1.81666666666667	492.654032536326\\
1.8	492.973938268911\\
1.78333333333333	493.169401297935\\
1.76666666666667	493.405018575103\\
1.75	493.530898383012\\
1.73333333333333	493.738947089293\\
1.71666666666667	493.925971838042\\
1.7	493.988846004377\\
1.68333333333333	493.981706468737\\
1.66666666666667	494.035962649285\\
1.65	494.013427672506\\
1.63333333333333	494.09757755303\\
1.61666666666667	494.284833257105\\
1.6	494.395804438207\\
1.58333333333333	494.468609703073\\
1.56666666666667	494.34833920619\\
1.55	494.088423346327\\
1.53333333333333	494.109422423874\\
1.51666666666667	493.940395977789\\
1.5	493.822735299494\\
1.48333333333333	493.871359977976\\
1.46666666666667	493.781894427856\\
1.45	493.874041611113\\
1.43333333333333	493.925874909287\\
1.41666666666667	494.094940812545\\
1.4	494.402407637636\\
1.38333333333333	494.626743860185\\
1.36666666666667	494.955109441526\\
1.35	495.14314812187\\
1.33333333333333	495.133586376831\\
1.31666666666667	495.29185067829\\
1.3	495.316970105136\\
1.28333333333333	495.215043506421\\
1.26666666666667	495.199538486054\\
1.25	495.102151336024\\
1.23333333333333	494.747720002944\\
1.21666666666667	494.429280380854\\
1.2	494.124391697038\\
1.18333333333333	494.122622045017\\
1.16666666666667	493.712424366873\\
1.15	493.443912797075\\
1.13333333333333	493.364615834927\\
1.11666666666667	493.00799106612\\
1.1	492.482564449464\\
1.08333333333333	492.503557884879\\
1.06666666666667	492.357845664094\\
1.05	492.427106396926\\
1.03333333333333	492.679667891972\\
1.01666666666667	492.994900406879\\
1	493.082909516943\\
0.983333333333333	493.445787913947\\
0.966666666666667	493.569020402992\\
0.95	493.760771491669\\
0.933333333333333	493.690723292927\\
0.916666666666667	493.822644861836\\
0.9	493.825101434836\\
0.883333333333333	493.933219055316\\
0.866666666666667	494.087254232497\\
0.85	493.997357725885\\
0.833333333333333	494.121174439444\\
0.816666666666667	494.438360651488\\
0.8	494.460573175369\\
0.783333333333333	494.443502455703\\
0.766666666666667	494.411922146105\\
0.75	494.141877031224\\
0.733333333333333	494.213555358957\\
0.716666666666667	494.087423855218\\
0.7	494.115849496701\\
0.683333333333333	494.034685072687\\
0.666666666666667	493.904667761577\\
0.65	494.093448599175\\
0.633333333333333	494.310143365668\\
0.616666666666667	494.418968567497\\
0.6	494.61508811558\\
0.583333333333333	494.9711284421\\
0.566666666666667	495.215466108585\\
0.55	495.278085240976\\
0.533333333333333	495.274453952523\\
0.516666666666667	495.30748329911\\
0.5	495.174231418647\\
0.483333333333333	495.225853537865\\
0.466666666666667	495.278439725294\\
0.45	494.974914397526\\
0.433333333333333	494.796716239744\\
0.416666666666667	494.358378631811\\
0.4	494.23409263319\\
0.383333333333333	494.071290373787\\
0.366666666666667	493.85266406658\\
0.35	493.532281697956\\
0.333333333333333	493.364515266103\\
0.316666666666667	493.011609047886\\
0.3	492.912501588046\\
0.283333333333333	492.750514311363\\
0.266666666666667	492.578282850327\\
0.25	492.904449783107\\
0.233333333333333	493.04272467674\\
0.216666666666667	493.137422144499\\
0.2	493.309391496664\\
0.183333333333333	493.43238881851\\
0.166666666666667	493.638067235875\\
0.15	493.581020189557\\
0.133333333333333	493.603251047218\\
0.116666666666667	493.701617379061\\
0.1	493.592369940786\\
0.0833333333333333	493.70818797241\\
0.0666666666666667	493.662660837876\\
0.05	493.680889903775\\
0.0333333333333333	494.048618560013\\
0.0166666666666667	494.202943116364\\
0	494.124157924808\\
}--cycle;
\addplot [color=mycolor2]
  table[row sep=crcr]{%
0	422.715049656226\\
0.0166666666666515	422.927119938885\\
0.0333333333333599	422.944843391902\\
0.0500000000000114	422.881894576012\\
0.0666666666666629	422.981512605042\\
0.0833333333333144	423.118411000764\\
0.100000000000023	423.111993888464\\
0.116666666666674	423.348815889992\\
0.133333333333326	423.272421695951\\
0.149999999999977	423.310924369748\\
0.166666666666686	423.379373567609\\
0.183333333333337	423.256531703591\\
0.199999999999989	423.073491214668\\
0.21666666666664	422.760275019099\\
0.233333333333348	422.730634071811\\
0.25	422.579984721161\\
0.266666666666652	422.135981665393\\
0.28333333333336	422.091978609626\\
0.300000000000011	422.105118411001\\
0.316666666666663	422.176012223071\\
0.333333333333314	422.414056531704\\
0.416666666666686	423.182887700535\\
0.433333333333337	423.476852559205\\
0.449999999999989	423.671199388846\\
0.46666666666664	424.054392666157\\
0.483333333333348	424.114896867838\\
0.5	424.161650114591\\
0.516666666666652	424.283880825057\\
0.53333333333336	424.242016806723\\
0.550000000000011	424.283269671505\\
0.566666666666663	424.229793735676\\
0.600000000000023	423.763483575248\\
0.616666666666674	423.426126814362\\
0.633333333333326	423.294117647059\\
0.649999999999977	423.026737967914\\
0.666666666666686	422.698548510313\\
0.683333333333337	422.709549274255\\
0.699999999999989	422.672880061115\\
0.71666666666664	422.557372039725\\
0.733333333333348	422.63346065699\\
0.75	422.547899159664\\
0.766666666666652	422.876088617265\\
0.800000000000011	423.035294117647\\
0.816666666666663	423.203666921314\\
0.833333333333314	423.15538579068\\
0.850000000000023	423.273644003056\\
0.866666666666674	423.501298701299\\
0.883333333333326	423.533384262796\\
0.899999999999977	423.623529411765\\
0.916666666666686	423.624140565317\\
0.933333333333337	423.56577540107\\
0.949999999999989	423.738426279603\\
0.96666666666664	423.51871657754\\
0.983333333333348	423.391291061879\\
1	423.094270435447\\
1.01666666666665	422.956760886173\\
1.05000000000001	422.421390374332\\
1.06666666666666	422.201986249045\\
1.08333333333331	422.193735676089\\
1.10000000000002	422.000305576776\\
1.11666666666667	422.366692131398\\
1.13333333333333	422.653934300993\\
1.14999999999998	422.627043544691\\
1.16666666666669	422.815889992361\\
1.18333333333334	423.13705118411\\
1.19999999999999	423.131550802139\\
1.21666666666664	423.389457601222\\
1.25	424.03269671505\\
1.26666666666665	424.083116883117\\
1.30000000000001	424.216042780749\\
1.31666666666666	424.286631016043\\
1.33333333333331	424.14820473644\\
1.35000000000002	424.288770053476\\
1.36666666666667	424.126203208556\\
1.41666666666669	423.122383498854\\
1.44999999999999	422.669824293354\\
1.46666666666664	422.542704354469\\
1.5	422.335217723453\\
1.51666666666665	422.414667685256\\
1.53333333333336	422.523147440795\\
1.55000000000001	422.607486631016\\
1.56666666666666	422.900840336134\\
1.58333333333331	423.038044308633\\
1.61666666666667	423.237891520244\\
1.63333333333333	423.264171122995\\
1.64999999999998	423.422765469824\\
1.66666666666669	423.653475935829\\
1.68333333333334	423.698701298701\\
1.69999999999999	423.817876241406\\
1.71666666666664	423.718869365928\\
1.73333333333335	423.669060351413\\
1.76666666666665	423.373262032086\\
1.80000000000001	422.967150496562\\
1.81666666666666	422.665546218487\\
1.83333333333331	422.412834224599\\
1.85000000000002	422.258517952636\\
1.86666666666667	422.073032849503\\
1.88333333333333	422.049503437739\\
1.89999999999998	422.263101604278\\
1.91666666666669	422.415584415584\\
1.93333333333334	422.646294881589\\
1.94999999999999	422.791138273491\\
1.96666666666664	423.144079449962\\
1.98333333333335	423.367150496562\\
2	423.305118411001\\
2.03333333333336	423.912910618793\\
2.05000000000001	424.066615737204\\
2.06666666666666	424.085561497326\\
2.08333333333331	424.180290297937\\
2.10000000000002	424.313521772345\\
2.13333333333333	424.404278074866\\
2.16666666666669	423.830099312452\\
2.19999999999999	423.323758594347\\
2.21666666666664	422.966233766234\\
2.23333333333335	422.791443850267\\
2.25	422.657601222307\\
2.26666666666665	422.394805194805\\
2.28333333333336	422.57845683728\\
2.30000000000001	422.541176470588\\
2.31666666666666	422.739190221543\\
2.35000000000002	422.941176470588\\
2.36666666666667	423.377540106952\\
2.38333333333333	423.491520244461\\
2.39999999999998	423.533384262796\\
2.41666666666669	423.840488922842\\
2.43333333333334	423.895187165775\\
2.44999999999999	424.057754010695\\
2.46666666666664	424.017417876241\\
2.48333333333335	423.997249809015\\
2.5	424.130786860199\\
2.51666666666665	423.832849503438\\
2.53333333333336	423.696867838044\\
2.55000000000001	423.487547746371\\
2.56666666666666	423.342093200917\\
2.58333333333331	423.236974789916\\
2.60000000000002	422.89320091673\\
2.61666666666667	422.494728800611\\
2.63333333333333	422.36577540107\\
2.64999999999998	422.288158899924\\
2.66666666666669	422.164094728801\\
2.68333333333334	422.268296409473\\
2.69999999999999	422.408556149733\\
2.75	423.038044308633\\
2.76666666666665	423.3179526356\\
2.78333333333336	423.527883880825\\
2.80000000000001	423.56974789916\\
2.81666666666666	423.986249045073\\
2.83333333333331	423.977081741788\\
2.85000000000002	424.120091673033\\
2.86666666666667	424.240794499618\\
2.88333333333333	424.240794499618\\
2.89999999999998	424.321466768526\\
2.91666666666669	424.246600458365\\
2.93333333333334	424.064782276547\\
2.96666666666664	423.535828877005\\
2.98333333333335	423.316730328495\\
3.01666666666665	422.700687547746\\
3.05000000000001	422.342245989305\\
3.06666666666666	422.299770817418\\
3.08333333333331	422.406722689076\\
3.10000000000002	422.438808250573\\
3.14999999999998	423.18961038961\\
3.16666666666669	423.493353705118\\
3.18333333333334	423.709396485867\\
3.19999999999999	423.848433919022\\
3.23333333333335	424.208403361345\\
3.25	424.156455309397\\
3.26666666666665	424.03269671505\\
3.28333333333336	423.966080977846\\
3.30000000000001	424.091367456073\\
3.33333333333331	423.432238349885\\
3.35000000000002	423.426126814362\\
3.36666666666667	423.269977081742\\
3.38333333333333	423.04782276547\\
3.41666666666669	422.388693659282\\
3.43333333333334	422.295187165775\\
3.44999999999999	422.160427807487\\
3.46666666666664	422.159205500382\\
3.48333333333335	422.337662337662\\
3.5	422.3410236822\\
3.51666666666665	422.475783040489\\
3.53333333333336	422.823223834989\\
3.56666666666666	423.320702826585\\
3.58333333333331	423.479602750191\\
3.60000000000002	423.548051948052\\
3.61666666666667	423.851489686784\\
3.63333333333333	424.061420932009\\
3.64999999999998	424.139648586707\\
3.66666666666669	424.094423223835\\
3.68333333333334	424.209625668449\\
3.69999999999999	424.207792207792\\
3.71666666666664	424.043086325439\\
3.75	423.516577540107\\
3.76666666666665	423.350038197097\\
3.78333333333336	423.041100076394\\
3.83333333333331	422.391749427044\\
3.85000000000002	422.287547746371\\
3.86666666666667	422.268296409473\\
3.88333333333333	422.467532467532\\
3.89999999999998	422.585179526356\\
3.93333333333334	423.456684491979\\
3.96666666666664	424.036974789916\\
3.98333333333335	424.246294881589\\
4	424.33705118411\\
4.01666666666665	424.53384262796\\
4.03333333333336	424.56653934301\\
4.05000000000001	424.391443850267\\
4.06666666666666	424.312605042017\\
4.08333333333331	424.159816653934\\
4.10000000000002	424.120091673033\\
4.11666666666667	423.825821237586\\
4.13333333333333	423.588388082506\\
4.14999999999998	423.522077922078\\
4.16666666666669	423.435905271199\\
4.21666666666664	422.437585943468\\
4.23333333333335	422.312910618793\\
4.26666666666665	422.275935828877\\
4.28333333333336	422.197402597403\\
4.35000000000002	423.251948051948\\
4.36666666666667	423.604583651642\\
4.39999999999998	423.913827349121\\
4.43333333333334	424.117647058824\\
4.44999999999999	424.113368983957\\
4.46666666666664	424.01833460657\\
4.48333333333335	424.159205500382\\
4.51666666666665	423.720397249809\\
4.53333333333336	423.399847211612\\
4.55000000000001	423.222001527884\\
4.56666666666666	423.087242169595\\
4.58333333333331	422.879449961803\\
4.60000000000002	422.621848739496\\
4.61666666666667	422.439113827349\\
4.63333333333333	422.317188693659\\
4.64999999999998	422.283575248281\\
4.66666666666669	422.344690603514\\
4.69999999999999	422.89564553094\\
4.75	423.94499618029\\
4.76666666666665	424.146371275783\\
4.78333333333336	424.484339190222\\
4.80000000000001	424.557066462949\\
4.81666666666666	424.749274255157\\
4.83333333333331	424.58090145149\\
4.85000000000002	424.386249045073\\
4.86666666666667	424.291825821238\\
4.88333333333333	424.031474407945\\
4.89999999999998	423.907715813598\\
4.91666666666669	423.865851795264\\
4.96666666666664	423.240336134454\\
4.98333333333335	422.789304812834\\
5	422.52192513369\\
5.01666666666665	422.361497326203\\
5.03333333333336	422.414973262032\\
5.05000000000001	422.254239877769\\
5.06666666666666	422.126508785332\\
5.08333333333331	422.225821237586\\
5.10000000000002	422.572650878533\\
5.11666666666667	422.825668449198\\
5.13333333333333	423.222612681436\\
5.14999999999998	423.440488922842\\
5.16666666666669	423.744537815126\\
5.18333333333334	423.923911382735\\
5.19999999999999	423.992666157372\\
5.21666666666664	424.251184110008\\
5.23333333333335	424.136898395722\\
5.25	424.156455309397\\
5.26666666666665	424.09961802903\\
5.28333333333336	424.067532467532\\
5.30000000000001	423.726508785332\\
5.33333333333331	423.342093200917\\
5.35000000000002	423.233919022154\\
5.36666666666667	423.081741787624\\
5.38333333333333	422.854698242934\\
5.39999999999998	422.763025210084\\
5.41666666666669	422.56653934301\\
5.43333333333334	422.504812834225\\
5.44999999999999	422.484950343774\\
5.53333333333336	424.232849503438\\
5.56666666666666	424.665240641711\\
5.58333333333331	424.870282658518\\
5.60000000000002	425.012681436211\\
5.63333333333333	424.704660045837\\
5.64999999999998	424.371581359817\\
5.66666666666669	424.319633307869\\
5.68333333333334	424.129258976318\\
5.69999999999999	424.047058823529\\
5.71666666666664	423.773567608862\\
5.73333333333335	423.639724980901\\
5.76666666666665	423.099770817418\\
5.78333333333336	422.682658517953\\
5.80000000000001	422.519480519481\\
5.81666666666666	422.575401069519\\
5.83333333333331	422.347746371276\\
5.85000000000002	422.349885408709\\
5.86666666666667	422.283880825057\\
5.88333333333333	422.429640947288\\
5.91666666666669	423.194499618029\\
5.94999999999999	423.596944232238\\
5.96666666666664	423.857601222307\\
5.98333333333335	423.97372039725\\
6	424.018945760122\\
6.01666666666665	424.081894576012\\
6.03333333333336	424.163789152024\\
6.05000000000001	424.124675324675\\
6.06666666666666	423.981359816654\\
6.08333333333331	423.795874713522\\
6.10000000000002	423.411459129106\\
6.11666666666667	423.227501909855\\
6.13333333333333	423.199388846448\\
6.14999999999998	423.077158135982\\
6.16666666666669	422.825668449198\\
6.18333333333334	422.629793735676\\
6.19999999999999	422.741329258976\\
6.21666666666664	422.561344537815\\
6.23333333333335	422.475783040489\\
6.25	422.820779220779\\
6.26666666666665	423.0820473644\\
6.28333333333336	423.559663865546\\
6.30000000000001	423.881741787624\\
6.31666666666666	424.1179526356\\
6.33333333333331	424.293353705118\\
6.35000000000002	424.645072574484\\
6.36666666666667	424.933231474408\\
6.38333333333333	425.026432391138\\
6.39999999999998	424.787776928953\\
6.41666666666669	424.614514896868\\
6.43333333333334	424.543315508021\\
6.44999999999999	424.316577540107\\
6.48333333333335	423.986249045073\\
6.51666666666665	423.578609625668\\
6.55000000000001	423.286172650879\\
6.58333333333331	422.708632543927\\
6.60000000000002	422.466004583652\\
6.61666666666667	422.48525592055\\
6.63333333333333	422.350190985485\\
6.64999999999998	422.308632543927\\
6.66666666666669	422.216042780749\\
6.68333333333334	422.626432391138\\
6.69999999999999	422.959816653934\\
6.71666666666664	423.377845683728\\
6.73333333333335	423.56730328495\\
6.75	423.81909854851\\
6.76666666666665	423.726203208556\\
6.78333333333336	423.97372039725\\
6.80000000000001	423.898242933537\\
6.81666666666666	423.933689839572\\
6.83333333333331	423.978304048892\\
6.85000000000002	423.713368983957\\
6.86666666666667	423.616806722689\\
6.88333333333333	423.384262796027\\
6.89999999999998	423.199083269672\\
6.91666666666669	423.176470588235\\
6.93333333333334	423.046906035141\\
6.94999999999999	422.99281894576\\
6.98333333333335	422.82872421696\\
7	422.823529411765\\
7.01666666666665	422.7410236822\\
7.03333333333336	423.168525592055\\
7.05000000000001	423.311229946524\\
7.06666666666666	423.606722689076\\
7.08333333333331	423.999694423224\\
7.10000000000002	424.222459893048\\
7.11666666666667	424.517341482047\\
7.13333333333333	424.71871657754\\
7.14999999999998	425.052711993888\\
7.16666666666669	425.14499618029\\
7.18333333333334	425.054851031322\\
7.19999999999999	424.864171122995\\
7.21666666666664	424.703437738732\\
7.26666666666665	423.986860198625\\
7.28333333333336	423.842016806723\\
7.31666666666666	423.607028265852\\
7.33333333333331	423.414209320092\\
7.35000000000002	423.085714285714\\
7.36666666666667	422.968983957219\\
7.38333333333333	422.618181818182\\
7.39999999999998	422.581512605042\\
7.41666666666669	422.371581359817\\
7.43333333333334	422.472421695951\\
7.44999999999999	422.396638655462\\
7.46666666666664	422.504201680672\\
7.48333333333335	422.802139037433\\
7.5	423.172192513369\\
7.51666666666665	423.248281130634\\
7.53333333333336	423.55538579068\\
7.55000000000001	423.661115355233\\
7.56666666666666	423.737509549274\\
7.58333333333331	423.85179526356\\
7.60000000000002	423.939495798319\\
7.61666666666667	423.904048892284\\
7.63333333333333	423.774789915966\\
7.64999999999998	423.573414820474\\
7.66666666666669	423.442322383499\\
7.68333333333334	423.158441558442\\
7.69999999999999	423.191138273491\\
7.73333333333335	423.120550038197\\
7.75	423.065240641711\\
7.76666666666665	423.055156608098\\
7.78333333333336	422.976012223071\\
7.80000000000001	422.989152024446\\
7.81666666666666	423.266004583652\\
7.83333333333331	423.498854087089\\
7.85000000000002	423.632391138273\\
7.86666666666667	423.915966386555\\
7.88333333333333	424.088617265088\\
7.91666666666669	424.597097020626\\
7.93333333333334	424.997708174179\\
7.94999999999999	425.042627960275\\
7.96666666666664	425.037127578304\\
7.98333333333335	424.921313980137\\
8	424.655462184874\\
8.01666666666665	424.519786096257\\
8.03333333333336	424.283880825057\\
8.06666666666666	423.676394194041\\
8.10000000000002	423.498548510313\\
8.11666666666667	423.371734148205\\
8.13333333333333	423.075630252101\\
8.14999999999998	423.025210084034\\
8.16666666666669	422.805500381971\\
8.18333333333334	422.804278074866\\
8.19999999999999	422.524675324675\\
8.21666666666664	422.447669977082\\
8.23333333333335	422.47486631016\\
8.25	422.530481283422\\
8.26666666666665	422.563789152024\\
8.28333333333336	422.850725744843\\
8.30000000000001	423.042016806723\\
8.31666666666666	423.358288770053\\
8.33333333333331	423.577998472116\\
8.35000000000002	423.68128342246\\
8.36666666666667	423.861573720397\\
8.38333333333333	423.966692131398\\
8.39999999999998	423.746065699007\\
8.41666666666669	423.717035905271\\
8.43333333333334	423.539495798319\\
8.44999999999999	423.426432391138\\
8.46666666666664	423.272727272727\\
8.48333333333335	423.334453781513\\
8.5	423.204583651642\\
8.51666666666665	423.204583651642\\
8.53333333333336	423.228113063407\\
8.55000000000001	423.237280366692\\
8.56666666666666	423.152941176471\\
8.60000000000002	423.501604278075\\
8.61666666666667	423.463407181054\\
8.63333333333333	423.786401833461\\
8.64999999999998	423.891520244461\\
8.66666666666669	424.107257448434\\
8.69999999999999	424.617876241406\\
8.71666666666664	424.996485867074\\
8.73333333333335	424.966233766234\\
8.75	424.876394194041\\
8.78333333333336	424.669824293354\\
8.80000000000001	424.408556149733\\
8.81666666666666	424.348051948052\\
8.83333333333331	423.937356760886\\
8.85000000000002	423.795874713522\\
8.88333333333333	423.469518716578\\
8.89999999999998	423.355233002292\\
8.91666666666669	423.18808250573\\
8.93333333333334	423.065240641711\\
8.94999999999999	422.831474407945\\
8.96666666666664	422.732773109244\\
8.98333333333335	422.540565317036\\
9	422.441558441558\\
9.01666666666665	422.31871657754\\
9.03333333333336	422.320244461421\\
9.05000000000001	422.386860198625\\
9.06666666666666	422.536898395722\\
9.08333333333331	422.839419404125\\
9.11666666666667	423.311229946524\\
9.13333333333333	423.493964858671\\
9.14999999999998	423.511688311688\\
9.16666666666669	423.700840336134\\
9.18333333333334	423.7115355233\\
9.19999999999999	423.474713521772\\
9.21666666666664	423.507410236822\\
9.23333333333335	423.288922841864\\
9.25	423.27333842628\\
9.26666666666665	423.425821237586\\
9.28333333333336	423.254698242934\\
9.30000000000001	423.416653934301\\
9.31666666666666	423.394346829641\\
9.33333333333331	423.3769289534\\
9.35000000000002	423.342704354469\\
9.36666666666667	423.366844919786\\
9.38333333333333	423.562108479756\\
9.39999999999998	423.479297173415\\
9.41666666666669	423.631779984721\\
9.43333333333334	423.952330022918\\
9.44999999999999	423.890909090909\\
9.46666666666664	424.086172650879\\
9.48333333333335	424.354469060351\\
9.5	424.526814362108\\
9.53333333333336	424.719938884645\\
9.55000000000001	424.635294117647\\
9.56666666666666	424.577540106952\\
9.58333333333331	424.425668449198\\
9.60000000000002	424.2230710466\\
9.61666666666667	424.081894576012\\
9.63333333333333	423.830710466005\\
9.64999999999998	423.672116119175\\
9.68333333333334	423.240947288006\\
9.69999999999999	423.189304812834\\
9.71666666666664	422.998624904507\\
9.73333333333335	422.971734148205\\
9.75	422.751107715814\\
9.76666666666665	422.646600458365\\
9.78333333333336	422.375553857907\\
9.80000000000001	422.225210084034\\
9.81666666666666	422.291825821238\\
9.83333333333331	422.388693659282\\
9.85000000000002	422.353552330023\\
9.86666666666667	422.605958747135\\
9.88333333333333	422.923147440794\\
9.89999999999998	423.135523300229\\
9.91666666666669	423.112910618793\\
9.93333333333334	423.352788388083\\
9.94999999999999	423.438961038961\\
9.96666666666664	423.625362872422\\
9.98333333333335	423.481436210848\\
10	423.566386554622\\
10.0166666666667	423.410847975554\\
10.0333333333334	423.335064935065\\
10.05	423.587165775401\\
10.0666666666667	423.606722689076\\
10.0833333333333	423.415737203973\\
10.1	423.620168067227\\
10.1166666666667	423.519022154316\\
10.1333333333333	423.64064171123\\
10.15	423.601833460657\\
10.1666666666667	423.581054239878\\
10.1833333333333	423.818487394958\\
10.2	423.803819709702\\
10.2166666666666	423.849656226127\\
10.2333333333333	423.877463712758\\
10.25	424.167150496562\\
10.2666666666667	424.216348357525\\
10.2833333333334	424.444003055768\\
10.3	424.521313980138\\
10.3166666666667	424.571734148205\\
10.3333333333333	424.502979373568\\
10.35	424.339190221543\\
10.3666666666667	424.482811306341\\
10.3833333333333	424.423529411765\\
10.4	424.18487394958\\
10.4166666666667	424.12192513369\\
10.4333333333333	423.765928189458\\
10.4666666666666	423.332009167303\\
10.4833333333333	423.179831932773\\
10.5	423.185332314744\\
10.5166666666667	422.846142093201\\
10.5333333333334	422.677769289534\\
10.5666666666667	422.109396485867\\
10.5833333333333	422.09961802903\\
10.6	422.056531703591\\
10.6166666666667	422.134759358289\\
10.6333333333333	422.267074102368\\
10.65	422.4589763178\\
10.6666666666667	422.587929717341\\
10.7	422.906646294882\\
10.7166666666666	423.159052711994\\
10.7333333333333	423.312452253629\\
10.75	423.2525592055\\
10.7666666666667	423.445683728037\\
10.7833333333334	423.401375095493\\
10.8	423.406569900688\\
10.8166666666667	423.74025974026\\
10.8333333333333	423.702368220015\\
10.85	423.790679908327\\
10.8666666666667	423.698395721925\\
10.8833333333333	423.522077922078\\
10.9	423.646142093201\\
10.9166666666667	423.640030557678\\
10.9333333333333	423.801069518717\\
10.95	423.848739495798\\
10.9666666666666	423.723758594347\\
10.9833333333333	423.793430099312\\
11	423.774178762414\\
11.0166666666667	423.845072574484\\
11.05	424.016806722689\\
11.0666666666667	424.012528647823\\
11.0833333333333	424.176623376623\\
11.1	424.260351413293\\
11.1166666666667	424.154927425516\\
11.1333333333333	424.218487394958\\
11.15	424.327578304049\\
11.1666666666667	424.394194041253\\
11.1833333333333	424.171122994652\\
11.2	423.901909854851\\
11.2166666666666	423.886631016043\\
11.2333333333333	423.397402597403\\
11.25	423.229335370512\\
11.2666666666667	422.999541634836\\
11.2833333333334	422.824446142093\\
11.3	422.744385026738\\
11.3166666666667	422.72513368984\\
11.3333333333333	422.156455309397\\
11.35	422.179679144385\\
11.3666666666667	421.903437738732\\
11.3833333333333	422.057754010695\\
11.4	422.056837280367\\
11.4166666666667	422.255156608098\\
11.4333333333333	422.282352941176\\
11.45	422.423223834989\\
11.4666666666666	422.616042780749\\
11.4833333333333	422.879755538579\\
11.5	423.061268143621\\
11.5333333333334	423.511688311688\\
11.55	423.574637127578\\
11.5666666666667	423.529411764706\\
11.6	423.901298701299\\
11.6166666666667	424.083116883117\\
11.6333333333333	424.031779984721\\
11.65	423.832238349885\\
11.6666666666667	423.789152024446\\
11.6833333333333	423.736898395722\\
11.7166666666666	423.911688311688\\
11.7333333333333	423.981359816654\\
11.75	424.124369747899\\
11.7666666666667	424.062643239114\\
11.7833333333334	424.043697478992\\
11.8	423.892742551566\\
11.8166666666667	424.126814362108\\
11.8333333333333	424.017417876241\\
11.85	424.198013750955\\
11.8666666666667	424.069365928189\\
11.8833333333333	424.313521772345\\
11.9	424.129564553094\\
11.9166666666667	424.162261268144\\
11.9333333333333	424.350496562261\\
11.95	424.382887700535\\
11.9666666666666	424.283575248281\\
11.9833333333333	424.064782276547\\
12	423.810847975554\\
};
\addlegendentry{$\text{\textless{}100 ms}$}

\end{axis}
\end{tikzpicture}%
        \captionsetup{justification=centering}
        \caption{Number of available \glspl{isl} available at each satellite, that experience a latency below 10~ms (blue) and 100~ms (orange).
        The solid line and the shaded area represent the average and the standard deviation, respectively.}
        \label{fig:isl_latency}
    \end{subfigure}
    \caption{Propagation delay for ground-to-satellite (a) and inter-satellite (b) links as observed over a 12-hour interval (12 AM-12 PM, Nov. 11th, 2024, 60 s sampling time). }
     \vspace{-10pt}
\end{figure*}

\subsection{Use Cases}

The following examples highlight how addressing critical integration challenges will support advanced applications.


\textbf{Rural and Localized Networks.} Deploying localized networks in remote or underserved areas demands flexible resource allocation and low-latency communication. Standalone \glspl{ntn} often struggle to meet these requirements due to limited adaptability. Integrated architectures address these limitations by enabling terrestrial base stations to act as ground stations, efficiently managing spectrum and resources across both terrestrial and non-terrestrial areas, removing the reliance on specific feeder links.

In rural regions, where the deployment and maintenance of terrestrial infrastructure is prohibitively expensive, \glspl{ntn} can provide high-speed broadband backhaul to existing base stations. By serving multiple regions globally, \glspl{ntn} distribute deployment costs across a vast user base, significantly reducing the financial burden on local communities compared to establishing fixed terrestrial networks. During tactical operations, integrated systems can ensure seamless service delivery by prioritizing low-latency command-and-control communication via terrestrial nodes, while leveraging \glspl{ntn} for robust data aggregation and high-latency bulk transmission. This dual capability supports critical applications without relying on pre-existing terrestrial infrastructure.

\textbf{Autonomous Disaster Response.} In disaster scenarios where the terrestrial infrastructure is severely disrupted or destroyed, integrated terrestrial and non-terrestrial systems can ensure the continuity of communication. Using \glspl{isl} for backhaul and dynamic routing of traffic through unaffected terrestrial nodes, these systems enable real-time situational awareness, emergency coordination, and remote command-and-control. For example, a satellite-enabled backhaul can restore connectivity by dynamically allocating bandwidth to the affected region while prioritizing critical traffic, such as video feeds for search-and-rescue operations. Additionally, real-time telemetry from \glspl{ntn} enhances decision-making by complementing terrestrial sensing capabilities.


\textbf{Ubiquitous AI-Driven Applications.} Satellites provide unmatched global coverage for data collection, but face significant limitations in on-board storage, processing power, and energy efficiency, making local \gls{ai} inference impractical \cite{mahboob2023revolutionizing, oranntn2025}. Integrated systems overcome these challenges by offloading data pre-processing, model training, and resource-intensive inference to terrestrial cloud or edge infrastructure. Federated AI would allow the use of terrestrial resources for distributed model training while leveraging \glspl{ntn} for global data collection. This approach would improve system-wide efficiency, reduce latency, and support scalable deployment of \gls{ai}-driven ubiquitous applications, such as autonomous navigation, predictive maintenance, and real-time analytics in remote or resource-constrained environments.


\section{Proposed Solution: Space-O-RAN}
\label{sec:architecture}


%Explain the constellation and why is it relevant here for the architecture

The Space-O-RAN architecture integrates terrestrial and \glspl{ntn}, addressing key challenges such as {\em dynamic connectivity, resource constraints, and scalability}. By applying open \gls{ran} principles and standardized interfaces, the architecture ensures seamless interaction between terrestrial and non-terrestrial segments.

To efficiently manage satellite operations within a large constellation, we adopt a clustering approach~\cite{clustering}. Satellites are grouped based on orbital proximity, reducing system-wide signaling overhead and enabling localized decision-making. Instead of requiring continuous coordination across the entire constellation, each cluster autonomously optimizes its resources while maintaining alignment with broader network objectives. This decentralized approach enhances adaptability to dynamic topology changes and intermittent connectivity.

The architecture consists of two main domains: space-based and ground-based components (see \cref{fig:oran_space_arch}). The onboard AI-enabled edge network, optimized for resource-constrained environments, hosts modular network elements such as \glspl{sdu} and \glspl{scu}. At the core of this system is the \gls{spaceric}, a near-real-time \gls{ric} responsible for cluster-wide coordination and real-time adaptation. Unlike dApps, which directly interact with individual RAN functions, the \gls{spaceric} orchestrates intra-cluster decision-making, managing inter-satellite handovers, spectrum allocation, and resource scheduling while ensuring alignment with terrestrial directives. \glspl{sapp} operate within the \gls{spaceric}, functioning similarly to terrestrial \glspl{xapp} but designed for execution in orbit within a coordinated cluster.

Real-time connectivity is maintained by \glspl{sru}, each managing a dedicated link for \glspl{isl}, \glspl{fl}/\glspl{gsl}, and \glspl{sl}, ensuring cohesive communication with terrestrial systems and end-users. These links facilitate efficient task execution across network layers, minimizing latency in inter-satellite coordination and user-plane operations.

On the ground, the Terrestrial Cloud and \gls{smo} oversee long-term strategic operations, including \gls{ai} model training and \gls{dt} modeling for predictive optimization. Standardized O-RAN interfaces such as \gls{a1} and \gls{o1} enable closed-loop communication, integrating satellite telemetry to refine global policies and ensure continuous adaptation.

A functional split ensures efficient task allocation between the space and ground domains. Real-time tasks, such as beamforming, resource scheduling, and spectrum management, are handled at the satellite and cluster levels, making use of lower-latency \glspl{isl} for coordinated decision-making. \glspl{dapp} complement \glspl{sapp} and \glspl{rapp}, enabling dynamic control adjustments. Meanwhile, computationally intensive processes, including AI model training, telemetry aggregation, and long-term optimization, are offloaded to terrestrial cloud infrastructure via \glspl{gsl}, ensuring scalability without compromising real-time responsiveness.%Clustering neighboring satellites in overlapping orbital planes reduces system-wide coordination overhead, enabling localized optimization and efficient resource reuse. Each cluster operates under a leader-follower model, where a designated leader satellite dynamically manages intra-cluster tasks such as inter-satellite handovers, spectrum allocation, and resource scheduling, while ensuring alignment with global network directives. "

\vskip -2\baselineskip plus -1fil

%XXX Two questions to address here. What is a cluster, and why do we need a Space RIC? Isn't it the same as dApps - applications interacting directly with RAN Nodes? What's fundamentally different? XXX


\subsection{Control Knobs and Optimization Surface.}
The architecture defines three hierarchical control levels, each operating on different timescales to address the complex demands of integrated terrestrial and \glspl{ntn}. The \textbf{strategic level}, supported by \gls{dt} and \gls{ai} large-scale models in the terrestrial cloud, gathers and processes data such as cluster-specific or satellite-level \glspl{kpi}. This level focuses on monitoring and optimizing the overall performance of the constellation in terms of coverage, energy efficiency, capacity utilization, resilience, and service life.

The \textbf{coordination level} manages resource allocation among satellite clusters, optimizing efficiency based on traffic conditions and the state of \gls{ai} pipelines. While this level operates on the on-board edge, it communicates with a cloud-based orchestrator to align with global objectives. However, it retains the ability to function autonomously in scenarios requiring localized decision-making.

At the \textbf{operational} level, control occurs at the individual satellite scale, where radio parameters and physical links are dynamically adjusted in real-time to respond to user demands and environmental variations. This ensures a seamless adaptation to traffic fluctuations, enhancing network responsiveness and overall efficiency.

The \textbf{optimization surface}  unifies these control layers, ensuring that all levels of the hierarchy dynamically adjust to maintain optimal balance between latency, throughput, energy consumption and spectral efficiency. 

\subsection{Feasibility Bounds}
\label{sec:feasibility}

The feasibility of Space-O-RAN is evaluated by examining the latency characteristics of \glspl{ntn} and their implications for the hierarchical control framework. Simulations of the Starlink constellation, comprising 6545 satellites and 33 ground gateways, provide latency insights critical for aligning real-time and non real-time operations. 

Due to the Starlink constellation design and density, each satellite has a consistent number of available \glspl{isl} below 10~ms delays (\cref{fig:isl_latency}), offering a reliable foundation for real-time tasks. %In contrast, due to the \gls{leo} satellite rapid motion, \glspl{gsl} exhibit higher and more variable delays (\cref{fig:g2s_latency}), making them suitable for non-time-critical operations.
Conversely, due to the \gls{leo} satellite rapid motion, the \glspl{gsl} delay exceeds 10 ms for 
% ranges from 10 to over 40 ms 
80\% of the possible elevation angles (\cref{fig:g2s_latency}).
% depending on satellite elevation. 
More of half of the satellites visible from the ground experience a propagation delay above 40~ms, which limits their suitability for real-time operations but makes them viable for strategic, non-critical tasks like policy updates and long-term resource allocation.

This latency stratification aligns with the functional split strategy, as previously described. Real-time, latency-sensitive decisions are localized to the operational and coordination levels, using \glspl{isl} for immediate adjustments and intra-cluster communication. Computationally intensive tasks, which can tolerate higher delays, are offloaded to the terrestrial cloud via \glspl{gsl}, ensuring efficient utilization of satellite and terrestrial resources.

The analysis also highlights critical trade-offs in the integration of terrestrial and non-terrestrial networks. Meanwhile reducing on-board processing delays can improve responsiveness, it places greater reliance on \glspl{isl} for high-bandwidth, low-latency communication. Conversely, centralized cloud-based processing risks overloading \glspl{gsl}, reducing spectral efficiency and introducing bottlenecks. The proposed hybrid approach mitigates these trade-offs by combining localized real-time decision-making at the cluster level with scalable computational resources in the terrestrial cloud.



\begin{table*}[h]
\centering
\scriptsize % Reduce font size
\setlength{\tabcolsep}{2pt} % Reduce space between columns
\caption{Dynamic Interface-Link Mapping for NTNs}
\label{tab:interface_mapping_flexibility}
\begin{tabular}{|p{2.5cm}|p{2.5cm}|p{2.5cm}|p{4cm}|p{4cm}|p{2cm}|} % Adjust column widths to fit the page
\hline
\textbf{Link Type} & \textbf{Primary Function} & \textbf{Supported Interfaces} & \textbf{Role in the Integrated Network} & \textbf{Dynamic Re-Mapping Capability} & \textbf{Delay Quantiles}\\
\hline
Service Link (SL) & End-user connection & F1, E2, NG & Provides QoS for critical slices and user-plane traffic; re-routes traffic to coordination or feeder roles during disruptions & Supports high-priority traffic under feeder link outages by re-allocating resources dynamically & Median: 30 ms, 90\% perc.: 40 ms\\
\hline
Feeder Link (FL) & Satellite-to-ground backhaul & A1, E2, F1, O1 & Handles satellite-terrestrial integration, telemetry updates, and user-plane data routing; ensures system-wide traffic stability & Re-routes telemetry updates and bulk data transfers during congestion or outages & Median: 50 ms, 90\% perc.: 70 ms\\
\hline
Ground to Satellite Link (GSL) & Terrestrial communication uplink & A1, F1, O1 & Supports strategic updates, model synchronization, and policy distribution from terrestrial systems & Dynamically re-maps for priority updates during satellite connectivity disruptions & Median: 40 ms, 90\% perc.: 60 ms\\
\hline
Inter-Satellite Link (ISL) & Satellite coordination & E2, F1, A1, Xn & Enables real-time resource sharing, inter-satellite routing, and CU-DU splits for critical operations & Re-allocates traffic to feeder backhaul or user-plane under congestion or feeder link loss & Median: 5 ms, 90\% perc.: 10 ms\\
\hline
\end{tabular}
\end{table*}




%The \gls{spaceric} centrally coordinates the satellite cluster, housing satellite-specific \glspl{xapp} and \glspl{sapp} for near-real-time tasks like interference detection and management, and aiding constellation operations. 

%A distinctive feature of the architecture is its ability to dynamically map O-RAN interface functions onto physical links, enabling efficient bandwidth utilization and traffic management. This capability ensures that the architecture remains scalable and adaptable, supporting larger constellations and more complex \gls{ntn} deployments in the future.

\begin{figure}[h]
\centering
\includegraphics[width=0.9\linewidth, trim=0 0 100 0, clip]{figures/SpaceRIC.pdf}
\captionsetup{justification=centering}
\caption{Hierarchical coordination workflow}
\label{fig:spaceRIC}
\end{figure}





\subsection{SpaceRIC and Hierarchical Coordination of r/s/dApps}
\label{subsec:hierarchical_coordination}

The \gls{spaceric}, illustrated in Fig.~\ref{fig:spaceRIC}, extends the O-RAN \gls{ric} framework to satellite clusters, integrating global policies from terrestrial \glspl{rapp} with real-time operations in space. Unlike terrestrial \glspl{ric}, which rely on stable infrastructure, \gls{spaceric} is designed for decentralized control, enabling satellites to adapt dynamically to topology changes, intermittent connectivity, and constrained computational resources.

Each satellite runs an instance of \gls{spaceric}, but coordination follows a Leader-Follower model to optimize system-wide efficiency. The Leader interfaces with terrestrial non-RT \gls{ric}, receiving policies via \gls{a1} and \gls{o1}, processing real-time telemetry from \gls{e2}, and issuing intra-cluster control commands. It dynamically manages inter-satellite handovers, spectrum allocation, and resource scheduling, ensuring local adaptations align with global directives.

Followers execute Leader's control decisions via \glspl{dapp}, handling near-real-time tasks such as beam adjustments, modulation selection, and power control. To enhance resilience, all satellites maintain active but passive Coord \gls{sapp}, continuously synchronizing state information. If the Leader fails or loses connectivity, the \gls{fede2} protocol triggers an autonomous re-election, selecting the most suitable Follower based on link stability, computational capacity, and telemetry quality. Since Followers remain pre-synchronized, the transition occurs seamlessly with minimal service disruption.

\glspl{sapp} in \gls{spaceric} optimize cluster-wide decision making similar to terrestrial \glspl{xapp}, but are designed for in-orbit execution. These applications operate in a fully distributed manner, synchronizing state updates via inter-satellite links (\glspl{isl}) to prevent conflicts in resource allocation. The Leader enables these processes while avoiding a single point of failure, maintaining distributed control if it becomes unavailable until a new election occurs.

When a satellite detects degrading link conditions or prepares to leave the range of a cluster, the Leader synchronizes the beam parameters, link telemetry, and \gls{ai} model states with the next cluster, ensuring a seamless transition with minimal latency.

%The Space-RIC is a software-defined, distributed control system that extends the O-RAN RIC framework to satellite clusters, adapting it to the dynamic topology, constrained resources, and intermittent connectivity of space networks. Unlike terrestrial RICs, which rely on stable, high-bandwidth infrastructure, Space-RIC is designed for decentralized operation, enabling autonomous decision-making within satellite clusters while maintaining global coordination with terrestrial controllers.

%Each satellite runs an instance of the Space-RIC, but cluster-level coordination is handled by a dynamically elected Leader, which interfaces with the terrestrial non-RT RIC to align local operations with global directives. The Leader receives policies via A1 and O1 interfaces, processes real-time telemetry from E2, and distributes intra-cluster control commands. This ensures that all satellites within a cluster operate cohesively, adapting to traffic variations, link conditions, and resource constraints in near real-time. The Leader role is not static; it is assigned based on link stability, computational availability, and telemetry quality. In the event of failure or prolonged loss of connectivity, the FedE2 protocol triggers an autonomous re-election among the remaining satellites. Each satellite maintains a passive but synchronized control state, ensuring that leadership transitions occur seamlessly without service disruption.

%Space-RIC executes sApps, which are specifically designed for space-based O-RAN control, optimizing functions such as adaptive beamforming, dynamic spectrum reallocation, and power-aware scheduling. Unlike conventional terrestrial xApps, these applications are fully distributed, running across multiple satellites within a cluster and synchronized through the Leader. Each sApp instance processes local telemetry and periodically exchanges state information with its cluster peers via inter-satellite links. This ensures that updates such as beam adjustments and scheduling decisions are made with cluster-wide awareness, avoiding conflicts and maintaining stability. The Leader coordinates these processes but does not act as a central point of failure; if it goes offline, the remaining satellites continue executing distributed control autonomously until a new Leader is elected.

%When a satellite detects degrading service link conditions or prepares to exit a cluster’s operational boundary, the Leader initiates a predictive handover by synchronizing beam parameters, link telemetry, and AI model states with the next cluster. This ensures continuity with minimal latency, as the receiving satellite can immediately apply pre-computed adjustments upon taking over service responsibilities.



\subsection{Dynamic Link-Interface Mapping}
\label{subsec:link_mapping}

Dynamic link-interface mapping is a cornerstone of the architecture, addressing challenges in spectrum management and resource prioritization by enabling the adaptive assignment of O-RAN interfaces to specific link roles based on real-time network conditions. This flexibility is critical in the On-Board Compute-AI-Network Edge, where varying connectivity needs and diverse application demands require seamless coordination across network layers.

The system maps multiple O-RAN interfaces to a single physical link to support concurrent signaling and user-plane traffic. For example, an \gls{isl} can manage both the \gls{e2} interface for inter-satellite coordination within the \gls{spaceric} and the F1 interface for communication between the \gls{scu} and the \gls{sdu}, ensuring operational continuity across distributed satellite clusters. Similarly, \glspl{fl} and \glspl{sl} are dynamically allocated to prioritize critical services, such as emergency response or high-priority \gls{iot} traffic, using low-latency links like \glspl{isl}. Conversely, non-critical bulk data is routed through higher-latency paths, preserving premium resources for time-sensitive operations.

%This mapping strategy enhances spectral efficiency and resource utilization, seamlessly adapting to evolving network demands while ensuring that critical applications maintain priority. 
Table~\ref{tab:interface_mapping_flexibility} highlights the dynamical allocation capabilities, illustrating the optimized use of available links for balanced performance across the network.


% \subsection{Propagation Delay and Real-Time Control}

% Efficient management of propagation delay is fundamental for enabling real-time operations in \glspl{ntn}. \glspl{leo} constellations face frequent handovers and dynamic connectivity, which requires the use of low latency \glspl{isl} to ensure that delays remain below 50ms, as demonstrated in Fig.\ref{fig:isl_latency}. These links serve as the backbone for latency-sensitive tasks by leveraging the \gls{e2} interface for real-time signaling and satellite coordination.

% The results, depicted in Fig.\ref{fig:g2s_latency}, reveal that \glspl{isl} achieve consistent sub-10ms delays, enabling time-critical tasks such as adaptive beamforming, interference mitigation, and high priority traffic management. In contrast, \glspl{gsl} exhibit variable delays ranging from 10ms to over 40ms based on satellite elevation. These characteristics are taken advantage of the architecture's functional split strategies: \glspl{isl} are prioritized for real-time intra-cluster operations that require low latency, while \glspl{gsl} are reserved for non-critical tasks such as policy updates, model synchronization, and bulk data transfers to the terrestrial cloud. By integrating telemetry from distributed satellites and directives from the coordination level, \gls{spaceric} dynamically orchestrates operations through \glspl{sapp} and \glspl{dapp}. Parameters such as beam patterns are fine-tuned in real time, informed by on-board \gls{ai} inference models and closed-loop feedback mechanisms. These adjustments ensure that delay-sensitive operations align with global network objectives while adapting to localized conditions.


%\hl{Explicitly map out pipelines that would support various AI-related tasks, from training to inference. Either here as an architectural component or as a standalone subsection}
\subsection{AI Pipelines and Digital Twin}
The architecture integrates tailored \gls{ai} pipelines and \glspl{dt} to enhance adaptability and decision-making across terrestrial and non-terrestrial domains. Satellites locally preprocess telemetry data, including \glspl{kpi} such as latency and throughput, to reduce bandwidth demands before transmitting them to the terrestrial cloud.

The terrestrial cloud hosts \glspl{dt}, which simulate network states and inform global policy optimization. Federated learning enables distributed model updates, leveraging localized telemetry to refine global models while maintaining consistency across the constellation. Trained models are disseminated hierarchically to cluster leaders, which adapt and coordinate updates within their clusters based on real-time telemetry and localized needs. Retraining and iterative refinement occur through standardized feedback loops ensuring models remain effective under dynamic conditions. Real-time applications, such as adaptive beamforming, leverage these models onboard satellites, while computationally intensive processes are handled in the cloud. 


\subsection{Security and Interoperability}


The framework incorporates advanced security mechanisms to address threats associated with the integration of terrestrial and \gls{ntn} networks. While quantum-resistant cryptographic protocols represent a forward-looking approach, their feasibility in resource-constrained satellite environments remains subject to further evaluation. In the interim, lightweight encryption mechanisms secure \gls{isl} and \gls{gsl} communication, ensuring data integrity and confidentiality. Multi-factor authentication restricts unauthorized satellites from assuming critical roles such as cluster leadership, while the \gls{fede2} protocol enforces cryptographic safeguards to prevent unauthorized leader elections and preserve cluster coordination integrity.

Ephemeral key generation enhances security for transient data exchanges during link disruptions, mitigating risks of interception and replay attacks. Session-based authentication, combined with temporary token caching, enables rapid reestablishment of secure links during satellite handovers, reducing authentication latency while maintaining network resilience. Preauthentication mechanisms further optimize link reliability in dynamic environments, while automated token rotation and expiration policies mitigate credential compromise risks. Standardized O-RAN interfaces ensure secure interoperability, supporting scalable and resilient NTN-TN integration.


\section{Future Outlook}
\label{sec:future}

%Future integrated networks must address the challenges of scalability and autonomy posed by dynamic and resource-constrained environments. Autonomous self-healing and decentralized resource coordination will be critical for \glspl{ntn}, particularly for deep-space and extraterrestrial missions. Protocols such as \gls{hdtn} and the Bundle Protocol \cite{dudukovich2024advances} can mitigate latency and connectivity disruptions in these scenarios, enabling reliable operations independent of Earth-based communication. Advances in \gls{ai} and \gls{ml}, including distributed inference, federated learning, and model compression, will optimize satellite operations by reducing bandwidth and computational overhead. 

%As constellations expand, modular design principles are essential for the scalability and integration of planetary exploration nodes. This flexibility could enable applications like autonomous scientific research or precise space environmental monitoring. Integrated communication and sensing capabilities should simplify data processing by shifting resource-heavy tasks to ground centers. Additionally, using the THz spectrum for \glspl{isl} and terrestrial links \cite{alqaraghuli2023road} offers promise for ultra-high-throughput communication but faces challenges such as atmospheric attenuation. Dynamic spectrum management mechanisms within the proposed framework can address these by switching between THz and robust bands (e.g., Ka or Ku) based on real-time conditions. Addressing satellite power and processing limitations, improving network layer synchronization, and advancing standardization are crucial next steps for efficient and reliable systems.


Future integrated networks must support scalability, autonomy, and efficient resource utilization in highly dynamic and constrained environments. Autonomous self-healing mechanisms and decentralized coordination will be essential for \glspl{ntn}, particularly in deep-space exploration and extraterrestrial missions. Protocols such as \gls{hdtn} and the Bundle Protocol \cite{dudukovich2024advances} can enhance connectivity resilience, reducing dependency on Earth-based infrastructure. Advances in \gls{ai} and \gls{ml}, including distributed inference, federated learning, and model compression, will improve in-orbit decision-making by optimizing bandwidth use and computational efficiency.

Beyond planetary exploration, satellite-to-satellite services are expected to play a critical role in future space-based infrastructure. Applications such as in-orbit data relay, autonomous servicing, coordinated sensing, and distributed space computing will require high-throughput, low-latency interconnectivity. These capabilities will be particularly relevant for private space stations, orbital research facilities, and future space-based industrial operations, where seamless integration will be necessary to support real-time control and high-volume data exchanges.

Expanding \gls{ntn} will require \glspl{isl} with ultra-high capacity and adaptive spectrum management to support scalable and reliable communication. While the THz spectrum presents challenges such as atmospheric attenuation and hardware constraints, ongoing research is steadily addressing these limitations, making its integration into future \glspl{ntn} increasingly feasible \cite{alqaraghuli2023road}.


\section{Conclusions}
\label{sec:conclusions}

The proposed Space-O-RAN architecture integrates satellite and terrestrial networks, addressing key challenges such as dynamic connectivity, resource constraints, and scalability. By leveraging 
 open \gls{ran} principles and standardized interfaces, it enables hierarchical control across strategic, coordination, and operational levels, ensuring seamless network alignment. 


A key contribution is the dynamic link-interface mapping mechanism, which optimizes resource utilization by adapting O-RAN interfaces to the varying roles of onboard links, mitigating NTN-specific challenges such as intermittent connectivity and latency variations. The \gls{spaceric} framework further enhances cluster-level coordination, supporting scalable operations and efficient resource sharing. 

These advancements unlock new 6G applications, including real-time inter-satellite coordination for disaster response and federated AI for distributed intelligence. Simulations validate the feasibility showing that intra-cluster coordination operates within latency bounds while keeping non-real-time tasks. The framework effectively balances real-time decision making with cloud-assisted optimization. Future work will refine AI-driven orchestration and optimize onboard processing to enhance scalability and resilience in NTN-TN integration.



\bibliographystyle{IEEEtran}
\bibliography{biblio}

\vskip -3\baselineskip plus -1fil



\begin{IEEEbiographynophoto}{Eduardo Baena} is a postdoctoral research fellow at Northeastern University, with a Ph.D. in Telecommunication Engineering from the University of Malaga. His experience includes roles in the international private sector (2010–2017) and as Co-PI in several research projects.
\end{IEEEbiographynophoto}

\vskip -2\baselineskip plus -1fil

\begin{IEEEbiographynophoto}{Paolo Testolina} received a Ph.D. in information engineering from the University of Padova in 2022. He is a Post-Doctoral Researcher at Northeastern University, with research interests in mmWave and sub-THz networks, including channel modeling, physical layer simulation, Non-Terrestrial Networks and spectrum sharing and coexistence.
\end{IEEEbiographynophoto}

\vskip -2\baselineskip plus -1fil

\begin{IEEEbiographynophoto}{Michele Polese} is a Research Assistant Professor at Northeastern University’s Institute for the Wireless Internet of Things. He received his Ph.D. from the University of Padova in 2020 and focuses on protocols for 5G and beyond, mmWave/THz networks, spectrum sharing, and open RAN development.
\end{IEEEbiographynophoto}

\vskip -2\baselineskip plus -1fil

\begin{IEEEbiographynophoto}{Dimitrios Koutsonikolas} is an Associate Professor in the department of Electrical and Computer Engineering and a member of the Institute for the Wireless Internet of Things at Northeastern University. He specializes in experimental wireless networking and mobile computing, with a focus on 5G/NextG networks and applications. He is a recipient of the IEEE Region 1 Technological Innovation Award and the NSF CAREER Award.
\end{IEEEbiographynophoto}

\vskip -2\baselineskip plus -1fil

\begin{IEEEbiographynophoto}{Josep M. Jornet} (M'13--SM'20--F'24) is a Professor and Associate Director of the Institute for the Wireless Internet of Things at Northeastern University. His research focuses on terahertz communications and networking, pioneering advancements in nanoscale communications and wireless systems. He has authored over 250 peer-reviewed publications and holds five US patents.
\end{IEEEbiographynophoto}

\vskip -2\baselineskip plus -1fil

\begin{IEEEbiographynophoto}{Tommaso Melodia} is the William Lincoln Smith Chair Professor at Northeastern University and Founding Director of the Institute for the Wireless Internet of Things. A recipient of the NSF CAREER award, he has served as an Associate Editor for leading IEEE journals.
\end{IEEEbiographynophoto}

\end{document}
