\section{Background and Related Work}
\label{sec:background}

Existing integrated terrestrial-\gls{ntn} architectures, such as \gls{sagin}~\cite{cui2022architecture}, focus primarily on user-plane solutions and lack closed-loop control mechanisms. While \gls{sagin} envisions seamless \gls{ntn}-terrestrial integration, its practical implementation remains limited by dynamic link variability, intermittent connectivity, and latency constraints. These factors make efficient real-time control loops difficult to implement, particularly when relying on existing \gls{ai}-driven approaches, which struggle with inconsistent data availability and computational constraints.

The integration of \gls{ai} into \glspl{ntn} faces significant challenges. Limited onboard resources restrict the execution of complex learning algorithms, while intermittent satellite connectivity and long propagation delays disrupt data collection, model training, and inference processes. Current solutions often fail to address real-time resource allocation, adaptive control, and precise timing synchronization, leaving interoperability between orbital and terrestrial layers largely conceptual, with no standardized implementations for protocol or interface alignment \cite{10716597}.

Recent research on \gls{ntn}-terrestrial integration has identified key trade-offs in \gls{ran} architectures. Lower-layer functional splits reduce latency but impose strict timing constraints that are difficult to meet in high-latency \gls{ntn} environments. Centralized architectures, while easier to implement, introduce bandwidth inefficiencies over feeder links \cite{Muro2024}. Modular architectures that leverage standardized interfaces, such as near-RT and non-RT \glspl{ric}, show potential for adaptive resource management and real-time control. However, they remain constrained by scalability issues, high latency, and the complexity of ensuring seamless handovers between satellite clusters \cite{oranntn2025}. Addressing these limitations requires innovations that balance distributed decision-making, adaptive resource allocation, and efficient multi-domain integration.

On the industry side, \gls{3gpp} has made significant progress in interoperability with \gls{ntn}. Release 17 introduced extensions to \gls{5g} \gls{nr} for direct-to-device satellite connectivity, while Release 18 builds on these enhancements with dynamic handovers and spectrum sharing.  ITU-T Y.3207 (2024) defines INCA for Fixed, Mobile, and Satellite Convergence (FMSC), proposing a centralized control framework, but its feasibility in dynamic \gls{ntn} environments remains uncertain due to latency, overhead, and computational constraints. These developments provide a foundation for integration, but remain insufficient without additional architectural enhancements to address specific constraints in \gls{ntn}, such as real-time coordination, cross-layer control, and resource-aware \gls{ai} processing.
%The lack of closed-loop control procedures remains a major limitation in state-of-the-art approaches like \gls{sagin}~\cite{cui2022architecture}, which provide only conceptual models for integrated architectures and focus primarily on user plane solutions. While \gls{sagin} frameworks envision seamless integration of terrestrial and \glspl{ntn}, their practical implementation is hindered by the dynamic nature of \gls{ntn} links, intermittent connectivity, and significant latency. These challenges complicate the design of stable and efficient control loops and are exacerbated by the limitations of existing \gls{ai}-driven approaches.

%The incorporation of \gls{ai} technologies into \glspl{ntn} faces significant hurdles, as limited satellite resources constrain the deployment of complex algorithms, while intermittent connectivity and long propagation delays disrupt data collection and training cycles. Moreover, current solutions frequently do not succeed in address critical operational requirements, including real-time resource allocation, adaptive control, and precise timing synchronization. Furthermore, interoperability between orbital and terrestrial layers remains largely conceptual, with no concrete implementations for protocol or interface alignment \cite{10716597}.

%Recent efforts to integrate \glspl{ntn} with terrestrial networks have highlighted critical trade-offs in \gls{ran} architectures. Lower-layer splits reduce latency, but face strict delay constraints that are difficult to meet in high-latency \gls{ntn} environments, while centralized approaches suffer from bandwidth inefficiencies over feeder links \cite{seeram2024feasibilitystudyfunctionsplits}. Modular architectures that leverage standardized interfaces like near-RT and non-RT \glspl{ric} show potential for adaptive resource management and real-time control, but are challenged by scalability, high latency, and seamless handovers \cite{oranntn2025}. Addressing these limitations requires innovations that balance distributed decision-making, adaptive resource allocation, and efficient integration across network segments.

%On the industry side, \gls{3gpp} standardization has significantly advanced \gls{ntn} interoperability. Release 17 introduced extensions to \gls{5g} \gls{nr} for direct-to-cell connectivity, and Release 18 builds on these developments by incorporating dynamic handovers and spectrum sharing. These provide a solid foundation for the bridge between terrestrial and satellite networks, but remain insufficient without additional architectural enhancements to address NTN-specific constraints.

 \begin{figure*}[h]
    \centering
    % Combinar width, height, trim y clip
    \includegraphics[width=0.77\textwidth]{figures/arch.pdf}
    \captionsetup{singlelinecheck=false}
    \caption{The proposed framework coordinates on board Space-RIC (with s-Apps) with terrestrial SMO via flexible link mapping}
    \label{fig:oran_space_arch}
\end{figure*}
\vskip -2\baselineskip plus -1fil