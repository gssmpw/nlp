\documentclass[journal]{IEEEtran}
\IEEEoverridecommandlockouts
% The preceding line is only needed to identify funding in the first footnote. If that is unneeded, please comment it out.
\usepackage{cite}
\usepackage{amsmath,amssymb,amsfonts}
\usepackage{algorithmic}
\usepackage{graphicx}
\usepackage{textcomp}
\usepackage{xcolor}
\usepackage{multirow}
\usepackage{graphicx}
\usepackage{array}
\usepackage{caption}
\usepackage{booktabs}
\usepackage{multirow}

\usepackage{tikz}
\usetikzlibrary{fit}
\makeatletter
\tikzset{
  fitting node/.style={
    inner sep=0pt,
    fill=none,
    draw=none,
    reset transform,
    fit={(\pgf@pathminx,\pgf@pathminy) (\pgf@pathmaxx,\pgf@pathmaxy)}
  },
  reset transform/.code={\pgftransformreset}
}
\makeatother

\usepackage{pgfplots}
\pgfplotsset{compat=newest} 
\pgfplotsset{plot coordinates/math parser=false} 

\newlength\fheight
\setlength{\fheight}{0.5\columnwidth}
\newlength\fwidth
\setlength{\fwidth}{0.8\columnwidth}

\def\BibTeX{{\rm B\kern-.05em{\sc i\kern-.025em b}\kern-.08em
    T\kern-.1667em\lower.7ex\hbox{E}\kern-.125emX}}
\begin{document}



%\title{Integrating O-RAN with LEO Satellite Networks: A Path to Unified Non-Terrestrial and Terrestrial Communications in 6G

%\title{ASTRO-RAN: A Path to Unified Space and Terrestrial Communications in 6G

\title{O-RAN-in-Space: Enabling Intelligent, Open, and Interoperable Non Terrestrial Networks in 6G

%\title{Integrating O-RAN with Non-Terrestrial Networks: A Path to Unified Space and Terrestrial Communications in 6G


\thanks{Identify applicable funding agency here. If none, delete this.}
}



\author{\IEEEauthorblockN{
Eduardo Baena\IEEEauthorrefmark{1}, Paolo Testolina\IEEEauthorrefmark{1}, 
Michele Polese\IEEEauthorrefmark{1}, 
 Dimitrios Koutsonikolas\IEEEauthorrefmark{1}, Josep Jornet\IEEEauthorrefmark{1}, 
%Salvatore D'Oro\IEEEauthorrefmark{1},
Tommaso Melodia\IEEEauthorrefmark{1}
}

\IEEEauthorblockA{\IEEEauthorrefmark{1}Institute for the Wireless Internet of Things, Northeastern University, Boston, MA, U.S.A.}

}

\maketitle

\begin{abstract}
In this paper, we present an innovative architecture that integrates Open Radio Access Network principles leveraged by O-RAN ALLIANCE with Non-Terrestrial Networks (NTNs), fostering intelligent, open, and interoperable space-terrestrial communications for the 6G era. The architecture encompasses a wide range of NTN components, including satellites in Low Earth Orbit (LEO), Medium Earth Orbit (MEO), Geostationary Orbit (GEO), and High Altitude Platform Stations (HAPS). By leveraging open interfaces, virtualization, and AI-driven optimization, this framework unifies resource management and orchestration across heterogeneous space and aerial networks, creating a cohesive and adaptable network ecosystem.

The architecture enables dynamic spectrum management, automated configuration, and AI-enhanced service delivery through a federated approach that includes both ground-based and on-board elements, as well as flexible xApps and dApps deployed across satellite constellations. A federated satellite RAN Intelligent Controller (RIC) supports real-time and strategic optimizations across satellite nodes, while digital twin technology allows for cost-effective testing and predictive maintenance. By integrating large-scale and local AI models with advanced resource allocation addressing current NTN limitations. 

\end{abstract}

\begin{IEEEkeywords}
component, formatting, style, styling, insert
\end{IEEEkeywords}

\section{Introduction}


The rise of space-based and aerial communication technologies has created an interconnected ecosystem of non-terrestrial networks (NTNs) that complements and extends terrestrial connectivity. This ecosystem includes satellites in Low Earth Orbit (LEO), Medium Earth Orbit (MEO), Geostationary Orbit (GEO), and emerging infrastructures for deep space communications. Major players like SpaceX and Amazon’s Project Kuiper are deploying vast constellations of LEO satellites to address connectivity gaps in remote and underserved regions, demonstrating capabilities that traditional cellular networks cannot achieve \cite{abdelsadek2023future}. However, NTNs face substantial challenges in scaling to meet the rigorous demands of 6G integration \cite{mahboob2023revolutionizing}. 

A critical limitation lies in the discontinuous nature of NTN communications, which can result in underutilization of onboard resources and limited ability to offload vast amounts of collected data to terrestrial stations \cite{ren2023review}. This discontinuity, present across multiple orbital layers, poses significant constraints on real-time responsiveness.

Additionally, orchestrating thousands of interconnected satellites across multiple orbits requires a level of operational control far beyond what traditional networks support \cite{ ma2024leo}. The variability in signal propagation, coexistence of frequency bands, and potential interference from terrestrial and other space-based systems add further complexity \cite{beyaz2024satellite}. Moreover, the scarcity of spectrum resources for NTNs, despite new allocations, imposes additional constraints. Each frequency band has inherent limitations due to propagation characteristics and interference risk, particularly when shared with terrestrial networks \cite{jia2020virtual}. 

Computational constraints onboard satellites further complicate the deployment of advanced services. While Artificial Intelligence (AI) techniques have proven effective for optimizing terrestrial networks, the strict power, thermal, and spatial limitations of satellite hardware restrict their use in space. Implementing advanced AI-driven applications requires significant adaptation to meet the constraints of spaceborne systems \cite{ren2023review}, often demanding reductions in model size and complexity to remain feasible for satellite deployment. Current AI algorithms frequently demand more processing power than is practical for most satellites, limiting the scope of real-time decision-making and adaptability onboard.

Beyond computational challenges, interoperability remains a key barrier to effective NTN and terrestrial network integration. Unlike terrestrial systems, where standardized protocols facilitate seamless handovers and coordination, NTNs operate across diverse constellations, orbital layers, and frequency bands. This fragmentation results in a lack of unified mechanisms for effective, secure inter-satellite and satellite-to-ground communication, hindering the formation of a cohesive, globally integrated network infrastructure \cite{mahboob2023revolutionizing}.

To overcome these challenges, we introduce an integrated Open-Space-Integrated-RAN (OSIRAN) architecture. This design offers coordinated, closed-loop control across multiple timescales to meet NTNs’ dynamic demands by treating satellites as adaptive edge computing nodes within a scalable, interoperable framework. Through distributed RAN Intelligent Controllers (RICs) on satellites and their integration with terrestrial infrastructure, OSIRAN enables real-time adaptations, optimized resource distribution, and AI-enhanced service delivery across space and terrestrial networks.

The paper is structured as follows. Section~\ref{sec:background}  provides a detailed background on NTN architectures, discussing key challenges in 6G integration. Section~\ref{sec:challenges}  highlights specific use cases that underscore the need for an NTN-specific architecture. Section~\ref{sec:architecture}  presents the architecture, detailing its structure and closed-loop control approach. Section~\ref{sec:feasibility} discusses performance evaluations in terms of latency and resource optimization. Lastly, Section~\ref{sec:future}  concludes with future perspectives.

\section{Background and Related Work}
\label{sec:background}

The integration of terrestrial and non-terrestrial network (TN-NTN) infrastructures is a critical advancement in modern telecommunications, enabling seamless, high-speed global connectivity. Known as Space-Air-Ground Integrated Networks (SAGIN), this convergence unifies satellite, aerial, and ground networks to address rising demands for ubiquitous data access. Industry-standardization efforts, including 3GPP Releases 17 and 18 and the O-RAN Next Generation Research Group’s 6G roadmap \cite{3gpp_ntn}, lay the groundwork for this integration, highlighting interoperable and flexible interfaces essential for TN-NTN convergence.

AI has emerged as a powerful enabler for enhancing TN-NTN operations. Recent studies underscore AI-driven optimization for resource allocation, emphasizing its role in dynamically adapting network conditions to improve efficiency. However, current research often overlooks multi-layered synchronization across TN and NTN segments, a key element for true end-to-end service coherence \cite{iqbal2023ai, 10716597}. AI’s potential to create resilient, service-oriented architectures has also been explored within NTN contexts, though gaps remain in achieving fully dynamic adaptability across the inherently evolving topologies of non-terrestrial networks \cite{10716597, cheng2022service,ray2022review}.

Further advancements focus on technological solutions to support TN-NTN backhaul and edge processing. The potential of Integrated Access Backhaul (IAB) technology has been evaluated within 5G networks to enhance NTN backhaul; however, this approach primarily addresses satellite-ground links, without extending orchestration across SAGIN layers \cite{pugliese2024iab}. Similarly, Mobile Edge Computing (MEC) has been proposed to reduce latency through edge processing, though it alone may not sufficiently address the frequent connectivity disruptions encountered in NTNs, underscoring the need for more comprehensive, persistent connectivity solutions \cite{qiu2022mec}.

Digital Twin (DT) technology offers a promising approach for resource management within NTNs, enabling real-time simulations and AI-driven analytics for dynamic resource optimization \cite{DTNTN}. While DT frameworks enhance adaptive resource management, challenges persist in maintaining computational efficiency and interoperability in real-time DT environments, especially in the context of diverse NTN requirements.

While these efforts collectively advance the feasibility of TN-NTN integration, limitations such as a lack of adaptive interoperability, real-time synchronization, and resilient connectivity in dynamic NTN scenarios persist. Addressing these challenges requires the development of an interoperable, scalable architecture that integrates flexible resource management, AI-driven intelligence, and digital twin capabilities to achieve cohesive TN-NTN integration and realize the full potential of future networks.




\section{Challenges and Use cases}
\label{sec:challenges}

As it has been mentioned, future NTNs will need to cope with reliable and adaptive operations in high-mobility, resource-limited environments. Advanced architectures are essential to meet these demands and ensure seamless global connectivity. The following sections dive into key challenges and use cases defining this need.

\subsection{Resilience to Link Disruptions}

Ensuring resilient NTNs that maintain high service quality in the face of frequent link disruptions is essential as satellite constellations scale in size and complexity. Unlike terrestrial networks, which typically benefit from stable infrastructure and consistent connectivity, NTNs—particularly in LEO—must contend with the constant movement of satellites, resulting in regular interruptions in connectivity both with ground stations and within the constellation. A robust architecture that supports dynamic link reconfiguration and adaptive resource allocation across multiple timescales is therefore vital to sustaining seamless service in NTNs.

\textbf{Use Case: Autonomous Disaster Response and Critical Infrastructure Monitoring}

In scenarios such as natural disasters or emergencies, NTNs play a pivotal role by enabling rapid communication across affected regions, where terrestrial infrastructure is either compromised or entirely absent. A resilient NTN architecture must support autonomous, real-time reconfiguration of its links and resources to ensure that connectivity remains unbroken, especially under the unpredictable conditions characteristic of disaster-stricken areas. When ground stations are unavailable, or when certain satellite links degrade due to environmental or orbital dynamics, the network must be able to seamlessly reroute traffic through alternative paths in the constellation, such as ISL or dynamically reassigning link roles between feeder and user connections.

In this context, maintaining service quality and reliability is critical, as the network must support data-intensive applications like live video feeds for situational awareness, remote command-and-control operations, and IoT-based environmental monitoring. The architecture's adaptive link management capabilities allow it to autonomously prioritize high-priority data traffic, minimize latency through optimized path selection, and manage congestion, ensuring uninterrupted service for emergency responders. 


\subsection{Multi-Scale Closed Earth-Space Control Loop}

A multi-scale closed-loop control system tailored for NTNs provides dynamic adaptability across various temporal layers, crucial for managing the high-mobility and distributed nature of satellite networks. Unlike terrestrial networks, NTNs demand autonomous, real-time control at the satellite level to minimize latency and ensure rapid reconfiguration without relying on Earth-based feedback. Immediate adjustments, such as link reallocation and interference suppression, are executed on board to uphold connectivity in response to changing link conditions. For strategic tasks requiring more computation, such as resource optimization and predictive analytics, processing is delegated to terrestrial systems, where greater processing power and historical data enhance decision-making. This layered, responsive control enables NTNs to sustain reliable connectivity even in remote and unpredictable environments, achieving a flexibility that surpasses conventional terrestrial systems.

.

.
.

\textbf{Use Case: QoS-aware, Sliced Private Virtual NTNs}

The closed-loop control enables NTNs to adapt satellite resources with precision, delivering differentiated QoS across network slices that respond to the unique demands of each use case. Each NTN slice can autonomously instantiate and scale virtualized network functions based on the specific signaling and user traffic requirements of its application, adapting both link assignments and signaling pathways in real-time. This flexibility is essential in NTN settings, where constraints such as satellite mobility, latency variations, and intermittent link availability necessitate agile reconfiguration to maintain consistent service quality.

For latency-sensitive slices, such as those required for remote healthcare or autonomous monitoring, the closed-loop prioritizes high-speed ISL and feeder links, dynamically adjusting link roles in milliseconds to counteract connectivity disruptions. These slices benefit from near-instant adaptation, with signaling and link resources recalibrated on demand to meet stringent latency requirements, ensuring reliable service for critical applications.

In contrast, applications with more lenient latency needs, such as global IoT sensor networks, are configured to utilize lower-priority links or schedule transmissions during off-peak times. The control loop intelligently routes these slices through secondary or less congested ISLs, freeing high-priority resources for mission-critical applications. For private NTN networks, on-demand slices can be created with customized signaling configurations and security protocols, tailored to enterprise or government specifications, providing secure, isolated access. 





\subsection{Distributed Computing and Task Offloading}

In NTNs, the computational constraints of satellite platforms pose a challenge to deploying AI-driven tasks essential for advanced joint communication and sensing capabilities. Tasks such as interference prediction, adaptive beamforming, and joint communication-sensing optimizations are computationally intensive and often require significant data handling, particularly when involving raw IQ data. Without a robust mechanism to balance computational load and data flow across both space and terrestrial infrastructure, NTNs risk overloading their limited link capacity and onboard resources.

An integrated computer-communication resource-aware framework mitigates these challenges by enabling NTNs to offload demanding tasks, such as AI model training and data aggregation, to terrestrial data centers equipped with high-performance processing capabilities. This offloading allows satellites to perform localized inference tasks onboard, while sending select, critical data or model updates to ground-based systems where more intensive computations can occur. Federated learning within this framework enables satellites to contribute local model updates without transferring vast quantities of raw data, preserving link capacity and improving model accuracy over time.

\textbf{Use Case: Federated AI-Driven Optimization for Joint Communication and Sensing}

The federated AI approach facilitates the efficient deployment of AI models across NTNs for optimized joint communication and sensing. Each satellite node performs real-time inference locally, updating its AI models based on the latest ground-driven improvements while reserving bandwidth by only transmitting essential model updates. This federated structure enhances real-time applications like environmental monitoring, where satellites collaboratively detect and analyze patterns without congesting the network. Satellites share insights with a ground-based digital twin model, which aggregates multi-node data to adjust resource allocation, optimize beamforming, and predict interference patterns across the network.

In dynamically changing environments such as disaster response or remote surveillance, NTNs benefit from adaptive offloading strategies that adjust AI model training and deployment based on network conditions. When link quality allows, satellites offload data to ground systems for full-scale model updates. In scenarios with restricted connectivity, the framework limits updates to high-priority information, preserving critical resources. This flexible AI deployment and offloading strategy, responsive to available processing power and network conditions, enables NTNs to support high-demand applications with efficient communication and sensing, ensuring robust, scalable performance for diverse next-generation use cases.




\begin{figure*}[h]
    \centering
    \includegraphics[width=0.81\linewidth]{figures/architecture.pdf}
    \captionsetup{justification=centering}
    \caption{OSIRAN Framework.}
    \label{fig:oran_space_arch}
\end{figure*}

\section{Proposed Solution: OSIRAN}
\label{sec:architecture}

The OSIRAN architecture is designed to address the challenges of NTNs by providing a robust and adaptable framework that integrates terrestrial cloud resources with satellite-based edge computing. This architecture operates on multiple layers, ensuring that network functions can be deployed dynamically across the Earth and space segments to meet real-time and long-term demands without relying solely on terrestrial connectivity. 

\subsection{Core Components}


\textbf{Terrestrial Cloud:} The terrestrial cloud functions as the centralized decision-making layer, handling long-term strategies and resource-intensive processing that exceed the constraints of satellite hardware. It hosts the non-real-time RIC within the SMO layer, where AI-driven optimization, digital twin maintenance, and overarching spectrum and policy management are executed. With ample computational resources, the cloud manages tasks such as model training, predictive analytics, and network-wide configuration, streamlining satellite network performance while preserving onboard resources. This centralized control coordinates high-level network configurations, which are securely transmitted to the satellite network, supporting a seamless integration between terrestrial and space layers.

\textbf{On-Board Edge:} The space-based edge architecture is the operational core of each satellite, enabling near-real-time network responsiveness. The on-board edge includes the Space-RIC, which executes localized decision-making and control processes, significantly reducing latency and ensuring the network can react swiftly to in-orbit changes. Each Space-RIC hosts specialized sApps (including xApps and dApps) that enable autonomous handling of critical functions like beamforming, spectral resource management, and link reconfiguration. These sApps are precisely optimized for the dynamic conditions of NTNs, executing in-situ to minimize dependency on terrestrial feedback and supporting real-time service continuity.

\textbf{Hierarchical Leader-Follower Configuration:} Within each satellite cluster, a leader-follower arrangement is established to distribute network responsibilities and maintain resilience. The leader satellite coordinates resource allocation, handovers, and link optimization, using real-time metrics collected from follower satellites. Follower satellites, in turn, perform local adjustments and relay performance metrics to the leader, enabling the cluster to function autonomously. If the leader satellite experiences a loss of connectivity or fails, leadership shifts dynamically to one of the follower satellites, ensuring uninterrupted operations without necessitating terrestrial intervention. 

\textbf{Hardware Layer and Virtualization:} The hardware layer onboard each satellite integrates CPUs, GPUs, and FPGAs, which meet the computational demands of the Space-RIC and its applications. Managed by an OS middleware, these hardware resources are dynamically allocated based on active network demands, supporting tasks such as RF management, sensing, and adaptive beamforming. Additionally, a virtualization layer allows flexible deployment of several virtual network functions, optimizing network slicing and enhancing the satellite’s capacity to handle real-time data processing and control functions within the onboard edge.


\subsection{Dynamic Link-Interface Mapping}

A key feature of the architecture is its flexible mapping of interfaces to physical links within the satellite network, adapting to network function and priority requirements in real-time. This adaptability is crucial in NTNs, where physical links often need to support multiple roles due to dynamic conditions. For example, an ISL can support the E2 interface for signaling and coordination between space-RICs, while simultaneously facilitating F1 communication for DU-CU connections if network functions are distributed across satellites. This dual-purpose configuration enhances resilience and continuity, adjusting seamlessly to link variability and service demands.

Similarly, the User and Feeder Links can dynamically accommodate both signaling and user traffic based on priority and latency requirements. High-priority slices with stringent latency demands can leverage low-latency ISLs or User Links, while lower-priority data traffic can be directed to alternate paths, optimizing resource use across the network. This approach allows NTNs to provide slice-specific QoS tailored precisely to the latency and reliability requirements of each application—from critical, low-latency services to best-effort data collection. Table~\ref{tab:interface_mapping_flexibility} outlines the primary and adaptive roles of each link in the network's interface mapping strategy.


\begin{table*}[h]
\centering
\scriptsize % Reduce font size
\setlength{\tabcolsep}{2pt} % Reduce space between columns
\caption{Dynamic Interface-Link Mapping}
\label{tab:interface_mapping_flexibility}
\begin{tabular}{|c|p{3cm}|p{4cm}|p{4cm}|p{4.5cm}|} % Adjust column widths to fit the page
\hline
\textbf{Link Type} & \textbf{Primary Function} & \textbf{ORAN Interfaces Supported} & \textbf{Primary OSIRAN Strategy} & \textbf{Dynamic Re-mapping Adaptability} \\
\hline
User Link (UL) & Connects satellite to end-user devices & F1 (User plane), E2 (Control plane) & Primarily supports direct user data transmission with rerouting through ISL in case of disruption & Can temporarily re-map to feeder link functions for signaling and configuration when feeder connectivity is lost, providing essential network stability during outages \\
\hline
Feeder Link (FL) & Connects satellite to ground stations for backhaul and signaling & A1 (Policy/Config), E2 (Coordination) & Manages backhaul and policy updates, offloading non-real-time tasks to the terrestrial network & Adapts to support user plane functions like F1 or even ISL-like coordination for critical slices needing prioritized backhaul during peak demand \\
\hline
Inter-Satellite Link (ISL) & Connects satellites within constellations for distributed coordination & E2 (Intra-satellite signaling), F1 (Distributed CU-DU), A1 (Policy sync) & Primarily supports intra-satellite control and user-plane data sharing for continuity across constellation & Dynamic support for feeder link backhaul if feeder is unavailable, providing temporary terrestrial uplink roles; can also offload user data if UL is congested or down \\
\hline
\end{tabular}
\end{table*}

\subsection{Hierarchical r/s/dApp Coordination}

This architecture leverages a hierarchical coordination structure spanning from terrestrial cloud resources to satellite-based edge systems, where the orchestration of rApps, xApps, and dApps provides a seamless control loop tailored to the dynamic needs of NTNs. Figure~\ref{fig:spaceRIC} illustrates the workflow for a beamforming application in this structured framework, where each layer, from ground to satellite, operates in concert to deliver real-time adaptability.

At the terrestrial level, the SMO layer hosts rApps that perform high-level, non-real-time functions like long-term spectrum management, digital twin modeling, and policy generation based on AI-driven analytics. These rApps process large-scale historical data and develop strategic directives, which are then transmitted to the SpaceRIC Leader via the A1 interface for real-time operational integration.

The SpaceRIC Leader, situated on a primary satellite, receives these high-level directives and combines them with real-time telemetry from within the constellation to coordinate adjustments and resource allocation. It directly manages xApps, which handle immediate tasks such as spectrum reallocation, beamforming, and handover management, processing both SMO directives and data from neighboring satellites to maintain optimal performance. This localized management at the edge allows for latency-sensitive operations and efficient response to fast-changing conditions in orbit without the delays associated with ground control.

Within each satellite cluster, the SpaceRIC Leader communicates with SpaceRIC Followers through inter-satellite links, using E2 for control signaling and other interfaces as needed. Followers, equipped with dApps, execute specific monitoring and adaptation tasks, like beam switching or interference mitigation, based on metrics such as spectral efficiency and signal-to-noise ratios. These dApps adjust parameters in real-time, aligning with the Leader’s instructions and ensuring that each satellite optimally contributes to the cluster’s performance.

This multi-layered, hierarchical approach enables precise and efficient control across the constellation, maintaining a continuous feedback loop between the SMO, SpaceRIC Leader, and Followers. SpaceRIC Followers relay performance data back to the Leader, which aggregates it for ongoing dynamic optimization and sends it periodically to the SMO for strategic evaluation. This coordinated effort across applications and interfaces supports adaptive, resilient NTN operations in the face of changing link conditions and operational demands.

\begin{figure}[h]
    \centering
    \includegraphics[width=0.81\linewidth]{figures/SpaceRIC.pdf}
       \captionsetup{justification=centering}
    \caption{SpaceRIC workflow}
    \label{fig:spaceRIC}
\end{figure}

\subsection{Security and Interoperability}

To meet the high-security demands of NTNs, the architecture enforces strict encryption and secure protocols across all satellite links, both inter-satellite and satellite-to-ground. This setup ensures data integrity and safeguards against unauthorized access, addressing the vulnerability of non-terrestrial links to potential threats. The architecture also supports seamless interoperability with ORAN-compatible terrestrial networks, enabling flexible integration with existing 5G and future 6G systems. Through standard ORAN interfaces, it facilitates dynamic connectivity with ORAN-enabled infrastructures, including high-altitude platforms and aerial networks. This compatibility allows for secure, flexible resource sharing across terrestrial and non-terrestrial domains, enhancing the overall resilience and adaptability of the integrated network.

\subsection{Propagation Delay and Real-Time Control}

For latency-critical applications, inter-satellite links are prioritized to keep total propagation delays under a 50 ms threshold in LEO constellations, leveraging the E2 interface. This configuration reduces reliance on ground-based routing, which often incurs higher latency, thus supporting the low-delay requirements of real-time NTN applications. Applications like IoT monitoring and emergency response benefit directly from this low-latency setup, as do adaptive beamforming adjustments that require rapid response to shifting user demands. This setup not only minimizes service interruptions but also enhances responsiveness for dynamic applications, paving the way for a feasibility analysis on real-time control under various network conditions.


\section{Feasibility Study}
\label{sec:feasibility}

 
To evaluate latency performance, simulations were conducted to assess the proposed framework ability to meet stringent delay requirements by prioritizing inter-satellite links within a 50 ms threshold for LEO constellations, using the E2 interface. This approach substantially reduces dependency on terrestrial routing paths, which typically add significant delays. Real-time applications such as IoT monitoring and emergency response, as well as adaptive beamforming adjustments that require frequent recalibration, exhibit marked performance improvements due to this reduced latency. 

Figure~\ref{fig:latency} presents the simulation results, confirming that OSIRAN effectively meets the latency constraints necessary for reliable, real-time NTN applications.

\begin{figure}[h]
    \centering
    % \includegraphics[width=0.81\linewidth]{figures/latency.pdf}
    \setlength{\fheight}{0.4\columnwidth}
    \setlength{\fwidth}{0.8\columnwidth}
    \input{figures/latency_g2s.tikz}
       \captionsetup{justification=centering}
    \caption{Latency result}
    \label{fig:latency}
    % \vspace{-10pt}
\end{figure}

\section{Future Outlook}
\label{sec:future}
The proposed architecture addresses current NTN demands while also laying the groundwork for future, autonomous networks fit for deep-space missions and extraterrestrial environments. As NTNs extend beyond Earth’s orbit, autonomous features—such as self-healing capabilities and independent resource management—become essential for maintaining network functionality when delays prevent timely control from Earth. High-Delay Tolerant Networking (HDTN) protocols, using standards like the Bundle Protocol \cite{sabbagh2017bundle}, will play a crucial role in ensuring robust communication under high-latency, intermittent connectivity typical in deep-space environments.

To fully leverage OSIRAN in resource-limited satellite environments, future work will focus on optimizing network functions and AI models for constrained conditions. Techniques like model compression, quantization, and edge AI support adaptive applications that fit within onboard resource limits, while federated learning can enable collaborative model training across satellite nodes, conserving bandwidth and energy. These advancements would further OSIRAN's capacity to provide intelligent, adaptive services in both high-latency and resource-scarce settings.

Another promising avenue is enhancing joint communication and sensing. By dynamically offloading intensive sensing tasks—such as environmental monitoring, anomaly detection, and predictive analytics—OSIRAN allows for complex sensing without overwhelming onboard resources. Integrating sensing with communication processes fosters real-time analysis and enriches network intelligence, adapting the entire NTN system to evolving demands.

As NTNs grow to support dense data flows, THz spectrum management becomes increasingly vital. THz bands offer ultra-high-throughput potential but require dynamic, adaptive spectrum management due to atmospheric sensitivities. Incorporating intelligent spectrum allocation mechanisms within OSIRAN enables seamless transitions between THz and lower-frequency bands, optimizing throughput and ensuring robust operation across orbital environments.




\section{Conclusions}
\label{sec:conclusion}
This paper presents an open architecture tailored to address NTNs’ unique challenges in integrating with terrestrial 6G systems. Through decentralized control and task offloading, OSIRAN enhances network efficiency across multiple orbital layers, meeting stringent timing requirements for applications in LEO, MEO, and GEO.

Key advantages include OSIRAN’s capacity to offload computational tasks to terrestrial cloud resources, which reduces onboard power demands and extends satellite lifespan. Simulation results confirm latency reductions of up to 70 ms through ISL-based paths confirming OSIRAN’s real-time adaptability.

The architecture’s modular design supports real-time adjustments and autonomous responses, improving spectrum utilization and minimizing interference, particularly in multi-orbit setups. By reducing ground station reliance, OSIRAN facilitates resilient NTN-TN coordination, achieving flexibility and robustness across diverse operational conditions.

In future, this flexible architecture paves the way for deep-space applications, where self-healing capabilities and independence from terrestrial control will be essential, marking a significant step toward NTNs that are 6G-ready and future-proofed for beyond-Earth missions.

%\section*{Acknowledgment}

%The preferred spelling of the word ``acknowledgment'' in America is without 
%an ``e'' after the ``g''. Avoid the stilted expression ``one of us (R. B. 
%G.) thanks $\ldots$''. Instead, try ``R. B. G. thanks$\ldots$''. Put sponsor 
%acknowledgments in the unnumbered footnote on the first page.

\bibliographystyle{IEEEtran}
\bibliography{biblio}

\vskip -3\baselineskip plus -1fil


\begin{IEEEbiography}[{\includegraphics[width=1in,height=1.25in,clip,keepaspectratio]{figures/headshots/biopic_ebm.jpg}}]{Eduardo Baena} is a postdoctoral research fellow at Northeastern University. Holding a Ph.D. in Telecommunication Engineering from the University of Malaga, his experience spans various roles within the international private sector from 2010 to 2017. Later he joined UMA as a lecturer and researcher contributing to several H2020 projects and as a Co-IP of national and regional funded projects. 
\end{IEEEbiography}

\vskip -2\baselineskip plus -1fil

\begin{IEEEbiography}[{\includegraphics[width=1in,height=1.25in,clip,keepaspectratio]{figures/headshots/paolo.jpg}}]{Paolo Testolina} received the Ph.D. degree in information engineering from the University of Padova in 2022. He spent research periods with Northeastern University, Boston, MA, USA. He is currently a Post-Doctoral Researcher with the University of Padova. His research interests include mmWave networks, from channel modeling to link layer simulation, traffic modeling, radio frequency interference analysis, and vehicular networks
\end{IEEEbiography}


\vskip -2\baselineskip plus -1fil

\begin{IEEEbiography}[{\includegraphics[width=1in,height=1.1in,clip,keepaspectratio]{figures/headshots/Michele.png}}]{Michele Polese} is a Research Assistant Professor at the Institute for the Wireless Internet of Things, Northeastern University, Boston, since October 2023. He received his Ph.D. at the Department of Information Engineering of the University of Padova in 2020. He then joined Northeastern University as a research scientist and part-time lecturer in 2020. During his Ph.D., he visited New York University (NYU), AT\&T Labs in Bedminster, NJ, and Northeastern University.
His research interests are in the analysis and development of protocols and architectures for future generations of cellular networks (5G and beyond), in particular for millimeter-wave and terahertz networks, spectrum sharing and passive/active user coexistence, open RAN development, and the performance evaluation of end-to-end, complex networks. 
\end{IEEEbiography}

\vskip -2\baselineskip plus -1fil

\begin{IEEEbiography}[{\includegraphics[width=1in,height=1.25in,clip,keepaspectratio]{figures/headshots/koutsonikolas3.jpg}}]{Dimitrios Koutsonikolas} is an Associate Professor in the Department of Electrical and Computer Engineering and a member of the Institute for the Wireless Internet of Things at Northeastern University. Between January 2011 and December 2020, he was in the Computer Science and Engineering Department at the University at Buffalo, first as an Assistant Professor (2011-2016) and then as an Associate Professor (2016-2020) and Director of Graduate Studies (2018-2020). He received his PhD in Electrical and Computer Engineering from Purdue University in 2010.  His research interests are broadly in experimental wireless networking and mobile computing, with a current focus on 5G networks and latency-critical applications (AR, VR, CAVs) over 5G, millimeter-wave networking, and energy-aware protocol design for smartphones. He has served as the General Co-Chair for IEEE LANMAN 2024, IEEE WoWMoM 2023, and ACM EWSN 2018, and TPC Co-Chair for IEEE LANMAN 2023, IEEE HPSR 2023, IEEE DCOSS 2022, IEEE WoWMoM 2021, and IFIP Networking 2021. He received the IEEE Region 1 Technological Innovation (Academic) Award in 2019 and the NSF CAREER Award in 2016. He is a senior member of the IEEE and the ACM and a member of USENIX.

 
\end{IEEEbiography}

\vskip -2\baselineskip plus -1fil

\begin{IEEEbiography}[{\includegraphics[width=1in,height=1.25in,clip,keepaspectratio]{figures/headshots/jornet.jpg}}]{Josep M. Jornet} (M'13--SM'20--F'24) is a Professor in the Department of Electrical and Computer Engineering, the director of the Ultrabroadband Nanonetworking (UN) Laboratory, and the Associate Director of the Institute for the Wireless Internet of Things at Northeastern University (NU). He received his Ph.D. degree in Electrical and Computer Engineering from the Georgia Institute of Technology, Atlanta, GA, in August 2013. He is a leading expert in terahertz communications, in addition to wireless nano-bio-communication networks and the Internet of Nano-Things. In these areas, he has co-authored over 250 peer-reviewed scientific publications, including one book, and has been granted five US patents. 
\end{IEEEbiography}

\vskip -2\baselineskip plus -1fil

\begin{IEEEbiography}[{\includegraphics[width=1in,height=1.25in,clip,keepaspectratio]{figures/headshots/melodia.jpg}}]{Tommaso Melodia}
is the William Lincoln Smith Chair Professor with the Department of Electrical and Computer Engineering at Northeastern University in Boston. He is also the Founding Director of the Institute for the Wireless Internet of Things and the Director of Research for the PAWR Project Office. He received his Ph.D. in Electrical and Computer Engineering from the Georgia Institute of Technology in 2007. He is a recipient of the National Science Foundation CAREER award. Prof. Melodia has served as Associate Editor of IEEE Transactions on Wireless Communications, IEEE Transactions on Mobile Computing, Elsevier Computer Networks, among others. 
\end{IEEEbiography}



\end{document}
