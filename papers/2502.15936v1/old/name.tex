\documentclass[journal]{IEEEtran}
\IEEEoverridecommandlockouts
\usepackage{cite}
\usepackage{amsmath,amssymb,amsfonts}
\usepackage{algorithmic}
\usepackage{graphicx}
\usepackage{textcomp}
\usepackage{xcolor}
\usepackage{multirow}
\usepackage{array}
\usepackage{caption}
\usepackage{booktabs}
\usepackage{subcaption}
\usepackage{balance}
\usepackage[capitalise]{cleveref}
\usepackage{tikz}
\usetikzlibrary{fit}
\makeatletter
\tikzset{
  fitting node/.style={
    inner sep=0pt,
    fill=none,
    draw=none,
    reset transform,
    fit={(\pgf@pathminx,\pgf@pathminy) (\pgf@pathmaxx,\pgf@pathmaxy)}
  },
  reset transform/.code={\pgftransformreset}
}
\makeatother

\usepackage{pgfplots}
\pgfplotsset{compat=newest} 
\pgfplotsset{plot coordinates/math parser=false} 

\usepackage[acronyms,nonumberlist,nopostdot,nomain,nogroupskip,acronymlists={hidden}]{glossaries}
\newlength\fheight
\setlength{\fheight}{0.5\columnwidth}
\newlength\fwidth
\setlength{\fwidth}{0.8\columnwidth}

\def\BibTeX{{\rm B\kern-.05em{\sc i\kern-.025em b}\kern-.08em
    T\kern-.1667em\lower.7ex\hbox{E}\kern-.125emX}}
\begin{document}

\title{O-RAN-en-el-Espacio: Habilitando Redes Inteligentes, Abiertas e Interoperables en 6G}

\author{\IEEEauthorblockN{
Eduardo Baena, 
Paolo Testolina, 
Michele Polese, 
Dimitrios Koutsonikolas, 
Josep Jornet, 
Tommaso Melodia}

\IEEEauthorblockA{Instituto para el Internet Inalámbrico de las Cosas, Universidad Northeastern, Boston, MA, EE.UU.}}

\maketitle

\begin{abstract}
Aunque las redes no terrestres (NTN) han avanzado significativamente en la expansión de la conectividad, su integración con las redes terrestres sigue siendo un desafío complejo debido a la falta de marcos de gestión cohesivos. La naturaleza descentralizada de las NTN, junto con los mecanismos de control fragmentados existentes y la falta de interfaces de gestión estandarizadas, complica la coordinación entre los dominios terrestre y orbital, resultando en ineficiencias críticas y discontinuidades operativas. Este artículo presenta la arquitectura O-RAN-en-el-Espacio, un marco novedoso diseñado para unificar los sistemas terrestres y no terrestres a través de una gestión jerárquica y distribuida. En su núcleo, el "SpaceRIC" permite la coordinación a nivel de constelación, integrando perfectamente las directivas estratégicas de los sistemas terrestres impulsados por IA y Gemelos Digitales con las operaciones en tiempo real de los satélites. La arquitectura incorpora aplicaciones distribuidas (DAPP y SAPP), lo que permite bucles de control adaptativos y garantiza la robustez operativa bajo diversas condiciones de red. Un enfoque dinámico de mapeo de interfaz-enlace mejora la flexibilidad al alinear las funciones de red con las demandas específicas de las aplicaciones, incluso bajo escenarios de conectividad intermitente y recursos limitados. Los resultados de simulaciones validan su viabilidad al demostrar el cumplimiento de los requisitos de retardo de señalización, destacando su potencial para habilitar una gestión global robusta y eficiente de la conectividad hacia las redes 6G.
\end{abstract}

\begin{IEEEkeywords}
NTN, O-RAN, Integración, 6G, Gestión
\end{IEEEkeywords}

\glsresetall

\section{Introducción}
La creciente demanda de conectividad global de alta capacidad y confiabilidad ha convertido a las redes no terrestres (NTN) en un componente clave de los sistemas de comunicación futuros. Estas redes, que incluyen satélites en órbitas bajas (LEO), medias (MEO), y geoestacionarias (GEO), así como plataformas de gran altitud (HAP), amplían la conectividad a regiones remotas y desatendidas, superando las limitaciones de la infraestructura terrestre. Sin embargo, la integración de las redes NTN con las terrestres sigue siendo compleja debido a sus diferencias operativas y restricciones de recursos.

Los satélites en órbita baja, por ejemplo, operan en entornos altamente dinámicos donde su movimiento constante requiere traspasos frecuentes y actualizaciones continuas a los sistemas de gestión de conectividad. Estas dinámicas introducen desafíos como latencia variable, conectividad intermitente y pérdidas frecuentes de línea de visión (LoS), lo que afecta la continuidad del servicio y compromete la estabilidad de la red.

\section{Arquitectura Propuesta}
\label{sec:architecture}
La arquitectura O-RAN-en-el-Espacio aborda los principales desafíos de integración de las NTN con los sistemas terrestres compatibles con O-RAN, enfocándose en un marco de control jerárquico que une la infraestructura terrestre en la nube y el cómputo en el borde a bordo de los satélites. Esta integración permite una asignación dinámica de recursos y distribución de tareas a lo largo de los segmentos terrestre y orbital, garantizando eficiencia operativa bajo condiciones dinámicas y reduciendo la dependencia de la conectividad terrestre continua.

\subsection{Componentes Clave}
\textbf{Nube Terrestre:} La nube terrestre gestiona funciones de red de alto nivel, incluyendo planificación a largo plazo, modelado de gemelos digitales y optimización de políticas. La capacidad computacional de esta capa soporta procesos no críticos en tiempo real, como entrenamiento de modelos de aprendizaje automático y análisis de datos a gran escala.

\textbf{Cómputo en el Borde a Bordo:} Los recursos a bordo, como CPU y GPU, son gestionados por un middleware avanzado que asigna recursos dinámicamente según la demanda. Este middleware, junto con una capa de virtualización, facilita el despliegue de múltiples funciones de red y garantiza la adaptabilidad y eficiencia operativa en condiciones de tráfico variables.

\section{Conclusión}
La arquitectura O-RAN-en-el-Espacio proporciona un marco robusto para la integración de redes terrestres y no terrestres, habilitando comunicaciones globales inteligentes, abiertas e interoperables hacia la era 6G.
\end{document}