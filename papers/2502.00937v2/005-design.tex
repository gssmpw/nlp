\section{\sysname{} Design and Implementation}
Based on our insights from the characterization study of open-source \lmm{} benchmarks and production \lmm{} workloads, we propose \sysname, a novel decoupled architecture for scalable and resource-efficient \lmm{} serving.

The key idea in \sysname{} is to separate image- and text-specific inference stages into distinct instances, given the need to optimize each stage separately (\Cref{insight:1,insight:3}) and enable seamless interaction between stages.
Unlike monolithic infrastructures, \sysname{} enables independent optimization of each stage, improving resource efficiency while meeting performance SLOs.
This decoupling also enables modality-aware request serving, addressing tail latency, heterogeneous bursts, and resource contention.

\myparagraph{Overview}
\Cref{fig:overview} shows \sysname{}'s design.
A pool of \emph{Image Instances} handles image preprocessing and encoding of image-text requests.
The resulting image tokens are passed to a pool of \emph{Text Instances}, which performs LLM prefill and decode operations.
Text-only requests bypass the image components and are queued directly at the \emph{Text Instances}.
Two pools are managed by the \emph{Image and Text Pool Managers}.

\sysname{} adopts a hierarchical architecture inspired by DynamoLLM~\cite{stojkovic2024dynamollm}.
Onboarding any new \lmms{} (\eg{}, Llama3.2-11B) starts with an offline profiling phase to build model-stage profiles that capture how model configurations and load impact performance (\Cref{sec:design:profiling}).
\sysname{} uses these profiles to guide model configuration and instance scaling.
After model deployment, \sysname{} then reconfigures resources periodically to adapt to workload patterns, scaling for increased image-text requests (or vice versa) (\Cref{sec:design:resource-management}).
For each request, \sysname{} selects the optimal \lmm{} Text or Image Instances for execution (\Cref{sec:design:modality-aware-serving}).

\begin{figure}[!t]
    \raggedright
    \includegraphics[width=\linewidth]{figs/modserve-overview.pdf}
    \caption{Overview of the \sysname{} architecture.}
    \label{fig:overview}
\end{figure}

\subsection{Offline \lmm{} Profile Generation}
\label{sec:design:profiling}
When onboarding a new \lmm{}, \sysname{} generates resource-performance profiles by characterizing the image encoder and LLM backend independently.
This profiling runs controlled inference workloads with varying model parallelisms (\eg{}, TP-2 and TP-8), batch sizes, and load (\ie{}, image tokens per second for image encoders and prompt tokens per second for LLM backends) to capture per-stage performance characteristics.
To efficiently model performance across different load conditions, \sysname{} profiles a set of representative load levels (up to the maximum throughput) and extrapolates the behavior for intermediate loads.
The resulting profiles take load, parallelism, and batch size as inputs to predict key performance metrics, including encoding latency for \emph{Image Instances} and prefill time and TBT for \emph{Text Instances}.
The \emph{Pool Managers} use these profiles to guide operational decisions (\Cref{sec:design:resource-management}):
(1) pool autoscaling to meet latency SLOs without overprovisioning,
(2) model sharding that selects optimal TP degrees, and
(3) max batch sizing for each stage.

Since multiple \lmms{} may share the same image encoder or LLM backend, \sysname{} minimizes overhead by reusing model profiles across deployments.
These profiles are cached in cluster-local storage and synchronized via a global repository, enabling efficient sharing across clusters.


\subsection{Decoupled Resource Management}
\label{sec:design:resource-management}
\sysname{}'s decoupled approach to resource management stems from our insights on stage-specific performance disparities in batching (\Cref{insight:4}), independent scaling benefits (\Cref{insight:5}), and modality-specific traffic patterns (\Cref{insight:6}).
Specifically, \sysname{} periodically reconfigures resources (\ie{}, every five minutes to match the autoscaling overhead) to align with workload demands.

The \emph{Image Pool Manager} maintains a pool of \emph{Image Instances}, which preprocess images on CPU and encode image workloads on GPU for image-text requests.
Meanwhile, the \emph{Text Pool Manager} manages a pool of \emph{Text Instances} responsible for the prefill and decode stages of both image-text and text-only requests.
Based on model profiles (\Cref{sec:design:profiling}), each manager independently optimizes pool autoscaling, model sharding, and max batch sizing to minimize costs while meeting performance SLOs.

Before their online operation, the initial number of Image Instances ($N_i$) is determined using the median image QPS multiplied by the median image encoding latency.
The number of Text Instances ($N_t$) is set as $N_i$ divided by the median number of images per request, based on historical \lmm{} inference traces.
If no history is available, \sysname{} initially overprovisions resources to ensure reliability.

\myparagraph{Token-Aware Pool Autoscaling}
The \emph{Pool Managers} dynamically scale the number of \emph{Image and Text Instances} based on real-time workload demands.
For example, a surge in image-heavy requests leads to more image preprocessors and encoders, while an increase in text requests or requests' prompt lengths triggers the \emph{Text Pool Manager} to scale LLM replicas to handle variations in prefill.

The number of replicas per stage is computed as $\lceil{\frac{ML}{MC}}\rceil$
where $ML$ is the modality-specific load (\eg{}, prompt tokens/sec for \emph{Text Instances}, image tokens/sec for \emph{Image Instances}) and $MC$ is the maximum capacity each stage can handle without violating SLOs, based on the offline \lmm{} profiling data.
Unlike traditional web service autoscaling, which reacts to request rates, \sysname{} optimizes scaling based on token throughput (tokens/s), capturing variations in both request rates and request sizes (\Cref{insight:6} and \Cref{fig:text-len-prod}).

For \emph{Image Instances}, image token counts are precomputed based on static mapping from image dimensions (\Cref{fig:img-token-distribution-open}).
Autoscaling of \emph{Text Instances} is based on text token load in CroAttn models but total tokens in DecOnly models due to homogeneous self-attention across modalities.
Advanced autoscaling hysteresis prevention techniques~\cite{optscaler} can be employed to avoid excessive scaling actions caused by transient workload fluctuations but are not covered in this paper.

\myparagraph{Model Sharding}
The \emph{Pool Managers} also determine instance sharding for optimal tensor parallelism (TP) for image encoders and LLM backends.
Our characterization (\Cref{sec:characterization:open-source}) shows image encoders achieve peak throughput with lower TP than LLMs.
Therefore, the model sharding degree for each instance is configured separately for maximum throughput while ensuring SLO attainment on TTFT and TBT.
By decoupling the components, \sysname{} ensures independent sharding, optimizing parallelism without unnecessary synchronization overhead.

When scaling beyond a single GPU server, \sysname{} prioritizes autoscaling over pipeline parallelism (PP)~\cite{shazeer2020megatron} to maximize throughput, seamlessly transitioning to batch-level optimizations as needed.

\myparagraph{Identifying Max Batch Size}
For each stage, the maximum batch size is configured to maximize throughput while meeting latency SLOs.
Batch sizing decisions are guided by the offline model-stage profiles, which predict their impact on encoding and decoding latencies.
\emph{Image Instances} may forgo batching as small max batch sizes often achieve optimal GPU utilization (\Cref{insight:3}).
In contrast, \emph{Text Instances} batch requests when beneficial, optimizing token throughput during prefill/decode based on TTFT and TBT SLOs, particularly for CroAttn \lmms{} (\Cref{insight:4}).

\subsection{Modality-Aware \lmm{} Request Serving}
\label{sec:design:modality-aware-serving}
For each incoming \lmm{} request, \sysname{} dynamically routes and schedules workloads to balance load across \emph{Image and Text Instances}.
The \emph{Pool Managers} optimize this process to minimize queueing delays and improve TTFT latency.

\myparagraph{Request Routing Across Instances}
To mitigate tail TTFT latency surges caused by modality-specific bursts and queuing delays, \sysname{} employs a modality-aware routing strategy that balances image and text workloads independently.
Traditional request-level LLM load balancing (\eg{}, round-robin, memory-based~\cite{sun2024llumnix}) overlooks the computational intensity of image encoding (\Cref{insight:2}), making them vulnerable to load imbalances during image bursts (\Cref{insight:6}), leading to high tail latencies.

Instead, \sysname{} routes requests by input modality.
Image-text requests are assigned to \emph{Image Instances} with the least image-token load.
Large requests (\ie{}, those with more images) are consequently distributed across multiple \emph{Image Instances} for parallel processing and encoding (\Cref{insight:2}), preventing degraded batching performance that would occur if all images were routed to a single instance.
This effectively enables a form of request chunking~\cite{sarathi-serve}, where images in a large request can be processed in an interleaved manner with other requests, reducing HoL blocking and improving scheduling flexibility.

To route traffic between Text Instances, text-only requests and image-text requests with completed image tokens are directed to the \emph{Text Instance} with the least total pending tokens (text+image) for DecOnly models and the least total pending text tokens for CroAttn models because of the attention mechanism difference between the two model architectures (\Cref{insight:7}).
Modality-aware routing enables parallel image encoding and dynamically adapts to image or text traffic bursts, reducing queueing delays and improving TTFT, particularly at the tail.

\myparagraph{Instance Request Scheduling}
At the instance level, \sysname{} minimizes resource contention between image-text and text-only requests with priority scheduling based on modality and prompt size.
While decoupling isolates image and text processing, contention can still arise in \emph{Text Instances}, where both request types share prefill processing.
This issue is particularly pronounced in DecOnly \lmms{}, which exhibits lower efficiency during the prefill stage (\Cref{fig:batch-size}).
Performance degradation occurs from increased batching latency for all requests, while non-batched processing introduces HoL blocking and high queueing delays at tails due to request heterogeneity (\Cref{fig:prod-input}).

To address these challenges, \sysname{} replaces traditional FIFO scheduling--which may exacerbate HoL blocking~\cite{patke2024queue,aiops2024qiu}--with an SLO-driven scheduling strategy that can prioritize shorter requests (\eg{}, text-only queries or small image-text requests with tight SLOs) to maintain low latency.

\emph{Pool Managers} continuously monitor SLO attainment and trigger pool autoscaling when the rate falls below a predefined threshold (default 0.99 with a sensitivity study in \Cref{sec:eval:sensitivity}), ensuring adaptive resource allocation under dynamic workloads (especially in cases of unpredictable traffic).
\sysname{} can work with state-of-the-art batch scheduling techniques~\cite{sarathi-serve,patel2024splitwise} to optimize TBT during the decode stage, which we leave to future work as we do not observe TBT degradation in \lmm{} characterization (\Cref{insight:4}).

\myparagraph{Image Token Transfer}
Once image processing is complete, \sysname{} transfers image tokens from \emph{Image Instances} to \emph{Text Instances} via a pull-based RDMA mechanism.
Unlike a push-based approach that immediately sends image tokens upon availability, this strategy reduces synchronization overhead and optimizes load balancing.

Since image-text requests may involve multiple images processed across different \emph{Image Instances}, the transfer follows a many-to-one pattern, aggregating all image tokens before text processing begins.
This defers transfer until all tokens are ready, allowing for better scheduling decisions by routing requests to the least-loaded \emph{Text Instance}.
Meanwhile, queuing at the \emph{Text Instance} overlaps with token transfer, effectively hiding transfer latency.

\sysname{} colocates \emph{Image and Text Instances} in the same server when each \emph{Text Instance} is not taking up all GPUs on a server to avoid image transfer overhead and unallocated idle GPUs.
For example, it may place one TP-4 Text Instance and two TP-2 Image Instances within the same 8-GPU server.
Unlike monolithic deployments, colocated instances remain independently configurable and can serve corresponding stages of different requests independently.


\subsection{Implementation}
\label{sec:design:implementation}

We implement \sysname{} using 5,000 lines of Python code.
We base the \emph{Text Instance} on vLLM~\cite{vllm} (v0.7.2), a state-of-the-art generative model inference platform, and build the \emph{Image Instance} on HuggingFace Transformers~\cite{hf-transformers}.
The modular architecture of \sysname{} enables easy integration with other serving engines (\eg{}, TensorRT~\cite{tensor-rt} and DeepSpeed~\cite{deepspeed}).
We use \texttt{numactl} to restrict CPU and memory usage of image preprocessing to a single NUMA node, which reduces memory access latency and performance variation.
To ensure efficient GPU-to-GPU memory transfer of image tokens, we use PyTorch's distributed communication with the NCCL backend and GPU Direct RDMA.
The P99 transfer latency of image tokens per request is 5 ms.

We implement the \emph{Image and Text Pool Managers} as lightweight gRPC servers (hosted on dedicated VMs) with low memory and compute requirements, drawing inspiration from DynamoLLM~\cite{stojkovic2024dynamollm}.
For failure detection and recovery, the \emph{Pool Managers} use a simple heartbeat-based membership management protocol~\cite{gupta2001scalable}.
However, \sysname{} can be easily extended to adopt more robust leader election (\eg{}, Raft~\cite{raft}) and fault-tolerance algorithms~\cite{pb}.