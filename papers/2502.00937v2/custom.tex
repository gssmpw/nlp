\newcommand{\jheading}[1]{\vspace{0.05in}\noindent\textbf{#1.}}
\newcommand{\sref}[1]{\S\ref{#1}}
\newcommand{\wrt}{w.r.t\xspace}
\newcommand{\ra}[1]{\renewcommand{\arraystretch}{#1}}

\newcommand{\myparagraph}[1]{\vspace{\smallskipamount}\noindent\textbf{#1.\xspace}}
\newcommand{\myparagraphnodot}[1]{\vspace{\smallskipamount}\noindent\textbf{#1\xspace}}
\newcommand{\myparagraphemph}[1]{\vspace{\smallskipamount}\noindent\emph{#1.\xspace}}
\newcommand{\eg}{\emph{e.g.}\xspace}
\newcommand{\etc}{etc.\@\xspace}
\newcommand{\cf}{{cf.}\xspace}
\newcommand{\ie}{\emph{i.e.}\xspace}
\newcommand{\etal}{\emph{et al.}\xspace}
\newcommand*{\rom}[1]{\uppercase\expandafter{\romannumeral #1\relax}}

\crefname{insight}{Insight}{Insight}
\newcounter{insight}
\newcommand{\boxinsight}[1]{%
  \refstepcounter{insight}%
  \vspace{\smallskipamount}%
  \noindent \simplebox{
    \textbf{\uline{Insight~\theinsight:}} 
    \textit{#1}% \par
  }
}


% Comments
\usepackage{ifthen}
\newboolean{publicversion}
%\setboolean{publicversion}{true}
\setboolean{publicversion}{false}

\newcommand{\RED}[1]{{\bfseries \color{red} #1}}
\newcommand{\red}[1]{{\color{red} #1}}
\newcommand{\blue}[1]{{\color{blue} #1}}

\ifthenelse{\boolean{publicversion}}{
    \newcommand{\grumbler}[3]{}
    \newcommand{\esha}[1]{}
    \newcommand{\inigo}[1]{}
    \newcommand{\haoran}[1]{}
    \newcommand{\alind}[1]{}
    \newcommand{\anish}[1]{}
    \newcommand{\rr}[1]{}
    \newcommand{\hq}[1]{}
    \newcommand{\todo}[1]{}
}
{
    \newcommand{\grumbler}[3]{\xspace\textcolor{#3}{\bf #1: #2}}
    \newcommand{\esha}[1]{\grumbler{Esha}{#1}{brown}}
    \newcommand{\inigo}[1]{\grumbler{Inigo}{#1}{violet}}
    \newcommand{\haoran}[1]{\grumbler{Haoran}{#1}{blue}}
    \newcommand{\jm}[1]{\grumbler{Jayashree}{#1}{magenta}}
    \newcommand{\alind}[1]{\grumbler{Alind}{#1}{ForestGreen}}
    \newcommand{\anish}[1]{\grumbler{Anish}{#1}{teal}}
    \newcommand{\rr}[1]{\grumbler{Ram}{#1}{purple}}
    \newcommand{\hq}[1]{\textcolor{blue}{hqiu: #1}}
    \newcommand{\todo}[1]{\textcolor{blue}{TODO: #1}}
}

\hypersetup{
    unicode=true,
    bookmarksnumbered=true,
    colorlinks=true,
    linkcolor={red!70!black},
    citecolor={red!70!black},
    urlcolor={blue!70!black},
    pdfborder={0 0 0}
}

% highlight box
\usepackage[most]{tcolorbox}
\tcbset{textmarker/.style={%
        enhanced,
        parbox=false,boxrule=0mm,boxsep=0mm,arc=1.5mm,
        outer arc=1.5mm,left=2mm,right=2mm,top=4pt,bottom=3pt,
        toptitle=1mm,bottomtitle=1mm,oversize}}


\newtcolorbox{simplenoteBox}{colback=white, colframe=black, boxrule=0.2mm, arc=0mm, auto outer arc, boxsep=0mm, left=2mm, right=2mm, top=1mm, bottom=1mm} 
% define new colorboxes
\newtcolorbox{noteBox}{textmarker,
    % borderline west={0pt}{0pt}{gray},
    colback=gray!8!white}
% define commands for easy access
\newcommand{\takeaway}[1]{\begin{noteBox} \textbf{Takeaway:} #1 \end{noteBox}}
\newcommand{\note}[2]{\begin{noteBox} \textbf{#1} #2 \end{noteBox}}
\newcommand{\plainbox}[1]{\begin{noteBox} #1 \end{noteBox}}

\newcommand{\simplebox}[1]{\begin{simplenoteBox} #1 \end{simplenoteBox}}

% no splitting footnotes
\interfootnotelinepenalty=10000

\definecolor{mygray}{gray}{0.95}
\newmdenv[
  backgroundcolor=mygray,
  linecolor=black,
  linewidth=1pt,
  roundcorner=3pt,
  nobreak=true,
]{keyfinding}

% \captionsetup{skip=2pt}