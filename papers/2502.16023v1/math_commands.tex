% \usepackage{amsthm}
% \smartqed  % flush right qed marks, e.g. at end of proof
\usepackage{graphicx}

%-------%
% Fonts % 
%-------%
\usepackage[T1]{fontenc}    % use 8-bit T1 fonts
\usepackage{booktabs}       % professional-quality tables


% General Packages
%------------------%
\usepackage{amsfonts,amssymb,bbm}
\usepackage{amsmath}
\usepackage{enumitem}
\usepackage{graphicx}
% \usepackage{xcolor}
\usepackage{adjustbox}
\usepackage[UKenglish]{isodate}
\usepackage{graphicx}
\usepackage{silence}
\WarningFilter{caption}{Unknown document class (or package)}
\usepackage[format=default,font=small,labelfont=it]{caption}
\usepackage{subcaption}
% \usepackage{hyperref} 
%     \hypersetup{
%     	colorlinks = true,
%     	linkcolor = blue,
%     	anchorcolor = blue,
%     	citecolor = blue,
%     	filecolor = blue,
%     	urlcolor = blue
%     }

    


\usepackage{float}
\usepackage{enumitem}
\usepackage{multirow}
\usepackage{fancyvrb}
\usepackage{rotating}
    %%% Tikz
    \usepackage{tikz}
    \usepackage{tikz-cd} 
    \pgfdeclarelayer{edgelayer}
    \pgfdeclarelayer{nodelayer}
    \pgfsetlayers{edgelayer,nodelayer,main}
    %%%% Custom Node styles
    \tikzstyle{new style 0}=[fill={rgb,255: red,255; green,94; blue,247}, draw=black, shape=circle]
    \tikzstyle{pointy}=[fill=white, draw=black, shape=circle]
    %%%% Custom Edge styles
    \tikzstyle{pointy}=[->]
\usepackage{fontawesome}


%To display labels in the PDF. Comment at the end!
% Code
\usepackage{algpseudocode}
\usepackage[linesnumbered,lined,boxed,commentsnumbered,ruled,longend]{algorithm2e}
\newcommand\mycommfont[1]{\footnotesize\ttfamily\textcolor{blue}{#1}}
\SetCommentSty{mycommfont}
%\SetSideCommentRight

% TABLE Formatting
\raggedbottom
\newcommand{\ra}[1]{\renewcommand{\arraystretch}{#1}}
\setlength\heavyrulewidth{0.3ex}
\usepackage{lscape}



% \makeatletter
\newcommand{\pushright}[1]{\ifmeasuring@#1\else\omit\hfill$\displaystyle#1$\fi\ignorespaces}
\newcommand{\pushleft}[1]{\ifmeasuring@#1\else\omit$\displaystyle#1$\hfill\fi\ignorespaces}
\makeatother


\DeclareMathOperator{\reshape}{reshape}
\DeclareMathOperator{\vectorize}{vec}
\DeclareMathOperator{\layernorm}{LayerNorm}
\DeclareMathOperator{\concat}{cat}
\DeclareMathOperator{\softmax}{softmax}
\DeclareMathOperator{\softplus}{softplus}
\DeclareMathOperator{\sigmoid}{sig}
\DeclareMathOperator{\attention}{Attention}
\DeclareMathOperator{\multihead}{MultiHead}
\DeclareMathOperator{\tblock}{\mathcal{TB}}
\DeclareMathOperator{\transformer}{\mathcal{T}}

\DeclareMathOperator{\Leb}{Leb}
\DeclareMathOperator{\supp}{supp}
\DeclareMathOperator{\inter}{int}
\DeclareMathOperator*{\esssup}{ess\,sup}
\DeclareMathOperator{\tr}{tr}
\newcommand{\n}{\mathbbm{n}}
\newcommand{\bE}{\mathbb{E}}
\newcommand{\bP}{\mathbb{P}}
\newcommand{\cL}{\mathcal{L}}
\newcommand{\cY}{\mathcal{Y}}
\newcommand{\cB}{\mathcal{B}}
\newcommand{\cM}{\mathcal{M}}
\newcommand{\1}{\mathbbm{1}}
\newcommand{\II}{I_{\varepsilon,\Gamma}}

%Modifiche comode 
\renewcommand{\phi}{\varphi}
\newcommand{\eps}{\varepsilon}
\newcommand{\al}{\alpha}
\newcommand{\om}{\omega}
\newcommand{\Om}{\Omega}
\newcommand{\lam}{\lambda}
\newcommand{\Ll}{\left}
\newcommand{\rr}{\mathbb{R}}
\newcommand*\diff{\mathop{}\!\mathrm{d}}
%\def\mathsf{1}{\mathds{1}}
\let\oldemptyset\emptyset
\let\emptyset\varnothing
\newcommand{\essinf}{\operatornamewithlimits{ess\,inf}}
\newcommand{\paren}[1]{\left( \left. #1 \right. \right)}
\DeclareMathOperator*{\argmax}{arg\,max}
\DeclareMathOperator*{\argmin}{arg\,min}
\DeclareMathOperator*{\BL}{BL}
\newcommand\sbullet[1][.5]{\mathbin{\vcenter{\hbox{\scalebox{#1}{$\bullet$}}}}}
\newcommand{\rhoseason}{m_{\rho}}%\rho_{\mathrm{season}}}% name of the seasonality function: m_{\rho} in [Arlot 2004]
\newcommand{\rhoseasondiscrete}[1]{m_{\rho,#1}}
\newcommand{\rhomin}{\ensuremath{\mu_{\min}}} %changed by Duccio
\newcommand{\rhomax}{\ensuremath{\mu_{\max}}} %changed by Duccio
\newcommand{\croch}[1]{\left[ \left. #1 \right. \right]}
\newcommand{\survival}{\ensuremath{\mathbb{S}}}

% Processes
\newcommand{\process}[3]{{#1}({#2})_{#3}}
\newcommand{\myset}[2]{\lbrace{#1}, \ldots, {#2}\rbrace}
\newcommand{\mySet}[2]{\left\lbrace {#1} \middle\vert {#2} \right\rbrace}


% Mathbb
\newcommand{\C}{\mathbb{C}}
\newcommand{\D}{\mathbf{D}}
\newcommand{\E}{\mathbb{E}}
\newcommand{\N}{\mathbb{N}}
\renewcommand{\P}{\mathbb{P}}
\newcommand{\Q}{\mathbb{Q}}
\newcommand{\R}{\mathbb{R}}
\newcommand{\T}{\mathbb{T}}
\newcommand{\Z}{\mathbb{Z}}
\newcommand{\K}{\mathbb{K}}
\newcommand{\Cg}{\mathfrak{C}}
\newcommand{\bdot}{\boldsymbol{\cdot}}


% Calligrafiche
\newcommand{\Aa}{\mathcal{A}}
\newcommand{\Bb}{\mathcal{B}}
\newcommand{\Cc}{\mathcal{C}}
\newcommand{\Ee}{\mathcal{E}}
\newcommand{\Ff}{\mathcal{F}}
\newcommand{\Hh}{\mathcal{H}}
\newcommand{\Ii}{\mathcal{I}}
\newcommand{\Kk}{K}
\newcommand{\Mm}{\mathcal{M}}
\newcommand{\Nn}{\mathcal{N}}
\newcommand{\Qq}{\mathcal{Q}}
\newcommand{\Ss}{\mathcal{S}}
\newcommand{\Xx}{\mathcal{X}}

\newcommand{\eqdef}{\ensuremath{\,\raisebox{-1pt}{$\stackrel{\mbox{\upshape\tiny def.}}{=}$}}\,}
% \newcommand{\eqdef}{\stackrel{\mathclap{\tiny\mbox{def.}}}{=}}


\newcommand{\sdim}{{N_\mathrm{sim}}} %Similarity score encoding dimension
\newcommand{\pdim}{{N_\mathrm{pos}}} %Positional encoding dimension
\newcommand{\edim}{{N}} % Encoding dimension
\newcommand{\ddim}{{N'}} % Decoding dimension (target dimension of MLP)
\newcommand{\rdim}{{N_\mathrm{ref}}} %Number of reference paths
\newcommand{\tdim}{{N_\mathrm{time}}} %Number of reference time points
\newcommand{\pldim}{{N_\mathrm{p}}} %Number of time points between linear pieces
\newcommand{\ldim}{{d_K}} %Latent dimension, was {d_K}

% Cose del modello
\newcommand{\rhosummer}{\rho}% name of the length of summer
\newcommand{\rhowinter}{\rho}% name of the length of winter

\RequirePackage{amsthm}
\newtheorem{atheorem}{Theorem}
\renewcommand{\theatheorem}{\Alph{atheorem}} % "letter-numbered" theorems
%\newtheorem{InformalTheorem}{Informal Theorem}
\newtheorem{definition}{Definition}
\newtheorem{proposition}{Proposition}
% \newtheorem{assumption}{Assumption}
\newtheorem{theorem}{Theorem}
\newtheorem{lemma}{Lemma}
% \newtheorem{corollary}{Corollary}
\newtheorem{notation}{Notation}
\newtheorem{setting}{Setting}[section]
\newtheorem{assumptions}{Assumptions}[section]
\theoremstyle{remark}
% \newtheorem*{remark}{Remark}
% \newtheorem{example}{Example}
% \theoremstyle{definition}
\newtheorem{question}{Question}

% QED Symbol 
% \usepackage{etoolbox}
% \newcommand{\addQEDstyle}[2]{\AtBeginEnvironment{#1}{\pushQED{\qed}\renewcommand{\qedsymbol}{#2}}\AtEndEnvironment{#1}{\popQED}}
% \addQEDstyle{remark}{$\triangleleft$}
% \addQEDstyle{example}{$\triangleleft$}



%%%% COMMENTING (Temporary)
\NewDocumentCommand{\luca}{mo}{
    \IfValueF{#2}{
    %%%%% NO Margin Note
                        {{\scriptsize
                            \textcolor{green}{ 
                            \textbf{L:}
                            \textit{{#1}}
                            }
                        }}
        }
    %%%% Margin Note
    \IfValueT{#2}{
                        \marginnote{{\scriptsize
                            \textcolor{green}{ 
                            \textbf{L:}
                            \textit{{#1}}
                            }
                        }}
        }
                    }
\NewDocumentCommand{\giulia}{mo}{
    \IfValueF{#2}{
    %%%%% NO Margin Note
                        {{\scriptsize
                            \textcolor{red}{ 
                            \textbf{GL:}
                            \textit{{#1}}
                            }
                        }}
        }
    %%%% Margin Note
    \IfValueT{#2}{
                        \marginnote{{\scriptsize
                            \textcolor{red}{ 
                            \textbf{GL:}
                            \textit{{#1}}
                            }
                        }}
        }
}
\NewDocumentCommand{\anastasis}{mo}{
    \IfValueF{#2}{
    %%%%% NO Margin Note
                        {{\scriptsize
                            \textcolor{violet}{ 
                            \textbf{A:}
                            \textit{{#1}}
                            }
                        }}
        }
    %%%% Margin Note
    \IfValueT{#2}{
                        \marginnote{{\scriptsize
                            \textcolor{violet}{ 
                            \textbf{A:}
                            \textit{{#1}}
                            }
                        }}
        }
                    }
                    
\NewDocumentCommand{\cody}{mo}{
    \IfValueF{#2}{
    %%%%% NO Margin Note
                        {{\scriptsize
                            \textcolor{orange}{ 
                            \textbf{A:}
                            \textit{{#1}}
                            }
                        }}
        }
    %%%% Margin Note
    \IfValueT{#2}{
                        \marginnote{{\scriptsize
                            \textcolor{orange}{ 
                            \textbf{A:}
                            \textit{{#1}}
                            }
                        }}
        }
                    }

\newcommand{\byyan}[1]{\textcolor{cyan}{#1}}
\NewDocumentCommand{\yannick}{mo}{
    \IfValueF{#2}{
    %%%%% NO Margin Note
                        {{\scriptsize
                            \textcolor{cyan}{ 
                            \textbf{Y:}
                            \textit{{#1}}
                            }
                        }}
        }
    %%%% Margin Note
    \IfValueT{#2}{
                        \marginnote{{\scriptsize
                            \textcolor{cyan}{ 
                            \textbf{Y:}
                            \textit{{#1}}
                            }
                        }}
        }
                    } 

\definecolor{darkgreen}{rgb}{0.0, 0.2, 0.13}
\NewDocumentCommand{\xuwei}{mo}{
    \IfValueF{#2}{
    %%%%% NO Margin Note
                        {{\scriptsize
                            \textcolor{darkgreen}{ 
                            \textbf{X:}
                            \textit{{#1}}
                            }
                        }}
        }
    %%%% Margin Note
    \IfValueT{#2}{
                        \marginnote{{\scriptsize
                            \textcolor{darkgreen}{ 
                            \textbf{X:}
                            \textit{{#1}}
                            }
                        }}
        }
                    }

\usepackage{appendix}



\usepackage{cleveref}
\newcounter{termcounter}
\renewcommand{\thetermcounter}{\Roman{termcounter}}
\crefname{term}{term}{terms}
\creflabelformat{term}{#2\textup{(#1)}#3}

\makeatletter
\def\term{\@ifnextchar[\term@optarg\term@noarg}%]
\def\term@optarg[#1]#2{%
  \textup{#1}%
  \def\@currentlabel{#1}%
  \def\cref@currentlabel{[][2147483647][]#1}%
  \cref@label[term]{#2}}
\def\term@noarg#1{%
  \refstepcounter{termcounter}%
  \textup{(\thetermcounter)}%
  \cref@label[term]{#1}}
\makeatother


%  \setcounter{termcounter}{0}

\crefname{lemma}{lemma}{lemmata}
\Crefname{lemma}{Lemma}{Lemmata}
\crefname{assumption}{assumption}{assumptions}
\Crefname{assumption}{Assumption}{Assumptions}
\crefname{example}{Example}{Examples}
\crefname{proposition}{Proposition}{Proposition}


\usepackage{comment}
% \usepackage{microtype}



