\section{Proof of results regarding MCMC}\label{app:proof-MCMC}
Define $\bm{u}: \mathcal{S} \to \mathbb{R}^d$ as the solution to the \emph{Poission equation}
\begin{align*}
\bm{g}(x) := \bm{f}(x) - \mathbb{E}_{\mu}[\bm{f}] = \bm{u}(x) - \mathcal{P}\bm{u}(x), \quad \forall x \in \mathcal{S}.
\end{align*}
In fact, it is easy to verify that $\bm{u}(x)$ admits the following expression:
\begin{align*}
\bm{u}(x) = \sum_{k=0}^{\infty} \mathcal{P}^k \bm{g}(x) = \sum_{k=0}^{\infty} [\mathcal{P}^k \bm{f}(x)-\mu(\bm{f})].
\end{align*}
The mixing assumption \ref{as:mixing}, combined with the basic property of TV distance, directly implies that
\begin{align*}
\|\mathcal{P}^k \bm{f}(x) - \mu(\bm{f})\|_2   
&= \left\|\mathbb{E}_{s \sim P^k(\cdot \mid x)}[\bm{f}(s)] - \mathbb{E}_{s \sim \mu}[\bm{f}(s)]\right\|_2 \\ 
&\leq \sup_{x \in \mathcal{S}}\|\bm{f}(x)\|_2\cdot d_{\mathsf{TV}}(P^k(\cdot \mid x),\mu) \leq Mm\rho^k.
\end{align*}
Hence the norm of $\bm{u}(\cdot)$ is bounded uniformly by
\begin{align}\label{eq:MCMC-u-bound}
\|\bm{u}(x)\|_2 \leq \sum_{k=0}^{\infty} \|\mathcal{P}^k \bm{f}(x)-\mu(\bm{f})\|_2 \leq \frac{Mm}{1-\rho}, \quad \forall x \in \mathcal{S}.
\end{align}
Using the function $\bm{u}(\cdot)$, $\bar{\bm{f}}_n -\mu(\bm{f})$ can be represented through a telescoping technique by
\begin{align}\label{eq:MCMC-decompose}
\bar{\bm{f}}_n - \mu(\bm{f}) &= \frac{1}{n} \sum_{i=1}^n [\bm{f}(s_i) - \mu(\bm{f})] \nonumber \\ 
&=  \frac{1}{n} \sum_{i=1}^n[\bm{u}(s_i) - \mathcal{P}\bm{u}(s_i)] \nonumber \\ 
&= \frac{1}{n} \sum_{i=1}^n[\bm{u}(s_i) - \mathbb{E}_{i}\bm{u}(s_{i+1})] \nonumber \\ 
&= \frac{1}{n} \sum_{i=1}^n[\bm{u}(s_i) - \mathbb{E}_{i-1}\bm{u}(s_{i})] + \frac{1}{n} \sum_{i=1}^n[\mathbb{E}_{i-1}\bm{u}(s_{i})-\mathbb{E}_{i}\bm{u}(s_{i+1})]\nonumber \\
&= \frac{1}{n} \sum_{i=1}^n[\bm{u}(s_i) - \mathbb{E}_{i-1}\bm{u}(s_{i})] + \frac{1}{n} \left(\mathbb{E}_0[\bm{u}(s_1)] - \mathbb{E}_n[\bm{u}(s_{n+1})]\right)
\end{align}
For simplicity, we define a vector-valued function $\bm{m}: \mathcal{S}^2 \to \mathbb{R}^d$ as
\begin{align}\label{eq:MCMC-defn-m}
\bm{m}(s,s') = \bm{u}(s')-\mathcal{P}(s)
\end{align}
It can be easily verified that
\begin{align}
&\mathbb{E}_{s' \sim P(\cdot \mid s)}\bm{m}(s,s') = \bm{0}, \quad \forall s \in \mathcal{S}; \quad \text{and}\\ 
&\mathbb{E}_{s \sim \mu, s' \sim P(\cdot \mid s)}[\bm{m}(s,s')\bm{m}^\top(s,s')]=\tilde{\bm{\Sigma}}.
\end{align}
Furthermore, \eqref{eq:MCMC-decompose} and \eqref{eq:MCMC-defn-m} indicates that
\begin{align}\label{eq:MCMC-decompose-2}
\bar{\bm{f}}_n - \mu(\bm{f}) = \frac{1}{n}\sum_{i=1}^n \bm{m}(s_{i-1},s_i) + \frac{1}{n} \left(\mathbb{E}_0[\bm{u}(s_1)] - \mathbb{E}_n[\bm{u}(s_{n+1})]\right)
\end{align}
The Bernstein's inequality and Berry-Esseen bound for $\bar{\bm{f}}_n$ are both based on this decomposition.

\subsection{Proof of Theorem \ref{thm:MCMC-bernstein}}
By triangle inequality, the decomposition \eqref{eq:MCMC-decompose-2} and the universal boundedness of $\bm{u}(\cdot)$ \eqref{eq:MCMC-u-bound} directly implies that
\begin{align}\label{eq:MCMC-bernstein-triangle}
\left\|\bar{\bm{f}}_n - \mu(\bm{f})\right\|_2 \leq \left\|\frac{1}{n}\sum_{i=1}^n \bm{m}(s_{i-1},s_i)\right\|_2 + \frac{2Mm}{(1-\rho)}.
\end{align}
In order to bound the first term on the right-hand-side, consider the matrix-valued function $\bm{F}: \mathcal{S}^2 \to \mathbb{R}^{(d+1) \times (d+1)}$, defined as 
\begin{align*}
\bm{F}(s,s') = \begin{pmatrix}
0 & \bm{m}^\top(s,s') \\ 
\bm{m}(s,s') & \bm{0}_{d\times d}.
\end{pmatrix}, \quad \forall i \in [n].
\end{align*}
It can then be verified that 
\begin{align*}
\left\|\mathbb{E}_{s\sim \mu, s' \sim P(\cdot \mid s)}[\bm{F}^2(s,s')] \right\|= \mathbb{E}_{s\sim \mu, s' \sim P(\cdot \mid s)}\|\bm{m}(s,s')\|_2^2 = \mathsf{Tr}(\tilde{\bm{\Sigma}})
\end{align*}
and that
\begin{align*}
\|\bm{F}(s,s')\| \leq \frac{Mm}{1-\rho}, \quad \forall s,s' \in \mathcal{S}.
\end{align*}
Therefore, a direct application of Theorem \ref{thm:matrix-bernstein-mtg} on the sequence $\{\bm{F}(s_{i-1},s_i)\}_{i \in [n]}$, combined with \eqref{eq:MCMC-bernstein-triangle}, yields the conclusion of the theorem. 
