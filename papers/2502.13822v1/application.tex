\section{Application to TD learning with linear function approximation}\label{sec:TD}

In the second part of the paper, we apply our results to study the properties of the TD learning algorithm with linear function approximation under Markovian samples in RL settings. 
We begin by introducing some background on TD learning; see, e.g., \cite{sutton2018reinforcement}.
% with Polyak-Ruppert averaging~\eqref{eq:TD-update-all}. 
% with Markov samples and decaying stepsizes $\eta_t = \eta_0 t^{-\alpha}$ for $\alpha \in (\frac{1}{2},1)$. 



To illustrate equilibria and dynamics of performative prediction games, we focus on a scenario in which a \emph{duopoly} of mortgage companies, i.e. banks, compete to sell loans to customers.

\paragraph{Customer Model:} In our game, each bank is trying to attract customers from a given population $\mathcal{P}$. We model this population as comprised of individuals with a single-dimensional type: we denote individual $j$'s type as $y_j \in [0,1]$. For simplicity, we assume that \(y\) represents the customer’s probability of repaying the loan\footnote{In practice, a customer's (normalized) credit score can be interpreted as a noisy observation of $y_j$. This also corresponds to credit scores being \emph{calibrated}.}, i.e., $y_j := \P[Y_j = 1]$, where $Y_j$ is a random variable such that $Y_j = 0$ means that $j$ defaults on their loan, and $Y_j = 1$ means they repay their loan. Customer types in the population are drawn from a known distribution $D_y$ supported on $[0,1]$. 

\paragraph{Game between Banks:} Each Bank \(i \in \{1, 2\}\) selects two parameters \( (\tau_i, \gamma_i) := \theta_i\), where:
\begin{itemize}
    \item \(\tau_i \in \{\tau_l,\tau_h\}\) is the credit score threshold for approving a customer\footnote{We restrict the bank to only pick between two thresholds, $\tau_l$ and $\tau_h$. However, we highlight how our results are affected when we expand the strategy space to $n > 2$ actions in our experiments of Appendix \ref{app:3gamma}.}. Specifically, a customer $j$ with credit score \(y_j\) is approved by Bank $i$ if and only if \(y_j \geq \tau_i\);
    \item \(\gamma_i \in \{\gamma_l, \gamma_h\}\) is the interest rate offered to approved customers.
\end{itemize}
We denote as shorthand the space of allowable thresholds by $\Gamma := [0,1]$ and allowable interests rates by $\Lambda := [0,1]$. %The latter is set without loss of generality---we simply normalize the rates to be at most $1$. 
% {\color{red} Vidya: just thinking about this but is it natural to restrict interest rate to $1$? I don't think it would affect the equilibrium structure of the game but theoretically I think the interest rate could be anything in $[0,\infty)$.} {\color{green} Guanghui: Could we say something like this is without loss of generality} \gua{changed.}\juba{I think we repeated this twice, the next sentence already had this}
The loan amount is normalized to $1$ in the entire paper, without loss of generality; in this case, if a customer chooses Bank $i$, and the customer is approved by the bank at an interest rate of $\gamma_i$, the expected utility for the bank is equal to
\[
(1+\gamma_i)\cdot \P[Y_i = 1]-\P[Y_i = 0] = (1+\gamma_i)y_i-(1-y_i).
\]


%In practice, the credit score \(y\) serves as a noisy observation of the true likelihood of the customer's repayment. 

\paragraph{Banks' Utilities:} For given parameter choices \(\theta_1 = (\tau_1, \gamma_1)\) by Bank 1 and \(\theta_2 = (\tau_2, \gamma_2)\) by Bank 2, a \emph{rational} customer with credit score $y$ acts as follows:

\begin{enumerate}
    \item \textbf{Qualified for a single bank}: 
        \begin{itemize}
        \item If \(\tau_1 \leq y < \tau_2\), the customer goes to Bank 1, as the score qualifies for Bank 1 but not Bank 2. Conversely, if \(\tau_2 \leq y < \tau_1\), the customer chooses Bank 2.
    \end{itemize}
    \item \textbf{Qualified for both banks}:
     \begin{itemize}
        \item If \(\tau_1, \tau_2 \leq y\) and \(\gamma_1 < \gamma_2\), the customer selects Bank 1 for its lower interest rate. Conversely, if \(\gamma_1 > \gamma_2\), the customer chooses Bank 2.
        \item If \(\gamma_1 = \gamma_2\), the customer picks each bank with probability $1/2$. 
    \end{itemize}
    \item \textbf{Unqualified for both banks}:
    \begin{itemize}
        \item If \(y < \tau_1\) and \(y < \tau_2\), the customer is rejected by both banks.
    \end{itemize}
\end{enumerate}

The expected reward for Bank 1, denoted as \(u_1(\theta_1, \theta_2)\), can then be expressed as:
\begin{align}\label{eq:utility}
    u_1(\theta_1, \theta_2) 
    &=  \mathbb{E}_{y \sim D_y} \left[ \mathbb{I}\{\underbrace{\tau_1 \leq y < \tau_2 \ \cup \ (\tau_1, \tau_2 \leq y \ \cap \ \gamma_1 < \gamma_2)}_{\text{accepted by Bank 1}}\} \cdot \big((1+\gamma_1)y - (1-y)\big) \right] \nonumber\\
    & + \frac{1}{2} \mathbb{E}_{y \sim D_y} \left[ \mathbb{I}\{\underbrace{\tau_1, \tau_2 \leq y \ \cap \ \gamma_1 = \gamma_2}_{\text{accepted by both Banks}}\} \cdot \big((1+\gamma_1)y - (1-y)\big) \right].
\end{align}
Note that the problem is \emph{symmetric}, i.e., the utility function for Bank 2 can be derived by swapping the roles of \(\theta_1\) and \(\theta_2\). I.e., $u_2(\theta_1, \theta_2) = u_1(\theta_2, \theta_1)$. 

% If a bank only attracts customers between thresholds $\tau_a$ and $\tau_b$, for $\tau_a<\tau_b$, we call $[\tau_a,\tau_b]$ the \emph{threshold} range for that bank. For example, if Bank $1$ sets a threshold of $\tau_1$, Bank $2$ a threshold of $\tau_2 > \tau_1$, and $\gamma_1 > \gamma_2$, then Bank 1 has a threshold range of $[\tau_1,\tau_2]$, while bank $2$ has a threshold range of $[\tau_2,1]$.
% Note that the parameters set by \emph{both} banks, i.e. $(\theta_1,\theta_2)$ both influence the threshold range for each of Bank 1 and 2.  If $\tau_1>\tau_2$, $\gamma_1>\gamma_2$, then $\tau_a>\tau_b$, and the bank does not attract any customers. 
% {\color{red} is it possible for $\tau_a > \tau_b$, leading to the bank never attracting customers?} \gua{if $\gamma_1>\gamma_2$, $\tau_1>\tau_2$, then it gets no customer. I think it also makes sense.}\juba{I think we said we wanted to delete the discussion of the threshold range, no?}

% \noindent \textbf{Discrete Model}   
% We now present the discrete version of our model, where the interest rates and thresholds are selected from finite sets \(\Gamma\) and \(\Lambda\), respectively, with $\tau\in[0,1], \gamma\in[0,1]$,  for all $\tau\in\Lambda$ and $\gamma\in\Gamma$, \(|\Gamma| = n\) and \(|\Lambda| = m\). Let \(p_1, p_2 \in \Delta(\Gamma \times \Lambda)\) represent the mixed strategies of the two banks, where \(\Delta(\Gamma \times \Lambda)\) denotes the set of probability distributions over the discrete decision space \(\Gamma \times \Lambda\).


% \begin{Remark}
%    Note that our proposed problem can be reformulated as a standard multi-player performative prediction problem \citep{narang2023multiplayer}. However, in our problem, the data distribution faced by each learner breaks the Lipschitzness assumption of previous work~\citep{hardt2023performative,narang2023multiplayer}. A small modification in one of the learner's thresholds can completely change how demand is allocated across both learners, as is often the case in Bertrand-style games. 
% \end{Remark} 

% \gua{I made some changes to Remark 1, please have a look}
\begin{Remark}
   Previous works in multi-learner performative prediction~\citep{narang2023multiplayer} resort to an insensitivity assumption, i.e., the data distribution faced by each player can only changes slightly when the parameters also change slightly; formally, the data distribution faced by each player is Lipschitz in their decisions. This is immediately not true in our setting: the bank slightly changing its parameters can completely changes the demand distribution of customers it faces. Intuitively, this is because of Bertrand-competition-style effects, where if two banks have similar rates, one bank that lowers their rate by a small amount suddenly captures the entire customer demand that is eligible for that rate.%\juba{made further light edits adding intuition}
   
   In Appendix \ref{Appendix:refumulation}, we discuss this problem more carefully by reformulating our problem in the standard multi-learner performative prediction form given by~\citep{narang2023multiplayer}. We show the distribution is not Lipschitz with respect to the parameters, and thus does not satisfy the insensitivity assumption. 
%Prior work~\citep{hardt2023performative,narang2023multiplayer} showed that, for a general multi-agent performative prediction framework to work, insensitivity assumptions are needed: in the \textbf{worst case}, they can construct settings where the insensitivity assumption does not hold and simple dynamics do not converge anymore. We add nuance to this picture. We will show that our dynamics often converge, even absent insensitivity assumptions, highlighting that while the impossibility results of previous work hold in the worst case, they may not hold in the ``average case'' and especially not in problems motivated by applications. In particular, we will show convergence to a variety of equilibria of our game, and often to symmetric Nash equilibria where insensitivity is immediately violated.
     
\end{Remark}



% \paragraph{Relationship to Performative Prediction} A central point of our work is to highlight that \textcolor{red}{needs writing from intro}. We highlight how our work specifically ties to ``Performative Prediction'' below:


%\textcolor{red}{needs a definition environment}



%Here, \(\E_{\theta_1, \theta_2}\) represents the expected utility of the banks over their respective strategies \((\theta_1, \theta_2)\). These inequalities ensure that neither bank can unilaterally improve its expected utility by deviating from its mixed strategy in the equilibrium.



%and  for all $\tau\in\Gamma$, we have $\tau\in\Lambda$, $(\tau,\gamma)\in[0,1]^2$. Let $\Gamma\times\Lambda$
%In this paper, we focus on the most fundamental case, where there are two choices for each parameter: $0\leq\tau_{\ell}<\tau_{h}\leq 1$, and $0\leq \gamma_{\ell}< \gamma_{h}\leq 1$. In this case, the utility for each pair of decisions forms a $4\times4$ matrix (given in Table \ref{tab:my-table}). We consider the canonical case where $\tau_{\ell}=\frac{1}{2+\gamma_{h}}$, and $\tau_{h}=\frac{1}{2+\gamma_{\ell}}.$ Note that these are natural choices for the thresholds, in the sense that, if there is only one bank and the interest rate is set to be $\gamma$, then $\frac{1}{2+\gamma}$ is the optimal threshold corresponding to the fixed $\gamma$.


%and the thresholds are chosen in $\Lambda=\{\tau^{(1)},\dots,\tau^{(m)}\}$. Here, we only assume that, for each $\gamma\in\Gamma$, there at least exist one $\tau\in\Lambda$ such that $f(\gamma,\tau,1)>0$. Note that this is a very minor assumption, in the sense that, if for a $\gamma$ such that $f(\gamma,\tau,1)<0$ for all $\tau\in\Lambda$, then adopting this decision will lead to negative utility regardless of the opponent's decision, and thus is not an interesting case. 

%\textcolor{red}{The model section is missing the dynamic version of the game. We should clearly define the one-shot and the dynamic game}
% we only considered one-shot case in our paper




\subsection{Convergence and Berry-Essen bounds for TD estimator}

%\ale{We should cite \cite{fort2015central}. What other references are relevant here?}\weichen{checked.} 
It is known from results on stochastic approximations that, in fixed dimensions,  the TD estimator with polynomial-decaying stepsizes and Polyak-Ruppert averaging $\bar{\bm{\theta}}_T$ satisfies the central limit theorem (CLT) \citep[see, e.g.,][]{fort2015central,mou2020linear,li2023online}
\begin{align*}
    \sqrt{T}(\bar{\bm{\theta}}_T - \bm{\theta}^\star) \xrightarrow{d} \mathcal{N}(\bm{0},\tilde{\bm{\Lambda}}^\star),
\end{align*}
with asymptotic covariance matrix $\tilde{\bm{\Lambda}}^\star$ 
\begin{align}
    \label{eq:defn-tilde-Lambdastar}
    \tilde{\bm{\Lambda}}^\star = \bm{A}^{-1} \tilde{\bm{\Gamma}}\bm{A}^{-\top},
\end{align}
where $\tilde{\bm{\Gamma}}$ is the time-averaging covariance matrix
\begin{align}
\label{eq:defn-tilde-Gamma}
\tilde{\bm{\Gamma}}&=\lim_{T \to \infty} \mathbf{Var}_{s_0 \sim \mu,s_{t+1} \sim P(\cdot \mid s_t)} \left[\frac{1}{T}\sum_{t=1}^T (\bm{A}_t\bm{\theta}^\star - \bm{b}_t)\right]\nonumber \\ 
&= \mathbb{E}[(\bm{A}_1 \bm{\theta}^\star - \bm{b}_1)(\bm{A}_1 \bm{\theta}^\star - \bm{b}_1)^\top] + \sum_{t=2}^{\infty} \mathbb{E}[(\bm{A}_1 \bm{\theta}^\star - \bm{b}_1)(\bm{A}_{t} \bm{\theta}^\star - \bm{b}_{t})^\top + (\bm{A}_{t} \bm{\theta}^\star - \bm{b}_{t})(\bm{A}_1 \bm{\theta}^\star - \bm{b}_1)^\top].
\end{align}
% \yuting{use $t$ and $T$ instead of $i$ and $n$ to be consistent with theorem statement. } \weichen{changed the notation.}
However, the above results are asymptotic in nature and do not explicitly reveal the dependence on the dimension and other problem-related quantities. A line of recent work has made headway toward this goal, attaining non-asymptotic distributional characterization of the TD procedure. 
Nonetheless, most of the literature has focused on the independent setting where each $(\bm{A}_t,\bm{b}_t)$ pair is an independent and identically distributed random variable \citep[see, e.g.,][]{mou2020linear,wu2024statistical, samsonov2024gaussian}. 
To describe our results, throughout this section, we make the additional assumption that the Markov chain mixes exponentially fast, a standard condition in finite sample analyses of Markovian data. 
\medskip
\begin{customassumption}\label{as:mixing}
There exists constants $m>0,\rho\in (0,1)$, such that for every positive integer $t$,
\begin{align*}
\sup_{s \in \mathcal{S}} d_{\mathsf{TV}}(P^t(\cdot \mid s), \mu) \leq m \rho^t.
\end{align*}
\end{customassumption}
\medskip
We remark that, under this assumption, for any $\varepsilon \in (0,1)$, the corresponding \emph{mixing time} 
\begin{align}\label{eq:defn-tmix}
t_{\mix}(\varepsilon):= \min\{t:\sup_{s \in \mathcal{S}}d_{\mathsf{TV}}(P^t(\cdot \mid s),\mu) \leq \varepsilon\},
\end{align}
satisfies the bound
\begin{align}\label{eq:tmix-bound}
\tmix(\varepsilon) \leq \frac{ \log (m/\varepsilon)}{\log (1/\rho)}.
\end{align}

% \yuting{try to be consistent; sometimes we assume a unique stationary distribution, and sometimes we use irreducible and aperiodic}\weichen{checked.}
\medskip
\begin{customtheorem}[High-probability convergence of TD estimator]\label{thm:TD-whp}
Consider TD with Polyak-Ruppert averaging~\eqref{eq:TD-update-all} with Markov samples and decaying stepsizes $\eta_t = \eta_0 t^{-\alpha}$ for $\alpha \in (\frac{1}{2},1)$. %\ale{Forgot: why do we require $\alpha \in (\frac{1}{2},1)$?} \weichen{In principle, we need the stepsizes to satisfy $\sum_{t=1}^{\infty} \eta_t =\infty$ and $\sum_{t=1}^\infty \eta_t^2 \leq \infty$; in our proof, some upper bounds have $2\alpha-1$ and $1-\alpha$ in their denominators.}
Suppose that the Markov transition kernel has a unique stationary distribution, has a positive spectral gap, mixes exponentially as indicated by Assumption \ref{as:mixing}, and starts from a distribution satisfying Assumption \ref{as:nu}. Then for every tolerance level $\delta \in (0,1)$, there exists $\eta_0 = \eta_0(\delta)$ such that
\begin{align*}
\|\bar{\bm{\theta}}_T-\bm{\theta}^\star\|_2^2 \lesssim \sqrt{\frac{\mathsf{Tr}(\tilde{\bm{\Lambda}}^\star)}{T}\log \frac{d}{\delta}} + o\left(\sqrt{\frac{1}{T}}\log^{\frac{3}{2}} \frac{d}{\delta}\right),
\end{align*}
with probability at least $1-\delta$.
\end{customtheorem}
\medskip

Theorem~\ref{thm:TD-whp}, proved in Appendix \ref{app:proof-markov-deltat-convergence}, establishes the first high-probability convergence guarantee for the TD estimation error with Markov samples that matches the asymptotic variance $\tilde{\bm{\Lambda}}^\star/T$ up to a log factor.  It may be regarded as the Markovian counterpart of Theorem 3.1 in \cite{wu2024statistical}, which focuses on the TD estimation error with \emph{independent} samples under the same settings. 

% In a nutshell, Theorem \ref{thm:TD-whp} ensures that $\|\bar{\bm{\Delta}}_T\|_2^2$ is controlled by three terms: A \emph{linear leading term}, which converges by the rate of $O(\sqrt{{\mathsf{Tr}(\tilde{\bm{\Lambda}}_T)}\log \frac{d}{\delta}/{T}}).$
% \yuting{why not $\tilde{\bm{\Lambda}}^\star$?}
% Here, $\tilde{\bm{\Lambda}}_T$ is the \emph{non-asymptotic variance matrix}, whose exact form will be specified in the Appendix. We establish this bound by invoking our newly-established matrix Berstein inequality on Markovian martingales, Theorem \ref{thm:matrix-bernstein-mtg}; 
% A \emph{non-linear reminder}, which we analyze by tuning the induction argument for TD with independent samples to utilize the mixing property of the Markov chain;
% A \emph{Variance comparison term}, featuring the difference between the non-asymptotic variance $\tilde{\bm{\Lambda}}_T$ and the asymptotic variance $\bm{\Lambda}^\star$. This term is contained in the higher-order reminder in Theorem \ref{thm:TD-whp}. \yuting{This part is not super clear; are you trying to describe how you proved the result?}\weichen{Tried to illustrate the technical novelty. Maybe we can delete this paragraph and just briefly mention that we use Theorem \ref{thm:matrix-bernstein-mtg}, as I did in the next paragraph?}

Our analysis of the TD estimation error borrows tools and ideas from several previous contributions, starting from the seminal work of \cite{polyak1992acceleration} and including some of the most recent results; see, e.g, \cite{li2023online,samsonov2024gaussian,srikant2024rates}. In particular, we apply the induction technique of \cite{srikant2019finite}, \cite{li2023sharp} and \cite{wu2024statistical}, developed for TD learning with \emph{independent} samples. The generalization from independent to Markov samples is highly nontrivial, and the novel theoretical results obtained in the first part of the paper, especially the newly-established matrix Bernstein inequality for Markovian martingales (Corollary \ref{thm:matrix-bernstein-mtg}), play a critical role in our analysis.

\paragraph{Dependence on problem-related quantities.}
We point out that the initial stepsize $\eta_0$ depends on the probability tolerance level $\delta$ as well as other problem-related quantities, as specified in Eq.~\eqref{eq:deltat-condition-markov} in Appendix \ref{app:proof-TD-original}. Though not ideal, this dependence  %\weichen{I suggest we add the negative result illustrating this necessity in the appendix} \yuting{if reviewers ask}
also appears in the independent setting in \cite[Theorem 3.1]{wu2024statistical}, who further argues that it is unavoidable. It is also notable that the choice of $\eta_0$ does \emph{not} depend on the sample size $T$, as is the case in \cite{samsonov2023finitesample}, who also considers TD with Markov samples, though with a \emph{time-invariant} stepsize choice. 
% \yuting{I think it is very standard to have $\eta_0$ depending on problem parameters, do we need to emphasize this?} \weichen{Two AIStats reviewers raised this question, so I mentioned it. And also, the dependency of $\eta_0$ on problem-related quantities is different in this theorem and the next theorem.}
The upper bound in Theorem \ref{thm:TD-whp} depends on various problem-related quantities, including the mixing speed of the Markov chain (specifically, the mixing factor $\rho$ and the spectral gap $1-\lambda$), the discount factor $\gamma$, the initial stepsize $\eta_0$, the feature covariance matrix $\bm{\Sigma}$ and the time-averaging variance matrix of the TD error, $\tilde{\bm{\Gamma}}$. These intricate dependencies impact the higher-order (in $T$)  reminder terms, which is not shown in our bound but can be tracked in our proofs. %incorporated in the upper bound both through the asymptotic variance $\tilde{\bm{\Lambda}}^\star$ (which is independent of $\eta_0$), 

In our next and final result, we give a novel high-dimensional Berry-Esseen bound for the TD estimator assuming Markovian data. 

\medskip
\begin{customtheorem}[Berry-Esseen bound for TD estimator]\label{thm:TD-berry-esseen}
Consider TD with Polyak-Ruppert averaging~\eqref{eq:TD-update-all} with Markov samples and decaying stepsizes $\eta_t = \eta_0 t^{-\frac{3}{4}}$, where $\eta_0 < \frac{1}{2\lambda_{\Sigma}}$. Suppose that the Markov transition kernel has a unique stationary distribution, has a positive spectral gap, mixes exponentially as indicated by Assumption \ref{as:mixing}, and is initiated from a distribution $\nu$ satisfying Assumption \ref{as:nu}. Further assume that $\lambda_{\min}(\tilde{\bm{\Gamma}}) > 0$. Then when $T$ is sufficiently large\footnote{The exact constraint on $T$ is indicated in \eqref{eq:Lambda-T-condition} in the Appendix.}, 
\begin{align*}
d_{\mathsf{C}}(\sqrt{T}(\bar{\bm{\theta}}_T-\bm{\theta}^\star),\mathcal{N}(\bm{0},\tilde{\bm{\Lambda}}^\star)) \lesssim \tilde{C}T^{-\frac{1}{4}}\log T + o(T^{-\frac{1}{4}}),
\end{align*}
where $\widetilde{C}$ is a problem-related quantity independent of $T$, with exact form shown in Appendix \ref{app:proof-TD-Berry-Esseen}.
\end{customtheorem}
\medskip



% \yuting{Should we directly state the optimal choice inside Theorem~\ref{thm:TD-berry-esseen}? We can work with general $\alpha$ in the proof.}\weichen{checked}.

In a nutshell, Theorem~\ref{thm:TD-berry-esseen}, whose proof is given in Appendix \ref{app:proof-TD-Berry-Esseen}, ensures that the rescaled TD estimator with Polyak-Ruppert averaging converges to its Gaussian limit at a rate of $T^{-1/4}\log T$ (holding all other parameters fixed) with respect to the convex distance. 
The dependence on the feature dimension $d$, as well as other problem-related parameters, is unwieldy and thus not given explicitly in the statement of the theorem. However, in principle it can be tracked through the various steps of the proof.

The closest contribution to ours, and indeed the impetus for some of our work,  is the recent excellent paper by \cite{srikant2024rates}, which claims the bound 
\begin{align*}
    d_{\mathsf{W}}(\sqrt{T}(\bar{\bm{\theta}}_T-\bm{\theta}^\star),\mathcal{N}(\bm{0},\tilde{\bm{\Lambda}}^\star)) &\lesssim O(T^{-\frac{1}{4}}\log T).
\end{align*}
In our analysis, we have filled in the gaps in some of their arguments concerning vector-valued martingales and their application to TD learning, while also strengthening the bound from Wasserstein to convex distance.
Other recent and directly relevant contributions by \cite{samsonov2024gaussian} and \cite{wu2024statistical} have also produced high-dimensional Berry-Esseen bound for TD learning with \emph{independent samples},  with \cite{wu2024statistical} claiming the the state-of-the-art rate (in $T$) of $O(T^{-\frac{1}{3}})$. Though it remains to be seen whether the rate of convergence of order  $O(T^{-\frac{1}{4}}\log T)$ for Markovian data that we obtain in Theorem \ref{thm:TD-berry-esseen} is sharp, the generalization from independent to Markov samples is nonetheless highly nontrivial and, we believe,  significant. In any case, we are able to show in Appendix \ref{app:proof-Berry-Esseen-tight} that, for any $\alpha > \frac{3}{4}$ and when $T$ is sufficiently large,
\begin{align}\label{eq:TD-Berry-Esseen-tight}
d_{\mathsf{C}}(\sqrt{T}(\bar{\bm{\theta}}_T-\bm{\theta}^\star),\mathcal{N}(\bm{0},\tilde{\bm{\Lambda}}^\star)) &\gtrsim \Theta(T^{-\frac{1}{4}}\log T),\quad\text{for all } \alpha \in \left(\frac{3}{4},1\right),
\end{align}
a partial clue that our rate might in fact be optimal.


% Finally, we conclude by observing that choice of the stepsize in Theorem \ref{thm:TD-whp} and Theorem \ref{thm:TD-berry-esseen} are different: in Theorem \ref{thm:TD-whp}, $\eta_0$ depends on $\delta$ and other problem-related quantities like $\lambda_0$ and $\gamma$, and $\alpha$ can take any value between $\frac{1}{2}$ and $1$; however, in Theorem \ref{thm:TD-berry-esseen}, the initial stepsize $\eta_0$ can take any value less than $1/2\lambda_{\Sigma}$, while $\alpha$ is set to the specific value of $\frac{3}{4}$. In fact, our proof of Theorem~\ref{thm:TD-berry-esseen} allows for a general choice of $\alpha$; however, using other values of $\alpha$ other than $3/4$ appears to be suboptimal. \ale{Maybe we can move this comment to the appendix? It seems very detailed and technical. It would also save space}
% \paragraph{Comparison with previous works.} 
% To the best of our knowledge, Theorem \ref{thm:TD-berry-esseen} is the first \emph{valid} Berry-Esseen bound for TD with linear function approximation and Markov samples. 


% Similar to Theorem \ref{thm:TD-whp}, the non-asymptotic distribution of $\bar{\bm{\Delta}}_T$ is also decomposed into three parts. In our proof, shown in Appendix \ref{app:proof-TD-Berry-Esseen}, we address these three terms under any choice of $\alpha \in (\frac{1}{2},1)$ by:

% \begin{enumerate}
% \item The linear leading term is characterized by our newly established high-dimensional Berry-Esseen bound, Theorem \ref{thm:Berry-Esseen-mtg}. It yields a convergence rate of $O(T^{-\frac{1}{4}}\log T)$, independent of $\alpha$;
% \item The non-linear reminder is characterized by the mixing property of the Markov chain, as well as the L2 convergence rate of the original TD estimator. This term yields a convergence rate of $O((T^{\frac{1}{2}-\alpha} + T^{-\frac{\alpha}{3}})\log T)$;
% \item The variance comparison term is characterized by the TV distance between two zero-mean multi-dimensional Gaussian distributions, yielding a convergence rate of $O(T^{\alpha-1})$.
% \end{enumerate}

% \paragraph{Tightness of Theorem \ref{thm:TD-berry-esseen}.}





