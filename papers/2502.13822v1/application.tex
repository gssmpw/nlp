\section{Application to TD learning with linear function approximation}\label{sec:TD}

In the second part of the paper, we apply our results to study the properties of the TD learning algorithm with linear function approximation under Markovian samples in RL settings. 
We begin by introducing some background on TD learning; see, e.g., \cite{sutton2018reinforcement}.
% with Polyak-Ruppert averaging~\eqref{eq:TD-update-all}. 
% with Markov samples and decaying stepsizes $\eta_t = \eta_0 t^{-\alpha}$ for $\alpha \in (\frac{1}{2},1)$. 



\section{Model}
\label{sec:model}
Let $[N] = \{1, 2, \dots, N \}$ be a set of $N$ agents.
We examine an environment in which a system interacts with the agents over $T$ rounds.
Every round $t\leq T$ comprises $N$ \emph{sessions}, each session represents an encounter of the system with exactly one agent, and each agent interacts exactly once with the system every round.
I.e., in each round $t$ the agents arrive sequentially. 


\paragraph{Arrival order} The \emph{arrival order} of round $t$, denoted as $\ordv_t=(\ord_t(1),\dots, \ord_t(N))$, is an element from set of all permutations of $[N]$. Each entry $q$ in $\ordv_t$ is the index of the agent that arrives in the $q^{\text{th}}$ session of round $t$.
For example, if $\ord_t(1) = 2$, then agent $2$ arrives in the first session of round $t$.
Correspondingly, $\ord_t^{-1}(i)=q$ implies that agent $i$ arrives in the $q^{\text{th}}$ session of round $t$. 

As we demonstrate later, the arrival order has an immediate impact on agent rewards. We call the mechanism by which the arrival order is set \emph{arrival function} and denote it by $\ordname$. Throughout the paper, we consider several arrival functions such as the \emph{uniform arrival} function, denoted by $\uniord$, and the \emph{nudged arrival} $\sugord$; we introduce those formally in Sections~\ref{sec:uniform} and~\ref{sec:nudge}, respectively.

%We elaborate more on this concept in Section~\ref{sec: arrival}.


\paragraph{Arms} We consider a set of $K \geq 2$ arms, $A = \{a_1, \ldots, a_K\}$. The reward of arm $a_i$ in round $t$ is a random variable $X_i^t \sim \mathcal{D}^t_i$, where the rewards $(X_i^t)_{i,t}$ are mutually independent and bounded within the interval $[0,1]$. The reward distribution $\mathcal{D}^t_i$ of arm $a_i$, $i\in [K]$ at round $t\in T$ is assumed to be non-stationary but independent across arms and rounds. We denote the realized reward of arm $a_i$ in round $t$ by $x_i^t$. We assume \emph{reward consistency}, meaning that rewards may vary between rounds but remain constant within the sessions of a single round. Specifically, if an arm $a_i$ is selected multiple times during round~$t$, each selection yields the same reward $x_i^t$, where the superscript $t$ indicates its dependence on the round rather than the session. This consistency enables the system to leverage information obtained from earlier sessions to make more informed decisions in later sessions within the same round. We provide further details on this principle in Subsection~\ref{subsec:information}.


\paragraph{Algorithms} An algorithm is a mapping from histories to actions. We typically expect algorithms to maximize some aggregated agent metric like social welfare. Let $\mathcal H^{t,q}$ denote the information observed during all sessions of rounds $1$ to $t-1$ and sessions $1$ to $q-1$ in round $t$.  The history $\mathcal H^{t,q}$ is an element from $(A \times [0,1])^{(t-1) \cdot N +q-1}$, consisting of pairs of the form (pulled arm, realized reward). Notice that we restrict our attention to \emph{anonymous} algorithms, i.e., algorithms that do not distinguish between agents based on their identities. Instead, they only respond to the history of arms pulled and rewards observed, without conditioning on which specific agent performed each action.
%In the most general case, algorithms make decisions at session $q$ of round $t$  based on the entire history $\mathcal H^{t,q}$ and the index of the arriving agent $\ord_t(q)$. %Furthermore, we sometimes assume that algorithms have Bayesian information, i.e., algorithms are aware of the distributions $(\mathcal D_i)^K_{i=1}$. 
Furthermore, we sometimes assume that algorithms have Bayesian information, meaning they are aware of the reward distributions $(\mathcal{D}^t_i)_{i,t}$. If such an assumption is required to derive a result, we make it explicit. %Otherwise, we do not assume any additional knowledge about the algorithm’s information. %This distinction allows us to analyze both general algorithms without prior distributional knowledge and specialized algorithms that leverage Bayesian information.


\paragraph{Rewards} Let $\rt{i}$ denote the reward received by agent $i \in [N]$ at round $t$, and let $\Rt{i}$ denote her cumulative reward at the end of round $t$, i.e., $\Rt{i} = \sum_{\tau=1}^{t}{r^{\tau}_{i}}$. We further denote the \emph{social welfare} as the sum of the rewards all agents receive after $T$ rounds. Formally, $\sw=\sum^{N}_{i=1}{R^T_i}$. We emphasize that social welfare is independent of the arrival order. 


\paragraph{Envy}
We denote by $\adift{i}{j}$ the reward discrepancy of agents $i$ and $j$ in round $t$; namely, $\adift{i}{j}= \rt{i} - \rt{j}$. %We call this term \omer{name??} reward discrepancy in round $t$. 
The (cumulative) \emph{envy} between two agents at round $t$ is the difference in their cumulative rewards. Formally, $\env_{i,j}^t= \Rt{i} - \Rt{j}$ is the envy after $t$ rounds between agent $i$ and $j$. We can also formulate envy as the sum of reward discrepancies, $\env_{i,j}^t= \sum^{t}_{\tau=1}{\adif{i}{j}^\tau}$. Notice that envy is a signed quantity and can be either positive or negative. Specifically, if $\env_{i,j}^t < 0$, we say that agent $i$ envies agent $j$, and if $\env_{i,j}^t > 0$, agent $j$ envies agent $i$. The main goal of this paper is to investigate the behavior of the \emph{maximal envy}, defined as
\[
\env^t = \max_{i,j \in [N]} \env^t_{i,j}.
\]
For clarity, the term \emph{envy} will refer to the maximal envy.\footnote{ We address alternative definitions of envy in Section~\ref{sec:discussion}.} % Envy can also be defined in alternative ways, such as by averaging pairwise envy across all agents. We address average envy in Section~\ref{sec:avg_envy}.}
Note that $\env_{i,j}^t$ are random variables that depend on the decision-making algorithm, realized rewards, and the arrival order, and therefore, so is $\env^t$. If a result we obtain regarding envy depends on the arrival order $\ordname$, we write $\env^t(\ordname)$. Similarly, to ease notation, if $\ordname$ can be understood from the context, it is omitted.



\paragraph{Further Notation} We use the subscript $(q)$ to address elements of the $q^{\text{th}}$ session, for $q\in [N]$.
That is, we use the notation $\rt{(q)}$ to denote the reward granted to the agent that arrives in the $q^{\text{th}}$ session of round $t$ and $\Rt{(q)}$ to denote her cumulative reward. %Additionally, we introduce the notation $\at{(q)}$ to denote the arm pulled in that session.
Correspondingly, $\sdift{q}{w} = \rt{(q)} - \rt{(w)}$ is the reward discrepancy of the agents arriving in the $q^{\text{th}}$ and $w^{\text{th}}$ sessions of round $t$, respectively. 
To distinguish agents, arms, sessions and rounds, we use the letters $i,j$ to mark agents and arms, $q,w$ for sessions, and $t,\tau$ for rounds.


\subsection{Example}
\label{sec: example}
To illustrate the proposed setting and notation, we present the following example, which serves as a running example throughout the paper.

\begin{table}[t]
\centering
\begin{tabular}{|c|c|c|c|}
\hline
$t$ (round) & $\ordv_t$ (arrival order) & $x_1^t$ & $x_2^t$ \\ \hline
1           & 2, 1                     & 0.6     & 0.92    \\ \hline
2           & 1, 2                     & 0.48    & 0.1     \\ \hline
3           & 2, 1                     & 0.15    & 0.8     \\ \hline
\end{tabular}
\caption{
    Data for Example~\ref{example 1}.
}
\label{tbl: example}
\end{table}

\begin{algorithm}[t]
\caption{Algorithm for Example~\ref{example 1}}
\label{alguni}
\DontPrintSemicolon 
\For{round $t = 1$ to $T$}{
    pull $a_{1}$ in the first session\label{alguniexample: first}\\
    \lIf{$x^t_1 \geq \frac{1}{2}$}{pull $a_{1}$ again in second session \label{alguniexample: pulling a again}}
    \lElse{pull $a_{2}$ in second session \label{alguniexample: sopt else}}
}
\end{algorithm}


\begin{example}\label{example 1}
We consider $K=2$ uniform arms, $X_1,X_2 \sim \uni{0,1}$, and $N=2$ for some $T\geq 3$. We shall assume arm decision are made by Algorithm~\ref{alguni}: In the first session, the algorithm pulls $a_1$; if it yields a reward greater than $\nicefrac{1}{2}$, the algorithm pulls it again in the second session (the ``if'' clause). Otherwise, it pulls $a_2$.



We further assume that the arrival orders and rewards are as specified in Table~\ref{tbl: example}. Specifically, agent 2 arrives in the first session of round $t=1$, and pulling arm $a_2$ in this round would yield a reward of $x^1_2 = 0.92$. Importantly, \emph{this information is not available to the decision-making algorithm in advance} and is only revealed when or if the corresponding arms are pulled.




In the first round, $\boldsymbol{\eta}^1 = \left(2,1\right)$; thus, agent 2 arrives in the first session.
The algorithm pulls arm $a_1$, which means, $a^1_{(1)} = a_1$, and the agent receives $r_{2}^1=r_{(1)}^1=x_1^1=0.6$.
Later that round, in the second session, agent 1 arrives, and the algorithm pulls the same arm again since $x^1_1 = 0.6 \geq \nicefrac{1}{2}$ due to the ``if'' clause.
I.e., $a^1_{(2)} = a_1$ and $r_{1}^1 = r_{(2)}^1 = x_1^1 = 0.6$.
Even though the realized reward of arm $a_2$ in that round is higher ($0.92$), the algorithm is not aware of that value.
At the end of the first round, $R^1_1 = R^1_{(2)} = R^1_2 = R^1_{(1)} = 0.6$. The reward discrepancy is thus $\adif{1}{2}^1 = \adif{2}{1}^1= \sdif{2}{1}^1 = 0.6 - 0.6 =0$. 



In the second round, agent 1 arrives first, followed by agent 2.
Firstly, the algorithm pulls arm $a_1$ and agent 1 receives a reward of $r_{1}^2 = r_{(1)}^2 = x_1^2 = 0.48$.
Because the reward is lower than $\nicefrac{1}{2}$, in the second session the algorithm pulls the other arm ($a^2_{(2)} = a_2$), granting agent 2 a reward of $r_{2}^2 = r_{(2)}^2 = x_2^2 = 0.1$.
At the end of the second round, $R^2_1 = R^2_{(1)} = 0.6 + 0.48 = 1.08$ and $R^2_2 = R^2_{(2)} = 0.6 + 0.1 = 0.7$. Furthermore, $\sdif{2}{1}^2 = \adif{2}{1}^2 = r^2_{2} - r^2_{1} = 0.1 - 0.48 = -0.38$.

In the third and final round, agent 2 arrives first again, and receives a reward  of $0.15$ from $a_1$. When agent 1 arrives in the second session, the algorithm pulls arm $a_2$, and she receives a reward of $0.8$. As for the reward discrepancy, $\sdif{2}{1}^3 = \adif{2}{1}^3 = r^3_{2} - r^3_{1} = 0.15 - 0.8 = -0.75$. 

Finally, agent 1 has a cumulative reward of $R^3_1 = R^3_{(2)} = 0.6 + 0.48 + 0.8 = 1.88$, whereas agent~2 has a cumulative reward of $R^3_2 = R^3_{(1)} = 0.6 + 0.1 + 0.15 = 0.85$. In terms of envy, $\env^1_{1,2}= \adif{1}{2}^1 =0$, $\env^2_{1,2}=\adif{1}{2}^1+\adif{1}{2}^2= 0.38$, and $\env^3_{1,2} = -\env^3_{2,1} = R^3_1-R^3_2 = 1.88-0.85 = 1.03$, and consequently the envy in round 3 is $\env^3 = 1.03$.
\end{example}


\subsection{Information Exploitation}
\label{subsec:information}

In this subsection, we explain how algorithms can exploit intra-round information.
Since rewards are consistent in the sessions of each round, acquiring information in each session can be used to increase the reward of the following sessions.
In other words, the earlier sessions can be used for exploration, and we generally expect agents arriving in later sessions to receive higher rewards.
Taken to the extreme, an agent that arrives after all arms have been pulled could potentially obtain the highest reward of that round, depending on how the algorithm operates.

To further demonstrate the advantage of late arrival, we reconsider Example~\ref{example 1} and Algorithm~\ref{alguni}. 
The expected reward for the agent in the first session of round $t$ is $\E{\rt{(1)}}=\mu_1=\frac{1}{2}$, yet the expected reward of the agent in the second session is
\begin{align*}
\E{\rt{(2)}}=\E{\rt{(2)}\mid X^t_1 \geq \frac{1}{2} }\prb{X^t_1 \geq \frac{1}{2}} + \E{\rt{(2)}\mid X^t_1 < \frac{1}{2} }\prb{X^t_1 < \frac{1}{2}};
\end{align*}
thus, $\E{\rt{(2)}} =\E{X^t_1\mid X^t_1 \geq \frac{1}{2} }\cdot \frac{1}{2} + \mu_2\cdot\frac{1}{2} = \frac{5}{8}$.
Consequently, the expected welfare per round is $\E{\rt{(1)}+\rt{(2)}}=1+\frac{1}{8}$, and the benefit of arriving in the second session of any round $t$ is $\E{\rt{(2)} - \rt{(1)}} = \frac{1}{8}$. This gap creates envy over time, which we aim to measure and understand.
%This discrepancy generates envy over time, and our paper aims to better understand it.
\subsection{Socially Optimal Algorithms}
\label{sec: sw}
Since our model is novel, particularly in its focus on the reward consistency element, studying social welfare maximizing algorithms represents an important extension of our work. While the primary focus of this paper is to analyze envy under minimal assumptions about algorithmic operations, we also make progress in the direction of social welfare optimization. See more details in Section~\ref{sec:discussion}.%Due to space limitations, we defer the discussion on socially optimal algorithms to  \ifnum\Includeappendix=0{the appendix}\else{Section~\ref{appendix:sociallyopt}}\fi.




% Since our model is novel and specifically the reward consistency element, it might be interesting to study social welfare optimization. While the main focus of our paper is to study envy under minimal assumptions on how the algorithm operates, we take steps toward this direction as well. Due to space limitations, we defer the discussion on socially optimal algorithms to  \ifnum\Includeappendix=0{the appendix}\else{Section~\ref{appendix:sociallyopt}}\fi.  We devise a socially optimal algorithm for the two-agent case, offer efficient and optimal algorithms for special cases of $N>2$ agents, and an inefficient and approximately optimal algorithm for any instance with $N>2$. Moreover, we address the welfare-envy tradeoff in Section~\ref{sec:extensions}.


% Social welfare, unlike envy, is entirely independent of the arrival order. While the main focus of our paper is to study envy under minimal assumptions on how the algorithm operates, socially optimal algorithms might also be of interest. Due to space limitations, we defer the discussion on socially optimal algorithms to  \ifnum\Includeappendix=0{the appendix}\else{Section~\ref{appendix:sociallyopt}}\fi. We devise a socially optimal algorithm for the two-agent case, offer efficient and optimal algorithms for special cases of $N>2$ agents, and an inefficient and approximately optimal algorithm for any instance with $N>2$. %Furthermore, we treat the welfare-envy tradeoff of the special case of Example~\ref{example 1}.




\subsection{Convergence and Berry-Essen bounds for TD estimator}

%\ale{We should cite \cite{fort2015central}. What other references are relevant here?}\weichen{checked.} 
It is known from results on stochastic approximations that, in fixed dimensions,  the TD estimator with polynomial-decaying stepsizes and Polyak-Ruppert averaging $\bar{\bm{\theta}}_T$ satisfies the central limit theorem (CLT) \citep[see, e.g.,][]{fort2015central,mou2020linear,li2023online}
\begin{align*}
    \sqrt{T}(\bar{\bm{\theta}}_T - \bm{\theta}^\star) \xrightarrow{d} \mathcal{N}(\bm{0},\tilde{\bm{\Lambda}}^\star),
\end{align*}
with asymptotic covariance matrix $\tilde{\bm{\Lambda}}^\star$ 
\begin{align}
    \label{eq:defn-tilde-Lambdastar}
    \tilde{\bm{\Lambda}}^\star = \bm{A}^{-1} \tilde{\bm{\Gamma}}\bm{A}^{-\top},
\end{align}
where $\tilde{\bm{\Gamma}}$ is the time-averaging covariance matrix
\begin{align}
\label{eq:defn-tilde-Gamma}
\tilde{\bm{\Gamma}}&=\lim_{T \to \infty} \mathbf{Var}_{s_0 \sim \mu,s_{t+1} \sim P(\cdot \mid s_t)} \left[\frac{1}{T}\sum_{t=1}^T (\bm{A}_t\bm{\theta}^\star - \bm{b}_t)\right]\nonumber \\ 
&= \mathbb{E}[(\bm{A}_1 \bm{\theta}^\star - \bm{b}_1)(\bm{A}_1 \bm{\theta}^\star - \bm{b}_1)^\top] + \sum_{t=2}^{\infty} \mathbb{E}[(\bm{A}_1 \bm{\theta}^\star - \bm{b}_1)(\bm{A}_{t} \bm{\theta}^\star - \bm{b}_{t})^\top + (\bm{A}_{t} \bm{\theta}^\star - \bm{b}_{t})(\bm{A}_1 \bm{\theta}^\star - \bm{b}_1)^\top].
\end{align}
% \yuting{use $t$ and $T$ instead of $i$ and $n$ to be consistent with theorem statement. } \weichen{changed the notation.}
However, the above results are asymptotic in nature and do not explicitly reveal the dependence on the dimension and other problem-related quantities. A line of recent work has made headway toward this goal, attaining non-asymptotic distributional characterization of the TD procedure. 
Nonetheless, most of the literature has focused on the independent setting where each $(\bm{A}_t,\bm{b}_t)$ pair is an independent and identically distributed random variable \citep[see, e.g.,][]{mou2020linear,wu2024statistical, samsonov2024gaussian}. 
To describe our results, throughout this section, we make the additional assumption that the Markov chain mixes exponentially fast, a standard condition in finite sample analyses of Markovian data. 
\medskip
\begin{customassumption}\label{as:mixing}
There exists constants $m>0,\rho\in (0,1)$, such that for every positive integer $t$,
\begin{align*}
\sup_{s \in \mathcal{S}} d_{\mathsf{TV}}(P^t(\cdot \mid s), \mu) \leq m \rho^t.
\end{align*}
\end{customassumption}
\medskip
We remark that, under this assumption, for any $\varepsilon \in (0,1)$, the corresponding \emph{mixing time} 
\begin{align}\label{eq:defn-tmix}
t_{\mix}(\varepsilon):= \min\{t:\sup_{s \in \mathcal{S}}d_{\mathsf{TV}}(P^t(\cdot \mid s),\mu) \leq \varepsilon\},
\end{align}
satisfies the bound
\begin{align}\label{eq:tmix-bound}
\tmix(\varepsilon) \leq \frac{ \log (m/\varepsilon)}{\log (1/\rho)}.
\end{align}

% \yuting{try to be consistent; sometimes we assume a unique stationary distribution, and sometimes we use irreducible and aperiodic}\weichen{checked.}
\medskip
\begin{customtheorem}[High-probability convergence of TD estimator]\label{thm:TD-whp}
Consider TD with Polyak-Ruppert averaging~\eqref{eq:TD-update-all} with Markov samples and decaying stepsizes $\eta_t = \eta_0 t^{-\alpha}$ for $\alpha \in (\frac{1}{2},1)$. %\ale{Forgot: why do we require $\alpha \in (\frac{1}{2},1)$?} \weichen{In principle, we need the stepsizes to satisfy $\sum_{t=1}^{\infty} \eta_t =\infty$ and $\sum_{t=1}^\infty \eta_t^2 \leq \infty$; in our proof, some upper bounds have $2\alpha-1$ and $1-\alpha$ in their denominators.}
Suppose that the Markov transition kernel has a unique stationary distribution, has a positive spectral gap, mixes exponentially as indicated by Assumption \ref{as:mixing}, and starts from a distribution satisfying Assumption \ref{as:nu}. Then for every tolerance level $\delta \in (0,1)$, there exists $\eta_0 = \eta_0(\delta)$ such that
\begin{align*}
\|\bar{\bm{\theta}}_T-\bm{\theta}^\star\|_2^2 \lesssim \sqrt{\frac{\mathsf{Tr}(\tilde{\bm{\Lambda}}^\star)}{T}\log \frac{d}{\delta}} + o\left(\sqrt{\frac{1}{T}}\log^{\frac{3}{2}} \frac{d}{\delta}\right),
\end{align*}
with probability at least $1-\delta$.
\end{customtheorem}
\medskip

Theorem~\ref{thm:TD-whp}, proved in Appendix \ref{app:proof-markov-deltat-convergence}, establishes the first high-probability convergence guarantee for the TD estimation error with Markov samples that matches the asymptotic variance $\tilde{\bm{\Lambda}}^\star/T$ up to a log factor.  It may be regarded as the Markovian counterpart of Theorem 3.1 in \cite{wu2024statistical}, which focuses on the TD estimation error with \emph{independent} samples under the same settings. 

% In a nutshell, Theorem \ref{thm:TD-whp} ensures that $\|\bar{\bm{\Delta}}_T\|_2^2$ is controlled by three terms: A \emph{linear leading term}, which converges by the rate of $O(\sqrt{{\mathsf{Tr}(\tilde{\bm{\Lambda}}_T)}\log \frac{d}{\delta}/{T}}).$
% \yuting{why not $\tilde{\bm{\Lambda}}^\star$?}
% Here, $\tilde{\bm{\Lambda}}_T$ is the \emph{non-asymptotic variance matrix}, whose exact form will be specified in the Appendix. We establish this bound by invoking our newly-established matrix Berstein inequality on Markovian martingales, Theorem \ref{thm:matrix-bernstein-mtg}; 
% A \emph{non-linear reminder}, which we analyze by tuning the induction argument for TD with independent samples to utilize the mixing property of the Markov chain;
% A \emph{Variance comparison term}, featuring the difference between the non-asymptotic variance $\tilde{\bm{\Lambda}}_T$ and the asymptotic variance $\bm{\Lambda}^\star$. This term is contained in the higher-order reminder in Theorem \ref{thm:TD-whp}. \yuting{This part is not super clear; are you trying to describe how you proved the result?}\weichen{Tried to illustrate the technical novelty. Maybe we can delete this paragraph and just briefly mention that we use Theorem \ref{thm:matrix-bernstein-mtg}, as I did in the next paragraph?}

Our analysis of the TD estimation error borrows tools and ideas from several previous contributions, starting from the seminal work of \cite{polyak1992acceleration} and including some of the most recent results; see, e.g, \cite{li2023online,samsonov2024gaussian,srikant2024rates}. In particular, we apply the induction technique of \cite{srikant2019finite}, \cite{li2023sharp} and \cite{wu2024statistical}, developed for TD learning with \emph{independent} samples. The generalization from independent to Markov samples is highly nontrivial, and the novel theoretical results obtained in the first part of the paper, especially the newly-established matrix Bernstein inequality for Markovian martingales (Corollary \ref{thm:matrix-bernstein-mtg}), play a critical role in our analysis.

\paragraph{Dependence on problem-related quantities.}
We point out that the initial stepsize $\eta_0$ depends on the probability tolerance level $\delta$ as well as other problem-related quantities, as specified in Eq.~\eqref{eq:deltat-condition-markov} in Appendix \ref{app:proof-TD-original}. Though not ideal, this dependence  %\weichen{I suggest we add the negative result illustrating this necessity in the appendix} \yuting{if reviewers ask}
also appears in the independent setting in \cite[Theorem 3.1]{wu2024statistical}, who further argues that it is unavoidable. It is also notable that the choice of $\eta_0$ does \emph{not} depend on the sample size $T$, as is the case in \cite{samsonov2023finitesample}, who also considers TD with Markov samples, though with a \emph{time-invariant} stepsize choice. 
% \yuting{I think it is very standard to have $\eta_0$ depending on problem parameters, do we need to emphasize this?} \weichen{Two AIStats reviewers raised this question, so I mentioned it. And also, the dependency of $\eta_0$ on problem-related quantities is different in this theorem and the next theorem.}
The upper bound in Theorem \ref{thm:TD-whp} depends on various problem-related quantities, including the mixing speed of the Markov chain (specifically, the mixing factor $\rho$ and the spectral gap $1-\lambda$), the discount factor $\gamma$, the initial stepsize $\eta_0$, the feature covariance matrix $\bm{\Sigma}$ and the time-averaging variance matrix of the TD error, $\tilde{\bm{\Gamma}}$. These intricate dependencies impact the higher-order (in $T$)  reminder terms, which is not shown in our bound but can be tracked in our proofs. %incorporated in the upper bound both through the asymptotic variance $\tilde{\bm{\Lambda}}^\star$ (which is independent of $\eta_0$), 

In our next and final result, we give a novel high-dimensional Berry-Esseen bound for the TD estimator assuming Markovian data. 

\medskip
\begin{customtheorem}[Berry-Esseen bound for TD estimator]\label{thm:TD-berry-esseen}
Consider TD with Polyak-Ruppert averaging~\eqref{eq:TD-update-all} with Markov samples and decaying stepsizes $\eta_t = \eta_0 t^{-\frac{3}{4}}$, where $\eta_0 < \frac{1}{2\lambda_{\Sigma}}$. Suppose that the Markov transition kernel has a unique stationary distribution, has a positive spectral gap, mixes exponentially as indicated by Assumption \ref{as:mixing}, and is initiated from a distribution $\nu$ satisfying Assumption \ref{as:nu}. Further assume that $\lambda_{\min}(\tilde{\bm{\Gamma}}) > 0$. Then when $T$ is sufficiently large\footnote{The exact constraint on $T$ is indicated in \eqref{eq:Lambda-T-condition} in the Appendix.}, 
\begin{align*}
d_{\mathsf{C}}(\sqrt{T}(\bar{\bm{\theta}}_T-\bm{\theta}^\star),\mathcal{N}(\bm{0},\tilde{\bm{\Lambda}}^\star)) \lesssim \tilde{C}T^{-\frac{1}{4}}\log T + o(T^{-\frac{1}{4}}),
\end{align*}
where $\widetilde{C}$ is a problem-related quantity independent of $T$, with exact form shown in Appendix \ref{app:proof-TD-Berry-Esseen}.
\end{customtheorem}
\medskip



% \yuting{Should we directly state the optimal choice inside Theorem~\ref{thm:TD-berry-esseen}? We can work with general $\alpha$ in the proof.}\weichen{checked}.

In a nutshell, Theorem~\ref{thm:TD-berry-esseen}, whose proof is given in Appendix \ref{app:proof-TD-Berry-Esseen}, ensures that the rescaled TD estimator with Polyak-Ruppert averaging converges to its Gaussian limit at a rate of $T^{-1/4}\log T$ (holding all other parameters fixed) with respect to the convex distance. 
The dependence on the feature dimension $d$, as well as other problem-related parameters, is unwieldy and thus not given explicitly in the statement of the theorem. However, in principle it can be tracked through the various steps of the proof.

The closest contribution to ours, and indeed the impetus for some of our work,  is the recent excellent paper by \cite{srikant2024rates}, which claims the bound 
\begin{align*}
    d_{\mathsf{W}}(\sqrt{T}(\bar{\bm{\theta}}_T-\bm{\theta}^\star),\mathcal{N}(\bm{0},\tilde{\bm{\Lambda}}^\star)) &\lesssim O(T^{-\frac{1}{4}}\log T).
\end{align*}
In our analysis, we have filled in the gaps in some of their arguments concerning vector-valued martingales and their application to TD learning, while also strengthening the bound from Wasserstein to convex distance.
Other recent and directly relevant contributions by \cite{samsonov2024gaussian} and \cite{wu2024statistical} have also produced high-dimensional Berry-Esseen bound for TD learning with \emph{independent samples},  with \cite{wu2024statistical} claiming the the state-of-the-art rate (in $T$) of $O(T^{-\frac{1}{3}})$. Though it remains to be seen whether the rate of convergence of order  $O(T^{-\frac{1}{4}}\log T)$ for Markovian data that we obtain in Theorem \ref{thm:TD-berry-esseen} is sharp, the generalization from independent to Markov samples is nonetheless highly nontrivial and, we believe,  significant. In any case, we are able to show in Appendix \ref{app:proof-Berry-Esseen-tight} that, for any $\alpha > \frac{3}{4}$ and when $T$ is sufficiently large,
\begin{align}\label{eq:TD-Berry-Esseen-tight}
d_{\mathsf{C}}(\sqrt{T}(\bar{\bm{\theta}}_T-\bm{\theta}^\star),\mathcal{N}(\bm{0},\tilde{\bm{\Lambda}}^\star)) &\gtrsim \Theta(T^{-\frac{1}{4}}\log T),\quad\text{for all } \alpha \in \left(\frac{3}{4},1\right),
\end{align}
a partial clue that our rate might in fact be optimal.


% Finally, we conclude by observing that choice of the stepsize in Theorem \ref{thm:TD-whp} and Theorem \ref{thm:TD-berry-esseen} are different: in Theorem \ref{thm:TD-whp}, $\eta_0$ depends on $\delta$ and other problem-related quantities like $\lambda_0$ and $\gamma$, and $\alpha$ can take any value between $\frac{1}{2}$ and $1$; however, in Theorem \ref{thm:TD-berry-esseen}, the initial stepsize $\eta_0$ can take any value less than $1/2\lambda_{\Sigma}$, while $\alpha$ is set to the specific value of $\frac{3}{4}$. In fact, our proof of Theorem~\ref{thm:TD-berry-esseen} allows for a general choice of $\alpha$; however, using other values of $\alpha$ other than $3/4$ appears to be suboptimal. \ale{Maybe we can move this comment to the appendix? It seems very detailed and technical. It would also save space}
% \paragraph{Comparison with previous works.} 
% To the best of our knowledge, Theorem \ref{thm:TD-berry-esseen} is the first \emph{valid} Berry-Esseen bound for TD with linear function approximation and Markov samples. 


% Similar to Theorem \ref{thm:TD-whp}, the non-asymptotic distribution of $\bar{\bm{\Delta}}_T$ is also decomposed into three parts. In our proof, shown in Appendix \ref{app:proof-TD-Berry-Esseen}, we address these three terms under any choice of $\alpha \in (\frac{1}{2},1)$ by:

% \begin{enumerate}
% \item The linear leading term is characterized by our newly established high-dimensional Berry-Esseen bound, Theorem \ref{thm:Berry-Esseen-mtg}. It yields a convergence rate of $O(T^{-\frac{1}{4}}\log T)$, independent of $\alpha$;
% \item The non-linear reminder is characterized by the mixing property of the Markov chain, as well as the L2 convergence rate of the original TD estimator. This term yields a convergence rate of $O((T^{\frac{1}{2}-\alpha} + T^{-\frac{\alpha}{3}})\log T)$;
% \item The variance comparison term is characterized by the TV distance between two zero-mean multi-dimensional Gaussian distributions, yielding a convergence rate of $O(T^{\alpha-1})$.
% \end{enumerate}

% \paragraph{Tightness of Theorem \ref{thm:TD-berry-esseen}.}





