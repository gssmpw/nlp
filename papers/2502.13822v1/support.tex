\section{Proof of supportive lemmas and propositions}

\subsection{Proof of Proposition \ref{prop:PE}}\label{proof-prop-PE}
We address these recursive relations in order.
\paragraph{Proof of \eqref{eq:PE1}.} By definition and according to Proposition \ref{prop:P-top}, the left-hand-side is featured by
\begin{align*}
\left\|\left(\mathcal{P}^* \exp(t\bm{H})\bm{g}^{\parallel}\right)^{\parallel}\right\|_{\mu} &= \left\|\left(\exp(t\bm{H})\bm{g}^{\parallel}\right)^{\parallel}\right\|_{\mu} \\ 
&= \mathbb{E}_{\mu}\left[\exp(t\bm{H})\bm{g}^{\parallel}\right] = \left\|\mathbb{E}_{\mu} [\exp(t\bm{H})]\right\| \cdot \|\bm{g}^{\parallel}\|.
\end{align*}
Since $\mu(\bm{H}) = \bm{0}$, the expectation of $\exp(t\bm{H})$ can be bounded by
\begin{align*}
\left\|\mathbb{E}_{\mu}\exp(t\bm{H})\right\| = \left\|\sum_{k=0}^{\infty} \mathbb{E}_{\mu} \frac{1}{k!}(t\bm{H})^k\right\| &\leq \|\bm{I}\| + \sum_{k=2}^{\infty} \frac{1}{k!} \left\|\mathbb{E}_{\mu}(t\bm{H})^2 \right\| \\ 
&\leq 1 + \sum_{k=2}^{\infty} \frac{1}{k!} (tM)^k = \exp(tM) - tM.
\end{align*}
\paragraph{Proof of \eqref{eq:PE2}.} Since $\bm{g}^{\parallel}(x) = \mu(\bm{g}) = \|\bm{g}^{\parallel}\|_{\mu}$ is a constant function, the left hand side can be bounded by
\begin{align*}
\left\|\left(\mathcal{P}^* \exp(t\bm{H})\bm{g}^{\parallel}\right)^{\perp}\right\|_{\mu}&= \|\bm{g}^{\parallel}\|_{\mu} \cdot \left\|\left(\mathcal{P}^* \exp(t\bm{H})\right)^{\perp}\right\|_{\mu} = \|\bm{g}^{\parallel}\|_{\mu} \cdot \left\|\mathcal{P}^* \left(\exp(t\bm{H})\right)^{\perp}\right\|_{\mu} \\ 
&\leq \|\bm{g}^{\parallel}\|_{\mu} \cdot\lambda \left\| \left(\exp(t\bm{H})\right)^{\perp}\right\|_{\mu} \\ 
&= \|\bm{g}^{\parallel}\|_{\mu} \cdot\lambda \left\| \left(\exp(t\bm{H})-\bm{I}\right)^{\perp}\right\|_{\mu} \\ 
&\leq \|\bm{g}^{\parallel}\|_{\mu} \cdot\lambda \sup_{x \in \mathcal{S}} \left\| \exp(t\bm{H}(x))-\bm{I}\right\| \\ 
&\leq \|\bm{g}^{\parallel}\|_{\mu} \cdot\lambda (\exp(tM)-1).
\end{align*}
\paragraph{Proof of \eqref{eq:PE3}.} By definition, the left-hand-side can be represented as
\begin{align*}
\left\|\left(\mathcal{P}^* \exp(t\bm{H})\bm{g}^{\perp}\right)^{\parallel}\right\|_{\mu} = \left\|\mu\left(\mathcal{P}^* \exp(t\bm{H})\bm{g}^{\perp}\right)\right\| &= \left\|\mu(\exp(t\bm{H})\bm{g}^{\perp})\right\| \\ 
&= \left\|\mu((\exp(t\bm{H}) - \bm{I}) \bm{g}^{\perp})\right\| \\ 
&\leq \left\|(\exp(t\bm{H}) - \bm{I}) \bm{g}^{\perp}\right\|_{\mu} \\ 
&\leq \sup_{x \in \mathcal{S}} \left\|\exp(t\bm{H}) - \bm{I}\right\| \cdot \left\|\bm{g}^{\perp}\right\|_{\mu} \\ 
&\leq (\exp(tM)-1)  \left\|\bm{g}^{\perp}\right\|_{\mu}.
\end{align*}
\paragraph{Proof of \eqref{eq:PE4}.} As a direct implication of Proposition \ref{prop:P-top}, the left-hand-side is featured by
\begin{align*}
\left\|\left(\mathcal{P}^* \exp(t\bm{H})\bm{g}^{\perp}\right)^{\perp}\right\|_{\mu} = \left\|\mathcal{P}^* \left(\exp(t\bm{H})\bm{g}^{\perp}\right)^{\perp}\right\|_{\mu} 
&\leq \lambda \left\|\left(\exp(t\bm{H})\bm{g}^{\perp}\right)^{\perp}\right\|_{\mu} \\ 
&\leq \lambda \left\|\exp(t\bm{H})\bm{g}^{\perp}\right\|_{\mu} \\ 
&\leq \lambda \sup_{x \in \mathcal{S}}\|\exp(t\bm{H}(x))\| \|\bm{g}^{\perp}\|_{\mu} \\ 
&\leq \lambda \exp(tM) \|\bm{g}^{\perp}\|_{\mu}.
\end{align*}
 

\subsection{Proof of Lemma \ref{lemma:Uk}}\label{app:proof-lemma-Uk}
Direct calculation reveals
\begin{align}\label{eq:lemma-Uk-1}
\frac{\alpha_1 + \alpha_4}{2} + \frac{\sqrt{(\alpha_1 - \alpha_4)^2 + 4\alpha_3^2}}{2} \nonumber &= \alpha_1 + \frac{\sqrt{(\alpha_1 - \alpha_4)^2 + 4\alpha_3^2}}{2} - \frac{\alpha_1 -\alpha_4}{2} \nonumber \\ 
&= \alpha_1 + \frac{2\alpha_3^2}{\sqrt{(\alpha_1 - \alpha_4)^2 + 4\alpha_3^2} + (\alpha_1 -\alpha_4)}.
\end{align}
In what follows, we firstly illustrate that
\begin{align}\label{eq:lemma-Uk-2}
&\sqrt{(\alpha_1 - \alpha_4)^2 + 4\alpha_3^2} + (\alpha_1 -\alpha_4) \nonumber \\ 
&=\sqrt{((1-\lambda)e^x - x)^2 + 4(e^x-1)^2} + (1-\lambda)e^x - x\nonumber \\ 
&=: f(x) \geq 2(1-\lambda).
\end{align}
In fact, since $f(0) = 2(1-\lambda)$, it suffices to show that $f'(x) \geq 0$ for all $x \geq 0$. Towards this end, observe that
\begin{align*}
f'(x) &= \frac{((1-\lambda)e^x - x) ((1-\lambda) e^x-1) + 4(e^x-1)e^x}{\sqrt{((1-\lambda)e^x - x)^2 + 4(e^x-1)^2}} + (1-\lambda) e^x-1 \\ 
&= \frac{[(1-\lambda)e^x-1]\left[\sqrt{((1-\lambda)e^x - x)^2 + 4(e^x-1)^2} - ((1-\lambda)e^x - x)\right] + 4(e^x-1)e^x}{\sqrt{((1-\lambda)e^x - x)^2 + 4(e^x-1)^2}}
\end{align*}
Since the denominator is always positive, we now focus on showing that the numerator. Specifically, we discuss the following three cases:
\begin{enumerate}
\item If $(1-\lambda)e^x - 1 \geq 0$, then since $\sqrt{((1-\lambda)e^x - x)^2 + 4(e^x-1)^2} > (1-\lambda)e^x - x$, the numerator is positive;
\item If $(1-\lambda)e^x - 1 < 0$ and $(1-\lambda)e^x - x \geq 0$, then by triangle inequality,
\begin{align*}
\sqrt{((1-\lambda)e^x - x)^2 + 4(e^x-1)^2} - ((1-\lambda)e^x - x) \leq 2(e^x-1).
\end{align*}
Meanwhile, since $(1-\lambda)e^x - 1 > -1 > -e^x$, it can be guaranteed that
\begin{align*}
&[(1-\lambda)e^x-1]\left[\sqrt{((1-\lambda)e^x - x)^2 + 4(e^x-1)^2} - ((1-\lambda)e^x - x)\right] + 4(e^x-1)e^x \\ 
&> -e^x [2(e^x-1)]+ 4(e^x-1)e^x > 0.
\end{align*}
\item If $(1-\lambda)e^x - 1 < 0$ and $(1-\lambda)e^x - x < 0$, then also by triangle inequality,
\begin{align*}
\sqrt{((1-\lambda)e^x - x)^2 + 4(e^x-1)^2} - ((1-\lambda)e^x - x) &\leq 2(e^x-1) + 2(x-(1-\lambda)e^x) \\ 
& < 2(e^x-1 + x) < 4(e^x-1).
\end{align*}
Therefore, again because $(1-\lambda)e^x - 1 > -1 > -e^x$, the numerator is bounded below by
\begin{align*}
&[(1-\lambda)e^x-1]\left[\sqrt{((1-\lambda)e^x - x)^2 + 4(e^x-1)^2} - ((1-\lambda)e^x - x)\right] + 4(e^x-1)e^x \\ 
&> e^{-x} \cdot 4(e^x-1) + 4(e^x-1)e^x > 0.
\end{align*}
\end{enumerate}
In all three cases, we have proved that $f'(x) > 0$. This complete the proof of \eqref{eq:lemma-Uk-2}. As a direct consequence of \eqref{eq:lemma-Uk-1} and \eqref{eq:lemma-Uk-2}, we obtain
\begin{align*}
&\frac{\alpha_1 + \alpha_4}{2} + \frac{\sqrt{(\alpha_1 - \alpha_4)^2 + 4\alpha_3^2}}{2} \\ 
& \leq \alpha_1 + \frac{\alpha_3^2}{1-\lambda} = (e^x-x) + \frac{(e^x-1)^2}{1-\lambda}.
\end{align*}
We now proceed to further bound this upper bound. On one hand, when $x \in (0,1)$, It can be guaranteed that $e^x - x < 1+x^2$ and $e^x-1 < 2x$. Therefore,
\begin{align*}
(e^x-x) + \frac{(e^x-1)^2}{1-\lambda} &\leq 1 + x^2 + \frac{(2x)^2}{1-\lambda} \\ 
&< 1 + \frac{5x^2}{1-\lambda} < \exp\left(\frac{5x^2}{1-\lambda}\right)
\end{align*}
where we invoked the fact that $1+x < e^x$ in the last inequality. On the other hand, when $x > 1$, define
\begin{align*}
g(x) = \exp\left(\frac{5x^2}{1-\lambda}\right) - \left[(e^x-x) + \frac{(e^x-1)^2}{1-\lambda}\right];
\end{align*}
it is easy to illustrate that $g'(x) > 0$ for any $x > 1$, and therefore $g(x)$ is monotonically increasing with respect to $x$. This completes the proof of the Lemma.


\subsection{Proof of Proposition \ref{prop:Stein-smooth}}\label{app:proof-Stein-smooth}
This proposition is a generalization of Proposition 2.2 in \cite{gallouet2018regularity}, and the proofs are similar to each other. Recall from \cite{gallouet2018regularity}, proof of Proposition 2.2, that for any $\bm{\alpha} \in \mathbb{R}^d$ with $\|\bm{\alpha}\| =1$,
\begin{align*}
\bm{\alpha}^\top \nabla^2 f_g(\bm{x}) \bm{\alpha} &= -\int_0^1 \frac{1}{2(1-t)}\mathbb{E}\left[((\bm{\alpha}^\top \bm{z})^2 -1)g(\sqrt{t}\bm{x}+\sqrt{1-t}\bm{z})\right] \mathrm{d}t.
\end{align*}
Hence, the difference between $\nabla^2 f_g(\bm{x})$ and $\nabla^2 f_g(\bm{y})$ can be featured by
\begin{align}\label{eq:Stein-smooth-decompose}
&\bm{\alpha}^\top (\nabla^2 f_g(\bm{x})-\nabla^2 f_g(\bm{y})) \bm{\alpha} \nonumber \\ 
&= -\int_0^1 \frac{1}{2(1-t)}\mathbb{E}\left[((\bm{\alpha}^\top \bm{z})^2 -1)\left(g(\sqrt{t}\bm{x}+\sqrt{1-t}\bm{z})-g(\sqrt{t}\bm{y}+\sqrt{1-t}\bm{z})\right)\right] \mathrm{d}t \nonumber \\ 
&= -\underset{I_1}{\underbrace{\int_0^{1-\eta} \frac{1}{2(1-t)}\mathbb{E}\left[((\bm{\alpha}^\top \bm{z})^2 -1)\left(g(\sqrt{t}\bm{x}+\sqrt{1-t}\bm{z})-g(\sqrt{t}\bm{y}+\sqrt{1-t}\bm{z})\right)\right] \mathrm{d}t}} \nonumber \\ 
&- \underset{I_2}{\underbrace{\int_{1-\eta}^1 \frac{1}{2(1-t)}\mathbb{E}\left[((\bm{\alpha}^\top \bm{z})^2 -1)g(\sqrt{t}\bm{x}+\sqrt{1-t}\bm{z})\right] \mathrm{d}t}} \nonumber \\ 
&+ \underset{I_3}{\underbrace{\int_{1-\eta}^1 \frac{1}{2(1-t)}\mathbb{E}\left[((\bm{\alpha}^\top \bm{z})^2 -1)g(\sqrt{t}\bm{y}+\sqrt{1-t}\bm{z})\right] \mathrm{d}t}},
\end{align} 
where $\eta \in (0,1]$ is a variable to be determined later. We address the terms $I_1$, $I_2$ and $I_3$ accordingly.
\paragraph{Bounding $I_1$.} Since $g(\bm{x}) = h(\bm{\Sigma}^{\frac{1}{2}}\bm{x}+\bm{\mu})$ and $h \in \mathsf{Lip}_1$, it can be guaranteed that
\begin{align*}
\left|g(\sqrt{t}\bm{x}+\sqrt{1-t}\bm{z})-g(\sqrt{t}\bm{y}+\sqrt{1-t}\bm{z})\right| \leq \sqrt{t}\left\|\bm{\Sigma}^{\frac{1}{2}}(\bm{x} - \bm{y})\right\|_2;
\end{align*}
hence, $I_1$ can be bounded by
\begin{align*}
&\left|\int_0^{1-\eta} \frac{1}{2(1-t)}\mathbb{E}\left[((\bm{\alpha}^\top \bm{z})^2 -1)\left(g(\sqrt{t}\bm{x}+\sqrt{1-t}\bm{z})-g(\sqrt{t}\bm{y}+\sqrt{1-t}\bm{z})\right)\right] \mathrm{d}t\right| \\ 
&\leq \left\|\bm{\Sigma}^{\frac{1}{2}}(\bm{x} - \bm{y})\right\|_2 \mathbb{E}\left|(\bm{\alpha}^\top \bm{z})^2-1\right| \int_0^{1-\eta} \frac{\sqrt{t}}{2(1-t)}\mathrm{d}t.
\end{align*}
Here, since $\bm{z} \sim \mathcal{N}(\bm{0},\bm{I}_d)$ and $\|\bm{\alpha}\|_2 = 1$, we have $\bm{\alpha}^\top \bm{z} \sim \mathcal{N}(0,1)$ and therefore $(\bm{\alpha}^\top \bm{z})^2 \sim \chi^2(1)$. Consequently, $\mathbb{E}\left|(\bm{\alpha}^\top \bm{z})^2-1\right|$ is the standard error of $\chi^2(1)$ distribution, thus a universal constant. Meanwhile, the integral is bounded by
\begin{align*}
\int_0^{1-\eta} \frac{\sqrt{t}}{2(1-t)}\mathrm{d}t \leq \int_0^{1-\eta} \frac{1}{2(1-t)}\mathrm{d}t=-\frac{1}{2}(\log \eta).
\end{align*}
In combination, the term $I_1$ is bounded by
\begin{align}\label{eq:Stein-smooth-I1}
&\left|\int_0^{1-\eta} \frac{1}{2(1-t)}\mathbb{E}\left[((\bm{\alpha}^\top \bm{z})^2 -1)\left(g(\sqrt{t}\bm{x}+\sqrt{1-t}\bm{z})-g(\sqrt{t}\bm{y}+\sqrt{1-t}\bm{z})\right)\right] \mathrm{d}t\right| \nonumber \\ 
&\lesssim \left\|\bm{\Sigma}^{\frac{1}{2}}(\bm{x} - \bm{y})\right\|_2 (-\log \eta).
\end{align}

\paragraph{Bounding $I_2$.} Since $(\bm{\alpha}^\top \bm{z})^2 \sim \chi^2(1)$, we naturally have $\mathbb{E}[(\bm{\alpha}^\top \bm{z})^2] = 1$. Therefore, $I_2$ can be rephrased as
\begin{align*}
&\int_{1-\eta}^1 \frac{1}{2(1-t)}\mathbb{E}\left[((\bm{\alpha}^\top \bm{z})^2 -1)g(\sqrt{t}\bm{x}+\sqrt{1-t}\bm{z})\right] \mathrm{d}t\\ 
&= \int_{1-\eta}^1 \frac{1}{2(1-t)}\mathbb{E}\left[((\bm{\alpha}^\top \bm{z})^2 -1)\left(g(\sqrt{t}\bm{x}+\sqrt{1-t}\bm{z})-g(\sqrt{t}\bm{x})\right)\right] \mathrm{d}t,
\end{align*}
and its absolute value can be bounded by
\begin{align*}
&\left|\int_{1-\eta}^1 \frac{1}{2(1-t)}\mathbb{E}\left[((\bm{\alpha}^\top \bm{z})^2 -1)g(\sqrt{t}\bm{x}+\sqrt{1-t}\bm{z})\right] \mathrm{d}t\right| \\ 
&\leq \int_{1-\eta}^1 \frac{1}{2(1-t)}\mathbb{E}\left[|(\bm{\alpha}^\top \bm{z})^2 -1|\left|g(\sqrt{t}\bm{x}+\sqrt{1-t}\bm{z})-g(\sqrt{t}\bm{x})\right|\right] \mathrm{d}t \\ 
&\leq \int_{1-\eta}^1 \frac{1}{2(1-t)}\mathbb{E}\left[|(\bm{\alpha}^\top \bm{z})^2 -1| \sqrt{1-t}\|\bm{\Sigma}^{\frac{1}{2}}\bm{z}\|_2\right] \mathrm{d}t \\ 
&\leq \|\bm{\Sigma}^{\frac{1}{2}}\| \mathbb{E}[|(\bm{\alpha}^\top \bm{z})^2 -1|\|\bm{z}\|_2] \int_{1-\eta}^1 \frac{1}{2\sqrt{1-t}}\mathrm{d}t
\end{align*}
As is illustrated by Equation (26) in \cite{gallouet2018regularity}, the expectation is bounded by
\begin{align*}
\mathbb{E}[|(\bm{\alpha}^\top \bm{z})^2 -1|\|\bm{z}\|_2] \lesssim \sqrt{d},
\end{align*}
and the integral is bounded by
\begin{align*}
\int_{1-\eta}^1 \frac{1}{2\sqrt{1-t}}\mathrm{d}t \leq \sqrt{\eta}.
\end{align*}
So in combination, the term $I_2$ is bounded by
\begin{align}\label{eq:Stein-smooth-I2}
&\left|\int_{1-\eta}^1 \frac{1}{2(1-t)}\mathbb{E}\left[((\bm{\alpha}^\top \bm{z})^2 -1)g(\sqrt{t}\bm{x}+\sqrt{1-t}\bm{z})\right] \mathrm{d}t\right| \\ 
&\lesssim \sqrt{d}\|\bm{\Sigma}^{\frac{1}{2}}\| \sqrt{\eta}.
\end{align}
The term $I_3$ can be bounded by a similar manner.
\paragraph{Completing the proof.} Combining \eqref{eq:Stein-smooth-decompose}, \eqref{eq:Stein-smooth-I1} and \eqref{eq:Stein-smooth-I2} by triangle inequality, we obtain
\begin{align*}
&\left|\bm{\alpha}^\top (\nabla^2 f_g(\bm{x})-\nabla^2 f_g(\bm{y})) \bm{\alpha}\right| \\ 
&\lesssim \left\|\bm{\Sigma}^{\frac{1}{2}}(\bm{x} - \bm{y})\right\|_2 (-\log \eta) + \sqrt{d}\|\bm{\Sigma}^{\frac{1}{2}}\| \sqrt{\eta}.
\end{align*}
In the case where $2\|\bm{\Sigma}^{\frac{1}{2}}(\bm{x} - \bm{y})\|_2 >\sqrt{d}\|\bm{\Sigma}^{\frac{1}{2}}\|_2$, we can simply take $\eta=1$, yielding the bound
\begin{align}\label{eq:Stein-smooth-case1}
\left|\bm{\alpha}^\top (\nabla^2 f_g(\bm{x})-\nabla^2 f_g(\bm{y})) \bm{\alpha}\right| \lesssim \sqrt{d} \|\bm{\Sigma}^{\frac{1}{2}}\|;
\end{align}
otherwise, when $2\|\bm{\Sigma}^{\frac{1}{2}}(\bm{x} - \bm{y})\|_2 \leq \sqrt{d}\|\bm{\Sigma}^{\frac{1}{2}}\|$, we can set
\begin{align*}
\eta = \frac{4\|\bm{\Sigma}^{\frac{1}{2}}(\bm{x} - \bm{y})\|_2^2}{d\|\bm{\Sigma}\|},
\end{align*}
yielding the bound 
\begin{align}\label{eq:Stein-smooth-case2}
\left|\bm{\alpha}^\top (\nabla^2 f_g(\bm{x})-\nabla^2 f_g(\bm{y})) \bm{\alpha}\right| \lesssim 2\|\bm{\Sigma}^{\frac{1}{2}}(\bm{x} - \bm{y})\|_2 \left(1+\log \frac{\sqrt{d}\|\bm{\Sigma}^{\frac{1}{2}}\|}{\|\bm{\Sigma}^{\frac{1}{2}}(\bm{x} - \bm{y})\|_2}\right).
\end{align}
For simplicity, we use $f(x,a)$ to denote the piecewise function
\begin{align*}
f(x,a) = \begin{cases}
&x + x\log a - x \log x, \quad \text{if } x \in [0,a]; \\ 
&a, \quad \text{if } x > a.
\end{cases}
\end{align*}
It is easy to illustrate that
\begin{align*}
f(x) \leq (1+\log a)^+ x + e^{-1}.
\end{align*}
Therefore, by combining \eqref{eq:Stein-smooth-case1} and \eqref{eq:Stein-smooth-case2}, we obtain
\begin{align*}
\left|\bm{\alpha}^\top (\nabla^2 f_g(\bm{x})-\nabla^2 f_g(\bm{y})) \bm{\alpha}\right|  &\leq f(2\|\bm{\Sigma}^{\frac{1}{2}}(\bm{x} - \bm{y})\|_2,\sqrt{d}\|\bm{\Sigma}^{\frac{1}{2}}\|) \\ 
&\leq (2+\log (d\|\bm{\Sigma}\|))^+ \cdot \|\bm{\Sigma}^{\frac{1}{2}}(\bm{x} - \bm{y})\|_2 + e^{-1}.
\end{align*}

\subsection{Proof of Equation \eqref{eq:correlated-norms}}\label{app:proof-correlated-norms}
Essentially, it suffices to show that for any fixed matrix $\bm{A} \in \mathbb{R}^{d \times d}$ and random vector $\bm{x} \in \mathbb{R}^d$,
\begin{align}
\mathbb{E}\|\bm{Ax}\|_2^2 \mathbb{E}\|\bm{x}\|_2 \leq \mathbb{E}\|\bm{Ax}\|_2^2\|\bm{x}\|_2.
\end{align}
To see this, we use
\begin{align*}
\bm{A}^\top \bm{A} = \bm{PDP}^\top
\end{align*}
to denote the eigen decomposition of $\bm{A}^\top\bm{A}$, where $\bm{P}$ is a ortho-normal matrix and $\bm{D} = \text{diag}\{\lambda_1,...,\lambda_d\}$ where $\lambda_1 \geq \lambda_2 \geq ... \geq \lambda_d \geq 0$. Further denote $\bm{y} = \bm{Px}$, then the norms of $\bm{x}$ and $\bm{Ax}$ can be represented by
\begin{align*}
&\|\bm{Ax}\|_2^2 = \bm{x}^\top \bm{PDP}^\top \bm{x} = \bm{y}^\top \bm{Dy} = \sum_{i=1}^d \lambda_i y_i^2, \quad \text{and} \\ 
&\|\bm{x}\|_2 = \|\bm{y}\|_2 = \sqrt{\sum_{i=1}^d y_i^2}.
\end{align*}
For every $i \in [d]$, it is easy to verify that
\begin{align*}
y_i^2 \quad \text{and} \quad \|\bm{y}\|_2
\end{align*}
are positively correlated, and therefore
\begin{align*}
\mathbb{E}\|\bm{Ax}\|_2^2 \mathbb{E}\|\bm{x}\|_2 &= \mathbb{E}\left[\sum_{i=1}^d \lambda_i y_i^2\right]\mathbb{E}\|\bm{y}\|_2 = \sum_{i=1}^d \lambda_i \mathbb{E}[y_i^2] \mathbb{E}\|\bm{y}\|_2 \\ 
&\leq \sum_{i=1}^d \lambda_i \mathbb{E}[y_i^2 \|\bm{y}\|_2] = \mathbb{E}\left[\left(\sum_{i=1}^d \lambda_i y_i^2\right) \cdot \|\bm{y}\|_2 \right] = \mathbb{E}\|\bm{Ax}\|_2^2\|\bm{x}\|_2.
\end{align*}
Here, the inequality on the third line follows from the Chebyshev's association inequality.

\subsection{Proof of Lemma \ref{lemma:E-delta-tmix}}\label{app:proof-lemma-E-delta-tmix}
The TD update rule \eqref{eq:TD-update-all} directly implies that
\begin{align}\label{eq:theta-tmix-decompose}
\bm{\theta}_t - \bm{\theta}_{t-t_{\mix}}&= \sum_{i=t-t_{\mix}}^{t-1} (\bm{\theta}_{i+1} - \bm{\theta}_i) \nonumber \\ 
&= \sum_{i=t-t_{\mix}}^{t-1} \eta_i (\bm{A}_i\bm{\theta}_i-\bm{b}_i)\nonumber \\ 
&= \sum_{i=t-t_{\mix}}^{t-1} \eta_i (\bm{A}_i\bm{\theta}^\star-\bm{b}_i) + \sum_{i=t-t_{\mix}}^{t-1} \eta_i \bm{A}_i \bm{\Delta}_i.
\end{align}
We will apply this relation to prove the three bounds respectively.
\paragraph{Proof of Equation \eqref{eq:E-delta-tmix-1}.} By triangle inequality, \eqref{eq:theta-tmix-decompose} implies that
\begin{align*}
\mathbb{E}\|\bm{\theta}_t - \bm{\theta}_{t-t_{\mix}}\|_2 & \leq  \sum_{i=t-t_{\mix}}^{t-1} \eta_i \mathbb{E}\|(\bm{A}_i\bm{\theta}^\star-\bm{b}_i)\|_2 + \sum_{i=t-t_{\mix}}^{t-1} \eta_i \mathbb{E}\|\bm{A}_i \bm{\Delta}_i\| \\ 
&\leq \sum_{i=t-t_{\mix}}^{t-1} \eta_i (2\|\bm{\theta}^\star\|_2+1) + \sum_{i=t-t_{\mix}}^{t-1} \eta_i 2\mathbb{E}\|\bm{\Delta}_i\|_2\\ 
&\leq t_{\mix}\eta_{t-t_{\mix}}(2\|\bm{\theta^\star}\|_2 + 1)+ 2\eta_{t - t_{\mix}}\sum_{i=t-t_{\mix}}^{t-1}\mathbb{E}\|\bm{\Delta}_i\|_2,
\end{align*}
where the last line follows from the fact that the stepsizes $\{\eta_t\}_{t \geq 0}$ are non-increasing.
\paragraph{Proof of Equation \eqref{eq:E-delta-tmix-2}.} We firstly notice that for a set of $n$ vectors $\bm{x}_1,\bm{x}_2,...,\bm{x}_n$, it always holds true that
\begin{align*}
\left\|\sum_{i=1}^n \bm{x}_i\right\|_2^2 \leq n \sum_{i=1}^n \|\bm{x}_i\|_2^2.
\end{align*}
Therefore, \eqref{eq:theta-tmix-decompose} implies the following bound for $\mathbb{E}\|\bm{\theta}_t - \bm{\theta}_{t-t_{\mix}}\|_2^2$:
\begin{align*}
\mathbb{E}\|\bm{\theta}_t - \bm{\theta}_{t-t_{\mix}}\|_2^2 &\leq 2t_{\mix} \cdot \left\{\sum_{i=t-t_{\mix}}^{t-1} \eta_i^2 \mathbb{E}\|(\bm{A}_i\bm{\theta}^\star-\bm{b}_i)\|_2^2 + \sum_{i=t-t_{\mix}}^{t-1} \eta_i^2 \mathbb{E}\|\bm{A}_i \bm{\Delta}_i\|_2^2\right\} \\ 
&\leq 2t_{\mix}\cdot \left\{t_{\mix} \eta_{t-t_{\mix}}^2 (2\|\bm{\theta}^\star\|_2+1)^2 + 4 \eta_{t-t_{\mix}}^2 \sum_{i=t-t_{\mix}}^{t-1} \mathbb{E}\|\bm{\Delta}_i\|_2^2\right\}\\ 
&= 2t_{\mix}\eta_{t-t_{\mix}}^2\left[t_{\mix}(2\|\bm{\theta}^\star\|_2+1)^2 + 4 \sum_{i=t-t_{\mix}}^{t-1} \mathbb{E}\|\bm{\Delta}_i\|_2^2\right].
\end{align*}
\paragraph{Proof of Equation \eqref{eq:E-delta-tmix-3}.} By triangle inequality, \eqref{eq:theta-tmix-decompose} implies that
\begin{align*}
&\mathbb{E}[\|\bm{\Delta}_{t-t_{\mix}}\|_2 \|\bm{\theta}_{t} - \bm{\theta}_{t-t_{\mix}}\|_2] \\ 
&\leq \sum_{i=t-t_{\mix}}^{t-1} \eta_i \mathbb{E}[\|\bm{\Delta}_{t-t_{\mix}}\|_2 \|\bm{A}_i\bm{\theta}^\star-\bm{b}_i\|_2] + \sum_{i=t-t_{\mix}}^{t-1} \eta_i \mathbb{E}[\|\bm{\Delta}_{t-t_{\mix}}\|_2 \|\bm{A}_i\bm{\Delta}_i\|_2] \\ 
&\leq \sum_{i=t-t_{\mix}}^{t-1} \eta_i (2\|\bm{\theta}^\star\|_2+1)\mathbb{E}\|\bm{\Delta}_{t-t_{\mix}}\|_2 + \sum_{i=t-t_{\mix}}^{t-1} \eta_i \mathbb{E}[\frac{1}{2}\|\bm{\Delta}_{t-t_{\mix}}\|_2^2 + \frac{1}{2}\|\bm{A}_i\bm{\Delta}_i\|_2] \\ 
&\leq t_{\mix} \eta_{t-t_{\mix}}(2\|\bm{\theta}^\star\|_2+1)\mathbb{E}\|\bm{\Delta}_{t-t_{\mix}}\|_2 + \frac{1}{2}\eta_{t-t_{\mix}} \left(\mathbb{E}\|\bm{\Delta}_{t-t_{\mix}}\|_2^2 + 2\sum_{i=t-t_{\mix}}^{t-1} \|\bm{\Delta}_i\|_2^2\right).
\end{align*}
This completes the proof of the lemma. 

