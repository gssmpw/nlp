\documentclass[english]{article}

\usepackage{geometry}
\geometry{verbose,tmargin=1in,bmargin=1in,lmargin=1in,rmargin=1in}
%\usepackage{babel}
\usepackage[T1]{fontenc}
\usepackage[utf8]{inputenc}
\usepackage{bm}
\usepackage{graphicx}
\graphicspath{{../imgs/}}
\usepackage{tikz}
\usetikzlibrary{positioning, arrows.meta, calc}
\tikzset{
  box/.style={
    draw,
    rounded corners,
    minimum width=2cm,
    minimum height=1cm,
    align=center,
  },
  arrow/.style={
    ->,
    >=Latex,
    thick,
    draw=blue,
  },
  dashed-line/.style={
    dashed,
    thick,
    draw=blue,
  }
}

\usepackage{caption}
\usepackage{subcaption}
\usepackage{amssymb}
\usepackage{mathrsfs}
\usepackage{amsmath}
\usepackage{dsfont}
\usepackage{extarrows}
\usepackage{enumitem}
\usepackage{booktabs}
\usepackage{makecell}
\usepackage{tablefootnote}
%\usepackage{algorithm}
%\usepackage{algpseudocode}
\usepackage[linesnumbered,ruled,vlined]{algorithm2e}
\usepackage{xcolor}
\newcommand\mycommfont[1]{\footnotesize\ttfamily\textcolor{blue}{#1}}
\SetCommentSty{mycommfont}

\SetKwInput{KwInput}{Input}                % Set the Input
\SetKwInput{KwOutput}{Output}              % set the Output

\usepackage[unicode=true,
 bookmarks=false,
 breaklinks=false,pdfborder={0 0 1},colorlinks=false]
 {hyperref}
\hypersetup{
 colorlinks,citecolor=blue,filecolor=blue,linkcolor=blue,urlcolor=blue}

%\setlength{\parskip}{0.05 in}
\makeatletter
%%%%%%%%%%%%%%%%%%%%%%%%%%%%%% User specified LaTeX commands. 
\usepackage{amsthm}
%\usepackage{cite}  
\usepackage{comment}
\usepackage{natbib}

\graphicspath{ {./imgs/} }
%\frenchspacing

% \usepackage[inline]{showlabels}

\newtheorem{theorem}{Theorem}[section]
\newtheorem{proposition}[theorem]{Proposition}
\newtheorem{lemma}[theorem]{Lemma}
\newtheorem{corollary}[theorem]{Corollary}
\newtheorem{assumption}[theorem]{Assumption}
\newtheorem{conjecture}[theorem]{Conjecture}

% Create new counters for each theorem-like environment
\newcounter{theoremctr}
\newcounter{corollaryctr}
\newcounter{assumptionctr}
\newcounter{propositionctr}
\newcounter{lemmactr}


% Define new numbering for theorems, corollaries, and assumptions
\renewcommand{\thetheoremctr}{\arabic{theoremctr}}     
\renewcommand{\thecorollaryctr}{\arabic{corollaryctr}}  
\renewcommand{\theassumptionctr}{\arabic{assumptionctr}} 
\renewcommand{\thepropositionctr}{\arabic{propositionctr}} 
\renewcommand{\thelemmactr}{\arabic{lemmactr}}

% Define the new environments manually
\newenvironment{customtheorem}{\refstepcounter{theoremctr}\par\noindent\textbf{Theorem \thetheoremctr.}\itshape\ }{\par}
\newenvironment{customcorollary}{\refstepcounter{corollaryctr}\par\noindent\textbf{Corollary \thecorollaryctr.}\itshape\ }{\par}
\newenvironment{customassumption}{\refstepcounter{assumptionctr}\par\noindent\textbf{Assumption \theassumptionctr.}\itshape\ }{\par}
\newenvironment{customproposition}{\refstepcounter{propositionctr}\par\noindent\textbf{Proposition \thepropositionctr.}\itshape\ }{\par}
\newenvironment{customlemma}{\refstepcounter{lemmactr}\par\noindent\textbf{Lemma \thelemmactr.}\itshape\ }{\par}
\newcommand{\bfA}{\bm{A}}
\newcommand{\bfb}{\bm{b}}
\newcommand{\bfD}{\bm{D}}
\newcommand{\bfI}{\bm{I}}
\newcommand{\bfV}{\bm{V}}
\newcommand{\bfQ}{\bm{Q}}
\newcommand{\bfR}{\bm{R}}
\newcommand{\bfS}{\bm{S}}
\newcommand{\bfX}{\bm{X}}
\newcommand{\bfZ}{\bm{Z}}
\newcommand{\bfx}{\bm{x}}
\newcommand{\bfy}{\bm{y}}
\newcommand{\mcA}{\mathcal{A}}
\newcommand{\mcN}{\mathcal{N}}
\newcommand{\mcH}{\mathcal{H}}
\newcommand{\bfSigma}{\bm{\Sigma}}
\newcommand{\bfPi}{\bm{\Pi}}
\newcommand{\bftheta}{\bm{\theta}}
\newcommand{\bfPhi}{\bm{\Phi}}
\newcommand{\bfDelta}{\bm{\Delta}}
\newcommand{\bfzeta}{\bm{\zeta}}
\newcommand{\bfxi}{\bm{\xi}}
\newcommand{\bfomega}{\bm{\omega}}
\newcommand{\bbP}{\mathbb{P}}
\newcommand{\EA}{\bm{E}_A}
\newcommand{\eb}{\bm{e}_b}
\newcommand{\tmix}{t_{\mathsf{mix}}} 
\newcommand{\mix}{\mathsf{mix}}
\newcommand{\cS}{\mathcal{S}}
\newcommand{\cA}{\mathcal{A}}

\allowdisplaybreaks
\let\hat\widehat
\let\bar\overline
\let\tilde\widetilde

\newcommand{\red}[1]{\textcolor{red}{#1}}
\newcommand{\blue}[1]{\textcolor{blue}{#1}}
\newcommand{\violet}[1]{\textcolor{violet}{#1}}
\newcommand{\yuting}[1]{\red{Yuting: #1}}
\newcommand{\ale}[1]{\violet{*ALE* #1}}
\newcommand{\weichen}[1]{\blue{Weichen: #1}}

\title{Uncertainty quantification for Markov chains with application to temporal difference learning}

\author{
  Weichen Wu\thanks{The Voleon Group, Berkeley, CA 94704 USA.}\\
  \and
  Yuting Wei\thanks{Department of Statistics and Data Science, The Wharton School, University of Pennsylvania, Philadelphia, PA 19104, USA.}  \\
  \and 
  Alessandro Rinaldo\thanks{Department of Statistics and Data Sciences, University of Texas, Austin TX 78705, USA.}
  } 
\date{\today}

\begin{document}
% \begin{onehalfspace}

\maketitle

\begin{abstract}
Markov chains are fundamental to statistical machine learning, underpinning key methodologies such as Markov Chain Monte Carlo (MCMC) sampling and temporal difference (TD) learning in reinforcement learning (RL). Given their widespread use, it is crucial to establish rigorous probabilistic guarantees on their convergence, uncertainty, and stability.
In this work, we develop novel, high-dimensional concentration inequalities and Berry-Esseen bounds for vector- and matrix-valued functions of Markov chains, addressing key limitations in existing theoretical tools for handling dependent data. We leverage these results to analyze the TD learning algorithm, a widely used method for policy evaluation in RL. Our analysis yields a sharp high-probability consistency guarantee that matches the asymptotic variance up to logarithmic factors. Furthermore, we establish a $O(T^{-\frac{1}{4}}\log T)$ distributional convergence rate for the Gaussian approximation of the TD estimator, measured in convex distance. These findings provide new insights into statistical inference for RL algorithms, bridging the gaps between classical stochastic approximation theory and modern reinforcement learning applications.
\end{abstract}

\setcounter{tocdepth}{2}
\tableofcontents

\section{Introduction}


\begin{figure}[t]
\centering
\includegraphics[width=0.6\columnwidth]{figures/evaluation_desiderata_V5.pdf}
\vspace{-0.5cm}
\caption{\systemName is a platform for conducting realistic evaluations of code LLMs, collecting human preferences of coding models with real users, real tasks, and in realistic environments, aimed at addressing the limitations of existing evaluations.
}
\label{fig:motivation}
\end{figure}

\begin{figure*}[t]
\centering
\includegraphics[width=\textwidth]{figures/system_design_v2.png}
\caption{We introduce \systemName, a VSCode extension to collect human preferences of code directly in a developer's IDE. \systemName enables developers to use code completions from various models. The system comprises a) the interface in the user's IDE which presents paired completions to users (left), b) a sampling strategy that picks model pairs to reduce latency (right, top), and c) a prompting scheme that allows diverse LLMs to perform code completions with high fidelity.
Users can select between the top completion (green box) using \texttt{tab} or the bottom completion (blue box) using \texttt{shift+tab}.}
\label{fig:overview}
\end{figure*}

As model capabilities improve, large language models (LLMs) are increasingly integrated into user environments and workflows.
For example, software developers code with AI in integrated developer environments (IDEs)~\citep{peng2023impact}, doctors rely on notes generated through ambient listening~\citep{oberst2024science}, and lawyers consider case evidence identified by electronic discovery systems~\citep{yang2024beyond}.
Increasing deployment of models in productivity tools demands evaluation that more closely reflects real-world circumstances~\citep{hutchinson2022evaluation, saxon2024benchmarks, kapoor2024ai}.
While newer benchmarks and live platforms incorporate human feedback to capture real-world usage, they almost exclusively focus on evaluating LLMs in chat conversations~\citep{zheng2023judging,dubois2023alpacafarm,chiang2024chatbot, kirk2024the}.
Model evaluation must move beyond chat-based interactions and into specialized user environments.



 

In this work, we focus on evaluating LLM-based coding assistants. 
Despite the popularity of these tools---millions of developers use Github Copilot~\citep{Copilot}---existing
evaluations of the coding capabilities of new models exhibit multiple limitations (Figure~\ref{fig:motivation}, bottom).
Traditional ML benchmarks evaluate LLM capabilities by measuring how well a model can complete static, interview-style coding tasks~\citep{chen2021evaluating,austin2021program,jain2024livecodebench, white2024livebench} and lack \emph{real users}. 
User studies recruit real users to evaluate the effectiveness of LLMs as coding assistants, but are often limited to simple programming tasks as opposed to \emph{real tasks}~\citep{vaithilingam2022expectation,ross2023programmer, mozannar2024realhumaneval}.
Recent efforts to collect human feedback such as Chatbot Arena~\citep{chiang2024chatbot} are still removed from a \emph{realistic environment}, resulting in users and data that deviate from typical software development processes.
We introduce \systemName to address these limitations (Figure~\ref{fig:motivation}, top), and we describe our three main contributions below.


\textbf{We deploy \systemName in-the-wild to collect human preferences on code.} 
\systemName is a Visual Studio Code extension, collecting preferences directly in a developer's IDE within their actual workflow (Figure~\ref{fig:overview}).
\systemName provides developers with code completions, akin to the type of support provided by Github Copilot~\citep{Copilot}. 
Over the past 3 months, \systemName has served over~\completions suggestions from 10 state-of-the-art LLMs, 
gathering \sampleCount~votes from \userCount~users.
To collect user preferences,
\systemName presents a novel interface that shows users paired code completions from two different LLMs, which are determined based on a sampling strategy that aims to 
mitigate latency while preserving coverage across model comparisons.
Additionally, we devise a prompting scheme that allows a diverse set of models to perform code completions with high fidelity.
See Section~\ref{sec:system} and Section~\ref{sec:deployment} for details about system design and deployment respectively.



\textbf{We construct a leaderboard of user preferences and find notable differences from existing static benchmarks and human preference leaderboards.}
In general, we observe that smaller models seem to overperform in static benchmarks compared to our leaderboard, while performance among larger models is mixed (Section~\ref{sec:leaderboard_calculation}).
We attribute these differences to the fact that \systemName is exposed to users and tasks that differ drastically from code evaluations in the past. 
Our data spans 103 programming languages and 24 natural languages as well as a variety of real-world applications and code structures, while static benchmarks tend to focus on a specific programming and natural language and task (e.g. coding competition problems).
Additionally, while all of \systemName interactions contain code contexts and the majority involve infilling tasks, a much smaller fraction of Chatbot Arena's coding tasks contain code context, with infilling tasks appearing even more rarely. 
We analyze our data in depth in Section~\ref{subsec:comparison}.



\textbf{We derive new insights into user preferences of code by analyzing \systemName's diverse and distinct data distribution.}
We compare user preferences across different stratifications of input data (e.g., common versus rare languages) and observe which affect observed preferences most (Section~\ref{sec:analysis}).
For example, while user preferences stay relatively consistent across various programming languages, they differ drastically between different task categories (e.g. frontend/backend versus algorithm design).
We also observe variations in user preference due to different features related to code structure 
(e.g., context length and completion patterns).
We open-source \systemName and release a curated subset of code contexts.
Altogether, our results highlight the necessity of model evaluation in realistic and domain-specific settings.






%\section{Model}
\label{sec:model}
Let $[N] = \{1, 2, \dots, N \}$ be a set of $N$ agents.
We examine an environment in which a system interacts with the agents over $T$ rounds.
Every round $t\leq T$ comprises $N$ \emph{sessions}, each session represents an encounter of the system with exactly one agent, and each agent interacts exactly once with the system every round.
I.e., in each round $t$ the agents arrive sequentially. 


\paragraph{Arrival order} The \emph{arrival order} of round $t$, denoted as $\ordv_t=(\ord_t(1),\dots, \ord_t(N))$, is an element from set of all permutations of $[N]$. Each entry $q$ in $\ordv_t$ is the index of the agent that arrives in the $q^{\text{th}}$ session of round $t$.
For example, if $\ord_t(1) = 2$, then agent $2$ arrives in the first session of round $t$.
Correspondingly, $\ord_t^{-1}(i)=q$ implies that agent $i$ arrives in the $q^{\text{th}}$ session of round $t$. 

As we demonstrate later, the arrival order has an immediate impact on agent rewards. We call the mechanism by which the arrival order is set \emph{arrival function} and denote it by $\ordname$. Throughout the paper, we consider several arrival functions such as the \emph{uniform arrival} function, denoted by $\uniord$, and the \emph{nudged arrival} $\sugord$; we introduce those formally in Sections~\ref{sec:uniform} and~\ref{sec:nudge}, respectively.

%We elaborate more on this concept in Section~\ref{sec: arrival}.


\paragraph{Arms} We consider a set of $K \geq 2$ arms, $A = \{a_1, \ldots, a_K\}$. The reward of arm $a_i$ in round $t$ is a random variable $X_i^t \sim \mathcal{D}^t_i$, where the rewards $(X_i^t)_{i,t}$ are mutually independent and bounded within the interval $[0,1]$. The reward distribution $\mathcal{D}^t_i$ of arm $a_i$, $i\in [K]$ at round $t\in T$ is assumed to be non-stationary but independent across arms and rounds. We denote the realized reward of arm $a_i$ in round $t$ by $x_i^t$. We assume \emph{reward consistency}, meaning that rewards may vary between rounds but remain constant within the sessions of a single round. Specifically, if an arm $a_i$ is selected multiple times during round~$t$, each selection yields the same reward $x_i^t$, where the superscript $t$ indicates its dependence on the round rather than the session. This consistency enables the system to leverage information obtained from earlier sessions to make more informed decisions in later sessions within the same round. We provide further details on this principle in Subsection~\ref{subsec:information}.


\paragraph{Algorithms} An algorithm is a mapping from histories to actions. We typically expect algorithms to maximize some aggregated agent metric like social welfare. Let $\mathcal H^{t,q}$ denote the information observed during all sessions of rounds $1$ to $t-1$ and sessions $1$ to $q-1$ in round $t$.  The history $\mathcal H^{t,q}$ is an element from $(A \times [0,1])^{(t-1) \cdot N +q-1}$, consisting of pairs of the form (pulled arm, realized reward). Notice that we restrict our attention to \emph{anonymous} algorithms, i.e., algorithms that do not distinguish between agents based on their identities. Instead, they only respond to the history of arms pulled and rewards observed, without conditioning on which specific agent performed each action.
%In the most general case, algorithms make decisions at session $q$ of round $t$  based on the entire history $\mathcal H^{t,q}$ and the index of the arriving agent $\ord_t(q)$. %Furthermore, we sometimes assume that algorithms have Bayesian information, i.e., algorithms are aware of the distributions $(\mathcal D_i)^K_{i=1}$. 
Furthermore, we sometimes assume that algorithms have Bayesian information, meaning they are aware of the reward distributions $(\mathcal{D}^t_i)_{i,t}$. If such an assumption is required to derive a result, we make it explicit. %Otherwise, we do not assume any additional knowledge about the algorithm’s information. %This distinction allows us to analyze both general algorithms without prior distributional knowledge and specialized algorithms that leverage Bayesian information.


\paragraph{Rewards} Let $\rt{i}$ denote the reward received by agent $i \in [N]$ at round $t$, and let $\Rt{i}$ denote her cumulative reward at the end of round $t$, i.e., $\Rt{i} = \sum_{\tau=1}^{t}{r^{\tau}_{i}}$. We further denote the \emph{social welfare} as the sum of the rewards all agents receive after $T$ rounds. Formally, $\sw=\sum^{N}_{i=1}{R^T_i}$. We emphasize that social welfare is independent of the arrival order. 


\paragraph{Envy}
We denote by $\adift{i}{j}$ the reward discrepancy of agents $i$ and $j$ in round $t$; namely, $\adift{i}{j}= \rt{i} - \rt{j}$. %We call this term \omer{name??} reward discrepancy in round $t$. 
The (cumulative) \emph{envy} between two agents at round $t$ is the difference in their cumulative rewards. Formally, $\env_{i,j}^t= \Rt{i} - \Rt{j}$ is the envy after $t$ rounds between agent $i$ and $j$. We can also formulate envy as the sum of reward discrepancies, $\env_{i,j}^t= \sum^{t}_{\tau=1}{\adif{i}{j}^\tau}$. Notice that envy is a signed quantity and can be either positive or negative. Specifically, if $\env_{i,j}^t < 0$, we say that agent $i$ envies agent $j$, and if $\env_{i,j}^t > 0$, agent $j$ envies agent $i$. The main goal of this paper is to investigate the behavior of the \emph{maximal envy}, defined as
\[
\env^t = \max_{i,j \in [N]} \env^t_{i,j}.
\]
For clarity, the term \emph{envy} will refer to the maximal envy.\footnote{ We address alternative definitions of envy in Section~\ref{sec:discussion}.} % Envy can also be defined in alternative ways, such as by averaging pairwise envy across all agents. We address average envy in Section~\ref{sec:avg_envy}.}
Note that $\env_{i,j}^t$ are random variables that depend on the decision-making algorithm, realized rewards, and the arrival order, and therefore, so is $\env^t$. If a result we obtain regarding envy depends on the arrival order $\ordname$, we write $\env^t(\ordname)$. Similarly, to ease notation, if $\ordname$ can be understood from the context, it is omitted.



\paragraph{Further Notation} We use the subscript $(q)$ to address elements of the $q^{\text{th}}$ session, for $q\in [N]$.
That is, we use the notation $\rt{(q)}$ to denote the reward granted to the agent that arrives in the $q^{\text{th}}$ session of round $t$ and $\Rt{(q)}$ to denote her cumulative reward. %Additionally, we introduce the notation $\at{(q)}$ to denote the arm pulled in that session.
Correspondingly, $\sdift{q}{w} = \rt{(q)} - \rt{(w)}$ is the reward discrepancy of the agents arriving in the $q^{\text{th}}$ and $w^{\text{th}}$ sessions of round $t$, respectively. 
To distinguish agents, arms, sessions and rounds, we use the letters $i,j$ to mark agents and arms, $q,w$ for sessions, and $t,\tau$ for rounds.


\subsection{Example}
\label{sec: example}
To illustrate the proposed setting and notation, we present the following example, which serves as a running example throughout the paper.

\begin{table}[t]
\centering
\begin{tabular}{|c|c|c|c|}
\hline
$t$ (round) & $\ordv_t$ (arrival order) & $x_1^t$ & $x_2^t$ \\ \hline
1           & 2, 1                     & 0.6     & 0.92    \\ \hline
2           & 1, 2                     & 0.48    & 0.1     \\ \hline
3           & 2, 1                     & 0.15    & 0.8     \\ \hline
\end{tabular}
\caption{
    Data for Example~\ref{example 1}.
}
\label{tbl: example}
\end{table}

\begin{algorithm}[t]
\caption{Algorithm for Example~\ref{example 1}}
\label{alguni}
\DontPrintSemicolon 
\For{round $t = 1$ to $T$}{
    pull $a_{1}$ in the first session\label{alguniexample: first}\\
    \lIf{$x^t_1 \geq \frac{1}{2}$}{pull $a_{1}$ again in second session \label{alguniexample: pulling a again}}
    \lElse{pull $a_{2}$ in second session \label{alguniexample: sopt else}}
}
\end{algorithm}


\begin{example}\label{example 1}
We consider $K=2$ uniform arms, $X_1,X_2 \sim \uni{0,1}$, and $N=2$ for some $T\geq 3$. We shall assume arm decision are made by Algorithm~\ref{alguni}: In the first session, the algorithm pulls $a_1$; if it yields a reward greater than $\nicefrac{1}{2}$, the algorithm pulls it again in the second session (the ``if'' clause). Otherwise, it pulls $a_2$.



We further assume that the arrival orders and rewards are as specified in Table~\ref{tbl: example}. Specifically, agent 2 arrives in the first session of round $t=1$, and pulling arm $a_2$ in this round would yield a reward of $x^1_2 = 0.92$. Importantly, \emph{this information is not available to the decision-making algorithm in advance} and is only revealed when or if the corresponding arms are pulled.




In the first round, $\boldsymbol{\eta}^1 = \left(2,1\right)$; thus, agent 2 arrives in the first session.
The algorithm pulls arm $a_1$, which means, $a^1_{(1)} = a_1$, and the agent receives $r_{2}^1=r_{(1)}^1=x_1^1=0.6$.
Later that round, in the second session, agent 1 arrives, and the algorithm pulls the same arm again since $x^1_1 = 0.6 \geq \nicefrac{1}{2}$ due to the ``if'' clause.
I.e., $a^1_{(2)} = a_1$ and $r_{1}^1 = r_{(2)}^1 = x_1^1 = 0.6$.
Even though the realized reward of arm $a_2$ in that round is higher ($0.92$), the algorithm is not aware of that value.
At the end of the first round, $R^1_1 = R^1_{(2)} = R^1_2 = R^1_{(1)} = 0.6$. The reward discrepancy is thus $\adif{1}{2}^1 = \adif{2}{1}^1= \sdif{2}{1}^1 = 0.6 - 0.6 =0$. 



In the second round, agent 1 arrives first, followed by agent 2.
Firstly, the algorithm pulls arm $a_1$ and agent 1 receives a reward of $r_{1}^2 = r_{(1)}^2 = x_1^2 = 0.48$.
Because the reward is lower than $\nicefrac{1}{2}$, in the second session the algorithm pulls the other arm ($a^2_{(2)} = a_2$), granting agent 2 a reward of $r_{2}^2 = r_{(2)}^2 = x_2^2 = 0.1$.
At the end of the second round, $R^2_1 = R^2_{(1)} = 0.6 + 0.48 = 1.08$ and $R^2_2 = R^2_{(2)} = 0.6 + 0.1 = 0.7$. Furthermore, $\sdif{2}{1}^2 = \adif{2}{1}^2 = r^2_{2} - r^2_{1} = 0.1 - 0.48 = -0.38$.

In the third and final round, agent 2 arrives first again, and receives a reward  of $0.15$ from $a_1$. When agent 1 arrives in the second session, the algorithm pulls arm $a_2$, and she receives a reward of $0.8$. As for the reward discrepancy, $\sdif{2}{1}^3 = \adif{2}{1}^3 = r^3_{2} - r^3_{1} = 0.15 - 0.8 = -0.75$. 

Finally, agent 1 has a cumulative reward of $R^3_1 = R^3_{(2)} = 0.6 + 0.48 + 0.8 = 1.88$, whereas agent~2 has a cumulative reward of $R^3_2 = R^3_{(1)} = 0.6 + 0.1 + 0.15 = 0.85$. In terms of envy, $\env^1_{1,2}= \adif{1}{2}^1 =0$, $\env^2_{1,2}=\adif{1}{2}^1+\adif{1}{2}^2= 0.38$, and $\env^3_{1,2} = -\env^3_{2,1} = R^3_1-R^3_2 = 1.88-0.85 = 1.03$, and consequently the envy in round 3 is $\env^3 = 1.03$.
\end{example}


\subsection{Information Exploitation}
\label{subsec:information}

In this subsection, we explain how algorithms can exploit intra-round information.
Since rewards are consistent in the sessions of each round, acquiring information in each session can be used to increase the reward of the following sessions.
In other words, the earlier sessions can be used for exploration, and we generally expect agents arriving in later sessions to receive higher rewards.
Taken to the extreme, an agent that arrives after all arms have been pulled could potentially obtain the highest reward of that round, depending on how the algorithm operates.

To further demonstrate the advantage of late arrival, we reconsider Example~\ref{example 1} and Algorithm~\ref{alguni}. 
The expected reward for the agent in the first session of round $t$ is $\E{\rt{(1)}}=\mu_1=\frac{1}{2}$, yet the expected reward of the agent in the second session is
\begin{align*}
\E{\rt{(2)}}=\E{\rt{(2)}\mid X^t_1 \geq \frac{1}{2} }\prb{X^t_1 \geq \frac{1}{2}} + \E{\rt{(2)}\mid X^t_1 < \frac{1}{2} }\prb{X^t_1 < \frac{1}{2}};
\end{align*}
thus, $\E{\rt{(2)}} =\E{X^t_1\mid X^t_1 \geq \frac{1}{2} }\cdot \frac{1}{2} + \mu_2\cdot\frac{1}{2} = \frac{5}{8}$.
Consequently, the expected welfare per round is $\E{\rt{(1)}+\rt{(2)}}=1+\frac{1}{8}$, and the benefit of arriving in the second session of any round $t$ is $\E{\rt{(2)} - \rt{(1)}} = \frac{1}{8}$. This gap creates envy over time, which we aim to measure and understand.
%This discrepancy generates envy over time, and our paper aims to better understand it.
\subsection{Socially Optimal Algorithms}
\label{sec: sw}
Since our model is novel, particularly in its focus on the reward consistency element, studying social welfare maximizing algorithms represents an important extension of our work. While the primary focus of this paper is to analyze envy under minimal assumptions about algorithmic operations, we also make progress in the direction of social welfare optimization. See more details in Section~\ref{sec:discussion}.%Due to space limitations, we defer the discussion on socially optimal algorithms to  \ifnum\Includeappendix=0{the appendix}\else{Section~\ref{appendix:sociallyopt}}\fi.




% Since our model is novel and specifically the reward consistency element, it might be interesting to study social welfare optimization. While the main focus of our paper is to study envy under minimal assumptions on how the algorithm operates, we take steps toward this direction as well. Due to space limitations, we defer the discussion on socially optimal algorithms to  \ifnum\Includeappendix=0{the appendix}\else{Section~\ref{appendix:sociallyopt}}\fi.  We devise a socially optimal algorithm for the two-agent case, offer efficient and optimal algorithms for special cases of $N>2$ agents, and an inefficient and approximately optimal algorithm for any instance with $N>2$. Moreover, we address the welfare-envy tradeoff in Section~\ref{sec:extensions}.


% Social welfare, unlike envy, is entirely independent of the arrival order. While the main focus of our paper is to study envy under minimal assumptions on how the algorithm operates, socially optimal algorithms might also be of interest. Due to space limitations, we defer the discussion on socially optimal algorithms to  \ifnum\Includeappendix=0{the appendix}\else{Section~\ref{appendix:sociallyopt}}\fi. We devise a socially optimal algorithm for the two-agent case, offer efficient and optimal algorithms for special cases of $N>2$ agents, and an inefficient and approximately optimal algorithm for any instance with $N>2$. %Furthermore, we treat the welfare-envy tradeoff of the special case of Example~\ref{example 1}.




\section{High-dimensional concentration and Berry-Esseen bounds on Markov chains}%Theoretical results regarding general Markov chains}
%\yuting{define $\mu$-norm}\weichen{did in Section \ref{sec:notation}.}
In this section, we present our main results for uncertainty quantification of Markov chains. Section \ref{sec:MC-concentration} focuses on generalized and improved concentration inequalities for matrix-valued functions, while Section \ref{sec:MC-Berry-Esseen} focuses on multi-variate Berry-Esseen bounds on martingales. Section \ref{sec:MC-mtg} applies these results to the case where the martingale is generated from a Markov chain. 

Throughout the paper, we consider a Markov chain $\{s_t\}_{t \geq 0}$ with state space $\mathcal{S}$, transition kernel $P$ and a \emph{unique stationary distribution}, denoted as $\mu$. It can be guaranteed that any positive recurrent, irreducible and aperiodic Markov chain admits this property.
% \yuting{need to be positive recurrent for Markov chain with possibly infinite space to have unique stationary distribution}\weichen{checked.}
%It can be guaranteed that such a Markov chain admits a unique stationary distribution, denoted as $\mu$. 
Furthermore, let $\lambda$ denote the \emph{spectral expansion} of the Markov chain, and assume throughout the \emph{spectral gap condition} $1-\lambda > 0$. We refer the reader to Appendix \ref{app:MC-basics} for background on Markov chains. Notably, the theoretical results in this paper does \emph{not} require the Markov chain to be reversible. %As a final note, we will only consider Markov processes and martingales in discrete time.




% \paragraph{Markov chains}
% Throughout the paper, we consider Markov chains with transition kernel $\mathcal{P}$ that satisfies the following two properties:
% \begin{customassumption}
% \label{as:stationary}
% \yuting{shall we assume the Markov chain is irreducible and aperiodic? then it is always true that it has a unique stationary distribution}
% There exists a unique stationary distribution $\mu$, such that
% \begin{align*}
% \mu(B) = \int_{\mathcal{S}} P(x,B)\mu(\mathrm{d}x).
% \end{align*}
% \end{customassumption}
% \begin{customassumption}\label{as:spectral}
% The Markov chain admits a strictly positive spectral gap, namely, $1-\lambda > 0$.
% % \yuting{we use $\lambda$ for eigenvalues of $\Sigma$. Consider using $\rho$ for eigen-gaps.}
% \end{customassumption}



\subsection{Matrix Hoeffding's inequality on Markov chains}
\label{sec:MC-concentration}

In this section, we present a new matrix Hoeffding's inequality for sums of matrix-valued functions on Markov chains. 
Matrix Hoeffding's inequality, offering finite sample bounds for the spectral norm of sums of bounded random matrices, is a powerful and widely used result in statistics, machine learning and computer science.  
Initially developed for the sum of independent random matrices ~\citep[see, e.g.,][]{Tropp2011matrixtails,oliveira.matrix.hoef}, it has been recently generalized \citep{garg2018matrixexpanderchernoff,qiu2020matrix} to the case where the matrices are generated from a Markov chain. 
Our first theorem provides an extension and improvement of these results. See Appendix \ref{app:proof-matrix-hoeffding} for the proof.%  which turns out to be useful in developing statistical inference procedures for general Markov chains. 
\medskip
\begin{customtheorem}[Matrix Hoeffding's inequality for Markov Chains]\label{thm:matrix-hoeffding}
Consider a Markov chain $\{s_t\}_{t \geq 1}$ with a unique stationary distribution $\mu$ and a spectral gap $1-\lambda > 0$. Let $\{\bm{F}_i\}_{i \in [n]}$ be a sequence of matrix-valued functions from the state space $\mathcal{S}$ to $\mathbb{S}^{d \times d}$ satisfying $\mathbb{E}_{s \sim \mu}[\bm{F}_i(s)] = \bm{0}$ and $\|\bm{F}_i(s)\| \leq M_i$ almost surely for every $i \in [n]$. 
% \yuting{use $\|\cdot\|_2$ or $\|\cdot\|$ for spectral norm?}\weichen{checked.}
Then for any $\varepsilon > 0$, it can be guaranteed that when $s_1 \sim \mu$,
\begin{align}\label{eq:markov-matrix-hoeffding}
\mathbb{P}\left(\left\|\frac{1}{n}\sum_{i=1}^n \bm{F}_i(s_i)\right\| \geq \varepsilon \right) &\leq 2d^{2-\frac{\pi}{4}} \exp\left\{-\frac{1-\lambda}{20}\left(\frac{\pi}{4}\right)^2\frac{n^2\varepsilon^2 }{\sum_{j=1}^n M_j^2} \right\}.
\end{align}
\end{customtheorem}
\medskip
% \begin{proof}
% See 
% \end{proof}

% \yuting{provide links to the proof of each result} \weichen{checked.}

It is noteworthy that the right-hand-side of the above expression depends on the dimension $d$ through $d^{2-\frac{\pi}{4}}$, a worse dependence than  for independent matrices \citep[see, e.g.,][Theorem 5.2]{Tropp2011matrixtails}. The exponent $2-\frac{\pi}{4}$ stems from an application of a multi-matrix Golden-Thompson inequality due to \cite{garg2018matrixexpanderchernoff}.
We refer readers to inequality \eqref{eq:multi-matrix-golden-thompson} in Appendix \ref{app:proof-matrix-hoeffding}. Indeed, the proof of this theorem follows the framework developed by \cite{garg2018matrixexpanderchernoff} and \cite{qiu2020matrix} but makes improvements on multiple fronts. Firstly, our proof is presented in the language of measure theory, thus allowing for infinite state spaces; secondly, we implement novel recursive algebraic analysis to obtain a tighter Hoeffding's inequality for \emph{arbitrary tolerance level} $\varepsilon$, instead of $\varepsilon \in (0,1)$,  and \emph{time-dependent} functions $\{\bm{F}_i\}_{i \in [n]}$, instead of a \emph{time-invariant} function $\bm{F}$ throughout the steps $i = 1,2,...,n$. 
%Table \ref{table:Hoeffding-compare} compares Theorem \ref{thm:matrix-hoeffding} with previous results from various aspects. \yuting{no table now.}
%Throughout the proof, we also made a non-trivial correction to the previous attempt by \cite{qiu2020matrix} to handle non-stationary starting distribution $\nu$. See Appendix \ref{app:proof-matrix-hoeffding} for details.

It is also important to point out that Theorem \ref{thm:matrix-hoeffding} implies that, for any $\delta \in (0,1)$, 
\begin{align}
\label{eq:thm-matrix-hoeffding-whp}
\left\|\frac{1}{n}\sum_{i=1}^n \bm{F}_i(s_i)\right\|  \lesssim \sqrt{\frac{1}{1-\lambda} \frac{\sum_{j=1}^n M_j^2}{n} \log \Big(\frac{d}{\delta}\Big)} \cdot \frac{1}{\sqrt{n}},
\end{align} 
with probability at least $1-\delta$, where, here and throughout the paper, $\lesssim $ indicates weak inequality up to universal constants.
% Essentially, the operator norm of the sample mean is upper bounded with high probability by the following:
% \begin{enumerate}
% 	\itemsep0em
% \item The mixing property of the Markov chain, represented by $\frac{1}{\sqrt{1-\lambda}}$;
% \item The Root-Mean-Square of the upper bounds of the samples, $\sqrt{\frac{\sum_{j=1}^n M_j^2}{n}}$;
% \item The dimension of the matrix $d$, the tolerance level $\delta$, through the logarithm term $\sqrt{\log \frac{d}{\delta}}$;
% \item The sample size $n$, through the convergence rate $\frac{1}{\sqrt{n}}$.
% %\item The non-stationarity of the Markov chain, measured by $\sqrt{\log} \left\|\frac{\mathrm{d}\nu}{\mathrm{d}\mu}\right\|_{\mu}$.
% \end{enumerate}
% Specifically, it cannot be guaranteed for a series of matrices $\bm{X}_1,...,\bm{X}_n \in \mathbb{S}^{d \times d}$ that
% \begin{align*}
% \mathsf{Tr}\left(\exp\left(\sum_{i=1}^n \bm{X}_i \right)\right) \leq \mathsf{Tr}\left(\prod_{i=1}^n \exp(\bm{X}_i)\right).
% \end{align*}
% \yuting{comments on the $\pi/4$ exponent on $d$?}
% \weichen{That exponent comes from the multi-matrix Golden-Thompson inequality \eqref{eq:multi-matrix-golden-thompson}}
% \paragraph{Technical novelty.}
In the scalar case, i.e. when $d=1$, our result agrees, albeit with a worse dependence on the constant and on $\lambda$, with Theorem 1 in \cite{Fan2021Hoeffding}, which provides a sharp Hoeffding's inequality for averages of scalar functions on Markov chains. This is an indication that our bound is not loose. 




% \begin{table*}[t]
% \centering
% \renewcommand{\arraystretch}{2.3}
% \begin{tabular}{c|c|c|c|c} 
% \toprule
% paper & \makecell{multi- \\ dimension} & \makecell{arbitrary \\ tolerance level}  & \makecell{infinite \\ state space} & \makecell{time-dependent \\ functions}\\ 
% \toprule
% 	\cite{Fan2021Hoeffding}, Theorem 1 & No & Yes  & Yes & Yes \\ \hline

% 	\cite{garg2018matrixexpanderchernoff}, Theorem 3 & Yes & No  & No & No \\ \hline

% 	This work, Theorem \ref{thm:matrix-hoeffding} & Yes & Yes  & Yes & Yes \\ 
% \toprule
% \end{tabular}
% \caption{Comparisons between Theorem \ref{thm:matrix-hoeffding} and previous results. \yuting{should we include other comparisons too?}}
% \label{table:Hoeffding-compare}
% \end{table*}

In order to generalize Theorem \ref{thm:matrix-hoeffding} to non-stationary Markov chains, we impose the following assumption on the distribution of the initial state, denoted as $\nu$.
\medskip
\begin{customassumption}\label{as:nu}
The probability measure $\nu$ is absolutely continuous with respect to the stationary distribution $\mu$; furthermore, assume that there exists $p \in (1,\infty]$, such that the Radon-Nykodin derivative of $\nu$ with respect to $\mu$ satisfies 
% \begin{align*}
	$\left\|\frac{\mathrm{d}\nu}{\mathrm{d}\mu}\right\|_{\mu,p} < \infty.$
    % \text{and}
% \end{align*}
%with the norm defined in Eq.~\eqref{eq:mu-p-norm}.
\end{customassumption}
\medskip
% \yuting{why do you need $q$ in this assumption?} \weichen{This is just for notational simplicity.}
We also let $q \in [1,\infty)$ denote the conjugate of $p$, i.e. $\frac{1}{p} + \frac{1}{q} = 1$ when $p < \infty$ and  $q = 1$ if $p = \infty$.  For simplicity, we say $\nu$ \emph{satisfies Assumption \ref{as:nu} with parameters $(p,q)$} if these conditions hold true.  The following corollary generalizes Theorem \ref{thm:matrix-hoeffding} to the case where $s_1$, the first state of the Markov chain, is not generated from the stationary distribution $\mu$ but rather from a distribution $\nu$ satisfying Assumption \ref{as:nu}. 

\medskip
\begin{customcorollary}\label{cor:matrix-hoeffding} 
Assume the conditions of Theorem \ref{thm:matrix-hoeffding}  and that $s_1 \sim \nu$, where $\nu$ is a probability measure on the state space $\mathcal{S}$ satisfying Assumption \ref{as:nu} with parameters $(p,q)$. 
Then for any $\varepsilon>0$, 
\begin{align*}
\mathbb{P}\left(\left\|\frac{1}{n}\sum_{i=1}^n \bm{F}_i(s_i)\right\| \geq \varepsilon \right) &\leq 2d^{2-\frac{\pi}{4}} \left\|\frac{\mathrm{d}\nu}{\mathrm{d}\mu}\right\|_{\mu,p}  \exp\left\{-\frac{1-\lambda}{20q}\left(\frac{\pi}{4}\right)^2\frac{n^2\varepsilon^2 }{\sum_{k=1}^n M_k^2} \right\}.
\end{align*}
\end{customcorollary}
\medskip
This corollary follows directly from Theorem \ref{thm:matrix-hoeffding} and Holder's inequality. See Appendix \ref{app:proof-cor-matrix-hoeffding} for details. As an application, for any $\delta \in (0,1)$, we have that, with probability at least $1-\delta$,
\begin{align}\label{eq:cor-matrix-hoeffding-whp}
\left\|\frac{1}{n}\sum_{i=1}^n \bm{F}_i(s_i)\right\|  \lesssim \sqrt{\frac{q}{1-\lambda} \frac{\sum_{j=1}^n M_j^2}{n} \log \left(\frac{d}{\delta}\left\|\frac{\mathrm{d}\nu}{\mathrm{d}\mu}\right\|_{\mu,p}\right)} \cdot \frac{1}{\sqrt{n}}.
\end{align}
Comparing the above bound with the analogous one for stationary Markovian sequences (see  \ref{eq:thm-matrix-hoeffding-whp}), we observe that for a non-stationary Markov chain, the upper bound is larger by a factor of $\sqrt{q}$ and an additional factor in the logarithm term that reflects the difference between the starting distribution $\nu$ and the stationary distribution $\mu$. Of course, when $\nu = \mu$, then $\frac{\mathrm{d}\nu}{\mathrm{d}\mu} \equiv 1$ and we can take $p = \infty$ and $q = 1$, so that 
% \begin{align*}
$\left\|\frac{\mathrm{d}\nu}{\mathrm{d}\mu}\right\|_{\mu,p} = 1,$
% \end{align*}
and the bound \eqref{eq:cor-matrix-hoeffding-whp} coincides with \eqref{eq:thm-matrix-hoeffding-whp}.




\subsection{Berry-Esseen bounds on vector-valued martingales}
\label{sec:MC-Berry-Esseen}

In our next result, we derive novel high-dimensional Berry-Esseen bounds for vector-valued martingales in terms of the  Wasserstein distance. See Appendix \ref{app:Srikant-generalize} for the proof.

\medskip
\begin{customtheorem}[Berry-Esseen bound on vector-valued martingales]\label{thm:Srikant-generalize}
Let $\{\bm{x}_k\}_{k=1}^n$ be a martingale difference process in $\mathbb{R}^d$ with respect to the filtration $\{\mathscr{F}_k\}_{k=0}^n$.
For every $k \in [n]$, define 
\begin{align*}
&\bm{V}_k := \mathbb{E}[\bm{x}_k\bm{x}_k^\top \mid \mathscr{F}_{k-1}], \quad  \text{and}  \quad \bm{P}_k := \sum_{j=k}^n \bm{V}_k.
\end{align*}
Furthermore, define
\begin{align*}
\bm{\Sigma}_n := \frac{1}{n}\sum_{k=1}^n \mathbb{E}[\bm{x}_k \bm{x}_k^\top \mid \mathscr{F}_0], 
\end{align*}
and assume that
\begin{align}\label{eq:as-P1}
\bm{P}_1 = n\bm{\Sigma}_n \quad \text{almost surely.}
\end{align}
Then for any  $d$-dimensional symmetric positive semi-definite matrix $\bm{\Sigma}$, it can be guaranteed that
\begin{align}\label{eq:Srikant-Berry-Esseen}
d_{\mathsf{W}}\left(\frac{1}{\sqrt{n}}\sum_{k=1}^n \bm{x}_k,\mathcal{N}(\bm{0},\bm{\Sigma}_n)\right) &\lesssim  \frac{(2+\log(d\|(n\bm{\Sigma}_n + \bm{\Sigma})\|))^+}{\sqrt{n}} \sum_{k=1}^n \mathbb{E}\left[\mathbb{E}\left[\|(\bm{P}_k + \bm{\Sigma})^{-\frac{1}{2}}\bm{x}_k\|_2^2 \|\bm{x}_k\|_2 \bigg|\mathscr{F}_{k-1}\right]\right] \nonumber \\ 
&\qquad+ \frac{1}{\sqrt{n}}\left[\mathsf{Tr}(\log(n\bm{\Sigma}_n+\bm{\Sigma})) - \log(\bm{\Sigma}))\right]+ \sqrt{\frac{\mathsf{Tr}(\bm{\Sigma})}{n}}.
\end{align}
\end{customtheorem}
\medskip

% \begin{proof}
% See Appendix \ref{app:Srikant-generalize}.
% \end{proof}
% \yuting{comment on the assumption $\bm{P}_1 = n\bm{\Sigma}_n$ a.s.}\weichen{This follows the precedent of several papers considering the Berry-Esseen bound on martingales. I believe there are results showing even in uni-dimensional case that the Berry-Esseen bound is controlled by the difference between $\bm{P}_1$ and $n\bm{\Sigma}_n$.}

% \paragraph{Comparison with the uni-dimensional Berry-Esseen bound shown in Theorem 2.1 of \cite{rollin2018}.} 
Let us compare this result with analogous ones in the literature. When $d=1$, Theorem \ref{thm:Srikant-generalize} agrees with Theorem 2.1 of \cite{rollin2018}, aside from logarithmic  factors, by letting
\begin{align*}
	\bm{\Sigma}_n = s_n^2/n, \quad \bm{P}_k = \rho_k^2, \quad \text{and} \quad \bm{\Sigma} = a^2.
\end{align*}
In this sense,  Theorem \ref{thm:Srikant-generalize} may be be regarded as a multivariate generalization of the univariate bound of Theorem 2.1 of \cite{rollin2018}.
% \paragraph{Technical novelty.} 
% This theorem is an exact multi-dimensional generalization of Theorem 2.1 in \cite{rollin2018}, in the sense that when $d=1$, the upper bound \eqref{eq:Srikant-Berry-Esseen} reduces to the upper bound shown in Theorem 2.1 of \cite{rollin2018} up to a logarithm factor. 
Next, Theorem \ref{thm:Srikant-generalize} bears similarities with Theorem 1 of \cite{JMLR2019CLT}, also concerned with Gaussian approximations of multivariate martingale sequences. Our result offers a different but arguably more general guarantee because it establishes convergence in the weaker Wasserstein distance instead of the $d_2$ distance %and, furthermore, levetrages a refined result on the smoothness of the solution to Stein's equation which may be of independent interest (see Proposition \ref{prop:Stein-smooth} for details). 
We believe that these are meaningful improvements, since the Wasserstein distance can be directly related to the convex distance \citep{nourdin2021multivariate}, which, in turn,  is more amenable to statistical inference. %\ale{add a citation}
%\yuting{other differences?} \weichen{Not much. Actually their bound is smaller than ours, if we simply compare the terms regardless of the kind of distance in use. Maybe want to say $d_{\mathsf{W}}$ is more directly translated to $d_{\mathsf{C}}$, which is useful for statistical inference?} \yuting{yes, and add a sentence saying why this generalization is non-trivial?}\weichen{please see the amended text.}

Our strategy to prove Theorem \ref{thm:Srikant-generalize} above is heavily inspired by the recent, very interesting pre-print by \cite{srikant2024rates}, which deploys Stein's method and Lindeberg swapping. 
We refine the approach of \cite{srikant2024rates} in the following ways: firstly, we addressed a gap in their proof if assumption \eqref{eq:as-P1} does not hold; secondly, we obtained a tighter bound on the smoothness of the solution to the multivariate Stein's equation, which may be of independent interest; see Proposition \ref{prop:Stein-smooth} in Appendix \ref{app:Srikant-generalize}  and compare it to Proposition 2.2 and 2.3 in \cite{gallouët2018regularitysolutionssteinequation}. 
The following corollary offers a useful simplification of our upper bound under a slightly stronger condition; see 
Appendix \ref{app:proof-cor-Wu} for the proof.

% \paragraph{Comparison with Theorem 1 of \cite{srikant2024rates}.} While the framework of our proof of Theorem \ref{thm:Srikant-generalize} is mainly inspired by the proof of Theorem 1 of \cite{srikant2024rates}, there are some noteworthy differences. Most importantly, we observe that the assumption $P_1 = n\bm{\Sigma}_n$ almost surely is necessary to obtain a meaningful Berry-Esseen bound, which also follows the precedent of \cite{JMLR2019CLT}. The relaxation of this assumption would be addressed in Theorem \ref{thm:Berry-Esseen-mtg}. Another important improvement we made in Theroem \ref{thm:Srikant-generalize} is to tighten the upper bound through a closer scrutiny of the smoothness of the solution to the Stein's equation, as is indicated in Proposition \ref{prop:Stein-smooth}. This also paves the way for the following Corollary. 
\medskip
\begin{customcorollary}
\label{cor:Wu}
Under the settings of Theorem \ref{thm:Srikant-generalize}, further assume there exists a uniform constant $M > 0$, such that for any matrix $\bm{A} \in \mathbb{R}^{d \times d}$ and any $k \in [n]$, it satisfies 
\begin{align}\label{eq:3rd-momentum-condition}
\mathbb{E}\left[\|\bm{Ax}_k\|_2^2 \|\bm{x}_k\|_2 \bigg|\mathscr{F}_{k-1}\right] \leq M \mathbb{E}\left[\|\bm{Ax}_k\|_2^2 \bigg|\mathscr{F}_{k-1}\right].
\end{align}
Then,
\begin{align*}
	d_{\mathsf{W}}\left(\frac{1}{\sqrt{n}}\sum_{k=1}^n \bm{x}_k,\mathcal{N}(\bm{0},\bm{\Sigma}_n)\right) \lesssim \left[M(2+\log(dn\|\bm{\Sigma}_n\|))^+ + 1\right]\frac{d \log n}{\sqrt{n}} +\sqrt{\frac{\mathsf{Tr}(\bm{\Sigma}_n)}{n}}.
\end{align*}
\end{customcorollary}
\medskip
% \begin{proof}
% See Appendix \ref{app:proof-cor-Wu}.
% \end{proof}

% \paragraph{Technical novelty.} 
% Applying a delicate telescoping method, \yuting{telescoping method: be more specific?} \weichen{hard to tell it in text here. Maybe add a reference to the proof?} 
Corollary~\ref{cor:Wu} should be compared with Corollary 2.3 in \cite{rollin2018} when $d=1$, and  Corollary 3 in \cite{JMLR2019CLT}, valid in multivariate settings but for the stronger $d_2$ distance. In establishing our bound, we have lifted the requirement on the conditional third momentum of $\bm{x}_k$ and addressed a potential gap in the proof of Corollary 3 of \cite{JMLR2019CLT}.

%compares favortightens and simplifies the upper bound on the Wasserstein distance when compared with similar results in prior work (e.g., Corollary 2.3 in \cite{rollin2018} and Corollary 3 in \cite{JMLR2019CLT}) while lifting the requirement assumed on the conditional third momentum of $\bm{x}_k$. 
%When $d=1$, our result recovers Corollary 2.3 of \cite{rollin2018}.

  
% Corollary \ref{cor:Wu} illustrates that the difference between the sum of the martingale difference process $\{\bm{x}_k\}$ and its Gaussian approximation $\mathcal{N}(\bm{0},\bm{\Sigma}_n)$ converges by the rate of $O(n^{-\frac{1}{2}}\log n)$, when measured by Wasserstein distance. \yuting{I am a little torn on whether we should state the convergence rate like this without mentioning the $d$ dependence} \weichen{alternatively, we can use $O(M\sqrt{\frac{d}{n}})$ to describe the convergence rate up to log factors, and note that $M$ may also depend on $d$. Which one do you think is better?}
% This is the first Berry-Esseen bound on vector-valued martingales to reach that convergence speed. 


% With Corollary \ref{cor:Wu}, we are ready to present our result on characterizing the non-asymptotic convergence of a vector-valued Markovian martingale to its Gaussian limit.
% \yuting{is ``Markovian martingale'' a standard name? }

\subsection{Uncertainty quantification for martingales generated from Markov chains}\label{sec:MC-mtg}
In this section, we leverage the results presented above to investigate a specific structure of sample dependency arising when a matrix- or vector-valued function sequence is \emph{both a martingale difference and generated from a Markov chain}. The study of dependent data of this type is  motivated by our analysis of TD learning, but it can also be of independent interest as it applies to MCMC algorithms. Specifically, we consider functions $\bm{F}:\mathcal{S}^2 \to \mathbb{R}^{m \times n}$ that satisfy the following assumption. %: \yuting{$d\times d$?}\weichen{The Berstein's inequality is on matrices and the Berry-Esseen bound is on vectors. So I wrote $m \times n$ for generality.}
\medskip
\begin{customassumption}\label{as:markov-mtg}
For every $s \in \mathcal{S}$, $\mathbb{E}_{s' \sim P(\cdot \mid s)}\bm{F}(s,s') = \bm{0}$.
\end{customassumption}
\medskip

Notice that when a sequence of functions $\{\bm{F}_i\}_{1 \leq i \leq n}$ with the same dimensions all satisfy Assumption \ref{as:markov-mtg}, it can be guaranteed that $\{\bm{F}_i(s_{i-1},s_i)\}_{1 \leq i \leq n}$ is a martingale, and hence better statistical properties can be derived. 

The next result presents a Bernstein-style convergence guarantee on matrix-valued functions satisfying Assumption \ref{as:markov-mtg}.

% Theorem \ref{thm:matrix-hoeffding} and Corollary \ref{cor:matrix-hoeffding} are useful if the aim is to develop high-probability convergence guarantees for matrix-valued functions generated from Markov chains. For example, when coupled with the matrix Freedman's inequality on martingales, Corollary \ref{cor:matrix-hoeffding} induces the following matrix Bernstein's inequality on a martingale generated by a Markov chain.
\medskip
\begin{customcorollary}[Matrix Bernstein's inequality on martingales generated from Markov chains]\label{thm:matrix-bernstein-mtg}
Consider a Markov chain $\{s_t\}_{t \geq 0}$ with a unique stationary distribution $\mu$ and a spectral gap $1-\lambda > 0$. Let $\{\bm{F}_i\}_{i \in [n]}$ be a sequence of functions mapping from  $\mathcal{S}^2$ to $\mathbb{S}^{d \times d}$, satisfying Assumption \ref{as:markov-mtg} and such that $\|\bm{F}_i(s,s')\| \leq M$ for every $i \in [n]$ and $s,s' \in \mathcal{S}$. Further define
\begin{align}
\bm{\Sigma}_n = \frac{1}{n} \sum_{i=1}^n \mathbb{E}_{s \sim \mu, s' \sim P(\cdot \mid s)}[\bm{F}_i(s,s')\bm{F}_i^\top(s,s')].
\end{align} 
% \yuting{$\bm{F}_i^2$, you mean $\bm{F}_i \bm{F}_i^\top$?}
% \weichen{Since $\bm{F}_i$ is symmetric (as indicated in $\bm{F}_i \in \mathbb{S}^{d \times d}$), these are the same. Will clarify in the notation section.}
Let $\nu$ denote a probability distribution on $\mathcal{S}$  satisfying Assumption \ref{as:nu} with parameters $(p,q)$.
Then for any $\delta \in (0,1)$, it can be guaranteed that, when $s_0 \sim \nu$,
\begin{align}\label{eq:matrix-Bernstein}
\left\|\frac{1}{n}\sum_{i=1}^n \bm{F}_i(s_{i-1},s_i)\right\|  \lesssim \sqrt{\frac{\|\bm{\Sigma}_n\|}{n}\log \frac{d}{\delta}}+\frac{\sqrt{q}M}{(1-\lambda)^{\frac{1}{4}} n^{\frac{3}{4}}} \log^{\frac{3}{4}} \left(\frac{d}{\delta}\left\|\frac{\mathrm{d}\nu}{\mathrm{d}\mu}\right\|_{\mu,p}\right) + \frac{M}{n} \log \frac{d}{\delta},
\end{align}
with probability at least $1-\delta$.
\end{customcorollary}
\medskip
% \paragraph{Technical novelty.} 

To establish Corollary ~\ref{thm:matrix-bernstein-mtg}, whose proof is in Appendix~\ref{app:proof-matrix-bernstein-mtg}, we first deploy  matrix Freedman's inequality, which gives an upper bound that is dependent on the \emph{conditional} covariance matrix
\begin{align*}
\bar{\bm{\Sigma}}_n := \frac{1}{n}\sum_{i=1}^n \mathbb{E}[\bm{F}_i\bm{F}_i^\top\mid \mathcal{F}_{i-1}].
\end{align*}
However, the above quantity is not measurable with respect to the trivial $\sigma$-field $\mathscr{F}_0$.
% When the martingale difference process $\{\bm{F}_i\}$ is generated by a Markov chain, 
Thus, we control the difference between $\bar{\bm{\Sigma}}_n$ and $\bm{\Sigma}_n$ using Corollary \ref{cor:matrix-hoeffding}, which leads to the first two terms in the upper bound \eqref{eq:matrix-Bernstein}. Notice that the second and third terms on the right hand side of \eqref{eq:matrix-Bernstein} both converge faster than $n^{-1/2}$, so Theorem \ref{thm:matrix-bernstein-mtg} shows that when measured by spectral norm, the sample mean of the sequence $\{\bm{F}_i\}$ converges to $\bm{0}$ by a rate determined by $\bm{\Sigma}_n$ with high probability. This bound plays a key role in 
% result is useful in the high-dimensional, non-asymptotic analysis of ML algorithms, as we will illustrate 
our analysis of TD learning in Section~\ref{sec:TD} and can be potentially used to understand the concentration of other machine learning algorithms. 

The following result, which is a corollary to Theorem \ref{thm:Srikant-generalize}, presents a multi-dimensional Berry-Esseen bound on vector-valued functions satisfying Assumption \ref{as:markov-mtg}. See Appendix \ref{app:proof-Berry-Esseen-mtg} for the proof.

\medskip
\begin{customcorollary}[High-dimensional Berry-Esseen bound on martingales generated from Markov chains]\label{thm:Berry-Esseen-mtg}
Consider a Markov chain $\{s_t\}_{t \geq 0}$ with a unique stationary distribution $\mu$ and a spectral gap $1-\lambda > 0$. Let $\{\bm{f}_i\}_{i \in [n]}$ be a sequence of functions from $\mathcal{S}^2$ to $\mathbb{R}^{d }$ satisfying Assumption \ref{as:markov-mtg} and such that $\|\bm{f}_i(s,s')\|_2 \leq M_i \leq M$ for all $s,s' \in \mathcal{S}$ and $i \in [n]$. Further define
\begin{align*}
\bm{\Sigma}_n = \frac{1}{n} \sum_{i=1}^n \mathbb{E}_{s\sim\mu,s' \sim P(\cdot \mid s)}\Big[\bm{f}_i(s,s')\bm{f}_i^\top(s,s')\Big], \quad \text{and} \quad \bar{M} = \left(\frac{\sum_{i=1}^n M_i^4}{n}\right)^{\frac{1}{4}}.
\end{align*} 
% \yuting{in this definition, sum over $i$?}\ale{I agree}\weichen{yes, checked.}
Assume that $\lambda_{\min}(\bm{\Sigma}_n)\geq c $ for a constant $c>0$. For simplicity, we also assume that $M \geq 1$ and that $d\|\bm{\Sigma}_n\| \geq 1$.\footnote{These assumptions are made for ease of presentation, and are not essential.} 
Let $\nu$ denote a probability distribution on $\mathcal{S}$ satisfying Assumption \ref{as:nu} with parameters $(p,q)$. 
Then, when $s_0 \sim \nu$,
\begin{align}\label{eq:Berry-Esseen-mtg}
&d_{\mathsf{C}}\left(\frac{1}{\sqrt{n}}\sum_{i=1}^n \bm{f}_i(s_{i-1}, s_i), \mathcal{N}(\bm{0},\bm{\Sigma}_n)\right) \nonumber \\ 
&\lesssim \left\{\bar{M}\left(\frac{q}{1-\lambda}\right)^{\frac{1}{4}}\log^{\frac{1}{4}}\left(d\left\|\frac{\mathrm{d}\nu}{\mathrm{d}\mu}\right\|_{\mu,p}\right)\|\bm{\Sigma}_n\|_{\mathsf{F}}^{\frac{1}{2}}+ \sqrt{M} \log^{\frac{1}{2}} (d\|\bm{\Sigma}_n\|)\|\bm{\Sigma}_n\|_{\mathsf{F}}^{\frac{1}{4}}\right\}\sqrt{d} n^{-\frac{1}{4}}\log n.
\end{align}
\end{customcorollary}
\medskip

For readability, in the above expression we have omitted a lower-order term depending on the lower bound $c$ on $\lambda_{\min}(\bm{\Sigma}_n)$; see equation \eqref{eq:Wu-Gaussian-comparison} in Appendix \ref{app:proof-Berry-Esseen-mtg}.
To the best of our knowledge, Corollary \ref{thm:Berry-Esseen-mtg} presents the first high-dimensional Berry-Esseen bound on Markov chain-induced martingales in the convex distance. This distance is amenable for constructing confidence regions/sets, which would not be possible using the Wasserstein distance.
There are a number of notable differences between the conditions of Corollary \ref{thm:Berry-Esseen-mtg} and Corollary \ref{cor:Wu}: Firstly, Corollary \ref{thm:Berry-Esseen-mtg} does \emph{not} use the restrictive condition \eqref{eq:as-P1}, which demands a deterministic conditional variance, but instead only requires the martingale to be generated from a Markov chain with good mixing property, 
% \yuting{do you assume this anyway?} 
which means $\bm{P}_1$ is \emph{close to} $n \bm{\Sigma}_n$ with high probability. Secondly, Theorem \ref{thm:Berry-Esseen-mtg} requires that the norm of $\bm{f}_i$' s be \emph{uniformly bounded}, which is strictly stronger than \eqref{eq:3rd-momentum-condition}, as required by Corollary \ref{cor:Wu}. 

In order to demonstrate how the Berry-Esseen bound in Corollary \ref{thm:Berry-Esseen-mtg} depends on problem-related quantities, consider the scenario in which 
$\bm{f}_1 = \bm{f}_2 = \ldots = \bm{f}_n = \bm{f}$, and $M_1 = M_2 = \ldots = M_n = M$. Then, in this case,
% \yuting{we can even let $\nu = \mu$ for simplification?}
% \weichen{I personally prefer keeping the current form, to illustrate how the initial deviation from stationary distribution is reflected in the Berry-Esseen bound.}
\begin{align*}
\bm{\Sigma}_n = \frac{1}{n} \sum_{i=1}^n \mathbb{E}_{s_0 \sim \mu}\Big[\bm{f}_i(s_{i-1},s_i)\bm{f}_i^\top(s_{i-1},s_i)\Big] = \mathbb{E}_{s \sim \mu, s' \sim P(\cdot \mid s)}[\bm{f}(s,s')\bm{f}^\top(s,s')] =: \bm{\Sigma},
\end{align*}
and $\bar{M} = M$. Therefore, \eqref{eq:Berry-Esseen-mtg} implies that
\begin{align*}
d_{\mathsf{C}}\left(\frac{1}{\sqrt{n}}\sum_{i=1}^n \bm{f}(s_{i-1}, s_i), \mathcal{N}(\bm{0},\bm{\Sigma})\right)  \lesssim \left(\frac{q}{1-\lambda}\right)^{\frac{1}{4}}\log^{\frac{1}{4}}\left(d\left\|\frac{\mathrm{d}\nu}{\mathrm{d}\mu}\right\|_{\mu,p}\right)M\|\bm{\Sigma}\|_{\mathsf{F}}^{\frac{1}{2}}\sqrt{d}n^{-\frac{1}{4}}\log n.
\end{align*}
We remark that the dependence on $d$ appears both explicitly in the $\sqrt{d}$ term, and implicitly in the $\|\bm{\Sigma}\|_{\mathsf{F}}^{\frac{1}{2}}$ and $M$ terms. 
% In other words, the rate by which the sample mean converges to its asymptotic Gaussian distribution is upper bounded by the followng:
% \begin{enumerate}
% \item The mixing speed of the Markov chain, through the term $(1-\lambda)^{-\frac{1}{4}}$;
% \item The derivation from the stationary distribution, through the term $q^{\frac{1}{4}}\log^{\frac{1}{4}}\left\|\frac{\mathrm{d}\nu}{\mathrm{d}\mu}\right\|_{\mu,p}$;
% \item The properties of the function $\bm{f}$, including its dimension $d$ (through $\sqrt{d}\log^{\frac{1}{4}}d$), its variance matrix $\bm{\Sigma}$ (through $\|\bm{\Sigma}\|_{\mathsf{F}}^{\frac{1}{2}}$), and the bound on its norm $M$;
% \item The sample size $n$, through the convergence rate $n^{-\frac{1}{4}}\log n$. 
% \end{enumerate}

% \paragraph{Technical novelty.} 
The proof of Corollary \ref{thm:Berry-Esseen-mtg} is inspired by the arguments developed by \cite{rollin2018} to relax Assumption \eqref{eq:as-P1} in the uni-dimensional case, recently extended to the multivariate setting in 
\citet[][Lemma B.8]{cattaneo2024yurinskiiscouplingmartingales} and \citet[][Theorem 2.1]{
belloni2018highdimensionalcentrallimit}.
%A generalization to the multi-dimensional case is highly nontrivial, mainly due to the fact that the positive semi-definite order between matrices is incomplete. 
Specifically, we construct an auxiliary martingale satisfying \eqref{eq:as-P1}, and apply Corollary \ref{cor:Wu} to derive a Berry-Esseen bound. At the same time, we bound the difference between the target martingale and this auxiliary martingale by Corollary \ref{cor:matrix-hoeffding}. Finally, we combine these bounds using the properties of Gaussian distributions, as well as the relationship between convex distance and Wasserstein distance. 
See Appendix \ref{app:proof-Berry-Esseen-mtg}. %We refer the reader to  for more details.





\section{Application: Harnessing the Linearity}
\label{sec:application}
Leveraging the \emph{linearity} of DMD operator, as well as the intuition of bases exposed by the spectral decomposition, we have developed several novel applications that extend the capabilities of our Koopman-based reduced-order simulation pipeline. In this section, we explore these applications, demonstrating that our method's unique strengths translate into practical tools for graphics and simulation.

\subsection{Direct Editing Temporal Dynamics}
\label{sec:editing}
\begin{figure}[!ht]
    \centering
    \includegraphics[width=1\columnwidth]{figure/karman_vortex_street_editing.pdf}
    \caption{\textbf{Editing temporal dynamics of K\'arm\'an Vortex Street with the Koopman Operator Approximation}. The modifications are applied to the DMD basis coefficients: (a) Scaling the modulus of the DMD basis by factors of 0.5, 1.0, and 1.5, affecting overall amplitude; (b) Adjusting the real part of $\bm{\Omega}$, influencing growth and decay rates of modal contributions; (c) Modifying the imaginary part, altering phase dynamics and wave propagation characteristics. }
    \label{fig:karman_editing}
    \Description{}
\end{figure}


\begin{figure*}[!ht]
    \centering
    \includegraphics[width=1\linewidth]{figure/reversibility.pdf}
    \caption{\textbf{Reversibility of Flows with Inversed DMD Operator}. We compare the reconstruction of two distinct fluid flows using Dynamic Mode Decomposition (DMD). The top row in each panel shows the velocity L2-norm of the field used to train the DMD, while the second and third rows depict the temporal evolution of the reconstructed flow fields as applied to an initial density field. The forward-time training phase is followed by a backward-time testing phase to assess predictive accuracy when advecting backward in time. The bottom plots show the evolution of kinetic energy over time. From the buoyant case, we observe the inverted DMD operator $\bm{A^{-1}}$ can still reasonably trace backward in time without compromising much visual quality. The vortical case exhibits a more challenging example where the symmetry should be reconstructed backwards in time. We see that the inverse operator indeed recovers this symmetry, with some acceptable levels of incurred noise. Bottom plots show the evolution of the total kinetic energy over time, demonstrating that our inverse operator actually correctly reverses the arrow of time, reversing the dissipation-related entropy increase over time. Decreasing kinetic energy also validates the \emph{physical plausibility} of our result.}
    \label{fig:reverse_simulation}
    \Description{}
\end{figure*}


Since our method approximates \refeq{eqn:euler_equations} with a linear operator in the full space, this allows us to transform the operator acting on the velocity field into the evolution of different modes under a linear operator. Therefore, we can directly edit the temporal dynamics of the fluid system by modifying the modes of the reduced \koopman{} $\bm{\hat{K}}$:
we set $t_0$ to be the initial time, $\bm{\Omega} = \nicefrac{\log(\bm\Lambda)}{\Delta t}$, where $\Delta t$ is the time step of the dataset. With this, we can rewrite \refeq{eqn:reduced_koopman_simulation} in the following form:
\begin{equation}
    \begin{aligned}
    \bm{u}(t_0 + k\Delta t) &= \bm{\Phi}\exp(\bm{\Omega} t) \bm{z}(t_0) \\
    &= \bm{\Phi}\exp{\left(k(\log(r) + i\theta)\right)} \bm{z}(t_0) \\
    &= \sum_{i = 1}^{n} {w_i} \bm{\Phi_i} r_i^k \left(\cos(k\theta_i) + \sin(k\theta_i)\right) \bm{z_i}(t_0)\\
    \end{aligned}
    \label{eqn:edit_temporal}
\end{equation}
% explanation for the formula
where ${w_i}$ is a user-defined scalar weight, $r_i = \sqrt{\Re(\lambda_i)^2 + \Im(\lambda_i)^2}$ is the \emph{modulus} and $\theta_i = \arctan\left(\Im(\lambda_i), \Re(\lambda_i)\right)$ is the \emph{phase} of the $i$-th eigenvalue $\lambda_i$ in the diagonal \emph{complex} eigenvalue matrix $\bm{\Lambda}$. Notice that this implies that the modes of the spectral decomposition represent different scales of vorticity, completing the physical intuition of the reduced space modes.
% show the benefits of our method for artist to edit

As shown in \refeq{eqn:edit_temporal}, our method decomposes a simulation sequence into modes with different growth/decay rates and frequencies.
The growth/decay rate of a mode is reflected in $r_i$, where a larger $r_i$ indicates a higher growth rate (or a lower decay rate), and vice versa.
The frequency of a mode is represented by the absolute value of $\theta_i$, with a larger absolute value corresponding to a higher frequency mode, and vice versa.
Furthermore, the different modes are decoupled, allowing for the adjustment of the relative proportions between modes.
As a result, these properties provide the artist with powerful tools to edit the simulation playback. The artist can modify the overall velocity field by adjusting the proportion ($w_i$), growth/decay rate ($r_i$), and frequency ($\theta_i$) of specific modes.
% explanation for what we actually do in code
In the experiments, we directly adjust the real part of $\bm{\Omega_i}$ to control $r_i$, modify the imaginary part of $\bm{\Omega_i}$ to control $\theta_i$, and vary the modulus of $\bm{\Phi_i}$ to control $w_i$.
% explanation for what we did to edit in karman vortex street scene
\paragraph{Editing the K\'arm\'an Vortex Street}
The first example is editing on the classic K\'arm\'an vortex street. We filter the imaginary part of $\bm{\Omega}$ and cluster modes with an absolute value smaller than $0.01$ as \emph{low-frequency cluster}, and the rest as \emph{high-frequency cluster}.
The low-frequency mode manifests as a laminar flow, with its phase changing very slowly over time. The high-frequency mode is represented by vortical structures distributed on both sides of the cylinder, where the phase of this mode changes relatively quickly over time.
As seen in \reffig{fig:karman_editing}, when we adjust the modulus of the high-frequency cluster from $0.5$ to $1.5$, the intensity of the vortices increases, which is as we expected. When we set the real part of $\bm{\Omega}$ to $0.5$, it can be observed that the high-frequency motion decays faster than user input. When we set the real part of $\bm{\Omega}$ to $1.5$, it can be observed that the high-frequency motion decays slower than user input. Similarly, when we tune the imaginary part of $\bm{\Omega}$ from $0.5$ to $1.5$, we could observe the oscillation frequency of the fluid trail transitions from slow to fast compared to user input.
% explanation for what we did to edit in 3D plume scene
\paragraph{Editing the Plume with Bunny}
To evaluate the editing capability of our method, we scale our editing scenario to 3D. With the same filtering procedure as in the K\'arm\'an vortex street example, we set the low-frequency cluster to high-frequency cluster ratio to $4:1$, $2:1$, $1:2$, and $1:4$, and compared the results with the user input. From the results, we observe that when the proportion of low-frequency cluster is increased, with a ratio of $4:1$, the top of the plume lacks "wrinkles" and appears more "fluffy". This is because the velocity field is dominated by smoother, lower-frequency modes than the original user input. Conversely, when the proportion of high-frequency cluster is increased, with ratios of $1:4$, the plume developes more detailed plume structure around the top, as the velocity field now emphasizes more high-frequency details compared to the user input.

\subsection{Reversibility of the Reduced Simulation}
Although physically-based fluid simulations have the capability to generate stunning visuals, when artists aim to direct the fluid's evolution toward a predefined target shape, challenges arise. It is a long standing problem in the community that people aim to enable users with \emph{spatial control}. In this example, we aim to enable users to do \emph{temporal control}, motived by a prior work \citet{oborn2018time}. Compared to previous work \shortcite{oborn2018time} where the authors employ a self-attraction force to replace the arbitrary external forces, providing a stable, physics-motivated, but time-consuming approach, we propose a data-driven, fast, and easy to implement method to address the same problem.

\label{sec:reversibility}

We observe that that given $\bm{\tilde{K}} = \bm{\Phi} \bm{\Lambda} \bm{\Phi}^+$, we could easily compute the \emph{inverse} of the truncated \koopman{} $\bm{\tilde{K}}^{-1} = (\bm{\Phi} \bm{\Lambda} \bm{\Phi}^+)^{-1} = \bm{\Phi} \bm{\Lambda}^{-1} \bm{\Phi}^+$, which is essentially the approximate inverse time evolution $\bm{f}^{-1}(\bm u)$ of the fluid system. This allows us to reverse the simulation by applying the inverse truncated \koopman{} to the current state of the fluid system:
\begin{equation}
    \label{eqn:reverse_simulation}
    \begin{aligned}
        \bm{u}(t) &= \bm{A}^{-1} \bm{u}(t + \Delta t), \\
        \bm{u}(t) &= \bm{\Phi} \bm{\Lambda}^{-1}\bm{\Phi}^+ \bm{u}(t + \Delta t), \\
        \bm{u}(t) &= \bm{\Phi} \bm{\Lambda}^{-1} \bm{z}(t + \Delta t).
    \end{aligned}
\end{equation}

Similar to \refeq{eqn:reduced_koopman_projection}, we could train the reduced \koopman{} on the forward simulation data and then apply the inverse reduced \koopman{} to reverse the simulation, given a state of the fluid system.


\begin{figure*}[!ht]
    \centering
    \includegraphics[width=1\linewidth]{figure/upsample.pdf}
    \caption{\textbf{Upsampling and Generalization to Unseen Sequences with Trained DMD Operator}. Two different input low-resolution fluid simulations (bunny and strawberry) are upscaled using the same DMD operator trained on a different velocity field. Initial velocity fields are seeded as moving down based on the input density field.    
    Naive application of DMD shown in each middle column, and our \emph{augmented DMD upresolution} method shown on the right columns. 
    Schematic of our method presented on the far right. At each frame, we project the low-resolution artist-directed input into the low-order bases of our reduced representation, using these to replace the low-order terms of the DMD field. Notice that naive application of DMD simply moves towards the known input training data, while our augmented field matches the low-resolution input more closely, with extra high-order detail gained from the DMD operator.}
    \label{fig:upsample}
    \Description{}
\end{figure*}


% first explanation for buoyant reversibility
\paragraph{Reversibility of Buoyant Flow}
We experiment our approach on a simple buoyant flow setup (\reffig{fig:reverse_simulation}, left). Our dataset was initialized with a \textit{qian}, a density field shaped like a round coin with a square hole, with the density value set to $1$. A density value of $1$ density field was driven by a velocity field where an upwards velocity of $0.3$ is set within the qian and downwards elsewhere. We run the simulation for $300$ frames to construct the dataset, and trained the DMD operator on this dataset. The inverse operator $\bm{\tilde{K}}^{-1}$ was then applied to the initial velocity field of the dataset at $t=0$ (frame $0$). By iteratively applying the inverse operator, we obtained the velocity fields for the preceding frames, starting from frame $-1$, frame $-2$, and all the way back to frame $-300$.
% stability
When examining the evolution of the density field from frame -300 to frame 300, it is evident that the velocity field remains consistently upward and smooth, indicating that our method is both reasonable and effective.
% energy
Further analysis of the energy of the velocity field obtained through the inverse process and the velocity field from the dataset reveals a downward trend in energy, with a smooth and reasonable curve, consistent with fluids with dissipative properties. This demonstrates that our inverse operator has the ability to predict a \emph{physically-plausible} velocity field prior to the dataset.

% second explanation for vortical reversibility
\paragraph{Reversibility of Vortical Flow}
To challenge the method with a scene of nontrivial vortical structure, we initialized a vortex sheet by placing four vortices at the corners of the domain (\reffig{fig:reverse_simulation}, right). We generated the dataset using the same procedure as in the previous experiment, resulting in a collection of $500$ frames. Subsequently, we constructed the inverse operator to recover the velocity fields preceding the dataset.
% stability
The results show that the density field (counterclockwise) and the dataset (clockwise) rotate in the opposite direction, which indicates that the velocity field predicted by the inverse operator is correct. This is because the vortex sheet velocity field continuously rotates in a clockwise direction, and by examining the density field from frame -500 to frame 500, we observe that the field indeed undergoes continuous clockwise rotation.
% energy
From the energy field analysis, the results show that, except for the significant energy fluctuation between frames -500 and -450, the energy consistently decreases in the remaining frames, with a consistent slope. This further demonstrates the robustness of our method.

\subsection{Upsampling with Reduced Koopman Operator}

The scale of the imaginary part of eigenvalues in $\bm{\Lambda}$ encode different scales of turbulent modes, enabling us to use a trained DMD operator to add in secondary motion to an existing fluid simulation. This is particularly useful for \emph{upscaling} a low-resolution fluid, simulated using stable fluid for example, leveraging the DMD basis to add in turbulent modes that were too small for the low-res sim to capture. This upscaling problem has been explored in prior work \cite{kim2008wavelet, nielsen2009guiding}, but we show that due to the linearity of the Koopman operator, and the physical intuition on each of its reduced bases, this upscaling is essentially attained for \emph{free}, amounting to nothing more than a linear combination of two matrix multiplications. 

\subsubsection{Evolution} \label{sec:upres_direct}

Suppose we have frames of a low-res input velocity field $\{\bm{L}_0, \bm{L}_1, \bm{L}_2, \dots, \bm{L}_T\}$, a high-res initial condition $H_0$. Additionally, we have some DMD basis $\bm{\Phi}$ trained on some high-res simulation distinct from the low-res simulation, with corresponding eigenvalues $\bm{\Lambda}$, sorted by the length of their imaginary parts in increasing order. At the first frame, we can generate the reduced-space initial condition by simply using our basis mapping $R_0 = \bm{\Phi}^TH_0$.

Now, for every subsequent frame $t$, we generate $R_t$ by first applying the DMD evolution on the previous reduced space frame to produce an intermediate state $R^*_t=\bm{\Lambda}R_{t-1}$. We also produce a representation of the current frame of the low-res input in reduced space $P_t = \bm{\Phi}^TL_t$. Now, we have a representation of the \emph{current} frame of the low-res input, and the DMD \emph{time evolution} of the \emph{previous} reduced space frame. We want to keep the low-order bulk flow of the low-res input, and augment it with the high-order turbulent flow learned by the DMD basis. To that end, we split each reduced-space vector into a low-order and high-order part: $R^*_t = \left[R_t^{*L}\ R_t^{*H}\right]$, $P_t=\left[P_t^L\ P_t^H\right]$. Now, we take only the low-order modes of the input flow, and the high-order modes of the DMD-evolved flow, to produce our new reduced space velocity field $R_t=\left[P_t^L\ R_t^{*H}\right]$. From here, we can just apply the basis to return to high-resolution full-space $H_t=\bm{\Phi}R_t$.

We note that the composition operators here are linear. We can simply represent them with selection matrices $S^H$, $S_L$, for the high- and low-order bases respectively, such that $R_t=S^LP_t + S^HR_t^*$. Since the DMD operator is also linear, we note that this entire upscaling method is linear by construction.

Results are shown on Figure \ref{fig:upsample}. We see that even if the initial velocity field is significantly different from the input field, the low-order basis is able to capture the bulk flow of the low-resolution input, and modify the DMD-produced field accordingly. In particular, we note that naively applying the DMD operator, without passing the low-resolution input field into the low-order bases, ends up reconstructing the original training set, rather than a velocity field directed by our input. This is demonstrated by the results for the two initial conditions being very similar, whereas our augmented field matches the input much closer.

\subsubsection{Projection}

The above governs the time evolution of the velocity field. In some cases, where the input velocity field differs significantly from the training data used for the DMD basis, the above as written will still produce velocity fields that are unacceptably different from the input velocity field. This is largely representation error, fields that are far away from the training data are less representable by the reduced space. In these cases, we can again leverage our input low-res field, this time as a constraint. 

Essentially, we would like to project our velocity field $\bm{H}_t$ onto the space of velocity fields that are identical to the input low-res field when downsampled to that resolution. This can be represented as an equality-constrained quadratic problem,
\begin{align}
    &\argmin_x \frac{1}{2}(\bm{x}-\bm{H}_t)^T(\bm{x}-\bm{H}_t) \\
    &\text{subject to } \bm{Ax} = \bm{L}_t,
\end{align}
where $\bm{A}$ is a downsampling operator that converts from high-res to low-res. 
Notice that because the downsampling operator does not change for the duration of the simulation. Thus, the KKT (Karush-Kuhn-Tucker) matrix can be precomputed making the projection a single matrix multiply during runtime.

\begin{wrapfigure}{r}{0.5\columnwidth}
    \vspace{-2pt}
    \includegraphics[width=0.5\columnwidth]{figure/qr.pdf}
    \hspace{5pt}
    \label{fig:qp_project}
\end{wrapfigure}

As a sanity check, we show the effect of this projection here: it is apparent with the projection step,we can recover fields that are much closer to the input, yet retaining extra high-order detail. And of course, because these are all linear, linear combinations of the direct and projected fields can be taken. In particular, because the basis functions of reduced space are orthogonal, a diagonal matrix of linear weights can be taken, preferring projected for low-order modes and direct for high-order modes for example.

\begin{comment}
Given a high-resolution DMD matrix $A \in \mathbb{R}^{N_{hi} \times N_{hi}}$ trained on high-dimensional data, we reconstruct a high-resolution sequence $\bm{x}_{hi}(t) \in \mathbb{R}^{N_{hi}}$ using an initial high-resolution frame $\bm{x}_{hi}(0)$ and subsequent low-resolution frames $\bm{x}_{lo}(t) \in \mathbb{R}^{N_{lo}}$, where $N_{lo} < N_{hi}$. The matrix $A$ is structured as
\begin{equation}
    \label{eqn:slice_A}
    A = \begin{bmatrix} A_{ll} & A_{lh} \\ A_{hl} & A_{hh} \end{bmatrix}
\end{equation}, with $A_{ll} \in \mathbb{R}^{N_{lo} \times N_{lo}}$ map low frequency component to, $A_{lh} \in \mathbb{R}^{N_{lo} \times (N_{hi} - N_{lo})}$, $A_{hl} \in \mathbb{R}^{(N_{hi} - N_{lo}) \times N_{lo}}$, and $A_{hh} \in \mathbb{R}^{(N_{hi} - N_{lo}) \times (N_{hi} - N_{lo})}$.

Starting from the initial condition $\bm{x}_{hi}(0)$, the high-resolution state at time $t + \Delta t$ is updated using:
\begin{equation}
    \label{eqn:upsampling_advect}
    \bm{x}_{hi}(t + \Delta t) = A \bm{x}_{hi}(t) + \begin{bmatrix} \bm{x}_{lo}(t + \Delta t) - A_{ll} \bm{x}_{lo}(t) \\ A_{hl} \left( \bm{x}_{lo}(t + \Delta t) - A_{ll} \bm{x}_{lo}(t) \right) \end{bmatrix}.
\end{equation}

In this equation, $A \bm{x}_{hi}(t)$ evolves the high-resolution dynamics. The term $\bm{x}_{lo}(t + \Delta t) - A_{ll} \bm{x}_{lo}(t)$ represents the correction to the low-frequency component, and $A_{hl} \left( \bm{x}_{lo}(t + \Delta t) - A_{ll} \bm{x}_{lo}(t) \right)$ in-paints the missing high-frequency details. This process ensures the reconstructed high-resolution sequence remains consistent with the initial frame and the low-resolution input while leveraging the full dynamics encoded in $A$.

\end{comment}

Our findings challenge the conjecture that code-comment coherence, as measured by SIDE \cite{mastropaolo2024evaluating}, is a critical quality attribute for filtering instances of code summarization datasets. By selecting $\langle code, summary \rangle$ pairs with high-coherence for training allow to achieve the same results that would be achieved by randomly selecting such a number of instances. At the same time, we observed that reducing the datasets size up to 50\% of the training instances does not significantly affect the effectiveness of the models, even when the instances are randomly selected. These results have several implications.

First, code-comment consistency might not be a problem in state-of-the-art datasets in the first place, as also suggested in the results of RQ$_0$. Also, the DL models we adopted (and, probably, bigger models as well) are not affected by inconsistent code-comment pairs, even when these inconsistencies are present in the training set.
Despite the theoretical benefits of filtering by SIDE \cite{mastropaolo2024evaluating}, that is the state-of-the-art metric for measuring code-comment alignment, our results indicate its limitations in improving the \textit{overall} quality of the training sets for code summarization task.
Nevertheless, other quality aspects of code and comments that have not been explored yet (such as readability) may be important for smartly selecting the training instances.
Future work should explore such quality aspects further.

Our results clearly indicate that state-of-the-art datasets contain instances that do not contribute to improving the models' effectiveness. This finding is related to a general phenomenon observed in Machine Learning and Deep Learning. Models reach convergence when they are trained for a certain amount of time (epochs). Additional training provides smaller improvements and increases the risk of overfitting. We show that the same is true for data. In terms of effectiveness, model convergence is achieved with fewer training instances than previously assumed. Limiting the number of epochs may make it possible to reach model convergence with a subset of training data, maintaining model effectiveness, reducing resource demands and minimizing the risk of overfitting.
Future work could explore different criteria for data selection that identify the most informative subsets for training.
Conversely, this insight suggests that currently available datasets suffer from poor diversity (thus causing the previously discussed phenomenon).
This latter insight constitutes a clear warning for researchers interested in building code summarization datasets, which should include instances that add relevant information instead of adding more data, which might turn out to be useless.

Finally, it is worth pointing out that another benefit of the reduction we performed is the environmental impact. Reducing the number of training instances implies a reduced training time, which, in turn, lowers the resources necessary to perform training and, thus, energy consumption and CO$_2$ emissions.
We performed a rough estimation of the training time across different selections of \textit{TL-CodeSum} and \textit{Funcom} datasets and estimated a proxy of the CO$_2$ emissions for each model training phase by relying on the ML CO$_2$ impact calculator\footnote{\url{https://mlco2.github.io/impact/\#compute}} \cite{lacoste2019quantifying}. Such a calculator considers factors such as the total training time, the infrastructure used, the carbon efficiency, and the amount of carbon offset purchased. The estimation of CO$_{2}$ emissions needed to train the model with the \textit{Full} selection of \textit{Funcom} ($\sim$ 200 hours) is equal to 26.05 Kg, while with the optimized training set, \ie $SIDE_{0.9}$ ($\sim$ 90 hours), the estimation is 11.69 Kg of $CO_2$ (-55\% emissions).
While we recognize that this method provides an estimation rather than a precise measurement, it offers a glimpse into the environmental impact of applying data reduction.


\section*{Ackowledgement}
W. Wu and A.Rinaldo are supported in part by NIH under Grant R01 NS121913. 
Y. Wei is supported in part by the NSF grants CCF-2106778, CCF-2418156 and CAREER award DMS-2143215. 

% \newpage

\bibliography{refs.bib,bibfileRL,bibfileRL-2}
%\bibliographystyle{alpha}
\bibliographystyle{plainnat}

% \newpage
\appendix 
% Consider a lasso optimization procedure with potentially distinct regularization penalties:
% \begin{align}
%     \hat{\beta} = \arg\min_{\beta}\{\|y-X\beta\|^2_2+\sum_{i=1}^{N}\lambda_i|\beta_i|\}.
% \end{align}
\subsection{Supervised Data-Driven Learning}\label{subsec:supervised}
We consider a generic data-driven supervised learning procedure. Given a dataset \( \mathcal{D} \) consisting of \( n \) data points \( (x_i, y_i) \in \mathcal{X} \times \mathcal{Y} \) drawn from an underlying distribution \( p(\cdot|\theta) \), our goal is to estimate parameters \( \theta \in \Theta \) through a learning procedure, defined as \( f: (\mathcal{X} \times \mathcal{Y})^n \rightarrow \Theta \) 
that minimizes the predictive error on observed data. 
Specifically, the learning objective is defined as follows:
\begin{align}
\hat{\theta}_f := f(\mathcal{D}) = \arg\min_{\theta} \mathcal{L}(\theta, \mathcal{D}),
\end{align}
where \( \mathcal{L}(\cdot,\mathcal{D}) := \sum_{i=1}^{n} \mathcal{L}(\cdot, (x_i, y_i))\), and $\mathcal{L}$ is a loss function quantifying the error between predictions and true outcomes. 
Here, $\hat{\theta}_f$ is the parameter that best explains the observed data pairs \( (x_i, y_i) \) according to the chosen loss function \( \mathcal{L} (\cdot) \).

\paragraph{Feature Selection.}
Feature selection aims to improve model \( f \)'s predictive performance while minimizing redundancy. 
%Formally, given data \( X \), response \( y \), feature set \( \mathcal{F} \), loss function \( \mathcal{L}(\cdot) \), and a feature limit \( k \), the objective is:
% \begin{align}
% \mathcal{S}^* = \arg \min_{\mathcal{S} \subseteq \mathcal{F}, |\mathcal{S}| \leq k} \mathcal{L}(y, f(X_\mathcal{S})) + \lambda R(\mathcal{S}),
% \end{align}
% where \( X_\mathcal{S} \) is the submatrix of \( X \) for selected features \( \mathcal{S} \), \( \lambda \) is a regularization parameter, and \( R(\mathcal{S}) \) penalizes feature redundancy.
 State-of-the-art techniques fall into four categories: (i) filter methods, which rank features based on statistical properties like Fisher score \citep{duda2001pattern,song2012feature}; (ii) wrapper methods, which evaluate model performance on different feature subsets \citep{kohavi1997wrappers}; (iii) embedded methods, which integrate feature selection into the learning process using techniques like regularization \citep{tibshirani1996LASSO,lemhadri2021lassonet}; and (iv) hybrid methods, which combine elements of (i)-(iii) \citep{SINGH2021104396,li2022micq}. This paper focuses on embedded methods via Lasso, benchmarking against approaches from (i)-(iii).

\subsection{Language Modeling}
% The objective of language modeling is to learn a probability distribution \( p_{LM}(x) \) over sequences of text \( x = (X_1, \ldots, X_{|x|}) \), such that \( p_{LM}(x) \approx p_{text}(x) \), where \( p_{text}(x) \) represents the true distribution of natural language. This process involves estimating the likelihood of token sequences across variable lengths and diverse linguistic structures.
% Modern large language models (LLMs) are trained on vast datasets spanning encyclopedias, news, social media, books, and scientific papers \cite{gao2020pile}. This broad training enables them to generalize across domains, learn contextual knowledge, and perform zero-shot learning—tackling new tasks using only task descriptions without fine-tuning \cite{brown2020gpt3}.
Language modeling aims to approximate the true distribution of natural language \( p_{\text{text}}(x) \) by learning \( p_{\text{LM}}(x) \), a probability distribution over text sequences \( x = (X_1, \ldots, X_{|x|}) \). Modern large language models, trained on diverse datasets \citep{gao2020pile}, exhibit strong generalization across domains, acquire contextual knowledge, and perform zero-shot learning—solving new tasks using only task descriptions—or few-shot learning by leveraging a small number of demonstrations \citep{brown2020gpt3}.
\paragraph{Retrieval-Augmented Generation (RAG).} Retrieval-Augmented Generation (RAG) enhances the performance of generative language models by  integrating a domain-specific information retrieval process  \citep{lewis2020retrieval}. The RAG framework comprises two main components: \textit{retrieval}, which extracts relevant information from external knowledge sources, and \textit{generation}, where an LLM generates context-aware responses using the prompt combined with the retrieved context. Documents are indexed through various databases, such as relational, graph, or vector databases \citep{khattab2020colbert, douze2024faiss, peng2024graphretrievalaugmentedgenerationsurvey}, enabling efficient organization and retrieval via algorithms like semantic similarity search to match the prompt with relevant documents in the knowledge base. RAG has gained much traction recently due to its demonstrated ability to reduce incidence of hallucinations and boost LLMs' reliability as well as performance \citep{huang2023hallucination, zhang2023merging}. 
 
% image source: https://medium.com/@bindurani_22/retrieval-augmented-generation-815c1ae438d8
\begin{figure}
    \centering
\includegraphics[width=1.03\linewidth]{fig/fig1.pdf}
\vspace{-0.6cm}
\scriptsize 
    \caption{Retrieval Augmented Generation (RAG) based $\ell_1$-norm weights (penalty factors) for Lasso. Only feature names---no training data--- are included in LLM prompt.} 
    \label{fig:rag}
\end{figure}
% However, for the RAG model to be effective given the input token constraints of the LLM model used, we need to effectively process the retrieval documents through a procedure known as \textit{chunking}.

\subsection{Task-Specific Data-Driven Learning}
LLM-Lasso aims to bridge the gap between data-driven supervised learning and the predictive capabilities of LLMs trained on rich metadata. This fusion not only enhances traditional data-driven methods by incorporating key task-relevant contextual information often overlooked by such models, but can also be especially valuable in low-data regimes, where the learning algorithm $f:\mathcal{D}\rightarrow\Theta$ (seen as a map from datasets $\mathcal{D}$ to the space of decisions $\Theta$) is susceptible to overfitting.

The task-specific data-driven learning model $\tilde{f}:\mathcal{D}\times\mathcal{D}_\text{meta}\rightarrow\Theta$ can be described as a metadata-augmented version of $f$, where a link function $h(\cdot)$ integrates metadata (i.e. $\mathcal{D}_\text{meta}$) to refine the original learning process. This can be expressed as:
\[
\tilde{f}(\mathcal{D}, \mathcal{D}_\text{meta}) := \mathcal{T}(f(\mathcal{D}),  h(\mathcal{D}_{\text{meta}})),
\]
where the functional $\mathcal{T}$ takes the original learning algorithm $f(\mathcal{D})$ and transforms it into a task-specific learning algorithm $\tilde{f}(\mathcal{D}, \mathcal{D}_\text{meta})$ by incorporating the metadata $\mathcal{D}_\text{meta}$. 
% In particular, the link function $h(\mathcal{D}_{\text{meta}})$ provides a structured mechanism summarizing the contextual knowledge.

There are multiple approaches to formulate $\mathcal{T}$ and $h$.
%to ``inform" the data-driven model $f$ of %meta knowledge. 
For instance, LMPriors \citep{choi2022lmpriorspretrainedlanguagemodels} designed $h$ and $\mathcal{T}$ such that $h(\mathcal{D}_{\text{meta}})$ first specifies which features to retain (based on a probabilistic prior framework), and then $\mathcal{T}$ keeps the selected features and removes all the others from the original learning objective of $f$. 
Note that this approach inherently is restricted as it selects important features solely based on $\mathcal{D}_\text{meta}$ without seeing $\mathcal{D}$.

In contrast, we directly embed task-specific knowledge into the optimization landscape through regularization by introducing a structured inductive bias. This bias guides the learning process toward solutions that are consistent with metadata-informed insights, without relying on explicit probabilistic modeling. Abstractly, this can be expressed as:
\begin{align}
    \!\!\!\!\!\hat{\theta}_{\tilde{f}} := \tilde{f}(\mathcal{D},\mathcal{D}
    _\text{meta})= \arg\min_{\theta} \mathcal{L}(\theta, \mathcal{D}) + \lambda R(\theta, \mathcal{D}_{\text{meta}}),
\end{align}
where \( \lambda \) is a regularization parameter, \( R(\cdot) \) is a regularizer, and $\theta$ is the prediction parameter.
%We explain our framework with more details in the following section.


% Our research diverges from both aforementioned approaches by positioning the LLM not as a standalone feature selector but as an enhancement to data-driven models through an embedded feature selection method, L-LASSO. L-LASSO incorporates domain expertise—auxiliary natural language metadata about the task—via the LLM-informed LASSO penalty, which is then used in statistical models to enhance predictive performance. This method integrates the rich, context-sensitive insights of LLMs with the rigor and transparency of statistical modeling, bridging the gap between data-driven and knowledge-driven feature selection approaches. To approach this task, we need to tackle two key components: (i). train an LLM that is expert in the task-specific knowledge; (ii). inform data-driven feature selector LASSO with LLM knowledge.

% In practice, this involves combining techniques like prompt engineering and data engineering to develop an effective framework for integrating metadata into existing data-driven models. We will go through this in detail in Section \ref{mthd} and \ref{experiment}.



\section{Proof of theoretical results regarding general Markov chains}

\subsection{Proof of Theorem \ref{thm:matrix-hoeffding}}\label{app:proof-matrix-hoeffding}
A classic Chernoff argument indicates
\begin{align}\label{eq:matrix-chernoff}
\mathbb{P}\left(\left\|\frac{1}{n}\sum_{i=1}^n \bm{F}_i(s_i)\right\| \geq \varepsilon \right) &\leq 2\inf_{t \geq 0} \exp(-nt\varepsilon) \mathbb{E}\left[\mathsf{Tr}\left(\exp\left(t\sum_{i=1}^n \bm{F}_i(s_i)\right)\right)\right].
\end{align}
In order to bound the right-hand-side, \cite{garg2018matrixexpanderchernoff} illustrated in their Equation (11), through an application of the multi-matrix Golden-Thompson inequality, that there exists a probability distribution $\phi$ on the interval $[-\frac{\pi}{2},\frac{\pi}{2}]$, such that
% \yuting{give the theorem number}\weichen{added Equation number.}
\begin{align}\label{eq:multi-matrix-golden-thompson}
\mathsf{Tr}\left(\exp\left(t\sum_{i=1}^n \bm{F}_i(s_i)\right)\right) \leq d^{1-\frac{\pi}{4}}\int_{-\frac{\pi}{2}}^{\frac{\pi}{2}}\mathsf{Tr}\left[\prod_{i=1}^n \exp\left(\frac{2}{\pi}e^{\mathbf{i}\theta}t\bm{F}_i(s_i)\right)\prod_{i=n}^1 \exp\left(\frac{2}{\pi}e^{-\mathbf{i}\theta}t\bm{F}_i(s_i)\right)\right]\mathrm{d}\phi(\theta).
\end{align}
Furthermore, by repeatedly applying the basic properties of Kronecker products, the trace on the right-hand-side can be computed as 
\begin{align}\label{eq:matrix-hoeffding-kronecker}
&\mathsf{Tr}\left[\prod_{i=1}^n \exp\left(\frac{2}{\pi}e^{\mathbf{i}\theta}t\bm{F}_i(s_i)\right)\prod_{i=n}^1 \exp\left(\frac{2}{\pi}e^{-\mathbf{i}\theta}t\bm{F}_i(s_i)\right)\right] \nonumber \\ 
&= [\mathbf{vec}(\bm{I}_d)]^\top \left\{\prod_{i=n}^1 \exp\left(\frac{2}{\pi}e^{-\mathbf{i}\theta}t\bm{F}_i(s_i)\right) \otimes  \prod_{i=n}^1\exp\left(\frac{2}{\pi}e^{\mathbf{i}\theta}t\bm{F}_i(s_i)\right)\right\}\mathbf{vec}(\bm{I}_d) \nonumber \\ 
&= [\mathbf{vec}(\bm{I}_d)]^\top \prod_{i=n}^1\left\{\exp\left(\frac{2}{\pi}e^{-\mathbf{i}\theta}t\bm{F}_i(s_i)\right) \otimes \exp\left(\frac{2}{\pi}e^{\mathbf{i}\theta}t\bm{F}_i(s_i)\right)\right\}\mathbf{vec}(\bm{I}_d) \nonumber \\ 
&= [\mathbf{vec}(\bm{I}_d)]^\top \prod_{i=n}^1 \exp(t\bm{H}_i(s_i,\theta))\mathbf{vec}(\bm{I}_d),
\end{align}
% \yuting{add details for this equality}\weichen{a short explanation follows. Does that make things clear?}
where we define, for simplicity,
\begin{align}\label{eq:defn-Hi}
\bm{H}_i(s_i,\theta):= \frac{2}{\pi}e^{-\mathbf{i}\theta}\bm{F}_i(s_i) \otimes \bm{I}_{d^2} + \bm{I}_{d^2} \otimes \frac{2}{\pi}e^{\mathbf{i}\theta}\bm{F}_i(s_i).
\end{align}
Through the deduction of \eqref{eq:matrix-hoeffding-kronecker}, we applied properties (6), (1) and (7) in Theorem \ref{thm:Kronecker} in the second, third and fourth line respectively. 
It is easy to verify that the matrix function $\bm{H}_i$ defined as in \eqref{eq:defn-Hi} has the following properties:
\begin{align}\label{eq:Hi-properties}
&\mathbb{E}_{s_i \sim \mu} [\bm{H}_i(s_i,\theta)] = \bm{0}, \quad \forall i \in [n], \theta \in \left[-\frac{\pi}{2},\frac{\pi}{2}\right];\\ 
&\|\bm{H}_i(s_i,\theta)\| \leq \frac{4}{\pi} M_i, \quad \text{a.s.}, \quad \forall i \in [n],  \theta \in \left[-\frac{\pi}{2},\frac{\pi}{2}\right].
\end{align}

As a combination of \eqref{eq:matrix-chernoff}, \eqref{eq:multi-matrix-golden-thompson} and \eqref{eq:matrix-hoeffding-kronecker}, our goal is to bound
\begin{align*}
&\mathbb{P}\left(\left\|\frac{1}{n}\sum_{i=1}^n \bm{F}_i(s_i)\right\| \geq \varepsilon \right)  \\ 
&\leq 2d^{1-\frac{\pi}{4}} \inf_{t \geq 0} \exp(-nt\varepsilon) [\mathbf{vec}(\bm{I}_d)]^\top \mathbb{E}_{\theta \sim \phi} \mathbb{E}_s\left[\prod_{i=n}^1 \exp(t\bm{H}_i(s_i,\theta))\mathbf{vec}(\bm{I}_d)\right] \\ 
&\leq 2d^{1-\frac{\pi}{4}} \inf_{t \geq 0}  \sup_{\theta \in [-\frac{\pi}{2},\frac{\pi}{2}]} \exp(-nt\varepsilon) [\mathbf{vec}(\bm{I}_d)]^\top \mathbb{E}_s\left[\prod_{i=n}^1 \exp(t\bm{H}_i(s_i,\theta))\mathbf{vec}(\bm{I}_d)\right]
\end{align*}
% \red{expectation is over all the randomness. why only write $s_1$?}\weichen{checked.}
% \yuting{explain the notation $\mathbb{E}_{s_1 \sim \mu,s_{t+1}\sim P(\cdot \mid s_t)}$}
% \weichen{please see the explanation below.}
Here, we use $\mathbb{E}_s$ to denote the expectation taken over all the samples in the markov chain $s_1,s_2,...,s_n$, where $s_1 \sim \mu$ and $s_{i+1} \sim P(\cdot \mid s_i)$ for all $1 \leq i < n$. In order to bound this expectation, we define, for any $t>0$ and $\theta  \in [-\frac{\pi}{2},\frac{\pi}{2}]$, a sequence of vector-valued functions $\{\bm{g}_k\}_{k \in [n]}$ taking values in $\mathbb{R}^{d^2}$, where 
\begin{align*}
&\bm{g}_0 (x) = \textbf{vec}(\bm{I}_d), \quad \text{and} \\ 
&\bm{g}_k (x) = \mathbb{E}\left[\prod_{i=k}^1 \exp(t\bm{H}_i(s_i,\theta))\textbf{vec}(\bm{I}_d)\bigg| s_{k+1} = x\right], \quad k \geq 1.
\end{align*}
In this way, the left-hand-side of \eqref{eq:markov-matrix-hoeffding} is bounded by
\begin{align*}
\mathbb{P}\left(\left\|\frac{1}{n}\sum_{i=1}^n \bm{F}_i(s_i)\right\| \geq \varepsilon \right) &\leq   2d^{1-\frac{\pi}{4}} \inf_{t \geq 0}  \sup_{\theta \in [-\frac{\pi}{2},\frac{\pi}{2}]} \exp(-nt\varepsilon) [\mathbf{vec}(\bm{I}_d)]^\top \mu(\bm{g}_n) \\ 
&\leq 2d^{1-\frac{\pi}{4}} \inf_{t \geq 0}  \sup_{\theta \in [-\frac{\pi}{2},\frac{\pi}{2}]} \exp(-nt\varepsilon) \left\|\mathbf{vec}(\bm{I}_d)\right\|_2 \cdot \|\mu(\bm{g}_n)\|_2 \\ 
&= 2d^{1-\frac{\pi}{4}} \inf_{t \geq 0}  \sup_{\theta \in [-\frac{\pi}{2},\frac{\pi}{2}]} \exp(-nt\varepsilon) \sqrt{d} \cdot \|\bm{g}_n^{\parallel}\|_{\mu}
\end{align*}
where we applied the fact that $s_{n+1} \sim \mu$ when $s_1 \sim \mu$. Recall that by definition, $\mu(\bm{g}_n) = \mathbb{E}_{x \sim \mu}[\bm{g}_n(x)]$ and $\bm{g}_n^{\parallel}$ represents the constant function taking this value.
% \yuting{recall the definition of $\mu(\bm{g}_n)$, $\mu$ norm, and $\bm{g}_n^{\parallel}$.} \weichen{checked.}
Next, we construct a recursive relation among the sequence $\{\bm{g}_k\}_{k \in [n]}$ for the purpose of bounding the norm of $\bm{g}_n$. Notice that, on one hand,
\begin{align*}
\exp\left(t\bm{H}_k(x,\theta)\right)\bm{g}_{k-1}(x) = \mathbb{E}\left[\prod_{i=k}^1 \exp(t\bm{H}_i(s_i))\bigg| s_{k} = x\right];
\end{align*}
%\textcolor{violet}{in the last line, is it $\bm{F}_i(s_i)$ or $\bm{H}_i(s_i,\theta)$?}\weichen{checked.}
on the other hand, %$\bm{g}_k(x)$ can be represented by
\begin{align*}
\bm{g}_k(x) &= \mathbb{E}\left[\prod_{i=k}^1 \exp(t\bm{H}_i(s_i,\theta))\textbf{vec}(\bm{I}_d)\bigg| s_{k+1} = x\right] \\ 
&=  \int_{y \in \mathcal{S}} \mathbb{E}\left[\prod_{i=k}^1 \exp(t\bm{H}_i(s_i,\theta))\textbf{vec}(\bm{I}_d)\bigg| s_{k} = y\right] \mathrm{d}\mathbb{P}(s_k = y\mid s_{k+1} = x) \\ 
&= \int_{y \in \mathcal{S}} \left\{\mathbb{E}\left[\prod_{i=k}^1 \exp(t\bm{H}_i(s_i,\theta))\textbf{vec}(\bm{I}_d)\bigg| s_{k} = y\right] \right\}P^*(x,\mathrm{d}y);
\end{align*}
% \color{violet}
% working on this. I think it will be notationally easier to use densities instead of measures when representing conditional distributions
% \begin{align*}
% \bm{g}_k(x) &= \frac{\mathcal{P}^k\nu(\mathrm{d}x)}{\mu(\mathrm{d}x)}\mathbb{E}\left[\prod_{i=k}^1 \exp(t\bm{H}_i(s_i,\theta))\textbf{vec}(\bm{I}_d)\bigg| s_{k+1} = x\right] \\ 
% &=  \frac{\mathcal{P}^k\nu(\mathrm{d}x)}{\mu(\mathrm{d}x)}\int_{y \in \mathcal{S}} \mathbb{E}\left[\prod_{i=k}^1 \exp(t\bm{H}_i(s_i,\theta))\textbf{vec}(\bm{I}_d)\bigg| s_{k} = y, s_{k+1} = x\right] \mathrm{d}\mathbb{P}_{S_k|S_{k+1}}( y\mid  x) \\ 
% &= \int_{y \in \mathcal{S}} \mathbb{E}\left[\prod_{i=k}^1 \exp(t\bm{H}_i(s_i,\theta))\textbf{vec}(\bm{I}_d)\bigg| s_{k} = y\right] \frac{\mathrm{d} \mathcal{P}^k\nu(x)}{\mathrm{d} \mu(x)} \cdot \frac{\mathrm{d} \mathcal{P}^{k-1}\nu(y)\mathrm{d}P_{S_{k+1} \mid S_k}(y|x)}{\mathrm{d} \mathcal{P}^k\nu(x)}\\ 
% &= \int_{y \in \mathcal{S}} \mathbb{E}\left[\prod_{i=k}^1 \exp(t\bm{H}_i(s_i,\theta))\textbf{vec}(\bm{I}_d)\bigg| s_{k} = y\right] \frac{\mathrm{d} \mathcal{P}^k\nu(x)}{\mathrm{d} \mu(x)} \cdot \frac{\mathrm{d} \mathcal{P}^{k-1}\nu(y) P(x,dy)}{\mathrm{d} \mathcal{P}^k\nu(x)}\\ 
% &= \int_{y \in \mathcal{S}} \left\{\mathbb{E}\left[\prod_{i=k}^1 \exp(t\bm{H}_i(s_i,\theta))\textbf{vec}(\bm{I}_d)\bigg| s_{k} = y\right] \frac{\mathrm{d}\mathcal{P}^{k-1}\nu(y)}{\mu(\mathrm{d}y)}\right\}P^*(x,\mathrm{d}y);
% \end{align*}
% \color{black}
% \weichen{I think an easier way is to just assume $s_0 \sim \mu$, the stationary distribution. The non-stationary starting distribution case can be handled by applying Holder's inequlity, though this would introduce a constant in the upper bound.}
here the contents in the curly bracket is equal to $\exp\left(t\bm{H}_k(y,\theta)\right)\bm{g}_{k-1}(u)$. Therefore, the sequence $\{\bm{g}_k\}$ is featured by the recursive relation
\begin{align}\label{eq:markov-matrix-hoeffding-G-recursive}
\bm{g}_{k}(x) &= \int_{y \in \mathcal{S}}\exp\left(t\bm{H}_k(y,\theta)\right)\bm{g}_{k-1}(y) P^*(x,\mathrm{d}y) \nonumber \\ 
&= (\mathcal{P}^* (\exp(t\bm{H}_k)\bm{g}_{k-1}))(x).
\end{align}
Based on this recursive relationship, the following proposition comes in handy for bounding the norm of $\bm{g}_k$ recursively.

\begin{customproposition}\label{prop:PE}
For any matrix function $\bm{g}: \mathcal{S} \to \mathbb{R}^{m}$ and any bounded symmetric matrix function $\bm{H}: \mathcal{S} \to \mathbb{S}^{m \times m}$ with  $\mu(\bm{H}) = \bm{0}$ and $\|\bm{F}(x)\| \leq M$ almost surely. Then the following hold for any $t > 0$:
\begin{align}
&\left\|\left(\mathcal{P}^* \exp(t\bm{H})\bm{g}^{\parallel}\right)^{\parallel}\right\|_{\mu} \leq \alpha_1 \|\bm{g}^{\parallel}\|_{\mu},\quad \text{where} \quad \alpha_1 = \exp(tM)-tM;\label{eq:PE1}\\
&\left\|\left(\mathcal{P}^* \exp(t\bm{H})\bm{g}^{\parallel}\right)^{\perp}\right\|_{\mu} \leq \alpha_2 \|\bm{g}^{\parallel}\|_{\mu},\quad \text{where} \quad \alpha_2 = \lambda(\exp(tM)-1);\label{eq:PE2}\\ 
&\left\|\left(\mathcal{P}^* \exp(t\bm{H})\bm{g}^{\perp}\right)^{\parallel}\right\|_{\mu} \leq \alpha_3 \|\bm{g}^{\perp}\|_{\mu},\quad \text{where} \quad \alpha_3 = \exp(tM)-1; \label{eq:PE3}\\ 
&\left\|\left(\mathcal{P}^* \exp(t\bm{H})\bm{g}^{\perp}\right)^{\perp}\right\|_{\mu} \leq \alpha_4 \|\bm{g}^{\parallel}\|_{\mu},\quad \text{where} \quad \alpha_4 = \lambda \exp(tM).\label{eq:PE4}
\end{align}
\end{customproposition}
\begin{proof} 
See Appendix \ref{proof-prop-PE}.
\end{proof}

A recursive relation can be constructed to bound the norm of $\bm{g}_n^{\parallel}$. Specifically, for every $k \in [n]$, \eqref{eq:markov-matrix-hoeffding-G-recursive},the triangle inequality and \eqref{eq:PE1}, \eqref{eq:PE2} guarantee that
\begin{align*}
\|\bm{g}_k^{\parallel}\|_{\mu} &= \left\|(\mathcal{P}^* \exp(t\bm{H}_k)\bm{g}_{k-1})^{\parallel}\right\|_{\mu} \\ 
&\leq \left\|(\mathcal{P}^* \exp(t\bm{H}_k)\bm{g}_{k-1}^{\parallel})^{\parallel}\right\|_{\mu} + \left\|(\mathcal{P}^* \exp(t\bm{H}_k))\bm{g}_{k-1}^{\perp})^{\parallel}\right\|_{\mu} \\ 
&\leq \alpha_{1k} \|\bm{g}_{k-1}^{\parallel}\|_{\mu} + \alpha_{2k} \|\bm{g}_{k-1}^{\perp}\|_{\mu};
\end{align*}
here, we define $\alpha_{1k} = \exp(\frac{4}{\pi}tM_k) - \frac{4}{\pi}tM_k$ and $\alpha_{2k} = \lambda(\exp(\frac{4}{\pi}tM_k)-1)$. In order to iteratively bound the norm of $\bm{g}_k^{\parallel}$, we also need to bound the norm of $\bm{g}_{k-1}^{\perp}$. Towards this end, we observe, due to \eqref{eq:markov-matrix-hoeffding-G-recursive}, the triangle inequality and \eqref{eq:PE3}, \eqref{eq:PE4} that for every $k \in [n]$,
\begin{align*}
\|\bm{g}_k^{\perp}\|_{\mu} &= \left\|(\mathcal{P}^* \exp(t\bm{H}_k))\bm{g}_{k-1})^{\perp}\right\|_{\mu} \\ 
&\leq \left\|(\mathcal{P}^* \exp(t\bm{H}_k))\bm{g}_{k-1}^{\parallel})^{\perp}\right\|_{\mu} + \left\|(\mathcal{P}^* \exp(t\bm{H}_k))\bm{g}_{k-1}^{\perp})^{\perp}\right\|_{\mu} \\ 
&\leq \alpha_{3k} \|\bm{g}_{k-1}^{\parallel}\|_{\mu} + \alpha_{4k} \|\bm{g}_{k-1}^{\perp}\|_{\mu}.
\end{align*}
Here again, we define $\alpha_{3k} = \exp(\frac{4}{\pi}tM_k) - 1$ and $\alpha_{4k} = \lambda \exp(\frac{4}{\pi}tM_k)$. For simplicity, we denote
\begin{align*}
&\bm{x}_0 = \begin{pmatrix}
\|\bm{g}_0^{\parallel}\|_{\mu} \\ 
\|\bm{g}_0^{\perp}\|_{\mu} 
\end{pmatrix}=\begin{pmatrix}
\sqrt{d} \\ 
0
\end{pmatrix}, \quad \text{and} \\ 
&\bm{x}_k = \underset{\bm{U}_j}{\underbrace{\begin{pmatrix}
\alpha_{1k} & \alpha_{3k} \\ 
\alpha_{3k} & \alpha_{4k}
\end{pmatrix}}} \bm{x}_{k-1}, \quad \forall k \in [n].
\end{align*}
It can then be easily verified through an induction argument that 
\begin{align*}
\|\bm{g}_n^{\parallel}\|_{\mu} \leq x_{n1}.
\end{align*}
Notice here that we applied the fact $\alpha_{2k}< \alpha_{3k}$, since $\lambda < 1$. Consequently, 
\begin{align*}
\|\bm{g}_n^{\parallel}\|_{\mu} &\leq x_{n1} 
\leq \|\bm{x}_n\|_2 
= \left\|\prod_{k=1}^n \bm{U}_k \bm{x}_0\right\|_2 
 \leq \prod_{k=1}^n \|\bm{U}_k\| \|\bm{x}_0\|_2 =  \sqrt{d} \cdot \prod_{k=1}^n \|\bm{U}_k\|.
\end{align*}
Since $\bm{U}_k$ is a symmetric $2 \times 2$ matrix, its operator norm is featured by
\begin{align*}
\|\bm{U}_k\| = \max\{|\sigma_{k1}|,|\sigma_{k2}|\}
\end{align*}
where $\sigma_{k1}$ and $\sigma_{k2}$ are the two eigenvalues of $\bm{U}_k$. Recall from elementary linear algebra that $\sigma_{j1}$ and $\sigma_{j2}$ are solutions to the equation
\begin{align*}
(\alpha_{1k} - x)(\alpha_{4k}-x) -\alpha_{3k}^2 = 0.
\end{align*}
Since $\alpha_{1k} > 0$ and $\alpha_{4k} > 0$, it can be easily verified that
\begin{align*}
\|\bm{U}_k\| = \frac{\alpha_{1k} + \alpha_{4k}}{2} + \frac{\sqrt{(\alpha_{1k} - \alpha_{4k})^2 + 4\alpha_{3k}^2}}{2}.
\end{align*}
The following lemma comes in handy for bounding the norm of $\bm{U}_k$.
\begin{customlemma}\label{lemma:Uk}
For any $x>0$ and $\lambda \in (0,1)$, denote $\alpha_1 = e^x - x$, $\alpha_3 = e^x - 1$ and $\alpha_4 = \lambda e^x$. It can then be guaranteed that
\begin{align*}
\frac{\alpha_1 + \alpha_4}{2} + \frac{\sqrt{(\alpha_1 - \alpha_4)^2 + 4\alpha_3^2}}{2} \leq \exp\left(\frac{5}{1-\lambda}x^2\right).
\end{align*}
\end{customlemma}  
\begin{proof} 
See Appendix \ref{app:proof-lemma-Uk}.
\end{proof}

As a direct implication of lemma \ref{lemma:Uk}, the norm of $\bm{U}_k$ is bounded by
\begin{align*}
\|\bm{U}_k\| \leq \exp\left(\frac{5}{1-\lambda} (\frac{4}{\pi}t)^2 M_k^2\right);
\end{align*}
consequently, the norm of $\bm{g}_n^{\parallel}$ is bounded by
\begin{align*}
\|\bm{g}_n^{\parallel}\|_{\mu} &\leq \sqrt{d}\exp\left(\left(\frac{4}{\pi}\right)^2\frac{5t^2}{1-\lambda} \sum_{k=1}^n M_k^2 \right) 
\end{align*}
for any $t \geq 0$ and $\theta \in \left[-\frac{\pi}{2},\frac{\pi}{2}\right]$. Therefore, the left-hand-side of \eqref{eq:markov-matrix-hoeffding} is bounded by
\begin{align*}
\mathbb{P}\left(\left\|\frac{1}{n}\sum_{i=1}^n \bm{F}_i(s_i)\right\| \geq \varepsilon \right) &\leq   2d^{2-\frac{\pi}{4}} \inf_{t \geq 0}  \exp(-nt\varepsilon)\exp\left(\left(\frac{4}{\pi}\right)^2\frac{5t^2}{1-\lambda} \sum_{k=1}^n M_k^2 \right).
\end{align*}

By letting
\begin{align*}
t = \frac{1-\lambda}{10} \left(\frac{\pi}{4}\right)^2 \frac{n}{\sum_{k=1}^n M_k^2} \cdot \varepsilon,
\end{align*}
we obtain
\begin{align*}
\mathbb{P}\left(\left\|\frac{1}{n}\sum_{i=1}^n \bm{F}_i(s_i)\right\| \geq \varepsilon \right) &\leq 2d^{2-\frac{\pi}{4}} \exp\left\{-\frac{1-\lambda}{20}\left(\frac{\pi}{4}\right)^2\frac{n^2\varepsilon^2 }{\sum_{k=1}^n M_k^2} \right\}
\end{align*}
which completes the proof.


% \paragraph{An error in \cite{qiu2020matrix}.} In Claim 4 on page 24 of the paper, the authors attempted to bound the norm of $\|\bm{z}_i^{\perp}\|_{\pi}$; however, we note that the fourth line in the proof of this claim holds true only when $\|\bm{z}_0^{\perp}\|_{\pi} = 0$, which happens to not be the case since the initial distribution is non-stationary. This error seems to require a non-trivial correction, as we have done in the proof of Theorem \ref{thm:matrix-hoeffding} through the refinement of the iterative relationship along the sequence $\{\bm{g}_k\}_{k \in [n]}$.
% \yuting{will it be a problem if we assume $s_1\sim \mu$?}
% \weichen{Given our decision to only discuss $s_1 \sim \mu$ in this theorem and put non-stationary starting distribution to the next corollary, we may want to move this paragraph to the end of next subsection and delete the last sentence; or maybe remove it altogether.}

\subsection{Proof of Corollary \ref{cor:matrix-hoeffding}}\label{app:proof-cor-matrix-hoeffding}
For every $x \in \mathcal{S}$, define functions $f(x)$ and $g(x)$ as 
\begin{align*}
f(x):= \mathbb{P}\left(\left\|\frac{1}{n}\sum_{i=1}^n \bm{F}_i(s_i)\right\| \geq \varepsilon \Bigg| s_1 = x\right), \quad \text{and} \quad g(x) := \frac{\nu(\mathrm{d}x)}{\mu(\mathrm{d}x)}
\end{align*}
respectively. Then by definition, $g(x) \geq 0$ and $f(x) \in [0,1]$ for all $x \in \mathcal{S}$. Therefore, the Holder's inequality guarantees
\begin{align}\label{eq:matrix-hoeffding-holder}
\mathbb{P}_{s_1 \sim \nu}\left(\left\|\frac{1}{n}\sum_{i=1}^n \bm{F}_i(s_i)\right\| \geq \varepsilon \right)&=\int_{\mathcal{S}}f(x) \nu(\mathrm{d}x)\nonumber \\
&= \int_{\mathcal{S}}f(x) g(x) \mu(\mathrm{d}x) \nonumber \\ 
&=\|fg\|_{\mu,1} \leq \|f\|_{\mu,q} \|g\|_{\mu,p} = \|f\|_{\mu,q}\left\|\frac{\mathrm{d}\nu}{\mathrm{d}\mu}\right\|_{\mu,p},
\end{align}
where, since $q > 1$ and $f(x) \in [0,1]$ almost surely, $\|f\|_{\mu,q}$ is bounded by
\begin{align}\label{eq:matrix-hoeffding-qbound}
\|f\|_{\mu,q} &= \left(\int_{\mathcal{S}}f^q(x)\mathrm{d}\mu \right)^{\frac{1}{q}}  
\leq \left(\int_{\mathcal{S}}f(x)\mathrm{d}\mu \right)^{\frac{1}{q}}  
\leq \left[\mathbb{P}_{s_1 \sim \mu}\left(\left\|\frac{1}{n}\sum_{i=1}^n \bm{F}_i(s_i)\right\| \geq \varepsilon \right)\right]^{\frac{1}{q}}.
\end{align}
Corollary \ref{cor:matrix-hoeffding} follows by combining Theorem \ref{thm:matrix-hoeffding}, with \eqref{eq:matrix-hoeffding-holder} and \eqref{eq:matrix-hoeffding-qbound}.






\subsection{Proof of Theorem \ref{thm:Srikant-generalize}}\label{app:Srikant-generalize}
This proof is a correction and improvement of Theorem 1 in \cite{srikant2024rates}, which applies Stein's method and Lindeberg's decomposition.
Using notation from \cite{JMLR2019CLT}, we denote, for every $k \in [n]$,
\begin{align*}
\bm{S}_k = \sum_{j=1}^k \bm{x}_k,  \quad \bm{T}_k = \sum_{j=k}^n \bm{V}_j^{\frac{1}{2}}\bm{z}_j, \quad \text{and} \quad \bm{T}_k' = \bm{T}_k + \bm{\Sigma}^{\frac{1}{2}}\bm{z}'.
\end{align*}
Here, $\{\bm{z}_k\}_{k=1}^n$ and $\bm{z}'$ are $i.i.d.$ standard Gaussian random variables in $\mathbb{R}^d$ and independent of the filtration $\{\mathscr{F}_k\}_{k=0}^n$. Note that since we assume $\bm{P}_1 = n\bm{\Sigma}_n$ almost surely, and that $\{\bm{V}_j\}_{1 \leq j \leq k}$ are all measurable with respect to the filtration $\mathscr{F}_{k-1}$, the matrices 
\begin{align*}
&\bm{P}_k = \bm{P}_1 - \sum_{j=1}^{k-1} \bm{V}_j = n\bm{\Sigma}_n - \sum_{j=1}^{k-1} \bm{V}_j, \quad \text{and} \\ 
&\bm{P}_{k+1} = \bm{P}_1 - \sum_{j=1}^{k} \bm{V}_j = n\bm{\Sigma}_n - \sum_{j=1}^{k} \bm{V}_j
\end{align*}
are both measurable with respect to $\mathscr{F}_{k-1}$. 
%\yuting{a bit hand-wavy.}\weichen{Added a short explanation. Does this work?}
It can then be guaranteed that $\frac{1}{\sqrt{n}}\bm{T}_1 \sim \mathcal{N}(\bm{0},\bm{\Sigma}_n)$ and that 
\begin{align}\label{eq:Srikant-Wasserstein}
d_{\mathsf{W}}\left(\frac{1}{\sqrt{n}}\sum_{k=1}^n \bm{x}_k,\mathcal{N}(\bm{0},\bm{\Sigma}_n)\right) &= \sup_{h \in \mathsf{Lip}_1} \left|\mathbb{E}\left[h\left(\frac{1}{\sqrt{n}}\bm{S}_n\right)\right] - \mathbb{E}\left[h\left(\frac{1}{\sqrt{n}}\bm{T}_1\right)\right]\right| \nonumber \\ 
&= \frac{1}{\sqrt{n}} \sup_{h \in \mathsf{Lip}_1}|\mathbb{E}[h(\bm{S}_n)]-\mathbb{E}[h(\bm{T}_1)]|\nonumber \\ 
&\leq \frac{1}{\sqrt{n}} \sup_{h \in \mathsf{Lip}_1}|\mathbb{E}[h(\bm{S}_n)]-\mathbb{E}[h(\bm{T}_1')]| + \frac{1}{\sqrt{n}} \sup_{h \in \mathsf{Lip}_1}|\mathbb{E}[h(\bm{T}_1)]-\mathbb{E}[h(\bm{T}_1')]|\nonumber \\ 
&\overset{(i)}\leq \frac{1}{\sqrt{n}} \sup_{h \in \mathsf{Lip}_1}|\mathbb{E}[h(\bm{S}_n)]-\mathbb{E}[h(\bm{T}_1')]| + \frac{1}{\sqrt{n}} \sup_{h \in \mathsf{Lip}_1} \mathbb{E}|h(\bm{T}_1) - h(\bm{T}_1')| \nonumber \\ 
&\overset{(ii)}\leq \frac{1}{\sqrt{n}} \sup_{h \in \mathsf{Lip}_1}|\mathbb{E}[h(\bm{S}_n)]-\mathbb{E}[h(\bm{T}_1')]| + \frac{1}{\sqrt{n}} \mathbb{E}\|\bm{\Sigma}^{\frac{1}{2}}\bm{z}\|_2 \nonumber \\ 
&\overset{(iii)}\leq \frac{1}{\sqrt{n}} \sup_{h \in \mathsf{Lip}_1}|\mathbb{E}[h(\bm{S}_n)]-\mathbb{E}[h(\bm{T}_1')]| + \sqrt{\frac{\mathsf{Tr}(\bm{\Sigma})}{n}},
\end{align}
where we applied the Jensen's inequality in (i)(iii) and Lipchitz property in (ii). 
Invoking the Lindeberg's decomposition, the triangle inequality yields
\begin{align}\label{eq:Srikant-Lindeberg}
|\mathbb{E}[h(\bm{S}_n)]-\mathbb{E}[h(\bm{T}_1')]| &\leq \sum_{k=1}^n |\mathbb{E}[h(\bm{S}_k+\bm{T}_{k+1}')]-\mathbb{E}[h(\bm{S}_{k-1}+\bm{T}_k')]|,
\end{align}
where we define $\bm{S}_0=\bm{T}_{n+1}=0$ for consistency. For each $k \in [n]$, define
\begin{align*}
\tilde{h}_k(\bm{x}):= h(\bm{P}_k'^{\frac{1}{2}}\bm{x}+\bm{S}_{k-1}),
\end{align*}
where we denote, for simplicity, $$\bm{P}_k' = \bm{P}_k + \bm{\Sigma}.$$ 
Since $h \in \mathsf{Lip}_1$, $\tilde{h}_k$ has Lipchitz coefficient $L_k = \|\bm{P}_k'^{\frac{1}{2}}\|$. Therefore, Lemma 1 in \cite{srikant2024rates} guarantees that there exists a function $f_k: \mathbb{R}^d \to \mathbb{R}$ such that
\begin{align*}
\tilde{h}_k(\bm{x})-\mathbb{E}[\tilde{h}_k(\bm{z})] = \Delta f_k(\bm{x}) - \bm{x}^\top \nabla f_k(\bm{x})
\end{align*}
holds for any $\bm{x} \in \mathbb{R}^d$, where $\bm{z}$ is the $d$-dimensional standard Gaussian random variable. Therefore, it can be guaranteed that
\begin{align*}
&h(\bm{S}_k + \bm{T}_{k+1}') -\mathbb{E}[h(\bm{S}_{k-1} + \bm{T}_k')\mid \mathscr{F}_{k-1}]\\ 
&= h(\bm{S}_k + \bm{T}_{k+1}') -\mathbb{E}[h(\bm{S}_{k-1} + \bm{P}_k'^{\frac{1}{2}}\bm{z})\mid \mathscr{F}_{k-1}] \\ 
&= \tilde{h}_k(\bm{P}_k'^{-\frac{1}{2}}(\bm{x}_k + \bm{T}_{k+1}'))-\mathbb{E}[\tilde{h}_k(\bm{z})] \\ 
&= \Delta f_k(\bm{P}_k'^{-\frac{1}{2}}(\bm{x}_k + \bm{T}_{k+1})) - \left[\bm{P}_k'^{-\frac{1}{2}}(\bm{x}_k + \bm{T}_{k+1}')\right]^\top \nabla f_k(\bm{P}_k'^{-\frac{1}{2}}(\bm{x}_k + \bm{T}_{k+1}')).
\end{align*}
Notice that by definition, $\bm{P}_k'^{-\frac{1}{2}}\bm{T}_{k+1} \mid \mathscr{F}_{k-1} \sim \mathcal{N}(\bm{0},\bm{P}_k'^{-\frac{1}{2}}\bm{P}_{k+1}'\bm{P}_k'^{-\frac{1}{2}})$. Therefore, we can denote $\tilde{\bm{z}}_k$ such that $\tilde{\bm{z}}_k \mid \mathscr{F}_{k-1} \sim \mathcal{N}(\bm{0},\bm{P}_k'^{-\frac{1}{2}}\bm{P}_{k+1}'\bm{P}_k'^{-\frac{1}{2}})$ and that $\tilde{\bm{z}}_k$ is independent of $\bm{x}_k$ when conditioned on $\mathscr{F}_{k-1}$. By taking conditional expectations, we obtain
\begin{align*}
&\mathbb{E}\left[h(\bm{S}_k + \bm{T}_{k+1}')\mid \mathscr{F}_{k-1}\right] - \mathbb{E}[h(\bm{S}_{k-1} + \bm{T}_k')\mid \mathscr{F}_{k-1}]\\ 
&=\mathbb{E}[\Delta f_k(\bm{P}_k'^{-\frac{1}{2}}\bm{x}_k + \tilde{\bm{z}}_k)\mid \mathscr{F}_{k-1}]- \mathbb{E}\left[(\bm{P}_k'^{-\frac{1}{2}}\bm{x}_k + \tilde{\bm{z}}_k)^\top \nabla f_k(\bm{P}_k'^{-\frac{1}{2}}\bm{x}_k + \tilde{\bm{z}}_k)\mid \mathscr{F}_{k-1}\right].
\end{align*}
Since by definition, $\Delta f_k = \mathsf{Tr}(\nabla^2 f_k)$, this can be further decomposed into
\begin{align}\label{eq:srikant-decompose}
&\mathbb{E}\left[h(\bm{S}_k + \bm{T}_{k+1}')\mid \mathscr{F}_{k-1}\right] - \mathbb{E}[h(\bm{S}_{k-1} + \bm{T}_k')\mid \mathscr{F}_{k-1}]\nonumber \\ 
&= \underset{I_1}{\underbrace{\mathbb{E}\left[\mathsf{Tr}\left(\bm{P}_k'^{-\frac{1}{2}}\bm{P}_{k+1}'\bm{P}_k'^{-\frac{1}{2}}\nabla^2 f_k(\bm{P}_k'^{-\frac{1}{2}}\bm{x}_k + \tilde{\bm{z}}_k)\right)-\tilde{\bm{z}}_k^\top \nabla f_k(\bm{P}_k'^{-\frac{1}{2}}\bm{x}_k + \tilde{\bm{z}}_k) \bigg| \mathscr{F}_{k-1}\right]}} \nonumber \\ 
&+ \underset{I_2}{\underbrace{\mathbb{E}\left[\mathsf{Tr}\left(\left(\bm{I}-\bm{P}_k'^{-\frac{1}{2}}\bm{P}_{k+1}'\bm{P}_k'^{-\frac{1}{2}}\right)\nabla^2 f_k(\bm{P}_k'^{-\frac{1}{2}}\bm{x}_k + \tilde{\bm{z}}_k)\right)\bigg|\mathscr{F}_{k-1}\right]}} \nonumber \\ 
&- \underset{I_3}{\underbrace{\mathbb{E}\left[(\bm{P}_k'^{-\frac{1}{2}}\bm{x}_k)^\top \nabla f_k(\tilde{\bm{z}}_k) \bigg|\mathscr{F}_{k-1}\right]}} - \underset{I_4}{\underbrace{\mathbb{E}\left[(\bm{P}_k'^{-\frac{1}{2}}\bm{x}_k)^\top \left(\nabla f_k(\bm{P}_k'^{-\frac{1}{2}}\bm{x}_k + \tilde{\bm{z}}_k)- \nabla f_k(\tilde{\bm{z}}_k) \right)\bigg|\mathscr{F}_{k-1}\right]}}.
\end{align}
Of the four terms on the right-hand-side, $I_1$ is equal to $\bm{0}$ according to Lemma 2 in \cite{srikant2024rates}; $I_3$ is also $\bm{0}$ according to martingale property. 

\paragraph{Control the term $I_2$.} We first make the observation that
\begin{align*}
\bm{I}-\bm{P}_k'^{-\frac{1}{2}}\bm{P}_{k+1}'\bm{P}_k'^{-\frac{1}{2}}=\bm{P}_k'^{-\frac{1}{2}}(\bm{P}_k'-\bm{P}_{k+1}')\bm{P}_k'^{-\frac{1}{2}}=\bm{P}_k'^{-\frac{1}{2}}\bm{V}_k\bm{P}_k'^{-\frac{1}{2}};
\end{align*}
Therefore, the term $I_2$ can be represented as
\begin{align}\label{eq:Srikant-I2}
&\mathbb{E}\left[\mathsf{Tr}\left(\left(\bm{I}-\bm{P}_k'^{-\frac{1}{2}}\bm{P}_{k+1}'\bm{P}_k'^{-\frac{1}{2}}\right)\nabla^2 f_k(\bm{P}_k'^{-\frac{1}{2}}\bm{x}_k + \tilde{\bm{z}}_k)\right)\bigg|\mathscr{F}_{k-1}\right] \nonumber\\ 
&= \mathbb{E}\left[\mathsf{Tr}\left(\bm{P}_k'^{-\frac{1}{2}}\bm{V}_k\bm{P}_k'^{-\frac{1}{2}} \nabla^2 f_k(\bm{P}_k'^{-\frac{1}{2}}\bm{x}_k + \tilde{\bm{z}}_k)\right)\bigg|\mathscr{F}_{k-1}\right]\nonumber\\ 
&\overset{(i)}{=}\mathsf{Tr}\left\{\bm{P}_k'^{-\frac{1}{2}}\bm{V}_k\bm{P}_k'^{-\frac{1}{2}}\mathbb{E}\left[\nabla^2 f_k(\bm{P}_k'^{-\frac{1}{2}}\bm{x}_k + \tilde{\bm{z}}_k)\bigg|\mathscr{F}_{k-1}\right]\right\} \nonumber\\ 
&= \mathsf{Tr}\left\{\bm{P}_k'^{-\frac{1}{2}}\bm{V}_k\bm{P}_k'^{-\frac{1}{2}}\mathbb{E}\left[\nabla^2 f_k(\tilde{\bm{z}}_k)\bigg|\mathscr{F}_{k-1}\right]\right\}\nonumber \\ 
&+ \mathsf{Tr}\left\{\bm{P}_k'^{-\frac{1}{2}}\bm{V}_k\bm{P}_k'^{-\frac{1}{2}}\mathbb{E}\left[\nabla^2 f_k(\bm{P}_k'^{-\frac{1}{2}}\bm{x}_k + \tilde{\bm{z}}_k)-\nabla^2 f_k(\tilde{\bm{z}}_k)\bigg|\mathscr{F}_{k-1}\right]\right\},
\end{align}
where we invoked the linearity of trace and expectation in (i). 

% \paragraph{Control the term $I_4$.} We firstly represent the difference $\nabla f_k(\bm{P}_k^{-\frac{1}{2}}\bm{x}_k + \tilde{\bm{z}}_k)- \nabla f_k(\tilde{\bm{z}}_k)$ as an integral:
% \begin{align*}
% &\nabla f_k(\bm{P}_k^{-\frac{1}{2}}\bm{x}_k + \tilde{\bm{z}}_k)- \nabla f_k(\tilde{\bm{z}}_k) \\ 
% &= \left[\int_{t=0}^1 \nabla^2 f_k(t\bm{P}_k^{-\frac{1}{2}}\bm{x}_k + \tilde{\bm{z}}_k) \mathrm{d}t\right] \cdot \bm{P}_k^{-\frac{1}{2}}\bm{x}_k \\ 
% &= \nabla^2 f_k(\tilde{z}_k) \cdot \bm{P}_k^{-\frac{1}{2}}\bm{x}_k + \left\{\int_{t=0}^1 [\nabla^2 f_k(t\bm{P}_k^{-\frac{1}{2}}\bm{x}_k + \tilde{\bm{z}}_k)- \nabla^2 f_k(\tilde{z}_k)]\mathrm{d}t\right\} \cdot \bm{P}_k^{-\frac{1}{2}}\bm{x}_k
% \end{align*}

\paragraph{Controlling $I_4$.} For clarity, we define a uni-dimensional function
\begin{align*}
F(t) := (\bm{P}_k'^{-\frac{1}{2}}\bm{x}_k)^\top \nabla f_k(t\bm{P}_k'^{-\frac{1}{2}}\bm{x}_k + \tilde{\bm{z}}_k), \quad \forall t \in [0,1].
\end{align*}
Since $f_k$ is twice-differentiable, the derivative of $F(t)$ is featured by
\begin{align*}
F'(t) = (\bm{P}_k'^{-\frac{1}{2}}\bm{x}_k)^\top \nabla^2 f_k(t\bm{P}_k'^{-\frac{1}{2}}\bm{x}_k + \tilde{\bm{z}}_k) (\bm{P}_k'^{-\frac{1}{2}}\bm{x}_k).
\end{align*}
Consequently, the Lagrange's mid-value theorem guarantees the existence of $\Theta  \in [0,1]$, such that
\begin{align*}
(\bm{P}_k'^{-\frac{1}{2}}\bm{x}_k)^\top \left(\nabla f_k(\bm{P}_k'^{-\frac{1}{2}}\bm{x}_k + \tilde{\bm{z}}_k)- \nabla f_k(\tilde{\bm{z}}_k) \right) 
&= F(1) - F(0) \\ 
&= F'(\Theta) = (\bm{P}_k'^{-\frac{1}{2}}\bm{x}_k)^\top \nabla^2 f_k(\Theta\bm{P}_k'^{-\frac{1}{2}}\bm{x}_k + \tilde{\bm{z}}_k) (\bm{P}_k'^{-\frac{1}{2}}\bm{x}_k).
\end{align*}
% \yuting{write as integral form; there is no mid-value theorem in multivariate case} \weichen{Please check the updated version.} 
Consequently, the term $I_4$ is characterized by
\begin{align}\label{eq:Srikant-I4}
&\mathbb{E}\left[(\bm{P}_k'^{-\frac{1}{2}}\bm{x}_k)^\top \left(\nabla f_k(\bm{P}_k'^{-\frac{1}{2}}\bm{x}_k + \tilde{\bm{z}}_k)- \nabla f_k(\tilde{\bm{z}}_k) \right)\bigg|\mathscr{F}_{k-1}\right]\nonumber \\ 
&=\mathbb{E}\left[(\bm{P}_k'^{-\frac{1}{2}}\bm{x}_k)^\top \nabla^2f_k(\Theta \bm{P}_k'^{-\frac{1}{2}}\bm{x}_k + \tilde{\bm{z}}_k)(\bm{P}_k'^{-\frac{1}{2}}\bm{x}_k) \bigg|\mathscr{F}_{k-1}\right]\nonumber \\
&= \mathbb{E}\left[(\bm{P}_k'^{-\frac{1}{2}}\bm{x}_k)^\top \nabla^2f_k(\tilde{\bm{z}}_k)(\bm{P}_k'^{-\frac{1}{2}}\bm{x}_k) \bigg|\mathscr{F}_{k-1}\right] \nonumber \\ 
&+ \mathbb{E}\left[(\bm{P}_k'^{-\frac{1}{2}}\bm{x}_k)^\top (\nabla^2f_k(\Theta \bm{P}_k'^{-\frac{1}{2}}\bm{x}_k + \tilde{\bm{z}}_k) - \nabla^2 f_k(\tilde{\bm{z}}_k))(\bm{P}_k'^{-\frac{1}{2}}\bm{x}_k) \bigg|\mathscr{F}_{k-1}\right],
\end{align}
where $\Theta \in (0,1)$. Here, the first term on the right-most part of the equation can be further computed by
\begin{align}\label{eq:Srikant-I41}
&\mathbb{E}\left[(\bm{P}_k'^{-\frac{1}{2}}\bm{x}_k)^\top \nabla^2f_k(\tilde{\bm{z}}_k)(\bm{P}_k'^{-\frac{1}{2}}\bm{x}_k) \bigg|\mathscr{F}_{k-1}\right] = \mathbb{E}\left[\mathsf{Tr}\left((\bm{P}_k'^{-\frac{1}{2}}\bm{x}_k)^\top \nabla^2f_k(\tilde{\bm{z}}_k)(\bm{P}_k'^{-\frac{1}{2}}\bm{x}_k)\right)\bigg|\mathscr{F}_{k-1}\right]\nonumber \\ 
&\overset{(i)}{=} \mathbb{E}\left[\mathsf{Tr}\left((\bm{P}_k'^{-\frac{1}{2}}\bm{x}_k)(\bm{P}_k'^{-\frac{1}{2}}\bm{x}_k)^\top \nabla^2f_k(\tilde{\bm{z}}_k)\right)\bigg|\mathscr{F}_{k-1}\right]\nonumber\\
&\overset{(ii)}{=} \mathsf{Tr}\left\{\mathbb{E}\left[\bm{P}_k'^{-\frac{1}{2}}\bm{x}_k\bm{x}_k^\top \bm{P}_k'^{-\frac{1}{2}}\nabla^2f_k(\tilde{\bm{z}}_k)\bigg|\mathscr{F}_{k-1}\right]\right\}\nonumber \\ 
&\overset{(iii)}{=}\mathsf{Tr}\left\{\mathbb{E}[\bm{P}_k'^{-\frac{1}{2}}\bm{x}_k\bm{x}_k^\top \bm{P}_k'^{-\frac{1}{2}} \mid \mathscr{F}_{k-1}] \mathbb{E}[\nabla^2f_k(\tilde{\bm{z}}_k)\bigg|\mathscr{F}_{k-1}]\right\}\nonumber \\
&\overset{(iv)}{=}\mathsf{Tr}\left\{\bm{P}_k'^{-\frac{1}{2}}\bm{V}_k \bm{P}_k'^{-\frac{1}{2}}\mathbb{E}[\nabla^2f_k(\tilde{\bm{z}}_k)\bigg|\mathscr{F}_{k-1}]\right\};
\end{align}
notice that we applied the basic property of matrix trace in (i), the linearity of expectation in (ii), the conditional independence between $\bm{x}_k$ and $\tilde{\bm{z}}_k$ in (iii), and the definition of $\bm{V}_k$ in (iv). The right-most part of this equation is exactly the same as the first term on the right-most part of \eqref{eq:Srikant-I2}. 


\medskip
Consequently, putting relations \eqref{eq:srikant-decompose}, \eqref{eq:Srikant-I2}, \eqref{eq:Srikant-I4} and \eqref{eq:Srikant-I41}, we obtain
\begin{align}\label{eq:Srikant-reorganize}
&\mathbb{E}\left[h(\bm{S}_k + \bm{T}_{k+1}')\mid \mathscr{F}_{k-1}\right] - \mathbb{E}[h(\bm{S}_{k-1} + \bm{T}_k')\mid \mathscr{F}_{k-1}]\nonumber \\ 
&=\mathsf{Tr}\left\{\bm{P}_k'^{-\frac{1}{2}}\bm{V}_k\bm{P}_k'^{-\frac{1}{2}}\mathbb{E}\left[\nabla^2 f_k(\bm{P}_k^{-\frac{1}{2}}\bm{x}_k + \tilde{\bm{z}}_k)-\nabla^2 f_k(\tilde{\bm{z}}_k)\bigg|\mathscr{F}_{k-1}\right]\right\} \nonumber \\ 
&- \mathbb{E}\left[(\bm{P}_k'^{-\frac{1}{2}}\bm{x}_k)^\top (\nabla^2f_k(\Theta \bm{P}_k'^{-\frac{1}{2}}\bm{x}_k + \tilde{\bm{z}}_k) - \nabla^2 f_k(\tilde{\bm{z}}_k))(\bm{P}_k'^{-\frac{1}{2}}\bm{x}_k) \bigg|\mathscr{F}_{k-1}\right]
\end{align}
Both terms on the right-hand-side can be bounded by the Holder's property of $\nabla^2 f_k$. Specifically, we have the following proposition:
\begin{customproposition}\label{prop:Stein-smooth}
Let $h: \mathbb{R}^d \to \mathbb{R} \in \mathsf{Lip}_1$, $\bm{\mu} \in \mathbb{R}^d$ and $\bm{\Sigma} \succ \bm{0} \in \mathbb{S}^{d \times d}$. Further define $g: \mathbb{R}^d \to \mathbb{R}$ as
\begin{align*}
g(\bm{x})= h(\bm{\Sigma}^{\frac{1}{2}}\bm{x} +\bm{\mu}), \quad \forall x \in \mathbb{R}^d,
\end{align*}
and use $f_g$ to denote the solution to Stein's equation
\begin{align*}
\Delta f(\bm{x}) - \bm{x}^\top \nabla f(\bm{x}) = g(\bm{x}) - \mathbb{E}[g(\bm{z})]
\end{align*}
where $\bm{z}$ is the $d$-dimensional standard Gaussian distribution. It can then be guaranteed that
\begin{align}
\left\|\nabla^2 f_g(\bm{x}) - \nabla^2 f_g(\bm{y})\right\| \lesssim (2+\log (d\|\bm{\Sigma}\|))^+ \cdot \|\bm{\Sigma}^{\frac{1}{2}}(\bm{x} - \bm{y})\|_2 + e^{-1}.
\end{align}
\end{customproposition}
\begin{proof} 
See Appendix \ref{app:proof-Stein-smooth}. 
\end{proof}

Proposition \ref{prop:Stein-smooth} guarantees for every $\Theta \in [0,1]$ that
\begin{align}
%\label{eqn:brahms}
\notag \left\|\nabla^2f_k(\Theta \bm{P}_k'^{-\frac{1}{2}}\bm{x}_k + \tilde{\bm{z}}_k) - \nabla^2 f_k(\tilde{\bm{z}}_k)\right\| &\leq (2+\log(d\|\bm{P}_k'\|))^+ \|\bm{x}_k\|_2 + e^{-1}\nonumber \\ 
&\leq (2+\log(d\|(n\bm{\Sigma}_n + \bm{\Sigma})\|))^+ \|\bm{x}_k\|_2 + e^{-1}
\end{align}
% \yuting{not sure I understand this relation; isn't $\Sigma$ chosen arbitrary?} \weichen{checked.}
% Here, by taking $\beta = 1 - \frac{2}{\log n}$, we obtain
% \begin{align}
% \label{eqn:mozart}
% \left\|\nabla^2f_k(\Theta \bm{P}_k'^{-\frac{1}{2}}\bm{x}_k + \tilde{\bm{z}}_k) - \nabla^2 f_k(\tilde{\bm{z}}_k)\right\| &\leq (\sqrt{d} + \log n) \|\bm{x}_k\|_2.
% \end{align}
% \yuting{what about the dependence on $\Sigma_n$?} \weichen{checked.}

Consequently, the first term on the right hand side of \eqref{eq:Srikant-reorganize} can be bounded by
\begin{align*}
&\left|\mathsf{Tr}\left\{\bm{P}_k'^{-\frac{1}{2}}\bm{V}_k\bm{P}_k'^{-\frac{1}{2}}\mathbb{E}\left[\nabla^2 f_k(\bm{P}_k^{-\frac{1}{2}}\bm{x}_k + \tilde{\bm{z}}_k)-\nabla^2 f_k(\tilde{\bm{z}}_k)\bigg|\mathscr{F}_{k-1}\right]\right\}\right| \\ 
&\leq (2+\log(d\|(n\bm{\Sigma}_n + \bm{\Sigma})\|))^+\mathbb{E}\left[\|\bm{P}_k'^{-\frac{1}{2}}\bm{x}_k\|_2^2 \bigg|\mathscr{F}_{k-1}\right] \mathbb{E}[\|\bm{x}_k\|_2 |\mathscr{F}_{k-1}] + e^{-1} \mathbb{E}\left[\|\bm{P}_k'^{-\frac{1}{2}}\bm{x}_k\|_2^2 \bigg|\mathscr{F}_{k-1}\right] \\ 
\end{align*}
Here, we further notice that since $\bm{P}_k'$ is measurable with respect to $\mathscr{F}_{k-1}$, and that $\|\bm{P}_k'^{-\frac{1}{2}}\bm{x}_k\|_2^2$ and $\|\bm{x}_k\|_2$ are positively correlated \footnote{We refer readers to Appendix \ref{app:proof-correlated-norms} for the details of this claim.}, 
\begin{align}\label{eq:correlated-norms}
\mathbb{E}\left[\|\bm{P}_k'^{-\frac{1}{2}}\bm{x}_k\|_2^2 \bigg|\mathscr{F}_{k-1}\right] \mathbb{E}[\|\bm{x}_k\|_2 |\mathscr{F}_{k-1}] \leq \mathbb{E}\left[\|\bm{P}_k'^{-\frac{1}{2}}\bm{x}_k\|_2^2 \|\bm{x}_k\|_2 \bigg|\mathscr{F}_{k-1}\right].
\end{align}
Therefore, the upper bound can be further simplified by
\begin{align*}
&\left|\mathsf{Tr}\left\{\bm{P}_k'^{-\frac{1}{2}}\bm{V}_k\bm{P}_k'^{-\frac{1}{2}}\mathbb{E}\left[\nabla^2 f_k(\bm{P}_k^{-\frac{1}{2}}\bm{x}_k + \tilde{\bm{z}}_k)-\nabla^2 f_k(\tilde{\bm{z}}_k)\bigg|\mathscr{F}_{k-1}\right]\right\}\right| \\ 
&\leq (2+\log(d\|(n\bm{\Sigma}_n + \bm{\Sigma})\|))^+ \mathbb{E}\left[\|\bm{P}_k'^{-\frac{1}{2}}\bm{x}_k\|_2^2 \|\bm{x}_k\|_2 \bigg|\mathscr{F}_{k-1}\right]+ \mathbb{E}\left[\|\bm{P}_k'^{-\frac{1}{2}}\bm{x}_k\|_2^2 \bigg|\mathscr{F}_{k-1}\right].
\end{align*}
Meanwhile, the second term can also be bounded by
\begin{align*}
&\left|\mathbb{E}\left[(\bm{P}_k^{-\frac{1}{2}}\bm{x}_k)^\top (\nabla^2f_k(\Theta \bm{P}_k^{-\frac{1}{2}}\bm{x}_k + \tilde{\bm{z}}_k) - \nabla^2 f_k(\tilde{\bm{z}}_k))(\bm{P}_k^{-\frac{1}{2}}\bm{x}_k) \bigg|\mathscr{F}_{k-1}\right]\right| \\ 
&\leq (2+\log(d\|(n\bm{\Sigma}_n + \bm{\Sigma})\|))^+ \mathbb{E}\left[\|\bm{P}_k'^{-\frac{1}{2}}\bm{x}_k\|_2^2 \|\bm{x}_k\|_2 \bigg|\mathscr{F}_{k-1}\right]+ \mathbb{E}\left[\|\bm{P}_k'^{-\frac{1}{2}}\bm{x}_k\|_2^2 \bigg|\mathscr{F}_{k-1}\right].
\end{align*}
Since these two bounds are equivalent, in combination, the triangle inequality guarantees
\begin{align*}
&\Big|\mathbb{E}\left[h(\bm{S}_k + \bm{T}_{k+1})\mid \mathscr{F}_{k-1}\right] - \mathbb{E}[h(\bm{S}_{k-1} + \bm{T}_k)\mid \mathscr{F}_{k-1}]\Big|\\ 
& \lesssim (2+\log(d\|(n\bm{\Sigma}_n + \bm{\Sigma})\|))^+ \mathbb{E}\left[\|\bm{P}_k'^{-\frac{1}{2}}\bm{x}_k\|_2^2 \|\bm{x}_k\|_2 \bigg|\mathscr{F}_{k-1}\right]+ \mathbb{E}\left[\|\bm{P}_k'^{-\frac{1}{2}}\bm{x}_k\|_2^2 \bigg|\mathscr{F}_{k-1}\right].
\end{align*}
Plugging into \eqref{eq:Srikant-Lindeberg}, we obtain
\begin{align}\label{eq:Srikant-Lindeberg-2}
|\mathbb{E}[h(\bm{S}_n)]-\mathbb{E}[h(\bm{T}_1')]| &\leq \sum_{k=1}^n |\mathbb{E}[h(\bm{S}_k+\bm{T}_{k+1}')]-\mathbb{E}[h(\bm{S}_{k-1}+\bm{T}_k')]| \nonumber \\ 
&\lesssim (2+\log(d\|(n\bm{\Sigma}_n + \bm{\Sigma})\|))^+ \sum_{k=1}^n \mathbb{E}\left[\mathbb{E}\left[\|(\bm{P}_k + \bm{\Sigma})^{-\frac{1}{2}}\bm{x}_k\|_2^2 \|\bm{x}_k\|_2 \bigg|\mathscr{F}_{k-1}\right]\right] \nonumber \\ 
&+ \sum_{k=1}^n \mathbb{E}\left[\mathbb{E}\left[\|(\bm{P}_k + \bm{\Sigma})^{-\frac{1}{2}}\bm{x}_k\|_2^2 \bigg|\mathscr{F}_{k-1}\right]\right].
\end{align}
We now aim to proof the last summation on the right-hand-side of \eqref{eq:Srikant-Lindeberg-2}. The law of total expectation directly implies, for every $k \in [n]$, that
\begin{align*}
\mathbb{E}\left[\mathbb{E}\left[\|(\bm{P}_k + \bm{\Sigma})^{-\frac{1}{2}}\bm{x}_k\|_2^2 \bigg|\mathscr{F}_{k-1}\right]\right] = \mathbb{E}\left[\mathsf{Tr}(\bm{P}_k'^{-1}\bm{V}_k)\mid \mathscr{F}_0\right].
\end{align*}
To further feature the summation, we invoke a telescoping technique. By taking $\bm{A} = \bm{P}_k'$ and $\bm{B} = \bm{P}_{k+1}'$ in Theorem \ref{thm:Klein}, the summand can be bounded by
\begin{align*}
\mathsf{Tr}(\bm{P}_k'^{-1}\bm{V}_k)&= \mathsf{Tr}(\bm{P}_k'^{-1}(\bm{P}_k' - \bm{P}_{k+1}')) \leq \mathsf{Tr}(\log (\bm{P}_k')) - \mathsf{Tr}(\log(\bm{P}_{k+1}')).
\end{align*}
Summing from $k=1$ through $k=n$, we obtain
\begin{align}\label{eq:Srikant-telescope}
\sum_{k=1}^n \mathbb{E}\left[\mathbb{E}\left[\|(\bm{P}_k + \bm{\Sigma})^{-\frac{1}{2}}\bm{x}_k\|_2^2 \bigg|\mathscr{F}_{k-1}\right]\right] 
&=\sum_{k=1}^n \mathbb{E}\left[\mathsf{Tr}(\bm{P}_k'^{-1}\bm{V}_k)\mid\mathscr{F}_0\right] \nonumber \\
&\leq \sum_{k=1}^n \mathbb{E}\left[\mathsf{Tr}(\log (\bm{P}_k')) - \mathsf{Tr}(\log(\bm{P}_{k+1}')) \big|\mathscr{F}_0\right] \nonumber \\ 
&= \mathbb{E}[\mathsf{Tr}(\log \bm{P}_1')]-\mathbb{E}[\mathsf{Tr}(\log \bm{P}_n')]\nonumber \\
&= \mathsf{Tr}(\log(n\bm{\Sigma}_n+\bm{\Sigma})) - \log(\bm{\Sigma})).
\end{align}
Our target result follows by combining \eqref{eq:Srikant-Wasserstein}, \eqref{eq:Srikant-Lindeberg-2} and \eqref{eq:Srikant-telescope}. 


\subsection{Proof of Corollary \ref{cor:Wu}}\label{app:proof-cor-Wu}
In order to simplify the upper bound in Theorem \ref{thm:Srikant-generalize}, we firstly take $\bm{\Sigma} = \bm{\Sigma}_n$, yielding
\begin{align*}
d_{\mathsf{W}}\left(\frac{1}{\sqrt{n}}\sum_{k=1}^n \bm{x}_k,\mathcal{N}(\bm{0},\bm{\Sigma}_n)\right)  
& \lesssim  \frac{(2+\log(dn\|\bm{\Sigma}_n\|))^+}{\sqrt{n}} \sum_{k=1}^n \mathbb{E}\left[\mathbb{E}\left[\|(\bm{P}_k + \bm{\Sigma}_n)^{-\frac{1}{2}}\bm{x}_k\|_2^2 \|\bm{x}_k\|_2 \bigg|\mathscr{F}_{k-1}\right]\right] \\ 
&+ \frac{1}{\sqrt{n}}\left[\mathsf{Tr}(\log((n+1)\bm{\Sigma}_n)) - \log(\bm{\Sigma_n}))\right]+ \sqrt{\frac{\mathsf{Tr}(\bm{\Sigma_n})}{n}}.
\end{align*}
For the first term on the second line, we observe
\begin{align*}
\mathsf{Tr}(\log((n+1)\bm{\Sigma}_n)) - \log(\bm{\Sigma_n}))&= \sum_{i=1}^d \log (\lambda_i((n+1)\bm{\Sigma}_n)) - \sum_{i=1}^d \log (\lambda_i(\bm{\Sigma}_n))\\ 
&= \sum_{i=1}^n (\log (n+1) + \log \lambda_i(\bm{\Sigma}_n)) - \log \lambda_i(\bm{\Sigma}_n) = d \log(n+1).
\end{align*}
Meanwhile, the condition \eqref{eq:3rd-momentum-condition} can be invoked on the first term to obtain
\begin{align*}
\sum_{k=1}^n \mathbb{E}\left[\mathbb{E}\left[\|(\bm{P}_k + \bm{\Sigma}_n)^{-\frac{1}{2}}\bm{x}_k\|_2^2 \|\bm{x}_k\|_2 \bigg|\mathscr{F}_{k-1}\right]\right] &\leq M \cdot \sum_{k=1}^n\mathbb{E}\left[\mathbb{E}\left[\|(\bm{P}_k + \bm{\Sigma}_n)^{-\frac{1}{2}}\bm{x}_k\|_2^2 \bigg|\mathscr{F}_{k-1}\right]\right] \\ 
&= M \cdot \left[\mathsf{Tr}(\log((n+1)\bm{\Sigma}_n)) - \log(\bm{\Sigma_n}))\right]\\ 
& = Md\log(n+1),
\end{align*}
% In order to tackle the summation in the upper bound of Theorem \ref{thm:Srikant-generalize}, we firstly apply the assumption \eqref{eq:3rd-momentum-condition} to obtain
% \begin{align}\label{eq:Wu-Wasserstein}
% d_{\mathsf{W}}\left(\frac{1}{\sqrt{n}}\sum_{k=1}^n \bm{x}_k,\mathcal{N}(\bm{0},\bm{\Sigma}_n)\right)  
% & \lesssim  \frac{(2+\log(d\|(n\bm{\Sigma}_n + \bm{\Sigma})\|))^+}{\sqrt{n}} \sum_{k=1}^n \mathbb{E}\left[\mathbb{E}\left[\|(\bm{P}_k + \bm{\Sigma})^{-\frac{1}{2}}\bm{x}_k\|_2^2 \|\bm{x}_k\|_2 \bigg|\mathscr{F}_{k-1}\right]\right] \nonumber \\ 
% &+ \frac{1}{\sqrt{n}}\left[\mathsf{Tr}(\log(n\bm{\Sigma}_n+\bm{\Sigma})) - \mathsf{Tr}(\bm{\Sigma}))\right]+ \sqrt{\frac{\mathsf{Tr}(\bm{\Sigma})}{n}} \nonumber \\ 
% &\lesssim  \frac{M}{\sqrt{n}} (2+\log(d\|(n\bm{\Sigma}_n + \bm{\Sigma})\|))^+\sum_{k=1}^n \mathbb{E}\left[\mathbb{E}\left[\|(\bm{P}_k + \bm{\Sigma})^{-\frac{1}{2}}\bm{x}_k\|_2^2 \bigg|\mathscr{F}_{k-1}\right]\right]  \nonumber \\ 
% &+ \frac{1}{\sqrt{n}}\left[\mathsf{Tr}(\log(n\bm{\Sigma}_n+\bm{\Sigma})) - \mathsf{Tr}(\bm{\Sigma}))\right]+ \sqrt{\frac{\mathsf{Tr}(\bm{\Sigma})}{n}}}.
% \end{align}
where the third line follows from \eqref{eq:Srikant-telescope}. In combination, the Wasserstein distance can be bounded by
\begin{align*}
d_{\mathsf{W}}\left(\frac{1}{\sqrt{n}}\sum_{k=1}^n \bm{x}_k,\mathcal{N}(\bm{0},\bm{\Sigma}_n)\right) &\lesssim \frac{M(2+\log(dn\|\bm{\Sigma}_n\|))^+ + 1}{\sqrt{n}}\cdot d\log n+ \sqrt{\frac{\mathsf{Tr}(\bm{\Sigma_n})}{n}}.
\end{align*}


\paragraph{Comparison with the corresponding uni-dimensional Corollary 2.3 in \cite{rollin2018}.} When $d=1$, the condition \eqref{eq:3rd-momentum-condition} reduces to
\begin{align*}
\mathbb{E}|x_k|^3\mid \mathcal{F}_{k-1} \leq M \mathbb{E}[x_k^2]\mid \mathcal{F}_{k-1}, \quad \forall k \in [n].
\end{align*}
Notice that this is a weaker assumption than the one outlined in Equation (2.13) of \cite{rollin2018}, which further assumes that the conditional third momentum of $x_k$ is uniformly bounded. In our proof of Corollary \ref{cor:Wu}, we invoke a telescoping method that not only simplifies the proof but also yields a tighter upper bound.

\paragraph{Comments on Corollary 3 in \cite{JMLR2019CLT}.} In a similar attempt to generalize this uni-dimensional corollary from \cite{rollin2018} to multi-dimensional settings, Corollary 3 of \cite{JMLR2019CLT} made the assumption that 
\begin{align}\label{eq:JMLR-Cor-assumption}
\mathbb{E}[\|\bm{x}_k\|_2^3 \mid \mathcal{F}_{k-1}] \leq \beta \vee \delta \mathsf{Tr}(\bm{V}_k), \quad \text{a.s.}
\end{align}
and applied the technique used by \cite{rollin2018}, which involved defining a sequence of stopping times $\{\tau_k\}_{k \in [n]}$. However, this generalization is non-trivial since the positive semi-definite order of symmetric matrices is \emph{incomplete}; in other words, when $\bm{A},\bm{B} \in \mathbb{S}^{d \times d}$, $\bm{A} \npreceq \bm{B}$ does not imply $\bm{A} \succ \bm{B}$. Consequently, the derivation from Equation (A.36) to (A.37) in the proof of Corollary 3 of \cite{JMLR2019CLT} is invalid: apparently, the authors' reasoning was, under their notation:
\begin{align*}
\mathsf{Tr}(\bar{\bm{V}}_{\tau_k} - \bar{\bm{V}}_{\tau_{k-1}})&= \mathsf{Tr}(\bar{\bm{V}}_{\tau_k} - \bar{\bm{V}}_{\tau_{k-1} + 1}) + \mathsf{Tr}(\bm{V}_k) \\ 
&\leq\mathsf{Tr}\left( \frac{k}{n}\bm{\Sigma}_n - \frac{k-1}{n}\bm{\Sigma}_n\right) + \beta^{\frac{2}{3}} = \frac{1}{n}\mathsf{Tr}(\bm{\Sigma}_n)+ \beta^{\frac{2}{3}}
\end{align*}
where the inequality on the second line follows from $\bm{V}_{\tau_k} \preceq \frac{k}{n}\bm{\Sigma}_n$, and that $\bm{V}_{\tau_{k-1}+1} \npreceq \frac{k-1}{n}\bm{\Sigma}_n$. However, since the positive semi-definite order is incomplete, the fact that $\bm{V}_{\tau_{k-1}+1} \npreceq \frac{k-1}{n}\bm{\Sigma}_n$ is not equivalent to $\bm{V}_{\tau_{k-1}+1} \succ \frac{k-1}{n}\bm{\Sigma}_n$; neither is there any guarantee that $\mathsf{Tr}(\bm{V}_{\tau_{k-1}+1})$ is greater than $\frac{k-1}{n}\mathsf{Tr}(\bm{\Sigma}_n)$.

In our proof of Corollary \ref{cor:Wu}, we solve this problem by invoking a different assumption \eqref{eq:3rd-momentum-condition} than \eqref{eq:JMLR-Cor-assumption}, and the telescoping method. 

\subsection{Proof of Corollary \ref{thm:matrix-bernstein-mtg}}\label{app:proof-matrix-berstein-mtg}
\label{app:proof-matrix-bernstein-mtg}

% \yuting{switch the order of theorem 2 and 3}\weichen{checked}

Since $\mathbb{E}_{s' \sim P(\cdot \mid s)}[\bm{F}_i(s,s')] = \bm{0}$ holds almost surely on $\mathcal{S}$ for every $i \in [n]$, it can be guaranteed that
\begin{align*}
\frac{1}{n}\sum_{i=1}^n \bm{F}_i(s_{i-1},s_i)
\end{align*}
is a matrix-valued martingale; therefore, if we define
\begin{align*}
\bar{\bm{\Sigma}}_n = \frac{1}{n}\sum_{i=1}^n\mathbb{E}_{i-1}[\bm{F}_i^2(s_{i-1},s_i)],
\end{align*}
% \yuting{sum over $i$?}\weichen{checked}
the matrix Freedman's inequality can be invoked on this martingale. Specifically, for every $\sigma^2$ and $\delta \in (0,1)$, it can be guaranteed with probability at least $1-\frac{\delta}{2}$ that
\begin{align}\label{eq:matrix-freedman}
\left\|\frac{1}{n}\sum_{i=1}^n \bm{F}_i(s_{i-1},s_i)\right\| \lesssim \sqrt{\frac{\sigma^2}{n}\log \frac{d}{\delta}} + \frac{M}{n} \log \frac{d}{\delta}, \quad \text{or} \quad \|\bar{\bm{\Sigma}}_n\| \geq \sigma^2.
\end{align}
In what follows, we aim to bound the norm of $\bar{\bm{\Sigma}}_n$ by controlling its different from $\bm{\Sigma}_n$. Specifically, define
\begin{align*}
\bm{G}_{i-1}(s_{i-1}) = \mathbb{E}_{i-1} [\bm{F}_i^2(s_{i-1},s_i)] - \mathbb{E}_{s \sim \mu, s' \sim P(\cdot \mid s)} [\bm{F}_i^2(s,s')].
\end{align*}
It can be easily verified that $\mu(\bm{G}_{i-1}) = \bm{0}$ and $\|\bm{G}_{i-1}\| \leq M^2$ almost surely for every $i \in [n]$. Therefore, Theorem \ref{thm:matrix-hoeffding} can be applied to the sequence $\{\bm{G}_{i-1}\}_{i \in [n]}$ to obtain
\begin{align*}
\left\|\bar{\bm{\Sigma}}_n - \bm{\Sigma}_n\right\| = \left\|\frac{1}{n}\sum_{i=1}^n \bm{G}_{i-1}(s_{i-1})\right\| \lesssim \frac{p}{p-1}\frac{M^2}{(1-\lambda)^{\frac{1}{2}}n^{\frac{1}{2}}}\log^{\frac{1}{2}}\left(\frac{d}{\delta}\left\|\frac{\mathrm{d}\nu}{\mathrm{d}\mu}\right\|_{\mu,p}\right)
\end{align*}
with probability at least $1-\frac{\delta}{2}$. Hence, the triangle inequality directly yields
\begin{align}\label{eq:matrix-bernstein-quadratic}
\|\bar{\bm{\Sigma}}_n\|\leq \|\bm{\Sigma}_n\| + \frac{p}{p-1} \frac{M^2}{(1-\lambda)^{\frac{1}{2}}n^{\frac{1}{2}}}\log^{\frac{1}{2}}\left(\frac{d}{\delta}\left\|\frac{\mathrm{d}\nu}{\mathrm{d}\mu}\right\|_{\mu,p}\right).
\end{align}
The theorem follows by combining \eqref{eq:matrix-freedman}, \eqref{eq:matrix-bernstein-quadratic} using a union bound argument and taking
\begin{align*}
\sigma^2 = \|\bm{\Sigma}_n\| + \frac{p}{p-1} \frac{M^2}{(1-\lambda)^{\frac{1}{2}}n^{\frac{1}{2}}}\log^{\frac{1}{2}}\left(\frac{d}{\delta}\left\|\frac{\mathrm{d}\nu}{\mathrm{d}\mu}\right\|_{\mu,p}\right).
\end{align*}

\subsection{Proof of Corollary \ref{thm:Berry-Esseen-mtg}}\label{app:proof-Berry-Esseen-mtg}

Part of the proof is inspired by the proof of Lemma B.8 in \citet{cattaneo2024yurinskiiscouplingmartingales} and Theorem 2.1 in \cite{
belloni2018highdimensionalcentrallimit}.
Borrrowing the notation from the proof of Theorem \ref{thm:Srikant-generalize} and Corollary \ref{cor:Wu}, we define, for each $k=1,\ldots,n$,
\begin{align*}
&\bm{S}_k := \sum_{j=1}^k \bm{f}_j(s_{j-1},s_j), \\ 
&\bm{V}_k := \mathbb{E}[\bm{f}_k(s_{k-1},s_k) \bm{f}_k^\top(s_{k-1},s_k)\mid \mathscr{F}_{k-1}]   \\ 
&\bm{T}_k := \sum_{j=k}^n \bm{V}_j^{1/2}\bm{z}_j, \text{ and } \\ %\ale{\text{ why not } \sum_{j=k}^n \bm{V}_j^{1/2} \bm{z}_j?} \weichen{\text{checked.}} \\ 
&\bm{P}_k := \sum_{j=k}^n \bm{V}_j.
\end{align*}
%for every $k \in [n]$. 
This proof approaches the theorem in four steps:
\begin{enumerate}
\item Find a value $\kappa = \kappa(n) = O(\frac{\log n}{\sqrt{n}})$, such that 
\begin{align*}
\mathbb{P}(\|\bm{P}_1 - n\bm{\Sigma}_n\| \geq n\kappa) \leq n^{-\frac{1}{2}}.
\end{align*}
\item Construct a martingale $\{\tilde{\bm{S}}_{j}\}_{j=1}^N$, whose differentiation satisfies the condition of Theorem \ref{thm:Srikant-generalize}, 
% \yuting{bad choice of notation $\bm{\Sigma}_{N}$}\weichen{abolished this notation,} 
such that
\begin{align*}
\mathbb{E}[\tilde{\bm{S}}_N \tilde{\bm{S}}_N^\top] = n(\bm{\Sigma}_n + \kappa \bm{I}), \quad \text{and} \quad \mathbb{P}\left(\|\bm{S}_n - \tilde{\bm{S}}_{N}\|_2 > \sqrt{2d\kappa n  \log n}\right)\leq  n^{-\frac{1}{2}}.
\end{align*}
\item Apply Theorem \ref{thm:Srikant-generalize} to $\tilde{\bm{S}}_{N}$ to derive a Berry-Esseen bound between the distributions of $\tilde{\bm{S}}_{N}$ and $\mathcal{N}(\bm{0},n(\bm{\Sigma}_n + \kappa \bm{I}))$;
\item Combine the results above to achieve the desired Berry-Esseen bound.
\end{enumerate}
\paragraph{Step 1: find $\kappa$.} Due to the Markovian property, the matrix $\bm{V}_k$ is a function of $s_{k-1}$ for every $k \in [n]$. Define
\begin{align*}
\bar{\bm{V}}_k = \mathbb{E}_{s \sim \mu,s' \sim P(\cdot \mid s)}[\bm{f}_i(s,s')\bm{f}_i^\top (s,s')],
\end{align*}
then it can be guaranteed that
\begin{align*}
\mathbb{E}_{s_{k-1}\sim \mu} [\bm{V}_k] = \bar{\bm{V}}_k, 
\end{align*}
and that
\begin{align*}
\|\bm{V}_k - \bar{\bm{V}}_k\| \leq M_k^2, \quad \text{a.s.}
\end{align*}
hold for every $k \in [n]$.
Consequently, a direct application of Theorem \ref{thm:matrix-hoeffding} yields
\begin{align*}
\|\bm{P}_1 - n\bm{\Sigma}_n\|= n \cdot \left\|\frac{1}{n}\sum_{i=1}^n (\bm{V}_k - \bar{\bm{V}}_k)\right\| &\leq \sqrt{\sum_{i=1}^n M_i^4} \sqrt{\frac{20q}{1-\lambda} \log \left(\frac{2d}{n^{-\frac{1}{2}}}\left\|\frac{\mathrm{d}\nu}{\mathrm{d}\mu}\right\|_{\mu,p}\right)} \\ 
 &\leq \sqrt{\sum_{i=1}^n M_i^4}\sqrt{\frac{40q}{1-\lambda} \log \left(2d n \left\|\frac{\mathrm{d}\nu}{\mathrm{d}\mu}\right\|_{\mu,p}\right)},
\end{align*}
with probability at least $1-n^{-\frac{1}{2}}$. In what follows, we take
\begin{align*}
\kappa &= \frac{1}{\sqrt{n}}\sqrt{\frac{\sum_{i=1}^n M_i^4 }{n} \frac{40q}{1-\lambda} \log \left(2d n \left\|\frac{\mathrm{d}\nu}{\mathrm{d}\mu}\right\|_{\mu,p}\right)} = \frac{\bar{M}^2}{\sqrt{n}} \sqrt{\frac{40q}{1-\lambda} \log \left(2d n \left\|\frac{\mathrm{d}\nu}{\mathrm{d}\mu}\right\|_{\mu,p}\right)}.
\end{align*}
\paragraph{Step 2: Construct $\{\tilde{\bm{S}}_j\}_{j=1}^N$.} Define the stopping time
\begin{align*}
\tau := \sup\left\{t \leq n: \sum_{i=1}^t \bm{V}_i \preceq n (\bm{\Sigma}_n + \kappa I)\right\},
\end{align*}
and let
\begin{align*}
&m:= \left\lceil \frac{1}{M^2} \mathsf{Tr}\Big(n (\bm{\Sigma}_n + \kappa I) - \sum_{i=1}^\tau \bm{V}_i\Big) \right\rceil, \quad \text{and} \quad N:= \left\lceil \frac{\mathsf{Tr}(n (\bm{\Sigma}_n + \kappa I))}{M^2} \right\rceil + n.
\end{align*}
By definition, it can be guaranteed that $n+m \leq N$, and that $N \asymp n$. We now construct a martingale difference process $\{ \tilde{\bm{x}}_i \}_{i=1}^{N}$ in the following way: for $1 \leq i \leq \tau$, let $\tilde{\bm{x}}_i = \bm{f}_i(s_{i-1},s_i)$ and for $\tau < i \leq \tau + m$, let 
\[
\tilde{\bm{x}}_i = \frac{1}{\sqrt{m}}\sum_{j=1}^d \epsilon_{ij} \sqrt{\lambda}_j \bm{u}_j,
\]
where the $\lambda_j$'s and $\bm{u}_j$'s are (possibly random) eigenvalues and eigenvectors of the spectral decomposition 
\[
n (\bm{\Sigma}_n + \kappa I) - \sum_{i=1}^\tau \bm{V}_i = \sum_{j=1}^d \lambda_j \bm{u}_j \bm{u}_j^\top,
\]
and $\{\epsilon_{ij}\}_{\tau < i \leq \tau + m, 1 \leq j \leq d}$ are $i.i.d.$ Rademacher random variables independent of the $s_i$'s, i.e.
\begin{align*}
\epsilon_{ij} = \begin{cases}
+1, \quad \text{w.p. }\frac{1}{2}; \\ 
-1, \quad \text{w.p. }\frac{1}{2}.
\end{cases}
\end{align*}
In particular, it holds that, for any $ \tau < i \leq \tau + m$,  
\begin{align*}
\mathbb{E}[\tilde{\bm{x}}_i] = \bm{0}, \quad \mathbb{E}[\tilde{\bm{x}}_i \tilde{\bm{x}}_i^\top\mid \mathscr{F}_{i-1}] = \frac{1}{m} \sum_{j=1}^d \lambda_j \bm{u}_j \bm{u}_j^\top,
\end{align*}
%\ale{shouldn't it be $\mathbb{E}[\tilde{\bm{x}}_i \tilde{\bm{x}}_i^\top] = \frac{1}{m} \sum_{j=1}^d \lambda_j \bm{u}_j \bm{u}_j^\top$?} \weichen{Yes, checked.}
and 
\begin{align*}
\|\tilde{\bm{x}}_i\|_2^2 = \frac{1}{m} \mathsf{Tr}\left(n (\bm{\Sigma}_n + \kappa I) - \sum_{i=1}^\tau \bm{V}_i\right) \leq M^2l
\end{align*}
almost surely. Finally, if $\tau + m < i \leq  N$, we simply set $\tilde{\bm{x}}_i = \bm{0}$. 

The martingale $\{\tilde{\bm{S}}_j\}_{j=1}^N$ is naturally constructed by
\begin{align*}
\tilde{\bm{S}}_j = \sum_{i=1}^j \tilde{\bm{x}}_i.
\end{align*}
%\ale{Shouldn't be say $\{ \tilde{S}_j\}_{j=1}^N$?} \weichen{checked.} 
In this step, we explore the difference between $\bm{S}_n$ and $\tilde{\bm{S}}_N$. Specifically, observe that
\begin{align*}
&\mathbb{P}\left(\|\bm{S}_n - \tilde{\bm{S}}_{N}\|_2 > \sqrt{2d\kappa n\log n}\right) \\ 
&= \mathbb{P}(\|\bm{S}_n - \tilde{\bm{S}}_{N}\|_2 > \sqrt{2d\kappa n\log n}, \|\bm{P}_1 - n\bm{\Sigma}_n\| \leq \kappa n) \\ 
&+ \mathbb{P}(\|\bm{S}_n - \tilde{\bm{S}}_{N}\|_2 > \sqrt{2d\kappa n\log n}, \|\bm{P}_1 - n\bm{\Sigma}_n\| > \kappa n)  \\ 
&\leq \mathbb{P}(\|\bm{S}_n - \tilde{\bm{S}}_{N}\|_2 > \sqrt{2d\kappa n\log n}, \|\bm{P}_1 - n\bm{\Sigma}_n\| \leq \kappa n) + \mathbb{P}(\|\bm{P}_1 - n\bm{\Sigma}_n\| > \kappa n) \\ 
&\leq \mathbb{P}(\|\bm{S}_n - \tilde{\bm{S}}_{N}\|_2 > \sqrt{2d\kappa n\log n}, \|\bm{P}_1 - n\bm{\Sigma}_n\| \leq \kappa n ) + n^{-\frac{1}{2}}.
\end{align*}
To bound the first term on the left hand side of the last inequality, notice that, when $\|\bm{P}_1 - n\bm{\Sigma}_n\| \leq \kappa n$,
\begin{align*}
\bm{P}_1 = \sum_{i=1}^n \bm{V}_i \preceq n(\bm{\Sigma}_n + \kappa \bm{I}).
\end{align*}
Thus, on the same event, $\tau = n$ and for every $j \in [d]$, 
\begin{align*}
\lambda_j \leq \|n(\bm{\Sigma}_n + \kappa \bm{I}) - \bm{P}_1\| \leq \|n\bm{\Sigma}_n - \bm{P}_1\| + \|n\kappa \bm{I}\| = 2\kappa n.
\end{align*}

%\ale{I am confused. Shouldn't it be $\lambda_j \leq M^2 + \kappa n$?} \weichen{please see above.}
Consequently, 
\begin{align*}
\|\bm{S}_n - \tilde{\bm{S}}_{N}\|_2^2 &= \left\|\sum_{i=n+1}^{n+m} \tilde{\bm{x}}_i \right\|_2^2 
= \frac{1}{m}\left\|\sum_{j=1}^d \sqrt{\lambda_j} \left(\sum_{i=n+1}^{n+m}\epsilon_{ij}\right)\bm{u}_j\right\|_2^2 
\leq \frac{2\kappa n}{m}  \sum_{j=1}^d \left(\sum_{i=n+1}^{n+m}\epsilon_{ij}\right)^2.
\end{align*}
Since $\{\epsilon_{ij}\}_{n+1 \leq i \leq n+m, 1 \leq j \leq d}$ are $i.i.d.$ Rademacher random variables, the Hoeffding's inequality guarantees that
\begin{align*}
\left|\sum_{i=n+1}^{n+m}\epsilon_{ij}\right| \lesssim \sqrt{m \log n}
\end{align*}
% \yuting{why there is a $d$ factor?} \ale{I agree: there is no $d$ here. Also, the above relation is not an $\leq$ but a $\preceq$ because we are ignoring universal constants} \weichen{all checked.} 
with probability at least $1-2n^{-\frac{1}{2}}$. As a direct consequence, the difference between $\bm{S}_n$ and $\tilde{\bm{S}}_N$ can be bounded by
\begin{align*}
\|\bm{S}_n - \tilde{\bm{S}}_{N}\|_2^2 \lesssim 2 d \kappa n\log n
\end{align*}
with probability at least $1-2n^{-\frac{1}{2}}$. In combination, we obtain
\begin{align*}
\mathbb{P}(\|\bm{S}_n - \tilde{\bm{S}}_N\|_2 \gtrsim \sqrt{2d\kappa n\log n}) \leq 3n^{-\frac{1}{2}}.
\end{align*}

\paragraph{Step 3: Berry-Esseen bound on $\tilde{\bm{S}}_N$.} It can be easily verified that the sequence $\{ \tilde{\bm{x}}_i\}_{i=1}^N$ is a martingale difference such that
\begin{align*}
&\sum_{i=1}^N \mathbb{E}[\tilde{\bm{x}}_i \tilde{\bm{x}}_i^\top \mid \mathscr{F}_{i-1}] = n(\bm{\Sigma}_n + \kappa \bm{I}), \quad \text{and} \quad \|\bm{\tilde{x}}_i\|_2 \leq M \quad \text{a.s., } \forall i \in [N].
\end{align*}
Hence, Theorem \ref{thm:Srikant-generalize} can be applied on $\tilde{\bm{S}}_N$ to obtain 
%\ale{Minor comment, though subtle (and following up on an earlier comment of Yuting's about the notation $\bm{\Sigma}_N$, which seems to suggest a sum of $N$ terms): here we apply Theorem \ref{thm:Srikant-generalize} to the sum $S_N$ but divide by $\sqrt{n}$. So we are effectively using the ``sum'' and not the ``average'' version of that theorem. I think everything checks out but just wated to point this out. } \weichen{Yes, this is a bit tricky. I just abolished the notation $\bm{\Sigma}_N$. Hope this reduces confusion!} 
\begin{align*}
&d_{\mathsf{W}}\left(\frac{\tilde{\bm{S}}_N}{\sqrt{n}}, \mathcal{N}(\bm{0},\bm{\Sigma}_n + \kappa \bm{I})\right) \lesssim Md \log (d\|(\bm{\Sigma}_n + \kappa I)\|)\frac{\log^2 n}{\sqrt{n}},
\end{align*} 
where we applied the fact that $M\geq 1$ and $\|d\bm{\Sigma}_n\| \geq 1.$
\paragraph{Step 4: Completing the proof.} By the triangle inequality,
\begin{align}\label{eq:Wu-initial-decompose}
d_{\mathsf{C}}\left(\frac{\bm{S}_n}{\sqrt{n}},\mathcal{N}(\bm{0},\bm{\Sigma}_n)\right) \leq d_{\mathsf{C}}\left(\frac{\bm{S}_n}{\sqrt{n}},\mathcal{N}(\bm{0},\bm{\Sigma}_n + \kappa \bm{I})\right) + d_{\mathsf{C}}\left(\mathcal{N}(\bm{0},\bm{\Sigma}_n),\mathcal{N}(\bm{0},\bm{\Sigma}_n + \kappa \bm{I})\right)
\end{align}
where the second term on the right-hand-side can be bounded using a direct application of Theorem \ref{thm:DMR} by
\begin{align}\label{eq:Wu-Gaussian-comparison}
d_{\mathsf{C}}\left(\mathcal{N}(\bm{0},\bm{\Sigma}_n),\mathcal{N}(\bm{0},\bm{\Sigma}_n + \kappa \bm{I})\right) \lesssim \kappa \|\bm{\Sigma}_n^{-1}\|_{\mathsf{F}} \leq \kappa \frac{\sqrt{d}}{c}  = O\left( \sqrt{\frac{d}{c^2 n}} \log n \right),
\end{align}
where the last inequality follows from the assumption that $\lambda_{\min}(\bm{\Sigma}_n) \geq c$ and the choice of $\kappa$.
For the first term, consider any convex set $\mathcal{A} \subset \mathbb{R}^d$, the triangle inequality guarantees, for every $x > 0$, that
\begin{align}\label{eq:Wu-decompose-upper}
\mathbb{P}\left(\frac{\bm{S}_n}{\sqrt{n}} \in \mathcal{A}\right)&= \mathbb{P}\left(\frac{\bm{S}_n}{\sqrt{n}} \in \mathcal{A}, \left\|\frac{\bm{S}_n - \tilde{\bm{S}}_N}{\sqrt{n}}\right\|_2 \leq x\right) + \mathbb{P}\left(\frac{\bm{S}_n}{\sqrt{n}} \in \mathcal{A}, \left\|\frac{\bm{S}_n - \tilde{\bm{S}}_N}{\sqrt{n}}\right\|_2 > x\right) \nonumber \\ 
&\leq \mathbb{P}\left(\frac{\tilde{\bm{S}}_N}{\sqrt{n}} \in \mathcal{A}^x \right) + \mathbb{P}\left(\left\|\frac{\bm{S}_n - \tilde{\bm{S}}_N}{\sqrt{n}}\right\|_2 > x\right)\nonumber \\
&\leq \mathbb{P}\left(\frac{\tilde{\bm{T}}_N}{\sqrt{n}} \in \mathcal{A}^x \right) + d_{\mathsf{C}}\left(\frac{\tilde{\bm{S}}_N}{\sqrt{n}},\frac{\tilde{\bm{T}}_N}{\sqrt{n}}\right)+ \mathbb{P}\left(\left\|\frac{\bm{S}_n - \tilde{\bm{S}}_N}{\sqrt{n}}\right\|_2 > x\right) \nonumber \\ 
&\leq \mathbb{P}\left(\frac{\tilde{\bm{T}}_N}{\sqrt{n}} \in \mathcal{A} \right) + \left(\|\bm{\Sigma}_n\|_{\mathsf{F}}^{\frac{1}{2}} + \sqrt{\kappa} d^{\frac{1}{4}}\right) x \nonumber \\ 
&+ d_{\mathsf{C}}\left(\frac{\tilde{\bm{S}}_N}{\sqrt{n}},\frac{\tilde{\bm{T}}_N}{\sqrt{n}}\right)+ \mathbb{P}\left(\left\|\frac{\bm{S}_n - \tilde{\bm{S}}_N}{\sqrt{n}}\right\|_2 > x\right).
\end{align}
Here, $\tilde{\bm{T}}_N \sim \mathcal{N}(\bm{0},n(\bm{\Sigma}_n +\kappa \bm{I}))$ and we applied Theorem \ref{thm:Gaussian-reminder} on the last line. On the right-most part of the inequality, the third term can be bounded by the Berry-Esseen bound on $\tilde{\bm{S}}_N$ and Theorem \ref{thm:Gaussian-convex-Wass}:
\begin{align*}
d_{\mathsf{C}}\left(\frac{\tilde{\bm{S}}_N}{\sqrt{n}},\frac{\tilde{\bm{T}}_N}{\sqrt{n}}\right)
&\leq \|\bm{\Sigma}_n + \kappa \bm{I}\|_{\mathsf{F}}^{\frac{1}{4}} \sqrt{d_{\mathsf{W}}\left(\frac{\tilde{\bm{S}}_N}{\sqrt{n}},\frac{\tilde{\bm{T}}_N}{\sqrt{n}}\right)} \\ 
&\leq \left(\|\bm{\Sigma}_n\|_{\mathsf{F}}^{\frac{1}{4}} +  (\kappa \sqrt{d})^{\frac{1}{4}}\right)\sqrt{M} d^{\frac{1}{2}}\log^{\frac{1}{2}} (d\|\bm{\Sigma}_n+\kappa \bm{I}\|) n^{-\frac{1}{4}}\log n.
\end{align*}
Therefore, taking $x = \sqrt{2d\kappa \log n}$ in \eqref{eq:Wu-decompose-upper} yields
\begin{align*}
\mathbb{P}\left(\frac{\bm{S}_n}{\sqrt{n}} \in \mathcal{A}\right)& \leq \mathbb{P}\left(\frac{\tilde{\bm{T}}_N}{\sqrt{n}} \in \mathcal{A} \right) + \left(\|\bm{\Sigma}_n\|_{\mathsf{F}}^{\frac{1}{2}} + \sqrt{\kappa} d^{\frac{1}{4}}\right) \cdot \sqrt{2d\kappa\log n} \\
&+ \left(\|\bm{\Sigma}_n\|_{\mathsf{F}}^{\frac{1}{4}} +  (\kappa \sqrt{d})^{\frac{1}{4}}\right)\sqrt{M} d^{\frac{1}{2}}\log^{\frac{1}{2}} (d\|\bm{\Sigma}_n+\kappa \bm{I}\|) n^{-\frac{1}{4}}\log n + 3n^{-\frac{1}{2}}.
\end{align*}
Since $\sqrt{\kappa} = o(1)$, a simple reorganization yields
\begin{align}\label{eq:Wu-upper-bound}
&  \mathbb{P}\left(\frac{\bm{S}_n}{\sqrt{n}} \in \mathcal{A}\right)- \mathbb{P}\left(\frac{\tilde{\bm{T}}_N}{\sqrt{n}} \in \mathcal{A} \right)\nonumber \\ 
&\lesssim \left\{\bar{M}\left(\frac{q}{1-\lambda}\right)^{\frac{1}{4}}\log^{\frac{1}{4}}\left(d\left\|\frac{\mathrm{d}\nu}{\mathrm{d}\mu}\right\|_{\mu,p}\right)\|\bm{\Sigma}_n\|_{\mathsf{F}}^{\frac{1}{2}}+ \sqrt{M} \log^{\frac{1}{2}} (d\|\bm{\Sigma}_n\|)\|\bm{\Sigma}_n\|_{\mathsf{F}}^{\frac{1}{4}}\right\} \sqrt{d}n^{-\frac{1}{4}}\log n.
\end{align}
Meanwhile, a union bound argument and the triangle inequality guarantee
\begin{align*}
\mathbb{P}\left(\frac{\bm{S}_n}{\sqrt{n}} \in \mathcal{A}\right)& \geq \mathbb{P}\left(\frac{\bm{S}_n}{\sqrt{n}} \in \mathcal{A} \cup \left\|\frac{\bm{S}_n - \tilde{\bm{S}}_N}{\sqrt{n}}\right\|_2 > x\right) - \mathbb{P}\left(\left\|\frac{\bm{S}_n - \tilde{\bm{S}}_N}{\sqrt{n}}\right\|_2 > x\right) \\ 
&\geq \mathbb{P}\left(\frac{\tilde{\bm{S}}_N}{\sqrt{n}} \in \mathcal{A}^{-x} \right) - \mathbb{P}\left(\left\|\frac{\bm{S}_n - \tilde{\bm{S}}_N}{\sqrt{n}}\right\|_2 > x\right) \\ 
&\geq \mathbb{P}\left(\frac{\tilde{\bm{T}}_N}{\sqrt{n}} \in \mathcal{A}^{-x} \right) - d_{\mathsf{C}}\left(\frac{\tilde{\bm{S}}_N}{\sqrt{n}},\frac{\tilde{\bm{T}}_N}{\sqrt{n}}\right)- \mathbb{P}\left(\left\|\frac{\bm{S}_n - \tilde{\bm{S}}_N}{\sqrt{n}}\right\|_2 > x\right)  \\ 
&\geq \mathbb{P}\left(\frac{\tilde{\bm{T}}_N}{\sqrt{n}} \in \mathcal{A} \right) - \left(\|\bm{\Sigma}_n\|_{\mathsf{F}}^{\frac{1}{2}} + \sqrt{\kappa} d^{\frac{1}{4}}\right) x \\ 
&- d_{\mathsf{C}}\left(\frac{\tilde{\bm{S}}_N}{\sqrt{n}},\frac{\tilde{\bm{T}}_N}{\sqrt{n}}\right)- \mathbb{P}\left(\left\|\frac{\bm{S}_n - \tilde{\bm{S}}_N}{\sqrt{n}}\right\|_2 > x\right);
\end{align*}
consequently, it can be symmetrically proved that
\begin{align}\label{eq:Wu-lower-bound}
&\mathbb{P}\left(\frac{\tilde{\bm{T}}_N}{\sqrt{n}} \in \mathcal{A} \right)-\mathbb{P}\left(\frac{\bm{S}_n}{\sqrt{n}} \in \mathcal{A}\right) \nonumber  \\ 
&\leq \left\{\bar{M}\left(\frac{q}{1-\lambda}\right)^{\frac{1}{4}}\log^{\frac{1}{4}}\left(d\left\|\frac{\mathrm{d}\nu}{\mathrm{d}\mu}\right\|_{\mu,p}\right)\|\bm{\Sigma}_n\|_{\mathsf{F}}^{\frac{1}{2}}+ \sqrt{M} \log^{\frac{1}{2}} (d\|\bm{\Sigma}_n\|)\|\bm{\Sigma}_n\|_{\mathsf{F}}^{\frac{1}{4}}\right\}\sqrt{d} n^{-\frac{1}{4}}\log n.
\end{align}
The theorem follows by combining \eqref{eq:Wu-initial-decompose}, \eqref{eq:Wu-Gaussian-comparison}, \eqref{eq:Wu-upper-bound}, \eqref{eq:Wu-lower-bound} and taking a supremum over $\mathcal{A} \in \mathscr{C}$. 





%\section{Proof of results regarding MCMC}\label{app:proof-MCMC}
Define $\bm{u}: \mathcal{S} \to \mathbb{R}^d$ as the solution to the \emph{Poission equation}
\begin{align*}
\bm{g}(x) := \bm{f}(x) - \mathbb{E}_{\mu}[\bm{f}] = \bm{u}(x) - \mathcal{P}\bm{u}(x), \quad \forall x \in \mathcal{S}.
\end{align*}
In fact, it is easy to verify that $\bm{u}(x)$ admits the following expression:
\begin{align*}
\bm{u}(x) = \sum_{k=0}^{\infty} \mathcal{P}^k \bm{g}(x) = \sum_{k=0}^{\infty} [\mathcal{P}^k \bm{f}(x)-\mu(\bm{f})].
\end{align*}
The mixing assumption \ref{as:mixing}, combined with the basic property of TV distance, directly implies that
\begin{align*}
\|\mathcal{P}^k \bm{f}(x) - \mu(\bm{f})\|_2   
&= \left\|\mathbb{E}_{s \sim P^k(\cdot \mid x)}[\bm{f}(s)] - \mathbb{E}_{s \sim \mu}[\bm{f}(s)]\right\|_2 \\ 
&\leq \sup_{x \in \mathcal{S}}\|\bm{f}(x)\|_2\cdot d_{\mathsf{TV}}(P^k(\cdot \mid x),\mu) \leq Mm\rho^k.
\end{align*}
Hence the norm of $\bm{u}(\cdot)$ is bounded uniformly by
\begin{align}\label{eq:MCMC-u-bound}
\|\bm{u}(x)\|_2 \leq \sum_{k=0}^{\infty} \|\mathcal{P}^k \bm{f}(x)-\mu(\bm{f})\|_2 \leq \frac{Mm}{1-\rho}, \quad \forall x \in \mathcal{S}.
\end{align}
Using the function $\bm{u}(\cdot)$, $\bar{\bm{f}}_n -\mu(\bm{f})$ can be represented through a telescoping technique by
\begin{align}\label{eq:MCMC-decompose}
\bar{\bm{f}}_n - \mu(\bm{f}) &= \frac{1}{n} \sum_{i=1}^n [\bm{f}(s_i) - \mu(\bm{f})] \nonumber \\ 
&=  \frac{1}{n} \sum_{i=1}^n[\bm{u}(s_i) - \mathcal{P}\bm{u}(s_i)] \nonumber \\ 
&= \frac{1}{n} \sum_{i=1}^n[\bm{u}(s_i) - \mathbb{E}_{i}\bm{u}(s_{i+1})] \nonumber \\ 
&= \frac{1}{n} \sum_{i=1}^n[\bm{u}(s_i) - \mathbb{E}_{i-1}\bm{u}(s_{i})] + \frac{1}{n} \sum_{i=1}^n[\mathbb{E}_{i-1}\bm{u}(s_{i})-\mathbb{E}_{i}\bm{u}(s_{i+1})]\nonumber \\
&= \frac{1}{n} \sum_{i=1}^n[\bm{u}(s_i) - \mathbb{E}_{i-1}\bm{u}(s_{i})] + \frac{1}{n} \left(\mathbb{E}_0[\bm{u}(s_1)] - \mathbb{E}_n[\bm{u}(s_{n+1})]\right)
\end{align}
For simplicity, we define a vector-valued function $\bm{m}: \mathcal{S}^2 \to \mathbb{R}^d$ as
\begin{align}\label{eq:MCMC-defn-m}
\bm{m}(s,s') = \bm{u}(s')-\mathcal{P}(s)
\end{align}
It can be easily verified that
\begin{align}
&\mathbb{E}_{s' \sim P(\cdot \mid s)}\bm{m}(s,s') = \bm{0}, \quad \forall s \in \mathcal{S}; \quad \text{and}\\ 
&\mathbb{E}_{s \sim \mu, s' \sim P(\cdot \mid s)}[\bm{m}(s,s')\bm{m}^\top(s,s')]=\tilde{\bm{\Sigma}}.
\end{align}
Furthermore, \eqref{eq:MCMC-decompose} and \eqref{eq:MCMC-defn-m} indicates that
\begin{align}\label{eq:MCMC-decompose-2}
\bar{\bm{f}}_n - \mu(\bm{f}) = \frac{1}{n}\sum_{i=1}^n \bm{m}(s_{i-1},s_i) + \frac{1}{n} \left(\mathbb{E}_0[\bm{u}(s_1)] - \mathbb{E}_n[\bm{u}(s_{n+1})]\right)
\end{align}
The Bernstein's inequality and Berry-Esseen bound for $\bar{\bm{f}}_n$ are both based on this decomposition.

\subsection{Proof of Theorem \ref{thm:MCMC-bernstein}}
By triangle inequality, the decomposition \eqref{eq:MCMC-decompose-2} and the universal boundedness of $\bm{u}(\cdot)$ \eqref{eq:MCMC-u-bound} directly implies that
\begin{align}\label{eq:MCMC-bernstein-triangle}
\left\|\bar{\bm{f}}_n - \mu(\bm{f})\right\|_2 \leq \left\|\frac{1}{n}\sum_{i=1}^n \bm{m}(s_{i-1},s_i)\right\|_2 + \frac{2Mm}{(1-\rho)}.
\end{align}
In order to bound the first term on the right-hand-side, consider the matrix-valued function $\bm{F}: \mathcal{S}^2 \to \mathbb{R}^{(d+1) \times (d+1)}$, defined as 
\begin{align*}
\bm{F}(s,s') = \begin{pmatrix}
0 & \bm{m}^\top(s,s') \\ 
\bm{m}(s,s') & \bm{0}_{d\times d}.
\end{pmatrix}, \quad \forall i \in [n].
\end{align*}
It can then be verified that 
\begin{align*}
\left\|\mathbb{E}_{s\sim \mu, s' \sim P(\cdot \mid s)}[\bm{F}^2(s,s')] \right\|= \mathbb{E}_{s\sim \mu, s' \sim P(\cdot \mid s)}\|\bm{m}(s,s')\|_2^2 = \mathsf{Tr}(\tilde{\bm{\Sigma}})
\end{align*}
and that
\begin{align*}
\|\bm{F}(s,s')\| \leq \frac{Mm}{1-\rho}, \quad \forall s,s' \in \mathcal{S}.
\end{align*}
Therefore, a direct application of Theorem \ref{thm:matrix-bernstein-mtg} on the sequence $\{\bm{F}(s_{i-1},s_i)\}_{i \in [n]}$, combined with \eqref{eq:MCMC-bernstein-triangle}, yields the conclusion of the theorem. 


\section{Proof of results regarding TD learning}\label{app:proof-TD}

Throughout this section, we denote
\begin{align}
&\bm{\Delta}_t = \bm{\theta}_t - \bm{\theta}^\star, \quad \forall t \in [T], \quad \text{and}  \\
&\bar{\bm{\Delta}}_T = \bar{\bm{\theta}}_T - \bm{\theta}^\star.
\end{align}

Furthermore, for every $t = 0,1,...,T$, we denote 
\begin{align*}
\mathbb{E}_t[\cdot] := \mathbb{E}[\cdot \mid s_0,s_1,...,s_t].
\end{align*}
Without any subscript, the operator $\mathbb{E}$ represents taking expectation with respect to all the samples starting from $s_0$.

\subsection{$L^2$ convergence of the TD estimation error}
The following theorem captures the asymptotic property of $\mathbb{E}\|\bm{\Delta}_t\|_2^2$ with Markov samples and is useful in our proofs for other results. 
Note that the bound holds, non-asymptotically, for all $t \geq t^\star$ where $t^\star$ is a problem-dependent quantity; we state it as an asymptotic result only for convenience.

\begin{theorem}\label{thm:markov-L2-convergence}
Consider TD with Polyak-Ruppert averaging~\eqref{eq:TD-update-all} with Markov samples and decaying stepsizes $\eta_t = \eta_0 t^{-\alpha}$ for $\alpha \in (\frac{1}{2},1)$. Suppose that the Markov transition kernel has a unique stationary distribution, a strictly positive spectral gap, and mixes exponentially as indicated by Assumption \ref{as:mixing}.  It can then be guaranteed that when $t \to \infty$,
\begin{align*}
\mathbb{E}\big[\|\bm{\Delta}_t\|_2^2\big] \lesssim (2\|\bm{\theta}^\star\|_2+1)^2 \left[\frac{1}{(1-\rho)^2}\frac{\eta_0}{\lambda_0(1-\gamma)}t^{-\alpha} \log^2 t + o\left(t^{-\alpha} \log^2 t\right)\right].
\end{align*}
%The following properties hold for the expected squared norm of the TD estimation error $\bm{\Delta}_t$:
% \begin{enumerate}
% \item Under Scenario \ref{case:alpha}, 
% \begin{align*}
% \mathbb{E}\|\bm{\Delta}_t\|_2^2 \leq O(t^{-\alpha} \log^2 t);
% \end{align*}
% \item Under Scenario \ref{case:nu},
% \begin{align*}
% \mathbb{E}\|\bm{\Delta}_t\|_2^2 \leq O((t+\nu)^{-1} \log^2 (t+\nu));
% \end{align*}
% \item Under Scenario \ref{case:eta} with the proviso that
% \begin{align}\label{eq:S3-eta-condition-L2}
% \eta_0 \frac{\log(m/\eta)}{\log(1/\rho)} \leq \frac{\lambda_0(1-\gamma)}{43},
% \end{align}
% it can be guaranteed that
% \begin{align*}
% \mathbb{E}\|\bm{\Delta}_t\|_2^2 \leq (2\|\bm{\theta}^\star\|_2+1)^2 \quad \text{for all} \quad t \in \mathbb{N}.
% \end{align*}
% \end{enumerate}
\end{theorem}
% \ale{Add a comment to say that the proof is non-asymptotic and holds for all  $t$ satisfying $t > 2 t_{\mix}$ and $\eta_t (1+\gamma)^2 < \lambda_0(1-\gamma)$ and such that  but for convenience we have expressed the statement as an asymptotic one} \weichen{added one sentence before the theorem.}

\begin{proof}
We firstly construct an iterative relation along the sequence $\{\mathbb{E}\|\bm{\Delta}_t\|_2^2\}_{t \geq 0}$ in general, and then refine our analysis using a specific choice of stepsizes. The TD iteration rule \eqref{eq:TD-update-all} directly implies that
\begin{align*}
\|\bm{\Delta}_{t}\|_2^2 &= \|\bm{\Delta}_{t-1}\|_2^2 -2\eta_t \bm{\Delta}_{t-1}^\top (\bm{A}_t \bm{\theta}_{t-1} - \bm{b}_t) + \eta_t^2 \|\bm{A}_t \bm{\theta}_{t-1} - \bm{b}_t\|_2^2 \\ 
&\leq \|\bm{\Delta}_{t-1}\|_2^2 -2\eta_t \bm{\Delta}_{t-1}^\top (\bm{A}_t \bm{\theta}^\star + \bm{A}_t \bm{\Delta}_{t-1} - \bm{b}_t) + 2\eta_t^2 (\|\bm{A}_t \bm{\Delta}_{t-1}\|_2^2 + \|\bm{A}_t \bm{\theta}^\star - \bm{b}_t\|_2^2)\\
&= \|\bm{\Delta}_{t-1}\|_2^2 -2\eta_t \bm{\Delta}_{t-1}^\top \bm{A \Delta}_{t-1} - 2\eta_t\bm{\Delta}_{t-1}^\top(\bm{A}_t-\bm{A})\bm{\Delta}_{t-1}\\ 
& - 2\eta_t \bm{\Delta}_{t-1}^\top (\bm{A}_t \bm{\theta}^\star - \bm{b}_t)+ 2\eta_t^2 (\|\bm{A}_t \bm{\Delta}_{t-1}\|_2^2 + \|\bm{A}_t \bm{\theta}^\star - \bm{b}_t\|_2^2).
\end{align*}
Since $\bm{\Delta}_{t-1}^\top \bm{A \Delta}_{t-1} \geq \lambda_0(1-\gamma)\|\bm{\Delta}\|_{t-1}$ due to \eqref{eq:lemma-A-1} and $\|\bm{A}_t\|\leq 1+\gamma$, we can bound $\mathbb{E}\|\bm{\Delta}_{t}\|_2^2$ by 
\begin{align}\label{eq:markov-L2-iter}
\mathbb{E}\|\bm{\Delta}_{t}\|_2^2 &\leq \mathbb{E}\|\bm{\Delta}_{t-1}\|_2^2 -2\lambda_0(1-\gamma) \eta_{t} \mathbb{E}\|\bm{\Delta}_{t-1}\|_2^2 +2\eta_t^2(1+\gamma)^2  \mathbb{E}\|\bm{\Delta}_{t-1}\|_2^2 \nonumber \\ 
&- 2\eta_t \mathbb{E}[\bm{\Delta}_{t-1}^\top(\bm{A}_t - \bm{A})\bm{\Delta}_{t-1}] - 2\eta_t \mathbb{E}[\bm{\Delta}_{t-1}^\top (\bm{A}_t \bm{\theta}^\star - \bm{b}_t)] + 2\eta_t^2 \mathbb{E}\|\bm{A}_t \bm{\theta}^\star - \bm{b}_t\|_2^2 \nonumber \\ 
&= \underset{I_1}{\underbrace{\left(1-2\lambda_0(1-\gamma)\eta_t + 2\eta_t^2(1+\gamma)^2\right) \mathbb{E}\|\bm{\Delta}_{t-1}\|_2^2 }}+ \underset{I_2}{\underbrace{2\eta_t^2 \mathbb{E}\|\bm{A}_t \bm{\theta}^\star - \bm{b}_t\|_2^2}} \nonumber \\ 
&- 2\eta_t\underset{I_3}{\underbrace{ \mathbb{E}[\bm{\Delta}_{t-1}^\top(\bm{A}_t - \bm{A})\bm{\Delta}_{t-1}]}} - 2 \eta_t \underset{I_4} {\underbrace{\mathbb{E}[\bm{\Delta}_{t-1}^\top (\bm{A}_t \bm{\theta}^\star - \bm{b}_t)]}}.
\end{align}

In this expression, $I_1$ is contractive with respect to $\mathbb{E}\|\bm{\Delta}_{t-1}\|_2^2$ as long as $\eta_t$ is sufficiently small, while $I_2$ is proportional to $\eta_t^2$ since $\mathbb{E}\|\bm{A}_t\bm{\theta}^\star - \bm{b}_t\|_2^2$ is independent of $t$. These two terms are desirable and can be left as they are; 

The difficulty of this proof lies in bounding $I_3$ and $I_4$ using Markov samples. Notice that with $i.i.d.$ sampling, both terms are actually $0$; hence, we aim to bound them by applying the mixing property of the Markov chain. 

To simplify notation, throughout the proof, we denote 
\begin{align*}
t_{\mix}:=t_{\mix}(\eta_t) + 1,
\end{align*}
so that with Markov samples, $s_{t-1}\mid \mathscr{F}_{t-t_{\mix}} \sim P^{t_{\mix}-1}(\cdot \mid s_{t-t_{\mix}})$, and that
\begin{align*}
d_{\mathsf{TV}}(P^{t_{\mix}-1}(\cdot \mid s_{t-t_{\mix}}),\mu) \leq \eta_t.
\end{align*}
Meanwhile, Assumption \ref{as:mixing} implies that

\begin{align}\label{eq:tmix-bound-L2}
t_{\mix} \leq \frac{\log(m/\eta_t)}{\log(1/\rho)} +1 = \frac{\log(m/\eta_0) + \alpha \log t}{\log(1/\rho)} + 1< \frac{\log(m/\eta_0) + \alpha \log t}{1-\rho} + 1.
\end{align}

%\ale{I think it would be good here to introduce $\bm{\Delta}_{t-t_{\mix}}$ and explain its role and the proof strategy.}\weichen{please see below.}
In other words, $t_{\mix}(\eta_t)$ grows at most by $O(\log t)$; therefore, in what follows, we can assume that $t$ is large enough such that $t \geq 2t_{\mix}$. The essential idea of bounding $I_3$ and $I_4$ involves decomposing $\bm{\Delta}_{t-1}$ by
\begin{align*}
\bm{\Delta}_{t-1} = (\bm{\Delta}_{t-1} - \bm{\Delta}_{t-t_{\mix}}) + \bm{\Delta}_{t-t_{\mix}},
\end{align*}
where the norm of $(\bm{\Delta}_{t-1} - \bm{\Delta}_{t-t_{\mix}})$ is bounded by the decaying stepsizes, while the correlation between $\bm{\Delta}_{t-t_{\mix}}$ and $\bm{A}_t, \bm{b}_t$ is bounded by the mixing property of the Markov chain.

We address $I_3$ and $I_4$ respectively.

\paragraph{Bounding $I_3$.} The definition of $t_{\mix}$ implies
\begin{align*}
&\left|\mathbb{E}[\bm{\Delta}_{t-t_{\mix}}^\top (\bm{A}_t - \bm{A})\bm{\Delta}_{t-t_{\mix}}]\right|\\
&= \left|\mathbb{E}[\mathbb{E}_{t-t_{\mix}}[\bm{\Delta}_{t-t_{\mix}}^\top (\bm{A}_t - \bm{A})\bm{\Delta}_{t-t_{\mix}}]]\right| \\ 
&= \left|\mathbb{E} \left[\mathbb{E}_{s_{t-1} \sim P^{t_{\mix}-1}(\cdot \mid s_{t-t_{\mix}})}[\mathbb{E}_{s_t \sim P(\cdot \mid s_{t-1})}[\bm{\Delta}_{t-t_{\mix}}^\top\bm{A}_t \bm{\Delta}_{t-t_{\mix}}]] - \mathbb{E}_{s_{t-1} \sim \mu}[\mathbb{E}_{s_t \sim P(\cdot \mid s_{t-1})}[\bm{\Delta}_{t-t_{\mix}}^\top\bm{A}_t \bm{\Delta}_{t-t_{\mix}}]  \right]] \right|\\ 
&\leq \mathbb{E}\sup_{s_{t-1}} |\mathbb{E}_{s_t \sim P(\cdot \mid s_{t-1})}\bm{\Delta}_{t-t_{\mix}}^\top\bm{A}_t \bm{\Delta}_{t-t_{\mix}}| \cdot d_{\mathsf{TV}}(P^{t_{\mix}-1}(\cdot \mid s_{t-t_{\mix}}),\mu) \\ 
&\leq \mathbb{E}[2\|\bm{\Delta}_{t - t_{\mix}}\|_2^2 ] \cdot \eta_t;
\end{align*}
notice that the inequality on the fourth line follows from the basic property of the TV distance.
As a direct consequence, $I_3$ is featured by %\ale{Here we need to say explicitly that $t > t_{\mix}$}\weichen{checked.}
\begin{align*}
I_3 &= \mathbb{E}[\bm{\Delta}_{t-t_{\mix}}^\top (\bm{A}_t - \bm{A})\bm{\Delta}_{t-t_{\mix}}] + 2\mathbb{E}[\bm{\Delta}_{t-t_{\mix}}^\top (\bm{A}_t - \bm{A})(\bm{\Delta}_{t-1} - \bm{\Delta}_{t-t_{\mix}})] \\ 
&+ \mathbb{E}[(\bm{\Delta}_{t-1} - \bm{\Delta}_{t-t_{\mix}})^\top (\bm{A}_t - \bm{A})(\bm{\Delta}_{t-1} - \bm{\Delta}_{t-t_{\mix}})] \\ 
&\geq -2\eta_t\mathbb{E}\|\bm{\Delta}_{t-t_{\mix}}\|_2^2 -4 \mathbb{E}[\|\bm{\Delta}_{t-t_{\mix}}\|_2\|\bm{\Delta}_{t-1} - \bm{\Delta}_{t-t_{\mix}}\|_2]-2 \mathbb{E}\|\bm{\Delta}_{t-1} - \bm{\Delta}_{t-t_{\mix}}\|_2^2 \\ 
&= -2\eta_t\mathbb{E}\|\bm{\Delta}_{t-t_{\mix}}\|_2^2 -4 \mathbb{E}[\|\bm{\Delta}_{t-t_{\mix}}\|_2\|\bm{\theta}_{t-1} - \bm{\theta}_{t-t_{\mix}}\|_2]-2 \mathbb{E}\|\bm{\theta}_{t-1} - \bm{\theta}_{t-t_{\mix}}\|_2^2.
\end{align*}
%\ale{Why is it the case that $| \mathbb{E}[\bm{\Delta}_{t-t_{\mix}}^\top (\bm{A}_t - \bm{A})\bm{\Delta}_{t-t_{\mix}}] | \leq 2\eta_t\mathbb{E}\|\bm{\Delta}_{t-t_{\mix}}\|_2^2 $? Where does the $\eta_t$ come from? This looks like a repeating typo.}\weichen{This actually comes from the mixing property and the definition of $t_{\mix}$. Added an explanation earlier.}
To lower bound the right-hand-side of the last expression, the following lemma comes in handy, with its proof postponed to Appendix \ref{app:proof-lemma-E-delta-tmix}.
\begin{customlemma}\label{lemma:E-delta-tmix}
For the TD iterations \eqref{eq:TD-update-all} with Markov samples and non-increasing stepsizes $\eta_1 \geq ... \geq \eta_T$ it holds that, for all $t \geq t_{\mix}$, %\ale{add brief proof?}\weichen{added a reference to the section of its proof.}
\begin{subequations}
\begin{align}
&\mathbb{E}\|\bm{\theta}_{t-1} - \bm{\theta}_{t-t_{\mix}}\|_2 \leq t_{\mix}\eta_{t-t_{\mix}}(2\|\bm{\theta^\star}\|_2 + 1)+ 2\eta_{t - t_{\mix}}\sum_{i=t-t_{\mix}}^{t-2}\mathbb{E}\|\bm{\Delta}_i\|_2; \label{eq:E-delta-tmix-1}\\ 
&\mathbb{E}\|\bm{\theta}_{t-1} - \bm{\theta}_{t-t_{\mix}}\|_2^2 \leq 2t_{\mix}\eta_{t-t_{\mix}}^2\left[t_{\mix}(2\|\bm{\theta}^\star\|_2+1)^2 + 4 \sum_{i=t-t_{\mix}}^{t-2} \mathbb{E}\|\bm{\Delta}_i\|_2^2\right]; \label{eq:E-delta-tmix-2}\\ 
&\mathbb{E}[\|\bm{\Delta}_{t-t_{\mix}}\|_2 \|\bm{\theta}_{t-1} - \bm{\theta}_{t-t_{\mix}}\|_2] \nonumber \\ 
&\leq t_{\mix}\eta_{t-t_{\mix}}(2\|\bm{\theta}^\star\|_2 +1)\mathbb{E}\|\bm{\Delta}_{t- t_{\mix}}\|_2+ \eta_{t - t_{\mix}}\sum_{i=t-t_{\mix}}^{t-2}\mathbb{E}\|\bm{\Delta}_{i}\|_2^2 + t_{\mix} \eta_{t - t_{\mix}} \mathbb{E}\|\bm{\Delta}_{t- t_{\mix}}\|_2^2.\label{eq:E-delta-tmix-3}
\end{align}
\end{subequations}
\end{customlemma}

Lemma \ref{lemma:E-delta-tmix} directly leads to  
\begin{align}\label{eq:markov-L2-I3-bound}
I_3 &\geq -2\eta_t\mathbb{E}\|\bm{\Delta}_{t-t_{\mix}}\|_2^2 - 4t_{\mix}^2 \eta_{t-t_{\mix}}^2(2\|\bm{\theta^\star}\|_2 + 1)^2 - 16 t_{\mix} \eta_{t-t_{\mix}}^2\sum_{i=t-t_{\mix}}^{t-2}\mathbb{E}\|\bm{\Delta}_i\|_2^2 \nonumber \\ 
&- 4t_{\mix}\eta_{t-t_{\mix}}(2\|\bm{\theta}^\star\|_2 +1)\mathbb{E}\|\bm{\Delta}_{t- t_{\mix}}\|_2 - 4\eta_{t - t_{\mix}}\sum_{i=t-t_{\mix}}^{t-2}\mathbb{E}\|\bm{\Delta}_i\|_2^2  - 4t_{\mix} \eta_{t - t_{\mix}} \mathbb{E}\|\bm{\Delta}_{t- t_{\mix}}\|_2^2 \nonumber \\ 
&= -(2\eta_t +4t_{\mix}\eta_{t-t_{\mix}})\mathbb{E}\|\bm{\Delta}_{t-t_{\mix}}\|_2^2 - (16 t_{\mix} \eta_{t-t_{\mix}}^2+ 4\eta_{t - t_{\mix}})\sum_{i=t-t_{\mix}}^{t-2}\mathbb{E}\|\bm{\Delta}_i\|_2^2 \nonumber \\ 
&-4t_{\mix}^2 \eta_{t-t_{\mix}}^2(2\|\bm{\theta^\star}\|_2 + 1)^2 -4t_{\mix}\eta_{t-t_{\mix}}(2\|\bm{\theta}^\star\|_2 +1)\mathbb{E}\|\bm{\Delta}_{t- t_{\mix}}\|_2.
\end{align}

\paragraph{Bounding $I_4$.} Similarly, we decompose $I_4$ as
\begin{align*}
I_4 &=  \mathbb{E}[\bm{\Delta}_{t-t_{\mix}}^\top (\bm{A}_t \bm{\theta}^\star - \bm{b}_t)] + \mathbb{E}[(\bm{\Delta}_{t-1} - \bm{\Delta}_{t-t_{\mix}})^\top (\bm{A}_t \bm{\theta}^\star - \bm{b}_t)]\\ 
&= \mathbb{E}[\bm{\Delta}_{t-t_{\mix}}^\top (\bm{A}_t \bm{\theta}^\star - \bm{b}_t)] + \mathbb{E}[(\bm{\theta}_{t-1} - \bm{\theta}_{t-t_{\mix}})^\top (\bm{A}_t \bm{\theta}^\star - \bm{b}_t)].
\end{align*}
The first term can be bounded using  the $t_{\mix}$ separation: %\ale{Here we need to define $\mathbb{E}_{t-t_{\mix}}[\cdot]$}\weichen{yes, at the beginning of the section}.
\begin{align*}
&|\mathbb{E}[\bm{\Delta}_{t-t_{\mix}}^\top (\bm{A}_t \bm{\theta}^\star - \bm{b}_t)]| \\ 
&= |\mathbb{E} [\mathbb{E}_{t-t_{\mix}}[\bm{\Delta}_{t-t_{\mix}}^\top (\bm{A}_t \bm{\theta}^\star - \bm{b}_t)]]|\\ 
&= |\mathbb{E} [\mathbb{E}_{s_{t-1} \sim P^{t_{\mix}-1}(\cdot \mid s_{t-t_{\mix}})}[\mathbb{E}_{s_t \sim P(\cdot \mid s_{t-1})}[\bm{\Delta}_{t-t_{\mix}}^\top(\bm{A}_t \bm{\theta}^\star - \bm{b}_t)]] -  \mathbb{E}_{s_{t-1}\sim \mu }[\mathbb{E}_{s_t \sim P(\cdot \mid s_{t-1}) }[\bm{\Delta}_{t-t_{\mix}}^\top (\bm{A}_t \bm{\theta}^\star - \bm{b}_t)]]]|\\
&\leq \mathbb{E} \sup_{s_{t-1}} |\mathbb{E}_{s_t \sim P(\cdot \mid s_{t-1})}\bm{\Delta}_{t-t_{\mix}}^\top(\bm{A}_t \bm{\theta}^\star - \bm{b}_t)| \cdot d_{\mathsf{TV}}(P^{t_{\mix}-1}(\cdot \mid s_{t-t_{\mix}}),\mu)\\
&\leq  \mathbb{E}\|\bm{\Delta}_{t-t_{\mix}}\|_2 (2\|\bm{\theta}^\star\|_2 + 1) \cdot \eta_t,
\end{align*}
%\ale{where does $\eta_t$ come from in the last equation? } \weichen{please see above.}
while the second term can be bounded by stepsizes:
\begin{align*}
&\mathbb{E}[(\bm{\theta}_{t-1} - \bm{\theta}_{t-t_{\mix}})^\top (\bm{A}_t \bm{\theta}^\star - \bm{b}_t)]\\ 
&\geq -(2\|\bm{\theta}^\star\|_2 + 1)\mathbb{E}\|\bm{\theta}_{t-1} - \bm{\theta}_{t-t_{\mix}}\|_2 \\ 
&\geq -(2\|\bm{\theta}^\star\|_2 + 1) \left[t_{\mix} \eta_{t-t_{\mix}} (2\|\bm{\theta}^\star\|_2 + 1) + 2\eta_{t-t_{\mix}} \sum_{i=t-t_{\mix}}^{t-2} \mathbb{E}\|\bm{\Delta}_i\|_2\right]\\ 
&= -\eta_{t-t_{\mix}}(2\|\bm{\theta}^\star\|_2 + 1) \left[t_{\mix}(2\|\bm{\theta}^\star\|_2 + 1) + 2\sum_{i=t-t_{\mix}}^{t-2} \mathbb{E}\|\bm{\Delta}_i\|_2 \right].
\end{align*}
Therefore, $I_4$ can be bounded by
\begin{align}\label{eq:markov-L2-I4-bound}
I_4 &\geq -\eta_t \mathbb{E}\|\bm{\Delta}_{t-t_{\mix}}\|_2 (2\|\bm{\theta}^\star\|_2 + 1) \nonumber \\ 
&-\eta_{t-t_{\mix}}(2\|\bm{\theta}^\star\|_2 + 1) \left[t_{\mix}(2\|\bm{\theta}^\star\|_2 + 1) + 2\sum_{i=t-t_{\mix}}^{t-2} \mathbb{E}\|\bm{\Delta}_i\|_2 \right].
\end{align}
\paragraph{Combining terms.} With $I_3$ and $I_4$ bounded, we now return to Equation \eqref{eq:markov-L2-iter}. $\mathbb{E}\|\bm{\Delta}_{t}\|_2^2$ can be upper bounded by 
\begin{align}\label{eq:markov-L2-iter-bound}
\mathbb{E}\|\bm{\Delta}_{t}\|_2^2 &\leq I_1 + I_2 - 2\eta_t(I_3 + I_4) \nonumber \\ 
&\leq [1-2\lambda_0(1-\gamma)\eta_t + 2\eta_t^2(1+\gamma)^2]\mathbb{E}\|\bm{\Delta}_{t-1}\|_2^2 + 2\eta_t^2(2\|\bm{\theta}\|_2+1)^2 \nonumber \\
&+2\eta_t (2\eta_t +4t_{\mix}\eta_{t-t_{\mix}})\mathbb{E}\|\bm{\Delta}_{t-t_{\mix}}\|_2^2 + 2\eta_t (16 t_{\mix} \eta_{t-t_{\mix}}^2+ 4\eta_{t - t_{\mix}})\sum_{i=t-t_{\mix}}^{t-2}\mathbb{E}\|\bm{\Delta}_i\|_2^2 \nonumber \\ 
&+8\eta_t t_{\mix}^2 \eta_{t-t_{\mix}}^2 (2\|\bm{\theta^\star}\|_2 + 1)^2 + 8\eta_t t_{\mix}\eta_{t-t_{\mix}}(2\|\bm{\theta}^\star\|_2 +1)\mathbb{E}\|\bm{\Delta}_{t- t_{\mix}}\|_2 \nonumber \\ 
&+ 2\eta_t^2 \mathbb{E}\|\bm{\Delta}_{t-t_{\mix}}\|_2 (2\|\bm{\theta}^\star\|_2 + 1) +2\eta_t \eta_{t-t_{\mix}}(2\|\bm{\theta}^\star\|_2 + 1)t_{\mix}(2\|\bm{\theta}^\star\|_2 + 1) \nonumber \\ 
&+4\eta_t \eta_{t-t_{\mix}}(2\|\bm{\theta}^\star\|_2 + 1)\sum_{i=t-t_{\mix}}^{t-2} \mathbb{E}\|\bm{\Delta}_i\|_2 .
\end{align}
\paragraph{Specifying the polynomially-decaying stepsizes.} With polynomially-decaying stepsizes, when $t$ is sufficiently large, it can be guaranteed that $t > 2t_{\mix}$, and therefore $\eta_{t-t_{\mix}} \geq 2^{-\alpha} \eta_t$. Furthermore, for sufficiently large $t$, $\eta_t (1+\gamma)^2 < \lambda_0(1-\gamma)$. Therefore, by dividing $(2\|\bm{\theta}^\star\|_2+1)^2$ on both sides and combining terms, we can simplify Equation \eqref{eq:markov-L2-iter-bound} as
\begin{align}\label{eq:markov-L2-iter-simplify}
\frac{\mathbb{E}\|\bm{\Delta}_{t}\|_2^2}{(2\|\bm{\theta}^\star\|_2+1)^2} &\leq (1-\widetilde{C}_1t^{-\alpha})\frac{\mathbb{E}\|\bm{\Delta}_{t-1}\|_2^2}{(2\|\bm{\theta}^\star\|_2+1)^2}  + \widetilde{C}_2 t^{-2\alpha}\log^2 t + \widetilde{C}_3 t^{-2\alpha} \log t \frac{\mathbb{E}\|\bm{\Delta}_{t-t_{\mix}}\|_2^2}{{(2\|\bm{\theta}^\star\|_2+1)^2} } \nonumber \\ 
&+ \widetilde{C}_4 t^{-2\alpha} \log t \sum_{i=t-t_{\mix}}^{t-2}\frac{\mathbb{E}\|\bm{\Delta}_i\|_2^2}{{(2\|\bm{\theta}^\star\|_2+1)^2} } + \widetilde{C}_5 t^{-2\alpha} \sum_{i=t-t_{\mix}}^{t-2}\frac{\mathbb{E}\|\bm{\Delta}_i\|_2}{2\|\bm{\theta}^\star\|_2+1} ,
\end{align}
where $\widetilde{C}_1$ throught $\widetilde{C}_5$ are constants depending on $\alpha,\eta_0,m$ and $\rho$. Notice that the $\log t$ terms occur due to $t_{\mix} = O(\log t)$; see \eqref{eq:tmix-bound-L2}. We will use an induction argument based on the relation \eqref{eq:markov-L2-iter-simplify}. For simplicity, let
\begin{align*}
X_t = \frac{\|\bm{\Delta}_t\|_2}{2\|\bm{\theta}^\star\|_2+1};
\end{align*}
now suppose that
\begin{align}\label{eq:markov-L2-induction-assumption}
\mathbb{E}[X_t^2] \leq \widetilde{C} \cdot \frac{\log^2 t}{t^{\alpha}}, \quad \forall 1 < t \leq k,
\end{align}
for some $\widetilde{C}$. 
% \ale{above we must have $t>0$} \weichen{checked.} \yuting{check the base case. $t=1$ does not work either?}\weichen{Yes.... this is annoying. Used $t > 1$ instead.}



Our goal is to demonstrate, inductively, that
\begin{align}\label{eq:markov-L2-induction-goal}
\mathbb{E}[X_{k+1}^2] \leq \widetilde{C} \cdot \frac{\log^2 (k+1)}{(k+1)^{\alpha}}.
\end{align}
%\ale{is this the induction relation?}\weichen{yes, just added the word "inductively".}
Towards this end, the iterative relation \eqref{eq:markov-L2-iter-simplify} implies that
\begin{align}\label{eq:markov-L2-induction-1}
\mathbb{E}[X_{k+1}^2] &\leq \left(1-\widetilde{C}_1(k+1)^{-\alpha}\right)\mathbb{E}[X_k^2] + \widetilde{C}_2 (k+1)^{-2\alpha}\log^2(k+1) \nonumber \\ 
&+ \widetilde{C}_3(k+1)^{-2\alpha} \log (k+1) \mathbb{E}[X_{k+1-t_{\mix}}^2] \nonumber \\ 
&+ \widetilde{C}_4(k+1)^{-2\alpha} \log(k+1) \sum_{i=k+1-t_{\mix}}^{k-1} \mathbb{E}[X_i^2] \nonumber \\ 
&+ \widetilde{C}_5(k+1)^{-2\alpha}\sum_{i=k+1-t_{\mix}}^{k-1} \mathbb{E}[X_i].
\end{align}
Here, the induction assumption guarantees that, as long as $k > 2t_{\mix}$, 
\begin{align*}
&\mathbb{E}[X_k^2] \leq \widetilde{C} k^{-\alpha} \log^2 k, \\ 
&\mathbb{E}[X_{k+1-t_{\mix}}^2] \leq \widetilde{C} (k+1-t_{\mix})^{-\alpha} \log^2(k+1-t_{\mix}) < 2^{-\alpha} \widetilde{C} (k+1)^{-\alpha} \log^2(k+1), 
\end{align*}
and that 
\begin{align*}
\sum_{i=k+1-t_{\mix}}^{k-1} \mathbb{E}[X_i] &\leq t_{\mix} \cdot \widetilde{C} (k+1-t_{\mix})^{-\frac{\alpha}{2}} \log(k+1-t_{\mix}) \\ 
&\lesssim \widetilde{C} \cdot (k+1)^{-\frac{\alpha}{2}} \log^2(k+1),\\ 
\sum_{i=k+1-t_{\mix}}^{k-1} \mathbb{E}[X_i^2] &\leq t_{\mix} \cdot \widetilde{C} (k+1-t_{\mix})^{-\alpha} \log^2(k+1-t_{\mix}) \\ 
&\lesssim \widetilde{C} \cdot (k+1)^{-\alpha} \log^3(k+1).
\end{align*}
Plugging these inequalities into the iteration relation \eqref{eq:markov-L2-induction-1}, we obtain that for sufficiently large $k$,
\begin{align*}
\mathbb{E}[X_{k+1}^2] &\leq \widetilde{C} \cdot \left[k^{-\alpha} \log^2 k - \widetilde{C}_1(k+1)^{-\alpha} k^{-\alpha} \log^2 k + \widetilde{C}_3 (k+1)^{-\frac{5}{2}\alpha}\log^2(k+1)\right]\\ 
& + \widetilde{C}_2(k+1)^{-2\alpha} \log^2(k+1).
\end{align*}
Here, $\widetilde{C}_1, \widetilde{C}_2$ and $\widetilde{C}_3$ are again constants independent of $t$, with there exact values can change from \eqref{eq:markov-L2-induction-1}. Therefore, it suffices to prove that
\begin{align}\label{eq:markov-L2-induction-2}
&\widetilde{C} \cdot \left[k^{-\alpha} \log^2 k - \widetilde{C}_1(k+1)^{-\alpha} k^{-\alpha} \log^2 k + \widetilde{C}_3 (k+1)^{-\frac{5}{2}\alpha}\log^2(k+1)\right]\nonumber \\ 
& + \widetilde{C}_2(k+1)^{-2\alpha} \log^2(k+1) \leq \widetilde{C} (k+1)^{-\alpha} \log^2(k+1).
\end{align}
Notice that when $x$ is sufficiently large, the function $f(x) = x^{-\alpha}\log^2(x)$ is monotonically decreasing; therefore, for sufficiently large $k$, it can be guaranteed that $k^{-\alpha} \log^2 k > (k+1)^{-\alpha} \log (k+1)$. Therefore, the left-hand-side of \eqref{eq:markov-L2-induction-2} is upper bounded by
\begin{align*}
&\widetilde{C} \cdot \left[k^{-\alpha} \log^2 k - \widetilde{C}_1(k+1)^{-\alpha} k^{-\alpha} \log^2 k + \widetilde{C}_3 (k+1)^{-\frac{5}{2}\alpha}\log^2(k+1)\right]\\ 
 &+\widetilde{C}_2(k+1)^{-2\alpha} \log^2(k+1) \\ 
&\leq \widetilde{C} \cdot \left[k^{-\alpha} \log^2 (k+1) - \widetilde{C}_1(k+1)^{-\alpha} (k+1)^{-\alpha} \log^2 (k+1) + \widetilde{C}_3 (k+1)^{-\frac{5}{2}\alpha}\log^2(k+1)\right]\\ 
 &+\widetilde{C}_2(k+1)^{-2\alpha} \log^2(k+1) \\ 
 &= \widetilde{C} \log^2(k+1) \cdot \left[k^{-\alpha} +\left(\frac{\widetilde{C}_2}{\widetilde{C}}- \widetilde{C}_1\right)(k+1)^{-2\alpha} + \widetilde{C}_3 (k+1)^{-\frac{5}{2}\alpha} \right].
\end{align*}
Hence, in order to prove \eqref{eq:markov-L2-induction-2}, it suffices to show that
\begin{align*}
k^{-\alpha} +\left(\frac{\widetilde{C}_2}{\widetilde{C}}- \widetilde{C}_1\right)(k+1)^{-2\alpha} + \widetilde{C}_3 (k+1)^{-\frac{5}{2}\alpha} \leq (k+1)^{-\alpha},
\end{align*}
which is equivalent to
\begin{align*}
(k+1)^{2\alpha}\left[k^{-\alpha} - (k+1)^{-\alpha}\right] + \widetilde{C}_3 (k+1)^{-\frac{\alpha}{2}} \leq \widetilde{C}_1 - \frac{\widetilde{C}_2}{\widetilde{C}}.
\end{align*}
Here, we further notice that the function $f(x) = x^{-\alpha}$ is monotonically decreasing and convex, so $k^{-\alpha}-(k+1)^{-\alpha} = f(k) -f(k+1) \leq -f'(k+1) = \alpha(k+1)^{-\alpha-1}$. Hence, the proof boils down to showing
\begin{align*}
\widetilde{C}_1 - \frac{\widetilde{C}_2}{\widetilde{C}} &\geq (k+1)^{2\alpha}\cdot \alpha(k+1)^{-\alpha-1} + \widetilde{C}_3 (k+1)^{-\frac{\alpha}{2}} \\ 
&= \alpha(k+1)^{\alpha-1} + \widetilde{C}_3 (k+1)^{-\frac{\alpha}{2}}
\end{align*}
for an appropriate $\widetilde{C}$ that is independent of $t$ and satisfies the induction assumption \eqref{eq:markov-L2-induction-assumption}. 
Towards this end, we define a function $f(\widetilde{C},k)$ as
\begin{align*}
f(\widetilde{C}, k):= \widetilde{C}_1 - \frac{\widetilde{C}_2}{\widetilde{C}} - \alpha(k+1)^{\alpha-1}- \widetilde{C}_3 (k+1)^{-\frac{\alpha}{2}}
\end{align*}
It is easy to verify that for any $\widetilde{C}$,
\begin{align*}
\lim_{k \to \infty} f(\widetilde{C}, k) = \widetilde{C}_1 - \frac{\widetilde{C}_2}{\widetilde{C}}.
\end{align*}
Therefore, we can take 
\begin{align*}
&k^\star = \min\left\{k:f\left(\max_{1 \leq t \leq k} \frac{t^\alpha}{\log^2 t} \mathbb{E}[X_t^2], k\right) \geq 0 \right\}, \quad \text{and} \\
&\widetilde{C} = \max_{1 \leq t \leq k^\star} \frac{t^\alpha}{\log^2 t} \mathbb{E}[X_t^2].
\end{align*}
On one hand, if $k^\star$ does not exist, then from our analysis we can conclude that 
\begin{align*}
\mathbb{E}[X_t^2] \leq \frac{\widetilde{C}_2}{\widetilde{C}_1} \frac{\log^2 t}{t^{\alpha}}
\end{align*}
for all $t \geq 1$; on the other hand, if $k^\star$ does exist, then an induction argument guarantees that
\begin{align*}
\mathbb{E}[X_t^2] \leq \widetilde{C} \frac{\log^2 t}{t^{\alpha}}
\end{align*}
for all $t \geq 1$. In both cases, \eqref{eq:markov-L2-induction-goal} holds true. 
\paragraph{Specification of the coefficient.} We next try to specify the coefficient corresponding to the leading term of the upper bound. In the previous paragraph, we have essentially proved that, there exists a $t^\star \in \mathbb{N}$ depending on $\alpha,\eta_0,\lambda_0,m$ and $\rho$ such that
\begin{align*}
\mathbb{E}[X_t^2] \leq 1, \quad \text{for all }\quad t \geq t^\star.
\end{align*}
Hence, when $t > t^\star$, a closer examination of \eqref{eq:markov-L2-iter-bound} yields
\begin{align*}
\mathbb{E}[X_t^2] \leq (1-\lambda_0(1-\gamma)\eta_t) \mathbb{E}[X_{t-1}^2] + C  \frac{\eta_0^2}{(1-\rho)^2} t^{-2\alpha}\log^2 t
\end{align*}
for a \emph{universal constant} $C$. Hence by iteration, it can be guaranteed that when $t > t^\star$,
\begin{align*}
\mathbb{E}[X_t^2] \leq \underset{I_1}{\underbrace{\prod_{i=t^\star+1}^t (1-\beta i^{-\alpha})X_{t^\star}}} + C  \underset{I_2}{\underbrace{\frac{\eta_0^2}{(1-\rho)^2}\sum_{i=t^\star}^t (i^{-2\alpha} \log^2 i)\prod_{k=i+1}^t (1-\beta k^{-\alpha})}}.
\end{align*}
Here, it is easy to verify that $I_1$ converges exponentially with respect to $t$, and that $I_2$ is upper bounded by
\begin{align*}
I_2 &\leq \frac{\eta_0^2}{(1-\rho)^2} \log^2 t \sum_{i=t^\star}^t i^{-2\alpha} \prod_{k=i+1}^t (1-\beta k^{-\alpha}) \\ 
&\leq \frac{\eta_0^2}{(1-\rho)^2} \log^2 t \sum_{i=1}^t i^{-2\alpha} \prod_{k=i+1}^t (1-\beta k^{-\alpha}) \\ 
&\overset{(i)}{=} \frac{\eta_0^2}{(1-\rho)^2} \log^2 t \left(\frac{1}{\beta} t^{-\alpha} + O(t^{-1})\right)\\ 
&= \frac{\eta_0}{(1-\rho)^2 \lambda_0(1-\gamma)} t^{-\alpha}\log^2 t + o(t^{-\alpha}\log^2 t).
\end{align*}
The theorem follows immediately.
% \paragraph{Specify to Scenario \ref{case:nu}.} The convergence of $\mathbb{E}\|\bm{\Delta}_t\|_2^2$ under Scenario \ref{case:nu} can be guaranteed by an induction argument similar to our reasoning under Scenario \ref{case:alpha}, and the details are omitted.
% \paragraph{Specify to Scenario \ref{case:eta}.} By replacing $\eta_t$, $\eta_{t-t_{\mix}}$ by $\eta_0$ and inductively bounding $\mathbb{E}\|\bm{\Delta}_i\|_2^2$ by $(2\|\bm{\theta}^\star\|_2+1)^2$ for all $i < t$, the iterative relation \eqref{eq:markov-L2-iter-bound} is translated to
% \begin{align*}
% \mathbb{E}\|\bm{\Delta}_t\|_2^2 &\leq (1-2\lambda_0(1-\gamma)\eta) (2\|\bm{\theta}^\star\|_2+1)^2 \\ 
% &+ \eta^2 (2\|\bm{\theta}^\star\|_2+1)^2 \cdot \bigg\{2(1+\gamma)^2 + 2 + 4 + 8\tmix + 32\eta \tmix^2 + 8\tmix \\ 
% &\qquad \qquad \qquad \qquad \quad + 8\eta \tmix^2 + 8\tmix + 2 + 2\tmix + 4\tmix \bigg\};
% \end{align*}
% furthermore, the condition \eqref{eq:S3-eta-condition-L2} guarantees $\eta \tmix \ll 1$, and hence
% \begin{align*}
% \mathbb{E}\|\bm{\Delta}_t\|_2^2 &\leq (1-2\lambda_0(1-\gamma)\eta) (2\|\bm{\theta}^\star\|_2+1)^2 + 86\eta^2 \tmix (2\|\bm{\theta}^\star\|_2+1)^2 \\ 
% &\leq (2\|\bm{\theta}^\star\|_2+1)^2,
% \end{align*}
% where the second inequality is also implied by condition \eqref{eq:S3-eta-condition-L2}.

\end{proof}


\subsection{High-probability convergence guarantee for the original TD estimation error}\label{app:proof-TD-original}
Similar to the case with $i.i.d.$ samples, we firstly state the following theorem for the high-probability convergence rate for the original TD estimation error $\bm{\Delta}_t$ with Markov samples.
\begin{theorem}
\label{thm:markov-deltat-convergence}
Consider TD with Polyak-Ruppert averaging~\eqref{eq:TD-update-all} with Markov samples and decaying stepsizes $\eta_t = \eta_0 t^{-\alpha}$ for $\alpha \in (\frac{1}{2},1)$. Suppose that the Markov transition kernel has a unique stationary distribution $\mu$, a strictly positive spectral gap $1-\lambda > 0$, and Assumption \ref{as:mixing} holds true. Then for any $\delta \in (0,1)$, there exists $\eta_0 > 0$, such that with probability at least $1-{\delta}$,
\begin{align*}
\|\bm{\Delta}_t\|_2 &\leq \frac{13}{2}\|\bm{\theta}^\star\|_2 + \frac{5}{4} \quad \text{and}\\
\|\bm{\Delta}_t\|_2 &\lesssim \eta_0 \sqrt{\frac{2\tmix(t^{-\frac{\alpha}{2}})}{(2\alpha-1)}\log \frac{9T\tmix(t^{-\frac{\alpha}{2}})}{\delta}}(2\|\bm{\theta}^\star\|_2  + 1)\left(\frac{(1-\gamma)\lambda_0 \eta_0}{4\alpha}\right)^{-\frac{\alpha}{2(1-\alpha)}}t^{-\frac{\alpha}{2}}
\end{align*}
hold simultaneously for all $t \in [T]$.
\end{theorem}

\begin{proof}
Recalling the TD update rule \eqref{eq:TD-update-all}, we represent $\bm{\Delta}_t$ as
\begin{align*}
\bm{\Delta}_t = \bm{\theta}_t - \bm{\theta}^\star
&= (\bm{\theta}_{t-1}-\eta_t(\bm{A}_{t}\bm{\theta}_{t-1}-\bm{b}_{t})) - \bm{\theta}^\star\\
&= \bm{\Delta}_{t-1} - \eta_t (\bm{A}\bm{\theta}_{t-1}-\bm{b} + \bm{\zeta}_{t})\\
&= (\bm{I}-\eta_t \bm{A})\bm{\Delta}_{t-1} -\eta_t \bm{\zeta}_{t},
\end{align*}
where $\bm{\zeta}_t$ is defined as
\begin{align}\label{eq:defn-zetat}
\bm{\zeta}_t := (\bm{A}_t -\bm{A})\bm{\theta}_{t-1} - (\bm{b}_t-\bm{b}).
\end{align}
Therefore by induction, $\bm{\Delta}_t$ can be expressed as a weighted sum of $\{\bm{\zeta}_i\}_{0 \leq i < t}$, namely
\begin{align}
& \bm{\Delta}_t = \prod_{k=1}^{t} (\bm{I}-\eta_k \bm{A}) \bm{\Delta}_0 -\sum_{i=0}^{t-1} \bm{R}_i^t \bm{\zeta}_i,  \label{eq:delta-t-markov} \\ 
&  \text{where} \quad \bm{R}_i^t = \eta_i \prod_{k=i+1}^{t-1} (\bm{I}-\eta_k \bm{A}).
\end{align}

The difficulty in bounding the second term of \eqref{eq:delta-t-markov} lies in the fact that with Markov samples, $\{\bm{\zeta_i}\}_{i > 0}$ is no longer a martingale difference process. Therefore, we further decompose $\bm{\zeta}_i$ into three parts, namely
%In order to characterize the second term under Markov samples, we further decompose $\bm{\zeta}_i$ into three parts, namely
\begin{align}\label{eq:markov-zetai-decompose}
\bm{\zeta}_i = \mathbb{E}_{i_{\mix}}[\bm{\zeta}_{i,\mix}] + (\bm{\zeta}_{i,\mix} - \mathbb{E}_{i_{\mix}}[\bm{\zeta}_{i,\mix}]) + (\bm{\zeta}_i - \bm{\zeta}_{i,\mix}).
\end{align}
%\ale{What is $\mathbb{E}_{i_{\mix}}$? The conditional expectation given $s_{i_{\mix}}$?}\weichen{Yes, added notation at the beginning of the section.}
In order to simplify notation, throughout this proof we denote
\begin{align*}
t_{\mix} = t_{\mix}(\varepsilon) + 1,
\end{align*}
where $\varepsilon \in (0,1)$ is to be specified later. 
Furthermore, for every $t > 0$, we denote
\begin{align} 
&i_{\mix} = \max\{0,i - t_{\mix}(\varepsilon)\}, \quad \text{and} \label{eq:defn-t-imix}\\ 
&\bm{\zeta}_{i,\mix} = (\bm{A}_i - \bm{A})\bm{\theta}_{i_{\mix}} - (\bm{b}_i - \bm{b}),\label{eq:defn-zeta-imix}
\end{align}
%\yuting{$t_{\mix}(\varepsilon)$ should be defined earlier when you first use this.} \weichen{changed to the main text.}

The intuition behind the construction of $\bm{\zeta}_{i,\mix}$ is to guarantee that the samples $(\bm{A}_i,\bm{b}_i)$ and the iterated estimator $\bm{\theta}_{i_{\mix}}$ are separated in the Markov chain by at least $t_{\mix}$ samples, so that their distributions are close to independent. Recall from \eqref{eq:tmix.bound} that the mixing property of the Markov chain featured by Assumption \ref{as:mixing} guarantees
\begin{align}
\tmix \leq \frac{ \log (m/\varepsilon)}{\log (1/\rho)} + 1.
\end{align}

Furthermore, the difference bewteen $\bm{\zeta}_i $ and $\bm{\zeta}_{i,\mix}$ is 
\begin{align}
\bm{\zeta}_i - \bm{\zeta}_{i,\mix} = (\bm{A}_i - \bm{A}) (\bm{\theta}_{i-1} - \bm{\theta}_{i,\mix}).
\end{align}
Therefore, with the decomposition \eqref{eq:markov-zetai-decompose}, $\bm{\Delta}_t$ can be characterized as
\begin{align}\label{eq:markov-deltat-decompose}
\bm{\Delta}_t &= \underset{I_1}{\underbrace{\prod_{k=1}^t (\bm{I}-\eta_k \bm{A}) \bm{\Delta}_0}} - \underset{I_2}{\underbrace{\sum_{i=1}^t \bm{R}_i^t \mathbb{E}_{i_{\mix}}[\bm{\zeta}_{i,\mix}]}} - \underset{I_3}{\underbrace{\sum_{i=1}^t \bm{R}_i^t (\bm{A}_i - \bm{A}) (\bm{\theta}_{i-1} - \bm{\theta}_{i,\mix})}} \nonumber \\ 
&- \underset{I_4}{\underbrace{\sum_{i=1}^t \bm{R}_i^t (\bm{\zeta}_{i,\mix} - \mathbb{E}_{i_{\mix}}[\bm{\zeta}_{i,\mix}])}}.
\end{align}

In what follows, we denote
\begin{align*}
& \beta=\frac{1-\gamma}{2}\lambda_0\eta_0,\\ %\quad
& R = \frac{13}{2}\|\bm{\theta}^\star\|_2 + \frac{5}{4}, \quad \text{and}\\
& \mathcal{H}_t = \left\{\max_{1 \leq i \leq t}\|\bm{\Delta}_i\|_2 \leq R\right\}.
\end{align*}
Furthermore, for any given $\delta$, let\footnote{Notice that the existence of $t^\star(\delta)$ is guaranteed by a similar reasoning to that in the proof of Theorem B.1 in \cite{wu2024statistical}.}
\begin{align}\label{eq:defn-tstar-markov}
t^\star = t^\star(\delta):= \inf\Bigg\{t \in \mathbb{N}^+: &\int_t^{\infty} \exp \left(-\frac{2\alpha-1}{2^{11}\eta_0^2} \frac{\log(1/\rho)}{\log(8m\eta_0/\beta)}\left(\frac{\beta}{2\alpha}\right)^{\frac{\alpha}{1-\alpha}}x^{-\alpha}\right)\mathrm{d}x \nonumber \\ 
&\leq \frac{\delta}{27} \cdot \frac{\log(8m\eta_0/\beta)}{\log(1/\rho)}\Bigg\},
\end{align}
and assume that 
\begin{align}\label{eq:deltat-condition-markov}
\eta_0 \sqrt{\frac{2}{2\alpha-1} \frac{\log(8m\eta_0/\beta)}{\log(1/\rho)}} \max\left\{\frac{1}{4\sqrt{1-\alpha}},\log \frac{9\log(8m\eta_0/\beta)t^\star}{\log(1/\rho)\delta}\right\}\leq \frac{1}{32}.
\end{align}
We break down the proof of the theorem into a sequence of steps:
\begin{enumerate}
\item We obtain convergence rates for the four terms on the right-hand-side of \eqref{eq:markov-deltat-decompose};
\item We lower bound  the probability of $\mathcal{H}_{t^\star}$ by %$\mathbb{P}(\mathcal{H}_{t^\star})$ 
 $1-\frac{\delta}{3}$;
\item We lower bound the probability of $\mathcal{H}_{\infty}$ by %$\mathbb{P}(\mathcal{H}_{\infty}) \geq$ 
  $1-\frac{2\delta}{3}$;
\item Using the results from the first steps, we arrive at a final bound on $\|\bm{\Delta}_t\|_2$.
\end{enumerate}

\paragraph{Step 1: Basic convergence properties of the four terms on the right-hand-side of \eqref{eq:markov-deltat-decompose}.} As is shown in the proof of Theorem B.1 in \cite{wu2024statistical}, the norm of $I_1$ is bounded by
\begin{align}\label{eq:markov-deltat-I1-bound}
\left\|\prod_{k=0}^{t-1} (\bm{I}-\eta_k \bm{A}) \bm{\Delta}_0\right\| \leq \left(\frac{1-\gamma}{2}\lambda_0 \eta_0\right)^{-\frac{\alpha}{1-\alpha}} t^{-\alpha} \|\bm{\Delta}_0\|_2.
\end{align}
For the term $I_2$, we observe that for $i \leq t_{\mix}$, since $i_{\mix} = 0$, 
\begin{align*}
\mathbb{E}_{i_{\mix}}\left[\bm{\zeta}_{i,\mix}\right]= \mathbb{E}_0 [(\bm{A}_i-\bm{A})\bm{\theta}_0 - (\bm{b}_i - \bm{b})] = \bm{0};
\end{align*}
otherwise when $i > t_{\mix}$, since $i_{\mix} = i-t_{\mix}$, 
\begin{align}\label{eq:E-zeta-imix}
\left\|\mathbb{E}_{i_\mix}[\bm{\zeta}_{i,\mix}]\right\|_2&\leq \left\| (\mathbb{E}_{i_{\mix}}[\bm{A}_i] - \mathbb{E}[\bm{A}_i]) \bm{\theta}_{i_{\mix}}\right\|_2  + \left\|\mathbb{E}_{i_{\mix}}[\bm{b}_i] - \mathbb{E}[\bm{b}_i]\right\|_2\nonumber \\
&\leq d_{\text{TV}}(P^{t_{\mix}}(\cdot|s_{i_{\mix}}),\mu) \left(\sup_{s_{i-1},s_i \in \mathcal{S}} \| \bm{A}_i \bm{\theta}_{i_{\mix}} \|_2 + \sup_{s_i \in \mathcal{S}} \|\bm{b}_i\|_2\right)\nonumber \\
&\leq (2\max_{1 \leq i < t}\|\bm{\Delta}_i\|_2+ 2\|\bm{\theta}^\star\|_2 + 1) d_{\text{TV}}(P^{t_{\mix}}(\cdot|s_{i_{\mix}}),\mu) \nonumber \\ 
&\leq \varepsilon (2\max_{1 \leq i < t}\|\bm{\Delta}_i\|_2+ 2\|\bm{\theta}^\star\|_2 + 1).
\end{align}
%\ale{in the first line above, replace $b$ with $\mathbb{E}[b_i]$}
Meanwhile, we observe that the sum of $\|\bm{R}_i^t\|$ can be bounded by %\yuting{use consistent notation for spectral norm} \weichen{checked.} 
\begin{align*}
\sum_{i=1}^t \|\bm{R}_i^t\| &\leq \sum_{i=1}^t \eta_i \prod_{k=i+1}^t \|\bm{I}-\eta_k \bm{A}\|\\ 
&\leq \sum_{i=1}^t \eta_0 i^{-\alpha} \prod_{k=i+1}^t \left(1-\frac{1-\gamma}{2}\lambda_0 \eta_0 k^{-\alpha}\right)\\ 
&= \sum_{i=1}^t \eta_0 i^{-\alpha} \prod_{k=i+1}^t (1-\beta k^{-\alpha})\\ 
&= \frac{\eta_0}{\beta} \sum_{i=1}^t \left(\prod_{k=i+1}^t - \prod_{k=i}^t\right)(1-\beta k^{-\alpha}) \\ 
&= \frac{\eta_0}{\beta}\left(1-\prod_{k=1}^t (1-\beta k^{-\alpha})\right) < \frac{\eta_0}{\beta},
\end{align*}
where the second inequality follows from \eqref{eq:lemma-A-5} in Lemma \ref{lemma:A}.
%\ale{I could not find what results in the manuscript we are using to justify these bounds - we can just cite bounds from the AISTATS paper}\weichen{checked; it's from Lemma \ref{lemma:A}.}
Therefore, the norm of $I_2$ is bounded by 
\begin{align}\label{eq:markov-deltat-I2-bound}
\left\| \sum_{i=1}^t \bm{R}_i^t \mathbb{E}_{i_{\mix}}[\bm{\zeta}_{i,\mix}] \right\|&\leq \sum_{i=1}^t \left\|\bm{R}_i^t\right\| \left\|\mathbb{E}_{i_\mix}[\bm{\zeta}_{i,\mix}]\right\|_2 \nonumber \\ 
&\leq \frac{\varepsilon \eta_0}{\beta} (2\max_{1 \leq i \leq t}\|\bm{\Delta}_i\|_2 + 2\|\bm{\theta}^\star\|_2  + 1).
\end{align}
For the term $I_3$, we firstly bound the difference betweeen $\bm{\theta}_i$ and $\bm{\theta}_{i,\mix}$ by $R$ and the initial stepsize $\eta_0$. Notice that given the induction assumption, 
\begin{align*}
\left\|\bm{\theta}_{i-1} - \bm{\theta}_{i,\mix}\right\|_2 = \left\|\sum_{j=i_{\mix}+1}^{i-1} (\bm{\theta}_{j} - \bm{\theta}_{j-1}) \right\|_2 &= \left\| \sum_{j=i_{\mix}+1}^{i-1} \eta_j (\bm{A}_j \bm{\theta}_{j-1} - \bm{b}_j)\right\|_2 \\ 
&\leq \sum_{j=i_{\mix}}^{i-1} \eta_j \|\bm{A}_j \bm{\theta}_{j-1} - \bm{b}_j\|_2 \\ 
&\leq \frac{t_{\mix}}{1-\alpha} \eta_i (2 \max_{1 \leq i < t} \|\bm{\Delta}_i\|_2 + 2\|\bm{\theta}^\star\|_2  + 1).
\end{align*}
Hence, the norm of $I_{3}$ can be bounded by
\begin{align}\label{eq:markov-deltat-I3-bound}
&\left\|\sum_{i=1}^t \bm{R}_i^t (\bm{A}_i - \bm{A}) (\bm{\theta}_i - \bm{\theta}_{i,\mix})\right\|_2  \nonumber \\
& \leq \sum_{i=1}^t \|\bm{R}_i^t\| \|\bm{A}_i - \bm{A}\| \left\|\bm{\theta}_{i-1} - \bm{\theta}_{i,\mix}\right\|_2 \nonumber \\ 
&\leq \sum_{i=1}^t \left(\eta_i \prod_{k=i+1}^t \left(1-\beta k^{-\alpha} \right)\right) \cdot 4 \cdot \frac{t_{\mix}}{1-\alpha} \eta_i (2 \max_{1 \leq i < t} \|\bm{\Delta}_i\|_2 + 2\|\bm{\theta}^\star\|_2  + 1)\nonumber \\ 
&\lesssim \sum_{i=1}^t \eta_i^2 \prod_{k=i+1}^t \left(1-\beta k^{-\alpha} \right) \cdot  \frac{t_{\mix}}{1-\alpha} (2\max_{1 \leq i < t} \|\bm{\Delta}_i\|_2 + 2\|\bm{\theta}^\star\|_2  + 1)\nonumber \\ 
&\leq \frac{\eta_0^2t_{\mix}}{(1-\alpha)} (2\max_{1 \leq i < t} \|\bm{\Delta}_i\|_2 + 2\|\bm{\theta}^\star\|_2  + 1) \cdot \sum_{i=1}^t i^{-2\alpha} \prod_{k=i+1}^t (1-\beta k^{-\alpha})\nonumber \\
&\leq \frac{\eta_0^2t_{\mix}}{1-\alpha} (2\max_{1 \leq i < t} \|\bm{\Delta}_i\|_2 + 2\|\bm{\theta}^\star\|_2  + 1) \cdot \frac{1}{2\alpha-1}\left(\frac{\beta}{2\alpha}\right)^{\frac{\alpha}{1-\alpha}}t^{-\alpha}.
\end{align}
Notice that the last inequality follows by taking $\nu = 2\alpha$ in \eqref{eq:lemma-R-2} in Lemma \ref{lemma:R}. %\ale{I am confused - which results from \ref{lemma:R}?}\weichen{added more details.}

For the term $I_4$, we again invoke the vector Azuma's inequality (Theorem \ref{thm:vector-Azuma}). %\ale{I think this is Corollary A.16 in the AISTATS paper, right?}\weichen{yes,checked}.
For simplicity, we define
\begin{align}
	\bm{X}_i := \bm{\zeta}_{i,\text{mix}}-\mathbb{E}_{i_{\text{mix}}}[\bm{\zeta}_{i,\text{mix}}].
\end{align}
By definition, we can see that for every integer $r \in [\tmix]$, the sequence $\{ \bm{X}_{r + i t_{\text{mix}}} \}_{i=0,1,\ldots}$
%\begin{align*}
%\bm{X}_r, \bm{X}_{r+t_{\text{mix}}}, \bm{X}_{r+2t_{\text{mix}}},......,\bm{X}_{r+it_{\text{mix}}},.......
%\end{align*}
form a martingale difference process. Throughout this part, we assume, without loss of generality that $t = t' \cdot \tmix$ for a positive integer $t'$. Hence, we can first consider the norm of the summation
\begin{align}
\sum_{i'=0}^{t'} \bm{R}_{r+i'\tmix}^t \bm{X}_{r+i'\tmix},
\end{align}
which we will bound using vector Azuma's inequality. Towards that goal, we define 
\begin{align*}
W_{\max}^r :=  \sum_{i'=0}^{t'} \sup\left\|\bm{R}_{r+i'\tmix}^t \bm{X}_{r+i'\tmix}\right\|_2^2.
\end{align*}
By definition, this term is bounded by
\begin{align*}
W_{\max}^r &\leq \sum_{i'=0}^{t'}\|\bm{R}_{r+i'\tmix}^t\|^2 \sup \|\bm{X}_{r+i'\tmix}\|_2^2 \leq  \sum_{i'=0}^{t'}\|\bm{R}_{r+i'\tmix}^t\|^2 \cdot \max_{1 \leq i \leq t} \sup \|\bm{X}_i\|_2^2;
\end{align*}
by summing over $r = 1$ through $r = \tmix$, we obtain
\begin{align*}
\sum_{r=1}^{\tmix} W_{\max}^r &\leq \sum_{r=1}^{\tmix} \sum_{i'=0}^{t'}\|\bm{R}_{r+i'\tmix}^t\|^2 \cdot  \max_{1 \leq i \leq t} \sup \|\bm{X}_i\|_2^2 \\ 
&\leq  \left(\max_{1 \leq i \leq t} \sup \|\bm{X}_i\|_2^2\right)\cdot \sum_{i=1}^t \|\bm{R}_i^t\|^2 .
\end{align*}
Meanwhile, the triangle inequality directly implies
\begin{align*}
\max_{1 \leq i \leq t} \sup \|\bm{X}_i\|_2^2 &= \max_{1 \leq i \leq t} \sup \|\bm{\zeta}_{i,\text{mix}}-\mathbb{E}_{i_{\text{mix}}}[\bm{\zeta}_{i,\text{mix}}]\| \\ 
&\leq 4\max_{1 \leq i < t} \|\bm{\Delta}_i\|_2 + 2\|\bm{\theta}^\star\|_2  + 1.
\end{align*}
In combination, we have that
\begin{align*}
\sum_{r=1}^{\tmix} \sqrt{W_{\max}^r} &\leq \sqrt{\tmix} \sqrt{\sum_{r=1}^{\tmix} W_{\max}^r} \\ 
&\leq \sqrt{\tmix} \sup \|\bm{X}_i\|_2 \sqrt{\sum_{i=1}^t \|\bm{R}_i^t\|^2} \\ 
&\leq \sqrt{\tmix} (4\max_{1 \leq i < t} \|\bm{\Delta}_i\|_2 + 2\|\bm{\theta}^\star\|_2  + 1)\cdot \sqrt{\sum_{i=1}^t \|\bm{R}_i^t\|^2} \\ 
&\leq \sqrt{\tmix} (4\max_{1 \leq i < t} \|\bm{\Delta}_i\|_2 + 2\|\bm{\theta}^\star\|_2  + 1)\cdot \sqrt{\sum_{i=1}^t \eta_i^2 \prod_{k=i+1}^t (1-\frac{1-\gamma}{2}\lambda_0\eta_k)^2}\\
&\leq \sqrt{\tmix} (4\max_{1 \leq i < t} \|\bm{\Delta}_i\|_2 + 2\|\bm{\theta}^\star\|_2  + 1)\cdot \eta_0 \cdot \sqrt{\sum_{i=1}^t i^{-2\alpha} \prod_{k=i+1}^t (1-\beta k^{-\alpha})}\\
&\leq \sqrt{\tmix} (4\max_{1 \leq i < t} \|\bm{\Delta}_i\|_2 + 2\|\bm{\theta}^\star\|_2  + 1)\cdot \eta_0 \cdot \sqrt{\frac{1}{2\alpha-1}}\left(\frac{\beta}{2\alpha}\right)^{\frac{\alpha}{2(1-\alpha)}}t^{-\frac{\alpha}{2}},
\end{align*}
%\yuting{can you explain why the second and the third inequalities holds?}\weichen{added some explanation before this long equation.}
where we invoked Lemma \ref{lemma:R} in the last inequality, following the same logic as in the last line of \eqref{eq:markov-deltat-I3-bound}. %\ale{I am confused - which results from \ref{lemma:R}?}\weichen{Added more details. Hope this helps!} 
Consequently, the vector Azuma's inequality (Theorem \ref{thm:vector-Azuma}) %\ale{Again we should cite Corollary A.16 of the AISTATS paper or include this for reference}\weichen{checked}
, combined with a union bound argument, yields the bound on the norm of $I_4$
\begin{align}\label{eq:markov-deltat-I4-bound}
&\left\|\sum_{i=1}^t \bm{R}_i^t (\bm{\zeta}_{i,\mix} - \mathbb{E}_{i_{\mix}}[\bm{\zeta}_{i,\mix}])\right\|_2\nonumber \leq 2\sqrt{2\log \frac{9t_{\mix}}{\delta_t}} \sum_{r=1}^{\tmix} \sqrt{W_{\max}^r} \nonumber \\ 
&\leq 2 \eta_0 \sqrt{\frac{2\tmix}{2\alpha-1}\log \frac{3\tmix}{\delta_t}}(4\max_{1 \leq i < t} \|\bm{\Delta}_i\|_2 + 2\|\bm{\theta}^\star\|_2  + 1)\cdot \left(\frac{\beta}{2\alpha}\right)^{\frac{\alpha}{2(1-\alpha)}}t^{-\frac{\alpha}{2}},
\end{align}
with probability at least $1-\delta_t/3$. 
\paragraph{Step 2: Bounding $\mathbb{P}(\mathcal{H}_{t^\star})$.} By definition, $\mathbb{P}(\mathcal{H}_0) = 1$. We will show via induction that, for all other $1 \leq t \leq t^\star$
\begin{align*}
\mathbb{P}(\mathcal{H}_t) \geq 1-\frac{t}{3t^\star} \delta.%, \quad \forall 1 \leq t \leq t^\star.
\end{align*}
By taking $\varepsilon = \frac{\beta}{8\eta_0}$ in \eqref{eq:markov-deltat-I2-bound}, we obtain that
\begin{align*}
\left\| \sum_{i=1}^t \bm{R}_i^t \mathbb{E}_{i_{\mix}}[\bm{\zeta}_{i,\mix}] \right\|_2 \leq \frac{1}{8}\left(2\max_{1 \leq i < t} \|\bm{\Delta}_i\|_2 + 2\|\bm{\theta}^\star\|_2  + 1\right).
\end{align*}
Next, putting together \eqref{eq:markov-deltat-I3-bound}, condition \eqref{eq:deltat-condition-markov} and the bound on $\tmix$ as specified by \eqref{eq:tmix-bound} guarantees that
\begin{align*}
\left\|\sum_{i=1}^t \bm{R}_i^t (\bm{A}_i - \bm{A}) (\bm{\theta}_i - \bm{\theta}_{i,\mix})\right\|_2 \leq \frac{1}{8}\left(2\max_{1 \leq i < t} \|\bm{\Delta}_i\|_2 + 2\|\bm{\theta}^\star\|_2  + 1\right).
\end{align*}
Similarly, setting $\delta_t = \frac{\delta}{3t^\star}$ in \eqref{eq:markov-deltat-I4-bound} and using the condition \eqref{eq:deltat-condition-markov}, we have that, with probability at least $1-\frac{\delta}{3t^\star}$, 
\begin{align*}
\left\|\sum_{i=1}^t \bm{R}_i^t (\bm{\zeta}_{i,\mix} - \mathbb{E}_{i_{\mix}}[\bm{\zeta}_{i,\mix}])\right\|_2 &\leq 2\tmix \eta_0 \sqrt{2\log \frac{9\tmix t^\star}{\delta}}(4\max_{1 \leq i < t} \|\bm{\Delta}_i\|_2 + 2\|\bm{\theta}^\star\|_2  + 1) \\ 
&\leq \frac{1}{16}(4\max_{1 \leq i < t} \|\bm{\Delta}_i\|_2 + 2\|\bm{\theta}^\star\|_2  + 1).
\end{align*}
Therefore, when $\mathcal{H}_{t-1}$ holds true, the norm of $\bm{\Delta}_t$ can be bounded using the triangle inequality as
\begin{align}\label{eq:markov-deltat-induction}
\|\bm{\Delta}_t\|_2 &\leq \|\bm{\theta}^\star\|_2 + \frac{1}{8}\left(2R + 2\|\bm{\theta}^\star\|_2  + 1\right) + \frac{1}{8}\left(2R + 2\|\bm{\theta}^\star\|_2  + 1\right) + \frac{1}{16}(4R + 2\|\bm{\theta}^\star\|_2  + 1) \nonumber \\ 
&= \frac{3}{4}R + \frac{13}{8}\|\bm{\theta}^\star\|_2 + \frac{5}{16} = R,
\end{align}
with probability of at least $1-\frac{\delta}{3t^\star}$. It then follows that
\begin{align*}
\mathbb{P}(\mathcal{H}_{t-1} \setminus \mathcal{H}_t) \leq \frac{\delta}{3t^\star},
\end{align*}
and thus that
\begin{align*}
\mathbb{P}(\mathcal{H}_{t^\star}) \geq 1-\frac{\delta}{3}.
\end{align*}
\paragraph{Step 3: Bounding $\mathbb{P}(\mathcal{H}_{\infty})$.} For $t > t^\star$, we sharpen the induction argument by a more refined choice of $\delta_t$. In detail, let 
\begin{align*}
\delta_t = 3\tmix(\beta/8\eta_0)\exp\left\{-\frac{2\alpha-1}{2^{11} \eta_0^2} \left(\frac{1}{\tmix(\beta/8\eta_0)}\right)\left(\frac{\beta}{2\alpha}\right)^{\frac{\alpha}{1-\alpha}} t^{-\alpha}\right\}.
\end{align*}
Then the norm of $I_4$ is bounded by
\begin{align*}
&\left\|\sum_{i=1}^t \bm{R}_i^t (\bm{\zeta}_{i,\mix} - \mathbb{E}_{i_{\mix}}[\bm{\zeta}_{i,\mix}])\right\|_2\\ 
&\leq 2 \eta_0 \sqrt{\frac{2\tmix}{2\alpha-1}\log \frac{3\tmix}{\delta_t}}(4\max_{1 \leq i < t} \|\bm{\Delta}_i\|_2 + 2\|\bm{\theta}^\star\|_2  + 1)\cdot \left(\frac{\beta}{2\alpha}\right)^{\frac{\alpha}{2(1-\alpha)}}t^{-\frac{\alpha}{2}} \\ 
&\leq \frac{1}{16}(4\max_{1 \leq i < t} \|\bm{\Delta}_i\|_2 + 2\|\bm{\theta}^\star\|_2  + 1).
\end{align*}
with probability at least $1-\delta_t$. Hence, using induction, when $\mathcal{H}_{t-1}$ holds true,  the bound in \eqref{eq:markov-deltat-induction} implies that $\mathcal{H}_t$ also holds true with probability at least $1-\delta_t$. In other words,
\begin{align*}
\mathbb{P}(\mathcal{H}_{t-1}) - \mathbb{P}(\mathcal{H}_t) = \mathcal{P}(\mathcal{H}_{t-1} \setminus \mathcal{H}_t) \leq \delta_t.
\end{align*}
Consequently, the definition of $t^\star$ \eqref{eq:defn-tstar-markov} guarantees
\begin{align*}
\mathbb{P}(\mathcal{H}_{\infty}) &= \mathbb{P}(\mathcal{H}_{t^\star}) - \sum_{t=t^\star+1}^{\infty} \mathbb{P}(\mathcal{H}_{t-1}) - \mathbb{P}(\mathcal{H}_t) \\ 
&\geq \left(1-\frac{\delta}{3}\right) - \sum_{t=t^\star+1}^{\infty} \delta_t \\ 
&\geq \left(1-\frac{\delta}{3}\right) - \int_{t^\star}^{\infty} 3\tmix(\beta/8\eta_0)\exp\left\{-\frac{2\alpha-1}{2^{11} \eta_0^2} \left(\frac{1}{\tmix(\beta/8\eta_0)}\right)\left(\frac{\beta}{2\alpha}\right)^{\frac{\alpha}{1-\alpha}} x^{-\alpha}\right\}\mathrm{d}x \\ 
&\geq \left(1-\frac{\delta}{3}\right) - \frac{\delta}{3} = 1-\frac{2\delta}{3}.
\end{align*}
\paragraph{Step 4: Refining the bound on $\|\bm{\Delta}_t\|_2$.} In order to bound the norm of $\bm{\Delta}_t$ by $O(t^{-\frac{\alpha}{2}})$ and thus conclude the prooof, we take $\varepsilon = t^{-\frac{\alpha}{2}}$. Then,
\begin{align*}
\tmix(\varepsilon) \leq \frac{\log m + \frac{\alpha}{2}\log t}{\log(1/\rho)}.
\end{align*}
With this bound, \eqref{eq:markov-deltat-I2-bound}, \eqref{eq:markov-deltat-I3-bound} and \eqref{eq:markov-deltat-I4-bound} yield that, with probability at least $1-\frac{\delta}{3T}$,
\begin{align*}
&\left\| \sum_{i=1}^t \bm{R}_i^t \mathbb{E}_{i_{\mix}}[\bm{\zeta}_{i,\mix}] \right\|\leq \frac{\eta_0}{\beta}(2\max_{1 \leq i < t} \|\bm{\Delta}_i\|_2 + 2\|\bm{\theta}^\star\|_2  + 1)t^{-\frac{\alpha}{2}}, \\ 
&\left\|\sum_{i=1}^t \bm{R}_i^t (\bm{A}_i - \bm{A}) (\bm{\theta}_i - \bm{\theta}_{i,\mix})\right\|_2 \leq \frac{\eta_0^2}{(1-\alpha)(2\alpha-1)} \cdot \tmix(2\max_{1 \leq i < t} \|\bm{\Delta}_i\|_2 + 2\|\bm{\theta}^\star\|_2  + 1) \left(\frac{\beta}{2\alpha}\right)^{-\frac{\alpha}{1-\alpha}}t^{-\alpha}, \\ 
&\left\|\sum_{i=1}^t \bm{R}_i^t (\bm{\zeta}_{i,\mix} - \mathbb{E}_{i_{\mix}}[\bm{\zeta}_{i,\mix}])\right\|_2 \leq 2 \eta_0 \sqrt{\frac{2\tmix}{(2\alpha-1)}\log \frac{9T\tmix}{\delta}}(4\max_{1 \leq i < t} \|\bm{\Delta}_i\|_2 + 2\|\bm{\theta}^\star\|_2  + 1)\left(\frac{\beta}{2\alpha}\right)^{-\frac{\alpha}{2(1-\alpha)}}t^{-\frac{\alpha}{2}}.
\end{align*}
The final result follows from the triangle inequality and the union bound. 
\end{proof}

%With the convergence of $\bm{\Delta}_t$ guaranteed with high probability, we now proceed to the proof of Theorem \ref{thm:TD-whp}.

\subsection{Proof of Theorem \ref{thm:TD-whp}}\label{app:proof-markov-deltat-convergence}
Recall from Theorem \ref{thm:Lambda} that
\begin{align*}
\mathsf{Tr}(\tilde{\bm{\Lambda}}_T - \bm{\Lambda}^\star) = T^{\alpha-1}\mathsf{Tr}(\bm{X}(\tilde{\bm{\Lambda}}^\star)) + O(T^{2\alpha-2}),
\end{align*}
where $\bm{X}(\tilde{\bm{\Lambda}}^\star)$ is the solution to the Lyapunov equation
\begin{align*}
\eta_0(\bm{AX+XA}^\top) = \tilde{\bm{\Lambda}}^\star.
\end{align*}
By combining Lemma \ref{lemma:A}, Lemma \ref{lemma:Lyapunov} and Lemma \ref{lemma:Gamma}, we obtain
\begin{align*}
\mathsf{Tr}(\bm{X}(\tilde{\bm{\Lambda}}^\star)) &\leq \frac{\mathsf{Tr}(\tilde{\bm{\Lambda}}^\star)}{\eta_0\lambda_0(1-\gamma)} \leq \frac{\|\bm{A}^{-1}\|^2 \mathsf{Tr}(\tilde{\bm{\Gamma}})}{\eta_0\lambda_0(1-\gamma)} 
\leq \frac{\mathsf{Tr}(\tilde{\bm{\Gamma}})}{\eta_0\lambda_0^3(1-\gamma)^3} \\ 
&\lesssim \frac{m}{1-\rho} \cdot \frac{1}{\eta_0\lambda_0^5(1-\gamma)^5}.
\end{align*}
Hence, the difference between $\tilde{\bm{\Lambda}}_T$ and $\bm{\Lambda}^\star$ is given by
\begin{align*}
\mathsf{Tr}(\tilde{\bm{\Lambda}}_T - \bm{\Lambda}^\star)  \leq \widetilde{C}T^{\alpha-1},
\end{align*}
where $\widetilde{C}$ can be represented by $\lambda_0,\eta_0$ and $\gamma$.
%\ale{You mean the operator norm of the difference is $\bm{O}(T^{\alpha-1})$}
%\weichen{Actually, it can also be proved that the difference is $\bm{O}(T^{\alpha-1})$ when measured by trace, without introducing a $d$. Please see above.}
Therefore, it suffices to show that with probability at least $1-\delta$, the averaged TD error can be bounded by
\begin{align*}
\|\bar{\bm{\Delta}}_T\|_2 &\lesssim 2\sqrt{\frac{2\mathsf{Tr}(\widetilde{\bm{\Lambda}}_T)}{T} \log \frac{6d}{\delta}} + o\left(T^{-\frac{1}{2}}\log^{\frac{3}{2}}\frac{dT}{\delta}\right).
\end{align*}
%\ale{2 comments here: (1) the remainder term is larger than in the theorem statement because of the term $\log^{3/2}(dT)$ and (2) Theorem \ref{thm:Lambda} bounds the operator norm of the difference between $\tilde{\bm{\Lambda}}_T$ and $\bm{\Lambda}^\star$ but here we need a bound on $\mathsf{Tr}(\bm{\Lambda}^\star - \tilde{\bm{\Lambda}}_T)$, which will pick up an additional $d$. The same comment applies to the bounds in the AISTATS paper, I think. That is, all is correct but we need to include an extra $\sqrt{d}$ in the remainder term.}
As a direct implication of \eqref{eq:delta-t-markov}, $\bar{\bm{\Delta}}_T$ can be decomposed as 
\begin{align*}
\bar{\bm{\Delta}}_T &= \frac{1}{T}\sum_{t=1}^T \bm{\Delta}_t \\ 
 &= \frac{1}{T}\sum_{t=1}^T \left(\prod_{k=1}^t (\bm{I}-\eta_k \bm{A})\Delta_0 - \sum_{i=1}^{t} \bm{R}_{i}^t (\bm{A}_i\bm{\theta}^\star -\bm{b}_i) - \sum_{i=1}^{t} \bm{R}_{i}^t (\bm{A}_i - \bm{A})\bm{\Delta}_{i-1}\right) \\ 
&= \frac{1}{T}\sum_{t=1}^T\prod_{k=1}^t (\bm{I}-\eta_k \bm{A})\Delta_0 - \frac{1}{T}\sum_{t=1}^T\sum_{i=1}^{t} \bm{R}_{i}^t (\bm{A}_i\bm{\theta}^\star -\bm{b}_i) - \frac{1}{T}\sum_{t=1}^T\sum_{i=1}^{t}\bm{R}_{i}^t (\bm{A}_i - \bm{A})\bm{\Delta}_{i-1} \\ 
&= \frac{1}{T}\sum_{t=1}^T\prod_{k=1}^t (\bm{I}-\eta_k \bm{A})\Delta_0 - \frac{1}{T} \sum_{i=1}^T \sum_{t=i}^T \bm{R}_{i}^t (\bm{A}_i\bm{\theta}^\star -\bm{b}_i) - \frac{1}{T} \sum_{i=1}^T \sum_{t=i}^T \bm{R}_{i}^t (\bm{A}_i - \bm{A})\bm{\Delta}_{i-1},
\end{align*}
where we have switched the order of summation in the last equation. The definition of $\bm{Q}_t$ \eqref{eq:defn-Qt} implies 
\begin{align*}
\bar{\bm{\Delta}}_T = \underset{I_1}{\underbrace{\frac{1}{T\eta_0} \bm{Q}_0 \bm{\Delta}_0}} - \underset{I_2}{\underbrace{\frac{1}{T} \sum_{i=1}^T \bm{Q}_i(\bm{A}_i \bm{\theta}^\star-\bm{b}_i)}} - \underset{I_3}{\underbrace{\frac{1}{T}\sum_{i=1}^T \bm{Q}_i (\bm{A}_i-\bm{A})\bm{\Delta}_{i-1}}},
\end{align*}
where $I_1$ can be bounded by
\begin{align*}
\left\|\frac{1}{T\eta_0} \bm{Q}_0 \bm{\Delta}_0\right\|_2 \leq \frac{1}{T\eta_0} \|\bm{Q}_0\| \|\bm{\Delta}_0\|\leq \frac{3}{T}\left(\frac{2}{\beta}\right)^{\frac{1}{1-\alpha}} \|\bm{\Delta}_0\|_2. 
\end{align*}
%\yuting{why almost surely? Is it a deterministic bound?} \weichen{checked.}


We now proceed to bounding $I_2$ and $I_3$respectively. 

%\yuting{$I_4$?} 
Throughout the proof, we let
\begin{align*}
&\beta = \frac{1-\gamma}{2}\lambda_0 \eta_0,\\ 
& R = \frac{13}{2}\|\bm{\theta}^\star\|_2 + \frac{5}{4}, \\ 
&t_{\mix} = \tmix(T^{-\frac{\alpha}{2}}) \leq \frac{\log m + (\alpha/2)\log T}{\log(1/\rho)} \quad \text{and}\\ 
&R' = \eta_0 \sqrt{\frac{2\tmix}{2\alpha-1}\log \frac{27T\tmix}{\delta}} (2\|\bm{\theta}^\star\|_2+1)\left(\frac{\beta}{2\alpha}\right)^{-\frac{\alpha}{2(1-\alpha)}}, 
\end{align*}
and, for each $1 \leq t \leq T$,
\[
\mathcal{H}_t = \Big\{ \|\bm{\Delta}_j\|_2 \leq \min\{R'j^{-\frac{\alpha}{2}},R\}, \forall j \leq t \Big\} \quad \text{and} \quad \tilde{\bm{\Delta}}_t = \bm{\Delta}_t \mathds{1}(\mathcal{H}_t).
\]
%Furthermore, we use $\mathcal{H}_t$ to denote the event that
%\begin{align*}
%\|\bm{\Delta}_j\|_2 \leq \min\{R'j^{-\frac{\alpha}{2}},R\}
%\end{align*}
%holds true for all $j \leq t$, and let $\tilde{\bm{\Delta}}_t = \bm{\Delta}_t \mathds{1}(\mathcal{H}_t)$. 
Theorem \ref{thm:markov-deltat-convergence} shows that $\mathbb{P}(\mathcal{H}_T) \geq 1-\frac{\delta}{3}$.
\paragraph{Bounding $I_2$.} In order to invoke the matrix Freedman's inequality on the term $I_2$, we firstly relate it to a martingale. Specifically, for every $i \in [T]$, we define $\bm{U}_i$ as
\begin{align}\label{eq:defn-Ui}
\bm{U}_i = \mathbb{E}_i \left[\sum_{j=i}^{\infty} (\bm{A}_j \bm{\theta}^\star - \bm{b}_j)\right].
\end{align}
It is then easy to verify that on one hand, the norm of $\bm{U}_i$ is uniformly bounded due to the exponential convergence of the Markov chain. Specifically, since for any positive integers $i<j$, it can be guaranteed that
\begin{align*}
&\left\|\mathbb{E}_i[\bm{A}_j \bm{\theta}^\star - \bm{b}_j]\right\|_2 \\ 
&=\left\|\mathbb{E}_{s_{j-1} \sim P^{j-i-1}(\cdot \mid s_i),s_j \sim P(\cdot \mid s_{j-1})}[\bm{A}_j \bm{\theta}^\star - \bm{b}_j]\right\|_2 \\ 
&= \left\|\mathbb{E}_{s_{j-1} \sim P^{j-i-1}(\cdot \mid s_i),s_j \sim P(\cdot \mid s_{j-1})}[\bm{A}_j \bm{\theta}^\star - \bm{b}_j]-\mathbb{E}_{s_{j-1} \sim \mu,s_j \sim P(\cdot \mid s_{j-1})}[\bm{A}_j \bm{\theta}^\star - \bm{b}_j]\right\|_2 \\ 
&\leq d_{\mathsf{TV}}(P^{j-i-1}(\cdot\mid s_i),\mu) \cdot \sup_{s_{j-1},s_j}\|\bm{A}_j \bm{\theta}^\star - \bm{b}_j\|_2 \\ 
&\leq m\rho^{j-i-1} (2\|\bm{\theta}^\star\|_2+1).
\end{align*}
Therefore, the norm of $\bm{U}_i$ is bounded by
\begin{align}\label{eq:U-bound}
\|\bm{U}_i\|_2 &\leq \|\bm{A}_i \bm{\theta}^\star - \bm{b}_i\|_2 + \sum_{j=i+1}^{\infty}\mathbb{E}_i \|\bm{A}_j \bm{\theta}^\star - \bm{b}_j\|_2 \nonumber \\ 
&\leq (2\|\bm{\theta}^\star\|_2+1) \left(1+\sum_{j=i+1}^{\infty} m\rho^{j-i-1}\right)\nonumber \\ 
& \lesssim \frac{1}{1-\rho} (2\|\bm{\theta}^\star\|_2+1);
\end{align}
%\yuting{add explanation here.}
On the other hand, $\bm{A}_i \bm{\theta}^\star - \bm{b}_i$ can be represented as
\begin{align}\label{eq:Ui-telescope}
\bm{A}_i \bm{\theta}^\star - \bm{b}_i &= \bm{U}_i - \mathbb{E}_i[\bm{U}_{i+1}]\nonumber \\
&= (\bm{U}_i - \mathbb{E}_{i-1}[\bm{U}_i]) + (\mathbb{E}_{i-1}[\bm{U}_i] - \mathbb{E}_i[\bm{U}_{i+1}])\nonumber \\ 
&=: \bm{m}_i + (\mathbb{E}_{i-1}[\bm{U}_i] - \mathbb{E}_i[\bm{U}_{i+1}])
\end{align}
Here, the first term $\bm{U}_i -\bm{U}_{i+1}$ can be analyzed by the telescoping technique, while 
\begin{align}\label{eq:defn-mi}
\bm{m}_i:= \bm{U}_{i} - \mathbb{E}_{i-1}[\bm{U}_{i}]
\end{align}
is a martingale difference process. Furthermore, we observe that when $s_0$ is drawn from the stationary distribution $\mu$, the covariance matrix $\mathbb{E}_{0}\left[\bm{m}_i \bm{m}_i^\top\right]$ is time-invariant, and can be expressed as
\begin{align}\label{eq:var-mi}
\mathbb{E}[\bm{m}_i \bm{m}_i^\top ] &= \mathbb{E}[\bm{m}_1 \bm{m}_1^\top]\nonumber \\ 
&= \mathbb{E}[(\bm{U}_1 - \mathbb{E}_0[\bm{U}_1])(\bm{U}_1 - \mathbb{E}_0[\bm{U}_1])^\top]\nonumber\\
&= \mathbb{E}[\bm{U}_1 \bm{U}_1^\top] - \mathbb{E}[\mathbb{E}_0[\bm{U}_1]\mathbb{E}_0[\bm{U}_1^\top]]\nonumber\\
&\overset{(i)}{=} \mathbb{E}[\bm{U}_1 \bm{U}_1^\top] - \mathbb{E}[\mathbb{E}_1[\bm{U}_2]\mathbb{E}_1[\bm{U}_2^\top]]\nonumber\\
&= \mathbb{E}[(\bm{A}_1 \bm{\theta}^\star -\bm{b}_1 + \mathbb{E}_1[\bm{U}_2])(\bm{A}_1 \bm{\theta}^\star -\bm{b}_1 + \mathbb{E}_1[\bm{U}_2])^\top] - \mathbb{E}[\mathbb{E}_1[\bm{U}_2]\mathbb{E}_1[\bm{U}_2^\top]]\nonumber\\
&= \mathbb{E}[(\bm{A}_1 \bm{\theta}^\star -\bm{b}_1)(\bm{A}_1 \bm{\theta}^\star -\bm{b}_1)^\top] + \mathbb{E}[(\bm{A}_1 \bm{\theta}^\star -\bm{b}_1)\mathbb{E}_1[\bm{U}_2]^\top] + \mathbb{E}[\mathbb{E}_1[\bm{U}_2](\bm{A}_1 \bm{\theta}^\star -\bm{b}_1)^\top]\nonumber\\
&= \mathbb{E}[(\bm{A}_1 \bm{\theta}^\star -\bm{b}_1)(\bm{A}_1 \bm{\theta}^\star -\bm{b}_1)^\top]\nonumber \\ 
&+ \sum_{j=2}^{\infty} \mathbb{E}[(\bm{A}_1 \bm{\theta}^\star -\bm{b}_1)(\bm{A}_j \bm{\theta}^\star - \bm{b}_j)^\top + (\bm{A}_j \bm{\theta}^\star - \bm{b}_j)(\bm{A}_1 \bm{\theta}^\star -\bm{b}_1)^\top]\nonumber\\
&= \widetilde{\bm{\Gamma}},
\end{align}
according to the definition of $\widetilde{\bm{\Gamma}}$ (as in eq.~\eqref{eq:defn-tilde-Gamma}). Notice here that we applied the rule of total expectation throughout this deduction, and took advantage of the time-invariant property of the distribution of $\{\bm{U}_i\}_{1 \leq i \leq T}$ in (i).

In order to relate $I_2$ to the martingale difference process $\bm{m}_i$, we invoke the  relation \eqref{eq:Ui-telescope} to obtain
\begin{align*}
\frac{1}{T}\sum_{i=1}^T \bm{Q}_i (\bm{A}_i \bm{\theta}^\star - \bm{b}_i)
&= \frac{1}{T}\sum_{i=1}^T \bm{Q}_i \bm{m}_i + \frac{1}{T}\sum_{i=1}^T \bm{Q}_i (\mathbb{E}_{i-1}[\bm{U}_i] - \mathbb{E}_i[\bm{U}_{i+1}]) \\ 
&= \frac{1}{T}\sum_{i=1}^T \bm{Q}_i \bm{m}_i + \frac{1}{T}\sum_{i=1}^T (\bm{Q}_{i-1}\mathbb{E}_{i-1}[\bm{U}_i] - \bm{Q}_{i}\mathbb{E}_i[\bm{U}_{i+1}]) + \frac{1}{T}\sum_{i=1}^T (\bm{Q}_i - \bm{Q}_{i-1})\mathbb{E}_{i-1}[\bm{U}_i] \\ 
&= \underset{I_{21}}{\underbrace{\frac{1}{T}\sum_{i=1}^T \bm{Q}_i \bm{m}_i}} + \underset{I_{22}}{\underbrace{\frac{1}{T}(\bm{Q}_0 \mathbb{E}_0[\bm{U}_1] - \bm{Q}_T \mathbb{E}_T [\bm{U}_{T+1}])}}+ \underset{I_{23}}{\underbrace{\frac{1}{T}\sum_{i=1}^T (\bm{Q}_i - \bm{Q}_{i-1})\mathbb{E}_{i-1}[\bm{U}_i]}}
\end{align*}
where we applied the telescoping technique in the last equation. The uniform boundedness of $\|\bm{Q}_t\|$, as indicated by Lemma \ref{lemma:Q-bound}, and the uniform boundedness of $\|\bm{U}_i\|_2$, as indicated by \eqref{eq:U-bound}, guarantee that
\begin{align}
\label{eq:markov-bar-deltat-I22-bound}
\left\|\frac{1}{T}(\bm{Q}_0 \mathbb{E}_0[\bm{U}_1] - \bm{Q}_T \mathbb{E}_T [\bm{U}_{T+1}])\right\|_2  
&\leq \frac{1}{T} (\|\bm{Q}_0\| \sup \|\bm{U}_1\|_2 + \|\bm{Q}_T\| \sup \|\bm{U}_T\|_2)\notag\\ 
&\lesssim  \left(\frac{2}{\beta}\right)^{\frac{1}{1-\alpha}}\left(\frac{2m}{1-\rho}\right)(2\|\bm{\theta}^\star\|_2+1)
\end{align}
%\yuting{add details} \weichen{added one line, please check if this clarifies the point.} 
deterministically.
Meanwhile, the norm of $I_{23}$ is bounded by invoking Lemma \ref{lemma:delta-Q}:
\begin{align}
\label{eq:markov-bar-deltat-I23-bound}
\left\|\frac{1}{T}\sum_{i=1}^T (\bm{Q}_i - \bm{Q}_{i-1})\mathbb{E}_{i-1}[\bm{U}_i]\right\| \lesssim \eta_0 \left[\eta_0 \Gamma\left(\frac{1}{1-\alpha}\right)+\alpha\right]\left(\frac{1}{\beta}\right)^{\frac{1}{1-\alpha}} \left(\frac{2m}{1-\rho}\right)(2\|\bm{\theta}^\star\|_2+1)\frac{\log T}{T}
\end{align}
almost surely. 

It now boils down to bounding the norm of $I_{21}$. Towards this end, we firstly observe that
\begin{align*}
\frac{1}{T}\sum_{i=1}^T \mathbb{E}_{s_{i-1}\sim\mu,s_i \sim P(\cdot \mid s_{i-1})}\|\bm{Q}_i\bm{m}_i\|_2^2 = \mathsf{Tr}(\tilde{\bm{\Lambda}}_T),
\end{align*}
and that 
\begin{align*}
\frac{1}{T}\|\bm{Q}_i\bm{m}_i\|_2 \leq \frac{1}{T} \eta_0 \left(\frac{2}{\beta}\right)^{\frac{1}{1-\alpha}}\left(\frac{2m}{1-\rho}\right)(2\|\bm{\theta}^\star\|_2+1)
\end{align*}
almost surely for all $i \in [T]$, according to Lemma \ref{lemma:Q-bound} and Equation \eqref{eq:U-bound}. Now consider a sequence of matrix-valued functions $\bm{F}_i:\mathcal{S} \times \mathcal{S} \to \mathbb{R}^{(d+1) \times (d+1)}$, defined as
\begin{align*}
\bm{F}_i(s,s') = \begin{pmatrix}
0 & (\bm{Q}_i\bm{m}_i(s,s'))^\top \\ 
\bm{Q}_i\bm{m}_i(s,s') & \bm{0}_{d\times d}.
\end{pmatrix}, \quad \forall i \in [T].
\end{align*}
It can then be verified that 
\begin{align*}
\left\|\mathbb{E}_{s\sim \mu, s' \sim P(\cdot \mid s)}[\bm{F}_i^2(s,s')] \right\|= \mathbb{E}_{s\sim \mu, s' \sim P(\cdot \mid s)}\|\bm{Q}_i\bm{m}_i(s,s')\|_2^2,
\end{align*}
and that
\begin{align*}
\|\bm{F}_i(s,s')\| = \|\bm{Q}_i\bm{m}_i(s,s')\|_2, \quad \forall s,s' \in \mathcal{S}.
\end{align*}
Therefore, a direct application of Corollary \ref{thm:matrix-bernstein-mtg} yields
\begin{align}\label{eq:markov-bar-deltat-I21-bound}
\left\|\frac{1}{T}\sum_{i=1}^T \bm{Q}_i \bm{m}_i\right\|_2 &\lesssim 2\sqrt{\frac{2\mathsf{Tr}(\tilde{\bm{\Lambda}}_T)}{T}\log \frac{12d}{\delta}} \nonumber \\ 
&+ \eta_0 \left(\frac{2}{\beta}\right)^{\frac{1}{1-\alpha}}\left(\frac{2m}{1-\rho}\right)(2\|\bm{\theta}^\star\|_2+1)(1-\lambda)^{-\frac{1}{4}} \frac{1}{T}\log^{\frac{3}{2}}\frac{6d}{\delta},
\end{align}
with probability at least $1-\frac{\delta}{3}$.


% \paragraph{Bounding $I_2$.} In order to invoke the matrix Freedman's inequality on the term $I_2$, we firstly relate it to a martingale. Specifically, we define $\widetilde{\bm{U}}_i$ as
% \begin{align}\label{eq:defn-tilde-Ui}
% \widetilde{\bm{U}}_i = \mathbb{E}_i \left[\sum_{j=i}^T \bm{Q}_j (\bm{A}_j \bm{\theta}^\star - \bm{b}_j)\right],
% \end{align}
% for all integers $i \geq 1$. Assumption \ref{as:mixing} and Lemma \ref{lemma:Q} directly implies that the norm of $\widetilde{\bm{U}}_i$ are uniformly bounded by
% \begin{align}\label{eq:tilde-Ui-bound}
% \|\widetilde{\bm{U}}_i\|_2 &\leq \sum_{j=i}^T \|\bm{Q}_j\|\mathbb{E}_i  \|(\bm{A}_j \bm{\theta}^\star - \bm{b}_j)\|_2\nonumber  \\ 
% &\lesssim \left(\frac{2}{\beta}\right)^{\frac{1}{1-\alpha}} \sum_{j=i}^T (2\|\bm{\theta}^\star\|_2+1) \cdot m \rho^{j-i}\nonumber \\ 
% &< \left(\frac{2}{\beta}\right)^{\frac{1}{1-\alpha}} \frac{m}{1-\rho}(2\|\bm{\theta}^\star\|_2+1).
% \end{align}
% Furthermore, we observe that the sequence $\{\bm{Q}_i(\bm{A}_i \bm{\theta}^\star - \bm{b}_i)\}$ is related to the sequence $\{\widetilde{\bm{U}}_i\}$ by
% \begin{align*}
% \widetilde{\bm{U}_i} &= \bm{Q}_i (\bm{A}_i\bm{\theta}^\star - \bm{b}_i) + \mathbb{E}_i \left[\sum_{j=i+1}^T \bm{Q}_j (\bm{A}_j \bm{\theta}^\star - \bm{b}_j)\right] \\ 
% &= \bm{Q}_i (\bm{A}_i\bm{\theta}^\star - \bm{b}_i) + \mathbb{E}_i [\widetilde{\bm{U}}_{i+1}].
% \end{align*}
% Hence, the term $I_2$ can be expressed as
% \begin{align*}
% &\frac{1}{T}\sum_{i=1}^T \bm{Q}_i (\bm{A}_i\bm{\theta}^\star - \bm{b}_i)\\ 
% &= \frac{1}{T}\frac{1}{T}\sum_{i=1}^T (\widetilde{\bm{U}}_i - \mathbb{E}_i[\widetilde{\bm{U}}_{i+1}])\\ 
% &= \frac{1}{T}\sum_{i=1}^T (\widetilde{\bm{U}}_i - \mathbb{E}_{i-1}[\widetilde{\bm{U}}_i] + \mathbb{E}_{i-1}[\widetilde{\bm{U}}_i]-\mathbb{E}_i[\widetilde{\bm{U}}_{i+1}]) \\ 
% &= \frac{1}{T}\sum_{i=1}^T (\widetilde{\bm{U}}_i - \mathbb{E}_{i-1}[\widetilde{\bm{U}}_i]) + \frac{1}{T}\sum_{i=1}^T(\mathbb{E}_{i-1}[\widetilde{\bm{U}}_i]-\mathbb{E}_i[\widetilde{\bm{U}}_{i+1}]) \\ 
% &= \frac{1}{T}\sum_{i=1}^T (\widetilde{\bm{U}}_i - \mathbb{E}_{i-1}[\widetilde{\bm{U}}_i]) + \frac{1}{T}\mathbb{E}_0[\widetilde{\bm{U}}_1] - \frac{1}{T}\mathbb{E}_T [\widetilde{\bm{U}}_{T+1}],
% \end{align*}
% where we applied the telescoping method in the last equation. By definition, $\mathbb{E}_0[\widetilde{\bm{U}}_1] = \mathbb{E}_T [\widetilde{\bm{U}}_{T+1}] = \bm{0}$, so we obtain
% \begin{align}\label{eq:markov-bar-deltat-I2-decompose}
% \sum_{i=1}^T \bm{Q}_i (\bm{A}_i\bm{\theta}^\star - \bm{b}_i) = \sum_{i=1}^T (\widetilde{\bm{U}}_i - \mathbb{E}_{i-1}[\widetilde{\bm{U}}_i]).
% \end{align}
% Define $\widetilde{\bm{m}}_i = \widetilde{\bm{U}_i} - \mathbb{E}_{i-1}[\widetilde{\bm{U}}_i]$, then it is easy to confirm that $\widetilde{\bm{m}}_i$ is a martingale difference process with respect to the filtration $\{\mathcal{F}_i\}$; equation \eqref{eq:markov-bar-deltat-I2-decompose} then indicates that the term $I_2$ can be expressed as a martingale. Consequently, the matrix Freedman's inequality can be applied to this term and yields
% \begin{align}\label{eq:markov-bar-deltat-Freedman}
% \left\|\frac{1}{T}\sum_{i=1}^T \bm{Q}_i (\bm{A}_i \bm{\theta}^\star - \bm{b}_i)\right\|_2 \leq 2\sqrt{2W_{\max} \log \frac{3d}{\delta}} + \frac{4}{3}B_{\max} \log \frac{3d}{\delta}
% \end{align}
% with probability at least $1-\frac{\delta}{3}$, in which $W_{\max}$ and $B_{\max}$ are defined respectively by
% \begin{align*}
% &W_{\max} := \frac{1}{T^2}\sum_{i=1}^T \mathbb{E}_{i-1} \|\tilde{\bm{m}}_i\|_2^2, \quad \text{and} \\ 
% &B_{\max} := \frac{1}{T} \max_{1 \leq i \leq t} \sup \|\tilde{\bm{m}}_i\|_2.
% \end{align*}
% Notice that the uniform boundedness of $\tilde{\bm{U}}_i$ directly implies
% \begin{align}\label{eq:markov-bar-deltat-Bmax-bound}
% B_{\max} \leq \frac{1}{T} \left(\frac{2}{\beta}\right)^{\frac{1}{1-\alpha}} \frac{m}{1-\rho}(2\|\bm{\theta}^\star\|_2+1);
% \end{align}
% meanwhile, $W_{\max}$ can be represented directly by
% \begin{align}\label{eq:markov-bar-deltat-Wmax-bound}
% W_{\max} &= \frac{1}{T^2} \mathbb{E}\left[\left(\sum_{i=1}^T \tilde{\bm{m}}_i\right)\left(\sum_{i=1}^T \tilde{\bm{m}}_i\right)^\top\right]\nonumber  \\ 
% &= \frac{1}{T^2} \mathbb{E}\left[\left(\sum_{i=1}^t \bm{Q}_i(\bm{A}_i\bm{\theta}^\star - \bm{b}_i)\right)\left(\sum_{i=1}^t \bm{Q}_i(\bm{A}_i\bm{\theta}^\star - \bm{b}_i)\right)^\top\right] \nonumber \\ 
% &= \frac{\mathsf{Tr}(\tilde{\bm{\Lambda}}_T)}{T}.
% \end{align}
% Plugging \eqref{eq:markov-bar-deltat-Wmax-bound} and \eqref{eq:markov-bar-deltat-Bmax-bound} into \eqref{eq:markov-bar-deltat-Freedman}, we obtain
% \begin{align}\label{eq:markov-bar-deltat-I2-bound}.
% &\left\|\frac{1}{T}\sum_{i=1}^T \bm{Q}_i (\bm{A}_i \bm{\theta}^\star - \bm{b}_i)\right\|_2 \nonumber \\ 
% &\leq 2\sqrt{\frac{2\mathsf{Tr}(\tilde{\bm{\Lambda}}_T)}{T}\log \frac{6d}{\delta}} + \frac{4}{3T}\left(\frac{2}{\beta}\right)^{\frac{1}{1-\alpha}} \frac{m}{1-\rho}(2\|\bm{\theta}^\star\|_2+1)
% \end{align}
% with probability at least $1-\frac{\delta}{3}$.

\paragraph{Bounding $I_3$.} Applying a similar technique as in the proof of Theorem \ref{thm:markov-deltat-convergence}, we decompose the term $I_3$ as
\begin{align}\label{eq:markov-bar-deltat-I3-decompose}
&\frac{1}{T}\sum_{i=1}^T \bm{Q}_i (\bm{A}_i-\bm{A})\bm{\Delta}_{i-1} \nonumber \\ 
&=\frac{1}{T} \sum_{i=1}^T \bm{Q}_i(\bm{A}_i - \bm{A}) (\bm{\Delta}_{i-1} -\bm{\Delta}_{i_{\mix}}) + \frac{1}{T}\sum_{i=1}^T \bm{Q}_i(\bm{A}_i - \bm{A}) \bm{\Delta}_{i_{\mix}} \nonumber \\
&= \underset{I_{31}}{\underbrace{\frac{1}{T} \sum_{i=1}^T \bm{Q}_i(\bm{A}_i - \bm{A}) (\bm{\Delta}_{i-1} -\bm{\Delta}_{i_{\mix}})}} + \underset{I_{32}}{\underbrace{\frac{1}{T}\sum_{i=1}^T \bm{Q}_i(\bm{A}_i - \mathbb{E}_{i_{\mix}}[\bm{A}_i])\bm{\Delta}_{i_{\mix}} }}+ \underset{I_{33}}{\underbrace{\frac{1}{T}\sum_{i=1}^T \bm{Q}_i(\mathbb{E}_{i_{\mix}}[\bm{A}_i] - \bm{A}) \bm{\Delta}_{i_{\mix}} }}.
\end{align}
Recall from the proof of Theorem \ref{thm:markov-deltat-convergence} that 
\begin{align*}
\left\|\bm{\Delta}_{i-1} -\bm{\Delta}_{i_{\mix}}\right\|_2&= \left\|\bm{\theta}_{i-1} -\bm{\theta}_{i_{\mix}}\right\|_2 \\ 
&\leq \frac{t_{\mix}}{1-\alpha} \eta_i (2 \max_{1 \leq j < i} \|\bm{\Delta}_j\|_2 + 2\|\bm{\theta}^\star\|_2  + 1);
\end{align*}
hence the norm of $I_{31}$ can be bounded by
\begin{align}
\label{eq:markov-bar-deltat-I31-bound}
\left\|\frac{1}{T} \sum_{i=1}^T \bm{Q}_i(\bm{A}_i - \bm{A}) (\bm{\Delta}_{i-1} -\bm{\Delta}_{i_{\mix}})\right\|_2
&\lesssim \frac{1}{T} \sum_{i=1}^T \|\bm{Q}_i\| \cdot \frac{t_{\mix}}{1-\alpha} \eta_i (2 \max_{1 \leq j < i} \|\bm{\Delta}_j\|_2 + 2\|\bm{\theta}^\star\|_2  + 1) \nonumber \\ 
&\lesssim \frac{\tmix}{(1-\alpha)T}\left(\frac{2}{\beta}\right)^{\frac{1}{1-\alpha}}\sum_{i=1}^T \eta_i (2 \max_{1 \leq j < T} \|\bm{\Delta}_j\|_2 + 2\|\bm{\theta}^\star\|_2  + 1) \nonumber \\ 
&\lesssim \frac{\tmix \eta_0 }{(1-\alpha)^2}  \left(\frac{2}{\beta}\right)^{\frac{1}{1-\alpha}} (2 \max_{1 \leq j < T} \|\bm{\Delta}_j\|_2 + 2\|\bm{\theta}^\star\|_2  + 1) T^{-\alpha}.
\end{align}
The term $I_{32}$ can be decomposed into $\tmix$ martingales and bounded by the vector Azuma's inequality, invoking a similar technique to the tackling of the term $I_4$ in the proof of Theorem \ref{thm:markov-deltat-convergence}. With details omitted, we obtain with probability at least $1-\frac{\delta}{3}$ that
\begin{align}\label{eq:markov-bar-deltat-I32-bound}
&\left\|\frac{1}{T}\sum_{i=1}^T \bm{Q}_i(\bm{A}_i - \mathbb{E}_{i_{\mix}}[\bm{A}_i])\tilde{\bm{\Delta}}_{i_{\mix}}\right\|_2^2 \lesssim \left(\frac{2}{\beta}\right)^{\frac{1}{1-\alpha}}  \sqrt{\frac{\tmix}{1-\alpha}\log \frac{9\tmix}{\delta}}R' T^{-\frac{\alpha+1}{2}}.
\end{align}
The term $I_{33}$ is bounded by the mixing property of the Markov chain, specifically
\begin{align}\label{eq:markov-bar-deltat-I33-bound}
\left\|\frac{1}{T}\sum_{i=1}^T \bm{Q}_i(\mathbb{E}_{i_{\mix}}[\bm{A}_i] - \bm{A}) \tilde{\bm{\Delta}}_{i_{\mix}}\right\|_2 \lesssim \frac{1}{T} \left(\frac{2}{\beta}\right)^{\frac{1}{1-\alpha}} T^{-\frac{\alpha}{2}} \sum_{i=1}^T R' (i_{\mix})^{-\frac{\alpha}{2}} \lesssim \left(\frac{2}{\beta}\right)^{\frac{1}{1-\alpha}} R' T^{-\alpha}.
\end{align}
\paragraph{Completing the proof.} Combining \eqref{eq:markov-bar-deltat-I21-bound}, \eqref{eq:markov-bar-deltat-I22-bound}, \eqref{eq:markov-bar-deltat-I23-bound}, \eqref{eq:markov-bar-deltat-I31-bound}, \eqref{eq:markov-bar-deltat-I32-bound}, and \eqref{eq:markov-bar-deltat-I33-bound} by a union bound argument and plugging in the definition of $R'$, we obtain
\begin{align*}
\|\bar{\bm{\Delta}}_T\|_2 &\lesssim 2\sqrt{\frac{2\mathsf{Tr}(\widetilde{\bm{\Lambda}}_T)}{T} \log \frac{6d}{\delta}} \\ 
&+ \frac{\eta_0\tmix(T^{-\frac{\alpha}{2}})}{(1-\alpha)^2}  (2\|\bm{\theta}^\star\|_2+1)\left(\frac{1-\gamma}{2}\lambda_0\eta_0\right)^{-\frac{1}{1-\alpha}}T^{-\alpha} \\ 
&+ \eta_0 \sqrt{\frac{2\tmix(T^{-\frac{\alpha}{2}})}{2\alpha-1}\log \frac{27T \tmix(T^{-\frac{\alpha}{2}})}{\delta}}(2\|\bm{\theta}^\star\|_2+1)\left(\frac{1-\gamma}{2}\lambda_0\eta_0\right)^{-\frac{2+\alpha}{2(1-\alpha)}}T^{-\alpha} \\ 
&+ \frac{\eta_0\tmix(T^{-\frac{\alpha}{2}})}{\sqrt{(1-\alpha)(2\alpha-1)}} \log \frac{27T \tmix(T^{-\frac{\alpha}{2}})}{\delta}(2\|\bm{\theta}^\star\|_2+1)\left(\frac{1-\gamma}{2}\lambda_0\eta_0\right)^{-\frac{2+\alpha}{2(1-\alpha)}}T^{-\frac{\alpha+1}{2}} \\ 
&+ \eta_0 \frac{m}{1-\rho} (2\|\bm{\theta}^\star\|_2+1)\left(\frac{1-\gamma}{4}\lambda_0\eta_0\right)^{-\frac{1}{1-\alpha}}T^{-1} \\ 
&\cdot \left[(1-\lambda)^{-\frac{1}{4}}\log^{\frac{3}{2}}\frac{6d}{\delta} + \left(\eta_0 \Gamma\left(\frac{1}{1-\alpha}\right)+\alpha\right) \log T\right].
\end{align*}
Notice that all the terms beginning from the second line can all be bounded by
\begin{align*}
\widetilde{C}T^{-\alpha}\log^{\frac{3}{2}} \frac{dT}{\delta},
\end{align*}
where $\widetilde{C}$ is a problem-related quantity depending on $\alpha,\eta_0,\lambda_0, \gamma, m,\rho$ and $\lambda$. The theorem follows immediately.

\subsection{Proof of Theorem \ref{thm:TD-berry-esseen}} \label{app:proof-TD-Berry-Esseen}
Following the precedent of \cite{wu2024statistical}, we approach this Berry-Esseen bound by introducing a Gaussian comparison term. Specifically, the triangle inequality indicates
\begin{align}\label{eq:TD-Berry-Esseen-decompose}
d_{\mathsf{C}}(\sqrt{T} \bar{\bm{\Delta}}_T,\mathcal{N}(\bm{0},\widetilde{\bm{\Lambda}}^{\star}))\leq d_{\mathsf{C}}(\sqrt{T} \bar{\bm{\Delta}}_T,\mathcal{N}(\bm{0},\widetilde{\bm{\Lambda}}_T)) + d_{\mathsf{C}}(\mathcal{N}(\bm{0},\widetilde{\bm{\Lambda}}_T), \mathcal{N}(\bm{0},\widetilde{\bm{\Lambda}}^{\star}))
\end{align}
where the second term on the right-hand-side can be bounded by the following proposition.
\begin{customlemma}\label{lemma:Gaussian-comparison}
With $\tilde{\bm{\Lambda}}^\star$ and $\tilde{\bm{\Lambda}}_T$ defined as in \eqref{eq:defn-tilde-Lambdastar} and \eqref{eq:defn-tilde-LambdaT} respectively, it can be guaranteed for any $\eta_0 \leq \frac{1}{2\lambda_{\Sigma}}$ that
\begin{align*}
d_{\mathsf{C}}(\mathcal{N}(\bm{0},\widetilde{\bm{\Lambda}}_T), \mathcal{N}(\bm{0},\widetilde{\bm{\Lambda}}^{\star})) \lesssim \frac{\sqrt{d\mathsf{cond}(\bm{\tilde{\Gamma}})}}{(1-\gamma)\lambda_0\eta_0} T^{\alpha-1} + O(T^{2\alpha-2}).
\end{align*}
\end{customlemma}
\begin{proof}
This lemma is a direct generalization of Theorem 3.3 in \cite{wu2024statistical}, where $\bar{\bm{\Lambda}}_T$ is replaced by $\tilde{\bm{\Lambda}}_T$ and $\bm{\Lambda}^\star$ is replaced by $\tilde{\bm{\Lambda}}^\star$. 
\end{proof}

We next focus on the first term on the right-hand-side of \eqref{eq:TD-Berry-Esseen-decompose}. For this, we notice that 
\begin{align*}
d_{\mathsf{C}}(\sqrt{T} \bar{\bm{\Delta}}_T,\mathcal{N}(\bm{0},\widetilde{\bm{\Lambda}}_T)) = d_{\mathsf{C}}(\sqrt{T} \bm{A}\bar{\bm{\Delta}}_T,\mathcal{N}(\bm{0},\bm{A}\widetilde{\bm{\Lambda}}_T\bm{A}^\top));
\end{align*}
and we will focus on bounding the latter.

Recall that $\sqrt{T}\bm{A}\bar{\bm{\Delta}}_T$ can be decomposed as
\begin{align}\label{eq:delta-decomposition-markov}
\sqrt{T}\bm{A}\bar{\bm{\Delta}}_T&= \underset{I_1}{\underbrace{\frac{\bm{A}}{\sqrt{T} \eta_0} \bm{Q}_0 \bm{\Delta}_0}} - \underset{I_2}{\underbrace{\frac{\bm{A}}{\sqrt{T}} \sum_{i=1}^T \bm{Q}_i (\bm{A}_i  - \bm{A})\bm{\Delta}_{i-1}}} -\underset{I_3} {\underbrace{\frac{\bm{A}}{\sqrt{T}} \sum_{i=1}^T \bm{Q}_i (\bm{A}_i \bm{\theta}^\star - \bm{b}_i) }}.
% &= \underset{I_1}{\underbrace{\frac{1}{\sqrt{T} \eta_0} \bm{Q}_0 \bm{\Delta}_0}} - \underset{I_2}{\underbrace{\frac{1}{\sqrt{T}} \sum_{i=1}^T \bm{Q}_i (\bm{A}_i  - \bm{A})\bm{\Delta}_i }} - \underset{I_3}{\underbrace{\frac{1}{\sqrt{T}} \sum_{i=1}^T (\bm{Q}_i - \bm{A}^{-1})(\bm{A}_i \bm{\theta}^\star - \bm{b}_i)}} - \underset{I_4}{\underbrace{\frac{1}{\sqrt{T}} \sum_{i=1}^T \bm{A}^{-1}(\bm{A}_i\bm{\theta}^\star - \bm{b}_i)}}.
\end{align}
In order to derive the non-asymptotic rate at which $\sqrt{T}\bm{A}\bar{\bm{\Delta}}_T$ converges to its Gaussian distribution, we derive the convergence of $I_1$, $I_2$ and $I_3$ accordingly in the following paragraphs. For readability concerns, we will only keep track of dependence on $T$ and $d$ in this proof, and use $\widetilde{C}$ to denote any problem-related parameters that are related to $\alpha,\gamma,\eta_0,\lambda_0,m,\rho$.

\paragraph{The $a.s.$ convergence of $I_1$.} 
Lemma \ref{lemma:Q-bound} directly implies that as $T \to \infty$, $I_1$ is bounded by
% \blue{there exists a constant $\widetilde{C}$ such that}
% \yuting{do you want to use $\lesssim$ or this statement}
\begin{align}\label{eq:markov-CLT-I1-converge}
\left\|\frac{\bm{A}}{\sqrt{T} \eta_0} \bm{Q}_0 \bm{\Delta}_0\right\|_2 \leq \frac{1}{\sqrt{T} \eta_0} \|\bm{A}\bm{Q}_0\| \|\bm{\Delta}_0\|_2 \lesssim \lambda_{\Sigma} \left(\frac{2}{\beta}\right)^{\frac{1}{1-\alpha}}\|\bm{\theta}^\star\|_2 T^{-\frac{1}{2}}.
\end{align}
almost surely.

\paragraph{Bounding $I_2$ with high probability.} The convergence of $I_2$ is result of the uniform boundedness of  $\bm{Q}_i$, the convergence of $\{\bm{\Delta}_t\}$, and the mixing property of the Markov chain. Specifically, we again apply the technique in the proof of Theorem \ref{thm:markov-deltat-convergence} and define
\begin{align}\label{eq:defn-markov-CLT-tmix}
&\tmix = \tmix(T^{-\frac{1}{2}}), \quad \text{and} \quad i_{\mix} = \max\left\{i-\tmix, 0\right\}.
\end{align}
Assumption \ref{as:mixing} implies that (see \ref{eq:tmix-bound-L2}) 
\begin{align}
\label{eq:tmix.bound}
\tmix \leq \frac{\log m + \frac{1}{2}\log T}{\log(1/\rho)} \lesssim \frac{\log T}{1-\rho}.
\end{align}
The term $I_2$ can be decomposed as
\begin{align*}
&\frac{1}{\sqrt{T}} \sum_{i=1}^T \bm{AQ}_i(\bm{A}_i - \bm{A}) (\bm{\Delta}_i -\bm{\Delta}_{i_{\mix}}) + \frac{1}{\sqrt{T}}\sum_{i=1}^T \bm{AQ}_i(\bm{A}_i - \bm{A}) \bm{\Delta}_{i_{\mix}} \\
&= \underset{I_{21}}{\underbrace{\frac{1}{\sqrt{T}} \sum_{i=1}^T \bm{AQ}_i(\bm{A}_i - \bm{A}) (\bm{\Delta}_{i-1} -\bm{\Delta}_{i_{\mix}})}} + \underset{I_{22}}{\underbrace{\frac{1}{\sqrt{T}}\sum_{i=1}^T \bm{AQ}_i(\bm{A}_i - \mathbb{E}_{i_{\mix}}[\bm{A}_i])\bm{\Delta}_{i_{\mix}} }}\\ 
&+ \underset{I_{23}}{\underbrace{\frac{1}{\sqrt{T}}\sum_{i=1}^T \bm{AQ}_i(\mathbb{E}_{i_{\mix}}[\bm{A}_i] - \bm{A}) \bm{\Delta}_{i_{\mix}} }};
\end{align*}
for the term $I_{21}$, recall that the difference between $\bm{\Delta}_{i-1}$ and $\bm{\Delta}_{i_{\mix}}$ can be further decomposed into
\begin{align*}
\bm{\Delta}_{i-1}-\bm{\Delta}_{i_{\mix}} = \bm{\theta}_{i-1} - \bm{\theta}_{i_{\mix}} &= \sum_{j=i_{\mix}+1}^{i-1} \eta_j (\bm{\theta}_{j} - \bm{\theta}_{j-1})\\  
&= -\sum_{j=i_{\mix}+1}^{i-1} \eta_j(\bm{A}_j \bm{\theta}_{j-1} - \bm{b}_j) \\ 
&= -\sum_{j=i_{\mix}+1}^{i-1} \eta_j(\bm{A}_j \bm{\theta}^\star - \bm{b}_j) - \sum_{j=i_{\mix}+1}^{i-1} \eta_j\bm{A}_j \bm{\Delta}_{j-1}.
\end{align*}
% \color{violet}
% *ALE* - minor corrections:
% \begin{align*}
% \bm{\Delta}_{i-1}-\bm{\Delta}_{i_{\mix}} = \bm{\theta}_{i-1} - \bm{\theta}_{i_{\mix}} &= \sum_{j=i_{\mix}+1}^{i-1}  (\bm{\theta}_{j} - \bm{\theta}_{j-1})\\  
% &= - \sum_{j=i_{\mix}+1}^{i-1} \eta_j(\bm{A}_j \bm{\theta}_{j-1} - \bm{b}_j) \\ 
% &= - \sum_{j=i_{\mix}+1}^{i-1} \eta_j(\bm{A}_j \bm{\theta}^\star - \bm{b}_j) - \sum_{j=i_{\mix}+1}^{i-1} \eta_j\bm{A}_j \bm{\Delta}_{j-1}.
% \end{align*}
% \weichen{checked}

\color{black}

Hence, the decomposition of $I_2$ can be expressed as
\begin{align}\label{eq:markov-Berry-Esseen-I2-decompose}
I_2 &= -\underset{I_{20}}{\underbrace{\frac{1}{\sqrt{T}}\sum_{i=1}^T \left[\bm{AQ}_i (\bm{A}_i - \bm{A})\sum_{j=i_{\mix}+1}^{i-1} \eta_j (\bm{A}_j \bm{\theta}^\star - \bm{b}_j)\right]}} \nonumber \\ 
&- \underset{I_{21}'}{\underbrace{\frac{1}{\sqrt{T}}\sum_{i=1}^T \left[\bm{AQ}_i (\bm{A}_i - \bm{A})\sum_{j=i_{\mix}+1}^{i-1} \eta_j \bm{A}_j \bm{\Delta}_{j-1}\right]}} + I_{22} + I_{23},
\end{align}
% \color{violet}
% *ALE* - minor correction:
% \[
% I_2 = - \underset{I_{20}}{\underbrace{\frac{1}{\sqrt{T}}\sum_{i=1}^T \left[\bm{AQ}_i (\bm{A}_i - \bm{A})\sum_{j=i_{\mix}+1}^{i-1} \eta_j (\bm{A}_j \bm{\theta}^\star - \bm{b}_j)\right]}} + \underset{I_{21}'}{\underbrace{\frac{1}{\sqrt{T}}\sum_{i=1}^T \left[\bm{AQ}_i (\bm{A}_i - \bm{A})\sum_{j=i_{\mix}+1}^{i-1} \eta_j \bm{A}_j \bm{\Delta}_{j-1}\right]}} + I_{22} + I_{23},
% \]
% \color{black}
where the norm of $I_{20}$ is bounded almost surely by
\begin{align}\label{eq:markov-Berry-Esseen-I20}
\|I_{20}\|_2&\leq \frac{1}{\sqrt{T}}\sum_{i=1}^T \|\bm{AQ}_i\| \|\bm{A}_i - \bm{A}\| \cdot \sum_{j=i_{\mix}+1}^{i-1} \eta_j (2\|\bm{\theta}^\star\|_2 + 1)\nonumber \\ 
&\lesssim \frac{1}{\sqrt{T}}\sum_{i=1}^T (2+\widetilde{C}i^{\alpha-1})\cdot \tmix \eta_i (2\|\bm{\theta}^\star\|_2 + 1)\nonumber \\ 
&\lesssim (2\|\bm{\theta}^\star\|_2 + 1) \left[\frac{\eta_0 }{1-\rho}T^{\frac{1}{2}-\alpha}\log T + \widetilde{C}T^{-\frac{1}{2}}\log^2 T\right]\nonumber \\
&= \frac{\eta_0 }{1-\rho}(2\|\bm{\theta}^\star\|_2 + 1) T^{\frac{1}{2}-\alpha}\log T + o(T^{\frac{1}{2}-\alpha}).
\end{align}
%\yuting{quote necessary inequalities.} 
Here, the second line follows from Lemma \ref{lemma:Q-bound}, and the third line uses the bound \eqref{eq:tmix.bound} on $\tmix$.

% \ale{Mention that the second line follows from Lemma \ref{lemma:Q-bound}, the third lines uses the bound \eqref{eq:tmix.bound} on $\tmix$ and $\tilde{C}$ is independent of $T$ and $d$.}
% \weichen{Near the beginning of the proof, we have said that we use $\widetilde{C}$ to denote parameters independent of $T$.}

For the term $I_{21}'$, we invoke the fact that for any vectors $\bm{x}_1,\bm{x}_2,...,\bm{x}_n \in \mathbb{R}^d$, it can be guaranteed that
\begin{align*}
\left\|\sum_{i=1}^n \bm{x}_i\right\|_2^2 \leq n \sum_{i=1}^n \|\bm{x}_i\|_2^2;
\end{align*}
Consequently, the norm of $I_{21}'$ is bounded, in expectation, by
\begin{align}\label{eq:markov-Berry-Esseen-I21}
\mathbb{E}\|I_{21}'\|_2^2 &= \frac{1}{T}  \mathbb{E}\left\|\sum_{i=1}^T \sum_{j=i_{\mix}+1}^{i-1}\bm{AQ}_i (\bm{A}_i - \bm{A}) \eta_j \bm{A}_j \bm{\Delta}_{j-1}\right\|_2^2 \nonumber \\ 
&\leq \frac{1}{T} \cdot (T\tmix) \sum_{i=1}^T \sum_{j=i_{\mix}+1}^{i-1} \mathbb{E}\|\bm{AQ}_i (\bm{A}_i - \bm{A}) \eta_j \bm{A}_j \bm{\Delta}_{j-1}\|_2^2 \nonumber 
\\ 
&\leq \tmix \sum_{i=1}^T \left\|\bm{AQ}_i\right\|^2 \|\bm{A}_i - \bm{A}\|^2 \cdot \left(4  \sum_{j=i_{\mix}+1}^{i-1} \eta_j^2 \mathbb{E}\|\bm{\Delta}_{j-1}\|_2^2\right)\nonumber \\ 
&\lesssim \eta_0^2 \tmix^2 (2\|\bm{\theta}^\star\|_2+1)^2 \sum_{i=1}^T (2+\widetilde{C}i^{\alpha-1})i^{-2\alpha} \left(\frac{\eta_0}{\lambda_0(1-\gamma)}\frac{1}{(1-\rho)^2}i^{-\alpha} \log^2 i + \widetilde{C}'i^{-1}\log^2 i\right)\nonumber \\ 
& \lesssim \widetilde{C}(2\|\bm{\theta}^\star\|_2+1)^2 T^{1-3\alpha} \log^4 T = o(T^{\frac{3}{2}-3\alpha}),
\end{align}
where we invoke Theorem \ref{thm:markov-L2-convergence} in the fourth line.

The term $I_{22}$ can be decomposed into $\tmix$ martingales:%, and we apply the AM-GM inequality to obtain that
\begin{align*}
&\left\|\frac{1}{\sqrt{T}}\sum_{i=1}^T \bm{AQ}_i(\bm{A}_i - \mathbb{E}_{i_{\mix}}[\bm{A}_i])\bm{\Delta}_{i_{\mix}}\right\|_2^2   \\
&= \frac{1}{T}\left\|\sum_{r=1}^{\tmix} \sum_{i=0}^{T'-1} \bm{AQ}_{i\tmix + r}(\bm{A}_{i\tmix + r} - \mathbb{E}_{(i-1)\tmix + r}[\bm{A}_{i\tmix + r}])\bm{\Delta}_{(i-1)\tmix + r}\right\|_2^2 \\ 
&\leq \frac{\tmix}{T} \sum_{r=1}^{\tmix} \left\|\sum_{i=0}^{T'-1} \bm{AQ}_{i\tmix + r}(\bm{A}_{i\tmix + r} - \mathbb{E}_{(i-1)\tmix + r}[\bm{A}_{i\tmix + r}])\bm{\Delta}_{(i-1)\tmix + r}\right\|.
\end{align*}
Notice here that for any $r \in [\tmix]$, the sequence
\begin{align*}
\left\{\bm{AQ}_{i\tmix + r}(\bm{A}_{i\tmix + r} - \mathbb{E}_{(i-1)\tmix + r}[\bm{A}_{i\tmix + r}])\bm{\Delta}_{(i-1)\tmix + r}\right\}_{i=0}^{T'-1}
\end{align*}
is a martingale difference. Therefore, its expected norm can be bounded by
\begin{align*}
&\mathbb{E}\left\|\sum_{i=0}^{T'-1}\bm{AQ}_{i\tmix + r}(\bm{A}_{i\tmix + r} - \mathbb{E}_{(i-1)\tmix + r}[\bm{A}_{i\tmix + r}])\bm{\Delta}_{(i-1)\tmix + r}\right\|_2^2 \\ 
&= \sum_{i=0}^{T'-1} \mathbb{E}\left\|\bm{AQ}_{i\tmix + r}(\bm{A}_{i\tmix + r} - \mathbb{E}_{(i-1)\tmix + r}[\bm{A}_{i\tmix + r}])\bm{\Delta}_{(i-1)\tmix + r} \right\|_2^2 \\ 
&\lesssim  \sum_{i=0}^{T'-1} \|\bm{AQ}_{i\tmix + r}\|^2 \mathbb{E}\|\bm{\Delta}_{(i-1)\tmix + r}\|_2^2 
\end{align*}
Therefore, the norm of $I_{22}$ is bounded by
\begin{align}
\label{eq:markov-CLT-I22-bound}
&\mathbb{E}\left\|\frac{1}{\sqrt{T}}\sum_{i=1}^T \bm{AQ}_i(\bm{A}_i - \mathbb{E}_{i_{\mix}}[\bm{A}_i])\bm{\Delta}_{i_{\mix}}\right\|_2^2 \nonumber \\
&\leq \frac{\tmix}{T}  \sum_{i=1}^T \|\bm{AQ}_i\|^2\mathbb{E}\|\bm{\Delta}_{i_{\mix}}\|_2^2 \nonumber \\ 
&\leq \frac{\log T}{(1-\rho) T} \sum_{i=1}^T (2+O(i^{\alpha-1}))^2  \left[\frac{\eta_0}{\lambda_0(1-\gamma)} \frac{1}{(1-\rho)^2} (2\|\bm{\theta}^\star\|_2+1)^2 i_{\mix}^{-\alpha} \log^2 i + O(i_{\mix}^{-1} \log^2 i)\right] \nonumber \\ 
&\leq \frac{\eta_0}{\lambda_0(1-\gamma)} \frac{1}{(1-\rho)^3}  (2\|\bm{\theta}^\star\|_2+1)^2 T^{-\alpha} \log^3 T + O(T^{-1}\log^3 T)\nonumber \\ 
&= \frac{\eta_0}{\lambda_0(1-\gamma)} \frac{1}{(1-\rho)^3}  (2\|\bm{\theta}^\star\|_2+1)^2 T^{-\alpha} \log^3 T + o(T^{-\alpha}).
\end{align}

For $I_{23}$, we make use of the fact that since 
\begin{align*}
\max_{s \in \mathcal{S}} d_{\mathsf{TV}}(P^{\tmix}(\cdot \mid s), \mu) \leq T^{-1/2},
\end{align*}
the difference between $\mathbb{E}_{i_{\mix}}[\bm{A}_i]$ and $\bm{A} = \mathbb{E}_{\mu}[\bm{A}_i]$ is bounded by
\begin{align*}
\left\|\mathbb{E}_{i_{\mix}}[\bm{A}_i] - \mathbb{E}_{\mu}[\bm{A}_i] \right\|\leq \max_{s \in \mathcal{S}} d_{\mathsf{TV}}(P^{\tmix}(\cdot \mid s), \mu) \cdot \sup_{s_{i-1},s_{i}} \|\bm{A}_i\| \leq 2T^{-1/2}.
\end{align*}
Hence, by AM-GM inequality, the expected norm of $I_{23}$ is bounded by
\begin{align}\label{eq:markov-CLT-I23-bound}
&\mathbb{E}\left\|\frac{1}{\sqrt{T}}\sum_{i=1}^T \bm{AQ}_i(\mathbb{E}_{i_{\mix}}[\bm{A}_i] - \bm{A}) \bm{\Delta}_{i_{\mix}}\right\|_2^2 \nonumber \\
&\leq \sum_{i=1}^T \mathbb{E}\left\|\bm{AQ}_i(\mathbb{E}_{i_{\mix}}[\bm{A}_i] - \bm{A}) \bm{\Delta}_{i_{\mix}} \right\|_2^2 \nonumber \\ 
&\leq \sum_{i=1}^T \mathbb{E} \left\{ \|\bm{AQ}_i\|^2 \|\mathbb{E}_{i_{\mix}}[\bm{A}_i] - \bm{A}\|^2 \|\bm{\Delta}_{i_{\mix}} \|_2^2 \right\}\nonumber \\ 
&\lesssim \sum_{i=1}^T (2+O(i^{\alpha-1})) (T^{-1}) \left[\frac{\eta_0}{\lambda_0(1-\gamma)} \frac{1}{(1-\rho)^2} (2\|\bm{\theta}^\star\|_2+1)^2 i_{\mix}^{-\alpha} \log^2 i + O(i_{\mix}^{-1} \log^2 i)\right]\nonumber \\ 
&\lesssim \frac{\eta_0}{\lambda_0(1-\gamma)} \frac{1}{(1-\rho)^2} (2\|\bm{\theta}^\star\|_2+1)^2 T^{-\alpha}\log^2 T + O(T^{-1}\log^2 T) \nonumber \\ 
&=  \frac{\eta_0}{\lambda_0(1-\gamma)} \frac{1}{(1-\rho)^2} (2\|\bm{\theta}^\star\|_2+1)^2 T^{-\alpha}\log^2 T +o(T^{-\alpha}). 
\end{align}
% Combining \eqref{eq:markov-CLT-I21-bound}, \eqref{eq:markov-CLT-I22-bound} and \eqref{eq:markov-CLT-I23-bound}, we obtain
% \begin{align}\label{eq:markov-CLT-I2-converge}
% \mathbb{E}\left\|\frac{1}{\sqrt{T}} \sum_{i=1}^T \bm{Q}_i (\bm{A}_i  - \bm{A})\bm{\Delta}_i\right\|_2^2 \leq \widetilde{C} T^{1-2\alpha} \log^2(T)
% \end{align}
% for a constant $\widetilde{C}$ independent of $T$.
Combining \eqref{eq:markov-Berry-Esseen-I2-decompose}, \eqref{eq:markov-Berry-Esseen-I20}, \eqref{eq:markov-Berry-Esseen-I21}, \eqref{eq:markov-CLT-I22-bound} and \eqref{eq:markov-CLT-I23-bound}, we obtain
\begin{align*}
\mathbb{E}\left\|I_2 - \widetilde{C}_1'T^{\frac{1-2\alpha}{2}} \log T + o(T^{\frac{1}{2}-\alpha})\right\|_2^2 
&= \mathbb{E}\left\|I_2 - I_{20}\right\|_2^2 \\ 
&\leq 3\left(\mathbb{E}\|I_{21}'\|_2^2 + \mathbb{E}\|I_{22}\|_2^2 + \mathbb{E}\|I_{23}\|_2^2\right)\\ 
&\lesssim (\widetilde{C}_2')^3 T^{-\alpha}\log^3 T + o(T^{\frac{3}{2}-3\alpha} + T^{-\alpha}),
\end{align*}
where we use $\widetilde{C}_1'$ and $\widetilde{C}_2'$ to denote problem-related quantities
\begin{align}
&\widetilde{C}_1' = \frac{\eta_0}{1-\rho}(2\|\bm{\theta}^\star\|_2+1), \quad \text{and} \label{eq:Berry-Esseen-C10}\\ 
&\widetilde{C}_2' = \frac{1}{1-\rho} \left(\frac{\eta_0(2\|\bm{\theta}^\star\|_2+1)^2}{\lambda_0(1-\gamma)}\right)^{\frac{1}{3}}. \label{eq:Berry-Esseen-C20}
\end{align}
Therefore, the Chebyshev's inequality  directly implies that
\begin{align*}
&\mathbb{P}\left(\left\|I_2 - \widetilde{C}_1'T^{\frac{1}{2}-\alpha}\log T -o(T^{\frac{1}{2}-\alpha}) \right\|_2 \gtrsim \widetilde{C}_2'T^{-\frac{\alpha}{3}}\log T + o(T^{\frac{1}{2}-\alpha} + T^{-\frac{\alpha}{3}})\right)\\
 &\lesssim \widetilde{C}_2'T^{-\frac{\alpha}{3}}\log T + o(T^{\frac{1}{2}-\alpha} + T^{-\frac{\alpha}{3}}).
\end{align*}
Applying the triangle inequality, we obtain the bound on $I_2$ with high probability by triangle inequality:
\begin{align}\label{eq:markov-Berry-Esseen-I2}
\mathbb{P}\left(\left\|I_2 \right\|_2 \gtrsim \widetilde{C}_1'T^{\frac{1}{2}-\alpha}\log T + \widetilde{C}_2'T^{-\frac{\alpha}{3}}\log T + o(T^{\frac{1}{2}-\alpha} + T^{-\frac{\alpha}{3}})\right)
 \lesssim \widetilde{C}_2'T^{-\frac{\alpha}{3}}\log T + o(T^{\frac{1}{2}-\alpha} + T^{-\frac{\alpha}{3}}).
\end{align}

\paragraph{A Berry-Esseen bound for $I_3$.}

% \textcolor{red}{An idea of deriving a tighter Berry-Esseen bound on $I_3$: Using technique developed by \cite{srikant2024rates}, firstly prove that for every $\beta \in (0,1)$,
% \begin{align*}
% &d_{\mathsf{W}}(\frac{1}{\sqrt{T}}\sum_{i=1}^T \bm{AQ}_i \bm{m}_i, \mathcal{N}(\bm{0},\bm{A}\check{\bm{\Lambda}}_T\bm{A}^\top)) \\ 
% &\lesssim \frac{1}{\sqrt{T}}\tilde{C}_1(d,\beta) \sum_{k=1}^T \mathbb{E}\|\bm{P}_k^{-1} \bm{Q}_k \bm{m}_k \|_2^{2+\beta} \|\bm{A}\bm{P}_k\|.
% \end{align*}
% Here, the matrix $\bm{P}_k$ is defined as
% \begin{align*}
% \bm{P}_k = \left(\sum_{j=k}^T \bm{Q}_j \mathbb{E}[\bm{m}_j\bm{m}_j^\top \mid \mathcal{F}_{k-1}]\bm{Q}_j^\top \right)^{\frac{1}{2}}.
% \end{align*}
% I can guarantee that this step is correct. And secondly, prove that
% \begin{align*}
% \|\bm{A}\bm{P}_k\| \lesssim (T-k+1)^{\frac{1}{2}}.
% \end{align*}
% THis is also guaranteed to be true. And the last remaining part is to prove that
% \begin{align*}
% \|\bm{P}_k^{-1} \bm{Q}_k\| \lesssim (T-k+1)^{-\frac{1}{2}}.
% \end{align*}
% }
Following the decomposition of the term $I_2$ in the proof of Theorem \ref{thm:TD-whp}, we represent $I_3$ as
\begin{align*}
&\frac{1}{\sqrt{T}}\sum_{i=1}^T \bm{AQ}_i(\bm{A}_i \bm{\theta}^\star - \bm{b}_i )\\
 &= \underset{I_{31}}{\underbrace{\frac{1}{\sqrt{T}}\sum_{i=1}^T \bm{AQ}_i \bm{m}_i}} + \underset{I_{32}}{\underbrace{\frac{\bm{A}}{\sqrt{T}}(\bm{Q}_0 \mathbb{E}_0[\bm{U}_1] - \bm{Q}_T \mathbb{E}_T [\bm{U}_{T+1}])}}+ \underset{I_{33}}{\underbrace{\frac{\bm{A}}{\sqrt{T}}\sum_{i=1}^T (\bm{Q}_i - \bm{Q}_{i-1})\mathbb{E}_{i-1}[\bm{U}_i]}}
\end{align*}
where $\bm{U}_i$ is defined as in \eqref{eq:defn-Ui} and $\bm{m}_i$ is defined as in \eqref{eq:defn-mi}. Here, the norm of $I_{32}$ and $I_{33}$ can be bounded by $O(T^{-\frac{1}{2}})$ almost surely; it now boils down to the term $I_{31}$, for which we aim to apply Corollary \ref{thm:Berry-Esseen-mtg}. Specifically, let 
\begin{align*}
\bm{f}_i(s_i,s_{i-1}) = \bm{AQ}_i \bm{m}_i,
\end{align*}
it is easy to verify that for all $i \in [T]$,
\begin{align*}
\|\bm{f}_i(s_i,s_{i-1})\|_2 &\leq \|\bm{A}\bm{Q}_i \bm{m}_i\|_2  \\ 
&\leq \|\bm{AQ}_i\|\|\bm{m}_i\|_2\\ 
&\leq (2+O(i^{\alpha-1})) \cdot \frac{m}{1-\rho}(2\|\bm{\theta}^\star\|_2+1)\\ 
&\lesssim \eta_0 \left(\frac{1}{\beta}\right)^{\frac{1}{1-\alpha}}\frac{m}{1-\rho}(2\|\bm{\theta}^\star\|_2+1), \quad \text{a.s.}
\end{align*}
Meanwhile,
\begin{align*}
&\frac{1}{T}\sum_{i=1}^T \mathbb{E}[\bm{f}_i\bm{f}_i^\top] = \bm{A}\tilde{\bm{\Lambda}}_T \bm{A}^\top, \quad \text{with} \\ 
&\|\bm{A}\tilde{\bm{\Lambda}}_T \bm{A}^\top - \tilde{\bm{\Gamma}}\| \leq O(T^{\alpha-1}).
\end{align*} 
%\yuting{similar to Ale's comment before. Specify matrix residual is in what norm.} \weichen{checked.}
and when $T$ satisfies \eqref{eq:Lambda-T-condition}, it can be guaranteed that $\lambda_{\min}(\bm{A}\tilde{\bm{\Lambda}}_T \bm{A}^\top) \geq \frac{1}{2}\lambda_{\min}(\tilde{\bm{\Gamma}})$. Hence, a direct application of Corollary \ref{thm:Berry-Esseen-mtg} reveals that
\begin{align}\label{eq:markov-Berry-Esseen-I31}
d_{\mathsf{C}}\left(\frac{1}{\sqrt{T}}\sum_{i=1}^T \bm{AQ}_i \bm{m}_i,\mathcal{N}(\bm{0},\bm{A}\tilde{\bm{\Lambda}}_T \bm{A}^\top)\right) \leq \widetilde{C}_3 T^{-\frac{1}{4}}\log T + o(T^{-\frac{1}{4}}),
\end{align}
where $\widetilde{C}_3$ is a problem-related quantity
\begin{align}
\widetilde{C}_3 &= \Bigg\{ \left(\frac{p}{(p-1)(1-\lambda)}\log\left(d\left\|\frac{\mathrm{d}\nu}{\mathrm{d}\mu}\right\|_{\mu,p}\right)\right)^{\frac{1}{4}}\cdot \frac{m}{1-\rho}(2\|\bm{\theta}^\star\|_2+1) \nonumber \\ 
&+ \sqrt{\frac{m}{1-\rho}(2\|\bm{\theta}^\star\|_2+1)} \cdot \eta_0 \left(\frac{1}{(1-\gamma)\lambda_0\eta_0}\right)^{\frac{1}{2(1-\alpha)}} \log^{\frac{1}{4}}(d\|\tilde{\bm{\Gamma}}\|) \Bigg\} \cdot \sqrt{d}\|\tilde{\bm{\Gamma}}\|_{\mathsf{F}}^{\frac{1}{2}}.\label{eq:Berry-Esseen-C3}
\end{align}

\paragraph{Completing the proof.} 
The proof now boils down to combining the convergence rate of $I_1,I_2$ and the Berry-Esseen bound on $I_3$. For simplicity, we denote
\begin{align*}
\bm{\delta}_T := \sqrt{T} \bm{A}\bar{\bm{\Delta}}_T - \frac{1}{\sqrt{T}}\sum_{i=1}^T\bm{AQ}_i{\bm{m}}_i = I_1 - I_2 - I_{32} - I_{33}.
\end{align*}
From the previous calculations, we have shown that
\begin{align*}
\mathbb{P}\left(\left\|\bm{\delta_T} \right\|_2 \gtrsim \widetilde{C}_1'T^{\frac{1}{2}-\alpha}\log T + \widetilde{C}_2'T^{-\frac{\alpha}{3}}\log T + o(T^{\frac{1}{2}-\alpha} + T^{-\frac{\alpha}{3}})\right)
 \lesssim \widetilde{C}_2'T^{-\frac{\alpha}{3}}\log T + o(T^{\frac{1}{2}-\alpha} + T^{-\frac{\alpha}{3}}),
\end{align*}
and that
\begin{align*}
\sup_{\mathcal{A} \in \mathscr{C}}\left|\mathbb{P}\left(\frac{1}{\sqrt{T}}\sum_{i=1}^T{\bm{AQ}_i\bm{m}}_i\in \mathcal{A}\right) - \mathbb{P}(\bm{A}\widetilde{\bm{\Lambda}}_T^{ \frac{1}{2}}\bm{z} \in \mathcal{A})\right| \leq \widetilde{C}_3 T^{-\frac{1}{4}}{\log T} + o(T^{-\frac{1}{4}}).
\end{align*}
We now combine these two results to bound the difference between the distributions of $\sqrt{T} \bm{A}\bar{\bm{\Delta}}_T$ and $\mathcal{N}(\bm{0},\bm{A}\widetilde{\bm{\Lambda}}_T\bm{A}^\top)$. Considering any convex set $\mathcal{A} \subset \mathbb{R}^d$, define
\begin{align*}
&\mathcal{A}^{\varepsilon} := \{\bm{x} \in \mathbb{R}^d: \inf_{y \in \mathcal{A}} \|\bm{x} - \bm{y}\|_2 \leq \varepsilon\}, \quad \text{and}  \qquad \mathcal{A}^{-\varepsilon}:=\{\bm{x} \in \mathbb{R}^d: B(\bm{x},\varepsilon) \subset \mathcal{A}\}. 
\end{align*}
% For any $\bm{r} \in \mathbb{R}^d$ and any $\varepsilon>0$, define
% \begin{align*}
% &\mathcal{R}_{\bm{r}}:= \prod_{i=1}^d (-\infty, r_i),\\
% &\mathcal{R}_{\bm{r}+\varepsilon}:= \prod_{i=1}^d (-\infty, r_i+\varepsilon) \quad \text{and} \\ 
% &\mathcal{R}_{\bm{r}-\varepsilon}:= \prod_{i=1}^d (-\infty, r_i-\varepsilon).
% \end{align*}
Direct calculation yields
\begin{align*}
\mathbb{P}(\sqrt{T}\bm{A}\bar{\bm{\Delta}}_T \in \mathcal{A}) &= \mathbb{P}(\sqrt{T}\bm{A}\bar{\bm{\Delta}}_T \in \mathcal{A},\|\bm{\delta}_T\|_2 > \varepsilon) + \mathbb{P}(\sqrt{T}\bm{A}\bar{\bm{\Delta}}_T \in \mathcal{A},\|\bm{\delta}_T\|_2 \leq \varepsilon) \\ 
&\leq \mathbb{P}(\|\bm{\delta}_T\|_2 > \varepsilon) + \mathbb{P}(\sqrt{T}\bm{A}\bar{\bm{\Delta}}_T \in \mathcal{A},\|\bm{\delta}_T\|_2 \leq \varepsilon).
\end{align*}
Here, the triangle inequality implies
\begin{align*}
\left(\sqrt{T}\bm{A}\bar{\bm{\Delta}}_T \in \mathcal{A},\|\bm{\delta}_T\|_2 \leq \varepsilon \right) \Rightarrow \frac{1}{\sqrt{T}}\sum_{i=1}^T\bm{AQ}_i{\bm{m}}_i \in \mathcal{A}^{\varepsilon}.
\end{align*}
Hence, $\mathbb{P}(\sqrt{T}\bm{A}\bar{\bm{\Delta}}_T \in \mathcal{A})$ is upper bounded by
\begin{align*}
\mathbb{P}(\sqrt{T}\bm{A}\bar{\bm{\Delta}}_T \in \mathcal{A}) 
&\leq \mathbb{P}(\|\bm{\delta}_T\|_2 > \varepsilon) + \mathbb{P}\left(\frac{1}{\sqrt{T}}\sum_{i=1}^T{\bm{AQ}_i\bm{m}}_i \in \mathcal{A}^{\varepsilon}\right)\\
&\leq \mathbb{P}(\|\bm{\delta}_T\|_2 > \varepsilon) + \mathbb{P}\left(\bm{A}\widetilde{\bm{\Lambda}}_T^{\frac{1}{2}}\bm{z} \in \mathcal{A}^{\varepsilon}\right)+\widetilde{C}_3 T^{-\frac{1}{4}}{\log T}  + o(T^{-\frac{1}{4}})\\ 
&\leq \mathbb{P}(\|\bm{\delta}_T\|_2 > \varepsilon) + \mathbb{P}\left(\bm{A}\widetilde{\bm{\Lambda}}_T^{\frac{1}{2}}\bm{z} \in \mathcal{A}\right) + \|\tilde{\bm{\Gamma}}\|_{\mathsf{F}}^{\frac{1}{2}}\varepsilon +\widetilde{C}_3 T^{-\frac{1}{4}}{\log T}  + o(T^{-\frac{1}{4}}),
\end{align*}
where we invoked Theorem \ref{thm:Gaussian-reminder} in the last inequality. By letting 
\begin{align*}
\varepsilon \asymp \widetilde{C}_1'T^{\frac{1}{2}-\alpha}\log T + \widetilde{C}_2'T^{-\frac{\alpha}{3}}\log T + o(T^{\frac{1}{2}-\alpha} + T^{-\frac{\alpha}{3}}),
\end{align*}
we obtain
\begin{align*}
\mathbb{P}(\sqrt{T}\bm{A}\bar{\bm{\Delta}}_T \in \mathcal{A}) 
&\leq \widetilde{C}_2'T^{-\frac{\alpha}{3}}\log T + o(T^{\frac{1}{2}-\alpha} + T^{-\frac{\alpha}{3}})  + \mathbb{P}\left(\bm{A}\widetilde{\bm{\Lambda}}_T^{\frac{1}{2}}\bm{z} \in \mathcal{A}\right) \\ 
&+ \|\tilde{\bm{\Gamma}}\|_{\mathsf{F}}^{\frac{1}{2}} \left(\widetilde{C}_1'T^{\frac{1}{2}-\alpha}\log T + \widetilde{C}_2'T^{-\frac{\alpha}{3}}\log T + o(T^{\frac{1}{2}-\alpha} + T^{-\frac{\alpha}{3}})\right) +\widetilde{C}_3 T^{-\frac{1}{4}}{\log T}  + o(T^{-\frac{1}{4}}) \\ 
&= \mathbb{P}\left(\bm{A}\widetilde{\bm{\Lambda}}_T^{\frac{1}{2}}\bm{z} \in \mathcal{A}\right) + (\widetilde{C}_1 T^{\frac{1}{2}-\alpha} + \widetilde{C}_2 T^{-\frac{\alpha}{3}}+ \widetilde{C}_3 T^{-\frac{1}{4}}){\log T}  + o(T^{\frac{1}{2}-\alpha} + T^{-\frac{\alpha}{3}} + T^{-\frac{1}{4}}).
\end{align*}
Here, in the last equality, we denote, for simplicity,
\begin{align}
&\widetilde{C}_1 = \|\tilde{\bm{\Gamma}}\|_{\mathsf{F}}^{\frac{1}{2}}\widetilde{C}_1' = \|\tilde{\bm{\Gamma}}\|_{\mathsf{F}}^{\frac{1}{2}}\frac{\eta_0}{1-\rho}(2\|\bm{\theta}^\star\|_2+1), \quad \text{and} \label{eq:Berry-Esseen-C1} \\ 
&\widetilde{C}_2 = (\|\tilde{\bm{\Gamma}}\|_{\mathsf{F}}^{\frac{1}{2}} + 1) \widetilde{C}_2' = (\|\tilde{\bm{\Gamma}}\|_{\mathsf{F}}^{\frac{1}{2}} + 1)\frac{1}{1-\rho} \left(\frac{\eta_0(2\|\bm{\theta}^\star\|_2+1)^2}{\lambda_0(1-\gamma)}\right)^{\frac{1}{3}} \label{eq:Berry-Esseen-C2}.
\end{align}
Using the same technique as in the proof of Corollary \ref{thm:Berry-Esseen-mtg}, a lower bound can be derived symmetrically. By taking a supremum over $\mathcal{A} \in \mathscr{C}$, it can be guaranteed that
\begin{align}\label{eq:TD-Berry-Esseen-intermediate}
d_\mathsf{C}(\sqrt{T}\bm{A}\bar{\bm{\Delta}}_T,\mathcal{N}(\bm{0},\bm{A}\tilde{\bm{\Lambda}}_T \bm{A}^\top)) &\lesssim (\widetilde{C}_1 T^{\frac{1}{2}-\alpha} + \widetilde{C}_2 T^{-\frac{\alpha}{3}}+ \widetilde{C}_3 T^{-\frac{1}{4}}){\log T} + o(T^{\frac{1}{2}-\alpha} + T^{-\frac{\alpha}{3}} + T^{-\frac{1}{4}}).
\end{align}
%\yuting{why do you have $\lesssim$ and those $\widetilde{C}_i$?}\weichen{$\tilde{C}_i$ does not include universal constants.}
Further combining \eqref{eq:TD-Berry-Esseen-intermediate} with \eqref{eq:TD-Berry-Esseen-decompose} and Lemma \ref{lemma:Gaussian-comparison}, we obtain the Berry-Esseen bound
\begin{align*}
d_{\mathsf{C}}(\sqrt{T}\bar{\bm{\Delta}}_T,\mathcal{N}(\bm{0},\tilde{\bm{\Lambda}}^\star)) &\lesssim (\widetilde{C}_1 T^{\frac{1}{2}-\alpha} + \widetilde{C}_2 T^{-\frac{\alpha}{3}}+ \widetilde{C}_3 T^{-\frac{1}{4}} + \widetilde{C}_4 T^{\alpha-1}){\log T} \\ 
& + o(T^{\frac{1}{2}-\alpha} + T^{-\frac{\alpha}{3}} + T^{-\frac{1}{4}} + T^{\alpha-1})
\end{align*}
with $\widetilde{C}_1$, $\widetilde{C}_2$, $\widetilde{C}_3$ defined as in \eqref{eq:Berry-Esseen-C1}, \eqref{eq:Berry-Esseen-C2}, \eqref{eq:Berry-Esseen-C3} respectively, and
\begin{align*}
\widetilde{C}_4 = \frac{\sqrt{d\mathsf{cond}(\bm{\tilde{\Gamma}})}}{(1-\gamma)\lambda_0\eta_0}.
\end{align*}
Finally, when $\alpha = \frac{3}{4}$, we have, coincidentally,
\begin{align*}
\frac{1}{2}-\alpha = -\frac{\alpha}{3} = -\frac{1}{4} = \alpha-1.
\end{align*} 
Hence, Theorem \ref{thm:TD-berry-esseen} follows from taking $\tilde{C} = \max\{\tilde{C}_1,\tilde{C}_2,\tilde{C}_3,\tilde{C}_4\}$.

\subsection{Proof of Relation~\eqref{eq:TD-Berry-Esseen-tight}}\label{app:proof-Berry-Esseen-tight} 
Following the same logic as Appendix B.4.1 in \cite{wu2024statistical}, we can obtain a lower bound on the difference between $\mathcal{N}(\bm{0},\tilde{\bm{\Lambda}}_T)$ and $\mathcal{N}(\bm{0},\tilde{\bm{\Lambda}}^\star)$. Specifically, when $T$ is sufficiently large,
\begin{align*}
d_{\mathsf{C}}(\mathcal{N}(\bm{0},\tilde{\bm{\Lambda}}_T),\mathcal{N}(\bm{0},\tilde{\bm{\Lambda}}^\star)) = d_{\mathsf{TV}}(\mathcal{N}(\bm{0},\tilde{\bm{\Lambda}}_T),\mathcal{N}(\bm{0},\tilde{\bm{\Lambda}}^\star)) \gtrsim O(T^{\alpha-1}).
\end{align*}
Meanwhile, when $\alpha > \frac{3}{4}$, both $\frac{1}{2}-\alpha$ and $-\frac{\alpha}{3}$ are less than $-\frac{1}{4}$. Therefore, the upper bound \eqref{eq:TD-Berry-Esseen-intermediate} is transformed as
\begin{align*}
d_\mathsf{C}(\sqrt{T}\bar{\bm{\Delta}}_T,\mathcal{N}(\bm{0},\tilde{\bm{\Lambda}}_T )) \leq O(T^{-\frac{1}{4}}).
\end{align*}
In combination, the triangle inequality reveals
\begin{align*}
d_{\mathsf{C}}(\sqrt{T}\bar{\bm{\Delta}}_T,\mathcal{N}(\bm{0},\tilde{\bm{\Lambda}}^\star)) &\geq d_{\mathsf{C}}(\mathcal{N}(\bm{0},\tilde{\bm{\Lambda}}_T),\mathcal{N}(\bm{0},\tilde{\bm{\Lambda}}^\star)) - d_\mathsf{C}(\sqrt{T}\bar{\bm{\Delta}}_T,\mathcal{N}(\bm{0},\tilde{\bm{\Lambda}}_T ))\\ 
&\gtrsim O(T^{\alpha-1}) - O(T^{-\frac{1}{4}})\\ 
&\gtrsim O(T^{\alpha-1}) \gtrsim O(T^{-\frac{1}{4}}).
\end{align*}
Here, in the last line, we applied the fact that since $\alpha > \frac{3}{4}$, $\alpha - 1 \geq -\frac{1}{4}$.

\paragraph{Choose of stepsizes.} We conclude by noting that choice of the stepsize in Theorem \ref{thm:TD-whp} and Theorem \ref{thm:TD-berry-esseen} are different: in Theorem \ref{thm:TD-whp}, $\eta_0$ depends on $\delta$ and other problem-related quantities like $\lambda_0$ and $\gamma$, and $\alpha$ can take any value between $\frac{1}{2}$ and $1$; however, in Theorem \ref{thm:TD-berry-esseen}, the initial stepsize $\eta_0$ can take any value less than $1/2\lambda_{\Sigma}$, while $\alpha$ is set to the specific value of $\frac{3}{4}$. In fact, our proof of Theorem~\ref{thm:TD-berry-esseen} allows for a general choice of $\alpha$; however, using other values of $\alpha$ other than $3/4$ appears to be suboptimal.


%Input Basic Packages

% \usepackage{natbib}
\usepackage{float}

%Input Math and Algo Related packages

\usepackage{amsmath, mathtools, amsthm}
\usepackage{algorithm, times}
\usepackage[noend]{algpseudocode}
% \usepackage[noend]{algorithmic}


% \allowdisplaybreaks
%Input Packages to Handle Images and Graphs

% \usepackage{wrapfig, epsfig, graphicx, subcaption, pgf, tikz, pgfplots}

%Tikz Setup

% \usetikzlibrary{calc, shapes.geometric, arrows, automata}
% \pgfplotsset{compat=1.18}

%Include additional various packages

% \usepackage{enumerate, thmtools, soul, setspace}

%Hyperref Setup

% \usepackage{hyperref}
% \hypersetup{
%     colorlinks=true,
%     linkcolor=blue,
%     filecolor=blue,      
%     urlcolor=blue,
%     citecolor=blue,
%     pdftitle={Notes - Topic},
%     pdfpagemode=FullScreen,
%     }

% Load the parskip package with skip and indent options
% \usepackage[skip=10pt, indent=15pt]{parskip}
\setlength{\marginparwidth}{2pc}
% \usepackage[textwidth=2pc,textsize=tiny]{todonotes}


% Include Theorem related packages

\theoremstyle{plain}
\newtheorem{theorem}{Theorem}
\newtheorem{lemma}[theorem]{Lemma}
\newtheorem{corollary}[theorem]{Corollary}
\newtheorem{proposition}[theorem]{Proposition}

\theoremstyle{definition}
\newtheorem{definition}[theorem]{Definition}
\newtheorem{example}[theorem]{Example}
\newtheorem{notation}[theorem]{Notation}
\newtheorem{problem}[theorem]{Problem}
\newtheorem{assumption}{Assumption}

% \theoremstyle{remark}
% \newtheorem{remark}[theorem]{Remark}

% \newcommand\numberthis{\addtocounter{equation}{1}\tag{\theequation}}

%Redefining Command

% \renewcommand{\baselinestretch}{1.25}
% \renewcommand{\chaptermark}[1]{\markboth{#1}{}}
% \renewcommand{\sectionmark}[1]{\markright{\thesection\ #1}}

%Citations Maintenance

% \usepackage[nottoc]{tocbibind}




%Little Boxes

\usepackage[most]{tcolorbox}
\newtcolorbox{idea}[1][]
{
colbacktitle=cyan,
colback=cyan!10,
arc=1pt,
boxrule=1pt,
title=#1 % I would like to make this (one of these in general) assignment optional depending on #1, #2...
}



\newtcolorbox{update}[1][]
{
colbacktitle=gray,
colback=gray!10,
arc=1pt,
boxrule=1pt,
title=#1 % I would like to make this (one of these in general) assignment optional depending on #1, #2...
}


\newtcolorbox{question}[1][]
{
coltitle=black,
colbacktitle=yellow,
colback=yellow!10,
arc=1pt,
boxrule=1pt,
title=#1 % I would like to make this (one of these in general) assignment optional depending on #1, #2...
}

\newtcolorbox{answer}[1][]
{
coltitle=black,
colbacktitle=violet!10,
colback=violet!5,
arc=1pt,
boxrule=1pt,
title=#1 % I would like to make this (one of these in general) assignment optional depending on #1, #2...
}

% \newtcolorbox{note}[1][]
% {
% coltitle=black,
% colbacktitle=green,
% colback=green!10,
% arc=1pt,
% boxrule=1pt,
% title=#1 % I would like to make this (one of these in general) assignment optional depending on #1, #2...
% }

\newcommand{\colnote}[3]{\textcolor{#1}{{\small $\ll$\textsf{#2}$\gg$\marginpar{\tiny\bf \textcolor[rgb]{1.00,0.00,0.00}{#3}}}}}
%Bold Symbols

\newcommand{\ba}{\boldsymbol{a}}
\newcommand{\bb}{\boldsymbol{b}}
\newcommand{\bc}{\boldsymbol{c}}
\newcommand{\bd}{\boldsymbol{d}}
\newcommand{\be}{\boldsymbol{e}}

\newcommand{\bg}{\boldsymbol{g}}
\newcommand{\bh}{\boldsymbol{h}}
\newcommand{\bi}{\boldsymbol{i}}
\newcommand{\bj}{\boldsymbol{j}}
\newcommand{\bl}{\boldsymbol{l}}
\newcommand{\bm}{\boldsymbol{m}}
\newcommand{\bn}{\boldsymbol{n}}
\newcommand{\bo}{\boldsymbol{o}}
\newcommand{\bp}{\boldsymbol{p}}
\newcommand{\bq}{\boldsymbol{q}}
\newcommand{\br}{\boldsymbol{r}}
\newcommand{\bs}{\boldsymbol{s}}
\newcommand{\bt}{\boldsymbol{t}}
\newcommand{\bu}{\boldsymbol{u}}
\newcommand{\bv}{\boldsymbol{v}}
\newcommand{\bw}{\boldsymbol{w}}
\newcommand{\bx}{\boldsymbol{x}}
\newcommand{\by}{\boldsymbol{y}}
\newcommand{\bz}{\boldsymbol{z}}


\newcommand{\bA}{\boldsymbol{A}}
\newcommand{\bB}{\boldsymbol{B}}
\newcommand{\bC}{\boldsymbol{C}}
\newcommand{\bD}{\boldsymbol{D}}
\newcommand{\bE}{\boldsymbol{E}}
\newcommand{\bF}{\boldsymbol{F}}
\newcommand{\bG}{\boldsymbol{G}}
\newcommand{\bH}{\boldsymbol{H}}
\newcommand{\bI}{\boldsymbol{I}}
\newcommand{\bJ}{\boldsymbol{J}}
\newcommand{\bK}{\boldsymbol{K}}
\newcommand{\bL}{\boldsymbol{L}}
\newcommand{\bM}{\boldsymbol{M}}
\newcommand{\bN}{\boldsymbol{N}}
\newcommand{\bO}{\boldsymbol{O}}
\newcommand{\bP}{\boldsymbol{P}}
\newcommand{\bQ}{\boldsymbol{Q}}
\newcommand{\bR}{\boldsymbol{R}}
\newcommand{\bS}{\boldsymbol{S}}
\newcommand{\bT}{\boldsymbol{T}}
\newcommand{\bU}{\boldsymbol{U}}
\newcommand{\bV}{\boldsymbol{V}}
\newcommand{\bW}{\boldsymbol{W}}
\newcommand{\bX}{\boldsymbol{X}}
\newcommand{\bY}{\boldsymbol{Y}}
\newcommand{\bZ}{\boldsymbol{Z}}

\newcommand{\balpha}{\boldsymbol{\alpha}}
\newcommand{\bbeta}{\boldsymbol{\beta}}
\newcommand{\bmu}{\boldsymbol{\mu}}
\newcommand{\bpi}{\boldsymbol{\pi}}
\newcommand{\bSigma}{\boldsymbol{\Sigma}}
\newcommand{\bsigma}{\boldsymbol{\sigma}}
\newcommand{\btheta}{\boldsymbol{\theta}}
\newcommand{\bTheta}{\boldsymbol{\Theta}}

%Mathcal Commands
\newcommand{\sA}{\mathcal{A}}
\newcommand{\sB}{\mathcal{B}}
\newcommand{\sC}{\mathcal{C}}
\newcommand{\sD}{\mathcal{D}}
\newcommand{\sE}{\mathcal{E}}
\newcommand{\sF}{\mathcal{F}}
\newcommand{\sG}{\mathcal{G}}
\newcommand{\sH}{\mathcal{H}}
\newcommand{\sI}{\mathcal{I}}
\newcommand{\sJ}{\mathcal{J}}
\newcommand{\sK}{\mathcal{K}}
\newcommand{\sL}{\mathcal{L}}
\newcommand{\sM}{\mathcal{M}}
\newcommand{\sN}{\mathcal{N}}
\newcommand{\sO}{\mathcal{O}}
\newcommand{\sP}{\mathcal{P}}
\newcommand{\sQ}{\mathcal{Q}}
\newcommand{\sR}{\mathcal{R}}
\newcommand{\sS}{\mathcal{S}}
\newcommand{\sT}{\mathcal{T}}
\newcommand{\sU}{\mathcal{U}}
\newcommand{\sV}{\mathcal{V}}
\newcommand{\sW}{\mathcal{W}}
\newcommand{\sX}{\mathcal{X}}
\newcommand{\sY}{\mathcal{Y}}
\newcommand{\sZ}{\mathcal{Z}}

%Further Specific Fonts (Miscellaneous)

\newcommand{\bhatY}{\boldsymbol{\hat{Y}}}
\newcommand{\bbary}{\boldsymbol{\bar{y}}}
\newcommand{\bhatX}{\boldsymbol{\hat{X}}}
\newcommand{\bbarx}{\boldsymbol{\bar{x}}}
\newcommand{\bbarZ}{\boldsymbol{\bar{Z}}}
\newcommand{\bbarz}{\boldsymbol{\bar{z}}}
\newcommand{\barz}{\bar{z}}
\newcommand{\bbarS}{\boldsymbol{\bar{S}}}


\newcommand{\bzero}{\boldsymbol{0}}
\newcommand{\bbartheta}{\boldsymbol{\bar{\theta}}}

\newcommand{\bfB}{\mathbf{B}}
\newcommand{\sbarZ}{\bar{\mathcal{Z}}}
\newcommand{\fbag}{\bold{F}}

%
\setlength\unitlength{1mm}
\newcommand{\twodots}{\mathinner {\ldotp \ldotp}}
% bb font symbols
\newcommand{\Rho}{\mathrm{P}}
\newcommand{\Tau}{\mathrm{T}}

\newfont{\bbb}{msbm10 scaled 700}
\newcommand{\CCC}{\mbox{\bbb C}}

\newfont{\bb}{msbm10 scaled 1100}
\newcommand{\CC}{\mbox{\bb C}}
\newcommand{\PP}{\mbox{\bb P}}
\newcommand{\RR}{\mbox{\bb R}}
\newcommand{\QQ}{\mbox{\bb Q}}
\newcommand{\ZZ}{\mbox{\bb Z}}
\newcommand{\FF}{\mbox{\bb F}}
\newcommand{\GG}{\mbox{\bb G}}
\newcommand{\EE}{\mbox{\bb E}}
\newcommand{\NN}{\mbox{\bb N}}
\newcommand{\KK}{\mbox{\bb K}}
\newcommand{\HH}{\mbox{\bb H}}
\newcommand{\SSS}{\mbox{\bb S}}
\newcommand{\UU}{\mbox{\bb U}}
\newcommand{\VV}{\mbox{\bb V}}


\newcommand{\yy}{\mathbbm{y}}
\newcommand{\xx}{\mathbbm{x}}
\newcommand{\zz}{\mathbbm{z}}
\newcommand{\sss}{\mathbbm{s}}
\newcommand{\rr}{\mathbbm{r}}
\newcommand{\pp}{\mathbbm{p}}
\newcommand{\qq}{\mathbbm{q}}
\newcommand{\ww}{\mathbbm{w}}
\newcommand{\hh}{\mathbbm{h}}
\newcommand{\vvv}{\mathbbm{v}}

% Vectors

\newcommand{\av}{{\bf a}}
\newcommand{\bv}{{\bf b}}
\newcommand{\cv}{{\bf c}}
\newcommand{\dv}{{\bf d}}
\newcommand{\ev}{{\bf e}}
\newcommand{\fv}{{\bf f}}
\newcommand{\gv}{{\bf g}}
\newcommand{\hv}{{\bf h}}
\newcommand{\iv}{{\bf i}}
\newcommand{\jv}{{\bf j}}
\newcommand{\kv}{{\bf k}}
\newcommand{\lv}{{\bf l}}
\newcommand{\mv}{{\bf m}}
\newcommand{\nv}{{\bf n}}
\newcommand{\ov}{{\bf o}}
\newcommand{\pv}{{\bf p}}
\newcommand{\qv}{{\bf q}}
\newcommand{\rv}{{\bf r}}
\newcommand{\sv}{{\bf s}}
\newcommand{\tv}{{\bf t}}
\newcommand{\uv}{{\bf u}}
\newcommand{\wv}{{\bf w}}
\newcommand{\vv}{{\bf v}}
\newcommand{\xv}{{\bf x}}
\newcommand{\yv}{{\bf y}}
\newcommand{\zv}{{\bf z}}
\newcommand{\zerov}{{\bf 0}}
\newcommand{\onev}{{\bf 1}}

% Matrices

\newcommand{\Am}{{\bf A}}
\newcommand{\Bm}{{\bf B}}
\newcommand{\Cm}{{\bf C}}
\newcommand{\Dm}{{\bf D}}
\newcommand{\Em}{{\bf E}}
\newcommand{\Fm}{{\bf F}}
\newcommand{\Gm}{{\bf G}}
\newcommand{\Hm}{{\bf H}}
\newcommand{\Id}{{\bf I}}
\newcommand{\Jm}{{\bf J}}
\newcommand{\Km}{{\bf K}}
\newcommand{\Lm}{{\bf L}}
\newcommand{\Mm}{{\bf M}}
\newcommand{\Nm}{{\bf N}}
\newcommand{\Om}{{\bf O}}
\newcommand{\Pm}{{\bf P}}
\newcommand{\Qm}{{\bf Q}}
\newcommand{\Rm}{{\bf R}}
\newcommand{\Sm}{{\bf S}}
\newcommand{\Tm}{{\bf T}}
\newcommand{\Um}{{\bf U}}
\newcommand{\Wm}{{\bf W}}
\newcommand{\Vm}{{\bf V}}
\newcommand{\Xm}{{\bf X}}
\newcommand{\Ym}{{\bf Y}}
\newcommand{\Zm}{{\bf Z}}

% Calligraphic

\newcommand{\Ac}{{\cal A}}
\newcommand{\Bc}{{\cal B}}
\newcommand{\Cc}{{\cal C}}
\newcommand{\Dc}{{\cal D}}
\newcommand{\Ec}{{\cal E}}
\newcommand{\Fc}{{\cal F}}
\newcommand{\Gc}{{\cal G}}
\newcommand{\Hc}{{\cal H}}
\newcommand{\Ic}{{\cal I}}
\newcommand{\Jc}{{\cal J}}
\newcommand{\Kc}{{\cal K}}
\newcommand{\Lc}{{\cal L}}
\newcommand{\Mc}{{\cal M}}
\newcommand{\Nc}{{\cal N}}
\newcommand{\nc}{{\cal n}}
\newcommand{\Oc}{{\cal O}}
\newcommand{\Pc}{{\cal P}}
\newcommand{\Qc}{{\cal Q}}
\newcommand{\Rc}{{\cal R}}
\newcommand{\Sc}{{\cal S}}
\newcommand{\Tc}{{\cal T}}
\newcommand{\Uc}{{\cal U}}
\newcommand{\Wc}{{\cal W}}
\newcommand{\Vc}{{\cal V}}
\newcommand{\Xc}{{\cal X}}
\newcommand{\Yc}{{\cal Y}}
\newcommand{\Zc}{{\cal Z}}

% Bold greek letters

\newcommand{\alphav}{\hbox{\boldmath$\alpha$}}
\newcommand{\betav}{\hbox{\boldmath$\beta$}}
\newcommand{\gammav}{\hbox{\boldmath$\gamma$}}
\newcommand{\deltav}{\hbox{\boldmath$\delta$}}
\newcommand{\etav}{\hbox{\boldmath$\eta$}}
\newcommand{\lambdav}{\hbox{\boldmath$\lambda$}}
\newcommand{\epsilonv}{\hbox{\boldmath$\epsilon$}}
\newcommand{\nuv}{\hbox{\boldmath$\nu$}}
\newcommand{\muv}{\hbox{\boldmath$\mu$}}
\newcommand{\zetav}{\hbox{\boldmath$\zeta$}}
\newcommand{\phiv}{\hbox{\boldmath$\phi$}}
\newcommand{\psiv}{\hbox{\boldmath$\psi$}}
\newcommand{\thetav}{\hbox{\boldmath$\theta$}}
\newcommand{\tauv}{\hbox{\boldmath$\tau$}}
\newcommand{\omegav}{\hbox{\boldmath$\omega$}}
\newcommand{\xiv}{\hbox{\boldmath$\xi$}}
\newcommand{\sigmav}{\hbox{\boldmath$\sigma$}}
\newcommand{\piv}{\hbox{\boldmath$\pi$}}
\newcommand{\rhov}{\hbox{\boldmath$\rho$}}
\newcommand{\upsilonv}{\hbox{\boldmath$\upsilon$}}

\newcommand{\Gammam}{\hbox{\boldmath$\Gamma$}}
\newcommand{\Lambdam}{\hbox{\boldmath$\Lambda$}}
\newcommand{\Deltam}{\hbox{\boldmath$\Delta$}}
\newcommand{\Sigmam}{\hbox{\boldmath$\Sigma$}}
\newcommand{\Phim}{\hbox{\boldmath$\Phi$}}
\newcommand{\Pim}{\hbox{\boldmath$\Pi$}}
\newcommand{\Psim}{\hbox{\boldmath$\Psi$}}
\newcommand{\Thetam}{\hbox{\boldmath$\Theta$}}
\newcommand{\Omegam}{\hbox{\boldmath$\Omega$}}
\newcommand{\Xim}{\hbox{\boldmath$\Xi$}}


% Sans Serif small case

\newcommand{\Gsf}{{\sf G}}

\newcommand{\asf}{{\sf a}}
\newcommand{\bsf}{{\sf b}}
\newcommand{\csf}{{\sf c}}
\newcommand{\dsf}{{\sf d}}
\newcommand{\esf}{{\sf e}}
\newcommand{\fsf}{{\sf f}}
\newcommand{\gsf}{{\sf g}}
\newcommand{\hsf}{{\sf h}}
\newcommand{\isf}{{\sf i}}
\newcommand{\jsf}{{\sf j}}
\newcommand{\ksf}{{\sf k}}
\newcommand{\lsf}{{\sf l}}
\newcommand{\msf}{{\sf m}}
\newcommand{\nsf}{{\sf n}}
\newcommand{\osf}{{\sf o}}
\newcommand{\psf}{{\sf p}}
\newcommand{\qsf}{{\sf q}}
\newcommand{\rsf}{{\sf r}}
\newcommand{\ssf}{{\sf s}}
\newcommand{\tsf}{{\sf t}}
\newcommand{\usf}{{\sf u}}
\newcommand{\wsf}{{\sf w}}
\newcommand{\vsf}{{\sf v}}
\newcommand{\xsf}{{\sf x}}
\newcommand{\ysf}{{\sf y}}
\newcommand{\zsf}{{\sf z}}


% mixed symbols

\newcommand{\sinc}{{\hbox{sinc}}}
\newcommand{\diag}{{\hbox{diag}}}
\renewcommand{\det}{{\hbox{det}}}
\newcommand{\trace}{{\hbox{tr}}}
\newcommand{\sign}{{\hbox{sign}}}
\renewcommand{\arg}{{\hbox{arg}}}
\newcommand{\var}{{\hbox{var}}}
\newcommand{\cov}{{\hbox{cov}}}
\newcommand{\Ei}{{\rm E}_{\rm i}}
\renewcommand{\Re}{{\rm Re}}
\renewcommand{\Im}{{\rm Im}}
\newcommand{\eqdef}{\stackrel{\Delta}{=}}
\newcommand{\defines}{{\,\,\stackrel{\scriptscriptstyle \bigtriangleup}{=}\,\,}}
\newcommand{\<}{\left\langle}
\renewcommand{\>}{\right\rangle}
\newcommand{\herm}{{\sf H}}
\newcommand{\trasp}{{\sf T}}
\newcommand{\transp}{{\sf T}}
\renewcommand{\vec}{{\rm vec}}
\newcommand{\Psf}{{\sf P}}
\newcommand{\SINR}{{\sf SINR}}
\newcommand{\SNR}{{\sf SNR}}
\newcommand{\MMSE}{{\sf MMSE}}
\newcommand{\REF}{{\RED [REF]}}

% Markov chain
\usepackage{stmaryrd} % for \mkv 
\newcommand{\mkv}{-\!\!\!\!\minuso\!\!\!\!-}

% Colors

\newcommand{\RED}{\color[rgb]{1.00,0.10,0.10}}
\newcommand{\BLUE}{\color[rgb]{0,0,0.90}}
\newcommand{\GREEN}{\color[rgb]{0,0.80,0.20}}

%%%%%%%%%%%%%%%%%%%%%%%%%%%%%%%%%%%%%%%%%%
\usepackage{hyperref}
\hypersetup{
    bookmarks=true,         % show bookmarks bar?
    unicode=false,          % non-Latin characters in AcrobatÕs bookmarks
    pdftoolbar=true,        % show AcrobatÕs toolbar?
    pdfmenubar=true,        % show AcrobatÕs menu?
    pdffitwindow=false,     % window fit to page when opened
    pdfstartview={FitH},    % fits the width of the page to the window
%    pdftitle={My title},    % title
%    pdfauthor={Author},     % author
%    pdfsubject={Subject},   % subject of the document
%    pdfcreator={Creator},   % creator of the document
%    pdfproducer={Producer}, % producer of the document
%    pdfkeywords={keyword1} {key2} {key3}, % list of keywords
    pdfnewwindow=true,      % links in new window
    colorlinks=true,       % false: boxed links; true: colored links
    linkcolor=red,          % color of internal links (change box color with linkbordercolor)
    citecolor=green,        % color of links to bibliography
    filecolor=blue,      % color of file links
    urlcolor=blue           % color of external links
}
%%%%%%%%%%%%%%%%%%%%%%%%%%%%%%%%%%%%%%%%%%%





%Note Taking Snippets
% \newcommand{\todo}[1]{\textcolor{red}{TODO: #1}}
\newcommand{\fixme}[1]{\textcolor{red}{FIXME: #1}}

%Math Fields
\newcommand{\field}[1]{\mathbb{#1}}
\newcommand{\fY}{\field{Y}}
\newcommand{\fX}{\field{X}}
\newcommand{\fH}{\field{H}}
\newcommand{\R}{\field{R}}
\newcommand{\Nat}{\field{N}}


%Operations
\newcommand\theset[2]{ \left\{ {#1} \,:\, {#2} \right\} }
\newcommand\inn[2]{ \left\langle {#1} \,,\, {#2} \right\rangle }
\newcommand\RE[2]{ D\left({#1} \| {#2}\right) }
\newcommand\Ind[1]{ \left\{{#1}\right\} }
\newcommand{\norm}[1]{\left\|{#1}\right\|}
\newcommand{\ltwonorm}[1]{\left\|{#1}\right\|_2}
\newcommand{\diag}[1]{\mbox{\rm diag}\!\left\{{#1}\right\}}
\newcommand{\func}[3]{{#1} : {#2} \rightarrow {#3}}
% \DeclarePairedDelimiter{\ceil}{\lceil}{\rceil}
\newcommand{\ceil}[1]{\left\lceil{#1}\right\rceil}
\DeclarePairedDelimiter{\floor}{\lfloor}{\rfloor}
\DeclareMathOperator{\Tr}{Tr}
\newcommand{\set}[1]{\left\{{#1}\right\}}
\newcommand{\size}[1]{\left|{#1}\right|}

\newcommand{\defeq}{\stackrel{\rm def}{=}}
\newcommand{\sign}{\mbox{\sc sgn}}


\newcommand{\dt}{\displaystyle}
\newcommand{\wh}{\widehat}
\newcommand{\ve}{\varepsilon}
\newcommand{\hlambda}{\wh{\lambda}}
\newcommand{\yhat}{\wh{y}}
\newcommand{\hDelta}{\wh{\Delta}}
\newcommand{\hdelta}{\wh{\delta}}
\newcommand{\spin}{\{-1,+1\}}

%Color Definitions
\newcommand{\blue}[1]{\textcolor{blue}{#1}}
\newcommand{\red}[1]{\textcolor{red} {#1}}
\newcommand{\green}{\color{OliveGreen}}
\newcommand{\violet}{\color{violet}}

%Notational Shortcuts
\newcommand{\sequence}[2]{{#1}_1,{#1}_2,...,{#1}_{#2}}
\newcommand{\fbrac}[1]{\left({#1}\right)}
\newcommand{\sbrac}[1]{\left\{{#1}\right\}}
\newcommand{\tbrac}[1]{\left[{#1}\right]}
\newcommand{\abs}[1]{\left|{#1}\right|}

%Optimization etc.
\newcommand{\argmin}{\mathop{\mathrm{argmin}}}
\newcommand{\argmax}{\mathop{\mathrm{argmax}}}
\newcommand{\conv}{\mathop{\mathrm{conv}}}
\newcommand{\interior}{\mathop{\mathrm{int}}}
\newcommand{\dom}{\mathop{\mathrm{Dom}}}
\newcommand{\range}{\mathop{\mathrm{Range}}}
\newcommand{\lipschitzconstant}{\ell}
\newcommand{\lipschitz}{$\lipschitzconstant$-lipschitz }


%Online Learning
\DeclareMathOperator{\Regret}{Regret}
\DeclareMathOperator{\Wealth}{Wealth}
\DeclareMathOperator{\Reward}{Reward}
\DeclareMathOperator{\Risk}{Risk}
\DeclareMathOperator{\Prox}{Prox}
\newcommand{\KL}[2]{\operatorname{KL}\left({#1};{#2}\right)}


% KL divergence
\newcommand{\indicator}{\mathbf{1}}
\newcommand{\Wasserstein}[3]{\sW_{#1}\left({#2},{#3}\right)}

%Probability

\DeclareMathOperator*{\Prob}{\field{P}}
\DeclareMathOperator*{\Exp}{\field{E}}
\DeclareMathOperator*{\Var}{\mathrm{Var}}
\newcommand{\sigalg}{\sigma\text{-Algebra}}

%Distributions
\newcommand{\distribution}{\sD}
% \newcommand{\laplace}{\mathrm{Lap}}
\newcommand{\uniform}{\mathrm{Unif}}
\newcommand{\normal}{\sN}
\newcommand{\subgaussian}{Sub-Gaussian }
\newcommand{\subexponential}{Sub-Exponential }

%Random-Approx
\newcommand{\fhat}{\hat{f}}
\newcommand{\invfhat}{\hat{f}^{-1}}
%\newcommand{\approxerror}{\epsilon}
\newcommand{\approxerror}{\varepsilon}
\newcommand{\confidence}{\delta}
\newcommand{\appcon}{\fbrac{\approxerror,\confidence}}
\newcommand{\inverse}[1]{{#1}^{-1}}

\newcommand{\bigo}[1]{O\fbrac{{#1}}}
\newcommand{\bigot}[1]{\widetilde{O}\fbrac{{#1}}}
\newcommand{\bigoted}[1]{\widetilde{O}_{\approxerror,\confidence}\fbrac{{#1}}}
\newcommand{\smallo}[1]{o\fbrac{{#1}}}
\newcommand{\smallot}[1]{\widetilde{o}\fbrac{{#1}}}
\newcommand{\bigomega}[1]{\Omega\fbrac{{#1}}}
\newcommand{\bigomegat}[1]{\widetilde{\Omega}\fbrac{{#1}}}
\newcommand{\bigomegated}[1]{\widetilde{\Omega}_{\approxerror,\confidence}\fbrac{{#1}}}
\newcommand{\smallomega}[1]{\omega\fbrac{{#1}}}
\newcommand{\smallomegat}[1]{\widetilde{\omega}\fbrac{{#1}}}
\newcommand{\bigtheta}[1]{\Theta\fbrac{{#1}}}
\newcommand{\bigthetat}[1]{\widetilde{\Theta}\fbrac{{#1}}}
\newcommand{\algo}{\sA}
\newcommand{\poly}[1]{poly\fbrac{#1}}
\newcommand{\polylog}[1]{\poly\log\fbrac{#1}}

%Work Specific
\usepackage{soul}
% \usepackage{eqparbox}
% \renewcommand{\algorithmiccomment}[1]{\hfill\eqparbox{COMMENT}{\# #1}}




\end{document}
