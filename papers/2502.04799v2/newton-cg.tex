\documentclass[arxiv,12pt]{colt2025} %


\title[Complexity of Regularized Newton for Nonconvex Optimization]{A Regularized Newton Method for Nonconvex Optimization with Global and Local Complexity Guarantees}
\usepackage{times}  %



\coltauthor{%
 \Name{Yuhao Zhou} \Email{yuhaoz.cs@gmail.com}\\
 \addr Department of Computer Science and Technology, Tsinghua AI Insitute, BNList Lab, \\ Tsinghua-Bosch Joint Center for ML, Tsinghua University
 \AND
 \Name{Jintao Xu} \Email{jintao.xu@polyu.edu.hk}\\
 \addr Department of Applied Mathematics, The Hong Kong Polytechnic University
 \AND
 \Name{Chenglong Bao} \Email{clbao@mail.tsinghua.edu.cn}\\
 \addr Yau Mathematical Sciences Center, Tsinghua University \\  Beijing Institute of Mathematical Sciences and Applications
 \AND
 \Name{Chao Ding} \Email{dingchao@amss.ac.cn}\\
 \addr Academy of Mathematics and Systems Science, Chinese Academy of Sciences
 \AND
 \Name{Jun Zhu} \Email{dcszj@tsinghua.edu.cn}\\
 \addr Department of Computer Science and Technology, Tsinghua AI Insitute, BNList Lab, \\ Tsinghua-Bosch Joint Center for ML, Tsinghua University
}

\newcommand{\ones}{\mathbf 1}
\newcommand{\reals}{{\mbox{\bf R}}}
\newcommand{\integers}{{\mbox{\bf Z}}}
\newcommand{\symm}{{\mbox{\bf S}}}  % symmetric matrices

\newcommand{\nullspace}{{\mathcal N}}
\newcommand{\range}{{\mathcal R}}
\newcommand{\Rank}{\mathop{\bf Rank}}
%\newcommand{\Tr}{\mathop{\bf Tr}}
\newcommand{\diag}{\mathop{\bf diag}}
\newcommand{\card}{\mathop{\bf card}}
\newcommand{\rank}{\mathop{\bf rank}}
\newcommand{\conv}{\mathop{\bf conv}}
\newcommand{\prox}{\mathbf{prox}}

\newcommand{\Expect}{\mathop{\bf E{}}}
\newcommand{\var}{\mathop{\bf var{}}}
\newcommand{\Prob}{\mathop{\bf Prob}}
\newcommand{\Co}{{\mathop {\bf Co}}} % convex hull
\newcommand{\dist}{\mathop{\bf dist{}}}
%\newcommand{\argmin}{\mathop{\rm argmin}}
%\newcommand{\argmax}{\mathop{\rm argmax}}
\newcommand{\epi}{\mathop{\bf epi}} % epigraph
\newcommand{\Vol}{\mathop{\bf vol}}
\newcommand{\dom}{\mathop{\bf dom}} % domain
\newcommand{\intr}{\mathop{\bf int}}
%\newcommand{\sign}{\mathop{\bf sign}}

\newcommand{\cf}{{\it cf.}}
\newcommand{\eg}{{\it e.g.}}
\newcommand{\ie}{{\it i.e.}}
\newcommand{\etc}{{\it etc.}}

\newcommand{\todo}{{\bf TODO}}

\newcommand{\bone}{\boldsymbol{1}}
\newcommand{\balpha}{\boldsymbol{\alpha}}
\newcommand{\bbeta}{\boldsymbol{\beta}}
\newcommand{\bdelta}{\boldsymbol{\delta}}
\newcommand{\bepsilon}{\boldsymbol{\epsilon}}
\newcommand{\blambda}{\boldsymbol{\lambda}}
\newcommand{\bomega}{\boldsymbol{\omega}}
\newcommand{\bpi}{\boldsymbol{\pi}}
\newcommand{\bnu}{\boldsymbol{\nu}}
\newcommand{\bphi}{\boldsymbol{\phi}}
\newcommand{\bvphi}{\boldsymbol{\varphi}}
\newcommand{\bpsi}{\boldsymbol{\psi}}
\newcommand{\bsigma}{\boldsymbol{\sigma}}
\newcommand{\btheta}{\boldsymbol{\theta}}
\newcommand{\bzeta}{\boldsymbol{\zeta}}
\newcommand{\bxi}{\boldsymbol{\xi}}
\newcommand{\ba}{\boldsymbol{a}}
\newcommand{\bb}{\boldsymbol{b}}
\newcommand{\bc}{\boldsymbol{c}}
\newcommand{\bd}{\boldsymbol{d}}
\newcommand{\be}{\boldsymbol{e}}
\newcommand{\boldf}{\boldsymbol{f}}
\newcommand{\bg}{\boldsymbol{g}}
\newcommand{\bh}{\boldsymbol{h}}
\newcommand{\bi}{\boldsymbol{i}}
\newcommand{\bj}{\boldsymbol{j}}
\newcommand{\bk}{\boldsymbol{k}}
\newcommand{\bell}{\boldsymbol{\ell}}
\newcommand{\bp}{\boldsymbol{p}}
\newcommand{\br}{\boldsymbol{r}}
\newcommand{\bs}{\boldsymbol{s}}
\newcommand{\bt}{\boldsymbol{t}}
\newcommand{\bu}{\boldsymbol{u}}
\newcommand{\bv}{\boldsymbol{v}}
\newcommand{\bw}{\boldsymbol{w}}
\newcommand{\bx}{{\boldsymbol{x}}}
\newcommand{\by}{\boldsymbol{y}}
\newcommand{\bz}{\boldsymbol{z}}
\newcommand{\bA}{\boldsymbol{A}}
\newcommand{\bB}{\boldsymbol{B}}
\newcommand{\bC}{\boldsymbol{C}}
\newcommand{\bD}{\boldsymbol{D}}
\newcommand{\bE}{\boldsymbol{E}}
\newcommand{\bF}{\boldsymbol{F}}
\newcommand{\bG}{\boldsymbol{G}}
\newcommand{\bH}{\boldsymbol{H}}
\newcommand{\bI}{\boldsymbol{I}}
\newcommand{\bJ}{\boldsymbol{J}}
\newcommand{\bL}{\boldsymbol{L}}
\newcommand{\bM}{\boldsymbol{M}}
\newcommand{\bP}{\boldsymbol{P}}
\newcommand{\bQ}{\boldsymbol{Q}}
\newcommand{\bR}{\boldsymbol{R}}
\newcommand{\bS}{\boldsymbol{S}}
\newcommand{\bT}{\boldsymbol{T}}
\newcommand{\bU}{\boldsymbol{U}}
\newcommand{\bV}{\boldsymbol{V}}
\newcommand{\bW}{\boldsymbol{W}}
\newcommand{\bX}{\boldsymbol{X}}
\newcommand{\bY}{\boldsymbol{Y}}
\newcommand{\bZ}{\boldsymbol{Z}}

% new theorems
% \newtheorem{theorem}{Theorem}
%\newtheorem*{proof}{Proof}

% usepackages
\usepackage{amsmath}
\usepackage{amsfonts}
\usepackage{textcomp} % for \textlangle and \textrangle macros
\newcommand{\qdist}[1]{\ifmmode\langle#1\rangle\else\textlangle#1\textrangle\fi}
\usepackage{xcolor}
\usepackage{algorithm} % for algorithms
\usepackage{algpseudocode} % for pseudocode
\usepackage{comment} % for large comments
\usepackage{bbm}
\usepackage{dsfont}
\usepackage{subfigure}
\usepackage{bm}
\usepackage{booktabs} % For better table lines
\usepackage{array} % For better column formatting
%\usepackage{appendix}
%\usepackage[english]{babel}
%\usepackage{amsthm}
\usepackage{graphicx} % for graphs





\usepackage[usestackEOL]{stackengine}
\usepackage{lipsum}%
\usepackage{tikz-cd}
\newcommand{\algname}[1]{\textbf{#1}}
\newcommand{\poly}{\text{poly}}
\newcommand{\expasymp}[1]{e^{-\Omega(#1)}}
\newcommand{\expasympsqr}{\expasymp{k^2}}
\newcommand{\posorthant}{\R_{\geq 0}^n}
\newcommand{\strictposorthant}{\R_{> 0}^n}
\newcommand{\sign}{\textrm{sign}}
\newcommand{\jintao}[1]{{{\color{red}[[{\textbf{JINTAO}}: #1]]}}}
\newcommand{\yuhao}[1]{{{\color{green}[[{\textbf{YUHAO}}: #1]]}}}
\newcommand{\chao}[1]{{{\color{olive}[[{\textbf{CHAO}}: #1]]}}}
\newcommand{\chenglong}[1]{{{\color{red}[[{\textbf{CL}}: #1]]}}}
\newcommand{\junz}[1]{{{\color{red}[[{\textbf{jz}}: #1]]}}}

\newcommand{\supsucc}{\mathrm{t}}
\newcommand{\supfallback}{\mathrm{f}}

\newcommand*{\propositionrefname}{Proposition}
\newcommand*{\propositionsrefname}{Propositions}
\newcommand*{\propositionref}[1]{%
  \objectref{#1}{\propositionrefname}{\propositionsrefname}{}{}}


\usepackage{chngcntr}
\counterwithin{table}{section}
\counterwithin{figure}{section}

\begin{document}

\maketitle

\begin{abstract}%
  We consider the problem of finding an $\epsilon$-stationary point of a nonconvex function with a Lipschitz continuous Hessian and propose a quadratic regularized Newton method incorporating a new class of regularizers constructed from the current and previous gradients. The method leverages a recently developed linear conjugate gradient approach with a negative curvature monitor to solve the regularized Newton equation. Notably, our algorithm is adaptive, requiring no prior knowledge of the Lipschitz constant of the Hessian, and achieves a global complexity of $O(\epsilon^{-\frac{3}{2}}) + \tilde O(1)$ in terms of the second-order oracle calls, and $\tilde O(\epsilon^{-\frac{7}{4}})$ for Hessian-vector products, respectively. Moreover, when the iterates converge to a point where the Hessian is positive definite, the method exhibits quadratic local convergence. Preliminary numerical results illustrate the competitiveness of our algorithm.
\end{abstract}

\begin{keywords}%
  smooth nonconvex optimization;
  Newton method;
  worst-case complexity;
  adaptive algorithm%
\end{keywords}



\section{Introduction}
\label{sec:intro}

\begin{figure*}[tb]
    \centering
    \includegraphics[width=0.848\linewidth]{figs/circuitnn.pdf} 
    \caption{Illustration of differentiable CircuitNN. CircuitNN is designed based on differentiable NAND gates. After DAS is guided by PI and PO pairs of the truth table, CircuitNN can get the precise circuit architecture logic equivalent to the truth table.}
    \label{fig:circuitnn}
\end{figure*}

% 1. Describe the importance of logic synthesis
% 2. Existing Problems
% (a) Neural Architecture Search: Unstable, Predefined Setting, etc.
% (b) Circuit Generation: Probabilistic Model, Logic Equivalence

With the rapid advancement of technology, the scale of integrated circuits (ICs) has expanded exponentially. 
This expansion has introduced significant challenges in chip manufacturing, particularly concerning power and area metrics.
A primary objective in IC design is achieving the same circuit function with fewer transistors, thereby reducing power usage and area occupancy.

Logic synthesis~\cite{hachtel2005logicsynth}, a critical step in electronic design automation (EDA), transforms behavioral-level circuit designs into optimized gate-level circuits, ultimately yielding the final IC layout. 
The primary goal of logic synthesis is to identify the physical implementation with the fewest gates for a given circuit function. 
This task constitutes a challenging NP-hard combinatorial optimization problem. 
Current logic synthesis tools~\cite{brayton2010abc, wolf2013yosys} rely on human-designed heuristics, often leading to sub-optimal outcomes.

Differentiable architecture search (DAS) techniques~\cite{liu2018darts, chu2020darts} offer novel perspectives on addressing challenges in this problem.
Circuit functions can be represented through truth tables, which map binary inputs to their corresponding outputs. 
Truth tables provide a precise representation of input-output relationships, ensuring the design of functionally equivalent circuits.
Inspired by this, researchers~\cite{deepmind2024ai4sys, wang2024tnet} have begun exploring the application of DAS to synthesize circuits directly from truth tables.
Specifically, \citet{deepmind2024ai4sys} proposed CircuitNN, a framework that learns differentiable connection structures with logic gates, enabling the automatic generation of logic circuits from truth tables.
This approach significantly reduces the complexity of traditional circuit generation. 
Building on this, \citet{wang2024tnet} introduced T-Net, a triangle-shaped variant of CircuitNN, incorporating regularization techniques to enhance the efficiency of DAS.

Despite these advancements, several challenges remain. 
The computational complexity of DAS grows quadratically with the number of gates, posing scalability issues.
Although triangle-shaped architecture~\cite{wang2024tnet} partially mitigates this problem, redundancy persists. 
%Additionally, DAS is susceptible to converging to local optima, limiting the ability to search architectures that satisfy the given truth tables~\cite{liu2018darts}. 
%Furthermore, hyperparameters (network depth and layer width) require extensive searches, introducing complexity and prolonging the synthesis process. 
Additionally, DAS is susceptible to converging to local optima~\cite{liu2018darts} and hyperparameters (network depth and layer width) require extensive searches. 
The challenges arise from the vast search space in DAS. 
% Even with predefined settings for CircuitNN, finding a configuration that meets the truth table requires extensive trial and error during the DAS process. 
Intuitively, limiting the search space through predefined parameters (network depth, gates per layer, and connection probabilities) can significantly reduce the complexity.

Recent advances~\cite{openai2023gpt4, abramson2024alphafold3, esser2024sd3, li2024mar} in conditional generative models have demonstrated remarkable performance across language, vision, and graph generation tasks. 
Motivated by these developments, we propose a novel approach to circuit generation that generates preliminary circuit structures to guide DAS in generating refined circuits matching specified truth tables. 
Firstly, we introduce CircuitVQ, a tokenizer with a discrete codebook for circuit tokenization. 
Built upon our Circuit AutoEncoder framework~\cite{hou2022graphmae,li2023maskgae,wu2025mgvga}, CircuitVQ is trained through a circuit reconstruction task. 
Specifically, the CircuitVQ encoder encodes input circuits into discrete tokens using a learnable codebook, while the decoder reconstructs the circuit adjacency matrix based on these tokens.
Subsequently, the CircuitVQ encoder serves as a circuit tokenizer for CircuitAR pretraining, which employs a masked autoregressive modeling paradigm~\cite{chang2022maskgit, li2023mage}. 
In this process, the discrete codes function as supervision signals. 
After training, CircuitAR can generate discrete tokens progressively, which can be decoded into initial circuit structures by the decoder of the CircuitVQ. 
These prior insights can guide DAS in producing refined circuits that match the target truth tables precisely.

Our key contributions can be summarized as follows:
\begin{itemize}
\item We introduce CircuitVQ, a circuit tokenizer that facilitates graph autoregressive modeling for circuit generation, based on our Circuit AutoEncoder framework;
\item Develop CircuitAR, a model trained using masked autoregressive modeling, which generates initial circuit structures conditioned on given truth tables;
\item Propose a refinement framework that integrates differentiable architecture search to produce functionally equivalent circuits guided by target truth tables;
\item Comprehensive experiments demonstrating the scalability and capability emergence of our CircuitAR and the superior performance of the proposed circuit generation approach.
\end{itemize}

% Motivation
% (a) Diffusion (Vision, Graph), Autoregressive (Language, Vision)
% (b) Circuit Generation for Predefined Setting
% (c) Neural Architecture Search for Strict Logic Equivalence

% Contribution
% (a) Circuit Tokenizer (new transformer arch, training strategy)
% (b) CircuitAR (train and gen strategies, post-ar strategy)
% (c) Extensive Evaluation including BitD (Bit Distance) for Scalability

\section{Basic Background: Supervised Learning and the PAC Model}
\label{sec:background}

At this point almost everyone has heard of machine learning (ML). Anyone likely to stumble upon this article will have also heard of its most influential special case, supervised learning, and those theoretically inclined will also be familiar with the PAC model. Nonetheless, I will set the stage by  recapping the basics.

\subsection{Basics of Supervised Learning}%Let's set the stage in any case

\emph{Supervised Learning} is the task of ``coming up'' with a function $f: \X \to \Y$ to ``explain'' or ``fit'' a sequence of input/output examples   $(x_1,y_1), \ldots, (x_n,y_n)$, with $x_i \in \X$ and $y_i \in \Y$.  Here $\X$ is a \emph{data domain} consisting of \emph{datapoints} $x \in \X$, $\Y$ is a \emph{label set} consisting of \emph{labels} $y \in \Y$, and the sequence $(x_1,y_1),\ldots,(x_n,y_n)$ is the \emph{training data} consisting of \emph{labeled examples (a.k.a. samples)}~$(x_i,y_i)$.  I~will refer to the chosen function $f$ as a \emph{predictor}, and to $n$ as the \emph{sample size}. A \emph{learning algorithm} takes as input training data, and outputs (some representation of) a predictor $f \in \Y^\X$.\footnote{Note that this describes the usual \emph{batch}, a.k.a.~\emph{offline}, setting of supervised learning. I do not discuss other paradigms such as online or active learning in this article.} 



Success in supervised learning is defined as \emph{generalization} to  future examples: For a typical \emph{test example}  $(x_{\tst},y_{\tst})$, the predicted label $y'_{\tst}=f(x_{\tst})$ should ``equal'' $y_{\tst}$, perhaps approximately. We usually assume the test example is drawn from the same  ``source'' as the training data  --- commonly, i.i.d.~from the same distribution. The quality of the prediction is quantified by $\ell(y'_{\tst},y_{\tst})$, where $\ell:~\Y~\times~\Y \to \RR_{\geq 0}$ is a \emph{loss function} chosen as part of the problem definition. Common loss functions include the 0-1 loss $\ell_{0-1}(y',y) = [y' \neq y]$ for \emph{classification} problems,\footnote{The notation $[P]$ denotes $1$ when predicate $P$ is true, and denotes $0$ when $P$ is false.} as well as the absolute loss $|y'-y|$ or squared loss $(y'-y)^2$ for \emph{regression problems} featuring $\Y  \sse \RR$.

Nontrivial generalization properties are typically only possible if one assumes something about the data.\footnote{The need for such an assumption is formalized by the  \emph{no free lunch theorems} of supervised learning \cite{wolpert_connection_1992,wolpert_lack_1996,schaffer_conservation_1994}.} The Bayesian approach to  machine learning, common in many applications, assumes some parametric form for the distribution generating the data, and postulates a prior on the parameters. This is not the approach I will take in this article. Instead, I will focus on the frequentist --- and some would say ``worst-case'' or ``adversarial'' ---  approach that is common in the computational learning theory community, embodied by the PAC model. Here we assume that the (training and test) data can be explained, perhaps approximately, by a function in some ``simple enough to learn'' class of functions $\H \sse \Y^\X$, often called the \emph{hypotheses}. Equivalently, we  seek a predictor which explains the unseen data roughly  as well as the best hypothesis $h^* \in \H$, whether or not we assume that $h^*$ itself provides a perfect explanation.



 \paragraph{Common Algorithmic Templates.} Perhaps the best known general-purpose supervised learning algorithm is \emph{empirical risk minimization (ERM)}, which chooses as its predictor a hypothesis $f \in \H$ minimizing $\frac{1}{n} \sum_{i=1}^n \ell(f(x_i),y_i)$ --- a quantity called the \emph{training error}, \emph{empirical error}, or \emph{empirical risk} of $f$. %\footnote{When multiple hypotheses minimize the empirical risk, we assume ERM breaks ties arbitrarily.}
A common template for generalizing ERM involves adding a \emph{regularization term} $\psi(f)$ to the  objective function, typically chosen to measure some notion of ``hypothesis complexity.'' An algorithm instantiating this template is known as a \emph{structural risk minimizer (SRM)}, and chooses as its predictor the hypothesis $f \in \H$ minimizing the \emph{structural risk} $\frac{1}{n} \sum_{i=1}^n \ell(f(x_i),y_i) + \psi(f)$. Other well-known algorithms, such as gradient descent and its variations,  can frequently be interpreted as approximate implementations of ERM or SRM.


\paragraph{Proper vs Improper Learning.} A learning algorithm is said to be \emph{proper} if its predictor $f$ is always chosen from the hypothesis class, i.e., $f \in \H$, otherwise it is said to be \emph{improper}. ERM  is an example of a proper learning algorithm, as are SRM algorithms of the form described above.  In the \emph{proper regime} of learning, algorithms are required to be proper. This article will be concerned with the more flexible \emph{improper regime} (a.k.a \emph{representation-independent learning}), where no such constraint is placed on the learner. In other words, all we care about is predictive power at test time, rather than any insights derived from the functional form or representation of the predictor~itself.


\subsection{The PAC Model}
A standard mathematical setup for evaluation of supervised learning algorithms, at least in the theoretical computer science community, is Valiant's \emph{Probably Approximately Correct (PAC) model} of learning (see e.g.~\cite{kearns_introduction_1994,mohri_foundations_2018}). Here, we assume there is an unknown distribution $\D$ on $\X \times \Y$ from which training and test data are  drawn.  Specifically, the labeled datapoints of the training set  $(x_1,y_1), \ldots, (x_n,y_n)$, as well as the test data  $(x_\tst,y_\tst)$, are i.i.d.~from $\D$. Often it is assumed that $\D$ lies in some class of distributions of interest. The \emph{true expected loss}, or simply \emph{loss}, of a predictor $f: \X \to \Y$ is the expected loss it incurs on draws from $\D$, written $L_\D(f) = \Ex_{(x,y) \sim \D} \ell(f(x),y)$.


There are two main ``settings'' in PAC learning. The  \emph{realizable setting} only requires that the data be perfectly explained by some hypothesis in $\H$. More generally, the \emph{agnostic setting} makes no assumption relating the data to the hypotheses, but shifts the goalposts as necessary to allow nontrivial guarantees: the expected loss at test time is evaluated only ``relative'' to that of the best hypothesis $h^* \in \H$. There are other settings which make more nuanced assumptions, such as $\D$ being of a particular parametric form or its support living in some (unknown) lower-dimensional space, etc. I will mostly discuss the realizable and agnostic settings in this article, those being the simplest and most studied from a theoretical perspective. %TODO:We will briefly discuss other settings in Section ??

The PAC model demands high probability guarantees of learners, in the worst case over distributions of interest. Consider first the realizable setting, where $\D$ is such that $\min_{h \in \H} L_{\D}(h) = 0$. A PAC learner has \emph{error} $\epsilon=\epsilon(n)$ and \emph{confidence} $\delta=\delta(n)$ if, when training data consists of $n$ i.i.d~samples from a realizable distribution $\D$, it produces a predictor $f$  satisfying $L_\D(f) \leq \epsilon$ with probability at least $1-\delta$. In the agnostic setting, where $\D$ can be arbitrary, we require $L_\D(f) - \min_{h \in \H} L_\D(h) \leq \epsilon$ with probability $1-\delta$.

In both the realizable and agnostic settings, we look for PAC learners with small $\epsilon$ and $\delta$ as a function of the sample size $n$. An equivalent perspective looks at the sample complexity $m(\epsilon,\delta)$, which is the minimum sample size which guarantees error  at most $\epsilon$ with probability at least $1-\delta$. We say a problem is \emph{PAC learnable} if its PAC sample complexity is finite whenever $\epsilon,\delta > 0$.

For most PAC learning problems, learnability and sample complexity are characterized in terms of a  ``dimension'' of the hypothesis class. Most prominently this is the \emph{VC dimension} for binary classification, the \emph{fat shattering dimension} for agnostic regression, and the \emph{DS dimension} for multiclass classification (see \cite{anthony_neural_1999,daniely_optimal_2014,brukhim_characterization_2022}). Treatment of these is beyond the scope of this article. The unfamiliar reader need not worry, however,  as dimensions will feature only tangentially in our~discussion.




%\paragraph{Learning settings: Realizable, Agnostic, etc.} In learning theory, evaluating a supervised learning algorithm requires specifying a data model and an objective. We will leave the details of the data model flexible for now, to allow for both the PAC model and the adversarial transductive model. Nonetheless we will describe two variations, which we call ``settings'', which cut across different models. The  \emph{realizable setting}  requires only that the data be perfectly explained by some hypothesis $h \in \H$ --- i.e., there exists a hypothesis which is guaranteed to suffer a loss of $0$ on training and test data. The performance of the learning algorithm is its expected loss at test time for some ``worst case'' realizable instance. More generally, the \emph{agnostic setting} makes no assumption relating the data to the hypotheses, but shifts the goalposts as necessary to allow nontrivial guarantees: the expected loss at test time is evaluated only ``relative'' to that of the best hypothesis $h^* \in \H$, again for some ``worst case'' instance. There are other settings which make more nuanced assumptions about the data, such as it is drawn from a distribution of a particular parametric form, or that it lives in some (unknown) lower-dimensional space, etc. We will mostly discuss the realizable and agnostic settings, those being the simplest and most studied from a theoretical perspective.




%%% Local Variables:
%%% mode: latex
%%% TeX-master: "learning_matching"
%%% End:



\section{Overview of the techniques} \label{sec:main/techniques-overview}


As mentioned in \Cref{sec:main/background}, the key to establishing a fast global rate is to show that the loss decreases by at least $L_H^{-\frac{1}{2}}\epsilon^{\frac{3}{2}}$ (i.e., \emph{sufficient descent}) for as many iterations as possible.
We summarize necessary properties of \Cref{alg:adap-newton-cg} in \lemmaref{lem:lipschitz-constant-estimation}, 
and will subsequently focus on how to leverage them to establish a global rate.


\begin{lemma}[Summarized descent lemma, see \Cref{sec:appendix/summarized-descent}]
    \label{lem:lipschitz-constant-estimation}
    Let $\{ x_k$, $M_k$, $\text{d\_type}_k$, $m_k \}_{k \ge 0}$ be the sequence generated by \Cref{alg:adap-newton-cg}, 
    and denote $\omega_k := \omega_k^{\supsucc}$ if the trial step is accepted and $\omega_k := \omega_k^{\supfallback}$ otherwise.
    Define the index sets $\cJ^{i} = \{ k : M_{k+1} = \gamma^{i} M_k \}$ for $i = -1, 0, 1$, and the constants
    $\tilde C_4 = \max\big ( 1, \tau_-^{-1}(9\beta)^{-\frac{1}{2}}, \tau_-^{-1}(3\beta(1 - 2\mu))^{-1}\big )$ and $\tilde C_5 = \min(2, 3 - 6\mu)^{-1}$,
    then 
    \begin{enumerate}
        \item If $k \in \cJ^1$, then $M_k \leq \tilde C_5 L_H$; %
        \item 
        For the regularizers in \theoremref{thm:newton-local-rate-boosted},
        if $M_k > \tilde C_4 L_H$ and $\tau_- \leq \min\big ( \delta_k^\alpha, \delta_{k+1}^\alpha \big )$, 
        then $k \in \cJ^{-1}$, where $\alpha = \max(2, 3\theta)$.
    \end{enumerate}
    Moreover, 
    we have
    $\bigcup_{i = -1, 0, 1} ( \cJ^{i} \cap I_{0,k}) = I_{0,k}$, and
    \begin{align}
        \label{eqn:cardinarlity-of-M-set-a}
        |\cJ^{1} \cap I_{0,k}| & \leq
        |\cJ^{-1} \cap I_{0,k}| + [\log_\gamma (\gamma \tilde C_5 M_0^{-1} L_H)]_+, \\
        \label{eqn:cardinarlity-of-M-set}
        k = | I_{0,k} | &\leq 2| \cJ^{-1} \cap I_{0,k} | + |\cJ^0 \cap I_{0,k}| + [\log_\gamma (\gamma \tilde C_5 M_0^{-1} L_H) ]_+,
    \end{align}
    and the following descent inequality holds:
    \begin{equation}
        \label{eqn:summarized-descent-inequality}
        \varphi(x_{k+1}) - \varphi(x_k) \leq 
         \begin{cases}
             0, & \text{ if } k \in \cJ^1, \\
             -\tilde C_1 M_k^{-\frac{1}{2}} D_k, & \text{ if } k \in \cJ^0 \cup \cJ^{-1},
        \end{cases}
    \end{equation}
    where
    $\tilde C_1 = \min\left( 9\beta^2(1 - 2\mu)^2\mu, 36\beta\mu(1-\mu)^2, 4\mu /33 \right)$, and
    \begin{equation}
        \label{eqn:summarized-descent-amount}
        D_k = \begin{cases}
            (\omega_k^{\supfallback})^3,
            & \text{ if } k \in \cJ^{-1}, \\
            \min\Big(  
                (\omega_k^{\supfallback})^3,
                \omega_k^3, 
                g_{k+1}^2 \omega_k^{-1}  
            \Big), & \text{ if } \text{d\_type}_k = \texttt{SOL} \text{ and } m_k = 0 \text{ and } k \notin \cJ^{-1}, 
            \\
            \min\Big( 
                (\omega_k^{\supfallback})^3,
                \omega_k^3
            \Big), & \text{ otherwise}.
        \end{cases}
    \end{equation}
\end{lemma}

Before proceeding, we discuss the dependence on $L_H$ in \eqref{eqn:summarized-descent-inequality}.
Since $M_k$ is increased (i.e., $k \in \cJ^{1}$) only if $M_k \leq O(L_H)$, 
then if there exists $k_{\mathrm{init}}$ such that $M_{k_{\mathrm{init}}} \le O(L_H)$, we know
$M_k \le O(L_H)$ for $k \geq k_{\mathrm{init}}$.
Furthermore, for $k \geq k_{\mathrm{init}}$, when $M_k$ remains unchanged or decreases (i.e., $k \in \cJ^0 \cup \cJ^{-1}$), 
the function descent satisfies $\varphi(x_{k+1}) - \varphi(x_k) \lesssim -L_H^{-\frac{1}{2}} D_k$,
which ensures the dependence of the sufficient descent on $L_H$. 
The only issue arises when $M_k$ needs to be increased. 
However, as shown in \eqref{eqn:cardinarlity-of-M-set-a},
the occurrence of such cases can be effectively controlled.

\subsection{The global iteration complexity}
\label{sec:main/sqrt-global-rate}

Since under the choices of regularizers, we have either $\omega_k^{\supfallback} = \sqrt{g_k}$ or $\omega_k^{\supfallback} = \sqrt{\epsilon_k}$, then ensuring sufficient descent reduces to counting the occurrences of the event $D_k \geq (\omega_k^{\supfallback})^3$. 
We outline the key steps for it in this section and defer the proofs and intermediate lemmas to \Cref{sec:appendix/global-rate-under-local-boosting,sec:appendix/global-rate-technical-lemmas}.

Throughout this section, we partition %
$\N$ 
into a disjoint union of intervals $\N = \bigcup_{j\geq 1} I_{\ell_j, \ell_{j+1}}$ such that 
 $0 = \ell_1$ and $\ell_j < \ell_{j+1}$ for $j \geq 1$, where $I_{i,j}=\{i, .., j-1\}$ is defined in the notation section. %
These intervals are constructed such that the following conditions hold for every $j \geq 1$:
\begin{align}
    g_{\ell_j} \geq g_{\ell_j + 1} \geq \dots \geq g_{\ell_{j+1} - 1}
    \text{ and }
    g_{\ell_{j+1} - 1} < g_{\ell_{j+1}}.
    \label{eqn:proof/newton-partition}
\end{align}
In other words, the sequence $\{x_k\}_{k \ge0}$ is divided into subsequences where the gradient norms are non-increasing.
The following lemma shows that sufficient descent occurs during the transition between adjacent subsequences, 
provided that $\ell_j - 1\notin \cJ^1$.
The fallback step is primarily designed to ensure this lemma holds. 
Without the fallback step, a sudden gradient decrease (i.e., a small $\delta_k$) could result in a small regularizer, causing the sufficient descent guaranteed by this lemma to vanish.

\begin{lemma}[Transition between adjacent subsequences, see \lemmaref{lem:proof/transition-between-subsequences-give-valid-regularizer}]
    \label{lem:main/transition-between-subsequences-give-valid-regularizer}
    Under the regularizers in \theoremref{thm:newton-local-rate-boosted} with $\theta \geq 0$, 
    we have $\omega_{\ell_{j}-1} = \omega_{\ell_{j}-1}^{\supfallback}$ for each $j > 1$, and 
    \begin{equation}
        \label{eqn:main/inexact-mixed-newton-boundary-bound}
        \varphi(x_{\ell_{j}})
        - \varphi(x_{\ell_{j}-1}) 
        \leq 
        -\tilde C_1 M_{\ell_j-1}^{-\frac{1}{2}} \bone_{\{\ell_{j}-1 \notin \cJ^{1}\}} (\omega_{\ell_{j}-1}^{\supfallback})^3.
    \end{equation}
    Moreover, if $M_{\ell_j-1} > \tilde C_4 L_H$, then $\ell_j -1 \in\cJ^{-1}$.
\end{lemma}

The following lemma characterizes the overall decrease of the function within a subsequence.
It roughly states that there are at most $O\big (\log\log \frac{g_{\ell_j}}{g_k}\big )$ iterations with insufficient descent in the subsequence $I_{\ell_j,\ell_{j+1}}$,
since otherwise the gradient decreases superlinearly below $g_k$.

\begin{lemma}[Iteration within a subsequence, see \lemmaref{lem:proof/iteration-in-a-subsequence}]
    \label{lem:main/iteration-in-a-subsequence}
    Under the regularizers in \theoremref{thm:newton-local-rate-boosted} with $\theta \geq 0$, 
    then for $j \geq 1$ and $\ell_j  < k < \ell_{j+1}$, we have
    \begin{align}
        \varphi(x_{k})
        - \varphi(x_{\ell_{j}})
        \leq
         - C_{\ell_j,k}
        \left ( 
            |I_{\ell_j,k} \cap \cJ^{-1} |
            + 
            \max\left ( 0, | I_{\ell_j,k} \cap \cJ^0 | - T_{\ell_j,k} - 5 \right ) 
        \right ) (\omega_k^{\supfallback})^3
         ,
         \label{eqn:main/inexact-mixed-newton-inner-bound-theta0}
    \end{align}
    where $C_{i,j} = \tilde C_1 %
    \min_{i \leq l < j} M_l^{-\frac{1}{2}}$ and $T_{i,j}=2\log\log\big (3 (\omega_i^{\supfallback})^2 (\omega_j^{\supfallback})^{-2}\big)$.
\end{lemma}
\begin{proof}[Sketch of the idea]
To demonstrate the key ideas, we use the square root gradient regularizer  $\omega_i^{\supsucc} = \omega_i^{\supfallback} = \sqrt{g_i}$ 
and assume the Lipschitz constant estimation is precise (i.e., $\N = \cJ^{0}$).
Under this choice, we observe that $D_i \geq g^2_{i+1} g_i^{-\frac{1}{2}}$ for iterations within a subsequence, i.e., $i \in I_{\ell_j,\ell_{j+1}}$.
We can divide $I_{\ell_j,k}$ into subsets 
$I_{\ell_j,k}^{(l)} = \{ i \in I_{\ell_j,k} :  \exp(4^l)g_k \leq g_i < \exp(4^{l+1})g_k \}$ for $l \geq 0$ and $I_{\ell_j,k}^{(-1)} = \{ i \in I_{\ell_j,k} : g_k \leq g_i < \mathrm{e} g_k \}$. 
Then, we find that $D_i \geq \mathrm{e}^{-\frac{1}{2}}g_k^{\frac{3}{2}}$ if $i$ and $i + 1$ belong to the same subinterval, and the number of non-empty subintervals is $O(T_{\ell_j,k})$ (see \lemmaref{lem:accumulated-descent-lower-bound} for details).
The general case follows a similar approach but involves additional technical complexities, which are detailed in \Cref{sec:appendix/global-rate-under-local-boosting}.
\end{proof}

Combining \lemmaref{lem:main/transition-between-subsequences-give-valid-regularizer,lem:main/iteration-in-a-subsequence}, we have the following proposition about the accumulated function descent, 
and find that there are $\Sigma_k$ iterations with sufficient descent.

\begin{proposition}[Accumulated descent, see \propositionref{prop:proof/accumulated-descent}]
    \label{prop:main/accumulated-descent}
    Under the choices of \theoremref{thm:newton-local-rate-boosted} with $\theta \geq 0$, 
    for each $k \geq 0$, we have
    \begin{align}
        \varphi(x_{k})
        - \varphi(x_0)
        \leq
         - C_{0,k}
        \big (\underbrace{
            |I_{0,k} \cap \cJ^{-1}|
            + \max\big( |S_k \cap \cJ^0|, |I_{0,k} \cap \cJ^0| - V_k - 5J_k \big)
            }_{\Sigma_k} \big )
         \epsilon_k^{\frac{3}{2}}
         ,
        \label{eqn:main/newton-global-final-inequality}
    \end{align}
    where $V_k = \sum_{j=1}^{J_k-1} T_{\ell_j,\ell_{j+1}} + T_{\ell_{J_k},k}$,
    and $S_k = \bigcup_{j=1}^{J_k-1}\{\ell_{j+1}-1\}$,
    and $J_k = \max\{ j : \ell_j \leq k \}$.
\end{proposition}

The difference of the logarithmic factor in the iteration complexity of \theoremref{thm:newton-local-rate-boosted} arises from the following lemma, which provides an upper bound for $V_k$.
This lemma shows that the choice $\omega_k^{\supfallback} = \sqrt{\epsilon_k}$ leads to a better control over $V_k$ due to the monotonicity of $\epsilon_k$, 
resulting in improved lower bound for $\Sigma_k$, as indicated by \lemmaref{lem:basic-counting-lemma}. %
\begin{lemma}[See \Cref{sec:appendix/proof-lower-bound-of-Vk}]
    \label{lem:main/lower-bound-of-Vk}
    Let $V_k, J_k$ be defined in \propositionref{prop:main/accumulated-descent}, then we have
    (1). If $\omega_k^{\supfallback} = \sqrt{g_k}$, then $V_k \leq J_k \log\log \frac{U_\varphi}{\epsilon_k}$;
    (2). If $\omega_k^{\supfallback} = \sqrt{\epsilon_k}$, then $V_k \leq \log \frac{\epsilon_0}{\epsilon_k} + J_k$.
\end{lemma}


Finally, we need to determine the aforementioned hitting time $k_{\mathrm{init}}$ such that $M_{k_\mathrm{init}} \leq O(L_H)$,
and apply \propositionref{prop:main/accumulated-descent} for $\{ x_k \}_{k \geq k_{\mathrm{init}}}$ to achieve the $L_H^{-\frac{1}{2}}$ dependence in the iteration complexity.
The idea behind the following lemma is that when $M_k > \Omega(L_H)$ but $k \in \cJ^0$, we will find that the gradient decreases linearly, implying that this event can occur at most $O\big ( \log \frac{U_\varphi}{\epsilon_{k_{\mathrm{init}}}} \big )$ times.

\begin{proposition}[Initial phase, see \propositionref{prop:proof/initial-phase-decreasing-Mk}]
    \label{prop:main/initial-phase-decreasing-Mk}
    Let $k_{\mathrm{init}} = \min\{ j : M_j \leq O(L_H) \}$ and assume $M_0 > \Omega(L_H)$, then 
    for the first choice in \theoremref{thm:newton-local-rate-boosted}, we have
        $k_{\mathrm{init}} 
        \leq 
        O\Big ( \log \frac{M_0}{L_H} \log \frac{U_\varphi}{\epsilon_{k_{\mathrm{init}}}}\Big)$;
    and for the second choice, we have
        $k_{\mathrm{init}} 
        \leq 
        O\Big ( \log \frac{M_0}{L_H} + \log \frac{U_\varphi}{\epsilon_{k_{\mathrm{init}}}}\Big)$.
\end{proposition}




\subsection{The local convergence order} \label{sec:main/boosting-local-rates}

From the compactness of $L_\varphi(x_0)$ in Assumption~\ref{assumption:liphess}, we know there exists a subsequence $\{ x_{k_j} \}_{j \geq 0}$ converging to some $x^*$ with $\nabla\varphi(x^*) = 0$ (see \theoremref{thm:appendix/global-newton-complexity}).
In the analysis of the local convergence rate, we need to assume the positive definiteness of $\nabla^2\varphi(x^*)$,
under which the whole sequence $\{x_k\}_{k\ge 0}$ also converges to $x^*$ (see \propositionref{prop:mixed-newton-nonconvex-phase-local-rates}).

The standard analysis of the local rates for Newton methods consists of two steps.
The first step shows that the Newton direction (i.e., $(\nabla^2 \varphi(x_k) + \omega_k \Id)^{-1} \nabla \varphi(x_k)$) yields superlinear convergence, and then the second step shows this direction is eventually taken.
Since there are some adjustments in our usage of these results, 
we provide the proofs in \Cref{sec:properties-of-newton-step} for completeness, 
and present the statements below.

\begin{lemma}%
    \label{lem:main/asymptotic-newton-properties}
    Assuming $\nabla^2\varphi(x^*) \succeq \alpha \Id$,
    if $\text{d\_type}_k = \texttt{SOL}$ and $m_k = 0$,
    and $x_k$ is close enough to $x^*$, we have 
    $g_{k+1} \leq O(g_k^2 + \omega_k g_k)$.
    Furthermore, under the choices of regularizers in \theoremref{thm:newton-local-rate-boosted},
    if $x_k$ is close enough to $x^*$,
    we know the trial step is accepted, and $\text{d\_type}_k = \texttt{SOL}$ and $m_k = 0$.
\end{lemma}


\begin{figure}[tbp]
    \centering
    \includegraphics{figures/local-order.pdf}
    \caption{
        The left plot illustrates the local order achievable by the regularizers in \theoremref{thm:newton-local-rate-boosted} for $\theta \in (0, 1]$.
        It can be made arbitrarily close to $1 + \nu_\infty$. 
        The right plot illustrates the local order for different $\theta$ using $\varphi(x) = \frac{1}{2}x^2$,
        and its slope reflects the local order and aligns with our predictions.
        }
    \label{fig:local-rate-for-nu1}
\end{figure}

We observe that when taking $\omega_k^{\supsucc} = \omega_k^{\supfallback} = O(g_k^{\bar\nu})$ with $\bar \nu \in (0, 1]$, the gradient norm converges superlinearly with order $1 + \bar\nu$.
For the choices in \theoremref{thm:newton-local-rate-boosted}, we find $\max(\omega_k^{\supsucc},\omega_k^{\supfallback}) \leq \sqrt{g_k}$ so a local rate of order $\frac{3}{2}$ can be achieved.
Furthermore, the following technical lemma shows that the local order can be improved to arbitrarily close to $1 + \nu_\infty \in \big(\frac{3}{2}, 2\big]$ for $\theta > 0$ with $\nu_\infty$ defined in \lemmaref{lem:superlinear-rate-boosting} (see \figureref{fig:local-rate-for-nu1} for an illustration), and achieves quadratic convergence for $\theta > 1$.
Its premise will be satisfied as long as the iteration is close to the solution according to \lemmaref{lem:main/asymptotic-newton-properties}.

\begin{lemma}[Local rate boosting]
    \label{lem:superlinear-rate-boosting}
    Let $\theta > 0$ and $\{ g_k  \}_{k \geq 0} \subseteq (0, \infty)$. 
    Suppose $g_1 \leq O \big( g_0^{\frac{3}{2}} \big)$ and
    $g_{k+1} \leq O \big (g_k^{2} + g_k^{\frac{3}{2}} \frac{g_k^\theta}{g_{k-1}^\theta} \big )$ holds for each $k \geq 1$,
    and $g_0$ is sufficiently small. 
    Then, 
    \begin{enumerate}
        \item If $\theta \in (0, 1]$, let $\nu_\infty$ be the positive root of the equation $\frac{1}{2} + \frac{\theta \nu_\infty}{1 + \nu_\infty} = \nu_\infty$, 
        then we have $g_{k + 1} \leq O\big ( g_k^{1 + \nu_\infty - (4\theta/9)^{k}}  \big )$,
        i.e., $g_k$ has local order $1 + \nu_\infty - \delta$ for any $\delta > 0$.
        \item If $\theta > 1$ and $k \geq2\log \frac{2\theta - 1}{2\theta - 2} + 1$, 
        then $g_{k+1} \leq O(g_k^2)$,
        i.e., $g_k$ converges quadratically.
    \end{enumerate}
\end{lemma}
\begin{proof}[Sketch of the idea]
    If $g_{k} = O(g_{k-1}^{\alpha})$ for $\alpha \in (1, 2]$, 
    then $g_{k-1}^{-\theta} = O\big (g_k^{-\frac{\theta}{\alpha}}\big )$.
    Thus, $g_{k+1} \leq O\big (g_k^2 + g_k^{\frac{3}{2} + \theta - \frac{\theta}{\alpha}}\big )$, 
    implying that the local order becomes $\min\big (2, \frac{3}{2} + \frac{\theta\alpha}{1 + \alpha} \big) > \frac{3}{2}$.
    By recursively applying this argument, we can gradually improve the local order.
    See \Cref{sec:appendix/local-rate-boosting} for details.
\end{proof}




\section{Preliminary numerical results} 
\label{sec:main/numerical}

\begin{figure}[tbp]
    \centering
    \includegraphics{figures/main-ss1-time.pdf} \hfill
    \includegraphics{figures/main-ss1-hesseval.pdf}
    \caption{
        Comparison of success rates as functions of elapsed time and Hessian evaluations for CUTEst benchmark problems.  
        \algname{ARNCG$_g$}, \algname{ARNCG$_\epsilon$}, and ``Fixed'' correspond to \Cref{alg:adap-newton-cg} with the first and second regularizers from \theoremref{thm:newton-local-rate-boosted}, and a fixed $\omega_k \equiv \sqrt{\epsilon}$, respectively.  
        For Hessian evaluations, 
        since our algorithm accesses this information only via Hessian-vector products, 
        we count multiple products involving $\nabla^2\varphi(x)$ at the same point $x$ as a single evaluation.
        }
    \label{fig:main-algoperf}
\end{figure}

In this section, we present some preliminary numerical results.\footnote{Our code is available at \url{https://github.com/miskcoo/ARNCG}.} %
Our primary goal is to provide an overall sense of our algorithm's performance and the effects of its components.
Detailed results are deferred to \Cref{sec:appendix/numerical-results}.

Since the recently proposed trust-region-type method \algname{CAT} has an optimal rate and shows competitiveness with state-of-the-art solvers~\citep{hamad2024simple}, we adopt their experimental setup and compare with it, as well as the regularized Newton-type method \algname{AN2CER} proposed by \citet{gratton2024yet}.
The experiments are conducted on the 124 unconstrained problems with more than 100 variables from the widely used CUTEst benchmark for nonlinear optimization~\citep{gould2015cutest}.
The algorithm is considered successful if it terminates with $\epsilon_k \leq \epsilon = 10^{-5}$ such that $k \leq 10^5$. If the algorithm fails to terminate within 5 hours, it is also recorded as a failure.

In \Cref{sec:appendix/numerical-results}, 
we observe that the fallback step has insignificant impact on  performance yet increases computational cost, suggesting it can be relaxed or removed.
Furthermore, $\theta \in [0.5, 1]$ balances computational efficiency and local behavior 
and a small $m_{\mathrm{max}}$ is preferable. 
Finally, the second linesearch step \eqref{eqn:smooth-line-search-sol-smaller-stepsize} and the \texttt{TERM} state of \texttt{CappedCG} are rarely taken in practice.

\figureref{fig:main-algoperf} shows our method without the fallback step (see \Cref{sec:appendix/numerical-results} for details). 
It is slightly faster than CAT and AN2CER, 
as each iteration uses only a few Hessian-vector products, 
whereas CAT relies on multiple Cholesky factorizations and AN2CER involves minimal eigenvalue computations. 
Meanwhile, our method requires a similar number of Hessian evaluations as CAT, and slightly fewer than AN2CER.
We also note that using a fixed $\omega_k = \sqrt{\epsilon}$ in \Cref{alg:adap-newton-cg}
may lead to failures when $g_k \gg \epsilon$, resulting in deteriorated performance.
Additionally, our method requires significantly less memory ($\sim$6GB) compared to CAT ($\sim$74GB) for the largest problem in the benchmark with 123200 variables, as it avoids  constructing the full Hessian.

\section{Routing in a general multi-objective framework}\label{sec:general-multi-obj}

In this section, we discuss a natural extension of the routing problem with general metrics. Since most of the constructions are very similar to the construction in the Introduction section \ref{sec:intro}, we keep our discussion short.  Say, we are provided $M$ pre-trained models, denoted as $f_1, \dots,f_M$. Given model $f_m$ and a sample point $(X, Y)$,  we are interested in $L$ different loss metrics, among which the first $L_1 $ many, denoted as $\ell_1\{f_m; X, Y\} , \dots, \ell_{L_1}\{f_m; X, Y\}$, depends on $f_m$, $X$ and $Y$, while the next $L_2 = L - L_1$ many, denoted as $\ell_{L_1+ 1}(f_m; X), \dots , \ell_{L}(f_m; X)$, depends only on $f_m$ and $X$.  In that case we consider a compromise between these $L$ losses: for an $\lambda = (\lambda_1 , \dots, \lambda_L) \in\Delta^{L-1} $ the linear loss trade-off in eq. \eqref{eq:linearized-loss} naturally extends to 
\begin{equation} \label{eq:linearized-loss-gen}
 \textstyle \eta_{\lambda, m}(X, Y) =  \sum_{l = 1}^{L_1}\lambda_l \ell_l\{f_m; X, Y\} + \sum_{l = L_1 + 1}^{L}\lambda_l \ell_l\{f_m; X\}   \,.
\end{equation} 
To make it even more general, assume that we have a classification task with $M$ classes, \ie \ the outcome $Y \in \{1, \dots, M\}$ is now categorical. Additionally, as a multi-objective problem we have $L$ different loss metrics, among which the first $L_1 $ many $\ell_1\{m; X, Y\} , \dots, \ell_{L_1}\{m; X, Y\}$ are losses  of predicting a sample $(X, Y)$ as the class $m$, while the next $L_2 = L - L_1$ many $\ell_{L_1+ 1}\{m; X\}, \dots , \ell_{L}\{m; X\}$ is the cost of predicting the sample $X$ as the class $m$. Then the aggregated loss is 
\begin{equation} \label{eq:linearized-loss-gen-2}
 \textstyle \eta_{\lambda, m}(X, Y) =  \sum_{l = 1}^{L_1}\lambda_l \ell_l\{m; X, Y\} + \sum_{l = L_1 + 1}^{L}\lambda_l \ell_l\{m; X\}   \,.
\end{equation} 

For a particular $\lambda \in \Delta^{L-1}$ we want to estimate the optimal/Bayes/admissible classifier $g_\lambda^\star$ that minimized the average of the aggregates loss trade-off:
\begin{equation}
    \textstyle g_\lambda^\star = \argmin_{g: \cX \to [M]}  \Ex_P\big[\sum_{m = 1}^M \bbI\{g(X) = m\} \eta_{\lambda, m} (X, Y) \big ]
\end{equation}
Similar to Lemma \ref{lemma:oracle-router}, defining $\Phi_{l, m}^\star(X) = \Ex[\ell_l\{f_m; X,Y\} \mid X]$ for $l \in [L_1]$ and $\Phi_{l, m}^\star(X) = \ell_l\{f_m; X\} $ for $l \in \{L_1 + 1, \dots, L\}$ the Bayes classifier $g_\lambda^\star$ has the explicit form  (\cf\ Lemma \ref{lemma:bayes-classifier-gen})
\begin{equation}\label{eq:oracle-router-gen}
  \textstyle  g_\lambda^\star (X) = \argmin_m  \eta_{\lambda, m}(X), ~~ \eta_{\lambda, m}(X) = \sum_{l = 1}^L \lambda_l \Phi_{l, m}^\star (X)\,.
\end{equation}
A natural extension of Algorithm \ref{alg:pareto-routers} requires us to estimate each of the functions $\Phi_{l, m}^\star$ as $\widehat \Phi_{l, m}$ then estimate all Bayes classifiers in a one-shot plug-in approach as:
\[
\textstyle \widehat g_\lambda(X) = \argmin_m  \{\sum_{l = 1}^L \lambda_l \widehat \Phi_{l, m} (X) \}\,.
\] 
Note that, for $l \ge L_1+ 1$ the functions $\Phi_{l, m}^\star$ are known, so they need not be estimated; in these cases, we simply let $\widehat \Phi_{l, m} = \Phi_{l ,m}^\star$. 
Having estimated $\widehat g_\lambda$, we can then evaluate them on a test data split with respect to each of the individual metrics to estimate the Pareto frontier and examine the optimal trade-off between the objectives.
We describe the procedure in Algorithm \ref{alg:pareto-routers-gen}.  

\begin{algorithm}
    \begin{algorithmic}[1]
\Require Dataset $\cD_n$
\State Randomly split the dataset into training and test splits: $\cD_n = \cD_{\text{tr}} \cup \cD_{\text{test}}$. 
\State  Learn estimates $\widehat \Phi_{l, m} (X)$ of the function $\Phi_{l, m}^\star(X)$ using the training split $\cD_{\text{tr}}$. 
\For{$\lambda \in \Delta^{L-1}$}
\State  Define $\widehat \eta_{\lambda, m}(X) =  \sum_{l = 1}^L \lambda_l \widehat \Phi_{l, m} (X)  $ and 
 $\widehat g_\lambda(X) = \argmin_m \widehat \eta_{\lambda, m}(X)$
 \For{$l \in \{1, \dots, L\}$}
 \State Calculate $\widehat \cE_{l, \lambda}  =  \frac1{|\cD_{\text{test}}|} \sum_{(X, Y) \in \cD_{\text{test}}}  \ell_l\{Y, f_{\widehat g_\lambda(X)}(X)\}$
 \EndFor
\EndFor

\Return $\{g_\lambda: \lambda \in \Delta^{L-1}\}$ and $\widehat\cF = \{(\widehat \cE_{1, \lambda}, \dots, \widehat \cE_{L, \lambda}): \lambda \in \Delta^{L-1}\}$. 
\end{algorithmic}
\caption{Learning of Bayes classifiers}
\label{alg:pareto-routers-gen}
\end{algorithm}


Except for some minor differences, the minimax rate analysis of the estimate of $g_\lambda^\star$ is identical to Section \ref{sec:lower-bound}. We assume that
\begin{enumerate}
    \item For $1 \le l \le L_1$ the $\Phi_{l, m}^\star$ functions are $(\beta_l, K_{\beta, l})$-smooth, which is similar to the Assumption \ref{assmp:smooth}. Recall the discussion following Assumption \ref{assmp:smooth}; (1) Since $\Phi_{l, m}$ are known for $l \ge L_1 + 1$ a smoothness assumption on them is not necessary, and (2) We could assume different smoothness for different $\Phi_{l, m}^\star$, \ie\ they are $\beta_{l, m}$ smooth, but the rate will only involve the smallest smoothness, \ie\ $\beta_{l, \min} = \min_m \beta_{l, m}$. Thus, for simplicity, we assume that for a given $l$ the $\beta_{l, m}$ are identical for different $m$. 
    \item The margin condition in Assumption \ref{assmp:margin} is satisfied with the new definition of $\eta_{\lambda, m}(X) = \Ex[\eta_{\lambda, m}(X, Y)\mid X]$ with parameters $(\alpha, K_\alpha)$.
    \item  The $P_X$ is compactly supported and satisfies the strong density condition, as in the assumption \ref{assmp:strong-density}.
\end{enumerate}

%  and that the margin condition is satisfied  Furthermore, we assume the 
% \SM{talk about how the worst smoothness parameter is the one that matters.}


Under these assumptions, we establish the upper and lower bound analysis for the minimax rate of convergence in excess risk
\[
\textstyle \cE_P(g, \lambda) = \cL_P(g, \lambda) - \cL_P(g_\lambda^\star, \lambda)\,.
\]
For this purpose, let us quickly recall the notation and definitions necessary to describe the lower and upper bounds. We denote the class of all probability distributions on $\cX \times \cY$ space by $\cP$.
We also denote $\Gamma = \{g: \cX \to [M]\}$ as the set of all classifiers and $\cA_n$ as the class of learning algorithms, such that an algorithm $\cA_n \ni A_n: \cZ^n \to \Gamma $ that takes the dataset $\cD_n $ as input and produces a classifier $A_n(\cD_n): \cX \to [M]$.

\begin{theorem}
\label{thm:bound-gen}
Suppose that $\max_{l \in  [L_1]}\beta_l <  \nicefrac{d}{\alpha}$. Then we have the following lower and upper bounds for the excess risk. 
\begin{itemize}
    \item {\bf Lower bound:} For $n \ge 1$  and $A_n \in \cA_n$ define $\cE_P(A_n, \lambda) = \Ex_{\cD_n}\big[\cE_P\big(A_n(\cD_n), \lambda\big)\big]$. There exist constants $c_1, c_2> 0$ that are independent of $n$ and $\lambda$ such that for any $n\ge 1$ and $ \lambda \in \Delta^{L-1}$ we have the lower bound
    \begin{equation} \label{eq:lower-bound-gen}
      \textstyle  \min\limits_{A_n \in \cA_n} \max\limits_{P \in \cP} ~~ \cE_P(A_n, \lambda) \ge c \big \{\sum_{l = 1}^{L_1}\lambda_l n^{- \frac{\beta_l}{2\beta_l + d}}\big\}^{1+\alpha} \,.
    \end{equation}  
    \item  {\bf Upper bound:} For $l \le L_1$ and $m\in [M]$ suppose that there are estimators $\widehat\Phi_{l, m}$ for $\Phi^\star_{l, m}$ that satisfies the following: for constants $\rho_{l, 1}, \rho_{l, 2} > 0$ and any $n \ge 1$ and $t > 0$ and almost all $X$ with respect to $P_X$ we have 
    \begin{equation} \label{eq:concentration-phi-gen}
        \max_{P\in \cP} P \big \{ \max_m \big |\widehat \Phi_{l, m} (X) - \Phi^\star_{l, m}  (X)\big |  \ge t\big \} \le  \rho_{l, 1} \exp\big (- \rho_{l,2} a_{l, n}^2 t^2 \big )\,,  
    \end{equation}  where the sequence $\{a_{l, n}; n \ge 1\}\subset (0, \infty)$  increases to $\infty$. Then there exists a $K> 0$ such that for any $n \ge 1$ and $\lambda \in \Delta^{L-1} $ the excess risk for the router $\widehat g_\lambda$ in Algorithm \ref{alg:pareto-routers-gen} is upper bounded as 
    \begin{equation} \label{eq:upper-bound-gen}
      \textstyle   \max\limits_{P\in \cP} \Ex_{\cD_n}\big [\cE_P(\widehat g_\lambda,\lambda)\big ] \le K \big\{ \sum_{l = 1}^{L_1}\lambda_l a_{l, n}^{-1}\big\} ^{1+ \alpha}\,. 
    \end{equation}
    \item  {\bf Local polynomial estimator:} Assume that $\ell_l \{Y_i, f_m(X_i)\}$ are sub-Gaussian, \ie\ there exists constants $c_{l, 1}$ and $c_{l, 2}$ such that  
    \[
    \textstyle P\big ( |\ell_l \{Y, f_m(X)\} | > t \mid X\big ) \le c_{l, 1} e^{-c_{l, 2}t^2}\,. 
    \] If $\psi$ satisfies the regularity conditions with parameter $\beta_l$ in (\cf\ Definition \ref{def:kernel-reg})  and $k = \lfloor \beta_l \rfloor$ then for $h = n^{-\frac{1}{2\beta_l + d}}$ the inequality in \eqref{eq:concentration-phi-gen} holds for 
    \begin{equation}\label{LPR-gen}
    \widehat \Phi_{l, m}(x_0) = \widehat\theta_{x_0}^{(m)}(0), ~~ \widehat \theta_x^{(m)} \in \argmin_{\theta \in \Theta(k)}  \textstyle \sum_{i = 1}^n \psi (\frac{X_i - x_0}{h}) \big [\ell \{Y_i, f_m(X_i)\} - \theta (X_i -x_0 )\big]^2 \,. 
\end{equation}
    with $a_{l, n} = n^{\nicefrac{2\beta_l}{(2\beta_l + d)}}$, where $\Theta(k)$ is the class of all $k$ degree polynomials. 
\end{itemize} 

  
\end{theorem}
% \SM{Discuss the optimality in remarks.} 
\SM{comment about $\max_{l \in  [L_1]}\beta_l <  \nicefrac{d}{\alpha}$.}
We end our section with a few remarks. 

\begin{remark}[Rate optimality for plug-in routers]

Firstly, given that the upper bound in eq.  \eqref{eq:upper-bound-gen} for plug-in router described in Algorithm \ref{alg:pareto-routers-gen} with local-polynomial estimator for $\widehat \Phi_{l, m}(X)$ achieves the same rate of convergence as the lower bound in \eqref{eq:lower-bound} we conclude that the minimax optimal rate of convergence in excess risk is 
\begin{equation} \label{eq:optimal-rate-gen}
    \textstyle \min\limits_{A_n \in \cA_n} \max\limits_{P \in \cP} ~~ \cE_P(A_n, \lambda) \asymp \cO \Big(\big \{\sum_{l = 1}^{L_1}\lambda_l n^{- \frac{\beta_l}{2\beta_l + d}}\big\}^{1+\alpha}\Big)\,. 
\end{equation} Moreover, this also implies that the plug-in router can achieve this optimal rate as long as the bounds in eq. \eqref{eq:concentration-phi-gen} are satisfied with $a_{l, n} = n^{\nicefrac{2\beta_l}{(2\beta_l + d)}}$. 
    
\end{remark}

\begin{remark}[Difficulty in routing with respect to $\lambda$]
Moreover, the rate in eq. \eqref{eq:optimal-rate-gen} implies that for a smaller $\lambda_l$ the errors in estimating $\Phi_{l, m}^\star$ have a lesser impact on the excess risk convergence. This conclusion is very much related to our Remark \ref{remark:diff-in-lambda-lb}, and thus we keep our discussion brief and ask readers to revisit the said remark. We end with a quick observation that the hardest instance to classify is when $\lambda_{l_{\min}} = 1$ for the lowest smoothness parameter, \ie\ $l_{\min} = \argmin_{l \le L_1} \beta_l$. 

    
\end{remark}


\begin{remark}

{\bf (Minimax rate study for non-parametric classification with multiple objectives)}
Compared to \citet{audibert2007Fast}, our study of the optimal minimax rate provided in this section generalizes the setting on three fronts: (1) the number of classes can be more than two, (2) general classification loss functions beyond $0/1$-loss and (3) multiple objectives. In this general setting, we show that the plug-in classifiers described in Algorithm \ref{alg:pareto-routers-gen} are both computationally and statistically (rate optimal) efficient for estimating the entire class of Bayes classifiers $\{g_\lambda^\star: \lambda\in \Delta^{L-1}\}$. 
    
\end{remark}


\acks{
Y. Zhou and J. Zhu are supported by the National Natural Science
Foundation of China (Nos. 92270001, 62350080, 62106120), Tsinghua Institute for Guo Qiang,
and the High Performance Computing Center, Tsinghua
University; J. Zhu
was also supported by the XPlorer Prize.
C. Bao is supported by the National Key R\&D Program of China (No.~2021YFA1001300) and the National Natural Science Foundation of China (No.~12271291). 
J. Xu is supported in part by PolyU postdoc matching fund scheme of the Hong Kong Polytechnic University (No.~1-W35A), 
and Huawei's Collaborative Grants ``Large scale linear programming solver'' and ``Solving large scale linear programming models for production planning''. 
C. Ding is supported in part by the National Key R\&D Program of China (No.~2021YFA1000300, No.~2021YFA1000301) and CAS Project for Young Scientists in Basic Research (No.~YSBR-034).}

\bibliography{references}

\clearpage
\appendix


\tableofcontents

\clearpage

\input{newton-cg/appendix-structure}

\section{Details and properties of capped CG} \label{sec:appendix/capped-cg}


\begin{algorithm2e}[htbp]
    \caption{Capped conjugate gradient~\citep[Algorithm 1]{royer2020newton}} 
    \label{alg:capped-cg}

    \SetKwInOut{Input}{Input}
    \SetKwInOut{Output}{Output}
    \DontPrintSemicolon

    \Input{A symmetric matrix $H \in \R^{d \times d}$, a vector $g \in \R^d$, a regularizer $\rho \in (0,\infty)$, a parameter $\bar\rho \in (0,\infty)$ used to decide whether to terminate the algorithm earlier, and a tolerance parameter $\xi\in(0,1)$.}
    \Output{$(\text{d\_type}, \tilde d)$ such that $\text{d\_type} \in \{ \texttt{NC}, \texttt{SOL}, \texttt{TERM} \}$ and \lemmaref{lem:capped-cg} holds.}
        
    \SetKwFunction{CappedCG}{CappedCG}
    \SetKwProg{Fn}{Subroutine}{}{}

    \Fn{\CappedCG{$H, g, \rho, \xi, \bar \rho$}} {
    $(y_0, r_0, p_0, j) \gets (0, g, -g, 0)$\;
    $\bar H \gets H + 2\rho \Id$\;
    $M \gets \frac{\|Hp_0\|}{\|p_0\|}$\;
    \lIf{$p_0^\top \bar H p_0 < \rho \|p_0\|^2$} {
        \Return $(\texttt{NC}, p_{0})$
    }
    \While{True}{

        \tcp{Beginning of standard CG}
         $\alpha_k \gets \frac{\|r_k\|^2}{p_k^\top \bar H p_k}$\;
         $y_{k+1} \gets y_k + \alpha_k p_k$\;
         $r_{k+1} \gets r_k + \alpha_k \bar H p_k$\;
         $\beta_{k+1} \gets \frac{\|r_{k+1}\|^2}{\|r_k\|^2}$\;
         $p_{k+1} \gets -r_{k+1} + \beta_{k+1} p_k$\;
        \tcp{End of standard CG}

        $k \gets k + 1$\;
        $M \gets \max\left( M, \frac{\|Hp_{k}\|}{\|p_{k}\|}, \frac{\|Hr_{k}\|}{\|r_{k}\|}, \frac{\|Hy_{k}\|}{\|y_{k}\|} \right)$
        \tcp*[f]{Estimate the norm of $H$}
        \\
        $(\kappa, \hat\xi, \tau, T) \gets \left( 
            \frac{M + 2\rho}{\rho},
            \frac{\xi}{3\kappa},
            \frac{\sqrt \kappa}{\sqrt \kappa + 1},
            \frac{4\kappa^4}{(1 - \sqrt \tau)^2}
            \right)$\;
        
        \lIf{$y_{k}^\top \bar H y_{k} < \rho \| y_{k} \|^2$} {
            \Return $(\texttt{NC}, y_{k})$
        }\lElseIf{$\|r_{k}\|  \leq \hat \xi \| r_0 \|$} {
            \Return $(\texttt{SOL}, y_{k})$
        }\lElseIf{$p_{k}^\top \bar H p_{k} < \rho \| p_{k} \|^2$} {
            \Return $(\texttt{NC}, p_{k})$
        }\uElseIf{$\|r_{k}\|>\sqrt T\tau^{\frac{k}{2}} \|r_0\|$}  {
        $\alpha_{k} \gets \frac{\|r_{k}\|^2}{p_{k}^\top \bar H p_{k}}$\;
        $y_{k+1} \gets y_{k} + \alpha_{k} p_{k}$\;
        Find %
        $i \in \{ 0, \dots, k-1 \}$
        such that 
        \begin{equation}
            \label{eqn:capped-cg-slow-decay-condition}
            \frac{(y_{k+1} - y_i)^\top \bar H (y_{k+1} - y_i)}{\|y_{k+1}-y_i\|^2} < \rho.
        \end{equation} \\
        \Return $(\texttt{NC}, y_{k+1} - y_i)$\;
        }
        \ElseIf{$k \geq J(M, \bar\rho, \xi) + 1$}
        { 
            \Return $(\texttt{TERM}, y_k)$ 
            \tcp*[f]{$J(M, \bar\rho, \xi)$ is defined in \eqref{eqn:capped-cg-hessvec-evals}}
        }
    }
    }
\end{algorithm2e}



The capped CG in \citet{royer2020newton} is presented in \Cref{alg:capped-cg}, with an additional termination condition $k \geq J(M, \bar\rho, \xi) + 1$ and type \texttt{TERM}. 
Note that in \Cref{alg:adap-newton-cg}, we will take $\rho = \sqrt M \omega$. The following lemma states the number of iterations for the original version of capped CG.
\begin{lemma}[Lemma 1 of \citet{royer2020newton}]
    \label{lem:capped-cg-iteration-complexity}
    When the termination condition for \texttt{TERM} is removed, \Cref{alg:capped-cg} terminates in $\min(n, J(M, \rho, \xi)) + 1 \leq \min(n, \tilde O(\rho^{-\frac{1}{2}}))$ iterations, where
    \begin{equation}
    \label{eqn:capped-cg-hessvec-evals}
    J(M, \rho, \xi) 
    = 1 + \left( \sqrt \kappa + \frac{1}{2} \right) \log \left( \frac{144\left( \sqrt \kappa + 1 \right)^2 \kappa^6}{\xi^2} \right),
    \quad 
    \kappa = \frac{M + \rho}{\rho}
    .
\end{equation}
\end{lemma}
The additional termination condition indicates that the regularizer $\rho$ may be too small to find a solution within the given computational budget. 

For the oracle complexity, each iteration of \Cref{alg:capped-cg} requires only one Hessian-vector product, since the quantities $H y_k$, $H p_k$ and $H r_k$ used in the negative curvature monitor can be recursively constructed from $\bar H p_k$ generated in the standard CG iteration.
When the residual decays slower than expected, one more CG iteration is performed, and if the historical iterations are stored, only one additional Hessian-vector product is needed. 

The properties of our version with the \texttt{TERM} state are summarized below.
\begin{lemma}
    \label{lem:capped-cg}
    Invoking the subroutine \texttt{CappedCG}$(H, g, \rho, \xi, \bar\rho)$ obtains $(\text{d\_type}, \tilde d)$, then we have the following properties.
    \begin{enumerate}
        \item When $\text{d\_type} = \texttt{SOL}$, $\tilde d$ is an approximated solution of $(H + 2\rho \Id)\tilde d=-g$ such that
    \begin{align}
        \label{eqn:capped-cg-hessian-lowerbound}
        \tilde d^\top (H + 2\rho \Id) \tilde d &\geq\rho \|  \tilde d\|^2,  \\
        \label{eqn:capped-cg-hessian-lowerbound-H}
        \tilde d^\top H \tilde d &\geq -\rho \|  \tilde d\|^2, 
        \\
        \label{eqn:capped-cg-io-diff}
        \| \tilde d \| &\leq 2 \rho^{-1} \|g\|, \\
        \label{eqn:capped-cg-hessian-upperbound}
        \| (H + 2\rho \Id) \tilde d + g \| &\leq \frac{1}{2}\rho \xi \|\tilde d\| \leq \xi \|g\|,\\
        \label{eqn:capped-cg-descent-direction}
        \tilde d^\top g &= - \tilde d^\top (H + 2\rho \Id) \tilde d \leq -\rho \|\tilde d \|^2.
    \end{align}
        \item When $\text{d\_type} = \texttt{NC}$, $\tilde d$ is a negative curvature direction such that 
    \begin{align}
        \tilde d^\top H\tilde d \leq -\rho\|\tilde d\|^2.
        \label{eqn:capped-cg-nc-direction-inequ}
    \end{align}
        \item When $\text{d\_type} = \texttt{TERM}$, then $\rho < \bar \rho$.
        In other words, if $\bar\rho \leq \rho$ the algorithm terminates with $\text{d\_type} \in \{\texttt{SOL}, \texttt{NC}\}$.
    \item 
    Suppose there exist $\alpha, a, b > 0$ such that
     $H \succeq \alpha \Id$, $\bar \rho \leq b\rho^a$ and $\rho \leq 1$, then 
     the algorithm terminates with $\text{d\_type} = \texttt{SOL}$ when $\xi = \rho \leq C(\alpha, a, b, \|H\|)$, where
     \begin{align*}
        C(\alpha, a, b, U) := 
        \min\left( 1,
            \left (\frac{\alpha^2}{b U} \right )^{\frac{1}{a}}, 
            \left( \frac{24\alpha^7}{b^7 \sqrt{U}(U+2)} \right)^{\frac{1}{7a}},
            \left( \frac{12\alpha^7}{b^7} \right)^{\frac{1}{7a+2}}
         \right).
     \end{align*}
    \end{enumerate}
\end{lemma}
\begin{proof}
    The first two cases directly follow from \citet[Lemma 3]{royer2020newton}.\footnote{This lemma assumes that $H = \nabla^2 \varphi(x)$, $g = \nabla \varphi(x)$, and $\varphi$ has Lipschitz Hessian. However, the statement of this lemma and the capped CG involve only the Hessian of $\varphi$ at a single point $x$, and hence the assumption can be removed.} 
    The third case follows from \lemmaref{lem:capped-cg-iteration-complexity} and the 
    monotonic non-increasing property of the map $\rho \mapsto J(M, \rho, \xi)$.

    The fourth case follows from the standard property of CG for positive definite equation,
    since $H \succeq \alpha \Id$ the capped CG reduces to the standard CG. 
    Specifically, let $\{ y_k, r_k \}_{k\ge 0}$ be the sequence generated by \Cref{alg:capped-cg}, then \citet[Equation (5.36)]{nocedal1999numerical} gives that 
    \begin{align*}
        \| e_k \|_{\bar H}
        \leq 2 \left( \frac{\sqrt{\kappa({\bar H})}-1}{\sqrt{\kappa({\bar H})}+1} \right)^k \|e_0\|_{\bar H}
        \leq 2 \exp\left( \frac{-2k}{\sqrt{\kappa({\bar H})}} \right) \|e_0\|_{\bar H},
    \end{align*}
    where $\|e_k\|_{\bar H}^2 := e_k^\top \bar H e_k$ and $\kappa(\bar H) = (\alpha + 2\rho)^{-1}(\|H\| + 2\rho)$ is the condition number, 
    and $e_k = y_k + \bar H^{-1} g = \bar H^{-1} r_k$ and $\bar H = H + 2\rho \Id$.
    Then, the above display becomes
    \begin{align*}
        \frac{1}{\|H\|+ 2\rho}
        \|r_k\|^2
        &\leq
        r_k^\top  \bar H^{-1} r_k 
        \leq 4 \exp\left( \frac{-4k}{\sqrt{\kappa({\bar H})}} \right) r_0^\top \bar H^{-1}  r_0 \\
        &\leq 
        4 \exp\left( \frac{-4k}{\sqrt{\kappa({\bar H})}} \right)  
        \frac{1}{\alpha + 2\rho}
        \|r_0\|^2.
    \end{align*}
    Let $M, \kappa, \hat \xi$ be the quantities updated in the algorithm. 
    Then, we have $M \geq \alpha$ 
    and $\kappa \leq \rho^{-1}\|H\| + 2$ and $\hat \xi = \frac{\xi}{3\kappa} \geq \frac{\xi}{3\rho^{-1}\|H\| + 6}$.
    Hence, when the \texttt{TERM} state is removed, and suppose \Cref{alg:capped-cg} terminates at $k_*$-th step with $\texttt{SOL}$.
    Then, we have
    \begin{align}
        \label{eqn:proof/cg-normal-termination}
        k_* \leq \left \lceil \frac{1}{2} \sqrt{\kappa(\bar H)}
        \log \frac{6\sqrt{\kappa(\bar H)}(\rho^{-1} \|H \| + 2)}{\xi} \right \rceil.
    \end{align}
    Since $\kappa(\bar H) \leq \frac{\|H\|}{\alpha}$ and $\rho\le1$, 
    we know 
    \begin{align*}
        k_*
        &\leq 
        \frac{1}{2} \sqrt{\frac{\|H\|}{\alpha}}
        \log \frac{6\sqrt{\|H\|}(\|H \| + 2)}{\sqrt{\alpha}\rho\xi} + 1 
        =: K(\rho, \xi).
    \end{align*}
    When incorporating the \texttt{TERM} state, and suppose it is triggered at the $\hat k$-th step, then 
    \begin{equation}
        \label{eqn:proof/capped-cg-term-condition}
    K(\rho, \xi) \geq k_* > \hat k \geq J(M, \bar \rho, \xi) + 1
    \geq J(M, \bar \rho, \xi)
    .
    \end{equation}
    However, when $b\rho^a \geq \bar \rho$, we have 
    \begin{align*}
        J(M, \bar\rho, \xi)
        \geq 
        J(\alpha, \bar\rho, \xi)
        \geq J(\alpha, b\rho^a, \xi)
        \geq \sqrt{\frac{\alpha}{b\rho^a}} \log \frac{144 \alpha^7}{\xi^2 b^7 \rho^{7a}}.
    \end{align*}
    Hence, when $\xi = \rho \leq C(\alpha, a, b,  \|H\|)$, 
    we have $\frac{\alpha}{b\rho^a} \geq \frac{\|H\|}{\alpha} \geq 1$ and 
    $\frac{144\alpha^7}{b^7 \rho^{7a + 2}} \geq \frac{6\sqrt{\|H\|} (\|H\|+2)}{\sqrt{\alpha} \rho^2}$
    and $\frac{144\alpha^7}{b^7 \rho^{7a + 2}} \geq 12$.
    Then, 
    \begin{align*}
        0 \overset{\eqref{eqn:proof/capped-cg-term-condition}}&{\geq} J(M, \bar \rho, \rho) - K(\rho, \rho) \\
        &\geq 
         \sqrt{\frac{\alpha}{b \rho^a}} \log \frac{144 \alpha^7}{b^7\rho^{7a+2}}
        - \frac{1}{2}\sqrt{\frac{\|H\|}{\alpha}}\log  
        \frac{6\sqrt{\|H\|} (\|H\|+2)}{\sqrt{\alpha} \rho^2} -1\\
        &\geq 
         \frac{1}{2}\sqrt{\frac{\|H\|}{\alpha}} \log \frac{144 \alpha^7}{b^7\rho^{7a + 2}} - 1
         \geq \frac{\log 12}{2} - 1
         > 0
        ,
    \end{align*}
    which leads to a contradiction. Therefore, the algorithm will terminate with \texttt{SOL}.
\end{proof}




\input{newton-cg/appendix-discussions}
\section{Main results for global rates}
\label{sec:appendix/global-rate-under-local-boosting}


Throughout this section, we follow the partition \eqref{eqn:proof/newton-partition} defined in \Cref{sec:main/sqrt-global-rate}, 
and provide detailed proofs for the global rates in \theoremref{thm:newton-local-rate-boosted} and corresponding lemmas described in \Cref{sec:main/sqrt-global-rate}.
For the sake of readability, we restate the lemmas mentioned in \Cref{sec:main/sqrt-global-rate}.

\subsection{Details in \Cref{sec:main/sqrt-global-rate}}

\begin{lemma}[Restatement of \lemmaref{lem:main/transition-between-subsequences-give-valid-regularizer}]
    \label{lem:proof/transition-between-subsequences-give-valid-regularizer}
    Under the regularizer choices of \theoremref{thm:newton-local-rate-boosted}, 
    we have $\omega_{\ell_{j}-1} = \omega_{\ell_{j}-1}^{\supfallback}$ for each %
    $j \ge 2$,
    and 
    \begin{equation}
        \label{eqn:inexact-mixed-newton-boundary-bound}
        \varphi(x_{\ell_{j}})
        - \varphi(x_{\ell_{j}-1}) 
        \leq 
        -\tilde C_1 M_{\ell_j-1}^{-\frac{1}{2}} \bone_{\{\ell_{j}-1 \notin \cJ^{1}\}} (\omega_{\ell_{j}-1}^{\supfallback})^3,
    \end{equation}
    where $\tilde C_1, \tilde C_4$ are defined in \lemmaref{lem:lipschitz-constant-estimation}.
    Moreover, if $M_{\ell_j-1} > \tilde C_4 L_H$, then $\ell_j -1 \in\cJ^{-1}$.
\end{lemma}
\begin{proof}
    Let $k = \ell_j - 1$.
    If the fallback step is taken, then $\omega_k = \omega_k^{\supfallback}$ holds.
    We consider the case where the trial step at $k$-th iteration is accepted, 
    then we know $g_{k + \frac{1}{2}} = g_{k+1} 
    > g_k$ by the partition rule \eqref{eqn:proof/newton-partition}.
    However, the acceptance rule of the trial step in \Cref{alg:adap-newton-cg} 
    gives that $g_k > g_{k-1}$, 
    and hence $\min(1, g_k^\theta g_{k-1}^{-\theta}) = 1$.
    Moreover, we have $g_{k-1} \geq \epsilon_{k-1}$ and then 
    \begin{align*}
    \epsilon_k 
    = \min(\epsilon_{k-1}, g_k) 
    \geq \min(\epsilon_{k-1}, g_{k-1}) 
    = \epsilon_{k-1} \geq \epsilon_k.
    \end{align*}
    Therefore, $\epsilon_k^\theta \epsilon_{k-1}^{-\theta} = 1$.
    Combining these discussions, we know $\omega_k = \omega_k^{\supfallback}$ for the two choices of regularizers.

    It remains to show that $D_k \geq(\omega_k^{\supfallback})^3$ for $D_k$ defined in \lemmaref{lem:lipschitz-constant-estimation},
    which holds since we know
    $g_{k+1} > g_k$
    by the partition rule \eqref{eqn:proof/newton-partition}, and $g_k \geq (\omega_k^{\supfallback})^2$ by the choice of regularizers, and therefore,
    \begin{align}
    D_k \overset{\eqref{eqn:summarized-descent-amount}}{\geq}
    \min((\omega_k^{\supfallback})^3, g_{k+1}^2(\omega_k^{\supfallback})^{-1}) 
    \geq \min((\omega_k^{\supfallback})^3, g_{k}^2(\omega_k^{\supfallback})^{-1}) 
    \geq (\omega_k^{\supfallback})^3.
    \label{eqn:proof/transition-descent-amount}
    \end{align}


Finally, when $M_k > \tilde C_4 L_H$, we use \corollaryref{cor:appendix/decreasing-Mk-condition} to show that $k \in \cJ^{-1}$.
For the first case in that corollary,
since $\tau_- < 1$, then  $\omega_{k} = \omega_{k}^{\supfallback} > \tau_- \omega_{k}^{\supfallback}$, 
then the corollary gives $k \in \cJ^{-1}$.
For the second case, the results follows from \eqref{eqn:proof/transition-descent-amount} and 
$\min(\omega_k^3, g_{k+1}^2\omega_k^{-1}) \geq (\omega_k^{\supfallback})^3 > \tau_- (\omega_k^{\supfallback})^3$.
\end{proof}

\begin{lemma}[Restatement of \lemmaref{lem:main/iteration-in-a-subsequence}]
    \label{lem:proof/iteration-in-a-subsequence}
    Under the regularizer choices of \theoremref{thm:newton-local-rate-boosted}, 
    we have $(\omega_k^{\supfallback})^{1 + 2\theta}(\omega^{\supfallback}_{k-1})^{-2\theta}\leq \omega_k \leq \omega^{\supfallback}_k$ for each %
    $k \ge 1$.
    Moreover, for $j \geq 1$ and $\ell_j  < k < \ell_{j+1}$,
    \begin{align}
        \varphi(x_{k})
        - \varphi(x_{\ell_{j}})
        \leq
         - C_{\ell_j,k}
        \left ( 
            |I_{\ell_j,k} \cap \cJ^{-1} |
            + 
            \max\left ( 0, | I_{\ell_j,k} \cap \cJ^0 | - T_{\ell_j,k} - 5 \right ) 
        \right ) (\omega_k^{\supfallback})^3
         ,
         \label{eqn:inexact-mixed-newton-inner-bound}
    \end{align}
    where $C_{i,j} = \tilde C_1 %
    \min_{i \leq l < j} M_l^{-\frac{1}{2}}$,
    $T_{i,j}=2\log\log\big (3 (\omega_i^{\supfallback})^2 (\omega_j^{\supfallback})^{-2}\big)$, 
    and $\tilde C_1$ is defined in \lemmaref{lem:lipschitz-constant-estimation}.
\end{lemma}
\begin{proof}
    Under the regularizers choices, we know for each $k \in \N$, $D_k$ defined in \eqref{eqn:summarized-descent-amount} satisfies that 
    \begin{align}
        \nonumber
    D_k 
    &\geq \min\left( (\omega_k^{\supfallback})^3, g_{k+1}^2\omega_k^{-1}, \omega_k^3 \right)
    = \min\left( g_{k+1}^2\omega_k^{-1}, \omega_k^3 \right) \\
    &\geq 
    \min\left( g_{k+1}^2(\omega_k^{\supfallback})^{-1}, (\omega_k^{\supfallback})^{3 + 6\theta}(\omega_{k-1}^{\supfallback})^{-6\theta} \right).
    \label{eqn:proof/lower-bound-of-descent-amount-general}
    \end{align}
    \paragraph{Case 1}
    For the first choice of regularizers, 
    we have $\omega_i^{\supfallback} = \sqrt{g_i}$ and
    $T_{i,j} = 2\log\log \frac{3g_i}{g_j}$, and
    \begin{align*}
        \varphi(x_{i+1}) - \varphi(x_i)
        \overset{\eqref{eqn:summarized-descent-inequality}}{\leq} 
        \begin{cases}
        -C_i\min\left( g_{i+1}^2 g_i^{-\frac{1}{2}}, g_i^{\frac{3}{2} + 3\theta} g_{i-1}^{-3\theta} \right),
        & \text{ if } i \notin \cJ^{-1}, \\
        -C_i g_i^{\frac{3}{2}}, 
        & \text{ if } i \in \cJ^{-1},
        \end{cases}
    \end{align*}
    where $C_i := \tilde C_1 M_i^{-\frac{1}{2}}$.
    
    When $\theta > 0$, for any $\ell_j < k \leq \ell_{j+1} - 1$, using \lemmaref{lem:accumulated-mixed-descent-lower-bound} with 
    \begin{equation}
     (p_1, q_1, p_2, q_2, a, A, K, S) = \left (2, \frac{1}{2}, \frac{3}{2} + 3\theta, 3\theta, g_{k}, g_{\ell_j}, k - \ell_j - 1, I_{\ell_{j},k} \cap \cJ^0 \right ), 
     \label{eqn:proof/parameter-choices-of-the-descent-lowerbound-lemma}
    \end{equation}
    we see that 
    \begin{align}
        \nonumber
        \varphi(x_{k})
        - \varphi(x_{\ell_{j}})
        \overset{\eqref{eqn:summarized-descent-inequality}}&{\leq}
         - \tilde C_1 \sum_{\substack{\ell_j \leq i < k\\ i\in \cJ^{-1}}} 
         M_i^{-\frac{1}{2}} g_i^{\frac{3}{2}}  
         - \tilde C_1 
         \sum_{\substack{\ell_j \leq i < k\\ i\in \cJ^0}} 
         M_i^{-\frac{1}{2}} 
         \min\left( g_{i+1}^2 g_i^{-\frac{1}{2}}, g_i^{\frac{3}{2} + 3\theta} g_{i-1}^{-3\theta} \right)  \\
         \nonumber
         &\leq 
         - C_{\ell_j,k}
         \sum_{\substack{\ell_j \leq i < k\\ i\in \cJ^{-1}}} 
        g_i^{\frac{3}{2}} 
         - C_{\ell_j,k}
         \sum_{\substack{\ell_j \leq i < k\\ i\in \cJ^{0}}} 
         \min\left( g_{i+1}^2 g_i^{-\frac{1}{2}}, g_i^{\frac{3}{2} + 3\theta} g_{i-1}^{-3\theta} \right)
         \\
        \overset{\eqref{eqn:accumulated-mixed-descent-lower-bound}}&{\leq} 
         - C_{\ell_j,k}
        \left ( 
            |I_{\ell_j,k} \cap \cJ^{-1} |
            + 
            \max\left ( 0, | I_{\ell_j,k} \cap \cJ^0 | - T_{\ell_j,k} - 5 \right ) 
        \right ) g_{k}^{\frac{3}{2}}
         .
         \label{eqn:proof/inexact-mixed-newton-inter-bound}
    \end{align}

    On the other hand, when $\theta = 0$, we know $\varphi(x_{i+1}) - \varphi(x_i) \leq -C_i g_{i+1}^2 g_i^{-\frac{1}{2}}$ for $i \notin \cJ^{-1}$, and \eqref{eqn:proof/inexact-mixed-newton-inter-bound} also holds by applying \lemmaref{lem:accumulated-descent-lower-bound} with
    \begin{equation*}
     (p, q, a, A, K, S) = \left (2, \frac{1}{2}, g_{k}, g_{\ell_j}, k - \ell_j - 1, I_{\ell_{j},k} \cap \cJ^0 \right ).
    \end{equation*}

    \paragraph{Case 2}
    For the second choice of the regularizers, we have $\omega_i^{\supfallback} = \sqrt{\epsilon_i}$
    and $T_{i,j} = 2\log\log \frac{3\epsilon_i}{\epsilon_j}$.

    Since $\epsilon_k$ is non-increasing and $\omega_k \leq \sqrt{\epsilon_k}$ for each $k \in \N$,  
    then for a fixed $i$ such that $\ell_j \leq i < \ell_{j+1} - 1$, we know $g_i \geq g_{i+1}$ and have the following two cases.
    \begin{enumerate}
        \item If $g_{i+1} \geq \epsilon_{i-1}$, we know $\epsilon_{i} 
        = \min( \epsilon_{i-1}, g_i) 
        \geq \min( \epsilon_{i-1}, g_{i+1}) 
        = \epsilon_{i-1} \geq \epsilon_i$. 
        Then, 
        \begin{align*}
        D_i 
        \overset{\eqref{eqn:proof/lower-bound-of-descent-amount-general}}{\geq} 
        \min \big (g_{i+1}^2\epsilon_i^{-\frac{1}{2}}, \epsilon_i^{\frac{3}{2} + 3\theta}\epsilon_{i-1}^{-3\theta} \big )
        \overset{(g_{i+1} \geq \epsilon_{i-1})}{\geq}
        \min \big (\epsilon_{i-1}^2\epsilon_i^{-\frac{1}{2}}, \epsilon_i^{\frac{3}{2} + 3\theta}\epsilon_{i-1}^{-3\theta} \big )
        \overset{(\epsilon_i = \epsilon_{i-1})}{=}
        \epsilon_i^{\frac{3}{2}}
        .
        \end{align*}
        \item If $g_{i+1} < \epsilon_{i-1}$, 
        then using $g_{i+1} \geq \min(g_{i+1}, \epsilon_i) =  \epsilon_{i+1}$,
        we have
        \begin{align*}
        D_i 
        \overset{\eqref{eqn:proof/lower-bound-of-descent-amount-general}}{\geq} 
        \min \big (g_{i+1}^2\epsilon_i^{-\frac{1}{2}}, \epsilon_i^{\frac{3}{2} + 3\theta}\epsilon_{i-1}^{-3\theta} \big )
        \overset{(g_{i+1} \geq \epsilon_{i+1})}{\geq}
        \min \big (\epsilon_{i+1}^2\epsilon_i^{-\frac{1}{2}}, \epsilon_i^{\frac{3}{2} + 3\theta}\epsilon_{i-1}^{-3\theta} \big )
        .
        \end{align*}
    \end{enumerate}
    Thus, from \lemmaref{lem:lipschitz-constant-estimation}, we know for $\ell_j \leq i < \ell_{j+1} - 1$, it holds that
    \begin{align*}
        \varphi(x_{i+1}) - \varphi(x_i)
        \overset{\eqref{eqn:summarized-descent-inequality}}{\leq} 
        \begin{cases}
        -C_i \min \left (\epsilon_{i+1}^2\epsilon_i^{-\frac{1}{2}}, \epsilon_i^{\frac{3}{2} + 3\theta}\epsilon_{i-1}^{-3\theta} \right ),
        & \text{ if } i \notin \cJ^{-1} \text{ and } g_{i+1} < \epsilon_{i-1}, \\
        -C_i \epsilon_i^{\frac{3}{2}}, 
        & \text{ if } i \in \cJ^{-1} \text{ or } g_{i+1} \geq \epsilon_{i-1}.
        \end{cases}
    \end{align*}

    Define $\cJ^0_+ = \cJ^0 \cap \{ i : g_{i+1} \geq \epsilon_{i-1} \}$ and $\cJ^0_- = \cJ^0 \setminus \cJ_+^0$.
    For any $\ell_j < k \leq \ell_{j+1} - 1$ and $\theta > 0$, we can apply \lemmaref{lem:accumulated-mixed-descent-lower-bound},
    with the parameters $a, A$, and $\{g_i\}_{0\le i \le K+1}$ therein chosen as $\epsilon_k, \epsilon_{\ell_j}$, and $\{\epsilon_i\}_{\ell_j \le i\le k}$, respectively, 
    and other parameter choices remain the same as \eqref{eqn:proof/parameter-choices-of-the-descent-lowerbound-lemma}.
    Then, we know
    \begin{align}
        \nonumber
        \varphi(x_{k})
        - \varphi(x_{\ell_{j}}) 
        \overset{\eqref{eqn:summarized-descent-inequality}}&{\leq}
         - C_{\ell_j,k}
         \sum_{\substack{\ell_j \leq i < k\\ i\in \cJ^{-1} \cup \cJ^0_+}} \epsilon_i^{\frac{3}{2}}  
         - C_{\ell_j,k}
         \sum_{\substack{\ell_j \leq i < k\\ i\in \cJ^0_-}} 
            \min \left (\epsilon_{i+1}^2\epsilon_i^{-\frac{1}{2}}, \epsilon_i^{\frac{3}{2} + 3\theta}\epsilon_{i-1}^{-3\theta} \right ) 
            \\
         \nonumber
        \overset{\eqref{eqn:accumulated-descent-lower-bound}}&{\leq} 
         - C_{\ell_j,k}
        \left ( 
            |I_{\ell_j,k} \cap (\cJ^{-1} \cup \cJ^0_+) | + 
            \max\left ( 0, | I_{\ell_j,k} \cap \cJ^0_- | - T_{\ell_j, k} - 5 \right ) 
        \right ) \epsilon_{k}^{\frac{3}{2}} \\
        \nonumber
        &= 
         - C_{\ell_j,k}
        \left ( 
            |I_{\ell_j,k} \cap \cJ^{-1} | + 
            \max\left ( | I_{\ell_j,k} \cap \cJ^0_+ |, | I_{\ell_j,k} \cap \cJ^0 | - T_{\ell_j,k} - 5 \right ) 
        \right ) \epsilon_{k}^{\frac{3}{2}} \\
        &\leq 
         - C_{\ell_j,k}
        \left ( 
            |I_{\ell_j,k} \cap \cJ^{-1} | + 
            \max\left ( 0, | I_{\ell_j,k} \cap \cJ^0 | - T_{\ell_j,k} - 5 \right ) 
        \right ) \epsilon_{k}^{\frac{3}{2}}
        .
        \label{eqn:proof/inexact-newton-inter-bound-loglog-removed}
    \end{align}
    Similarly, when $\theta = 0$ we can invoke \lemmaref{lem:accumulated-descent-lower-bound} to obtain the same result.
\end{proof}

\begin{proposition}[Restatement of \propositionref{prop:main/accumulated-descent}]
    \label{prop:proof/accumulated-descent}
    Under the regularizer choices of \theoremref{thm:newton-local-rate-boosted}, 
    for each $k \geq 0$, we have
    \begin{align}
        \varphi(x_{k})
        - \varphi(x_0)
        \leq
         - C_{0,k}
        \Big (\underbrace{
            |I_{0,k} 
            \cap \cJ^{-1}|
            + \max\left( |S_k \cap \cJ^0|, |I_{0,k}
             \cap \cJ^0| - V_k - 5J_k \right)
            }_{\Sigma_k} \Big )
         \epsilon_k^{\frac{3}{2}}
         ,
        \label{eqn:newton-global-final-inequality}
    \end{align}
    where $C_{0,k}$ is defined in \lemmaref{lem:proof/iteration-in-a-subsequence},
    and $V_k = \sum_{j=1}^{J_k-1} T_{\ell_j,\ell_{j+1}} + T_{\ell_{J_k},k}$,
    and $S_k = \bigcup_{j=1}^{J_k-1}\{\ell_{j+1}-1\}$,
    and $J_k = \max\{ j : \ell_j \leq k \}$.
\end{proposition}
\begin{proof}
    For each $j \geq 0$ such that $\ell_{j+1} - \ell_j \geq 2$, 
    using \eqref{eqn:inexact-mixed-newton-inner-bound} with $k = \ell_{j+1} - 1$ and \eqref{eqn:inexact-mixed-newton-boundary-bound}, and $\bone_{\{k\notin \cJ^1\}} = \bone_{\{k\in \cJ^{-1}\}} + \bone_{\{k\in \cJ^0\}}$, we find 
    \begin{align*}
        &\peq \varphi(x_{\ell_{j+1}})
        - \varphi(x_{\ell_{j}})  
        = 
        \left (\varphi(x_{\ell_{j+1}}) - \varphi(x_{\ell_{j+1}-1})  \right )
        + \left ( \varphi(x_{\ell_{j+1}-1}) - \varphi(x_{\ell_{j}+1}) \right ) 
        \\
        &\leq 
        - C_{\ell_j,\ell_{j+1}}
        \left ( 
            |I_{\ell_j,\ell_{j+1}} \cap \cJ^{-1} |
            +
            \max\left ( \bone_{\{\ell_{j+1}-1 \in \cJ^{0}\}}, | I_{\ell_j,\ell_{j+1}} \cap \cJ^0 | - T_{j} - 5 \right ) 
        \right )
        (\omega^{\supfallback}_{{\ell_{j+1}-1}})^3,
    \end{align*}
    where $T_j := T_{\ell_j,\ell_{j+1}}$ and $I_{i,j}, T_{i,j}, C_{i,j}$ are defined in \lemmaref{lem:proof/iteration-in-a-subsequence}.
    On the other hand,
    when $\ell_{j+1} - \ell_j = 1$, then the above inequality also holds since it reduces to \eqref{eqn:inexact-mixed-newton-boundary-bound}.

    Define $J_k = \max\left\{ j : \ell_j \leq k \right\}$,
     then $\ell_{J_k} \leq k < \ell_{J_{k}+1}$, and the following inequality holds by noticing that for each $i \in \N$, either $\omega^{\supfallback}_i = \sqrt{\epsilon_i}$ or $\omega^{\supfallback}_i = \sqrt{g_i} \geq \sqrt{\epsilon_i}$.
    \begin{align}
        \nonumber
        &\peq \varphi(x_k) - \varphi(x_0) 
        = 
        \varphi(x_k) - \varphi(x_{\ell_{J_k}})
        + \sum_{j = 1}^{J_k-1} 
        \left ( \varphi(x_{\ell_{j+1}}) - \varphi(x_{\ell_{j}}) \right ) \\
        \nonumber
        &\leq
        - C_{\ell_{J_k},k}
        \left ( 
            |I_{\ell_{J_k},k} \cap \cJ^{-1} |
            + 
            \max\left ( 0, | I_{\ell_{J_k},k} \cap \cJ^0 | - T_{\ell_{J_k},k} - 5 \right ) 
        \right ) \epsilon_{k}^{\frac{3}{2}} \\
        \nonumber
        &\peq 
        -
        \sum_{j=1}^{J_k-1}
        C_{\ell_j,\ell_{j+1}}
        \left ( 
            |I_{\ell_j,\ell_{j+1}} \cap \cJ^{-1} |
            +
            \max\left ( \bone_{\{\ell_{j+1}-1 \in \cJ^{0}\}}, | I_{\ell_j,\ell_{j+1}} \cap \cJ^0 | - T_{j} - 5 \right ) 
        \right )
        \epsilon_{{\ell_{j+1}-1}}^{\frac{3}{2}}
        \\
        &\leq -C_{0,k} \epsilon_k^{\frac{3}{2}}
        \left (
            |I_{0,k} \cap \cJ^{-1}|
            + \max\left( 
               |S_k \cap \cJ^0|,
                |I_{0,k} \cap \cJ^0| - V_k - 5J_k
                \right)
        \right ),%
    \end{align}
    where $V_k = \sum_{j=1}^{J_k-1} T_j + T_{\ell_{J_k},k}$, $S_k = \bigcup_{j=1}^{J_k-1}\{\ell_{j+1}-1\}$ 
    and the last inequality follows from $\max(a, b) + \max(c, d) \geq \max(a + c, b + d)$.
\end{proof}

\begin{proposition}[Restatement of \propositionref{prop:main/initial-phase-decreasing-Mk}]
    \label{prop:proof/initial-phase-decreasing-Mk}
    Let $k_{\mathrm{init}} = \min\{ j : M_j \leq \tilde C_4 L_H \}$ if $M_0 > \tilde C_4 L_H$ and $k_{\mathrm{init}} = 0$ otherwise, then 
    for the first choice of regularizers in \theoremref{thm:newton-local-rate-boosted}, we have
    \begin{equation}
        k_{\mathrm{init}}  
        \leq 
        \left[ \log_\gamma \frac{\gamma M_0}{\tilde C_4 L_H} \right]_+
        \left ( \tilde C_3 \log \frac{U_\varphi}{\epsilon_{k_{\mathrm{init}}}} + 3 \right )
        + 2
        ,
    \end{equation}
    where $\tilde C_3^{-1} = \frac{1}{2\max(2, 3\theta)} \log\frac{1}{\tau_-} > 0$ and $\tilde C_4$ is defined in \lemmaref{lem:lipschitz-constant-estimation},
    and $[x]_+$ denotes $\max(0, x)$.
    For the second choice of regularizers, we have
    \begin{equation}
        k_{\mathrm{init}} 
        \leq 
        \left[ \log_\gamma \frac{M_0}{\tilde C_4 L_H} \right]_+
        + \tilde C_3 \log \frac{U_\varphi}{\epsilon_{k_{\mathrm{init}}}}
        + 2
        .
    \end{equation}
\end{proposition}
\begin{proof}
    Using \lemmaref{lem:lipschitz-constant-estimation}
    and observing that the constants therein satisfy $\tilde C_4 \geq \tilde C_5$,
    then we know 
    $M_k$ is non-increasing for $k < k_{\mathrm{init}}$.
    Hence, $\tilde C_4 L_H < M_k = M_0 \gamma^{-|I_{0,k} \cap \cJ^{-1}|}$, and equivalently,
    \begin{equation}
        \label{eqn:proof/lipschitz-decay-count}
       \log_\gamma (\tilde C_4L_H) < \log_\gamma M_k = \log_\gamma M_0 - |I_{0,k} \cap \cJ^{-1}|.
    \end{equation}

    By definition of $\delta_k$ in \theoremref{thm:newton-local-rate-boosted}, we know $\delta_k^\theta = \omega_k^{\supsucc} (\omega_k^{\supfallback})^{-1} \leq 1$.
    Let %
    $\cI_{i,j} = \{ l \in I_{i,j} : \delta_l^\alpha < \tau_-\}$,
    and $\cI_{i,j}^+ = \{ l \in I_{i,j} : \delta_{l+1}^\alpha < \tau_- \}$. 
    From \lemmaref{lem:lipschitz-constant-estimation}, 
    when $M_k > \tilde C_4 L_H$ and $\tau_- \leq \min\big ( \delta_k^\alpha, \delta_{k+1}^\alpha \big )$, 
    we have $k \in \cJ^{-1}$.
    Equivalently, we have $(I_{i,j} \setminus \cI_{i,j}) \cap (I_{i,j} \setminus \cI_{i,j}^+) \subseteq I_{i,j} \cap \cJ^{-1}$ for $i < j < k_{\mathrm{init}}$.
    Then, 
    \begin{align}
        \nonumber
        |I_{i,j} \cap \cJ^{-1} |
        &\geq 
        |(I_{i,j} \setminus \cI_{i,j}) \cap (I_{i,j} \setminus \cI_{i,j}^+)|
        = 
        |I_{i,j} \setminus 
        ( \cI_{i,j}
        \cup  \cI^+_{i,j})| \\
        &\geq 
        |I_{i,j}| -  ( |\cI_{i,j}| +  |\cI^+_{i,j}|) 
        \geq 
        |I_{i,j}| -  2|\cI^+_{i-1,j}|
        ,
        \label{eqn:proof/initial-phase-newton-gradient-decay-raw}
    \end{align}
    where the last inequality follows from $\cI_{i,j} = \cI_{i-1,j-1}^+ \subseteq \cI_{i-1,j}^+$.
    Reformulating \eqref{eqn:proof/initial-phase-newton-gradient-decay-raw} obtains
    \begin{equation}
        |\cI^+_{i,j+1}|\geq \frac{1}{2}\left( |I_{i+1,j+1}| - |I_{i+1,j+1} \cap \cJ^{-1}| \right),  \forall\, 0\leq i < j < k_{\mathrm{init}} - 1.
        \label{eqn:proof/initial-phase-newton-gradient-decay}
    \end{equation}

    \paragraph{Case 1}
    We consider the first choice of regularizers, i.e., $\delta_k = \min(1, g_kg_{k-1}^{-1})$.
    Following the partition \eqref{eqn:proof/newton-partition},
    for any $\ell_j \leq l < \ell_{j+1}-1$ and $l < k_{\mathrm{init}} - 1$, 
    we know $g_{l+1} \leq g_l$ and $\delta_{l+1} = g_{l+1}g_l^{-1}$.
    Therefore,
    since $\log \delta_{l+1} \leq 0$ and $\log \tau_- < 0$, it holds that 
    \begin{align}
        \nonumber
        \log \frac{g_{l+1}}{g_{\ell_j}}
        &= \sum_{\ell_j \leq i \leq l} \log \delta_{i+1}
        \leq \sum_{i\in \cI^+_{\ell_j,l+1}} \log \delta_{i+1} \\
        &<
        \frac{\log\tau_-}{\alpha} | \cI^+_{\ell_j,l+1} |
        \overset{\eqref{eqn:proof/initial-phase-newton-gradient-decay}}{\leq}
        -A ( |I_{\ell_j+1,l+1}| - |I_{\ell_j+1,l+1} \cap \cJ^{-1}| ),
        \label{eqn:proof/initial-phase-newton-gradient-decay-inner-subsequence}
    \end{align}
    where $A = \frac{1}{2\alpha} \log \frac{1}{\tau_-} > 0$.
    Let $k < k_{\mathrm{init}} - 1$ and 
    $\hat J_k = \max\left\{ j : \ell_j \leq k + 1 \right\}$,
    then
    \begin{align}
        \nonumber
        \hat J_k \log \frac{\epsilon_{k+1}}{U_\varphi}
        &\leq \sum_{j=1}^{\hat J_k-1} \log \frac{g_{\ell_{j+1}-1}}{g_{\ell_j}}
        + \log\frac{g_{k+1}}{g_{\ell_{\hat J_k}}}  \\
        \nonumber
        \overset{\eqref{eqn:proof/initial-phase-newton-gradient-decay-inner-subsequence}}&{\leq}
        -A \sum_{j=1}^{\hat J_k-1} (
            |I_{\ell_j+1,\ell_{j+1}-1}|
            - |I_{\ell_j+1,\ell_{j+1}-1}\cap\cJ^{-1}|
            )  
            \\
        \nonumber
            &\peq 
        -A  (
            |I_{\ell_{\hat J_k}+1,k+1}|
           - |I_{\ell_{\hat J_k}+1,k+1}\cap \cJ^{-1}|
           )
        \\
        &\leq -A ( |I_{1,k+1}| - 2\hat J_k - |I_{1,k+1} \cap \cJ^{-1}|),
        \label{eqn:proof/initial-phase-newton-gradient-decay-summarized}
    \end{align}
    where the last inequality follows from 
    $|I_{\ell_j+1,\ell_{j+1}-1}| = |I_{\ell_j+1,\ell_{j+1}+1}| - 2$ and 
    $I_{\ell_j+1,\ell_{j+1}-1} \cap \cJ^{-1} \subseteq I_{\ell_j+1,\ell_{j+1}+1} \cap \cJ^{-1}$.

    For $1 \leq j \leq \hat J_k$, we have $\ell_j - 1 \leq k < k_{\mathrm{init}} - 1$, then \lemmaref{lem:proof/transition-between-subsequences-give-valid-regularizer}
    gives $\ell_j - 1 \in \cJ^{-1}$,
    Therefore, $|I_{0,k+1}\cap\cJ^{-1}|\geq \hat J_k$ and \eqref{eqn:proof/lipschitz-decay-count} yields 
    $\log_\gamma (\tilde C_4 L_H) < \log_\gamma M_0 - \hat J_k$.
    That is, $\hat J_k \leq \log_\gamma \frac{\gamma M_0}{\tilde C_4L_H}$.
    From \eqref{eqn:proof/lipschitz-decay-count}, we know
    \begin{align*}
         k = |I_{1,k+1}| 
         \overset{\eqref{eqn:proof/initial-phase-newton-gradient-decay-summarized}}&{\leq}
        J_k \left ( A^{-1}\log \frac{U_\varphi}{\epsilon_{k+1}} + 2 \right )
        + |I_{1,k+1} \cap \cJ^{-1}|  \\
        \overset{\eqref{eqn:proof/lipschitz-decay-count}}&{\leq}
        J_k \left ( A^{-1}\log \frac{U_\varphi}{\epsilon_{k+1}} + 2 \right )
        +  \log_\gamma \frac{M_0}{\tilde C_4L_H} 
        \leq 
        \log_\gamma \frac{\gamma M_0}{\tilde C_4 L_H} \left ( A^{-1}\log \frac{U_\varphi}{\epsilon_{k+1}} + 3 \right ).
    \end{align*}

    \paragraph{Case 2}
    When $\delta_k = \epsilon_k \epsilon_{k-1}^{-1}$ for each $k \in \N$. 
    For any $k < k_{\mathrm{init}}-1$, 
    we know a similar version of \eqref{eqn:proof/initial-phase-newton-gradient-decay-inner-subsequence} holds since $\log \delta_{i+1} \leq 0$:
    \begin{align*}
        \log\frac{\epsilon_{k+1}}{\epsilon_0}
        &= \sum_{i \in I_{0,k+1}} \log \delta_{i+1}
        \leq \sum_{i \in \cI^+_{0,k+1}} \log \delta_{i+1} \\
        &< -2A |\cI_{0,k+1}^+|
        \overset{\eqref{eqn:proof/initial-phase-newton-gradient-decay}}{\leq}
        -A ( |I_{1,k+1}| - | I_{1,k+1} \cap \cJ^{-1} |).
    \end{align*}
    Therefore, we have
    \begin{align*}
         k  = |I_{1,k+1}|
         &\leq 
         A^{-1} \log \frac{\epsilon_0}{\epsilon_{k+1}}
        + |I_{1,k+1} \cap \cJ^{-1}|  
        \overset{\eqref{eqn:proof/lipschitz-decay-count}}{\leq}
         A^{-1} \log \frac{\epsilon_0}{\epsilon_{k+1}}
         +
         \log_\gamma \frac{\gamma M_0}{\tilde C_4 L_H}.
    \end{align*}

    Finally, the proof is completed by setting $k = k_{\mathrm{init}} - 2$, and noticing that the conclusion automatically holds when $M_0 \leq \tilde C_4 L_H$.
\end{proof}



\subsection{Proof of the global rates in Theorem~\ref{thm:newton-local-rate-boosted}} \label{sec:appendix/global-rate-proof}

The following theorem provides a precise version of the global rates in \theoremref{thm:newton-local-rate-boosted}.  
It can be translated into \theoremref{thm:newton-local-rate-boosted} by using the identity $[\log L_H]_+ + [\log L_H^{-1}]_+ = |\log L_H|$.  

Since the right-hand sides of the following bounds are non-decreasing as $\epsilon_k$ decreases,  
whenever an $\epsilon$-stationary point is encountered such that $\epsilon_k \leq g_k \leq \epsilon$,  
the two inequalities below hold with $\epsilon_k$ replaced by $\epsilon$.  
Hence, the iteration bounds in \theoremref{thm:newton-local-rate-boosted} are valid.


\begin{theorem}[Precise statement of the global rates in \theoremref{thm:newton-local-rate-boosted}]
    \label{thm:appendix/global-newton-complexity}
    Let $\{ x_k \}_{k \ge 1}$ be generated by \Cref{alg:adap-newton-cg} with $\theta \geq 0$. 
    Under Assumption~\ref{assumption:liphess} and let $C = \max(\tilde C_4, \gamma \tilde C_5)^{\frac{1}{2}} \tilde C_1^{-1}$ 
    with the constants $\tilde C_1, \tilde C_4, \tilde C_5$ defined in \lemmaref{lem:lipschitz-constant-estimation},
    and let $\tilde C_3, k_{\mathrm{init}}$ be defined in \propositionref{prop:proof/initial-phase-decreasing-Mk},
    we have
    \begin{enumerate}
        \item
        If $\omega_k^{\supfallback} = \sqrt{g_k}$, and $\omega_k^{\supsucc} = \omega_k^{\supfallback} \min ( 1, g_k^\theta g_{k-1}^{-\theta} )$, 
        then 
        \begin{align*}
        k
        &\leq 
        \left[ \log_\gamma \frac{\gamma M_0}{\tilde C_4 L_H} \right]_+
        \left ( \tilde C_3 \log \frac{U_\varphi}{\epsilon_{k}} + 3 \right ) \\
        &\peq + 
        5\left ( C\Delta_\varphi L_H^{\frac{1}{2}} 
        \epsilon_k^{-\frac{3}{2}}
        + \left[ \log_\gamma \frac{\tilde C_5 L_H}{M_0} \right]_+ + 2 \right )
        \left ( \log\log \frac{U_\varphi}{\epsilon_k} + 7 \right )
        + 2
        .
        \end{align*}
        \item
        If $\omega_k^{\supfallback} = \sqrt{\epsilon_k}$, 
        and $\omega_k^{\supsucc} = \omega_k^{\supfallback} \epsilon_k^\theta \epsilon_{k-1}^{-\theta}$, 
        then 
        \begin{align*}
        k
        &\leq 
        40 \left ( C\Delta_\varphi L_H^{\frac{1}{2}} 
        \epsilon_k^{-\frac{3}{2}}
        + \left[ \log_\gamma \frac{\tilde C_5 L_H}{M_0} \right]_+ + 2 \right ) \\
        &\peq  \peq
        + \left[ \log_\gamma \frac{M_0}{\tilde C_4L_H} \right]_+
        + (24 + \tilde C_3)\log \frac{U_\varphi}{\epsilon_k}
        + 2
        .
        \end{align*}
    \end{enumerate}
    Moreover, there exists a subsequence $\{ x_{k_j} \}_{j \geq 0}$ such that $\lim_{j \to \infty} x_{k_j} = x^*$ with $\nabla \varphi(x^*) = 0$.
\end{theorem}
\begin{proof}
    Let $k_{\mathrm{init}}$ be defined in \propositionref{prop:proof/initial-phase-decreasing-Mk}, 
    without loss of generality,
    we can drop the iterations $\{ x_j \}_{j \leq k_{\mathrm{init}}}$
    and assume $M_0 \leq \tilde C_4 L_H$, where $\tilde C_4$ is defined in \lemmaref{lem:lipschitz-constant-estimation}.
    By \lemmaref{lem:lipschitz-constant-estimation}, 
    we know $k \in \cJ^1$ implies $M_k \leq \tilde C_5 L_H$, and hence 
    $\sup_{j \geq 0} M_j \leq \max(\tilde C_4, \gamma \tilde C_5) L_H$.

    By applying \propositionref{prop:proof/accumulated-descent}, we have
    \begin{align*}
        -\Delta_\varphi \leq \varphi(x_{k}) - 
        \varphi(x_{0})
        \overset{\eqref{eqn:newton-global-final-inequality}}&{\leq}
        -C_{0,k} \Sigma_k \epsilon_k^{\frac{3}{2}}
        \leq 
        -\tilde C_1 (\max(\tilde C_4, \gamma \tilde C_5) L_H)^{-\frac{1}{2}} \Sigma_k \epsilon_k^{\frac{3}{2}},
    \end{align*}
    which implies that $\Sigma_k \leq C L_H^{\frac{1}{2}} \Delta_\varphi \epsilon_k^{-\frac{3}{2}}$ with $C = \max(\tilde C_4, \gamma \tilde C_5)^{\frac{1}{2}} \tilde C_1^{-1}$, 
    and the theorem can be proved by find a lower bound over $\Sigma_k$.
    \paragraph{Case 1}
    For the first choice of regularizers, \lemmaref{lem:main/lower-bound-of-Vk} shows that $V_k \leq J_k \log\log\frac{U_\varphi}{\epsilon_k}$, and hence,
    \begin{align*}
       \Sigma_k &\geq 
       |I_{0,k}\cap \cJ^{-1}|
       + 
       \max \left ( 
        |S_k \cap \cJ^{-1}|,  
        |I_{0,k}\cap \cJ^0| - 
        J_k \left( \log\log \frac{U_\varphi}{\epsilon_k} + 5 \right)
        \right ) \\ 
    \overset{\eqref{eqn:basic-counting-lemma-loss-descent}}&{\geq}
      \frac{k}{5\left( \log\log \frac{U_\varphi}{\epsilon_k} + 7 \right)}
      - \left[ \log_\gamma \frac{\tilde C_5 L_H}{M_0} \right]_+ - 2,
    \end{align*}
    where \lemmaref{lem:basic-counting-lemma} is invoked with $W_k = 0$ and $U_k = \log\log\frac{U_\varphi}{\epsilon_k} + 5$.
    Reorganizing the above inequality and incorporating the initial phase in \propositionref{prop:main/initial-phase-decreasing-Mk} yields
        \begin{align*}
        k
        &\leq 
        k_{\mathrm{init}}
        + 
        5\left ( C\Delta_\varphi L_H^{\frac{1}{2}} 
        \epsilon_k^{-\frac{3}{2}}
        + \left[ \log_\gamma \frac{\tilde C_5 L_H}{M_0} \right]_+ + 2 \right )
        \left ( \log\log \frac{U_\varphi}{\epsilon_k} + 7 \right )
        .
        \end{align*}
    \paragraph{Case 2}
    For the second choice of regularizers,
    \lemmaref{lem:main/lower-bound-of-Vk} shows that $V_k \leq \log\frac{U_\varphi}{\epsilon_k} + J_k$, and 
    \begin{align*}
       \Sigma_k &\geq 
       |I_{0,k}\cap \cJ^{-1}|
       + 
       \max \left ( 
        |S_k \cap \cJ^{-1}|,  
        |I_{0,k}
        \cap \cJ^0| 
        - \log\frac{U_\varphi}{\epsilon_k}
        - 6J_k
        \right ).
    \end{align*}
    Using \lemmaref{lem:basic-counting-lemma} with $U_k = 6$ and $W_k = \log \frac{U_\varphi}{\epsilon_k}$, we know either $\log \frac{U_\varphi}{\epsilon_k} \geq k / 24$, or 
    \begin{align*}
        \Sigma_k & \geq 
        \frac{k}{40}
      - \left[ \log_\gamma \frac{\tilde C_5 L_H}{M_0} \right]_+ - 2.
    \end{align*}
    By incorporating the case $k \leq 24 \log\frac{U_\varphi}{\epsilon_k}$ and the initial phase in \propositionref{prop:main/initial-phase-decreasing-Mk}, the proof is completed.

    \paragraph{The subsequence convergence}
    From the global complexity we know $\lim_{k \to \infty}\epsilon_k = 0$.
    Since $\epsilon_k = \min(\epsilon_{k-1}, g_k)$, we can construct a subsequence $\{ x_{k_j} \}_{j \geq 0}$ such that $g_{k_j} = \epsilon_{k_j}$.
    Note $\varphi(x_{k_j}) \leq \varphi(x_0)$ and the compactness of the sublevel set $L_\varphi(x_0)$ in Assumption~\ref{assumption:liphess}, we know there is a further subsequence of $\{ x_{k_j} \}$ converging to some point $x^*$.
    Since $\nabla\varphi$ is a continuous map, we know $\nabla\varphi(x^*) = 0$.
\end{proof}



\subsection{Proof of Theorem~\ref{thm:newton-local-rate-boosted-oracle-complexity}}\label{sec:appendix/oracle-complexity-proof}

\begin{proof}
The two gradient evaluations come from $\nabla \varphi(x_k)$ and $\nabla \varphi(x_k + d_k)$.
The number of function value evaluations in a linesearch criterion
is upper bounded by $m_{\mathrm{max}} + 1$, 
In the \texttt{SOL} case, at most two criteria are tested, in the \texttt{NC} case one criterion is tested.
Thus, the total number of function evaluations is bounded by $2m_{\mathrm{max}} + 2$.
The number of Hessian-vector product evaluations can be bounded using \lemmaref{lem:capped-cg}.
\end{proof}

\section{Technical lemmas for global rates} \label{sec:appendix/global-rate-technical-lemmas}

\subsection{Descent lemmas and their proofs} \label{sec:proof-descent-lemma}

In this section we provide the descent lemmas for the \texttt{NC} case (\lemmaref{lem:newton-cg-nc}) and the \texttt{SOL} case (\lemmaref{lem:newton-cg-sol}).
The lemma for the \texttt{NC} case is the same as \citet[Lemma 6.3]{he2023newton}, and we include the proof for completeness.
However, %
the linesearch rules for the \texttt{SOL} case are different, so we need a complete proof.

\begin{lemma}
    \label{lem:newton-cg-nc}%
    Suppose $\text{d\_type}, d, \tilde d, m$ be the those in 
    the subroutine \texttt{NewtonStep} of \Cref{alg:adap-newton-cg}, and $x, \omega, M$ be its inputs.
    Suppose $\text{d\_type} = \texttt{NC}$ and let 
    $m_*$ be the smallest integer such that \eqref{eqn:smooth-line-search-nc} holds.
    If $0 < m_* \leq m_{\mathrm{max}}$, we have
    \begin{align}
        \label{eqn:newton-cg-nc-stepsize}
        \beta^{m_* - 1} &> \frac{3M(1 - 2\mu)}{L_H}, \\
        \label{eqn:newton-cg-nc-decay}
        \varphi(x + \beta^{m_*}d) - \varphi(x) & <
        -\frac{9\beta^2(1 - 2\mu)^2\mu}{L_H^2} M^{\frac{3}{2}} \omega^3
         .
    \end{align}
    When $m_* = 0$, the linesearch rule gives
    \begin{align}
        \label{eqn:newton-cg-nc-decay-ls0}
        \varphi(x + d) - \varphi(x) &\leq 
        -\mu M^{-\frac{1}{2}} \omega^3
         .
    \end{align}
    Finally, when $m_* > m_{\mathrm{max}}$, we have $M \leq (3 - 6\mu)^{-1} L_H$.
\end{lemma}
\begin{proof}
    Let $H = \nabla^2\varphi(x)$, 
    from \eqref{eqn:smooth-line-search-nc-direction} we can verify that
    $\|d \| = L(\bar d) = M^{-1} \|d\|^{-2} |d^\top H d|$, where $\bar d = \|\tilde d\|^{-1} \tilde d$ and $\tilde d$ is the direction satisfying \lemmaref{lem:capped-cg}. 
    Then, $d^\top Hd=-M\|d\|^3$ and $d^\top\nabla\varphi(x)\leq 0$.
    When $m_* \geq 1$, let $0 \leq j \leq m_* - 1$, then \eqref{eqn:smooth-line-search-nc} fails to hold with $m = j$, and 
    \begin{align}
        \nonumber
        -\mu\beta^{2j} M \| d \|^3
        < 
        \varphi(x + \beta^j d) - \varphi(x)
        \overset{\eqref{eqn:hessian-lip-value-inequ}}&{\leq} 
        \beta^j \nabla \varphi(x)^\top d + \frac{\beta^{2j}}{2} d^\top H d + \frac{L_H}{6} \beta^{3j} \| d \|^3 \\
        &\leq \frac{\beta^{2j}}{2} d^\top H d + \frac{L_H}{6} \beta^{3j} \| d \|^3 \\
        &= - \frac{\beta^{2j}}{2} M \|d\|^3 + \frac{L_H}{6} \beta^{3j} \| d \|^3.
        \label{eqn:proof/nc-descent-lemma-ls-nonzero}
    \end{align}
    Dividing both sides by $\beta^{2j} \| d \|^3$ we have
    \begin{align}
        \label{eqn:proof/linesearch-nc-failure}
        -M\mu
        < 
        - \frac{M}{2} + \frac{L_H}{6} \beta^{j}.
    \end{align}
    Therefore, rearranging the above inequality gives \eqref{eqn:newton-cg-nc-stepsize}.

    From \eqref{eqn:capped-cg-nc-direction-inequ} and \eqref{eqn:smooth-line-search-nc-direction},
    we know $\tilde d^\top H \tilde d \leq -\sqrt M \omega \| \tilde d \|^2$ and hence $\| d \| = M^{-1} \frac{|\tilde d^\top H \tilde d|}{\|\tilde d\|^2} \geq M^{-\frac{1}{2}} \omega$.
    By the linesearch rule \eqref{eqn:smooth-line-search-nc}, we have
    \begin{align*}
        \varphi(x + \beta^{m_*}d) - \varphi(x) 
        \leq - \mu\beta^{2m_*} M\| d \|^3
        \leq
        - \mu \beta^{2m_*} M^{-\frac{1}{2}} \omega^3
        \overset{\eqref{eqn:newton-cg-nc-stepsize}}{<} 
        -  \frac{9\beta^2(1 - 2\mu)^2\mu}{L_H^2} M^{\frac{3}{2}}\omega^3.
    \end{align*}
    When $m_* = 0$, \eqref{eqn:newton-cg-nc-decay-ls0} can be also proven using the above argument.

    Finally, when $m_* > m_{\mathrm{max}} \geq 0$, 
    we know \eqref{eqn:smooth-line-search-nc} fails to holds with $m = 0$, and then \eqref{eqn:proof/linesearch-nc-failure} holds with $j = 0$.
    Therefore, we have $M < (3 - 6\mu)^{-1}L_H$.

    \end{proof}

The following lemma summarizes the properties of \texttt{NewtonStep} for \texttt{SOL} case. 
Its first item is the necessary condition that the linesearch \eqref{eqn:smooth-line-search-sol} or \eqref{eqn:smooth-line-search-sol-smaller-stepsize} fails,
which will be used by subsequent items.
\begin{lemma}
    \label{lem:newton-cg-sol}%
    Suppose $\text{d\_type}, d, m, \hat m, \alpha$ be the those in the subroutine \texttt{NewtonStep} of \Cref{alg:adap-newton-cg}, and $x, \omega, M$ be its inputs.
    Suppose $\text{d\_type} = \texttt{SOL}$, and 
    let $m_* \geq 0$ be the smallest integer such that $\eqref{eqn:smooth-line-search-sol}$ holds, 
    and $\hat m_* \geq 0$ be the smallest integer such that $\eqref{eqn:smooth-line-search-sol-smaller-stepsize}$ holds,
    then we have
    \begin{enumerate}
        \item 
        Suppose $\mu \tau \beta^j d^\top \nabla \varphi(x) < \varphi(x + \tau \beta^j d) - \varphi(x)$ for some $\tau \in (0, 1]$ and $j \geq 0$, then 
    \begin{align}
        \label{eqn:newton-cg-sol-stepsize-when-linesearch-violated}
        \beta^{j} &
        > \sqrt{\frac{6(1 - \mu)M^{\frac{1}{2}}\omega}{L_H\tau^2\|d\|} }
        = \frac{\sqrt 2 C_M \omega^{\frac{1}{2}}}{\tau M^{\frac{1}{4}}\|d \|^{\frac{1}{2}}}
        ,
    \end{align}
        where $C_{M} := \sqrt{\frac{3(1 - \mu)M}{L_H} } \geq \sqrt{\frac{M}{L_H}}$.
        \item 
    If $m_{\mathrm{max}} \geq m_* > 0$, then $\alpha = \beta^{m_*}$ and 
    \begin{align}
        \label{eqn:newton-cg-sol-stepsize}
        \beta^{m_* - 1} &
        > \max\left ( \beta^{m_{\mathrm{max}} - 1}, C_{M} \| \nabla \varphi(x) \|^{-\frac{1}{2}}\omega \right )
        , \\
        \label{eqn:newton-cg-sol-decay}
        \varphi(x + \alpha d) - \varphi(x)
        & <
        - \frac{36\beta\mu(1 - \mu)^2}{L_H^2} M^{\frac{3}{2}} \omega^3
        .
    \end{align}
    \item If $m_* > m_{\mathrm{max}}$ but $m_{\mathrm{max}} \geq \hat m_* > 0$, then
        $\beta^{\hat m_* - 1}
        > \sqrt 2 C_{M}$.
    \item If $m_* > m_{\mathrm{max}}$ but $m_{\mathrm{max}} \geq \hat m_* \geq 0$, then $\alpha = \hat \alpha \beta^{\hat m_*}$ with $\hat \alpha = \min(1, \omega^{\frac{1}{2}}M^{-\frac{1}{4}}\|d\|^{-\frac{1}{2}})$, and 
    \begin{align}
        \label{eqn:newton-cg-sol-decay-smaller-stepsize}
        \varphi(x + \alpha d) - \varphi(x)
        &< %
    -\mu\beta^{\hat m_*} C_{M}^3 \min\left( C_{M}, 1  \right) M^{-\frac{1}{2}} \omega^3.
    \end{align}
    \item If both $m_* > m_{\mathrm{max}}$ and $\hat m_* > m_{\mathrm{max}}$, then $M \leq \frac{L_H}{2}$.
    \item If $m_* = 0$ (i.e., the stepsize $\alpha = 1$), then
    \begin{align}
        \label{eqn:newton-cg-sol-decay-ls0}
        \varphi(x + d) - \varphi(x)
        &\leq 
        -\frac{4\mu M^{-\frac{1}{2}}}{25 + 8L_HM^{-1}} \min \left( \| \nabla \varphi(x + d) \|^2 \omega^{-1}, \omega^3 \right).
    \end{align}
    \end{enumerate}
\end{lemma}
\begin{proof}
    Let $H = \nabla^2\varphi(x)$.
    We note that in the \texttt{SOL} setting, the direction $d$ is the same as $\tilde d$ returned by \texttt{CappedCG}, so \lemmaref{lem:capped-cg} holds for $d$.

    (1). By the assumption we have
    we have
\begin{align*}
    \mu \tau \beta^jd^\top \nabla \varphi(x)
    < 
    \varphi(x + \tau \beta^j d) - \varphi(x)
    \overset{\eqref{eqn:hessian-lip-value-inequ}}{\leq} \tau \beta^j d^\top \nabla \varphi(x)
    + \frac{\tau^2 \beta^{2j}}{2} d^\top H d + \frac{L_H}{6} \tau^3\beta^{3j} \| d \|^3,
\end{align*}
Rearranging the above inequality and dividing both sides by $\tau \beta^j$, we have
\begin{align}
    -(1 - \mu) d^\top \nabla \varphi(x)
    <
    \frac{\tau\beta^{j}}{2} d^\top H d + \frac{L_H}{6} \tau^2\beta^{2j} \| d \|^3.
    \label{eqn:newton-line-search-sol-rearrange}
\end{align}
From \lemmaref{lem:capped-cg}, we know that  
$d^\top \nabla \varphi(x) = - d^\top Hd - 2\sqrt M \omega \|d\|^2$, then since $\mu \in (0, 1/2)$, $j \geq 0$ and $\beta \in (0, 1)$, $\tau \in (0, 1]$, we have $1 - \mu > 1/2 \geq \beta^j / 2 \geq \tau \beta^j / 2$ and 
\begin{align*}
     \frac{L_H}{6}\tau^2 \beta^{2j} \| d \|^3 
     \overset{\eqref{eqn:newton-line-search-sol-rearrange}}&{>} 
    \left (1 - \mu - \frac{\tau\beta^j}{2} \right ) d^\top H d + 2\sqrt M\omega (1 - \mu) \|d\|^2 \\
    \overset{\eqref{eqn:capped-cg-hessian-lowerbound-H}}&{>} 
    -\sqrt M\omega \left (1 - \mu - \frac{\tau\beta^j}{2} \right ) \| d\|^2  + 2\sqrt M\omega (1 - \mu) \|d\|^2 \\
    &=
     \sqrt M\omega \left (1 - \mu + \frac{\tau\beta^j}{2} \right ) \|d\|^2.
\end{align*}
Therefore, we have
\begin{equation}
    \beta^{2j} 
    > \frac{6\sqrt M\omega (1 - \mu + \tau\beta^j / 2)}{L_H \tau^2 \| d\|}
    \geq \frac{6\sqrt M\omega (1 - \mu)}{L_H  \tau^2\| d\|}
    ,
    \label{eqn:capped-cg-sol-d-geq-omega}
\end{equation}
which proves \eqref{eqn:newton-cg-sol-stepsize-when-linesearch-violated}.

(2). In particular, when $m_* > 0$, we know \eqref{eqn:smooth-line-search-sol} is violated for $m = 0$, then \eqref{eqn:newton-cg-sol-stepsize-when-linesearch-violated} with $\tau = 1$ and $j = 0$ gives a lower bound of $d$:
\begin{equation}
    \label{eqn:capped-cg-sol-d-lower-bound}
    \| d \| 
    > 
    \frac{6\sqrt M\omega(1 - \mu)}{L_H}
    \geq 
     C_{M}^2 M^{-\frac{1}{2}}\omega
    .
\end{equation}
Note that \eqref{eqn:smooth-line-search-sol} is also violated for $m_* - 1$, then \eqref{eqn:newton-cg-sol-stepsize-when-linesearch-violated} holds with $(j, \tau) = (m_* - 1, 1)$, and we have
\begin{equation}
    \beta^{m_*-1} 
    \overset{\eqref{eqn:newton-cg-sol-stepsize-when-linesearch-violated}}{\geq} 
    \sqrt{\frac{6\sqrt M\omega (1 - \mu)}{L_H  \| d\|}}
    \overset{\eqref{eqn:capped-cg-io-diff}}{\geq} 
    \sqrt{\frac{3(1 - \mu)}{L_H } \frac{M\omega^2}{\| \nabla \varphi(x) \|}}
    = C_M\|\nabla\varphi(x) \|^{-\frac{1}{2}} \omega
    ,
    \label{eqn:capped-cg-sol-d-geq-omega-sol-normal-ls-case}
\end{equation}
which yields \eqref{eqn:newton-cg-sol-stepsize}.
Moreover, the descent of the function value can be bounded as follows:
\begin{align}
    \nonumber
    \varphi(x + \beta^{m_*}d) - \varphi(x)
    \overset{\eqref{eqn:smooth-line-search-sol}}&{\leq}
    \mu \beta^{m_*} d^\top\nabla\varphi(x) \\
    \nonumber
    \overset{\eqref{eqn:capped-cg-descent-direction}}&{=}
    -\mu \beta^{m_*} d^\top (H + 2\sqrt M\omega \Id)d 
    \overset{\eqref{eqn:capped-cg-hessian-lowerbound}}{\leq}
    -\mu \sqrt M\omega \beta^{m_*} \| d \|^2 \\
    \nonumber
    \overset{\eqref{eqn:capped-cg-sol-d-geq-omega-sol-normal-ls-case}}&{<}
    -\mu \beta \sqrt M\omega \| d \|^2
            \sqrt{\frac{6\sqrt M\omega(1 - \mu)}{L_H \|d \|}}
    = -\mu \beta (\sqrt M\omega \| d \|)^{\frac{3}{2}}
            \sqrt{\frac{6(1 - \mu)}{L_H}} \\
            \label{eqn:proof-sol-loss-descent}
    \overset{\eqref{eqn:capped-cg-sol-d-lower-bound}}&{<}
    - \frac{36\beta\mu(1 - \mu)^2}{L_H^2} M^{\frac{3}{2}} \omega^3.
\end{align}

(3).
The linesearch rule \eqref{eqn:smooth-line-search-sol-smaller-stepsize} can be regarded as using the rule in \eqref{eqn:smooth-line-search-sol} with a new direction $\hat \alpha d$, where $\hat \alpha = \min(1, \omega^{\frac{1}{2}} M^{-\frac{1}{4}} \|d \|^{-\frac{1}{2}})$.
Since $\hat m_* > 0$, then \eqref{eqn:smooth-line-search-sol-smaller-stepsize} is violated for $0 \leq j < \hat m_*$, and \eqref{eqn:newton-cg-sol-stepsize-when-linesearch-violated} with $\tau = \hat \alpha$ gives
\begin{align}
    \label{eqn:proof/sol-linesearch-failure-smaller-stepsize}
    \beta^{2j} 
    > \frac{6\sqrt M \omega (1 - \mu)}{L_H  \hat \alpha^2\| d \|}
    \geq \frac{6M (1 - \mu)}{L_H} = 2C_{M}^2
    .
\end{align}
Thus, the result follows from setting $j = \hat m_* - 1$.

(4).
Since $m_* > m_{\mathrm{max}} \geq 0$, 
then the linesearch rule \eqref{eqn:smooth-line-search-sol} is violated for $m = 0$ 
such that \eqref{eqn:capped-cg-sol-d-lower-bound} holds.
Hence, following the first two lines of the proof of \eqref{eqn:proof-sol-loss-descent}, we have
\begin{align*}
    \varphi(x + \hat \alpha\beta^{\hat m_*} d) - \varphi(x)
    &\leq 
    -\mu\beta^{\hat m_*}  M^{\frac{1}{2}} \omega \hat \alpha\|d\|^2  \\
    &=
    -\mu\beta^{\hat m_*} M^{\frac{1}{2}} \omega \min\left( \|d\|^2, \omega^{\frac{1}{2}} M^{-\frac{1}{4}} \|d\|^{\frac{3}{2}} \right) \\
    \overset{\eqref{eqn:capped-cg-sol-d-lower-bound}}&{\leq} 
    -\mu\beta^{\hat m_*}  M^{\frac{1}{2}} \omega \min\left( C_{M}^4M^{-1}  \omega^2, C_{M}^3 M^{-1}\omega^2  \right) \\
    &=
    -\mu\beta^{\hat m_*} C_{M}^3 \min\left( C_{M}, 1  \right) M^{-\frac{1}{2}} \omega^3.
\end{align*}


(5). Since $\hat m_* > m_{\mathrm{max}} \geq 0$, 
then \eqref{eqn:proof/sol-linesearch-failure-smaller-stepsize} holds with $j = 0$, which implies that $1 > 2 C_M^2$, i.e., $2M \leq L_H$.

(6). When $m_* = 0$, 
by the linesearch rule and \lemmaref{lem:capped-cg} we have
\begin{align}
    \label{eqn:newton-cg-sol-proof-ls0}
    \varphi(x + d) - \varphi(x)
    \leq \mu d^\top\nabla \varphi(x)
    \leq -\mu \sqrt M \omega \|d\|^2.
\end{align}
It remains to give a lower bound of $\|d\|$ as in \eqref{eqn:capped-cg-sol-d-lower-bound}, which is similar to the proof of \citet[Lemma 6.2]{he2023newton} with their $\epsilon_H$ and $\zeta$ replaced with our $\sqrt M\omega$ and $\tilde \eta$. 
Since special care must be taken with respect to $M$, we present the proof below.
Note that 
\begin{align*}
    \| \nabla \varphi(x + d) \|
    &\leq 
    \| \nabla \varphi(x + d) - \nabla \varphi(x) - \nabla^2\varphi(x) d \| \\
    &\peq 
    + \| \nabla \varphi(x) + (\nabla^2\varphi(x) + 2\sqrt M\omega \Id) d \|
      + 2 \sqrt M\omega \| d \| \\
    \overset{\eqref{eqn:capped-cg-hessian-upperbound}}&{\leq}
    \frac{L_H}{2} \| d \|^2
    + \sqrt M \left ( \frac{1}{2} \omega \tilde \eta
    + 2\omega \right ) \| d \|.
\end{align*}
Then, by the property of quadratic functions, we know 
\begin{align*}
    \| d \| 
    &\geq \frac{-(\tilde \eta + 4)+ \sqrt{(\tilde \eta + 4)^2 + 8L_H (\sqrt M\omega)^{-2}\| \nabla \varphi(x + d) \|}}{2L_H} \sqrt M\omega \\
    &\geq c_0 \sqrt M\omega \min\left( \omega^{-2}\| \nabla \varphi(x + d)\|, 1  \right),
\end{align*}
where 
$c_0 := \frac{4M^{-1}}{4 + \tilde \eta + \sqrt{(4 + \tilde \eta)^2 + 8M^{-1}L_H}} 
\geq \frac{2M^{-1}}{\sqrt{(4 + \tilde \eta)^2 + 8M^{-1}L_H}} 
\geq \frac{2M^{-1}}{\sqrt{25 + 8M^{-1}L_H}}$, 
and we have used the inequality $-a+\sqrt{a^2+bs}\geq(-a+\sqrt{a^2+b})\min(s,1)$ from \citet[Lemma 17]{royer2018complexity},
with $a=\tilde\eta+4 \leq 5$, $b=8L_H{M^{-1}}$ and $s = \omega^{-2}\|\nabla\varphi(x+d)\|$.
Combining with \eqref{eqn:newton-cg-sol-proof-ls0}, we get \eqref{eqn:newton-cg-sol-decay-ls0}. 
\end{proof}

\subsection{Proof of Lemma~\ref{lem:lipschitz-constant-estimation}} \label{sec:appendix/summarized-descent}

In this section, we provide the proof of \lemmaref{lem:lipschitz-constant-estimation}. 
It is highly technical but mostly based on the descent lemmas (\lemmaref{lem:newton-cg-nc,lem:newton-cg-sol}) and the choices of regularizers in \theoremref{thm:newton-local-rate-boosted}.

First, we give an auxiliary lemma for the claim about $k \in \cJ^{-1}$ in \lemmaref{lem:lipschitz-constant-estimation}. 
\begin{lemma}
    \label{lem:appendix/decreasing-Mk-condition}
    Suppose the following two properties are true:
    \begin{enumerate}
        \item Suppose $\text{d\_type}_k \neq \texttt{SOL}$ or $m_k > 0$. %
        If $M_k > \tilde C_4 L_H$ and $\omega_k \geq \tau_-\omega_k^{\supfallback}$, then $k \in \cJ^{-1}$;
        \item Suppose $\text{d\_type}_k = \texttt{SOL}$ and $m_k = 0$. 
        If $M_k > L_H$ and 
        $\min\big ( \omega^3_k, g_{k+1}^2 \omega_k^{-1} \big )
        \geq \tau_-(\omega_k^{\supfallback})^3$, then $k \in \cJ^{-1}$,
    \end{enumerate}
    where $\delta_k^\theta = \omega_k^{\supsucc} (\omega_k^{\supfallback})^{-1}$ is defined in \theoremref{thm:newton-local-rate-boosted}.
    Then, if $M_k > \tilde C_4 L_H$ and $\tau_- \leq \min\big ( \delta_k^\alpha, \delta_{k+1}^\alpha \big )$, 
    we know $k \in \cJ^{-1}$.
\end{lemma}
\begin{proof}
Let $\alpha = \max(2, 3\theta)$.
    We consider the following two cases: %
    \begin{enumerate}
        \item Note that $\tau_- < 1$. If $\omega_k < \tau_- \omega_k^{\supfallback}$, then we know the trial step is accepted since $\omega_k \neq \omega_k^{\supfallback}$, and hence, $\omega_k = \omega_k^{\supsucc}$ and $\tau_- > \delta_k^\theta \geq \delta_k^\alpha$ since $\delta_k \in (0, 1]$ and $\theta \leq \alpha$.
        \item  If $\min\big ( g_{k+1}^2 \omega_k^{-1}, \omega_k^3 \big ) < \tau_- (\omega_k^{\supfallback})^3$,
        we use the choice $\omega_k^{\supfallback} = \sqrt{g_k}$ as an example, the case for $\omega_k^{\supfallback} \sqrt{\epsilon_k}$ is similar and follows from $g_{k+1} \geq \epsilon_{k+1}$.
        In this case, we have $\delta_k = \min(1, g_kg_{k-1}^{-1})$.
        When the fallback step is taken, we have $\omega_k = \omega_k^{\supfallback}$, and
        \begin{align*}
            \tau_- > g_k^{-\frac{3}{2}}\min\big( g_{k+1}^2 g_k^{-\frac{1}{2}}, g_k^{\frac{3}{2}} \big) =  \delta^2_k.
        \end{align*}
        Since $\delta_k \in (0, 1]$ and $2 \leq \alpha$, we have $\tau_- > \delta_k^\alpha$.
    On the other hand, when the trial step is taken, we have $\omega_k = \omega_k^{\supsucc} = \sqrt{g_k} \delta_k^\theta$ and 
    \begin{align*}
    \tau_- 
    &> g_k^{-\frac{3}{2}}\min\big( g_{k+1}^2 g_k^{-\frac{1}{2}} \delta_k^{-\theta}, g_k^{\frac{3}{2}} \delta_k^{3\theta} \big)  
    \overset{(\delta_k \leq 1)}{\geq} 
    g_k^{-\frac{3}{2}}\min\big( g_{k+1}^2 g_k^{-\frac{1}{2}}, g_k^{\frac{3}{2}} \delta_k^{3\theta} \big)  \\
    &= \min\big( g_{k+1}^2 g_k^{-2}, \delta_k^{3\theta} \big)  
    \geq \min\big ( \delta_{k+1}^2, \delta_k^{3\theta}  \big ) 
    \geq \min\big ( \delta_{k+1}^\alpha, \delta_k^\alpha \big ).
    \end{align*}
    \end{enumerate}
    Conversely, we find when $\tau_- \leq \min\big ( \delta_k^\alpha, \delta_{k+1}^\alpha \big )$, 
    the assumptions of this lemma give that $k \in \cJ^{-1}$.
\end{proof}

We will also show that the two properties listed in \lemmaref{lem:appendix/decreasing-Mk-condition} hold in the proof of \lemmaref{lem:lipschitz-constant-estimation} below, and leave this fact as a corollary for our subsequent usage.
\begin{corollary}
    \label{cor:appendix/decreasing-Mk-condition}
    Under the regularizers in \theoremref{thm:newton-local-rate-boosted}, 
    the two properties in \lemmaref{lem:appendix/decreasing-Mk-condition} hold.
\end{corollary}

\begin{proof}[Proof of \lemmaref{lem:lipschitz-constant-estimation}]
    Define $\Delta_k = \varphi(x_k) - \varphi(x_{k+1})$.
    We denote $\omega_k = \omega_k^{\supsucc}$ if the trial step is taken, and $\omega_k = \omega_k^{\supfallback}$ otherwise.

    \paragraph{Case 1}
    When $\text{d\_type}_k = \texttt{SOL}$ and $m_k = 0$, i.e., $x_{k+1} = x_k + d_k$, 
    we define $E_k := \min \left(  g_{k+1}^2\omega_k^{-1}, \omega_k^3 \right)$.
    \begin{enumerate}
        \item When $k \in \cJ^1$, i.e., $M_{k + 1} = \gamma M_k$, we have
        \begin{align*}
            \frac{4\mu}{33} \tau_+ M_k^{-\frac{1}{2}} E_k
            \geq \Delta_k
            \overset{\eqref{eqn:newton-cg-sol-decay-ls0}}{\geq}
        \frac{4\mu M_k^{-\frac{1}{2}}}{25 + 8L_HM_k^{-1}} E_k,
        \end{align*}
        where the first inequality follows from the condition for increasing $M_k$ in \Cref{alg:adap-newton-cg}.
        The above display implies $25 + 8L_HM_k^{-1} \geq 33\tau_+^{-1} \geq 33$ as $\tau_+ \leq 1$, and hence, $M_k \leq L_H$.
        \item When $E_k \geq \tau_-(\omega_k^{\supfallback})^3$ 
        and $M_k > L_H$, 
        we have $k\in \cJ^{-1}$ since
        \begin{align*}
            \Delta_k
            \overset{\eqref{eqn:newton-cg-sol-decay-ls0}}{\geq}
            \frac{4\mu M_k^{-\frac{1}{2}} E_k}{25 + 8L_HM_k^{-1}}
            > \frac{4\mu M_k^{-\frac{1}{2}}\tau_- (\omega_k^{\supfallback})^3}{25 + 8}
            = \frac{4}{33}\mu\tau_-  M_k^{-\frac{1}{2}}(\omega_k^{\supfallback})^3,
        \end{align*}
        which satisfies the condition in \Cref{alg:adap-newton-cg} for decreasing $M_k$ since $\bar \omega$ therein is $\omega_k^{\supfallback}$.
        Thus, the second property of \lemmaref{lem:appendix/decreasing-Mk-condition} is true.
    \end{enumerate}

    \paragraph{Case 2}
    When $\text{d\_type}_k = \texttt{SOL}$, 
    and let $m_*$ and $\hat m_*$ be the smallest integer such that \eqref{eqn:smooth-line-search-sol} and \eqref{eqn:smooth-line-search-sol-smaller-stepsize} hold, respectively, as defined in \lemmaref{lem:newton-cg-sol}.
    We also recall that $C_{M_k}^2 = \frac{3(1-\mu)M_k}{L_H} \geq \frac{M_k}{L_H}$.

    Since the previous case addresses $m_* = 0$, we assume $m_* > 0$ here.
    Then, the condition for increasing $M_k$ in \Cref{alg:adap-newton-cg} is
    \begin{equation}
        \label{eqn:proof/cond-inc-Mk-SOL}
        \Delta_k \leq \tau_+ \beta \mu M_k^{-\frac{1}{2}} \omega_k^3.
    \end{equation}
    The condition for decreasing $M_k$ is
    \begin{equation}
        \label{eqn:proof/cond-dec-Mk-SOL}
        \Delta_k \geq \mu \tau_- M_k^{-\frac{1}{2}} (\omega_k^{\supfallback})^3.
    \end{equation}

    \begin{enumerate}
        \item When $k \in \cJ^1$ and $m_{\mathrm{max}} \geq m_* > 0$,
        i.e., $m_k = m_*$ and $x_{k+1} = x_k = \beta^{m_k} d_k$,
         we have
    \begin{align*}
         \tau_+ \beta \mu M_k^{-\frac{1}{2}} \omega_k^3
         \overset{\eqref{eqn:proof/cond-inc-Mk-SOL}}{\geq}
        \Delta_k
         \overset{\eqref{eqn:newton-cg-sol-decay}}{\geq}
         \frac{36\beta\mu(1 - \mu)^2}{L_H^2} M_k^{\frac{3}{2}} \omega_k^3
         \geq \frac{9\beta\mu}{L_H^2} M_k^{\frac{3}{2}} \omega_k^3,
    \end{align*}
    Since $\tau_+ \leq 1$, then we know $M_k \leq \tau_+^{\frac{1}{2}} L_H / 3 \leq L_H / 3$.
        \item When $m_{\mathrm{max}} \geq m_*  > 0$ and $M_{k} \geq \tau_-^{-1}(9\beta)^{-\frac{1}{2}}L_H$ 
        and $\omega_k \geq \tau_- \omega_k^{\supfallback}$,
    then 
    \begin{align*}
        \Delta_k
         \overset{\eqref{eqn:newton-cg-sol-decay}}{\geq}
        \frac{9\beta\mu}{L_H^2} M_k^{\frac{3}{2}}  \omega_k^3
        = \left (\frac{9\beta\mu}{L_H^2} M_k^{2} \right ) M_k^{-\frac{1}{2}} \omega_k^3
        \geq 
        \mu \tau_-^{-2} M_k^{-\frac{1}{2}} (\tau_-^{3}(\omega_k^{\supfallback})^3)
        =
        \mu \tau_- M_k^{-\frac{1}{2}} (\omega_k^{\supfallback})^3,
    \end{align*}
        which satisfies \eqref{eqn:proof/cond-dec-Mk-SOL}, and hence $k \in \cJ^{-1}$.

        \item When $k \in \cJ^1$ and $m_* > m_{\mathrm{max}}$ and $m_{\mathrm{max}} \geq \hat m_* \geq 0$, then we know 
        \begin{align*}
        \tau_+ \beta \mu M_k^{-\frac{1}{2}} \omega_k^3 
         \overset{\eqref{eqn:proof/cond-inc-Mk-SOL}}{\geq}
        \Delta_k
        \overset{\eqref{eqn:newton-cg-sol-decay-smaller-stepsize}}{\geq} 
        \mu \beta^{\hat m_*} C_{M_k}^3 \min\left( C_{M_k}, 1 \right) M_k^{-\frac{1}{2}} \omega_k^3
        ,
        \end{align*}
        which implies $\beta \geq \beta \tau_+ \geq  \beta^{\hat m_*} C_{M_k}^3 \min\left( C_{M_k}, 1 \right)$.
        If $C_{M_k} \leq 1$, then its definition implies that $M_k \leq 2L_H / 3$.
        Otherwise, we have 
    $\beta \geq \beta^{\hat m_*} C_{M_k}^3$.
    When $\hat m_* = 0$, we know $C_{M_k}^3 \leq \beta \leq 1$ and hence $M_k \leq 2L_H / 3$;
    when $\hat m_* > 0$, \lemmaref{lem:newton-cg-sol} shows $\beta^{\hat m_* - 1} > \sqrt 2 C_{M_k} > C_{M_k}$, 
    and hence $C_{M_k}^4 \leq 1$, leading to $M_k \leq 2L_H / 3$.
        \item When $m_* > m_{\mathrm{max}}$ and $m_{\mathrm{max}} \geq \hat m_* \geq 0$, and  $M_k \geq L_H$, we have $C_{M_k} \geq 1$ 
        and by \lemmaref{lem:newton-cg-sol}, $\hat m_* = 0$, 
        since otherwise we have $1 \geq \beta^{\hat m_* - 1} > \sqrt{2} C_{M_k} > 1$, leading to a contradiction.
    Then, \eqref{eqn:newton-cg-sol-decay-smaller-stepsize} gives  
    $\Delta_k \geq \mu M_k^{-\frac{1}{2}} \omega_k^3$, 
    and therefore $k \in \cJ^{-1}$ as long as $\omega_k \geq \tau_- \omega_k^{\supfallback}$.
    \item When $m_* > m_{\mathrm{max}}$ and $\hat m_* > m_{\mathrm{max}}$, then \lemmaref{lem:newton-cg-sol} shows that $M_k \leq L_H / 2$,
    and the algorithm directly increases $M_k$ so that $k \in \cJ^1$.
    \end{enumerate}
    The above arguments show that when $k \in \cJ^1$, we have $M_k \leq L_H \leq \tilde C_5 L_H$,
    and when $\omega_k \geq \tau_- \omega_k^{\supfallback}$ and $M_k > \tilde C_4 L_H  \geq  \max(1, \tau_-^{-1}(9\beta)^{-\frac{1}{2}}) L_H$, we have $k \in \cJ^{-1}$, i.e., the first property of \lemmaref{lem:appendix/decreasing-Mk-condition} is true for \texttt{SOL} case.

    \paragraph{Case 3}
     When $\text{d\_type}_k = \texttt{NC}$, let $m_*$ be the smallest integer such that \eqref{eqn:smooth-line-search-nc} holds, as defined in \lemmaref{lem:newton-cg-nc}.
     In this case, the condition for decreasing $M_k$ is also \eqref{eqn:proof/cond-dec-Mk-SOL}, and the condition for increasing it is 
    \begin{equation}
        \label{eqn:proof/cond-inc-Mk-NC}
        \Delta_k \leq \tau_+ (1 - 2\mu)^2 \beta^2 \mu M_k^{-\frac{1}{2}} \omega_k^3.
    \end{equation}
     \begin{enumerate}
        \item When $k \in \cJ^1$ and $m_* > 0$, we can similarly use  \eqref{eqn:newton-cg-nc-decay} in \lemmaref{lem:newton-cg-nc} and \eqref{eqn:proof/cond-inc-Mk-NC} to show that $M_k \leq L_H / 3$. 
        \item When $m_* > 0$ and  $M_k \geq \tau_-^{-1} (3\beta(1-2\mu))^{-1} L_H$ and $\tau_-\omega_k^{\supfallback}  \leq\omega_k$, then \lemmaref{lem:newton-cg-nc} shows that \eqref{eqn:proof/cond-dec-Mk-SOL} holds. Therefore,  $k \in \cJ^{-1}$.
        \item When $m_* = 0$, we show that $M_{k+1}$ will not increase, since otherwise \eqref{eqn:newton-cg-nc-decay-ls0} and \eqref{eqn:proof/cond-inc-Mk-NC} imply that $1 > (1 - 2\mu)^2\beta^2 \tau_+ \geq 1$, leading to a contradiction.
        \item When $m_* = 0$ and $\tau_-\omega_k^{\supfallback} \leq \omega_k$, we know \eqref{eqn:proof/cond-dec-Mk-SOL} holds from \eqref{eqn:newton-cg-nc-decay-ls0} and $\tau_- < 1$, and hence $k \in \cJ^{-1}$.
        \item When $m_* > m_{\mathrm{max}}$ and $\hat m_* > m_{\mathrm{max}}$, then \lemmaref{lem:newton-cg-nc} shows that $M_k \leq L_H / (3 - 6\mu)$,
        and the algorithm directly increases $M_k$ so that $k \in \cJ^1$.
     \end{enumerate}
    The above arguments show that when $k \in \cJ^1$, we have $M_k \leq L_H / \min(1, 3 - 6\mu) \leq \tilde C_5 L_H$,
    and when $\omega_k \geq \tau_- \omega_k^{\supfallback}$ and $M_k > \tilde C_4 L_H \geq \tau_-^{-1} (3\beta(1 - 2\mu))^{-1} L_H$, we have $k \in \cJ^{-1}$, i.e., the first property of \lemmaref{lem:appendix/decreasing-Mk-condition} is true for \texttt{NC} case.

    \paragraph{The cardinality of $\cJ^i$}
    By the definition of $\cJ^i$, we have 
    \begin{align*}
    \log_\gamma M_k = \log_\gamma M_0  + |I_{0,k} \cap \cJ^1|-|I_{0,k} \cap \cJ^{-1}|.
    \end{align*}
    For each $k$ we know $M_{k+1} > M_k$ only if $M_k \leq \tilde C_5 L_H$, 
    then $\sup_{k} M_k \leq \max(M_0, \gamma \tilde C_5 L_H)$, 
    and hence \eqref{eqn:cardinarlity-of-M-set-a} holds.
    Adding $|I_{0,k} \setminus \cJ^1|$ to both sides of \eqref{eqn:cardinarlity-of-M-set-a}, we find
    \eqref{eqn:cardinarlity-of-M-set} holds.

    \paragraph{The descent inequality}
    The $D_k$ dependence in \eqref{eqn:summarized-descent-inequality} directly follow from \lemmaref{lem:newton-cg-nc,lem:newton-cg-sol}.
     For the preleading coefficients, we consider the following three cases.
     (1). When $k \in \cJ^1$, the result also follows from the two lemmas and the fact that $M_k \geq 1$.
     We also note that the $L_H^{-\frac{5}{2}}$ dependence only comes from the case where $\text{d\_type} = \texttt{SOL}$ and $m$ does not exist, and for other cases the coefficient is of order $L_H^{-2}$;
     (2). When $k \in \cJ^{-1}$, the result follows from the algorithmic rule of decreasing $M_k$;
     (3). When $k \in \cJ^0$,
     we know the rules in the algorithm for increasing $M_k$ fail to hold, yielding an $M_k^{-\frac{1}{2}}$ dependence of the coefficient.
\end{proof}

\subsection{Proof of Lemma~\ref{lem:main/lower-bound-of-Vk}}
\label{sec:appendix/proof-lower-bound-of-Vk}

\begin{proof}[Proof of \lemmaref{lem:main/lower-bound-of-Vk}]
    When $\omega_k^{\supfallback} = \sqrt{g_k}$, the upper bound over $V_k$ follows from the monotonicity of $\log\log\frac{3A}{a}$.
    On the other hand, when $\omega_k^{\supfallback} = \sqrt{\epsilon_k}$, 
    we know $3\epsilon_{\ell_j-1} \geq 2\epsilon_{\ell_j-1} \geq 2\epsilon_{\ell_{j+1}-1}$ since $\{ \epsilon_k \}_{k \geq 0}$ is non-increasing.
    Then, we can apply \lemmaref{lem:summing-log-log-sequence} below with $a = 3$ to obtain
    \begin{align*}
        V_k 
        &\leq 
        \sum_{j=1}^{J_k-1}
        \log\log \frac{3\epsilon_{\ell_j-1}}{\epsilon_{\ell_{j+1}-1}}
        + \log \log \frac{3\epsilon_{\ell_{J_k-1}}}{\epsilon_{k}} 
        \\
        \overset{\eqref{eqn:summing-log-log-sequence}}&{\leq}
        \frac{1}{\log 3} \log \frac{\epsilon_{\ell_1-1}}{\epsilon_k} + J_k \log\log 3
        \leq \log \frac{\epsilon_{0}}{\epsilon_k} + J_k,
    \end{align*}
    where we have used the fact that $\log 3 \geq 1$ and $\log \log 3 \leq 1$.
\end{proof}

\begin{lemma}
        \label{lem:summing-log-log-sequence}
    Let $\{ b_j \}_{j \geq 1} \subseteq (0, \infty)$ be a sequence, and $a \geq 3$, $ab_j\geq 2b_{j+1}$, then we have for any $k \geq 1$,
    \begin{equation}
        \label{eqn:summing-log-log-sequence}
        \sum_{j=1}^k \log\log \frac{ab_{j}}{b_{j+1}}
        \leq\frac{1}{\log a} \log \frac{b_1}{b_{k+1}}+ k \log \log a. 
    \end{equation}
\end{lemma}
\begin{proof}
    Using the fact $\log(1 + x) \leq x$ for $x > -1$, and $\log b_j - \log b_{j+1} \geq -\log a + \log 2 > -\log a$, we have
    \begin{align*}
        \sum_{j=1}^k \log\log \frac{ab_{j}}{b_{j+1}}
        &= 
        \sum_{j=1}^k \log\left( 1  + \frac{\log b_j - \log b_{j+1}}{\log a} \right) + k \log \log a \\
        &\leq 
        \sum_{j=1}^k \left( \frac{\log b_j - \log b_{j+1}}{\log a} \right) + k \log \log a \\
        &= \frac{\log b_1 - \log b_{k+1}}{\log a}+ k \log \log a,
    \end{align*}
    which completes the proof.
\end{proof}

\subsection{The counting lemma}\label{sec:proof-counting-lemma}

\begin{lemma}[Counting lemma]
    \label{lem:basic-counting-lemma}
    Let $\cJ^{-1}, \cJ^0, \cJ^1 \subset \N$ be the sets in \lemmaref{lem:lipschitz-constant-estimation}, then we have
    at least one of the following inequalities holds:
    \begin{align}
        \label{eqn:basic-counting-lemma-loss-descent}
        \Sigma_k &\geq 
        \frac{k}{5(U_k+2)} - [\log_\gamma (\tilde C_5 M_0^{-1} L_H)]_+ - 2, \\
        \label{eqn:basic-counting-lemma-gradient-descent}
        W_k&\geq \frac{k}{3(U_k+2)},
    \end{align}
    where $\Sigma_k := |I_{0,k} 
    \cap \cJ^{-1}| + \max\left( |S_k \cap \cJ^0|, |I_{0,k}
    \cap\cJ^0| - W_k - U_kJ_k \right)$, and $S_k \subseteq I_{0,k}$, 
    $U_k \geq 0$, $J_k -1 = |S_k|$ and $W_k \in \R$, and $\tilde C_5$ is defined in \lemmaref{lem:lipschitz-constant-estimation}, $M_0$ is the input in \Cref{alg:adap-newton-cg}.
\end{lemma}
\begin{proof}
    Denote $B_k = (U_k + 2)^{-1}|
    I_{0,k} \cap \cJ^0|$
    and $\Gamma_k = [\log_\gamma(\gamma\tilde C_5 M_0^{-1}L_H)]_+$. 
    We consider the following five cases, where the first three cases deal with $J_k < B_k$, and the last two cases are the remaining parts.  
    We also note that the facts $|I_{0,k}|=k$ and  $1 \geq \frac{2}{U_k + 2}$ are frequently used.

    \paragraph{Case 1}
    When $J_k < B_k$ and $W_k < B_k$, we have 
    \begin{align*}
        \Sigma_k 
        &\geq 
        |%
        I_{0,k}\cap \cJ^{-1}| 
        + |%
        I_{0,k}\cap \cJ^{0}| 
        - U_kJ_k - W_k
        >
        |%
        I_{0,k}\cap \cJ^{-1}| 
        + \frac{|%
        I_{0,k}\cap \cJ^{0}|}{U_k+2} \\
        &\geq 
        \frac{2|%
        I_{0,k}\cap \cJ^{-1}| + |%
        I_{0,k}\cap \cJ^{0}|}{U_k+2}
        \overset{\eqref{eqn:cardinarlity-of-M-set}}{\geq} 
        \frac{k - \Gamma_k}{U_k+2}.
    \end{align*}
    
    \paragraph{Case 2}
    When $J_k < B_k \leq W_k$, and $|%
    I_{0,k}\cap \cJ^0| \leq \frac{k}{3}$, 
    then by \eqref{eqn:cardinarlity-of-M-set} we know 
    $k \leq 2|%
    I_{0,k}\cap \cJ^{-1}| + \frac{k}{3} + \Gamma_k$, and hence, 
    $\Sigma_k \geq |%
    I_{0,k}\cap \cJ^{-1}| \geq \frac{k}{3} - \frac{1}{2}\Gamma_k$.

    \paragraph{Case 3}
    When $J_k < B_k \leq W_k$, and $|%
    I_{0,k}\cap \cJ^0| > \frac{k}{3}$, 
    then $W_k \geq B_k > \frac{k}{3(U_k+2)}$.

    \paragraph{Case 4}
    When $|S_k \cap \cJ^0| > B_k/2$, we have
    \begin{align*}
        \Sigma_k 
        \geq 
        |%
        I_{0,k}\cap \cJ^{-1}| + |S_k\cap \cJ^0|
        \geq \frac{2|%
        I_{0,k}\cap \cJ^{-1}|+|%
        I_{0,k}\cap\cJ^0|}{2(U_k+2)}
        \overset{\eqref{eqn:cardinarlity-of-M-set}}{\geq} 
        \frac{k - \Gamma_k}{2(U_k+2)}.
    \end{align*}

    \paragraph{Case 5}
     When $J_k \geq B_k$ and $|S_k \cap \cJ^0| \leq B_k/2$, we have
    \begin{align*}
        B_k - 1 \leq J_k - 1
        = |S_k| 
        &=
         |S_k \cap \cJ^0|
        + |S_k \cap \cJ^1|
        + |S_k \cap \cJ^{-1}|\\
        &\leq \frac{B_k}{2}
        + |%
        I_{0,k}\cap \cJ^1|
        + |%
        I_{0,k} \cap \cJ^{-1}|\\
        \overset{\eqref{eqn:cardinarlity-of-M-set-a}}&{\leq}
        \frac{B_k}{2}
        + 2|%
        I_{0,k}\cap \cJ^{-1}|
        + \Gamma_k.
    \end{align*}
    Therefore, we have
    \begin{align*}
        \Sigma_k 
        &\geq 
        |%
        I_{0,k}\cap \cJ^{-1}|\\
        &=
        \frac{1}{5}|%
        I_{0,k} \cap \cJ^{-1}|
        + \frac{4}{5}|%
        I_{0,k} \cap \cJ^{-1}| \\
        &\geq 
        \frac{1}{5}\cdot\frac{8|%
        I_{0,k} \cap \cJ^{-1}|}{4(U_k+2)}
        + \frac{4}{5}\left( \frac{B_k}{4} - \frac{1}{2} - \frac{\Gamma_k}{2} \right) \\
        &=
        \frac{1}{5} \left ( 
            \frac{8|%
            I_{0,k}\cap\cJ^{-1}| + 4|%
            I_{0,k}\cap\cJ^0|}{4(U_k+2)}
            - 2 - 2\Gamma_k
            \right ) \\ 
        \overset{\eqref{eqn:cardinarlity-of-M-set}}&{\geq}
        \frac{1}{5} \left ( 
            \frac{k - \Gamma_k}{U_k+2}
            - 2 - 2\Gamma_k
            \right ).
    \end{align*}
    Summarizing the above cases, we conclude that 
    \begin{align*}
        \Sigma_k \geq 
        \frac{k}{5(U_k+2)} - \Gamma_k - \frac{2}{5}
        \geq \frac{k}{5(U_k+2)} - [\log_\gamma (\tilde C_5 M_0^{-1}L_H)]_+ - 2,
    \end{align*}
    and the proof is completed. %
\end{proof}

\subsection{Technical lemmas for Lemma~\ref{lem:main/iteration-in-a-subsequence}} \label{sec:summing-lemmas}

This section establishes two crucial lemmas for proving \lemmaref{lem:main/iteration-in-a-subsequence} (a.k.a. \lemmaref{lem:proof/iteration-in-a-subsequence} in the appendix). 
\lemmaref{lem:accumulated-descent-lower-bound}, mentioned in the ``sketch of the idea'' part of \lemmaref{lem:main/iteration-in-a-subsequence}, is  specifically applied to the case  $\theta = 0$.
For $\theta > 0$, we employ a modified version of this result as detailed in \lemmaref{lem:accumulated-mixed-descent-lower-bound}.

\begin{lemma}
    \label{lem:accumulated-descent-lower-bound}
    Given $K \in \N$, $p > q > 0$, and $A \geq  a > 0$, and let $\{ g_j \}_{0 \leq j \leq K+1}$ be such that 
    $A = g_0 \geq g_1 \geq \dots \geq g_K \geq g_{K+1} = a$.
    Then, for any subset $S \subseteq [K]$, we have
    \begin{align}
        \sum_{i \in S} \frac{g_{i+1}^p}{g_{i}^q} 
        \geq \max(0, |S| - R_a - 2) \ce^{-q} a^{p-q} 
        ,
        \label{eqn:accumulated-descent-lower-bound}
    \end{align}
    where $R_a := \left \lfloor \log \log \frac{3A}{a} - \log \log \frac{p}{q} \right \rfloor \leq \log \log \frac{3A}{a}$.
\end{lemma}
\begin{proof}
    It suffices to consider the case where $A = 1$, since for general cases, we can invoke the result of $A = 1$ with $g_j, a$ replaced with $g_j / A$, $a / A$, respectively.
    Let $\tau = p/q$ and $\cI_k = \{ j \in [K] : \exp(\tau^{k}) a \leq g_j < \exp(\tau^{k+1}) a \}$ with $0 \leq k \leq R_a$ and $\cI_{-1} = \{ j \in [K] : a \leq g_j < \ce a \}$. 
    Let $\zeta_k = \exp(\tau^k)$ for $k \geq 0$ and $\zeta_{-1} = 1$, then we have $\zeta_k^p \zeta_{k+1}^{-q} \geq \ce^{-q}$.
    Note that $\{ \cI_k \}_{-1 \leq k \leq R_a}$ is a partition of $[K]$, then we have
    \begin{align}
        \sum_{i\in S} \frac{g_{i+1}^p}{g_{i}^q}
        &= 
        \sum_{k=-1}^{R_a} \sum_{j \in \cI_k \cap S} \frac{g_{j+1}^p}{g_{j}^q} 
        = \sum_{k=-1}^{R_a}
        \left(  
         \sum_{\substack{j \in S\\ j,j+1 \in \cI_k}} \frac{g_{j+1}^p}{g_{j}^q} 
         + \sum_{\substack{j \in S \\j \in \cI_k, j+1\notin \cI_k}} \frac{g_{j+1}^p}{g_{j}^q} 
        \right)
         \nonumber\\
        &\geq \sum_{k=-1}^{R_a} 
        \sum_{\substack{j \in S \\j, j+1 \in \cI_k}} \frac{\left( \zeta_k a \right)^p}{\left (\zeta_{k+1}a\right )^q}
        \geq 
        \sum_{k=-1}^{R_a} 
        \sum_{\substack{j \in S \\j, j+1 \in \cI_k}} \ce^{-q}a^{p-q}
        = |\cI_S| \ce^{-q}a^{p-q}
        ,
        \label{eqn:newton-pq-superlinear-penalty}
    \end{align}
    where $\cI_S := \{ j \in S : j, j + 1 \in \cI_k, -1 \leq k \leq R_a \}$.
    By the monotonicity of $g_j$, we know for each $k$, there exists at most one $j \in \cI_k$ such that $j + 1 \notin \cI_k$. 
    Hence, $|\cI_S| \geq |S| - (R_a + 2)$.
\end{proof}

\begin{lemma}
    \label{lem:accumulated-mixed-descent-lower-bound}
    Given $K \in \N$, $p_1 > q_1 > 0$, $p_2 > q_2 > 0$ and $A \geq a > 0$, and let $\{ g_j \}_{0 \leq j \leq K+1}$ be such that 
    $A = g_0 \geq g_1 \geq \dots \geq g_K \geq g_{K+1} = a$.
    Then, for any subset $S \subseteq [K]$, we have
    \begin{align}
        &\sum_{i \in S}
        \min \left ( 
            A^{q_1 - p_1}\frac{g_{i+1}^{p_1}}{g_{i}^{q_1}},
            A^{q_2 - p_2}\frac{g_{i}^{p_2}}{g_{i-1}^{q_2}} 
        \right )  
        \nonumber
        \\
        &\quad\quad\quad \geq 
        \max(0, |S| - R_{a,1} - R_{a,2} - 4) \min \left ( \left( A^{-1}a \right)^{p_1 - q_1}, \left( A^{-1}a \right)^{p_2 - q_2} \right ).
        \label{eqn:accumulated-mixed-descent-lower-bound}
    \end{align}
    where 
    $R_{a,i} := \left \lfloor \log \log \frac{3A}{a} - \log \log \frac{p_i}{q_i} \right \rfloor \leq \log\log\frac{3A}{a}$ 
    for $i = 1, 2$.
\end{lemma}


\begin{proof}%
    Similar to \lemmaref{lem:accumulated-descent-lower-bound}, it suffices to show that \eqref{eqn:accumulated-mixed-descent-lower-bound} is true for $A = 1$.
    Let $\tau_i = p_i/q_i$ for $i = 1, 2$ and $\cI_k = \{ j \in [K] : \exp(\tau_1^{k}) a \leq g_j < \exp(\tau_1^{k+1}) a \}$ with $0 \leq k \leq R_{a,1}$ and $\cI_{-1} = \{ j \in [K] : a \leq g_j < \ce a \}$. 
    Note that $\{ \cI_k \}_{-1 \leq k \leq R_{a,1}}$ is a partition of $[K]$, then similar to \eqref{eqn:newton-pq-superlinear-penalty} we have
    \begin{align*}
        \sum_{i\in S} 
        \min \left ( 
            \frac{g_{i+1}^{p_1}}{g_{i}^{q_1}},
            \frac{g_{i}^{p_2}}{g_{i-1}^{q_2}} 
        \right )  
        &\geq 
        \sum_{k=-1}^{R_{a,1}} 
        \sum_{\substack{j \in S \\j, j+1 \in \cI_k} }
        \min \left ( 
            \ce^{-q_1}a^{p_1 - q_1},
            \frac{g_{i}^{p_2}}{g_{i-1}^{q_2}} 
        \right ) \\
        &\geq \sum_{j \in \cI_S}
        \min \left ( 
           \ce^{-q_1} a^{p_1 - q_1},
            \frac{g_{i}^{p_2}}{g_{i-1}^{q_2}} 
        \right )
        .
    \end{align*}
    where $\cI_S := \{ j \in S : j, j + 1 \in \cI_k, -1 \leq k \leq R_{a,1} \}$ and we have used the fact that $\min(\alpha_1, \beta) \geq \min(\alpha_2, \beta)$ if $\alpha_1 \geq \alpha_2$.
    Moreover, we can also conclude that $|\cI_S| \geq |S| - R_{a,1} - 2$.

    Next, we consider the partition of $\cI_S$ and lower bound the summation in the above display.
    Let $\cJ_k = \{ j \in \cI_S : \exp(\tau_2^k)a \leq g_j < \exp(\tau_2^{k+1})a \}$ with $0 \leq k \leq R_{a,2}$, $\cJ_{-1} = \{ j \in \cI_S : a \leq g_j < \ce a \}$, and $\cJ_S := \{ j \in S : j, j - 1 \in \cJ_k, -1 \leq k \leq R_{a,2} \}$. 
    Then, similar to \eqref{eqn:newton-pq-superlinear-penalty} we have
    \begin{align*}
        \sum_{j \in \cI_S}
        \min \left ( 
            \ce^{-q_1} a^{p_1 - q_1},
            \frac{g_{i}^{p_2}}{g_{i-1}^{q_2}} 
        \right ) 
        &\geq 
        \sum_{k=-1}^{R_{a,2}} 
        \sum_{j, j-1 \in \cJ_k} 
        \min \left ( 
           \ce^{-q_1} a^{p_1 - q_1},
           \ce^{-q_2} a^{p_2 - q_2}
        \right )
         \\
        &=
        |\cJ_S| \min \left ( 
            \ce^{-q_1}a^{p_1 - q_1},
            \ce^{-q_2}a^{p_2 - q_2}
        \right ).
    \end{align*}
    Therefore, the proof is completed by noticing that $|\cJ_S| \geq |\cI_S| - R_{a,2} - 2$.
\end{proof}

\section{Main results for local rates}


In this section, we first provide the precise version of \lemmaref{lem:main/asymptotic-newton-properties} in \lemmaref{lem:gradient-decay-of-newton-step,lem:asymptotic-newton-step}, and then prove the main result of the local convergence order. The proofs for technical lemmas are deferred to \Cref{sec:properties-of-newton-step,sec:appendix/local-rate-boosting}.

\begin{assumption}[Positive definiteness]
    \label{assumption:local-strong-convexity}
    There exists $\alpha > 0$ such that $\nabla^2 \varphi(x^*) \succeq \alpha \Id$.
\end{assumption}


Let $C(\alpha, a, b, U)$ be the constant defined in \lemmaref{lem:capped-cg},
$\alpha$ be defined in Assumption~\ref{assumption:local-strong-convexity}, 
and $\gamma, \mu, M_0, \eta$ be the inputs of \Cref{alg:adap-newton-cg},
and $\theta$ be defined in \theoremref{thm:newton-local-rate-boosted}.
We define the following constants which will be subsequently used in \lemmaref{lem:gradient-decay-of-newton-step,lem:asymptotic-newton-step}:
\begin{align*}
    U_M &= \max(M_0, \tilde C_5 \gamma L_H), 
    \delta_0 = \frac{\alpha}{2L_H}, 
    L_g = \| \nabla^2 \varphi(x^*) \| + L_H \delta_0,   \\
    \tilde c &= 
    C\left (\frac{\alpha}{2},
    (1 + 2\theta)^{-1},
    \tau U_\varphi^{-\theta(1 + 2\theta)^{-1}},
    U_M^{\frac{1 - \theta(1 + 2\theta)^{-1}}{2}},
    L_g
    \right ),
    \\
    \delta_1^{\frac{1}{2}} &= \min \Big ( 
        \delta_0^{\frac{1}{2}}, 
        \min(\eta, \tilde c) (U_ML_g)^{-\frac{1}{2}}
    \Big ), \\
    c_1 &= \frac{4}{\alpha} \max \left ( 
        L_H \delta_1^{\frac{1}{2}},
        2(U_M L_g)^{\frac{1}{2}}(1 + L_g) 
        \right ), \\
    \delta_2^{\frac{1}{2}} &= \min\left ( 
        \delta_1^{\frac{1}{2}},
        \frac{1}{2c_1},
        \frac{(1 - 2\mu) \alpha}{
            8L_g c_1\big(c_1 \delta_1^{\frac{1}{2}} + 1 \big) 
            + 32L_H \delta_1^{\frac{1}{2}}
            } 
        \right ), \\
    c_2 &=
    4\alpha^{-2} \max\left( 
        2\alpha^{-1}L_gL_H,
        (2 + \alpha)L_gU_M^{\frac{1}{2}}
     \right), \\
    \delta_3 &= 
    \min\left( \delta_2, 
    c_2^{-2} L_g^{-1} \big ( \delta_2^{\frac{1}{2}} + 1 \big )^{-2},
    \frac{\alpha^2}{4} (L_H + 2U_M^{\frac{1}{2}}L_g^{\frac{1}{2}}(1 + L_g))^{-2}
     \right)
    .
\end{align*}

\begin{lemma}[Newton direction yields superlinear convergence]
    \label{lem:gradient-decay-of-newton-step}
    Let $x, d, M$ and $\omega$ be those in the subroutine \texttt{NewtonStep} of \Cref{alg:adap-newton-cg} with $\text{d\_type} = \texttt{SOL}$.
    Let $x^*$ be such that $\nabla \varphi(x^*) = 0$ and $\nabla^2\varphi(x^*) \succeq \alpha\Id$, 
    then for $x \in B_{\delta_0}(x^*)$, 
    we have the following inequalities 
    \begin{align}
        \label{eqn:distance-decay-of-newton-step}
        \| x^* - (x + d) \|
        &\leq \frac{2}{\alpha}\left( L_H \| x - x^* \|^2 + 2M^{\frac{1}{2}}\omega (1 + L_g) \| x - x^* \| \right)
        ,
        \\
        \label{eqn:gradient-decay-of-newton-step}
        \| \nabla \varphi(x + d) \|
        &\leq \frac{8L_gL_H}{\alpha^3} \| \nabla \varphi(x) \|^2
        + \frac{4L_g(2 + \alpha)}{\alpha^2} M^{\frac{1}{2}} \omega \| \nabla \varphi(x) \|.
    \end{align}
\end{lemma}

The lemma below shows that the Newton direction will be taken when iterates are close enough to the solution.
\begin{lemma}[Newton direction is eventually taken]
    \label{lem:asymptotic-newton-step}
    Let $x^* \in \R^n$ be such that $\nabla \varphi(x^*) = 0$ and Assumption~\ref{assumption:local-strong-convexity} holds. 
    If $\max(\omega_k^{\supsucc}, \omega_k^{\supfallback}) \leq \sqrt{g_k}$, then $\text{d\_type}_k = \texttt{SOL}$ and $m_k = 0$ exists for $x_k \in B_{\delta_2}(x^*)$.
    Moreover, the trial step using $\omega_k^{\supsucc}$ is accepted for $x_k \in B_{\delta_3}(x^*)$.
\end{lemma}

\subsection{Proof of local rates in Theorem~\ref{thm:newton-local-rate-boosted}} \label{sec:appendix/proof-boosted-local-rates-theorem}

\begin{proposition}
    \label{prop:mixed-newton-nonconvex-phase-local-rates}
    Let $\{ x_k \}_{k \ge 0}$ be the points generated by \Cref{alg:adap-newton-cg} with the regularizer choices in \theoremref{thm:newton-local-rate-boosted} and $\theta \geq 0$;
    and $x^*, \{x_{k_j}\}_{j \geq 0}$ be those in \theoremref{thm:appendix/global-newton-complexity} such that $\lim_{j \to \infty} x_{k_j} = x^*$ and $\nabla\varphi(x^*) = 0$ and suppose Assumption~\ref{assumption:local-strong-convexity} holds, i.e., $\nabla^2\varphi(x^*) \succeq \alpha \Id$.

    Then, there exists $j_0$ such that $\epsilon_{j_0} = g_{j_0} < \min(1, (2c_2)^{-2})$ and $x_{j_0} \in B_{\delta_3}(x^*)$, and
    \begin{enumerate}
        \item  $\lim_{k \to \infty} x_k = x^*$.
        \item When $\theta \in (0, 1]$ and $j \geq 1$, we have
        \begin{equation*}
            \| \nabla \varphi(x_{j_0+j+1}) \| \leq 
            (2c_2)^3 \| \nabla \varphi(x_{j_0+j}) \|^{1 + \nu_\infty - (4\theta/9)^k},
        \end{equation*}
        where $\nu_\infty \in \left [\frac{1}{2}, 1\right ]$ is defined in \lemmaref{lem:appendix/superlinear-rate-boosting-generalized} and illustrated in \figureref{fig:local-rate-for-nu1}.
        \item When $\theta > 1$ and $j \geq \log_2\frac{2\theta - 1}{2\theta - 2} + 1$, we have
        \begin{equation*}
            \| \nabla \varphi(x_{j_0+j+1}) \| \leq 
            (2c_2)^{2\theta + 2} \| \nabla \varphi(x_{j_0+j}) \|^{2}.
        \end{equation*}
    \end{enumerate}
\end{proposition}
\begin{proof}
    Since $\lim_{j \to \infty} x_{k_j} = x^*$ and $\nabla\varphi(x^*) = 0$, we know $j_0$ exists.
We define the set
    \begin{equation}
        \cI = 
        \{
            j \in \N : 
    g_j = \epsilon_j
    \text{ and } x_j \in B_{\delta_3}(x^*)
        \}.
    \end{equation}

    By the existence of $j_0$, we know $j_0 \in \cI$. 
    Suppose $k \in \cI$, then we will show that $k + 1 \in \cI$.
    Since the choices of $\omega_{k}^{\supfallback}$ and $\omega_k^{\supsucc}$ in \theoremref{thm:newton-local-rate-boosted} fulfill the condition of \lemmaref{lem:asymptotic-newton-step}, 
    we know the trial step is taken and $x_{{k}+1} = x_{k} + d_{k}$,
    where $d_{k}$ is the direction in \texttt{NewtonStep} with $\omega = \omega_k^{\supsucc}$.

    From \lemmaref{lem:gradient-decay-of-newton-step} and \corollaryref{cor:perturbation-lemma},
    we have $g_{k} \leq L_g \| x_{k} - x^* \| \leq L_g \delta_3$, $\omega_k^{\supsucc} \leq \sqrt{g_k}$ and 
    \begin{align}
    g_{{k}+1} 
    \overset{\eqref{eqn:gradient-decay-of-newton-step}}{\leq} 
    c_2 g_{k}^2 + c_2 \omega^{\supsucc}_{k} g_{k}
    \leq c_2 \big ( L_g \delta_3 + (L_g \delta_3)^{\frac{1}{2}} \big ) g_{k} 
    \leq c_2 \big ( L_g \delta_2^{\frac{1}{2}} + L_g^{\frac{1}{2}} \big )\delta_3^{\frac{1}{2}} g_{k} 
    \leq g_{k}.
    \label{eqn:appendix/proof/basic-gradient-descent-of-newton}
    \end{align}
    Hence, $\epsilon_{{k}+1} = \min(\epsilon_{k}, g_{{k}+1}) = g_{{k}+1}$.
    Moreover, since $M_k \leq U_M$, then
    \begin{align*}
        \|x_{k+1} - x^* \|
        \overset{\eqref{eqn:distance-decay-of-newton-step}}&{\leq}
        \frac{2}{\alpha}\left( L_H \delta_3^2 + 2U_M^{\frac{1}{2}} (L_g \delta_3)^{\frac{1}{2}} (1 + L_g) \delta_3\right)   \\
        &\leq \frac{2}{\alpha}\left( L_H  + 2U_M^{\frac{1}{2}} L_g^{\frac{1}{2}} (1 + L_g) \right) \delta_3^{\frac{3}{2}}
        \leq \delta_3.
    \end{align*}
    Thus, we know $k + 1 \in \cI$.
    By induction, $k \in \cI$ for every $k \geq j_0$, 
    which also gives the convergence of the whole sequence $\{x_k\}$ since \lemmaref{lem:gradient-decay-of-newton-step} provides a superlinear convergence with order $\frac{3}{2}$ of the sequence $\{ \|x_k - x^*\| \}_{k \geq j_0}$.

    Furthermore, the regularizer $\omega_{k}^{\supsucc}$ reduces to $g_{k}^{\frac{1}{2} + \theta} g_{{k}-1}^{-\theta}$ for $k \geq j_0 + 1$
    and the premises of \lemmaref{lem:appendix/superlinear-rate-boosting-generalized} and \corollaryref{cor:appendix/quadratic-rate-boosting} are satisfied, 
    with the constants $c_0, c$, and $\nu$ therein chosen as $c_2, c_2$, and $1$, respectively.
    Then, the conclusion follows from \lemmaref{lem:appendix/superlinear-rate-boosting-generalized} and \corollaryref{cor:appendix/quadratic-rate-boosting}.
\end{proof}


\section{Technical lemmas for local rates}


\subsection{Standard properties of the Newton step}
\label{sec:properties-of-newton-step}

This section provides the proofs of \lemmaref{lem:gradient-decay-of-newton-step,lem:asymptotic-newton-step}, which are the detailed version of \lemmaref{lem:main/asymptotic-newton-properties}.

The following lemma is used to show that $\nabla^2\varphi(x)\succ 0$ in a neighborhood of $x^*$. 
It can be found in, e.g., \citet[Lemma 7.2.12]{facchinei2003finite}.
\begin{lemma}[Perturbation lemma]
    \label{lem:perturbation-lemma}
    Let $A, B \in \R^{n \times n}$ with $\| A^{-1} \| \leq \alpha$. 
    If $\| A - B \| \leq \beta$ and $\alpha\beta < 1$, then 
    \begin{equation}
        \label{eqn:perturbation-lemma}
        \|B^{-1} \| \leq \frac{\alpha}{1 - \alpha\beta}.
    \end{equation}
\end{lemma}

\begin{corollary}
    \label{cor:perturbation-lemma}
    Under Assumption~\ref{assumption:local-strong-convexity}, we have the following properties:
    \begin{enumerate}
        \item When $x \in B_{\delta_0}(x^*)$, we know $\nabla^2\varphi(x) \succeq \frac{\alpha}{2} \Id$ and $\| (\nabla^2\varphi(x))^{-1} \| \leq \frac{2}{\alpha}$.
        \item $\frac{\alpha}{2} \| x - y \| \leq \| \nabla \varphi(x) - \nabla \varphi(y) \| \leq L_g \| x - y \|$ for $x, y \in B_{\delta_0}(x^*)$.
    \end{enumerate}
\end{corollary}
\begin{proof}
    The first part directly follows from \lemmaref{lem:perturbation-lemma}.
    Since $\nabla^2\varphi$ is $L_H$-Lipschitz, then 
    \begin{align*}
    \sup_{x \in B_{\delta_0}(x^*)} \| \nabla^2 \varphi(x) \| 
    \leq \| \nabla^2\varphi(x^*) \| + L_H \delta_0 = L_g,
    \end{align*}
    implying that $\nabla \varphi$ is $L_g$-Lipschitz on $B_{\delta_0}(x^*)$.
    Then, the second part follows from \citet[Section 1]{nesterov2018lectures}.
\end{proof}

\begin{proof}[Proof of \lemmaref{lem:gradient-decay-of-newton-step}]
    From \corollaryref{cor:perturbation-lemma}, we know $H \succeq \frac{\alpha}{2} \Id$ and 
    $\| H^{-1} \| \leq \frac{2}{\alpha}$ for every $x \in B_{\delta}(x^*)$ and $H = \nabla^2\varphi(x)$.
    Then, let $\epsilon = M^{\frac{1}{2}}\omega$ and note that by the choice in \Cref{alg:adap-newton-cg}, $\tilde \eta \leq M^{\frac{1}{2}} \omega = \epsilon$, we have
    \begin{align}
    \nonumber
        \| x^* - (x + d) \|
        &\leq 
        \| (H + 2\epsilon \Id)^{-1} \nabla \varphi(x) + (x^* - x) \|
        + \| d + (H + 2\epsilon \Id)^{-1} \nabla \varphi(x)  \| \\
    \nonumber
        \overset{\eqref{eqn:capped-cg-hessian-upperbound}}&{\leq}
       \|(H + 2\epsilon \Id)^{-1} \|
       \left ( 
        \| \nabla \varphi(x) + H (x^* - x)  \|
        + 2\epsilon \| x^* - x \|
        + \tilde \eta    \| \nabla \varphi(x) \|
        \right ) \\
    \nonumber
        &\leq 
        \frac{2}{\alpha}
       \left ( 
        \| \nabla \varphi(x) + H (x^* - x)  \|
        + 2\epsilon \| x^* - x \|
        + 2\epsilon \| \nabla \varphi(x) \|
        \right ) \\
        \overset{\eqref{eqn:hessian-lip-gradient-inequ}}&{\leq}
        \frac{2}{\alpha}
       \left ( 
        L_H \| x^* - x \|^{2}
        + 2\epsilon \| x^* - x \|
        + 2\epsilon \| \nabla \varphi(x) \|
        \right ).
        \label{eqn:proof/newton-distance-decay-asymptotic}
    \end{align}
    From \corollaryref{cor:perturbation-lemma}, we know $\frac{\alpha}{2} \| x - x^*\| \leq \| \nabla \varphi(x) \| \leq L_g \| x - x^* \|$, 
    yielding \eqref{eqn:distance-decay-of-newton-step}.

    Furthermore, we have
    \begin{align*}
        \| \nabla \varphi(x + d)\|
        \leq  L_g \| x^* - (x + d) \|
        \overset{\eqref{eqn:proof/newton-distance-decay-asymptotic}}&{\leq} 
        \frac{2L_g}{\alpha}
       \left ( 
        L_H \| x^* - x \|^{2}
        + 2\epsilon \| x^* - x \|
        + 2\epsilon \| \nabla \varphi(x) \|
        \right ) \\
        &\leq 
        \frac{2L_g}{\alpha}
       \left ( 
        \frac{4L_H}{\alpha^2} \| \nabla \varphi(x) \|^{2}
        + \frac{4 + 2\alpha}{\alpha} \epsilon \| \nabla \varphi(x) \|
        \right ).
    \end{align*}
\end{proof}

\begin{proof}[Proof of \lemmaref{lem:asymptotic-newton-step}]
    Let $r_k = \| x_k - x^* \|$, the proof is divided to three steps.

    \paragraph{Step 1}
    We show that $\text{d\_type}_k = \texttt{SOL}$ for $x_k \in B_{\delta_1}(x^*)$ regardless of whether the trial step or the fallback step is taken.
    By \corollaryref{cor:perturbation-lemma}, we have
     $\nabla^2\varphi(x) \succeq \frac{\alpha}{2} \Id$ 
    for $x \in B_{\delta_0}(x^*)$.
    From \lemmaref{lem:capped-cg}, 
    when the fallback step is taken, then $\text{d\_type}_k = \texttt{SOL}$.
    On the other hand, if the trial step is taken, 
    we will also invoke \lemmaref{lem:capped-cg} as follows.
    Let $a = (1 + 2\theta)^{-1} \in (0, 1]$, we have
    \begin{enumerate}
        \item When $\omega^{\supsucc}_k = g_k^{\frac{1}{2}} \min(1, g_k^\theta g_{k-1}^{-\theta})$, 
        we know  $(\omega^{\supsucc}_k)^a \geq g_k^{\frac{1}{2}} U_\varphi^{-a\theta} = \omega_k^{\supfallback} U_\varphi^{-a\theta}$;
        \item When $\omega^{\supsucc}_k = \epsilon_k^{\frac{1}{2} + \theta} \epsilon_{k-1}^{-\theta}$, 
        it still holds that $(\omega^{\supsucc}_k)^a\geq \omega_k^{\supfallback} U_\varphi^{-a\theta}$.
    \end{enumerate}
    Therefore, let $\bar\rho = \tau \sqrt{M_k}\omega_k^{\supfallback}$ and $\rho = \sqrt{M_k} \omega_k^{\supsucc}$, 
    and note that from \lemmaref{lem:lipschitz-constant-estimation} we have $M_k \leq  U_M$, 
    then let $b = \tau U_\varphi^{a\theta} U_M^{\frac{1-a}{2}}$, we know
    \begin{align*}
        \rho^a &= M_k^{\frac{a}{2}} (\omega_k^{\supsucc})^a
        \geq M_k^{\frac{a}{2}} \omega_k^{\supfallback} U_{\varphi}^{-a\theta}
        =  \tau^{-1} U_{\varphi}^{-a\theta} M_k^{\frac{a-1}{2}} \bar \rho
        \overset{(a \leq 1)}{\geq}
        \tau^{-1} U_{\varphi}^{-a\theta} U_M^{\frac{a-1}{2}} \bar \rho
        = b^{-1} \bar \rho.
    \end{align*}
    Since the map $U \mapsto C(\alpha, a, b, U)$ defined in \lemmaref{lem:capped-cg} is non-increasing, 
    we know 
    \begin{align*}
    \inf_{x\in B_{\delta_0}(x^*)} C(\alpha / 2, a, b, \|\nabla^2\varphi(x)\|)
    \geq C(\alpha / 2, a, b, \|\nabla^2\varphi(x^*)\| + L_H \delta_0 ) =: \tilde c > 0.
    \end{align*}
    From \corollaryref{cor:perturbation-lemma}, 
    we know for $x_k \in B_{\delta_1}(x^*)$, 
    \begin{align*}
        \rho =  
        \sqrt{M_k} \omega_k^{\supsucc}
        \leq U_M^{\frac{1}{2}} g_k^{\frac{1}{2}}
        \leq U_M^{\frac{1}{2}} (L_g \delta_1)^{\frac{1}{2}}
        \leq \min\left( \eta, \tilde c \right).
    \end{align*}
    Thus, \texttt{CappedCG} is invoked with $\xi = \rho$ and the premises of the fourth item in \lemmaref{lem:capped-cg} are satisfied, which leads to $\text{d\_type}_k = \texttt{SOL}$.

    \paragraph{Step 2}
    This is a standard step showing that the Newton direction will be taken (see, e.g., \citet{facchinei1995minimization,facchinei2003finite}).

    We show that $m_k = 0$ for $x_k \in B_{\delta_2}(x^*)$ regardless of whether the trial step or the fallback step is taken.
    Define $\omega_k = \omega_k^{\supsucc}$ if the $k$-th step is accepted and $\omega_k = \omega_k^{\supfallback}$ otherwise, and denote $d_k$ as the direction generated in \texttt{NewtonStep} with such $\omega_k$. 
    By the assumption and \lemmaref{lem:gradient-decay-of-newton-step},
    we have for $x_k \in B_{\delta_1}(x^*)$, it holds that $\omega_k \leq g_k^{\frac{1}{2}} \leq L_g^{\frac{1}{2}} r_k^{\frac{1}{2}}$, 
    and $\sup_{x \in B_{\delta_1}(x^*)} \| \nabla^2\varphi(x) \| \leq L_g$, and 
    \begin{equation}
        \label{eqn:proof/newton-step-decay}
        \| x_k + d_k - x^* \| 
        \overset{\eqref{eqn:distance-decay-of-newton-step}}{\leq}  
        \frac{2}{\alpha}\left( 
            L_H r_k^2 + 2M_k^{\frac{1}{2}} (1 + L_g) r_k\omega_k
         \right) \leq c_1 r_k^{\frac{3}{2}},
    \end{equation}
    where we have used \lemmaref{lem:lipschitz-constant-estimation} to obtain $M_k \leq U_M$.
    Using the mean-value theorem and noticing that $\nabla \varphi(x^*) = 0$, 
    there exist $\zeta, \xi \in (0, 1)$ and $H_\zeta = \nabla^2 \varphi(x^* + \zeta (x_k - x^*))$, 
    $H_\xi = \nabla^2 \varphi(x^* + \xi(x_k + d_k - x^*))$ such that for $x_k \in B_{\delta_1}(x^*)$, 
    \begin{align*}
        \varphi(x_k) - \varphi(x^*) &= \frac{1}{2}(x_k - x^*)^\top H_\zeta(x_k - x^*), \\
        \varphi(x_k + d_k) - \varphi(x^*) &= \frac{1}{2}(x_k + d_k - x^*)^\top H_\xi(x_k + d_k - x^*)
         \overset{\eqref{eqn:proof/newton-step-decay}}{\leq} 
         \frac{L_gc_1^2}{2} r_k^3.
    \end{align*}
    Combining them, we have for $x_k \in B_{\delta_1}(x^*)$, 
    \begin{align}
        \nonumber
        &\peq \varphi(x_k +d_k) - \varphi(x_k) - \frac{1}{2} \nabla \varphi(x_k)^\top d_k \\
        \nonumber
        &\leq 
         \frac{L_gc_1^2}{2} r_k^{3} 
        -\frac{1}{2}(x_k - x^*)^\top H_\zeta(x_k - x^*)
        -\frac{1}{2}\nabla \varphi(x_k)^\top d_k
        \\
        &=
         \frac{L_gc_1^2}{2} r_k^{3}
        -\frac{1}{2}(x_k + d_k - x^*)^\top H_\zeta(x_k - x^*)
        -\frac{1}{2}(\nabla \varphi(x_k) - H_\zeta(x_k - x^*))^\top d_k 
        .
        \label{eqn:nonasymp-inexact-newton-line-search-final}
    \end{align}
    Let $\bar x = x^* + \zeta(x_k - x^*)$ and note $\nabla \varphi(x^*) = 0$, then
    \begin{align*}
        &\peq \| \nabla \varphi(x_k) - \zeta^{-1} \nabla \varphi(\bar x) \|
        = \| (\nabla \varphi(x_k) - \nabla\varphi(x^*)) - \zeta^{-1} (\nabla \varphi(\bar x) - \nabla \varphi(x^*)) \| \\
        &= \left \| \int_0^1 \nabla^2\varphi(x^* + t(x_k - x^*)) (x_k - x^*) \dd t
        - \zeta^{-1} \int_0^1 \nabla^2\varphi(x^* + t(\bar x  - x^*)) (\bar x - x^*) \dd t \right \| \\
        &= \left \| \int_0^1 (\nabla^2\varphi(x^* + t(x_k - x^*)) - \nabla^2\varphi(x^* + t(\bar x - x^*))) (x_k - x^*) \dd t \right \| \\
        &\leq L_H \int_0^1 t \| x_k - \bar x \| r_k \dd t 
        = L_H \int_0^1 t (1 - \zeta) \| x_k - x^* \| r_k \dd t 
        \leq L_H r_k^2.
    \end{align*}
    Therefore, we have for $x_k \in B_{\delta_1}(x^*)$,
    \begin{align}
        \nonumber
        &\peq \left \|  
          \nabla \varphi(x_k) - H_\zeta(x_k - x^*)
        \right \| \\ 
        \nonumber
        &\leq  
        \left \|  
          \zeta^{-1}\nabla \varphi(\bar x) - H_\zeta(x_k - x^*)
        \right \| + \| \zeta^{-1} \nabla \varphi(\bar x) - \nabla \varphi(x_k) \| \\
        \nonumber
        & = 
        \zeta^{-1}\left \|  
          \nabla \varphi(\bar x) - \nabla\varphi(x^*) - H_\zeta(\bar x - x^*)
        \right \| + \| \zeta^{-1} \nabla \varphi(\bar x) - \nabla \varphi(x_k) \| \\
        \label{eqn:appendix/proof-newton-step-intermediate-1}
        &\leq \zeta^{-1} L_H \| \bar x - x^* \|^2
        + L_H r_k^2
        = (\zeta + 1)L_H r_k^2
        \leq 2L_H r_k^2
        \leq 2L_H \delta_1^{\frac{1}{2}} r_k^{\frac{3}{2}}.
    \end{align}
    We also note that by the definition $\delta_2^{\frac{1}{2}} \leq 1 / (2c_1)$.
    Hence, $1 - c_1 \delta_2^{\frac{1}{2}} \geq 1/2$ and for $x_k \in B_{\delta_2}(x^*)$, 
    \begin{align}
        \label{eqn:inexact-descent-direction-upper-bound}
        \| d_k \| &\leq \| x_k + d_k - x_* \| + \| x_k - x_* \| 
        \overset{\eqref{eqn:proof/newton-step-decay}}{\leq} 
        c_1 r_k^{\frac{3}{2}} +  r_k
        \leq (c_1\delta_2^{\frac{1}{2}} + 1) r_k 
        \leq 2 r_k
        , \\
        \label{eqn:inexact-descent-direction-lower-bound}
        \| d_k \| &\geq \| x_k - x_* \| - \| x_k + d_k - x_* \| 
        \overset{\eqref{eqn:proof/newton-step-decay}}{\geq} 
        r_k- c_1 r_k^{\frac{3}{2}}
        \geq (1 - c_1\delta_2^{\frac{1}{2}}) r_k
        \geq \frac{r_k}{2}
        .
    \end{align}
    Combining the above two inequalities, we find for $x_k \in B_{\delta_2}(x^*)$, 
    \begin{align}
        | (\nabla \varphi(x_k) - H_\zeta (x_k - x_*))^\top d_k |
        \overset{\eqref{eqn:appendix/proof-newton-step-intermediate-1}}&{\leq} 
        4L_H \delta_1^{\frac{1}{2}} r_k^{\frac{5}{2}},
        \\
        | (x_k + d_k - x_*)^\top H_\zeta(x_k - x_*) |
        \overset{\eqref{eqn:proof/newton-step-decay}}&{\leq}
        L_g c_1 r_k^{\frac{5}{2}}.
    \end{align}
    Since $\text{d\_type}_k = \texttt{SOL}$, then using \lemmaref{lem:capped-cg} and note that $\nabla^2\varphi(x_k) \succeq \frac{\alpha}{2}\Id$, we know
    \begin{align}
        \nonumber
        \nabla \varphi(x_k)^\top d_k
        \overset{\eqref{eqn:capped-cg-descent-direction}}&{=} 
        -d_k^\top (\nabla^2\varphi(x_k) + 2M_k^{\frac{1}{2}}\omega_k I) d_k
        \leq 
        -\frac{\alpha}{2} \|d_k\|^2  
        \overset{\eqref{eqn:inexact-descent-direction-lower-bound}}{\leq} 
        - \frac{\alpha}{8}r_k^2
        .
    \end{align}
    Substituting them back to \eqref{eqn:nonasymp-inexact-newton-line-search-final}, and note that $\mu \in (0, 1/2)$, we have for $x_k \in B_{\delta_2}(x^*)$, 
    \begin{align*}
        &\peq \varphi(x_k +d_k) - \varphi(x_k) - \mu \nabla \varphi(x_k)^\top d_k  \\
        &\leq 
        \left( \frac{1}{2} - \mu \right)
        \nabla \varphi(x_k)^\top d_k
        + \left (
            \varphi(x_k + d_k) - \varphi(x_k) - \frac{1}{2} \nabla\varphi(x_k)^\top d_k
            \right ) \\
        &\leq 
        -\left( \frac{1}{2} - \mu \right)
        \frac{\alpha}{8} r_k^2
        + \frac{1}{2}\left( 
            L_gc_1^2 \delta_1^{\frac{1}{2}}
            + L_g c_1
            + 4L_H \delta_1^{\frac{1}{2}}
        \right) r_k^{\frac{5}{2}}
        .
    \end{align*}
    We can see that the above term is negative as long as $r_k \leq \delta_2$, and therefore, the linesearch \eqref{eqn:smooth-line-search-sol} holds with $m_k = 0$.

    \paragraph{Step 3}
    We show that the trial step (i.e., the step with using $\omega_k^{\supsucc}$) is accepted.
    Since $\text{d\_type}_k = \texttt{SOL}$, then \texttt{NewtonStep} will not return a \texttt{FAIL} state, so it suffices to show $g_{k+\frac{1}{2}} = \| \nabla \varphi(x_k + d_k) \| \leq g_k$, 
    where $d_k$ is the direction generated by \texttt{NewtonStep} with $\omega = \omega_k^{\supsucc} \leq \sqrt{g_k}$.
    Then, by \lemmaref{lem:gradient-decay-of-newton-step} and \eqref{eqn:appendix/proof/basic-gradient-descent-of-newton} we have 
    $g_{x + \frac{1}{2}} \leq g_k$ for $x_k \in B_{\delta_3}(x^*)$.
\end{proof}

\subsection{Local rate boosting lemma}
\label{sec:appendix/local-rate-boosting}

In this section, we establish a generalized version of \lemmaref{lem:superlinear-rate-boosting} in \lemmaref{lem:appendix/superlinear-rate-boosting-generalized} and \corollaryref{cor:appendix/quadratic-rate-boosting}, 
which extends to the case of a $\nu$-H\"older continuous Hessian and reduces the Lipschitz Hessian in Assumption~\ref{assumption:liphess} when $\nu = 1$.
The results in \lemmaref{lem:appendix/superlinear-rate-boosting-generalized} primarily characterize the behavior for $\theta \in [0, \nu]$,
while the case of $\theta > \nu$ is analyzed separately in \corollaryref{cor:appendix/quadratic-rate-boosting}.
This division into two cases is mainly a technical necessity, 
as merging them could result in the preleading coefficient $c_k$ in \eqref{eqn:appendix/superlinear-rate-boosting-logc} becoming unbounded.


\begin{lemma}
    \label{lem:appendix/superlinear-rate-boosting-generalized}
    Let $\{ g_k \}_{k \geq 0} \subseteq (0, \infty)$, $c_0\geq 1$, $c \geq 1$, $1\ge \nu>0$, %
    $\nu_0 = \bar \nu := \frac{\nu}{1+\nu}$,
    and $\theta \geq 0$.
    If $\log g_1 \leq \log c_0 + (1 + \nu_0) \log g_0$ and 
    the following inequality holds for $k \geq 1$,
    \begin{equation}
    g_{k+1} \leq c g_k^{1 + \nu} + c g_k^{1 + \bar\nu} \frac{g_k^\theta}{g_{k-1}^\theta},
    \label{eqn:appendix/superlinear-boosting-inequality}
    \end{equation}
    and $g_0 \leq \min\big (1, (2c)^{-\frac{1}{\bar \nu}}, c_0^{-\frac{1}{\bar \nu}} \big )$, 
    then we have $g_{k + 1} \leq g_k$ and the following inequality holds for every $k\geq 0$:
    \begin{equation}
        \label{eqn:appendix/superlinear-rate-boosting}
    \log g_{k+1} \leq \log c_k +  (1 + \nu_k) \log g_k,
    \end{equation}
    where we define $\bar \theta = \min(\theta, \nu)$ and 
    $\nu_\infty = -\frac{1}{2}(1 - \bar\nu-\bar\theta) + \frac{1}{2}\sqrt{(1-\bar\nu-\bar\theta)^2+4\bar \nu} \in [\bar\nu,\nu]$ 
    is  the positive root of the equation $\bar\nu + \frac{\bar\theta\nu_\infty}{1 + \nu_\infty} = \nu_\infty$, and\footnote{We define $\nu_{-1} = 0$.}
    \begin{align}
        \label{eqn:appendix/superlinear-rate-boosting-logc}
        \log c_k &:= \log(2c) + \frac{\bar\theta}{1 + \nu_{k-1}} \log c_{k-1}
        \leq \left(1+\frac{1}{\bar\nu}\right) \log (2c)  + \log c_0
        , \\
        \label{eqn:appendix/superlinear-rate-boosting-nu}
        \nu_k &:= \min\left( \nu, \bar\nu + \frac{\bar \theta \nu_{k-1}}{1 + \nu_{k-1}} \right) 
        \geq \nu_\infty - \frac{\bar\theta^k(\nu_\infty - \bar\nu)}{(1+\bar\nu)^{2k}} 
        \geq \nu_\infty - \frac{\bar\theta^k}{(1+\bar\nu)^{2k}} .
    \end{align}
    In particular, when $\theta\geq\nu$, we have $\nu_\infty = \nu$ and $v_k \geq \nu - \frac{\nu^k(\nu-\bar\nu)}{(1 + \bar\nu)^{2k}}$.
\end{lemma}
\begin{proof}
    We first show that $\nu_\infty \in [\bar \nu, \nu]$. 
    Define the map $T(\alpha) = \bar \nu + \frac{\bar \theta \alpha}{1 + \alpha} - \alpha$ for $\alpha \in [\bar\nu, \nu]$.
    By reformulating it as $T(\alpha) = \bar \nu + \bar \theta + 1 - \left( \frac{\bar \theta}{1 + \alpha} + (1 + \alpha) \right)$, 
    we see that $T$ is strictly decreasing whenever $1 + \alpha \geq \sqrt{\bar \theta}$,
    which holds since $1 + \alpha \geq 1 + \bar\nu > 1 \geq \nu \geq \bar \theta$.
    Then, there exists a unique $\nu_\infty \in [\bar\nu, \nu]$ such that $T(\nu_\infty) = 0$
    because $T(\bar\nu) = \frac{\bar\theta\bar\nu}{1 + \bar\nu} \geq 0$ and $T(\nu) = \frac{\nu(\bar\theta-\nu)}{1 + \nu} \leq 0$.

    Let $\cI \subseteq \N$ be the set such that $k \in \cI$ if and only if
    \begin{align*}
        g_{k+1} &\leq g_{k}, c_k \geq 1, \nu_k \leq  \nu_\infty,
        \text{ and }
        \eqref{eqn:appendix/superlinear-rate-boosting},
        \eqref{eqn:appendix/superlinear-rate-boosting-nu}
        \text{ hold, } \\
       & \text{ and }
        \log c_k 
        \leq \frac{1 - (1 + \bar\nu)^{-k}}{1 - (1 + \bar\nu)^{-1}} \log (2c) + \log c_0.
    \end{align*}

    First, we show that $0 \in \cI$.
    Since $\nu_0 = \bar \nu$ and $g_0^{\bar \nu} \leq c_0^{-1}$, 
    we have $g_1 \leq c_0g_0^{1 + \bar \nu} \leq g_0$.
    The other parts hold by assumption, and we have used $\nu_\infty \geq \bar \nu$ and the definition that $\nu_{-1} = 0$ in \eqref{eqn:appendix/superlinear-rate-boosting-nu} for $k = 0$.

    Next, we prove $\cI = \N$ by induction.
    Suppose $0, \dots, j - 1 \in \cI$ for some $j \geq 1$, we will show that $j\in \cI$.
    Since $j-1\in\cI$, from \eqref{eqn:appendix/superlinear-rate-boosting} we have 
    $g_j \leq c_{j-1} g_{j-1}^{1 + \nu_{j-1}}$, 
    and equivalently,
    $g_{j-1}^{-1} \leq \left (c_{j-1}^{-1}g_{j}\right )^{-\frac{1}{1 + \nu_{j-1}}}$.
    Note that $c_{j - 1} \geq 1$ 
    and $g_j \leq g_{j-1}$,
    and 
    $\frac{g_j^\theta}{g_{j-1}^\theta} \leq \frac{g_j^{\bar\theta}}{g_{j-1}^{\bar\theta}}$ for $\theta \geq \bar \theta$,
    we have
    \begin{align*}
        g_{j+1}
        \overset{\eqref{eqn:appendix/superlinear-boosting-inequality}}&{\leq}
        c g_j^{1 + \nu} + c g_j^{1 + \bar\nu} \frac{g_j^{\bar\theta}}{g_{j-1}^{\bar\theta}}
        \leq 
        c g_j^{1 + \nu} + c c_{j-1}^{\frac{\bar\theta}{1 + \nu_{j-1}}} g_j^{1 + \bar\nu + \frac{\bar\theta\nu_{j-1}}{1 + \nu_{j-1}}} \\
        \overset{(c, c_{j-1} \geq 1)}&{\leq}
        2 c c_{j-1}^{\frac{\bar\theta}{1 + \nu_{j-1}}} 
        \max \left ( g_j^{1 + \nu}, g_j^{1 + \bar\nu + \frac{\bar\theta\nu_{j-1}}{1 + \nu_{j-1}}} \right )
        .
    \end{align*}
    Therefore, we find that
    \begin{align}
        \log g_{j+1}
        \leq 
        \underbrace{\log (2c) + \frac{\bar\theta}{1 + \nu_{j-1}} \log c_{j-1} }_{\log c_{j}}
        + 
        \underbrace{\min\left(1 + \nu,  1 + \bar\nu + \frac{\bar\theta\nu_{j-1}}{1 + \nu_{j-1}} \right)}_{1 + \nu_{j}} \log g_j.
        \label{eqn:appendix/proof-superlinear-boosting-recursive}
    \end{align}
    Thus, \eqref{eqn:appendix/superlinear-rate-boosting} holds for $k=j$, and $\log c_j \geq \log(2c) \geq \log 2 \geq 0$, i.e., $c_j \geq 1$.

    Since $[j-1]\subseteq\cI$, we know $\{g_i\}_{0 \leq i \leq j}$ is non-increasing, $g_{j}^{\bar \nu} \leq g_0^{\bar\nu}\leq (2c)^{-1}$, and $g_j\leq g_{j-1}$. 
    Note that $\bar \nu \leq \nu$ and $g_j \leq g_0 \leq 1$, 
    then 
    $g_{j+1} 
    \leq c g_j^{1 + \nu} + cg_j^{1+\bar\nu} (g_j g_{j-1}^{-1})^{\theta}
    \leq 2cg_j^{1+\bar\nu} \leq g_j$.

    By \eqref{eqn:appendix/superlinear-rate-boosting-nu}, $\nu_{j-1} \geq \min(\bar\nu, \nu) = \bar\nu$ and we have 
    \begin{align*}
        \log c_j 
        &\leq \log(2c) + \frac{\bar\theta}{1 + \bar\nu} \log c_{j-1} \\
        \overset{(\bar\theta \leq 1)}&{\leq} \log(2c) + \frac{1}{1 + \bar\nu} \left ( \frac{1 - (1 + \bar\nu)^{-(j-1)}}{1 - (1 + \bar\nu)^{-1}} \log (2c) + \log c_0 \right )
        \\
        &\leq \frac{1 - (1 + \bar\nu)^{-j}}{1 - (1 + \bar\nu)^{-1}} \log (2c) + \log c_0.
    \end{align*}

    Finally, we show $\nu_j \leq \nu_\infty$ and \eqref{eqn:appendix/superlinear-rate-boosting-nu} holds for $k=j$. 
    Define the map $F(\alpha) = \bar\nu + \frac{\bar\theta\alpha}{1 + \alpha}$.
    We know $F(\alpha)$ is non-decreasing for $\alpha > 0$,
    and $F(\nu_\infty) = \nu_\infty$ by its definition.
    Since  $\nu_{j-1} \leq \nu_\infty$ 
    and $F(\nu_{j-1}) \leq F(\nu_\infty) = \nu_\infty \leq \nu$, 
    then
    $\nu_j = \min(\nu, F(\nu_{j-1})) = F(\nu_{j-1})
   \leq \nu_\infty$.
    Moreover,  we have
    \begin{align*}
        0 \leq \nu_\infty - \nu_j
        &= F(\nu_\infty) - F(\nu_{j-1})
        = \frac{\bar\theta(\nu_\infty-\nu_{j-1})}{(1+\nu_\infty)(1 + \nu_{j-1})} \\
        &\leq \frac{\bar\theta(\nu_\infty-\nu_{j-1})}{(1+\bar \nu)^2}
        \leq \frac{\bar\theta^j(\nu_\infty-\bar\nu)}{(1+\bar \nu)^{2j}}
        ,
    \end{align*}
    where the last inequality follows from the induction assumption.

    
    Thus, we have $j \in \cI$ and by induction $\cI = \N$.
\end{proof}


\begin{corollary}
    \label{cor:appendix/quadratic-rate-boosting}
    Under the assumptions of \lemmaref{lem:appendix/superlinear-rate-boosting-generalized},
    if $\theta > \nu$ and $k \geq k_0 := 
    \frac{\log\frac{\theta-\nu\bar\nu}{\theta-\nu}-\log\nu}{2\log(1+\bar\nu)-\log\nu} + 1$, 
    then $g_k$ converges superlinearly with order $1 + \nu$:
    \begin{align}
        \log g_{k}
        \leq
        \left( 1 + \theta + \frac{1}{\bar\nu} \right)\log (2c) + \theta  \log c_0
        + (1 + \nu) \log g_{k-1}.
    \end{align}
\end{corollary}
\begin{proof}
    Since the assumptions are the same as those in \lemmaref{lem:appendix/superlinear-rate-boosting-generalized}, the results therein are all valid.
    Furthermore, we note that in the proof of \lemmaref{lem:appendix/superlinear-rate-boosting-generalized}, 
    the following stronger variant of \eqref{eqn:appendix/proof-superlinear-boosting-recursive} can be obtained from \eqref{eqn:appendix/superlinear-boosting-inequality}:
    \begin{align}
        \log g_{j+1}
        \leq 
        \underbrace{\log (2c) + \frac{\theta}{1 + \nu_{j-1}} \log c_{j-1}}_{\hat c_j}
        + 
        \underbrace{\min\left(1 + \nu,  1 + \bar\nu + \frac{\theta\nu_{j-1}}{1 + \nu_{j-1}} \right)}_{1 + \hat \nu_j} \log g_j.
        \label{eqn:appendix/proof-superlinear-boosting-recursive-strong}
    \end{align}
    Let $\alpha = \left (\frac{\theta}{\nu - \bar \nu} - 1\right )^{-1} = \left (\frac{\theta}{\nu\bar\nu} - 1\right )^{-1}$.
    Since $\theta > \nu$, 
    then $\alpha > 0$ and $\frac{1}{\alpha} = \frac{\theta}{\nu\bar\nu} - 1 > \frac{1}{\bar\nu} - 1 = \frac{1}{\nu}$, i.e., $\alpha \in (0, \nu)$.
    When $\nu_{k-1} \geq \alpha$, we have
    \begin{align*}
       \hat \nu_{k} &=
       \min\left( \nu, \bar\nu + \frac{\theta\nu_{k-1}}{1+\nu_{k-1}} \right)
        =
        \min\left( \nu, \bar\nu + \frac{\theta}{\nu_{k-1}^{-1}+1} \right) \\
        &\geq
        \min\left( \nu, \bar\nu + \frac{\theta}{\alpha^{-1}+1} \right)
        =\nu.
    \end{align*}
    From \lemmaref{lem:appendix/superlinear-rate-boosting-generalized},
    we know $\nu_\infty = \nu$, and when $k - 1 \geq k_0 - 1 \geq \log_{\frac{\nu}{(1 + \bar\nu)^2}}(\nu - \alpha) = \frac{-\log(\nu-\alpha)}{2\log(1+\bar\nu)-\log\nu}$, the following inequality holds since $\nu \in (0, 1]$ and $1 + \bar\nu > 1$.
    \begin{align*}
       \nu_{k-1} \overset{\eqref{eqn:appendix/superlinear-rate-boosting-nu}}{\geq}
       \nu - \frac{\nu^{k-1}(\nu-\bar\nu)}{(1+\bar\nu)^{2(k-1)}}
       \geq \nu - \frac{\nu^{k-1}}{(1+\bar\nu)^{2(k-1)}}
       \geq \alpha.
    \end{align*}
    Thus, for any $k \geq k_0$, we have $\hat \nu_j = \nu$, and 
    \begin{align*}
        \log g_{k}
        \overset{\eqref{eqn:appendix/proof-superlinear-boosting-recursive-strong}}&{\leq}
        \log (2c) + \theta \log c_{k-1}
        + (1 + \nu) \log g_{k-1} \\
        \overset{\eqref{eqn:appendix/superlinear-rate-boosting-logc}}&{\leq}
        \left( 1 + \theta + \frac{1}{\bar\nu} \right)\log (2c) + \theta  \log c_0
        + (1 + \nu) \log g_{k-1}.
    \end{align*}
    Finally, the proof is completed by noticing that 
    $\nu - \alpha = \nu - \frac{\nu\bar\nu}{\theta-\nu\bar\nu} = \frac{\nu(\theta-\nu)}{\theta-\nu\bar\nu}$.
\end{proof}




\section{Additional numerical results} \label{sec:appendix/numerical-results}


This section provides a detailed description of the experimental setup and additional results to supplement \Cref{sec:main/numerical}.
We implement our algorithm in MATLAB R2023a 
and denote the variant using the first regularizer in \theoremref{thm:newton-local-rate-boosted} as \algname{ARNCG}$_g$,
and the variant using the second regularizer as \algname{ARNCG}$_\epsilon$. 
We use the official Julia implementation provided by \citet{hamad2024simple} for their \algname{CAT}.\footnote{See \url{https://github.com/fadihamad94/CAT-Journal}.}
As the code for \algname{AN2CER} is not publicly available, we investigate several ways to implement it in MATLAB and report the best results, as detailed in \Cref{sec:appendix/implementation-details}.

Our experimental settings follow those described by \citet{hamad2024simple}, we conduct all experiments in a single-threaded environment on a machine running Ubuntu Server 22.04, equipped with dual-socket Intel(R) Xeon(R) Silver 4210 CPUs 
and 192 GB of RAM.
Each socket is installed with three 32 GB RAM modules, running at 2400 MHz.
The algorithm is considered successful if it terminates when $\epsilon_k \leq  \epsilon =  10^{-5}$ such that $k \leq 10^5$. If the algorithm fails to terminate within 5 hours, it is also recorded as a failure.

We evaluate these algorithms using the standard CUTEst benchmark for nonlinear optimization~\citep{gould2015cutest}.
Specifically, we consider all unconstrained problems with more than 100 variables that are commonly available through the Julia and MATLAB interfaces%
\footnote{See \url{https://github.com/JuliaSmoothOptimizers/CUTEst.jl} for the Julia interface, and \url{https://github.com/matcutest/matcutest} for the MATLAB interface.}
of this benchmark,
comprising a total of 124 problems.
The dimensions of these problems range from 100 to 123200.

\subsection{Implementation details} \label{sec:appendix/implementation-details}
\paragraph{\algname{ARNCG}}
The initial point for each problem is provided by the benchmark itself.
Other parameters of \Cref{alg:adap-newton-cg} are set as follows:
\begin{align*}
\mu = 0.3,
\beta = 0.5,
\tau_- = 0.3, 
\tau = \tau_+ = 1.0,
\gamma = 5,
M_0 = 1 
\text{ and } 
\eta = 0.01.
\end{align*}
We consider two choices for $m_{\mathrm{max}}$:
\begin{enumerate}
    \item Setting $m_{\mathrm{max}} = 1$ so that at most 4 function evaluations per each iteration.
    \item Setting $m_{\mathrm{max}} = \lfloor \log_\beta 10^{-8} \rfloor$ to be the smallest integer such that $\beta^{m_{\mathrm{max}}+1} > 10^{-8}$.
\end{enumerate}
In our experiments, we find that $m_{\mathrm{max}} = 1$ works well, 
and the algorithm is not sensitive to the above parameters, so we do not perform further fine-tuning.
In the implementation of \texttt{CappedCG}, we do not keep the historical iterations to save memory.
Instead, we evaluate \eqref{eqn:capped-cg-slow-decay-condition} by regenerating the iterations. 
In practice, we observe that step \eqref{eqn:capped-cg-slow-decay-condition} is triggered very infrequently, resulting in minimal computational overhead.
The \texttt{TERM} state is primarily designed to ensure theoretical guarantees for Hessian-vector products in \theoremref{sec:appendix/oracle-complexity-proof}, 
and we find it is not triggered in practice. 
Since the termination condition of \texttt{CappedCG} using the error $\|r_k\| \leq \hat \xi \|r_0\|$ may not be appropriate for a large $\|r_0\|$, 
we instead require it to satisfy  $\|r_k\| \leq \min(\hat \xi \|r_0\|, 0.01)$.

The fallback step in the main loop of \Cref{alg:adap-newton-cg} is mainly designed for theoretical considerations, as described in \lemmaref{lem:main/transition-between-subsequences-give-valid-regularizer}.
It ensures that an abrupt increase in the gradient norm followed by a sudden drop does not compromise the validity of this lemma
but results in a wasted iteration.
However, we note that this condition can be relaxed to the following to enhance practical performance:
\begin{align}
    \label{eqn:appendix/fallback-relaxed}
    \lambda g_{k + \frac{1}{2}} >  g_k
    \text{ and }
    g_k \leq \lambda g_{k-1}, \text{ for } \lambda \in (0, 1].
\end{align}
When $\lambda = 1$,
this condition reduces to the original one. In our experiments, we explore the choices of 
$\lambda = 1$, $\lambda = 0.01$, and the impact of removing the fallback step (i.e., $\lambda = 0$).
Moreover, we note that when $\theta = 0$, the fallback step and the trial step are identical so the choices of $\lambda$ do not affect the results.
In practice, we suggest setting a small $\lambda$ or removing the fallback step. 

We also terminate the algorithm and \emph{mark it as a failure} if both the function value and gradient norm remain unchanged for 20 iterations 
or if the current search direction satisfies $\| d_k \| \leq 2 \times 10^{-16}$, 
or if the Lipschitz constant estimation satisfies $M_k \geq 10^{40}$,
as these scenarios may indicate numerical issues.
\figureref{fig:main-algoperf} in the main text is generated under the above settings with $\lambda = 0$ and $m_{\mathrm{max}} = 1$.

For the Hessian evaluations, we only access it through the Hessian-vector products, 
and count the evaluation number as the number of iterations minus the number of the linesearch failures.
Since when a linesearch failure occurs, the next point is the same as the current point and does not increase the oracle complexity of Hessian evaluations.

\paragraph{\algname{AN2CER}}
Our implementation follows the algorithm described in \citet[Section~2]{gratton2024yet}, with parameters adopted from their suggested values.  
The algorithm first attempts to solve the regularized Newton equation using the regularizer $\sqrt{\kappa_a M_k g_k}$.
If this attempt fails, the minimal eigenvalue $\lambda_{\mathrm{min}}(\nabla^2\varphi(x_k))$ is computed.  
The algorithm then switches to the regularizer $\sqrt{M_k g_k} + [-\lambda_{\mathrm{min}}(\nabla^2\varphi(x_k))]_+$ when $\lambda_{\mathrm{min}}(\nabla^2\varphi(x_k)) > \kappa_C \sqrt{M_k g_k}$, and directly uses the corresponding eigenvector otherwise.

In AN2CER, the authors suggest using Cholesky factorization to solve the Newton equation and invoking the full eigendecomposition (i.e., the \texttt{eig} function in MATLAB) to find the minimal eigenvalue when the factorization fails.  
We observe that, in the current benchmark, it is more efficient to use \texttt{CappedCG} as the equation solver and compute the minimal eigenvalue using MATLAB's \texttt{eigs} function when \texttt{NC} is returned.  
This modification preserves the success rate and oracle evaluations of the original implementation while significantly reducing computational cost.  
We also note that there are several variants of AN2CER in \citet{gratton2024yet}, and we find that the current version yields the best results among them.




\subsection{Results on the CUTEst benchmark}



Following \citet{hamad2024simple}, we report the shifted geometric mean%
\footnote{For a dataset $\{ a_i \}_{i\in[k]}$, the shifted geometric mean is defined as  $\exp\left( \frac{1}{k} \sum_{i=1}^k \log (a_i + 1)  \right)$, which accounts for cases where $a_i = 0$.} 
of Hessian, gradient and function evaluations, as well as the elapsed time in \tableref{tab:appendix-comparision-fallback,tab:appendix-comparision-theta}.
In our algorithm, we define normalized Hessian-vector products as the original products divided by the problem dimension $n$,
which can be interpreted as the fraction of information about the Hessian that is revealed to the algorithm;
the linesearch failure rate is the fraction of iterations that exceed the maximum allowed steps $m_{\mathrm{max}}$; 
and the second linesearch rate measures the fraction of times the linesearch rule \eqref{eqn:newton-cg-sol-decay-smaller-stepsize} is invoked.
The medians of these metrics are provided in \tableref{tab:appendix-comparision-fallback-median,tab:appendix-comparision-theta-median}.
The success rate as a function of oracle evaluations is plotted in \figureref{fig:appendix-comparision-fallback,fig:appendix-comparision-theta}.
When an algorithm fails, the elapsed time is recorded as twice the time limit (i.e., 10 hours), and the oracle evaluations are recorded as twice the iteration limit (i.e., $2 \times 10^5$).
We note that the choices for handling failure cases in the reported metrics of these tables may affect the relative comparison of results with different success rates,
although they follow the convention from previous works.
Therefore, we suggest that readers also focus on the figures for a detailed analysis of each algorithm's behavior.


\paragraph{The fallback parameter}
From \tableref{tab:appendix-comparision-fallback,tab:appendix-comparision-fallback-median} and \figureref{fig:appendix-comparision-fallback}, 
we observe that the choice of the fallback parameter $\lambda$ in \eqref{eqn:appendix/fallback-relaxed} 
does not significantly affect the success rate, 
and the overall performance remains similar across different values of $\lambda$.  
For larger $\lambda$, the fallback step is generally triggered more frequently (as indicated by the ``fallback rate''), leading to increased computational time and oracle evaluations.
Interestingly, ARNCG$_\epsilon$ with $m_{\mathrm{max}} = 1$ seems an exception that $\lambda = 1$ is beneficial for specific problems and gives a slightly higher success rate.

\paragraph{The regularization coefficients}
\tableref{tab:appendix-comparision-theta,tab:appendix-comparision-theta-median} and \figureref{fig:appendix-comparision-theta} present comparisons for different values of $\theta$.  
As $\theta$ increases, the performance initially improves but then declines.
Larger $\theta$ imposes stricter tolerance requirements on \texttt{CappedCG} (as indicated by the number of Hessian-vector products in these tables), 
and increases computational costs,
while smaller $\theta$ may lead to a slower local convergence.
Thus, we recommend choosing $\theta \in [0.5, 1]$ to balance computational efficiency and local behavior.

We also note that this tolerance requirement is designed for local convergence and is not necessary for global complexity,
so there may be room for improvement.
For example, we can use a fixed tolerance $\eta$ when the current gradient norm is larger than a threshold, and switch to the current choice $\min( \eta, \sqrt{M_k} \omega_k)$ otherwise.
We leave this for future exploration.

Although ARNCG$_g$ has a slightly higher worst-case complexity (by a double-logarithmic factor) than ARNCG$_\epsilon$, 
they exhibit similar empirical performance, and in some cases, ARNCG$_g$ even performs better.  

A potential failure case \emph{in practice} for ARNCG$_\epsilon$ occurs when the iteration enters a neighborhood with a small gradient norm and then escapes via a negative curvature direction.  
Consequently, $\epsilon_k$ stays small while $g_k$ may grow large, making the method resemble the fixed $\epsilon$ scenario.  
Interestingly, 
this same condition is also what introduces the logarithmic factor in
ARNCG$_g$ \emph{theoretically}.

\paragraph{The linesearch parameter}
Since our algorithm relies on a linesearch step, it requires more function evaluations than CAT for large $m_{\mathrm{max}}$.  
If evaluating the target function is expensive, we may need to set a small $m_{\mathrm{max}}$, or even $m_{\mathrm{max}} = 0$.  
Under the latter case, at most two tests of the line search criteria are performed, and the parameter $M_k$ is increased when these tests fail.  
Our theory guarantees that $M_k = O(L_H)$, so this choice remains valid.  
In practice, we observe that using a relatively small $m_{\mathrm{max}}$ gives better results.




\paragraph{Case studies for local behavior}
We present two benchmark problems that exhibit superlinear local convergence behavior.  
As illustrated in \figureref{fig:appendix-comparision-local}, a larger $\theta$ gives faster local convergence.
We only show the algorithm using the second regularizer in this figure, and note that the two regularizers have a similar behavior since in the local regime they reduce to $g_k^{\frac{1}{2} + \theta} g_{k-1}^{-\theta}$, as shown in the last paragraph of the proof of \propositionref{prop:mixed-newton-nonconvex-phase-local-rates}.
Generally, it is hard to identify when the algorithm enters the neighborhood for superlinear convergence.
For \texttt{HIMMELBG}, the algorithm appears to be initialized near the local regime. 
For \texttt{ROSENBR}, the algorithm enters the local regime after approximately 20 iterations.

\begin{figure}[!tbp]
    \centering
    \includegraphics{figures/single_HIMMELBG.pdf} \hfill
    \includegraphics{figures/single_ROSENBR.pdf} \\
    \caption{
        Illustration of the local behavior of our method on the \texttt{HIMMELBG} (left plot) and \texttt{ROSENBR} (right plot) problems from the CUTEst benchmark for $\lambda=0$ and $m_{\mathrm{max}} = 1$.
        All methods converge to the same point. 
        }
    \label{fig:appendix-comparision-local}
\end{figure}

\begin{figure}[!tbp]
    \centering
    \includegraphics{figures/fallback-ss1-gg1.0-time.pdf} \hfill
    \includegraphics{figures/fallback-ss1-gg1.0-hesseval.pdf} \\
    \vspace{1em}
    \includegraphics{figures/fallback-ss1-gg1.0-gradeval.pdf} \hfill
    \includegraphics{figures/fallback-ss1-gg1.0-funceval.pdf} 
    \caption{
        Comparison of success rates as functions of elapsed time, Hessian evaluations, gradient evaluations and function evaluations for solving problems in the CUTEst benchmark.
        The fallback parameter $\lambda$ in \eqref{eqn:appendix/fallback-relaxed} varies, and $m_{\mathrm{max}} = 1$.
        }
    \label{fig:appendix-comparision-fallback}
\end{figure}

\begin{table}[!tbp]
    \caption{
        Shifted geometric mean of the relevant metrics for different methods in the CUTEst benchmark.
        The fallback, second linesearch and linesearch failure rates are reported as mean values.
        The fallback parameter $\lambda$ in \eqref{eqn:appendix/fallback-relaxed} varies.
        }
    \resizebox{\columnwidth}{!}{%
    \begin{tabular}{cccccccccc}  \toprule
        & \textbf{\Centerstack{Elapsed   \\Time (s)}} & \textbf{\Centerstack{Hessian   \\Evaluations}} & \textbf{\Centerstack{Gradient  \\Evaluations}} & \textbf{\Centerstack{Function  \\Evaluations}} & \textbf{\Centerstack{Hessian-vector \\Products \\(normalzied)}} & \textbf{\Centerstack{Success \\Rate (\%)}} & \textbf{\Centerstack{Linesearch\\Failure \\Rate (\%)}} & \textbf{\Centerstack{Second\\Linesearch \\Rate (\%)}} & \textbf{\Centerstack{Fallback \\Rate (\%)}}\\ \midrule 
AN2CER                                             & 36.70 & 170.10 & 172.02 & 176.80 & 31.38 & 81.45 & N/A & N/A & N/A \\ 
CAT                                                & 23.34 & 88.47 & 96.61 & 125.56 & N/A & 85.48 & N/A & N/A & N/A\\ \midrule \multicolumn{10}{c}{ \textbf{Results for $m_{\mathrm{max}} = 1$ and $\theta = 1.0$  }  } \\ \midrule
ARNCG$_g$ ($\lambda = 0.00$)                       & 16.71 & 80.86 & 86.41 & 119.51 & 13.77 & 87.10 & 16.08 & 1.38 & 0.00 \\ 
ARNCG$_g$ ($\lambda = 0.01$)                       & 17.01 & 81.46 & 87.31 & 120.48 & 13.90 & 87.10 & 15.98 & 1.31 & 0.33 \\ 
ARNCG$_g$ ($\lambda = 1.00$)                       & 19.02 & 85.61 & 99.01 & 130.91 & 14.84 & 87.10 & 14.52 & 0.17 & 7.43 \\ \midrule
ARNCG$_\epsilon$ ($\lambda = 0.00$)                & 18.28 & 85.03 & 90.78 & 125.29 & 14.91 & 86.29 & 16.89 & 0.43 & 0.00 \\ 
ARNCG$_\epsilon$ ($\lambda = 0.01$)                & 18.39 & 85.03 & 90.78 & 125.29 & 14.91 & 86.29 & 16.89 & 0.43 & 0.00 \\ 
ARNCG$_\epsilon$ ($\lambda = 1.00$)                & 18.04 & 78.40 & 89.41 & 122.41 & 14.22 & 87.10 & 16.03 & 0.46 & 6.10
%
       \\ \midrule \multicolumn{10}{c}{ \textbf{Results for $m_{\mathrm{max}} = \lfloor\log_\beta 10^{-8}\rfloor$ and $\theta = 1.0$  }  } \\ \midrule
ARNCG$_g$ ($\lambda = 0.00$)                       & 22.89 & 113.82 & 121.08 & 184.09 & 19.14 & 83.87 & 0.08 & 0.00 & 0.00 \\ 
ARNCG$_g$ ($\lambda = 0.01$)                       & 23.81 & 117.02 & 125.50 & 189.01 & 19.77 & 83.87 & 0.08 & 0.00 & 0.90 \\ 
ARNCG$_g$ ($\lambda = 1.00$)                       & 26.68 & 125.53 & 147.89 & 218.05 & 22.53 & 83.87 & 0.08 & 0.00 & 11.43 \\ \midrule
ARNCG$_\epsilon$ ($\lambda = 0.00$)                & 22.58 & 105.95 & 112.68 & 176.50 & 17.81 & 84.68 & 0.10 & 0.00 & 0.00 \\ 
ARNCG$_\epsilon$ ($\lambda = 0.01$)                & 22.47 & 105.95 & 112.68 & 176.50 & 17.81 & 84.68 & 0.10 & 0.00 & 0.00 \\ 
ARNCG$_\epsilon$ ($\lambda = 1.00$)                & 25.80 & 118.41 & 137.31 & 214.58 & 20.79 & 83.06 & 0.29 & 0.00 & 9.94 \\ \bottomrule
%
    \end{tabular}%
    }
    \label{tab:appendix-comparision-fallback}
\end{table}
\begin{table}[!tbp]
    \caption{
        Median of the relevant metrics for different methods in the CUTEst benchmark.
        The fallback parameter $\lambda$ in \eqref{eqn:appendix/fallback-relaxed} varies.
        }
    \resizebox{\columnwidth}{!}{%
    \begin{tabular}{cccccccccc}  \toprule%
        & \textbf{\Centerstack{Elapsed   \\Time (s)}} & \textbf{\Centerstack{Hessian   \\Evaluations}} & \textbf{\Centerstack{Gradient  \\Evaluations}} & \textbf{\Centerstack{Function  \\Evaluations}} & \textbf{\Centerstack{Hessian-vector \\Products \\(normalzied)}} & \textbf{\Centerstack{Success \\Rate (\%)}} & \textbf{\Centerstack{Linesearch\\Failure \\Rate (\%)}} & \textbf{\Centerstack{Second\\Linesearch \\Rate (\%)}} & \textbf{\Centerstack{Fallback \\Rate (\%)}}\\ \midrule 
AN2CER                                             & 4.75 & 30.00 & 30.00 & 30.00 & 4.24 & 81.45 & N/A & N/A & N/A \\ 
CAT                                                & 2.13 & 21.00 & 22.00 & 34.50 & N/A & 85.48 & N/A & N/A & N/A\\ \midrule \multicolumn{10}{c}{ \textbf{Results for $m_{\mathrm{max}} = 1$ and $\theta = 1.0$  }  } \\ \midrule
ARNCG$_g$ ($\lambda = 0.00$)                       & 1.89 & 20.50 & 21.50 & 35.50 & 1.52 & 87.10 & 10.82 & 0.00 & 0.00 \\ 
ARNCG$_g$ ($\lambda = 0.01$)                       & 2.00 & 20.50 & 21.50 & 35.50 & 1.52 & 87.10 & 10.70 & 0.00 & 0.00 \\ 
ARNCG$_g$ ($\lambda = 1.00$)                       & 2.12 & 22.00 & 25.50 & 40.00 & 1.92 & 87.10 & 6.75 & 0.00 & 0.00 \\ \midrule
ARNCG$_\epsilon$ ($\lambda = 0.00$)                & 1.72 & 21.50 & 22.50 & 38.00 & 1.62 & 86.29 & 10.26 & 0.00 & 0.00 \\ 
ARNCG$_\epsilon$ ($\lambda = 0.01$)                & 1.86 & 21.50 & 22.50 & 38.00 & 1.62 & 86.29 & 10.26 & 0.00 & 0.00 \\ 
ARNCG$_\epsilon$ ($\lambda = 1.00$)                & 1.99 & 21.00 & 24.50 & 38.00 & 2.01 & 87.10 & 9.92 & 0.00 & 0.00
%
       \\ \midrule \multicolumn{10}{c}{ \textbf{Results for $m_{\mathrm{max}} = \lfloor\log_\beta 10^{-8}\rfloor$ and $\theta = 1.0$  }  } \\ \midrule
ARNCG$_g$ ($\lambda = 0.00$)                       & 2.84 & 25.00 & 26.00 & 53.00 & 2.13 & 83.87 & 0.00 & 0.00 & 0.00 \\ 
ARNCG$_g$ ($\lambda = 0.01$)                       & 2.89 & 25.00 & 26.00 & 53.00 & 2.34 & 83.87 & 0.00 & 0.00 & 0.00 \\ 
ARNCG$_g$ ($\lambda = 1.00$)                       & 3.28 & 24.00 & 30.50 & 61.50 & 2.34 & 83.87 & 0.00 & 0.00 & 9.09 \\ \midrule
ARNCG$_\epsilon$ ($\lambda = 0.00$)                & 2.49 & 26.00 & 27.00 & 55.50 & 1.40 & 84.68 & 0.00 & 0.00 & 0.00 \\ 
ARNCG$_\epsilon$ ($\lambda = 0.01$)                & 2.44 & 26.00 & 27.00 & 55.50 & 1.40 & 84.68 & 0.00 & 0.00 & 0.00 \\ 
ARNCG$_\epsilon$ ($\lambda = 1.00$)                & 2.90 & 25.00 & 30.50 & 69.00 & 1.68 & 83.06 & 0.00 & 0.00 & 8.33 \\ \bottomrule
%
    \end{tabular}%
    }
    \label{tab:appendix-comparision-fallback-median}
\end{table}


\begin{figure}[!tbp]
    \centering
    \includegraphics{figures/theta-ss1-gg-time.pdf} \hfill
    \includegraphics{figures/theta-ss1-gg-hesseval.pdf} \\
    \vspace{1em}
    \includegraphics{figures/theta-ss1-gg-gradeval.pdf} \hfill
    \includegraphics{figures/theta-ss1-gg-funceval.pdf}
    \caption{
        Comparison of success rates as functions of elapsed time, Hessian evaluations, gradient evaluations and function evaluations for solving problems in the CUTEst benchmark.
        The parameter $\theta$ in \theoremref{thm:newton-local-rate-boosted} varies, and the fallback step is removed, i.e., $\lambda = 0$ in \eqref{eqn:appendix/fallback-relaxed}, and $m_{\mathrm{max}} = 1$.
        }
    \label{fig:appendix-comparision-theta}
\end{figure}



\begin{table}[!tbp]
    \caption{
        Shifted geometric mean of the relevant metrics for different methods in the CUTEst benchmark.
        The linesearch failure rate is reported as mean values.
        The parameter $\theta$ in \theoremref{thm:newton-local-rate-boosted} and the linesearch parameter $m_{\mathrm{max}}$ vary, and $\lambda = 0$.
        }
    \resizebox{\columnwidth}{!}{%
    \begin{tabular}{ccccccccc}  \toprule%
        & \textbf{\Centerstack{Elapsed   \\Time (s)}} & \textbf{\Centerstack{Hessian   \\Evaluations}} & \textbf{\Centerstack{Gradient  \\Evaluations}} & \textbf{\Centerstack{Function  \\Evaluations}} & \textbf{\Centerstack{Hessian-vector \\Products \\(normalzied)}} & \textbf{\Centerstack{Success \\Rate (\%)}} & \textbf{\Centerstack{Linesearch\\Failure \\Rate (\%)}} & \textbf{\Centerstack{Second\\Linesearch \\Rate (\%)}}\\ \midrule 
AN2CER                                             & 36.70 & 170.10 & 172.02 & 176.80 & 31.38 & 81.45 & N/A & N/A \\ 
CAT                                                & 23.34 & 88.47 & 96.61 & 125.56 & N/A & 85.48 & N/A & N/A\\ \midrule \multicolumn{9}{c}{ \textbf{Results for $m_{\mathrm{max}} = 1$ and $\lambda = 0$  }  } \\ \midrule
Fixed ($\omega_k = \sqrt{\epsilon}$)               & 48.10 & 215.60 & 228.47 & 386.84 & 43.97 & 80.65 & 26.12 & 4.73 \\ \midrule
ARNCG$_g$ ($\theta = 0.0$)                         & 21.58 & 111.12 & 117.85 & 151.15 & 17.73 & 84.68 & 13.78 & 0.00 \\ 
ARNCG$_g$ ($\theta = 0.5$)                         & 18.62 & 87.10 & 92.89 & 126.92 & 14.85 & 86.29 & 15.48 & 1.31 \\ 
ARNCG$_g$ ($\theta = 1.0$)                         & 16.71 & 80.86 & 86.41 & 119.51 & 13.77 & 87.10 & 16.08 & 1.38 \\ 
ARNCG$_g$ ($\theta = 1.5$)                         & 19.22 & 87.83 & 93.84 & 129.00 & 15.29 & 86.29 & 15.38 & 1.58 \\ \midrule
ARNCG$_\epsilon$ ($\theta = 0.0$)                  & 18.39 & 90.95 & 96.67 & 129.71 & 15.28 & 85.48 & 15.49 & 0.50 \\ 
ARNCG$_\epsilon$ ($\theta = 0.5$)                  & 18.84 & 90.44 & 96.42 & 129.85 & 15.73 & 85.48 & 15.69 & 0.31 \\ 
ARNCG$_\epsilon$ ($\theta = 1.0$)                  & 18.28 & 85.03 & 90.78 & 125.29 & 14.91 & 86.29 & 16.89 & 0.43 \\ 
ARNCG$_\epsilon$ ($\theta = 1.5$)                  & 22.65 & 104.83 & 111.81 & 151.03 & 18.83 & 83.87 & 16.05 & 0.42
%
       \\ \midrule \multicolumn{9}{c}{ \textbf{Results for $m_{\mathrm{max}} = \lfloor\log_\beta 10^{-8}\rfloor$ and $\lambda = 0$  }  } \\ \midrule
Fixed ($\omega_k = \sqrt{\epsilon}$)               & 47.74 & 227.08 & 240.79 & 842.35 & 46.47 & 80.65 & 13.29 & 0.00 \\ \midrule
ARNCG$_g$ ($\theta = 0.0$)                         & 27.64 & 143.93 & 152.15 & 213.62 & 23.10 & 83.06 & 0.13 & 0.00 \\ 
ARNCG$_g$ ($\theta = 0.5$)                         & 21.20 & 101.86 & 108.25 & 167.06 & 15.96 & 85.48 & 0.15 & 0.00 \\ 
ARNCG$_g$ ($\theta = 1.0$)                         & 22.89 & 113.82 & 121.08 & 184.09 & 19.14 & 83.87 & 0.08 & 0.00 \\ 
ARNCG$_g$ ($\theta = 1.5$)                         & 22.36 & 109.75 & 116.82 & 185.25 & 18.60 & 84.68 & 0.09 & 0.00 \\ \midrule
ARNCG$_\epsilon$ ($\theta = 0.0$)                  & 22.09 & 113.33 & 120.03 & 179.29 & 18.35 & 83.87 & 0.09 & 0.00 \\ 
ARNCG$_\epsilon$ ($\theta = 0.5$)                  & 23.12 & 115.58 & 122.82 & 184.87 & 19.58 & 83.06 & 0.12 & 0.00 \\ 
ARNCG$_\epsilon$ ($\theta = 1.0$)                  & 22.58 & 105.95 & 112.68 & 176.50 & 17.81 & 84.68 & 0.10 & 0.00 \\ 
ARNCG$_\epsilon$ ($\theta = 1.5$)                  & 23.11 & 113.74 & 121.11 & 187.25 & 20.20 & 83.06 & 0.10 & 0.00 \\ \bottomrule
%
    \end{tabular}%
    }
    \label{tab:appendix-comparision-theta}
\end{table}
\begin{table}[!tbp]
    \caption{
        Median of the relevant metrics for different methods in the CUTEst benchmark.
        The parameter $\theta$ in \theoremref{thm:newton-local-rate-boosted} and the linesearch parameter $m_{\mathrm{max}}$ vary, and $\lambda = 0$.
        }
    \resizebox{\columnwidth}{!}{%
    \begin{tabular}{ccccccccc}  \toprule%
        & \textbf{\Centerstack{Elapsed   \\Time (s)}} & \textbf{\Centerstack{Hessian   \\Evaluations}} & \textbf{\Centerstack{Gradient  \\Evaluations}} & \textbf{\Centerstack{Function  \\Evaluations}} & \textbf{\Centerstack{Hessian-vector \\Products \\(normalzied)}} & \textbf{\Centerstack{Success \\Rate (\%)}} & \textbf{\Centerstack{Linesearch\\Failure \\Rate (\%)}} & \textbf{\Centerstack{Second\\Linesearch \\Rate (\%)}}\\ \midrule 
AN2CER                                             & 4.75 & 30.00 & 30.00 & 30.00 & 4.24 & 81.45 & N/A & N/A \\ 
CAT                                                & 2.13 & 21.00 & 22.00 & 34.50 & N/A & 85.48 & N/A & N/A\\ \midrule \multicolumn{9}{c}{ \textbf{Results for $m_{\mathrm{max}} = 1$ and $\lambda = 0$  }  } \\ \midrule
Fixed ($\omega_k = \sqrt{\epsilon}$)               & 10.75 & 36.50 & 37.50 & 90.00 & 7.29 & 80.65 & 33.16 & 0.00 \\ \midrule
ARNCG$_g$ ($\theta = 0.0$)                         & 2.04 & 22.50 & 23.50 & 37.00 & 1.52 & 84.68 & 1.72 & 0.00 \\ 
ARNCG$_g$ ($\theta = 0.5$)                         & 1.77 & 20.00 & 21.00 & 34.00 & 1.52 & 86.29 & 9.52 & 0.00 \\ 
ARNCG$_g$ ($\theta = 1.0$)                         & 1.89 & 20.50 & 21.50 & 35.50 & 1.52 & 87.10 & 10.82 & 0.00 \\ 
ARNCG$_g$ ($\theta = 1.5$)                         & 2.46 & 22.00 & 23.00 & 38.00 & 1.72 & 86.29 & 10.00 & 0.00 \\ \midrule
ARNCG$_\epsilon$ ($\theta = 0.0$)                  & 1.81 & 20.00 & 21.00 & 35.00 & 1.61 & 85.48 & 3.65 & 0.00 \\ 
ARNCG$_\epsilon$ ($\theta = 0.5$)                  & 1.91 & 20.00 & 21.00 & 35.00 & 1.74 & 85.48 & 7.12 & 0.00 \\ 
ARNCG$_\epsilon$ ($\theta = 1.0$)                  & 1.72 & 21.50 & 22.50 & 38.00 & 1.62 & 86.29 & 10.26 & 0.00 \\ 
ARNCG$_\epsilon$ ($\theta = 1.5$)                  & 1.95 & 22.00 & 23.00 & 40.50 & 1.93 & 83.87 & 10.00 & 0.00
%
       \\ \midrule \multicolumn{9}{c}{ \textbf{Results for $m_{\mathrm{max}} = \lfloor\log_\beta 10^{-8}\rfloor$ and $\lambda = 0$  }  } \\ \midrule
Fixed ($\omega_k = \sqrt{\epsilon}$)               & 12.27 & 39.50 & 40.50 & 323.50 & 7.59 & 80.65 & 0.00 & 0.00 \\ \midrule
ARNCG$_g$ ($\theta = 0.0$)                         & 3.49 & 25.50 & 26.50 & 53.50 & 1.95 & 83.06 & 0.00 & 0.00 \\ 
ARNCG$_g$ ($\theta = 0.5$)                         & 2.37 & 24.00 & 25.00 & 52.50 & 1.35 & 85.48 & 0.00 & 0.00 \\ 
ARNCG$_g$ ($\theta = 1.0$)                         & 2.84 & 25.00 & 26.00 & 53.00 & 2.13 & 83.87 & 0.00 & 0.00 \\ 
ARNCG$_g$ ($\theta = 1.5$)                         & 2.73 & 26.00 & 27.00 & 54.00 & 2.10 & 84.68 & 0.00 & 0.00 \\ \midrule
ARNCG$_\epsilon$ ($\theta = 0.0$)                  & 2.74 & 23.00 & 24.00 & 49.00 & 1.44 & 83.87 & 0.00 & 0.00 \\ 
ARNCG$_\epsilon$ ($\theta = 0.5$)                  & 2.31 & 24.00 & 25.00 & 53.50 & 1.43 & 83.06 & 0.00 & 0.00 \\ 
ARNCG$_\epsilon$ ($\theta = 1.0$)                  & 2.49 & 26.00 & 27.00 & 55.50 & 1.40 & 84.68 & 0.00 & 0.00 \\ 
ARNCG$_\epsilon$ ($\theta = 1.5$)                  & 2.86 & 25.50 & 26.50 & 55.50 & 2.10 & 83.06 & 0.00 & 0.00 \\ \bottomrule
%
    \end{tabular}%
    }
    \label{tab:appendix-comparision-theta-median}
\end{table}




\end{document}
