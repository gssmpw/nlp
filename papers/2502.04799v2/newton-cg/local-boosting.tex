
\subsection{The local convergence order} \label{sec:main/boosting-local-rates}

From the compactness of $L_\varphi(x_0)$ in Assumption~\ref{assumption:liphess}, we know there exists a subsequence $\{ x_{k_j} \}_{j \geq 0}$ converging to some $x^*$ with $\nabla\varphi(x^*) = 0$ (see \theoremref{thm:appendix/global-newton-complexity}).
In the analysis of the local convergence rate, we need to assume the positive definiteness of $\nabla^2\varphi(x^*)$,
under which the whole sequence $\{x_k\}_{k\ge 0}$ also converges to $x^*$ (see \propositionref{prop:mixed-newton-nonconvex-phase-local-rates}).

The standard analysis of the local rates for Newton methods consists of two steps.
The first step shows that the Newton direction (i.e., $(\nabla^2 \varphi(x_k) + \omega_k \Id)^{-1} \nabla \varphi(x_k)$) yields superlinear convergence, and then the second step shows this direction is eventually taken.
Since there are some adjustments in our usage of these results, 
we provide the proofs in \Cref{sec:properties-of-newton-step} for completeness, 
and present the statements below.

\begin{lemma}%
    \label{lem:main/asymptotic-newton-properties}
    Assuming $\nabla^2\varphi(x^*) \succeq \alpha \Id$,
    if $\text{d\_type}_k = \texttt{SOL}$ and $m_k = 0$,
    and $x_k$ is close enough to $x^*$, we have 
    $g_{k+1} \leq O(g_k^2 + \omega_k g_k)$.
    Furthermore, under the choices of regularizers in \theoremref{thm:newton-local-rate-boosted},
    if $x_k$ is close enough to $x^*$,
    we know the trial step is accepted, and $\text{d\_type}_k = \texttt{SOL}$ and $m_k = 0$.
\end{lemma}


\begin{figure}[tbp]
    \centering
    \includegraphics{figures/local-order.pdf}
    \caption{
        The left plot illustrates the local order achievable by the regularizers in \theoremref{thm:newton-local-rate-boosted} for $\theta \in (0, 1]$.
        It can be made arbitrarily close to $1 + \nu_\infty$. 
        The right plot illustrates the local order for different $\theta$ using $\varphi(x) = \frac{1}{2}x^2$,
        and its slope reflects the local order and aligns with our predictions.
        }
    \label{fig:local-rate-for-nu1}
\end{figure}

We observe that when taking $\omega_k^{\supsucc} = \omega_k^{\supfallback} = O(g_k^{\bar\nu})$ with $\bar \nu \in (0, 1]$, the gradient norm converges superlinearly with order $1 + \bar\nu$.
For the choices in \theoremref{thm:newton-local-rate-boosted}, we find $\max(\omega_k^{\supsucc},\omega_k^{\supfallback}) \leq \sqrt{g_k}$ so a local rate of order $\frac{3}{2}$ can be achieved.
Furthermore, the following technical lemma shows that the local order can be improved to arbitrarily close to $1 + \nu_\infty \in \big(\frac{3}{2}, 2\big]$ for $\theta > 0$ with $\nu_\infty$ defined in \lemmaref{lem:superlinear-rate-boosting} (see \figureref{fig:local-rate-for-nu1} for an illustration), and achieves quadratic convergence for $\theta > 1$.
Its premise will be satisfied as long as the iteration is close to the solution according to \lemmaref{lem:main/asymptotic-newton-properties}.

\begin{lemma}[Local rate boosting]
    \label{lem:superlinear-rate-boosting}
    Let $\theta > 0$ and $\{ g_k  \}_{k \geq 0} \subseteq (0, \infty)$. 
    Suppose $g_1 \leq O \big( g_0^{\frac{3}{2}} \big)$ and
    $g_{k+1} \leq O \big (g_k^{2} + g_k^{\frac{3}{2}} \frac{g_k^\theta}{g_{k-1}^\theta} \big )$ holds for each $k \geq 1$,
    and $g_0$ is sufficiently small. 
    Then, 
    \begin{enumerate}
        \item If $\theta \in (0, 1]$, let $\nu_\infty$ be the positive root of the equation $\frac{1}{2} + \frac{\theta \nu_\infty}{1 + \nu_\infty} = \nu_\infty$, 
        then we have $g_{k + 1} \leq O\big ( g_k^{1 + \nu_\infty - (4\theta/9)^{k}}  \big )$,
        i.e., $g_k$ has local order $1 + \nu_\infty - \delta$ for any $\delta > 0$.
        \item If $\theta > 1$ and $k \geq2\log \frac{2\theta - 1}{2\theta - 2} + 1$, 
        then $g_{k+1} \leq O(g_k^2)$,
        i.e., $g_k$ converges quadratically.
    \end{enumerate}
\end{lemma}
\begin{proof}[Sketch of the idea]
    If $g_{k} = O(g_{k-1}^{\alpha})$ for $\alpha \in (1, 2]$, 
    then $g_{k-1}^{-\theta} = O\big (g_k^{-\frac{\theta}{\alpha}}\big )$.
    Thus, $g_{k+1} \leq O\big (g_k^2 + g_k^{\frac{3}{2} + \theta - \frac{\theta}{\alpha}}\big )$, 
    implying that the local order becomes $\min\big (2, \frac{3}{2} + \frac{\theta\alpha}{1 + \alpha} \big) > \frac{3}{2}$.
    By recursively applying this argument, we can gradually improve the local order.
    See \Cref{sec:appendix/local-rate-boosting} for details.
\end{proof}



