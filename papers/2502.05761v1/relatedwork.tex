\section{Related Work}
\myparagraph{Anomaly Detection Datasets}: Datasets are crucial for defect detection research. Traditionally, algorithms are developed using specialized datasets for specific objects, such as pcbs \cite{(15)tang2019online}, tiles \cite{(16)huang2020surface}, and steel \cite{(17)he2019end}. These training datasets often required manual labeling, limiting their impact on advancing industrial anomaly detection (IAD). The release of MVTec-AD in 2019 is a significant milestone, as it supported the development of unsupervised IAD algorithms by providing a diverse dataset. Subsequently, datasets like BTAD \cite{(18)mishra2021vt}, MPDD \cite{(19)jezek2021deep}, and VisA \cite{(12)zou2022spot} have further propelled IAD research. Recently, the Real-IAD \cite{(14)wang2024real} dataset introduced a larger, multi-view dataset, presenting new challenges for IAD. %algorithms.

Anomaly synthetic: Artificially synthesizing anomalies allows researchers to augment datasets and improve model performance, even with limited original data. Recent unsupervised anomaly detection methods have increasingly utilized synthetic anomaly images. For example, CutPaste \cite{(21)li2021cutpaste} generates negative samples by cutting and pasting image segments, while DRAEM \cite{(22)zavrtanik2021draem} and DeSTSeg \cite{(23)zhang2023destseg} use Perlin noise and random uniform samples to create anomaly masks. Additionally, DTD \cite{(24)cimpoi2014describing} provides a source for blending anomalies into original images. The capabilities of diffusion models have further expanded synthetic data generation \cite{(25)hu2024anomalydiffusion, (26)zhang2024realnet}. However, synthetic approaches may yield unrealistic anomalies, and their diversity is limited by the inherent cognitive scope of the model.


\begin{figure*}[t]
    \centering
    % \centering\setlength{\abovecaptionskip}{2pt}
    \includegraphics[width=.99\linewidth]{./figures/figure2.pdf}
    \caption{3CAD dataset samples. The first row shows normal images, while the second row displays defective images.}
% \vspace{-0.2cm}
\label{fig2}
\end{figure*}


\myparagraph{Anomaly Detection}
Recovery-based methods train networks to restore defects in images to their normal state \cite{(27)zavrtanik2021reconstruction, (28)xing2023visual,(44)xing2024recover}. For example, RealNet \cite{(26)zhang2024realnet} employs a feature reconstruction network with pre-trained multi-scale features, adaptively selecting and reconstructing residuals. By avoiding equal inputs and outputs, these methods mitigate the identity mapping issue in traditional reconstruction approaches. Moreover, the adaptability of diffusion models to various downstream tasks has spurred advancements in anomaly detection \cite{(42)shen2023advancing,(43)he2024diffusion}.

Knowledge distillation (KD) methods align teacher-student outputs for normal regions while differentiating defective ones for precise localization \cite{(29)salehi2021multiresolution}. 
Identical network structures, however, may reduce feature diversity. Techniques like RD \cite{(30)deng2022anomaly} and AST \cite{(31)rudolph2023asymmetric} address this by adopting serial or asymmetric architectures, improving differentiation between normal and abnormal features. Our method enhances this further by using heterogeneous teacher-student networks to better separate normal and abnormal features.