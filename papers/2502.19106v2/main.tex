\documentclass[journal,comsoc]{IEEEtran}



\usepackage[T1]{fontenc}
\usepackage[pdftex]{graphicx}
\usepackage{float}
\usepackage{subfigure}
\usepackage{color}
\usepackage{subfigure}
\usepackage{multirow}
\usepackage{hyperref}




\ifCLASSINFOpdf

\else

\fi





\usepackage{amsmath}
\DeclareMathOperator*{\argmax}{arg\,max}

\interdisplaylinepenalty=2500

\usepackage[cmintegrals]{newtxmath}


\hyphenation{op-tical net-works semi-conduc-tor}


\begin{document}

\title{A Survey on Foundation-Model-Based Industrial Defect Detection}


\author{Tianle Yang$^{*}$, Luyao Chang$^{*}$, Jiadong Yan, Juntao Li, Zhi Wang, Ke Zhang

\thanks{Corresponding author: K. Zhang is with  Soochow University, Suzhou, China. (e-mail:kzhang19@suda.edu.cn).}
\thanks{T. Yang, J. Yan, and J. Li are with Soochow University, Suzhou, China. (e-mail:tlyang@stu.suda.edu.cn, jdyan24@stu.suda.edu.cn, ljt@suda.edu.cn).}
\thanks{Z. Wang is with Shenzhen International Graduate School, Tsinghua University, Beijing, China (e-mail: wangzhi@sz.tsinghua.edu.cn).}
\thanks{L. Chang is with Wuhan University of Science and Technology, Wuhan, China. (e-mail:changluyao001@163.com).}
\thanks{\textit{(T. Yang  and L. Chang contributed equally to this paper.)}}
}




        
        






\maketitle


\begin{abstract}
As industrial products become abundant and sophisticated, visual industrial defect detection receives much attention, including two-dimensional and three-dimensional visual feature modeling. Traditional methods use statistical analysis, abnormal data synthesis modeling, and generation-based models to separate product defect features and complete defect detection. Recently, the emergence of foundation models has brought visual and textual semantic prior knowledge. Many methods are based on foundation models (FM) to improve the accuracy of detection, but at the same time, increase model complexity and slow down inference speed. Some FM-based methods have begun to explore lightweight modeling ways, which have gradually attracted attention and deserve to be systematically analyzed. In this paper, we conduct a systematic survey with comparisons and discussions of foundation model methods from different aspects and briefly review non-foundation model (NFM) methods recently published. Furthermore, we discuss the differences between FM and NFM methods from training objectives, model structure and scale, model performance, and potential directions for future exploration. Through comparison, we find FM methods are more suitable for few-shot and zero-shot learning, which are more in line with actual industrial application scenarios and worthy of in-depth research.
\end{abstract}


\begin{IEEEkeywords}
Industrial defect detection, foundation model, large language model, segment anything model
\end{IEEEkeywords}







\IEEEpeerreviewmaketitle



\section{Introduction}

\IEEEPARstart{V}{isual} defect detection \cite{pang2021deep}, which is also called visual anomaly detection, is a key application area of artificial intelligence algorithms. This task plays a crucial role in ensuring the quality of industrial products. Traditional industrial anomaly detection algorithms \cite{bergmann2018improving,gong2019memorizing,liu2021unsupervised,deng2022anomaly} focus on modeling the statistical distribution of normal features and detecting anomalies by analyzing the deviations in input samples from these learned patterns. To enhance the model’s ability to identify anomalous patterns, some methods \cite{liu2023simplenet,liang2024tocoad} further explore contrastive learning mechanisms \cite{hu2024comprehensive} between normal and abnormal features. These methods typically rely on a large amount of high-quality training data to establish reliable feature distributions and contrastive relationships. However, in real industrial scenarios, it is challenging to acquire specific high-quality training data due to the diversity and complexity of products and defects \cite{bergmann2019mvtec,zou2022spot,wang2024real}. For example, in chip defect detection, there are many types of chips and numerous defect categories, including structural defects and texture defects, making it difficult to collect data for various products and defects. In such cases, traditional models struggle to achieve satisfactory detection results. Recently, with the release of foundation models in vision and language, such as CLIP \cite{radford2021clip}, GPT \cite{zhu2023minigpt4,yang2023gpt4v} and SAM \cite{kirillov2023sam}, industrial defect detection algorithms based on these models have made significant progress in both 2D and 3D visual environments \cite{rani2024advancements}, particularly in few-shot and zero-shot scenarios where data are limited. This has received a great deal of attention. \textbf{The foundation models themselves possess strong capabilities in understanding general vision and language, making it an important issue to explore how to effectively apply their foundational knowledge to industrial detection problems without additional training samples and annotations. }
We categorize the application of different foundation models in 2D and 3D industrial defect detection as follows:

\begin{figure*}[ht]
    \centering
    \includegraphics[width=1\linewidth]{Fig/Figure1.pdf}
    \caption{Organization of surveyed methods. We categorize the methods under investigation into two main categories: foundation models and non-foundation models. Each category is further divided into 2D and 3D scenarios. The foundation-model-based methods primarily include methods based on SAM, CLIP, and GPT, while non-foundation-model-based methods are classified into static methods, synthesis-based methods, methods combining 2D RGB and 3D point clouds, and 3D generative methods. Finally, we present the latest methods collected in this survey.}
\end{figure*}

\begin{enumerate}
    \item SAM-2D: \textbf{Application of visual prior knowledge}. As a powerful foundational model for visual segmentation, SAM provides semantic prior information acquired through extensive pre-training on vast amounts of data, significantly enhancing the accuracy of industrial defect detection. In 2D industrial defect detection tasks based on SAM, researchers have developed various methods \cite{li2024sam,cao2023segment,peng2024sam,liu2024unsupervised,li2024clipsam,yang2024spt} to prompt SAM specifically for industrial scenarios. Additionally, object matching based on the masks generated by SAM is used to identify defect regions.
    \item CLIP-2D: \textbf{Semantic matching of short texts and images}. Image-text foundation models such as CLIP demonstrate fine-grained image-text matching. This ability effectively links subtle visual cues with descriptive text, so it is especially beneficial for defect detection. In 2D industrial defect detection tasks based on CLIP \cite{jeong2023winclip,zhou2023anomalyclip,cao2025adaclip,qu2024vcp,deng2024simclip,chen2024clipad,zuo2024clipfsac,li2024clipsam,hu2024sowa,chen2023april,li2024promptad,gu2024filo,zhang2024dual}, it is essential to design and learn suitable text prompts while aligning image information at a fine-grained level to further enhance performance. The design of text prompt templates has been extensively studied.
    \item GPT-2D: \textbf{Long text semantic prior}. Large language models like GPT can generate long-form descriptions, making them very suitable for complex scenarios that require detailed explanations and structured descriptions. Therefore, a key challenge in GPT-based 2D industrial defect detection methods \cite{gu2024anomalygpt,cao2023towards,xu2024customizing,li2023myriad,zhu2024alfa,zhang2024gpt,zhang2024logicode} is designing prompts to obtain comprehensive text descriptions and effectively leveraging the textual information.
    \item CLIP-3D: \textbf{Short-text image semantic matching prior applied to cross-dimensional vision tasks}. 3D defect detection faces greater challenges due to its complex spatial information. To address this, image-text foundation models like CLIP offer a promising solution through cross-modal information complementarity \cite{zuo2024clip3d,zhou2024pointad,wang2024m3dm}. These models effectively combine visual information with textual descriptions, thereby enabling more precise high-dimensional spatial modeling that captures complex defects in 3D structures.
\end{enumerate}



Although FMs demonstrate promising application prospects in industrial defect detection, NFM methods still possess irreplaceable advantages in specific application scenarios due to their smaller parameter sizes and higher computational efficiency. Based on this, this paper also provides a review of NFM methods, including 2D statistical modeling \cite{zhang2024sofs,bae2023pni,lyu2024reb,yao2023bgad,qian2024friend}, 2D anomaly data synthesis \cite{chen2025glass,li2024adabldm,zhang2024realnet,jiang2024cagen,hu2024anomalyxfusion,hu2024anomalydiffusion,duan2023dfmgan,zhang2023destseg,qin2024cutswap,lin2024split,bai2024dfd,chen2024pbas}, 2D/3D cross-modal knowledge distillation \cite{chu2023shape,cao2024cpmf,horwitz2023back,fuvcka2025transfusion,zavrtanik20243dsr,wang2024m3dm,rudolph2023ast}, and algorithms based on 3D generative models \cite{zhou2025r3d,zhao2024pointcore,zuo2024clip3d,liu2024uni,zhu2024group3ad}. \textbf{We believe that these methods can provide effective insight for FM methods and some of them can be applied to FM models}. In addition, we systematically compare the differences between foundation and non-foundation approaches in terms of application scenarios, algorithm framework focus, detection performance, model complexity, and future development directions. Key areas for potential breakthroughs in both approaches are also highlighted. This paper aims to provide researchers and engineers with information on selecting the appropriate research methods for different scenarios and to offer valuable perspectives on the future development of industrial defect detection.

The organization of this survey paper is as follows. First, in Section 1, we introduce the challenges posed by FM in industrial defect detection, followed by a discussion of the mainstream methods currently adopted. In Section 2, we provide a detailed comparison between FM and NFM methods, focusing on their differences in training objectives, model architectures, algorithm framework and performance. Then we give an overview of the different types of FM methods applied to both 2D and 3D industrial defect detection in Section 3. Section 4 discusses the key approaches of NFM methods and insights they provide for FM methods. Finally, in Section 5, we examine the ongoing challenges faced by large models and highlight potential future directions for further exploration. A detailed organization of the methods we investigate is also shown in Figure 1.





\section{Comparison of FM and NFM methods}

With the diversification of industrial detection demands, the differences in model training objectives, structures, scales, and performance have become key factors influencing the choice of methods. The following comparison analyzes the performance of FM and NFM in industrial anomaly detection from the perspectives of training objectives, model structure and scale, algorithm framework and performance. A summary of the comparison is shown in Figure 2.

\begin{figure*}[t]
    \centering
    \includegraphics[width=1\linewidth]{Fig/compare.pdf}
    \caption{A summary of the comparison between FM and NFM methods. We conduct a systematic comparison of the FM and NFM methods from the following 5 aspects: 1) Model Training Objectives. 2) Model Structure. 3) Model Scale. 4) Model Performance (AUROC Performance, Inference Time, and Computational Complexity). 5) Advantages and Challenges.}
\end{figure*}


\subsection{Model Training Objectives}
FM and NFM exhibit significant differences in data requirements, training methods, computational resources, and the breadth of feature learning, which consequently leads to differences in their training objectives. 2D FM methods (e.g., CLIP, SAM, GPT) leverage the cross-modal capabilities\cite{liu2024speech,tu2025self,li2024learning} of vision-language models to improve the accuracy and efficiency of anomaly detection. Their main objectives include:
1) Identifying unknown anomaly categories through unsupervised or zero-shot learning, reducing the dependence on labeled data; 2) Generating interpretable detection results that describe anomalies in terms of color, shape, and category; 3) Enhancing accuracy by integrating specific anomaly observation modules with the FM, addressing complex anomalies; 4) Improving model scalability and adaptability, enabling rapid adaptation to different industrial scenarios; 5) Integrating multimodal information to achieve precise anomaly localization and identification. 
3D FM methods focus on the geometric features of point cloud data \cite{arav2024evaluating,ye2024po3ad} and multi-view fusion \cite{hao2024network}, addressing issues of incomplete data and noise interference \cite{dai2017scannet,uy2019revisiting}. They perform classification and segmentation through multi-view rendering, while also handling inconsistencies between multimodal data. 

In contrast, 2D NFM methods rely on traditional network architectures, utilizing techniques such as GANs \cite{al2024enhanced} and diffusion models \cite{bhosale2024anomaly} to generate diverse anomaly samples to compensate for insufficient data. They emphasize feature selection and reconstruction 
 \cite{kim2024rethinking,yao2025glad,rafiee2024dcor,patra2024revisiting}strategies. 3D NFM methods focus on detecting geometric defects and missing areas in point cloud data, using efficient architectures to reduce computational overhead. They also employ innovative techniques, such as normalization flows \cite{lee2024gdflow,zhou2024vq} and diffusion-based reconstruction mechanisms, to enhance accuracy and robustness, avoiding dependence on design files or model libraries.

In summary, FM methods focus on \textbf{cross-modal learning and generative capabilities}, excelling in data-scarce scenarios and suitable for multi-task and multi-domain detection. In contrast, NFM methods emphasize \textbf{feature selection, computational efficiency, and data synthesis}, making them more suitable for resource-constrained environments.

\begin{figure*}[t]
    \centering
    \includegraphics[width=1\linewidth]{Fig/framework.pdf}
    \caption{The left branch is framework of FM methods and the right one is of NFM methods. FM methods are primarily based on FM such as SAM, CLIP and GPT. During training, FM methods design appropriate loss functions to fine-tune the pre-trained foundational models, adapting them to the industrial defect detection domain. In contrast, NFM methods focus on designing task-specific models based on lightweight or specialized network architectures. Some NFM methods also design anomaly synthesis strategies to supplement training data.}
\end{figure*}

\subsection{Model Structure and Scale}
\subsubsection{Model structure}
FM methods rely on powerful \textbf{vision-language collaborative mechanisms}, integrating large-scale foundational models such as CLIP, SAM, and GPT. These models employ multi-level feature fusion to establish a collaborative workflow: CLIP performs multi-modal feature extraction and alignment on image and point cloud data, SAM carries out fine-grained segmentation to isolate potential anomaly regions, and GPT provides semantic understanding and description of the detection results, assisting users in quickly obtaining analytical conclusions. To address the challenges of few-shot and zero-shot learning \cite{chen2024survey}, CLIP’s pre-trained knowledge enables effective inference on unlabeled data, thereby enhancing the generalization ability of the detection model. NFM methods mainly include Teacher-Student Architecture \cite{sun2024memoryless,deng2024structural,chen2024filter,liu2024unlocking}, Distribution Map, Memory Bank \cite{xing2023visual}, Autoencoder-based \cite{liang2024automated}, GAN-based, Transformer-based, and Diffusion-based frameworks. These approaches do not rely on large-scale data or pretraining tasks, focusing more on \textbf{local feature selection, sample generation, and augmentation}. Their aim is to optimize feature learning and anomaly detection capabilities with limited data.
\subsubsection{Model scale}
FM methods typically rely on \textbf{large parameter sizes}, utilizing complex network architectures and cross-modal learning to handle intricate anomaly detection tasks. This results in higher training times and computational resource demands. For example, SimCLIP \cite{deng2024simclip} has parameter sizes of 428.77M. In contrast, NFM methods have smaller parameter sizes and primarily optimize models through efficient \textbf{feature selection, adversarial training, and self-supervised learning}. These methods can achieve more efficient training in resource-constrained environments. Since fast inference is an inevitable trend, the newly published FM methods are trying to explore ways to accelerate inference. For example, SAM-based STLM \cite{li2024sam} requires only 16.56M for inference, making it one of the most efficient methods. 



\subsection{Framework}
The frameworks of FM methods and NFM methods are shown in Figure 3. FM methods primarily leverage \textbf{the prior knowledge embedded in foundation models}, which have been pre-trained on large-scale general-purpose datasets and possess strong feature representation capabilities. Consequently, fine-tuning these models often requires only a small number of samples. Different types of foundation models, such as SAM and CLIP, can process data of different modalities, including images and textual information. During training, FM methods focus on designing suitable loss functions to adapt foundation models more effectively to anomaly detection tasks in industrial applications. Ultimately, the fine-tuned models achieve accurate segmentation or localization of anomalous regions in industrial images.

NFM methods focus on \textbf{designing task-specific models}. For example, reconstruction-based anomaly detection methods train a model that can accurately reconstruct normal data by learning the reconstruction process. During data preprocessing, some methods use anomaly synthesis strategies to expand the dataset since anomaly samples are rare. By training the model with the target data, it gradually improves and ultimately generates a model specialized in detecting or segmenting anomalous regions.


\subsection{Model Performance}
\subsubsection{AUROC performance}
As shown in Table 1, using the commonly employed MVTec dataset as an example, 2D NFM methods generally achieve AUROC values close to 99\%, performing the best. In contrast, some 2D FM methods have AUROC values around 98\%. In 3D methods, both FM and NFM exhibit average AUROC values below 95\%. \textbf{This indicates that 2D methods outperform 3D methods overall, and NFM is currently more mature than FM in this context.}

\begin{table*}
\centering
\caption{A brief summary and overview of different FM and NFM methods. The numbers of performance are all copied from their original paper. Specifically, “Performance” denotes the AUROC metric on the dataset shown behind.}
\resizebox{\textwidth}{!}{%
\begin{tabular}{c|c|c|c|c|c|c} 
\hline
Category & Sub-category & Method & Description & Publication & \multicolumn{2}{c}{Performance}\\ 
\hline
\multirow{29}{*}{Foundation}& \multirow{6}{*}{2D SAM Based}& ClipSAM \cite{li2024clipsam}& Hierarchical mask refinement with multi-level prompts&             ARXIV 2024&               92.3 &\multirow{28}{*}{MVTec AD}\\
 \multirow{30}{*}{Model}& & UCAD \cite{liu2024unsupervised}
& Structure-based contrastive learning with SAM& AAAI 2024& 93.0 &\\
 \multirow{31}{*}{Method}& & SAM-LAD \cite{peng2024sam}
& Use SAM to obtain object masks of the query and reference images and extract object features for matching& ARXIV 2024&98.4 &\\
 & & SAA+ \cite{cao2023segment}
& Hybrid prompt regularization& ARXIV 2023&- &\\ 
 & & STLM \cite{li2024sam}
& Utilize SAM as a teacher to guide student networks& ARXIV 2024&98.26 &\\
& & SPT \cite{yang2024spt}& Adapt SAM to better understand the relationships between different regions in the image& AAAI 2025& -&\\
\cline{2-6}& \multirow{15}{*}{2D CLIP Based}&WinCLIP \cite{jeong2023winclip}&Compositional prompt ensemble, reference association method&CVPR 2023&93.1 &\\ 
& &AnoCLIP \cite{deng2023anovl}&Local-aware visual tokens, domain-aware prompting, test-time adaptation method&ARXIV 2024&-
 &\\ 
& &AnomalyCLIP \cite{zhou2023anomalyclip}&An object-agnostic text prompt template, global abnormality loss function&ICLR 2024&-
 &\\ 
& &AdaCLIP \cite{cao2025adaclip}&Hybrid (static and dynamic) learnable prompts, hybrid-semantic fusion module&ECCV 2024&-
 &\\ 
& &VCP-CLIP \cite{qu2024vcp}&Visual context prompting model&ARXIV 2024&-
 &\\ 
& &SimCLIP \cite{deng2024simclip}&Multi-hierarchy vision adapter, implicit prompt learning, prior-aware optimization algorithm&ARXIV 2024&95.3
 &\\ 
& &CLIP-AD \cite{chen2024clipad}&Distribution of the text prompts, facilitate alignment via a linear layer&ARXIV 2023&-
 &\\ 
& &CLIP-FSAC \cite{zuo2024clipfsac}&Two-stage training strategy, visual-driven text features, fusion-text matching task&IJCAI 2024&95.5
 &\\ 
& &ClipSAM \cite{li2024clipsam}&CLIP and SAM Collaboration, unified multi- scale cross-modal interaction, multi-level mask refinement&ARXIV 2024&-
 &\\ 
& &SOWA \cite{hu2024sowa}&Hierarchical frozen window self-attention, dual Learnable Prompts&ARXIV 2024&-
 &\\ 
& &SAA+ \cite{cao2023segment}&Hybrid prompts, domain expert knowledge and target image context&ARXIV 2023&--
 &\\ 
& &APRIL-GAN \cite{chen2023april}& Employ a combination of state and template ensembles, memory bank-based approach &ARXIV 2023&92.0
 &\\ 
& &PromptAD \cite{li2024promptad}&Prompt learing, semantic concatenation, explicit anomaly margin&CVPR 2024&94.6
 &\\ 
& &FiLo \cite{gu2024filo}&Fine-grained description, learnable vectors, position-enhanced high-quality localization method&ARXIV 2024&-
 &\\ 
& &Dual-Image Enhanced CLIP \cite{zhang2024dual}&Dual image feature enhancement, test-time adaption with pseudo anomaly synthesis&ARXIV 2024&-
 &\\ 
\cline{2-6}& \multirow{7}{*}{2D GPT Based}&AnomalyGPT \cite{gu2024anomalygpt}&Lightweight and visual-textual feature-matching-based decoder, prompt embeddings&AAAI 2024&94.1
 &\\ 
& &Myriad \cite{li2023myriad}&Apply vision experts, vision expert tokenizer&ARXIV 2023&94.1
 &\\ 
& &ALFA \cite{zhu2024alfa}&Run-time prompt adaptation strategy, fine-grained aligner&ARXIV 2024&94.5 &\\ 
& &GPT-4V-AD \cite{zhang2024gpt}&Visual Question Answering paradigm, granular region division, prompt designing, Text2Segmentation method&ARXIV 2023&- &\\
& &Customizable-VLM \cite{xu2024customizing}&Enhance foundation models by integrating expert knowledge as external memory via prompting & ARXIV 2024& 82.9&\\
& &LogiCode \cite{zhang2024logicode}&Use LLMs to extract image logic and generate code for logical anomaly detection& ARXIV 2024& -&\\
\cline{2-7}& \multirow{3}{*}{3D CLIP Based}&CLIP3D-AD \cite{zuo2024clip3d}&Address both few-shot anomaly classification and segmentation without memory banks and plenty of training samples&ARXIV 2024&- &\multirow{3}{*}{MVTec3D-AD}\\ 
 & & PointAD \cite{zhou2024pointad}
 & Hybrid representation learning framework& ARXIV 2024&97.2 &\\
& & M3DM-NR \cite{wang2024m3dm}
& Use the suspected anomaly maps to achieve denoising& ARXIV 2024&94.5 &\\
\hline
\multirow{27}{*}{Non-Foundation}&\multirow{5}{*}{2D Statistic}&SOFS \cite{zhang2024sofs}
&Introduce an abnormal prior map and mixed normal Dice loss&CVPR 2024&93.3
 &\multirow{17}{*}{MVTec AD}\\ 
\multirow{28}{*}{Model}& &PNI \cite{bae2023pni}&Utilize position and neighborhood information&ARXIV 2023&99.56
 &\\ 
\multirow{29}{*}{Method}& &REB \cite{lyu2024reb}&Reduce domain and local density biases&ARXIV 2024&99.5&\\ 
& &BGAD \cite{yao2023bgad}&Strengthen the decision boundary by pulling together normal samples while pushing away anomalous samples & CVPR 2023& 99.3&\\
& &COAD \cite{qian2024friend}&Enhance model sensitivity to anomalies through controlled overfitting& ARXIV 2024& 99.9&\\
 \cline{2-6}& \multirow{12}{*}{2D Synthesis}& GLASS \cite{chen2025glass}
&                                              Anomaly synthesis based on Gaussian noise and gradient rise&             ARXIV 2024&               99.9 &\\
 & & AdaBLDM \cite{li2024adabldm}
& Latent diffusion model with feature editing& ARXIV 2024&- &\\
 & & RealNet \cite{zhang2024realnet}
& Strength-controllable diffusion anomaly synthesis& CVPR 2024&99.6 &\\
 & & CAGEN \cite{jiang2024cagen}
& Text-guided controllable anomaly generation& ICASSP 2024&97.7 &\\
 & & AnomalyXFusion \cite{hu2024anomalyxfusion}
& Multi-modal anomaly synthesis for enhanced sample fidelity& ARXIV 2024&99.2 &\\
 & & AnomalyDiffusion \cite{hu2024anomalydiffusion}
& Spatial anomaly embedding, adaptive attention re-weighting mechanism& AAAI 2024&99.2 &\\
 & & DFMGAN \cite{duan2023dfmgan}
& Use defect-aware residual blocks in StyleGAN2& AAAI 2023&- &\\
 & & DeSTSeg \cite{zhang2023destseg}
& Denoising student encoder-decoder, adaptive multi-level feature fusion& CVPR 2023&98.6 &\\
 & & CutSwap \cite{qin2024cutswap}
& Leverages saliency guidance to incorporate semantic cues& ARXIV 2023&98.0 &\\
 & & Split Training \cite{lin2024split}
& A split training strategy that alleviates the overftting issue& ARXIV 2024&98.3 &\\
 & & DFD \cite{bai2024dfd}
 & Frequency-domain analysis with dual-path frequency discriminators& ARXIV 2024&93.3 &\\
 & &PBAS \cite{chen2024pbas}&Use the compact distribution of normal sample features to guide the direction of feature-level anomaly synthesis  & TCSVT 2024& 99.8&\\
 \cline{2-7}& \multirow{7}{*}{2D RGB+3D PC}& Shape-Guided \cite{chu2023shape}
& Synergistic expert models for anomaly localization in color and shape& WACV 2024
& 94.7 
 &\multirow{7}{*}{MVTec3D-AD}\\ 
 & & CPMF \cite{cao2024cpmf}
& Combine handcrafted PCD descriptions with 
pre-trained 2D neural networks& Pattern Recognition 2023
&92.93 
 &\\
 & & Back to the Feature \cite{horwitz2023back}
& Handcrafted 3D representations with PatchCore& CVPR 2021
&97.8 
 &\\
 & & TransFusion \cite{fuvcka2025transfusion}
& Address the overgeneralization and loss-of-detail problems utilizing transparency-based diffusion& ECCV 2024&98.2 
 &\\
 & & 3DSR \cite{zavrtanik20243dsr}
& Depth-aware discrete autoencoder and the
simulated depth generation process& WACV 2024&97.8 
 &\\
 & & M3DM \cite{wang2024m3dm}
& A hybrid fusion scheme to reduce the disturbance between multimodal features and encourage feature interaction& CVPR 2023
&94.5 
 &\\
 & & AST \cite{rudolph2023ast}
& Introduce a network which compensates for wrongly estimated
likelihoods by a normalizing flow& WACV 2023
&93.7 
 &\\
 \cline{2-7}& \multirow{5}{*}{3D Generation}& R3D-AD \cite{zhou2025r3d}
& Overcome the inefficiencies due
to the memory bank module and low performance caused by incorrect rebuilds
with MAE& ECCV 2024
&73.4 
 &Real 3D-AD
\\
 & & Reg 3D-AD \cite{liu2024real3d}
& A dual-feature representation approach to preserve the training prototypes’
local and global features& NeurIPS 2023
&70.4 
 &Real 3D-AD
\\
 & & PointCore \cite{zhao2024pointcore}
& Reduce the computational cost and mismatching disturbance
in inference& ARXIV 2024
&82.9 
 &Real 3D-AD
\\
 & & Uni-3DAD \cite{liu2024uni}
& Notable adaptability to model-free industrial
products& ARXIV 2024
&-
 &MVTec 3D-AD\\
 & & Group3AD \cite{zhu2024group3ad}
 & Enhance the
resolution and accuracy of 3D anomaly detection through group level feature contrastive learning& ACM MM 2024&75.1 &Real 3D-AD\\ 
\hline
\end{tabular}
}
\label{tab:booktabs}
\end{table*}

\subsubsection{Inference time}
\textbf{Due to their large parameter sizes and complex model structures, FM methods generally require more inference time.} Although STLM \cite{li2024sam} has significantly optimized inference efficiency with an average inference time of 20ms, it still lags behind NFM methods like DRAEM \cite{zavrtanik2021draem}, FastFlow \cite{yu2021fastflow}, and PatchCore \cite{roth2022towards}, which have shorter inference times. 3D methods typically require longer inference times; however, some methods, such as SOWA \cite{hu2024sowa}, still demonstrate excellent inference speed, with a rate of 16.84 it/s (compared to 3.28 it/s for WinCLIP \cite{jeong2023winclip} and 1.82 it/s for April-GAN \cite{chen2023april}). Compared to BTF \cite{horwitz2023back}, M3DM \cite{wang2023multimodal}, and Reg 3D-AD \cite{liu2024real3d}, PointCore \cite{zhao2024pointcore} achieves the highest AUROC and is the fastest, with a mean inference time per object on Real3D-AD \cite{liu2024real3d} of 4.282s, excluding BTF \cite{horwitz2023back}(2.19s). The shape-guided \cite{chu2023shape} method has an inference time of 2.05s per sample, outperforming BTF \cite{horwitz2023back}.

\subsubsection{Computational complexity}
\textbf{FM methods typically have higher FLOPs than those of NFM methods due to their large model sizes.} For instance, STLM \cite{li2024sam} has a FLOPs of 55G. SAM-LAD  \cite{peng2024sam}, which employs transformers and upsampled feature maps, has a FLOPs of 54.7G, higher than that of CNN-based NFM methods such as AE \cite{bergmann2018improving}(5.0G) and f-AnoGan \cite{schlegl2019f}(7.7G). Addressing the computational complexity of FM has become a popular research direction. SimCLIP \cite{deng2024simclip}, optimized for inference efficiency, requires fewer FLOPs (513.75G) than SOTA prompt learning methods like CoOp \cite{zhou2022learning}(520.46G) and Co-CoOp \cite{zhou2022conditional}(520.46G) while maintaining the same parameter count. However, SimCLIP’s \cite{deng2024simclip} FLOPs are an order of magnitude higher than STLM \cite{li2024sam} and SAM-LAD \cite{peng2024sam}, as STLM \cite{li2024sam} uses distillation from a fixed SAM teacher, and SAM-LAD \cite{peng2024sam} employs FeatUp’s \cite{fu2024featup} Upsampling Factors. And both STLM \cite{li2024sam} and SAM-LAD \cite{peng2024sam} do not use foundation models during inference.

Based on the above analysis, FM methods demonstrate strong potential in complex industrial detection and cross-domain applications, thanks to their powerful multi-modal capabilities and large parameter sizes. However, challenges such as overfitting in data-scarce scenarios and inference efficiency remain bottlenecks, particularly when handling 3D high-dimensional data, which still offers ample room for exploration. NFM methods, on the other hand, rely on targeted feature extraction and efficient computation, making them more advantageous in real-time inference and industrial scenarios with limited computational resources.


\section{Foundation Model Methods}

In recent years, visual-language models have shown significant advantages in anomaly detection. These models are able to better understand and describe complex features in images by effectively combining visual information with linguistic cues. Compared to traditional anomaly detection methods, visual-language models are able to exploit rich contextual information and reduce the dependence on manual annotation and domain knowledge, thus achieving more accurate detection. In this part, three main classes of methods based on visual- language models are introduced: methods based on SAM, CLIP and GPT. In Table 1, we give a summary and overview of different
methods based on FM. And in Figure 4, the most important and popular works along the FM development are shown in the timeline.

\begin{figure*}[t]
    \centering
    \includegraphics[width=1\linewidth]{Fig/Figure2.pdf}
    \caption{Representative methods along the development of FM and NFM models. The orange box illustrates the evolution of FM methods. WinCLIP introduced the use of prompt ensemble and multi-scale feature extraction with CLIP. Subsequently, SAA+ and Anomaly GPT incorporated SAM and GPT techniques, fostering the exploration of cross-modal approaches exemplified by ClipSAM. 3D FM methods emerged later, with CLIP3D-AD and PointAD focusing on addressing inconsistencies in multimodal data. Recently, 2D FM methods have achieved improvements in inference speed and accuracy, such as STLM based on a teacher-student framework and CLIP-FSAC employing vision-driven textual strategies.The green box presents the progression of NFM methods. The early method Back to the Future proposed handcrafted 3D representations but suffered from low efficiency and accuracy. Diffusion-based approaches, including TransFusion, AnomalyDiffusion, and AnomalyXFusion, effectively addressed these issues. In recent years, 3D generative techniques have been explored, with efforts concentrated on enhancing computational and storage efficiency.}
\end{figure*}

\subsection{2D SAM-Based Methods}
As a foundation model, the Segment Anything Model (SAM) \cite{kirillov2023sam,cao2023collaborative,wan2021industrial} has a powerful ability to extract high-quality segmentation masks. By leveraging large-scale pre-training data, it can perform instance segmentation on any object in various scenarios without the need for task-specific training. Consequently, SAM-based methods demonstrate good performance in zero-shot anomaly detection tasks. Cao et al. \cite{cao2023segment} utilize SAM and cascading prompt-guided object detection models \cite{liu2025grounding} to construct a vanilla baseline, i.e., Segment Any Anomaly (SAA). SAA generates preliminary anomaly regions through simple language prompts such as “defect” or “anomaly,” followed by a refinement process. They further introduce a mixed prompt regularization technique, enhancing the framework into Segment Any Anomaly+ (SAA+). To better process the masks generated by SAM, Li et al. propose ClipSAM \cite{li2024clipsam}, which combines the strengths of both CLIP and SAM. ClipSAM uses CLIP’s semantic understanding capabilities for anomaly localization and rough segmentation, then using the results as prompt constraints for SAM to refine the segmentation outcomes. In SAM-LAD \cite{peng2024sam}, Peng et al. introduce SAM to obtain object masks of the query and reference images. Each object mask is multiplied with the entire image’s feature map to obtain object feature maps, which are then used for object matching. Building on this, an Anomaly Measurement Model (AMM) is proposed to detect logical and structural anomalies. In UCAD \cite{liu2024unsupervised}, Liu et al. use SAM to enhance anomaly detection through Structure-based Contrastive Learning (SCL). By treating SAM-generated masks as structure, features within the same mask are drawn closer together, while others are pushed apart. This improves feature representation for anomaly detection. Li et al. \cite{li2024sam} propose a SAM-guided Two-stream Lightweight Model (STLM), prioritizing efficiency and mobile compatibility. One stream extracts features to distinguish normal from anomalous regions, while the other reconstructs anomaly-free images to enhance differentiation. With a shared mask decoder and feature aggregation, STLM delivers precise anomaly maps. In SPT \cite{yang2024spt}, Yang et al. introduced a Visual-Relation-Aware Adapter (VRA-Adapter) to help SAM better understand the relationships between different regions in the image, enhancing SAM's fine-grained understanding of anomaly patterns.


\subsection{2D CLIP-Based Methods}
CLIP-based methods \textbf{use large-scale pre-trained visual-language models that combine image coding with textual cues}, thereby strengthening the relationship between visual features and linguistic information, and perform particularly well in zero- and few-shot scenarios.

\subsubsection{Text Prompt}
Jeong et al. \cite{jeong2023winclip} propose a window-level anomaly detection method called WinCLIP, which achieves zero-shot anomaly segmentation \cite{radford2021learning} through combinatorial prompt ensemble and multi-scale feature extraction. Zhou et al. \cite{zhou2023anomalyclip} design an object-agnostic text prompt template to capture anomalous regions in an image by learning generic normality and abnormality prompts \cite{zhou2022conditional} and combining global and local contextual information. APRIL-GAN \cite{chen2023april} combines a text prompt integration strategy with a linear layer \cite{ross2017focal,milletari2016v} to improve the performance of zero and few-shot anomaly detection. SimCLIP reduces the reliance on hand-crafted prompts by introducing multi-level visual adapters with implicit prompt tuning. Qu et al. \cite{qu2024vcp} use visual contextual prompts to activate CLIP's anomaly semantic capability, eliminating the need for product-specific prompts. Cao et al. \cite{cao2025adaclip} propose to optimise zero-shot anomaly detection performance by combining static and dynamic learnable prompts. Li et al. \cite{li2024promptad} propose to convert normal prompts into anomaly prompts via semantic connectivity to build a large number of negative samples for prompt learning.

\subsubsection{Fine-Grained Alignment}
Gu et al. \cite{gu2024filo} improve the accuracy of anomaly localization by fine-grained local descriptions and optimised visual encoders. Zhang et al. \cite{zhang2024dual} propose to improve the accuracy of anomaly detection by using dual-image as visual references. Zuo et al. \cite{zuo2024clipfsac} propose that the performance of few-shot anomaly classification can be effectively improved by a two-stage training strategy and an image-to-text cross-attention module. Chen et al. \cite{chen2024clipad} propose to achieve fine-grained alignment through representative vector selection and a staged dual-path model. FiLo uses fine-grained description and high-quality localization to improve the accuracy and interpretability of anomaly detection. SAA+ \cite{cao2023segment} achieves more accurate anomaly localization through prompt-guided object detection and refinement techniques. ClipSAM \cite{li2024clipsam} improves zero-shot anomaly segmentation by combining the strengths of CLIP and SAM \cite{wang2024sam}, and leverages the semantic understanding capability of CLIP for anomaly localization and fine-grained segmentation \cite{wang2022cris}. Hu et al. propose a hierarchical freezing window self-attention mechanism \cite{xing2024less} that captures features at different levels by combining multi-level adapters for fine-grained localization \cite{sun2022dualcoop}. 

\subsection{2D GPT-Based Methods}
GPT-based methods \textbf{exploit the advantages of large-scale language models in natural language understanding and adaptive learning} to support the anomaly detection task by generating concrete textual descriptions, while being able to adapt their processing strategies to changing detection environments and anomaly types based on real-time input. 

AnomalyGPT \cite{gu2024anomalygpt} is the first to apply Large Vision-Language Models (LVLMs) \cite{ouyang2022training,touvron2023open} to anomaly detection tasks and supports multiple rounds of conversations, demonstrating excellent few-shot learning capabilities. Subsequently, Cao et al. \cite{cao2023towards} and xu et al. \cite{xu2024customizing} explore how to use LVLMs for general anomaly detection tasks across various domains. At the same time, they incorporate information from different modalities, such as domain knowledge, class context, and reference images as prompts to improve LVLMs' detection performance. Myriad \cite{li2023myriad} reduces the reliance on labelled data by combining visual experts with a large-scale multimodal model. Zhu et al. \cite{zhu2024alfa} propose a run-time prompt adaptation strategy to generate informative anomaly prompts, which combined with a fine-grained aligner can achieve accurate anomaly localization and enhance the dynamic adaptability of the model, making it more useful in diverse industrial scenarios. Zhang et al. \cite{zhang2024gpt} explore the potential of Visual Question Answering (VQA)-oriented GPT-4V (ision) in anomaly detecion \cite{cao2023towards}, introducing a GPT-4V-AD framework that integrates Granular Region Division, Prompt Designing, and Text2Segmentation. LogiCode \cite{zhang2024logicode} fully leverages the reasoning capabilities of LLMs. It extracts logical relationships from normal images and generates executable Python code to automatically detect logical anomalies in test images. The system also provides the specific location and detailed explanation of the anomalies. The innovative framework of LogiCode breaks through traditional anomaly detection methods, offering a more intelligent solution.


\subsection{3D CLIP-Based Methods}
In contrast to traditional 2D CLIP models that primarily process RGB images, \textbf{3D CLIP handles three-dimensional data—point clouds— which encompass more complex spatial structures and geometric information}. Currently, research in 3D anomaly detection is less developed compared to its 2D counterpart, largely because CLIP was initially trained on 2D RGB images paired with text. Consequently, 3D CLIP faces challenges in integrating point cloud data with images in a multimodal framework. Recent studies have made significant progress in overcoming the modality gap in 3D data processing, employing techniques such as multimodal noise reduction, multi-view processing and fusion of 3D data, as well as the integration of zero-shot learning to improve performance.

Zuo et al. \cite{zuo2024clip3d} proposed a multi-view fusion module that integrates 2D image features from different perspectives, thereby enhancing the representation capability of point cloud data and overcoming the challenges posed by modality differences when processing 3D point clouds directly. PointAD \cite{zhou2024pointad} achieves zero-shot 3D anomaly detection by rendering 3D point clouds from multiple views into 2D images \cite{zhang2022pointclip}, and then jointly optimizing 2D and 3D features through Hybrid Representation Learning. M3DM-NR \cite{wang2024m3dm} significantly improves data quality and reduces noise interference through a three-stage multimodal noise removal method. It leverages pre-trained CLIP and Point-BIND models, and employs multi-scale feature comparison and weighting to enhance the quality of training samples and improve the overall data purity.



\section{Non-Foundation Model Methods}
Unlike large-scale model-based methods that depend on extensive pretraining and complex multimodal fusion techniques, lightweight models improve detection accuracy through optimized architectures, feature extraction techniques, and computational efficiency. These methods are particularly suited for resource-limited scenarios that require fast inference, offering significant benefits for real-world deployment in industrial environments. This section will discuss the four main types of current lightweight model methods: statistical approaches, anomaly synthesis strategies, detection methods combining 2D RGB images with 3D point clouds, and 3D generation techniques. \textbf{The methods discussed in this section can be used as references for FM methods in the future, such as statistics-related methods, generation model, and data synthesis}.
Table 1 also shows the summary and overview of different methods based on NFM. Figure 4 presents the main timeline of NFM development.

\subsection{Statistics-Related Methods}
The statistical methods provide \textbf{an effective theoretical foundation} for improving the performance of anomaly detection models. Zhang et al. \cite{zhang2024sofs} propose a mixed normal Dice loss to improve the Dice loss. This loss function imposes a large penalty when the model predicts false positives, thus prioritizing the prevention of such incorrect predictions. Bae et al. \cite{bae2023pni} propose the PNI algorithm to address the impact of location and neighborhood information on the distribution of normal features. This algorithm employs a conditional probability based on neighborhood features, using a Multi-Layer Perceptron (MLP) network to model the distribution of normal features. Additionally, the method effectively captures positional information by constructing histograms of representative features at each location. LYU et al. \cite{lyu2024reb} consider variations in local feature density and propose the Local Density K-Nearest Neighbors (LDKNN) method to reduce the density bias in patch-level features. COAD \cite{qian2024friend} views overfitting as a controllable mechanism that enhances sensitivity to anomalies through controlled overfitting. It introduces the Aberrance Retention Quotient (ARQ) metric to precisely quantify the degree of overfitting, thereby identifying an optimal "golden overfitting interval" (the optimal ARQ) to optimize anomaly detection performance. BGAD \cite{yao2023bgad} designs a boundary-guided semi-push-pull (BG-SPP) loss. First, it generates an explicit boundary by learning the normal sample feature distribution. Based on this, it pulls together normal samples while pushing away anomalous samples, thereby strengthening the decision boundary. BGAD enables the model to effectively distinguish between seen and unseen anomalies using only a small number of anomalous samples. However, the scarcity of anomalous samples may still lead to inefficient feature learning, and BGAD does not fully address this key issue.


\subsection{Anomaly Synthesis Strategies}
Anomaly synthesis strategies aim to enhance the performance of anomaly detection models by \textbf{generating diverse and realistic abnormal samples}. Broadly, anomaly synthesis strategies can be categorized into the following types: 

\subsubsection{Generative Models}
Based on Denoising Diffusion Probabilistic Models (DDPM), Zhang et al. \cite{zhang2024realnet} introduce additional noise in the reverse diffusion process to control the intensity of the generated anomalous samples. Besides, Hu et al. \cite{hu2024anomalyxfusion} aggregate multiple modality features and integrate them into a unified embedding space, optimizing modality alignment. They then facilitate controlled generation through adaptive adjustments of the embedding based on diffusion steps. Jiang et al. \cite{jiang2024cagen} enhance the controllability of anomaly generation through fine-tuning a ControlNet model with text prompts and binary masks. Hu et al. \cite{hu2024anomalydiffusion} propose AnomalyDiffusion, which uses a Latent Diffusion Model (LDM) to generate anomalous images. It combines spatial anomaly embedding with an adaptive attention mechanism to improve the alignment between the generated anomalies and their corresponding masks. Li et al. \cite{li2024adabldm} build upon the Blended Latent Diffusion Model (BLDM) \cite{rombach2022high} with several innovations. They design a novel ’defect trimap’ to delineate the object masks and defect regions in generated images. They also introduce a cascaded ’editing’ stage in latent and pixel spaces to ensure structural coherence and detail fidelity. Additionally, they propose an online adaptation of the image encoder to further enhance image quality. Duan et al. \cite{duan2023dfmgan} train a data-efficient StyleGAN2 on defect-free images as the backbone. Then, they add defect-aware residual blocks to generate defect masks and manipulate the features within the masked regions, generating new defect images.

\subsubsection{Data Augmentation Techniques}
Based on normal features, Chen et al. \cite{chen2025glass} guide Gaussian noise through gradient ascent and truncated projection to synthesize weak anomalies around normal points. Besides, they create binary masks using Perlin noise and combine them with external textures to synthesize strong anomalies that are further away from normal points. Zhang et al. \cite{zhang2023destseg} generate anomalous images using Perlin noise and use them as input for the student network. By training the student network to remove the synthetic anomalous noise, they enhance the student network’s ability to represent features of anomalous samples, thereby improving the performance of the teacher-student framework in anomaly detection. Qin et al. \cite{qin2024cutswap} introduce semantic information for the generation of anomalous samples. They utilize LayerCAM to extract salient features from images and conduct clustering to identify the most significant regions. Subsequently, they select similar patch pairs and swap their positions. The negative samples generated in this way are more subtle yet realistic. Lin et al. \cite{lin2024split} develope a comprehensive anomaly simulation framework that combines reconstruction strategies for both transparent and opaque anomalies. By using selective augmentation and segmentation-based training strategies, they address the challenges of anomaly generation diversity, reconstruction quality, and overfitting. Bai et al. \cite{bai2024dfd} discover that small anomalies become more noticeable in the frequency domain. By transforming spatial images into multi-frequency representations, the discriminator learns joint representations between normal images and pseudo-anomalies, thereby improving the performance of few-shot anomaly detection. PBAS \cite{chen2024pbas} first learns a compact distribution of normal sample features with center constraints as an approximate decision boundary, which is used to guide the direction of feature-level anomaly synthesis. Then, it performs binary classification between the synthesized anomalies and normal features, further optimizing the decision boundary to ensure that the synthesized anomalies do not overlap with normal samples.


\subsection{Methods Combining 2D RGB and 3D Point Clouds}
The method of combining 2D RGB images with 3D point clouds improves the detection capabilities of traditional approaches, which are often limited by the lack of data from a single modality. \textbf{This is done by fusing features from both modalities: the rich color and texture features of 2D RGB images and the spatial and geometric information provided by 3D point clouds.}

Chu et al. propose a shape-guided expert-based learning framework that employs two expert models to detect anomalies in 3D structure and color appearance, respectively, and locates defects in test samples using a dual memory bank and shape-guided reasoning method. The model utilizes neural implicit functions (NIFs) \cite{ma2022surface} to represent local shapes and refines the complex structure of point clouds through signed distance fields, enabling point-level anomaly prediction. This significantly improves the accuracy of anomaly localization while reducing computational and memory costs. CPMF \cite{cao2024cpmf} generates pseudo-2D representations by projecting point clouds onto 2D and extracts semantic features using a pre-trained 2D neural network. These features complement 3D local features extracted from handcrafted point cloud descriptors and are unified into a global semantic and local geometric point cloud representation through feature alignment and fusion modules. Horwitz et al. \cite{horwitz2023back} highlighted that 3D methods are currently outperformed by 2D methods and proposed a solution combining rotation-invariant handcrafted feature representations with deep learning-based color features to improve 3D anomaly detection performance. TransFusion \cite{fuvcka2025transfusion} addresses the overgeneralization and detail loss issues by iteratively increasing the transparency of anomalous regions and gradually replacing them with the normal appearance while preserving the normal appearance of non-anomalous regions. Zavrtanik et al. \cite{zavrtanik20243dsr} introduced 3DSR, where DADA learns a universal discrete latent space that jointly models RGB and depth data. 3DSR performs discriminative anomaly detection in the feature space learned by DADA. M3DM constructs three separate memory banks for RGB, 3D, and fused features and performs anomaly detection by considering decisions from these memory banks through Decision Layer Fusion (DLF). To better align 3D point cloud features with 2D RGB features, Point Feature Alignment (PFA) was introduced. Rudolph et al. \cite{rudolph2023ast} presented the Asymmetric Student-Teacher Network (AST), which employs a normalized flow for density estimation as the teacher network and a conventional feed-forward network as the student network, solving the issue of insufficient output differences for anomalous data caused by similar student and teacher architectures in previous methods.

\subsection{3D Generation Methods}
\textbf{3D generative techniques use generative models to reconstruct normal samples or missing regions,} aiming to reduce computational overhead and improve model robustness, particularly addressing the challenges of model-free products and the difficulty in identifying missing regions.

Zhou et al. \cite{zhou2025r3d} employed a diffusion model-based data distribution transformation to completely mask abnormal geometries in the input, learning gradual displacement during the reverse diffusion process and explicitly controlling the reconstruction of abnormal shapes. Additionally, they proposed a 3D anomaly simulation strategy called Patch-Gen, designed to generate realistic defect shapes and bridge the gap between training and testing data. R3D-AD addresses challenges in 3D anomaly detection related to computational storage overhead and the detection of unmasked region anomalies. PointCore requires only a single memory bank to store local (coordinate) and global (PointMAE) representations, assigning different priorities to these local-global features to reduce computational costs and mitigate feature misalignment during inference. A ranking-based normalization method is used to eliminate distribution discrepancies between different anomaly scores, while the Iterative Closest Point (ICP) algorithm is applied to locally optimize point cloud registration results, enhancing decision robustness. Liu et al. \cite{liu2024uni} proposed a dual-branch structure where the feature-based branch and reconstruction-based branch detect surface defects and missing regions, respectively, with the latter incorporating Generative Adversarial Network Inversion (GAN-Inversion) for the first time to generate normal samples most similar to the input, thereby reducing false positives. Zhu et al. \cite{zhu2024group3ad} introduced the Inter-cluster Uniformity Network (IUN) and Intra-cluster Alignment Network (IAN), which respectively achieve inter-cluster dispersion and intra-cluster alignment in feature space, enhancing the uniformity and consistency of features. Moreover, the adaptive group center selection design focuses on regions with potential issues, prioritizing areas with significant local geometric changes, thereby improving the model's sensitivity.

\subsection{Conclusion and Outlooks}
This paper reviews the methodologies in industrial defect detection, focusing on FM approaches. Section 1 introduces the challenges posed by FM methods. In Section 2, we compare FM and NFM systematically. Section 3 reviews FM methods for 2D and 3D defect detection, while Section 4 summarizes NFM approaches.

Despite progress, several challenges remain, and further exploration is needed in the following areas:

\begin{itemize}
\item \textbf{Improving Detection Accuracy on Single-Scene Datasets:} While FM show impressive generalization across diverse scenarios, there is still a need to optimize their performance on specific scene datasets. Enhancing accuracy for a given dataset requires refining model fine-tuning processes, incorporating scene-specific features, and exploring specialized training techniques, such as transfer learning or domain adaptation. Further investigation into balancing model generalization and overfitting on limited datasets will be crucial to improving single-scene detection accuracy.

\item \textbf{Increasing Inference Speed in Few-Shot and Zero-Shot Scenarios:} FM, due to their extensive parameters, face challenges in inference speed, particularly in few-shot or zero-shot learning contexts. Speed improvement strategies, such as knowledge distillation, quantization, and model pruning, hold promise. Moreover, methods for optimizing inference, like efficient transfer of learned knowledge from large datasets to smaller ones or leveraging feature extraction techniques, could be explored to accelerate inference while maintaining accuracy.

\item \textbf{Enhancing 3D Detection Performance:} The performance of large models in 3D defect detection remains suboptimal, especially in single-scene scenarios. Improving 3D detection requires incorporating advanced 3D data processing methods, such as multi-view fusion, improved point cloud processing, and novel geometric feature extraction techniques. Additionally, coupling these methods with large models could enhance their ability to detect anomalies in complex 3D environments, where context and spatial relationships play a critical role.

\item \textbf{Synthetic Data for Specific 3D Scenarios:} Synthetic data generation, particularly for specific 3D industrial environments, could significantly boost FM performance in these scenarios. By generating diverse, realistic 3D defect samples through simulation or augmentation techniques, we can alleviate data scarcity and improve model robustness. Exploring the synergy between synthetic data and large models, especially in underrepresented or highly specialized 3D defect scenarios, could provide new avenues for training and fine-tuning defect detection models in real-world applications.
\end{itemize}

It is our hope that this survey provides a systematic summary and offers inspiration to readers for conducting research in related fields.

\bibliographystyle{IEEEtran}
\documentclass{MITstyle}

%\usepackage[table]{xcolor}
\usepackage{chngcntr}
\usepackage{hyperref}
\usepackage{microtype}

\title{A Lightweight and Extensible Cell Segmentation and Classification Model for Whole Slide Images}

\author{Nikita Shvetsov~$^{1, }$\footnote{Correspondence e-mail: nikita.shvetsov@uit.no}, Thomas K. Kilvaer~$^{2, 3}$, Masoud Tafavvoghi~$^{4}$, Anders Sildnes~$^{1}$, \\ Kajsa Møllersen~$^{4}$, Lill-Tove Rasmussen Busund~$^{5, 6}$, Lars Ailo Bongo~$^{1}$ \\
%
\vspace{1em} % Space between authors and afilliations
%
\normalfont{\small $^{1}$Department of Computer Science, UiT The Arctic University of Norway}\\
\normalfont{\small $^{2}$Department of Oncology, University Hospital of North Norway}\\
\normalfont{\small $^{3}$Department of Clinical Medicine, UiT The Arctic University of Norway}\\
\normalfont{\small $^{4}$Department of Community Medicine, UiT The Arctic University of Norway}\\
\normalfont{\small $^{5}$Department of Medical Biology, UiT The Arctic University of Norway} \\
\normalfont{\small $^{6}$Department of Clinical Pathology, University Hospital of North Norway} %\vspace{2em}
}

\begin{document}
\maketitle

\section*{Abstract}

% \begin{abstract}
% Developing clinically useful cell-level analysis tools in digital pathology remains challenging due to limitations in dataset granularity, inconsistent annotations, computational demands of advanced models, and difficulties in integrating new technologies into clinical workflows. To address these challenges, we propose a multi-faceted solution that enhances data quality, model performance, and usability to create a lightweight and extensible cell segmentation and classification model.

% First, we update data labels by employing a cross-relabeling process that refines the labels of two existing datasets, PanNuke and MoNuSAC, to create a new unified dataset with enhanced granularity, encompassing seven distinct cell types. Second, we leverage the H-Optimus foundation model as a fixed encoder to improve feature representation for simultaneous cell segmentation and classification tasks. Third, to address the computational demands of foundation models, we employ knowledge distillation to reduce model size and complexity while maintaining comparable performance. Finally, to facilitate integration into clinical workflows, we integrate the distilled model into the QuPath software, a widely used open-source platform in digital pathology.

% Our results demonstrate improvements in cell segmentation and classification performance using the H‑Optimus-based model compared to a CNN-based model. Specifically, the average $R^2$ improved from 0.575 to 0.871, and the average $PQ$ score improved from 0.450 to 0.492, indicating better alignment with actual cell counts and enhanced segmentation and classification quality. Furthermore, the distilled student model maintains performance comparable to the larger foundation model while reducing the parameter count by a factor of 48.
% Overall, by reducing computational complexity and integrating it into existing workflows, the proposed approach may significantly impact diagnostic processes, reduce the workload of pathologists, and contribute to improved patient outcomes. Though our approach shows potential enhancements in efficiency and usability of cell segmentation and classification models in digital pathology, extensive validation is needed to deploy these models in clinical practice.
% \end{abstract}

%%% shortened abstract
\begin{abstract}
Developing clinically useful cell-level analysis tools in digital pathology remains challenging due to limitations in dataset granularity, inconsistent annotations, high computational demands, and difficulties integrating new technologies into workflows. To address these issues, we propose a solution that enhances data quality, model performance, and usability by creating a lightweight, extensible cell segmentation and classification model. 

First, we update data labels through cross-relabeling to refine annotations of PanNuke and MoNuSAC, producing a unified dataset with seven distinct cell types. Second, we leverage the H-Optimus foundation model as a fixed encoder to improve feature representation for simultaneous segmentation and classification tasks. Third, to address foundation models' computational demands, we distill knowledge to reduce model size and complexity while maintaining comparable performance. Finally, we integrate the distilled model into QuPath, a widely used open-source digital pathology platform. 

Results demonstrate improved segmentation and classification performance using the H-Optimus-based model compared to a CNN-based model. Specifically, average $R^2$ improved from 0.575 to 0.871, and average $PQ$ score improved from 0.450 to 0.492, indicating better alignment with actual cell counts and enhanced segmentation quality. The distilled model maintains comparable performance while reducing parameter count by a factor of 48. By reducing computational complexity and integrating into workflows, this approach may significantly impact diagnostics, reduce pathologist workload, and improve outcomes. Although the method shows promise, extensive validation is necessary prior to clinical deployment.
\end{abstract}
\clearpage

\section{Introduction}
In digital pathology, accurate segmentation and classification of cells are crucial for many diagnostic, prognostic, and predictive analyses \cite{Jaber_Beziaeva_etal._2019,Lin_Pan_etal._2022,Park_Ock_etal._2022,Shen_Choi_etal._2024}. Nowadays, developments in computational pathology offer multiple solutions \cite{H._Qu_P._Wu_etal._2020,Javed_Mahmood_etal._2020} to utilize cell-level datasets to train machine learning models that solve these problems. The quality and specificity of training datasets are critical for robust and accurate models. Adhering to the principle of "garbage in, garbage out", it is essential to ensure that these datasets are extensively and accurately labeled with distinct classes that reflect the diverse biological characteristics of different cell types. Unfortunately, the number of open-source datasets comprising such high-quality annotations is limited. Existing cell segmentation datasets \cite{Gamper_Koohbanani_etal._2019,Graham_Vu_etal._2019,Verma_Kumar_etal._2021} may offer extensive annotations for certain cell types while providing more general labels for others. For example, in PanNuke, which is one of the largest open-source datasets comprising labeled cells, various types of morphologically and functionally different inflammatory cells like macrophages and lymphocytes are clustered in a broad "inflammatory" class. Consequently, these classes are frequently omitted from analyses or aggregated into broader meta-classes \cite{Gamper_Koohbanani_etal._2020} and likely interfere with other cell classes included in the dataset. This and similar inconsistencies in annotation granularity limit the ability of machine learning models to learn the comprehensive and nuanced features necessary for accurate cell segmentation and classification. To address these challenges, methods for refining and standardizing dataset annotations are essential to enhance the quality of training data.

A complementary approach to mitigate the absence of high-quality training data is the use of foundation models. Foundation models as encoders are defined as large-scale, versatile networks pre-trained on vast, diverse datasets using self-supervised learning, contrasting with convolutional neural network (CNN) pre-trained encoders that rely on supervised learning with labeled data. In practice, foundation models leverage enormous amounts of weakly or unlabeled data from millions of whole slide images (WSIs) and employ self-attention mechanisms to capture long-range dependencies and global context \cite{Chen_Ding_etal._2024,Saillard_Jenatton_etal._2024,Vorontsov_Bozkurt_etal._2024,Xu_Usuyama_etal._2024}. As a consequence, foundation models are able to produce transferable feature representations across different cell types and tissue environments. The feature representations can be leveraged by decoder networks to produce segmentation masks and pixel-level classifications. Because foundation models have comprehensive feature representations, they can be effectively fine-tuned using much smaller amounts of cell-level data compared to the large datasets needed to train models from scratch. Furthermore, foundation models incorporate adversarial training elements or contrastive learning \cite{Chen_Ding_etal._2024,Xu_Usuyama_etal._2024}, enhancing their resilience and adaptability by exposing them to challenging and varied scenarios during training. This may result in more generalizable models, often making them well-suited for diverse and complex tasks in digital pathology.

Despite the inherent advantages of foundation models, their deployment for practical use faces its own obstacles. In particular, they require substantial computational power, financial investments and rigorous testing to ensure reliability and efficacy for a given task \cite{Akkus_Dangott_etal._2022,Dragomir_Cocuz_etal._2022,Go_2022,Jafri_Farooqui_etal._2024}. Moreover, while foundation models enhance feature representation and performance, they depend on the quality of available annotations for decoder fine-tuning and, like any other model, cannot resolve existing inconsistencies or ambiguities in data labels. Therefore, there remains a critical need for solutions that address both data quality and practical deployment considerations.
Further, integrating new technologies into existing clinical workflows often encounters resistance, as it necessitates adjustments to established diagnostic processes. So, there is a need to develop solutions that could be integrated into current practices, minimizing the burden on medical professionals to adopt new tools \cite{King_Williams_etal._2023}.

Existing solutions \cite{Goldsborough_Philps_etal._2024,Hörst_Rempe_etal._2024}, while addressing some aspects of these challenges, fall short in providing a comprehensive approach. To address the data quality and clinical deployment issues, we propose a multi-faceted solution that encompasses data refinement, model optimization, and integration with existing pathology tools (\hyperref[fig:fig1]{Figure 1}). The outcome is a lightweight cell segmentation and classification model that can be integrated into digital pathology workflows for practical clinical use.

\begin{figure}[h!]
    \centering
    \includegraphics[width=\textwidth, height=0.82\textheight, keepaspectratio]{images/Figure_1.pdf}
    \caption{Overview of the proposed solution, including 1) Data refinement using cross-relabeling, 2) Teacher model development and fine tuning, 3) Student model optimization with knowledge distillation and 4) Student model and QuPath integration}
    \label{fig:fig1}
\end{figure}
\clearpage

Our approach begins with preparing the data for the fine-tuning and training of the machine learning models. We create a refined dataset, acquired via cross-relabeling two cell-level datasets, enhancing annotation specificity and consistency of the labeled data. Subsequently, we create a cell segmentation and classification model based on the foundation model. We leverage the foundation model as a fixed encoder and fine-tune a decoder using the refined dataset to improve generalization across diverse tissue- and cell types.
To ensure that the model remains lightweight and deployable in a possibly resource-constrained environment, we employ knowledge distillation to approximate the functionality of the foundation model. Finally, to facilitate the practical application of our model in digital pathology workflows, we integrate it with the QuPath \cite{Bankhead_Loughrey_etal._2017} application. Each methodological component contributes to the overarching goal of enhancing model performance, generalizability, and usability in clinical settings.

The primary contributions of this paper are:
\begin{enumerate}
    \item \textit{Data labels refinement through cross-relabeling:}
    
    We propose a new method for refining labels of cell-level datasets through cross-relabeling. This method employs classification models to re-label broad and ambiguous instances, resulting in a more diverse dataset. Our evaluation demonstrates that these classification models achieve high accuracy on test subsets, indicating the reliability of the method for label refinement.

    \item \textit{Enhanced model performance via foundation models:}
    
    We employ a foundation model as a feature extractor for the cell segmentation and classification task. In comparison with training a CNN model from scratch, the foundation model backbone only needs fine-tuning, which significantly reduces training time, computational resources and data requirements. We show that using a foundation model encoder leads to better performance in cell segmentation and classification networks than using a CNN-based encoder. This improvement may enable the model to generalize more effectively across various tissue types and imaging methods.
    
    \item \textit{Model optimization through knowledge distillation:}
    
    We show that a smaller student model trained using knowledge distillation on the refined dataset obtained via our cross-relabeling approach from a foundation model achieves comparable performance in cell segmentation and quantification tasks. As a result, this model is more suitable for deployment in environments without high-performance computing resources.
    
    \item \textit{Integration with QuPath:}
    
    We integrate the distilled cell segmentation and classification model into QuPath, a widely used open-source digital pathology platform, to accelerate clinical adaptation by enabling pathologists to more easily incorporate advanced computational tools into their existing workflows.
\end{enumerate}

Through these methodological steps, we aim to bridge the gap between advanced machine learning techniques and practical clinical applications, making accurate and efficient digital pathology accessible in a broader range of healthcare settings.

\section{Refining Existing Datasets Using Cross-Relabeling}
To address the limitations of sparse and ambiguous labeling of cell-level datasets, we propose a generalizable cross-relabeling strategy that can be applied to any dataset containing broadly categorized or imprecisely labeled cell types. This approach involves training and subsequently leveraging classification models to refine broad categories into more specific or biologically relevant classes.
When applied to cell-level data, the methodology includes extracting individual cell images from the dataset patches, preprocessing these images to standardize the size and accommodate partial cells, and then training deep learning classifiers capable of distinguishing between the finer cell subtypes within the coarser categories. 
To illustrate our approach, we focus on the PanNuke \cite{Gamper_Koohbanani_etal._2020, Gamper_Koohbanani_etal._2019} and MoNuSAC \cite{Verma_Kumar_etal._2021} datasets that we have used to train models for cell quantification in our previous works \cite{Shvetsov_Grønnesby_etal._2022,Shvetsov_Sildnes_etal._2024}. We find that for better cell differentiation we have to introduce more granular labels. PanNuke includes a broad classification of "inflammatory" cells, encompassing lymphocytes, macrophages, and neutrophils. Each cell type differs significantly in structure, function, and clinical relevance. Conversely, MoNuSAC uses the label "epithelial" for a class that comprises both benign epithelial cells and malignant neoplastic cells. This practice makes it challenging to differentiate between benign and malignant epithelial cells in the dataset, which is a critical distinction when identifying tumor areas within tissue samples. To address these issues, we implement a cross-relabeling strategy as shown in \hyperref[fig:fig2]{Figure 2}. The key components are two classification models: one is trained on singular cell images from PanNuke data to classify the epithelial meta-class into epithelial and neoplastic classes. The other is trained on MoNuSAC to refine the inflammatory class into lymphocytes, neutrophils, and macrophages.

\begin{figure}[h!]
    \centering
    \includegraphics[width=\textwidth]{images/Figure_2.pdf}
    \caption{Refined dataset generation via cross relabeling}
    \label{fig:fig2}
\end{figure}

The refining approach consists of three consecutive steps. The first is the preprocessing step, in which we extract individual cells from both datasets (\hyperref[fig:fig3]{Figure 3}). The specifics of PanNuke and MoNuSAC patch preparation before cell preprocessing are provided in \hyperref[chap:S1]{Appendix S1}.

\begin{figure}[h!]
    \centering
    \includegraphics[width=\textwidth]{images/Figure_3.pdf}
    \caption{Cell instances preprocessing including (1) cell map extraction, (2) bounding box delineation, (3) adjusting cell boxes and (4) cropping and resizing of cell images}
    \label{fig:fig3}
\end{figure}

During preprocessing, we extract cell type maps from the ground truth label mask and calculate bounding boxes around each cell instance. To accommodate partial cells at patch borders, a common issue in cropped patch images, we employ mirror padding and extend the field of view of the cell label by 15 pixels to capture adjacent cells. We then crop and resize the identified regions to $64 \times 64$ pixels using bicubic interpolation.

The preprocessed PanNuke dataset comprises 68,031 neoplastic and 23,207 epithelial cell images, while MoNuSAC comprises  33,104 lymphocytes, 1,252 neutrophils, and 1,695 macrophages, which we subsequently use in training cell classification models and classifying the cell image data \hyperref[fig:S2]{Appendix Figure S2 (1)}. 

The next step is to train two distinct ResNet50-based classifiers tailored to address the specific labeling challenges inherent in each dataset. We use ResNet50 for classification models due to its proven effectiveness for image classification tasks in histopathology \cite{pan2022reviewmachinelearningapproaches}, and its compatibility with small images. For the PanNuke dataset, we design the classifier, trained on MoNuSAC data, to disaggregate the heterogeneous "inflammatory" cell category into distinct subtypes: lymphocytes, macrophages, and neutrophils. Similarly, for the MoNuSAC dataset, the classifier is trained on PanNuke data and distinguishes between benign and malignant epithelial cells within the overarching "epithelial" label. By applying these targeted classifiers to their respective datasets, we assign more specific labels to individual cell instances, thus enabling us to create a unified dataset.
To ensure a balanced representation of classes, we train both models on datasets that had been equalized to match the size of the least represented class. Thus, we obtain datasets comprising 23,207 samples per class for PanNuke and 1,252 samples per class for MoNuSAC data. Next, we partition both of them into training (70\%), validation (20\%), and testing (10\%) subsets. To mitigate the risk of overfitting, we use a single dropout layer with a rate of p=0.5 in both models and data augmentation using randomized color perturbations, rotation, and horizontal and vertical flipping. We employ AdamW optimizer and the cross-entropy loss function for the training criterion.

To evaluate the two trained models, we measure the classification accuracy on the respective test subsets. The accuracies on the test subset for both classifiers are presented in \hyperref[tab:1]{Table 1}. The PanNuke model achieves an average accuracy of 93.57\%, with higher accuracy for neoplastic cells (96.06\%) compared to epithelial cells (86.26\%). The confusion matrix in Figure A3.1 shows that the model predominantly distinguishes accurately between epithelial and neoplastic tissues, with a substantial number of correct classifications and relatively few misclassifications. The MoNuSAC model demonstrates an average accuracy of 98.92\%, excelling in classifying lymphocytes (99.67\%) and macrophages (94.12\%), with lower performance for neutrophils (85.71\%). The confusion matrix in Figure A3.2 shows that the model identifies lymphocytes and performs reasonably well with macrophages and neutrophils.

\begin{table}[h!]
\renewcommand{\arraystretch}{1.5}
  \centering
  \caption{Cell classification results for PanNuke and MoNuSAC trained models (CI 95\%).}
  \label{tab:1}
  \begin{tabular}{|l|c|c|}
   \hline
   %\rowcolor{gray!30}
    Accuracy               & PanNuke model              & MoNuSAC model              \\
    \hline
    Average      & 0.936 (0.931--0.941)         & 0.989 (0.986--0.993)        \\
    \hline
    Neoplastic   & 0.961 (0.956--0.965)         & -                          \\
    \hline
    Epithelial   & 0.863 (0.849--0.877)         & -                          \\
    \hline
    Lymphocytes  & -                          & 0.997 (0.995--0.999)        \\
    \hline
    Neutrophils  & -                          & 0.857 (0.796--0.918)        \\
    \hline
    Macrophages  & -                          & 0.941 (0.906--0.976)        \\
    \hline
  \end{tabular}
\end{table}

Finally, during the last step, we use the model trained on PanNuke data for epithelial cells in MoNuSAC and the model trained on MoNuSAC for the inflammatory cells class in PanNuke. Specifically, we use classifier models to relabel epithelial cells in MoNuSAC and inflammatory cells in PanNuke data. Then we combine cells with refined labels and the rest of the cells in both datasets to create a refined dataset (\hyperref[fig:S2]{Appendix Figure S2 (2)}). The process of relabeling cells and visualizing them on a patch is shown in \hyperref[fig:fig4]{Figure 4}. The cell counts in the refined dataset are provided in \hyperref[tab:S4]{Appendix Table S4}.

\begin{figure}[h!]
    \centering
    \includegraphics[width=\textwidth, height=0.42\textheight, keepaspectratio]{images/Figure_4.pdf}
    \caption{Cell relabeling procedure for epithelial and inflammatory cell classes}
    \label{fig:fig4}
\end{figure}

%\hfill

Relabeling and combining datasets have been explored in a prior study \cite{Parulekar_Kanwat_etal._2023}, where consecutive fine-tuning on multiple datasets was employed to account for hierarchical class label structures. While the method presented in \cite{Parulekar_Kanwat_etal._2023} is intuitive, it often lacks consistency and requires multiple fine-tuning runs, which can be cumbersome and time-consuming. 
In contrast, cross-relabeling simplifies this process by using specialized classification models tailored to each dataset's specific labeling challenges. This approach provides better transparency and produces a unified dataset encompassing seven distinct cell types across multiple tissue samples, enhancing data diversity for further model training or fine-tuning.

Despite these improvements, cross-relabeling does not entirely resolve issues related to poor labeling quality or the amount of labeled data. Specifically, our results show lower accuracies persist for underrepresented classes, such as macrophages, which may stem from a limited sample availability and intrinsic challenges in distinguishing these cells based solely on H\&E staining. Furthermore, while our method enhances label specificity, it relies on the initial quality of the broad labels; thus, any fundamental inaccuracies in the original annotations can propagate through the relabeling process. Addressing the overall problem of limited data labels may require integrating additional data sources or utilizing complementary immunohistochemical staining methods.
Although the reported performance metrics are obtained from evaluations on the native test sets of each dataset, it is important to note that the primary application of these classifiers is to perform cross-relabeling, where a model trained on one dataset (e.g., PanNuke) is applied to another (e.g., MoNuSAC) and vice versa. We acknowledge that a more systematic evaluation of cross-dataset generalization is needed and could be performed in future work.

Overall, the refined dataset produced by our approach can enhance the supervised training or fine-tuning of cell segmentation and classification models, especially those that utilize pre-trained foundation models to improve feature extraction robustness. In addition, these models can detect nuanced classes that enable researchers to conduct more detailed analyses of biological processes in computational pathology.

\section{Foundation models for robust cell segmentation and classification}

Accurate cell segmentation and classification in digital pathology are hindered by limited labeled data and the fact that conventional CNNs are unable to capture global contextual information due to their local receptive field constraints \cite{Gheflati_Rivaz_2022,Yang_Marcus_etal.}. Traditional approaches in cell quantification have predominantly relied on CNN encoders, such as ResNet50, given their proven effectiveness in semantic segmentation tasks \cite{Deshmane_2023,Graham_Vu_etal._2019,Mukasheva_Koishiyeva_etal._2024,Stringer_Wang_etal._2021}. However, approaches that include fine-tuning of pretrained CNNs, data augmentation, and stain normalization to partially increase data variability and address staining differences often fail to achieve the necessary generalization and robustness across diverse tissue types and staining conditions \cite{G._Wang_W._Li_etal._2018,Gao_Bagci_etal._2018,Karim_El_Khoury_Martin_Fockedey_etal._2021}.

To overcome these challenges, we leverage an encoder-decoder network that uses a foundation model as the encoder and a CNN upsampling decoder (\hyperref[fig:fig5]{Figure 5}) for simultaneous cell segmentation and classification in 2D patches extracted from WSIs. Foundation models with transformer-based architectures are viable alternatives to CNN-based encoders \cite{Shamshad_Khan_etal._2023,Sourget_2023}. They enable the creation of more advanced architectures that can decode or transform learned features more effectively \cite{Chen_Duan_etal._2023,Cheng_Misra_etal._2022,Xie_Wang_etal._2021}.

\begin{figure}[h!]
    \centering
    \includegraphics[width=\textwidth]{images/Figure_5.pdf}
    \caption{UNETR-like model with foundational model as backbone}
    \label{fig:fig5}
\end{figure}

By utilizing a transformer-based encoder, we incorporate global contextual information into the feature extraction process, which is a key advantage of such architectures \cite{Chen_Lu_etal._2021}. This foundation model integration facilitates accurate pixel-wise segmentation and classification without the need for extensive encoder training, thereby potentially improving generalization across varied cellular structures and tissue types.
In our implementation, we employ a modified UNETR \cite{Hatamizadeh_Tang_etal._2021} architecture that combines a vision transformer (ViT) \cite{Dosovitskiy_Beyer_etal._2021} encoder with a CNN-based decoder. The encoder utilizes the pretrained H-Optimus foundation model, which contains 1.1 billion parameters and is trained on over 500,000 H\&E stained WSIs \cite{Saillard_Jenatton_etal._2024}. We extract outputs from four evenly spaced transformer blocks $Z_i$, where $i \in [1, 14, 26, 38]$, to serve as residual connections for the CNN decoder. We select these blocks based on our observation that features from non-adjacent levels of the encoder lead to better overall performance on the test subset.

The CNN decoder upsamples the feature representations, acquired from the transformer blocks, to generate an intermediate vector that is handled by two task-specific layers that generate cell segmentation and classification masks. The first task-specific layer is the ‘Cellpose head’,  which is used to delineate cell instances. The layer generates horizontal and vertical gradient maps to form vector fields that are refined through gradient tracking in a post-processing step using the Cellpose algorithm \cite{Stringer_Wang_etal._2021}, known for its efficacy in cell segmentation tasks and generalizability across multiple domains \cite{Pachitariu_Stringer_2022,Stringer_Pachitariu_2024}. The second task-specific layer is the "Cell type head", which assigns labels to individual pixels. In the post-processing step, we determine the output classification label of each segmented cell instance by majority voting over the labeled pixels that comprise the cell in the segmentation map.

To evaluate model performance and measure the impact of adding a foundation model as backbone, we compare it to a ResNet50-based model. ResNet50 is a widely used solution for encoders in segmentation architectures in the medical domain \cite{Deshmane_2023,Graham_Vu_etal._2019,Mukasheva_Koishiyeva_etal._2024,Stringer_Wang_etal._2021}. For the H-Optimus-based model, we utilize frozen weights for the encoder and only fine-tune the decoder to take advantage of the extensive pre-training of the foundation model. For the ResNet50-based model we start with ImageNet \cite{Deng_Dong_etal.} weights and train both encoder and decoder parts. Hyperparameters for the training step are set to be identical, where possible, for comparable evaluation. 
For this evaluation, we deliberately use the PanNuke dataset to provide a standardized and controlled comparison between the H‑Optimus and ResNet50-based models (\hyperref[fig:S2]{Appendix Figure S2 (3)}). Specifically, we use two of the default PanNuke dataset splits (66\%) for training and validation, and reserve the third split (33\%) for testing.

To address the challenge of cell class imbalance in the PanNuke dataset, which is a common characteristic in most cell-level H\&E patch datasets, both models’ training processes employ a weighted loss function comprising cross-entropy and focal loss \cite{Lin_Goyal_etal._2018}. The focal loss component is adjusted with coefficients derived from each cell class' instance frequency, emphasizing learning from underrepresented classes and enhancing the model's sensitivity to rare but significant cellular patterns. The cross-entropy loss is augmented with spectral decoupling regularization \cite{Pezeshki_Kaba_etal._2021,Pohjonen_Stürenberg_etal._2022} and spatially varying label smoothing \cite{Islam_Glocker_2021}, which potentially stabilizes training and improves generalization in case of complex tissue morphologies. For optimization, we employ the \textit{AdamW} \cite{Loshchilov_Hutter_2019} to counter unbalanced class scenarios, with cosine annealing learning rate scheduler.

We utilize the scikit-learn library \cite{Van_der_Walt_Schönberger_etal._2014} and HoVer-Net \cite{Graham_Vu_etal._2019} implementations of $R^2$ (the coefficient of determination) and $PQ$ (panoptic quality) to evaluate our experiments. Complete mathematical formulations and detailed explanations of these metrics are provided in \hyperref[chap:S5]{Appendix S5}. To compute confidence intervals, we use nonparametric bootstrapping, where after calculating the metric on the full sample, we generated 1000 bootstrap replicates by resampling with replacement and then determined the 95\% confidence intervals as the 2.5th and 97.5th percentiles of the resulting empirical distribution.

%\hfill

The model comparisons are summarized in \hyperref[tab:2]{Table 2}. The H‑Optimus-based model achieves higher $R^2$ across all cell classes compared to the ResNet50-based model, which means that its predictions are more closely aligned with the PanNuke cell counts, indicating a stronger correlation with the observed data. Notably, the improvement of $R^2_{dead}$ may be an indicator of better global contextual representations provided by the foundation model backbone. In terms of segmentation and classification quality combined, measured by the PQ score, the H‑Optimus-based model demonstrates notable improvements across most cell classes. Overall, the average $R^2$ improved from 0.575 to 0.871, while the average $PQ$ score improved from 0.450 to 0.492, demonstrating better performance of the H-Optimus-based model.

\begin{table}[h!]
\renewcommand{\arraystretch}{1.5}
  \centering
  \caption{Cell quantification metrics for baseline and proposed models (CI 95\%).}
  \label{tab:2}
  \begin{tabular}{|l|c|c|}
    \hline
    %\rowcolor{gray!30}
    Metric             & Resnet50-based            & H-optimus-based              \\
    \hline
    $R^2_{neoplastic}$    & 0.681 (0.576--0.769)       & \textbf{0.941 (0.917--0.960)} \\
    \hline
    $R^2_{inflammatory}$  & 0.863 (0.778--0.903)       & \textbf{0.949 (0.918--0.966)} \\
    \hline
    $R^2_{connective}$    & 0.600 (0.488--0.698)       & 0.609 (0.436--0.772)          \\
    \hline
    $R^2_{dead}$          & 0.097 (-11.389--0.669)     & 0.925 (0.404--0.982)          \\
    \hline
    $R^2_{epithelial}$    & 0.635 (0.490--0.747)       & \textbf{0.930 (0.886--0.964)} \\
    \hline
    $PQ_{neoplastic}$       & 0.517 (0.499--0.535)       & \textbf{0.589 (0.575--0.604)} \\
    \hline
    $PQ_{inflammatory}$     & 0.455 (0.429--0.482)       & \textbf{0.528 (0.507--0.549)} \\
    \hline
    $PQ_{connective}$       & 0.416 (0.400--0.431)       & \textbf{0.451 (0.436--0.465)} \\
    \hline
    $PQ_{dead}$             & 0.374 (0.342--0.408)       & 0.292 (0.209--0.365)          \\
    \hline
    $PQ_{epithelial}$       & 0.488 (0.460--0.519)       & \textbf{0.599 (0.579--0.618)} \\
    \hline
  \end{tabular}
\end{table}

Our results  show that integrating the H‑Optimus foundation model within the UNETR architecture enhances the model's ability to segment and classify cells across diverse tissues from PanNuke data. The pretrained transformer encoder provides robust feature representations, resulting in higher average $R^2$ and $PQ$ scores compared to the CNN-based model. This leads to more reliable cell quantification and more accurate downstream analysis. Additionally, the streamlined fine-tuning process reduces computational overhead and training time, making the model more adaptable for new data.

Despite these advancements, the foundation model-based approach does not fully resolve all challenges related to cell segmentation and classification. We observe lower metric scores for underrepresented classes in the training data. Furthermore, foundation models typically encompass billions of parameters, resulting in substantial computational and memory requirements. It therefore poses challenges for deployment in resource-constrained environments, limiting their practical applicability in certain clinical settings.

\section{Model optimization via Knowledge Distillation}

To address the limitations posed by the extensive size of foundation models, we implement knowledge distillation — a model compression technique that leverages the teacher-student paradigm \cite{Hinton_Vinyals_etal._2015}. By training a smaller, more efficient student model to replicate the output of a larger, pre-trained teacher model, we retain performance while significantly reducing the model's complexity and resource requirements (\hyperref[fig:fig6]{Figure 6}).

\begin{figure}[h!]
    \centering
    \includegraphics[width=\textwidth, height=0.45\textheight, keepaspectratio]{images/Figure_6.pdf}
    \caption{Knowledge distillation framework for training a student model using a pre-trained teacher}
    \label{fig:fig6}
\end{figure}

We employ knowledge distillation to compress the H‑Optimus-based teacher model into a more efficient student model. The teacher model is the modified UNETR architecture with the H‑Optimus foundation model described in the previous chapter. The student model is based on a UNet architecture augmented with residual connections and incorporates a smaller ViT encoder with 9 million parameters \cite{Steiner_Kolesnikov_etal._2022,Wightman_2019}. 

First, we fine-tune the teacher model using the refined dataset from the cross-relabeling procedure (Section 2). Initially we train the decoder of the teacher model while keeping the encoder weights frozen. We split the refined dataset into train (70\%), validation (20\%) and test (10\%) subsets (\hyperref[fig:S2]{Appendix Figure S2 (4)}). During fine-tuning, we use the train and validation subsets, while leaving the test subset for model evaluation. We set the training procedure and model hyperparameters to be identical to those that were used to demonstrate the utility of foundation models for the simultaneous cell segmentation and classification task.

Next, we perform knowledge distillation from teacher to student using the refined dataset used to fine-tune the teacher model. The student model is trained to replicate the teacher model's outputs. We utilize a specialized loss function that aligns the student's predicted probability distribution with the teacher's, incorporating the teacher's class probability distribution derived from the output. Following the methodology of Hinton et al. \cite{Hinton_Vinyals_etal._2015}, we experiment with various hyperparameter settings for the temperature ($T$) and the balancing coefficients ($\alpha$ and $\beta$) in the loss function. We vary $T$ from 1 to 20 and adjust $\alpha$ and $\beta$ to balance the distillation and student losses. Through iterative tuning and evaluation, we identify that setting $T=14$, $\alpha=0.3$, and $\beta=0.7$ yields a configuration that converges and closely approximates the teacher model's performance during training.

Finally, we assess the performance of both models using the $R^2$ and $PQ$ (defined in \hyperref[chap:S5]{Appendix S5}) on the test set of the refined dataset (\hyperref[tab:3]{Table 3}). We observe that the 95\% confidence intervals overlap for most cell types, so we cannot claim statistically significant performance differences between the teacher and student models. One exception appears in the neoplastic class. The teacher model produces an $R^2$ of 0.919, while the student model shows an $R^2$ of 0.852. In addition, the student model achieves higher $PQ$ values for the neoplastic and connective classes, though the confidence intervals show overlap.

\begin{table}[h!]
\renewcommand{\arraystretch}{1.5}
  \centering
  \caption{Cell quantification metrics for teacher and distilled student models (CI 95\%).}
  \label{tab:3}
  \begin{tabular}{|l|c|c|}
    \hline
    %\rowcolor{gray!30}
    Metric & Teacher & Student \\
    \hline
    $R^2_{neoplastic}$    & \textbf{0.919} (0.898--0.939) & 0.852 (0.800--0.891) \\
    \hline
    $R^2_{lymphocyte}$    & 0.969 (0.956--0.977)         & 0.969 (0.956--0.978) \\
    \hline
    $R^2_{connective}$    & 0.694 (0.548--0.809)         & 0.618 (0.469--0.741) \\
    \hline
    $R^2_{dead}$          & 0.755 (0.400--0.908)         & 0.424 (0.100--0.731) \\
    \hline
    $R^2_{epithelial}$    & 0.922 (0.870--0.958)         & 0.843 (0.738--0.917) \\
    \hline
    $R^2_{macrophage}$    & 0.384 (-0.369--0.724)        & 0.704 (0.352--0.859) \\
    \hline
    $R^2_{neutrofil}$     & 0.854 (0.578--0.929)         & 0.833 (0.502--0.925) \\
    \hline
    $PQ_{neoplastic}$       & 0.581 (0.569--0.593)         & 0.601 (0.588--0.613) \\
    \hline
    $PQ_{lymphocyte}$       & 0.536 (0.520--0.553)         & 0.563 (0.544--0.579) \\
    \hline
    $PQ_{connective}$       & 0.436 (0.421--0.451)         & 0.457 (0.441--0.474) \\
    \hline
    $PQ_{dead}$             & 0.272 (0.235--0.315)         & 0.279 (0.201--0.369) \\
    \hline
    $PQ_{epithelial}$       & 0.522 (0.500--0.545)         & 0.530 (0.506--0.555) \\
    \hline
    $PQ_{macrophage}$       & 0.524 (0.459--0.588)         & 0.474 (0.405--0.543) \\
    \hline
    $PQ_{neutrofil}$        & 0.541 (0.490--0.592)         & 0.565 (0.522--0.607) \\
    \hline
  \end{tabular}
\end{table}


We further decompose the $PQ$ metric into its $SQ$ and $DQ$ components (\hyperref[tab:S6]{Appendix Table S6}). Both models produce nearly identical $SQ$ values, which indicates that they predict instance boundaries with similar precision. Although the student model shows some improvement in $DQ$ scores for certain classes, the confidence intervals overlap and do not confirm a statistically significant difference.

We observe that the student and teacher models yield comparable detection performance despite the student model using a much smaller and simpler architecture. A model with fewer parameters reduces the risk of overfitting when training data are scarce relative to the model’s complexity \cite{Farias_Ludermir_etal._2022}. The knowledge distillation process also encourages the student model to focus on the most generalizable detection features learned from the teacher. These factors enable the student model to achieve similar detection performance across different cell types.

Additionally, considering the model sizes reported in \hyperref[tab:4]{Table 4}, the distilled model achieves a significant reduction compared to the teacher model, with a 48-fold decrease in parameter count and a 5.5-fold reduction in on-disk size. In inference mode, the teacher model requires 16 GB of VRAM for a batch size of 32, while the distilled model only needs 3 GB of VRAM for the same batch size. These reductions make the distilled model significantly more practical for fine-tuning and deployment in resource-constrained environments.

\begin{table}[h!]
\renewcommand{\arraystretch}{1.5}
  \centering
  \caption{Parameter counts and size of teacher and distilled model}
  \label{tab:4}
  \adjustbox{max width=\textwidth}{%
  \begin{tabular}{|l|c|c|c|}
    \hline
    %\rowcolor{gray!30}
    Metric & H-optimus-based (Teacher) & mobileViT-based (Student) & Magnitude of difference \\
    \hline
    Parameters count       & 1,158,917,906   & \textbf{24,093,393}   & \textbf{48x}  \\
    \hline
    Estimated Total Size (MB) & 87,912       & \textbf{15,935}    & \textbf{5.5x} \\
    \hline
  \end{tabular}%
}
\end{table}

%\hfill

With recent advancements in complex network architectures and the use of pretrained encoders to achieve state-of-the-art performance \cite{Baumann_Dislich_etal._2024,Hörst_Rempe_etal._2024} in cell segmentation and classification tasks, model size, computational complexity, and processing times have increased. This limits the scalability and accessibility of these models. As we demonstrate, this may be mitigated using knowledge distillation. Studies in the field of natural language processing have demonstrated the efficacy of knowledge distillation in retaining the capabilities of the teacher model while achieving significant reductions in size and complexity \cite{Huangpu_Gao_2024,Sun_Yu_etal.}. 

We demonstrate the feasibility of knowledge distillation in digital pathology, specifically for cell segmentation and classification tasks. Moreover, we achieve this performance while also significantly reducing the parameter count. In addressing the challenge of knowledge transfer, we found that distillation from a transformer-based model to a smaller transformer is more straightforward than attempting to map transformer features to CNN blocks. In our experiments, using a CNN-based network as a student results in worse cell quantification performance due to the structural constraints of CNN feature space dimensions. 

Although our primary approach relies on a transformer-based student model that performs well, it can be further optimized to incorporate advantages from CNN architectures. For example, employing alternative techniques such as using ViT adapters \cite{Chen_Duan_etal._2023} or $1 \times 1$ convolutions to adjust feature map sizes may be beneficial for harnessing CNN advantages like enhanced local feature extraction. Moreover, if additional performance improvements are desired, the process can be further enhanced by applying supplementary knowledge distillation techniques, such as self-distillation \cite{Zhang_Song_etal._2019} or online distillation \cite{Houyon_Cioppa_etal._2023}.

Despite these promising results, further validation on independent datasets is necessary to fully understand the model's limitations. Underrepresented classes may pose challenges when addressing complex cases. Pathologists need to validate these models to adopt them in clinical settings. While the distilled models are smaller and more deployable, a technological gap persists because pathologists traditionally rely on established methods for inspecting WSIs and diagnosing diseases. Addressing the complexities involved in deploying models for inference and supporting pathologists in adopting new tools is essential for integrating these models into clinical workflows.

\section{Model integration with QuPath}
Digital pathology tools with graphical user interfaces are essential for visualizing and analyzing WSIs. To make our student model useful in clinical pathology workflows, it needs to be integrated into a tool that enables inspecting regions, creating annotations, and providing quantitative analyses of biomarkers. Therefore, we integrate the trained student model from the previous chapter into the QuPath open‑source platform \cite{Bankhead_Loughrey_etal._2017}. QuPath provides the required annotation, visualization, and analysis tools to interpret complex histological data, including workflows for cell segmentation, classification, and quantification (\hyperref[fig:fig7]{Figure 7}). 

\begin{figure}[h!]
    \centering
    \includegraphics[width=\textwidth]{images/Figure_7.pdf}
    \caption{Visualization of model-generated cell quantification annotations (left) and the corresponding unannotated slide (right) in QuPath}
    \label{fig:fig7}
\end{figure}

To identify the regions in a WSI critical for prognosticating tumor development, such as specific tumor areas or border regions without overlapping healthy tissue, the pathologist uses QuPath to outline these regions. Then, the pathologist initiates a cell segmentation and classification script through the QuPath interface for the selected regions. The resulting annotations and quantified cell information are then directly overlaid onto the WSI in the QuPath interface. Additional design and implementation details are in \hyperref[chap:S7]{Appendix S7}. 

Two common approaches for integrating deep learning models into QuPath are Java‑based native QuPath extensions \cite{Goldsborough_Philps_etal._2024} and the execution of RESTful API requests to a model server coupled with handling the response via an extension, as demonstrated in the application of cell segmentation models applied to immunofluorescence images \cite{Sugawara_2023}. While the community is actively working on these integration strategies, there is currently no universal solution that fully addresses all integration and performance requirements.

Extensions may offer better integration with QuPath, allowing slightly improved performance and more widespread usage of the built-in QuPath models, but they lack the flexibility to customize models and modify their behavior. For example, the newest version of QuPath includes models such as StarDist \cite{Weigert_Schmidt} and InstanSeg \cite{Goldsborough_Philps_etal._2024} that can perform cell segmentation. Both models pose limitations when applied to simultaneous cell segmentation and classification. StarDist performs well only on convex, round shapes by design, whereas some neoplastic, inflammatory, and connective cells exhibit complex and non-convex shapes. InstanSeg provides only semantic segmentation without assigning classes to the segmented cells.

%\hfill

In contrast, our approach offers an alternative integration strategy. It utilizes the paquo library to directly interact with QuPath’s internal application programming interface from within Python. This enables data exchange and processing without the need for intermediate conversion steps and provides greater control over model customization, retraining, and the incorporation of custom processing steps.

The integration of our custom model with QuPath underscores its potential to significantly enhance the diagnostic process by reducing the time burden on pathologists and enabling them to focus on more complex interpretative tasks using familiar software. Leveraging a tool that is already well-established among pathologists increases the likelihood of its adoption into daily clinical workflows. The quantitative data generated through the automated workflow is critical for both clinical decision-making and research, facilitating more accurate biomarker analysis, enabling robust statistical evaluations, and supporting hypothesis generation and testing. Additionally, by streamlining cell segmentation and classification, the tool enhances the scalability and reproducibility of pathological assessments, ultimately contributing to improved diagnostic accuracy and patient outcomes.

\section{Conclusion and future work}

In this study, we address critical challenges in digital pathology and tackle the usability and deployment issues of the developed models in standard computing environments without the need for high-performance computing systems. Our multi-faceted approach encompasses data refinement through cross-relabeling, leveraging foundation models for robust cell segmentation and classification, optimizing model performance via knowledge distillation, and integrating the optimized model into the QuPath software for practical application. This approach is used to construct a capable, versatile, and adjustable model for cell segmentation and classification, with enhanced performance and usability.

\begin{sloppypar}
While our approach shows potential in the field of computational pathology, certain limitations persist. 
For example, our implementation currently exhibits lower performance in detecting macrophages. 
This serves as an instance of the broader challenge of accurately identifying complex cell types. In order to address this issue, extending our approach to incorporate additional data sources, exploring alternative modeling approaches, and integrating other imaging modalities such as immunohistochemical staining may help improve detection accuracy. Moreover, although the distilled model reduces computational demands, integrating advanced deep learning models into clinical practice requires addressing technological gaps and potential resistance to adopting new tools within established diagnostic processes.
\end{sloppypar}

Future work could focus on several key areas to refine the proposed approach and facilitate its adoption in clinical environments. Enhancing the cell-relabeling process with additional datasets \cite{Graham_Jahanifar_etal._2021} could improve the representation of underrepresented cell types and enhance overall model performance. Also, incorporating additional data sources, such as multi-modal imaging or complementary staining methods, may address limitations related to cell type differentiation and class imbalance. Exploring other foundation models \cite{Vorontsov_Bozkurt_etal._2024,Zimmermann_Vorontsov_etal._2024} or introducing additional modalities \cite{Ding_Wagner_etal._2024,Vaidya_Zhang_etal._2025} may provide alternative architectures better suited to specific tasks or offer improved efficiency. Implementing more complex knowledge distillation techniques \cite{Houyon_Cioppa_etal._2023,Zhang_Song_etal._2019} could further optimize the model's performance and adaptability. Additionally, deeper integration with QuPath or other digital pathology software could provide pathologists more control over cell quantification analysis directly within the QuPath interface, thereby increasing accessibility and usability. Such enhancements would not only refine model performance but also ensure greater adaptability and scalability within various clinical environments. Finally, extensive validation of the model by pathologists and benchmarking against independent datasets are essential steps toward establishing the model's reliability and fostering confidence in its clinical utility.

\section*{Acknowledgments} 
This work was funded in part by the Research Council of Norway grant no. 309439 SFI Visual Intelligence, and the North Norwegian Health Authority grant no. HNF1521-20.

\bibliographystyle{IEEEtran}
\begin{sloppypar}
\begin{thebibliography}{99}

\bibitem{chaplot2020neural} Chaplot, Devendra Singh, et al. "Neural topological slam for visual navigation." Proceedings of the IEEE/CVF conference on computer vision and pattern recognition. 2020.

\bibitem{maksymets2021thda} Maksymets, Oleksandr, et al. "Thda: Treasure hunt data augmentation for semantic navigation." Proceedings of the IEEE/CVF International Conference on Computer Vision. 2021.

\bibitem{mezghan2022memory} Mezghan, Lina, et al. "Memory-augmented reinforcement learning for image-goal navigation." 2022 IEEE/RSJ International Conference on Intelligent Robots and Systems (IROS). IEEE, 2022.

\bibitem{al2022zero} Al-Halah, Ziad, Santhosh Kumar Ramakrishnan, and Kristen Grauman. "Zero experience required: Plug \& play modular transfer learning for semantic visual navigation." Proceedings of the IEEE/CVF Conference on Computer Vision and Pattern Recognition. 2022.

\bibitem{ye2021auxiliary} Ye, Joel, et al. "Auxiliary tasks and exploration enable objectgoal navigation." Proceedings of the IEEE/CVF international conference on computer vision. 2021.

\bibitem{chaplot2020object} Chaplot, Devendra Singh, et al. "Object goal navigation using goal-oriented semantic exploration." Advances in Neural Information Processing Systems 33 (2020)

\bibitem{ramakrishnan2022poni} Ramakrishnan, Santhosh Kumar, et al. "Poni: Potential functions for objectgoal navigation with interaction-free learning." Proceedings of the IEEE/CVF Conference on Computer Vision and Pattern Recognition. 2022.

\bibitem{ramrakhya2022habitat} Ramrakhya, Ram, et al. "Habitat-web: Learning embodied object-search strategies from human demonstrations at scale." Proceedings of the IEEE/CVF Conference on Computer Vision and Pattern Recognition. 2022.

\bibitem{mousavian2019visual} Mousavian, Arsalan, et al. "Visual representations for semantic target driven navigation." 2019 International Conference on Robotics and Automation (ICRA). IEEE, 2019.

\bibitem{dhariwal2021diffusion} Dhariwal, Prafulla, and Alexander Nichol. "Diffusion models beat gans on image synthesis." Advances in neural information processing systems 34 (2021)

\bibitem{ho2022classifier} Ho, Jonathan, and Tim Salimans. "Classifier-free diffusion guidance." arXiv preprint arXiv:2207.12598 (2022).

\bibitem{nichol2021glide} Nichol, Alex, et al. "Glide: Towards photorealistic image generation and editing with text-guided diffusion models." arXiv preprint arXiv:2112.10741 (2021)

\bibitem{brooks2023instructpix2pix} Brooks, Tim, Aleksander Holynski, and Alexei A. Efros. "Instructpix2pix: Learning to follow image editing instructions." Proceedings of the IEEE/CVF Conference on Computer Vision and Pattern Recognition. 2023.

\bibitem{fu2023guiding} Fu, Tsu-Jui, et al. "Guiding instruction-based image editing via multimodal large language models." arXiv preprint arXiv:2309.17102 (2023).

\bibitem{geng2024instructdiffusion} Geng, Zigang, et al. "Instructdiffusion: A generalist modeling interface for vision tasks." Proceedings of the IEEE/CVF Conference on Computer Vision and Pattern Recognition. 2024.

\bibitem{zhou2024minedreamer} Zhou, Enshen, et al. "Minedreamer: Learning to follow instructions via chain-of-imagination for simulated-world control." arXiv preprint arXiv:2403.12037 (2024).

\bibitem{zhou2023esc} Zhou, Kaiwen, et al. "Esc: Exploration with soft commonsense constraints for zero-shot object navigation." International Conference on Machine Learning. PMLR, 2023.

\bibitem{yu2023l3mvn} Yu, Bangguo, Hamidreza Kasaei, and Ming Cao. "L3mvn: Leveraging large language models for visual target navigation." 2023 IEEE/RSJ International Conference on Intelligent Robots and Systems (IROS). IEEE, 2023.

\bibitem{gadre2023cows} Gadre, Samir Yitzhak, et al. "Cows on pasture: Baselines and benchmarks for language-driven zero-shot object navigation." Proceedings of the IEEE/CVF Conference on Computer Vision and Pattern Recognition. 2023.

\bibitem{shah2023navigation} Shah, Dhruv, et al. "Navigation with large language models: Semantic guesswork as a heuristic for planning." Conference on Robot Learning. PMLR, 2023.

\bibitem{cai2024bridging} Cai, Wenzhe, et al. "Bridging zero-shot object navigation and foundation models through pixel-guided navigation skill." 2024 IEEE International Conference on Robotics and Automation (ICRA). IEEE, 2024.

\bibitem{yu2023co} Yu, Bangguo, Hamidreza Kasaei, and Ming Cao. "Co-NavGPT: Multi-robot cooperative visual semantic navigation using large language models." arXiv preprint arXiv:2310.07937 (2023).

\bibitem{wu2024voronav} Wu, Pengying, et al. "Voronav: Voronoi-based zero-shot object navigation with large language model." arXiv preprint arXiv:2401.02695 (2024).

\bibitem{qin2023mp5} Qin, Yiran, et al. "Mp5: A multi-modal open-ended embodied system in minecraft via active perception." arXiv preprint arXiv:2312.07472 (2023).

\bibitem{du2024learning} Du, Yilun, et al. "Learning universal policies via text-guided video generation." Advances in Neural Information Processing Systems 36 (2024).

\bibitem{ajay2024compositional} Ajay, Anurag, et al. "Compositional foundation models for hierarchical planning." Advances in Neural Information Processing Systems 36 (2024).

\bibitem{liang2024skilldiffuser} Liang, Zhixuan, et al. "Skilldiffuser: Interpretable hierarchical planning via skill abstractions in diffusion-based task execution." Proceedings of the IEEE/CVF Conference on Computer Vision and Pattern Recognition. 2024.

\bibitem{heusel2017gans} Heusel, Martin, et al. "Gans trained by a two time-scale update rule converge to a local nash equilibrium." Advances in neural information processing systems 30 (2017).

\bibitem{zhang2018unreasonable} Zhang, Richard, et al. "The unreasonable effectiveness of deep features as a perceptual metric." Proceedings of the IEEE conference on computer vision and pattern recognition. 2018.

\bibitem{brown2020language} Brown, Tom B. "Language models are few-shot learners." arXiv preprint arXiv:2005.14165 (2020).

\bibitem{podell2023sdxl} Podell, Dustin, et al. "Sdxl: Improving latent diffusion models for high-resolution image synthesis." arXiv preprint arXiv:2307.01952 (2023).

\bibitem{brohan2022rt} Brohan, Anthony, et al. "Rt-1: Robotics transformer for real-world control at scale." arXiv preprint arXiv:2212.06817 (2022).

\bibitem{brohan2023rt} Brohan, Anthony, et al. "Rt-2: Vision-language-action models transfer web knowledge to robotic control." arXiv preprint arXiv:2307.15818 (2023).

\bibitem{li2024manipllm} Li, Xiaoqi, et al. "Manipllm: Embodied multimodal large language model for object-centric robotic manipulation." Proceedings of the IEEE/CVF Conference on Computer Vision and Pattern Recognition. 2024.

\bibitem{shah2023vint} Shah, Dhruv, et al. "ViNT: A foundation model for visual navigation." arXiv preprint arXiv:2306.14846 (2023).

\bibitem{liu2024visual} Liu, Haotian, et al. "Visual instruction tuning." Advances in neural information processing systems 36 (2024).

\bibitem{hu2021lora} Hu, Edward J., et al. "Lora: Low-rank adaptation of large language models." arXiv preprint arXiv:2106.09685 (2021).

\bibitem{qin2023supfusion} Qin, Yiran, et al. "SupFusion: Supervised LiDAR-camera fusion for 3D object detection." Proceedings of the IEEE/CVF International Conference on Computer Vision. 2023.

\bibitem{qin2024worldsimbench} Qin, Yiran, et al. "Worldsimbench: Towards video generation models as world simulators." arXiv preprint arXiv:2410.18072 (2024).

\bibitem{yu2025gamefactory} Yu, Jiwen, et al. "GameFactory: Creating New Games with Generative Interactive Videos." arXiv preprint arXiv:2501.08325 (2025).

\bibitem{zhou2024code} Zhou, Enshen, et al. "Code-as-Monitor: Constraint-aware Visual Programming for Reactive and Proactive Robotic Failure Detection." arXiv preprint arXiv:2412.04455 (2024).

\bibitem{zhang2024ad} Zhang, Zaibin, et al. "AD-H: Autonomous Driving with Hierarchical Agents." arXiv preprint arXiv:2406.03474 (2024).

\bibitem{wang2024toward} Wang, Chaoqun, et al. "Toward Accurate Camera-based 3D Object Detection via Cascade Depth Estimation and Calibration." arXiv preprint arXiv:2402.04883 (2024).

\bibitem{huang2024story3d} Huang, Yuzhou, et al. "Story3d-agent: Exploring 3d storytelling visualization with large language models." arXiv preprint arXiv:2408.11801 (2024).

\bibitem{savinov2018semi} Savinov, Nikolay, Alexey Dosovitskiy, and Vladlen Koltun. "Semi-parametric topological memory for navigation." arXiv preprint arXiv:1803.00653 (2018).

\bibitem{majumdar2022zson} Majumdar, Arjun, et al. "Zson: Zero-shot object-goal navigation using multimodal goal embeddings." Advances in Neural Information Processing Systems 35 (2022): 32340-32352.

\bibitem{yadav2023offline} Yadav, Karmesh, et al. "Offline visual representation learning for embodied navigation." Workshop on Reincarnating Reinforcement Learning at ICLR 2023. 2023.

\bibitem{yadav2023ovrl} Yadav, Karmesh, et al. "Ovrl-v2: A simple state-of-art baseline for imagenav and objectnav." arXiv preprint arXiv:2303.07798 (2023).

\bibitem{sun2024fgprompt} Sun, Xinyu, et al. "FGPrompt: fine-grained goal prompting for image-goal navigation." Advances in Neural Information Processing Systems 36 (2024).

\bibitem{zhu2017target} Zhu, Yuke, et al. "Target-driven visual navigation in indoor scenes using deep reinforcement learning." 2017 IEEE international conference on robotics and automation (ICRA). IEEE, 2017.

\bibitem{koh2024generating} Koh, Jing Yu, Daniel Fried, and Russ R. Salakhutdinov. "Generating images with multimodal language models." Advances in Neural Information Processing Systems 36 (2024).

\bibitem{krantz2022instance} Krantz, Jacob, et al. "Instance-specific image goal navigation: Training embodied agents to find object instances." arXiv preprint arXiv:2211.15876 (2022).

\bibitem{schulman2017proximal} Schulman, John, et al. "Proximal policy optimization algorithms." arXiv preprint arXiv:1707.06347 (2017).

\bibitem{anderson2018evaluation} Anderson, Peter, et al. "On evaluation of embodied navigation agents." arXiv preprint arXiv:1807.06757 (2018).

\bibitem{lin2024navcot} Lin, Bingqian, et al. "NavCoT: Boosting LLM-Based Vision-and-Language Navigation via Learning Disentangled Reasoning." arXiv preprint arXiv:2403.07376 (2024).

\bibitem{NavGPT} Zhou, Gengze, Yicong Hong, and Qi Wu. "Navgpt: Explicit reasoning in vision-and-language navigation with large language models." Proceedings of the AAAI Conference on Artificial Intelligence.

\bibitem{hahn2021no} Hahn, Meera, et al. "No rl, no simulation: Learning to navigate without navigating." Advances in Neural Information Processing Systems 34 (2021): 26661-26673.

\bibitem{li2025t2isafety} Li, Lijun, et al. "T2ISafety: Benchmark for Assessing Fairness, Toxicity, and Privacy in Image Generation." arXiv preprint arXiv:2501.12612 (2025).

\bibitem{an2024agfsync} An, Jingkun, et al. "AGFSync: Leveraging AI-Generated Feedback for Preference Optimization in Text-to-Image Generation." arXiv preprint arXiv:2403.13352 (2024).


\end{thebibliography}
\end{sloppypar}

\clearpage
\beginsupplement
\section*{Appendix}
\renewcommand{\thesubsection}{S\arabic{subsection}}

\subsection{\label{chap:S1}PanNuke and MoNuSAC preprocessing}
The PanNuke dataset comprises a set of 7,901 RGB patches, each with dimensions of $256 \times 256$ pixels, which we set as the standard patch size for our analysis. In contrast, the MoNuSAC dataset encompasses 294 images of heterogeneous dimensions. To standardize the MoNuSAC images with our experiments, we implement a standardization protocol. Specifically, for images exceeding the dimensions of $256 \times 256$ pixels, we segment them into equal-sized patches and apply mirror padding to the remaining portions to avoid information loss at the peripherals. Patches with dimensions less than $128 \times 128$ pixels are excluded from the dataset due to the insufficient resolution to capture relevant cellular details. For patches where either dimension falls between 128 and 256 pixels, we employ upsampling to achieve the standard patch size. As a result, we obtain a total of 2,823 RGB patches derived from the MoNuSAC dataset for subsequent analysis. For additional details on the MoNuSAC data preparation process, refer to the source code \cite{Shvetsov_2025a}.
\clearpage

\subsection{\label{chap:S2}Data usage for the methodology}

\counterwithin{figure}{subsection}
\renewcommand{\thefigure}{S\arabic{subsection}}

\begin{figure}[h!]
    \centering
    \includegraphics[width=\textwidth, height=0.85\textheight, keepaspectratio]{images/A2.pdf}
    \caption{Overview of the methodology for cross-labeling, dataset refinement, and model comparison. (1) Cross-relabeling - training and testing cell classification models, (2) Cross-relabeling - using cell classification models to create refined dataset, (3) Fine-tuning and training models for comparison, (4) Student knowledge distillation with refined dataset}
    \label{fig:S2}
\end{figure}
\clearpage

\subsection{\label{chap:S3}Confusion matrices for classification models}
\counterwithin{figure}{subsection}
\renewcommand{\thefigure}{S\arabic{subsection}.\arabic{figure}}

\begin{figure}[h!]
    \centering
    \includegraphics[width=\textwidth, height=0.4\textheight, keepaspectratio]{images/A3_1.pdf}
    \caption{Confusion matrix for PanNuke trained model}
    \label{fig:S3.1}
\end{figure}

\begin{figure}[h!]
    \centering
    \includegraphics[width=\textwidth, height=0.4\textheight, keepaspectratio]{images/A3_2.pdf}
    \caption{Confusion matrix for MoNuSAC trained model}
    \label{fig:S3.2}
\end{figure}

\clearpage

\subsection{\label{chap:S4}Datasets cell counts}

\counterwithin{table}{subsection}
\renewcommand{\thetable}{S\arabic{subsection}}

\begin{table}[h!]
\renewcommand{\arraystretch}{2.0}
\centering
\caption{\label{tab:S4}Cell counts for PanNuke, MoNuSAC and refined datasets. Numbers in parentheses indicate preprocessed cell counts for cell classifier models training and testing.}
%\adjustbox{max width=\textwidth}{%
\begin{tabular}{|l|c|c|c|}
\hline
%\rowcolor{gray!30}
Cell type & PanNuke & MoNuSAC & Refined \\
\hline
Neoplastic & 77,403 (68,031) & - & 105,451 \\
\hline
Epithelial & 26,572 (23,207) & - & 29,926 \\
\hline
Epithelial (benign and malignant) & - & 31,402 & - \\
\hline
Inflammatory & 32,276 & - & - \\
\hline
Lymphocytes & - & 37,045 (33,104) & 65,275 \\
\hline
Neutrophils & - & 1,355 (1,252) & 3,833 \\
\hline
Macrophage & - & 1,842 (1,695) & 3,410 \\
\hline
Dead & 2,908 & - & 2,908 \\
\hline
Connective & 50,585 & - & 50,585 \\
\hline
\end{tabular}
%
%}
\end{table}



\clearpage

\subsection{\label{chap:S5}Definition of validation metrics}
\counterwithin{equation}{subsection}
\renewcommand{\theequation}{\arabic{equation}}

\subsubsection{\label{chap:S5.1}R\textsuperscript{2}}
The coefficient of determination, denoted as $R^2$, is a statistical measure that represents the proportion of variance in the dependent variable that is predictable from the independent variables. In the context of cell quantification in pathology, $R^2$ is used to assess how well the predicted quantities of different cell types in a patch align with the actual quantities observed in the ground truth data, with higher values representing more accurate quantification. $R^2$ is defined as
\begin{equation*}
R^2 = 1 - \frac{\sum_{i=1}^n (y_i - \hat{y}_i)^2}{\sum_{i=1}^n (y_i - \bar{y})^2},
\end{equation*}
where $y_i$ represents the actual number of cells of a specific type in the $i$-th image, $\hat{y}_i$ represents the predicted number of cells of that type in the $i$-th image, $\bar{y}$ is the mean of the actual numbers across all images, and $n$ is the total number of images in the dataset.

The $R^2$ metric has a range of $(-\infty, 1]$. An $R^2$ of 1 indicates perfect prediction, where all predicted values exactly match the actual values. An $R^2$ of 0 suggests that the model explains none of the variability of the response data around its mean. If $R^2$ is negative, it indicates that the model performs worse than a model that simply predicts the mean of the actual values for all observations.

\subsubsection{\label{chap:S5.2}PQ}
Panoptic Quality ($PQ$) is a comprehensive metric used to evaluate the performance of segmentation models in tasks that require both instance segmentation and classification. $PQ$ provides a single score that encapsulates both the detection accuracy (i.e., how many objects were correctly identified) and the segmentation quality (i.e., how accurately the objects' boundaries were delineated). This metric is particularly useful in multiclass scenarios where each pixel is classified into distinct categories, such as different cell types in pathology images.

$PQ$ is calculated as the product of two terms: Detection Quality ($DQ$) and Segmentation Quality ($SQ$). It can be expressed as
\begin{equation*}
PQ = DQ \cdot SQ,
\end{equation*}
where
\begin{equation*}
DQ = \frac{TP}{TP + 0.5\, FP + 0.5\, FN},
\end{equation*}
\begin{equation*}
SQ = \frac{\sum_{(p, g) \in \mathcal{M}} IoU(p, g)}{TP}.
\end{equation*}
In these formulas, $TP$ denotes the number of correctly matched instances between ground truth and prediction, $FP$ denotes the predicted instances that have no corresponding ground truth, $FN$ denotes the ground truth instances that were not detected, $IoU(p, g)$ is the Intersection over Union for a pair of matched instances $p$ (prediction) and $g$ (ground truth), and $\mathcal{M}$ is the set of matched pairs.

The $PQ$ metric is calculated for each class and is averaged across classes to provide a global performance measure.

The $PQ$ score has a range of $[0, 1.0]$, where a higher score indicates better performance in both detecting and segmenting the instances correctly. A $PQ$ of 1 signifies perfect identification and segmentation of all instances, whereas a $PQ$ of 0 indicates that no instances were correctly identified and segmented.

\clearpage

\subsection{\label{chap:S6}Segmentation and Detection quality metrics for teacher and student models}

\begin{table}[h!]
\renewcommand{\arraystretch}{2.0}
\centering
\caption{Segmentation and detection quality for student and teacher models (CI 95\%)}
\label{tab:S6}
%\adjustbox{max width=\textwidth}{%
\begin{tabular}{|l|c|c|}
\hline
%\rowcolor{gray!30}
Metric & Teacher & Student \\
\hline
$SQ_{neoplastic}$ & 0.819 (0.815--0.823) & 0.824 (0.819--0.828) \\
\hline
$SQ_{lymphocyte}$ & 0.795 (0.788--0.802) & 0.790 (0.783--0.796) \\
\hline
$SQ_{connective}$ & 0.770 (0.762--0.776) & 0.780 (0.772--0.786) \\
\hline
$SQ_{dead}$ & 0.659 (0.623--0.688) & 0.657 (0.624--0.695) \\
\hline
$SQ_{epithelial}$ & 0.780 (0.770--0.790) & 0.788 (0.779--0.797) \\
\hline
$SQ_{macrophage}$ & 0.788 (0.760--0.810) & 0.757 (0.730--0.783) \\
\hline
$SQ_{neutrofil}$ & 0.782 (0.761--0.801) & 0.775 (0.759--0.792) \\
\hline
$DQ_{neoplastic}$ & 0.706 (0.692--0.719) & 0.727 (0.712--0.741) \\
\hline
$DQ_{lymphocyte}$ & 0.675 (0.656--0.698) & 0.713 (0.691--0.734) \\
\hline
$DQ_{connective}$ & 0.566 (0.546--0.584) & 0.583 (0.565--0.602) \\
\hline
$DQ_{dead}$ & 0.410 (0.361--0.465) & 0.435 (0.306--0.561) \\
\hline
$DQ_{epithelial}$ & 0.668 (0.639--0.694) & 0.673 (0.644--0.702) \\
\hline
$DQ_{macrophage}$ & 0.657 (0.583--0.727) & 0.615 (0.531--0.703) \\
\hline
$DQ_{neutrofil}$ & 0.691 (0.625--0.753) & 0.729 (0.679--0.778) \\
\hline
\end{tabular}
%
%}
\end{table}

\clearpage

\subsection{\label{chap:S7}QuPath integration method}
We adopt an integration strategy leveraging the paquo \cite{Bayer_AG} library, a Python package that enables direct interaction with QuPath’s internal API, thereby facilitating seamless data exchange without intermediate conversion steps. The data processing pipeline (\hyperref[fig:S7]{Appendix Figure S7}) begins with the acquisition of WSIs and their associated annotations from QuPath, which are represented as Shapely \cite{Gillies_Wel_etal._2024} polygons. Utilizing paquo, we directly read, create, and modify these annotations and detections within a QuPath project in the Python environment. Images are then cropped using these polygons and processed by cell segmentation and classification models employing standard vision processing toolkits such as OpenCV, pyvips, and PyTorch. Additionally, QuPath employs Groovy scripts to initiate a Python process that starts the entire pipeline from QuPath graphical interface: fetching polygons, extracting images from them, and running deep learning model inference on the cropped images. 
The results are returned to QuPath, leveraging paquo's Python bindings to manipulate QuPath data while minimizing the computational overhead typically associated with cross-environment communication.

\counterwithin{figure}{subsection}
\renewcommand{\thefigure}{S\arabic{subsection}}

\begin{figure}[h!]
    \centering
    \includegraphics[width=\textwidth]{images/A7.pdf}
    \caption{QuPath integration workflow using Python environment}
    \label{fig:S7}
\end{figure}

Compared to traditional workflows that involve exporting annotations as GeoJSON, classifying them in Python, and reimporting them into QuPath, our approach offers several advantages. We eliminate the need to switch between programming languages, providing a cohesive and streamlined development process entirely within QuPath software and removing the necessity to use other tools. Meanwhile, we avoid storing annotations as intermediate JSON files unless required for external use or archiving. By conducting the entire inference and post-processing workflow within the Python environment, we leverage the power and flexibility of Python libraries for image processing and machine learning. This approach also enables adjustments to any set of labels and models, thereby improving its applicability.

%\hfill

The distilled model and QuPath integration code are packaged into a Docker container, enabling streamlined execution with the Docker engine. Detailed integration code and deployment instructions can be found in the GitHub repository \cite{Shvetsov_2025b}.

Despite these benefits, we acknowledge that the paquo library is a proof‑of‑concept project in its early development stage and has not been tested across all versions of QuPath.

\clearpage

\subsection{\label{chap:S8}Data and code availability statement}
All datasets, models, and code used in this study are publicly available and can be obtained from the repositories listed below. 
The PanNuke \cite{Gamper_Koohbanani_etal._2019} and MoNuSAC \cite{Verma_Kumar_etal._2021} datasets are publicly accessible, and download information along with detailed descriptions can be found in their respective articles. Preprocessing scripts for PanNuke and MoNuSAC data, as well as individual cell extraction scripts, are available on GitHub \cite{Shvetsov_2025a}. The H-Optimus foundation model used in our experiments can be downloaded from the HuggingFace repository \cite{hoptimus2024}, and model information is available on GitHub \cite{Saillard_Jenatton_etal._2024}. In addition, the integration code for QuPath and the distilled model packaged in a Docker container are provided in the repository \cite{Shvetsov_2025b}, and paquo Python library is available from the authors GitHub repository \cite{Bayer_AG}.
\clearpage

\end{document}


\clearpage
\begin{appendices}
\appendix[RESOURCES]
We collect open-source information for FM and NFM methods, including the paper URL, code address (Github), and deep learning tools. Table 2 and Table 3 present the summarized information for FM and NFM methods respectively.

\begin{table*}
\centering
    \caption{A Collection of Published Papers and Codes for FM Methods.}
\resizebox{\textwidth}{!}{%
    \begin{tabular}{|c|c|c|c|} \hline 
         Methods&  Paper URL&  Code URL& Framework\\ \hline 
         ClipSAM \cite{li2024clipsam}&  \href{https://arxiv.org/pdf/2401.12665}{https://arxiv.org/pdf/2401.12665}&  \href{https://github.com/Lszcoding/ClipSAM}{https://github.com/Lszcoding/ClipSAM}& -\\ \hline 
         UCAD \cite{liu2024unsupervised}&  \href{https://arxiv.org/pdf/2401.01010}{https://arxiv.org/pdf/2401.01010}&  \href{https://github.com/shirowalker/UCAD}{https://github.com/shirowalker/UCAD}& PyTorch\\ \hline 
         SAM-LAD \cite{peng2024sam}& \href{https://arxiv.org/pdf/2406.00625}{https://arxiv.org/pdf/2406.00625}& -&-\\ \hline 
         SAA+ \cite{cao2023segment}& \href{https://arxiv.org/pdf/2305.10724}{https://arxiv.org/pdf/2305.10724}&  \href{https://github.com/caoyunkang/Segment-Any-Anomaly}{https://github.com/caoyunkang/Segment-Any-Anomaly}&PyTorch\\  \hline
         STLM \cite{li2024sam}& \href{https://arxiv.org/pdf/2402.19145}{https://arxiv.org/pdf/2402.19145}& \href{https://github.com/Qi5Lei/STLM}{https://github.com/Qi5Lei/STLM}&PyTorch\\\hline
          SPT \cite{yang2024spt}& \href{https://arxiv.org/pdf/2411.17217}{https://arxiv.org/pdf/2411.17217}& \href{https://github.com/THU-MIG/SAM-SPT}{https://github.com/THU-MIG/SAM-SPT}&-\\\hline
         WinCLIP \cite{jeong2023winclip}&\href{https://arxiv.org/pdf/2303.14814v1}{https://arxiv.org/pdf/2303.14814v1}&\href{https://github.com/openvinotoolkit/anomalib}{https://github.com/openvinotoolkit/anomalib}&PyTorch\\ \hline 
        AnoCLIP \cite{deng2023anovl}&\href{https://arxiv.org/pdf/2308.15939v2}{https://arxiv.org/pdf/2308.15939v2}&-&-\\ \hline 
        AnomalyCLIP \cite{zhou2023anomalyclip}&\href{https://arxiv.org/pdf/2310.18961v7}{https://arxiv.org/pdf/2310.18961v7}&\href{https://github.com/zqhang/anomalyclip}{https://github.com/zqhang/anomalyclip}&PyTorch\\ \hline
        AdaCLIP \cite{cao2025adaclip}&\href{https://arxiv.org/pdf/2407.15795v1}{https://arxiv.org/pdf/2407.15795v1}&\href{https://github.com/caoyunkang/adaclip}{https://github.com/caoyunkang/adaclip}&PyTorch\\ \hline 
        VCP-CLIP \cite{qu2024vcp}&\href{https://arxiv.org/pdf/2407.12276v1}{https://arxiv.org/pdf/2407.12276v1}&\href{https://github.com/xiaozhen228/vcp-clip}{https://github.com/xiaozhen228/vcp-clip}&PyTorch\\ \hline 
        SimCLIP \cite{deng2024simclip}&\href{https://openreview.net/pdf?id=kiH6PqRhwE}{https://openreview.net/pdf?id=kiH6PqRhwE}&\href{https://anonymous.4open.science/r/SimCLIP-CAEC}{https://anonymous.4open.science/r/SimCLIP-CAEC}&-\\ \hline 
        CLIP-AD \cite{chen2024clipad}&\href{https://arxiv.org/pdf/2311.00453v2}{https://arxiv.org/pdf/2311.00453v2}&-&-\\ \hline 
        CLIP-FSAC \cite{zuo2024clipfsac}&\href{https://www.ijcai.org/proceedings/2024/0203.pdf}{https://www.ijcai.org/proceedings/2024/0203.pdf}&-&-\\ \hline
        ClipSAM \cite{li2024clipsam}&\href{https://arxiv.org/pdf/2401.12665v2/2024/0203.pdf}{https://arxiv.org/pdf/2401.12665v2/2024/0203.pdf}&\href{https://github.com/lszcoding/clipsam}{https://github.com/lszcoding/clipsam}&-\\\hline
        SOWA \cite{hu2024sowa}&\href{https://arxiv.org/pdf/2407.03634v2}{https://arxiv.org/pdf/2407.03634v2}&\href{https://github.com/huzongxiang/sowa}{https://github.com/huzongxiang/sowa}&PyTorch\\\hline
        SAA+ \cite{cao2023segment}&\href{https://arxiv.org/pdf/2305.10724v1}{https://arxiv.org/pdf/2305.10724v1}&\href{https://github.com/caoyunkang/segment-any-anomaly}{https://github.com/caoyunkang/segment-any-anomaly}&PyTorch\\\hline
        APRIL-GAN \cite{chen2023april}&\href{https://arxiv.org/pdf/2305.17382v3}{https://arxiv.org/pdf/2305.17382v3}&\href{https://github.com/bychelsea/vand-april-gan}{https://github.com/bychelsea/vand-april-gan}&PyTorch\\\hline
        PromptAD \cite{li2024promptad}&\href{https://arxiv.org/pdf/2404.05231v2}{https://arxiv.org/pdf/2404.05231v2}&\href{https://github.com/funz-0/promptad}{https://github.com/funz-0/promptad}&PyTorch\\\hline
        FiLo \cite{gu2024filo}&\href{https://arxiv.org/pdf/2404.13671v2}{https://arxiv.org/pdf/2404.13671v2}&\href{https://github.com/casia-iva-lab/filo}{https://github.com/casia-iva-lab/filo}&PyTorch\\\hline
        Dual-Image Enhanced CLIP \cite{zhang2024dual}&\href{https://arxiv.org/pdf/2405.04782v1}{https://arxiv.org/pdf/2405.04782v1}&-&-\\\hline
        AnomalyGPT \cite{gu2024anomalygpt}&\href{https://arxiv.org/pdf/2308.15366v4}{https://arxiv.org/pdf/2308.15366v4}&\href{https://github.com/casia-iva-lab/anomalygpt}{https://github.com/casia-iva-lab/anomalygpt}&PyTorch\\\hline
        Myriad \cite{li2023myriad}&\href{https://arxiv.org/pdf/2310.19070v2}{https://arxiv.org/pdf/2310.19070v2}&-&-\\\hline
        ALFA \cite{zhu2024alfa}&\href{https://arxiv.org/pdf/2404.09654v2}{https://arxiv.org/pdf/2404.09654v2}&-&-\\ \hline
        GPT-4V-AD \cite{zhang2024gpt}& \href{https://arxiv.org/pdf/2311.02612}{https://arxiv.org/pdf/2311.02612}& \href{https://github.com/zhangzjn/GPT-4V-AD}{https://github.com/zhangzjn/GPT-4V-AD}&PyTorch\\\hline
        Customizable-VLM \cite{xu2024customizing}& \href{https://arxiv.org/pdf/2403.11083}{https://arxiv.org/pdf/2403.11083}& \href{https://github.com/Xiaohao-Xu/Customizable-VLM}{https://github.com/Xiaohao-Xu/Customizable-VLM}&PyTorch\\\hline
        LogiCode \cite{zhang2024logicode}& \href{https://arxiv.org/pdf/2406.04687}{https://arxiv.org/pdf/2406.04687}& -&-\\\hline
        CLIP3D-AD \cite{zuo2024clip3d}& \href{https://arxiv.org/pdf/2406.18941}{https://arxiv.org/pdf/2406.18941}& -&-\\ \hline
        PointAD \cite{zhou2024pointad}& \href{https://arxiv.org/pdf/2410.00320}{https://arxiv.org/pdf/2410.00320}& \href{https://github.com/zqhang/PointAD}{https://github.com/zqhang/PointAD}&PyTorch\\\hline
        M3DM-NR \cite{wang2024m3dm}& \href{https://arxiv.org/pdf/2406.02263}{https://arxiv.org/pdf/2406.02263}& -&-\\\hline
         Echo \cite{chen2025can}&\href{https://arxiv.org/pdf/2501.15795}{https://arxiv.org/pdf/2501.15795}& -&-\\\hline 
         KAnoCLIP \cite{li2025kanoclip}&\href{https://arxiv.org/pdf/2501.03786}{https://arxiv.org/pdf/2501.03786}&-&-\\\hline
    \end{tabular}
}
\end{table*}


\begin{table*}
\centering
    \caption{A Collection of Published Papers and Codes for NFM Methods.}
\resizebox{\textwidth}{!}{%
\begin{tabular}{|c|c|c|c|} \hline 
    Methods&  Paper URL&  Code URL& Framework\\ \hline 
    SOFS \cite{zhang2024sofs}& \href{https://arxiv.org/pdf/2407.21351}{https://arxiv.org/pdf/2407.21351}& \href{https://github.com/zhangzilongc/SOFS}{https://github.com/zhangzilongc/SOFS}&PyTorch\\\hline
     PNI \cite{bae2023pni}& \href{https://arxiv.org/pdf/2211.12634}{https://arxiv.org/pdf/2211.12634}& \href{https://github.com/wogur110/PNI_Anomaly_Detection}{https://github.com/wogur110/PNI\_Anomaly\_Detection}&PyTorch\\\hline
     REB \cite{lyu2024reb}& \href{https://arxiv.org/pdf/2308.12577}{https://arxiv.org/pdf/2308.12577}& \href{https://github.com/ShuaiLYU/REB}{https://github.com/ShuaiLYU/REB}&PyTorch\\\hline
     BGAD \cite{yao2023bgad}& \href{https://arxiv.org/pdf/2207.01463}{https://arxiv.org/pdf/2207.01463}& \href{https://github.com/xcyao00/BGAD}{https://github.com/xcyao00/BGAD}&PyTorch\\\hline
     COAD \cite{qian2024friend}& \href{https://arxiv.org/pdf/2412.06510}{https://arxiv.org/pdf/2412.06510}& -&-\\\hline
     GLASS \cite{chen2025glass}& \href{https://arxiv.org/pdf/2407.09359}{https://arxiv.org/pdf/2407.09359}& \href{https://github.com/cqylunlun/GLASS}{https://github.com/cqylunlun/GLASS}&PyTorch\\\hline
     AdaBLDM \cite{li2024adabldm}& \href{https://arxiv.org/pdf/2402.19330}{https://arxiv.org/pdf/2402.19330}& \href{https://github.com/GrandpaXun242/AdaBLDM.git}{https://github.com/GrandpaXun242/AdaBLDM.git}&PyTorch\\\hline
     RealNet \cite{zhang2024realnet}& \href{https://arxiv.org/pdf/2403.05897}{https://arxiv.org/pdf/2403.05897}& \href{https://github.com/cnulab/RealNet}{https://github.com/cnulab/RealNet}&PyTorch\\\hline
     CAGEN \cite{jiang2024cagen}& \href{https://ieeexplore.ieee.org/document/10447663}{https://ieeexplore.ieee.org/document/10447663}& -&-\\\hline
     AnomalyXFusion \cite{hu2024anomalyxfusion}& \href{https://arxiv.org/pdf/2404.19444}{https://arxiv.org/pdf/2404.19444}& \href{https://github.com/hujiecpp/MVTec-Caption}{https://github.com/hujiecpp/MVTec-Caption}&-\\\hline
     AnomalyDiffusion \cite{hu2024anomalydiffusion}& \href{https://arxiv.org/pdf/2312.05767}{https://arxiv.org/pdf/2312.05767}& \href{https://github.com/sjtuplayer/anomalydiffusion}{https://github.com/sjtuplayer/anomalydiffusion}&PyTorch\\\hline
     DFMGAN \cite{duan2023dfmgan}& \href{https://arxiv.org/pdf/2303.02389}{https://arxiv.org/pdf/2303.02389}& \href{https://github.com/Ldhlwh/DFMGAN}{https://github.com/Ldhlwh/DFMGAN}&PyTorch\\\hline
     DeSTSeg \cite{zhang2023destseg}& \href{https://arxiv.org/pdf/2211.11317}{https://arxiv.org/pdf/2211.11317}& -&-\\\hline
     CutSwap \cite{qin2024cutswap}& \href{https://arxiv.org/pdf/2311.18332}{https://arxiv.org/pdf/2311.18332}& -&-\\\hline
     Split Training \cite{lin2024split}& \href{https://arxiv.org/pdf/2308.15068}{https://arxiv.org/pdf/2308.15068}& -&-\\\hline
     DFD \cite{bai2024dfd}& \href{https://arxiv.org/pdf/2403.04151}{https://arxiv.org/pdf/2403.04151}& \href{https://github.com/yuhbai/DFD}{https://github.com/yuhbai/DFD}&PyTorch\\\hline
     PBAS \cite{chen2024pbas}& \href{https://ieeexplore.ieee.org/stamp/stamp.jsp?tp=\&arnumber=10716437}{https://ieeexplore.ieee.org/stamp/stamp.jsp?tp=\&arnumber=10716437}& \href{https://github.com/cqylunlun/PBAS}{https://github.com/cqylunlun/PBAS}&PyTorch\\\hline
    Shape-Guided \cite{chu2023shape}&  \href{https://openreview.net/pdf?id=IkSGn9fcPz}{https://openreview.net/pdf?id=IkSGn9fcPz}&  \href{https://github.com/jayliu0313/Shape-Guided}{https://github.com/jayliu0313/Shape-Guided}& PyTorch\\ \hline 
    CPMF \cite{cao2024cpmf}&  \href{https://arxiv.org/pdf/2303.13194v1}{https://arxiv.org/pdf/2303.13194v1}&  \href{https://github.com/caoyunkang/CPMF}{https://github.com/caoyunkang/CPMF}& PyTorch\\ \hline 
    Back to the Feature \cite{horwitz2023back}&  \href{https://arxiv.org/pdf/2203.05550}{https://arxiv.org/pdf/2203.05550}&  \href{https://github.com/eliahuhorwitz/3D-ADS}{https://github.com/eliahuhorwitz/3D-ADS}& PyTorch\\ \hline 
    TransFusion \cite{fuvcka2025transfusion}&  \href{https://arxiv.org/pdf/2311.09999v2}{https://arxiv.org/pdf/2311.09999v2}&  \href{https://github.com/maticfuc/eccv_transfusion}{https://github.com/maticfuc/eccv\_transfusion}& PyTorch\\ \hline 
    3DSR \cite{zavrtanik20243dsr}&  \href{https://arxiv.org/pdf/2311.01117v1}{https://arxiv.org/pdf/2311.01117v1}&  \href{https://github.com/vitjanz/3dsr}{https://github.com/vitjanz/3dsr}& PyTorch\\ \hline 
    M3DM \cite{wang2024m3dm}& \href{https://arxiv.org/pdf/2303.00601v2}{https://arxiv.org/pdf/2303.00601v2}& \href{https://github.com/nomewang/m3dm}{https://github.com/nomewang/m3dm}&PyTorch\\ \hline 
    AST \cite{rudolph2023ast}& \href{https://arxiv.org/pdf/2210.07829v2}{https://arxiv.org/pdf/2210.07829v2}& \href{https://github.com/marco-rudolph/ast}{https://github.com/marco-rudolph/ast}&PyTorch\\ \hline 
    R3D-AD \cite{zhou2025r3d}& \href{https://arxiv.org/pdf/2407.10862v1}{https://arxiv.org/pdf/2407.10862v1}&  -&-\\ \hline
    Reg 3D-AD \cite{liu2024real3d}& \href{https://arxiv.org/pdf/2309.13226}{https://arxiv.org/pdf/2309.13226}& \href{https://github.com/M-3LAB/Real3D-AD}{https://github.com/M-3LAB/Real3D-AD}&PyTorch\\\hline
    PointCore \cite{zhao2024pointcore}& \href{https://arxiv.org/pdf/2403.01804v1}{https://arxiv.org/pdf/2403.01804v1}& -&-\\ \hline 
    Uni-3DAD \cite{liu2024uni}& \href{https://arxiv.org/pdf/2408.16201}{https://arxiv.org/pdf/2408.16201}& -&-\\ \hline
    Group3AD \cite{zhu2024group3ad}& \href{https://arxiv.org/pdf/2408.04604}{https://arxiv.org/pdf/2408.04604}& -&-\\ \hline
    PVQAE \cite{cheng2024patch}&\href{https://arxiv.org/pdf/2501.09187}{https://arxiv.org/pdf/2501.09187}&-&-\\\hline 
    CAI \cite{wang2025stones}&\href{https://arxiv.org/pdf/2501.15211}{https://arxiv.org/pdf/2501.15211}&-&-\\\hline 
 \end{tabular}
 }
\end{table*}

\end{appendices}
\end{document}


