
\documentclass{article} % For LaTeX2e
\usepackage{iclr2025_re-align_workshop,times}

% Optional math commands from https://github.com/goodfeli/dlbook_notation.
%%%%% NEW MATH DEFINITIONS %%%%%

\usepackage{amsmath,amsfonts,bm}
\usepackage{derivative}
% Mark sections of captions for referring to divisions of figures
\newcommand{\figleft}{{\em (Left)}}
\newcommand{\figcenter}{{\em (Center)}}
\newcommand{\figright}{{\em (Right)}}
\newcommand{\figtop}{{\em (Top)}}
\newcommand{\figbottom}{{\em (Bottom)}}
\newcommand{\captiona}{{\em (a)}}
\newcommand{\captionb}{{\em (b)}}
\newcommand{\captionc}{{\em (c)}}
\newcommand{\captiond}{{\em (d)}}

% Highlight a newly defined term
\newcommand{\newterm}[1]{{\bf #1}}

% Derivative d 
\newcommand{\deriv}{{\mathrm{d}}}

% Figure reference, lower-case.
\def\figref#1{figure~\ref{#1}}
% Figure reference, capital. For start of sentence
\def\Figref#1{Figure~\ref{#1}}
\def\twofigref#1#2{figures \ref{#1} and \ref{#2}}
\def\quadfigref#1#2#3#4{figures \ref{#1}, \ref{#2}, \ref{#3} and \ref{#4}}
% Section reference, lower-case.
\def\secref#1{section~\ref{#1}}
% Section reference, capital.
\def\Secref#1{Section~\ref{#1}}
% Reference to two sections.
\def\twosecrefs#1#2{sections \ref{#1} and \ref{#2}}
% Reference to three sections.
\def\secrefs#1#2#3{sections \ref{#1}, \ref{#2} and \ref{#3}}
% Reference to an equation, lower-case.
\def\eqref#1{equation~\ref{#1}}
% Reference to an equation, upper case
\def\Eqref#1{Equation~\ref{#1}}
% A raw reference to an equation---avoid using if possible
\def\plaineqref#1{\ref{#1}}
% Reference to a chapter, lower-case.
\def\chapref#1{chapter~\ref{#1}}
% Reference to an equation, upper case.
\def\Chapref#1{Chapter~\ref{#1}}
% Reference to a range of chapters
\def\rangechapref#1#2{chapters\ref{#1}--\ref{#2}}
% Reference to an algorithm, lower-case.
\def\algref#1{algorithm~\ref{#1}}
% Reference to an algorithm, upper case.
\def\Algref#1{Algorithm~\ref{#1}}
\def\twoalgref#1#2{algorithms \ref{#1} and \ref{#2}}
\def\Twoalgref#1#2{Algorithms \ref{#1} and \ref{#2}}
% Reference to a part, lower case
\def\partref#1{part~\ref{#1}}
% Reference to a part, upper case
\def\Partref#1{Part~\ref{#1}}
\def\twopartref#1#2{parts \ref{#1} and \ref{#2}}

\def\ceil#1{\lceil #1 \rceil}
\def\floor#1{\lfloor #1 \rfloor}
\def\1{\bm{1}}
\newcommand{\train}{\mathcal{D}}
\newcommand{\valid}{\mathcal{D_{\mathrm{valid}}}}
\newcommand{\test}{\mathcal{D_{\mathrm{test}}}}

\def\eps{{\epsilon}}


% Random variables
\def\reta{{\textnormal{$\eta$}}}
\def\ra{{\textnormal{a}}}
\def\rb{{\textnormal{b}}}
\def\rc{{\textnormal{c}}}
\def\rd{{\textnormal{d}}}
\def\re{{\textnormal{e}}}
\def\rf{{\textnormal{f}}}
\def\rg{{\textnormal{g}}}
\def\rh{{\textnormal{h}}}
\def\ri{{\textnormal{i}}}
\def\rj{{\textnormal{j}}}
\def\rk{{\textnormal{k}}}
\def\rl{{\textnormal{l}}}
% rm is already a command, just don't name any random variables m
\def\rn{{\textnormal{n}}}
\def\ro{{\textnormal{o}}}
\def\rp{{\textnormal{p}}}
\def\rq{{\textnormal{q}}}
\def\rr{{\textnormal{r}}}
\def\rs{{\textnormal{s}}}
\def\rt{{\textnormal{t}}}
\def\ru{{\textnormal{u}}}
\def\rv{{\textnormal{v}}}
\def\rw{{\textnormal{w}}}
\def\rx{{\textnormal{x}}}
\def\ry{{\textnormal{y}}}
\def\rz{{\textnormal{z}}}

% Random vectors
\def\rvepsilon{{\mathbf{\epsilon}}}
\def\rvphi{{\mathbf{\phi}}}
\def\rvtheta{{\mathbf{\theta}}}
\def\rva{{\mathbf{a}}}
\def\rvb{{\mathbf{b}}}
\def\rvc{{\mathbf{c}}}
\def\rvd{{\mathbf{d}}}
\def\rve{{\mathbf{e}}}
\def\rvf{{\mathbf{f}}}
\def\rvg{{\mathbf{g}}}
\def\rvh{{\mathbf{h}}}
\def\rvu{{\mathbf{i}}}
\def\rvj{{\mathbf{j}}}
\def\rvk{{\mathbf{k}}}
\def\rvl{{\mathbf{l}}}
\def\rvm{{\mathbf{m}}}
\def\rvn{{\mathbf{n}}}
\def\rvo{{\mathbf{o}}}
\def\rvp{{\mathbf{p}}}
\def\rvq{{\mathbf{q}}}
\def\rvr{{\mathbf{r}}}
\def\rvs{{\mathbf{s}}}
\def\rvt{{\mathbf{t}}}
\def\rvu{{\mathbf{u}}}
\def\rvv{{\mathbf{v}}}
\def\rvw{{\mathbf{w}}}
\def\rvx{{\mathbf{x}}}
\def\rvy{{\mathbf{y}}}
\def\rvz{{\mathbf{z}}}

% Elements of random vectors
\def\erva{{\textnormal{a}}}
\def\ervb{{\textnormal{b}}}
\def\ervc{{\textnormal{c}}}
\def\ervd{{\textnormal{d}}}
\def\erve{{\textnormal{e}}}
\def\ervf{{\textnormal{f}}}
\def\ervg{{\textnormal{g}}}
\def\ervh{{\textnormal{h}}}
\def\ervi{{\textnormal{i}}}
\def\ervj{{\textnormal{j}}}
\def\ervk{{\textnormal{k}}}
\def\ervl{{\textnormal{l}}}
\def\ervm{{\textnormal{m}}}
\def\ervn{{\textnormal{n}}}
\def\ervo{{\textnormal{o}}}
\def\ervp{{\textnormal{p}}}
\def\ervq{{\textnormal{q}}}
\def\ervr{{\textnormal{r}}}
\def\ervs{{\textnormal{s}}}
\def\ervt{{\textnormal{t}}}
\def\ervu{{\textnormal{u}}}
\def\ervv{{\textnormal{v}}}
\def\ervw{{\textnormal{w}}}
\def\ervx{{\textnormal{x}}}
\def\ervy{{\textnormal{y}}}
\def\ervz{{\textnormal{z}}}

% Random matrices
\def\rmA{{\mathbf{A}}}
\def\rmB{{\mathbf{B}}}
\def\rmC{{\mathbf{C}}}
\def\rmD{{\mathbf{D}}}
\def\rmE{{\mathbf{E}}}
\def\rmF{{\mathbf{F}}}
\def\rmG{{\mathbf{G}}}
\def\rmH{{\mathbf{H}}}
\def\rmI{{\mathbf{I}}}
\def\rmJ{{\mathbf{J}}}
\def\rmK{{\mathbf{K}}}
\def\rmL{{\mathbf{L}}}
\def\rmM{{\mathbf{M}}}
\def\rmN{{\mathbf{N}}}
\def\rmO{{\mathbf{O}}}
\def\rmP{{\mathbf{P}}}
\def\rmQ{{\mathbf{Q}}}
\def\rmR{{\mathbf{R}}}
\def\rmS{{\mathbf{S}}}
\def\rmT{{\mathbf{T}}}
\def\rmU{{\mathbf{U}}}
\def\rmV{{\mathbf{V}}}
\def\rmW{{\mathbf{W}}}
\def\rmX{{\mathbf{X}}}
\def\rmY{{\mathbf{Y}}}
\def\rmZ{{\mathbf{Z}}}

% Elements of random matrices
\def\ermA{{\textnormal{A}}}
\def\ermB{{\textnormal{B}}}
\def\ermC{{\textnormal{C}}}
\def\ermD{{\textnormal{D}}}
\def\ermE{{\textnormal{E}}}
\def\ermF{{\textnormal{F}}}
\def\ermG{{\textnormal{G}}}
\def\ermH{{\textnormal{H}}}
\def\ermI{{\textnormal{I}}}
\def\ermJ{{\textnormal{J}}}
\def\ermK{{\textnormal{K}}}
\def\ermL{{\textnormal{L}}}
\def\ermM{{\textnormal{M}}}
\def\ermN{{\textnormal{N}}}
\def\ermO{{\textnormal{O}}}
\def\ermP{{\textnormal{P}}}
\def\ermQ{{\textnormal{Q}}}
\def\ermR{{\textnormal{R}}}
\def\ermS{{\textnormal{S}}}
\def\ermT{{\textnormal{T}}}
\def\ermU{{\textnormal{U}}}
\def\ermV{{\textnormal{V}}}
\def\ermW{{\textnormal{W}}}
\def\ermX{{\textnormal{X}}}
\def\ermY{{\textnormal{Y}}}
\def\ermZ{{\textnormal{Z}}}

% Vectors
\def\vzero{{\bm{0}}}
\def\vone{{\bm{1}}}
\def\vmu{{\bm{\mu}}}
\def\vtheta{{\bm{\theta}}}
\def\vphi{{\bm{\phi}}}
\def\va{{\bm{a}}}
\def\vb{{\bm{b}}}
\def\vc{{\bm{c}}}
\def\vd{{\bm{d}}}
\def\ve{{\bm{e}}}
\def\vf{{\bm{f}}}
\def\vg{{\bm{g}}}
\def\vh{{\bm{h}}}
\def\vi{{\bm{i}}}
\def\vj{{\bm{j}}}
\def\vk{{\bm{k}}}
\def\vl{{\bm{l}}}
\def\vm{{\bm{m}}}
\def\vn{{\bm{n}}}
\def\vo{{\bm{o}}}
\def\vp{{\bm{p}}}
\def\vq{{\bm{q}}}
\def\vr{{\bm{r}}}
\def\vs{{\bm{s}}}
\def\vt{{\bm{t}}}
\def\vu{{\bm{u}}}
\def\vv{{\bm{v}}}
\def\vw{{\bm{w}}}
\def\vx{{\bm{x}}}
\def\vy{{\bm{y}}}
\def\vz{{\bm{z}}}

% Elements of vectors
\def\evalpha{{\alpha}}
\def\evbeta{{\beta}}
\def\evepsilon{{\epsilon}}
\def\evlambda{{\lambda}}
\def\evomega{{\omega}}
\def\evmu{{\mu}}
\def\evpsi{{\psi}}
\def\evsigma{{\sigma}}
\def\evtheta{{\theta}}
\def\eva{{a}}
\def\evb{{b}}
\def\evc{{c}}
\def\evd{{d}}
\def\eve{{e}}
\def\evf{{f}}
\def\evg{{g}}
\def\evh{{h}}
\def\evi{{i}}
\def\evj{{j}}
\def\evk{{k}}
\def\evl{{l}}
\def\evm{{m}}
\def\evn{{n}}
\def\evo{{o}}
\def\evp{{p}}
\def\evq{{q}}
\def\evr{{r}}
\def\evs{{s}}
\def\evt{{t}}
\def\evu{{u}}
\def\evv{{v}}
\def\evw{{w}}
\def\evx{{x}}
\def\evy{{y}}
\def\evz{{z}}

% Matrix
\def\mA{{\bm{A}}}
\def\mB{{\bm{B}}}
\def\mC{{\bm{C}}}
\def\mD{{\bm{D}}}
\def\mE{{\bm{E}}}
\def\mF{{\bm{F}}}
\def\mG{{\bm{G}}}
\def\mH{{\bm{H}}}
\def\mI{{\bm{I}}}
\def\mJ{{\bm{J}}}
\def\mK{{\bm{K}}}
\def\mL{{\bm{L}}}
\def\mM{{\bm{M}}}
\def\mN{{\bm{N}}}
\def\mO{{\bm{O}}}
\def\mP{{\bm{P}}}
\def\mQ{{\bm{Q}}}
\def\mR{{\bm{R}}}
\def\mS{{\bm{S}}}
\def\mT{{\bm{T}}}
\def\mU{{\bm{U}}}
\def\mV{{\bm{V}}}
\def\mW{{\bm{W}}}
\def\mX{{\bm{X}}}
\def\mY{{\bm{Y}}}
\def\mZ{{\bm{Z}}}
\def\mBeta{{\bm{\beta}}}
\def\mPhi{{\bm{\Phi}}}
\def\mLambda{{\bm{\Lambda}}}
\def\mSigma{{\bm{\Sigma}}}

% Tensor
\DeclareMathAlphabet{\mathsfit}{\encodingdefault}{\sfdefault}{m}{sl}
\SetMathAlphabet{\mathsfit}{bold}{\encodingdefault}{\sfdefault}{bx}{n}
\newcommand{\tens}[1]{\bm{\mathsfit{#1}}}
\def\tA{{\tens{A}}}
\def\tB{{\tens{B}}}
\def\tC{{\tens{C}}}
\def\tD{{\tens{D}}}
\def\tE{{\tens{E}}}
\def\tF{{\tens{F}}}
\def\tG{{\tens{G}}}
\def\tH{{\tens{H}}}
\def\tI{{\tens{I}}}
\def\tJ{{\tens{J}}}
\def\tK{{\tens{K}}}
\def\tL{{\tens{L}}}
\def\tM{{\tens{M}}}
\def\tN{{\tens{N}}}
\def\tO{{\tens{O}}}
\def\tP{{\tens{P}}}
\def\tQ{{\tens{Q}}}
\def\tR{{\tens{R}}}
\def\tS{{\tens{S}}}
\def\tT{{\tens{T}}}
\def\tU{{\tens{U}}}
\def\tV{{\tens{V}}}
\def\tW{{\tens{W}}}
\def\tX{{\tens{X}}}
\def\tY{{\tens{Y}}}
\def\tZ{{\tens{Z}}}


% Graph
\def\gA{{\mathcal{A}}}
\def\gB{{\mathcal{B}}}
\def\gC{{\mathcal{C}}}
\def\gD{{\mathcal{D}}}
\def\gE{{\mathcal{E}}}
\def\gF{{\mathcal{F}}}
\def\gG{{\mathcal{G}}}
\def\gH{{\mathcal{H}}}
\def\gI{{\mathcal{I}}}
\def\gJ{{\mathcal{J}}}
\def\gK{{\mathcal{K}}}
\def\gL{{\mathcal{L}}}
\def\gM{{\mathcal{M}}}
\def\gN{{\mathcal{N}}}
\def\gO{{\mathcal{O}}}
\def\gP{{\mathcal{P}}}
\def\gQ{{\mathcal{Q}}}
\def\gR{{\mathcal{R}}}
\def\gS{{\mathcal{S}}}
\def\gT{{\mathcal{T}}}
\def\gU{{\mathcal{U}}}
\def\gV{{\mathcal{V}}}
\def\gW{{\mathcal{W}}}
\def\gX{{\mathcal{X}}}
\def\gY{{\mathcal{Y}}}
\def\gZ{{\mathcal{Z}}}

% Sets
\def\sA{{\mathbb{A}}}
\def\sB{{\mathbb{B}}}
\def\sC{{\mathbb{C}}}
\def\sD{{\mathbb{D}}}
% Don't use a set called E, because this would be the same as our symbol
% for expectation.
\def\sF{{\mathbb{F}}}
\def\sG{{\mathbb{G}}}
\def\sH{{\mathbb{H}}}
\def\sI{{\mathbb{I}}}
\def\sJ{{\mathbb{J}}}
\def\sK{{\mathbb{K}}}
\def\sL{{\mathbb{L}}}
\def\sM{{\mathbb{M}}}
\def\sN{{\mathbb{N}}}
\def\sO{{\mathbb{O}}}
\def\sP{{\mathbb{P}}}
\def\sQ{{\mathbb{Q}}}
\def\sR{{\mathbb{R}}}
\def\sS{{\mathbb{S}}}
\def\sT{{\mathbb{T}}}
\def\sU{{\mathbb{U}}}
\def\sV{{\mathbb{V}}}
\def\sW{{\mathbb{W}}}
\def\sX{{\mathbb{X}}}
\def\sY{{\mathbb{Y}}}
\def\sZ{{\mathbb{Z}}}

% Entries of a matrix
\def\emLambda{{\Lambda}}
\def\emA{{A}}
\def\emB{{B}}
\def\emC{{C}}
\def\emD{{D}}
\def\emE{{E}}
\def\emF{{F}}
\def\emG{{G}}
\def\emH{{H}}
\def\emI{{I}}
\def\emJ{{J}}
\def\emK{{K}}
\def\emL{{L}}
\def\emM{{M}}
\def\emN{{N}}
\def\emO{{O}}
\def\emP{{P}}
\def\emQ{{Q}}
\def\emR{{R}}
\def\emS{{S}}
\def\emT{{T}}
\def\emU{{U}}
\def\emV{{V}}
\def\emW{{W}}
\def\emX{{X}}
\def\emY{{Y}}
\def\emZ{{Z}}
\def\emSigma{{\Sigma}}

% entries of a tensor
% Same font as tensor, without \bm wrapper
\newcommand{\etens}[1]{\mathsfit{#1}}
\def\etLambda{{\etens{\Lambda}}}
\def\etA{{\etens{A}}}
\def\etB{{\etens{B}}}
\def\etC{{\etens{C}}}
\def\etD{{\etens{D}}}
\def\etE{{\etens{E}}}
\def\etF{{\etens{F}}}
\def\etG{{\etens{G}}}
\def\etH{{\etens{H}}}
\def\etI{{\etens{I}}}
\def\etJ{{\etens{J}}}
\def\etK{{\etens{K}}}
\def\etL{{\etens{L}}}
\def\etM{{\etens{M}}}
\def\etN{{\etens{N}}}
\def\etO{{\etens{O}}}
\def\etP{{\etens{P}}}
\def\etQ{{\etens{Q}}}
\def\etR{{\etens{R}}}
\def\etS{{\etens{S}}}
\def\etT{{\etens{T}}}
\def\etU{{\etens{U}}}
\def\etV{{\etens{V}}}
\def\etW{{\etens{W}}}
\def\etX{{\etens{X}}}
\def\etY{{\etens{Y}}}
\def\etZ{{\etens{Z}}}

% The true underlying data generating distribution
\newcommand{\pdata}{p_{\rm{data}}}
\newcommand{\ptarget}{p_{\rm{target}}}
\newcommand{\pprior}{p_{\rm{prior}}}
\newcommand{\pbase}{p_{\rm{base}}}
\newcommand{\pref}{p_{\rm{ref}}}

% The empirical distribution defined by the training set
\newcommand{\ptrain}{\hat{p}_{\rm{data}}}
\newcommand{\Ptrain}{\hat{P}_{\rm{data}}}
% The model distribution
\newcommand{\pmodel}{p_{\rm{model}}}
\newcommand{\Pmodel}{P_{\rm{model}}}
\newcommand{\ptildemodel}{\tilde{p}_{\rm{model}}}
% Stochastic autoencoder distributions
\newcommand{\pencode}{p_{\rm{encoder}}}
\newcommand{\pdecode}{p_{\rm{decoder}}}
\newcommand{\precons}{p_{\rm{reconstruct}}}

\newcommand{\laplace}{\mathrm{Laplace}} % Laplace distribution

\newcommand{\E}{\mathbb{E}}
\newcommand{\Ls}{\mathcal{L}}
\newcommand{\R}{\mathbb{R}}
\newcommand{\emp}{\tilde{p}}
\newcommand{\lr}{\alpha}
\newcommand{\reg}{\lambda}
\newcommand{\rect}{\mathrm{rectifier}}
\newcommand{\softmax}{\mathrm{softmax}}
\newcommand{\sigmoid}{\sigma}
\newcommand{\softplus}{\zeta}
\newcommand{\KL}{D_{\mathrm{KL}}}
\newcommand{\Var}{\mathrm{Var}}
\newcommand{\standarderror}{\mathrm{SE}}
\newcommand{\Cov}{\mathrm{Cov}}
% Wolfram Mathworld says $L^2$ is for function spaces and $\ell^2$ is for vectors
% But then they seem to use $L^2$ for vectors throughout the site, and so does
% wikipedia.
\newcommand{\normlzero}{L^0}
\newcommand{\normlone}{L^1}
\newcommand{\normltwo}{L^2}
\newcommand{\normlp}{L^p}
\newcommand{\normmax}{L^\infty}

\newcommand{\parents}{Pa} % See usage in notation.tex. Chosen to match Daphne's book.

\DeclareMathOperator*{\argmax}{arg\,max}
\DeclareMathOperator*{\argmin}{arg\,min}

\DeclareMathOperator{\sign}{sign}
\DeclareMathOperator{\Tr}{Tr}
\let\ab\allowbreak


\usepackage{hyperref}
\usepackage{url}
\usepackage{xcolor}
\usepackage{graphicx}
\usepackage{subfig}
\usepackage{booktabs}
\newcommand\todo[1]{\textcolor{red}{#1}}
\newcommand{\nummodels}{118}
\renewcommand{\sectionautorefname}{Section}
\renewcommand{\subsectionautorefname}{Section}
\renewcommand{\subsubsectionautorefname}{Section}
\renewcommand{\appendixautorefname}{Appendix}
% \usepackage[subtle]{savetrees}


\title{Alignment and Adversarial Robustness: Are More Human-Like Models More Secure?}

% Authors must not appear in the submitted version. They should be hidden
% as long as the \iclrfinalcopy macro remains commented out below.
% Non-anonymous submissions will be rejected without review.

\author{Blaine Hoak\thanks{Equal contribution.}, ~ Kunyang Li\footnotemark[1] ~ \& Patrick McDaniel \\
Department of Computer Science\\
University of Wisconsin-Madison\\
\texttt{\{bhoak, kli253, mcdaniel\}@cs.wisc.edu}
}

% \author{Blaine Hoak\thanks{Equal contribution.}\\ 
% Department of Computer Science\\
% University of Wisconsin-Madison\\
% \texttt{bhoak@cs.wisc.edu} \\
% \And
% Kunyang Li\footnotemark[1] \\
% Department of Computer Science\\
% University of Wisconsin-Madison\\
% \texttt{kli253@cs.wisc.edu} \\
% \And
% Patrick McDaniel \\
% Department of Computer Science\\
% University of Wisconsin-Madison\\
% \texttt{mcdaniel@cs.wisc.edu}
% }

% The \author macro works with any number of authors. There are two commands
% used to separate the names and addresses of multiple authors: \And and \AND.
%
% Using \And between authors leaves it to \LaTeX{} to determine where to break
% the lines. Using \AND forces a linebreak at that point. So, if \LaTeX{}
% puts 3 of 4 authors names on the first line, and the last on the second
% line, try using \AND instead of \And before the third author name.

\newcommand{\fix}{\marginpar{FIX}}
\newcommand{\new}{\marginpar{NEW}}
\newcommand{\shortsection}{\noindent\textbf}

\iclrfinalcopy % Uncomment for camera-ready version, but NOT for submission.
\begin{document}


\maketitle

\begin{abstract}
Representational alignment refers to the extent to which a model’s internal representations mirror biological vision, offering insights into both neural similarity and functional correspondence. Recently, some more aligned models have demonstrated higher resiliency to adversarial examples, raising the question of whether more human-aligned models are inherently more secure. 
In this work, we conduct a large-scale empirical analysis to systematically investigate the relationship between representational alignment and adversarial robustness. We evaluate \nummodels{} models spanning diverse architectures and training paradigms, measuring their neural and behavioral alignment and engineering task performance across 106 benchmarks as well as their adversarial robustness via AutoAttack. Our findings reveal that while average alignment and robustness exhibit a weak overall correlation, \textit{specific} alignment benchmarks serve as strong predictors of adversarial robustness, particularly those that measure selectivity towards texture or shape. These results suggest that different forms of alignment play distinct roles in model robustness, motivating further investigation into how alignment-driven approaches can be leveraged to build more secure and perceptually-grounded vision models.


\end{abstract}

\begin{figure}
    \centering
    \begin{tikzpicture}[font=\footnotesize]
        \node (img) {\includegraphics[width=0.7\columnwidth]{figpaper/nfe_vs_fvd_vs_ep_ffs_teaser.pdf}};
            \node[anchor=north west, xshift=25pt, yshift=-5pt] at (img.north west) {
                \begin{tabular}{ll}
                \scriptsize
                    \textcolor[HTML]{A0A0A0}{\rule{6pt}{6pt}} &Rolling Diffusion \cite{ruhe2024rollingdiffusionmodels} \\
                    \textcolor[HTML]{e9cbc4}{\rule{6pt}{6pt}} &Diffusion Forcing \cite{chen2024diffusionforcing} \\
                    \textcolor[HTML]{F4A700}{\rule{6pt}{6pt}} &MaskFlow (\textit{Ours})
                \end{tabular}
            };
    \end{tikzpicture}
    \vspace{-7pt}
    \caption{\textbf{Our method (MaskFlow) improves video quality compared to baselines while simultaneously requiring fewer function evaluations (NFE)} when generating videos $2\times$, $5\times$, and $10\times$ longer than the training window.
}
    \label{fig:teaser}
    \vspace{-10pt}
\end{figure}

\section{Introduction}

Due to the high computational demands of both training and sampling processes, long video generation remains a challenging task in computer vision. Many recent state-of-the-art video generation approaches train on fixed sequence lengths \cite{blattmann2023stable,blattmann2023align_videoldm,ho2022video} and thus struggle to scale to longer sampling horizons. Many use cases not only require long video generation, but also require the ability to generate videos with varying length. A common way to address this is by adopting an autoregressive diffusion approach similar to LLMs \cite{gao2024vid}, where videos are generated frame by frame. This has other downsides, since it requires traversing the entire denoising chain for every frame individually, which is computationally expensive. Since autoregressive models condition the generative process recursively on previously generated frames, error accumulation, specifically when rolling out to videos longer than the training videos, is another challenge.
\par
Several recent works \cite{ruhe2024rollingdiffusionmodels, chen2024diffusionforcing} have attempted to unify the flexibility of autoregressive generation approaches with the advantages of full sequence generation. These approaches are built on the intuition that the data corruption process in diffusion models can serve as an intermediary for injecting temporal inductive bias. Progressively increasing noise schedules \cite{xie2024progressive,ruhe2024rollingdiffusionmodels} are an example of a sampling schedule enabled by this paradigm. These works impose monotonically increasing noise schedules w.r.t. frame position in the window during training, limiting their flexibility in interpolating between fully autoregressive, frame-by-frame generation and full-sequence generation. This is alleviated in \cite{chen2024diffusionforcing}, where independent, uniformly sampled noise levels are applied to frames during training, and the diffusion model is trained to denoise arbitrary sequences of noisy frames. All of these works use continuous representations.
\par
We transfer this idea to a discrete token space for two main reasons: First, it allows us to use a masking-based data corruption process, which enables confidence-based heuristic sampling that drastically speeds up the generative process. This becomes especially relevant when considering frame-by-frame autoregressive generation. Second, it allows us to use discrete flow matching dynamics, which provide a more flexible design space and the ability to further increase our sampling speed. Specifically, we adopt a \emph{frame-level masking} scheme in training (versus a \emph{constant-level masking} baseline, see Figure~\ref{fig:training}), which allows us to condition on an arbitrary number of previously generated frames while still being consistent with the training task. This makes our method inherently versatile, allowing us to generate videos using both full-sequence and autoregressive frame-by-frame generation, and use different sampling modes. We show that confidence-based masked generative model (MGM) style sampling is uniquely suited to this setting, generating high-quality results with a low number of function evaluations (NFE), and does not degrade quality compared to diffusion-like flow matching (FM)-style sampling that uses larger NFE. 
Combining frame-level masking during training with MGM-style sampling enables highly efficient long-horizon rollouts of our video generation models beyond $10 \times$ training frame lengths without degradation. We also demonstrate that this sampling method can be applied in a timestep-\emph{independent} setting that omits explicit timestep conditioning, even when models were trained in a timestep-dependent manner, which further underlines the flexibility of our approach. In summary, our contributions are the following:

\begin{itemize}
    \item To the best of our knowledge, we are the first to unify the paradigms of discrete representations in video flow matching with rolling out generative models to generate arbitrary-length videos. 
    \item We introduce MaskFlow, a frame-level masking approach that supports highly flexible sampling methods in a single unified model architecture.
    \item We demonstrate that MaskFlow with MGM-style sampling generates long videos faster while simultaneously preserving high visual quality (as shown in Figure~\ref{fig:teaser}).
    \item Additionally, we demonstrate an additional increase in quality when using full autoregressive generation or partial context guidance combined with MaskFlow for very long sampling horizons.
    \item We show that we can apply MaskFlow to both timestep-dependent and timestep-independent model backbones without re-training.
\end{itemize}

\begin{figure}
    \centering
    \includegraphics[width=0.75\linewidth]{figpaper/training.pdf}
    \caption{\textbf{MaskFlow Training:} For each video, Baseline training applies a single masking ratios to all frames, whereas our method samples masking ratios independently for each frame.}
    \vspace{-10pt}
    \label{fig:training}
\end{figure}

















\section{Background and Related Work\label{sec:background}}

\begin{figure*}[t]
\centering
\subfloat[Crypto]{\label{fig:Crypto-Graph}\includegraphics[width=0.33\textwidth]{fig/kw_crypto_rank-crop.pdf}}\hfill
\subfloat[Interface-GPIO]{\label{fig:GPIO-Graph}\includegraphics[width=0.33\textwidth]{fig/kw_gpio-crop.pdf}}\hfill
\subfloat[Interface-Peripheral]{\label{fig:Peripheral-Graph}\includegraphics[width=0.33\textwidth]{fig/kw_periph-crop.pdf}}
\caption{The number of occurrences for our identified partial keywords for three IP families.\label{fig:explanations_of_general_image}
}
\end{figure*}


An asset is any physical or logical component of \textit{value} and is essential for the proper functioning or security of the system\cite{holdings2009arm}.
The elements in an \ac{IP} that process, control, and store important values and interact with other \acp{IP} in an \ac{SoC} and communicate with the external peripherals are considered as \textit{primary} assets. 
Secondary assets are mostly internal design components of an \ac{IP} that help to propagate and handle the primary assets throughout the entire \ac{IP}.
It is helpful to know the assets in the system for several purposes, such as formulating security properties or identifying potential attack points (e.g., as in Accellera's \ac{SA-EDI} standard~\cite{accellera}). 
Recently, an IEEE P3164 white paper proposed a Conceptual and Structural Asset (CSA) methodology to help manually identify primary assets~\cite{ieee_p3164_working_group_asset_2024}, especially considering security objectives of \textbf{confidentiality}, \textbf{integrity}, and \textbf{availability} (the ``CIA triad'')~\cite{noauthor_what_nodate}, as well as the risk for ``undermined expected behavior'' in normal operation. 
Additionally, initiatives like MITRE's \acp{CWE}~\cite{mitreCommonWeakness} offer various examples of hardware weakness for identifying and mitigating vulnerabilities in hardware systems. 
This standard and methodologies, along with \ac{CWE} examples, give us valuable insights into current security challenges and how to potentially avoid them. 

Even so, as observed by the authors in~\cite{10140100}, engineers face challenges in identifying assets in the initial stages of hardware verification. 
Engineers must assess asset weaknesses, including proper initialization, information flow, and access controls. This process demands deep knowledge of security assets~\cite{ray_system--chip_2018}. However, there is a lack of tools to aid and automate this task, making it even more challenging for engineers.
Prior work attempted to perform generalized security analysis of \ac{HDL} code~\cite{Ahmad_2022} using syntactical patterns but found that more information from a designer (such as assets) was needed to address false positives. 
The authors of~\cite{polian_introduction_2017} described two situations where an element or the \ac{IP} itself as a whole can be an asset.
The authors of~\cite{meza_security_2023} demonstrated the security properties verification method with the help of security assets, where identifying hardware assets is the most crucial (manual) part. 
The authors of~\cite{farzana_saif_2021, Ayalasomayajula_Automatic_2024} proposed an automatic \textit{secondary} asset detection algorithm, that assumes that \textit{manual} \textbf{primary} security asset identification was performed. 
To the best of our knowledge, no prior work currently exists for \textbf{automated \underline{primary} asset identification} in Verilog source code, so our work complements prior works by offering an automated approach to identify potential primary assets. 

\section{Methods}\label{sec:methods}

\shortsection{Alignment.} To measure alignment and download candidate models, we leverage the BrainScore~\cite{schrimpf_brain-score_2018} library. BrainScore provides a standardized framework for evaluating model similarity to biological vision through a set of neural, behavioral, and engineering benchmarks, supplying 106 benchmarks in total. These benchmarks quantify how closely a model’s internal representations and outputs correspond to neurophysiological recordings, human psychophysical behavior, and performance on engineered vision tasks. Neural alignment is measured by comparing activations from DNNs to neural recordings from primate visual cortex regions (e.g., V1, V2, V4, and IT), using similarity metrics like Representational Similarity Analysis (RSA) \cite{kriegeskorte_representational_2008}. Behavioral alignment assesses whether models replicate human psychophysical responses in object recognition and perturbation tests, while engineering alignment evaluates model robustness to controlled distortions, such as contrast reductions, or performance on out of distribution data. 


In total, the BrainScore library has documented benchmark scores for 434 models. Out of those, there are 197 models available in their registry (the remaining 237 models were either submitted privately or have been deprecated). From the 197 models in the registry, we removed an additional 72 models because either loading the model produced a ClientError due to a moved or removed model hosting location or the model was incompatible with ImageNet (either does not output 1000 classes or expects video streams). After this, we had to discard an additional 7 models, which represented all the VOne class models~\cite{dapello_simulating_2020} because they were not able to run on AutoAttack due to gradient alteration or masking, suggesting that previous results finding that VOne models are more robust to adversarial examples could have been due to overestimated robustness and highlighting the importance of evaluating robustness under comprehensive attack strategies. After this filtering process, we were left with \nummodels{} models (see \autoref{sec:appendix}) for our evaluation.

\shortsection{Robustness.} To evaluate the robustness of our models, we use AutoAttack~\cite{croce_reliable_2020, croce_robustbench_2021}, which serves as the standard for evaluating the robustness of neural networks due to its strong attack performance and fully automated parameter-free design. AutoAttack contains 4 attacks: APGD-CE, APGD-DLR, FAB, and Square Attack. By evaluating on AutoAttack, we are not only evaluating on the most performant attacks, but also integrating in both white-box attacks and black-box attacks which has been recommended in previous works to combat reporting overestimated robustness due to gradient masking or obfuscation~\cite{carlini_towards_2017}.

To better understand how the relationship between adversarial robustness and alignment changes as attacks change, we evaluate the $\ell_\infty$ robustness of our models at three different epsilon levels: $\epsilon = \{\frac{0.25}{255}, \frac{0.5}{255}, \frac{1}{255}\}$ to represent adversaries at different capability levels and small, medium, and large image distortion levels. While these values are typically lower than what would be benchmarked on platforms such as RobustBench~\cite{croce_robustbench_2021}, we choose these values with the goal of having a wide distribution of robust accuracies to identify separability between models, rather than the goal of bringing the model down to 0\% accuracy as what is typically done. 
\definecolor{ReliableGreen}{RGB}{131,251,133}
\definecolor{ReliableYellow}{RGB}{255,251,128}
\definecolor{ReliableRed}{RGB}{255,184,184}
\definecolor{ReliableOrange}{RGB}{255,208,143}
\definecolor{ReliableGray}{RGB}{230,230,230}

\begin{table*}[t]
    \vspace{-4pt}
    
    \centering
    \small


    \setlength{\tabcolsep}{4pt}
    \begin{tabular}{l|p{0.45cm}p{0.45cm}p{0.45cm}p{0.45cm}p{0.45cm}p{0.45cm}|p{0.45cm}p{0.45cm}p{0.45cm}|l|p{0.45cm}p{0.45cm}p{0.45cm}|l|p{0.45cm}|c}
    \multicolumn{1}{l}{} & \rot{SingleOp} & \rot{SingleEq} & \rot{MultiArith} & \rot{SVAMP} & \rot{GSM8K} & \rot{MMLU HS Math} & \rot{Logic Ded. 3-Obj} & \rot{Object Counting} & \rot{Navigate} & \rot{TabFact} & \rot{HotpotQA} & \rot{SQuAD2.0} & \rot{DROP} & \rot{Winograd WSC} & \rot{VQA v2.0} &  \\ \toprule
    
        Type                    & \multicolumn{6}{c|}{Math} & \multicolumn{3}{c|}{Logic}        & Tab & \multicolumn{3}{c|}{RC} &   CR & Vis & \multirow{2}{*}{\textbf{Score}}     \\
    

    \# Platinum Questions & 150 & 100 & 171 & 273 & 274 & 268 & 200 & 191 & 200 & 173 & 184 & 164 & 209 & 195 & 248  \\ \midrule


\ \ o1-2024-12-17 (high) & \cellcolor{ReliableGreen}0 & \cellcolor{ReliableGreen}0 & \cellcolor{ReliableGreen}0 & \cellcolor{ReliableGreen}0 & \cellcolor{ReliableYellow}2 & \cellcolor{ReliableYellow}1 & \cellcolor{ReliableGreen}0 & \cellcolor{ReliableGreen}0 & \cellcolor{ReliableGreen}0 & \cellcolor{ReliableGreen}0 & \cellcolor{ReliableGreen}0 & \cellcolor{ReliableOrange}5 & \cellcolor{ReliableGreen}0 & \cellcolor{ReliableOrange}5 & \cellcolor{ReliableOrange}10 & \cellcolor{ReliableGray}0.75\% \\
\ \ Claude 3.5 Sonnet (Oct) & \cellcolor{ReliableGreen}0 & \cellcolor{ReliableGreen}0 & \cellcolor{ReliableGreen}0 & \cellcolor{ReliableYellow}1 & \cellcolor{ReliableYellow}3 & 21 & \cellcolor{ReliableGreen}0 & \cellcolor{ReliableGreen}0 & \cellcolor{ReliableYellow}1 & \cellcolor{ReliableYellow}1 & \cellcolor{ReliableGreen}0 & \cellcolor{ReliableYellow}3 & \cellcolor{ReliableOrange}6 & \cellcolor{ReliableYellow}3 & 17 & \cellcolor{ReliableGray}1.08\% \\
\ \ o1-2024-12-17 (med) & \cellcolor{ReliableGreen}0 & \cellcolor{ReliableGreen}0 & \cellcolor{ReliableGreen}0 & \cellcolor{ReliableYellow}1 & \cellcolor{ReliableYellow}2 & \cellcolor{ReliableYellow}2 & \cellcolor{ReliableGreen}0 & \cellcolor{ReliableGreen}0 & \cellcolor{ReliableGreen}0 & \cellcolor{ReliableGreen}0 & \cellcolor{ReliableGreen}0 & \cellcolor{ReliableOrange}8 & \cellcolor{ReliableYellow}2 & \cellcolor{ReliableOrange}8 & \cellcolor{ReliableOrange}6 & \cellcolor{ReliableGray}1.27\% \\
\ \ DeepSeek-R1 & \cellcolor{ReliableGreen}0 & \cellcolor{ReliableGreen}0 & \cellcolor{ReliableYellow}1 & \cellcolor{ReliableYellow}1 & \cellcolor{ReliableYellow}1 & \cellcolor{ReliableYellow}2 & \cellcolor{ReliableGreen}0 & \cellcolor{ReliableGreen}0 & \cellcolor{ReliableYellow}1 & \cellcolor{ReliableYellow}3 & \cellcolor{ReliableYellow}1 & \cellcolor{ReliableOrange}6 & \cellcolor{ReliableOrange}6 & \cellcolor{ReliableOrange}7 &  & \cellcolor{ReliableGray}1.64\% \\
\ \ Claude 3.5 Sonnet (June) & \cellcolor{ReliableGreen}0 & \cellcolor{ReliableGreen}0 & \cellcolor{ReliableGreen}0 & \cellcolor{ReliableYellow}2 & \cellcolor{ReliableYellow}5 & 29 & \cellcolor{ReliableGreen}0 & \cellcolor{ReliableGreen}0 & \cellcolor{ReliableGreen}0 & \cellcolor{ReliableYellow}3 & \cellcolor{ReliableGreen}0 & \cellcolor{ReliableYellow}3 & \cellcolor{ReliableYellow}3 & \cellcolor{ReliableOrange}8 & 21 & \cellcolor{ReliableGray}1.83\% \\
\ \ Llama 3.1 405B Inst & \cellcolor{ReliableGreen}0 & \cellcolor{ReliableGreen}0 & \cellcolor{ReliableGreen}0 & \cellcolor{ReliableYellow}3 & \cellcolor{ReliableYellow}2 & 28 & \cellcolor{ReliableGreen}0 & \cellcolor{ReliableYellow}1 & \cellcolor{ReliableOrange}5 & \cellcolor{ReliableYellow}1 & \cellcolor{ReliableYellow}2 & \cellcolor{ReliableOrange}4 & \cellcolor{ReliableYellow}4 & \cellcolor{ReliableOrange}8 &  & \cellcolor{ReliableGray}1.91\% \\
\ \ GPT-4o (Aug) & \cellcolor{ReliableGreen}0 & \cellcolor{ReliableGreen}0 & \cellcolor{ReliableGreen}0 & \cellcolor{ReliableOrange}7 & \cellcolor{ReliableYellow}4 & 22 & \cellcolor{ReliableGreen}0 & \cellcolor{ReliableGreen}0 & \cellcolor{ReliableYellow}4 & \cellcolor{ReliableYellow}2 & \cellcolor{ReliableYellow}3 & 9 & \cellcolor{ReliableYellow}4 & 10 & \cellcolor{ReliableOrange}8 & \cellcolor{ReliableGray}2.40\% \\
\ \ GPT-4o (Nov) & \cellcolor{ReliableGreen}0 & \cellcolor{ReliableGreen}0 & \cellcolor{ReliableGreen}0 & \cellcolor{ReliableOrange}6 & \cellcolor{ReliableOrange}7 & 20 & \cellcolor{ReliableGreen}0 & \cellcolor{ReliableOrange}4 & \cellcolor{ReliableYellow}4 & \cellcolor{ReliableYellow}1 & \cellcolor{ReliableYellow}2 & \cellcolor{ReliableOrange}7 & \cellcolor{ReliableYellow}3 & 12 & \cellcolor{ReliableOrange}6 & \cellcolor{ReliableGray}2.48\% \\
\ \ o1-preview & \cellcolor{ReliableGreen}0 & \cellcolor{ReliableGreen}0 & \cellcolor{ReliableGreen}0 & \cellcolor{ReliableYellow}1 & \cellcolor{ReliableYellow}2 & \cellcolor{ReliableYellow}3 & \cellcolor{ReliableGreen}0 & 15 & \cellcolor{ReliableYellow}3 & \cellcolor{ReliableOrange}4 & \cellcolor{ReliableYellow}3 & 11 & \cellcolor{ReliableOrange}5 & \cellcolor{ReliableOrange}6 &  & \cellcolor{ReliableGray}2.49\% \\
\ \ DeepSeek-V3 & \cellcolor{ReliableYellow}1 & \cellcolor{ReliableGreen}0 & \cellcolor{ReliableGreen}0 & \cellcolor{ReliableYellow}3 & \cellcolor{ReliableYellow}3 & \cellcolor{ReliableOrange}12 & \cellcolor{ReliableGreen}0 & 10 & \cellcolor{ReliableGreen}0 & \cellcolor{ReliableYellow}1 & \cellcolor{ReliableYellow}1 & 9 & \cellcolor{ReliableYellow}3 & 16 &  & \cellcolor{ReliableGray}2.85\% \\
\ \ o1-mini & \cellcolor{ReliableYellow}1 & \cellcolor{ReliableGreen}0 & \cellcolor{ReliableYellow}1 & \cellcolor{ReliableYellow}1 & \cellcolor{ReliableYellow}2 & \cellcolor{ReliableYellow}5 & \cellcolor{ReliableYellow}1 & \cellcolor{ReliableYellow}2 & \cellcolor{ReliableYellow}1 & \cellcolor{ReliableYellow}2 & \cellcolor{ReliableYellow}3 & 9 & \cellcolor{ReliableYellow}3 & 19 &  & \cellcolor{ReliableGray}3.03\% \\
\ \ Gemini Thinking (12/19) & \cellcolor{ReliableGreen}0 & \cellcolor{ReliableGreen}0 & \cellcolor{ReliableYellow}1 & \cellcolor{ReliableGreen}0 & \cellcolor{ReliableYellow}4 & \cellcolor{ReliableYellow}3 & \cellcolor{ReliableYellow}4 & \cellcolor{ReliableOrange}5 & \cellcolor{ReliableYellow}3 & \cellcolor{ReliableYellow}2 & \cellcolor{ReliableYellow}3 & \cellcolor{ReliableOrange}6 & \cellcolor{ReliableOrange}6 & 17 & \cellcolor{ReliableOrange}10 & \cellcolor{ReliableGray}3.03\% \\
\ \ Qwen 2.5 72B Inst & \cellcolor{ReliableGreen}0 & \cellcolor{ReliableGreen}0 & \cellcolor{ReliableGreen}0 & \cellcolor{ReliableYellow}4 & \cellcolor{ReliableOrange}7 & 16 & \cellcolor{ReliableYellow}3 & 10 & \cellcolor{ReliableYellow}3 & \cellcolor{ReliableYellow}3 & \cellcolor{ReliableOrange}4 & \cellcolor{ReliableOrange}8 & \cellcolor{ReliableOrange}5 & 12 &  & \cellcolor{ReliableGray}3.09\% \\
\ \ o3-mini-2025-01-31 (high) & \cellcolor{ReliableGreen}0 & \cellcolor{ReliableGreen}0 & \cellcolor{ReliableYellow}2 & \cellcolor{ReliableYellow}1 & \cellcolor{ReliableYellow}1 & \cellcolor{ReliableYellow}2 & \cellcolor{ReliableYellow}1 & \cellcolor{ReliableYellow}1 & \cellcolor{ReliableGreen}0 & \cellcolor{ReliableGreen}0 & \cellcolor{ReliableYellow}2 & 35 & \cellcolor{ReliableYellow}3 & 14 &  & \cellcolor{ReliableGray}3.18\% \\
\ \ Grok 2 & \cellcolor{ReliableYellow}1 & \cellcolor{ReliableGreen}0 & \cellcolor{ReliableGreen}0 & \cellcolor{ReliableYellow}4 & \cellcolor{ReliableYellow}3 & 15 & \cellcolor{ReliableYellow}1 & \cellcolor{ReliableYellow}2 & \cellcolor{ReliableOrange}5 & \cellcolor{ReliableOrange}5 & \cellcolor{ReliableYellow}1 & \cellcolor{ReliableOrange}8 & \cellcolor{ReliableOrange}5 & 15 & 27 & \cellcolor{ReliableGray}3.20\% \\
\ \ Mistral Large & \cellcolor{ReliableGreen}0 & \cellcolor{ReliableGreen}0 & \cellcolor{ReliableGreen}0 & \cellcolor{ReliableOrange}7 & \cellcolor{ReliableYellow}3 & 34 & \cellcolor{ReliableYellow}4 & \cellcolor{ReliableYellow}1 & 11 & \cellcolor{ReliableYellow}3 & \cellcolor{ReliableOrange}4 & 11 & 11 & 11 &  & \cellcolor{ReliableGray}3.5\% \\
\ \ Gemini 2.0 Flash & \cellcolor{ReliableGreen}0 & \cellcolor{ReliableGreen}0 & \cellcolor{ReliableYellow}1 & \cellcolor{ReliableYellow}4 & \cellcolor{ReliableOrange}8 & \cellcolor{ReliableOrange}6 & \cellcolor{ReliableGreen}0 & \cellcolor{ReliableGreen}0 & \cellcolor{ReliableOrange}7 & \cellcolor{ReliableYellow}2 & \cellcolor{ReliableYellow}3 & \cellcolor{ReliableOrange}7 & \cellcolor{ReliableOrange}6 & 22 & \cellcolor{ReliableOrange}7 & \cellcolor{ReliableGray}3.55\% \\
\ \ Llama 3.3 70B Inst & \cellcolor{ReliableGreen}0 & \cellcolor{ReliableGreen}0 & \cellcolor{ReliableGreen}0 & \cellcolor{ReliableOrange}7 & \cellcolor{ReliableOrange}7 & 44 & \cellcolor{ReliableGreen}0 & \cellcolor{ReliableYellow}1 & \cellcolor{ReliableOrange}10 & \cellcolor{ReliableOrange}4 & \cellcolor{ReliableYellow}2 & \cellcolor{ReliableOrange}8 & \cellcolor{ReliableOrange}7 & 14 &  & \cellcolor{ReliableGray}3.61\% \\
\ \ Llama 3.1 70B Inst & \cellcolor{ReliableYellow}2 & \cellcolor{ReliableGreen}0 & \cellcolor{ReliableGreen}0 & \cellcolor{ReliableOrange}7 & \cellcolor{ReliableOrange}7 & 39 & \cellcolor{ReliableYellow}4 & \cellcolor{ReliableYellow}2 & \cellcolor{ReliableOrange}9 & \cellcolor{ReliableOrange}4 & \cellcolor{ReliableYellow}1 & \cellcolor{ReliableOrange}7 & \cellcolor{ReliableOrange}9 & 17 &  & \cellcolor{ReliableGray}4.02\% \\
\ \ Gemini 1.5 Pro & \cellcolor{ReliableGreen}0 & \cellcolor{ReliableGreen}0 & \cellcolor{ReliableYellow}1 & \cellcolor{ReliableOrange}6 & \cellcolor{ReliableOrange}6 & \cellcolor{ReliableOrange}13 & \cellcolor{ReliableYellow}2 & \cellcolor{ReliableOrange}9 & \cellcolor{ReliableOrange}9 & \cellcolor{ReliableOrange}5 & \cellcolor{ReliableYellow}1 & 14 & \cellcolor{ReliableOrange}7 & 17 & \cellcolor{ReliableOrange}9 & \cellcolor{ReliableGray}4.16\% \\
\ \ GPT-4o mini & \cellcolor{ReliableGreen}0 & \cellcolor{ReliableYellow}1 & \cellcolor{ReliableYellow}1 & \cellcolor{ReliableOrange}6 & \cellcolor{ReliableOrange}6 & 24 & \cellcolor{ReliableYellow}2 & 14 & \cellcolor{ReliableOrange}7 & 14 & \cellcolor{ReliableOrange}4 & 16 & 13 & 27 & 29 & \cellcolor{ReliableGray}6.88\% \\
\ \ Claude 3.5 Haiku & \cellcolor{ReliableGreen}0 & \cellcolor{ReliableGreen}0 & \cellcolor{ReliableYellow}1 & \cellcolor{ReliableOrange}8 & \cellcolor{ReliableOrange}10 & 43 & \cellcolor{ReliableYellow}4 & \cellcolor{ReliableOrange}5 & 11 & 13 & \cellcolor{ReliableYellow}3 & 10 & \cellcolor{ReliableOrange}10 & 31 &  & \cellcolor{ReliableGray}6.88\% \\
\ \ Gemini 1.5 Flash & \cellcolor{ReliableGreen}0 & \cellcolor{ReliableYellow}1 & \cellcolor{ReliableGreen}0 & \cellcolor{ReliableOrange}13 & \cellcolor{ReliableOrange}11 & 23 & \cellcolor{ReliableOrange}5 & 15 & 18 & 17 & \cellcolor{ReliableYellow}3 & 13 & 12 & 21 & \cellcolor{ReliableOrange}11 & \cellcolor{ReliableGray}6.96\% \\
\ \ Mistral Small & \cellcolor{ReliableYellow}1 & \cellcolor{ReliableGreen}0 & \cellcolor{ReliableGreen}0 & \cellcolor{ReliableOrange}11 & 19 & 62 & 27 & 18 & 30 & 12 & 10 & 21 & 25 & 39 &  & \cellcolor{ReliableGray}11.09\% \\
    \bottomrule
    \end{tabular}
    \vspace{-2pt}
    \caption{\textbf{Frontier language models are not reliable on simple tasks.} Here we report the number of errors made by each model on our platinum benchmarks, where every error is manually verified. We observe that even after cleaning benchmarks for errors and ambiguities, almost every model makes mistakes on almost every benchmark. VQA v2.0 is only evaluated for models that support image inputs. The score is computed as the average percentage of errors per benchmark within each category, averaged across the five categories (excluding vision). We color results where models demonstrate \fboxsep2pt\colorbox{green!25}{\bf no errors}, \colorbox{yellow!40}{\bf$\leq$2\% errors}, or \colorbox{orange!40}{\bf$\leq$5\% errors}. \textit{RC: Reading comprehension, Tab: Table understanding, CR: Commonsense reasoning, Vis: Vision.}}
    \label{tab:error_results}

\end{table*}

\begin{table*}
    
    \centering
    \setlength{\tabcolsep}{4pt}
    \begin{tabular}{m{3.3cm}|b{0.6cm}b{0.6cm}b{0.6cm}b{0.6cm}b{0.6cm}b{0.6cm}b{0.6cm}b{0.6cm}b{0.6cm}b{0.6cm}b{0.6cm}b{0.6cm}b{0.6cm}b{0.6cm}}
    \multicolumn{1}{l}{}  & \rot{SingleOp} & \rot{SingleEq} & \rot{MultiArith} & \rot{SVAMP} & \rot{GSM8K} & \rot{MMLU HS Math} & \rot{Logic Ded. 3-Obj} & \rot{Object Counting} & \rot{Navigate} & \rot{TabFact} & \rot{HotpotQA} & \rot{SQuAD2.0} & \rot{DROP} & \rot{Winograd WSC} \\
    \toprule
Avg \# errors, original & 2.5 & 1.1 & 3.5 & 18.8 & 10.1 & 20.5 & 2.4 & 6.3 & 5.9 & 17.3 & 26.4 & 56.2 & 28.4 & 16.6 \\
     Avg \# errors, cleaned &  0.3 & 0.1 & 0.4 & 4.3 & 5.2 & 19.5 & 2.4 & 4.8 & 5.9 & 4.3 & 2.3 & 9.9 & 6.6 & 15.0 \\ \midrule
     
    \textbf{\% errors caused by \newline benchmark errors}  &  90\% & 93\% & 89\% & 77\% & 48\% & 5\% & 0\% & 24\% & 0\% & 75\% & 91\% & 82\% & 77\% & 10\% \\
    
    \bottomrule
    \end{tabular}
    \caption{Here we report the average number of errors across models on each benchmark before and after our revision process. For most benchmarks the number of model errors decreased significantly, often by over 50\%, suggesting that the majority of errors in the original benchmarks can be attributed to label noise rather than genuine model failures. We exclude VQA V2.0, as the original benchmark did not include one single ground-truth label to compare against (see Appendix \ref{app:vqa}).
    For reading comprehension benchmarks, there are often many potential correct answers that are not enumerated within the original label (e.g. answering ``five'' when the solution is ``5''). We use an LLM to resolve such cases in the original benchmark; see Appendix \ref{app:hotpotqa} for details.}
    \label{tab:original_vs_cleaned}
\end{table*}


Now that we have constructed a set of platinum benchmarks, we can use them to measure the reliability of frontier LLMs.

\subsection{Pinpointing the Reliability Frontier}
Within a given category of capabilities (e.g., solving math problems), there exist tasks of a wide spectrum of difficulty (e.g., ranging from a simple addition problem to graduate-level math problems).
So, in order to identify the reliability frontier with greater granularity, we need platinum benchmarks at varying levels of difficulty. Then, we can estimate a model's reliability frontier by identifying the most difficult benchmark it is able to pass (i.e., score 100\% on).


Towards this end, six of the benchmarks that we revised are mathematics benchmarks ranging in difficulty from single operations (SingleOP~\cite{roy2015reasoning}) to high school math problems (MMLU High School Math~\citep{hendrycks2020measuring}). We expect that models might exhibit reliability on sufficiently simple math problems (e.g., current models can generally complete single-digit multiplication 100\% of the time). So, this range should allow is to pinpoint the difficulty at which models begin to lose reliability.






\subsection{Findings}
In Table~\ref{tab:error_results} we report the number of errors made by each model on the revised benchmarks, and in Table~\ref{tab:original_vs_cleaned} we compare the frequency of errors on these benchmarks to the original versions before our revisions. 
We manually inspected all model errors during our revision process, so we can be confident that every model failure we report on our platinum benchmarks is genuine.
Our primary findings are:

\begin{enumerate}
    \item \textbf{Reliability challenges are significant and widespread.} Almost every model makes simple mistakes on almost {\em every} dataset, with the exception of particularly simple math datasets (SingleOP, SingleEq, MultiArith). Considering that these frontier models are now evaluated with PhD-level questions~\cite{rein2023gpqa}, it is alarming that they continue to make such simple mistakes. Several examples of these failures are presented in Appendix~\ref{app:example-failures}.
    \item \textbf{Most ``saturated'' benchmarks are too noisy to evaluate reliability.} Table~\ref{tab:original_vs_cleaned} confirms significant differences in the number of errors made by models on the original and cleaned benchmarks. For most of the original benchmarks, the majority of errors can be attributed to mislabeling or bad questions. For instance, about 75\% the errors that models make on the original SVAMP benchmark are on poorly written or mislabeled questions. Logic datasets show few or no errors, but only because these datasets were programmatically generated~\cite{srivastava2022beyond}.

    \item \textbf{More capable models are also more reliable.} Models that are considered to be more capable (e.g., GPT-4o is more capable than GPT-4o mini, Llama 3.1 405B Instruct is more capable than Llama 3.1 70B Instruct) tend to perform better on our benchmarks. Notably, these more capable models are able to perform perfectly on a few of the benchmarks, whereas our least capable models cannot do so for any benchmark.

    \item \textbf{Models' reliability varies depending on the specific capability.} We find that the o1 series and DeepSeek-R1 exhibit the greatest reliability among mathematics benchmarks, while Claude 3.5 Sonnet (Oct) exhibits the greatest reliability for commonsense reasoning. This finding emphasizes the importance of diverse reliability benchmarking; one might want to choose a different frontier LLM depending on the specific task of interest.
\end{enumerate}


\subsection{Platinum benchmarks allow us to discover new patterns of model failures} 
By investigating model errors on platinum benchmarks and their corresponding chain of thought processes, we can discover \textit{patterns} of failures. We identify two such patterns of questions that lead to consistent collapses in reasoning of frontier LLMs. We initially found an instance of each failure mode by examining models' reasoning processes on failures from our platinum benchmarks. We then verified the consistency of such failures by procedurally constructing similar examples. We outline these failure modes below, and provide further details and a more complete analysis in Appendix \ref{app:patterns}.

\paragraph{Example pattern 1: First event bias} We find that when asked: "What happened second: \{\textit{some event}\} or \{\textit{some other event}\}" given some context, three models (Gemini 1.5 Flash, Gemini 1.5 Pro, and Mistral Small) almost always answer with the first event, and will even explicitly acknowledge they are identifying the first event rather than the second:

\begin{quote}
\small
\textbf{Question:} \textit{\{context paragraph\}} What happened second: Russians blocked Azov or Treaty of Constantinople? \\
\textbf{Solution:}  Treaty of Constantinople\\
\textbf{Gemini 1.5 Pro:} \ldots we can conclude that the Russians blocking Azov happened before the Treaty of Constantinople. Answer: Russians blocked Azov
\end{quote}

\noindent We procedurally generate further examples in this question format and find that each of these three models fails on over 85\% of the examples (see Figure \ref{appfig:event_bias}). Notably, this style of question elicits high failure rates in some models while having no detrimental effect on others.

\paragraph{Example pattern 2: Rounding up primes} We find that when Claude 3.5 Sonnet (June) is asked to solve grade-school math problems involving division, it often incorrectly rounds up the quotient even though the division already resulted in a whole number, leading to an answer greater than the solution by one:

\begin{quote}
\small
\textbf{Question:} The school is planning a field trip. The school has 67 classrooms. There are 66 students in each classroom in the school. If there are 6 seats on each school bus. How many buses are needed to take the trip?\\%(Solution: 737)\\
\textbf{Solution:} 737\\
\textbf{Claude 3.5 Sonnet:} \ldots To find the number of buses needed, we divide the total number of students by the number of seats per bus:
4,422 ÷ 6 = 737. However, since we can't have a fraction of a bus, we need to round up to the next whole number to ensure all students have a seat. Therefore, the school needs 738 buses to take all students on the trip. Answer: 738
\end{quote}

\noindent Interestingly, we find that this behavior is significantly more common when the final answer is closer to being prime (i.e., it has few divisors). For example, when we query Claude 3.5 Sonnet on similarly constructed problems with an answer that is prime, the model makes this exact rounding error 20\% of the time (see Figure \ref{appfig:rounding}).

\section{Related Works}
\label{sec:related}
This work takes inspiration from recent methods that probe the hidden states of LMs to observe interpretable patterns~\citep{alain2017understanding,Kim2017InterpretabilityBF, petroni2019language, hewitt-liang-2019-designing, akyrek2023what} identify false statements~\citep{azaria-mitchell-2023-internal, li2023inferencetime, liu-etal-2024-universal, yuksekgonul2024attention} and hallucinations~\citep{chuang-etal-2024-lookback, su-etal-2024-unsupervised, jiang2024large}. Our work also has connections to literature on early exiting during the forward pass of NN models, with works often using signals from the hidden states to prematurely exit with a prediction on the next token~\citep{xin-etal-2020-deebert, zhou2020bert, xin-etal-2021-berxit, schuster-etal-2021-consistent, jazbec2023towards}. Notably, ~\citet{schuster-etal-2021-consistent} also uses a conformal prediction framework to give a provable error bound on the next token approximation. 

We are the first to show that the internal states can predict a range of behaviors, uninferable from the next token alone, \textbf{before any} of the output tokens are generated. Additionally, we are the first to show that conformal prediction can be used to create early warning systems for a wide range of behaviors like question abstention to format following errors. Our work advances research on understanding the nature of the information contained in the hidden states of LMs~\citep{petroni2019language, anonymous2023does, nylund-etal-2024-time, liu-etal-2024-probing, tighidet-etal-2024-probing}. Specifically, we show that the information contained in the hidden states is relevant not just to the next token, but to behaviors that manifest several tokens later during the LMs generation. 

\section{Conclusions}
In this work, we find that, perhaps surprisingly, representational alignment and adversarial robustness in vision systems are not always correlated. However, we do observe that certain individual benchmarks serve as strong indicators of robust accuracy, particularly those that assess a model's preference for texture information over shape. From this, we hope to encourage future work to leverage insights found in both areas to build more secure and aligned vision systems.


\subsubsection*{Acknowledgments}
% Use unnumbered third level headings for the acknowledgments. All
% acknowledgments, including those to funding agencies, go at the end of the paper.
This material is based upon work supported by, or in part by, the National Science Foundation under Grant No. CNS 2343611, and by the Combat Capabilities Development Command Army Research Office under Grant No. W911NF-21-1-0317 (ARO MURI). Any opinions, findings, and conclusions or recommendations expressed in this publication are those of the author(s) and do not necessarily reflect the views of the National Science Foundation, the U.S. Government, or the Department of Defense. The U.S. Government is authorized to reproduce and distribute reprints for government purposes notwithstanding any copyright notation hereon.


\bibliography{references.bib}
\bibliographystyle{iclr2025_conference}

\appendix
\subsection{Lloyd-Max Algorithm}
\label{subsec:Lloyd-Max}
For a given quantization bitwidth $B$ and an operand $\bm{X}$, the Lloyd-Max algorithm finds $2^B$ quantization levels $\{\hat{x}_i\}_{i=1}^{2^B}$ such that quantizing $\bm{X}$ by rounding each scalar in $\bm{X}$ to the nearest quantization level minimizes the quantization MSE. 

The algorithm starts with an initial guess of quantization levels and then iteratively computes quantization thresholds $\{\tau_i\}_{i=1}^{2^B-1}$ and updates quantization levels $\{\hat{x}_i\}_{i=1}^{2^B}$. Specifically, at iteration $n$, thresholds are set to the midpoints of the previous iteration's levels:
\begin{align*}
    \tau_i^{(n)}=\frac{\hat{x}_i^{(n-1)}+\hat{x}_{i+1}^{(n-1)}}2 \text{ for } i=1\ldots 2^B-1
\end{align*}
Subsequently, the quantization levels are re-computed as conditional means of the data regions defined by the new thresholds:
\begin{align*}
    \hat{x}_i^{(n)}=\mathbb{E}\left[ \bm{X} \big| \bm{X}\in [\tau_{i-1}^{(n)},\tau_i^{(n)}] \right] \text{ for } i=1\ldots 2^B
\end{align*}
where to satisfy boundary conditions we have $\tau_0=-\infty$ and $\tau_{2^B}=\infty$. The algorithm iterates the above steps until convergence.

Figure \ref{fig:lm_quant} compares the quantization levels of a $7$-bit floating point (E3M3) quantizer (left) to a $7$-bit Lloyd-Max quantizer (right) when quantizing a layer of weights from the GPT3-126M model at a per-tensor granularity. As shown, the Lloyd-Max quantizer achieves substantially lower quantization MSE. Further, Table \ref{tab:FP7_vs_LM7} shows the superior perplexity achieved by Lloyd-Max quantizers for bitwidths of $7$, $6$ and $5$. The difference between the quantizers is clear at 5 bits, where per-tensor FP quantization incurs a drastic and unacceptable increase in perplexity, while Lloyd-Max quantization incurs a much smaller increase. Nevertheless, we note that even the optimal Lloyd-Max quantizer incurs a notable ($\sim 1.5$) increase in perplexity due to the coarse granularity of quantization. 

\begin{figure}[h]
  \centering
  \includegraphics[width=0.7\linewidth]{sections/figures/LM7_FP7.pdf}
  \caption{\small Quantization levels and the corresponding quantization MSE of Floating Point (left) vs Lloyd-Max (right) Quantizers for a layer of weights in the GPT3-126M model.}
  \label{fig:lm_quant}
\end{figure}

\begin{table}[h]\scriptsize
\begin{center}
\caption{\label{tab:FP7_vs_LM7} \small Comparing perplexity (lower is better) achieved by floating point quantizers and Lloyd-Max quantizers on a GPT3-126M model for the Wikitext-103 dataset.}
\begin{tabular}{c|cc|c}
\hline
 \multirow{2}{*}{\textbf{Bitwidth}} & \multicolumn{2}{|c|}{\textbf{Floating-Point Quantizer}} & \textbf{Lloyd-Max Quantizer} \\
 & Best Format & Wikitext-103 Perplexity & Wikitext-103 Perplexity \\
\hline
7 & E3M3 & 18.32 & 18.27 \\
6 & E3M2 & 19.07 & 18.51 \\
5 & E4M0 & 43.89 & 19.71 \\
\hline
\end{tabular}
\end{center}
\end{table}

\subsection{Proof of Local Optimality of LO-BCQ}
\label{subsec:lobcq_opt_proof}
For a given block $\bm{b}_j$, the quantization MSE during LO-BCQ can be empirically evaluated as $\frac{1}{L_b}\lVert \bm{b}_j- \bm{\hat{b}}_j\rVert^2_2$ where $\bm{\hat{b}}_j$ is computed from equation (\ref{eq:clustered_quantization_definition}) as $C_{f(\bm{b}_j)}(\bm{b}_j)$. Further, for a given block cluster $\mathcal{B}_i$, we compute the quantization MSE as $\frac{1}{|\mathcal{B}_{i}|}\sum_{\bm{b} \in \mathcal{B}_{i}} \frac{1}{L_b}\lVert \bm{b}- C_i^{(n)}(\bm{b})\rVert^2_2$. Therefore, at the end of iteration $n$, we evaluate the overall quantization MSE $J^{(n)}$ for a given operand $\bm{X}$ composed of $N_c$ block clusters as:
\begin{align*}
    \label{eq:mse_iter_n}
    J^{(n)} = \frac{1}{N_c} \sum_{i=1}^{N_c} \frac{1}{|\mathcal{B}_{i}^{(n)}|}\sum_{\bm{v} \in \mathcal{B}_{i}^{(n)}} \frac{1}{L_b}\lVert \bm{b}- B_i^{(n)}(\bm{b})\rVert^2_2
\end{align*}

At the end of iteration $n$, the codebooks are updated from $\mathcal{C}^{(n-1)}$ to $\mathcal{C}^{(n)}$. However, the mapping of a given vector $\bm{b}_j$ to quantizers $\mathcal{C}^{(n)}$ remains as  $f^{(n)}(\bm{b}_j)$. At the next iteration, during the vector clustering step, $f^{(n+1)}(\bm{b}_j)$ finds new mapping of $\bm{b}_j$ to updated codebooks $\mathcal{C}^{(n)}$ such that the quantization MSE over the candidate codebooks is minimized. Therefore, we obtain the following result for $\bm{b}_j$:
\begin{align*}
\frac{1}{L_b}\lVert \bm{b}_j - C_{f^{(n+1)}(\bm{b}_j)}^{(n)}(\bm{b}_j)\rVert^2_2 \le \frac{1}{L_b}\lVert \bm{b}_j - C_{f^{(n)}(\bm{b}_j)}^{(n)}(\bm{b}_j)\rVert^2_2
\end{align*}

That is, quantizing $\bm{b}_j$ at the end of the block clustering step of iteration $n+1$ results in lower quantization MSE compared to quantizing at the end of iteration $n$. Since this is true for all $\bm{b} \in \bm{X}$, we assert the following:
\begin{equation}
\begin{split}
\label{eq:mse_ineq_1}
    \tilde{J}^{(n+1)} &= \frac{1}{N_c} \sum_{i=1}^{N_c} \frac{1}{|\mathcal{B}_{i}^{(n+1)}|}\sum_{\bm{b} \in \mathcal{B}_{i}^{(n+1)}} \frac{1}{L_b}\lVert \bm{b} - C_i^{(n)}(b)\rVert^2_2 \le J^{(n)}
\end{split}
\end{equation}
where $\tilde{J}^{(n+1)}$ is the the quantization MSE after the vector clustering step at iteration $n+1$.

Next, during the codebook update step (\ref{eq:quantizers_update}) at iteration $n+1$, the per-cluster codebooks $\mathcal{C}^{(n)}$ are updated to $\mathcal{C}^{(n+1)}$ by invoking the Lloyd-Max algorithm \citep{Lloyd}. We know that for any given value distribution, the Lloyd-Max algorithm minimizes the quantization MSE. Therefore, for a given vector cluster $\mathcal{B}_i$ we obtain the following result:

\begin{equation}
    \frac{1}{|\mathcal{B}_{i}^{(n+1)}|}\sum_{\bm{b} \in \mathcal{B}_{i}^{(n+1)}} \frac{1}{L_b}\lVert \bm{b}- C_i^{(n+1)}(\bm{b})\rVert^2_2 \le \frac{1}{|\mathcal{B}_{i}^{(n+1)}|}\sum_{\bm{b} \in \mathcal{B}_{i}^{(n+1)}} \frac{1}{L_b}\lVert \bm{b}- C_i^{(n)}(\bm{b})\rVert^2_2
\end{equation}

The above equation states that quantizing the given block cluster $\mathcal{B}_i$ after updating the associated codebook from $C_i^{(n)}$ to $C_i^{(n+1)}$ results in lower quantization MSE. Since this is true for all the block clusters, we derive the following result: 
\begin{equation}
\begin{split}
\label{eq:mse_ineq_2}
     J^{(n+1)} &= \frac{1}{N_c} \sum_{i=1}^{N_c} \frac{1}{|\mathcal{B}_{i}^{(n+1)}|}\sum_{\bm{b} \in \mathcal{B}_{i}^{(n+1)}} \frac{1}{L_b}\lVert \bm{b}- C_i^{(n+1)}(\bm{b})\rVert^2_2  \le \tilde{J}^{(n+1)}   
\end{split}
\end{equation}

Following (\ref{eq:mse_ineq_1}) and (\ref{eq:mse_ineq_2}), we find that the quantization MSE is non-increasing for each iteration, that is, $J^{(1)} \ge J^{(2)} \ge J^{(3)} \ge \ldots \ge J^{(M)}$ where $M$ is the maximum number of iterations. 
%Therefore, we can say that if the algorithm converges, then it must be that it has converged to a local minimum. 
\hfill $\blacksquare$


\begin{figure}
    \begin{center}
    \includegraphics[width=0.5\textwidth]{sections//figures/mse_vs_iter.pdf}
    \end{center}
    \caption{\small NMSE vs iterations during LO-BCQ compared to other block quantization proposals}
    \label{fig:nmse_vs_iter}
\end{figure}

Figure \ref{fig:nmse_vs_iter} shows the empirical convergence of LO-BCQ across several block lengths and number of codebooks. Also, the MSE achieved by LO-BCQ is compared to baselines such as MXFP and VSQ. As shown, LO-BCQ converges to a lower MSE than the baselines. Further, we achieve better convergence for larger number of codebooks ($N_c$) and for a smaller block length ($L_b$), both of which increase the bitwidth of BCQ (see Eq \ref{eq:bitwidth_bcq}).


\subsection{Additional Accuracy Results}
%Table \ref{tab:lobcq_config} lists the various LOBCQ configurations and their corresponding bitwidths.
\begin{table}
\setlength{\tabcolsep}{4.75pt}
\begin{center}
\caption{\label{tab:lobcq_config} Various LO-BCQ configurations and their bitwidths.}
\begin{tabular}{|c||c|c|c|c||c|c||c|} 
\hline
 & \multicolumn{4}{|c||}{$L_b=8$} & \multicolumn{2}{|c||}{$L_b=4$} & $L_b=2$ \\
 \hline
 \backslashbox{$L_A$\kern-1em}{\kern-1em$N_c$} & 2 & 4 & 8 & 16 & 2 & 4 & 2 \\
 \hline
 64 & 4.25 & 4.375 & 4.5 & 4.625 & 4.375 & 4.625 & 4.625\\
 \hline
 32 & 4.375 & 4.5 & 4.625& 4.75 & 4.5 & 4.75 & 4.75 \\
 \hline
 16 & 4.625 & 4.75& 4.875 & 5 & 4.75 & 5 & 5 \\
 \hline
\end{tabular}
\end{center}
\end{table}

%\subsection{Perplexity achieved by various LO-BCQ configurations on Wikitext-103 dataset}

\begin{table} \centering
\begin{tabular}{|c||c|c|c|c||c|c||c|} 
\hline
 $L_b \rightarrow$& \multicolumn{4}{c||}{8} & \multicolumn{2}{c||}{4} & 2\\
 \hline
 \backslashbox{$L_A$\kern-1em}{\kern-1em$N_c$} & 2 & 4 & 8 & 16 & 2 & 4 & 2  \\
 %$N_c \rightarrow$ & 2 & 4 & 8 & 16 & 2 & 4 & 2 \\
 \hline
 \hline
 \multicolumn{8}{c}{GPT3-1.3B (FP32 PPL = 9.98)} \\ 
 \hline
 \hline
 64 & 10.40 & 10.23 & 10.17 & 10.15 &  10.28 & 10.18 & 10.19 \\
 \hline
 32 & 10.25 & 10.20 & 10.15 & 10.12 &  10.23 & 10.17 & 10.17 \\
 \hline
 16 & 10.22 & 10.16 & 10.10 & 10.09 &  10.21 & 10.14 & 10.16 \\
 \hline
  \hline
 \multicolumn{8}{c}{GPT3-8B (FP32 PPL = 7.38)} \\ 
 \hline
 \hline
 64 & 7.61 & 7.52 & 7.48 &  7.47 &  7.55 &  7.49 & 7.50 \\
 \hline
 32 & 7.52 & 7.50 & 7.46 &  7.45 &  7.52 &  7.48 & 7.48  \\
 \hline
 16 & 7.51 & 7.48 & 7.44 &  7.44 &  7.51 &  7.49 & 7.47  \\
 \hline
\end{tabular}
\caption{\label{tab:ppl_gpt3_abalation} Wikitext-103 perplexity across GPT3-1.3B and 8B models.}
\end{table}

\begin{table} \centering
\begin{tabular}{|c||c|c|c|c||} 
\hline
 $L_b \rightarrow$& \multicolumn{4}{c||}{8}\\
 \hline
 \backslashbox{$L_A$\kern-1em}{\kern-1em$N_c$} & 2 & 4 & 8 & 16 \\
 %$N_c \rightarrow$ & 2 & 4 & 8 & 16 & 2 & 4 & 2 \\
 \hline
 \hline
 \multicolumn{5}{|c|}{Llama2-7B (FP32 PPL = 5.06)} \\ 
 \hline
 \hline
 64 & 5.31 & 5.26 & 5.19 & 5.18  \\
 \hline
 32 & 5.23 & 5.25 & 5.18 & 5.15  \\
 \hline
 16 & 5.23 & 5.19 & 5.16 & 5.14  \\
 \hline
 \multicolumn{5}{|c|}{Nemotron4-15B (FP32 PPL = 5.87)} \\ 
 \hline
 \hline
 64  & 6.3 & 6.20 & 6.13 & 6.08  \\
 \hline
 32  & 6.24 & 6.12 & 6.07 & 6.03  \\
 \hline
 16  & 6.12 & 6.14 & 6.04 & 6.02  \\
 \hline
 \multicolumn{5}{|c|}{Nemotron4-340B (FP32 PPL = 3.48)} \\ 
 \hline
 \hline
 64 & 3.67 & 3.62 & 3.60 & 3.59 \\
 \hline
 32 & 3.63 & 3.61 & 3.59 & 3.56 \\
 \hline
 16 & 3.61 & 3.58 & 3.57 & 3.55 \\
 \hline
\end{tabular}
\caption{\label{tab:ppl_llama7B_nemo15B} Wikitext-103 perplexity compared to FP32 baseline in Llama2-7B and Nemotron4-15B, 340B models}
\end{table}

%\subsection{Perplexity achieved by various LO-BCQ configurations on MMLU dataset}


\begin{table} \centering
\begin{tabular}{|c||c|c|c|c||c|c|c|c|} 
\hline
 $L_b \rightarrow$& \multicolumn{4}{c||}{8} & \multicolumn{4}{c||}{8}\\
 \hline
 \backslashbox{$L_A$\kern-1em}{\kern-1em$N_c$} & 2 & 4 & 8 & 16 & 2 & 4 & 8 & 16  \\
 %$N_c \rightarrow$ & 2 & 4 & 8 & 16 & 2 & 4 & 2 \\
 \hline
 \hline
 \multicolumn{5}{|c|}{Llama2-7B (FP32 Accuracy = 45.8\%)} & \multicolumn{4}{|c|}{Llama2-70B (FP32 Accuracy = 69.12\%)} \\ 
 \hline
 \hline
 64 & 43.9 & 43.4 & 43.9 & 44.9 & 68.07 & 68.27 & 68.17 & 68.75 \\
 \hline
 32 & 44.5 & 43.8 & 44.9 & 44.5 & 68.37 & 68.51 & 68.35 & 68.27  \\
 \hline
 16 & 43.9 & 42.7 & 44.9 & 45 & 68.12 & 68.77 & 68.31 & 68.59  \\
 \hline
 \hline
 \multicolumn{5}{|c|}{GPT3-22B (FP32 Accuracy = 38.75\%)} & \multicolumn{4}{|c|}{Nemotron4-15B (FP32 Accuracy = 64.3\%)} \\ 
 \hline
 \hline
 64 & 36.71 & 38.85 & 38.13 & 38.92 & 63.17 & 62.36 & 63.72 & 64.09 \\
 \hline
 32 & 37.95 & 38.69 & 39.45 & 38.34 & 64.05 & 62.30 & 63.8 & 64.33  \\
 \hline
 16 & 38.88 & 38.80 & 38.31 & 38.92 & 63.22 & 63.51 & 63.93 & 64.43  \\
 \hline
\end{tabular}
\caption{\label{tab:mmlu_abalation} Accuracy on MMLU dataset across GPT3-22B, Llama2-7B, 70B and Nemotron4-15B models.}
\end{table}


%\subsection{Perplexity achieved by various LO-BCQ configurations on LM evaluation harness}

\begin{table} \centering
\begin{tabular}{|c||c|c|c|c||c|c|c|c|} 
\hline
 $L_b \rightarrow$& \multicolumn{4}{c||}{8} & \multicolumn{4}{c||}{8}\\
 \hline
 \backslashbox{$L_A$\kern-1em}{\kern-1em$N_c$} & 2 & 4 & 8 & 16 & 2 & 4 & 8 & 16  \\
 %$N_c \rightarrow$ & 2 & 4 & 8 & 16 & 2 & 4 & 2 \\
 \hline
 \hline
 \multicolumn{5}{|c|}{Race (FP32 Accuracy = 37.51\%)} & \multicolumn{4}{|c|}{Boolq (FP32 Accuracy = 64.62\%)} \\ 
 \hline
 \hline
 64 & 36.94 & 37.13 & 36.27 & 37.13 & 63.73 & 62.26 & 63.49 & 63.36 \\
 \hline
 32 & 37.03 & 36.36 & 36.08 & 37.03 & 62.54 & 63.51 & 63.49 & 63.55  \\
 \hline
 16 & 37.03 & 37.03 & 36.46 & 37.03 & 61.1 & 63.79 & 63.58 & 63.33  \\
 \hline
 \hline
 \multicolumn{5}{|c|}{Winogrande (FP32 Accuracy = 58.01\%)} & \multicolumn{4}{|c|}{Piqa (FP32 Accuracy = 74.21\%)} \\ 
 \hline
 \hline
 64 & 58.17 & 57.22 & 57.85 & 58.33 & 73.01 & 73.07 & 73.07 & 72.80 \\
 \hline
 32 & 59.12 & 58.09 & 57.85 & 58.41 & 73.01 & 73.94 & 72.74 & 73.18  \\
 \hline
 16 & 57.93 & 58.88 & 57.93 & 58.56 & 73.94 & 72.80 & 73.01 & 73.94  \\
 \hline
\end{tabular}
\caption{\label{tab:mmlu_abalation} Accuracy on LM evaluation harness tasks on GPT3-1.3B model.}
\end{table}

\begin{table} \centering
\begin{tabular}{|c||c|c|c|c||c|c|c|c|} 
\hline
 $L_b \rightarrow$& \multicolumn{4}{c||}{8} & \multicolumn{4}{c||}{8}\\
 \hline
 \backslashbox{$L_A$\kern-1em}{\kern-1em$N_c$} & 2 & 4 & 8 & 16 & 2 & 4 & 8 & 16  \\
 %$N_c \rightarrow$ & 2 & 4 & 8 & 16 & 2 & 4 & 2 \\
 \hline
 \hline
 \multicolumn{5}{|c|}{Race (FP32 Accuracy = 41.34\%)} & \multicolumn{4}{|c|}{Boolq (FP32 Accuracy = 68.32\%)} \\ 
 \hline
 \hline
 64 & 40.48 & 40.10 & 39.43 & 39.90 & 69.20 & 68.41 & 69.45 & 68.56 \\
 \hline
 32 & 39.52 & 39.52 & 40.77 & 39.62 & 68.32 & 67.43 & 68.17 & 69.30  \\
 \hline
 16 & 39.81 & 39.71 & 39.90 & 40.38 & 68.10 & 66.33 & 69.51 & 69.42  \\
 \hline
 \hline
 \multicolumn{5}{|c|}{Winogrande (FP32 Accuracy = 67.88\%)} & \multicolumn{4}{|c|}{Piqa (FP32 Accuracy = 78.78\%)} \\ 
 \hline
 \hline
 64 & 66.85 & 66.61 & 67.72 & 67.88 & 77.31 & 77.42 & 77.75 & 77.64 \\
 \hline
 32 & 67.25 & 67.72 & 67.72 & 67.00 & 77.31 & 77.04 & 77.80 & 77.37  \\
 \hline
 16 & 68.11 & 68.90 & 67.88 & 67.48 & 77.37 & 78.13 & 78.13 & 77.69  \\
 \hline
\end{tabular}
\caption{\label{tab:mmlu_abalation} Accuracy on LM evaluation harness tasks on GPT3-8B model.}
\end{table}

\begin{table} \centering
\begin{tabular}{|c||c|c|c|c||c|c|c|c|} 
\hline
 $L_b \rightarrow$& \multicolumn{4}{c||}{8} & \multicolumn{4}{c||}{8}\\
 \hline
 \backslashbox{$L_A$\kern-1em}{\kern-1em$N_c$} & 2 & 4 & 8 & 16 & 2 & 4 & 8 & 16  \\
 %$N_c \rightarrow$ & 2 & 4 & 8 & 16 & 2 & 4 & 2 \\
 \hline
 \hline
 \multicolumn{5}{|c|}{Race (FP32 Accuracy = 40.67\%)} & \multicolumn{4}{|c|}{Boolq (FP32 Accuracy = 76.54\%)} \\ 
 \hline
 \hline
 64 & 40.48 & 40.10 & 39.43 & 39.90 & 75.41 & 75.11 & 77.09 & 75.66 \\
 \hline
 32 & 39.52 & 39.52 & 40.77 & 39.62 & 76.02 & 76.02 & 75.96 & 75.35  \\
 \hline
 16 & 39.81 & 39.71 & 39.90 & 40.38 & 75.05 & 73.82 & 75.72 & 76.09  \\
 \hline
 \hline
 \multicolumn{5}{|c|}{Winogrande (FP32 Accuracy = 70.64\%)} & \multicolumn{4}{|c|}{Piqa (FP32 Accuracy = 79.16\%)} \\ 
 \hline
 \hline
 64 & 69.14 & 70.17 & 70.17 & 70.56 & 78.24 & 79.00 & 78.62 & 78.73 \\
 \hline
 32 & 70.96 & 69.69 & 71.27 & 69.30 & 78.56 & 79.49 & 79.16 & 78.89  \\
 \hline
 16 & 71.03 & 69.53 & 69.69 & 70.40 & 78.13 & 79.16 & 79.00 & 79.00  \\
 \hline
\end{tabular}
\caption{\label{tab:mmlu_abalation} Accuracy on LM evaluation harness tasks on GPT3-22B model.}
\end{table}

\begin{table} \centering
\begin{tabular}{|c||c|c|c|c||c|c|c|c|} 
\hline
 $L_b \rightarrow$& \multicolumn{4}{c||}{8} & \multicolumn{4}{c||}{8}\\
 \hline
 \backslashbox{$L_A$\kern-1em}{\kern-1em$N_c$} & 2 & 4 & 8 & 16 & 2 & 4 & 8 & 16  \\
 %$N_c \rightarrow$ & 2 & 4 & 8 & 16 & 2 & 4 & 2 \\
 \hline
 \hline
 \multicolumn{5}{|c|}{Race (FP32 Accuracy = 44.4\%)} & \multicolumn{4}{|c|}{Boolq (FP32 Accuracy = 79.29\%)} \\ 
 \hline
 \hline
 64 & 42.49 & 42.51 & 42.58 & 43.45 & 77.58 & 77.37 & 77.43 & 78.1 \\
 \hline
 32 & 43.35 & 42.49 & 43.64 & 43.73 & 77.86 & 75.32 & 77.28 & 77.86  \\
 \hline
 16 & 44.21 & 44.21 & 43.64 & 42.97 & 78.65 & 77 & 76.94 & 77.98  \\
 \hline
 \hline
 \multicolumn{5}{|c|}{Winogrande (FP32 Accuracy = 69.38\%)} & \multicolumn{4}{|c|}{Piqa (FP32 Accuracy = 78.07\%)} \\ 
 \hline
 \hline
 64 & 68.9 & 68.43 & 69.77 & 68.19 & 77.09 & 76.82 & 77.09 & 77.86 \\
 \hline
 32 & 69.38 & 68.51 & 68.82 & 68.90 & 78.07 & 76.71 & 78.07 & 77.86  \\
 \hline
 16 & 69.53 & 67.09 & 69.38 & 68.90 & 77.37 & 77.8 & 77.91 & 77.69  \\
 \hline
\end{tabular}
\caption{\label{tab:mmlu_abalation} Accuracy on LM evaluation harness tasks on Llama2-7B model.}
\end{table}

\begin{table} \centering
\begin{tabular}{|c||c|c|c|c||c|c|c|c|} 
\hline
 $L_b \rightarrow$& \multicolumn{4}{c||}{8} & \multicolumn{4}{c||}{8}\\
 \hline
 \backslashbox{$L_A$\kern-1em}{\kern-1em$N_c$} & 2 & 4 & 8 & 16 & 2 & 4 & 8 & 16  \\
 %$N_c \rightarrow$ & 2 & 4 & 8 & 16 & 2 & 4 & 2 \\
 \hline
 \hline
 \multicolumn{5}{|c|}{Race (FP32 Accuracy = 48.8\%)} & \multicolumn{4}{|c|}{Boolq (FP32 Accuracy = 85.23\%)} \\ 
 \hline
 \hline
 64 & 49.00 & 49.00 & 49.28 & 48.71 & 82.82 & 84.28 & 84.03 & 84.25 \\
 \hline
 32 & 49.57 & 48.52 & 48.33 & 49.28 & 83.85 & 84.46 & 84.31 & 84.93  \\
 \hline
 16 & 49.85 & 49.09 & 49.28 & 48.99 & 85.11 & 84.46 & 84.61 & 83.94  \\
 \hline
 \hline
 \multicolumn{5}{|c|}{Winogrande (FP32 Accuracy = 79.95\%)} & \multicolumn{4}{|c|}{Piqa (FP32 Accuracy = 81.56\%)} \\ 
 \hline
 \hline
 64 & 78.77 & 78.45 & 78.37 & 79.16 & 81.45 & 80.69 & 81.45 & 81.5 \\
 \hline
 32 & 78.45 & 79.01 & 78.69 & 80.66 & 81.56 & 80.58 & 81.18 & 81.34  \\
 \hline
 16 & 79.95 & 79.56 & 79.79 & 79.72 & 81.28 & 81.66 & 81.28 & 80.96  \\
 \hline
\end{tabular}
\caption{\label{tab:mmlu_abalation} Accuracy on LM evaluation harness tasks on Llama2-70B model.}
\end{table}

%\section{MSE Studies}
%\textcolor{red}{TODO}


\subsection{Number Formats and Quantization Method}
\label{subsec:numFormats_quantMethod}
\subsubsection{Integer Format}
An $n$-bit signed integer (INT) is typically represented with a 2s-complement format \citep{yao2022zeroquant,xiao2023smoothquant,dai2021vsq}, where the most significant bit denotes the sign.

\subsubsection{Floating Point Format}
An $n$-bit signed floating point (FP) number $x$ comprises of a 1-bit sign ($x_{\mathrm{sign}}$), $B_m$-bit mantissa ($x_{\mathrm{mant}}$) and $B_e$-bit exponent ($x_{\mathrm{exp}}$) such that $B_m+B_e=n-1$. The associated constant exponent bias ($E_{\mathrm{bias}}$) is computed as $(2^{{B_e}-1}-1)$. We denote this format as $E_{B_e}M_{B_m}$.  

\subsubsection{Quantization Scheme}
\label{subsec:quant_method}
A quantization scheme dictates how a given unquantized tensor is converted to its quantized representation. We consider FP formats for the purpose of illustration. Given an unquantized tensor $\bm{X}$ and an FP format $E_{B_e}M_{B_m}$, we first, we compute the quantization scale factor $s_X$ that maps the maximum absolute value of $\bm{X}$ to the maximum quantization level of the $E_{B_e}M_{B_m}$ format as follows:
\begin{align}
\label{eq:sf}
    s_X = \frac{\mathrm{max}(|\bm{X}|)}{\mathrm{max}(E_{B_e}M_{B_m})}
\end{align}
In the above equation, $|\cdot|$ denotes the absolute value function.

Next, we scale $\bm{X}$ by $s_X$ and quantize it to $\hat{\bm{X}}$ by rounding it to the nearest quantization level of $E_{B_e}M_{B_m}$ as:

\begin{align}
\label{eq:tensor_quant}
    \hat{\bm{X}} = \text{round-to-nearest}\left(\frac{\bm{X}}{s_X}, E_{B_e}M_{B_m}\right)
\end{align}

We perform dynamic max-scaled quantization \citep{wu2020integer}, where the scale factor $s$ for activations is dynamically computed during runtime.

\subsection{Vector Scaled Quantization}
\begin{wrapfigure}{r}{0.35\linewidth}
  \centering
  \includegraphics[width=\linewidth]{sections/figures/vsquant.jpg}
  \caption{\small Vectorwise decomposition for per-vector scaled quantization (VSQ \citep{dai2021vsq}).}
  \label{fig:vsquant}
\end{wrapfigure}
During VSQ \citep{dai2021vsq}, the operand tensors are decomposed into 1D vectors in a hardware friendly manner as shown in Figure \ref{fig:vsquant}. Since the decomposed tensors are used as operands in matrix multiplications during inference, it is beneficial to perform this decomposition along the reduction dimension of the multiplication. The vectorwise quantization is performed similar to tensorwise quantization described in Equations \ref{eq:sf} and \ref{eq:tensor_quant}, where a scale factor $s_v$ is required for each vector $\bm{v}$ that maps the maximum absolute value of that vector to the maximum quantization level. While smaller vector lengths can lead to larger accuracy gains, the associated memory and computational overheads due to the per-vector scale factors increases. To alleviate these overheads, VSQ \citep{dai2021vsq} proposed a second level quantization of the per-vector scale factors to unsigned integers, while MX \citep{rouhani2023shared} quantizes them to integer powers of 2 (denoted as $2^{INT}$).

\subsubsection{MX Format}
The MX format proposed in \citep{rouhani2023microscaling} introduces the concept of sub-block shifting. For every two scalar elements of $b$-bits each, there is a shared exponent bit. The value of this exponent bit is determined through an empirical analysis that targets minimizing quantization MSE. We note that the FP format $E_{1}M_{b}$ is strictly better than MX from an accuracy perspective since it allocates a dedicated exponent bit to each scalar as opposed to sharing it across two scalars. Therefore, we conservatively bound the accuracy of a $b+2$-bit signed MX format with that of a $E_{1}M_{b}$ format in our comparisons. For instance, we use E1M2 format as a proxy for MX4.

\begin{figure}
    \centering
    \includegraphics[width=1\linewidth]{sections//figures/BlockFormats.pdf}
    \caption{\small Comparing LO-BCQ to MX format.}
    \label{fig:block_formats}
\end{figure}

Figure \ref{fig:block_formats} compares our $4$-bit LO-BCQ block format to MX \citep{rouhani2023microscaling}. As shown, both LO-BCQ and MX decompose a given operand tensor into block arrays and each block array into blocks. Similar to MX, we find that per-block quantization ($L_b < L_A$) leads to better accuracy due to increased flexibility. While MX achieves this through per-block $1$-bit micro-scales, we associate a dedicated codebook to each block through a per-block codebook selector. Further, MX quantizes the per-block array scale-factor to E8M0 format without per-tensor scaling. In contrast during LO-BCQ, we find that per-tensor scaling combined with quantization of per-block array scale-factor to E4M3 format results in superior inference accuracy across models. 



\end{document}
