\section{Motivation}
\label{sec:motivation}

\begin{quote}
\textit{If you can't measure it, you can't improve it.}

\hfill ---Lord Kelvin
\end{quote}

\noindent
This section outlines several challenges with existing systems that motivate \sysname's design.

\noindent\textbf{Manual performance tuning.}
Optimizing the performance of RSM-based protocols necessitates precise tuning to the conditions of their deployment environment.
Performance parameters---such as the replication degree, batch size, concurrency, shard count, and network timeouts---must be carefully calibrated to balance efficiency and reliability.
For instance, RSMs operating over wide area networks with inherently higher and less predictable latencies require more conservative timeouts compared to those on local area networks~\cite{quepaxa,atlas}.
For leader-based protocols, choosing a short timeout may lead to consecutive leader changes~\cite{mytumbler}, while a longer timeout results in slower recovery from failures.
A single leader often becomes a performance bottleneck~\cite{hovercraft,scalablesmr}, and rotating the leader-role~\cite{hotstuff} can result in significant performance degradation when some replicas are faulty~\cite{mytumbler,beegees}.
Despite these challenges, the prevalent practice of manual performance tuning is fraught with complexities~\cite{etcd-tuning}, often due to inadequate feedback mechanisms, resulting in configurations that are less than ideal.

\noindent\textbf{Limitations of local-only measurements.}
Local measurements~\cite{bft-smart,local1,local2,local3,local4}, where individual replicas capture and use metrics without sharing this information with their counterparts, present inherent limitations.
Such local-only observations can lead to discrepancies, as replicas might draw inconsistent conclusions about their operating environment.
Furthermore, replicas cannot be held accountable for their configuration decisions based on local measurements as they are not verifiable by other replicas.

\noindent\textbf{We don't measure enough.}
RSM-based production systems use telemetry frameworks~\cite{algorand, tendermint} to collect various metrics such as latency, throughput, CPU and memory load, and so on.
These metrics are typically used for offline analysis to understand and improve the RSM's behavior under various workloads and network conditions.
However, it is less common for these metrics to be used as input to real-time automated reconfiguration, e.g., to optimize the RSM's operational performance.
Such optimization hinges on accurate and agreed-upon measurements.

\noindent\textbf{Measurement-based role assignment.}
A wide range of strategies exists for RSM optimization~\cite{berger2023sok} and analysis, including pipelining~\cite{rcc,sbft}, committee selection~\cite{dumbo,proteus,proof-of-qos}, hierarchical consensus~\cite{kauri, byzcoin}, and forensics~\cite{bft-forensics}.
Typically, these optimizations rely on special roles, like coordinators, validators, or committee members.
Conventionally, these roles are assigned in a randomized~\cite{algorand, kauri} or predefined fashion~\cite{zorfu, mirbft, mencius}, requiring trial-and-error to find a working assignment.
This precludes informed decisions and learning from past configuration failures.
Others use an external service~\cite{zookeeper, etcd, king}, which may pose a single point of failure and does not align with the BFT trust model.
Assignment based on measurements and prior experience is promising but requires an infrastructure to record measurements and decisions and draw conclusions.
Moreover, the infrastructure must be suitable for the BFT trust model and sufficiently flexible to support a variety of optimization strategies.

\noindent\textbf{Collaborative optimization.}
Optimizing RSM configurations often involves exploring a vast search space, and centralizing this task at a single replica leads to performance bottlenecks.
A better solution is partitioning the search space into smaller tasks and distributing them across the replicas using scatter-gather~\cite{mpi} or map-reduce~\cite{map-reduce, spark} techniques.
Despite their potential, these methods are surprisingly underutilized for optimizing RSM configurations.

\noindent\textbf{Random configuration decisions are bad.}
Given a set of replicas which are deployed across the world, any two randomly selected configurations can have huge variance in performance.
The distribution of the configuration latencies can be non-uniform, and the performance of the RSM can be significantly impacted by the configuration decision.
The possibility of randomly selecting a bad configuration is high therefore choosing a random configuration is not a good idea.




\noindent\textbf{Non-deterministic configuration optimization.}
Finding an effective configuration can be framed as a combinatorial optimization problem.
Due to the vastness of possible configurations, finding an optimal solution is often unfeasible.
This makes heuristic approaches, including machine learning, more viable than deterministic methods.
However, due to the inherent non-determinism of these heuristic solutions, existing systems~\cite{aware} do not support non-deterministic configuration optimization.





