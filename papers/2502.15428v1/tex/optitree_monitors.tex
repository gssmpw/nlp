\subsection{\optitree Monitors}
\label{sec:sm-trees}

This section describes how we adjust the existing sensors and monitors to work with tree topologies.

First, we do not record latencies while the RSM is in a tree configuration.
This is because the tree structure prevents direct communication between replicas, making it difficult to measure individual latencies.
Instead we measure latencies only in a star topology.
Thus, we do not need to adjust the latency monitoring for trees.
At first glance, this lack of recorded tree latencies may seem like a weakness that an adversary could exploit.
A Byzantine replica may initially behave correctly in order to be selected as an internal node.
However, if it later misbehaves, e.g., by delaying messages, it will be suspected by the suspicion monitor, resulting in a failed tree.
When a tree fails, the replicas revert to the star topology to record new measurements.
In \cref{sec:tree-formation}, we adapt the configuration sensor to search for efficient trees.
In \cref{sec:tree-suspicion}, we modify the suspicion sensor to detect slow behavior in the tree overlay.
We also optimize the suspicion monitor to compute a candidate set more efficiently, limiting the number of reconfigurations caused by faulty replicas.

\subsubsection{Configuration Monitoring in Trees}
\label{sec:tree-formation}

Configuration monitoring in \optitree is responsible for finding efficient tree configurations and activating the best one.
As outlined in \cref{sec:config_sensor}, the \cfgsensor may start a search periodically or when the current configuration is no longer valid.
A tree configuration is valid if all internal nodes are part of the candidate set \Cand from the \susmonitor.
Thus, when a tree fails, \optitree reverts to a star topology to record suspicions.
Once these suspicions result in a change of \Cand, the \cfgsensor initiates a search for a new tree.

The \cfgsensor searches for a valid, low-latency tree using latencies \lm from the \latmonitor and the candidate set \Cand from the \susmonitor.
Additionally, the \cfgsensor uses the estimated number of misbehaving replicas \un to search for a tree that provides low latency despite \un many unresponsive leaves.

Since searching for optimal trees in large systems is intractable, we use a heuristic optimization technique based on simulated annealing~\cite{simulatedannealing}.
To ensure that only valid trees are considered, internal nodes must be selected from \Cand.
We initiate the search with a random tree, ensuring that internal nodes are only selected from \Cand.
Simulated annealing uses a \textit{mutate} function to randomly swap two nodes in the tree.
Our mutate function ensures that internal nodes are only swapped with other nodes from \Cand.

Simulated annealing then compares the latency of the mutated tree with that of the best tree found so far and probabilistically selects the lower-latency tree for further mutation.
The tree latency is computed using the scoring function in~\cref{def:score} with parameter $\kn=\q+u$.
This score is the latency required to collect $\q$ votes from the tree, assuming that \un leaf nodes are unresponsive.
While simulated annealing allows exploration beyond local minima, it does not guarantee a global minimum.
The search ends when the \textit{search timer} expires or when simulated annealing converges.
At this point, the \cfgsensor logs the lowest-latency tree found.

The \cfgmonitor waits for $\f+1$ valid tree configurations to be committed to the log and ranks them using the scoring function.
It then selects the lowest-latency tree for reconfiguration.

\begin{definition}\label{def:score}
  For a tree $\tree$, the $\score(\kn, \tree)$ is the minimum latency required to collect votes from $\kn$ nodes.

  This score is computed using the link latencies in the tree, derived from $\lm$, taking into account that intermediate nodes first collect votes from their children before forwarding them to the root.

  The \textit{aggregation latency} $\agt{\nin}$, for intermediate node $\nin$, is the maximum latency from $\nin$ to any of its children, \Ch{\nin}:
  \[
    \agt{\nin} = \max_{\nv \in \Ch{\nin}} \lme{\nin}{\nv}
  \]
  Let $\Intermediate_k$ represent subsets of all intermediate nodes such that the subtrees rooted at these nodes contain a total of at least $k$ nodes:
  \[
    \Intermediate_k = \{M \subset \Intermediate \mid \sum_{\nin \in M} |\Ch{\nin}| + 1 \geq k \}
  \]
  Thus, the time to collect $\kn$ votes is the minimum time to collect aggregates from subtrees in a set from $\Intermediate_{\kn-1}$, since the root $\nr$'s vote is added separately.
  \[
    \score(\kn, \tree) = \min_{M \in \Intermediate_{\kn-1}} \left( \max_{\nin \in M} \left( \agt{\nin} + \lme{\nin}{\nr} \right) \right)
  \]
\end{definition}

\subsubsection{Misbehavior Monitoring in Trees}

We augment misbehavior monitoring for trees with an additional rule for invalid vote aggregation.
In a tree configuration, intermediate nodes are responsible for aggregating votes from their children and forwarding them to the root.
However, if an intermediate node does not receive a vote from a child node, the aggregate vote must include a suspicion for the missing vote.
That is, the aggregate must include $\bn + 1$ votes or suspicions.
Failure to meet this requirement allows the root to use the aggregation vote as \textit{proof-of-misbehavior} against the intermediate node.

\subsubsection{Suspicion Monitoring in Trees}
\label{sec:tree-suspicion}

The tree's more complex communication pattern requires adjusting how suspicions are raised.
\cref{tab:tree-suspicion} shows the adjusted conditions for raising suspicions in a tree configuration.
In a star configuration (\cref{sec:suspicion}), the leader suspects replicas that fail to reply in time~(a).
In a tree, internal nodes suspect their children if they do not respond in time \tcond{a}.
However, the root's timeout must also be adjusted to account for the time it takes the intermediate node to collect votes from its children \tcond{a'}.

Similarly, in a star configuration, replicas suspect the leader upon a view timeout (b).
In a tree configuration, we use the score of the configuration $\score(\q+\un,\tree)$ as timeout, adjusted by $\delta$. 
But only intermediate nodes directly suspect the root upon a view timeout \tcond{b}.
However, if faulty intermediate nodes prevent the root from collecting enough votes to advance the view, the root instead forwards suspicions to avoid being suspected.
If a leaf does not receive a proposal, it could be caused by the root not sending the proposal or by the intermediate node failing to forward it.
Therefore, leaf nodes do not immediately issue suspicions on view timeout; instead, they wait for the internal nodes to raise suspicions.
If no such suspicion is raised, the leaf suspects both the root and the intermediate node \tcond{b'}.
Finally, correct replicas reciprocate suspicions raised by faulty replicas (c), as in the star configuration.

\begin{table}[ht]
  \caption{Suspicion cases in a tree topology.}
  \label{tab:tree-suspicion}
  \footnotesize
  \centering
  \begin{tabular}{@{}p{0.7cm}@{}p{0.35cm}@{}p{2.6cm}@{}p{2.4cm}@{}p{1.95cm}@{}}
    \toprule
    \multicolumn{3}{@{}l}{\textbf{Condition for Suspicion}} & \textbf{Timeout}                & \textbf{Suspicion} \\
    \midrule
    \tcond{a}  & \nin:    & No response from $\nleafT$  & $\timeout{\nin}{\nleafT}$           & \suspm{Slow}{\nin}{\nleafT} \\
    \tcond{a'} & \nr:     & No response from $\nin$     & $\df{(\agt{\nin}+\lme{\nr}{\nin})}$ & \suspm{Slow}{\nr}{\nin} \\
    \tcond{b}  & \nin:    & View timeout and            & $\df{\score(\q+\un,\tree)}$             & \suspm{Slow}{\nin}{\nr} \\
    &          & no suspicion                &                                     & \\
    \tcond{b'} & \nleafT: & View timeout and            & $\df{\score(\q+\un,\tree)}$             & \suspm{Slow}{\nleafT}{\nr} \\
    &          & no suspicion during         & $\df{\score(\q+\un,\tree)}$             & $\wedge\suspm{Slow}{\nleafT}{\nin}$ \\
    \tcond{c}  & \na:     & False suspicion from $\nb$  &                                     & \suspm{False}{\na}{\nb} \\
    \bottomrule
  \end{tabular}
\end{table}

When computing a maximum independent set for candidate selection (\cref{sec:suspicion}), a single intermediate node suspected by a leaf may cause the candidate set to update and require a reconfiguration.
Faulty replicas may force $\Omega(\f^2)$ reconfigurations before they are excluded from the candidate set~\cite{jehl2019quorum}.

We use a different approach to compute $\un$ and the candidate set $\Cand$ for trees.
This method produces smaller candidate sets and requires at most $2\f$ reconfigurations, while avoiding the computational complexity of finding a maximum independent set.

We introduce two additional data structures, \MG and \T, both derived from $\G$.
\MG is a maximal set of disjoint edges in $\G$.
For every edge in \MG, at least one of the adjacent vertices is a faulty replica.
We therefore exclude both replicas from $\Cand$.
Whenever an edge is added to \G, the suspicion monitor checks if \MG is still maximal.
This check includes possibly removing one edge from \MG and adding two new ones.
We also determine $\T$, the set of vertices that are not adjacent to an edge in \MG but part of a triangle in $\G$ with an edge in \MG.
We also exclude replicas in $\T$ from $\Cand$.

Finally, we return $\Cand$ containing replicas (vertices) in $\G$ that are not in $\T$ or adjacent to an edge in \MG.
We also return the estimated number of faulty replicas $\un=|\MG|+|\T|$.

\begin{figure}
  \centering\
  \footnotesize
  \begin{tikzpicture}
    [vertex/.style={circle, draw=black, fill=black!10!white,thick, inner sep=0pt, minimum size=3.5mm},
      new/.style={draw=red, fill=red!30!white},
      exclude/.style={fill=red!40!white},
      tree/.style={draw=blue!60!white, thick},
    label/.style={right=4pt}]
    \node[vertex, exclude] (A) at (0.5,0){\susn{1}};
    \node[vertex, exclude] (B) at (1.5,0){\susn{2}};
    \node[vertex, exclude] (C) at (2.5,0){\susn{3}};
    \node[vertex, exclude] (D) at (0,1){\susn{4}};
    \node[vertex, exclude] (E) at (1,1){$\naT$};
    \node[vertex] (F) at (2,1){\cn{1}};
    \node[vertex] (G) at (3,1){\cn{2}};
    \node[vertex, exclude] (H) at (0.5,2){$\nbC$};
    \node[vertex] (I) at (1.5,2){\cn{3}};
    \node[vertex] (J) at (2.5,2){$\nr$};

    \foreach \p/\c in {J/I, J/G, J/F, I/E, I/D, F/H, F/C, G/B, G/A} {
      \draw[tree] (\p) -- (\c);
    }

    \draw[thick, new] (A) -- (D);
    \draw[thick, new] (B) -- (C);
    \draw[dashed, thick, <->] (A) -- (D);
    \draw[dashed, thick, <->] (A) -- (E);
    \draw[dashed, thick, <->] (D) -- (E);
    \draw[dashed, thick, <->] (B) -- (C);
    \draw[dashed, thick, <->] (C) -- (G);
    \draw[dotted, thick, ->] (I) -- (H);

    \node[vertex,exclude] (X2) at (3.5,0){};
    \node[label] at (X2) {Excluded from internal nodes};

    \node[vertex] (X) at (3.5,0.5){};
    \node[vertex,exclude] (Y) at (4.5,0.5){};
    \draw[dotted, thick, ->] (X) -- (Y);
    \node[label] at (Y) {One-way suspicion};

    \node[vertex] (XX) at (3.5,1){};
    \node[vertex] (YY) at (4.5,1){};
    \draw[dashed, thick, <->] (XX) -- (YY);
    \node[label] at (YY) {Two-way suspicion};

    \node[vertex,exclude] (X3) at (3.5,1.5){};
    \node[vertex,exclude] (Y3) at (4.5,1.5){};
    \draw[thick, new] (X3) -- (Y3);
    \draw[dashed, thick, <->] (X3) -- (Y3);
    \node[label] at (Y3) {Two-way suspicion in $\MG$};

    \node[vertex] (X4) at (3.5,2){};
    \node[vertex] (Y4) at (4.5,2){};
    \draw[tree] (X4) -- (Y4);
    \node[label] at (Y4) {Edge in tree};

  \end{tikzpicture}

  \caption{Example of suspicions leading to removal of nodes as candidates for internal tree positions.}
  \Description{Graph depicting an example of suspicions leading to removal of nodes as candidates for internal tree positions.}
  \label{fig:suspicionsexample}
\end{figure}


\cref{fig:suspicionsexample} illustrates a suspicion graph $\G$ and the resulting tree configuration with excluded nodes.
In this example, $\MG = \{(\susn{1},\susn{4}), (\susn{2},\susn{3})\}$, as adding any other suspicion would violate the disjoint property of \MG, which is already maximal.
The set $\T = \{\naT\}$ includes nodes that form a triangle with one edge in $\MG$ and two others in $\G$.
As $\nbC$ has a one-way suspicion, it is added to $\Crash$.
Since we exclude all replicas (vertices) from $\T$, $\MG$, and $\Crash$ from the configuration $\C{}$ making $\Cand = \{\cn{1},\cn{2},\cn{3},\nr\}$.

