\section{\optitree Correctness}
\label{sec:security_analysis}

In this section, we proof correctness properties for \optitree.
\optitree provides the following guarantees:
\begin{enumerate}[label=C\arabic*]
  \item\label{p:candidates} There are always enough candidates available to form a tree. (\cref{thm:treeform})
  \item\label{p:suspicions} If a tree configuration fails and the system reverts to a star topology, sufficient suspicions will be recorded to update the candidate set and invalidate the failed tree configuration. (\cref{lem:suspicions} \&~\ref{lem:incu})
  \item\label{p:no-susp} After GST, no two correct replicas raise suspicions against each other. (\cref{lem:c2c})
  \item\label{p:limit} After GST, at most $2\f$ reconfigurations are needed to form a correct tree. Specifically, if $\tn < \f$ replicas are faulty, at most $2\tn$ reconfigurations are needed. (\cref{thm:2t})
\end{enumerate}



\noindent
Faulty replicas can cast suspicions on non-faulty replicas to exclude them from being selected as internal nodes.
However, \optitree can continue to form and reconfigure to new trees if there are enough candidate replicas to select the internal nodes.
The process for selecting candidate internal nodes is detailed in Section 5.3.3.
It involves constructing $\MG$ from the suspicion graph \G.
\cref{thm:treeform} shows that enough candidate replicas are always available to form a tree.

\begin{theorem}
  \label{thm:treeform}
  There are always enough candidate replicas available for selecting the internal nodes to form a tree.
\end{theorem}

\begin{proof}
  A tree configuration of size $\n \ge 3\f+1$ with a branch factor $\bn \leq \sqn$ requires at most $\sqn+1$ replicas as internal nodes.
  As explained in Section 4.1.3, the graph $\G$ always contains an independent set~($\IS$) of at least $\n-\f$ replicas.
  Then, we construct $\MG$ and $\T$ from $\G$, where $\MG$ is a maximum set of disjoint edges and $\T$ is a set of nodes that form a triangle in $\G$ with an edge in $\MG$.
  For any edge in $\MG$, at most one of the two vertices is in $\IS$.
  Since $\MG$ is disjoint, each vertex belongs to at most one edge in $\MG$.
  Moreover, for any edge in $\MG$ that forms a triangle in $\G$, only one of the three vertices forming the triangle can be in $\IS$.

  Due to the maximality of $\MG$, any edge in $\MG$ can be part of at most one triangle involving a replica in $\T$.
  Therefore, if at most $\f$ replicas are outside $\IS$, the number of vertices in the independent set adjacent to edges in $\MG$ or in $\T$ is at most $\f$.
  Consequently, the remaining replicas in $\IS$ (at least $\f+1$) are available as candidates for internal nodes.

  For $\n \ge 13$, we have $\sqn < \f$, implying that for configuration sizes of 13 and above, there are always enough candidate replicas available for selecting internal nodes.
\end{proof}

\begin{lemma}
  \label{lem:suspicions}
  When a tree fails, either (1) one suspicion between internal nodes is raised, or (2) $\un+1$ leaves, not in $\Crash$ get suspected.
\end{lemma}
\begin{proof}
  A tree fails, if the replica at the root fails, or cannot collect $\n-\f$ votes in time.
  If the root fails, it is suspected by other internal nodes, matching (1) from the Lemma.
  Otherwise, the root waits for aggregates from internal nodes, which together aggregate more $\n-\f+\un$ votes.
  Thus, there are two possibilities. Either at least $\un+1$ leaves are missing and suspected by their parents, matching (2), or one internal node is either missing or did send a faulty aggregate, and is suspected by the root, matching (1).
\end{proof}

\begin{lemma}
  \label{lem:incu}
  If a tree failed, either: (i) $|\MG|$ increases, or (ii) $|\T|$ increases and $|\MG^n|$ stays unchanged.
\end{lemma}

\begin{proof}
  All suspicions are added to the graph $\G$.
  In case (1) of Lemma~\ref{lem:suspicions}, the new suspicion can be added to $\MG$ and $\T$ remains unchanged.
  In case (2), $\un+1$ edges are added to $\G$.
  These new edges include $\un+1$ different leaves from the failed tree. These leaves, can be already adjacent to an edge in $\MG$ or part of $\T$.
  There are three cases:
a) one leaf was not part of $\MG$ or $\T$. The new edge can be added to $\MG$ giving case (i) of the Lemma.
b) one leaf was part of $\T$. The new edge can be added to $\MG$ and $|\T|$ is reduced by 1, giving case (i) of the Lemma.
(3) all leaves are part of $\MG$, but since $|\MG|\leq \un$, at least two suspected leaves are already connected in $\MG$. This allows to either replace one edge in $\MG$ by two new ones, or add a new node to $\T$.
\end{proof}
