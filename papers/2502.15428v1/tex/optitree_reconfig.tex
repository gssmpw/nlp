\subsubsection{Configuration Monitoring in Trees}
\label{sec:tree-formation}

Configuration monitoring in \optitree is responsible for finding efficient tree configurations and activating the best one.
As outlined in \cref{sec:config_sensor}, the \cfgsensor may start a search periodically or when the current configuration is no longer valid.
A tree configuration is valid if all internal nodes are part of the candidate set \Cand from the \susmonitor.
Thus, when a tree fails, \optitree reverts to a star topology to record suspicions.
Once these suspicions result in a change of \Cand, the \cfgsensor initiates a search for a new tree.

The \cfgsensor searches for a valid, low-latency tree using latencies \lm from the \latmonitor and the candidate set \Cand from the \susmonitor.
Additionally, the \cfgsensor uses the estimated number of misbehaving replicas \un to search for a tree that provides low latency despite \un many unresponsive leaves.

Since searching for optimal trees in large systems is intractable, we use a heuristic optimization technique based on simulated annealing~\cite{simulatedannealing}.
To ensure that only valid trees are considered, internal nodes must be selected from \Cand.
We initiate the search with a random tree, ensuring that internal nodes are only selected from \Cand.
Simulated annealing uses a \textit{mutate} function to randomly swap two nodes in the tree.
Our mutate function ensures that internal nodes are only swapped with other nodes from \Cand.

Simulated annealing then compares the latency of the mutated tree with that of the best tree found so far and probabilistically selects the lower-latency tree for further mutation.
The tree latency is computed using the scoring function in~\cref{def:score} with parameter $\kn=\q+u$.
This score is the latency required to collect $\q$ votes from the tree, assuming that \un leaf nodes are unresponsive.
While simulated annealing allows exploration beyond local minima, it does not guarantee a global minimum.
The search ends when the \textit{search timer} expires or when simulated annealing converges.
At this point, the \cfgsensor logs the lowest-latency tree found.

The \cfgmonitor waits for $\f+1$ valid tree configurations to be committed to the log and ranks them using the scoring function.
It then selects the lowest-latency tree for reconfiguration.
