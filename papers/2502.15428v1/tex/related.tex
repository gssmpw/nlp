\section{Related Work}

Several prior works~\cite{p2p-acc, network-acc} use accountability as a tool to enhance security by supporting forensics to identify misbehaving replicas.
PeerReview~\cite{peerreview} introduced a comprehensive method to establish accountability in distributed systems by auditing messages from other replicas.
However, due to the extensive message exchange among replicas, this approach may be too costly for RSMs.
Polygraph~\cite{polygraph} introduced accountable Byzantine agreement by logging protocol messages to justify the decisions made by replicas.
IA-CCF~\cite{ia-ccf} improves on Polygraph by providing individual accountability through the use of \textit{ledgers} and \textit{receipts}.
It relies on trusted execution environments to secure replicas.
Given a set of conflicting receipts and a ledger, any third party can generate proof-of-misbehavior against at least $\n/3$ replicas.
\sysname, however, does not require logging all protocol messages to ensure accountable configurations.
Instead, it captures and records the underlying metrics involved in system configuration decisions.
Moreover, \sysname's flexible architecture can support sensor-monitor pairs to provide even stronger accountability if needed.

Kauri~\cite{kauri} uses random partitions to create a tree topology for scaling BFT protocols.
While random selection offers a probabilistically working tree, these trees can suffer from poor performance.
Several previous works~\cite{archer, droopy, mencius} have optimized configuration latency by selecting leaders based on location.
Other approaches~\cite{wheat, aware, eval} rank configurations to identify the optimal one.
Wheat~\cite{wheat} assumes a topology-aware configuration, relying on external systems for replica monitoring and weight-vote allocation during view changes, which may not align with the BFT trust model.
Aware~\cite{aware} improves on Wheat by globally logging latency measurements to compute the weight-vote assignment deterministically.
However, these mechanisms provide limited accountability and focus more on optimizing configurations than selecting a working configuration.
\sysname takes a comprehensive approach to select a working configuration within a fixed number of reconfigurations, ensuring accountable decisions while supporting non-deterministic configuration methods.

Abstract~\cite{Abstract} introduced a switching mechanism to enable more efficient protocols under favorable conditions and reverts to more resilient protocols when conditions deteriorate.
Similarly, Dahlia et al.~\cite{Dahlia} propose a system to collect and replicate configuration metrics and use them in a reinforcement learning-based mechanism to switch between protocols.
Adapt~\cite{adapt} presents a quality control system that assesses the suitability of a configuration and protocol based on the current environment.
While protocol switching is a useful fallback mechanism, these systems do not address the specific needs of RSM optimizations~\cite{berger2023sok}, such as committee selection, hierarchical consensus, and pipelining.
\sysname provides a flexible framework to log relevant measurements and use them to enhance such optimizations.
