\subsubsection{Suspicion Monitoring}
\label{sec:suspicion}

Gathering enough evidence to substantiate a proof-of-misbehavior complaint against a replica is often impossible.
Therefore, we introduce suspicion monitoring to detect timing and omission faults.
These suspicions are processed to produce a candidate set that avoids suspected replicas.

The \sussensor is embedded in the \consmod, allowing it to monitor both replies and view timeouts.
It raises suspicions on (a)~slow replicas, (b)~slow leaders, and (c)~replicas issuing false suspicions.
Let \susp{\na}{\nb} denote that $\na$ suspects $\nb$.
The sensor at replica $\na$ logs a suspicion message whenever condition (a), (b), or (c) is met, as outlined below.

\begin{table}[ht]
  \footnotesize
  \centering
  \begin{tabular}{@{}p{4.25cm}p{1.75cm}p{1.78cm}@{}}
    \toprule
    \textbf{Condition for Suspicion}           & \textbf{Timeout}   & \textbf{Suspicion} \\ \midrule
    (a) No response from $\nb$ within timeout  & \timeout{\na}{\nb} & \suspm{Slow}{\na}{\nb} \\
    (b) View with leader $\nl$ timed out       & \viewtimeout{\nl}  & \suspm{Slow}{\na}{\nl} \\
    (c) False suspicion from $\nb$ on $\na$    &                    & \suspm{False}{\na}{\nb} \\
    \bottomrule
  \end{tabular}
\end{table}

\noindent
First, \susp{\na}{\nb} if \nb fails to respond within a timeout of $\timeout{\na}{\nb}$, where $\lme{\na}{\nb}$ is the latency reported by the \latmonitor.
This mechanism ensures that omission failures also lead to suspicion.
Second, \susp{\na}{\nl} if \na does not receive a proposal message within the view timeout.
The view timeout is also adjusted based on latencies from the \latmonitor.
However, since the leader must wait for slower replicas before responding, the view timeout is based on the maximum latency in the leader's row.
Finally, if a \suspm{\_}{\nb}{\na} is raised, $\na$ reciprocates by raising a \suspm{False}{\na}{\nb}.

After \ac{GST}, no two correct processes will raise suspicions against each other.
This holds because the latencies between correct replicas have stabilized, and latency measurements and timeouts have been adjusted accordingly.

To produce a candidate set, the \susmonitor first removes provably faulty replicas \Faulty, reported by the \mismonitor.
Next, the \susmonitor distinguishes between crash suspicions and misbehavior suspicions.
It then filters out certain suspicions raised by misbehaving replicas to prevent a single Byzantine replica from excluding many correct replicas.
The monitor then outputs a candidate set of replicas considered correct, along with an estimate of the number of misbehaving replicas.

Condition (c) allows us to distinguish suspicions against crashed replicas from those caused by misbehaving replicas.
A crashed replica $\nb$ will be suspected by others through $\suspm{Slow}{\na}{\nb}$.
If a faulty replica $\nb$ raises $\suspm{Slow}{\nb}{\na}$ against a correct replica $\na$, then $\na$ will reciprocate with $\suspm{False}{\na}{\nb}$.
Thus, a replica suspected of being slow without raising a counter-suspicion is \textit{considered crashed}.
Conversely, if two replicas suspect each other, it remains unclear which one is faulty.
The \susmonitor maintains separate data structures for these two cases, a set $\Crash$ for crashed replicas, and a graph $\G$ for other suspicions.
Formally, $\G=(\V,\E)$ is an undirected graph where the vertices $\V=\{\Pi\setminus\Faulty\setminus\Crash\}$ represent possible candidate replicas, and an edge $(\na, \nb) \in \E$ indicates a two-way suspicion, $\susptwo{\na}{\nb}$.

When a suspicion $\suspm{Slow}{\na}{\nb}$ is raised against a correct replica $\nb$, it may take several views before a reciprocation $\suspm{False}{\nb}{\na}$ is logged, especially if faulty replicas attempt to censor such suspicions.
We therefore treat every new suspicion between two replicas in $\V$ as a two-way suspicion by adding the corresponding edge to $\G$.
If no reciprocation is received within $\f+1$ leader changes (views), the edge is treated as a one-way suspicion, and $\nb$ is added to $\Crash$.

The \susmonitor computes a maximum independent set from the vertices in $\G$ and uses it as the candidate set $\Cand$.
This computation must be deterministic, to ensure that all correct replicas reach the same conclusion.
Additionally, the monitor outputs the estimated number of misbehaving replicas, $\un=|\V|-|\Cand|$.

Before GST, unstable latencies may cause suspicions, even between correct replicas.
To prevent these suspicions from persisting indefinitely, the \susmonitor employs two mechanisms.
First, if the system remains stable with no new suspicions for \winlen views, the monitor begins removing old suspicions in the order they appear in the log.
Second, if too many suspicions are received, it also starts discarding old suspicions.
Too many suspicions occur when $\G$ no longer contains an independent set of size $\n-\f$.
