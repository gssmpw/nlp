\subsubsection{Misbehavior Monitoring}
\label{sec:mms}

A significant challenge for scalable RSMs is to stay optimized despite faulty replicas disrupting the configuration.
\sysname provides two monitors for detecting faulty replicas: one precise (this section) and one for suspicions (\cref{sec:suspicion}).

Precise detection is possible using the \textit{proof-of-misbehavior} technique~\cite{ia-ccf, BFTFD, byzid,zyzzyva, prime}.
The \missensor, integrated into the \consmod, observes the replicas' protocol-specific behavior and raises a \textit{complaint} when detecting a provable violation.
Complaints are signed and proposed via the \sensapp and logged.
Detected misbehaviors include invalid threshold signatures, proposals, votes, complaints, and equivocation.
An equivocation is identified when replicas detect the leader sending different messages, when the protocol dictates that it should have sent the same message.

When a complaint proposal is received, each replica's \mismonitor verifies its validity.
If valid, the monitor updates its \textit{list of provably faulty replicas} \Faulty, which can be used to exclude them from special protocol roles, such as leader.





