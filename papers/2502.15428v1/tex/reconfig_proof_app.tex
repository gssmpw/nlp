\subsection{Reconfiguration Proof}
\label{sec:proof}

In the following, we show that \optitree ensures that after GST, a working configuration is found after at most $2\tn$ reconfigurations, where $\tn$ is the actual number of faulty replicas in the system, i.e. $\tn \leq \f$.
This is Theorem~\ref{thm:2t}.

We first proof \ref{p:no-susp} as \cref{lem:c2c}.


\begin{lemma}
  \label{lem:c2c}
  After GST, no two correct replicas suspect each other.
\end{lemma}
\begin{proof}
  \cref{tab:tree-suspicion} describes the conditions for raising suspicions in a tree.
  After GST, the round-trip time between two correct replicas $\na$ and $\nb$ is in the interval $[\lat[a]{\na}{\nb}, \delta\cdot\lat[a]{\na}{\nb}]$.
  It follows that no suspicions between $\na$ and $\nb$ will be raised based on conditions \tcond{a}. 
  For \tcond{a'} the same follows, since a correct intermediate node waits at most $\df{(\agt{\nin})}$ for votes from leaf nodes and then send an aggregate to the root.

  For conditions \tcond{b} and \tcond{b'}, a correct root will either:
  (1)~suspect one of the intermediate nodes based on \tcond{a'},
  (2)~receive $\un+1$ suspicions on leaf nodes, or
  (3)~collect and forwards a quorum of votes in time to prevent intermediate nodes from triggering \tcond{b}.
  
  In cases (1) and (2) above, where the root collects or issues suspicions, the intermediate not will not trigger case \tcond{b}.

  Similarly, a correct leaf will receive suspicions from a correct root before triggering \tcond{b'}.
  Finally, a correct intermediate node will either forward proposals to the leaf or broadcast a suspicion against the root before triggering \tcond{b'}.

  If no suspicions are raised due to timeouts, condition \tcond{c} will not be triggered either.
\end{proof}

To prove our main theorem, about the number of reconfigurations needed to find a correct tree,
we consider new subsets $\MG^n\subset\MG$, $\T^n\subset\T$, and $\Crash^n\subset\Crash$ which are added to the respective sets after GST.
Lemma~\ref{lem:found} shows that the size of these data-structures reflects the number of faulty replicas removed from the candidate set.
Lemma~\ref{lem:incu} shows that for any two failed trees, the size of these data-structures increases by at least one.
Lemmas~\ref{lem:nodecu} and~\ref{lem:oldremove} show that other mechanisms do not reduce the size.
We denote the total size of these structures as $\tn^n=|\Crash^n|+|\MG^n|+|\T^n|$.

\begin{lemma}
  \label{lem:found}
  If $\tn^n=\tn$, then a working configuration is found.
\end{lemma}
\begin{proof}
  After GST, no two correct replicas raise suspicions on each other and all suspicions on correct replicas are reciprocated. Therefore, if a replica was added to $\Crash^n$ it is indeed faulty.
  Further, if a new edge was added to $\G$, then at least one of the adjacent replicas is faulty.
  Thus, for every edge in $\MG^n$, at least one of the adjacent nodes is faulty, and for each triangle build from a replica in $\T^n$ and an edge in $\MG^n$, at least two of the adjacent replicas are faulty.
  Finally, for a triangle in $\G$ containing a replica in $\T^n$ and an edge in $\MG$ but not in $\MG^n$, at least one of the adjacent replicas is faulty.
  Thus $\tn^n$ many faulty replicas have been removed from the candidate set for internal nodes.
\end{proof}



\begin{lemma}
  \label{lem:incugst}
  If a tree failed after GST, either: (i) $|\MG^n|$ increases, or (ii) $|\T^n|$ increases and $|\MG^n|$ stays unchanged.
  Also, either $\tn'$ increases, or $\tn'$ stays constant and $\T^n$ decreases.
\end{lemma}
This Lemma is a variation of \cref{lem:incu} and the proof follows the same schema.

\begin{proof}
  All suspicions are added to the graph $\G$.
  In case (1) of \cref{lem:suspicions}, the new suspicion can be added to $\MG^n$ and $\T$ remains unchanged, increasing $\tn'$.
  In case (2), $\un+1$ edges are added to $\G$.
  These new edges include $\un+1$ different leaves from the failed tree. These leaves, can be already adjacent to an edge in $\MG$ or part of $\T$.
  There are three cases:
a) one leaf was not part of $\MG$ or $\T$. The new edge can be added to $\MG^n$ giving case (i) of the Lemma.
b) one leaf was part of $\T$. The new edge can be added to $\MG^n$ and $|\T|$, possibly also $|\T^n|$ is reduced by 1, giving case (ii) of the Lemma. In this case, $\tn'$ stays constant.
(3) all leaves are part of $\MG$, but since $|\MG|\leq \un$, at least two suspected leaves are already connected in $\MG$. This allows to either replace one edge in $\MG$ by two in $\MG^n$, or add a new node to $\T^n$. Also here $\tn'$ increases.
\end{proof}

If a new suspicion is raised between two nodes in $G$, it is first treated as a two way suspicion and added to $\G$. If no reciprocation happens during the next $\f+1$ views, then the suspected replica is added to $\Crash$.
The next Lemma proves that these changes do not decrease $\tn'$.

\begin{lemma}
\label{lem:nodecu}
If nodes are added to $\Crash$ due to not reciprocated suspicions, this does not change $\tn'$, and does not increase $\T^n$.
\end{lemma}

\begin{proof}
Clearly, if an edge is removed from $\MG^n$, then a replica is added to $\Crash^n$.
In some cases the removal removing an edge from $\MG$ causes that a replica is removed from $\T^n$ and instead, a new edge is added to $\MG^n$.
Thus $\tn'$ does not change.
\end{proof}

All suspicions added after GST include at least one faulty replica. Thus these suspicions leave an independent set of $\n-\f$ correct replicas.
However, suspicions made before GST may be successively removed after GST.

\begin{lemma}
\label{lem:oldremove}
If suspicions made before GST are discarded, removing replicas from $\Crash$, or removing edges from $\G$, $\tn'$ does not decrease, and $\T^n$ does not increase.
\end{lemma}
\begin{proof}
Clearly this does not reduce $\Crash^n$ and $\MG^n$.
However, if an edge in $\MG$ is removed, it can be that a replica in $\T^n$ is no longer part of a triangle.
However, the replica in $\T^n$ is adjacent to an edge added after GST. This edge can now be added to $\MG^n$, leaving $\tn'$ unchanged.
\end{proof}

\begin{theorem}
\label{thm:2t}
After GST, \optitree finds a working configuration after at most $2\tn$ reconfigurations.
\end{theorem}
\begin{proof}
Based on Lemma~\ref{lem:found}, we only have to show that after $2\tn$ reconfigurations, $\tn'=\tn$.
According to Lemma~\ref{lem:incu}, after a reconfiguration, either $\tn'$ increases, or $\tn'$ stays constant and $\T^n$ decreases.
Since $\T^n$ can only decrease by $k$ after $\T^n$ and $\tn'$
have been increased by $k$. Thus at most $\tn'$ reconfigurations can leave $\tn'$ unchanged.
\end{proof}
