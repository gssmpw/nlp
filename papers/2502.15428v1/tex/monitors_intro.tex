\subsection{\Sensors and \Monapps}
\label{sec:sm}

This section presents \sysname's mechanisms for collecting and processing metrics, namely \sensors and \monitors.
Each sensor-monitor pair has a one-to-one relationship, as shown with \sens{1}--\mons{1} and \sens{2}--\mons{2} in \cref{fig:sensors-and-monitors}.

A \textit{\sensor} abstracts the capture of local-only metrics related to a replica's operational context.
\Sensors can be integrated into system components, such as the \consmod, or can query the operating system for metrics such as CPU load.
A sensor can also use information from local monitors, as exemplified by the dashed arrow from \mons{1} to \sens{1} in \cref{fig:sensors-and-monitors}.
A \sensor's output is assumed to be non-deterministic and may vary across replicas.
\Sensors may also perform non-deterministic computations, e.g., to support heuristic and distributed optimization algorithms.
We refer to both sensor output and compute results as \textit{measurements}.
These measurements are recorded in the log, providing a consistent record of the system's operation.
The recorded measurements are later accessed by \monitors, as discussed below.

In a Byzantine environment, \sensors at up to \f faulty replicas may report incorrect measurements.
Consequently, any \monitor using these measurements must account for potential inaccuracies.
Despite this, retaining a record of incorrect measurements can be invaluable for forensic analysis.

The \textit{\monitor} is the core abstraction linking sensor measurements from individual replicas with system-wide operations, enabling dynamic adjustments to the RSM.
When a sensor records a measurement, the corresponding \monitor updates its data structures through the following steps:

\begin{enumerate}
  \item \textbf{Data Collection:}
    The monitor collects metrics from its associated sensor via the log and may also use data from other local monitors.
  \item \textbf{Deterministic Computation:}
    The monitor processes collected metrics into consistent data structures.
\end{enumerate}

\noindent
Since monitors on each replica operate on the same ordered set of measurements, they maintain consistent data structures across the system.
This consistency provides a consolidated, global view of the measurements, allowing replicas to coordinate system-wide operations reliably, such as activating a new tree configuration with \optitree~(\cref{sec:optitree}).
\cref{tab:sensors_monitors} summarizes the key properties of sensors and monitors.

\begin{table}[htb]
  \footnotesize
  \centering
  \caption{Summary of sensor and monitor properties.}
  \label{tab:sensors_monitors}
  \begin{tabular}{@{}p{1.8cm}|p{2.9cm}p{2.9cm}@{}}
    \toprule
    \textbf{Property}        & \textbf{Sensors}          & \textbf{Monitors}          \\ \midrule
    \textbf{Input}           & System \& local monitors  & Log \& local monitors      \\
    \textbf{Computation}     & Non-deterministic         & Deterministic              \\
    \textbf{Output}          & Variable across replicas  & Consistent across replicas \\
    \bottomrule
  \end{tabular}
\end{table}

\noindent
In the following sections we present the sensors and monitors we implemented for \sysname, and \cref{fig:star-sens-mons} shows how these interact to ensure robust configurations.

\begin{figure*}[hbt]
  \centering
  \def\smsize{0.6cm}
\def\smpos{1}
\def\rsize{1cm}
\def\dist{1.0}
\def\logpos{0}
\def\logh{0.5}
\def\logw{12.7}

\definecolor{darkcola}{RGB}{140, 140, 180} %
\definecolor{cola}{RGB}{230, 230, 250} %
\definecolor{colb}{RGB}{100, 200, 100} %
\definecolor{colc}{RGB}{200, 100, 100} %
\definecolor{cold}{RGB}{100, 180, 180} %

\tikzset{
  arrow-record-a/.style={->, thick, color=#1},
  arrow-record-b/.style={->, thick, >={latex'}, color=#1},
  arrow-notify-a/.style={<-, thick, color=#1},
  arrow-notify-b/.style={<-, thick, >={latex'}, color=#1},
  arrow-local/.style={->, thick, dashed},
  arrow-sh/.style={xshift=#1*\arrowspacing},
}

\tikzset{
  label/.style={midway, yshift=-2pt, font=\scriptsize, fill=white, fill opacity=1, text opacity=1, inner sep=1.5pt},
  legend/.style={left, font=\scriptsize},
  box/.style={draw, rectangle, minimum width=\smsize, minimum height=\smsize},
  replica/.style={draw, rectangle, minimum width=\rsize, minimum height=\smsize},
  sensor-box/.style={box, text=white, preaction={fill=#1}},
  monitor-box/.style={box, text=white, preaction={fill=#1}},
  replica-box/.style={replica, preaction={fill=#1}},
  legend-box/.style={font=\scriptsize, draw, rectangle, minimum width=4cm, minimum height=0.35cm, text=white, preaction={fill=#1}, anchor=west, inner sep=0pt, align=right, text height=1.5ex, text depth=.25ex},
}

\begin{tikzpicture}

  \draw[thick, fill=cola] (0.5, 2.1) rectangle (\logw, \logpos+0.5) node[anchor=north east] at (\logw, 2.1) {Replica \na};

  \node[sensor-box=light-yellow]  (LS) at (1*\dist, \smpos)   {\textsf{LS}};
  \node[monitor-box=light-yellow] (LM) at (2*\dist, \smpos)   {\textsf{LM}};
  \node[sensor-box=light-blue]    (MS) at (3.5*\dist, \smpos) {\textsf{MS}};
  \node[monitor-box=light-blue]   (MM) at (4.5*\dist, \smpos) {\textsf{MM}};
  \node[sensor-box=light-red]     (SS) at (6*\dist, \smpos)   {\textsf{SS}};
  \node[monitor-box=light-red]    (SM) at (7*\dist, \smpos)   {\textsf{SM}};
  \node[sensor-box=light-green]   (CS) at (8.5*\dist, \smpos) {\textsf{CS}};
  \node[monitor-box=light-green]  (CM) at (9.5*\dist, \smpos) {\textsf{CM}};
  \node[replica-box=light-blue]  (RSM) at (12*\dist, \smpos) {RSM};

  \draw[arrow-local] (LM.north) to[out=60, in=120, looseness=0.5] (SS.north);
  \draw[arrow-local] (LM.north) node[font=\scriptsize, above] {\lm} to[out=60, in=120, looseness=0.4] (CS.north);
  \draw[arrow-local] (MM.north) node[font=\scriptsize, above] {\Faulty} to[out=60, in=120, looseness=0.5] (SM.north);
  \draw[arrow-local] (SM.east) -- (CS.west) node[midway, yshift=-2pt, font=\scriptsize, above] {\Cand, \un};
  \draw[arrow-local] (CM.east) -- (RSM.west) node[midway, yshift=-3pt, font=\scriptsize, above] {Reconfigure};

  \draw[arrow-record-a=darkcola] (LS.south) -- (1*\dist, \logpos) node[label] {\latm{Prop}{A}{B}};
  \draw[arrow-record-a=darkcola] (MS.south) -- (3.5*\dist, \logpos) node[label] {\compm};
  \draw[arrow-record-a=darkcola] (SS.south) -- (6*\dist, \logpos) node[label] {\suspm{Slow}{A}{B}};
  \draw[arrow-record-a=darkcola] (CS.south) -- (8.5*\dist, \logpos) node[label] {\cfgm};

  \begin{scope}[shift={(\logw+0.5, \logpos+0.5)}]
    \node[legend-box=light-yellow] at (0, 1.5) {\textsf{LS/LM: Latency Sensor/Monitor}};
    \node[legend-box=light-blue]   at (0, 1.0) {\textsf{MS/MM: Misbehavior Sensor/Monitor}};
    \node[legend-box=light-red]    at (0, 0.5) {\textsf{SS/SM: Suspicion Sensor/Monitor}};
    \node[legend-box=light-green]  at (0, 0.0) {\textsf{CS/CM: Config Sensor/Monitor}};
  \end{scope}
\end{tikzpicture}

  \caption{Connections between Replica \na's sensors and monitors.}
  \Description{Diagram illustrating the connections between sensors and monitors for a star topology.}
  \label{fig:star-sens-mons}
\end{figure*}
