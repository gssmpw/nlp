\section{Materials and Methods}
\label{sec:methods}
\subsection{Soft Arm Design and Manufacturing}
\begin{figure}
    \centering
    \includegraphics[width=0.9\columnwidth]{Manufacturing_compressed.pdf}
    \caption{Materials and geometry of the variable stiffness soft arm. Soft regions are printed with 40A Shore hardness, while rigid ones reach 83-86D Shore hardness. A clearance $g$ of 0.2\,mm allows for the printing of the beads without them fusing together, while openings on the side walls between the bellow actuators enable easier removal of the support material during post-processing of the print. 
    }
    \label{fig:man}
\end{figure}
As displayed in \cref{fig:man}, the soft bellows are printed on a Stratasys J735 \textsuperscript{\textregistered} (Stratasys Ltd, USA) multi-material 3D printer using a digital material blend of VeroCyan\textsuperscript{\textregistered} (Stratasys Ltd, USA) and a rubber-like soft material Agilus30\textsuperscript{\textregistered} (Stratasys Ltd, USA) to achieve a Shore hardness of 40A, while the end plates and the beads are fabricated using a rigid plastic-like material VeroCyan\textsuperscript{\textregistered} (Stratasys Ltd, USA) with a Shore hardness of 83-86D. Soluble support material SUP706\textsuperscript{\textregistered} (Stratasys Ltd, USA) was also used during the print for the voids.

The base geometry for the bellow actuators is the result of an optimization process our group developed in \cite{bellow_actuator}. Three bellows are joined together at 120$^\circ$ to form the soft arm and a further optimization process employing \gls{fea} was conducted to get to a geometry of the soft arm that can achieve 90$^\circ$ omnidirectional bending at 20\,kPa \cite{bellowdesign2}. The resulting geometrical parameters are as follows: inner radius $r$=4\,mm, wall thickness $t$=1.5\,mm, average radius $R$=7\,mm, module length $L$=7.2\,mm, and module number $12$ for each bellow pneumatic chamber. These constraints allowed for a central column with a diameter of 10\,mm to be cut out, to house the beads for the bead jamming mechanism.

The design of the bowl-shaped beads results from the integration of a pre-existing design ~\cite{bead_design} with our monolithic manufacturing goal. With respect to \cref{fig:man}, large bead diameter $D$ maximizes the holding torque of the beaded-string for a given jamming tension \cite{bead_design}. A minimum clearance $g$ between each bead and between the beads and the central column wall must be granted for the inclusion of support material, to avoid the various elements of the design fusing together while printing. Whilst this clearance is mandatory, minimizing it is crucial for effective jamming performance, as this minimizes tendon travel to engage the beads. Clearance test prints were performed to empirically evaluate the 0.2\,mm minimum reliable clearance $g$ achievable with the printer used. Given these considerations, the beads were designed with an external diameter $D$ of 9.2\,mm. Each bead features a central hole with diameter $d$ of 1.3\,mm, for the 1\,mm thick nylon jamming tendon, while the angle $\alpha$ of of $30^{\circ}$ coupled with the spherical joint resulting from the bead design ensures unrestricted range of motion of the soft arm. The dimensions allowed for 17 beads to be included in the soft arm, with the top and bottom ones fused into the soft arm's end plates.

To enable the effective removal of the support material, openings were designed in correspondence of each bead pair as show in the detailed view of \cref{fig:man}. Openings at the inlet and outlet of the soft bellows are also present for the same reason. During post processing, a steel wire was run through the central tendon channel, to remove the support material. A pressurized water jet was then flushed through the tendon channel and this forced most of the support material in and in between the beads to be expelled through the side openings. The soft arm was finally put in a solution of $0.02\,$kg/L Sodium Hydroxide and $0.01\,$kg/L Sodium Metasilicate at 30\,$^\circ\mathrm{C}$ for 3 days, with additional manual cleaning under warm water every day, for the removal of the remaining support material. Once clean, bellows were sealed at the bottom with high-viscosity superglue (UN3334 Everbuild Building Products, UK), while tubing was applied at the inlets to enable their pressurization. The nylon jamming tendon was routed last, knotting its end to allow it to compress the beads through the bottom plate.


\subsection{Finite Element Analyis of the Beads}
\begin{figure}
    \centering
    \includegraphics[width=0.9\columnwidth]{FEM.pdf}
    \caption{The FEM results for determining the maximum tension force: (left) the maximum volumetric von Mises stress as a function of the applied force; (right) the volumetric maximum von Mises stress distribution of an intermediate bead, shown from a cross-sectional view through the center for T=25\,N which is used as the upper limit for the tension applied to the real system. 
    }
    \label{fig:FEM}
\end{figure}
To determine the maximum allowable tension force for the jamming tendon, a \gls{fea} was employed using COMSOL Multiphysics\textsuperscript{\textregistered} (COMSOL Inc., Sweden). As mentioned, the beads are fabricated using the rigid material VeroCyan, with a Young's modulus ranging from 2000 to 3000\,MPa, tensile strength between 50 and 65\,MPa, and density between 1.17 and 1.18\,g/cm³. Therefore, in the \gls{fea}, a linear elastic material model was used, with a representative Young’s modulus of 2500\,MPa and a density of 1.175\,g/cm³, alongside a Poisson’s ratio of 0.38~\cite{vero_poisson_ratio}. During the simulation, the beads were arranged in a vertical stack, replicating their configuration in the soft arm at rest. The uppermost bead was fixed in place, and an upward body force was incrementally applied to the bottom-most bead, ranging from 0\,N to 25\,N, with a step size of 5\,N. The contact interfaces between adjacent beads were modelled using identity pairs to replicate the no-slip condition between the beads when jammed. The maximum jamming force was limited to 25\,N to ensure a safety factor of 12 given the bead material tensile strength of 50\,MPa. This was done to guarantee the system's long-term operational integrity and account for mechanical fatigue and uncertainties in material properties.
%
\Cref{fig:FEM} shows the maximum volumetric Von Mises stress as a function of the applied jamming force. 









