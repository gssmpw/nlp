

\section{Experiments and Results}
\label{sec:exp_descr}

The experiments aim at 
characterizing the soft arm \gls{rom}, in both jammed and unjammed states, and evaluating the resulting sag %\textcolor{blue}{("sag" sounds a bit casual)} 
under load at different actuation pressures and jamming conditions. 
% to assess two key performance metrics of the soft arm: its \gls{rom} and stiffness change. These metrics are critical in evaluating a soft arm's capacity for large deformation and force application.

\subsection{Experimental Setup}
\label{sec:setup}

As illustrated in~\cref{fig:setup}, the experimental setup utilized a Motive OptiTrack\textsuperscript{\texttrademark} system (NaturalPoint, Inc., USA) with four Flex 3 cameras to track the positions and orientations of the soft arm, mounted on a $780\ \mathrm{mm} \times 700\ \mathrm{mm} \times \ 563\ \mathrm{mm}$ aluminum frame. 
Three reflective markers were placed for traking purpose on both end plates of the soft arm.
The geometric centre of the three makers defines the position of the base and tip of the arm in Cartesian space.
The rotation angle $\theta$, as indicated in \cref{fig:setup}, was computed as the angle between the normal vectors of these two rigid bodies. 

\begin{figure}
    \centering
    \includegraphics[width=0.9\columnwidth]{ExpSetup.pdf}
    \caption{The setup includes a tracking system with four cameras mounted on a $780\ \mathrm{mm} \times 700\ \mathrm{mm} \times \ 563\ \mathrm{mm}$ aluminum frame, reflective markers on the soft arm’s top and bottom plates for angle measurement, a servo for tendon tensioning.}
    \label{fig:setup}
\end{figure}

A DYNAMIXEL MX-106R Servo Motor \textsuperscript{\textregistered} (Robotis, UK) was mounted above the soft arm to actuate the jamming tendon through a pulley system. An Anest Iwata\textsuperscript{\textregistered} air compressor (Anest Iwata Corporation, Japan) supplied compressed air to a VPPE-3-1-1/8-2-010-E1 Festo\textsuperscript{\textregistered} proportional-pressure regulator (Festo AG \& Co. KG, Germany), used to control the actuation pressure.

\subsection{Range of Motion Characterization}
\label{sec:charact}

The purpose of this characterization is to evaluate the \gls{rom} of the soft arm under unjammed and jammed conditions.
\begin{figure}[]
    \centering
    \includegraphics[width=0.9\columnwidth]{ThreeDirection_compressed.pdf}
    \caption{Range of motion and uniform deformation test: rotation angle results for each direction, with two pneumatic chambers actuated simultaneously at 20 kPa.}
    \label{fig:threedir}
    \vspace{-2em}
\end{figure}

Before comparing these two conditions, a preliminary test was conducted to assess whether the arm deforms uniformly in different directions. In this test, the arm was actuated in the three principal directions of motion by simultaneously actuating the pneumatic chambers in pairs at 20\,kPa, maximum pressure the bellow actuators could withstand without permanent damage due to excessive material stretch. 

Data was collected at a steady state during the final $2$ seconds after allowing the system to settle for at least $15$ seconds following pressure application. For each pair of actuated chambers, the test was repeated three times. The averaged rotation angles and the corresponding standard deviations are displayed in \cref{fig:threedir}. The results show that the rotation angles in the three directions are $86.13^{\circ}$, $85.78^{\circ}$, and $86.32^{\circ}$, with standard deviations of $0.96^{\circ}$, $0.07^{\circ}$, and $2.03^{\circ}$, indicating good repeatability. A one-way ANOVA test performed on the experimental results showed no statistically significant difference between the three directions.

The characterization of the effect of jamming on the \gls{rom} then followed.
To begin with, three different actuation sequences were considered:  
\begin{itemize}
    \item \textbf{Sequence 1}: The arm was first actuated to 20 kPa, followed by the application of the full tension force of 25 N. 
    \item \textbf{Sequence 2}: A pre-tension of approximately 0.48 N was applied to engage the beads before actuating the arm at 20 kPa, followed by the full tension force. 
    The pre-tension force represents the minimum empirically determined force, resulting from the servo motor’s resolution (0.4 N) and the friction present in the system.
    \item \textbf{Sequence 3}: The arm was first actuated to 20 kPa, followed by the application of pre-tension, and finally the full tension force.
\end{itemize}
These three sequences were selected to assess the effect of the 0.2\,mm inter-bead gap and the jamming tension transient on the stiffening behavior of the arm. Sequence 2 in particular was developed to evaluate the \gls{rom} of the soft arm when the gap was removed through the application pre-tension force.

During each sequence, the bending angle after pressurization and the one after full jamming were recorded through the OptiTrack system. The results of the experiment are shown in \cref{fig:RoM}a for each sequence, in terms of the rotation angle $\theta$. This was computed, as shown in \cref{fig:setup}, as the angle between the vectors $\overrightarrow{n_1}$ and $\overrightarrow{n_2}$. $\overrightarrow{n_1}$ is orthogonal to the plane defined by the markers of the base plate, while $\overrightarrow{n_2}$ is orthogonal to the plane defined by the markers at the tip of the arm. During the test, we noticed that the application of the jamming tension caused the rotation angle of the soft arm to change and $\overrightarrow{n_2}$ to drift out-of-plane. In the presentation and analysis of the results, we decided to neglect the out-of-plane rotation of the arm as this was an order of magnitude smaller than the in-plane one (e.g. $3.66^{\circ}$ compared to $32.29^{\circ}$ for sequence 3).

For each experimental trial, we allowed at least 15 seconds after each applied action for the system to reach a steady state. Data was then averaged over the final 2 seconds of this period to generate the corresponding data points. We then computed the mean rotation angle of the arm before and after jamming for each sequence across all trials, along with their standard deviations for comparison. 
The initial rotation angles before jamming for Sequences 1 and 3 were $86.53^\circ$ and $84.39^\circ$, respectively. The small difference between them is due to the accuracy of the pressure regulator, which has a minumum controllable pressure of 1kPa~\cite{festo_vppe_datasheet}. Sequence 2 exhibited the smallest initial rotation angle before jamming at $72.48^\circ$.
This result is to be expected, as Sequence 2 starts from a lightly jammed configuration, where the increased internal friction between beads reduces the bending angle $\theta$ compared to the other two initial unjammed states. Despite this difference, the final rotation angle after the soft arm is fully jammed are on average very similar ($50.60^{\circ}$, $51.42^{\circ}$, $52.69^{\circ}$ for sequences 1,2 and 3 respectively), with no statistically significant difference between the three (one-way ANOVA test with p-value=0.224).


\begin{figure}[]
    \centering
    \includegraphics[width=0.9\columnwidth]{sequences.pdf}
    \caption{Range of motion test. (a) Preliminary results for three actuation-jamming sequences: The upper left plot shows the unjammed and jammed results for each sequence, while the upper right provides a schematic highlighting the actuation and tension force conditions for each sequence. The bottom row presents real photos of the soft arm demonstrating the status for each sequence. (b) Further range-of-motion characterization showing the unjammed and jammed rotation angles for Sequences 2 and 3 as pressure changes.}
    \label{fig:RoM}
    \vspace{-2em}
\end{figure}

From these observations, we can conclude that, when unjammed, the arm benefited from a large \gls{rom}, while, a noticeable reduction of the bending angle occurred when jamming was enabled. 

Due to the similar resulting behavior between Sequences 1 and 3, only Sequences 2 and 3 were utilized in the remainder of \gls{rom} characterization as they respectively led to the smallest and the largest average variation in the angle $\theta$ after jamming. \Cref{fig:RoM}b reports the value of $\theta$ at increasing levels of actuation pressures, for both Sequence 2 and 3.
The actuation pressure ranged from $0$ to 20\,kPa with 5\,kPa intervals, and with the full tension force of 25\,N applied. Three trials were conducted for each pressure level, and the average rotation angle and its standard deviation across trials were computed. Overall, the results starting from 5\,kPa indicate that Sequence 2 experienced a smaller reduction in rotation angle after jamming, while Sequence 3 maintained a larger jammed rotation angle across the pressure range, consistent with the preliminary findings. For the unjammed results, the pressure-only condition (yellow dashed line, Sequence 3) shows a steady increase in rotation angle as pressure increases. The pre-tension condition followed by actuation pressure (red dashed line, Sequence 2) demonstrates a similar trend but with slightly lower angles starting from 5 kPa. Finally, the jammed results (solid lines) for both sequences reveal a more substantial decrease in rotation angle compared to the unjammed results of Sequence 2, suggesting a stronger effect of full-tension over pre-tension. From the analysis, Sequence 3 represents a better choice with respect to the other two as it allows for a slightly larger \gls{rom}.

It is worth noting that when no pressure and no tension were applied (unjammed results of Sequence 3), the rotation angle was $1.21^{\circ}$. This angle increased to $6.31^{\circ}$ when pre-tension was applied (unjammed results of Sequence 2), and further increased to $8.33^{\circ}$ with full tension in both sequences (jammed results of both sequences). This increase results from the inherent asymmetry of the three bellow chambers, as well as gaps and slight misalignments between beads, leading to an unavoidable change in orientation when the tendon was pulled upwards to fully engage the beads. 

\subsection{Variable Stiffness Validation}

\begin{figure}[t]
    \centering
    \includegraphics[width=0.95\columnwidth]{load.pdf}
    \caption{Stiffness characterization results: the change in rotation angle with incremental loads under different actuation and jamming conditions. The dashed line in the chart defines two regions based on load range to indicate when jamming (to the right of the line) or no jamming (to the left of the line) is preferable concerning the soft arm’s sag.}
    \label{fig:stiffness}
    \vspace{-1.2em}
\end{figure}

The \gls{vs} capability of the arm was evaluated by measuring the sag occurring when a load was applied to the robot's tip. Experiments were performed with weights ranging from $0$ to 300\,g (equal to 3.3 times the weight of the soft arm at 90.1\,g), in increments of 20\,g. 

Furthermore, the sag was evaluated under actuations and jamming conditions corresponding to Sequence 3:
\begin{itemize}
    \item Pressure-only actuation.
    \item Pressurization, followed by 0.48\,N jamming tension.
    \item Pressurization, followed by 0.48\,N and finally 12.5\,N jamming tension.
    \item Pressurization, followed by 0.48\,N and finally 25\,N jamming tension.
    \item Pressurization, followed by 0.48\,N and 25\,N jamming tension and finally depressurization.
\end{itemize}
Similarly to the \gls{rom} characterization tests, after applying the jamming/actuation sequence, we let the system reach the steady state before applying the loads to the tip of the soft arm. The measured position of the tip was recorded after settling and averaged considering a time window of 2\,s.

\Cref{fig:stiffness} illustrates the change in bending angle as a function of the external load ~\cite{force-angle1,force-angle2} for different jamming conditions.
As expected, when no jamming was applied, the system reached the highest initial bending angle of 86.99$^{\circ}$, and the angle decreased dramatically even for the lower weights. A similar trend was observed when a low jamming tension of 0.48\,N was applied to the system after the initial actuation: the initial angle of 77.77$^{\circ}$ decreased rapidly and finally reached 24.70$^{\circ}$. The remaining three conditions demonstrated comparable performance, with rotation angles starting at 52.6$^{\circ}$, 52.24$^{\circ}$, and 53.46$^{\circ}$, and ending at 39.85$^{\circ}$, 46.04$^{\circ}$, and 45.4$^{\circ}$, with the lower angle achieved with the smaller jamming tension of 12.5\,N. 

As shown in \cref{fig:stiffness}, two distinct regions can be identified, separated by the vertical dashed line. When the external load is below 70\,g (region to the left of the line) one would get a larger sag by jamming the arm compared to supporting the load without jamming. Unjammed operation should therefore be preferred. Jamming becomes advantageous for loads higher than 70\,g, where it successfully manages to drastically reduce the sagging of the robot arm tip (region to the right of the line). 

\subsection{Demonstration in a Switch-Toggling Task}
\begin{figure}[t]
    \centering
    \includegraphics[width=0.95\columnwidth]{ValidationExperiment_compressed.pdf}
    \caption{Experimental validation in a switch-toggling task. The top part shows the arm's behavior under unjammed (left) and jammed (right) conditions. The bottom plot replicates Figure 6, with the switch actuation load (110\,g) indicated. The plot demonstrates that the arm in the unjammed state cannot apply sufficient force to toggle the switch, while the jammed state increases stiffness and ensures successful task completion.}
    \label{fig:task}
\end{figure}
The effectiveness of the proposed bead jamming mechanism in increasing the force application capabilities of the soft arm was validated in a task where the robot was controlled to toggle a lever switch.

A SG-90 servo motor, capable of providing 2.5\,kg$\cdot$cm of torque, served as the end effector and was mounted on the tip of the soft arm, with a 4\,cm servo horn connected to its output shaft. A lever switch with an 80\,g actuation load was wired to an off-the-shelf light box. The arm was pressurized at 20\,kPa and the servo motor was then actuated to toggle the switch. Two trials were performed, one with the arm in an unjammed state and the second with jamming activated. The results are displayed in \cref{fig:task}. When the soft arm was driven solely by an actuation pressure of 20\,kPa, the task failed as the arm was pushed up by the lowering servo horn. In the second scenario, instead, where a jamming tension of 25\,N was applied following the initial pressurization,  the switch was successfully toggled, activating the light box. A detailed demonstration is provided in the supplementary video.




