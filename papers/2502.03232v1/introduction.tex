\section{INTRODUCTION}
\begin{figure}[!t]
    \centering
    \includegraphics[width=0.95\columnwidth]{BeadJammingActuator_compressed.pdf}
    \caption{The variable stiffness soft arm. (a) The design of the soft arm enables monolithic fabrication and variable stiffness through beaded string jamming. (b) Performance of the soft arm while inflated, but without jamming enabled. (c) By engaging the integrated bead jamming mechanism, the load bearing capability of the arm drastically improves.}
    \label{fig:figure1}
\end{figure}

Soft robots offer a key advantage over traditional rigid robots - their inherent compliance allows safe interaction with unstructured environments and enables adaptation to complex tasks~\cite{softrobot1, softrobot2}. 
Pneumatic actuators play a central role in this domain due to their lightweight construction, high power-to-weight ratio, and ease of manufacturing and actuation ~\cite{xavier2022soft}. 
Recent advancements in 3D printing have enabled the automation of the fabrication of complex actuators and integrated systems, significantly reducing production time and cost while improving precision and repeatability~\cite{3dprinting, wang2022modular, zhai2023desktop, wehner2016integrated}.
However, the selection of commercial soft materials remains limited~\cite{3dprintingmaterial}, with elongation at break and actuation pressures significantly lower than traditional silicones, restricting load-bearing capacities and overall stiffness.
%

To address these limitations, designs such as pleated or bellow actuators have been developed, using macroscopic geometric deformations rather than material strain to achieve desired performance~\cite{pleated/bellow}. While effective at enabling high deformations at low pressures, these actuators exhibit low stiffness, limiting their ability to exert significant forces or support heavier loads. This constraint restricts their application in tasks requiring both strength and flexibility.

A promising approach to overcoming this challenge are variable stiffness mechanisms~\cite{stiffness_review}. Existing methods, such as tendon-based or pneumatic antagonistic actuation, often require complex control and scale poorly for high degrees-of-freedom systems~\cite{tendon_stiffening, pneumatic_stiffening}.
%
Hybrid soft-rigid designs and jamming-based techniques (e.g., granular or fiber jamming) have also been explored~\cite{soft-rigid,selective_jamming, Strings_of_Beads, Jammkle, Jeg}. 
%
While these solutions enable stiffness modulation, they often involve complex assembly steps, which prevent monolithic fabrication, require airtight chambers for jamming media, and can lead to uneven stiffness distribution due to media settling under external forces~\cite{VSspine}.

In this paper, we propose a novel variable stiffness soft pneumatic arm design based on bead jamming. The proposed design, shown in \cref{fig:figure1}a, is monolithically 3D-printed, integrating a central channel for a tendon that applies jamming tension. This design minimizes manufacturing steps while achieving effective stiffness modulation. The spherical beads in the central chain allow omnidirectional motion and maintain \gls{rom} comparable to other soft arms. Experimental results demonstrate significant stiffness enhancement under load, as indicated in \cref{fig:figure1}b and c, by reduced sag when jamming is applied. The remainder of the paper is structured as follows. Section \ref{sec:methods} details the design and manufacturing process. Section \ref{sec:setup} outlines the experimental setup, and Section \ref{sec:exp_descr} presents validation experiments and results. Conclusion follows.



