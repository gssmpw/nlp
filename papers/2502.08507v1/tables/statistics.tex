\begin{table*}
\centering
\small
% \resizebox{\columnwidth}{!}{
\begin{tabular}{l|lll|ll}
\hline
 & \multicolumn{3}{c|}{\textbf{Database}} & \multicolumn{2}{c}{\textbf{Test Dataset}} \\ \hline
\textbf{Language} & \textbf{Origin} & \textbf{\#Erroneous} & \textbf{\#Correct} & \textbf{Origin} & \textbf{\#Total} \\ \hline
\textbf{English} & W\&I+LOCNESS & 20185 & 6839 & CoNLL-14 & 1312 \\
\textbf{Chinese} & HSK & 25000* & 25000* & NLPCC-18 & 2000 \\
\textbf{German} & Falko-Merlin & 11801 & 1916 & Falko-Merlin & 2337 \\
\textbf{Russian} & RULEC & 961 & 913 & RULEC & 5000 \\
\textbf{Estonian} & Tartu-L2-Corpus & 7156 & 2** & Tartu-L1-Corpus & 1453 \\ \hline
\end{tabular}
% }
\caption{
The GEC data quantity used. For the database, \#Erroneous represents the number of erroneous samples, and \#Correct represents the number of correct samples. For the test data, \#Total indicates the total number of samples. *For the HSK dataset, due to its large size, we randomly selected 25,000 erroneous and 25,000 correct samples for database construction. **The Tartu-L2-Corpus contains only 4 correct samples, and after filtering, 2 samples remained.
}
\label{tab:statistics}
\end{table*}