%% 
%% Copyright 2007-2024 Elsevier Ltd
%% 
%% This file is part of the 'Elsarticle Bundle'.
%% ---------------------------------------------
%% 
%% It may be distributed under the conditions of the LaTeX Project Public
%% License, either version 1.3 of this license or (at your option) any
%% later version.  The latest version of this license is in
%%    http://www.latex-project.org/lppl.txt
%% and version 1.3 or later is part of all distributions of LaTeX
%% version 1999/12/01 or later.
%% 
%% The list of all files belonging to the 'Elsarticle Bundle' is
%% given in the file `manifest.txt'.
%% 
%% Template article for Elsevier's document class `elsarticle'
%% with numbered style bibliographic references
%% SP 2008/03/01
%% $Id: elsarticle-template-num.tex 249 2024-04-06 10:51:24Z rishi $
%%
\documentclass[preprint,12pt]{elsarticle}

%% Use the option review to obtain double line spacing
%% \documentclass[authoryear,preprint,review,12pt]{elsarticle}

%% Use the options 1p,twocolumn; 3p; 3p,twocolumn; 5p; or 5p,twocolumn
%% for a journal layout:
%% \documentclass[final,1p,times]{elsarticle}
%% \documentclass[final,1p,times,twocolumn]{elsarticle}
%% \documentclass[final,3p,times]{elsarticle}
%% \documentclass[final,3p,times,twocolumn]{elsarticle}
%% \documentclass[final,5p,times]{elsarticle}
%% \documentclass[final,5p,times,twocolumn]{elsarticle}

%% For including figures, graphicx.sty has been loaded in
%% elsarticle.cls. If you prefer to use the old commands
%% please give \usepackage{epsfig}

%% The amssymb package provides various useful mathematical symbols
\usepackage{amssymb}
%% The amsmath package provides various useful equation environments.
\usepackage{amsmath}
%% The amsthm package provides extended theorem environments
%% \usepackage{amsthm}
\usepackage{booktabs}
%% The lineno packages adds line numbers. Start line numbering with
%% \begin{linenumbers}, end it with \end{linenumbers}. Or switch it on
%% for the whole article with \linenumbers.
%% \usepackage{lineno}

%% Other packages
\usepackage{booktabs}
\usepackage{hyperref} 
\usepackage{subcaption}

\journal{Journal}

\begin{document}

\begin{frontmatter}

%% Title, authors and addresses

%% use the tnoteref command within \title for footnotes;
%% use the tnotetext command for theassociated footnote;
%% use the fnref command within \author or \affiliation for footnotes;
%% use the fntext command for theassociated footnote;
%% use the corref command within \author for corresponding author footnotes;
%% use the cortext command for theassociated footnote;
%% use the ead command for the email address,
%% and the form \ead[url] for the home page:
%% \title{Title\tnoteref{label1}}
%% \tnotetext[label1]{}
%% \author{Name\corref{cor1}\fnref{label2}}
%% \ead{email address}
%% \ead[url]{home page}
%% \fntext[label2]{}
%% \cortext[cor1]{}
%% \affiliation{organization={},
%%             addressline={},
%%             city={},
%%             postcode={},
%%             state={},
%%             country={}}
%% \fntext[label3]{}

\title{A Comparison of Human and Machine Learning Errors in Face Recognition}

%% use optional labels to link authors explicitly to addresses:
%% \author[label1,label2]{}
%% \affiliation[label1]{organization={},
%%             addressline={},
%%             city={},
%%             postcode={},
%%             state={},
%%             country={}}
%%
%% \affiliation[label2]{organization={},
%%             addressline={},
%%             city={},
%%             postcode={},
%%             state={},
%%             country={}}

%% Author name
%\author[]{ANONYMOUS AUTHOR(S)}

\author[inst1]{Marina Estévez-Almenzar}
%% Author affiliation
\affiliation[inst1]{organization={Universitat Pompeu Fabra},%Department and Organization
            %addressline={}, 
            city={Barcelona},
            %postcode={}, 
            %state={},
            country={Spain}}

%% Author name            
\author[inst1,inst2]{Ricardo Baeza-Yates}
%% Author affiliation
\affiliation[inst2]{organization={EAI, Northeastern University},%Department and Organization
            %addressline={}, 
            city={Silicon Valley},
            %postcode={}, 
            state={California},
            country={USA}}

%% Author name            
\author[inst1,inst3]{Carlos Castillo}
%% Author affiliation
\affiliation[inst3]{organization={ICREA},%Department and Organization
            %addressline={}, 
            city={Barcelona},
            %postcode={}, 
            %state={},
            country={Spain}}
%% Abstract
\begin{abstract}
%% Text of abstract
% Motivation
Machine learning applications in high-stakes scenarios should always operate under human oversight.
% Goal
Developing an optimal combination of human and machine intelligence requires an understanding of their complementarities, particularly regarding the similarities and differences in the way they make mistakes.
% Method
We perform extensive experiments in the area of face recognition and compare two automated face recognition systems against human annotators through a demographically balanced user study. 
% Results
Our research uncovers important ways in which machine learning errors and human errors differ from each other, and suggests potential strategies in which human-machine collaboration can improve accuracy in face recognition.
%
%This allows us to propose a human-centred approach for the understanding of machine learning errors.
\end{abstract}

%%Graphical abstract
%\begin{graphicalabstract}
%\includegraphics{grabs}
%\end{graphicalabstract}

%%Research highlights
%\begin{highlights}
%\item We conduct comprehensive experiments in the field of face recognition, evaluating two automated face recognition systems in comparison with human annotators using a demographically balanced user study. 
%\item The face recognition system shows a marked disparity in similarity scores between correctly and incorrectly solved tasks,  which acts as an effective indicator of errors.
%\item There is a correlation between the difficulties shared by more than one model and the difficulties experienced by humans, while the latter find it easier to solve tasks where at least one of the models solves them correctly.
%\item Humans perform substantially better than facial recognition models in assessing negative pairs (pairs consisting of different identities).
%\item Leveraging the results of the comparative study enabled us to design a manual evaluation strategy that achieves maximum joint human-machine precision with a very low number of annotations.
%\end{highlights}

%% Keywords
\begin{keyword}
%% keywords here, in the form: keyword \sep keyword
Human-centered computing \sep User studies \sep Face recognition \sep Machine learning errors
%% PACS codes here, in the form: \PACS code \sep code

%% MSC codes here, in the form: \MSC code \sep code
%% or \MSC[2008] code \sep code (2000 is the default)

\end{keyword}

\end{frontmatter}

%% Add \usepackage{lineno} before \begin{document} and uncomment 
%% following line to enable line numbers
%% \linenumbers

%% main text
%%

%% Use \section commands to start a section
\section{Introduction}
\label{introduction}
%% Labels are used to cross-reference an item using \ref command.

Decision support systems powered by machine learning (ML) are increasingly used in high-stakes scenarios including immigration, video-surveillance, healthcare, justice, and access to labor and education, among many others.
%
In these application domains, ML systems should not be autonomous, but rely on a human operator or expert, who should be responsible for the final decision.
%
%This poses a scenario in which the algorithm is no longer the only agent involved in solving the task, as the human operator becomes an indispensable part of the decision-making system. 
An in-depth understanding of the dynamics of human-algorithm interactions is crucial for developing safe, trustworthy systems \cite{matias2023humans}.

%
%Human operators need to consider carefully the characteristics of the models being used, and particularly they need to tread a fine line between algorithmic aversion and automation bias.
%
%\emph{Algorithmic aversion} is a biased, overly negative assessment of an algorithm, often arising after learning that an algorithm is imperfect~\cite{madoc2020aversion}.
%
%\emph{Automation bias} is the opposite, an over-reliance on automated decision support mechanisms, typically observed when the human operator is under high cognitive load~\cite{lyell2017automation}.
%
%Recognizing both phenomena requires an in-depth understanding of error rates, modes of failure, and limitations of the algorithms being used.
Understanding the complementarities between human and machine intelligence is crucial.
%
%However, creating hybrid systems that integrate human and machine intelligence presents a challenge, as only very few human-algorithm patterns are solely beneficial or detrimental to society \cite{matias2023humans}.
%
In an ``ideal'' scenario, there is perfect complementarity: cases challenging for the ML system are easily handled by the human operator.
%
Conversely, the worst case is when there is total overlap: cases that are difficult or uncertain for the ML also lead to human errors.
%
In practice, we may find applications that are somewhere between these extremes, as our empirical findings demonstrate within the context of face recognition.


% Occupying the niche
  % outlining purposes (what is the aim of this research paper?) ... 
Among other areas that need exploration, little has been done to understand when human and machine errors are similar and when they are different. Analyzing these similarities and differences is particularly important because the presence of algorithmic errors influences well-known patterns of human-machine interaction, such as \emph{algorithmic aversion} \cite{madoc2020aversion,dietvorst2015algorithm}, a biased and overly negative human evaluation of an algorithm.
%
The question arises as to whether this aversion also varies depending on whether or not the errors presented by the model resemble those that a human agent might make.
%
If we are able to avoid this type of bias, there is another risk: \emph{automation bias,} an over-reliance on automated decision support mechanisms \cite{lyell2017automation}.
%
In this case we can ask an analogous question: Does a high similarity between human and machine errors influence the human agent's ability to judge the accuracy of the model? 
%Answering these questions requires an in-depth understanding of human and machine complementarities.


The main goal of this research is to compare ML errors and human errors. The results obtained from this comparative study serve as inputs for the development of straightforward yet impactful strategies to combine human and machine intelligence.
%
%Human-machine complementarities in face matching enable us to offer straightforward yet impactful strategies.
%When implemented, these strategies indicate that collaborations between humans and machines constitute a reliable and secure approach to decision-making.

The use of the concept ``human error'' suggests an homogeneity that is almost non-existent in real life. Human perception varies from individual to individual, either due to variations in physiological structures or external influences such as culture.
%
It becomes even more complex when it comes to assessing human perception in distinguishing between other human identities, such as in the context of a face recognition task.
%
We tackle this complexity through a demographically diverse user study for face matching in which possible inter-individual differences and disagreements are considered. 
%
Proposing hybrid human-machine strategies in the field of face recognition is crucial. For instance, in an automated system integrated in a police surveillance scenario, interrogating or detaining someone just because a face recognition system has erroneously matched their face in a database of persons of interest deserves special attention in the current use of facial recognition technologies. In January 2020, Robert Julian-Borchak Williams became the first documented example in the U.S. of someone being wrongfully arrested based on a false hit produced by facial recognition technology. \footnote{\href{https://www.nytimes.com/2020/06/24/technology/facial-recognition-arrest.html}{\em Wrongfully Accused by an Algorithm}, The New York Times, 24 June 2020}
%
Many more cases have been documented, and often there are racial biases.\footnote{\href{https://innocenceproject.org/when-artificial-intelligence-gets-it-wrong/}{\em When AI Gets It Wrong}. Innocence Project, 19 September 2023.}

% Occupying the niche (continued):
% ... main findings, value of the research
Our main findings for the face recognition task we study are:
(1) humans rarely produce false positives;
(2) the ML similarity score is a potential error predictor;
(3) humans find it easier to address mistakes made by an individual model compared to addressing shared errors between two models; and
(4) in face recognition the human perception of gender expression and ethnic appearance is determinant.
%
These findings provide a method for detecting potential errors in automated facial recognition, and help us find potential errors that a human annotator has a high chance of correcting.
%
Applying this approach in a practical setting enables us to develop an effective evaluation strategy that maximizes joint human-machine accuracy while controlling human annotation effort.
%
Unlike other approaches that strictly emphasize the enhancement of accuracy through algorithmic advancements, this work underscores not only the importance of incorporating the human factor in this race for accuracy maximization, but also the effectiveness of this approach.


The rest of the paper is organized as follows. In \S\ref{sec:related} we review related work, followed by our research questions and our methodology in \S\ref{sec:research} and \S\ref{sec:methods}, respectively. In \S\ref{sec:results} we present our results, while in \S\ref{sec:discussion} we discuss our results and present a human-computer collaboration strategy based on our findings. We also outline some of the limitations encountered in the development of this work, as well as possible future directions for it. We conclude in \S\ref{sec:conclusions}.

\section{Related Work}
\label{sec:related}

Even though automated facial recognition systems are not influenced by factors that affect human ability to match faces (e.g., time pressure \cite{fysh2017effects}, fatigue \cite{behrens2023fatigue}, processing capabilities in real-time \cite{curry2003capability}), they are affected by other factors. % that make the evaluation of automated face recognition systems a difficult task in real-life scenarios. 

\subsection{Human and ML performance}

%  1. human outperform algorithms
ML systems may outperform human annotators in tasks considered simple or moderately difficult, but they tend to struggle when faced with more complex conditions that mirror real-life scenarios. In challenging tasks, non-expert observers showed performance comparable to that of some facial recognition algorithms, while in some cases experts outperformed these algorithms \cite{white2015perceptual}. Some systems did not make accurate identifications, while humans exceeded random chance \cite{rice2013unaware}.
%
Throughout the years, numerous competitions have been conducted to assess human-algorithm performance in different face recognition tasks, opposing algorithmic accuracy to human accuracy, thus establishing a distinction between two solving agents that, instead of collaborating, compete. Phillips {\em et al.} \cite{phillips2014comparison} conducted a cross-modal study to evaluate the results of selected human-algorithm competitions in facial recognition. Their findings revealed that algorithms outperformed humans in the case of simple frontal static images, whereas humans demonstrated superiority in challenging static images and videos.
Rice {\em et al.} \cite{rice2013unaware} were interested in the specific cases where the facial recognition system fails, and investigated how humans performed in these cases. They documented instances where facial recognition algorithms did not achieve any successful matches, while humans outperformed random chance. White {\em et al.} \cite{white2015perceptual} observed that in forensic facial identification algorithms performed similarly to certain observers and were outperformed by experts. 

\subsection{Combining human and machine intelligence}
% 2+3. human does not trust algorithms: algorithm aversion + control
%This human-machine competition approach to facial identification is opposed to other approaches that seek to find distinct patterns of behavior derived from an interaction between the two agents. Several discussions have taken place on the reliability future potential users may have in facial recognition systems. 
Researchers have seek to uncover how human and machine intelligence can be reliably combined.
%
\textit{Algorithm aversion} has been extensively studied \cite{dzindolet2002perceived,reich2023overcome}, and has been shown that it becomes particularly noticeable when users witness mistakes made by the algorithm \cite{dietvorst2015algorithm}. This aversion is reduced when the user has some level of control (even if little) over the prediction process \cite{dietvorst2018overcoming,roy2019automation}.

% 4. human-computer interaction, oversight and evaluating algorithms
This situation illustrates that the successful integration of facial recognition systems in practical settings necessitates more than just technological progress. According to the EU AI Act \cite{EUAIAct}, the implementation of facial recognition should be proportionate and deployed only when strictly necessary. In \cite{negri2024framework}, Negri {\em et al.} present a framework aimed at determining whether a facial recognition intervention is appropriate for a particular usage scenario.
%
Other factors such as the application context, including the prospective end-users and the demographic characteristics of the population on which the system will operate, must be thoroughly taken into account. These approaches are closely connected to investigating human-centered ML techniques \cite{papenmeier2022accurate}, such as human oversight strategies for assessing and enhancing system outcomes \cite{hupont2022landscape,kyriakou2023humans}, as well as mechanisms for preserving the essential human element in decision-making in areas where safeguarding fundamental rights is particularly crucial \cite{koulu2020proceduralizing}. 

% 5. human factors in algorithms
\subsection{Human factors in decision support}
Numerous studies have been conducted to explore methods for incorporating human factors into ML systems. Han {\em et al.} \cite{han2017hard} introduced an emotion detection model that leveraged inter-annotator agreement to provide a prediction that is more akin to human judgment. Their approach diverged from the conventional belief that a person's state can be simply classified into a \textit{hard} category or single value.
% -- 
The idea of representing the human element using a continuous distribution is endorsed by Peterson {\em et al.} \cite{peterson2019human}, who introduced a novel image dataset that includes a comprehensive range of human annotations for each image. By representing human-like uncertainty, they achieved favorable results in terms of robustness and out-of-training-set performance, demonstrating that errors in human classification can be just as enlightening as accurate responses. 
%

Related to facial recognition technologies, Andrews {\em et al.} \cite{andrews2023view} raised doubts about the ability of categorical labels to capture the continuous spectrum of human phenotype diversity, particularly in the context of deducing delicate characteristics like social identities (as only a few facial recognition datasets include self-identified categories). They presented a dataset with human perception of face similarities that can be used to learn an embedding space aligned with human perception.
% -- Continuous labels to capture human perception as a spectrum
In relation to the distinctive features present in human perception, Makino et al. \cite{makino2022differences} examined the differences between deep neural networks (DNNs) and human perception in medical diagnosis, focusing on breast cancer screening. They discovered that DNNs utilize features that radiologists often ignore and are outside areas that they considered suspicious. This underscores the importance of incorporating domain knowledge into comparisons of human and machine perception to prevent erroneous outcomes.
%
In a similar vein, Huber {\em et al.} \cite{huber2022stating} proposed the propagation of model uncertainties to the final output to enhance transparency of facial recognition systems and offer a deeper understanding of the verification process. 
%
Additionally, Papenmeir {\em et al.} \cite{papenmeier2022accurate} conducted a study involving users and discovered that the perceived accuracy of a model was notably more reduced when users saw the model failing on a simple task compared to when it made mistakes on more challenging tasks. This research suggests that algorithm aversion may be impacted differently based on the type of model errors encountered.
%

% human influence on algorithms: other-race effect and 
While it is desirable to have greater consideration of human factors in the automated decision-making process, researchers have also studied the negative consequences of mimicking certain human biases \cite{hupont2019demogpairs}. The \textit{other-race effect} for face recognition (our ability to best recognize the identity of faces from our own race) has been observed in several human studies \cite{meissner2001thirty,feliciano2016shades}. Philips {\em et al.} \cite{phillips2011other} showed an other-race effect for the algorithms, concluding that their performance varies as a function of the demographic origin of the algorithm and the demographic contents of the test population. Similarly, motivated by this human bias, Flores-Saviaga {\em et al.} \cite{flores2023inclusive} propose an alternative interface to the classical human-in-the-loop interface and suggest that deriving the classification of facial image pairs as a function of annotators' race will improve the efficiency of the system. But, as they point out, such design decisions carry delicate ethical implications that underscore the importance of work along other lines.
%

There is a recent line of research investigating how human annotators can be effectively introduced into the loop so the algorithm can pass the final decision to the human when certain conditions are given \cite{hemmer2023learning,mozannar2023should,keswani2021towards}. These conditions are often related to the low confidence of an automated system, which can be used to determine what type of human-machine interaction is most appropriate in a hybrid system \cite{punzi2024ai}, as well as to distinguish which annotations flows should be adopted to make human-machine collaboration more efficient \cite{lee2022towards}. Combining decisions of systems and humans based on (weighted by) their perceived individual similarities has also been investigated \cite{phillips2018face}.

% "Our contribution"
\bigskip
To the best of our knowledge, most of the efforts in integrating human factors into technology have mainly focused on encoding specific human traits and enhancing model performance — observing humans and refining models independently. Some more recent efforts have gone further, proposing novel techniques to combine human and machine performance. However, there is still a lack of understanding of the key differences of decision-makers, especially in contexts where the task involves a certain subjectivity. In a scenario where there is no longer only the final decision related to the task at hand, but also the decision as to which agent — human, algorithmic or combination of both — should make the final decision, it is important to know the strengths and weaknesses of both agents, which of these are shared and which diverge, and how these differences and similarities can be exploited. Here, we propose to study human and machine similarities and differences to capture and understand their complementary aspects so that this knowledge enables efficient and accountable human-machine interaction paradigms.
%
For this purpose, we establish an error-centred comparison of human and machine performance in solving a face matching task. Examining these distinctions and similarities is crucial for enhancing the effectiveness of human oversight of algorithms, and for gaining insights into integrating the human factor into decision-making processes. Additionally, we explore how gender and ethnicity can influence human errors, building on earlier findings that have identified their significance in tasks that involve facial recognition \cite{phillips2011other,wright2003own}.

\section{Research Questions}
\label{sec:research}

The main goal of this work is to study model errors in face recognition, comparing them with human errors. 
%
We also investigate how human conceptions of gender and ethnicity affect these errors.

Our experimental setting, described in detail in \S\ref{sec:methods}, is based on a number of face recognition tasks that are performed by two automated systems, as well as by human annotators hired through a crowdsourcing platform. These tasks consist of matching facial images: given a pair of facial images, determining whether they belong to the same individual or to two distinct individuals. Both the two automated models and the set of annotators performed this task independently.

\subsection{Error consistency} 
\label{subsec:consistency}
We would like to characterize similarities and differences between human errors and ML system errors.
%
% between the mechanisms that induce errors in humans and ML systems.
%
To achieve this, first we need to determine if errors and successes are consistent, {\em i.e.}, if we can determine which are the subsets of face recognition tasks in which errors and successes are concentrated.

% References later our previous paper
\begin{quote}
\textbf{RQ1a}
\textit{Are human annotators consistent when solving a face recognition task?}


%likely to correctly and incorrectly identify faces in the same subsets of face recognition tasks?}
%alternatives: Is there enough agreement among participants to clearly categorize the errors made by the models as “equal/close to human error” or “different/far from human error”?
%\end{quote}
%\begin{quote}

\textbf{RQ1b}
\textit{Are ML systems consistent when solving a face recognition task?}
% likely to correctly and incorrectly identify faces in the same subsets of face recognition tasks?}
%alternatives: Is there enough agreement among participants to clearly categorize the errors made by the models as “equal/close to human error” or “different/far from human error”?
\end{quote}

If humans are consistent in their errors and successes, then we can define for a ML system in general, and for a face recognition system in our case, ``human-like'' errors as those machine errors that a human would also tend to make, and ``non-human-like'' errors as those machine errors that a human would not be likely to make.


\subsection{Error alignment} We want to uncover whether there are common difficulties between ML systems and human annotators.
%
We expect these common difficulties to manifest as incorrect human annotations on those face recognition tasks where the ML system erred. We also expect to obtain more incorrect annotations in cases where more than one ML system errs.
%
If human annotators and ML systems know whether they are likely to be making a mistake ({\em i.e.}, provide a low-confidence annotation), then we would like to test whether their confidence in annotation aligns. 
%
This would indicate that not only there are face recognition tasks that are likely to lead to errors by human annotators and ML systems, but that there are also tasks that are more challenging and elicit less certainty in both situations.
\begin{quote}
\textbf{RQ2a} \textit{Are human annotators more likely to make a mistake in a face recognition task if a ML system also gives an incorrect answer for that task, compared to tasks for which the system is correct?}
%\end{quote}
%\begin{quote}

\textbf{RQ2b} \textit{Are human annotators even more likely to make a mistake if more than one ML system is incorrect?}
%\end{quote}
%\begin{quote}

\textbf{RQ2c} \textit{Are human annotators' perception of similarity and ML computations of similarity correlated?}
\end{quote}

\subsection{The role of gender and ethnicity in errors} We define, as detailed in the next section, a false positive in face recognition as the incorrect identification of two images of different people as the same person.
%
False positives involving people of different gender and/or ethnicity are unlikely to be made by human annotators, as differences related to gender expression and ethnic appearance are determining factors in humans when establishing an identity judgment \cite{phillips2011other,wright2003own}.
%
%We would like to know whether this is so prevalent that we can use equal/different gender/ethnicity as a proxy for ``human-like'' and ``non-human-like'' errors.
% RIC: redundante
%
In addition to errors, we would like to know whether cases where images are perceived to have different gender/ethnicity by human annotators lead to lower confidence by a ML system.

We remark that ``human perception of gender and ethnicity,'' refers to differences and similarities in terms of gender expression and ethnic appearance, and not in terms of gender identity or self-ascription to an ethnicity. 
%
% [...] It is important to note that The judgment suggested here is based on how two people's expressions are perceived by a person judging through facial images. This judgment should not be understood as a truth about the identity of the persons being judged.
As described in the next section, in some datasets labels are not provided by the photo subjects themselves, but are inferred through other means.
\begin{quote}
\textbf{RQ3a} \textit{Are ML errors on pairs of images labeled as depicting different gender expression, or eliciting different perceptions of ethnicity unlikely to be made by human annotators, to the extent that this can be used to characterize ``human-like'' and ``non-human-like'' errors?}
%\end{quote}
%\begin{quote}

\textbf{RQ3b} \textit{Are human perception of similarity and/or ML similarity score correlated with human perception of gender and/or ethnicity similarity?}
\end{quote}
%\begin{quote}
%\textbf{RQ3(c)} \textit{Annotator ethnicity vs ethnicity in item (other-race effect)}
%\end{quote}

\subsection{Exploratory study of error-based human-machine collaboration} With the above questions we want to know if we can develop a strategy to optimise human-machine collaboration in the context of solving a face recognition task. The consistency raised in the first question would allow us to generalise in this context, while the study of the alignment between human and machine error patterns would allow us to detect the key points of complementarity for the development of a successful strategy.
\begin{quote}
\textbf{RQ4} \textit{Can we design a human-computer collaboration strategy based on the results obtained from this comparative study?}
\end{quote}

\section{Experimental Setup and Ethical Considerations}
\label{sec:methods}

In the next five subsections we detail the data used and the methodological aspects of this work. In the last subsection we detail some of the ethical aspects that have been taken into account in the development of this work.

\subsection{Datasets}

\subsubsection{Training data.} We used two pre-trained face recognition models.
%
Both were trained by their respective authors on \emph{MS-Celeb-1M} \cite{guo2016ms}, a dataset released by Microsoft in 2016. According to its authors, it was the largest publicly available face recognition dataset in the world. It contains about 10M images of nearly 100K people. %As the name suggests, this set of individuals contains celebrities, mostly American and British, but also other non-Hollywood personalities, such as policymakers, writers and academics. It even contains critics of the technology for which Microsoft is using their names and biometric information: digital rights activists, artists critical of surveillance, and founders of institutions researching the social implications of data in AI. 
After an investigation by Financial Times in 2019,\footnote{\href{https://www.ft.com/content/cf19b956-60a2-11e9-b285-3acd5d43599e}{\em Who’s using your face? The ugly truth about facial recognition} Financial Times, 18 September 2019} it was found that many of the people who appeared in the images were not asked for their consent, nor were they aware that their faces appeared in this database. Some time after this finding, without warning, 
Microsoft removed MS-Celeb-1M and its web page\footnote{\href{https://www.msceleb.org/}{https://www.msceleb.org/}} is currently offline.
%
Before its demise, the dataset was widely used and still exists in several forms, such as trained models. 
%
MS-Celeb-1M is fairly unbalanced demographically (see Table \ref{tab:datasets}).

\begin{table}[t]
\centering
\caption{Characteristics of MS-Celeb-1M \cite{guo2016ms,wang2019racial} and DemogPairs \cite{hupont2019demogpairs}.}
\label{tab:datasets}
\begin{tabular}{ccc} \toprule
 & ~~MS-Celeb-1M~~ & ~~DemogPairs~~  \\ \midrule
\# Images & 10M  & 10.8K  \\
\# People & 100K & 600 \\ \midrule
\% Female & $\approx$80\% & 50\% \\
\% Male   & $\approx$20\% & 50\% \\ \midrule
\% White  & 76.3\%        & 33.3\% \\
\% Black  & 14.5\%        & 33.3\% \\
\% Asian  & 6.6\%         & 33.3\% \\
\% Other  & 2.6\%         & - \\
\bottomrule
\end{tabular}
\end{table}

\subsubsection{Testing data.} We used \emph{DemogPairs} \cite{hupont2019demogpairs} as the evaluation dataset.
%
It contains 10,800 facial images corresponding to 600 people divided into 6 balanced demographic labeled folds: \{ female, male \} $\times$ \{ Black, Asian, White \}. %Asian females, Asian males, black females, black males, white females and white males. 
These labels were manually annotated by their authors. We will use \textit{labels} when we refer to those from the original dataset, and \textit{annotations} for those obtained from our user study. 
%
Each demographic fold has 100 subjects, with 18 images per subject (see Table \ref{tab:datasets}). DemogPairs was created and released by its authors with the explicit objective of being used as a tool to test for demographic biases on face recognition models.
%It was tested with some of the state-of-the-art deep face recognition models (SphereFace, FaceNet and ResNet50) and it was shown that these models suffer from a highly detrimental demographic bias.

\subsection{Models}

The two face recognition models used in this work were IR50+ArcFace \cite{deng2019arcface} and LightCNN v4 \cite{wu2018light}, both trained, as explained before, over MS-Celeb-1M by the respective authors. We did not do any additional training or fine-tuning for this work. Both pre-trained models can be found in their original sources.\footnote{\href{https://github.com/ZhaoJ9014/face.evoLVe}{IR50+ArcFace pre-trained model on MS-Celeb-1M}, \href{https://github.com/AlfredXiangWu/LightCNN}{LightCNN v4 pre-trained model on MS-Celeb-1M}}

\textbf{IR50+ArcFace} is an extension of ResNet50 \cite{guo2016ms,he2016deep}, a residual network that has been extensively applied to many image tasks, with an ArcFace loss function \cite{deng2019arcface}. It reaches an accuracy of 99.78\% in the well-known LFW (Labeled Faces in the Wild) public benchmark for pair matching \cite{huang2008labeled}.
%
\textbf{LightCNN} was created to learn a compact embedding on large-scale face data with noisy labels. It has been reported to achieve state-of-the-art results on various face benchmarks without fine-tuning \cite{wu2018light}. In this work, we used the 29-layer model version, which reaches an accuracy of 99.40\% in LFW.

For evaluation, we used \texttt{face.evoLVe} \cite{wang2021face}, a face recognition library that provides a standard interface and can be used with various models for face-related analytics and applications. For the purposes of this research, the library was instrumented to keep track of individual errors. The instrumented library is available with our code release.
%According to the documentation, it includes various backbones (e.g., ResNet, IR, IR-SE, ResNeXt, SE-ResNeXt, DenseNet, LightCNN, MobileNet, ShuffleNet, DPN, etc.), various losses (e.g., Softmax, Focal, Center, SphereFace, CosFace, AmSoftmax, ArcFace, Triplet, etc.) and some other tricks for improving performance (e.g., training refinements, model tweaks, knowledge distillation...). However, as far as backbones are concerned, we found that only one of those mentioned (IR) is working in this library.

\subsection{Procedure}\label{subsec:procedure}

We performed an online user study, with the following structure. 

\subsubsection{Participant recruitment}
We recruited participants through a crowdsourcing platform for experimentation named Prolific.\footnote{\href{https://www.prolific.co/}{www.prolific.co}}
%
We considered four countries in continental Europe in which Prolific has large user bases: France, Germany, Italy, and Spain, plus the United Kingdom and Turkey. 
%
The crowdsourcing platform provides gender information and allows users to self-identify with a ``simplified ethnic group,'' which is made available as a criterion for participant selection.
%
We made sure that our sets of participants were gender balanced, and that for each pair of images, at least one person from each simplified ethnic group (White, Black, and Asian) participated in their evaluation. So, for every pair of images, we collected at least 3 annotations. For the subsequent analysis, for each pair of images, we take into account exactly 3 annotations, one from each simplified ethnic group.

In total, we recruited 235 participants, excluding 2 of them from our data due to failed attention checks. For the subsequent analysis, based on ethnic self-identification, we selected 162 participants.
%
Participants were paid 0.70 GBP to label 10 pairs of images, with an average completion time of 5 minutes. This amounts to 8.4 GBP per hour, which is slightly above the recommended payment by this platform (8 GBP/h).

\subsubsection{Demographic questionnaire}
Participants were asked about their age, gender identity, and ethnic background (see Figure \ref{fig:survey}).

\begin{figure}[p!]
\centering
    \begin{subfigure}[b]{0.9\textwidth}
        %\begin{subfigure}[b]{0.89\textwidth}
        \begin{subfigure}[b]{0.475\textwidth}
             \centering
             \caption{Demographics questionnaire.}
             \includegraphics[width=\textwidth]{images/survey_demog.png}
             \label{fig:survey_demog}
         \end{subfigure}
         %begin{subfigure}[b]{0.89\textwidth}
         \begin{subfigure}[b]{0.475\textwidth}
             \centering
             \caption{Pair shown to the participant in survey. First question.}
             \includegraphics[width=\textwidth]{images/survey_pair.png}
             \label{fig:survey_pair}
        \end{subfigure}
    \end{subfigure}
    \hfill
    \begin{subfigure}[b]{0.9\textwidth}
          \centering
          \caption{Second question shown to the participant only when they answered something different to \textit{Yes} in the first question.}
         %\includegraphics[height=\textwidth, angle=-90]{images/survey_clar.png}
         \includegraphics[width=\textwidth]{images/survey_clar.png}
         \label{fig:survey_clar}
    \end{subfigure}
    \caption{Survey screenshots. First, participants were asked about their age, gender identity and ethnic background. Then, participants started to evaluate the pairs of images. For those where the participant was not completely sure of both identities being the same person (answering something different to \textit{Yes} in question \ref{fig:survey_pair}), participants were asked to provide some details relate to gender expression and ethnic appearance similarities, as shown in \ref{fig:survey_clar}.}
    %\Description{Three screenshots showing the three different pages that the survey participant had to answer. The first page shows the question regarding the participant's demographics. The second shows the pair of facial images and the question of whether or not they are the same person. The third page shows the pair again and asks how similar they are in terms of gender expression, ethnic appearance, and age appearance.}
    \label{fig:survey}
\end{figure}

\subsubsection{Face recognition tasks}
Participants evaluated one pair of images at a time. The participant had to answer the question \textit{Are they the same person?}, with the possible options: \textit{No}, \textit{Probably not}, \textit{Not sure}, \textit{Probably yes} or \textit{Yes}.

% We can remove this and the text is still understandable
%If the answer was \textit{Not}, then the next pair of images to be evaluated was shown.
%
If the answer was different from \textit{Yes}, then the same pair of images was shown one more time, and the participant was asked about some of the differences between the two images.
%
These differences referred to gender expression, ethnic appearance, and age appearance (see Figure \ref{fig:survey_clar} for details). The participant had to answer three questions: \textit{How are these persons in terms of \{ gender expression | ethnic appearance | age appearance\}}. Each question had to be answered independently on a scale with five options: \textit{Different}, \textit{Probably different}, \textit{Not sure}, \textit{Probably equal} and \textit{Equal}.
%
We remark that we asked about ``expression'' and ``appearance'' because the participants do not know the identities of the photo subjects.

\subsubsection{Task selection} We found that the joint accuracy of the face recognition models (see \S\ref{subsec:measurements}) was correct above 95\% of the tasks.
%
Hence, due to budget constraints, we annotated all the cases where the models were wrong (``misses''), and a sample of cases in which both models were right (``hits'').
%
First, we annotated 363 ``misses'' (237 false negatives and 126 false positives, see Table \ref{tab:models_errors}), which were shown to a total of 164 participants, from which we selected a demographically balanced set of 108 participants. % (79 women, 82 men and 2 non-binary people).
%
Next, we annotated 180 model ``hits,'' which were shown to a total of 69 participants, from which we selected a demographically balanced set of 54 participants. % (35 women, 33 men).
This selection of ``hits'' was a random sample that was demographically balanced for the true positive set (90 pairs) and for the true negative set (90 pairs). 

\subsection{Measurements}
\label{subsec:measurements}
We measured the following dependent variables.

\paragraph{Accuracy} Accuracy is defined as the fraction of correct responses with respect to the ground truth.

\begin{enumerate}
    \item \textbf{Machine accuracy}: Joint accuracy of the models. Each individual accuracy is calculated as the number of correct answers divided by the total number of pairs. To calculate the joint accuracy, in those cases where there is a disagreement between both models (for pairs labeled as positive by one model and as negative by the other) the average of their calibrated similarity scores is calculated and the label is decided based on this average (positive if it is above $0.5$ and negative if it is below).
    %$$ \frac{\textrm{number of correct responses by model}}{\textrm{total number of pairs}}~.$$
    \item \textbf{Human accuracy}: Accuracy of the human annotators, as a group of three annotators. This is computed as a macro average, {\em i.e.}, first all the human evaluations on a pair of images are averaged, and then we determine whether that average is correct or not, computing human accuracy as number of correct responses by group of participants divided by the total number of pairs.
    %$$ \frac{\textrm{number of correct responses by group of participants}}{\textrm{total number of pairs}}~.$$ 
\end{enumerate}

\paragraph{Similarity} This is a measurement of how similar the model or the human annotator perceives the persons in the images.
\begin{enumerate}
    \item \textbf{ML similarity score}: Given two images, the model computes two embeddings or feature vectors (one per image). The numerical distance between these embeddings, $d$, is compared against a threshold $\theta$ to determine the output (if $d < \theta$, the pair of images is labeled as positive, while if $d > \theta$, the pair of images is labeled as negative). After normalizing this distance, we take $1-d$ as the similarity of the pair. Because the original scores are not calibrated, we calibrate this similarity, so it can be interpreted as a probability lying in the $[0, 1]$ interval. Scores close to 0.5 can be interpreted as a low model confidence.
    \item \textbf{Human perception of similarity}: this is inferred from the distance between the answer \textit{Not sure} and the annotator's actual answer to the questions specified in section \S\ref{subsec:procedure}. From this measurement we can infer human confidence: the answers in the extremes (\textit{No} and \textit{Yes}) correspond to the highest confidence, while answer \textit{Not sure} corresponds to the lowest confidence.
\end{enumerate}

%\spara{Gender expression and ethnic appearance.} In both cases, we convert the ordinal responses from \textit{Different} to \textit{Equal} to a numerical value between 0.0 and 1.0, with steps of 0.2 between each of the five options.

\subsection{Ethical Considerations}

Our research plan was reviewed and approved by %the Institutional Committee for Ethical Review of Projects (CIREP) at Universitat Pompeu Fabra.
the Ethics Review Board of our university. % ANONYMIZED
%
The review included compliance with internationally accepted ethical principles in research, and with personal data protection guided by the EU General Data Protection Regulation (2016/679).

Regarding the gender and ethnicity discussed in this paper, it is important to note two things: (1) the original labels regarding gender and ethnicity in the testing database were inferred by means other than directly asking the person in the image about their demographics, so they should in no way be assumed to be true, and (2) in our study the participants were shown a pair of images and were asked about the similarity of \textit{gender expression} and \textit{ethnic appearance}, these being different concepts to those relating to the social identities of people in the images.


\section{Results}
\label{sec:results}
In what follows, we will consider a \textit{human error} when the mean response of the three annotators solving the same task corresponds to a wrong response, and a \textit{human success} when the mean response corresponds to a correct one. Equivalently, we will consider a \textit{machine error} when at least one of the two models solves the task incorrectly and a \textit{machine success} when both models are correct. For brevity, we will use "false positives", "false negatives", "true positives" and "true negatives" when we refer to the responses given by the models. In case we refer to the annotators' responses, we will do so explicitly ({\em e.g.}, Human False Negatives for positive pairs that annotators classified as negative). We will also show some significance test results (p-values, noted as $p$). Since what we want is to compare unknown a priori distributions, and by virtue of the continuity of our data, all these tests correspond to Kolmogorov-Smirnov tests.

%Annotators were shown the 363 pairs in which at least one of the models made an error (126 false positive pairs and 237 false negative pairs), and a demographically balanced selection of 180 pairs in which both models successed (90 true positive pairs and 90 true negative pairs). These successes were selected from the total of 5097 successful model outcomes.

\subsection{Participant Demographics}\label{subsec:participants}

Participants were on average 27.3 years old (SD=12.0 years). 
%
%On average, participants were M = 27.33 years old (SD = 12.04 years).
Out of the 82 participants that indicated their gender, 45 (55\%) identified as female, 34 (41\%) as male, and 3 (4\%) as non-binary.
%To the question \textit{Which gender identity do you most identify with?}, 45 participants selected "Female", 34 "Male", and 3 "Non-binary".
The majority of the 205 participants that indicated an ethnicity identified as ``White'' (46\%), followed by ``Non-Arab African'' (19\%), ``South Asian`` (13\%), and ``East Asian'' (9\%). 
%
The remaining ethnicities accounted for less than 5\% of the participants each.
%To the question \textit{Which category best describes your ethnicity? Select all that apply}, 95 people identified as "White", 7 as "Middle East and North Africa", 39 as "Non-Arab African", 6 as "Latin American", 27 as "South Asian", 13 as "Southeast Asian", 18 as "East Asian", 14 selected "Other", 9 selected "I prefer not to answer", and 4 participants selected more than one ethnicity.

\subsection{Error Consistency (RQ1)}

We now consider the agreement of human annotations, {\em i.e.}, the extent to which multiple people agree on whether a pair of images represents the same person or not.
%
% 
Annotators were shown a total of 543 pairs of face images: 363 machine errors and 180 machine successes. Since the successes shown to the annotators are only a sample of all the successes from the models, we oversampled them to balance the workload. We also transformed every human annotation, originally based on a numeric 5-point scale, into a binary annotation in order to stablish a fair comparison between human and machine agreement. We obtained a \textit{moderate} multi-rater agreement among annotator responses (Fleiss' kappa = 0.47), which suggest that there is a mixture of agreement and disagreement between annotators (RQ1a). The moderate agreement present among the annotators is mainly due to the agreement they reach in those cases where the machine successes (Fleiss' kappa = 0.51), while in the cases where the machine makes a mistake we find no better agreement than would be the case by chance (Fleiss' kappa = -0.05).
%
As Figure \ref{fig:RQ1_neg} shows, human annotators are almost always correct in negative pairs, {\em i.e.}, when both images represent different people, as less than 5\% of pairs are incorrectly classified as positive by the annotators.
%
However, when images represent the same person, results are mixed. Figure \ref{fig:RQ1_pos} shows that although most of the positive pairs were correctly classified by the annotators, approximately 30\% of those pairs were incorrectly categorized as negative.
%
Differences in the distributions of labels on negative and positive pairs (as shown in the comparison of Figure~\ref{fig:RQ1_neg_and_pos_vio}) are significant at $p \ll 0.0001$.
%Differences between human performance on negative pairs and human performance on positive pairs were significant (KS=$0.616$, \textit{p} = $2.083e{-46}$).

For the models, we obtained an \textit{almost perfect} inter-rater agreement between the outputs of IR50 and LightCNN (Fleiss' kappa = 0.92), which suggests that the agreement between models is much better than would be expected by chance (RQ1b).
%
The human tendency to err with higher probability in positive pairs is similar to the models' way of erring: more than 65\% of model errors are false negatives (see Table~\ref{tab:models_errors}).
%
The agreement among human annotators becomes significantly lower when we consider only \textit{human errors} (Fleiss' kappa = $-0.05$), suggesting a poor agreement among annotators when their mean outcome is erroneous. This reduction in agreement is even more pronounced with the inter-rater agreement between models for \textit{machine errors} (Fleiss' kappa = $-0.29$), which suggests a great disagreement in tasks where at least one of the models made a mistake. The interpretation of negative values for Fleiss' kappa are based on \cite{landis1977measurement}.

% Conclusions: If we consider all evaluations, both the inter-agreement of the models and the multi-agreement of the annotators are significant. Humans false positives are practically non-existing, whereas for models, false positives account for almost 35% of the total errors. When studying agreement over the set of annotators' errors and the set of models' errors, human multi-rater agreement is still fair, while machine inter-agreement becomes slight. This suggests a lack of machine error consistency compared to human error consistency.

\begin{figure}[t!]
\centering
    \begin{subfigure}[b]{0.32\textwidth}
         \centering
         \includegraphics[width=\textwidth]{images/RQ1_neg.png}
         \caption{Different people}
         \label{fig:RQ1_neg}
    \end{subfigure}
    \hfill
    \begin{subfigure}[b]{0.32\textwidth}
        \centering
         \includegraphics[width=\textwidth]{images/RQ1_pos.png}
         \caption{Same person}
         \label{fig:RQ1_pos}
    \end{subfigure}
    \hfill
    \begin{subfigure}[b]{0.32\textwidth}
          \centering
         \includegraphics[width=\textwidth]{images/RQ1_neg_and_pos_vio.png}
         \caption{Comparison}
         \label{fig:RQ1_neg_and_pos_vio}
    \end{subfigure}
\caption{Human evaluations over 543 pairs of facial images. Negative pairs correspond of images of different people (216 negative pairs, see Figure \ref{fig:RQ1_neg}), while positive pairs correspond to images of the same person (327 negative pairs, see Figure \ref{fig:RQ1_pos}). In Figure \ref{fig:RQ1_neg_and_pos_vio}, we see the comparison of the distribution of negative pairs evaluations and positive pairs evaluations.
%: annotators' responses to the question "Are they the same person?" (see Figure \ref{fig:survey_pair}) for negative pairs and positive pairs. Response -2 corresponds to the answer "Not", -1 to "Probably not", 0 to "Not sure", 1 to "Probably yes", and 2 to "Yes".
Responses range from -2 (``No'') to +2 (``Yes'').
}
%\Description{Two histogram plots and a two-violin plot. The first histogram shows the distribution of human evaluation on negative pairs, with the vast majority being correct. Approximately 65\% of the answers to the question "Are they the same person?" were "No". The rest were distributed between "Probably not" and "Not sure". The second histogram shows the distribution of the human evaluation of the positive pairs, which is now more uniformly distributed. The violin graph shows these two previous distributions in violin form, one next to the other.}
\label{fig:RQ1}
\end{figure}

\begin{table}[t]
\caption{Model errors, from a total of 5,460 evaluations. M1 stands for IR50, M2 stands for LightCNN, M1$\cap$M2 stands for the common cases, and M1$\cup$M2 stands for the union of cases with no repetition.}
%\begin{adjustbox}{width=0.8\columnwidth,center}
    \centering
    \begin{tabular}{lcccc}
        \textbf{} & \textbf{ Model 1 } & \textbf{ Model 2 } & \textbf{ M1$\cap$M2 } & \textbf{ M1$\cup$M2 }\rule{0pt}{2.8ex}\\ \hline
        \textbf{ False Negatives } & 160 & 181 & 104 & 237 \rule{0pt}{2.8ex} \\ \hline
        \textbf{ False Positives } & 71 & 88 & 33 & 126 \rule{0pt}{2.8ex} \\ \hline
        \textbf{ Total } & 231 & 269 & 137 & 363 \rule{0pt}{2.8ex} \\ \hline
    \end{tabular}
%\end{adjustbox}
\label{tab:models_errors}
\end{table}

\subsection{Error Alignment (RQ2)}

%RQ2(a): Are human annotators more likely to make a mistake in a face recognition task if a machine learning system also gives an incorrect answer for that task, compared to tasks for which the system is correct? Figure \ref{fig:RQ2a}

%RQ2(b): Are human annotators even more likely to make a mistake if more than one machine learning system is incorrect? \ref{fig:RQ2b}

%RQ2(c): Are human annotators confidence and machine learning confidence correlated? \ref{fig:RQ2c}

Next, we studied the extent to which human successes/errors are aligned with machine successes/errors.
%
%For that, we used annotators' responses to the question in Figure \ref{fig:survey_pair}. This allowed us to
%
We considered four categories of model outcomes: True Positives, False Negatives, True Negatives, and False Positives. 
%
Human performance when evaluating True Negative pairs was significantly different from human performance when evaluating False Positive pairs (\textit{p} $\ll$ 0.0001).
%
Differences were also significant in the case of human performance in the two subsets of positive pairs ($p \ll 0.0001$).

In the case of negative pairs, in which human annotators are almost always correct, Figure~\ref{fig:RQ2a_NEGVIO2} shows less certainty and a possibility of error in the pairs in which ML models make a mistake.
%
Human annotators are less likely to select the option ``No'' and more likely to select the option ``Probably not'' when asked about a pair of images of different people for which the ML models mistakenly indicated that they were the same person. 
%
In the case of positive pairs, shown in Figure~\ref{fig:RQ2a_POSVIO2}, we see a similar trend. In this situation, human errors are concentrated in the cases in which the models also made an error. In other words, there are some pairs of images of the same person for which both human annotators and models are likely to err.
%
This, put together with the significance above, suggests that humans find cases where the machine erred more difficult in comparison to those where the machine succeed (RQ2a).
% Conclusions: Las evaluaciones humanas sobre parejas positivas cambian significativamente cuando se comparan aquéllas que han sido mal clasificadas por el modelo con aquéllas que han sido correctamente clasificadas por el modelo. Esta diferencia se debe a que el anotador tiene más probabilidad de errar en aquéllas parejas donde el modelo también erró. En el caso de las parejas negativas, aunque la forma humana de evaluar dichas parejas también cambia significativamente en función de si el modelo erró o acertó, la respuesta final del anotador no cambia, sino que tiende a ser mayoritariamente correcta para ambos casos. Esto sugiere que, únicamente si la pareja es positiva, la probabilidad de que un humano cometa error en una tarea de reconocimiento facial erróneamente resuelta por un sistema de reconocimiento facial es mayor comparada que la probabilidad de que el humano cometa error en una tarea de reconocimiento facial satisfactoriamente resuelta por dicho sistema. 
\begin{figure}[t!]
\centering
%\caption{Human evaluations over machine errors and successes: for each item, annotators answered the question "Are they the same person?" (see Figure \ref{fig:survey_pair}). Response -2 corresponds to the answer "Not", -1 to "Probably not", 0 to "Not sure", 1 to "Probably yes", and 2 to "Yes".}
    \begin{subfigure}[b]{0.4\textwidth}
        \centering
        \captionsetup{width=.7\linewidth}
        \includegraphics[width=\textwidth]{images/RQ2a_NEGVIO.png}
        \caption{Different people}
        \label{fig:RQ2a_NEGVIO2}
    \end{subfigure}
    \begin{subfigure}[b]{0.4\textwidth}
          \centering
          \captionsetup{width=.7\linewidth}
         \includegraphics[width=\textwidth]{images/RQ2a_POSVIO_.png}
         \caption{Same person}
         \label{fig:RQ2a_POSVIO2}
    \end{subfigure}
\caption{Human evaluations of machine errors (in orange, right violin in every figure) and successes (in blue, left violin in every figure). Negative pairs (a) correspond to images of different people, with False Positives indicating pairs that the models mistakenly labeled as the same person.
%
Positive pairs (b) correspond to images of the same person, with False Negatives indicating pairs that the models mistakenly labeled as different people.
%
Responses range from -2 (``No'') to +2 (``Yes'').}
%\Description{Two violin graphs with two violins each. The first graph plots the participants' responses for the negative pairs: one violin shows the distribution of true negative responses and the other the distribution of false positive responses. The first violin is more concentrated at the value -2 (answer ‘No’, with mean at -2), while the second violin is more distributed, still concentrated below 0, corresponding to answer "Not sure" (mean at -1.5). In the second graph the participants' responses for the positive pairs are represented: one violin shows the distribution of the true positive responses (mean at 0.5) and the other the distribution of the false negative responses (mean at -0.25).}
\label{fig:RQ2a2}
\end{figure}


%RQ2(b): Are human annotators even more likely to make a mistake if more than one machine learning system is incorrect? \ref{fig:RQ2b}

%
In general, annotators were more likely to make a mistake on pairs in which both models made an error (RQ2b); with human certainty (preference for ``No'' over ``Probably not'') reduced in false positives of both models, and human error more likely in false negatives of both models.
%For False Positive cases (Figure \ref{fig:RQ2b_FP}), most of the model errors are correctly evaluated as negative pairs by the annotators. However, annotators certainty is reduced (i.e. their response is closer to "Not sure" response) when evaluating an error made by both models. For False Negative cases (Figure \ref{fig:RQ2b_FN}) annotators do not perform as well as in the case of False Positives. On average, the response of the annotators is closer to "Not sure" for those errors that have been solely made by either IR50 or LightCNN. This certainty increases when annotators evaluate errors committed by both models, but it is then that these evaluations tend to be even more erroneous.
%
In the case of False Positives, human evaluation over those errors committed solely by IR50 are significantly different from human evaluations over those committed by both models, at $p<0.001$. %(KS = 0.476, p = 0.00036). 
%
However, human evaluations over False Positives committed solely by LightCNN is not significantly different from human evaluation over False Positives committed by both models ($p=0.17$). % (KS = 0.236, p = 0.1736).
%
We depict these differences in Figure \ref{fig:RQ2b_FP}.
%
In the case of False Negatives, human evaluation over those committed solely by IR50 is significantly different from human evaluation over those committed by both models ($p<0.001$). % (K = 0.3198, p = 0.001).
%
Similarly, human evaluations over False Negatives committed solely by LightCNN are significantly different from human evaluation over False Negatives committed by both models ($p \ll 0.0001$). % (KS = 0.34, p = 4.712e-05).
%
We depict these differences in Figure \ref{fig:RQ2b_FN}.
%

\begin{figure}[t!]
\centering
%\caption{Annotators' responses to the question "Are they the same person?" (see Figure \ref{fig:survey_pair}) for three disjoint groups of errors: errors made by model IR50 only, errors made by model LightCNN only, and in common errors made by both models. Response -2 corresponds to the answer "Not", -1 to "Probably not", 0 to "Not sure", 1 to "Probably yes", and 2 to "Yes".}
    \begin{subfigure}[b]{0.4\textwidth}
         \centering
         \includegraphics[width=\textwidth]{images/RQ2b_FP_custom.png}
         \caption{}
         \label{fig:RQ2b_FP}
    \end{subfigure}
    \begin{subfigure}[b]{0.4\textwidth}
          \centering
         \includegraphics[width=\textwidth]{images/RQ2b_FN_custom.png}
         \caption{}
         \label{fig:RQ2b_FN}
    \end{subfigure}
\caption{Human evaluations of machine errors made by model IR50 only, errors made by model LightCNN only, and errors made by both models. Responses range from -2 (``No'') to +2 (``Yes'').}
%\Description{Two violin graphs with three vertical violins each. The first graph plots the participants' responses for the false positives committed by the machine: the first violin shows the distribution of participants' responses for those false positive made only by the model IR50 (mean at -1.75, interquartile range from -2.00 to -1.25), the second violin shows the distribution of participants' responses for those false positive made only by the model LightCNN (mean at -1.25, interquartile range from -1.75 to -1.00), and the third violin shows the distribution of participants' responses for those common false positive made by both models (mean at -1.00, interquartile range from -1.50 to -0.75). The second graph plots the participants' responses for the false negatives committed by the machine: first violin corresponding to false negatives made only by the model IR50 (mean at 0.00, interquartile range from -0.75 to 0.75), second violin corresponding to false negatives made only by the model LightCNN (mean at 0.00, interquartile range from -0.50 to -0.50), and third violin corresponding to false negatives made by both models (mean at -0.75, interquartile range from -1.50 to 0.00).}
\label{fig:RQ2b}
\end{figure}

% Conclusion: De nuevo, la intuición de que un humano tendrá más probabilidad de cometer un error en una tarea que ha sido erróneamente resuelta por más de un sistema de reconocimiento facial se cumple para aquéllas tareas consistentes en comparar parejas positivas. Cuando se comparan parejas negativas, aunque la certeza del humano disminuya al resolver dicha tarea, aún es capaz de resolverlo correctamente.

%RQ2(c): Are human annotators' perception of similarity and machine similarity score correlated? \ref{fig:RQ2c}

We examined human annotators' perception of similarity and compared them with model-computed similarity scores.
%
This time we distinguished between eight overlapping categories of human and model errors and successes: \{ Human, Machine \} $\times$ \{ True Positives, False Positives, True Negatives, False Negatives \}.
%
When both models and annotators gave correct responses, there were differences between the machine similarity score and annotator's perception of similarity (see blue violins in Figure \ref{fig:RQ2c}).
%
We found significant differences between both similarities for positive cases ($p \ll 0.0001$), and for negative cases ($p \ll 0.0001$).

This analysis reveals differences in the distribution of machine similarities, which tend to be bimodal and concentrated on the extremes, while human perceptions of similarity are more nuanced and dispersed (RQ2c). 
%
We found significant differences when both models and annotators gave incorrect responses (see orange violins in Figure \ref{fig:RQ2c}) for negative pairs ($p \ll 0.0001$), %(K = 0.304, p = 2.019e-07),
but not for positive pairs ($p=0.35$).
%
In the case of False Positives, annotators' perception of similarity when claiming a negative pair as positive tended to accumulate close to 0.5, indicating a low confidence in their answers (for comparison with machine similarity scores, human similarity 0.5 corresponds to the case ``Not sure'').
%
The yellow band around similarity 0.5 in Figure \ref{fig:RQ2c} includes machine errors that based on these observations could be predicted in advance as potential errors.
% Conclusion: solo cuando máquina y humano cometen un falso negativo, sus respectivos índices de similaridad podrían llegar a ser similares. En todos los demás casos de aciertos y errores, humano y máquina dan respuesta basándose en índices de similaridad significativamente distintos. Además, la gran diferencia entre el puntaje de similaridad de la máquina cuando devuelve un positivo verdadero y cuando devuelve un falso verdadero podría permitir identificar potenciales falsos positivos. Dada la elevada tasa de acierto que tienen lo humanos a la hora de evaluar parejas negativas, tiene cabida proponer que una revisión manual sobre estos potenciales errores por parte de un conjunto de anotadores podría ser compensatoria.


\begin{figure}[t!]
\centering
    \includegraphics[width=0.95\textwidth]{images/RQ2c_yellow.png}
    \caption{Human annotators perception of similarity and machine similarity score for different categories of human/machine errors (in orange) and successes (in blue). Note that machine confidence can be inferred from the similarity score (the further the similarity is from 50\%, the higher the confidence). The yellow band near a similarity score of 0.5 includes machine errors that can be anticipated as possible errors.}
    %\Description{One violin graph with eight vertical violins. These violins show the distribution of similarity score associated to Machine True Positives (mean at 1.00), Machine False Positives (mean at 0.65), Human True Positives (mean at 0.70), Human False Positives (mean at 0.65), Machine True Negatives (mean at 0.00), Machine False Negatives (mean at 0.20), Human True Negatives (mean at 0.00), and Human False Negatives (mean at 0.30). There is a horizontal yellow band around similarity 0.5 that collapses with part of the violins corresponding to Machine False cases.}
    \label{fig:RQ2c}
\end{figure}

\subsection{The Role of Gender and Ethnicity (RQ3)}

%RQ3(a): Are machine learning errors on pairs of images of different gender/ethnicity unlikely to be made by human annotators, to the extent that this can be used to characterize ``human-like'' and ``non-human-like'' errors?

\begin{figure}[t!]
\centering
    %\begin{subfigure}[b]{0.4\textwidth}
        \centering
        \includegraphics[width=0.4\textwidth]{images/RQ3a_annotations.png}
         \caption{Human evaluation over False Positive machine errors. We examined pairs of images annotated as having different gender expression \textbf{or} different ethnicity appearance, compared to pairs annotated as having similar gender expression \textbf{and} similar ethnicity appearance. This indicates that annotators are less confident in differentiating between two distinct identities when they observe similarities in terms of gender expression and ethnic appearance.}
         %\Description{One violin graph with two vertical violins. The first violin corresponds to the distribution of human response for those machine false positives that were annotated as different in terms of gender expression or ethnic appearance (mean at -2.00, interquartile range from -2.00 to -1.50). The second violin corresponds to the distribution of human response for those machine false positives that were annotated as similar in terms of gender expression and ethnic appearance (mean at -1.25, interquartile range from -1.75 to -0.75).}
         \label{fig:RQ3a_annotations}
    %\end{subfigure}
\end{figure}


We examined human evaluations over two categories of False Positive errors, {\em i.e}, cases of the different people mistakenly identified by a model as being the same person.
%
We compared pairs of facial images annotated as different in terms of gender expression \textbf{or} ethnic appearance, against pairs of facial images annotated as similar in gender expression \textbf{and} similar ethnic appearance.
%
Figure \ref{fig:RQ3a_annotations} shows the results, and indicates that humans are less certain about their answer to the question on whether both images depict the same person ({\em i.e.}, more likely to indicate ``Probably not'' and less likely to indicate ``No'') for those pairs annotated as having equal or similar gender expression and similar or equal ethnicity appearance (RQ3a).
%
Differences are significant at $p \ll 0.0001$.
%We found significant differences between human evaluations over these two groups (K = 0.644, p = 4.719e-10).  suggests that this difference can be explained by the polarization of human responses when evaluating a pair of images of people with different gender labels or different ethnicity labels. In these cases, human responses to the question "Are they the same person?" are predominantly "Not". For pairs of images of people with the same gender and ethnicity labels, human responses tend to be more scattered, and centered on response "Probably not". 
%
%Conclusions: even when human confidence/certainty differs from one category to another, humans do not perform significantly better with "equal gender and ethnicity" pairs, since they are capable of correctly annotating both categories. This suggests that differences on gender or ethnicity labels is not a suitable proxy for "non-human errors".

%RQ3(b): Are human similarity perception and/or machine similarity score correlated with human perception on gender and/or ethnicity similarity?

Figure~\ref{fig:RQ3c_gender} depicts two-dimensional plots in which we compare the human perception of gender expression similarity (in the $x$ axis) against the human perception of similarity of the images (in the $y$ axis).
For the human perception of ethnicity similarity, the visual patterns are very similar (see Figure \ref{fig:RQ3c_ethnicity}).
%
%Figure~\ref{fig:RQ3c_ethnicity} does the same but for the perception of whether people perceive their ethnic appearance as similar.
%
The user interface for this question is the one showed in Figure \ref{fig:survey}.

\begin{figure}[t]
\centering
    \includegraphics[width=0.85\textwidth]{images/RQ3b_mas.png}
    \caption{Distribution of human perception of gender expression similarity (1 - Different, 5 - Equal) for different human and machine outcomes, compared to human similarity perception, and machine similarity score, respectively.}
    %\Description{Eight graphs, each showing the two-dimensional distribution of human perception on gender similarity (x axis) and {human/machine} similarity score (y axis) for every possible combination of  {Human/Machine} {True/False} {Positives/Negatives}.}
    \label{fig:RQ3c_gender}
\end{figure}

\begin{figure}[t]
\centering
\includegraphics[width=0.85\textwidth]{images/RQ3b_mas_e.png}
    \caption{Distribution of human perception of ethnicity appearance similarity (1 - Different, 5 - Equal) for different human and machine outcomes, compared to human similarity perception and machine similarity score, respectively.}
    %\Description{Eight graphs, each showing the two-dimensional distribution of human perception on ethnicity similarity (x axis) and {human/machine} similarity score (y axis) for every possible combination of  {Human/Machine} {True/False} {Positives/Negatives}.}
    \label{fig:RQ3c_ethnicity}
\end{figure}

%\begin{figure}[t]
%\centering
%\includegraphics[width=0.85\textwidth]{images/RQ3b_mas_e.png}
%    \caption{Distribution of human perception of ethnicity similarity for different human and machine outcomes, compared to human similarity perception and machine similarity score, respectively.}
%    \label{fig:RQ3c_ethnicity}
%\end{figure}

%When comparing Figure~\ref{fig:RQ3c_gender} and \ref{fig:RQ3c_ethnicity}, we can see that both elements play a role in human labeling.
%
People state that images of the same person portray the same gender expression and ethnic appearance.
%
This is more evident (less noisy) in the case of gender expression, suggesting that this signal determines more directly the human label than ethnic appearance. Another possible explanation is that ethnic appearance might be more affected by different lightning conditions in the images (RQ3b).

In our experiments, we found partial evidence of the ``other-race'' effect (see Table \ref{tab:other-race}). We calculated the error rates for the three self-ascribed ethnicities: White, %(annotators who chose "White" as the category that best described their ethnicity), 
Black, 
%(annotators who chose "Non-Arab African"), 
and Asian. % (annotators who chose "South Asian", "Southeast Asian", or "East Asian").
%
We considered only pairs of images with the same ethnicity label in both images and computed the error rate for each of these sets of pairs.
%
``White'' annotators are the most accurate when annotating images of ``White'' people, and ``Black'' annotators are the most accurate when annotating images of ``Black'' people, but this was not the case for ``Asians''.
%No ``other-race'' effect was found in the models.

\begin{table}[t]
\caption{Human and Machine error rate. First three rows are demographic groups evaluating different set of pairs. "White-white" pairs stands for pairs containing images of two people labeled as white, and so forth.}
\centering
\setlength{\tabcolsep}{10pt}
%\begin{adjustbox}{width=0.5\columnwidth,center}
\begin{tabular}{@{}lccc@{}}
\toprule
& white-white pairs  & black-black pairs & asian-asian pairs \\ \midrule
White & 0.10 & 0.45 & 0.20 \\ 
Black & 0.55 & 0.04 & 0.02 \\
Asian & 0.18 & 0.06 & 0.09 \\
Machine & 0.09 & 0.07 & 0.09 \\ \bottomrule
\end{tabular}
%\end{adjustbox}
\label{tab:other-race}
\end{table}
%

\subsection{Exploratory study of error-based human-machine collaboration (RQ4)}

We conducted a study with the intention of illustrating the consequences of applying a human supervision strategy based on the results previously obtained. We studied the improvement over model accuracy that would result from manually reviewing the pairs evaluated by the machine. As explained in the experimental settings, the ``machine accuracy'' is the accuracy resulting from the combined performance between the IR50 and LightCNN models. Under these considerations, the accuracy achieved by both models jointly is 93.5\%. 

The first improvement is based on  the results obtained related to RQ2c: the use of machine confidence to prioritize those cases that have a high probability of being corrected by the human annotator. Note that machine confidence can be inferred from the similarity score (the further the similarity is from 50\%, the higher the confidence, see Figure \ref{fig:RQ2c}). The evolution of joint accuracy when this prioritization is implemented can be seen in the colored line in Figure \ref{fig:pilot_similarity}. We can observe that the pairs that human annotators are able to solve correctly are concentrated at the beginning of the workflow, leading to an early and rapid growth of the joint accuracy. This marked improvement in accuracy contrasts with the results we would obtain if this strategy were not taken into account (see the black line in Figure \ref{fig:pilot_similarity}).

\begin{figure}[t!]
\centering
    \begin{subfigure}[b]{0.45\textwidth}
         \centering
         \includegraphics[width=\textwidth]{images/pilot_similarity.png}
         \caption{}
         \label{fig:pilot_similarity}
    \end{subfigure}
    \begin{subfigure}[b]{0.45\textwidth}
         \centering
         \includegraphics[width=\textwidth]{images/pilot_similarity_both.png}
         \caption{}
         \label{fig:pilot_similarity_both}
    \end{subfigure}
    \caption{Human evaluation of pairs classified by the machine. On the left, (a) shows the evolution of joint human-machine accuracy when annotators evaluate pairs in increasing order of machine certainty, inferred from machine similarity score. On the right, (b) shows the evolution of joint human-machine accuracy if, in addition to the previous strategy, we prioritized those pairs where the models gave different answers (blue line). This accuracy exceeds the joint accuracy obtained when this priority is not taken into account (orange line).
    %
    The initial machine accuracy was 93.5\%. Cost represents the rate of the number of annotators.}
    %\Description{Two line graphs where x axis represents the cost of the pilot study, and y axis represents the joint human-machine accuracy. First graph shows two lines. The one attributed to the collaboration strategy proposed in this section has a sharply increasing trend for the first cost values, while the line attributed to no strategy has a low increasing.}
    \label{fig:pilot_similarity_}
\end{figure}

The second improvement is based on the results obtained when investigating RQ2b: prioritizing those pairs where the models gave different answers, {\em i.e.}, only one of the two models correctly classified the pair. As we can see in Figure \ref{fig:pilot_similarity_both}, the joint accuracy obtained if these pairs are prioritized (blue line) exceeds the joint accuracy obtained when this priority is not implemented (orange line) during most of the human annotation flow, especially at the beginning.

\section{Discussion}
\label{sec:discussion}
This study comparing human errors and machine errors allows us to develop strategies for enhancing the efficiency and effectiveness of human-computer collaboration in the domain of face recognition tasks. 
%
In particular, we should try to make the most of human capabilities to complement machine deficits, and viceversa.
%More specifically, in a system composed of algorithmic and human agents that should make decisions in a joint manner, it is desirable that the decision to leave the final decision to the human agent is not only based on whether the model is able to solve the task by meeting the stated requirements, but it is also desirable to base the choice on making the most of the human capabilities that complement this machine deficit.
%
Three main observations can be deduced from the results obtained by investigating the first three research questions posed at the beginning of this paper.

First, sufficient consistency was observed in both the annotations provided by the humans and the responses generated by the model. The consistency in human responses enables us to identify ``non-human-like" errors, which are errors that models make but are unlikely to be made by humans.
%
%An example of these errors is seen in false positive pairs.
In our setting, these are false positive pairs.
%
When examining the correlation between human errors and machine errors, it was observed that when the machine makes a mistake, humans are more likely to make a mistake as well, particularly in the case of positive pairs, where humans have a high error rate in comparison with the machine false negative error rate. However, in negative pairs, although human confidence decreases for those cases where the machine fails, human responses are mostly accurate. While humans encounter more challenges with machine false negatives compared to true positives, humans seem to find no challenge with machine false positives, suggesting that the machine struggles with certain negative pairs, while humans do not find them difficult.

Next, when we categorize machine errors into those occurring in just one of the two models and those occurring in both models, we observe two distinct patterns for positive and negative pairs when compared to human assessments. In the case of a machine false negative pair, humans are notably more prone to error when that false negative is committed by both models. This reveals a correlation between the challenges faced by both models and those faced by humans when assessing positive pairs. However, when a human assesses a machine false positive pair, the probability of error is not significantly influenced by whether the error is common to both models or not, which is consistent with the observation above. However, what is significantly influenced is the change in human certainty, which is smaller, but still correct.

These two observations indicate that (1) humans have a significantly better capacity to distinguish negative pairs compared to machines. Of the 126 false positives made by the machine, humans made only 6 errors (4.8\%), and successfully identified all negative pairs correctly classified by the models. And (2) humans have a significantly better capacity to correctly classify those pairs in which both models disagree, over those pairs in which both models are wrong.

Finally, we observed that only when humans and machine make an error by failing to detect a positive pair, their similarity scores could become similar. In all other scenarios of correct and incorrect identifications, the human and the machine provide responses based on notably different similarity scores. Moreover, there is a substantial disparity between the machine's similarity rating for a correct identification and the machine's similarity score for an incorrect one. This difference is even more pronounced when the machine classifies a pair as positive. 
This, combined with the high accuracy mentioned above of humans over machines in evaluating certain pairs, could help to anticipate potential errors and suggest that a manual examination of these cases by a group of annotators could be beneficial.

Based on these observations, following the suggested strategies we could improve the accuracy of the system by approximately 3 percentage points by assuming 10\% of the total cost only ({\em i.e.}, by assuming the cost of 546 human annotations). This improvement of 3\% is equivalent in our case to the correction of 148 pairs (98 negative and 50 positive) misclassified by the machine, which outweighs the improvement of just 0.4\% (equivalent to 27 corrections, 19 negative and 8 positive pairs) that we would achieve if we did not follow the proposed strategies. This study is a clear example of how the findings of such comparative and exploratory analysis can facilitate the proposal of simple but powerful human-machine interaction paradigms.

\subsection{Limitations and future work}
Approaching more real-world use cases also highlights a possible limitation of the work developed here. In use cases for face recognition technologies, the nature of the domain determines under which thresholds of similarity score (and, therefore, machine confidence) the machine's response is considered positive or negative. In cases where, for example, it is desirable to prioritize the reduction of false positives without the possible increase of false negatives being detrimental ({\em e.g.}, face recognition methods for private access controls \cite{ibrahim2011study}), the similarity score is set at a higher value than in other scenarios where it is desirable to prioritize the reduction of false negatives ({\it e.g.}, face recognition methods for law enforcement \cite{raposo2023use}). This work has been approached from a symmetric costs point of view (with a threshold for the similarity score of 50\%) and thus serves as an example or starting point for possible different scenarios to be readjusted.

Also, the error behavior of a system (and beyond errors, its overall behavior) is highly dependent on the training data and the architecture of the chosen model. In this work we have chosen two specific pre-trained models (IR-50 \cite{he2016deep} and LightCNN \cite{wu2018light}), well known in the literature, taking care that the training was based on the same dataset MS-Celeb-1M \cite{guo2016ms} for a fair comparison. In this sense, RQ1a and RQ1b in \S\ref{subsec:consistency} could be interpreted, rather than as research questions, as prerequisites. The consistency alluded in RQ1 allows us to propose an error characterization, a differentiating axis between human errors and machine errors, on which the exploration and comparisons developed in this work are then based. It is therefore important to bear in mind that in other different scenarios this condition might not be present.

%\textbf{TO DO}: IR-50 and LightCNN are old... IGNORE: models are from 2018 and 2019 respectively, tbh I don't think they are old
A possible future work is based on revisiting some of the biases that may occur in a human-machine interaction scenario taking into account the results of this analysis. Biases such as algorithmic aversion, overconfidence, or confirmation bias can vary significantly depending on whether the resolution offered by the machine is more or less similar to the resolution that a human agent could offer. More specifically, our results suggest that machine aversion is more likely to be found in scenarios where minimizing false negatives is prioritized, as this will increase the proportion of false positives and, as this is a rare error in humans, may cause more rejection.

\section{Conclusions}
\label{sec:conclusions}
The main conclusions drawn from this work are the following: 
\begin{enumerate}
    \item The facial recognition models shows a marked disparity in similarity scores between correctly and incorrectly resolved pairs.
    \item There is a correlation between the shared challenges faced by the models (errors made by both models and not just one) and the difficulties experienced by humans, whereas humans encounter fewer issues when classifying pairs where the models provided different results.
    \item Humans perform substantially better than facial recognition models in assessing negative pairs (pairs consisting of different identities).
\end{enumerate}

Observation (1) enabled us to detect potential errors in the facial recognition models, while observations (2) and (3) helped us prioritize those potential errors that a human annotator has a high chance of correcting. Implementing this in practice allowed us to design a manual evaluation strategy that achieves maximum joint human-machine precision with a very low number of annotations.

In addition to the quantitative improvements shown in this work, it is worth paying attention to the impact that some of these conclusions could have on facial recognition tasks in real-world contexts.
As we saw in the human-machine collaboration paradigm proposed above, most of the machine errors corrected by a human annotator are negative pairs that were predicted as positive by the model. 
%This aligns with the finding that humans are significantly more effective at accurately resolving negative pairs.
%
It is worth noting that this should not only be taken into account when a face recognition system is already in the development or deployment phase, but also when evaluating the suitability of integrating an automatic face recognition system in the specific application domain.
%
In scenarios where the occurrence of false positives might have serious consequences and potentially affect fundamental rights, if there is a concern of lack of adequate human oversight, the integration of facial recognition technologies demands a rigorous and thoughtful reconsideration.
%In situations where false positives might result in breaches of human rights, and there is a concerning lack of sufficient human oversight, the implementation of facial recognition technologies requires careful and conscientious reevaluation.

In use cases where the resolution of face recognition tasks by a machine learning system can be conveniently monitored by human reviewers, it is still imperative to implement oversight strategies that acknowledge and address the disparate error patterns exhibited by humans and machines. 
%

A noteworthy aspect is the observation that the human advantage over machines in assessing negative pairs might be linked to the perception that humans have built on notions of similarity and difference in gender expression and ethnic appearance. 
%
Given a pair of images corresponding to two different identities, if a human makes the mistake of saying that they are the same person (which, as we have seen, happens infrequently), it does so in the belief that both identities share a similar gender expression and ethnic appearance. 
%But over the machine's false positive pairs, human perception of gender similarity is more dispersed, as is the machine's confidence when compared to the human's confidence. Furthermore,
When the human correctly classifies a negative pair that was classified as positive by the machine, both perceptions of gender and ethnicity seem to play a distinctive role in the final human decision.
This apparent human tendency to use gender and ethnicity-related characteristics to differentiate negative pairs could be due not only to gender and racial stereotypes perpetuated in society, but also to the predominant presence of stereotypical images in face recognition databases \cite{dominguez2022gender, keyes2018misgendering}.

Finally, our results suggest that facial recognition algorithms are not advanced enough to fully replace human roles in real world scenarios. This may also not be desirable, especially in light of the recent ethical and legal concerns that have been raised about the use of this technology. The current draft of the EU AI Act \cite{EUAIAct} contains many explicit and implicit allusions to facial processing, whose applications are considered at different risk levels, including \textit{high risk} and \textit{forbidden}. This envisions a future scenario for face recognition technologies in which permanent human oversight will be essential, highlighting the value of preserving human input in decision-making.

\section*{Acknowledgement}
MCIN/AEI/10.13039/501100011033 under the Maria de Maeztu Units of Excellence Programme (CEX2021-001195-M).
%% The Appendices part is started with the command \appendix;
%% appendix sections are then done as normal sections
%%\appendix
%\section{Example Appendix Section}
%\label{app1}

%Appendix text.

%% For citations use: 
%%       \cite{<label>} ==> [1]

%%

%% If you have bib database file and want bibtex to generate the
%% bibitems, please use
%%
\bibliographystyle{elsarticle-num} 
\bibliography{mybib.bib}

%% else use the following coding to input the bibitems directly in the
%% TeX file.

%% Refer following link for more details about bibliography and citations.
%% https://en.wikibooks.org/wiki/LaTeX/Bibliography_Management

%\begin{thebibliography}{00}

%% For numbered reference style
%% \bibitem{label}
%% Text of bibliographic item

%%\bibitem{lamport94}
%%  Leslie Lamport,
%%  \textit{\LaTeX: a document preparation system},
%%  Addison Wesley, Massachusetts,
%%  2nd edition,
%%  1994.

%\end{thebibliography}
\end{document}

\endinput
%%
%% End of file `elsarticle-template-num.tex'.
