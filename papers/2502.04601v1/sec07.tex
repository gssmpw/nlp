\begin{figure*}[h]
\centering
  \includegraphics[width=\textwidth]{Figures/Results/TrainingResults-1.png} 
  \vspace{-0.23in}
  \caption{Vanilla vs GOOD based asynchronous federated learning.}
  \label{fig:detection_correction} 
  \vspace{-0.15in}
\end{figure*}

\section{SECURITY ANALYSIS}
\label{formalsec} 
We prove the security of our constructions in the well-known Universal Composability (UC) framework~\cite{Canetti01}. The UC paradigm captures the conditions under which a given distributed protocol is secure, by comparing it to an ideal realization of the protocol. To accomplish this, the UC framework defines two “worlds”, firstly, a real-world, where the protocol $\pi$, to be proved secure, runs in the presence of a real-world adversary, $\adv$. Secondly, the ideal world, where the entire protocol $\phi$ is executed by an ideal and trusted functionality, in the presence of a simulator, $\simu$, which models the ideal-world adversary. All users only talk to an ideal functionality via secure and authenticated channels, the ideal functionality takes input from users, performs computations and other operations, and returns the output to the calling users. Our goal is to prove that no entity represented by $\env$, can successfully distinguish between the execution of the two worlds.

\subsection{Design of Ideal Functionalities}
We define an ideal functionality, $\fdctc$, consisting of five independent ideal functionalities, $\fregister, \fresp, \ftee, %
\fsmt, \fsig$. %
Specifically, $\fregister$ models the service setup, coordinator enclave and aggregate enclave onboarding processes, as well as encryption and decryption functionality for symmetric key encryption and attribute-based encryption. $\fresp$ models the user's request for the aggregation service and the response of the edge server to the user. We use the helper functionality $\ftee$~\cite{pass2017formal} to model the ideal functionality of a secure enclave.%
We also use two other helper functionalities, $\fsig$~\cite{canetti2004universally} and $\fsmt$~\cite{Canetti01}, to model ideal functionalities for digital signatures and secure/authenticated channels, respectively.
We assume that $\fdctc$ maintains an internal state that is accessible at any time to all the ideal functionalities.
We describe the functionalities of $\fdctc$, discuss some of their motivating design choices, and give the proof of the following theorem in Appendix~\ref{sec:full-sec-analysis}.


\begin{theorem}
\label{thm:uc}
 Let $\fdctc$ %
 be an ideal functionality for \sysname. Let $\adv$ be a probabilistic polynomial-time (PPT) adversary for \sysname, and let $\simu$ be an ideal-world PPT simulator for $\fdctc$. %
 \sysname UC-realizes $\fdctc$ for any PPT distinguishing environment $\env$.
\end{theorem}
