\section{Design Overview}
\label{sec04}
We propose a novel AsyncFL framework called \sysname (Figure~\ref{fig:overview}), that integrates three key components: a client-side gradient obfuscation mechanism to safeguard client data, a secure aggregation service mesh for efficient (aggregation) enclave deployment at the edge, and an implicit attestation mechanism for scalable and vendor-agnostic enclave verification. Together, these components satisfy the following design goals (DGs):
    
\noindent $\bullet$ \,{\bf DG 1:} Obfuscate privacy-sensitive information from global model updates while maintaining performance with minimal overhead. 

\noindent $\bullet$ \,{\bf DG 2:} To develop a secure aggregation solution that protects against a malicious server exfiltrating client information, is not bound to a specific hardware vendor, and can utilize TEE hardware.

\noindent $\bullet$ \,{\bf DG 3:} To enable vendor-agnostic enclave attestation, allowing the participating client to authenticate the aggregator irrespective of the TEE provider (\eg Intel SGX or Nvidia Confidential Computing).

\noindent $\bullet$ \,{\bf DG 4:} To develop a more scalable and asynchronous authentication solution than current state-of-the-art frameworks.



    
\subsection{TEE Mesh-based Secure Aggregation (TMSA)}
\label{subsec04-01}
The first component of the \sysname~framework is {\it TEE mesh-based secure aggregation (TMSA)}, which is a hierarchical enclave structure. The key feature of TMSA is the {\bf abstraction of all hardware-specific attestation primitives to make them vendor independent} and provisioning credentials, and these primitives to a central attestation enclave. 
The first layer of TMSA, the \textit{Coordinator Enclave}, centrally handles all attestation and provisioning required to deploy TMSA's second layer. %
Once the coordinator enclave is deployed and attested by the service provider, it is then given the attestation primitives and can attest and deploy the second layer, the \textit{Aggregation Enclave}.
This second layer, responsible for ML aggregation tasks, spans across multiple servers and utilizes various attestation primitives and methods.

The initial provisioning phase, with the deployment of the coordinator enclave, allows the service provider to utilize cross-platform aggregation enclaves at the aggregation server during the training stage without actively participating in the hardware attestation process. Moreover, running the aggregation process inside an enclave protects client-sensitive gradient updates and metadata from a malicious server. These properties satisfy \textbf{DG 2}. The design of TMSA is detailed in Section~\ref{subsec05-01}.


\subsection{Data-centric Attestation (DCA)}
\label{subsec04-02}
The second component of \sysname~is a novel {\it data-centric attestation mechanism (DCAM)}, which allows clients to attest the aggregation enclaves without relying on any enclave-specific measurements or collaterals using a MABE construct~\cite{ChaCho09}. %
In particular, aggregation enclaves are provisioned with private attributes, \ie credentials needed for secure aggregation, only after the coordinator enclave has successfully attested them. 
This ensures that only trusted enclaves will be provisioned with the correct attributes for aggregation. This enables the key feature of DCAM, which is {\bf allowing the client to perform an implicit attestation of the aggregation enclave} by virtue of the enclave's possession of the correct attributes.
This allows the client to perform the attestation without relying on any hardware-specific or vendor-specific measurement or information. 
The MABE construct will be used in a hybrid encryption scheme to achieve cost-effective secure communication between the client and the aggregation enclave. The client utilizes the attributes provisioned to the aggregation enclave to securely share a symmetric key (\eg AES) with the aggregation enclave for secure data sharing. 
Performing implicit attestation eliminates reliance on enclave-specific measurements or collateral, removing the dependency on specific TEE providers and hardware architectures. Moreover, replacing TLS with MABE-based hybrid encryption speeds up the attestation process, as detailed in Section~\ref{subsec05-02}. These properties satisfy \textbf{DG 3} and \textbf{DG 4}. %



\subsection{Gradient Obfuscation via Orthogonal Distribution (GOOD)} 
\label{subsec04-03}
The final component of \sysname is a novel mechanism, called {\it Gradient Obfuscation via Orthogonal Distribution} (GOOD), designed to protect the privacy of gradient updates. The key feature of GOOD is \textbf{obfuscating the client's private gradient updates by entangling them with the gradients of an orthogonal distribution}--a distribution that shares minimal overlap with the distribution of the primary task. To achieve this, GOOD employs a loss function derived from a modified version of the Soft Nearest Neighbor Loss (SNNL)~\cite{FroPapHin19}. This loss ensures the entanglement between the primary and the orthogonal tasks, making it infeasible to disentangle the gradients of them without any strong priors, enhancing privacy.

The orthogonal distribution may consist of natural images, such as those from the GTSRB dataset~\cite{stallkamp2012man}, or their synthetic variations~\cite{synthImages}. However, in many real-world training scenarios, the availability of these images may be infeasible, or their generation may become prohibitively expensive due to the computational demands of adversarial networks and diffusion models~\cite{ulhaq2022efficient,chen2024opportunities}. To address this, we leverage fractal images generated through Iterated Function Systems (IFS)~\cite{AndFar22}, which offer a scalable and efficient method for dynamically creating the desired orthogonal distribution, either during training or in an offline phase. The low cost of generating fractal distributions enables their real-time generation on client devices with minimal overhead~\cite{chu2000fast,brown2010highly}, thereby satisfying \textbf{DG 1}. The design of GOOD is detailed in Section~\ref{subsec05-03}.
