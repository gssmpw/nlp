




\subsection{Proof}

We now prove Theorem~\ref{thm:uc}.

\begin{proof}
We give a series of games, each of which is indistinguishable from its predecessor by a PPT $\env$.

$\gzero$: This is the same as the real-world \sysname. $\env$ interacts directly with \sysname and $\adv$.

$\gone$: $\simu$ internally runs $\adv$ and simulates the secure and authenticated channels functionality $\fsmt$. 
    \begin{lemma}
    \label{lem:lem1}
    For all PPT adversaries $\adv$ and PPT environments $\env$, there exists a simulator $\simu$ such that 
    $$ \execgzero \approx \execgone
    $$ \end{lemma}
    The two games are trivially indistinguishable since $\simu$ just executes the simulator for $\fsmt$.

$\gtwo$: $\simu$ communicates with the honest parties and $\adv$, and simulates the protocols of \sysname with the help of $\fdctc$. $\adv$ can corrupt any user or $EC$ at any point in time by sending a message ``corrupt'' to them. Once an entity is corrupted, all their information is sent to $\adv$ and all further communication to and from the corrupted party is routed through $\adv$. We now state and prove the following lemma:

\begin{lemma}
\label{lem:lem2}
For all PPT adversaries $\adv$ and PPT environments $\env$, there exists a simulator $\simu$ such that  
$$\execgone \approx \execgtwo$$
\end{lemma}

 After system setup, $SP$s can independently call the ABE setup function in $\fregister$ to set up their corresponding public keys and attributes. The attributes act as descriptors for specific services aggregation enclaves and coordinator enclaves will provide to users and these attribute-specific keys will be accessible to the enclave applications through RA-TLS connection in the real world, which is modeled in the ideal world by $\ftee$ accessing the $\abesktable$. $SP$s would then create their individual symmetric keys $K_A$ and $K_C$ by calling the $\symmenc$ function in $\fregister$ for encrypting their aggregator and coordinator apps, respectively. $SP$s will call the $\symmenc$ function and pass their coordinator and aggregator apps to generate $encCoorApp$ and $encAggApp$. Service providers call the $\mathrm{SP\_setup}$ function of $\fregister$ to store their encrypted applications and $BootApp$ in $\spidapptable$. 

Edge server entities in $\mathbb{E}$ which will provide coordinator services can call the $registerCoorEnc$ function of $\fregister$ to receive the $BootApp$ ($bA$) and the $encCoorApp$ for a specific service provider $SP \in \mathbb{SP}$ identified by $\spid$. 
Edge servers that will provide aggregation services can call the $registerAggEnc$ function of $\fregister$ to receive the $BootApp$ ($bA$) and the $encAggApp$ for a specific service provider $SP \in \mathbb{SP}$ identified by $\spid$. 
After receiving the $bA$ and the encrypted enclave app, the edge servers will initialize the bootapp using $\ftee$. In the real world, the enclave connects to the service provider to participate in remote attestation to verify that it has been initialized correctly before receiving the symmetric key $K_C$ or $K_A$ to decrypt the encrypted coordinator app or aggregator app, respectively. In the ideal world, $\ftee$ has access to the internal state maintained by all other functionalities, and hence, $\ftee$ can retrieve the $K_C$ and $K_A$ stored by the service provider in the $\spidapptable$. All the steps inside the aggregation and coordinator enclave in the ideal world as executed by $\ftee$ and the edge server has no access to the internal state of the $\ftee$. When an aggregation enclave is initialized, the verification in the real world that occurs between the aggregation enclave and coordinator enclave is handled within $\ftee$ as well, which models the secure communication between secure enclaves in the real world since no user interaction is involved during this stage.

When a user $u$ in the system needs to create a request for the aggregation service, they will first access the ABE public keys for the specific service provider. In the real world, this data is publicly accessible and this is modeled in the ideal world by calling the $retrieve\_mpks$ function which returns the public keys as well as the attributes for all the $SP$s in the system. The user $u$ generates a symmetric key $K_U$ using $\fregister$ and calls $\symmenc$ to encrypt their data using $K_U$, which results in $C_1$. The user further encrypts $K_U$ using the $\abeenc$ function of $\fregister$ using the service provider public key and the attribute associated with the aggregation service, resulting in $C_2$. Both the encrypted data are then sent to an aggregation server $e$ by calling the $userRequest$ function provided by $\fresp$. The edge server $e$ has to load the encrypted data into the enclave, which in the ideal world requires sending the data to $\ftee$ which is currently executing the $AggApp$ for $e$. 
$\ftee$ internally decrypts $C_2$ using the $\abedec$ function and then uses the decrypted key $K_U$ to access user data in $C_1$. The result of the computation is encrypted by $\ftee$ using $K_U$  and sent to the edge server $e$. 
At this point, $e$ calls the $serverResponse$ function of $\fresp$ and forwards the encrypted response from the enclave to $\fresp$ who forwards the response to $u$. 
Since $u$ has access to $K_U$, it calls the $\symmdec$ function in $\fregister$ to decrypt the response and concludes the protocol.

We note that this proof only accounts for the communication security aspect of \sysname and does not handle reconstruction attacks the user can mount given its access to the response from the aggregation service. The ideal world adversary can only try to access the user data before it is passed to the enclave and the responses from the aggregation enclave. Since this information is protected with symmetric key encryption and attribute-based encryption, as long as the security guarantees of the encryption schemes hold and the trusted enclave is secure, the adversary cannot access the user data which is only available to the enclaves and the individual users in the system.   

 


















     
    


   
    
    
    
    
    
    
    
    

    \qed

\end{proof}




