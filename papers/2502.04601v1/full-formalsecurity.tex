
 



\section{\sysname Security Analysis}
\label{sec:full-sec-analysis} 
\subsection{Formal Security Analysis}
We now provide a formal analysis of \sysname in the Universal Composability (UC) Framework~\cite{Canetti01}. The notion of UC security is captured by the pair of definitions below:

\begin{definition}{(UC-emulation~\cite{Canetti01})}
\label{def:ucdef1}
Let $\pi$ and $\phi$ be probabilistic polynomial-time (PPT) protocols. We say that $\pi$ UC-emulates $\phi$ if for any PPT adversary $\adv$ there exists a PPT adversary $\simu$ such that for any balanced PPT environment $\env$ we have
$$ \mathrm{EXEC}_{\phi, \simu, \env}  \approx \mathrm{EXEC}_{\pi, \adv, \env} $$
\end{definition}
 
\begin{definition}{(UC-realization~\cite{Canetti01})}
\label{def:ucdef2}
Let $\mathcal{F}$ be an ideal functionality and let $\pi$ be a protocol. We say that $\pi$ UC-realizes $\mathcal{F}$ if $\pi$ UC-emulates the ideal protocol for $\mathcal{F}$.
\end{definition}

We define an ideal functionality, $\fdctc$, consisting of five independent ideal functionalities, $\fregister, \fresp, \ftee%
, \fsmt, \fsig$%
.
The parties that interact with the ideal functionalities are the members of sets of service providers, $\mathbb{SP}$, edge servers with enclave execution capability, $\mathbb{E}$, %
and users $\mathbb{U}$. We assume that each member of the sets has a unique identifier.
We assume that $\fdctc$ maintains an internal state that is accessible at any time to all the ideal functionalities, specifically  %
tables,  $\stable$, $\spidapptable$, $\ABEetable$, $\abesktable$, $\symmktable$, and $\symmetable$. %
$\stable$ contains service providers information that an edge server provides services on behalf of, such as the set of attributes used by the enclaves and the service provider's ABE public key. $\spidapptable$ contains encrypted apps and information stored by each service provider that are 


\begin{figure}[H]
\begin{mdframed}
\begin{center}
\textbf{Functionality} $\fregister$
\end{center}


\begin{enumerate}%

\item \textbf{system setup:} on receiving request $(systemsetup,\mathbb{SP},\lambda,sid)$ from $\env$, $\fregister$ sets the system parameter $\lambda$, the set of service providers in the system  as $\mathbb{SP}$, set of edge servers $\mathbb{E}$, and set of users as $\mathbb{U}$. %

\item \textbf{ABE setup:} on receiving request $(\abesetup,$ $\spid, [a1,\dots,$ $an],$ $sid)$ from a $SP \in \mathbb{SP}$ identified by $\spid$, $\fregister$  %
picks $mpk_{\spid} \sample \{0,1\}^{\lambda}$ and adds 
$(\spid,[a1,\dots, an],$ $mpk_{\spid})$ 
to $\stable$. It returns $(\spid,[a1,\dots, an],mpk_{\spid})$ to $\simu$ and $\adv$. %

\item \textbf{ABE keygen:} On receiving request $(\abekeygen,[ai],sid)$ from a $SP \in \mathbb{SP}$ identified by $\spid$, $\F{register}$ checks if $(\spid,[\dots ai,\dots],mpk_{\spid})$ exists in $\stable$, if so, it picks $sk_{a1} \sample \{0,1\}^{\lambda}$, stores $(\spid, sk_{a1},a1)$ in $\abesktable$, and returns $sk_{a1}$ to $\spid$.

\item \textbf{ABE encryption:} On receiving request $(\abeenc,policy,mpk_{\spid},m,sid)$ from user $u$, $\F{register}$ checks if a tuple $(\cdot,[a1,\dots, an],mpk_{\spid})$ exists in $\stable$ and that attributes in $policy$ also exist in $[a1,\dots,an]$. If not, return $\bot$, else it picks $c \sample \{0,1\}^{\lambda}$, and adds  $(mpk_{\spid},policy,c,m)$ to $\ABEetable$ and returns $c$ to $u$, $\adv$, and $\simu$. 

\item \textbf{ABE decryption:} On receiving request $(\abedec,[sk_{aj},\dots,sk_{ak}],policy,c,sid)$ from user $u$, $\F{register}$ checks if a tuple $(mpk_{\spid},policy,c,\cdot)$ exists in $\ABEetable$, that attributes $[sk_{aj},\dots,sk_{ak}]$ satisfy $policy$, and  there exist tuples $(\cdot,sk_{ai},ai)$ in $\abesktable$ for each $sk_{ai} \in [sk_{aj},\dots,sk_{ak}]$. %
If previous checks pass then it retrieves the tuple $(mpk_{\spid},policy,c,m)$ from $\ABEetable$ and returns $m$ to $u$, else return $\bot$.

\item \textbf{Symmetric key generation:} On receiving request $(\symmkeygen,sid)$ from user $u$, $\F{register}$ picks $k \sample \{0,1\}^{\lambda}$, and adds  $(k)$ to $\symmktable$ and returns $k$ to user and $\simu$. %

\item \textbf{Symmetric key encryption:} On receiving request $(\symmenc,k,m,sid)$ from user $u$, if $k$  exists in $\symmktable$, $\F{register}$ picks $c \sample \{0,1\}^{\lambda}$, and adds  $(k,c,m)$ to $\symmetable$ and returns $c$ to $u$, $\adv$, and $\simu$. Else return $\bot$. %

\item \textbf{Symmetric key decryption:} On receiving request $(\symmdec,k,c,sid)$ from user $u$,  $\F{register}$ checks if a tuple $(k,c,\cdot)$ exists in $\symmetable$, if so it retrieves the tuple $(k,c,m)$ and returns $m$ to $u$, else return $\bot$.%

\item \textbf{Service provider setup:} on receiving request $(\mathrm{SP\_setup},$ $\spid,$ $encCoorApp,K_C,$  $encAggApp,K_A,$$bA, attestPrim,sid)$ from a $SP \in \mathbb{SP}$ identified by $\spid$, $\fregister$ adds $(\spid, encCoorApp, K_C$ $encAggApp, K_A,bA, spec)$ to $\spidapptable$.

\end{enumerate}


\end{mdframed}
\vspace{-0.15in}
\caption{Ideal functionality for Service Registration}
\label{fig:ucregister1}
\end{figure}


\noindent sent to edge servers when they register for a service. $\abesktable$ stores the secret keys for the ABE attributes generated by the different service providers and $\ABEetable$ stores the ABE encrypted messages, thus helping us simulate ABE functionality in the ideal world. $\symmetable$ and $\symmktable$ help realize symmetric key encryption operations in the ideal world by storing ciphertexts and symmetric keys respectively.  %
Lastly, $\etable$ helps keep track of edge servers registered with service providers to provide their services.

We note that the service providers in \sysname are trusted entities and do not act maliciously and that the users in the system are not authenticated. Any user in the system has access to all service providers' public keys and can encrypt an aggregation request for an aggregation enclave. The edge servers running the coordinator enclave and the aggregation enclave are honest but curious entities and will try to undermine the system security by accessing the user data that the enclaves use. We note that user authentication can be easily achieved in the application of \sysname by plugging in off-the-shelf protocols such as token-based (Java Web Token) and certificate-based authentication schemes.
We now briefly describe the design of our ideal functionalities. 


 



\addtocounter{figure}{-1}

\begin{figure}[h!]
\begin{mdframed}
\begin{center}
\textbf{Functionality} $\fregister$
\end{center}

\begin{enumerate}\addtocounter{enumi}{9}%




\item \textbf{Get all public keys:} on receiving request $(retrieve\_mpks, sid)$ from user $u$,  $\fregister$ retrieves all tuples $(\cdot,[a1 \dots an],mpk_{\spid})$ from $\stable$, generates a list $[(mpk_{\spid},[a1,\dots]),\dots]$ %
with each $mpk_{\spid}$ in the list corresponding to an $SP \in \mathbb{SP}$ and the corresponding attribute list for the $SP$. $\fregister$ sends list to $u$, $\adv$, and $\simu$.

\item \textbf{Coordinator edge server registration:} On receiving request $(registerCoorEnc$$,$$\spid,$$e,$$sid)$ from $e$, $\fregister$ checks if a tuple $(\spid,$ $encCoorApp,$ $\cdot,$$encAggApp,$$\cdot,$$bA,$$attestPrim)$ exists in $\spidapptable$, if true, sends $(encCoorApp,$ $bA)$ to $e$, $\simu$, and $\adv$, else return $\bot$. 

\item \textbf{Aggregation edge server registration:} %
On receiving request $(registerAggEnc,$ $\spid,$ $pid,$ $sid)$ from $e$, $\fregister$ retrieves $(\spid, encCoorApp,\cdot,encAggApp,\cdot,bA,attestPrim)$ from $\spidapptable$,  %
and sends $(encAggApp,bA)$  to $e$, $\simu$, and $\adv$, else return $\bot$. 


\end{enumerate}
\end{mdframed}
\vspace{-0.15in}
\caption{Ideal functionality for Service Registration}
\label{fig:ucregister2}
\end{figure}


















$\underline{\fregister}$: The $\fregister$ functionality shown in Figure~\ref{fig:ucregister1} handles the system setup, setup of service providers, coordinator and aggregation enclave server registrations, and functions to support attribute-based encryption functionality and symmetric key encryption functionality. %
$\env$ first initializes the system by sending the tuple $(systemsetup,\mathbb{SP},\lambda,sid)$ to $\fregister$. This initializes the set of entities ($\mathbb{SP}$) in the system that act as service providers, $\mathbb{E}$ as the set of edge servers, and the set of users in the system $\mathbb{U}$. 
Each service provider in the system, $SP \in \mathbb{SP}$,  contacts $\fregister$ by sending a tuple  $(\abesetup,$ $\spid, [a1,\dots,$ $an],$$sid)$  where $\spid$ denotes the unique identifier of $SP$ and $[a1,\dots,$ $an]$ denotes the list of attributes for which the $SP$ wants to generate ABE keys corresponding to the services the $SP$ is offering. %
$\fregister$ stores the information in $\stable$ and returns a unique $mpk_{\spid}$ to $\spid$. This function allows for overwriting a previous tuple in $\stable$ to account for $SP$s changing the attribute set or services they offer. %
$SP$s send $(\abekeygen,[ai],sid)$ tuple to $\fregister$ to generate secret keys associated with the attribute $ai$. $\fregister$ checks $\stable$ for the existence of the attribute $ai$ before generating the key and returning it to $SP$ identified by $\spid$. Any user in the system can call $\fregister$ with tuple $(\abeenc,policy,$ $mpk_{\spid},$ $m,sid)$ to get a message $m$ encrypted using public key $mpk_{\spid}$ and an ABE policy ($policy$) containing valid attributes of $SP$ identified by $\spid$. Any user in the system with valid secret keys for attributes in a give encryption policy can call $\fregister$ with tuple $(\abedec,[sk_{aj},\dots,sk_{ak}],policy,c,sid)$. $\fregister$ checks that the supplied secret keys are valid for the given policy and have been generated previously. If all the checks pass then $\fregister$ retrieves the decrypted message from $\ABEetable$ and returns it to the calling entity. $\fregister$ also provides symmetric key generation, encryption, and decryption functions. $\fregister$ creates symmetric keys and returns them to the calling user after storing them in the $\symmktable$. For encryption, a user calls $\fregister$ with $(\symmenc,k,m,sid)$, if the key $k$ exists in the $\symmktable$, then $\fregister$ stores a ciphertext $c$ corresponding to $m$ in $\symmetable$ and returns $c$ to the calling user. Any user in the system with cipher text $c$ and a valid key $k$ corresponding to $c$ can call the decryption function of $\fregister$ by sending the tuple $(symmdec,k,c,sid)$. $\fregister$ checks for the existence of a tuple containing $c$ and $k$ in $\symmetable$, and if found, returns the corresponding plaintext message $m$, thus providing the symmetric key decryption functionality.
\par 

After creating their corresponding ABE and symmetric keys, service providers in the system store the encrypted coordinator and aggregator applications with $\fregister$ by calling $(\mathrm{SP\_setup},$ $\spid,$ $encCoorApp,K_C,$  $encAggApp,K_A,$ $bA,$ $attestPrim,$ $sid)$. The encrypted apps and $bA$ will be sent to edge servers upon their registration as coordinator or aggregation enclaves. The symmetric encryption keys $K_C$ and $K_A$ are accessed by $\ftee$ when it decrypts the $encCoorApp$ and $encAggApp$ inside the enclave, respectively. \\$attestPrim$ is information that is used by the service providers and coordinator enclaves in the real world for remote attestation of provisioned enclaves and verifying that they were set up correctly before sending decryption keys for the applications and user data.  
Any user in the system can call $\fregister$ by sending tuple $(retrieve\_mpks, sid)$ to retrieve the set of ABE public keys for all $SP$s in the system as well as the attributes associated with the public keys. 
This models the fact that all service providers' public keys are available to any user in the system in the real world. 
An edge server $e$ in the system can register for providing coordinator enclave service by sending the tuple $(registerCoorEnc,\spid,e,sid)$ to $\fregister$. $\fregister$ retrieves the encrypted coordinator app $encCoorApp$ and the boot app $bA$ and sends it to $e$. %
An edge server $e$ in the system can register for providing aggregation enclave service by sending the tuple $(registerAggEnc,\spid,e,sid)$ to $\fregister$. $\fregister$ retrieves the encrypted aggregation app, $encAggApp$ and the boot app $bA$ and sends it to $e$. %









\begin{figure}[h]
\vspace{-0.0in}
\begin{mdframed}
\begin{center}
\textbf{Functionality} $\fresp$
\end{center}
\begin{enumerate}%


\item \textbf{User request:} On receiving request $(userRequest,C1,C2,Cert_{u},e,u,sid)$ from user $u \in \mathbb{U}$, $\fresp$ returns $\bot$ to $u$ if no such $e$ exists, else forwards request to edge server $e$, $\simu$, and $\adv$. %

\item \textbf{Edge server response:} On receiving message $(serverResponse,C3,u,sid)$ from edge server $e$, $\fresp$ forwards request to user $u$, $\simu$, and $\adv$. 









 



\end{enumerate}
\end{mdframed}
\vspace{-0.15in}
\caption{Ideal functionality for Responding to Requests}
\label{fig:ucresponse}
\end{figure}








 





$\underline{\fresp}$: %
The $\fresp$ functionality shown in Figure~\ref{fig:ucresponse} handles a service request from a user identified by $u$. When $u$ submits a request to $\fresp$ for aggregation service provided by an edge server $e$, it sends a request containing $(userRequest,C1,C2,Cert_{u},e,u,sid)$. %
$C1$ and $C2$ in the request are encrypted by the user using symmetric key encryption and attribute based encryption, respectively, as in the real world. $e$ helps the UC functionality forward the request to the appropriate edge server running the aggregation enclave.
Once $\fresp$ receives the request from the user, it forwards the tuple to $e$, $\adv$, and $\simu$. If no such server $e$ exists then $\bot$ is returned to $u$. 
After processing a user $u$'s request, an edge server $e$ can call $\fresp$ by sending the tuple $(serverResponse,C3,u,sid)$, where $C3$ is a symmetric key encrypted data meant for user $u$. $\fresp$ forwards the request to user $u$, $\simu$, and $\adv$.

$\underline{\ftee}$: Figure~\ref{fig:ftee} represents the ideal functionality for a secure enclave as described in Pass \kETAL~\cite{pass2017formal}. $\ftee$ is used by edge servers to launch the $BootApp$, represented by $bA$ in the ideal world by calling $(install,idx,\prog)$, where $\prog$ is $bA$. The $BootApp$ executes the coordinator app as well as the aggregation app within the enclave. The steps executed within secure enclaves described in the protocols in Section~\ref{sec05} are executed inside the $\ftee$ functionality in the ideal world. %
$\ftee$ also encrypts the user data $C3$ before returning it in the form of $\outp$ to the calling server $e$ which would have executed the aggregation enclave application when calling $\ftee$ using $install$ function. $\ftee$ can internally access all the tables maintained by $\fregister$ which will allow $bA$ running inside $\ftee$ to retrieve decryption keys from $\symmetable$ and decrypt $encCoorApp$ and $encAggApp$ as needed.  






\begin{figure}[h]
\begin{mdframed}
\begin{center}
\textbf{Functionality} $\ftee$
\end{center}


\textbf{initialize:}
On receiving ($ini$,$\sid$), set $mpk,msk$ = $\sum.\KeyGen(1^{\lambda})$, $T = \emptyset$. %

\textbf{public query:}
on receiving $\getpk()$ from some $\party$, send $mpk$ to $\party$.
\vspace{0.1in}
\begin{mdframed}
\begin{center}
Enclave operations
\end{center}

\textbf{public query:} on receiving $(install,idx,\prog)$ from some $\party \in \reg$, if $\party$ is honest, assert $idx == sid$, generate nonce $\eid \in \{ 0,1 \}^{\lambda}$, store $T[ \eid,\party] = (idx, \prog,\overrightarrow{0})$, send $\eid$ to $\party$.

\textbf{public query:} on receiving $(resume, \eid, \inp)$ from some $\party \in \reg$,\\
let $(idx, \prog, \mem) = T[\eid,\party]$, abort if not found.\\
let $(\outp,\mem) = \prog(\inp,\mem)$, update $T[\eid,\party] = (idx,\prog,\mem)$\\
let $\sigma = \sum.Sig_{msk}(idx, \eid, \prog, \outp)$, and send $(\outp, \sigma)$ to $\party$.
\end{mdframed}


\end{mdframed}
\caption{Ideal functionality for the Secure Enclave~\cite{pass2017formal}}
\label{fig:ftee}
\end{figure}



