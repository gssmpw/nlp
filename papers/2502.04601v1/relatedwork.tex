\section{Related Work}
\label{sec02}
Despite the privacy advantages of the FL paradigm, it remains vulnerable to various privacy and security attacks\cite{nasshohou2019comprehensive,zhuLiGu2024evaluating,shostrmarc2017membership,melsondec2019exploiting,picromveg2023Perfectly,pasfraate2022Eluding}. A significant attack vector in distributed learning is data reconstruction~\cite{yanGeXia2023using, BoeDziSchu2023when,ZhaShaElk2024Large-scale}


These attacks undermine one of the primary motivations for adopting FL: client privacy. To address these vulnerabilities, various security enhancements have been proposed, including sharing only portions of the gradients during updates and employing participant-level differential privacy~\cite{ShoRezShm2015Privacy,weilima2023personalized,Hewancai2024clustered,HuGuoGon2023Federated,liuligao2023privacy-encoded}, albeit often at the cost of model accuracy. Another defense strategy enhances client confidentiality through a Double-Masking protocol for gradients and counters adversarial aggregators by enforcing proof of correctness during aggregation~\cite{xuliliu2020verifynet}.
Numerous novel defenses against reconstruction attacks have also been proposed. For instance, the approach in~\cite{NaHyeJun2022Closing} introduces a low-cost method for obscuring gradients, while another approach incorporates Trusted Execution Enclaves (TEEs) into a synchronous FL (SyncFL) environment, isolating sensitive components of the aggregation process from the rest of the operating system~\cite{Caozhazha2024SRFL}.


These solutions were designed for synchronous FL (SyncFL) systems, with limited consideration for asynchronous FL (AsyncFL) systems. Due to the intrinsic assumption that clients and aggregators operate in a synchronized manner, many of these solutions are not directly applicable to AsyncFL. As a result, they often fail to provide sufficient security or achieve convergence in asynchronous environments.
Nonetheless, some defenses have been developed specifically for AsyncFL. For instance, certain approaches focus on detecting and preventing malicious clients~\cite{FanLiuGon2022AFLGuard,tiacheyu2021towards,liuyuzon2024Delay}. Others propose modifications to the AsyncFL system to enhance speed while maintaining some of the privacy and security levels of SyncFL~\cite{ngumalzha2022federated}. Nonetheless, these solutions still have gaps, particularly in defending against malicious clients and dishonest aggregators while preserving system efficiency--gaps that we addressed.