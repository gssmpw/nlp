\documentclass[11pt]{article}

% \usepackage{blindtext}

% Any additional packages needed should be included after jmlr2e.
% Note that jmlr2e.sty includes epsfig, amssymb, natbib and graphicx,
% and defines many common macros, such as 'proof' and 'example'.
%
% It also sets the bibliographystyle to plainnat; for more information on
% natbib citation styles, see the natbib documentation, a copy of which
% is archived at http://www.jmlr.org/format/natbib.pdf

% Available options for package jmlr2e are:
%
%   - abbrvbib : use abbrvnat for the bibliography style
%   - nohyperref : do not load the hyperref package
%   - preprint : remove JMLR specific information from the template,
%         useful for example for posting to preprint servers.
%
% Example of using the package with custom options:
%
% \usepackage[abbrvbib, preprint]{jmlr2e}

\usepackage{nicefrac}              
\usepackage{microtype}             
\usepackage{xcolor}                
\usepackage{algorithm, algorithmicx}
\usepackage{mathtools}
\usepackage{enumitem}
\usepackage{subfigure}
\usepackage{threeparttable}
\usepackage{multirow}
\usepackage{multicol}
\usepackage{graphicx}
\usepackage{caption}
\usepackage{MnSymbol}

% \usepackage{stmaryrd}
% Definitions of handy macros can go here

% \newtheorem{definition}{Definition}
% \newtheorem{assumption}{Assumption}
% \newtheorem{theorem}{Theorem}
% \newtheorem{lemma}{Proposition}
% \newtheorem{corollary}{Corollary}
% \newtheorem{proposition}{Proposition}
% \newtheorem{remark}{Remark}

\newcommand{\dataset}{{\cal D}}
\newcommand{\fracpartial}[2]{\frac{\partial #1}{\partial  #2}}
\newcommand{\model}{\textsc{DIFFormer}\xspace}
\newcommand{\xingbo}[1]{{\color{blue} #1}}

\usepackage{geometry}
\geometry{a4paper,left=2.7cm,right=2.7cm,top=3cm,bottom=3cm}

\begin{document}

\noindent Dear Editor:

\hspace*{\fill}

We are writing to submit our manuscript entitled ``Wholly-WOOD: Wholly Leveraging Diversified-quality Labels for Weakly-supervised Oriented Object Detection” to the prestigious \emph{IEEE Transactions on Pattern Analysis and Machine Intelligence (TPAMI)}. 

This work is an expanded version from the paper ``H2RBox-v2: Incorporating Symmetry for Boosting Horizontal Box Supervised Oriented Object Detection” which is published in \emph{Advances in Neural Information Processing Systems (NeurIPS)} and also from the paper ``Point2RBox: Combine Knowledge from Synthetic Visual Patterns for End-to-end Oriented Object Detection with Single Point Supervision" \emph{Computer Vision and Pattern Recognition (CVPR)}. To be specific, the extended contributions are outlined below:
\begin{itemize}
    \item We rewrite the full text with a unified perspective and build a more complete and more unified framework that enables support for multiple annotation formats among Point/HBox/RBox.
\item  A more stringent theoretical foundation of symmetry-aware learning is elucidated to provide insight into why the network can discern object angles through consistency losses.
\item  We technically simplify the paradigm with only one transformed view, resulting in a more concise architecture and a significant reduction in RAM usage.
\item  The proposed Wholly-WOOD exhibits further improvements in accuracy compared to the conference versions, especially the performance of Point-to-RBox has increased by 22.36\%, benefiting from our new unified architecture and the newly devised P2R subnet. 
\item  The model is applied to more Point/HBox-annotated scenarios, proving its effectiveness in reducing manual labeling in various applications.
\item  We have released PyTorch and Jittor version codes for H2RBox-v2, Point2RBox, and Wholly-WOOD.
\end{itemize}

We believe the new results in this paper are valuable for the  community and this work may inspire large audience interested in rotated object detection, especially the weakly-supervised setting.

\hspace*{\fill}

\noindent Sincerely,

\hspace*{\fill}

\noindent Junchi Yan (correspondence author)
\end{document}