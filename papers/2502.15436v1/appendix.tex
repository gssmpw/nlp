\clearpage


\section{Proof of Lemma \ref{lemma:privacy} } \label{app:proof_priv}
\begin{tcolorbox}[colback=cyan!10,colframe=black]
\begin{lemma*}
    Consider a model with \( d \) learnable parameters trained using DP-SGD. The privacy parameter \( \epsilon \) for \( \delta \)-approximate differential privacy, given \( T \) training steps and a batch size of \( q \), is expressed as:
\begin{align}
    \epsilon = O(q \sqrt{T d \log (1 / \delta)}) = O(\sqrt{d}).
\end{align}
\end{lemma*}
\end{tcolorbox}

\begin{proof}
    The following result from \citet{dgsgd} describes the relationship between noise variance, privacy parameters, number of optimization steps, batch size, and sample size in DP-SGD.

\begin{theorem*}
    There exist constants \( c_1 \) and \( c_2 \) such that, given the sampling probability \( q = L / N \) and the number of optimization steps \( T \), for any \( \epsilon < c_1 q^2 T \), DP-SGD is \( (\epsilon, \delta) \)-differentially private for any \( \delta > 0 \) if the noise scale satisfies:
    \begin{align}
        \sigma \geq c_2 \frac{q \sqrt{T \log (1 / \delta)}}{\epsilon}.
    \end{align}
\end{theorem*}

Each DP-SGD step introduces noise following \( \mathcal{N}\left(0, \sigma^2 C^2 \mathbf{I}_d\right) \) and satisfies \( (\alpha, \alpha / (2 \sigma^2)) \)-RDP (Rényi DP) for the Gaussian mechanism. For a function with \( \ell_2 \)-sensitivity \( \Delta_2 \), the Gaussian mechanism satisfies \( (\alpha, \epsilon) \)-RDP with:
\begin{align}
    \epsilon(\alpha) = \frac{\alpha \Delta_2^2}{2 \sigma_{\text{noise}}^2}.
\end{align}
Since DP-SGD has \( \Delta_2 = C \) and \( \sigma_{\text{noise}} = \sigma C \), applying privacy amplification due to sampling probability \( q \) results in each step satisfying \( (\alpha, \gamma) \)-RDP, where, for small \( q \):
\begin{align}
    \gamma = O\left(\frac{q^2 \alpha}{\sigma^2}\right).
\end{align}
Using composition over \( T \) steps, the total RDP privacy parameter becomes:
\begin{align}
    \gamma_{\text{total}} = O\left(\frac{q^2 T \alpha}{\sigma^2}\right).
\end{align}
Converting this RDP bound back to \( (\epsilon, \delta) \)-DP and setting \( \alpha \) proportional to \( 1 / \sqrt{d} \), given that the \( \ell_2 \)-norm of the gradient scales as \( \sqrt{d} \), we obtain:
\begin{align}
    \epsilon = O\left(\frac{q^2 T \alpha}{\sigma^2} + \frac{\log (1 / \delta)}{\alpha - 1}\right).
\end{align}
Substituting \( \sigma \propto 1 / \sqrt{d} \), we derive:
\begin{align}
    \epsilon = O(q \sqrt{T d \log (1 / \delta)}) = O(\sqrt{d}).
\end{align}
\end{proof}


% \begin{proof}
%     The following result from \cite{dgsgd} describes the dependence between the noise variance, the privacy parameters, number of optimization steps, batch-size and sample-size upon applying DP-SGD. 

% \begin{theorem*}
%     There exist constants $c_1$ and $c_2$ so that given the sampling probability $q=L / N$ and the number of steps $T$, for any $\epsilon<c_1 q^2 T$, Algorithm 1 is $(\epsilon, \delta)$-differentially private for any $\delta>0$ if we choose

% $$
% \sigma \geq c_2 \frac{q \sqrt{T \log (1 / \delta)}}{\epsilon}
% $$
% \end{theorem*}
% Each DP-SGD step has noise $\mathcal{N}\left(0, \sigma^2 C^2 \mathbf{I}_d\right)$ and is $\left(\alpha, \alpha /\left(2 \sigma^2\right)\right)$-RDP for the Gaussian mechanism. This is because, for functions with $\ell_2$-sensitivity $\Delta_2$, the Gaussian mechanism satisfies $(\alpha, \epsilon)$-RDP with $
% \epsilon(\alpha)=\frac{\alpha \Delta_2^2}{2 \sigma_{\text {noise }}^2}
% $ and in DP-SGD, we have $\Delta_2=C$ and $\sigma_{\text {noise }}=\sigma C$. Now, even upon using privacy amplification due to sampling with probability $q$, each step becomes $(\alpha, \gamma)$-RDP, where $\gamma=O\left(\frac{q^2 \alpha}{\sigma^2}\right)$ for small $q$. Upon applying
% composition over $T$ steps, the total RDP privacy parameter becomes
% $$
% \gamma_{\mathrm{total}}=O\left(\frac{q^2 T \alpha}{\sigma^2}\right)
% $$ Upon converting this RDP back to $(\epsilon, \delta)$-DP, we get that for $\delta>0$, and setting $\alpha$ to be proportional to $1 / \sqrt{d}$, as the $\ell_2$-norm of the gradient grows as $\sqrt{d}$, we have
% $$
% \epsilon=O\left(\frac{q^2 T \alpha}{\sigma^2}+\frac{\log (1 / \delta)}{\alpha-1}\right)
% $$ Upon substituting $\sigma \propto 1 / \sqrt{d}$ ,
% $$
% \epsilon=O(q \sqrt{T d \log (1 / \delta)})=\Omega(\sqrt{d})
% $$
% \end{proof}


%\section{Related Work}
%\label{sec:related-work}

%\subsection{Background}

%Defect detection is critical to ensure the yield of integrated circuit manufacturing lines and reduce faults. Previous research has primarily focused on wafer map data, which engineers produce by marking faulty chips with different colors based on test results. The specific spatial distribution of defects on a wafer can provide insights into the causes, thereby helping to determine which stage of the manufacturing process is responsible for the issues. Although such research is relatively mature, the continual miniaturization of integrated circuits and the increasing complexity and density of chip components have made chip-level detection more challenging, leading to potential risks\cite{ma2023review}. Consequently, there is a need to combine this approach with magnified imaging of the wafer surface using scanning electron microscopes (SEMs) to detect, classify, and analyze specific microscopic defects, thus helping to identify the particular process steps where defects originate.

%Previously, wafer surface defect classification and detection were primarily conducted by experienced engineers. However, this method relies heavily on the engineers' expertise and involves significant time expenditure and subjectivity, lacking uniform standards. With the ongoing development of artificial intelligence, deep learning methods using multi-layer neural networks to extract and learn target features have proven highly effective for this task\cite{gao2022review}.

%In the task of defect classification, it is typical to use a model structure that initially extracts features through convolutional and pooling layers, followed by classification via fully connected layers. Researchers have recently developed numerous classification model structures tailored to specific problems. These models primarily focus on how to extract defect features effectively. For instance, Chen et al. presented a defect recognition and classification algorithm rooted in PCA and classification SVM\cite{chen2008defect}. Chang et al. utilized SVM, drawing on features like smoothness and texture intricacy, for classifying high-intensity defect images\cite{chang2013hybrid}. The classification of defect images requires the formulation of numerous classifiers tailored for myriad inspection steps and an Abundance of accurately labeled data, making data acquisition challenging. Cheon et al. proposed a single CNN model adept at feature extraction\cite{cheon2019convolutional}. They achieved a granular classification of wafer surface defects by recognizing misclassified images and employing a k-nearest neighbors (k-NN) classifier algorithm to gauge the aggregate squared distance between each image feature vector and its k-neighbors within the same category. However, when applied to new or unseen defects, such models necessitate retraining, incurring computational overheads. Moreover, with escalating CNN complexity, the computational demands surge.

%Segmentation of defects is necessary to locate defect positions and gather information such as the size of defects. Unlike classification networks, segmentation networks often use classic encoder-decoder structures such as UNet\cite{ronneberger2015u} and SegNet\cite{badrinarayanan2017segnet}, which focus on effectively leveraging both local and global feature information. Han Hui et al. proposed integrating a Region Proposal Network (RPN) with a UNet architecture to suggest defect areas before conducting defect segmentation \cite{han2020polycrystalline}. This approach enables the segmentation of various defects in wafers with only a limited set of roughly labeled images, enhancing the efficiency of training and application in environments where detailed annotations are scarce. Subhrajit Nag et al. introduced a new network structure, WaferSegClassNet, which extracts multi-scale local features in the encoder and performs classification and segmentation tasks in the decoder \cite{nag2022wafersegclassnet}. This model represents the first detection system capable of simultaneously classifying and segmenting surface defects on wafers. However, it relies on extensive data training and annotation for high accuracy and reliability. 

%Recently, Vic De Ridder et al. introduced a novel approach for defect segmentation using diffusion models\cite{de2023semi}. This approach treats the instance segmentation task as a denoising process from noise to a filter, utilizing diffusion models to predict and reconstruct instance masks for semiconductor defects. This method achieves high precision and improved defect classification and segmentation detection performance. However, the complex network structure and the computational process of the diffusion model require substantial computational resources. Moreover, the performance of this model heavily relies on high-quality and large amounts of training data. These issues make it less suitable for industrial applications. Additionally, the model has only been applied to detecting and segmenting a single type of defect(bridges) following a specific manufacturing process step, limiting its practical utility in diverse industrial scenarios.

%\subsection{Few-shot Anomaly Detection}
%Traditional anomaly detection techniques typically rely on extensive training data to train models for identifying and locating anomalies. However, these methods often face limitations in rapidly changing production environments and diverse anomaly types. Recent research has started exploring effective anomaly detection using few or zero samples to address these challenges.

%Huang et al. developed the anomaly detection method RegAD, based on image registration technology. This method pre-trains an object-agnostic registration network with various images to establish the normality of unseen objects. It identifies anomalies by aligning image features and has achieved promising results. Despite these advancements, implementing few-shot settings in anomaly detection remains an area ripe for further exploration. Recent studies show that pre-trained vision-language models such as CLIP and MiniGPT can significantly enhance performance in anomaly detection tasks.

%Dong et al. introduced the MaskCLIP framework, which employs masked self-distillation to enhance contrastive language-image pretraining\cite{zhou2022maskclip}. This approach strengthens the visual encoder's learning of local image patches and uses indirect language supervision to enhance semantic understanding. It significantly improves transferability and pretraining outcomes across various visual tasks, although it requires substantial computational resources.
%Jeong et al. crafted the WinCLIP framework by integrating state words and prompt templates to characterize normal and anomalous states more accurately\cite{Jeong_2023_CVPR}. This framework introduces a novel window-based technique for extracting and aggregating multi-scale spatial features, significantly boosting the anomaly detection performance of the pre-trained CLIP model.
%Subsequently, Li et al. have further contributed to the field by creating a new expansive multimodal model named Myriad\cite{li2023myriad}. This model, which incorporates a pre-trained Industrial Anomaly Detection (IAD) model to act as a vision expert, embeds anomaly images as tokens interpretable by the language model, thus providing both detailed descriptions and accurate anomaly detection capabilities.
%Recently, Chen et al. introduced CLIP-AD\cite{chen2023clip}, and Li et al. proposed PromptAD\cite{li2024promptad}, both employing language-guided, tiered dual-path model structures and feature manipulation strategies. These approaches effectively address issues encountered when directly calculating anomaly maps using the CLIP model, such as reversed predictions and highlighting irrelevant areas. Specifically, CLIP-AD optimizes the utilization of multi-layer features, corrects feature misalignment, and enhances model performance through additional linear layer fine-tuning. PromptAD connects normal prompts with anomaly suffixes to form anomaly prompts, enabling contrastive learning in a single-class setting.

%These studies extend the boundaries of traditional anomaly detection techniques and demonstrate how to effectively address rapidly changing and sample-scarce production environments through the synergy of few-shot learning and deep learning models. Building on this foundation, our research further explores wafer surface defect detection based on the CLIP model, especially focusing on achieving efficient and accurate anomaly detection in the highly specialized and variable semiconductor manufacturing process using a minimal amount of labeled data.


\section{Experiment Details} \label{app:hyperparams}

We conduct experiments on a single NVIDIA A6000 GPU (48 GB) and report the average results from three independent runs. All non-private models are trained using the AdamW optimizer \citep{loshchilov2019decoupledweightdecayregularization}. 
In line with \citet{ponkshe2024initialization}, we initialize the adapter matrices using just $1/{1000}$ of the respective training dataset size.
\\

\textbf{Instruction Tuning.}
Table \ref{tab:hyper_it} presents the key hyperparameters and configurations for Mistral-7B, Gemma-2 9B, and Llama-3.2 3B. 
Our setup closely follows previous works \citep{cr-dataset, ponkshe2024initialization}, ensuring consistency with established best practices.
For the baseline experiments, we further set $\alpha = 16$, consistent with prior literature \citep{singhal2024exact, sun2024improving}.
We additionally perform a sweep over the learning rate for our experiments.
\\

\textbf{(Federated) Private Fine-Tuning.}
Table \ref{tab:hyper_bert} outlines the key hyperparameters and configurations for BERT-base in both centralized private and federated private settings. 
We train our models using the Opacus library \citep{yousefpour2021opacus} with the DP-SGD optimizer \citep{dgsgd}. 
Following standard DP practices, we set the privacy parameter as \(\delta = \frac{1}{|\text{trainset}|}\). 
To ensure adherence to best practices, we adopt hyperparameter choices from prior works \citep{singhal2024exact, lora}. For baseline experiments, we additionally set \(\alpha = 16\), aligning with previous literature \citep{singhal2024exact, sun2024improving}. 
We additionally perform a sweep over the learning rate and maximum gradient norm in DP-SGD for our experiments.


\begin{table*}[!h]
\centering
\begin{tabular}{lccc}
\toprule
 & \textbf{Mistral-7B} & \textbf{Gemma-2 9B} & \textbf{Llama-3.2 3B}\\
\midrule
Optimizer        & AdamW      & AdamW      & AdamW      \\
Learning Rate    & $5\mathrm{e}{-4}$ & $5\mathrm{e}{-4}$ & $2\mathrm{e}{-4}$ \\
LR Scheduler     & Cosine     & Cosine     & Linear     \\
Warmup Ratio     & $0.02$     & $0.02$     & $0.02$     \\
Batch Size       & $1$        & $1$        & $8$        \\
Grad Acc. Steps  & $32$       & $32$       & $24$       \\
Max. Seq. Len    & $512$      & $512$      & $256$      \\
Dropout          & $0$        & $0$        & $0$        \\
\# Clients   & $25$       & $25$       & $5$        \\
Local Epochs     & $1$        & $2$        & $2$        \\
Rounds           & $1$        & $1$        & $1$        \\
\bottomrule
\end{tabular}
\caption{Hyperparameter settings for Mistral-7B, Gemma-2 9B, and Llama-3.2 3B.}
\label{tab:hyper_it}
\end{table*}

\begin{table*}[!h]
\centering
\begin{tabular}{lcc}
\toprule
 & \textbf{BERT-base (centralized)} & \textbf{BERT-base (federated)} \\
\midrule
Optimizer        & DP-SGD      & DP-SGD      \\
Learning Rate    & $5\mathrm{e}{-4}$ & $5\mathrm{e}{-4}$ \\
LR Scheduler     & -    & -   \\
Warmup Ratio     & 0     & 0 \\
Batch Size       & $32$        & $32$\\
Max. Phy. Batch Size       & $8$        & $8$\\
Max. Seq. Len    & $128$      & $128$\\
Dropout          & $0.05$        & $0.05$\\
Max. Grad. Norm & $0.1$ & $0.1$ \\
Epochs & $3$ & - \\
\midrule
\# Clients   & -      & $3$\\
Local Epochs     & -        & $6$\\
Rounds           & -        & $1$\\
\bottomrule
\end{tabular}
\caption{Hyperparameter settings for BERT-base in centralized private and federated private setups.}
\label{tab:hyper_bert}
\end{table*}



\section{Dataset Details} \label{app:datasets}


\textsc{\textbf{CommonSense170K}} is a large-scale dataset that brings together eight benchmarks designed to assess various aspects of commonsense reasoning \citep{cr-dataset}. Below is an overview of its constituent datasets:

\begin{enumerate}
\item \textbf{PIQA} \citep{bisk2020piqa} evaluates physical commonsense by asking models to determine the most reasonable action in a given scenario.
\item \textbf{ARC Easy (ARC-e)} \citep{clark2018think} consists of elementary-level science questions, serving as a fundamental test of a model’s reasoning abilities.
\item \textbf{OBQA} \citep{mihaylov2018can} presents knowledge-intensive, open-book multiple-choice questions that require multi-step reasoning and retrieval.
\item \textbf{HellaSwag} \citep{zellers2019hellaswag} tests contextual reasoning by asking models to predict the most plausible continuation of a passage from a set of candidates.
\item \textbf{SIQA} \citep{sap2019socialiqa} examines social intelligence, requiring models to predict human actions and their social consequences.
\item \textbf{ARC Challenge (ARC-c)} \citep{clark2018think} includes difficult multiple-choice science questions that demand deeper logical inference beyond statistical co-occurrence.
\item \textbf{BoolQ} \citep{clark2019boolq} consists of naturally occurring yes/no questions, requiring models to infer relevant information from provided contexts.
\item \textbf{WinoGrande} \citep{sakaguchi2021winogrande} assesses commonsense knowledge through binary-choice sentence completion tasks that require resolving ambiguities.
\end{enumerate}


The \textbf{MetaMathQA} dataset \citep{metamathqa} constructs mathematical questions by reformulating them from different viewpoints while preserving their original knowledge content. We assess its performance using two well-established benchmarks: (1) \textbf{GSM8K} \citep{gsm8k}, a collection of grade-school-level math problems requiring step-by-step reasoning to reach a solution, and (2) \textbf{MATH} \citep{math}, which consists of high-difficulty, competition-style problems designed to test advanced mathematical skills.
\\

\textbf{Stanford Natural Language Inference (SNLI)} is a widely used benchmark for assessing textual entailment models in natural language understanding. 
It contains approximately 570,000 sentence pairs, each categorized into one of three classes: entailment, contradiction, or neutral, requiring models to infer the relationship between a given premise and hypothesis.

\section{Additional Plots} \label{app:plots}
\begin{figure*}[!h]
    \centering
    % First subfigure
    \begin{subfigure}{0.49\textwidth}
        \centering
        \includegraphics[width=\textwidth]{latex/figures/pareto-mistral-final-2.png}
        \caption{Mistral-7B (GSM8K)}
        \label{fig:fed-mistral}
    \end{subfigure}
    \begin{subfigure}{0.49\textwidth}
        \centering
        \includegraphics[width=\textwidth]{latex/figures/pareto-gemma-final-2.png}
        \caption{Gemma-2 9B (MATH)}
        \label{fig:fed-math}
    \end{subfigure}
    \begin{subfigure}{0.49\textwidth}
        \centering
        \includegraphics[width=\textwidth]{latex/figures/pareto-llama.png}
        \caption{Llama-3.2 3B (Commonsense)}
        \label{fig:fed-llama}
    \end{subfigure}
    \caption{Performance vs. number of communicated parameters (in log scale) for various methods in federated fine-tuning across multiple models on arithmetic and commonsense reasoning tasks.}
    \label{fig:results-it}
\end{figure*}

\begin{figure*}[!h]
    \centering
    % First subfigure
    \begin{subfigure}{0.49\textwidth}
        \centering
        \includegraphics[width=\textwidth]{latex/figures/dp_central_vary_eps.png}
        \caption{Centralized Private}
        \label{fig:dp-central-vary}
    \end{subfigure}
    \begin{subfigure}{0.49\textwidth}
        \centering
        \includegraphics[width=\textwidth]{latex/figures/dp_fed_vary_eps.png}
        \caption{Federated Private}
        \label{fig:dp-fed-vary}
    \end{subfigure}
    \caption{Performance comparison of various methods in centralized (Cent.) private and federated private fine-tuning (BERT-base) on SNLI across varying values of $\epsilon$.}
    \label{fig:plots-dp-vary}
\end{figure*}




\begin{figure*}[!h]
    \centering
    % First subfigure
    \begin{subfigure}{0.49\textwidth}
        \centering
        \includegraphics[width=\textwidth]{latex/figures/dp_central_eps_1.png}
        \caption{$\epsilon = 1$}
        \label{fig:dp-central-eps-1}
    \end{subfigure}
    \begin{subfigure}{0.49\textwidth}
        \centering
        \includegraphics[width=\textwidth]{latex/figures/dp_central_eps_3.png}
        \caption{$\epsilon = 3$}
        \label{fig:dp-central-eps-3}
    \end{subfigure}
    \begin{subfigure}{0.49\textwidth}
        \centering
        \includegraphics[width=\textwidth]{latex/figures/dp_central_eps_5.png}
        \caption{$\epsilon = 5$}
        \label{fig:dp-central-eps-5}
    \end{subfigure}
    %\hfill
    % Second subfigure
    \begin{subfigure}{0.49\textwidth}
        \centering
        \includegraphics[width=\textwidth]{latex/figures/dp_central_eps_7.5.png}
        \caption{$\epsilon = 7.5$}
        \label{fig:dp-central-eps-7.5}
    \end{subfigure}
    \begin{subfigure}{0.49\textwidth}
        \centering
        \includegraphics[width=\textwidth]{latex/figures/dp_central_eps_10.png}
        \caption{$\epsilon = 10$}
        \label{fig:dp-central-eps-10}
    \end{subfigure}
    \caption{Performance vs. number of trainable parameters (in log scale) for various methods in centralized private fine-tuning (BERT-base) across different privacy budgets ($\epsilon$).}
    \label{fig:plots-dp-central-eps}
\end{figure*}


\begin{figure*}[ht]
    \centering
    % First subfigure
    \begin{subfigure}{0.49\textwidth}
        \centering
        \includegraphics[width=\textwidth]{latex/figures/dp_fed_eps_1.png}
        \caption{$\epsilon = 1$}
        \label{fig:dp-fed-eps-1}
    \end{subfigure}
    \begin{subfigure}{0.49\textwidth}
        \centering
        \includegraphics[width=\textwidth]{latex/figures/dp_fed_eps_3.png}
        \caption{$\epsilon = 3$}
        \label{fig:dp-fed-eps-3}
    \end{subfigure}
    \begin{subfigure}{0.49\textwidth}
        \centering
        \includegraphics[width=\textwidth]{latex/figures/dp_fed_eps_5.png}
        \caption{$\epsilon = 5$}
        \label{fig:dp-fed-eps-5}
    \end{subfigure}
    %\hfill
    % Second subfigure
    \begin{subfigure}{0.49\textwidth}
        \centering
        \includegraphics[width=\textwidth]{latex/figures/dp_fed_eps_7.5.png}
        \caption{$\epsilon = 7.5$}
        \label{fig:dp-fed-eps-7.5}
    \end{subfigure}
    \begin{subfigure}{0.49\textwidth}
        \centering
        \includegraphics[width=\textwidth]{latex/figures/dp_fed_eps_10.png}
        \caption{$\epsilon = 10$}
        \label{fig:dp-fed-eps-10}
    \end{subfigure}
    \caption{Performance vs. number of communicated parameters (in log scale) for various methods in federated private fine-tuning (BERT-base) across different privacy budgets ($\epsilon$).}
    \label{fig:plots-dp-fed-eps}
\end{figure*}