% Can be in separate page
\vspace{-0.15cm}
\section{Limitations and Potential Risks}
\vspace{-0.15cm}
There are several limitations and potential risks. One limitation pertains to the ethical and fairness considerations of personalization. While \methodname\ improves inclusivity by modeling diverse preferences, the risk of reinforcing user biases (echo chambers) or inadvertently amplifying harmful viewpoints requires careful scrutiny. Future work should explore mechanisms to balance personalization with ethical safeguards, ensuring that models remain aligned with fairness principles while respecting user individuality. Additionally, our human study was preliminary with control over the questions that a user may ask, format normalization where formatting details such as markdown are removed, and view normalization comparing the same number of viewpoints for both \methodname\ and the baselines. However, to the best of our knowledge, we are the first approach to perform such a human study for personalization to open-ended question answering. Future work should do further ablations with human evaluation for personalization. Additionally, due to compute constraints, we work with models in the parameter range of 3B (specifically Llama 3.2 Instruct 3B) with a limited context window of 128K, and without context optimization such as sequence parallelism~\citep{li2022sequenceparallelismlongsequence, yang2024contextparallelismscalablemilliontoken}, further limiting the effective context window. It is an open question on how fine-tuning base models with better long-context and reasoning capabilities would help with \methodname\ for personalization, such as the 2M context window of Gemini  Flash Thinking models, especially in the case of COT.