%%
%% This is file `sample-sigconf.tex',
%% generated with the docstrip utility.
%%
%% The original source files were:
%%
%% samples.dtx  (with options: `all,proceedings,bibtex,sigconf')
%% 
%% IMPORTANT NOTICE:
%% 
%% For the copyright see the source file.
%% 
%% Any modified versions of this file must be renamed
%% with new filenames distinct from sample-sigconf.tex.
%% 
%% For distribution of the original source see the terms
%% for copying and modification in the file samples.dtx.
%% 
%% This generated file may be distributed as long as the
%% original source files, as listed above, are part of the
%% same distribution. (The sources need not necessarily be
%% in the same archive or directory.)
%%
%%
%% Commands for TeXCount
%TC:macro \cite [option:text,text]
%TC:macro \citep [option:text,text]
%TC:macro \citet [option:text,text]
%TC:envir table 0 1
%TC:envir table* 0 1
%TC:envir tabular [ignore] word
%TC:envir displaymath 0 word
%TC:envir math 0 word
%TC:envir comment 0 0
%%
%%
%% The first command in your LaTeX source must be the \documentclass
%% command.
%%
%% For submission and review of your manuscript please change the
%% command to \documentclass[manuscript, screen, review]{acmart}.
%%
%% When submitting camera ready or to TAPS, please change the command
%% to \documentclass[sigconf]{acmart} or whichever template is required
%% for your publication.
%%
%%
% \documentclass[manuscript,anonymous,review]{acmart}
\documentclass[sigconf]{acmart}

%%
%% \BibTeX command to typeset BibTeX logo in the docs
\AtBeginDocument{%
  \providecommand\BibTeX{{%
    Bib\TeX}}}

%% Rights management information.  This information is sent to you
%% when you complete the rights form.  These commands have SAMPLE
%% values in them; it is your responsibility as an author to replace
%% the commands and values with those provided to you when you
%% complete the rights form.
\copyrightyear{2025}
\acmYear{2025}
\setcopyright{cc}
\setcctype{by}
\acmConference[CHI '25]{CHI Conference on Human Factors in Computing Systems}{April 26-May 1, 2025}{Yokohama, Japan}
\acmBooktitle{CHI Conference on Human Factors in Computing Systems (CHI '25), April 26-May 1, 2025, Yokohama, Japan}\acmDOI{10.1145/3706598.3713103}
\acmISBN{979-8-4007-1394-1/25/04}


%%
%% Submission ID.
%% Use this when submitting an article to a sponsored event. You'll
%% receive a unique submission ID from the organizers
%% of the event, and this ID should be used as the parameter to this command.
%%\acmSubmissionID{123-A56-BU3}

%%
%% For managing citations, it is recommended to use bibliography
%% files in BibTeX format.
%%
%% You can then either use BibTeX with the ACM-Reference-Format style,
%% or BibLaTeX with the acmnumeric or acmauthoryear sytles, that include
%% support for advanced citation of software artefact from the
%% biblatex-software package, also separately available on CTAN.
%%
%% Look at the sample-*-biblatex.tex files for templates showcasing
%% the biblatex styles.
%%

%%
%% The majority of ACM publications use numbered citations and
%% references.  The command \citestyle{authoryear} switches to the
%% "author year" style.
%%
%% If you are preparing content for an event
%% sponsored by ACM SIGGRAPH, you must use the "author year" style of
%% citations and references.
%% Uncommenting
%% the next command will enable that style.
%%\citestyle{acmauthoryear}

\usepackage{enumitem}
\newcounter{goalCounter}

% \usepackage{color-edits}
% \addauthor{}{blue}

\usepackage{xr}

\externaldocument{supplementary}

%%
%% end of the preamble, start of the body of the document source.
\begin{document}

%%
%% The "title" command has an optional parameter,
%% allowing the author to define a "short title" to be used in page headers.
\title{Divisi: Interactive Search and Visualization for Scalable Exploratory Subgroup Analysis}

%%
%% The "author" command and its associated commands are used to define
%% the authors and their affiliations.
%% Of note is the shared affiliation of the first two authors, and the
%% "authornote" and "authornotemark" commands
%% used to denote shared contribution to the research.
\author{Venkatesh Sivaraman}
\email{venkats@cmu.edu}
\orcid{0000-0002-6965-3961}
\affiliation{%
  \institution{Carnegie Mellon University}
  \city{Pittsburgh}
  \state{Pennsylvania}
  \country{USA}
}

\author{Zexuan Li}
\email{zexuanli@umich.edu}
\affiliation{%
  \institution{University of Michigan}
  \city{Ann Arbor}
  \state{Michigan}
  \country{USA}
}

\author{Adam Perer}
\email{adamperer@cmu.edu}
\affiliation{%
  \institution{Carnegie Mellon University}
  \city{Pittsburgh}
  \state{Pennsylvania}
  \country{USA}
}

%%
%% By default, the full list of authors will be used in the page
%% headers. Often, this list is too long, and will overlap
%% other information printed in the page headers. This command allows
%% the author to define a more concise list
%% of authors' names for this purpose.
\renewcommand{\shortauthors}{Sivaraman et al.}

%%
%% The abstract is a short summary of the work to be presented in the
%% article.
\begin{abstract}
Analyzing data subgroups is a common data science task to build intuition about a dataset and identify areas to improve model performance. 
However, subgroup analysis is prohibitively difficult in datasets with many features, and existing tools limit unexpected discoveries by relying on user-defined or static subgroups.
We propose exploratory subgroup analysis as a set of tasks in which practitioners discover, evaluate, and curate interesting subgroups to build understanding about datasets and models. 
To support these tasks we introduce Divisi, an interactive notebook-based tool underpinned by a fast approximate subgroup discovery algorithm. 
Divisi's interface allows data scientists to interactively re-rank and refine subgroups and to visualize their overlap and coverage in the novel Subgroup Map.
Through a think-aloud study with 13 practitioners, we find that Divisi can help uncover surprising patterns in data features and their interactions, and that it encourages more thorough exploration of subtypes in complex data.
% Analyzing subgroups in a dataset that have different outcomes than average is an important way to build intuition about data. However, subgroup analysis is difficult when the dataset is large, contains many features, or requires intersecting multiple features together to find meaningful subgroups. Automatic subgroup discovery is a promising strategy to mine interesting groups of instances from large datasets, but existing approaches take too long to run interactively, do not extend to multiple outcomes of interest, and produce a static list of results. We propose Divisi, an interactive search-based tool for data scientists to find and curate subgroups of interest within large datasets with discrete features. Divisi incorporates a novel approximate subgroup discovery algorithm that allows balancing between multiple ranking functions, and it performs faster than existing approaches while maintaining high accuracy compared to exhaustive search. The computational notebook-based visual interface provides interactive re-ranking, editing, and curation features including a novel Subgroup Map visualization. Through a think-aloud study with 13 data scientists, we find that Divisi enables analyses that would have previously been prohibitively time-consuming, and that it facilitates an understanding of the overlap and coverage of subgroups of interest.
\end{abstract}

%%
%% The code below is generated by the tool at http://dl.acm.org/ccs.cfm.
%% Please copy and paste the code instead of the example below.
%%
\begin{CCSXML}
<ccs2012>
   <concept>
       <concept_id>10003120.10003145.10003151</concept_id>
       <concept_desc>Human-centered computing~Visualization systems and tools</concept_desc>
       <concept_significance>500</concept_significance>
       </concept>
   <concept>
       <concept_id>10002951.10003227.10003351.10003443</concept_id>
       <concept_desc>Information systems~Association rules</concept_desc>
       <concept_significance>500</concept_significance>
       </concept>
   <concept>
       <concept_id>10002951.10003317.10003331</concept_id>
       <concept_desc>Information systems~Users and interactive retrieval</concept_desc>
       <concept_significance>300</concept_significance>
       </concept>
 </ccs2012>
\end{CCSXML}

\ccsdesc[500]{Human-centered computing~Visualization systems and tools}
\ccsdesc[500]{Information systems~Association rules}
\ccsdesc[300]{Information systems~Users and interactive retrieval}

%%
%% Keywords. The author(s) should pick words that accurately describe
%% the work being presented. Separate the keywords with commas.
\keywords{Exploratory Data Analysis, Model Evaluation, Slice Discovery, Subgroup Analysis}
%% A "teaser" image appears between the author and affiliation
%% information and the body of the document, and typically spans the
%% page.
\begin{teaserfigure}
  \includegraphics[width=\textwidth, alt={Screenshot of Divisi with six callouts for key features, including running a subgroup discovery algorithm, interactively re-ranking subgroups, comparing metrics, editing subgroup definitions, visualizing overlap and coverage, and saving subgroups for review. The left side of the interface lists ranking functions: error coverage, error rate, simple rule, and subgroup size. The middle shows the Subgroups Table, where subgroups include Inflight wifi service = "neutral", Seat comfort = "neutral", Class = "business", and six others. Two subgroups are selected and plotted in the Subgroup Map at right, showing that they have a small degree of overlap but are mostly distinct.}]{figures/teaser.pdf}
  \caption{Divisi is an interactive visualization system to help data scientists perform \textit{exploratory subgroup analysis} on large datasets with many feature dimensions, such as the dataset of airline passenger satisfaction ratings shown~\cite{noauthor_airline_2019}. Implemented as a computational notebook widget, Divisi includes a novel approximate subgroup discovery algorithm (A) which allows interactive re-ranking by customizable functions, such as error rate and coverage (B). Users can compare metrics across subgroups (C) and test alternative rule definitions (D) to evaluate subgroups. Finally, the Subgroup Map (E) depicts overlap and coverage between groups, so users can curate the most representative subgroups for review (F).}
  \Description{Enjoying the baseball game from the third-base
  seats. Ichiro Suzuki preparing to bat.}
  \label{fig:teaser}
\end{teaserfigure}

% \received{20 February 2007}
% \received[revised]{12 March 2009}
% \received[accepted]{5 June 2009}

%%
%% This command processes the author and affiliation and title
%% information and builds the first part of the formatted document.
\maketitle


% humans are sensitive to the way information is presented.

% introduce framing as the way we address framing. say something about political views and how information is represented.

% in this paper we explore if models show similar sensitivity.

% why is it important/interesting.



% thought - it would be interesting to test it on real world data, but it would be hard to test humans because they come already biased about real world stuff, so we tested artificial.


% LLMs have recently been shown to mimic cognitive biases, typically associated with human behavior~\citep{ malberg2024comprehensive, itzhak-etal-2024-instructed}. This resemblance has significant implications for how we perceive these models and what we can expect from them in real-world interactions and decisionmaking~\citep{eigner2024determinants, echterhoff-etal-2024-cognitive}.

The \textit{framing effect} is a well-known cognitive phenomenon, where different presentations of the same underlying facts affect human perception towards them~\citep{tversky1981framing}.
For example, presenting an economic policy as only creating 50,000 new jobs, versus also reporting that it would cost 2B USD, can dramatically shift public opinion~\cite{sniderman2004structure}. 
%%%%%%%% 图1:  %%%%%%%%%%%%%%%%
\begin{figure}[t]
    \centering
    \includegraphics[width=\columnwidth]{Figs/01.pdf}
    \caption{Performance comparison (Top-1 Acc (\%)) under various open-vocabulary evaluation settings where the video learners except for CLIP are tuned on Kinetics-400~\cite{k400} with frozen text encoders. The satisfying in-context generalizability on UCF101~\cite{UCF101} (a) can be severely affected by static bias when evaluating on out-of-context SCUBA-UCF101~\cite{li2023mitigating} (b) by replacing the video background with other images.}
    \label{fig:teaser}
\end{figure}


Previous research has shown that LLMs exhibit various cognitive biases, including the framing effect~\cite{lore2024strategic,shaikh2024cbeval,malberg2024comprehensive,echterhoff-etal-2024-cognitive}. However, these either rely on synthetic datasets or evaluate LLMs on different data from what humans were tested on. In addition, comparisons between models and humans typically treat human performance as a baseline rather than comparing patterns in human behavior. 
% \gabis{looks good! what do we mean by ``most studies'' or ``rarely'' can we remove those? or we want to say that we don't know of previous work doing both at the same time?}\gili{yeah the main point is that some work has done each separated, but not all of it together. how about now?}

In this work, we evaluate LLMs on real-world data. Rather than measuring model performance in terms of accuracy, we analyze how closely their responses align with human annotations. Furthermore, while previous studies have examined the effect of framing on decision making, we extend this analysis to sentiment analysis, as sentiment perception plays a key explanatory role in decision-making \cite{lerner2015emotion}. 
%Based on this, we argue that examining sentiment shifts in response to reframing can provide deeper insights into the framing effect. \gabis{I don't understand this last claim. Maybe remove and just say we extend to sentiment analysis?}

% Understanding how LLMs respond to framing is crucial, as they are increasingly integrated into real-world applications~\citep{gan2024application, hurlin2024fairness}.
% In some applications, e.g., in virtual companions, framing can be harnessed to produce human-like behavior leading to better engagement.
% In contrast, in other applications, such as financial or legal advice, mitigating the effect of framing can lead to less biased decisions.
% In both cases, a better understanding of the framing effect on LLMs can help develop strategies to mitigate its negative impacts,
% while utilizing its positive aspects. \gabis{$\leftarrow$ reading this again, maybe this isn't the right place for this paragraph. Consider putting in the conclusion? I think that after we said that people have worked on it, we don't necessarily need this here and will shorten the long intro}


% If framing can influence their outputs, this could have significant societal effects,
% from spreading biases in automated decision-making~\citep{ghasemaghaei2024understanding} to reducing public trust in AI-generated content~\citep{afroogh2024trust}. 
% However, framing is not inherently negative -- understanding how it affects LLM outputs can offer valuable insights into both human and machine cognition.
% By systematically investigating the framing effect,


%It is therefore crucial to systematically investigate the framing effect, to better understand and mitigate its impact. \gabis{This paragraph is important - I think that right now it's saying that we don't want models to be influenced by framing (since we want to mitigate its impact, right?) When we talked I think we had a more nuanced position?}




To better understand the framing effect in LLMs in comparison to human behavior,
we introduce the \name{} dataset (Section~\ref{sec:data}), comprising 1,000 statements, constructed through a three-step process, as shown in Figure~\ref{fig:fig1}.
First, we collect a set of real-world statements that express a clear negative or positive sentiment (e.g., ``I won the highest prize'').
%as exemplified in Figure~\ref{fig:fig1} -- ``I won the highest prize'' positive base statement. (2) next,
Second, we \emph{reframe} the text by adding a prefix or suffix with an opposite sentiment (e.g., ``I won the highest prize, \emph{although I lost all my friends on the way}'').
Finally, we collect human annotations by asking different participants
if they consider the reframed statement to be overall positive or negative.
% \gabist{This allows us to quantify the extent of \textit{sentiment shifts}, which is defined as labeling the sentiment aligning with the opposite framing, rather then the base sentiment -- e.g., voting ``negative'' for the statement ``I won the highest prize, although I lost all my friends on the way'', as it aligns with the opposite framing sentiment.}
We choose to annotate Amazon reviews, where sentiment is more robust, compared to e.g., the news domain which introduces confounding variables such as prior political leaning~\cite{druckman2004political}.


%While the implications of framing on sensitive and controversial topics like politics or economics are highly relevant to real-world applications, testing these subjects in a controlled setting is challenging. Such topics can introduce confounding variables, as annotators might rely on their personal beliefs or emotions rather than focusing solely on the framing, particularly when the content is emotionally charged~\cite{druckman2004political}. To balance real-world relevance with experimental reliability, we chose to focus on statements derived from Amazon reviews. These are naturally occurring, sentiment-rich texts that are less likely to trigger strong preexisting biases or emotional reactions. For instance, a review like ``The book was engaging'' can be framed negatively without invoking specific cultural or political associations. 

 In Section~\ref{sec:results}, we evaluate eight state-of-the-art LLMs
 % including \gpt{}~\cite{openai2024gpt4osystemcard}, \llama{}~\cite{dubey2024llama}, \mistral{}~\cite{jiang2023mistral}, \mixtral{}~\cite{mistral2023mixtral}, and \gemma{}~\cite{team2024gemma}, 
on the \name{} dataset and compare them against human annotations. We find  that LLMs are influenced by framing, somewhat similar to human behavior. All models show a \emph{strong} correlation ($r>0.57$) with human behavior.
%All models show a correlation with human responses of more than $0.55$ in Pearson's $r$ \gabis{@Gili check how people report this?}.
Moreover, we find that both humans and LLMs are more influenced by positive reframing rather than negative reframing. We also find that larger models tend to be more correlated with human behavior. Interestingly, \gpt{} shows the lowest correlation with human behavior. This raises questions about how architectural or training differences might influence susceptibility to framing. 
%\gabis{this last finding about \gpt{} stands in opposition to the start of the statement, right? Even though it's probably one of the largest models, it doesn't correlate with humans? If so, better to state this explicitly}

This work contributes to understanding the parallels between LLM and human cognition, offering insights into how cognitive mechanisms such as the framing effect emerge in LLMs.\footnote{\name{} data available at \url{https://huggingface.co/datasets/gililior/WildFrame}\\Code: ~\url{https://github.com/SLAB-NLP/WildFrame-Eval}}

%\gabist{It also raises fundamental philosophical and practical questions -- should LLMs aim to emulate human-like behavior, even when such behavior is susceptible to harmful cognitive biases? or should they strive to deviate from human tendencies to avoid reproducing these pitfalls?}\gabis{$\leftarrow$ also following Itay's comment, maybe this is better in the dicsussion, since we don't address these questions in the paper.} %\gabis{This last statement brings the nuance back, so I think it contradicts the previous parapgraph where we talked about ``mitigating'' the effect of framing. Also, I think it would be nice to discuss this a bit more in depth, maybe in the discussion section.}






\section{Related Work}
Our work builds on several lines of research: exploring the role of text with visualizations, visualization and text systems, and image and text authoring interfaces.

\subsection{The role of text with visualizations}
The interplay between text and visual elements in data visualization has been a significant area of interest with increased advocacy for treating text as co-equal to visualization~\cite{stokesgive, lundgard2021accessible}. Kim et al.~\cite{kim2021towards} conducted a study to understand how readers integrate charts and captions in line charts. The study findings indicated that when both the chart and text emphasize the same prominent features, readers take away insights from both modalities. Their research underscores the importance of coherence between visual and textual elements and how external context provided by captions can enhance the reader's comprehension of the chart's message. Building on these insights, Lundgard and Satyanarayan~\cite{lundgard2021accessible} proposed a four-level model for content conveyed by natural language descriptions of visualizations. Their model delineates semantic content into four distinct levels: elemental and encoded properties (Level 1), statistical concepts (Level 2), perceptual and cognitive phenomena (Level 3), and contextual insights (Level 4).

Focusing on the role of textual annotations in visualization, Stokes et al.~\cite{stokes2022striking} observed that readers favored heavily annotated charts over less annotated charts or text alone. This preference highlights the added value of textual annotations in aiding data interpretation, with specific emphasis on how different types of semantic content impact the takeaways drawn by readers. Further contributions by Quadri et al.~\cite{quadri2024you} and Fan et al.~\cite{fan2024understanding} explored high-level visualization comprehension and the impact of text details and spatial autocorrelation on reader takeaways in thematic maps. These studies collectively underline the critical role of textual elements in shaping viewer perceptions, understanding, and accessibility of visual data. Ottley et al.~\cite{ottley2019curious} and Stokes et al.~\cite{stokes2023role} have also contributed to this body of research, focusing on how annotations influence perceptions of bias and predictions, reinforcing the multifaceted impact of text on visual data interpretation.

Our work further explores how text and charts can be better aligned with one another by offering a mixed-initiative authoring interface. Specifically, \pluto~allows leveraging both direct manipulation interactions and user-drafted text to generate recommendations for communicative text and chart design. Furthermore, \pluto's text recommendations explicitly incorporate Lundgard and Satyanarayan's model~\cite{lundgard2021accessible} for semantic information conveyed by visualization descriptions.
In doing so, the system ensures that the generated text has good semantic coverage and structure (e.g., generated descriptions start by conveying the chart's encodings and then list high-level trends) and is appropriate for the intended communicative use (e.g., the semantic information conveyed by titles is different from descriptions accompanying a chart or annotations on the chart).


\subsection{Visualization and text systems}

The integration of visualization and text has led to the development of various systems designed to facilitate the creation, interpretation, and enhancement of data visualizations with textual elements. He et al.~\cite{he2024leveraging} surveyed the leveraging of large models for crafting narrative visualizations, highlighting the potential of AI in supporting the narrative aspect of data visualization. This is complemented by AutoTitle, an interactive title generator for visualizations~\cite{liu2023autotitle}, and Vistext, a benchmark for semantically rich chart captioning~\cite{tang2023vistext}. VizFlow demonstrates the effectiveness of facilitating author-reader interaction by dynamically connecting text segments to corresponding chart elements to help enrich the storytelling experience~\cite{sultanum2021}. This body of research highlights the need for tools to support more nuanced integration of text and visualization.

Supporting the co-authoring of text and charts, Latif et al. introduced Kori~\cite{latif2021kori}, an interactive system for synthesizing text and charts in data documents, emphasizing the seamless integration of visual and textual data for enhanced communication.
\new{CrossData~\cite{chen2022crossdata} presents an interactive coupling between text and data in documents, enabling actions based on the document text and adjusting data values in the text through direct manipulation on the chart.
Such systems illustrate the potential for the bidirectional linking between text and charts to assist rich authoring of data-driven narratives.
}
Furthermore, systems like EmphasisChecker~\cite{kim2023emphasischecker}, Intentable~\cite{choi2022intentable}, Chart-to-text~\cite{obeid2020chart}, DataDive~\cite{kim2024datadive},
\new{InkSight~\cite{lin2023inksight}},
and FigurA11y~\cite{singh2024figura11y} focus on guiding chart and caption creation, supporting readers' contextualization of statistical statements, and assisting in writing scientific alt text. Recent work like SciCapenter supports the composition of scientific figure captions using AI-generated content and quality ratings \cite{hsu2024scicapenter}.
DataTales~\cite{sultanum2023datatales} is another example of a recent system using a large language model for authoring data-driven articles, indicating the growing interest in AI-assisted data storytelling.
These systems collectively illustrate the expanding scope of text integration into visualization, from enhancing data document creation to improving accessibility and data-driven communication. Reviewing the aforementioned tools and the use of generative AI for visualization more broadly, Basole and Major~\cite{basole2024generative} discuss how generative AI methods and tools offer creativity assistance and automation within the visualization workflow, specifically highlighting a shift towards ``human-led AI-assisted'' paradigms, where generative AI not only augments the creative process but also becomes a co-creator.

Aligned with this paradigm shift, \pluto~adopts a mixed-initiative approach that leverages the capabilities of generative AI to help create semantic alignment between the chart and its corresponding text for effective data-driven communication.
However, \pluto~differs from existing chart-and-text authoring tools in three significant ways.
First, going beyond existing systems that primarily leverage unimodal information from the chart to generate text, \pluto~supports multimodal authoring combining information from both the chart (including any direct interactions with marks) and user-drafted text.
Furthermore, unlike prior tools that focus exclusively on generating complete descriptions/captions or titles, \pluto's recommendations can be leveraged in flexible ways to author not only titles and descriptions but also more fine-grained annotations and sentence completions. Second, while existing tools primarily recommend text for a given chart, \pluto's recommendations are bidirectional.
Specifically, the system suggests chart design changes like sorting or adding embellishments based on the authored text, resulting in artifacts that more clearly communicate takeaways via a combination of text and charts. Lastly, unlike existing tools that primarily rely on pre-trained knowledge in generative AI models, \pluto's recommendations are grounded in a theoretical research-based model of semantic information conveyed in visualization text~\cite{lundgard2021accessible}, ensuring the generated text covers the appropriate level of detail and is effective for communication \emph{alongside} the chart.
\section{Formative Design: Exploratory Subgroup Analysis Workflow}
\label{sec:formative}
% We developed a novel slice discovery algorithm called Divisi that extends this previous work by addressing the following design requirements:
% \begin{enumerate}[label={\bfseries G\arabic*.}, ref={\bfseries G\arabic*},itemsep=1ex]
%     \item \textit{Flexible slicing by multiple weighted criteria.} Errors are often just one of many phenomena data scientists seek to understand about the models they develop and their underlying data~\cite{holstein_improving_2019}. Rather than searching only for slices by a single binary metric, our approach was designed to support ranking by several metrics at once (e.g., positive true labels and negative predictions). \label{goal:sd-score-fns}
%     \item \textit{Configurable to prefer faster performance or more thorough search.} Existing slicing algorithms~\cite{sagadeeva_sliceline_2021,pastor_looking_2021} typically perform exhaustive search over slices that meet the minimum support constraints, which can be prohibitively slow for large, wide datasets. We aimed to allow users to view \textit{approximations} of the optimal set of slices at customizable levels of speed and coverage. \label{goal:sd-approximate}
% \end{enumerate}

To establish the design goals for our system, we adapted \citeauthor{pirolli_sensemaking_2005}'s sense-making framework~\cite{pirolli_sensemaking_2005} with insights gained from semi-structured interviews with three data scientists experienced in subgroup analysis.
Sense-making captures how analysts move between individual observations and larger-scale, more rigorous hypotheses, similar to the process of insight discovery in EDA~\cite{tukey_exploratory_1970,cabrera_what_2022}. 
However, it is unclear how the stages of sense-making might correspond to exploratory analysis steps on \textit{subgroup} data.
As such, we aimed to understand how data scientists who regularly work with subgroups interpret and reason about this type of data and translate their specific observations to high-level assessments.

We recruited three data scientists with extensive experience in subgroup analysis in different domains, including health technology (participant E1), algorithmic fairness (E2), and biomedical informatics (E3).
We asked these practitioners about the current role that subgroup analysis plays in their workflow when analyzing data or building models, and about the challenges they face when analyzing subgroups (see Sec. B.1 in the Supplementary Material for the questions used).
We also gave them open-ended interpretation tasks using example subgroups in three different interfaces, one consisting of plain text and two early variants of the Divisi interface.
Through these tasks we sought to understand how they would make sense of these subgroups and what they thought of the system design.
Each interview took place remotely, lasted around 60 minutes and was recorded and transcribed for analysis.

The practitioners we interviewed used subgroups for a variety of reasons, ranging from evaluations of model performance to understanding patient subpopulations with different treatment needs in medical data.
For example, E1 described using subgroup analysis after building a model for screening clinical trial participants, with the goal of determining what kinds of instances led to more predicted exclusions.
To validate that the model was not discriminating by any factors such as race, gender, or socioeconomic status, they explored subgroups sliced by each factor individually as well as intersections of these factors.

Experts particularly valued subgroups defined by clear-cut rules as a way to communicate model behavior and dataset patterns to less technical stakeholders.
However, consistent with the lack of a dominant subgroup analysis method in the literature, all three experts reported primarily relying on prior or external knowledge and ad-hoc visualizations to perform subgroup analysis.
They faced challenges in deciding which features to use for defining subgroups, assessing whether the differences they saw were real, making sense of interactions between features, and comparing metrics across multiple subgroups.

Based on the practices and challenges identified in these interviews, we condensed the relevant parts of the \citeauthor{pirolli_sensemaking_2005}  framework into four successively higher-level representations: the original data, sets of subgroups, schemas or mental models of behavior, and a synthesized understanding of the dataset.
In the resulting workflow, shown in Fig. \ref{fig:esa-model}, the analyst moves between these representations through three iterative activities: discovery, evaluation, and curation.
We note that this adaptation of the sense-making process is similar to other process models proposed in visual analytics, such as workflows for exploratory analysis of machine learning models by \citeauthor{cabrera_what_2022}~\cite{cabrera_what_2022} and \citeauthor{zhang_manifold_2019}~\cite{zhang_manifold_2019}.
Though the overall analysis goals are consistent with prior work, specialized algorithmic and interactive tools may be required to conduct this sense-making process on data subgroups, informing the design of Divisi.
Below we describe how the activities of discovery, evaluation, and curation map to tasks requiring special affordances for subgroup analysis, following \citeauthor{munzner_nested_2009}'s taxonomy of visual analytics design~\cite{munzner_nested_2009}.

% Key findings from the formative study: practitioners value subgroup analysis because it provides intuition about the data, and because subgroup analyses can easily be interpreted by less technical stakeholders.
% However, people most often just use off-the-shelf tools for subgroup analysis and prior knowledge to decide what to slice by.
% They find it challenging to do subgroup analysis in large datasets with many features, and they would like to explore possible ways to pre-process the data before slicing.

% \begin{enumerate}
%     \item Automatic subgroup discovery to improve rigor and help sort through many features
%     \item Balance multiple metrics of interest.
%     \item Support reasoning about interactions between multiple features.
%     \item Use subgroups to help characterize the makeup of the dataset.
% \end{enumerate}

\begin{figure}
    \centering
    \includegraphics[width=\linewidth, alt={Flowchart showing four stages, Dataset, Subgroups, Schema, and Synthesis, connected by three cyclic paths labeled Discover, Evaluate, and Curate. The Discover cycle consists of (1) Find Subgroups going from Dataset to Subgroups, and (2) Investigate Data Features going back to Dataset. The Evaluate cycle consists of (3) Schematize going from Subgroups to Schema, and (4) Search for Evidence going back to Subgroups. The Curate cycle consists of (5) Build Case going from Schema to Synthesis, and (6) Find Gaps going back to Schema.}]{figures/ESA_model.pdf}
    \caption{Proposed workflow for exploratory subgroup analysis, adapted from \citeauthor{pirolli_sensemaking_2005}'s sense-making framework~\cite{pirolli_sensemaking_2005} and informed by three expert interviews.}
    \label{fig:esa-model}
\end{figure}

\textbf{Discovery.} Identifying important subgroups to look at was a major challenge for the experts we interviewed, particularly E1 and E2.
For example, E2 recounted a project about measuring disparities in medical record data:
\begin{quote}
\textit{``There's, I don't know, thousands of diagnosis codes at different levels.... So we had to do some literature review to justify why we're including those particular ones [rather] than the other ones. The ones we included are the ones that are known to [have disparities].''}
\end{quote}
% Not only was it challenging to find the initial features to analyze, the experts also found it difficult to visualize and reason about interactions about features (E1, E2).
Given this complexity, participants wanted tools that automatically identify subgroups to provide a \textit{``less biased view of the data''} (E3).
At the same time, they expressed the need to incorporate their prior knowledge, and that of stakeholders, to guide subgroup definitions.

\begin{enumerate}[label={\bfseries T\arabic*.}, ref={\bfseries T\arabic*},itemsep=1ex]
    \setcounter{enumi}{\value{goalCounter}}
    \item \textit{Find Subgroups:} The system should provide fast, approximate algorithmic subgroup recommendations as a starting point for analysis. \label{task:find-subgroups}
    \item \textit{Investigate Data Features:} Users should be able to define and store custom rules to probe the effects of features they identify as interesting. \label{task:investigate-data-features}
    \setcounter{goalCounter}{\value{enumi}}
\end{enumerate}

\textbf{Evaluation.} After identifying potentially interesting subgroups, experts wanted quantitative, actionable ways to detect if the subgroup was meaningful to their analysis.
% While statistical significance was a factor in assessing subgroups, our respondents were most interested in just looking at the metrics that would be most relevant to downstream decisions: \textit{``It's usually associated more with some type of business outcome or cost metric... [such as] bad press, bad marketing, or to make sure they're doing things the right way''} (E1).
E1 described the difficulty of comparing metrics across several different subgroups to decide which were the most meaningful, while E2 emphasized that the relevant performance metrics could be different for different subgroups.
Participants also wanted to see how their metrics of interest would change as they added and removed features from the rules to assess intersectionality (E1, E2).

\begin{enumerate}[label={\bfseries T\arabic*.}, ref={\bfseries T\arabic*},itemsep=1ex]
    \setcounter{enumi}{\value{goalCounter}}
    \item \textit{Schematize:} The system should enable comparison of the metrics in and overlap between multiple subgroups. \label{task:schematize}
    \item \textit{Search for Evidence:} Users should be able to create variants of subgroups that reveal how individual features contribute to the metrics and interact with each other. \label{task:search-evidence}
    % \item \textit{Visualize overlap between subgroups and their coverage of the dataset.}
    \setcounter{goalCounter}{\value{enumi}}
\end{enumerate}

\textbf{Curation.} The final product in all three experts' subgroup analyses was a presentation that could allow stakeholders to interpret the subgroups and the differences in metrics associated with them.
This was particularly true for problems in which domain experts were needed to interpret the subgroups: \textit{``In our analyses we usually... present the information and then let [the medical expert] make decisions based on their results''} (E3).
To make sure they were conveying accurate information to these non-data-scientist stakeholders, participants expressed the desire to make sure they captured an overall sense of the dataset during their exploration (E1, E2, E3).
They also suggested that the subgroups they identified could be used to familiarize others with the data provided they had attained good coverage.
\begin{enumerate}[label={\bfseries T\arabic*.}, ref={\bfseries T\arabic*},itemsep=1ex]
    \setcounter{enumi}{\value{goalCounter}}
    \item \textit{Build Case:} Users should be able to build a collection of subgroups that can be presented to stakeholders.\label{task:build-case}
    \item \textit{Find Gaps:} The system should surface under-explored areas of the dataset for further schema development and refinement of the search criteria.\label{task:find-gaps}
\end{enumerate}
\section{Subgroup Discovery Algorithm}
\label{sec:algorithm}

% \item \textit{Provide ways for users to configure algorithmic results for faster performance or more thorough search.} Existing slice discovery algorithms~\cite{sagadeeva_sliceline_2021,pastor_looking_2021} typically perform exhaustive search over subgroups that meet the minimum size constraints, which can be prohibitively slow for large, wide datasets. We aimed to allow users to view \textit{approximations} of the optimal set of subgroups at customizable levels of speed and coverage. \label{goal:sd-approximate}
    % \item \textit{Support balancing multiple criteria of interest.}
Existing subgroup discovery algorithms (as reviewed in Sec. \ref{sec:related-subgroup-analysis}) are predominantly designed to be run once and to retrieve subgroups that are mathematically optimal by some predefined criterion.
While these approaches are useful when the analyst has a specific goal in mind, their running time and lack of flexibility suggest they may be less compatible with the interactive exploratory subgroup analysis process described above.
We therefore developed a novel algorithm that better supports this iterative sense-making process by addressing two key requirements implied by the design goals in Sec. \ref{sec:formative}: 

\begin{itemize}
    \item \textit{Configurably approximate search.} Scalability is a major consideration for exploratory subgroup analysis, as data scientists may want to perform subgroup discovery many times as they refine their goals and intents. To support task \ref{task:find-subgroups} (Find Subgroups), the algorithm should allow users to control how deeply to search for subgroups, allowing them to get initial results quickly.
    \item \textit{Defining and weighing multiple criteria of interest.} Existing algorithms~\cite{chung_slice_2020,pastor_looking_2021} primarily focus on a single criterion (such as error rate), and they find problematic subgroups relative to the entire dataset. However, task \ref{task:find-gaps} (Find Gaps) requires data scientists to change their search criteria on-the-fly or target the search to specific regions of the dataset. Allowing users to search by multiple criteria at once could help them more flexibly express what constitutes an ``interesting'' subgroup.
\end{itemize}

Below we describe how our algorithm meets these requirements, and we present an evaluation of its accuracy and performance on three datasets compared to prior slice discovery approaches.
Throughout these sections, we use the running example of the UCI Census Income dataset~\cite{adult_2}, which consists of records for 48,842 individuals with 12 distinct features.
We train a classification model to predict whether each individual has an income of at least \$50K/yr, resulting in an error rate of 11.5\%.
Our goal in the running example is to find meaningful subsets of the data in which the model errs more often than average.

\subsection{Problem Setup}

Intuitively, the objective of subgroup discovery is to find subsets of a dataset, each defined using a clear-cut rule, that have interesting differences in some metric compared to the overall dataset.
For example, in the Census Income dataset, we could define a subgroup using the rule \texttt{relationship = "Husband" \& age = "45 - 65"}.
We can formalize this task by defining some notation: Let $X$ be a matrix of values representing a dataset, with $N$ instances and $M$ features per instance.
Note that each feature has to be discrete or categorical in order to define rules, so numerical features need to be binned into discrete categories.
Our goal is to find subgroups $S$, where each group is defined by a rule combining up to $L$ features in the form \texttt{X1 = "v1" \& X2 = "v2" \& ...}.
We then score and order these subgroups according to \textit{ranking functions}, which are each defined as functions of some length-$N$ outcome vector $Y$ and the subgroup $S$.

The simplest way to achieve the above, i.e. generate the top rule-based subgroups for any given set of ranking functions, would be to exhaustively score and rank all possible intersections of up to $L$ features.
However, even for a relatively small dataset with 100 features and two values per feature, there are over 1.3 million rules possible with $L = 3$. % (200+200*198/2+200*198*196/6)
Divisi therefore aims to ensure that the subgroups it returns are close to those that would be returned by naive iteration, but \textit{without} enumerating all possible feature combinations.

% We note that this formulation of subgroup discovery is similar to previous slice-based evaluation techniques such as Slice Finder and SliceTeller~\cite{chung_slice_2020,zhang_sliceteller_2022}, with the addition of arbitrary user-defined ranking functions.
% Unlike other approaches such as Domino and the Spotlight~\cite{eyuboglu_domino_2022,deon_spotlight_2022}, Divisi does not depend on learned representations for each instance, so in order to work with unstructured data it requires tabular metadata features to be extracted for each instance.
% However, it is often more straightforward to generate a large set of discrete-valued features to describe unstructured data, than it is to identify which features are most discriminative for subgroup analysis.
% We demonstrate the feasibility of this approach using the Reviews dataset in Sec. \ref{sec:performance-eval}, and we discuss further applications in the Discussion (Sec. \ref{sec:discussion}).

% As with other slice discovery methods, Divisi requires that the input features be discrete or categorical. Users can specify the list of variables to be used for slicing through an external configuration in Tempo query language syntax, either by adapting the variables used for modeling with the \tqlinline{cut} command or by defining new variables altogether. In our running example, health data analyst Ava uses the binary and categorical variables they had previously defined as-is, and they discretize the continuous variables into ``low,'' ``normal,'' and ``high'' bins.

\subsection{Sampling Approach}
\label{sec:sampling-approach}

Given the discrete-valued input matrix described above, Divisi works by sampling a small number of data points, then performing approximate subgroup discovery constrained to rules that \textit{match} the sampled point. 
In other words, the results of the subgroup discovery algorithm will consist exclusively of groups that contain at least one of the sampled points. 
Our key insight is that as more rows are sampled, the likelihood of finding any subgroup that matches a reasonable proportion of the dataset increases \textit{independently} of the size of the dataset. 
Users can therefore configure how many points to sample (and by extension, the running time of the algorithm) based on the size of subgroup they are looking for.
Furthermore, by sampling specifically from rows that contain outcomes of interest (such as positive labels or errors), we can further reduce the computation needed to find the most relevant groups.

Divisi uses a beam search algorithm to progressively find high-scoring rules with more input features (illustrated in Fig. \ref{fig:slice-finding}). For each of the $n$ sampled ``source rows'' (highlighted row in Figs. \ref{fig:slice-finding}A and B), the algorithm first scores and ranks all univariate rules that contain the row according to all ranking functions separately (Fig. \ref{fig:slice-finding}C). Then, the top $k$ rules according to \textit{each} ranking function are expanded by one feature, again testing all single-feature additions that match the source row (Fig. \ref{fig:slice-finding}D). Subgroups that contain a smaller proportion of the dataset than the user-defined minimum size $p_\text{min}$ are filtered out. This process continues until the user-defined maximum number of rule features $L$ has been reached, and the algorithm returns the ranked results over all subgroups that were evaluated during any iteration. 

In summary, Divisi provides five parameters to address the goal of configurably approximate search: the sample count $n$ (default 100), a binary mask within which to sample source rows (default none), the minimum subgroup size $p_\text{min}$ (default 0.01), the beam size $k$ (default 50), and the maximum rule length $L$ (default 3).
In practice, it is sufficient to set $n$ and $p_\text{min}$ based on the size of the desired subgroups.
% \edit{In summary, Divisi provides the following parameters to address the goal of configurably approximate search, though in practice only $n$ and $p_\text{min}$ need to be set based on the user's preferences:
% \begin{itemize}
%     \item \textbf{Sample count $n$ (default 100):} the number of rows that serve as a source row for the beam search algorithm. Larger values increase running time and the likelihood of finding smaller subgroups. This is distinct from downsampling the dataset, as subgroups are still scored over the entire dataset by default. Moreover, this does not limit the number of subgroups that can be found.
%     \item \textbf{Sampling mask (default none):} a binary mask over the rows of the dataset indicating where to sample source rows from. For large datasets, sampling specifically from rows that contain the outcome of interest can reduce the number of samples needed.
%     \item \textbf{Minimum support $p_\text{min}$ (default 0.01):} the smallest proportion of the dataset that a returned subgroup can have. Larger values decrease running time by eliminating subgroups that are too small.
%     \item \textbf{Beam size $k$ (default 50):} the number of top rules that are expanded in each iteration of the beam search algorithm. Larger values increase running time by including more possible intersections.
%     \item \textbf{Maximum rule length $L$ (default 3)}: the maximum number of features that can be included in a rule. Larger values increase running time by evaluating more complex rules.
% \end{itemize}
% }

\begin{figure*}
    \centering
    \includegraphics[width=0.75\linewidth, alt={Diagram showing how the Divisi sampling approach works on a toy dataset with five columns X1-5, and an outcome metric Y. The fourth row is sampled, with values X1 = 0, X2 = 1, X3 = 1, X4 = 0, X5 = 0, and Y = 0. From this, five rules are scored, one for each feature value. The top two rules are X2 = 1 and X4 = 0, and these are expanded in the next stage by testing adding an additional feature to them. This process repeats until the maximum rule length is reached.}]{figures/slice_finding.pdf}
    \caption{Divisi's subgroup discovery algorithm takes as input a matrix of discrete-valued input features (A) and one or more score functions, in this case a Binary Outcome Rate score over the outcomes in (B). For each sampled row (C), the algorithm first scores each single-feature slice containing that row (D), then iteratively expands the top $k$ slices using other features that match the sampled row (E). In this example, $k = 2$ and the minimum slice size is 2 instances.}
    \label{fig:slice-finding}
\end{figure*}

\subsubsection{Testing Robustness} \label{sec:testing-robustness}
Prior approaches for slice discovery typically either do not account for the robustness of subgroups or they use statistical methods such as $\alpha$-investing to mitigate the false discovery rate problem~\cite{chung_slice_2020,pastor_looking_2021}. Assessing the reliability of a subgroup is important for Divisi, particularly when the group is small relative to the overall dataset; however, $p$-value thresholds may be less appropriate for exploratory analysis across several different metrics. Therefore, we split the data into \textit{discovery} and \textit{evaluation} sets~\cite{green_subgroup_2021}, such that all initial scoring and ranking occurs on the discovery set while interactive re-ranking utilizes the evaluation set. This ensures that the metrics displayed for each subgroup are obtained separately from those used to develop the results.

\subsection{Ranking Functions}

Ranking functions compute a non-negative value for each subgroup that is higher if the subgroup more closely matches a criterion of interest. Divisi can in theory support any function that can be implemented in Python and that operates on an outcome variable and a binary subgroup mask.
In the interactive interface, the user can select from the six pre-defined ranking functions below. Through experimentation on datasets with binary, categorical, and continuous target variables, we found that these functions covered the majority of use cases:
\begin{itemize}[leftmargin=*]
    \item \textit{Binary Outcome Rate (Precision).} This function simply measures the ratio of the mean of a particular outcome variable within the subgroup to the overall mean. More formally, given an outcome vector $Y$ and a subgroup $S$ containing a subset of instances, the outcome rate score is calculated as
    \begin{equation}
    \text{Score}(S; Y) = \frac{\sum_{i \in S} y_i}{|S|}
    \end{equation}
    This ranking function can be used to rank subgroups by the rate of any binary metric, such as positive labels, positive predictions, model errors, or even similarity to another subgroup. By taking the inverse of the function, we can find subgroups with lower rates than average.

    \item \textit{Binary Outcome Coverage (Recall).} Complementary to the outcome rate, outcome coverage measures the proportion of all instances with a positive outcome that are captured within a subgroup, also known as recall:
    \begin{equation}
        \text{Score}(S; Y) = \frac{\sum_{i \in S} y_i}{\sum_{i = 1}^{N} y_i}
    \end{equation}
    % Similar to the above, the coverage metric can be used with any binary metric. Although the definitions are equivalent, we refer to these functions as ``rate'' and ``coverage'' to avoid confusion with precision and recall as these typically relate to a specific set of ground-truth labels.

    \item \textit{Interaction Effect.} This function penalizes rules with extraneous slicing features by taking the ratio of the binary outcome rate in the current subgroup against the maximum rate in all subgroups defined by subsets of the current group's slicing features. For example, for the rule \texttt{A = 1 \& B = 3}, this function would divide the outcome rate by the maximum of the rates for the rules \texttt{A = 1}, \texttt{B = 3}, and the overall dataset. A value greater than 1 indicates that all features play a role in elevating the outcome rate.
    
    \item \textit{Mean Difference.} This ranking function measures the difference between the mean of a continuous outcome in a subgroup and the overall mean:
    \begin{equation}
        \text{Score}(S; Y) = \left|\frac{\sum_{i \in S} y_i}{|S|} - \frac{\sum_{i=1}^{N} y_i}{N}\right|
    \end{equation}
    The Mean Difference score can be used to rank subgroups in regression models by how different the in-group predictions are from the true values.

    \item \textit{Group Size.} To prioritize groups that have good support while not being too large, we score subgroups using a Gaussian function of the group size with a configurable ideal mean and spread.

    \item \textit{Simple Rule.} This function penalizes subgroups that are defined using too many features in the rule, using an inverse logarithmic function.
\end{itemize}
% Note that all of these functions are non-negative by definition, guaranteeing that the top $k$ slices for any weighted sum of the score functions will be among the top $k$ slices for each individual score function. This enables us to efficiently re-rank slices for different combinations of score functions, addressing \ref{goal:sd-score-fns}.

\subsection{Accuracy and Performance Evaluation}
\label{sec:performance-eval}

\begin{figure*}
    \centering
    \includegraphics[width=0.75\textwidth, alt={Six performance evaluation charts, three for running time in seconds and three for recall, within which each shows results for Census Income, Airline, and Reviews. Divisi was tested with 10, 20, 50, 100, and 200 samples taken, and compared to Lattice Search and Frequent Itemset implementations. In all results, running time decreases as the minimum subgroup size increases from 1\% to 10\%, and recall increases. Recall is always 100\% for Lattice Search and Frequent Itemset as these methods are not approximate, and it is 0.5 or above for all settings of Divisi. In Census Income, all running times are within 0.3-3 seconds. In Airline, Lattice Search performs comparably to Divisi (around 10 seconds) while Frequent Itemset ranges from 15 to 100 seconds. In Reviews, Lattice Search ranges from 100 to 4,000 seconds, while Divisi always remains under 300 seconds (Frequent Itemset fails to run).}]{figures/performance_eval.pdf}
    \caption{Average running times and accuracy (recall in top 50 returned results) for different parameter settings of Divisi, compared against a Lattice Search and Frequent Itemset approach. (We were unable to run the Frequent Itemset approach on the Reviews dataset due to excessive memory consumption, so we only report its performance on the Census Income and Airline datasets.) Shaded regions represent one standard deviation over 10 trials.}
    \label{fig:performance-eval}
\end{figure*}

To test the Divisi algorithm's ability to return results with configurable levels of approximation and speed, we conducted an evaluation of its accuracy and time performance under different parameter settings.
We performed subgroup discovery on three datasets of varying size and dimensionality: UCI Census Income~\cite{adult_2} (48K instances, 12 features), an Airline Passenger Satisfaction dataset sourced from Kaggle~\cite{noauthor_airline_2019} (129K instances, 22 features), and a subset of the Yelp review dataset\footnote{To produce a discrete-valued tabular representation of the review dataset, we used a grouped bag-of-words approach, similar to the analysis shown in Sec. \ref{sec:use-case}.}~\cite{zhang_2015_yelp} (200K instances, 2,000 features).
Further details on the datasets used are provided in Sec. A of the Supplementary Material.
% The size of the datasets ranged from 48,000 in Census Income to over 200,000 instances in Reviews, and from 12 features in Census Income to 2,000 in Reviews.

For comparison, we tested two baseline approaches: a Lattice Search algorithm similar to Slice Finder~\cite{chung_slice_2020} and a Frequent Itemset approach based on DivExplorer~\cite{pastor_looking_2021}. 
To ensure that each method would rank subgroups the same way, we used their underlying algorithms to retrieve candidate subgroups, then used Divisi's ranking functions (a Binary Outcome Rate score for error and a Group Size function) to rank the top 100 subgroups.
We measured the recall of each Divisi output against the outputs of Lattice Search and Frequent Itemset, since these two approaches are both exhaustive and produce identical results.

As shown in Fig. \ref{fig:performance-eval}, different parameter settings of Divisi can result in either more accurate or faster results. Divisi's running time (upper panels) is mostly dependent on the size of the dataset and the number of sampled rows $n$. 
In contrast, the Lattice Search and Frequent Itemset methods are markedly slower for smaller minimum subgroup sizes, which increase the number of viable subgroups. 
Divisi's recall (lower panels) generally increases as the minimum subgroup size increases and as the number of sampled instances increases, yielding a more extensive search.

Notably, Divisi remains feasible to run as data dimensionality grows regardless of parameter setting.
Its running time is comparable to the Lattice Search algorithm in the Census Income and Airline datasets, but up to two orders of magnitude faster in the wider Reviews dataset.
Moreover, Divisi can be parallelized to sample different subsets of the data in different worker threads, leading to even faster performance.
These results suggest that Divisi could be used in place of existing Lattice Search methods without sacrificing runtime or accuracy in smaller data settings, while making subgroup analysis more feasible in larger datasets.
Additionally, while rule-based subgroup discovery has been largely infeasible on large text datasets using existing methods, Divisi's efficiency makes this approach much more practical, even with thousands of word- or concept-level features.
\section{Visualization Interface}
\begin{figure*}
    \centering
    \includegraphics[width=\textwidth, alt={Screenshot of the Configuration sidebar and Subgroups Table in Divisi loaded on the Census Income dataset. A dropdown is open under a "marital-status" feature to edit the allowed values for that feature in one subgroup.}]{figures/divisi_interface.pdf}
    \caption{The Configuration sidebar (A) and the Subgroups Table (B) allow users to run the subgroup discovery algorithm and browse the rules it returns. For example, in the Census Income dataset, the first returned subgroup (C) represents people with no capital gains or losses who are married to a civilian spouse. This subgroup comprises 38\% of the dataset, and has an error rate of 25.3\%, compared to 11.6\% in the overall Evaluation Set. By clicking the dropdown next to the \texttt{marital-status} feature (D), we can test alternative values for that feature.}
    \label{fig:divisi-interface}
\end{figure*}

Integrating the algorithm described above, we developed the Divisi visualization system to support the exploratory subgroup analysis workflow synthesized in Sec. \ref{sec:formative}.
The system is designed to be installed as a Python package and run in a Jupyter Notebook environment, which is widely used in data science workflows~\cite{shen_interactive_2014}.

To summarize how Divisi's interface might support a typical workflow, let us imagine a data scientist looking for subgroups with high prediction error from the UCI Census Income dataset~\cite{adult_2}.
Upon opening Divisi in a Jupyter Notebook, the data scientist clicks \textit{Find Subgroups} to run the sampling algorithm described above. 
The results appear in the Subgroups Table (Fig. \ref{fig:divisi-interface}B), where the first subgroup consists of people who are married and have no capital gains or losses. 
The data scientist could then refine and re-rank the results using the Configuration sidebar (Fig. \ref{fig:divisi-interface}A); for example, they might upweight ``Error Rate High'' to see subgroups with higher error rates regardless of size. 
The Subgroups Table also includes controls to edit subgroups and define custom rules; for example, the user can change the value of the \texttt{marital-status} feature in a rule to see how alternative values affect the error rate (Fig. \ref{fig:divisi-interface}D).
Finally, the data scientist can drag-and-drop rules into the Subgroup Map to see how much they overlap and cover the dataset (Fig. \ref{fig:subgroup-map}).

Below we discuss in more detail how these interface components enable the three exploratory subgroup analysis activities of discovery, evaluation, and curation.
% The interface consists of three main components: the Configuration sidebar (Fig. \ref{fig:divisi-interface}A), the Subgroups Table (Fig. \ref{fig:divisi-interface}B), and the Subgroup Map (Fig. \ref{fig:subgroup-map}).
%The Configuration sidebar provides controls for the subgroup discovery process, which in turn determine the results shown in the Subgroups Table.
%Meanwhile, the Subgroup Map depicts  a bubble chart of the dataset with visual encodings to show relevant outcome metrics as well as subgroup membership.
% Users can select subgroups from the table or drag-and-drop them into the map to visualize their overlap and coverage.

\begin{figure*}
    \centering
    \includegraphics[width=\textwidth, alt={Three screenshots of the Subgroup Map shown on the Census Income dataset in different states. In (A), no subgroup is selected, and shaded bubbles spread around the chart can be seen to represent model errors. In (B), three subgroups are selected, and the groups overlap such that the first overlaps with the second and third, but the second and third do not overlap with each other. In (C), a subgroup is hovered in the Subgroups Table and all bubbles are grayed out except those that match the hovered group.}]{figures/subgroup_map.pdf}
    \caption{Different states of the Subgroup Map on the UCI Census Income dataset~\cite{adult_2}: (A) an overview of the dataset with no subgroups selected, (B) intersections between three selected subgroups, and (C) highlighting the points that match a subgroup when hovered in the Subgroups Table. Filled-in bubbles indicate classification errors for the income prediction task; each bubble's size indicates the number of instances it contains.}
    \label{fig:subgroup-map}
\end{figure*}

\subsection{Finding and Ranking Subgroups}
\label{sec:vis-discovery}

Users initialize Divisi in a Jupyter Notebook by providing a dataset containing discrete feature values and one or more outcome metrics. 
For example, for the Census Income dataset, we could provide a binary indicator of whether each instance was mispredicted as the outcome metric, as well as the true and predicted values.
Upon launching Divisi, users can click the \textit{Find Subgroups} button to run the subgroup discovery algorithm and populate the Subgroups Table with an initial list of rules, completing task \ref{task:find-subgroups}.
Ranking functions to order the subgroups are automatically generated based on the provided outcomes, and they can be edited on the fly in the Ranking Functions section of the Configuration sidebar (Fig. \ref{fig:teaser}B).
For example, after passing in the ``Error'' metric, Divisi automatically generates a Binary Outcome Rate function to find subgroups where that metric is higher than average.

Top subgroups are computed for every ranking function provided during discovery, so the user can later sort subgroups by any weighted combination of these functions.
The Ranking Functions area allows users to toggle functions on or off, as well as to choose a weight for each function in four increments.
Subgroups are instantly re-ranked as the user adjusts the ranking function configuration, providing rapid feedback about what combination of functions leads to the most interesting results.

To perform task \ref{task:investigate-data-features} (Investigate Data Features), users can use the \textit{+ New Rule} button to define a custom subgroup.
Rules can be defined using a simple syntax, such as \texttt{"marital-status" = "Married-civ-spouse" \& "education" = "Some-college"}.
% This editing functionality also provides users the ability to experiment with ``or'' and ``not'' operations, which are not currently used in the subgroup discovery algorithm because it would drastically increase the number of possible subgroups.
Upon entering the rule, Divisi automatically computes all of the metrics for the new subgroup.

After finding one or more subgroups of interest, users can save those groups to the Favorites, which can later be viewed separately from the search results. This helps in building a case that can be presented to others (task \ref{task:build-case}).

\subsection{Assessing Feature Interactions}
\label{sec:vis-evaluation}

Addressing task \ref{task:schematize} (Schematize), the Subgroups Table lists each retrieved or custom rule along with a summary of the metrics within the subgroup represented by the rule.
Sparkline-style charts, similar to prior subgroup analysis interfaces~\cite{kahng_visual_2016,zhang_sliceteller_2022}, allow users to quickly scan over the list of subgroups and identify general patterns in the metrics as well as the surfaced data features.
% However, as with many subgroup discovery algorithms, it is important to note that multiple results could represent very similar sets of instances.
% To assess whether multiple subgroups have a high degree of overlap with one another, users can select the subgroups of interest or drag them into the Subgroup Map
This can help users combine insights across multiple subgroup results to build a higher-level understanding of one area of the dataset.

% \begin{figure}
%     \centering
%     \includegraphics[width=0.5\linewidth]{figures/slice_feature_editing.pdf}
%     \caption{The lightweight editing tools allow users to interactively test alternative values (or combinations of values) and see how their metrics change.}
%     \label{fig:slice-feature-editing}
% \end{figure}
The Subgroups Table also provides lightweight rule editing functionality to help users quickly test hypotheses about feature interactions, addressing task \ref{task:search-evidence} (Search for Evidence).
By clicking a feature's name in a rule, users can toggle that feature on and off to instantly see how the metrics change.
The feature values can also be adjusted through a dropdown menu, allowing users to select one or more alternative values as shown in Fig. \ref{fig:divisi-interface}D.
For more fine-grained changes, users can use the same syntax described in Sec. \ref{sec:vis-discovery} to edit any subgroup's definition.

\subsection{Visualizing Subgroup Overlap and Coverage}
\label{sec:vis-curation}

As with many subgroup discovery algorithms, Divisi can return subgroups that have many instances in common despite being based on different features (e.g., people who are married would likely overlap with people with a relationship type of ``husband'').
However, existing tools do not help the user assess overlap and coverage, potentially leading to analyses that focus too heavily on small areas of the data.
Divisi includes a novel Subgroup Map visualization that serves three purposes: (a) helping data scientists check whether multiple subgroups have a high degree of overlap with one another, (b) showing how much of the outcome has been covered by the selected subgroups, and (c) providing an overview of the dataset structure that can point to new areas to explore.
Early designs of Divisi used UpSet plots~\cite{2014_infovis_upset} or Venn diagrams, but in initial feedback with data scientists we found that users preferred the spatial dataset overview provided by a scatter plot, as is common in subgroup analysis tools for unstructured data~\cite{suresh_kaleidoscope_2023,robertson_angler_2023,Liu2019}.
% \footnote{Currently only binary outcomes are supported in the Subgroup Map, but the visual encodings could easily be extended to continuous outcomes in the future.}

We opted to use a dimensionality reduction plot using the t-SNE algorithm~\cite{maaten_visualizing_2008} as the starting point for the Subgroup Map.
Although dimensionality reduction can introduce distortions in proximities between points, it provides a useful way to navigate the dataset and depict groups of points that is well-established in visual analytics~\cite{Brehmer_2014_dr}.
To mitigate overdraw for large datasets, we visually simplify the plot by grouping together points that are within a threshold distance and have identical properties (e.g. same outcome value, same subgroup membership). 
This yields a bubble chart in which each bubble's size represents the count of its constituent points, its color represents the outcome value of those points, and its position is the centroid of their 2-D coordinates.
Finally, we perform a very short force-directed relaxation of the layout to remove any overlaps between bubbles, producing the starting view of the dataset as shown in Fig. \ref{fig:subgroup-map}A.

When the user selects a subgroup of interest or drags one from the Subgroups Table into the Subgroup Map, borders around the points are added to show which bubbles represent instances contained in the subgroup. 
Fig. \ref{fig:subgroup-map}B shows the Subgroup Map after selecting three groups from Census Income: married individuals with some college education (blue), married men aged 45-65 with no capital gains (orange), and married women (green).
For any bubble that contains instances in one or more subgroups, the border is divided into an equal-length arc for each subgroup that contains the bubble.
For instance, we can see that the blue subgroup intersects the orange group in Fig. \ref{fig:subgroup-map}B, based on the presence of bubbles with half-blue, half-orange borders.
Towards task \ref{task:schematize}, this visual depiction of subgroups enables users to visually map color-coded subgroups to regions of the plot and to identify when subgroups overlap with each other.

The Subgroup Map includes several secondary interactions to help users gain more information about subgroup overlap and coverage, further addressing task \ref{task:schematize}.
The Subgroup Intersections panel in the bottom-right corner of the map summarizes the size and outcome rate within each combination of subgroups (represented by a bubble glyph).
Hovering on a subgroup in the Subgroups Table grays out all bubbles on the map except those that contain instances matching the subgroup, helping users quickly check the location and overlap of a rule (Fig. \ref{fig:subgroup-map}C).
Hovering on or selecting a region of the map, meanwhile, shows a Distinguishing Feature that is most unique to that selection in the bottom-left corner of the map.

If a user finds an area of the dataset that they would like to describe using subgroups, they can lasso-select that area and conduct a targeted subgroup search, addressing task \ref{task:find-gaps}.
The Divisi algorithm enables efficient search within a selected region by simply adding another ranking function called the Selection Score, which is simply a Binary Outcome Rate function where the outcome is true if the instance is part of the selection. %\footnote{As described in Sec. \ref{sec:testing-robustness}, Divisi splits the data into a discovery and an evaluation set, and only visualizes the evaluation set to avoid generating and testing hypotheses on the same data. To build the Selection Score, we take the selected evaluation set instances and collect a similarly-sized set of their nearest neighbors in the discovery set.}
Using the Subgroup Map, the user can transition between high-level assessments of the dataset and retrieved subgroups within areas that they discover.

\subsection{Implementation Details}

Divisi is implemented as a Jupyter Notebook widget using the \texttt{anywidget}\footnote{\url{https://http://anywidget.dev}} library, with a Python backend and a Svelte\footnote{\url{https://svelte.dev}} frontend.
Visualizations are created using D3.js\footnote{\url{https://d3js.org}} for rendering and layout, and Counterpoint~\cite{sivaraman_2024_counterpoint} for state management.
Divisi is open-source and can be installed from PyPI and GitHub\footnote{\url{https://github.com/cmudig/divisi-toolkit}}.
\section{Use Case}
\label{sec:use-case}

% ** This dataset might not be the best to use for our use case because they created the dataset following 19 harm categories, which we are basically just replicating using Divisi **

To demonstrate a possible analytical process using Divisi on real data, we applied the system to the task of identifying large language model (LLM) prompts that can lead to unsafe responses.
While LLMs' emergent capabilities enable remarkable performance on many tasks, these models can also generate offensive or harmful content if not appropriately aligned with the intended values of its developers~\cite{ji_beavertails_2023}.
What kinds of prompts would result in unsafe responses? Moreover, what prompts might lead to more \textit{heterogeneity} in response safety (some responses are safe, others unsafe)? 
% We aimed to evaluate whether these questions could be answered using an exploratory subgroup analysis approach with Divisi.
To answer these questions, we analyzed the PKU-SafeRLHF dataset~\cite{ji_pku-saferlhf_2024}, which consists of about 145K LLM-generated responses to about 38K prompts. %, served as the context of our use case.
Each row contains a free-text prompt, two alternative responses to the prompt, and labels about each response's perceived safety. % assigned by a team of annotators with AI assistance.
Because the prompts were themselves created by asking various LLMs to generate requests in ``harm categories'' predefined by the dataset creators, we were able to corroborate our analysis against these categories.
However, such metadata is not required to use Divisi.

\begin{figure*}
    \centering
    \includegraphics[width=\textwidth, alt={Two screenshots of Divisi showing subgroups on the PKU-SafeRLHF dataset. In (A), the top subgroups for responses that are Both Unsafe are "launder, laundered", "poison, poisonous" and "boss, don, sir"', and "phishing, ddos, botnet.'" In (B), two subgroups are selected that have high rates of Different Safety: "address, addresses" and "try, trying". The Subgroup Map shows that the overlap of these two subgroups has an even higher rate of heterogeneity than either subgroup individually.}]{figures/llm_safety.pdf}
    \caption{Subgroups identified during exploratory subgroup analysis on LLM safety evaluation data~\cite{ji_pku-saferlhf_2024}. Divisi surfaces prompts involving criminal activity as consistently unsafe (A). For prompts with more heterogeneous responses (B), one subgroup involves asking the LLM to provide people's addresses (blue group), which becomes more heterogeneous when adding the words ``try'' or ``trying'' (orange group).}
    \label{fig:llm-safety}
\end{figure*}

We used a grouped bag-of-words approach to derive metadata features for each prompt.
The unique words in the prompts were clustered using a simple word embedding model, after which we took the top 5,000 word clusters and used the presence of any word in a cluster as each metadata feature.
Because these clusters were derived from an embedding model, they represented a variety of concepts, including topics, keywords, and grammatical functions.
For instance, one feature checked whether the prompt contained cyberattack-related words (\texttt{phishing, ddos, botnet}), while another checked for quantifier words (\texttt{any, some, few, enough, least}).
For each prompt, we defined the binary output metric ``Both Unsafe'' to measure whether both candidate responses were rated unsafe, and ``Different Safety'' to capture when one response was rated safe and the other unsafe.
Running the Divisi algorithm on the 75K training data instances with 5,000 features and 200 sampled rows took roughly 6 seconds.

In the Divisi interface, we first enabled a ``Both Unsafe'' Binary Outcome Rate ranking function to find out what kinds of prompts were associated with consistently unsafe responses.
We used Divisi's Jupyter Notebook integration to pull matching instances for each returned subgroup, allowing us to peruse example prompts.
As shown in Fig. \ref{fig:llm-safety}A, many of the subgroups with the most consistently unsafe responses had to do with specific criminal activity, such as money laundering and cyber-attacks.
The fifth subgroup on the list involved an intersection of \texttt{enforcement, personnel} and \texttt{law, laws}, containing prompts about evading the authorities for which responses were both unsafe 84\% of the time.
Interestingly, by disabling the individual features one by one, we could see that the presence of \texttt{enforcement, personnel} was almost sufficient to yield consistently unsafe responses (80\%), but \texttt{law, laws} alone was only consistently unsafe 52\% of the time.
This could suggest that responses to prompts about violating laws that do not directly mention law enforcement are considered less ``unsafe.''

We also analyzed prompts in which one response was labeled safe and the other unsafe, to understand areas of greater disagreement or heterogeneity.
As shown in Fig. \ref{fig:llm-safety}B, a subgroup of prompts with requests for addresses occupied one corner of the Subgroup Map and showed a Different Safety rate of 29\%, compared to 15.9\% on average.
By hovering on each rule in the Subgroups Table, we looked for other subgroups that might cover the same area of the plot, and we found that the feature \texttt{try, trying} overlapped with the addresses bubbles substantially.
In fact, the Subgroup Intersections chart showed that for prompts that matched both rules, the rate of heterogeneity was over 38\%.
This intersection, which was small and would have been difficult to find without Divisi's visualization of overlap, could point to linguistic features that might make LLMs more likely to avoid completing the user's request.

Using Divisi, we were able to build a case comprised of several subgroups defined by interpretable, clear-cut rules, without significant manual analysis but still curated for meaningfulness.
After plotting these subgroups on the Subgroup Map, we observed that there were still many regions of heterogeneous safety that were uncovered by our selected groups.
With further analysis, we could examine this ``long tail'' of prompts by creating progressively smaller subgroups, to see if additional patterns emerge.
\section{User Study}
\label{sec:user-study}

To evaluate data scientists' use of exploratory subgroup analysis, we conducted a think-aloud study with Divisi centered around three research questions:
\begin{enumerate}
\item What is the role of subgroup analysis in data scientists’ existing workflows?
\item What insights do data scientists uncover during the different stages of exploratory subgroup analysis?
\item What opportunities do data scientists perceive to use exploratory subgroup analysis in their work?
\end{enumerate}

\subsection{Study Design}

Each think-aloud session lasted around one hour and consisted of an introduction and pre-study survey, two hands-on tasks using the Divisi interface, and a final debriefing interview (see Sec. B.2 in the Supplementary Material for the questions used). 
% We aimed to understand the current role of subgroup analysis in data scientists’ existing workflows through pre-study questions and by asking participants how they would approach the subgroup analysis tasks before using Divisi. 
% % Participants were encouraged to  throughout the analysis process.
% Then, we gathered insights into the patterns and discoveries data scientists could uncover through the two tasks. 
% Then, with the debrief questions, we gathered new opportunities participants perceived, both from their experiences with Divisi and from their own project contexts. 
% Additionally, we also evaluated how data scientists perceived Divisi’s different features and how well the features support the needs of data scientists for subgroup analysis, by gathering their feedback on the tool throughout the study session. 
%add more details of participants?
In the introduction phase, we gathered background information about the participants’ experience with data science and subgroup analysis, setting the context for the tasks that followed. Participants completed a pre-study survey that contained questions pertaining their occupation, years of experience with data and machine learning, their familiarity with Python, and their familiarity with and approach to subgroup analysis. 
%Participants were also encouraged to provide additional context verbally.

The two tasks in the study required participants to use Divisi to analyze the Airline Passenger Satisfaction dataset mentioned in Sec. \ref{sec:performance-eval}. 
The dataset contains results of a survey of 129,880 passengers’ satisfaction with the airline based on various aspects of the airline services. 
The 22 features in the dataset included demographic information (e.g., Gender and Age), traveler information (Customer Type, Type of Travel, Class), flight information (Flight Distance, Departure Delay in Minutes, Arrival Delay in Minutes), ratings for individual aspects of the flight experience (inflight Wi-Fi service, cleanliness, etc.), and an overall satisfaction rating of satisfied or not satisfied. We converted the 5-point ratings for aspects of the flight experience into ``not satisfied,'' ``neutral,'' and ``satisfied'' to simplify the space of possible subgroups for the user study.
The two tasks were as follows:
\begin{enumerate}[label={\bfseries Study Task \arabic*:}, ref={Study Task \arabic*},itemsep=1ex,labelindent=0pt, wide=0pt]
    \item \textbf{Dissatisfaction.} Participants were asked to discover and interpret subgroups of data that could provide insights into what types of customers tended to be dissatisfied with their experience. Overall, 57\% of instances in the dataset had an overall rating of dissatisfied, meaning that the outcome of interest was fairly common. We asked participants to perform any analysis they would want to be able to present a comprehensive report to stakeholders at the airline company.
    %An example of how the interface looked during this task, and the subgroup results that appeared, is shown in Fig. \ref{fig:airline-example}. 
    \label{study-task:dissatisfaction}

    \item \textbf{Model Errors.} We trained a binary classification model to predict each passenger's overall satisfaction rating, resulting in an error rate of 4.8\%. Participants were then asked to find out what subgroups tended to have higher-than-average model error rates. Similar to the dissatisfaction analysis task, participants were told to curate their insights for stakeholders of the airline company. This task was slightly more difficult because of the involvement of a classification model as well as the lower outcome base rate, which made the discovered rules slightly more complex to interpret. \label{study-task:error}
\end{enumerate}
Before the tasks began, the interviewers provided a brief tutorial of how to use Divisi using an annotated screenshot. Participants were also given guidance during the session on how to use the tool for specific tasks that they expressed interest in performing.
After completing the tasks, we asked participants to evaluate the usefulness of the various features of Divisi, and to reflect on how the system could apply to their past projects. 

We recruited 13 data scientists with some experience with machine learning and data science, particularly involving large-scale tabular or text data. 
(None of the experts from the formative work also participated in this study.)
Participants were compensated \$20 USD, and the protocol was approved by our institutional IRB.
Each one-hour session was conducted on Zoom and was recorded and transcribed for analysis.

\subsection{Analysis}
To analyze the data collected during the study, we took an open coding approach on the transcripts and video recordings of the study sessions. Two of the authors each coded all transcripts, discussing disagreements and reaching consensus as needed, to understand participants' perspectives and answer the three research questions above. The videos were used to supplement the transcripts with specifics about the participants' actions using the system. The coders generated 483 codes after annotating all of the study transcripts. We then used affinity diagramming to group codes into higher-level categories to reveal broader themes and opportunities.

\subsection{Results}
Overall, participants were able to uncover a broad range of insights into both dissatisfaction and model errors in the Airline dataset using Divisi.
They also compared the process of exploratory subgroup analysis to their current workflows, and they suggested ways to make Divisi easier to learn and more broadly applicable.

\subsubsection{Current Practices: Participants have few structured ways to perform subgroup analysis}
\label{sec:results-current-practices}
Participants in our study had varying degrees of experience with subgroup analysis (9/13 reported having used it in a prior project), but none used tools specific to subgroup analysis.
Many of those who used subgroup analysis mentioned using off-the-shelf programming libraries such as \texttt{pandas} and \texttt{matplotlib} (5/13 participants) to manually create filtering rules, which limited their opportunities for discovery: 
\begin{quote}
\textit{``When I'm doing data analysis, most of the time it's just for creating the plots using matplotlib. I'm just exploring the features that I'm interested in, but I cannot explore all of the things.''} (participant 1, denoted P1)
\end{quote}
When it came to interpreting models or identifying features associated with an outcome, there were few interactive ways to do so. 
Some participants described using clustering analyses to identify useful subgroups in an unsupervised manner (3/13).
However, these participants found clustering to be cumbersome for exploratory analysis because of the challenge of going into each cluster and interpreting what kinds of instances it contained.
Instead, most others described training simple models such as logistic regression to predict the outcome of interest, \textit{``and then just [seeing] what that spits out''} (P10).
Participants (8/13) mentioned that compared to these strategies, using Divisi would make it easier and faster to narrow down which variables to focus on:
% ``I think it'd be really helpful to see the slices that i'm not aware of'' (P8)
\textit{``Now I have a starting point. I can just start diving in''} (P6).
% ``this is the equivalent of like diving straight into the patient and being like, these people had symptoms, these people didn't'' (P6).

\subsubsection{Discovery: Participants identify surprising insights in subgroup results}
\label{sec:results-discovery}
All participants were able to quickly make assessments of what types of instances led to both dissatisfaction (\ref{study-task:dissatisfaction}) and model errors (\ref{study-task:error}), performing \ref{task:find-subgroups} (Find Subgroups) using the results in the Subgroups Table.
A strategy taken by several participants (6/13) was to glance over the table and find individual features and values that appeared frequently: \textit{``From this page overall I could see that people who are not satisfied with the Wi-Fi service have higher dissatisfaction rate''} (P7).
In this way, participants found that people with dissatisfied or neutral ratings with Wi-Fi service, gate location, and online boarding often tended to be dissatisfied overall.
Likewise, in \ref{study-task:error}, customers who were not part of the airline's loyalty program (identified by 8/13 participants) and those with several neutral ratings (7/13) were frequently found to be mispredicted.

In many cases, the patterns participants observed in the subgroup results went against their intuitions, suggesting that they might not have uncovered the same subgroups had they been guided only by their own expectations.
For instance, several users described expecting to see passengers who were dissatisfied with the flight delay (4/13 participants) or baggage handling (P5), neither of which were commonly-surfaced subgrouping features: \textit{``It is kind of surprising to me, because I think the departure delay... will be the most important thing that people will complain about''} (P1).
To verify that dissatisfaction with delays was indeed less than the other returned subgroups, P1 created a custom rule representing high departure delay and compared its metrics to the rest of the table (an instance of \ref{task:investigate-data-features}, Investigate Data Features).

\begin{figure*}
    \centering
    \includegraphics[width=\textwidth, alt={Screenshot of Divisi from the user study in which a participant has three subgroups selected and is visualizing their overlap in the Subgroup Map. Almost all shaded bubbles (Dissatisfaction instances) are covered by the selection. In the Ranking Functions, the user has both Dissatisfaction and Dissatisfaction Coverage enabled with equal weighting.}]{figures/airline_example_user.pdf}
    \caption{During \ref{study-task:dissatisfaction}, P13 investigated the intersections of three rules involving the passenger's fare class and their satisfaction with the Wi-Fi service and online boarding features. They decided to interpret these subgroups because as shown in the Subgroup Map at right, they cover almost all of the shaded (dissatisfied) bubbles.}
    \label{fig:airline-example}
\end{figure*}

Participants used the Ranking Functions extensively during the discovery process.
Although many participants (5/13) found the ranking functionality confusing at first, they ultimately found it helpful to tailor the results to the kinds of subgroups they were looking for.
The most common criterion was ranking by the Binary Outcome Rate function for dissatisfaction or error (11/13 participants); however, many (7/13) also spent time prioritizing subgroups using the Binary Outcome Coverage function.
To get a holistic sense of the dissatisfied or mispredicted instances, participants saw coverage as a way to find subgroups that were \textit{``actually most representative of that group''} (P8).
Users (6/13) also adjusted the Simple Rule function for the purposes of familiarizing themselves with the dataset's variables and finding subgroups that would be easier to explain to a stakeholder: \textit{``Whatever conclusions you have, you want them to be simple enough. People usually don't care about if you... go very deep on different clusters of customers''} (P12).
Putting these desiderata together, P11 described how they set up their ranking weights:
\begin{quote}
\textit{``First, I don't want the rules to be too complex, so I want to keep it simple, but not that simple. Not just one [feature]... And I want these rules to be representative. So actually, I want the size to be like around 10\% [of the dataset]... And also I want the percentage of the errors as high as possible.''}
\end{quote}
Since different users expressed different criteria for what a ``good'' subgroup would look like, we observed more variation in analysis paths than if only one set of rankings were available.

% \begin{enumerate}
    % \item By looking over the slice results list, participants identify single features that seem to contribute highly to the outcome of interest. ``so from this page overall I could see that people who are not satisfied with the Wi-fi service have higher dissatisfaction rate.'' (P7) ``I see customer type in most of these subgroups, and there are all disloyal customers, which could indicate that the model's not really good at using the variable customer type.'' (P7) ``Well, customer type seems to be really important seems to... be, I guess, a tricky predictor'' (P8) ``It's interesting to me that all of these are like neutral predictors. Which in retrospect makes sense. If I feel neutral about it, then I don't want to say I'm dissatisfied or not about it.'' (P6) 
    % \item Participants frequently find surprising subgroups and patterns in the dataset. ``It is kind of surprising to me, because I think the departure delay... will be the most important thing that people will complain about'' (P1).
    % \item Participants take diverse strategies to manipulate on the ranking functions to achieve the slices they think will be most useful. ``That's like my 1st kind of criteria I have for slicing the data like, which of these slices are actually most representative of that group.'' (P8) ``so at this point, I want to see the error is as much as as high as possible.'' (P11) ``the 1st thing was, I wanted to know where was the model performing worse? Where was the highest error? ... So I upped the weight of the error.'' (P8) ``whatever conclusions you have. You want them to be simple enough. People usually don't care about if you like, go like very deep on, like different clusters of customers and stuff.'' (P12) Mixed criteria: ``First, I don't want the rules to be too complex, so I want to keep it simple, but not that simple. Not just one. Yeah, so not that simple. And I want these, I want these rules to be representative. So actually, I want the cons to be like around 10\% that's what I mentioned. And a score functions here. And also I want the percentage of the errors as high as possible.'' (P11) ``I kind of want to, you want to see like, the main effects before you see the like, these are kind of like, oh, the interaction effects.'' (P5)
% \end{enumerate}

\subsubsection{Evaluation: Participants use subgroup editing tools extensively to test feature interactions}
\label{sec:results-evaluation}

Participants schematized their understanding of subgroup behaviors (\ref{task:schematize}) and searched for additional evidence (\ref{task:search-evidence}) in diverse ways, most often by editing existing subgroup rules and creating new ones.
Most straightforwardly, several participants tested alternative values for features to observe correlations in outcomes: \textit{``The longer the flight [the greater the satisfaction]. Your... 10,000-mile business flight has only 24\% dissatisfaction, but at 500 [miles] it's 52\%''} (P10).
By performing this lightweight editing in subgroups defined by multiple features, participants (6/13) reasoned about interaction effects, i.e., feature values that had more pronounced impacts on outcomes when a different feature took a particular value.
For instance, by interactively toggling between economy and business class customers, P12 found that \textit{``in-flight service doesn't matter as much for economy customers.''}
This type of interaction helped 3/13 participants conclude that combining ratings with different polarities could cause higher error rates: \textit{``People are maybe very satisfied with something, but they may be unsatisfied with something else''} (P1).

In addition to editing rules, 6/13 participants used the custom rule feature to test hypotheses, driven by both prior knowledge and the patterns they observed.
For example, some participants (4/13) tried to extend rules they found in the Subgroups Table with additional features:
\begin{quote}
\textit{``I think I have to look closer [at] people who were neutral on the in-flight service, and then combine more within that subgroup to see if there's something actionable... Maybe that experience intersects with something else, and those two things are intertwined and actionable to improve the service.''} (P8)
\end{quote}
Except for one experienced user who wanted to add features to a rule in bulk using Python code (P10), Divisi's rule editor was sufficient for any query that users wanted to make.
Custom rules thereby helped participants both investigate features of interest (\ref{task:investigate-data-features}) and build up a more robust understanding of feature interactions (\ref{task:schematize}).

While evaluating subgroups, some participants (5/13) wondered whether the subgrouping features could be appearing due to randomness or spurious correlations, an evaluation aspect that Divisi does not currently support.
While Divisi splits the data into discovery and evaluation sets to mitigate false discoveries (Sec. \ref{sec:testing-robustness}), it is always possible that two features are highly correlated so that slicing by one of them is sufficient to slice by both.
For example, looking at a rule with high overall dissatisfaction but a satisfied rating on the departure/arrival time, P7 expressed the desire to dig deeper into other feature values within the subgroup to test whether \textit{``people in this subgroup are not satisfied with other aspects''}.

% \begin{enumerate}
    % \item Participants use slice customization heavily to understand how different feature values interact with each other. Correlations in feature values: ``The longer the flight. You're most like a 10,000-mile business flight has only 24\% of satisfaction. But at 500 it's 52.'' (P10) ``If you like. Have a lot of dissatisfaction ... you kind of can say that like in flight service doesn't matter as much for economy customers as [for business]'' (P12) This helped a few participants intuit that conflicting ratings could cause higher error rates: ``So people are maybe very satisfied with something, but they may be unsatisfied with something else.'' (P1)

    % \item Participants envision creating custom slices to test hypotheses driven by both prior knowledge and the patterns they observe. ``So I think I have to look closer of people who were neutral on the in-flight service, and then combine more with within that subgroup to see if there's something actionable. That and how like, maybe that experience intersects with something else, and those 2 things are intertwined and actionable to improve the service.'' (P8) ``what I would actually do in this situation is I would probably get go like into Python myself, and, like, find all these columns, and like, make all the neutral.'' (P10)
    % \item Participants express skepticism that the features and values shown in the tool are always representative and reliable due to correlation vs. causation. ``Maybe it's just pure random chances. Yeah, I mean, because models are random, like they? Probably it's probably just due to chance.'' (P9) being very satisfied with dep/arr time convenient: ``maybe people in these subgroups are not satisfied with other aspects.'' (P7)
% \end{enumerate}

\subsubsection{Curation: The Subgroup Map helps more experienced participants choose subgroups to prioritize.}
\label{sec:results-curation}

We observed that Curation actions, i.e. building a case (\ref{task:build-case}) and finding gaps (\ref{task:find-gaps}), were primarily done by participants with more familiarity with subgroup analysis. 
Nine out of 13 participants used the Subgroup Map and targeted search features to build understanding about the contents of the dataset and to choose which subgroups to analyze further.
Some (4 out of 9 who used the map) looked at the Distinguishing Features for each bubble to associate types of people with areas of the map: \textit{``I always expect... there's, say, a portion of the graph to be something like satisfied loyal customers.''} (P9)
Others (3/9) performed slice search within selected regions to characterize them using rules.
Furthermore, by plotting subgroups on the map and seeing how they were distributed, participants (4/9) noticed that some subgroups could be further divided into more detailed categories: \textit{``So you have like, multiple kinds of disloyal customers''} (P12).

Consistent with the Subgroup Map's goal of communicating overlap and coverage, participants (6/9) used the visualization to eliminate subgroups that covered very similar sets of instances on the plot: \textit{``They're just like, not satisfied with various aspects, but they fall in the same place''} (P12).
Others (3/9) used it to assess at-a-glance how many of the positive outcomes were covered by their selected subgroups, to decide when they had completed the task to their satisfaction.
For example, P13 began by selecting the top five subgroups, visualizing them in the Subgroup Map, then narrowing them down to capture the same population with just three groups (Fig. \ref{fig:airline-example}).
Two participants followed an iterative methodology (similar to \ref{task:find-gaps}, Find Gaps) of adding and removing subgroups from the map to cover as many shaded bubbles (positive outcomes) as possible, while conducting targeted search within the white bubbles (negative outcomes) to find subgroups to remove.
Participants who used the Subgroup Map to understand coverage found it particularly helpful:
\begin{quote}
\textit{``Without... the visualization, I would think I get the representative groups. But with it, I'm thinking, still... many errors are missing. So I think it's a good way to use the visualization, because it makes me think of if I really grasped the majority of the errors.''} (P11)
\end{quote}

% \begin{enumerate}
    % \item Participants use the subgroup map to spatialize their understanding of the different characteristics in the dataset. lasso selecting and finding subgroups within a selection ``Because they are close to each other, and I want to see like, what kind of people they will be.'' (P1) ``I didn't even realize age is a variable in this. We've not looked at age.'' (P10) ``I always expect, like, you know, there's say, a portion of the graph to be something like, you know, satisfied loyal customers, or something like that.'' (P9) ``so you have like, multiple kinds of disloyal customers'' (P12) 
    % \item Participants use the subgroup map to choose the slices they want to analyze further. `` they're just like, not satisfied with various aspects, but they fall in the same place'' (P12) ***``Because if without, without the image or the visualization, I would think I get the representative groups, but with with it, I'm thinking, still other still, many errors are missing. So I think it's a good, good way to use the visualizations, because it makes me think of if I really grasp the majority of the errors.'' (P11)
% \end{enumerate}

\subsubsection{Perceived Opportunities: Making Divisi more actionable and broadly accessible}
\label{sec:results-perceived-opps}
Participants envisioned many ways in which Divisi could support their current and prior projects, including better understanding clinical trial results (P4), analyzing test results in educational experiments (P8), and debugging image generation models using keywords in their caption text (P13).
They suggested ways to extend the exploratory subgroup analysis workflow to accommodate these use cases, such as measuring the statistical significance of subgroup differences (P7) or displaying metrics that were more relevant to the use case (P9).
Participants also wondered if Divisi's workflow could be applied to data types that were more difficult to slice by, such as images or representations of molecules (2/13).
These data types would likely require clever ways to create metadata features for rules that are both clear-cut and interpretable.

Participants considered how metadata features were defined in our user study tasks, and some wanted to adjust the feature definitions to help them mine better subgroups. 
For example, some participants (4/13) found rules with ``neutral'' rating values not very actionable, and they wanted to either exclude that feature value from subgroups or re-bin the ratings into ``satisfied'' and ``not satisfied.''
Others described wanting to remove features that they suspected would mostly contain spurious correlations, though they wanted to do so without injecting personal bias into the process: \textit{``I have my own mental heuristic, for which features... are very low signal-to-noise, and I'm trying to remove the noise''} (P6).
While participants could in theory have edited the code to create the grouping features within the Jupyter Notebook, it would be helpful to directly support this aspect of subgroup analysis within the tool.

Finally, several participants commented that the Subgroup Map (6/13) and the Ranking Functions (5/13) were initially difficult to understand.
These users suggested simplifying how subgroups were initially presented to them, such as by enabling only one ranking function at first (P4).
Those confused by the Subgroup Map commented that clearly indicating how the chart was generated would likely make it more trustworthy (P12) and clarify the relationship between bubbles and subgroups (P1).
While these features were largely perceived as useful (8/13) once participants were familiar with how they worked, more guidance could help those less experienced with subgroup analysis leverage them effectively.

% \begin{enumerate}
    % \item Participants want the tool to provide subgroup results that directly apply into their work, such as statistical tests. ``it's not involving the tests that I'm usually using or the metrics that I'm usually using, so I am not sure how to interpret the results academically.'' (P7) ``the numbers are there, and it's something that I can tangibly report.'' (P9)
    % \item Participants want to filter the subgroup space for actionability and simplicity, provided that they can do so without biasing their search. ``Online boarding of neutral... I don't know what neutral means.'' (P5) ``So basically, I want to get rid of this one type of travel and class. Actually, we can do nothing about that, like the class eco class is just normally or naturally bad. So if you want to pick class of eco, you will not, like, definitely not get that good service.'' (P11) ``I have my own mental heuristic, for which features, probably like don't are like very low signal to noise, and I'm trying to remove the noise'' (P6)
    % \item Participants find the searching and mapping features unintuitive without explanation. ``It's good to be a user to see this visualization. But how to interpret it to audience?'' (P3) ``I can't really interpret like say, why, say, for example, these 2 are close.'' (P9) ``I think that's really cool. I was just slightly lost'' (P2) ``for some people this could be difficult to handle, as you may not know what kind of score functionality you're using now, because it's a combination of different functions.'' (P1)
% \end{enumerate}
% 1. Analyses on the airline dataset
% 2. Perceptions and usage of features of the tool
% 3. How participants see subgroup discovery fitting in their workflow
% 4. Improvements to the tool
\section{Discussion}



This section delves into user preferences for creating visual blends and the generalizability of our method. We also provide insights for designing with generative AI and highlight the study's limitations.



\subsection{Design Preferences of Visual Blends}



During the evaluation of \sysname, we observed a variety of user designs reflecting their individual preferences.
These preferences were apparent not only in our interviews but also in how they interacted with both systems. 
We can categorize these preferences into three main design options: a preference for the visual expression of metaphors, the features prioritized for blending, and the visual style of blended results.




\subsubsection{Explicit vs. Implicit Expression of Visual Metaphors}





The design content of visual blends can directly or indirectly represent the topic.
For instance, in Figure~\ref{fig:direction}, when creating visual blends for ``global warming'', we can choose the ``fireplace'' from the explicit direction to establish a direct metaphorical connection with the concept of ``warming''. Alternatively, we can select the ``ice cream'' from the implicit direction to represent this concept through its melting state. Similarly, for the task ``Pepper sauce: be aware of the heat'', we can opt for the ``lighter'' from the explicit direction to directly represent the concept of ``heat''. Or, we can choose the ``fire extinguisher'' from the implicit direction to represent this concept through its fire-extinguishing or heat-reducing function.


\begin{figure}[t]
  \centering
  \includegraphics[width=0.9\columnwidth]{figure/example.pdf}
  \caption{Designers have two options when choosing visual metaphors: explicit (left) or implicit (right).
  % There are two distinct design ideation options available. Designers can choose to create objects in an explicit manner (left) or an implicit manner (right) for the desired results.
  % Two different design ideation choices. 
  % Designers can design objects in an explicit way (left) or an implicit way (right) for the results.
  }
  % \Description{Two different design ideation choices on visual metaphors.}
  \label{fig:direction}
\end{figure}


Almost half of the participants (13 out of 24) preferred implicit expression, finding it more subtle (P20, Female, 31; P24, Female, 23) and thought-provoking (P13, Female, 25). This approach allows the audience to gain a deeper understanding of the design output. These advantages contribute to the engagement (P11, Male, 27) and imagination (P9, Male, 21) of the resulting outcomes (P12, Female, 24). P9 (Male, 21) noted that implicit expressions provide richer metaphorical information.
While implicit expressions offer these benefits, explicit visual metaphors succeed in directly representing the literal information, providing a clear and straightforward understanding.




\subsubsection{Prioritizing Semantic vs. Visual Features when Blending}

When visually representing abstract concepts, a debate exists regarding the optimal object features to utilize.
Participants' preferences were evenly distributed, with eight out of 24 participants favoring semantic features, while nine favored visual features.
Those who preferred semantic features believed that semantic meaning captures the contextual nuances of concepts, enabling accurate representation (P20, Female, 31) and facilitating the understanding of abstract concepts and relationships that are not immediately apparent visually (P18, Female, 23). 
They hypothesized that blended results could convey the topic more directly, clearly, and comprehensively (P15, Male, 23).
However, participants also recognized the potential ambiguity of semantic meanings arising from polysemy (words with multiple meanings) and homonymy (different words that share the same spelling).
Conversely, those who preferred visual features favored a straightforward utilization of visual elements such as color, shape, and texture when creating visual blends (P8, Female, 21; P14, Male, 22). 
They believed that higher visual similarity would result in more harmonious blended visual outcomes (P6, Female, 23). 
Nevertheless, the superficiality of visual representation can also lead to misinterpretations of meaning.
We propose that when creating visual blends, designers should seek suitable objects based on semantic features and then identify the most appropriate forms of related objects for blending using visual features to achieve more effective conceptual expression.


Beyond using relevant features to facilitate harmonious blending, similarity scores within each feature also play a pivotal role in guiding designers toward diverse exploration paths. 
While high similarity scores are often desirable, designers may intentionally select object combinations with lower similarity scores to achieve more exaggerated and visually striking results.
As illustrated by the example ``exercise is your daily dose of body vitamins'' in Figure \ref{fig:results}, ``orange'' and ``basketball'' exhibit greater visual compatibility than ``orange'' and ``badminton'' and are easier to combine; however, this may lead to a less innovative visual outcome.
This approach encourages designers to explore a broader range of possibilities, stimulating their creativity and accelerating the ideation process.






\subsubsection{The Integrated vs. Independent Styles of Blended Outcomes}

Upon analyzing the generated visual blends, we identified two primary blending styles: objects merging into a unified form while retaining their individual identities or remaining independent and connected by shared attributes.
For instance, the visual blends of ``global warming'' (first row) in Figure~\ref{fig:results} exhibit a cohesive appearance between different objects.
The integrated representation creates a seamless and unified visual, which can be aesthetically pleasing and easier to interpret as a single entity.
However, achieving a balanced and harmonious blend can be challenging and may necessitate a thorough analysis of the objects' attributes.
In contrast, the visual blends of ``smoking is like a warm welcome to death'' (last row) in Figure~\ref{fig:results} present the distinct characteristics of each object.
Such independent representation allows for more flexibility in design, as objects can be repositioned or modified independently without affecting the overall composition. 
Using common attributes to connect objects can highlight the relationships or interactions among them, providing additional information.
However, the separated objects may not convey a sense of harmony and unity as effectively as the integrated design, and an excessive focus on individual characteristics might detract from the overall message.
Overall, the visual blend's style is largely shaped by the attributes of the combined objects. 
Designers have the flexibility to optimize the representation by substituting associated objects or adjusting visual attributes.







\begin{figure}[b]
  \centering
  \includegraphics[width=0.9\columnwidth]{figure/compare-visiblends.pdf}
  \caption{\sysname\ draws upon topics from \textit{VisiBlends}~\cite{10.1145/3290605.3300402} to showcase the related physical objects and the interconnections among their attributes.
  % Two different design ideation choices. 
  % Designers can design objects in an explicit way (left) or an implicit way (right) for the results.
  }
  % \Description{Results comparison.}
  \label{fig:compare-visiblend}
\end{figure}





\subsection{Generalization of \sysname}




\sysname\ is designed and implemented to create visual blends by merging two objects representing distinct concepts.
To test its generalizability, we applied the system to additional topics studied in prior research and expanded our experiments to evaluate its ability to combine multiple objects.




\subsubsection{Create Visual Blends with Previous Topics}



To demonstrate \sysname's performance in generating visual blends, we employed the same topics used in \textit{VisiBlends}~\cite{10.1145/3290605.3300402}.
Unlike the manual brainstorming and image selection processes in \textit{VisiBlends}, our system automatically identifies metaphorically related objects and offers a variety of blending options based on shared attributes. 
This facilitates creative ideation and reduces user effort. 
Figure~\ref{fig:compare-visiblend} demonstrates two topics from \textit{VisiBlends}. 
Beyond the high-quality visual results achieved through advanced image generation techniques, our approach provides a broader range of blending possibilities. 
By considering attribute combinations rather than solely relying on object shapes, we enable more creative blending directions. 
Additionally, our method focuses on the essential aspects of physical objects that connect to abstract concepts, leading to natural and visually appealing blended outcomes.






\begin{figure*}[t]
  \centering
  \includegraphics[width=0.79\linewidth]{figure/compare-spy.pdf}
  \caption{The results generated by our approach when processing more than two concepts. The outcomes presented are in line with the topics explored in \textit{I Spy a Metaphor}~\cite{chakrabarty-etal-2023-spy}.}
  % \Description{Results comparison.}
  \label{fig:compare-spy}
\end{figure*}



% \subsubsection{Objects Extension}
\subsubsection{Increase the Number of Blended Objects}

When blending three or more concepts, our approach emphasizes selecting two main concepts to merge into a primary subject, while the remaining concepts are incorporated sequentially as secondary elements in the image.
By considering similarity and sentiment scores, we strategically select objects and attributes for these additional concepts based on similarity and sentiment scores to create the final blended composition.
Previous work, such as \textit{I Spy a Metaphor}~\cite{chakrabarty-etal-2023-spy}, also explores metaphorical representations.
% \att{Previous work, such as I Spy a Metaphor [10], also explores metaphorical representations.}
Unlike its method, which treats all concepts equally, our approach distinguishes itself by emphasizing the blending of specific concepts with metaphorically related objects.
This not only facilitates creative exploration but also enhances the visual representation of abstract ideas~\cite{8eb8812e-2a0d-3ce5-ba0f-6a9667472863}.
Figure~\ref{fig:compare-spy} demonstrates four topics that highlight the differences between our method and the previous approach.
In the expression ``Books are the mirror to the soul'', our system merges ``books'' and ``mirror'' into a primary subject, strengthening the metaphorical connection and establishing it as the main focus of the image. 
Additionally, we incorporate the ``phoenix'' element to represent the ``soul'', showcasing the remaining concepts within the expression.
In terms of the final output, the previous method relied on visual elaboration without clear semantic mapping, leading to scattered visual messaging. 
% \att{In terms of the final output, the previous method relied on visual elaboration without clear semantic mapping, leading to scattered visual messaging.}
Our method, however, prioritizes element integration while preserving individual characteristics, making the intended message more easily recognizable by the audience.






\subsection{Design with Generative AI}

During our experiment, we found that the general T2I models do not always reflect users' intent from their generated results. 
Participants commented that ``\textit{sometimes it is tough to keep the AI model on track}'' (P11, Male, 27). 
In some circumstances, users ``\textit{... have no confidence or idea of how the AI will interpret what (they) say}'', and ``\textit{...each prompt is a bit of a gamble}'' (P15, Male, 23). 
These responses revealed two major issues in designing with generative AI.
One is that natural language can be limited in its ability to convey certain design intentions.
The second issue is that the capabilities of generative AI can sometimes be confusing to users due to their unclear boundaries.




The conversational style of current generative AI interfaces has catalyzed a novel trend in interaction design, enabling more intuitive communication with machines through natural language~\cite{10.1007/978-3-031-66329-1_41, 10.1007/978-3-031-48038-6_36}.
However, the lessons we learned have revealed the limitations of natural language in conveying abstract concepts and representing tacit human knowledge, which is often learned through experience rather than explicit instruction.
While natural language interfaces work well in domains where user intent is easily articulated, truly versatile AI interactions require systems that can understand abstract intent and respond to subtle cues, much like human communication.
To mitigate this limitation, specialized tools can tailor the user interface to suit the needs of target users better.
Given the potential shortcomings of natural language interfaces in handling multimedia content~\cite{voigt-etal-2021-challenges}, specialized tools can effectively leverage alternative user interfaces (e.g., graphical user interfaces, or tangible user interfaces) to enhance user support for interactive tasks. 



While generative AI offers the potential for processing multi-modal information, it currently requires advancements in task decomposition and domain-specific expertise to address complex problems effectively.
The evolution of software development shows that specialized techniques are needed to harness the unique capabilities of diverse applications. 
Despite the potential to integrate various information sources, current generative AI models sometimes struggle to produce coherent and reliable outputs, frequently generating unrealistic or nonsensical results.
For example, Figure~\ref{fig:user-study} illustrates how P6's result in T1 exhibits a visual discrepancy, such as an improperly lit cigarette. 
While \sysname{} focuses on design ideation, resolving such visual issues falls outside its scope. 
These common failures, often due to training data bias or model misalignment, can be mitigated through techniques such as retrieval-augmented generation or fine-tuning~\cite {liu2024survey}, though these methods require technical expertise and additional resources~\cite{gao2024retrievalaugmentedgenerationlargelanguage}.
From a human-computer interaction perspective, challenges such as ambiguous input, lack of context, and inadequate instructions hinder users' ability to leverage generative AI for complex tasks. 
To improve usability, we propose using prompt engineering to break tasks into smaller sub-tasks and employing a guided task decomposition interface tailored to specific tasks, ultimately enhancing user understanding and problem-solving efficiency.


\subsection{Limitations and Future Work}





Our research is subject to several limitations that we intend to address in future work.
Firstly, our evaluation was conducted using two design ideation tasks, each limited to under 20 minutes. 
In contrast, real-world design processes often span longer durations and encompass multiple iterative rounds of refinement and exploration. 
As a result, our analysis of user interactions and behaviors in creating visual blends was deliberately constrained to the ideation stage, reflecting the system’s primary focus.
In our evaluation, we employed a baseline consisting of Google Search and ChatGPT (GPT-3.5 with DALL·E 3) for comparison, aiming to balance technical sophistication due to the automatic generation capabilities of AI models. 
Other comparisons might involve existing ideation methods like moodboarding~\cite{10.1145/3290605.3300863} or mind-mapping~\cite{10.1145/3411764.3445325} to investigate strengths and weaknesses further.
Future work can also integrate \sysname\ into real-world design cases to observe how the system can facilitate designers' visual blend design processes.


Secondly, our system is designed primarily to inspire users during the ideation process, which currently limits the flexibility of the provided prompts for re-editing.
% our system currently focuses on inspiring users during their ideation process. 
% Consequently, the provided prompts have limited flexibility in re-editing. 
Moreover, if used to generate the final result, the system still requires multiple trial-and-error iterations to achieve a satisfactory outcome for users. 
Future enhancements could include improving prompt editability to allow professional users greater flexibility in expressing their creativity, incorporating more control modalities to streamline the trial-and-error process, and extending support to the refinement stage further to enhance the system’s utility and creative capabilities.



Thirdly, since our research primarily focused on understanding the system's potential within the design ideation context, a comprehensive analysis of prompt quality and generated result attributes, such as style and layout, was not undertaken.
Nonetheless, we recognize that the quality of user-supplied prompts directly influences the generated outcomes.
Future investigations could delve deeper into how the user's knowledge or experience with AI can affect their experience in generating visual blends. Additionally, exploring the impact of varying prompt formats on AI behavior and the resulting visual blend outputs could provide valuable insights into optimizing the creative process.



\section{Conclusion}

Subgroup analysis is an important, yet under-utilized tool in data science.
Our results suggest that combining algorithm-generated, rule-based insights with human intuition and experimentation in an interactive workflow can help practitioners develop a thorough understanding of complex datasets.
By implementing these interactions in a lightweight notebook-based tool, we hope to lower the barrier for data scientists to try subgroup discovery and to curate unexpected, interesting subpopulations in their data.
Divisi is available as an open-source package so that data scientists and HCI researchers can build on this work, helping to make exploratory subgroup analysis more feasible for a wider range of contexts.

%%
%% The acknowledgments section is defined using the "acks" environment
%% (and NOT an unnumbered section). This ensures the proper
%% identification of the section in the article metadata, and the
%% consistent spelling of the heading.
\begin{acks}
We thank the many data scientists and machine learning practitioners who participated in our user studies and shared their perspectives with us.
We also thank Will Epperson, Katelyn Morrison, and Dominik Moritz for providing feedback on the manuscript.
This work was supported by a National Science Foundation Graduate Research Fellowship (DGE2140739) and by the Carnegie Mellon University Center of Machine Learning and Health.
\end{acks}

%%
%% The next two lines define the bibliography style to be used, and
%% the bibliography file.
\bibliographystyle{ACM-Reference-Format}
\bibliography{references}


%%
%% If your work has an appendix, this is the place to put it.
\appendix
\section{Accuracy and Performance Evaluation Details}
\label{app:perf-evaluation-details}
Table \ref{tab:performance-eval-datasets} provides an overview of the datasets used in the performance and accuracy evaluation in Section \ref{sec:performance-eval}.
The default values were used for the beam size ($k = 50$) and rule length ($L = 3$) in all runs of the Divisi algorithm, while $n$ and $p_\text{min}$ were varied as part of the experiment.
For the source mask, we targeted row sampling to rows that have a positive value for the target outcomes specified in the table.

\begin{table*} \small
\begin{tabular}{rp{3.2cm}p{3.2cm}p{3.2cm}}
\toprule
\textbf{Dataset} & \textbf{Census Income} & \textbf{Airline} & \textbf{Reviews} \\ \midrule
\# \textbf{Instances} & 48,842 & 129,880 & 211,443 \\
\# \textbf{Features} & 12 & 22 & 2,000 \\
\textbf{Target Outcome} & Income classification error & Overall dissatisfaction & Error in predicted star rating (difference of $\geq 3$ stars) \\
\textbf{Base Rate} & 11.5\% & 56.6\% & 0.63\% \\
\midrule
\textbf{Example Subgroup} & Relationship = ``husband'' \& Education = ``assoc-voc'' \& Capital Gain = ``\textless 1'' & Gate location = ``neutral'' \& Inflight wifi service = ``not satisfied'' \& Ease of online booking = ``not satisfied'' & (back, follow, following) \& (years, year) \& (said, told, asked, thought, knew) \\
\textbf{Subgroup Size} & 1.6\% & 5.7\% & 1.1\% \\
\textbf{Outcome Rate in Subgroup} & 31.6\% & 100\% & 5.1\% \\
\bottomrule
\end{tabular}
\caption{Datasets used to evaluate the running time and accuracy of Divisi, and examples of subgroups returned by the Divisi algorithm for each task.}
\label{tab:performance-eval-datasets}
\end{table*}


\section{User Study Details}

This appendix presents condensed study protocols used for the formative design and the user study. The formative design (Sec. \ref{sec:formative}) involved semi-structured interviews with three data scientists experienced in subgroup analysis to inform tool development. The user study (Sec. \ref{sec:user-study}) consisted of a think-aloud session using the Divisi tool, designed to test how data scientists might use exploratory subgroup analysis to understand a dataset.

\subsection{Formative Design Study Protocol}
\label{app:formative-design}

\noindent \textbf{Introduction.} We’re developing a tool to interactively conduct subgroup or slice analysis. As an expert in \textit{[insert field]} with experience on subgroup analysis, I’d like to learn about how you’ve used these techniques in the past, and if there are ways we can build tools to help you do that more easily.

\noindent \textbf{Part 0: Metadata (5 mins).} 
\begin{itemize}
    \item Could you describe your research or work interests at a very high level?
\end{itemize}

\noindent \textbf{Part 1: Previous Experience with Subgroup Analysis (10–20 mins).}
To ensure we’re on the same page, subgroup analysis seeks to identify data segments (e.g., "female, PhD") where a model’s performance is significantly worse compared to overall performance. Insights from these segments can guide data collection, rules implementation, or model improvement.

\begin{itemize}
    \item How many projects (\textgreater 1 month) have you worked on involving subgroup analysis or similar processes?
    \item Can you describe your most recent project using subgroup analysis?
    \begin{itemize}
        \item What kind of data features did you work with?
        \item Did you need to transform or discretize these features? How did you make those decisions?
        \item What outcomes did you compare within the subgroups?
        \item How did you perform the subgroup analysis? What did you find?
        \item How did the findings help you analyze your data or improve your model?
        \item If any, what were the main obstacles in the project?
    \end{itemize}
    \item Do you typically evaluate on subgroups that you already know about or do you also need to discover new subgroups?
    \begin{itemize}
        \item How do you discover these subgroups? What are the characteristics that would make an interesting subgroup?
    \end{itemize}
    \item Have you used any pre-existing tools, either visual or programmatic, to do subgroup analysis? What tools?
    \begin{itemize}
        \item \textit{(If yes)} Did you find anything particularly helpful, unhelpful, or missing in each tool?
        \item What kinds of visualization did you create or want to create?
    \end{itemize}
    \item Have you ever been surprised by something you found in a subgroup analysis? How did you find it and what was its significance?
    \item What do you do with the results of a subgroup analysis? 
    \begin{itemize}
        \item What is most useful or necessary to communicate when you present these results to others?
    \end{itemize}
\end{itemize}

\noindent \textbf{Part 2: Questions on Divisi Prototypes (25 mins).}
\textit{Interviewer opens a Jupyter Notebook with three prototypes of Divisi loaded and shares their screen.}
Imagine you’re a data scientist working for an airline company, tasked with analyzing passenger satisfaction data to predict satisfaction and identify failure points.

\textbf{Subgroup Results from Divisi Algorithm (No Interface).}
We trained a model to predict the overall satisfaction, and extracted the top 10 results from the subgroup discovery algorithm where the error rate was higher than average.
\begin{itemize}
    \item Interpret the subgroups shown. What types of ratings seem harder for the model to classify?
    \item Is this similar to any analysis you’ve done? If so, how helpful was it? If not, would such an analysis be useful for your data?
    \item How do these subgroups compare to the subgroups you’ve looked at in the past in terms of features and their selection?
    \item If this were part of a tool, what additional features or information would you need to better understand the data?
\end{itemize}

\textbf{Basic Interactive Subgroup Discovery Interface.}
The same 10 subgroups as above are now shown in a subgroup discovery interface with basic re-ranking and interactive search functionalities.
\begin{itemize}
    \item What do you think about this interactive interface?
    \begin{itemize}
        \item Would you find it helpful to re-rank subgroups by different metrics? What metrics might you be interested in?
    \end{itemize}
    \item How understandable are these subgroups? Would you find it helpful to have additional explanations of them, such as using a language model or other summarization technique?
    \item Using this tool as a Jupyter notebook widget, how would you see a tool like this fitting into your workflow?
    \item What additional features or information do you think would be most helpful to better understand the data?
\end{itemize}

\textbf{Visualization: Subgroup Map.}
Finally, we want to show you an additional version of the interface where you are able to interact with a visualization of the dataset in addition to the features available for the last prototype. \textit{Interviewer explains how the Subgroup Map works.}
\begin{itemize}
    \item How would you interpret the overlaps between the subgroups as shown on the plot?
    \item How might this ability to see how subgroups overlap influence your analysis?
    \item What kind of visualization style (for example, dimensionality reduction, Venn diagram) would be more helpful for your analyses? Why?
    \item Do you have other suggestions for improving this interface?
\end{itemize}

\noindent \textbf{Part 3: Final Thoughts (5 mins).}
Let’s reflect on your earlier projects:  
\begin{itemize}
    \item After seeing these tools, how would you envision using subgroup analysis in your workflow?
    \item \textit{(If not interested in subgroup analysis)} Do you see this tool being more helpful for other data science problems?
    \item Which features of the tools did you find the most and least useful?
    \item Are there any analyses you would be interested in performing in the future? What would you need to make those analyses possible?
    \item Is there anything else you’d like to add to what we’ve discussed?
\end{itemize}


\subsection{User Study Protocol}
\label{app:user-study}

\noindent \textbf{Introduction.} Thank you for participating in this study. In this session we’ll have you try out a new tool to discover subgroups in a dataset, and we’re interested in learning how you might use this type of analysis in your work. We appreciate any feedback you can give us!

\noindent \textbf{Part 0: Pre-Study Survey (10 mins).}
\begin{itemize}
    \item Are you familiar with the term subgroup analysis? What does it mean to you? \textit{(If clarification is needed)} Subgroup analysis identifies data segments (e.g., combinations like ``female, PhD degree'') where a model performs poorly.  
    \item Please complete this survey about your prior experience with data science and subgroup analysis. Please share your screen while filling out the survey, and feel free to explain your responses as you go.
    \begin{itemize}
        \item Which of the following best describes your current occupation?
        \item How many years of experience do you have working with data and/or machine learning?
        \item What is your level of comfortability with writing Python code?
        \item How many projects (> 1 month long) have you worked on that have involved subgroup analysis?
        \item Do you typically evaluate on subgroups that you already know about or do you also need to discover new subgroups?
        \item Have you used any pre-existing tools, either visual or programmatic, to do subgroup analysis?
    \end{itemize}
    \item Among your prior projects, which do you think would most benefit from subgroup analysis? Did you use any subgroup analysis for that project?
\end{itemize}

\noindent \textbf{Part 1: First Task (15 mins).}
Imagine you are a data scientist analyzing airline passenger satisfaction data. Your task is to predict customer satisfaction and identify where predictions fail. \textit{[Interviewer walks participant through the dataset using a description shown in the survey form.]}
\begin{itemize}
    \item How would you approach this task using any methods you know?  
\end{itemize}

Please open the tool via the provided link. \textit{[Participant opens Jupyter Notebook interface while sharing screen. Interviewer gives a brief tutorial of how to use the interface, using an annotated screenshot embedded in the notebook.]}

For the first task, we’re just going to be looking at the customers’ overall satisfaction rating. For the next 15 minutes, we’d like you to pretend like you truly are the data scientist at the airline company. It’s your job to find insights in the data about what led to customer dissatisfaction. Please talk aloud as you do your analysis, and explain any insights you come up with. If there’s something you want to do with the tool but aren’t sure how, just ask and we can help.

\textit{Follow-up questions to ask during the study:}
\begin{itemize}
    \item Why did you focus on this particular subgroup?  
    \item What made you decide on the way you have the ranking functions set?  
    \item Would it matter to you if the subgroups you selected represented very similar sets of points? Would you want to check that before you present these subgroups?
    \item Would you want to see how much each feature contributed to the overall rate?  
    \item What would you want to do to make the search results more relevant?
\end{itemize}

\noindent \textbf{Part 2: Second Task (20 mins).}
We can imagine that the data science team at the airline company has now built a classification model that can pretty accurately determine whether a customer is going to be dissatisfied based on their survey ratings. But we’re interested in making this model as accurate as possible, and so the team would like you to look at subgroups where the model error is higher than average. Again, for the next 15 minutes you are truly a data scientist on the team, so please do the task to the best of your ability. \textit{[Use the same set of follow-up questions from Part 1 as needed.]}

\noindent \textbf{Part 3: Final Thoughts (10 mins).}
For our final questions, let’s think back to at the beginning of the session when you mentioned that it could be helpful to analyze subgroups for \textit{[a project that the participant mentioned]}.
\begin{itemize}
    \item Which of your prior projects would most benefit from subgroup analysis?  
    \item Did you do anything similar to the subgroup discovery process we did today in that prior project?
    \item Do you think it would be important to be able to discover new subgroups to analyze the dataset in that project? Or would it be sufficient to use slices that you were already aware of?  
    \item Do you see any other ways that this tool could help you get a different angle on your data?  
    \item Is there anything that the tool would need to do differently to be most helpful to you?
    \item Is there anything else you'd like to add to what we've already discussed?
\end{itemize}


\end{document}
\endinput
%%
%% End of file `sample-sigconf.tex'.
