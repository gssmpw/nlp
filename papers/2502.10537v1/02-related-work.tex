\section{Related Work}

Our work is informed by foundational paradigms in visual analytics including exploratory data analysis and exploratory search (Sec. \ref{sec:related-eda}). 
We also build on many prior methods for subgroup analysis from data mining and machine learning, the design space of which we describe in Sec. \ref{sec:related-subgroup-analysis}.

\subsection{Exploratory Data Analysis and Search}
\label{sec:related-eda}

\citeauthor{tukey_exploratory_1970} describes \textit{exploratory data analysis} (EDA) as ``looking at data to see what it seems to say''~\cite{tukey_exploratory_1970}.
EDA is therefore distinct from hypothesis testing, or confirmatory data analysis, in its emphasis on generating insight from the \textit{data} rather than prior knowledge and expectations.
Many systems for EDA are informed by interaction techniques for \textit{exploratory search}, in which people navigate through and query information resources to build understanding about some latent concept of interest~\cite{white_exploratory_2009}.
In these interactive search settings, features such as sorting, filtering, and faceted searches~\cite{yee_2003_faceted} play a key role in helping users uncover useful information.
Applied to EDA, these techniques can enable steerable recommendations of how to visualize data features~\cite{wongsuphasawat_voyager_2016,lee_2021_lux} or efficient overviews of text data~\cite{felix_texttile_2017}.
We draw inspiration from these search techniques in the design of Divisi.

A wide variety of EDA techniques have been developed for different types of data, including small-scale tabular settings~\cite{wongsuphasawat_voyager_2016,lee_2021_lux}, high-dimensional data~\cite{Liu2017}, text data~\cite{felix_texttile_2017}, and general unstructured data~\cite{Smilkov2016}.
It is often easiest to find useful insights in EDA on tabular data because the features are generally intrinsically interpretable. 
In contrast, for text or image data the ``features'' (words or pixels) may not have any meaning on their own, making it difficult to interpret what instances have in common.
As datasets grow larger, there may also be many different subtypes within the dataset, limiting the insight provided by top-level metrics and distributions.
For this reason, many prior works aim to mitigate the complexity of large, high-dimensional datasets by automatically deriving semantically meaningful features or ``concepts'' to bootstrap the analysis process~\cite{suresh_kaleidoscope_2023,kim_interpretability_2018}.
Alternatively, some systems allow the user to define their own features of interest~\cite{wu_errudite_2020,cabrera_zeno_2023}.
However, these methods require the user to already know roughly what concept they are looking for, limiting their opportunities to explore and find unexpected patterns.
Our work relies on the presence of interpretable tabular features for every instance; however, we design for use cases in which the data scientist wants to find the relevant features out of a large set of potentially-meaningful set of descriptors.
This can afford the simplicity of working with tabular data while not restricting the analysis to the user's prior hypotheses.

% \begin{enumerate}
%     \item EDA \cite{tukey_exploratory_1970}, exploratory search \cite{white_exploratory_2009,marchionini_exploratory_2006} - what are the activities involved in each?
%     \item Faceted browsing~\cite{yee_2003_faceted}, sort and filter
%     \item More modern notions of EDA: text exploration, image exploration, embedding analysis
%     \item Benefits of traditional EDA
%     \item Challenges in extending the traditional notions of EDA to modern, large-scale datasets: multiple driving phenomena or subtypes, many variables (possibly more than can be reasoned about), uninterpretable variables
% \end{enumerate}

\subsection{Tools for Subgroup Analysis}
\label{sec:related-subgroup-analysis}

Sometimes called slice discovery, cluster analysis, or rule mining, subgroup analysis is an important part of data science that can help people understand phenomena in a dataset~\cite{liu_exploratory_2020,gamberger_active_2003}, help model builders diagnose and fix issues~\cite{piorkowski_aimee_2023,zhang_drml_2022,cabrera2021deblinder,robertson_angler_2023,zhang_sliceteller_2022, jain_distilling_2022}, explain model predictions~\cite{ribeiro_anchors_2018}, or even be used in place of a model~\cite{lavrac_decision_2004}.
However, it is usually all but impossible to define clear-cut, interpretable subgroups that exactly capture the outcome of interest (e.g., model errors), creating a design space of trade-offs for how to produce useful insights.
A wide array of subgroup analysis techniques have been developed, varying across several dimensions:

\textit{Conceptualization of a subgroup.} Differences in data types, user needs, and algorithm formulations give rise to different definitions of what a subgroup is. 
At the most subjective level, subgroups can be any semantic human-readable description of instances, regardless of whether it is encoded in the data, such as ``images of people with glasses''~\cite{cabrera2021deblinder}. 
They can also be defined by numerical proximity to some conceptual entity, such as a direction or neighborhood around an instance in an embedding space~\cite{eyuboglu_domino_2022,kim_interpretability_2018,ahn_escape_2023}. 
Finally, subgroups can be defined more precisely by constructing rules for membership, such as textual patterns~\cite{wu_errudite_2020,robertson_angler_2023} or predicates on tabular features~\cite{kwon_rmexplorer_2022,hurley_interactive_2022}. 
While the latter results in the clearest subgroup definitions, it also requires crafting or mining high-quality rules.

\textit{Source of initiative.} Many subgroup discovery methods require the data scientist to define subgroups themselves, using the affordances of the various subgroup concepts described above~\cite{cabrera_zeno_2023,wu_errudite_2020,kwon_rmexplorer_2022}. 
These methods are flexible and often provide useful insights on known areas of interest, but it can be difficult to find \textit{new} subgroups without spending time perusing individual instances. 
Algorithm-initiated approaches can provide a strong initial set of subgroups to explore~\cite{chung_slice_2020,zhang_sliceteller_2022}; however, these techniques heavily focus on producing the most relevant set of subgroups in the initial query.
There is currently a lack of \textit{mixed-initiative} subgroup analysis approaches that allow the user to interactively steer the algorithm's output to produce more relevant slices.
When subgroup analysis tools do offer mixed-initiative interactions, it is typically to \textit{refine} the subgroup definitions~\cite{slyman_vlslice_2023} or to characterize and assess their validity~\cite{hurley_interactive_2022}, both of which are supported in Divisi within our broader mixed-initiative workflow.

\textit{Visualization.} Designs to visualize and compare subgroup-level data characteristics are largely dependent on the way the subgroups are conceptualized.
For example, most clustering-based tools use dimensionality reduction scatter plots, which provide a valuable overview of the dataset but are difficult to map to data features~\cite{Liu2019,slyman_vlslice_2023,xuan_attributionscanner_2024,suresh_kaleidoscope_2023,sivaraman_emblaze_2022}.
For handcrafted subgroups on tabular data, brushable histograms can serve as controls to define predicates that are then visualized in strip plots~\cite{cabrera_fairvis_2019} or domain-specific visualizations~\cite{kwon_rmexplorer_2022}.
To visualize rule-based subgroups generated by an algorithm, table representations with sparkline charts or glyphs are often preferred as they can efficiently present summary statistics over many subgroups~\cite{kahng_visual_2016,kerrigan_slicelens_2023,zhang_sliceteller_2022}.
Similarly, UpSet plots~\cite{2014_infovis_upset} provide a dense visual representation of metrics within multiple set intersections.
Divisi combines several of these elements, including the scatter plot and the subgroup table with sparklines, with novel adaptations for tasks such as assessing overlap and coverage.

\textit{Algorithmic approach.} We can divide prior algorithms for subgroup discovery into four broad classes: lattice search, frequent itemsets, classification, and clustering.
Lattice search methods, such as Slice Finder~\cite{chung_slice_2020,sagadeeva_sliceline_2021}, \textsc{Premise}~\cite{hedderich_label-descriptive_2022}, and the Nugget Browser~\cite{guo_nugget_2011}, perform combinatorial search of a space of discrete rules to find those that most satisfy the algorithm's desirability criteria.
These methods can result in easily-interpretable subgroups, but they tend to scale poorly to datasets with hundreds or thousands of features due to combinatorial explosion.
Frequent itemset-based methods, such as DivExplorer~\cite{pastor_looking_2021} and the method developed by \citeauthor{suzuki_rule_2023}~\cite{suzuki_rule_2023}, draw on efficient algorithms from data mining such as FPgrowth, then score and rank the returned subgroups.
Similarly, these methods work best with a relatively small number of possible feature combinations.
Classification-based methods can overcome some of the performance considerations of lattice search and frequent itemset approaches \cite{yuan_isea_2022,yuan_visual_2022}, but their results often require significant work to interpret.
Finally, clustering-based methods aim to group together instances by similarity in a high-dimensional space such as a learned embedding~\cite{zhang_manifold_2019,eyuboglu_domino_2022,kim_interpretability_2018}.
Though these methods can provide insight into unstructured data, they often require a trained model, sometimes one that is jointly trained with natural-language representations, limiting their applicability.
Moreover, like classification methods, the resulting clusters and concepts are not always straightforward to interpret because of their reliance on learned embeddings.
Divisi builds on this extensive space of previous algorithms, adopting a modified lattice search approach that addresses scalability issues using approximation.
While it is most directly applicable to tabular datasets as a result, we propose ways to use it in unstructured data contexts in Sec. \ref{sec:use-case}.

Because there are so many alternative techniques for subgroup analysis, each with their own specific associated data types and challenges, there is not a clear consensus of what approach should be applied to a given problem.
As a result, data scientists may not typically include subgroup analysis in the exploratory phase of their workflows.
Our work aims to make it easier to perform subgroup analyses interactively within a typical programming environment, and we assess in our study whether they might find such capabilities useful in their daily work.

% \begin{enumerate}
%     \item Why subgroup analysis? It can be used in place of classifiers ~\cite{lavrac_decision_2004}, it is useful for experts to understand phenomena in a dataset \cite{liu_exploratory_2020,gamberger_active_2003} or explain predictions~\cite{ribeiro_anchors_2018}, and it can help model builders diagnose and fix issues in their models~\cite{piorkowski_aimee_2023,zhang_drml_2022,cabrera2021deblinder,robertson_angler_2023,zhang_sliceteller_2022, jain_distilling_2022}.
%     \item There are many different subgroup analysis approaches, which vary in how the subgroup is conceptualized:
%     \begin{itemize}
%         \item Conceptualization of a subgroup: can be defined by a semantic human-readable description~\cite{cabrera2021deblinder}, defined by a pattern for text data~\cite{wu_errudite_2020,robertson_angler_2023,hedderich_label-descriptive_2022}, or a rule based on tabular features~\cite{kwon_rmexplorer_2022,hurley_interactive_2022}, based on proximity to some concept or a direction in an embedding space~\cite{suresh_kaleidoscope_nodate}, or based on clusters~\cite{Cavallo2019}
        
%     \end{itemize}
%     \item Source of initiative: often entirely human-initiated~\cite{cabrera_zeno_2023,wu_errudite_2020,kwon_rmexplorer_2022}, or algorithm-initiated. Slice discovery involves approaches to automatically generate the subgroups of interest by mining them from patterns in the data. Algorithmic approaches vary:
%     \begin{itemize}
%         \item rule mining - enumerate possible combinations of features and score them~\cite{chung_slice_2020,sagadeeva_sliceline_2021}
%         \item frequent itemsets~\cite{pastor_looking_2021,zhang_sliceteller_2022}
%         \item embedding-representation approaches~\cite{eyuboglu_domino_2022,kim_interpretability_2018}. For unstructured data we can also use cross-modal representation spaces to label clusters~\cite{slyman_vlslice_2023,eyuboglu_domino_2022}
%     \end{itemize}
%     \item Literature gap: mixed-initiative systems for subgroup discovery. Some interactive systems incorporating subgroup discovery allow users to refine the subgroup definitions~\cite{slyman_vlslice_2023,} or to investigate the characteristics of the subgroups and assess their validity~\cite{hurley_interactive_2022}. few systems have been developed that allow 
% \end{enumerate}
% rule-based explanations