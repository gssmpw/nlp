\section{Visualization Interface}
\begin{figure*}
    \centering
    \includegraphics[width=\textwidth, alt={Screenshot of the Configuration sidebar and Subgroups Table in Divisi loaded on the Census Income dataset. A dropdown is open under a "marital-status" feature to edit the allowed values for that feature in one subgroup.}]{figures/divisi_interface.pdf}
    \caption{The Configuration sidebar (A) and the Subgroups Table (B) allow users to run the subgroup discovery algorithm and browse the rules it returns. For example, in the Census Income dataset, the first returned subgroup (C) represents people with no capital gains or losses who are married to a civilian spouse. This subgroup comprises 38\% of the dataset, and has an error rate of 25.3\%, compared to 11.6\% in the overall Evaluation Set. By clicking the dropdown next to the \texttt{marital-status} feature (D), we can test alternative values for that feature.}
    \label{fig:divisi-interface}
\end{figure*}

Integrating the algorithm described above, we developed the Divisi visualization system to support the exploratory subgroup analysis workflow synthesized in Sec. \ref{sec:formative}.
The system is designed to be installed as a Python package and run in a Jupyter Notebook environment, which is widely used in data science workflows~\cite{shen_interactive_2014}.

To summarize how Divisi's interface might support a typical workflow, let us imagine a data scientist looking for subgroups with high prediction error from the UCI Census Income dataset~\cite{adult_2}.
Upon opening Divisi in a Jupyter Notebook, the data scientist clicks \textit{Find Subgroups} to run the sampling algorithm described above. 
The results appear in the Subgroups Table (Fig. \ref{fig:divisi-interface}B), where the first subgroup consists of people who are married and have no capital gains or losses. 
The data scientist could then refine and re-rank the results using the Configuration sidebar (Fig. \ref{fig:divisi-interface}A); for example, they might upweight ``Error Rate High'' to see subgroups with higher error rates regardless of size. 
The Subgroups Table also includes controls to edit subgroups and define custom rules; for example, the user can change the value of the \texttt{marital-status} feature in a rule to see how alternative values affect the error rate (Fig. \ref{fig:divisi-interface}D).
Finally, the data scientist can drag-and-drop rules into the Subgroup Map to see how much they overlap and cover the dataset (Fig. \ref{fig:subgroup-map}).

Below we discuss in more detail how these interface components enable the three exploratory subgroup analysis activities of discovery, evaluation, and curation.
% The interface consists of three main components: the Configuration sidebar (Fig. \ref{fig:divisi-interface}A), the Subgroups Table (Fig. \ref{fig:divisi-interface}B), and the Subgroup Map (Fig. \ref{fig:subgroup-map}).
%The Configuration sidebar provides controls for the subgroup discovery process, which in turn determine the results shown in the Subgroups Table.
%Meanwhile, the Subgroup Map depicts  a bubble chart of the dataset with visual encodings to show relevant outcome metrics as well as subgroup membership.
% Users can select subgroups from the table or drag-and-drop them into the map to visualize their overlap and coverage.

\begin{figure*}
    \centering
    \includegraphics[width=\textwidth, alt={Three screenshots of the Subgroup Map shown on the Census Income dataset in different states. In (A), no subgroup is selected, and shaded bubbles spread around the chart can be seen to represent model errors. In (B), three subgroups are selected, and the groups overlap such that the first overlaps with the second and third, but the second and third do not overlap with each other. In (C), a subgroup is hovered in the Subgroups Table and all bubbles are grayed out except those that match the hovered group.}]{figures/subgroup_map.pdf}
    \caption{Different states of the Subgroup Map on the UCI Census Income dataset~\cite{adult_2}: (A) an overview of the dataset with no subgroups selected, (B) intersections between three selected subgroups, and (C) highlighting the points that match a subgroup when hovered in the Subgroups Table. Filled-in bubbles indicate classification errors for the income prediction task; each bubble's size indicates the number of instances it contains.}
    \label{fig:subgroup-map}
\end{figure*}

\subsection{Finding and Ranking Subgroups}
\label{sec:vis-discovery}

Users initialize Divisi in a Jupyter Notebook by providing a dataset containing discrete feature values and one or more outcome metrics. 
For example, for the Census Income dataset, we could provide a binary indicator of whether each instance was mispredicted as the outcome metric, as well as the true and predicted values.
Upon launching Divisi, users can click the \textit{Find Subgroups} button to run the subgroup discovery algorithm and populate the Subgroups Table with an initial list of rules, completing task \ref{task:find-subgroups}.
Ranking functions to order the subgroups are automatically generated based on the provided outcomes, and they can be edited on the fly in the Ranking Functions section of the Configuration sidebar (Fig. \ref{fig:teaser}B).
For example, after passing in the ``Error'' metric, Divisi automatically generates a Binary Outcome Rate function to find subgroups where that metric is higher than average.

Top subgroups are computed for every ranking function provided during discovery, so the user can later sort subgroups by any weighted combination of these functions.
The Ranking Functions area allows users to toggle functions on or off, as well as to choose a weight for each function in four increments.
Subgroups are instantly re-ranked as the user adjusts the ranking function configuration, providing rapid feedback about what combination of functions leads to the most interesting results.

To perform task \ref{task:investigate-data-features} (Investigate Data Features), users can use the \textit{+ New Rule} button to define a custom subgroup.
Rules can be defined using a simple syntax, such as \texttt{"marital-status" = "Married-civ-spouse" \& "education" = "Some-college"}.
% This editing functionality also provides users the ability to experiment with ``or'' and ``not'' operations, which are not currently used in the subgroup discovery algorithm because it would drastically increase the number of possible subgroups.
Upon entering the rule, Divisi automatically computes all of the metrics for the new subgroup.

After finding one or more subgroups of interest, users can save those groups to the Favorites, which can later be viewed separately from the search results. This helps in building a case that can be presented to others (task \ref{task:build-case}).

\subsection{Assessing Feature Interactions}
\label{sec:vis-evaluation}

Addressing task \ref{task:schematize} (Schematize), the Subgroups Table lists each retrieved or custom rule along with a summary of the metrics within the subgroup represented by the rule.
Sparkline-style charts, similar to prior subgroup analysis interfaces~\cite{kahng_visual_2016,zhang_sliceteller_2022}, allow users to quickly scan over the list of subgroups and identify general patterns in the metrics as well as the surfaced data features.
% However, as with many subgroup discovery algorithms, it is important to note that multiple results could represent very similar sets of instances.
% To assess whether multiple subgroups have a high degree of overlap with one another, users can select the subgroups of interest or drag them into the Subgroup Map
This can help users combine insights across multiple subgroup results to build a higher-level understanding of one area of the dataset.

% \begin{figure}
%     \centering
%     \includegraphics[width=0.5\linewidth]{figures/slice_feature_editing.pdf}
%     \caption{The lightweight editing tools allow users to interactively test alternative values (or combinations of values) and see how their metrics change.}
%     \label{fig:slice-feature-editing}
% \end{figure}
The Subgroups Table also provides lightweight rule editing functionality to help users quickly test hypotheses about feature interactions, addressing task \ref{task:search-evidence} (Search for Evidence).
By clicking a feature's name in a rule, users can toggle that feature on and off to instantly see how the metrics change.
The feature values can also be adjusted through a dropdown menu, allowing users to select one or more alternative values as shown in Fig. \ref{fig:divisi-interface}D.
For more fine-grained changes, users can use the same syntax described in Sec. \ref{sec:vis-discovery} to edit any subgroup's definition.

\subsection{Visualizing Subgroup Overlap and Coverage}
\label{sec:vis-curation}

As with many subgroup discovery algorithms, Divisi can return subgroups that have many instances in common despite being based on different features (e.g., people who are married would likely overlap with people with a relationship type of ``husband'').
However, existing tools do not help the user assess overlap and coverage, potentially leading to analyses that focus too heavily on small areas of the data.
Divisi includes a novel Subgroup Map visualization that serves three purposes: (a) helping data scientists check whether multiple subgroups have a high degree of overlap with one another, (b) showing how much of the outcome has been covered by the selected subgroups, and (c) providing an overview of the dataset structure that can point to new areas to explore.
Early designs of Divisi used UpSet plots~\cite{2014_infovis_upset} or Venn diagrams, but in initial feedback with data scientists we found that users preferred the spatial dataset overview provided by a scatter plot, as is common in subgroup analysis tools for unstructured data~\cite{suresh_kaleidoscope_2023,robertson_angler_2023,Liu2019}.
% \footnote{Currently only binary outcomes are supported in the Subgroup Map, but the visual encodings could easily be extended to continuous outcomes in the future.}

We opted to use a dimensionality reduction plot using the t-SNE algorithm~\cite{maaten_visualizing_2008} as the starting point for the Subgroup Map.
Although dimensionality reduction can introduce distortions in proximities between points, it provides a useful way to navigate the dataset and depict groups of points that is well-established in visual analytics~\cite{Brehmer_2014_dr}.
To mitigate overdraw for large datasets, we visually simplify the plot by grouping together points that are within a threshold distance and have identical properties (e.g. same outcome value, same subgroup membership). 
This yields a bubble chart in which each bubble's size represents the count of its constituent points, its color represents the outcome value of those points, and its position is the centroid of their 2-D coordinates.
Finally, we perform a very short force-directed relaxation of the layout to remove any overlaps between bubbles, producing the starting view of the dataset as shown in Fig. \ref{fig:subgroup-map}A.

When the user selects a subgroup of interest or drags one from the Subgroups Table into the Subgroup Map, borders around the points are added to show which bubbles represent instances contained in the subgroup. 
Fig. \ref{fig:subgroup-map}B shows the Subgroup Map after selecting three groups from Census Income: married individuals with some college education (blue), married men aged 45-65 with no capital gains (orange), and married women (green).
For any bubble that contains instances in one or more subgroups, the border is divided into an equal-length arc for each subgroup that contains the bubble.
For instance, we can see that the blue subgroup intersects the orange group in Fig. \ref{fig:subgroup-map}B, based on the presence of bubbles with half-blue, half-orange borders.
Towards task \ref{task:schematize}, this visual depiction of subgroups enables users to visually map color-coded subgroups to regions of the plot and to identify when subgroups overlap with each other.

The Subgroup Map includes several secondary interactions to help users gain more information about subgroup overlap and coverage, further addressing task \ref{task:schematize}.
The Subgroup Intersections panel in the bottom-right corner of the map summarizes the size and outcome rate within each combination of subgroups (represented by a bubble glyph).
Hovering on a subgroup in the Subgroups Table grays out all bubbles on the map except those that contain instances matching the subgroup, helping users quickly check the location and overlap of a rule (Fig. \ref{fig:subgroup-map}C).
Hovering on or selecting a region of the map, meanwhile, shows a Distinguishing Feature that is most unique to that selection in the bottom-left corner of the map.

If a user finds an area of the dataset that they would like to describe using subgroups, they can lasso-select that area and conduct a targeted subgroup search, addressing task \ref{task:find-gaps}.
The Divisi algorithm enables efficient search within a selected region by simply adding another ranking function called the Selection Score, which is simply a Binary Outcome Rate function where the outcome is true if the instance is part of the selection. %\footnote{As described in Sec. \ref{sec:testing-robustness}, Divisi splits the data into a discovery and an evaluation set, and only visualizes the evaluation set to avoid generating and testing hypotheses on the same data. To build the Selection Score, we take the selected evaluation set instances and collect a similarly-sized set of their nearest neighbors in the discovery set.}
Using the Subgroup Map, the user can transition between high-level assessments of the dataset and retrieved subgroups within areas that they discover.

\subsection{Implementation Details}

Divisi is implemented as a Jupyter Notebook widget using the \texttt{anywidget}\footnote{\url{https://http://anywidget.dev}} library, with a Python backend and a Svelte\footnote{\url{https://svelte.dev}} frontend.
Visualizations are created using D3.js\footnote{\url{https://d3js.org}} for rendering and layout, and Counterpoint~\cite{sivaraman_2024_counterpoint} for state management.
Divisi is open-source and can be installed from PyPI and GitHub\footnote{\url{https://github.com/cmudig/divisi-toolkit}}.