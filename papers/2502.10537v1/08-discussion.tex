\section{Discussion and Limitations}
\label{sec:discussion}

This work presented a workflow for \textit{exploratory subgroup analysis} instantiated in a novel algorithm and interactive system called Divisi.
Throughout our formative interviews and think-aloud sessions, we saw that data scientists currently take a handcrafted approach to subgroup analysis, and they want techniques to help them discover and evaluate subgroups in more principled ways.
Moreover, they want to ensure that the subgroups they analyze are representative, cover most of the outcomes of interest, and can be easily interpreted by non-technical stakeholders.
Divisi helped our study participants perform these tasks by providing starting points for each stage of their analyses, allowing them to explore ``corners'' of the dataset that would have previously been hard to anticipate.
These results, and the current limitations of our system, suggest future directions for supporting exploratory subgroup analysis.

Rule mining has long been studied~\cite{han_mining_2000} as a way to help people find patterns in data, make interpretable predictions~\cite{furnkranz_brief_2015}, and generate model explanations~\cite{ribeiro_anchors_2018}.
Divisi builds on many of these prior works, applying ideas from data mining in our construction of an approximate subgroup discovery algorithm.
However, despite the abundance of available methods to discover interpretable rules, data scientists still typically construct subgroups based on their prior knowledge and using ad-hoc processes.
We aimed to bridge this gap by framing subgroup analysis as an \textit{exploratory} task that does not require clear goals or a specific stage in the data science process.
Divisi's design proposes that exploratory subgroup analysis tools should not only support the discovery of new subgroups, but also allow for human evaluation and curation of subgroups to juxtapose algorithmic results with expert intuitions.
Participants in our study envisioned ways to integrate all three of these activities into their work, suggesting that \textbf{an exploratory approach based on our proposed workflow may make subgroup analysis more applicable and practical for today's data science practices}. %, where datasets can evolve and models are frequently updated,
% by supporting greater flexibility and interactivity, to support an exploratory workflow
% Our work builds on these prior advances
% What has changed? Data types, multiple metrics of interest, . What hasn't changed - people still need to be able to high-level describe the patterns in a dataset

Our evaluations suggest that the novel aspects of our subgroup discovery algorithm make Divisi more amenable to large datasets and support interactive sense-making.
As we showed in the performance evaluation, the ability to configure levels of approximation through Divisi's parameters could make the tool more feasible than existing algorithms to use in large, wide datasets such as text data. 
Moreover, since Divisi's algorithm made re-ranking essentially instantaneous, participants extensively tested the effects of balancing multiple ranking functions in ways that would have been time-consuming otherwise.
Many systems developed in HCI and data visualization research have required new computational formalisms~\cite{suresh_kaleidoscope_2023,robertson_angler_2023} or algorithms~\cite{perer_frequence_2014,lam_concept_2024} to enable new insights into data or machine learning models.
Similarly, our evaluation of Divisi underscores how \textbf{formulating data mining algorithms specifically for the requirements of \textit{interactive} data science workflows can facilitate more thorough exploration.}

One current limitation of Divisi's curation features surfaced by several participants was deciding how to parameterize the space of possible subgroups.
As described in Sec. \ref{sec:results-perceived-opps}, participants pointed out that which variables were included and how they were transformed into discrete values (e.g. binning 5-point ratings into ``not satisfied,'' ``neutral,'' and ``satisfied'') could have a large impact on what subgroups were identified.
Similarly, when discovering subgroups in text data as we demonstrated in Sec. \ref{sec:use-case}, the choice and clustering of words used for grouping can greatly impact the meaningfulness of the returned rules.
Defining subgroups using syntactic features, topic models, or LLM-based concept extraction techniques could all provide different lenses on the same data.
\textbf{Exploring how to help data scientists create and test feature spaces for subgroups is an important direction for future work.}
% Participants' attentiveness to how the subgrouping space was parameterized could suggest that when using Divisi, the bulk of the analytical effort essentially shifted from mining and interpreting clusters to deciding how to \textit{express} their data using interpretable features.
For example, it is likely that these interpretable features could be constructed so that they generalize across many datasets, or generated automatically using language models to kickstart analysis. %, similar to how LIWC dictionaries can be used to measure psychological characteristics in many kinds of texts~\cite{tausczik_psychological_2010}.
% Such metadata could also potentially be automatically extracted for other types of data, such as images or graphs, further broadening the applicability of rule-based subgroup analysis.
% Future work can explore how to design these ``subgrouping vocabularies'' to kickstart the exploratory subgroup analysis process on new datasets.

When integrating exploratory subgroup analysis into a data science workflow, it is important to consider how the subgroups will be used downstream, both for statistical and, more broadly, ethical reasons. 
While subgroup or slice discovery in machine learning is most often framed in terms of identifying semantically meaningful model errors, users in our studies saw Divisi as potentially useful for a wider array of exploratory data analysis tasks.
For example, participants imagined using it to dig deeper into analyses of clinical trial results to see what kinds of patients received the most benefit.
However, these analyses come with the caveat that testing too many subgroups can lead to false positive observations, particularly in smaller datasets~\cite{wang_statistics_2007}.
\textbf{Making clear distinctions about when subgroup analysis is to be used for exploratory and confirmatory purposes~\cite{hullman_designing_2021}, and constraining the space of analytical actions for each case, could mitigate subgroup analyses that lead to spurious interpretations.}

We also note that defining black-and-white rules to categorize and separate subsets of data can be at odds with the nature of the data, particularly when it represents humans.
Just as prior HCI critiques have pointed out that most ``useful'' data constructs are convenient approximations of the true concept of interest~\cite{tal_target_2023}, exploratory subgroup analysis requires an understanding that any rule-defined subgroup is an approximation of the true subpopulation.
For example, discretizing medical lab values into low, normal, and high categories elides the reality that many people's normal lab values could be considered extreme relative to the population~\cite{cohen_personalized_2021}, leading to measurement errors when defining subgroups by any fixed cutoff.
\textbf{Future research could design ways to support reflection on where rule-based subgroups might not fully capture the intended population due to nuances in the data.}
% We also note that rules to categorize and separate subsets of data can have important implications when used to inform decisions, particularly when the data represent humans.
% Just as predictive models can implicitly create black-and-white categories to drive decisions about individuals~\cite{alkhatib_street-level_2019}, decisions driven by subgroup-level insights 

Although our evaluation allowed us to observe diverse ways that data scientists could use Divisi, we note the limitation that our participants only used the tool for about one hour, with an environment and dataset we set up. 
We chose this method to ensure consistency across the tasks performed, and to minimize the time burden on participants.
Although participants were able to conduct multiple analytical tasks during their sessions, spending more time with the system and using it on their own data could provide a richer picture of how participants might use it in the long term.
Additionally, because there is no commonly-used alternative workflow that data scientists use for subgroup analysis, we did not quantitatively assess the impact of using Divisi with respect to a baseline system.
Future work is needed to validate that the findings from exploratory subgroup analysis lead to more effective models or improved dataset understanding.

% \begin{enumerate}
% \item Using Divisi on unstructured data. For these types of tasks, Divisi's approach essentially shifts the analytical effort from interpreting clusters to selecting and curating from a large set of inherently interpretable features.
%     \item Deciding how to parameterize the slicing space
%     \item In fact, extensive subgroup analysis is sometimes discouraged in these settings because with small dataset sizes, testing too many subgroups can lead to false positive observations~\cite{wang_statistics_2007}.
% \end{enumerate}