\section{Introduction}

Core to data science work is making sense of large datasets to understand what patterns they contain and how they can be used~\cite{cao_data_2018}.
Along with the ``datafication'' of our personal, commercial, and medical lives, datasets aim to capture increasingly diverse and complete representations of the world~\cite{lycett_datafication_2013}.
To build a robust understanding of these datasets, data scientists and analysts often break the data into meaningful \textit{subgroups}, or smaller subsets of the data, and characterize how these subgroups differ from the overall population.
For example, businesses use customer segmentation, a form of subgroup analysis, to understand what types of customers are most likely to purchase a product or respond to a marketing campaign~\cite{noauthor_customer_2024}.
In medicine, researchers gain insight into clinical trial results by searching for subpopulations (e.g., men over 65 with kidney failure) where a treatment had greater benefits or risks~\cite{wang_statistics_2007}.

However, many questions that could be answered using subgroup analysis remain challenging to explore, particularly in the larger, higher-dimensional datasets used in modern machine learning~\cite{helal_subgroup_2016}:
What kinds of customers most often buy products or cancel subscriptions based on their purchase history?
What combinations of prescriptions result in patients getting more frequently admitted to the hospital?
What kinds of large language model (LLM) prompts are more often correctly answered by one model versus another?
Instances in these cases could have hundreds or thousands of data features, representing diverse subtypes that are hard to isolate using handcrafted subgroups.
% We present new algorithmic pattern mining techniques, and new interactive data science workflows to interpret their results, that can help build a complete understanding of these datasets.
% Advances in algorithmic pattern mining techniques, and new interactive data science workflows to interpret their results, are needed to build a complete understanding of these datasets.
Comprehensively answering such questions requires new algorithmic pattern mining techniques that are designed to work in tandem with interactive data science workflows.
% Despite the greater need for tools to assist with these tasks, there are few well-established solutions that directly help people characterize meaningful subgroups in large datasets.

% When interpreting any set of data, a common question to ask is ``\textit{What kinds of instances lead to an outcome?}'' 
% The outcome of interest and the ways ``kinds of instances'' are defined may vary widely from task to task.
% Yet the structure of these questions reflects a widespread need for humans working with data: to break down into understandable chunks the relationship between a potentially large number of variables and an outcome of interest.
% However, because data seldom follows clear-cut rules or fully predictable patterns, these questions cannot be answered directly from the outputs of any algorithm.
% Building a rigorous understanding of what kinds of instances lead to an outcome demands not only computational precision and efficiency to quantitatively survey large datasets, but also human interpretation to draw out local patterns into broader inferences about the data.

% Subgroup analysis, the process of identifying subsets of a dataset that exhibit differences in an outcome compared to average, has long been a part of statistical practice.
% For example, in medicine, clinical trial results are often analyzed by separating participants into groups by demographics or other features, in order to find patients for whom the treatment may have been most beneficial~\cite{wang_statistics_2007}.
% In these situations, the set of grouping variables is often small enough that experts can manually choose how to define and evaluate subgroups.
% However, subgroup analysis can be both more fruitful and more challenging in data mining and machine learning, where datasets tend to be much larger and contain many more features of potential interest~\cite{helal_subgroup_2016}. 
% For example, a data scientist or business analyst may want to understand what kinds of customers most often buy products or cancel subscriptions, what kinds of patients frequently get readmitted to a hospital, or what kinds of prompts are more often correctly answered by one chatbot versus another.
% Instances in these cases could be characterized by hundreds or thousands of possible features, and there may be multiple outcome metrics that the analyst needs to balance.
% Yet despite the greater need for tools to assist with these tasks, there are still few well-established solutions that directly help people characterize meaningful subgroups in large datasets.

Most existing approaches for subgroup analysis either depend on the user to define subgroups to analyze~\cite{cabrera_zeno_2023,cabrera_fairvis_2019,wu_errudite_2020}, or they return a static list of groups (often termed ``data slices'') that the user must interpret ~\cite{chung_slice_2020,pastor_looking_2021,eyuboglu_domino_2022}.
For complex datasets, subgroup analysis requires a combination of the two approaches to leverage both expert knowledge and algorithmic scalability.
However, the interactions in prior visual analytics tools for subgroup analysis~\cite{suresh_kaleidoscope_2023,zhang_sliceteller_2022,kwon_rmexplorer_2022,cabrera_zeno_2023} focus on refining individual slices rather than helping the user curate a more diverse set of candidates.
As a result, these tools are useful in identifying and validating patterns that already fit within an expert's mental model, but they are less effective at cultivating a more holistic understanding of the data through unexpected discoveries. % when posing the question ``what kinds of instances lead to an outcome,'' 

We propose to conceptualize subgroup analysis not as a hypothesis testing or machine learning problem, as has been done in prior work~\cite{chung_slice_2020,eyuboglu_domino_2022}, but rather as part of exploratory data analysis (EDA).
Data scientists use EDA to summarize the main characteristics of a dataset and build an intuitive understanding of its contents as they prepare it for modeling or other downstream purposes~\cite{tukey_exploratory_1970}.
However, traditional EDA techniques focus on understanding one feature at a time or interactions between two features across the dataset, rather than characterizing subgroups~\cite{wongsuphasawat_voyager_2016,epperson_dead_2023,stolte_2002_polaris}.
We define \textit{exploratory subgroup analysis} as a process analogous to EDA in which data scientists can discover, evaluate, and curate interesting and meaningful subsets of a dataset. This framing encompasses existing techniques such as rule mining~\cite{zhang_sliceteller_2022}, cluster interpretation~\cite{Cavallo2019}, and iterative subgroup refinement~\cite{slyman_vlslice_2023}, and creates opportunities for new interactions.
Exploratory subgroup analysis complements existing EDA techniques by providing an intermediate level between the overall dataset, which is often too complex to understand in its entirety, and individual instances, which can be time-consuming and misleading to interpret.

To materialize our proposed workflow for exploratory subgroup analysis and assess how it might influence data scientists' workflows, we developed a subgroup discovery algorithm and computational notebook-based visualization tool called Divisi.
The algorithm underlying Divisi expands on previous slice discovery approaches to support sense-making in interactive data science by (1) allowing for multiple arbitrary subgroup metrics and (2) using approximation to scale more efficiently to large, wide datasets.
These advances enable novel interactions in the Divisi interface, such as dynamic re-ranking and targeting subgroup search to a user-defined subset of the data.
The system also provides a novel Subgroup Map visualization which helps users relate subgroups to the dataset's overall structure.
As we illustrate through a performance evaluation and a use case with large language model (LLM) prompts that provoke unsafe responses, Divisi's algorithm and interactive workflow can reveal patterns in real-world datasets in a lightweight, efficient manner.

We explored how Divisi could help perform exploratory subgroup analysis tasks in a think-aloud study with 13 experienced data scientists.
Our results showed that data scientists particularly value the \textit{discovery} of unexpectedly important variables in the rules surfaced by Divisi's algorithm.
They heavily utilized subgroup editing and refinement tools to \textit{evaluate} whether a group was meaningful and how each feature in the rule contributed to its deviation from average.
Participants also found ways to \textit{curate} the subgroups they found for stakeholders, ensuring that their selections represented distinct subpopulations and covered most of the interesting outcomes.
Finally, they envisioned ways to fit the various features of Divisi into their projects, suggesting opportunities for future tools to more thoroughly and rigorously support exploratory subgroup analysis.

This work makes the following contributions to the literature on interactive tools for data science:
\begin{enumerate}
    \item \textbf{A proposed workflow and design goals for exploratory subgroup analysis} informed by interviews with three experts in subgroup analysis;
    \item \textbf{A novel subgroup discovery algorithm} designed specifically to support the needs of interactive applications, such as configuring the level of approximation and weighing multiple metrics of interest;
    \item \textbf{An interactive system called Divisi} that incorporates tools to tailor the discovery process, evaluate subgroup rules, and visualize their overlap and coverage; and
    \item \textbf{Results from a think-aloud study} with 13 data scientists, yielding implications for future exploratory subgroup analysis tools.
\end{enumerate}