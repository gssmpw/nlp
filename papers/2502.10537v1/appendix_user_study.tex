\section{User Study Details}

This appendix presents condensed study protocols used for the formative design and the user study. The formative design (Sec. \ref{sec:formative}) involved semi-structured interviews with three data scientists experienced in subgroup analysis to inform tool development. The user study (Sec. \ref{sec:user-study}) consisted of a think-aloud session using the Divisi tool, designed to test how data scientists might use exploratory subgroup analysis to understand a dataset.

\subsection{Formative Design Study Protocol}
\label{app:formative-design}

\noindent \textbf{Introduction.} We’re developing a tool to interactively conduct subgroup or slice analysis. As an expert in \textit{[insert field]} with experience on subgroup analysis, I’d like to learn about how you’ve used these techniques in the past, and if there are ways we can build tools to help you do that more easily.

\noindent \textbf{Part 0: Metadata (5 mins).} 
\begin{itemize}
    \item Could you describe your research or work interests at a very high level?
\end{itemize}

\noindent \textbf{Part 1: Previous Experience with Subgroup Analysis (10–20 mins).}
To ensure we’re on the same page, subgroup analysis seeks to identify data segments (e.g., "female, PhD") where a model’s performance is significantly worse compared to overall performance. Insights from these segments can guide data collection, rules implementation, or model improvement.

\begin{itemize}
    \item How many projects (\textgreater 1 month) have you worked on involving subgroup analysis or similar processes?
    \item Can you describe your most recent project using subgroup analysis?
    \begin{itemize}
        \item What kind of data features did you work with?
        \item Did you need to transform or discretize these features? How did you make those decisions?
        \item What outcomes did you compare within the subgroups?
        \item How did you perform the subgroup analysis? What did you find?
        \item How did the findings help you analyze your data or improve your model?
        \item If any, what were the main obstacles in the project?
    \end{itemize}
    \item Do you typically evaluate on subgroups that you already know about or do you also need to discover new subgroups?
    \begin{itemize}
        \item How do you discover these subgroups? What are the characteristics that would make an interesting subgroup?
    \end{itemize}
    \item Have you used any pre-existing tools, either visual or programmatic, to do subgroup analysis? What tools?
    \begin{itemize}
        \item \textit{(If yes)} Did you find anything particularly helpful, unhelpful, or missing in each tool?
        \item What kinds of visualization did you create or want to create?
    \end{itemize}
    \item Have you ever been surprised by something you found in a subgroup analysis? How did you find it and what was its significance?
    \item What do you do with the results of a subgroup analysis? 
    \begin{itemize}
        \item What is most useful or necessary to communicate when you present these results to others?
    \end{itemize}
\end{itemize}

\noindent \textbf{Part 2: Questions on Divisi Prototypes (25 mins).}
\textit{Interviewer opens a Jupyter Notebook with three prototypes of Divisi loaded and shares their screen.}
Imagine you’re a data scientist working for an airline company, tasked with analyzing passenger satisfaction data to predict satisfaction and identify failure points.

\textbf{Subgroup Results from Divisi Algorithm (No Interface).}
We trained a model to predict the overall satisfaction, and extracted the top 10 results from the subgroup discovery algorithm where the error rate was higher than average.
\begin{itemize}
    \item Interpret the subgroups shown. What types of ratings seem harder for the model to classify?
    \item Is this similar to any analysis you’ve done? If so, how helpful was it? If not, would such an analysis be useful for your data?
    \item How do these subgroups compare to the subgroups you’ve looked at in the past in terms of features and their selection?
    \item If this were part of a tool, what additional features or information would you need to better understand the data?
\end{itemize}

\textbf{Basic Interactive Subgroup Discovery Interface.}
The same 10 subgroups as above are now shown in a subgroup discovery interface with basic re-ranking and interactive search functionalities.
\begin{itemize}
    \item What do you think about this interactive interface?
    \begin{itemize}
        \item Would you find it helpful to re-rank subgroups by different metrics? What metrics might you be interested in?
    \end{itemize}
    \item How understandable are these subgroups? Would you find it helpful to have additional explanations of them, such as using a language model or other summarization technique?
    \item Using this tool as a Jupyter notebook widget, how would you see a tool like this fitting into your workflow?
    \item What additional features or information do you think would be most helpful to better understand the data?
\end{itemize}

\textbf{Visualization: Subgroup Map.}
Finally, we want to show you an additional version of the interface where you are able to interact with a visualization of the dataset in addition to the features available for the last prototype. \textit{Interviewer explains how the Subgroup Map works.}
\begin{itemize}
    \item How would you interpret the overlaps between the subgroups as shown on the plot?
    \item How might this ability to see how subgroups overlap influence your analysis?
    \item What kind of visualization style (for example, dimensionality reduction, Venn diagram) would be more helpful for your analyses? Why?
    \item Do you have other suggestions for improving this interface?
\end{itemize}

\noindent \textbf{Part 3: Final Thoughts (5 mins).}
Let’s reflect on your earlier projects:  
\begin{itemize}
    \item After seeing these tools, how would you envision using subgroup analysis in your workflow?
    \item \textit{(If not interested in subgroup analysis)} Do you see this tool being more helpful for other data science problems?
    \item Which features of the tools did you find the most and least useful?
    \item Are there any analyses you would be interested in performing in the future? What would you need to make those analyses possible?
    \item Is there anything else you’d like to add to what we’ve discussed?
\end{itemize}


\subsection{User Study Protocol}
\label{app:user-study}

\noindent \textbf{Introduction.} Thank you for participating in this study. In this session we’ll have you try out a new tool to discover subgroups in a dataset, and we’re interested in learning how you might use this type of analysis in your work. We appreciate any feedback you can give us!

\noindent \textbf{Part 0: Pre-Study Survey (10 mins).}
\begin{itemize}
    \item Are you familiar with the term subgroup analysis? What does it mean to you? \textit{(If clarification is needed)} Subgroup analysis identifies data segments (e.g., combinations like ``female, PhD degree'') where a model performs poorly.  
    \item Please complete this survey about your prior experience with data science and subgroup analysis. Please share your screen while filling out the survey, and feel free to explain your responses as you go.
    \begin{itemize}
        \item Which of the following best describes your current occupation?
        \item How many years of experience do you have working with data and/or machine learning?
        \item What is your level of comfortability with writing Python code?
        \item How many projects (> 1 month long) have you worked on that have involved subgroup analysis?
        \item Do you typically evaluate on subgroups that you already know about or do you also need to discover new subgroups?
        \item Have you used any pre-existing tools, either visual or programmatic, to do subgroup analysis?
    \end{itemize}
    \item Among your prior projects, which do you think would most benefit from subgroup analysis? Did you use any subgroup analysis for that project?
\end{itemize}

\noindent \textbf{Part 1: First Task (15 mins).}
Imagine you are a data scientist analyzing airline passenger satisfaction data. Your task is to predict customer satisfaction and identify where predictions fail. \textit{[Interviewer walks participant through the dataset using a description shown in the survey form.]}
\begin{itemize}
    \item How would you approach this task using any methods you know?  
\end{itemize}

Please open the tool via the provided link. \textit{[Participant opens Jupyter Notebook interface while sharing screen. Interviewer gives a brief tutorial of how to use the interface, using an annotated screenshot embedded in the notebook.]}

For the first task, we’re just going to be looking at the customers’ overall satisfaction rating. For the next 15 minutes, we’d like you to pretend like you truly are the data scientist at the airline company. It’s your job to find insights in the data about what led to customer dissatisfaction. Please talk aloud as you do your analysis, and explain any insights you come up with. If there’s something you want to do with the tool but aren’t sure how, just ask and we can help.

\textit{Follow-up questions to ask during the study:}
\begin{itemize}
    \item Why did you focus on this particular subgroup?  
    \item What made you decide on the way you have the ranking functions set?  
    \item Would it matter to you if the subgroups you selected represented very similar sets of points? Would you want to check that before you present these subgroups?
    \item Would you want to see how much each feature contributed to the overall rate?  
    \item What would you want to do to make the search results more relevant?
\end{itemize}

\noindent \textbf{Part 2: Second Task (20 mins).}
We can imagine that the data science team at the airline company has now built a classification model that can pretty accurately determine whether a customer is going to be dissatisfied based on their survey ratings. But we’re interested in making this model as accurate as possible, and so the team would like you to look at subgroups where the model error is higher than average. Again, for the next 15 minutes you are truly a data scientist on the team, so please do the task to the best of your ability. \textit{[Use the same set of follow-up questions from Part 1 as needed.]}

\noindent \textbf{Part 3: Final Thoughts (10 mins).}
For our final questions, let’s think back to at the beginning of the session when you mentioned that it could be helpful to analyze subgroups for \textit{[a project that the participant mentioned]}.
\begin{itemize}
    \item Which of your prior projects would most benefit from subgroup analysis?  
    \item Did you do anything similar to the subgroup discovery process we did today in that prior project?
    \item Do you think it would be important to be able to discover new subgroups to analyze the dataset in that project? Or would it be sufficient to use slices that you were already aware of?  
    \item Do you see any other ways that this tool could help you get a different angle on your data?  
    \item Is there anything that the tool would need to do differently to be most helpful to you?
    \item Is there anything else you'd like to add to what we've already discussed?
\end{itemize}
