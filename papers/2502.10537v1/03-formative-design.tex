\section{Formative Design: Exploratory Subgroup Analysis Workflow}
\label{sec:formative}
% We developed a novel slice discovery algorithm called Divisi that extends this previous work by addressing the following design requirements:
% \begin{enumerate}[label={\bfseries G\arabic*.}, ref={\bfseries G\arabic*},itemsep=1ex]
%     \item \textit{Flexible slicing by multiple weighted criteria.} Errors are often just one of many phenomena data scientists seek to understand about the models they develop and their underlying data~\cite{holstein_improving_2019}. Rather than searching only for slices by a single binary metric, our approach was designed to support ranking by several metrics at once (e.g., positive true labels and negative predictions). \label{goal:sd-score-fns}
%     \item \textit{Configurable to prefer faster performance or more thorough search.} Existing slicing algorithms~\cite{sagadeeva_sliceline_2021,pastor_looking_2021} typically perform exhaustive search over slices that meet the minimum support constraints, which can be prohibitively slow for large, wide datasets. We aimed to allow users to view \textit{approximations} of the optimal set of slices at customizable levels of speed and coverage. \label{goal:sd-approximate}
% \end{enumerate}

To establish the design goals for our system, we adapted \citeauthor{pirolli_sensemaking_2005}'s sense-making framework~\cite{pirolli_sensemaking_2005} with insights gained from semi-structured interviews with three data scientists experienced in subgroup analysis.
Sense-making captures how analysts move between individual observations and larger-scale, more rigorous hypotheses, similar to the process of insight discovery in EDA~\cite{tukey_exploratory_1970,cabrera_what_2022}. 
However, it is unclear how the stages of sense-making might correspond to exploratory analysis steps on \textit{subgroup} data.
As such, we aimed to understand how data scientists who regularly work with subgroups interpret and reason about this type of data and translate their specific observations to high-level assessments.

We recruited three data scientists with extensive experience in subgroup analysis in different domains, including health technology (participant E1), algorithmic fairness (E2), and biomedical informatics (E3).
We asked these practitioners about the current role that subgroup analysis plays in their workflow when analyzing data or building models, and about the challenges they face when analyzing subgroups (see Sec. B.1 in the Supplementary Material for the questions used).
We also gave them open-ended interpretation tasks using example subgroups in three different interfaces, one consisting of plain text and two early variants of the Divisi interface.
Through these tasks we sought to understand how they would make sense of these subgroups and what they thought of the system design.
Each interview took place remotely, lasted around 60 minutes and was recorded and transcribed for analysis.

The practitioners we interviewed used subgroups for a variety of reasons, ranging from evaluations of model performance to understanding patient subpopulations with different treatment needs in medical data.
For example, E1 described using subgroup analysis after building a model for screening clinical trial participants, with the goal of determining what kinds of instances led to more predicted exclusions.
To validate that the model was not discriminating by any factors such as race, gender, or socioeconomic status, they explored subgroups sliced by each factor individually as well as intersections of these factors.

Experts particularly valued subgroups defined by clear-cut rules as a way to communicate model behavior and dataset patterns to less technical stakeholders.
However, consistent with the lack of a dominant subgroup analysis method in the literature, all three experts reported primarily relying on prior or external knowledge and ad-hoc visualizations to perform subgroup analysis.
They faced challenges in deciding which features to use for defining subgroups, assessing whether the differences they saw were real, making sense of interactions between features, and comparing metrics across multiple subgroups.

Based on the practices and challenges identified in these interviews, we condensed the relevant parts of the \citeauthor{pirolli_sensemaking_2005}  framework into four successively higher-level representations: the original data, sets of subgroups, schemas or mental models of behavior, and a synthesized understanding of the dataset.
In the resulting workflow, shown in Fig. \ref{fig:esa-model}, the analyst moves between these representations through three iterative activities: discovery, evaluation, and curation.
We note that this adaptation of the sense-making process is similar to other process models proposed in visual analytics, such as workflows for exploratory analysis of machine learning models by \citeauthor{cabrera_what_2022}~\cite{cabrera_what_2022} and \citeauthor{zhang_manifold_2019}~\cite{zhang_manifold_2019}.
Though the overall analysis goals are consistent with prior work, specialized algorithmic and interactive tools may be required to conduct this sense-making process on data subgroups, informing the design of Divisi.
Below we describe how the activities of discovery, evaluation, and curation map to tasks requiring special affordances for subgroup analysis, following \citeauthor{munzner_nested_2009}'s taxonomy of visual analytics design~\cite{munzner_nested_2009}.

% Key findings from the formative study: practitioners value subgroup analysis because it provides intuition about the data, and because subgroup analyses can easily be interpreted by less technical stakeholders.
% However, people most often just use off-the-shelf tools for subgroup analysis and prior knowledge to decide what to slice by.
% They find it challenging to do subgroup analysis in large datasets with many features, and they would like to explore possible ways to pre-process the data before slicing.

% \begin{enumerate}
%     \item Automatic subgroup discovery to improve rigor and help sort through many features
%     \item Balance multiple metrics of interest.
%     \item Support reasoning about interactions between multiple features.
%     \item Use subgroups to help characterize the makeup of the dataset.
% \end{enumerate}

\begin{figure}
    \centering
    \includegraphics[width=\linewidth, alt={Flowchart showing four stages, Dataset, Subgroups, Schema, and Synthesis, connected by three cyclic paths labeled Discover, Evaluate, and Curate. The Discover cycle consists of (1) Find Subgroups going from Dataset to Subgroups, and (2) Investigate Data Features going back to Dataset. The Evaluate cycle consists of (3) Schematize going from Subgroups to Schema, and (4) Search for Evidence going back to Subgroups. The Curate cycle consists of (5) Build Case going from Schema to Synthesis, and (6) Find Gaps going back to Schema.}]{figures/ESA_model.pdf}
    \caption{Proposed workflow for exploratory subgroup analysis, adapted from \citeauthor{pirolli_sensemaking_2005}'s sense-making framework~\cite{pirolli_sensemaking_2005} and informed by three expert interviews.}
    \label{fig:esa-model}
\end{figure}

\textbf{Discovery.} Identifying important subgroups to look at was a major challenge for the experts we interviewed, particularly E1 and E2.
For example, E2 recounted a project about measuring disparities in medical record data:
\begin{quote}
\textit{``There's, I don't know, thousands of diagnosis codes at different levels.... So we had to do some literature review to justify why we're including those particular ones [rather] than the other ones. The ones we included are the ones that are known to [have disparities].''}
\end{quote}
% Not only was it challenging to find the initial features to analyze, the experts also found it difficult to visualize and reason about interactions about features (E1, E2).
Given this complexity, participants wanted tools that automatically identify subgroups to provide a \textit{``less biased view of the data''} (E3).
At the same time, they expressed the need to incorporate their prior knowledge, and that of stakeholders, to guide subgroup definitions.

\begin{enumerate}[label={\bfseries T\arabic*.}, ref={\bfseries T\arabic*},itemsep=1ex]
    \setcounter{enumi}{\value{goalCounter}}
    \item \textit{Find Subgroups:} The system should provide fast, approximate algorithmic subgroup recommendations as a starting point for analysis. \label{task:find-subgroups}
    \item \textit{Investigate Data Features:} Users should be able to define and store custom rules to probe the effects of features they identify as interesting. \label{task:investigate-data-features}
    \setcounter{goalCounter}{\value{enumi}}
\end{enumerate}

\textbf{Evaluation.} After identifying potentially interesting subgroups, experts wanted quantitative, actionable ways to detect if the subgroup was meaningful to their analysis.
% While statistical significance was a factor in assessing subgroups, our respondents were most interested in just looking at the metrics that would be most relevant to downstream decisions: \textit{``It's usually associated more with some type of business outcome or cost metric... [such as] bad press, bad marketing, or to make sure they're doing things the right way''} (E1).
E1 described the difficulty of comparing metrics across several different subgroups to decide which were the most meaningful, while E2 emphasized that the relevant performance metrics could be different for different subgroups.
Participants also wanted to see how their metrics of interest would change as they added and removed features from the rules to assess intersectionality (E1, E2).

\begin{enumerate}[label={\bfseries T\arabic*.}, ref={\bfseries T\arabic*},itemsep=1ex]
    \setcounter{enumi}{\value{goalCounter}}
    \item \textit{Schematize:} The system should enable comparison of the metrics in and overlap between multiple subgroups. \label{task:schematize}
    \item \textit{Search for Evidence:} Users should be able to create variants of subgroups that reveal how individual features contribute to the metrics and interact with each other. \label{task:search-evidence}
    % \item \textit{Visualize overlap between subgroups and their coverage of the dataset.}
    \setcounter{goalCounter}{\value{enumi}}
\end{enumerate}

\textbf{Curation.} The final product in all three experts' subgroup analyses was a presentation that could allow stakeholders to interpret the subgroups and the differences in metrics associated with them.
This was particularly true for problems in which domain experts were needed to interpret the subgroups: \textit{``In our analyses we usually... present the information and then let [the medical expert] make decisions based on their results''} (E3).
To make sure they were conveying accurate information to these non-data-scientist stakeholders, participants expressed the desire to make sure they captured an overall sense of the dataset during their exploration (E1, E2, E3).
They also suggested that the subgroups they identified could be used to familiarize others with the data provided they had attained good coverage.
\begin{enumerate}[label={\bfseries T\arabic*.}, ref={\bfseries T\arabic*},itemsep=1ex]
    \setcounter{enumi}{\value{goalCounter}}
    \item \textit{Build Case:} Users should be able to build a collection of subgroups that can be presented to stakeholders.\label{task:build-case}
    \item \textit{Find Gaps:} The system should surface under-explored areas of the dataset for further schema development and refinement of the search criteria.\label{task:find-gaps}
\end{enumerate}