\section{Model} \label{sec: model}
We consider a sequential setting over \( T \) discrete rounds, where in each round, users interact either with Generative AI (GenAI) or a complementary human-driven platform, Forum. An instance of our problem is represented by the tuple \( \langle a, \gamma, r, \beta, w^s \rangle \), and we now elaborate on the components of the model.

\paragraph{GenAI.}  
GenAI adopts a selective response strategy $\mathbf{x} = (x_1, x_2, \ldots, x_T)$, where $x_t \in [0, 1]$ represents the proportion of users who receive answers in round $t$ among those who have already chosen GenAI. For example, $x_t = 1$ means that GenAI answers all users who selected it in round $t$, whereas $x_t = 0$ means it answers none. The performance of GenAI depends on the cumulative amount of data it has collected and trained on at the start of each round $t$, denoted $\mathcal{D}_t(\mathbf{x})$. The quality of GenAI is represented by the \emph{accuracy function} $a(\mathcal{D}_t(\mathbf{x}))$, a strictly increasing function $a : [0, T] \to [0, 1]$, satisfying $\frac{d a(\mathcal{D})}{d \mathcal{D}} > 0$ for all $\mathcal{D} \in \mathbb{R}_{\geq 0}$.\footnote{We use the term accuracy for simplicity, allowing us to address user satisfaction abstractly. Evaluating the performance of GenAI is significantly more complex.}   

We use superscripts $g$ and $s$ to denote the utility users receive from GenAI and Forum, respectively. The utility users derive from GenAI in round $t$, denoted $w^g_t(\mathbf{x})$, reflects the expected quality $a(\mathcal{D}_t(\mathbf{x}))$ that users obtain from GenAI. It is given by  
\begin{equation}\label{eqdef wg}
w^g_t(\mathbf{x}) = a(\mathcal{D}_t(\mathbf{x})) \cdot x_t.    
\end{equation}
Crucially, GenAI can intentionally respond less accurately than its maximum capability. In each round $t$, the proportion of users who choose GenAI is denoted by $p_t(\mathbf{x})$. This fraction is determined by the selective response strategy $\bft{x}$ and user decisions, which will be discussed shortly.

The (time-discounted) revenue of GenAI over $T$ rounds, $U(\mathbf{x})$, is defined by  
\[
U(\mathbf{x}) = \sum_{t=1}^T \gamma^t r(p_t(\mathbf{x})),  
\]
where $\gamma^t$ represents the discount factor applied to the revenue at round $t$, reflecting the decreasing value of future revenue. The function $r : [0, 1] \to \mathbb{R}_{\geq 0}$ maps the proportion of users $p_t(\mathbf{x})$ in round $t$ to revenue, and is assumed to be both non-decreasing and $L_r$-Lipschitz. For instance, a superlinear $r$ captures the compounding market effects of GenAI, where revenue grows at an accelerating rate as the proportion of users increases \cite{katz1985network, bailey2022peer, mcintyre2017networks}. For example, the case where a higher user base attracts disproportionately more offers for collaborations and investment opportunities.

\paragraph{Data Accumulation.}  
The cumulative data available to GenAI evolves as users interact with Forum. At the start of round $t$, the cumulative data $\mathcal{D}_t(\mathbf{x})$ is defined recursively as:  
\[
\mathcal{D}_t(\mathbf{x}) = \mathcal{D}_{t-1}(\mathbf{x}) + (1 - p_{t-1}(\mathbf{x})),
\]  
with the initial condition $\mathcal{D}_1(\mathbf{x}) = 0$. This initial condition represents the emergence of a new topic, where GenAI has not acquired any relevant data from previous training sets.  

\paragraph{Forum.}  
Forum provides a human-driven platform where users can post and answer questions. The utility users derive from Forum, $w^s$, is constant across rounds and satisfies $w^s \in [0, 1]$.  

\paragraph{Users.}  
Users decide between GenAI and Forum by comparing the expected utility they derive from each platform. We model user decisions using a softmax function:  
\[
\sigma_t(\mathbf{x}) = \frac{e^{\beta w^g_t(\mathbf{x})}}{e^{\beta w^g_t(\mathbf{x})} + e^{\beta w^s}},
\]  
where $\beta > 0$ is a sensitivity parameter that captures users' responsiveness to utility differences.  

Recall that $x_t$ represents the proportion of users in $\sigma_t(\mathbf{x})$ who receive an answer from GenAI. The remaining users, who do not receive an answer, can either post their question on Forum or leave them unanswered. We assume the former, meaning that $p_t(\mathbf{x}) = x_t \sigma_t(\mathbf{x})$ is the proportion of users who receive an answer from GenAI, while the rest contribute to data generation by posting their question on Forum.

\paragraph{User Welfare.}  
The \textit{instantaneous user welfare} $w_t(\bft{x})$ accounts for the utilities derived from both platforms in round $t$. It is defined by
\begin{equation}\label{eq:def inst}
w_t(\bft{x}) = p_t(\mathbf{x}) \cdot w^g_t(\mathbf{x}) + (1 - p_t(\mathbf{x})) w^s    
\end{equation}
The \textit{cumulative user welfare}, $W$ is therefore the sum of the instantaneous welfare over all the rounds $W(\bft{x}) = \sum_{t = 1}^T w_t(\bft{x})$.



\paragraph{Assumptions and Useful Notations}
As we explain later, the following assumption on the structure of the accuracy function is crucial for analyzing the dynamics of the data generation process.
\begin{assumption} \label{assumption: data lip}
The accuracy function $a(\mathcal{D})$ is $L_a$-Lipschitz with constant $L_a \leq \frac{4}{\beta}$. 
\end{assumption}
We further discuss this assumption in Section~\ref{sec:discussion}. Additionally, we use the following notions throughout the paper. Given an arbitrary strategy $\bft{x}$, any strategy $\bft{x}^\tau$ that is obtained by reducing the response level in round $\tau$ and maintaining the other entries of $\bft x$ is called a \emph{$\tau$-selective modification of $\bft x$}. That is,  $\bft{x}^\tau$ is any strategy that is identical to strategy $\bft{x}$ except for round $\tau$, in which it \emph{answers less than $x_\tau$}. Formally, $x^\tau_{\tau} \in [0, x_{\tau})$ and $x^\tau_t = x_t$ for every $t \neq \tau$. For brevity, if $\bft x$ is clear for the context, we use $\bft{x}^\tau$ as any arbitrary $\tau$-selective modification. Another useful notation in $\Bar{\bft{x}}$, where $\Bar{\bft{x}} = (1, 1, \dots, 1)$ is \emph{full response} or the \emph{always-responding} strategy; we use these interchangeably. We use this strategy as a point of comparison, establishing a baseline to test other strategies.
\begin{example}\label{example}
\normalfont
Consider the instance $T = 10$, $a(\mathcal{D}) = 1-e^{-0.3 \mathcal{D}}$, $\gamma = 0.9$, $r(p) = p^2$, $\beta = 10$, and $w^s = 0.5$. Consider the following selective response strategy $\Bar{\bft{x}} = (1, \ldots, 1)$ and $\bft{x}$ which is defined by.
\begin{align*}
x_t = \begin{cases}
    0 & \mbox{$t \leq 4$} \\
    1 & \mbox{otherwise}
\end{cases}.
\end{align*}

At $t = 1$ it holds that $d_1(\bft{x}) = d_1(\Bar{\bft{x}}) = 0$. Notice that $a(0) = 0$ and therefore $p_1(\Bar{\bft{x}}) = 1 \cdot \frac{1}{1 + e^{\beta w^s}}  \approx 0.0067$.
Similarly, for $\bft{x}$ it is $p_1(\bft{x}) = 0 \cdot \frac{1}{1 + e^{\beta w^s}} = 0$. Thus, the generated data is $d_2(\Bar{\bft{x}}) \approx 1-0.0067 = 0.9933$ and $d_2(\bft{x}) = 1$.

With that, we have the ingredients to calculate the instantaneous welfare at time $t = 1$. 
\begin{align*}
    & w_1(\Bar{\bft{x}}) = p_1(\Bar{\bft{x}}) w^g_1(\Bar{\bft{x}}) + (1-p_1(\Bar{\bft{x}}))w^s \\
    & \qquad \approx 0.0067 \cdot 0 + 0.9933 \cdot 0.5 \approx 0.4966 \\
    & w_1(\bft{x}) = 0 \cdot 0 + 1 \cdot w^s = 0.5
\end{align*}
Figure~\ref{fig:example 1} demonstrates the proportions of the strategies $\Bar{\bft{x}}$ and $\bft{x}$ as a function of the round for $t \in [T]$. Notice that the selective response $\bft{x}$ induces lower user proportions in the earlier rounds, but it eventually surpasses the full response strategy $\Bar{\bft{x}}$.
\begin{figure}[t]
    \centering

\def\datagraphexample{Figures/example.csv}

\definecolor{O2}{RGB}{0,0,255}
\def\unitwo{O2}

\definecolor{O1}{RGB}{255,0,0}
\def\unicolor{O1}

\begin{tikzpicture}[scale=0.85]

\begin{axis}[
    axis lines = left,
    xlabel = Round $t$,
    ylabel = Proportions,
    legend pos=south east,
    legend cell align = left,
    %legend style={font=\large, nodes={scale=1.2, transform shape}}
]
\addplot [\unitwo, mark=*, mark options={solid}, line width=1.1pt] table[x=round, y=Px, col sep=comma] {\datagraphexample};
\addlegendentry{Selective response $\bft{x}$}

\addplot [\unicolor, mark=square*, mark options={solid}, line width=1.1pt] table[x=round, y=Px_new, col sep=comma] {\datagraphexample};
\addlegendentry{Full response $\Bar{\bft{x}}$}

\end{axis}

\end{tikzpicture}
    \caption{A visualization for Example~\ref{example}. The blue (circle) curve shows the proportion of users $p_t(\bft{x})$ for the selective response strategy $\bft x$ at each round $t$. The red (square) curve depicts the corresponding proportion for the full response.}
    \label{fig:example 1}
\end{figure}


Finally, the revenue is attained by calculating $U(\bft{x}) = \sum_{t = 1}^{10} \gamma^{t-1} (p_t(\bft{x}))^2$. Computing this for the two strategies, we see that $U(\bft{x})$ is roughly $5\%$ higher than $U(\Bar{\bft{x}})$. Similarly, the welfare $W(\bft{x})$ is about $8\%$ higher than $W(\Bar{\bft{x}})$. As this example suggests, selective response can improve both revenue and welfare. Indeed, this is the focus of the next section.
%\omer{to drop this paragraph?}We observe several key findings. First, using a selective response strategy can lead to a significantly faster increase in user proportions, enabling GenAI to expand its user base more quickly. Second, choosing a strategy with a selective response can be advantageous for both GenAI and users, boosting both revenue and overall welfare.
\end{example}