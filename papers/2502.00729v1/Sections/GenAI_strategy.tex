\section{The Benefits of Selective Response} \label{sec: motivation}
This section motivates our work by showing that selective response may benefit both GenAI and its users. We first demonstrate a qualitative result: Selective response can improve revenue, welfare, or both. Then, in Subsection~\ref{sec: price of full response}, we quantify the extent of these improvements.


Recall the definition of the full response strategy $\Bar{\bft{x}}$. We use it as a benchmark in evaluating the potential impact of adopting a selective response strategy on GenAI and its users.
\begin{observation} \label{obs: withholding impact}
There exist instances and a selective response strategy $\bft{x}$ that satisfy each one of the following inequalities:
\begin{enumerate}
    \item $U(\bft{x}) > U(\Bar{\bft{x}})$ and $W(\bft{x}) > W(\Bar{\bft{x}})$,
    \item $U(\bft{x}) < U(\Bar{\bft{x}})$ and $W(\bft{x}) > W(\Bar{\bft{x}})$,
    \item $U(\bft{x}) > U(\Bar{\bft{x}})$ and $W(\bft{x}) < W(\Bar{\bft{x}})$.
\end{enumerate}
\end{observation}
The first inequality in Observation~\ref{obs: withholding impact} indicates that 
there exists a selective response strategy that Pareto dominates the always-responding strategy. The subsequent two inequalities imply that increasing either GenAI's revenue or the users' social welfare may come at the expense of the other.


\subsection{Price of Always Responding}\label{sec: price of full response}
In this subsection, we quantify the negative impact of always answering users' queries. We introduce two indices: $\rpfr$, an abbreviation for \textbf{R}evenue's \textbf{P}rice of \textbf{A}lways \textbf{R}esponse, and $\wpfr$, which stand for \textbf{W}elfare's \textbf{P}rice of \textbf{A}lways \textbf{R}esponse. Formally, $\rpfr \triangleq \frac{\max_{\bft{x}} U(\bft{x})}{U(\Bar{\bft{x}})}$ and $\wpfr \triangleq \frac{\max_{\bft{x}} W(\bft{x})}{W(\Bar{\bft{x}})}$ are the price of always answering with respect to revenue and social welfare, respectively. These metrics capture the inefficiencies in revenue and welfare that arise when GenAI always responds to all user queries. Our next result demonstrates that the revenue inefficiency is unbounded.

\begin{proposition} \label{revenue price of answering}
For every $M \in \mathbb{R}_{>0}$ there exists an instance $I$ with $L_r = \Theta(\ln(M))$ such that $\rpfr(I) > M$.
\end{proposition}
Proposition~\ref{revenue price of answering} relies on the revenue scaling function $r(p)$, which can bias GenAI's incentives toward data generation rather than immediate revenue. For example, when $r(p)$ takes the form of a sigmoid function, the parameter $L_r$ controls the steepness of the curve. If the sigmoid is sufficiently steep, $r(p)$ approximates a step function, requiring GenAI to surpass a certain user proportion threshold to generate revenue. This mirrors threshold-based incentives, where substantial rewards are only provided once a predefined threshold is met.

Our next proposition shows that there exist instances where selective responses can result in social welfare nearly twice as large as that of the always-responding strategy.
\begin{proposition} \label{SW Price of answering}
For every $\varepsilon > 0$ there exists an instance $I$ with $\wpfr(I) > 2-\varepsilon$.
\end{proposition}
We end this section by analyzing Price-of-Anarchy~\cite{koutsoupias1999worst,roughgarden2005selfish}, a standard economic concept that measures the harm due to strategic behavior of GenAI. Formally, $PoA = \frac{\max_{\bft{x}} W(\bft{x})}{\min_{\bft{x} \in \mathcal{R}} W(\bft{x})}$, where $\mathcal{R}$ is the set of revenue-maximizing strategies. We show that it can increase with the smoothness parameter of the reward function $L_r$. Since this analysis depends on the revenue-optimal strategy of GenAI, which we only examine in later sections, we defer this analysis to the \appnx{Appendix~\ref{appn: full response}}
