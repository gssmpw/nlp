\section{Regulating Selective Response for Improved Social Welfare with Minimal Intervention} \label{sec: regulation}
In this section, we adopt the perspective of a regulator aiming to benefit users through interventions. We show how to use the results from the previous section to ensure that the intervention will be beneficial from a welfare perspective. Additionally, we bound the revenue gap that such an intervention may create. A crucial part of our approach is that the regulator can see previous actions, but not future actions, making it closer to real-world scenarios. Specifically, for any arbitrary round $\tau$, we assume the regulator observes $x_1, \dots x_{\tau}$, but has no access to GenAI's future strategy $(x_{t})_{t = \tau + 1}^T$. 
\subsection{Sufficient Conditions for Increasing Social Welfare}
We focus on $\tau$-selective modifications that guarantee to increase welfare w.r.t. a base strategy $\bft x$. We further assume GenAI commits to a 0 response level as long as its quality is below $C$, where $C$ is the threshold from Theorem~\ref{thm: sw silence effect}. %We term this \emph{GenAI commitment}. \omer{explain why make sense}.
This commitment, formally given by $\min_{t > \tau} \{w^g_t(\bft{x}) \mid w^g_t(\bft{x}) > 0\} > C$, represents the minimum utility required from GenAI for rounds $t > \tau$. %It ensures that the potential positive effect on welfare of increased proportions after the selective response are realized, guaranteeing a net increase in social welfare.
\begin{corollary}\label{cor: welfare suficient condition}
Assume that $w^g_\tau(\bft{x}) < C$ and that GenAI commits, i.e., $\min_{t > \tau} \{w^g_t(\bft{x}) \mid w^g_t(\bft{x}) > 0\} > C$ holds for all $t > \tau$. Then, $W(\bft{x}^\tau) \geq W(\bft{x})$.
\end{corollary}
Intuitively, \Cref{cor: welfare suficient condition} ensures that the welfare improvement due to this intervention (the green shaded region in Figure~\ref{fig:welfare+}) surpasses the welfare reduction (the gray region).
\subsection{Bounding GenAI's Revenue Gap} \label{sec: rev diff}
A complementary question is to what extent \emph{forcing} a $\tau$-selective response can harm GenAI's revenue. Our goal is to establish a bound on the revenue gap between the base strategy $\bft{x}$ and the modified strategy $\bft{x}^\tau$, where the selective response occurs in round $\tau$. We stress that incomplete information about future actions makes this analysis challenging. 

By definition, $\bft{x}$ and $\bft{x}^\tau$ are identical except for round $\tau$. Consequently, they generate the same amount of data in all rounds \emph{before} $\tau$. Using a $\tau$-selective response reduces the proportion of answers in that round, which in turn increases the accumulated data available in round $\tau + 1$. Therefore, the revenue gap can be decomposed into two components: (1) The immediate effect of the proportion change in round $\tau$, $r(p_\tau(\bft{x}^\tau)) - r(p_\tau(\bft{x}))$; and (2) the downstream effects on subsequent rounds due to the change in the data generation process. Using several technical lemmas that we prove in \appnx{Appendix~\ref{appendix:regulation}}, we show that:
\begin{corollary}\label{cor: loose bound}
It holds that
\begin{align*}
&U(\bft{x}^\tau) - U(\bft{x}) \leq \\
&\gamma^{\tau - 1} \left( r(p_\tau(\bft{x}^\tau)) - r(p_\tau(\bft{x})) \right) + L_r \gamma^{\tau} \frac{p_\tau(\bft{x}^\tau) - p_\tau(\bft{x})}{1 - \gamma}.
\end{align*}
\end{corollary}
The above bound is less informative as $\gamma$ approaches~1. In \appnx{Theorem~\ref{thm: revenue bounds}}, we obtain a tighter bound by having some additional assumptions.


% The above bound becomes less informative as $\gamma$ approaches~1. Next, we obtain a tighter bound by having some additional assumptions. We consider the case where $\beta L_a \leq 1$, which is a stricter condition that the one in Assumption~\ref{assumption: data lip}. Further, let
% \[
% k = \frac{\beta}{4(1+e^{\beta w^s})^2} \min_{\mathcal{D} \in [0, T]} \frac{da(\mathcal{D})}{d\mathcal{D}}.
% \]
% \omer{Boaz to explain these parameters in color}
% Using this notation, we present the following tighter bound.
% \begin{theorem} \label{thm: revenue bounds}
% Let $\unx = \min \{ x_t \mid t > \tau, x_t > 0 \}$. If $\beta L_a \leq 1$, then
% \begin{align*}
% &U(\bft{x}^\tau) - U(\bft{x}) < \\
% &\gamma^{\tau - 1} \left( r(p_\tau(\bft{x}^\tau)) - r(p_\tau(\bft{x})) \right) + L_r \gamma^{\tau} \frac{p_\tau(\bft{x}^\tau)-p_\tau(\bft{x})}{1-\gamma \left( 1 - k \unx^2 \right)}.
% \end{align*}
% \end{theorem}
% \omer{Boaz to explain contraction}
% \boaz{Theorem~\ref{thm: revenue bounds} relies on a contraction property of the data, specifically that the data gap after round \(\tau\) decreases at an exponential rate in \((1 - k \unx^2)\).}
% To leverage the bound provided by Theorem~\ref{thm: revenue bounds}, the regulator must elicit a commitment from GenAI regarding a minimum selective response, as the theorem's applicability depends on the value of $\unx$ for $t > \tau$.
