\section{The Impact of Selective Response on Social Welfare}  \label{sec: welfare} 
In this section, we flesh out the impact of implementing $\tau$-selective modifications on social welfare. Specifically, we focus on modifying an arbitrary initial strategy $\bft{x}$ by applying a selective response in a single round $\tau$. By restricting our analysis to single-round modifications, we aim to provide insights into the welfare dynamics.

The next Theorem~\ref{thm: sw silence effect} provides a powerful tool in characterizing the change in the instantaneous user welfare in $\tau$-selective modifications. It reveals the role of the threshold $C \approx w^s -\frac{1.28}{\beta}$ in determining how the instantaneous welfare changes compared to the based strategy. We first present the theorem and then analyze its consequences. 
% round $\tau$ and future rounds $t>\tau$, revealing 
% Intuitively, the positive impact of using a $\tau$-selective response on social welfare arises from two key factors. First, suppose GenAI's quality is inferior to the utility provided by Forum at time $\tau$, $w^s > w^g_\tau (\bft x)$. In that case, selective response drives users to turn to Forum, allowing them to benefit from Forum's superior utility.
% Second, there is a compounding effect due to Theorem~\ref{thm: not answering increase proportions}. Suppose the utility derived from GenAI surpasses Forum's utility $w^s$ in any round $t$. In that case, selective responses in earlier rounds increase the proportion of users who can access GenAI's superior content in round $t$ \omer{the above -inst welfare, points two and three}. 
% Notably, the relationship between GenAI’s quality and Forum’s utility is central in determining whether using selective response enhances users' welfare. 
\begin{theorem} \label{thm: sw silence effect} 
Fix any strategy $\bft{x}$. For $C =w^s -\frac{\mathcal{W}(e^{-1}) + 1}{\beta}$, where $\mathcal{W}$ is the Lambert function, it holds that
\begin{enumerate}  
\item\label{thm-p1} In round $\tau$, if $w^g_\tau(\bft{x}) < C$ then $w_\tau(\bft{x}^\tau) > w_\tau(\bft{x})$;
\item\label{thm-p2} For every round $t > \tau$ such that $w^g_t(\bft{x}^\tau) < C$, it holds that $w_t(\bft{x}^\tau) < w_t(\bft{x})$;
\item\label{thm-p3} For every round $t > \tau$ such that $w^g_t(\bft{x}) > C$, it holds that $w_t(\bft{x}^\tau) > w_t(\bft{x})$.  
\end{enumerate}  
\end{theorem}  
We interpret the theorem using the illustration in Figure~\ref{fig:welfare+}. The horizontal axis is the round number and the vertical axis is the quality users obtain from GenAI, $w^g_t$. There are three curves: The red (circle) is the base strategy $\bft x$; the blue (triangle) represents a $\tau$-selective modification; and the orange (dashed) line is the threshold $C \approx w^s -\frac{1.28}{\beta}$. We observe several phenomena.

% \begin{itemize}[leftmargin=*, itemsep=0pt]%,      % Align list with surrounding text
%     \item 
Before round $\tau$, the utilities of the two strategies are the same, as they agree on the response levels. Furthermore, in round $\tau$, the $\tau$-selective modification a has lower response level, i.e., $x^\tau_\tau < x_\tau$; hence, since $a(\mathcal{D}_\tau(\mathbf{x^\tau}))  = a(\mathcal{D}_\tau(\mathbf{x}))$, the definition of $w^g_t$ in Equation~\eqref{eqdef wg} implies $w^g_{\tau}(\bft x^\tau) < w^g_{\tau}(\bft x)$. By definition of the instantaneous welfare in Equation~\eqref{eq:def inst}, we thus conclude that $w_\tau(\bft{x}^\tau) > w_\tau(\bft{x})$, like in Part~\ref{thm-p1} of the theorem.

For any round $t$, $t>\tau$, we see that the blue curve is above the red curve. Namely, GenAI's quality of the $\tau$-selective modification $\bft x^\tau$ is greater than that of the base strategy $\bft x$. This is a direct corollary of Theorem~\ref{thm: not answering increase proportions}: We know that more data is created ($\mathcal{D}_t(\bft{x}^\tau) > \mathcal{D}_t(\bft{x})$) and more users choose GenAI ($p_t(\bft x^\tau) \geq p_t(\bft x)$); hence $w^{g}_t(\bft{x}^\tau) > w^{g}_t(\bft{x})$. 

Parts~\ref{thm-p2} and~\ref{thm-p3} are demonstrated by the shaded gray (featuring horizontal lines) and green (featuring vertical lines) areas. In the gray, we see rounds $t$ with $t> \tau$, and the blue curve is below the orange line; hence, $w^g_t(\bft{x}^\tau) < C$. Consequently, Part~\ref{thm-p2} implies that the instantaneous welfare of $\bft x^\tau$ is lower than that of $\bft x$. Similarly, the green background belongs to rounds in which the red curve is above the threshold $C$, i.e., $w^g_t(\bft{x}) > C$; therefore, Part~\ref{thm-p3} suggests that the instantaneous welfare of $\bft x^\tau$ is higher than that of~$\bft x$. Notably, combined together, we see how $\tau$-selective response can reduce the (cumulative) social welfare, which is aligned with Observation~\ref{obs: withholding impact}. 
\begin{figure}[t]
    \centering
    % First figure
        %\includegraphics[scale=1]%, width=\textwidth]
        %{Figures/welfare_genAI_utility_graph.pdf}
        \usetikzlibrary{patterns}
\usepgfplotslibrary{fillbetween}

\def\datagraphwf1{Figures/welfare_diff_graph1.csv}
\def\datagraphwfdecwf{Figures/welfare_diff_graph_dec_wf.csv}
\def\datagraphwfincwf{Figures/welfare_diff_graph_inc_wf.csv}

\definecolor{O2}{RGB}{0,0,255}
\def\unitwo{O2}

\definecolor{O1}{RGB}{255,0,0}
\def\unicolor{O1}

\definecolor{O3}{RGB}{255, 140, 0}
\def\treecolor{O3}

\begin{tikzpicture}[scale=0.9]

\begin{axis}[
    axis lines = left,
    xlabel = Round $t$,
    ylabel = $w_t^g(\bft{x})$,
    ylabel near ticks,
    xlabel near ticks,
    legend style={at={(0.95,0.05)}, anchor=south east}, % Aligns legend exactly to the right bottom
    legend cell align = left,
    ytick style = {draw=none},
    yticklabels = {}, % Remove tick labels
    xtick = {0.75}, 
    xticklabels = {$\tau$}, 
]

% Plotting the strategies
\addplot [\unicolor, mark=*, mark options={solid}, line width=1.1pt] 
    table[x=x, y=y, col sep=comma] {\datagraphwf1};
\addlegendentry{\footnotesize $\bft{x}$}

\addplot [\unitwo, mark=triangle*, mark options={solid}, line width=1.1pt] 
    table[x=x, y=y_tau, col sep=comma] {\datagraphwf1};
\addlegendentry{\footnotesize $\bft{x}^\tau$}

\addplot [\treecolor, dashed, line width=1.1pt] 
    table[x=x, y=treshold, col sep=comma] {\datagraphwf1};
\addlegendentry{\footnotesize $C$}

% --- Gray Background with Vertical Lines Pattern ---
\addplot [name path=dec_above, draw=none] 
    table[x=x, y=y_tau_above, col sep=comma] {\datagraphwfdecwf};
\addplot [name path=dec_below, draw=none] 
    table[x=x, y=y_tau_below, col sep=comma] {\datagraphwfdecwf};

% 1. Fill with Gray Color
\addplot [
    fill=gray,                        % Fill color set to gray
    opacity=0.2                       % Opacity level
] fill between [of=dec_above and dec_below];

% 2. Overlay with Vertical Lines Pattern
\addplot [
    pattern=horizontal lines,           % Vertical lines pattern
    pattern color=black,              % Pattern lines colored black
    draw=none,                        % No border
    opacity=1                         % Full opacity for pattern
] fill between [of=dec_above and dec_below];

% --- Green Background with Diagonal Lines Pattern ---
\addplot [name path=inc_above, draw=none] 
    table[x=x, y=y_tau_above, col sep=comma] {\datagraphwfincwf};
\addplot [name path=inc_below, draw=none] 
    table[x=x, y=y_tau_below, col sep=comma] {\datagraphwfincwf};

% 1. Fill with Green Color
\addplot [
    fill=green,                       % Fill color set to green
    opacity=0.4                       % Opacity level
] fill between [of=inc_above and inc_below];

% 2. Overlay with Diagonal Lines Pattern
\addplot [
    pattern=vertical lines,         % Diagonal lines pattern
    pattern color=black,              % Pattern lines colored black
    draw=none,                        % No border
    opacity=1                         % Full opacity for pattern
] fill between [of=inc_above and inc_below];

\end{axis}

\end{tikzpicture}

% \usetikzlibrary{patterns}
% \usepgfplotslibrary{fillbetween}

% \def\datagraphwf1{Figures/welfare_diff_graph1.csv}
% \def\datagraphwfdecwf{Figures/welfare_diff_graph_dec_wf.csv}
% \def\datagraphwfincwf{Figures/welfare_diff_graph_inc_wf.csv}

% \definecolor{O2}{RGB}{0,0,255}
% \def\unitwo{O2}

% \definecolor{O1}{RGB}{255,0,0}
% \def\unicolor{O1}

% \definecolor{O3}{RGB}{128, 0, 128}
% \def\treecolor{O3}

% \begin{tikzpicture}[scale=0.85]

% \begin{axis}[
%     axis lines = left,
%     xlabel = Round $t$,
%     ylabel = $w_t^g(\bft{x})$,
%     legend style={at={(1.15,0.05)}, anchor=south east}, % Aligns legend exactly to the right bottom
%     legend cell align = left,
%     ytick style = {draw=none},
%     yticklabels = {}, % Remove tick labels
%     xtick = {0.75}, 
%     xticklabels = {$\tau$}, 
% ]

% \addplot [\unicolor, mark=*, mark options={solid}, line width=1.1pt] table[x=x, y=y, col sep=comma] {\datagraphwf1};
% \addlegendentry{\footnotesize $\bft{x}$}

% \addplot [\unitwo, mark=triangle*, mark options={solid}, line width=1.1pt] table[x=x, y=y_tau, col sep=comma] {\datagraphwf1};
% \addlegendentry{\footnotesize $\bft{x}^\tau$}


% \addplot [\treecolor, dashed, line width=1.1pt] table[x=x, y=treshold, col sep=comma] {\datagraphwf1};
% \addlegendentry{\footnotesize $C$}

% \addplot [name path=dec_above, draw=none] table[x=x, y=y_tau_above, col sep=comma] {\datagraphwfdecwf};
% \addplot [name path=dec_below, draw=none] table[x=x, y=y_tau_below, col sep=comma] {\datagraphwfdecwf};
% \addplot [
%     green, 
%     opacity=0.7,
%     pattern=north west lines, % You can choose different patterns
%     pattern color=gray % Set the pattern color
% ] fill between [of=dec_above and dec_below];

% \addplot [name path=inc_above, draw=none] table[x=x, y=y_tau_above, col sep=comma] {\datagraphwfincwf};
% \addplot [name path=inc_below, draw=none] table[x=x, y=y_tau_below, col sep=comma] {\datagraphwfincwf};
% \addplot [
%     green, 
%     opacity=0.7,
%     pattern=north west lines, % You can choose different patterns
%     pattern color=green % Set the pattern color
% ] fill between [of=inc_above and inc_below];


% \end{axis}

% \end{tikzpicture}
        
        \caption{Illustrating Theorem~\ref{thm: sw silence effect}, GenAI's instantaneous user utility vs round index. The red (circle), blue (triangle), and orange (dashed) curves represent the base strategy $\bft x$, a $\tau$-selective modification $\bft x^\tau$, and the threshold $C$, respectively. %The gray area represents rounds in which $\bft x^\tau$ has a lower instantaneous welfare, while the green area represents rounds for which $\bft x^\tau$ has a greater instantaneous welfare.
        The gray and green shaded areas highlight specific rounds: The gray region indicates rounds where $\bft x^\tau$ results in lower instantaneous welfare, while the green region denotes rounds where $\bft x^\tau$ leads to higher instantaneous welfare. \label{fig:welfare+}}
    % Second figure
\end{figure}