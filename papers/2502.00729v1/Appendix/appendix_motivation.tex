\section{Proofs Omitted from Section~\ref{sec: motivation}}

\begin{proofof}{obs: withholding impact}
We prove each clause separately.

\paragraph{1. Pareto dominance} This is shown in Example~\ref{example}, for which it holds that
\begin{itemize}
    \item $U(\Bar{\bft{x}}) < 2.356$.
    \item $U(\bft{x}) > 2.483$.
    \item $W(\Bar{\bft{x}}) < 5.73$.
    \item $W(\bft{x}) > 6.2$.
\end{itemize}


\paragraph{2. Decreases revenue and increases welfare} 
Let $T = 5$ and consider the instance $a(\mathcal{D}) = 1-e^{-0.4 \mathcal{D}}$, $\gamma = 1$, $\beta = 3$, $w^s = 0.7$ and $r(p) = p$.

We calculate the revenue and the social welfare induced by $\Bar{\bft{x}}$ by calculating the proportions for every $t \in [T]$. Therefore, the induced revenue is $U(\Bar{\bft{x}}) > 1.6$ and the social welfare $W(\Bar{\bft{x}}) < 3.3$.

Next, we denote $\bft{x} = (0, 0, \ldots, 0)$, the strategy for which GenAI never answers. By definition we have that $U(\bft{x}) = 0 < U(\Bar{\bft{x}})$ and $W(\bft{x}) = T w^s = 3.5 > W(\Bar{\bft{x}})$.


\paragraph{3. Increases revenue and decreases welfare} 
Let $T = 5$ and consider the instance $a(\mathcal{D}) = 1-e^{-0.4 \mathcal{D}}$, $\gamma = 1$, $\beta = 3$, $w^s = 0.1$ and $r(p)$ is the step function defined as
\begin{align*}
r(p) = \begin{cases}
    1 & \mbox{$p \geq q(4, 1)$} \\
    0 & \mbox{Otherwise}
\end{cases}.
\end{align*}

We denote $\bft{x}$ the strategy that satisfies
\begin{align*}
x_t = \begin{cases}
    1 & \mbox{$t = T$} \\
    0 & \mbox{Otherwise}
\end{cases}.
\end{align*}

Notice that $p_1(\Bar{\bft{x}}) > 0$ and therefore for every $t \in [5]$ it holds that $D_t(\Bar{\bft{x}}) < 4$. Thus, $U(\Bar{\bft{x}}) = 0$.

The revenue induced by $\bft{x}$ is equal to the revenue induced at round $T$. This is true since $x_t = 0$ for every $t < T$ and therefore $p_t(\bft{x}) = 0$. At round $T$, the total generated data is $\mathcal{D}_T(\bft{x}) = T-1 = 4$. Thus, $U(\bft{x}) = r(p_T(\bft{x})) = q(4, 1) > 0.89$

Calculating the welfare induced by $\Bar{\bft{x}}$ can be done by calculating $p_t(\Bar{\bft{x}})$, resulting in $W(\Bar{\bft{x}}) > 1.17$.

Similarly, we can calculate the welfare induced by strategy $\bft{x}$. Repeating the same calculation leads to $W(\bft{x}) < 1.122$; thus, we can conclude that $U(\Bar{\bft{x}}) < U(\bft{x})$ and $W(\Bar{\bft{x}}) > W(\bft{x})$. This completes the proof of Observation~\ref{obs: withholding impact}.
\end{proofof}





\subsection{Proofs Omitted from Subsection~\ref{sec: price of full response}} \label{appn: full response}

\begin{proofof}{revenue price of answering}
Consider the instance $a(\mathcal{D}) = \frac{1 + \mathcal{D}}{T}$, $\gamma = 1$, $\beta = 1$, $w^s = \frac{1}{T}$ and $r(p)$ is the sigmoid function defined as $r(p) = \frac{1}{1+e^{-\xi \left(q(T-1, 1) - p\right)}}$, such that $\xi = \frac{\ln (2TM)}{q(T-1, 1) - q(\frac{T-1}{2}, 1)}$.

Notice that for every $t \in T$ it holds that
$w^g_t(\Bar{\bft{x}}) = a(\mathcal{D}_t(\Bar{\bft{x}})) = \frac{1 + \mathcal{D}_t(\Bar{\bft{x}})}{T} > \frac{1}{T} = w^s$.
Therefore, we get that $p_t(\Bar{\bft{x}}) > 0.5$ and $\mathcal{D}_t < \frac{t-1}{2}$.

we now bound the revenue induced by $\Bar{\bft{x}}$.
\begin{align*}
U(\Bar{\bft{x}}) = \sum_{t = 1}^T r(p_t(\bft{x})) < T r(q(\frac{T-1}{2}, 1)).
\end{align*}

Next, we define the scheme that answers only at the last round $\bft{x}^\star = (0, 0, \ldots, 0, 1)$.
Notice that the revenue induced by $\bft{x}^\star$ is $U(\bft{x}^\star) = r(q(T-1, 1)) = 0.5$. Therefore,

\begin{align*}
\rpfr &= \frac{\max_{\bft{x}} U(\bft{x})}{U(\Bar{\bft{x}})} > \frac{0.5}{T r(q(\frac{T-1}{2}, 1))} = \frac{1+e^{\xi \left(q(T-1) - q(\frac{T-1}{2}) \right)}}{2T} \\
&> \frac{e^{\xi \left(q(T-1) - q(\frac{T-1}{2}) \right)}}{2T} = \frac{1+e^{\ln (2TM - 1)}}{2T} = M.
\end{align*}

Notice that it holds that 
\begin{align*}
L_r = \max_{p \in [0, 1]} \frac{dr}{dp} = \max_{p \in [0, 1]} r(p) (1-r(p)) \xi \leq \frac{\xi}{4}.
\end{align*}

For $T = 10$, we get that $L_r \approx 15.26 \ln(M)$. This completes the proof of Proposition~\ref{revenue price of answering}.
\end{proofof}










\begin{proofof}{SW Price of answering}
Let $T \in \mathbb{R}_{>0}$ and consider the instance $a(\mathcal{D}) = \frac{\mathcal{D}}{T^3}$, $\gamma = 1$, $\beta = 1$, $w^s = \frac{1}{T}$ and $r(p) = p$.

Notice that the utility of the users from GenAI is bounded by
\begin{align*}
w^g_t(\bft{x}) = a(\mathcal{D}_t) x_t = \frac{\mathcal{D}}{T^3} x_t < \frac{T}{T^3} = \frac{1}{T^2} \leq \frac{1}{T} = w^s.
\end{align*}

Furthermore, we can bound the proportions by
\begin{align*}
p_t(\Bar{\bft{x}}) = \frac{1}{1 + e^{\beta \left(w^s - w^g_t\right)}} > \frac{1}{1 + e^{\beta w^s}}.
\end{align*}

Therefore, the users' social welfare satisfies that
\begin{align*}
w_t(\Bar{\bft{x}}) &= w^g_t(\Bar{\bft{x}}) p_t(\Bar{\bft{x}}) + (1-p_t(\Bar{\bft{x}})) w^s \\
& \leq \frac{1}{T^2} p_t(\Bar{\bft{x}}) + (1-p_t(\Bar{\bft{x}})) w^s \\
& \leq \frac{1}{T^2}  \frac{1}{1 + e^{\beta w^s}} + (1- \frac{1}{1 + e^{\beta w^s}}) w^s \\
&= \frac{1}{T^2}  \frac{1}{1 + e^{\frac{\beta}{T}}} + (1- \frac{1}{1 + e^{\frac{\beta}{T} }}) w^s.
\end{align*}

Next, denote $\tilde{\bft{x}} = (0, 0, \ldots 0)$, the strategy for which GenAI does not answer any query. Therefore, by definition it holds that $w_t(\tilde{\bft{x}}) = w^s$.

We now bound the price of anarchy:
\begin{align*}
\wpfr &= \frac{\max_{\bft{x}}W(\bft{x})}{W(\Bar{\bft{x}})} \geq \frac{W(\tilde{\bft{x}})}{W(\Bar{\bft{x}})} \\
&=  \frac{T w^s}{W(\Bar{\bft{x}})} \geq \frac{T}{T} \frac{w^s}{\frac{1}{T^2}  \frac{1}{1 + e^{\frac{\beta}{T}}} + (1- \frac{1}{1 + e^{\frac{\beta}{T} }}) w^s} \\
&= \frac{1}{\frac{\beta}{T}  \frac{1}{1 + e^{\frac{\beta}{T}}} + (1- \frac{1}{1 + e^{\frac{\beta}{T} }})}.
\end{align*}

Notice that $\frac{1}{2 - \varepsilon} > 0.5$. Next, denote $h(T) = \frac{\beta}{T}  \frac{1}{1 + e^{\frac{\beta}{T}}} + (1- \frac{1}{1 + e^{\frac{\beta}{T} }})$. Observe that $h(t)$ is continuous in $T$ and satisfies the following properties:
\begin{enumerate}
    \item $h(1) = 1$,
    \item $\lim_{T \rightarrow \infty} h(T) \rightarrow 0.5$.
\end{enumerate}

Therefore, by the intermediate value theorem, there exists $T_0$ such that $h(T_0) = \frac{1}{2-\varepsilon}$. 
Furthermore,
\begin{align*}
\frac{dh}{dT} &= -\frac{\beta}{T^2} \frac{1}{1 + e^{\frac{\beta}{T}}} - \frac{\beta}{T}\frac{1}{1 + e^{\frac{\beta}{T}}} \left(1 - \frac{1}{1 + e^{\frac{\beta}{T}}}\right) \frac{\beta}{T^2} + \frac{1}{1 + e^{\frac{\beta}{T}}} \left( 1- \frac{1}{1 + e^{\frac{\beta}{T}}}\right) \frac{\beta}{T^2} \\
&= - \frac{\beta}{T}\frac{1}{1 + e^{\frac{\beta}{T}}} \left(1 - \frac{1}{1 + e^{\frac{\beta}{T}}}\right) \frac{\beta}{T^2} - \frac{1}{1 + e^{\frac{\beta}{T}}} \frac{1}{1 + e^{\frac{\beta}{T}}} \frac{\beta}{T^2} < 0;
\end{align*}
hence, for every $T > T_0$ it holds that 
\begin{align*}
\wpfr \geq \frac{1}{h(T)} \geq \frac{1}{h(T_0)} = \frac{1}{\frac{1}{2 - \varepsilon}} = 2 - \varepsilon.
\end{align*}
This completes the proof of Proposition~\ref{SW Price of answering}.
\end{proofof}










\begin{theorem} \label{PoA lower bound}
For every $M \in \mathbb{R}_{\geq 0}$ there exists an instance $I$ with $PoA(I) > M$.
\end{theorem}

\begin{proofof}{PoA lower bound}
Let $T \in \mathbb{R}_{> 0}$ and Consider the instance $a(\mathcal{D}) = \frac{\mathcal{D}}{T}$, $\gamma = 1$, $\beta = 3$, $w^s = \frac{1}{T}$. We let $r(p)$ be the step function
\begin{align*}
r(p) = \begin{cases}
    1 & \mbox{$p \geq q(T-1, 1)$} \\
    0 & \mbox{Otherwise}
\end{cases}.
\end{align*}

The purpose of choosing $r(p)$ as a step function is to show that GenAI's revenue-maximizing strategy is $(0, \ldots, 0, 1)$. Notice that we can also represent this function as a sigmoid $r(p) \approx \frac{1}{1+e^{\xi \left(q(T-1, 1) - p \right)}}$ for $\xi \rightarrow \infty$.

Notice that in each turn, the maximal amount of data that can be generated is $1$, which occurs for $x_t = 0$. Therefore, for $T-1$ rounds, the maximum amount of data that can be generated is $T-1$, which is induced by the strategy that uses $x_t = 0$ for every $t \leq T-1$.
Answering any query before round $T$ results in $r(p_t) = 0$ for every $t \in [T]$. Therefore, GenAI's optimal strategy is:
\begin{align*}
    x^\star_t = \begin{cases}
        0 & \mbox{$t < T$} \\
        1 & \mbox{Otherwise}
    \end{cases}.
\end{align*}

We now evaluate the welfare for the schemes $\bft{x}^\star$ and $\Bar{\bft{x}}$.
We start with $\bft{x}^\star$:
\begin{align*}
W(\bft{x}^\star) = \sum_{t=1}^{T-1} w_t(\bft{x}^\star) + w_T(\bft{x}^\star) = (T-1)w_1(\bft{x}^\star) + w_T(\bft{x}^\star) \leq (T-1)w^s + 1 \leq T w^s + 1 = 2.
\end{align*}

We move on to evaluate the social welfare induced by $\Bar{\bft{x}}$. First, notice that for every $T \geq 1$ it holds that
\begin{align*}
p_t(\Bar{\bft{x}}) \geq p_1(\Bar{\bft{x}}) = q(0, 1) = \frac{1}{1+e^{\beta (w^s - a(0))}} = \frac{1}{1+e^{\frac{\beta}{T}}} \geq \frac{1}{1+e^\beta} > 0.04.
\end{align*}

Similarly, we develop an upper bound on the proportions:
\begin{align*}
p_t(\Bar{\bft{x}}) = \frac{1}{1 + e^{\beta \left(w^s - a(\mathcal{D}_t(\Bar{\bft{x}})) \right)}} = \frac{1}{1 + e^{\beta \left(\frac{1}{T} - \frac{\mathcal{D}_t(\Bar{\bft{x}})}{T}\right)}} < \frac{1}{1 + e^{-\beta \frac{T}{T}}} < 0.96.
\end{align*}
Using the bound on the proportions, we can get a lower bound on the total amount of data at each round
\begin{align*}
\mathcal{D}_t(\Bar{\bft{x}}) = \sum_{t' = 1}^{t - 1} \left(1 - p_{t'}(\Bar{\bft{x}}) \right) > 0.04(t-1).
\end{align*}

This allows us to evaluate the minimal welfare induced by strategy $\Bar{\bft{x}}$:
\begin{align*}
W(\Bar{\bft{x}}) &= \sum_{t = 1}^T p_t(\Bar{\bft{x}}) w^g_t(\Bar{\bft{x}}) + (1-p_t(\Bar{\bft{x}})) w^s > \sum_{t = 1}^T p_t(\Bar{\bft{x}}) w^g_t(\Bar{\bft{x}}) > 0.04 \sum_{t = 1}^T w^g_t(\Bar{\bft{x}}) \\
&= 0.04 \sum_{t = 1}^T a(\mathcal{D}_t(\Bar{\bft{x}})) > 0.04 \sum_{t = 1}^T \frac{0.04(t-1)}{T}= \frac{0.04^2}{2}(T-1).
\end{align*}

We are now ready to plug everything we calculated so far into the definition of the PoA.

\begin{align*}
PoA &= \frac{\max_{\bft{x}}W(\bft{x})}{\min_{\bft{x} \in \max U(\bft{x})} W(\bft{x})} = \frac{\max_{\bft{x}}W(\bft{x})}{W(\bft{x}^\star)} \\ &\geq \frac{W(\Bar{\bft{x}})}{W(\bft{x}^\star)} > \frac{\frac{0.04^2}{2}(T-1)}{2} = \frac{0.04^2}{4}(T-1).
\end{align*}

Therefore, for every $T > \frac{4M}{0.04^2} + 1$, it holds that
\begin{align*}
PoA > \frac{0.04^2}{4}(\frac{4M}{0.04^2} + 1 -1) = M.
\end{align*}

This completes the proof of Theorem~\ref{PoA lower bound}.
\end{proofof}