\section{Proofs Omitted from Section~\ref{sec: welfare}}
\begin{proofof}{thm: sw silence effect}
We use the following lemma.

\begin{lemma} \label{lemma: welfare genai derivative}
Fix $x \in (0, 1]$ and let $g(y) = x\frac{e^{\beta y}}{e^{\beta y} + e^{\beta w^s}} y + \left(1-x\frac{e^{\beta y}}{e^{\beta y} + e^{\beta w^s}}\right) w^s$ then
\begin{align*}
sign\left(\frac{d g(y)}{dy}\right) = sign(y - C).
\end{align*}
\end{lemma}

First, notice that $w_t(\bft x) = g(w^g_t(\bft x))$ and we analyze each property separately. 

\begin{enumerate}
\item 
If $ w^g_t(\bft{x}) \geq C$. From Proposition~\ref{prop: monotonicity}, for every $t > \tau$ it holds that $d_t(\bft{x}^\tau) > d_t(\bft{x})$ and therefore $w^g_t(\bft{x}^\tau) \geq w^g_t(\bft{x}) \geq C$. Thus, according to Lemma~\ref{lemma: welfare genai derivative} it holds that $g(w^g_t(\bft{x}^\tau)) \geq g(w^g_t(\bft{x}))$.

\item
Using the arguments from the previous property, for every $t > \tau$ it holds that $w^g_t(\bft{x}) < w^g_t(\bft{x}^\tau) < C$ and therefore according to Lemma~\ref{lemma: welfare genai derivative} it holds that $g(w^g_t(\bft{x}^\tau)) < g(w^g_t(\bft{x}))$.

\item Assume that $w^g_t(\bft{x}) < C$. Since $x^\tau_t = x_t$ for every $t \leq \tau$ it holds that $d_t(\bft{x}^\tau) = d_t(\bft{x})$ for every $t \leq \tau$. Furthermore, since $x^\tau_\tau < x_\tau$ it holds that 
\begin{align*}
w^g_\tau(\bft{x}^\tau) = a(d_\tau(\bft{x}^\tau)) x^\tau_\tau < a(d_\tau(\bft{x})) x_\tau = w^g_\tau(\bft{x}) < C
\end{align*}
Therefore, from Lemma~\ref{lemma: welfare genai derivative} it holds that $g(w^g_\tau(\bft{x}^\tau)) > g(w^g_\tau(\bft{x}))$.
\end{enumerate}

This completes the proof of Theorem~\ref{thm: sw silence effect}.
\end{proofof}

\begin{proofof}{lemma: welfare genai derivative}
Fix $x \in [0, 1]$ and we denote $\tilde{q}(y) = \frac{e^{\beta y}}{e^{\beta y} + e^{\beta w^s}}$. Therefore, $g(y)$ can be written as $g(y) = xq(y) y + (1-xq(y)) w^s$.
The derivative $g(y)$ is

\begin{align}
\frac{dg(y)}{dy} = x\frac{d \tilde{q}(y)}{dy} y + x\tilde{q}(y) - x\frac{d \tilde{q}(y)}{dy} w^s = x\frac{d \tilde{q}(y)}{dy} (y - w^s) + x\tilde{q}(y). \label{eq: welfare genai derivative}
\end{align}

Notice that $q(y)$ is a sigmoid function and therefore $\frac{d\tilde{q}(y)}{dy} = \beta \tilde{q}(y) \left(1-\tilde{q}(y) \right)$. Plugging this result in Equation~\ref{eq: welfare genai derivative} results in
\begin{align*}
\frac{dg(y)}{dy} = x\tilde{q}(y) \left(1-\tilde{q}(y) \right) \beta (y-w^s) + x\tilde{q}(y).
\end{align*}

Next, notice that $\tilde{q}(y) = \frac{e^{\beta y}}{e^{\beta y} + e^{\beta w^s}} = \frac{1}{1 + e^{\beta (w^s - y)}}$. We denote $z = \beta(y-w^s)$ and get
\begin{align*}
\frac{dg(y)}{dy} &= x\frac{1}{1 + e^{-z}} \frac{1}{1 + e^z} z + x\frac{1}{1 + e^{-z}} \\
&= x\frac{1}{1 + e^{-z}} \left(\frac{1}{1 + e^z} z + 1 \right) \\
&= x\frac{1}{1 + e^{-z}} \frac{z + 1 + e^z}{1 + e^z} \\
&= x\frac{1}{1 + e^{-z}} \frac{e^{z + 1}}{1 + e^z} \left( (z + 1)e^{-(z+1)} + e^{-1} \right).
\end{align*}

Therefore, to find the $y_0$ that results in $\frac{dg}{dy}|_{y = y_0} = 0$ is equivalent to finding the solution of 
\[(z + 1)e^{-(z+1)} + e^{-1} = 0. \]

Denote $\tilde{z} = -(z+1)$ and we have the inverse of the Lambert function
\begin{align*}
\tilde{z} e^{\tilde{z}} = e^{-1}
\end{align*}
and therefore $\tilde{z} = \mathcal{W}(e^{-1})$, which leads to $z_0 = -\mathcal{W}(e^-1) - 1$ and $y_0 = -\frac{\mathcal{W}(e^{-1}) + 1}{\beta} + w^s = C$.

Next, denote $h(z) = (z+1)e^{-(z+1)} + e^{-1}$ and notice that the sign of $\frac{dg}{dy}$ is determined by the sign of $h(z)$, that is $sign(\frac{dg}{dy}) = sign(h(z))$.

The derivative of $h(z)$ is given by
\begin{align*}
\frac{dh(z)}{dz} = e^{-(z+1)} - (z+1)e^{-(z+1)} = \left( 1 - (z+1) \right)e^{-(z+1)} = -ze^{-(z+1)}.
\end{align*}
Therefore $h(z)$ is an increasing function for $z < 0$ and a decreasing function for $z > 0$. Recall that $h(z_0) = 0$ and $z_0 < -1 < 0$ thus $h(z) < 0$ for every $z < z_0$.
Furthermore, $h(z)$ is an increasing function in $z \in [z_0, 0)$, therefore it holds that $h(z) > 0$ for every $z \in (z_0, 0)$. Lastly, notice that for every $z > 0$ it holds that $z + 1 > 0$ and $e^{-(z+1)} > 0$. and as such we can summarize that $h(z) > 0$ for every $z > z_0$.

This completes the proof of Lemma~\ref{lemma: welfare genai derivative}.
\end{proofof}