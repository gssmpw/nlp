\section{Proofs Omitted from Section~\ref{sec: regulation}}\label{appendix:regulation}

\subsection{Proofs Omitted from Subsection~\ref{sec: rev diff}}

\begin{comment}
\begin{corollary}\label{cor: loose bound}
It holds that
\begin{align*}
&U(\bft{x}^\tau) - U(\bft{x}) \leq \\
&\gamma^{\tau - 1} \left( r(p_\tau(\bft{x}^\tau)) - r(p_\tau(\bft{x})) \right) + L_r \gamma^{\tau} \frac{p_\tau(\bft{x}^\tau) - p_\tau(\bft{x})}{1 - \gamma}.
\end{align*}
\end{corollary}
\end{comment}


\begin{proofof}{cor: loose bound}
First, notice that for every $t \leq \tau$ it holds that $\mathcal{D}_t(\bft{x}) = \mathcal{D}_t(\bft{x}^\tau)$. Next, from Lemma~\ref{lemma: non-expanding data} it holds that for every $t > \tau$, the data satisfies
\begin{align*}
\left| \mathcal{D}_t(\bft{x}) - \mathcal{D}_t(\bft{x}^\tau) \right| \leq \left| \mathcal{D}_{\tau + 1}(\bft{x}) - \mathcal{D}_{\tau + 1}(\bft{x}^\tau) \right| = \left| p_\tau(\bft{x}) - p_{\tau}(\bft{x}^\tau) \right|.
\end{align*}

Therefore, we can bound the revenue:
\begin{align*}
U(\bft{x}^\tau) - U(\bft{x}) \leq \gamma^{\tau - 1} \left( r(p_\tau(\bft{x}^\tau)) - r(p_\tau(\bft{x})) \right) + L_r \sum_{t = \tau + 1}^T \gamma^t \left( p_t(\bft{x}^\tau) - p_t(\bft{x})\right).
\end{align*}

By Lemma~\ref{lemma: limited diff q same action} we get that
\begin{align*}
U(\bft{x}^\tau) - U(\bft{x}) &\leq \gamma^{\tau - 1} \left( r(p_\tau(\bft{x}^\tau)) - r(p_\tau(\bft{x})) \right) + L_r \sum_{t = \tau + 1}^T \gamma^t \left( \mathcal{D}_t(\bft{x}^\tau) - \mathcal{D}_t(\bft{x})\right) \\
&\leq \gamma^{\tau - 1} \left( r(p_\tau(\bft{x}^\tau)) - r(p_\tau(\bft{x})) \right) + L_r \left( p_\tau(\bft{x}) - p_\tau(\bft{x}^\tau)\right) \sum_{t = \tau + 1}^T \gamma^t \\
&\leq \gamma^{\tau - 1} \left( r(p_\tau(\bft{x}^\tau)) - r(p_\tau(\bft{x})) \right) + L_r \gamma^{\tau} \frac{p_\tau(\bft{x}^\tau) - p_\tau(\bft{x})}{1 - \gamma}.
\end{align*}

This completes the proof of Corollary~\ref{cor: loose bound}.
\end{proofof}

\begin{theorem} \label{thm: revenue bounds}
Let $\unx = \min \{ x_t \mid t > \tau, x_t > 0 \}$ and $k = \frac{\beta \min_{\mathcal{D} \in [0, T]} \frac{da(\mathcal{D})}{\mathcal{D}}}{4 \left(1+e^{\beta w^s} \right)^2} $. If $\beta L_a \leq 1$, then
\begin{align*}
&U(\bft{x}^\tau) - U(\bft{x}) < \\
&\gamma^{\tau - 1} \left( r(p_\tau(\bft{x}^\tau)) - r(p_\tau(\bft{x})) \right) + L_r \gamma^{\tau} \frac{p_\tau(\bft{x}^\tau)-p_\tau(\bft{x})}{1-\gamma \left( 1 - k \unx^2 \right)}.
\end{align*}
\end{theorem}


\begin{proofof}{thm: revenue bounds}
By definition, we get that
\begin{align*}
U(\bft{x}^\tau) - U(\bft{x}) &= \sum_{t = 1}^T \gamma^{t-1} r(p_t(\bft{x}^\tau)) - \sum_{t = 1}^T \gamma^{t-1} r(p_t(\bft{x})) \\
&= \sum_{t = 1}^T \gamma^{t-1} \left(r(p_t(\bft{x}^\tau)) - r(p_t(\bft{x})) \right) \\
&= \sum_{t = \tau}^T \gamma^{t-1} \left( r(p_t(\bft{x}^\tau)) - r(p_t(\bft{x})) \right) \\
&= \gamma^{\tau - 1} \left( r(p_\tau(\bft{x}^\tau)) - r(p_\tau(\bft{x})) \right) + \sum_{t = \tau + 1}^T \gamma^{t-1} \left( r(p_t(\bft{x}^\tau)) - r(p_t(\bft{x})) \right) \\
&\leq \gamma^{\tau - 1} \left( r(p_\tau(\bft{x}^\tau)) - r(p_\tau(\bft{x})) \right) + L_r \sum_{t = \tau + 1}^T \gamma^{t-1} \left( p_t(\bft{x}^\tau) - p_t(\bft{x}) \right).
\end{align*}

Next, we use the following lemma to get an upper bound on $p_t(\bft{x}^\tau) - p_t(\bft{x})$.

\begin{lemma} \label{lemma: proportions geometric}
For every $t > \tau$ it holds that
\begin{align*}
0 \leq p_t(\bft{x}^\tau) - p_t(\bft{x}) \leq \left| p_\tau(\bft{x}) - p_\tau(\bft{x}^\tau) \right| \prod_{i = \tau + 1}^{t-1} \left( 1 - \frac{\unsig ^2}{4} x_i^2 \beta \unla \right).
\end{align*}
\end{lemma}

Therefore, according to Lemma~\ref{lemma: proportions geometric}, it holds that
\begin{align*}
U(\bft{x}^\tau) - U(\bft{x}) &\leq \gamma^{\tau - 1} \left( r(p_\tau(\bft{x}^\tau)) - r(p_\tau(\bft{x})) \right) + L_r \sum_{t = \tau + 1}^T \gamma^{t-1} \left| p_\tau(\bft{x}) - p_\tau(\bft{x}^\tau) \right| \prod_{i = \tau + 1}^{t-1} \left( 1 - \frac{\unsig ^2}{4} x_i^2 \beta \unla \right) \\
&\leq \gamma^{\tau - 1} \left( r(p_\tau(\bft{x}^\tau)) - r(p_\tau(\bft{x})) \right) +  \left| p_\tau(\bft{x}) - p_\tau(\bft{x}^\tau) \right| L_r \sum_{t = \tau + 1}^T \gamma^{t-1} x_t^2 \prod_{i = \tau + 1}^{t-1} \left( 1 - \frac{\unsig ^2}{4} x_i^2 \beta \unla \right).
\end{align*}

We now simplify the second term using the following lemma.
\begin{lemma} \label{lemma: simplify bound summation}
It holds that
\begin{align*}
\sum_{t = \tau + 1}^T \gamma^{t-1} x_t^2 \prod_{i = \tau + 1}^{t-1} \left( 1 - \frac{\unsig ^2}{4} x_i^2 \beta \unla \right) \leq \sum_{t = \tau + 1}^{T} \gamma^{t-1} \left( 1 - \frac{\unsig ^2}{4} \unx^2 \beta \unla \right)^{t - \tau - 1}.
\end{align*}
\end{lemma}

Therefore, we get that
\begin{align*}
U(\bft{x}^\tau) - U(\bft{x}) &\leq \gamma^{\tau - 1} \left( r(p_\tau(\bft{x}^\tau)) - r(p_\tau(\bft{x})) \right) +  \left| p_\tau(\bft{x}) - p_\tau(\bft{x}^\tau) \right| L_r \sum_{t = \tau + 1}^{T} \gamma^{t-1} \left( 1 - \frac{\unsig ^2}{4} \unx^2 \beta \unla \right)^{t - \tau - 1} \\
&= \gamma^{\tau - 1} \left( r(p_\tau(\bft{x}^\tau)) - r(p_\tau(\bft{x})) \right) +  \left| p_\tau(\bft{x}) - p_\tau(\bft{x}^\tau) \right| L_r \gamma^\tau \sum_{t = 1}^{T - \tau} \gamma^{t-1} \left( 1 - \frac{\unsig ^2}{4} \unx^2 \beta \unla \right)^{t - 1}.
\end{align*}

Notice that $\sum_{t = 1}^{T-\tau} \gamma^{t-1} 
\left( 1 - \frac{\unsig ^2}{4} \unx^2 \beta \unla \right) ^{t - 1}$ is a sum of a geometric series, and therefore it holds that
\begin{align*}
\sum_{t = 1}^{T-\tau} \gamma^{t-1} 
\left( 1 - \frac{\unsig ^2}{4} \unx^2 \beta \unla \right) ^{t - 1} = \frac{\left(\gamma \left( 1 - \frac{\unsig ^2}{4} \unx^2 \beta \unla \right) \right)^{T - \tau} - 1}{\gamma \left( 1 - \frac{\unsig ^2}{4} \unx^2 \beta \unla \right) - 1}.
\end{align*}

Thus, we conclude that
\begin{align*}
U(\bft{x}^\tau) - U(\bft{x}) &\leq  \gamma^{\tau - 1} \left( r(p_\tau(\bft{x}^\tau)) - r(p_\tau(\bft{x})) \right) +  \left| p_\tau(\bft{x}) - p_\tau(\bft{x}^\tau) \right| L_r \gamma^{\tau} \frac{\left(\gamma \left( 1 - \frac{\unsig ^2}{4} \unx^2 \beta \unla \right) \right)^{T - \tau} - 1}{\gamma \left( 1 - \frac{\unsig ^2}{4} \unx^2 \beta \unla \right) - 1}.
\end{align*}

This completes the proof of Theorem~\ref{thm: revenue bounds}.
\end{proofof}






























\begin{proofof}{lemma: proportions geometric}
We start from the left inequality. From Theorem~\ref{thm: not answering increase proportions} it holds that $p_t(\bft{x}^\tau) \geq p_t(\bft{x})$ for every $t > \tau$.

We move on to the right inequality. For that, we use Lemma~\ref{lemma: limited diff q same action} and get that
\begin{align*}
p_t(\bft{x}^\tau) - p_t(\bft{x}) < \mathcal{D}_t(\bft{x}^\tau) - \mathcal{D}_t(\bft{x}).
\end{align*}

Next, we couple it with the following lemma.
\begin{lemma} \label{converging proportions}
For every $t > \tau$ it holds that
\begin{align*}
0 < \mathcal{D}_t(\bft{x}^\tau) - \mathcal{D}_t(\bft{x}) \leq \left| p_\tau(\bft{x}) - p_\tau(\bft{x}^\tau) \right| \prod_{i = \tau + 1}^{t-1} \left( 1 - \frac{\unsig ^2}{4} x_i^2 \beta \unla \right).
\end{align*}
\end{lemma}
Therefore, we conclude that
\begin{align*}
p_t(\bft{x}^\tau) - p_t(\bft{x}) &< \mathcal{D}_t(\bft{x}^\tau) - \mathcal{D}_t(\bft{x})\\
&\leq \left| p_\tau(\bft{x}) - p_\tau(\bft{x}^\tau) \right| \prod_{i = \tau + 1}^{t-1} \left( 1 - \frac{\unsig ^2}{4} x_i^2 \beta \unla \right).
\end{align*}
This completes the proof of Lemma~\ref{lemma: proportions geometric}.
\end{proofof}
















\begin{proofof}{lemma: q equality}
We expand it according to the definition:
\begin{align*}
\left| q(\mathcal{D}^1, x^1) - q(\mathcal{D}^2, x^2) \right| &= \left| \frac{e^{\beta a(\mathcal{D}^1) x^1}}{e^{\beta a(\mathcal{D}^1) x^1} + e^{\beta w^s}} - \frac{e^{\beta a(\mathcal{D}^2) x^2}}{e^{\beta a(\mathcal{D}^2) x^2} + e^{\beta w^s}}\right| \\
&= \left| \frac{e^{\beta a(\mathcal{D}^1) x^1} \left( e^{\beta a(\mathcal{D}^2) x^2} + e^{\beta w^s} \right) - e^{\beta a(\mathcal{D}^2) x^2} \left( e^{\beta a(\mathcal{D}^1) x^1} + e^{\beta w^s} \right)}{\left( e^{\beta a(\mathcal{D}^1) x^1} + e^{\beta w^s} \right) \left( e^{\beta a(\mathcal{D}^2) x^2} + e^{\beta w^s} \right)} \right| \\
&= \left| e^{\beta w^s} \frac{e^{\beta a(\mathcal{D}^1) x^1} - e^{\beta a(\mathcal{D}^2) x^2}}{\left( e^{\beta a(\mathcal{D}^1) x^1} + e^{\beta w^s} \right) \left( e^{\beta a(\mathcal{D}^2) x^2} + e^{\beta w^s} \right)} \right| \\
&= \left| e^{\beta w^s} e^{\beta a(\mathcal{D}^1) x^1} \frac{1 - e^{\beta \left(x^2 a(\mathcal{D}^2) - x^1 a(\mathcal{D}^1) \right)}}{\left( e^{\beta a(\mathcal{D}^1) x^1} + e^{\beta w^s} \right) \left( e^{\beta a(\mathcal{D}^2) x^2} + e^{\beta w^s} \right)} \right| \\
&= \left| q(\mathcal{D}^1, x^1) \left(1-q(\mathcal{D}^2, x^2) \right) \left( 1 - e^{\beta \left(x^2 a(\mathcal{D}^2) - x^1 a(\mathcal{D}^1) \right)} \right) \right| \\
&= q(\mathcal{D}^1, x^1) \left(1-q(\mathcal{D}^2, x^2)\right) \left| 1 - e^{\beta \left(x^2 a(\mathcal{D}^2) - x^1 a(\mathcal{D}^1) \right)} \right|.
\end{align*}

This completes the proof of Lemma~\ref{lemma: q equality}.
\end{proofof}















\begin{proofof}{converging proportions}
We prove it by induction, starting with the base case at $t = \tau + 1$. By definition,
\begin{align*}
\left| \mathcal{D}_{\tau + 1}(\bft{x}) - \mathcal{D}_{\tau + 1}(\bft{x}^\tau) \right| &= \left| \mathcal{D}_{\tau}(\bft{x}) - p_\tau (\bft{x}) - \mathcal{D}_{\tau}(\bft{x}^\tau) + p_\tau (\bft{x}^\tau) \right|.
\end{align*}
Since $\mathcal{D}_{\tau}(\bft{x}) = \mathcal{D}_{\tau}(\bft{x}^\tau)$ we get that
\begin{align*}
\left| \mathcal{D}_{\tau + 1}(\bft{x}) - \mathcal{D}_{\tau + 1}(\bft{x}^\tau) \right| &= \left| p_\tau (\bft{x}) -  p_\tau (\bft{x}^\tau) \right|.
\end{align*}

Therefore, we can conclude the base case. Next, assume that the inequality holds for $t > \tau + 1$, and we prove for $t + 1$.

We use the following lemma:
\begin{lemma} \label{lemma: data contracting}
For every $t > \tau + 1$ it holds that
\begin{align*}
\left| \mathcal{D}_{t+1}(\bft{x}) - \mathcal{D}_{t+1}(\bft{x}^\tau) \right| \leq \left(1 - \frac{\unsig ^2}{4} x_t^2 \beta \unla \right) \left| \mathcal{D}_{t}(\bft{x}) - \mathcal{D}_{t}(\bft{x}^\tau) \right|.
\end{align*}
\end{lemma}

We plug the inequality from our assumption into the inequality of lemma~\ref{lemma: data contracting}, Therefore, we get that
\begin{align*}
\left| \mathcal{D}_{t+1}(\bft{x}) - \mathcal{D}_{t+1}(\bft{x}^\tau) \right| &\leq \left(1 - \frac{\unsig ^2}{4} x_t^2 \beta \unla \right) \left| \mathcal{D}_{t}(\bft{x}) - \mathcal{D}_{t}(\bft{x}^\tau) \right| \\
&= \left(1 - \frac{\unsig ^2}{4} x_t^2 \beta \unla \right) \left( \mathcal{D}_{t}(\bft{x}^\tau) - \mathcal{D}_{t}(\bft{x}) \right) \\
&\leq \left(1 - \frac{\unsig ^2}{4} x_t^2 \beta \unla \right) \left| p_\tau(\bft{x}) - p_\tau(\bft{x}^\tau) \right| \prod_{i = \tau + 1}^{t-1} \left( 1 - \frac{\unsig ^2}{4} x_i^2 \beta \unla \right) \\
&= \left| p_\tau(\bft{x}) - p_\tau(\bft{x}^\tau) \right| \prod_{i = \tau + 1}^{t} \left( 1 - \frac{\unsig ^2}{4} x_i^2 \beta \unla \right).
\end{align*}

This completes the proof of Lemma~\ref{converging proportions}.
\end{proofof}


\begin{proofof}{lemma: data contracting}
By definition,
\begin{align*}
\left| \mathcal{D}_{t+1}(\bft{x}) - \mathcal{D}_{t+1}(\bft{x}^\tau) \right| = \left| \mathcal{D}_{t}(\bft{x}) - \mathcal{D}_{t}(\bft{x}^\tau) + p_t(\bft{x}^\tau) - p_t(\bft{x}) \right|.
\end{align*}

Since $y < x_\tau$ then from Theorem~\ref{thm: not answering increase proportions} it holds that $\mathcal{D}_t(\bft{x}^\tau) > \mathcal{D}_t(\bft{x})$ and $p_t(\bft{x}^\tau) > p_t(\bft{x})$ for every $t > \tau$. Next, we get an upper bound using the following lemma, which suggests a lower bound for the proportions.
\begin{lemma} \label{lemma: prop data lowerbound}
For every $t > \tau$, it holds that
\begin{align*}
q(\mathcal{D}_t(\bft{x}^\tau), x_t) - q(\mathcal{D}_t(\bft{x}), x_t) \geq \frac{\unsig ^2}{4} x \beta \unla \left( \mathcal{D}_t(\bft{x}^\tau) - \mathcal{D}_t(\bft{x}) \right).
\end{align*}
\end{lemma}
Using Lemma~\ref{lemma: prop data lowerbound}, we get that
\begin{align*}
\left| \mathcal{D}_{t+1}(\bft{x}) - \mathcal{D}_{t+1}(\bft{x}^\tau) \right| &= \mathcal{D}_{t}(\bft{x}^\tau) - \mathcal{D}_{t}(\bft{x}) + p_t(\bft{x}) - p_t(\bft{x}^\tau) \\
&\leq \mathcal{D}_{t}(\bft{x}^\tau) - \mathcal{D}_{t}(\bft{x}) - \frac{\unsig ^2}{4} x_t^2 \beta \unla \left( \mathcal{D}_t(\bft{x}^\tau) - \mathcal{D}_t(\bft{x}) \right) \\
&= \left(\mathcal{D}_{t}(\bft{x}^\tau) - \mathcal{D}_{t}(\bft{x}) \right) \left(1 - \frac{\unsig ^2}{4} x_t^2 \beta \unla \right) \\
&= \left(1 - \frac{\unsig ^2}{4} x_t^2 \beta \unla \right) \left| \mathcal{D}_{t}(\bft{x}) - \mathcal{D}_{t}(\bft{x}^\tau) \right|.
\end{align*}

This completes the proof of Lemma~\ref{lemma: data contracting}.
\end{proofof}




\begin{proofof}{lemma: prop data lowerbound}
From Theorem~\ref{thm: not answering increase proportions}, for every $t > \tau$ it holds that $\mathcal{D}_t(\bft{x}^\tau) > \mathcal{D}_t(\bft{x})$. Therefore, we get that $a(\mathcal{D}_t(\bft{x}^\tau)) > a(\mathcal{D}_t(\bft{x}))$. Furthermore, from Proposition~\ref{lemma: property data monotone} it holds that $q(\mathcal{D}_t(\bft{x}^\tau), x_t) \geq q(\mathcal{D}_t(\bft{x}), x_t)$. Thus, we use the following lemma to write $q(\mathcal{D}_t(\bft{x}^\tau), x) - q(\mathcal{D}_t(\bft{x}), x)$ differently:
\begin{lemma} \label{lemma: q equality}
For every $\mathcal{D}^1, \mathcal{D}^2 \in [0, T]$ and $x^1, x^2 \in [0, 1]$ it holds that
\begin{align*}
\left| q(\mathcal{D}^1, x^1) - q(\mathcal{D}^2, x^2) \right| = q(\mathcal{D}^1, x^1) \left(1-q(\mathcal{D}^2, x^2)\right) \left| 1 - e^{\beta \left(x^2 a(\mathcal{D}^2) - x^1 a(\mathcal{D}^1) \right)} \right|.
\end{align*}
\end{lemma}

Therefore,
\begin{align}
q(\mathcal{D}_t(\bft{x}^\tau), x) - q(\mathcal{D}_t(\bft{x}), x) &= \left| q(\mathcal{D}_t(\bft{x}^\tau), x) - q(\mathcal{D}_t(\bft{x}), x) \right| \label{eq: sub q equality} \\
\nonumber &= q(\mathcal{D}_t(\bft{x}^\tau), x) \left(1-q(\mathcal{D}_t(\bft{x}), x)\right) \left| 1 - e^{x \beta \left( a(\mathcal{D}_t(\bft{x})) - a(\mathcal{D}_t(\bft{x}^\tau)) \right)} \right|.
\end{align}

Notice that $q(\mathcal{D}, x), 1-q(\mathcal{D}, x) \geq \unsig$ for every $\mathcal{D} \in [0, T]$ and $x \in [0, 1]$. Furthermore, it holds that $\left| a(\mathcal{D}^2) - a(\mathcal{D}^1) \right| \geq \unla \left| \mathcal{D}^2 - \mathcal{D}^1 \right|$. Therefore,
\begin{align*}
a(\mathcal{D}_t(\bft{x})) - a(\mathcal{D}_t(\bft{x}^\tau)) = -\left| a(\mathcal{D}_t(\bft{x})) - a(\mathcal{D}_t(\bft{x}^\tau)) \right| \leq -\unla \left| \mathcal{D}_t(\bft{x}) - \mathcal{D}_t(\bft{x}^\tau) \right| = \unla (\mathcal{D}_t(\bft{x}) - \mathcal{D}_t(\bft{x}^\tau)).
\end{align*}

Notice that $\unla (\mathcal{D}_t(\bft{x}) - \mathcal{D}_t(\bft{x}^\tau)) \leq 0$ and therefore $\left| 1 - e^{x \beta \left( a(\mathcal{D}_t(\bft{x})) - a(\mathcal{D}_t(\bft{x}^\tau)) \right)} \right| > \left| 1 - e^{x \beta \unla \left(\mathcal{D}_t(\bft{x}) - \mathcal{D}_t(\bft{x}^\tau) \right)} \right|$.
Plugging everything into Equation~\eqref{eq: sub q equality} results in the following inequality:

\begin{align*}
q(\mathcal{D}_t(\bft{x}^\tau), x) - q(\mathcal{D}_t(\bft{x}), x) \geq \unsig^2 \left| 1- e^{x \beta \unla \left( \mathcal{D}_t(\bft{x}) - \mathcal{D}_t(\bft{x}^\tau) \right)} \right|.
\end{align*}

Next, we show that $x\beta \unla \left| \mathcal{D}_t(\bft{x}) - \mathcal{D}_t(\bft{x}^\tau) \right| \leq 1$. For that, we use the following lemma.
\begin{lemma} \label{lemma: bounded data at 1}
For every $t > \tau$ it holds that $\left| \mathcal{D}_t(\bft{x}) - \mathcal{D}_t(\bft{x}^\tau) \right| \leq 1$.
\end{lemma}

Therefore, we get that
\begin{align*}
x\beta \unla \left| \mathcal{D}_t(\bft{x}) - \mathcal{D}_t(\bft{x}^\tau) \right| \leq x\beta \unla \leq \beta \unla \leq \beta L_a \leq 1.
\end{align*}

Thus, we can use the inequality $\left| 1 - e^\alpha \right| \geq \frac{\left|\alpha\right|}{4}$ for $\left| \alpha \right| \leq 1$ and conclude that

\begin{align*}
q(\mathcal{D}_t(\bft{x}^\tau), x) - q(\mathcal{D}_t(\bft{x}), x) &\geq \frac{\unsig^2}{4} x \beta \unla \left| \mathcal{D}_t(\bft{x}) - \mathcal{D}_t(\bft{x}^\tau) \right| \\
& \frac{\unsig^2}{4} x \beta \unla \left( \mathcal{D}_t(\bft{x}^\tau) - \mathcal{D}_t(\bft{x}) \right).
\end{align*}

This completes the proof of Lemma~\ref{lemma: prop data lowerbound}.
\end{proofof}

\begin{proofof}{lemma: bounded data at 1}
By definition, we get that
\begin{align*}
\left| \mathcal{D}_t(\bft{x}) - \mathcal{D}_t(\bft{x}^\tau) \right| &= \left| \mathcal{D}_{t-1}(\bft{x}) + (1-p_{t-1}(\bft{x})) - \mathcal{D}_{t-1}(\bft{x}^\tau) - (1-p_{t-1}(\bft{x}^\tau)) \right| \\
&= \left| \mathcal{D}_{t-1}(\bft{x}) -p_{t-1}(\bft{x}) - \mathcal{D}_{t-1}(\bft{x}^\tau) + p_{t-1}(\bft{x}^\tau) \right| \\
&= \mathcal{D}_{t-1}(\bft{x}^\tau) - \mathcal{D}_{t-1}(\bft{x}) + p_{t-1}(\bft{x}) - p_{t-1}(\bft{x}^\tau).
\end{align*}

Observe that the proportions satisfies that $p_{t-1}(\bft{x}) - p_{t-1}(\bft{x}^\tau) \leq 0$. Therefore,
\begin{align*}    
\left| \mathcal{D}_t(\bft{x}) - \mathcal{D}_t(\bft{x}^\tau) \right| \leq \left| \mathcal{D}_{t-1}(\bft{x}^\tau) - \mathcal{D}_{t-1}(\bft{x}) \right|.
\end{align*}
Thus, by induction it follows that
\begin{align*}
\left| \mathcal{D}_t(\bft{x}) - \mathcal{D}_t(\bft{x}^\tau) \right| &\leq \left| \mathcal{D}_{\tau + 1}(\bft{x}) - \mathcal{D}_{\tau + 1}(\bft{x}^\tau) \right| \\
&= \left| \mathcal{D}_{\tau}(\bft{x}) + (1-p_{\tau}(\bft{x})) - \mathcal{D}_{\tau}(\bft{x}^\tau) - (1-p_{\tau}(\bft{x}^\tau)) \right| \\
&= \left| p_{\tau}(\bft{x}^\tau) - p_{\tau}(\bft{x}) \right| \leq 1.
\end{align*}

This completes the proof of Lemma~\ref{lemma: bounded data at 1}.
\end{proofof}



\begin{proofof}{lemma: simplify bound summation}
Let $t' > \tau$ be the maximum $t' \in [T]$ such that $x_{t'} = 0$. Therefore,
\begin{align*}
& \sum_{t = \tau + 1}^T \gamma^{t-1} x_t^2 \prod_{i = \tau + 1}^{t-1} \left( 1 - \frac{\unsig ^2}{4} x_i^2 \beta \unla \right) \\
&= \sum_{t = \tau + 1}^{t' - 1} \gamma^{t-1} x_t^2 \prod_{i = \tau + 1}^{t-1} \left( 1 - \frac{\unsig ^2}{4} x_i^2 \beta \unla \right) + \sum_{t = t' + 1}^T \gamma^{t-1} x_t^2 \prod_{i = \tau + 1}^{t-1} \left( 1 - \frac{\unsig ^2}{4} x_i^2 \beta \unla \right) \\
&\leq \sum_{t = \tau + 1}^{t' - 1} \gamma^{t-1} x_t^2 \prod_{i = \tau + 1}^{t-1} \left( 1 - \frac{\unsig ^2}{4} x_i^2 \beta \unla \right) + \sum_{t = t' + 1}^T \gamma^{t-1} \prod_{i = \tau + 1}^{t-1} \left( 1 - \frac{\unsig ^2}{4} x_i^2 \beta \unla \right).
\end{align*}

We now focus on the second term:
\begin{align*}
\sum_{t = t' + 1}^T \gamma^{t-1} \prod_{i = \tau + 1}^{t-1} \left( 1 - \frac{\unsig ^2}{4} x_i^2 \beta \unla \right) &= \sum_{t = t' + 1}^T \gamma^{t-1} \prod_{\substack{i = \tau + 1 \\ i \neq t'}}^{t-1} \left( 1 - \frac{\unsig ^2}{4} x_i^2 \beta \unla \right) \\
&\leq \sum_{t = t' + 1}^T \gamma^{t-1} \prod_{\substack{i = \tau + 1 \\ i \neq t'}}^{t-1} \left( 1 - \frac{\unsig ^2}{4} \unx^2 \beta \unla \right) \\
&= \sum_{t = t' + 1}^T \gamma^{t-1} \left( 1 - \frac{\unsig ^2}{4} \unx^2 \beta \unla \right)^{t - \tau - 2} \\
&= \sum_{t = t'}^{T-1} \gamma^{t} \left( 1 - \frac{\unsig ^2}{4} \unx^2 \beta \unla \right)^{t - \tau - 1} \\
&\leq \sum_{t = t'}^{T-1} \gamma^{t-1} \left( 1 - \frac{\unsig ^2}{4} \unx^2 \beta \unla \right)^{t - \tau - 1} \\
&\leq \sum_{t = t'}^{T} \gamma^{t-1} \left( 1 - \frac{\unsig ^2}{4} \unx^2 \beta \unla \right)^{t - \tau - 1}.
\end{align*}
Therefore, we conclude that
\begin{align*}
\sum_{t = \tau + 1}^T \gamma^{t-1} x_t^2 \prod_{i = \tau + 1}^{t-1} \left( 1 - \frac{\unsig ^2}{4} x_i^2 \beta \unla \right) \leq \sum_{t = \tau + 1}^{t' - 1} \gamma^{t-1} x_t^2 \prod_{i = \tau + 1}^{t-1} \left( 1 - \frac{\unsig ^2}{4} x_i^2 \beta \unla \right) + \sum_{t = t'}^{T} \gamma^{t-1} \left( 1 - \frac{\unsig ^2}{4} \unx^2 \beta \unla \right)^{t - \tau - 1}.
\end{align*}

At this point, we iteratively apply it while going backward using backward induction. In each step, we take the latest round $t'$ such that $x_t' = 0$ and apply the equation above to the first term. Ultimately, we get that

\begin{align*}
\sum_{t = \tau + 1}^T \gamma^{t-1} x_t^2 \prod_{i = \tau + 1}^{t-1} \left( 1 - \frac{\unsig ^2}{4} x_i^2 \beta \unla \right) \leq \sum_{t = \tau + 1}^{T} \gamma^{t-1} \left( 1 - \frac{\unsig ^2}{4} \unx^2 \beta \unla \right)^{t - \tau - 1}.
\end{align*}

This completes the proof of Lemma~\ref{lemma: simplify bound summation}.
\end{proofof}