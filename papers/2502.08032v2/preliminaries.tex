\section{Preliminaries} \label{sec:prelim}

\paragraph{Graph terminology.} Given a directed graph (digraph) $G=(V,E)$, a directed edge, denoted by an ordered pair of vertices $(u,v)\in E$, is called an in-edge for $v$ and an out-edge for $u$. The vertices $u$ and $v$ are called the endpoints of the edge $(u,v)$. The out-degree of a node $u\in V$ denotes the number of out-edges $(u,v)\in E$, and the in-degree denotes the number of in-edges $(v,u)\in E$. The maximum in-degree and out-degree of $G$ are the highest in-degree and out-degree values among all nodes, respectively.

A path $p$ of length $\ell$ is a vertex sequence $(v_0,v_1,...,v_\ell)$ where $(v_i,v_{i+1})\in E$ for any $0\le i<\ell$. The following phrases are equivalent and used interchangeably: $p$ is a path from $v_0$ to $v_\ell$, $v_0$ has a path to $v_\ell$, $v_0$ can reach $v_\ell$, $v_\ell$ is reachable from $v_0$, and $(v_0,v_\ell)$ is a reachable (ordered) pair. The distance between a pair of vertices $(v_0, v_\ell)$ is defined as the minimum length of paths from $v_0$ to $v_\ell$, denoted by $\distt{G}{v_0,v_\ell}$. The diameter of graph $G$ is defined as the maximum distance between any two reachable vertex pairs.

Given a digraph $G$, we use $G^T = (V, E^T )$ to represent the transitive closure of $G$. In other words, $(u, v) \in E^T $ if and only if vertex $u$ can reach vertex $v$ in $G$. For an edge $(u, v) \in E$, vertices $u$ and $v$ are referred to as the endpoints of the edge $(u, v)$. A subgraph of $G$ is a graph $G' = (V', E')$, where $E' \subseteq E$ and $V'$ contain all vertices that are endpoints of edges in $E'$. We will slightly stretch the terminology and use $E'$ to also denote the subgraph $G'$. For a given set of vertices $V' \subseteq V$, the subgraph induced by $V'$ is denoted by $G[V']$. 

\subsection{Shortcut and \tc{}}\label{subsec:shortcut}

We first introduce the definition of \tc{}.

\begin{definition}[\tc{}]
For a digraph $G=(V,E)$, any subset of edges $E' \subseteq E^T \backslash E$ that have the same transitive closure of $G$ is called a \emph{\tc{}} of $G$. If $E'$ has diameter at most $d$ and size at most $s$, we call $E'$ as \TCs{s}{d}.

\end{definition}


The following problem is about \tc{} approximation.


\begin{definition}[\TC{\apxS}{\apxD}]
Given a directed graph $G$ and integers $d$ and $s$, such that $G$ admits a \TCs{s}{d}, the goal is to find a \TCs{\apxS s}{\apxD d}.
\end{definition}


\begin{definition}[Shortcut]
For a digraph $G=(V,E)$, any subset of edges $E' \subseteq E^T$ is called a \emph{\ss{}} of $G$. For $s, d \in \N$, we say $E'$ is a \emph{\ssss{s}{d}} of $G$ if $|E'| \le s$ and the diameter of $E\cup E'$ is at most $d$.
\end{definition}





The following problem is about \ss{} approximation.

\begin{definition}[\oss{\apxS}{\apxD}]
Given a directed graph $G$ and integers $d$ and $s$, such that $G$ admits an \ssss{s}{d}, the goal is to find an \ssss{\apxS s}{\apxD d}.
\end{definition}

When we write $\apxS,\apxD$ as a function of $n$, the variable $n$ corresponds to the number of nodes present in graph $G$.





\subsection{Reductions between Shortcut and \tc{}}\label{subsec:reductions}

In this section, we prove an approximate equivalence between the two problems. 

\begin{lemma}\label{lem:redTCtoSh}
	If there is a polynomial time algorithm solving \oss{\apxS}{\apxD}, then there is a polynomial time algorithm solving \TC{\apxS+1}{\apxD}.  
\end{lemma}


\begin{proof}
	Given a graph $G$ that admits a \TCs{s}{d}, we will show how to find a \TCs{(\apxS+1) s}{\apxD d}.
	
	According to Theorem 1~\cite{AhoGU72}, a unique subgraph $H$ exists such that deleting any edge of $H$ will make the transitive closure of $G^T$ and $H$ unequal. Therefore, we can find $H$ in polynomial time: start with $G^T$, and repeatedly delete edges if deleting an edge will not change the transitive closure until no edge can be deleted. Since $G$ admits a \TCs{s}{d}, and a \tc{} must have the same transitive closure as $G$, we have $|E(H)|\le s$. %
 
 After finding $H$, we use the \ssss{\apxS}{\apxD} algorithm with input $H$ and $s,d$ to find a \ssss{\apxS s}{\apxD d} $E'$. $H$ admits a \ssss{s}{d} because the \TCs{s}{d} of $G$ is also a \ssss{s}{d} of $H$. We claim that the graph $(V(H), E'\cup E(H))$ is a \TCs{(\apxS+1) s}{\apxD d}. According to the definition of shortcut, $(V(H), E'\cup E(H))$ has a diameter at most $\apxD d$. The size is at most $\apxS s+|E(H)|\le (\apxS+1) s$ since $|E(H)|\le s$.
	
\end{proof}


\begin{lemma}\label{lem:redShtoTC}
	If there is a polynomial time algorithm solving \TC{\apxS}{\apxD}, then there is a polynomial time algorithm solving \oss{2\apxS}{\apxD} with input $s$ restricted to $s\ge m$.
\end{lemma}
\begin{proof}
	Given a graph $G$ with $m$ edges and $s\ge m,d$ such that $G$ admits a \ssss{s}{d}, we will show how to find a \ssss{2\apxS s}{\apxD d}. 
 
        We first prove that $G$ admits a \TCs{2s}{d}. Let the \ssss{s}{d} of $G$ be $E'$, then $E'\cup E$ has size at most $2s$ since $s\ge m$, and has diameter $d$. Thus, $E'\cup E$ is a \TCs{2s}{d} of $G$. 
        
        We apply the \TC{\apxS}{\apxD} algorithm on $G$ with inputs $2s,d$ to get a \TCs{2\apxS s}{\apxD d}, which is also a \ssss{2\apxS s}{\apxD d}. 
 
	
\end{proof}

\subsection{Label cover and PGC.}\label{subsec:labcov}

In this section, we introduce the label cover problem and state the projection game conjecture formally. 


\begin{definition}[\labcov{}]\label{def:labcov}
	The \labcov{} problem takes as input an instance $\I=(A, B,E,\L,$ $(\pi_e)_{e\in E})$ described as follows:
	\begin{itemize}
		\item $(A \cup B, E)$ is a bipartite regular graph with partitions $|A|=|B|$, 
		\item a label set $\L$ and an non-empty acceptable label pair associated with each each $e=(u,v)\in E$ denoted by $\pi_e\subseteq \L\times \L$.
	\end{itemize}
	
	A \emph{labeling} $\psi: V \to \cL$ is a function that gives a label to each vertex $\psi(v)\in\L$. $\psi$ is said to cover an edge $e=(u,v)\in E$ if $(\psi(u),\psi(v))\in\pi_e$. The goal of the problem is to find a labeling that covers the most number of edges.
	
\end{definition}


\begin{conjecture}[PGC]\label{con:pgc}
	There exists a universal constant $\epsilon >0$ such that, given a \labcov{} instance on input of size $N$, it is hard to distinguish between the following two cases: 
	\begin{itemize}
		\item (Completeness:) There is a labeling that covers every edge. 
		
		\item (Soundness:) Any labeling covers at most $N^{-\epsilon}$ fraction of the edges. 
	\end{itemize}
\end{conjecture} 
We will use a slightly stronger hardness result where the soundness case is allowed to assign multiple labels. The hardness of this variant is a simple implication of the PGC.  A \mlab{} $\psi:V \to 2^\cL$ gives a set of labels  $\psi(v)\subseteq\L$ to a vertex $v$. Such a multilabeling  is said to cover an edge $e=(u,v)\in E$ if there exists $a\in\psi(u),b\in\psi(v)$ such that $(a,b)\in\pi_e$. 
The \textit{cost} of $\psi$ is denoted by $\sum_{u \in A \cup B} |\psi(u)|$. Notice that a valid labeling is a multi-labeling of cost $|A|+|B|$.

\begin{lemma}\label{lem:pgc}
	Assuming PGC, 
	there exists a sufficiently small constant $\epsl{} >0$ such that, given a \labcov{} instance of size $N$, there is no randomized polynomial time algorithm that distinguishes betweehn the following two cases: 
	\begin{itemize}
		\item (Completeness:) There is a labeling that covers every edge. 
		
		\item (Soundness:) Any multilabeling of cost at most $N^{\epsl{}}(|A|+|B|)$ covers at most $N^{-\epsl{}}$ fraction of edges. 
	\end{itemize}
\end{lemma}
\begin{proof}
    Suppose there is an algorithm that can distinguish the two cases described in~\cref{lem:pgc}, we will show that it also distinguishes the two cases in~\cref{con:pgc}. Completeness in~\cref{con:pgc} trivially implies completeness in~\cref{lem:pgc}, we only need to show that it also holds for soundness. In other words, we want to show that if there exists a \mlab{} $\phi$ of cost at most $N^{\epsl}(|A|+|B|)$ that covers more than $N^{-\epsl}$ fraction of edges for sufficiently small constant $\epsl$, then there exists a labeling $\phi'$ that covers more than $N^{-\epsilon}$ fraction of edges.

    To construct $\phi'$, we uniformly at random sample $1$ label from $\phi(v)$ for every $v\in A\cup B$ as the label $\phi'(v)$. For each edge $(u,v)$ covered by $\phi$, the probability that $(u,v)$ is covered by $\phi'$ is at least $1/(|\phi(u)|\cdot |\phi(v)|)$. Let $P$ contain all edges $(u,v)$ covered by $\phi$ such that $|\phi(u)|<N^{3\epsl}$ and $|\phi(v)|<N^{3\epsl}$. Since $\sum_{u\in A\cup B}|\phi(u)|\le N^{\epsl}(|A|+|B|)$, the number of nodes $u\in A\cup B$ that $|\phi(u)|\ge N^{3\epsl}$ is bounded by $(|A|+|B|)/N^{2\epsl}$. Remember that the bipartite graph is regular, so the number of edges $(u,v)$ with $|\phi(u)|\ge N^{3\epsl}$ or $|\phi(v)|\ge N^{3\epsl}$ is at most 
    \[\frac{2|E|}{|A|+|B|}\cdot \frac{|A|+|B|}{N^{2\epsl}}=\frac{2|E|}{N^{2\epsl}}\]

    Since the number of edges covered by $\phi$ is at least $N^{-\epsl}|E|$, we have
    \[|P|\ge \frac{|E|}{N^{\epsl}}-\frac{2|E|}{N^{2\epsl}}\ge \frac{|E|}{N^{2\epsl}}\]
    The expectation of the number of covered edges by $\phi'$ is at least
    \[\sum_{(u,v)\in P}\frac{1}{|\phi(u)|\cdot |\phi(v)|}\ge \frac{|E|}{N^{8\epsl}}\]

    By setting $\epsl<\epsilon/8$, the expectation is at least $\frac{|E|}{N^{\epsilon}}$, which means there must exists a labeling $\phi$ that covers more than $N^{-\epsilon}$ fraction of edges.
\end{proof}

\subsection{Main Results}\label{subsec:mainresults}

We restate the main results in the introduction in a formal way.

\begin{theorem}[Lower bound]\label{thm:main}
		Under \conj{}, there exists a small constant $\epsilon>0$, such that no polynomial-time algorithm exists for solving \oss{n^{\epsilon}}{n^{\epsilon}} as well as \TC{n^{\epsilon}}{n^{\epsilon}}, even if we restricted the input $s=\Omega(m^{1+\epsilon}),d=\Omega(n^{\epsilon})$.
\end{theorem}

Recall that the Las Vegas algorithm is one that either generates a correct output or, with probability at most $1/2$, outputs $\bot$ to indicate its failure. 

\begin{theorem}[Upper bound]\label{thm:upperbound}

  There is a polynomial time Las Vegas algorithm solving \oss{\apxS}{\apxD} whenever the inputs $G,s,d$ satisfy the following conditions.
  \begin{enumerate}
      \item $s\ge |V(G)|$  and %
      \item $\apxS\ge\frac{Cn\log^2 n}{\sqrt{sd^2\apxD^3}}$ for sufficiently large constant $C$.
  \end{enumerate}

Moreover, if we further require $\apxS\ge 2$, then there is a polynomial time  Las Vegas algorithm solving \TC{\apxS}{\apxD}.  
\end{theorem}

Notice that the upper bound result is stated here in a slightly different form, as this form will be more convenient to derive formally. It is easy to show that Theorem~\ref{thm:upperbound} implies $(n^{\gamma_D}, n^{\gamma_S})$-approximation for $3\gamma_D + 2\gamma_S >1$ (using the fact that $s \geq n$ and $d \geq 1$.)   
