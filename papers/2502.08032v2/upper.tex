
\section{Upper Bounds}\label{sec:upperbound}
We use the algorithm idea of a previous spanner approximation algorithm~\cite{BermanBMRY13}. We explain our algorithm in the language of the shortcut problem. 

\subsection{Preliminaries}



\paragraph{Large diameter case:} If we aim for a relatively large diameter, i.e., $d \cdot \alpha_D \geq n^{0.34}$, we can simply use the known result. 

\begin{theorem}[\cite{KoganP22}]
There is an efficient algorithm that, given input graph $G$, computes a $(d,s)$-shortcut for $d = \tilde{O}(n^{1/3})$ and $s = \tilde{O}(n)$.    
\end{theorem}

Using the above theorem, if $\apxD{}d=\Omega(n^{0.34})$, we can easily achieve $(\alpha_D, \alpha_S)$-approximation for all $\alpha_S = \tilde{O}(1)$. 
Therefore, we make the following assumption throughout this section. 

\begin{remark}\label{rem:assumption}
We can assume w.l.o.g. that $\alpha_D d = O(n^{0.34})$. 
\end{remark}

\paragraph{Reduction to DAGs:} We argue that DAGs, in some sense, capture hard instances for our problems. This will allow us to focus on DAGs in the subsequent sections.
The formal statement is encapsulated in the following lemma. 

\begin{lemma}
If there exists an efficient $(\alpha_D,\alpha_S)$ approximation algorithm for DAGs, then there exists an efficient $(3\alpha_D, 3\alpha_S)$-approximation algorithm for all directed graphs.  
\end{lemma}
\begin{proof}
Assume that we are given an access to the algorithm $\aset(G,s,d)$ that produces $(\alpha_D,\alpha_S)$ approximation algorithm for the DAG case. 
Let $G$ be an input digraph, together with the input parameters $(d,s)$.  
Compute a collection ${\mathcal S}$ of strongly connected components (SCC) of $G$, and let $G'$ be the DAG obtained by contracting each SCC into a single node. Invoke the algorithm $\aset(G',s,d)$. Let $E' \subseteq E(G')$ be the shortcut edges so that $|E'| \leq \alpha_S \cdot s$ and the diameter of $G' \cup E'$ is at most $\alpha_D \cdot d$. 
These edges would be responsible for connecting the pairs whose endpoints are in distinct SCCs. 

For each SCC $C \in {\mathcal S}$, we pick an arbitrary ``center'' $u_C \in V(C)$ and connect it to every other vertex in $V(C)$. These edges are called $E''_C$. Define $E'' = \bigcup_{C \in {\mathcal S}} E''_C$. 
The final shortcut $F$ can be constructed by combining these two sets $E'$ and $E''$: Edges in $E''$ can be added into $F$ directly. For each edge that connects $C$ to $C'$ in $E'$, we create the corresponding edge $(u_C, u_C')$ connecting the centers. Notice that $|F| \leq |E'| + |E''| \leq \alpha_S \cdot s + 2n \leq 3\alpha_S \cdot s$ (here we used the assumption that $s \geq n$). Moreover, it is easy to verify that, for each reachable pair $(v,w)$ in $G$ where $v \in C$ and $w \in C'$, there is a path from $v$ to $w$ of length at most $3 \alpha_D \cdot d$.     
\end{proof}




\subsection{Overview} \label{sec:ub-overview}

Suppose $G=(V,E)$ is a directed acyclic graph and $G^T=(V,E^T)$ be its transitive closure, i.e., $(u,v)\in E^T$ if $u$ has a directed path with length at least $1$ to $v$ in $G$. Since $G$ is acyclic, $G^T$ contains no self loop. 

\begin{definition}[Local graphs]\label{def:localgraph}
For a pair $u,v \in V(G)$, we define the local graph $G^{u,v}$. 
 Let $G^{u,v}=(V^{u,v},E^{u,v})$ be the subgraph of $G^T$ induced by the vertices that can reach $v$ and can be reached from $u$ (i.e., these vertices lie on at least one path from $u$ to $v$). 
\end{definition}
\begin{definition}[Thick and thin pairs]\label{def:thickthinedge}
 Let $u,v \in V(G)$. 
 If $|V^{u,v}|\ge \beta$ ($1\le \beta\le n$ will be determined later), the corresponding edge $(s,t)$ is said to be $\beta$-thick, and otherwise, it is $\beta$-thin. When $\beta$ is clear from context, we will simply write thick and thin pairs respectively. 
\end{definition}

Denote by $\pset$ the set of pairs of vertices that are reachable in $G$. We can partition $\pset$ into $\pset_{thick} \cup \pset_{thin}$. 

\begin{definition}
	Let $d' \in {\mathbb N}$. A set $E'\subseteq E^T \setminus E$ is said to $d'$-settle a pair $(u,v)\in E$ if $(V,E \cup E')$ contains a path of length at most $d'$ from $u$ to $v$. 
\end{definition}


Our algorithm will find two edge sets $F_1,F_2\subseteq E^T \setminus E$ such that the set $F_1$ is responsbile for $(\apxD{} d)$-settling all thick pairs, while $F_2$ will $(\apxD{} d)$-settle all thin pairs. The final solution be $F_1\cup F_2$ (Notice that $F_1\cup F_2$ $(\apxD{} d)$-settles all the pairs.)  These are encapsulated in the following two lemmas: 

\begin{restatable}[thick pairs]{lemma}{settlethick}\label{lem:settlethickedges}
We can efficiently compute $F_1$ such that $|F_1| \le O\left(\frac{n^2\log^2 n}{\beta\apxD{}^2 d^2}+
n \log n
\right)$ and it $(\apxD{} d)$-settles all thick pairs w.h.p. 
\end{restatable}

\begin{lemma}[thin pairs] \label{lem:settlethinpairs}
We can efficiently compute $F_2$ such that $|F_2| \leq O\left(\frac{\beta \log^2 n s}{\alpha_D}\right)$ and $(\alpha_D d)$-settles all thin pairs with high probability.      
\end{lemma}

We will prove these two lemmas later. Meanwhile, we complete the proof of Theorem~\ref{thm:upperbound}. 
We minimize the sum $|F_1| + |F_2|$ by setting the value of $\beta=\frac{n}{d\sqrt{s\apxD{}}}$.\footnote{The only problem is that this value could be much less than $1$ when $d\sqrt{s\apxD{}}>n$, which leads to $(d\apxD{})^2s>n^2$. However, in that case,  we can use the known tradeoff~\cite{KoganP22} to construct a \ssss{s}{\apxD{}d} when $d\sqrt{s\apxD{}}>n$.} 
	Now we have 
 \[|F_1\cup F_2|=O\left(\frac{ns^{0.5}\log^{2}n}{d\apxD{}^{1.5}}+n\log n\right)\le s\cdot O\left(\frac{n\log^{2}n}{d\apxD{}^{1.5}s^{0.5}}\right).\]

\subsection{Settling the thick pairs}\label{subsec:thickedges}

In this Section, we prove Lemma~\ref{lem:settlethickedges}.  For convenience we assume $\apxD{}d=\omega(\log n)$. We will show the case when $\apxD{}d=O(\log n)$ later.
We will use the idea from~\cite{KoganP22} to construct $F_1$. First, the following lemma allows us to decompose a DAG into a collection of paths and independent sets. 

\begin{lemma}[Theorem 3.2 \cite{GrandoniILPU21}]
	There is a polynomial time algorithm given an $n$-vertex acyclic graph $G=(V,E)$ and an integer $k \in[1,n]$, partition $G$ into $k$ directed paths  $P_1,...,P_{k}$ and at most $2n/k$ independent set  $Q_1,...,Q_{2n/k}$ in $G^T$. In other words, $P_1,...,P_{k},Q_1,...,Q_{2n/k}$ are disjoint and $\left(\cup_{i\in[k]}P_i\right)\cup \left(\cup_{i\in[2n/k]}Q_i\right)=V$.
\end{lemma}

We first apply the following lemma with $k=n/(\apxD{} d)$ to get $P_1,...,P_{8n/(\apxD{} d)}$ and $Q_1,...,Q_{\apxD{} d/4}$. Notice that $\apxD d=\omega(\log n)$ and $\apxD d= O(n^{0.34})$, we can safely assume $8n/(\apxD{} d)$ and $\apxD{} d/4$ are integers without loss of generality.
We will use Lemma 1.1~\cite{Raskhodnikova10}.
\begin{lemma}[Lemma 1.1~\cite{Raskhodnikova10}]\label{lem:pathshortcut}
    For any integer $n\ge 3$, the directed path with length $\ell$ has a $2$-shortcut with at most $\ell\log \ell$ edges.
\end{lemma}

We first add $n\log n$ edges to $F_1$ and reduce the diameter of every path $P_i$ to $2$. To accomplish this, for each path, we use \cref{lem:pathshortcut}. Since different paths are disjoint regarding vertices, $n\log n$ edges suffice.


Next, let $R \subseteq V$ be obtained by sampling $\min\left((9\log n)\cdot n/\beta, n\right)$ vertices from $V$ uniformly at random. The algorithm also samples $\min\left((999\log n)\cdot n/(\apxD{} d)^2,n/(\apxD{} d)\right)$ paths from $P_1,...,$ $P_{n/(\apxD{} d)}$ uniformly at random; let ${\mathcal Q}$ denote the set of sampled paths. For each vertex $u\in R$ and path $p\in {\mathcal Q}$, add $(u,v_1)$ to $F_1$ where $v_1$ is the first node in $p$ that $u$ can reach (if it exists), and add $(v_2,u)$ to $F_1$ where $v_2$ is the last node in $p$ that can reach $u$ (if it exists). Observe that $\beta$ is between $1$ and $n$, $(\apxD d)$ is between $\omega(\log n)$ and $O(n^{0.34})$, so both $(9\log n)\cdot n/\beta$ and $(999\log n)\cdot n/(\apxD{} d)^2$ will not be too small, and we can safely assume they are integers without loss of generality.




\settlethick*
\begin{proof}
It is straightforward to verify that $|F_1| \le (999n^2\log^2 n)/(\beta\apxD{}^2 d^2)+n\log n$ by the construction.

Suppose $(s,t)\in E^T$ is a thick pair. Since $|V^{s,t}|\ge \beta$, a vertex $u\in R\cap V^{s,t}$ exists with high probability (w.h.p.). %
Let $G'$ be the graph obtained after adding all edges that reduce the diameter for each $P_i$. Denote the shortest path between $s$ and $t$ in $G'$ as $P_{s,t}$. 
The path $P_{s,t}$ intersects each path $P_i$ at no more than three nodes. Otherwise, there are four nodes in $P_i\cap P_{s,t}$ $v_1,v_2,v_3,v_4$ such that $v_1$ has distance at least $3$ to $v_4$ since $P_{s,t}$ is a shortest path, which contradict the fact that $P_i$ has diameter $2$. The path $P_{s,t}$ intersects each $Q_i$ at no more than one node since $Q_i$ is an independent set on $E^T$.
Therefore, if we disregard all nodes in $Q_i$ from $i=1$to $i=\apxD{} d/ 4$ (which incurs an additional $\apxD{} d/4$ steps) and examine the first $\apxD{} d/4$ vertices and the last $\apxD{} d/4$ vertices of $P_{s,t}$, both of them will intersect a path in ${\mathcal Q}$ w.h.p. since we sample $(999\log n)\cdot n/(\apxD{} d)^2$ paths into ${\mathcal Q}$ among all $8n/(\apxD d)$ paths. This implies that $s$ can first use a $\apxD{} d/4+\apxD{} d/9+1$ path to reach $u$, and then $u$ can use another $1+\apxD{} d/9+\apxD{} d/4$ path to reach $t$. 
\end{proof}

For the case when $\apxD{}d=O(\log n),\apxD{}d>1$, to get similar result as~\cref{lem:settlethickedges}, we want $|F_1|=O(n^2\log n/\beta+n)$. This is easy to construct by sampling $(n/\beta)\log n$ nodes, where w.h.p. one of them will be in $V^{s,t}$ for every thick pair $s,t$. For every sample node $u$, we just need to include $(v,u)$ to $F_1$ for every $v$ that can reach $u$, and $(u,v)$ to $F_1$ for every $v$ that $u$ can reach. Now every thick pair has distance $2$ in $F_1$. 

For the extreme case when $\apxD{}d=1$, their is a unique way of adding edges, which is connecting every reachable pair by one edge, thus, we ignore this situation.

\subsection{Settling the thin pairs}\label{subsec:thinedges}



\begin{definition}[Critical sets]\label{def:antispanner}
A set $A\subseteq E^T$ is a $k$-critical set of a pair $(u,v)\in E^T$ if $E^T\backslash A$ contains no path from $uv$ to $v$ with a length of at most $k$. If there does not exist an $A'\subset A$ such that $A'$ is also a $k$-critical, then we say $A$ is minimal.
\end{definition}


\begin{definition}[$\mathcal{A}_k$]\label{def:antispannerset}
Let $\mathcal{A}_k$ consist of all sets $A$ satisfying both (i) $A$ is a minimal critical set of some thin pair, and (ii) $A\cap E=\emptyset$.
\end{definition}

This gives an alternate characterization of a shortcut set. 

\begin{claim}[Adapted from \cite{BermanBMRY13}] 
    An edge set $E'$ is a $d$-shortcut set for all thin pairs if and only if $E' \cap A \neq \emptyset$  for all $A\in\mathcal{A}_{d}$. 
\end{claim}

\begin{proof}
We briefly sketch the proof of this claim below.
\begin{itemize}
    \item[($\Rightarrow$)] We first argue that $E'$ must intersect each $A\in\mathcal{A}_d$ with at least one edge. Suppose, on the contrary, that $E'$ is disjoint from a set $A\in\mathcal{A}_d$, which is a $d$-critical set of some pair $(u,v)$. Then, $E'\cup E\subseteq E^T \backslash A$, and the distance between $u$ and $v$ in $E'\cup E$ is larger than $d$, which contradicts our assumption. 
    
    \item[($\Leftarrow$)] On the other hand, assume that $E'$ intersects every $A\in\mathcal{A}_d$ with at least one edge then it must be a $d$-shortcut. Otherwise, there exists a pair $(u,v)\in E^T $ such that, in $E'\cup E$, $u$ has a distance larger than $d$ to $v$. This implies that $A= E^T\backslash(E'\cup E)$ is a $d$-critical set of $u,v$ and $A \cap E' = \emptyset$ (a contradiction). 
\end{itemize}
\end{proof}


We define the following polytope $\cP_{G,d}$. It includes at most $n^2$ variables: for each $e\in E^T$, there is a corresponding variable $x_e$. The polytope has an exponentially large number of constraints. However, it must be non-empty since we assume $G$ admits a \ssss{s}{d}.


\begin{align*}
	\text{Polytope }\cP_{G,d}: \qquad\sum_{e\in E^T\backslash E}x_e &\le s \\
	\sum_{e\in A}x_e &\ge 1\qquad \forall A\in \mathcal{A}_d\\
	x_e &\geq 0\qquad \forall e\in E^T
\end{align*}

The algorithm employs the cutting-plane method to attempt to find a point within the polytope. According to established analyses, the running time of the cutting-plane method is polynomial with respect to the number of variables, provided that a separation oracle with a polynomial running time exists based on the number of variables. A separation oracle either asserts that the point lies within the polytope or returns a violated constraint, which can function as a cutting plane. Since the constraints $\sum_{e\in E^T\backslash E}x_e\le s$ and $x_e\ge 0$ can be verified in polynomial time, identifying a cutting plane is straightforward if either of these constraints is violated. Consequently, we assume $\sum_{e\in E^T\backslash E}x_e\le s$ and $x_e\ge 0$.

In the subsequent algorithm, we accept a point as input and either return a non-trivial violated constraint (one of $\sum_{e\in A}x_e\ge 1$), which can act as a separation plane, or output a set $E_2$ that settles all thin pairs.




\begin{algorithm}[H]
	
	\caption{{\sc Cut-or-Round}$({\bf x})$}\label{alg:seperationoracle}
	
	\KwData{A vector ${\bf x} \in [0,1]^{E^T \setminus E}$ satisfying $\sum_{e\in E^T\backslash E}x_e\le s$ and $x_e\ge 0$.}
	\KwResult{A set $A \in  \mathcal{A}_d$ s.t. $\sum_{e\in A}x_e<1$, or a set $F_2\subset E^T \setminus E$ that $(\apxD{} d)$-settles all thin pairs.}
	Include  edge edge $e\in E^T$ independently into $F_2$ with probability $(500\log n)(\beta/\apxD{})x_e$\;
	
	\If{all thin pairs are $(\apxD{} d)$-settled}{\If{$|F_2\backslash E|\le (1000\log^2 n)(\beta/\apxD{})s$}{\Return{$F_2$}\;}\Else{\Return{fail}\;}}
	\Else{
		Use $F_2$ to find a $(\apxD{} d)$-critical set $A'\in\mathcal{A}_{\apxD{} d}$ with $F_2\cap A'=\emptyset$ using the algorithm from Claim 2.4 in~\cite{BermanBMRY13}\;
		\If{$\sum_{e\in A'}x_e<\apxD{}/9$}{
			Find a $d$-critical set $A\subseteq \mathcal{A}_d$  such that $\sum_{e\in A}x_e\ge 1$ using~\cref{lem:findantispanner}\;
			\Return violated constraint $\sum_{e\in A}x_e\ge 1$\;}
		\Else{\Return{fail}\;}
	}
\end{algorithm}

\begin{claim}\label{lem:notfail}
	The algorithm will fail with probability at most $\frac{1}{n^{\omega(1)}}$.
\end{claim}
\begin{proof}
	The first fail condition is $|F_2\backslash E|>(1000\log^2 n)(\beta/\apxD{})s$. Notice that each edge $e$ is included in $F_2$ independently with probability $(500\log n)(\beta/\apxD{})x_e$ where $\sum_{e\in E^T\backslash E}x_e\le s$. By chernoff bound, $|F_2\backslash E|>(1000\log^2 n)(\beta/\apxD{}) s$ happens with probability at most $e^{-(\log n)(\beta/\apxD{}) s}$. Remember that $s\ge n$ and $\apxD{}=O(n^{0.34})$, which means we have $e^{-(\log n)(\beta/\apxD{}) s}\le n^{-\omega(1)}$.
	
	Now we show that the second fail condition happens with small probability. We define event $\mathcal{E}$ as ``all $B\subseteq A_{\apxD{} d}$ with $\sum_{e\in B}x_e\ge \apxD{}/9$ satisfies $B\cap F_2\not=\emptyset$''. If the second fail is triggered, then there exists a set $A'\in\mathcal{A}_{\apxD{} d}$ with $\sum_{e\in A'}x_e\ge \apxD{}/9$ satisfies $A'\cap F_2=\emptyset$, which means $\mathcal{E}$ does not happen. Thus, the probability that the second fail is triggered is bounded by $1-\Pr[\mathcal{E}]$. For $B\subseteq A_{\apxD{} d}$ with $\sum_{e\in B}x_e\ge \apxD{}/9$, the probability that $B\cap F_2=\emptyset$ is bounded by $\exp\left(-9\beta\log n\right)$ according to Chernoff bound (remember that each edges is included in $F_2$ with probability $(500\log n)(\beta/\apxD{})x_e$). Now we count the size of $\mathcal{A}_{\apxD{} d}$. According to claim 2.5 in~\cite{BermanBMRY13}, we have $|\mathcal{A}_{\apxD{} d}|\le |E|\cdot \beta^\beta\le \exp\left(2\beta\log\beta\right)$. Finally, by using union bound, $\Pr[\mathcal{E}]\ge 1-\frac{1}{n^{\omega(1)}}$.
\end{proof}
\begin{lemma}[critical set decomposition]\label{lem:findantispanner}
	There exists a polynomial time algorithm that, given $A'\in\mathcal{A}_{\apxD{} d}$ with $\sum_{e\in A'}x_e< \apxD{}/9$, outputs $A\in\mathcal{A}_{d}$ such that $\sum_{e\in A}x_e< 1$. 
\end{lemma}
\begin{proof}
	The algorithm first use polynomial time to find $(s,t)\in E^T$ such that $A'$ is a $(\apxD{} d)$-critical set of $(s,t)$. Then the algorithm constructs a shortest path tree rooted at $s$ on the subgraph $E^T\backslash A'$, where all the nodes in the $i$-th laryer of the tree has distance $i$ from $s$. Denote the vertex set of the $i$-th layer as $L_i$. According to the definition of critical set, $t$ is on at least the $(\apxD{} d+1)$-th layer. The algorithm devides the first $\apxD{} d$ layers into at least $\apxD{}/3$ batches, where the $i$-th batch contains all vertices between layer $2(i-1)d$ to $2i\cdot d-1$. Denote the set of all the edges in $A'$ that has at least one end point in the $i$-th batch as $A_i$. Since each edge in $A'$ will be included in at most two $A_i$, it is easy to see that at least one of $A_i$ has the property $\sum_{e\in A_i}x_e<1$. 
	
	Now we prove that for any $i$, $A_i\in\mathcal{A}_d$. Since $A_i$ is a subset of $A'$, the second condition, i.e., $A_i\cap E=\emptyset$ is satisfied trivially. We only need to verify that $A_i$ is a $d$-critical set of some edge in $E^T$. The idea is to choose a vertex in $L_{2(i-1)d}$ and another vertex in $L_{2id-1}$, and argue that they are reachable and have distance more than $d$ in $E^T\backslash A_i$. Let $S$ contain all the vertices in $L_{2(i-1)d}$ that has at least one edge towards a vertex in $L_{2id-1}$. Notice that $G$ (and also $G^T$) is acyclic (\cref{rem:assumption}), which also means the induced subgraph $G^T[S]$ is acyclic. Thus, there exists a vertex $u\in S$ such that it has no edge in $E^T$ towards other vertices in $S$. Take an arbitrary vertex $v\in L_{2id-1}$ such that $(u,v)\in E^T$, now we argue that $u$ has distance more than $d$ to $v$ in $E^T\backslash A_i$.  
	
	We prove it by induction. Induction hypothesis: any vertex that $u$ has distance $x$ to in $E^T\backslash A_i$ is one of the following two types (i) a vertex that cannot reach $v$ in $G$ (ii) a vertex in $L_{2(i-1)d+x}$. If the induction hypothesis is correct for any $0\le x\le d+10$, then $u$ cannot have distance at most $d$ to $v$ in $E^T\backslash A_i$. Now we prove the induction hypothesis. When $x=0$, the hypothesis is correct. For $x>0$, suppose $w_2$ is a vertex with $\distt{E^T\backslash A_i}{u,w_2}=x$, then there exists an edge $(w_1,w_2)\in E^T\backslash A_i$ such that $\distt{E^T\backslash A_i}{u,w_1}=x-1$. According to induction hypothesis, we can assume $w_1$ is either type (i) or type (ii). If $w_1$ is type (i), then $w_2$ cannot reach $v$ in $G$, which means $w_2$ is also type (i). Now suppose $w_1$ is type (ii) and can reach $v$ in $G$. First of all, $w_2$ cannot be in the first $2(i-1)d$-th layer, otherwise $w_2$ has a path $p$ in $G$ to $v$ where $p$ must contain a vertex in $L_{2(i-1)d}$ (two consecutive vertices in $p$ cannot skip a layer since all edges in $p$ is in $E$, which is also in $E^T\backslash A'$), which means $u$ has an edge in $E^T$ to another vertex in $L_{2(i-1)d}$, leading to a contradiction. Then, $w_2$ cannot be a vertex in $L_{2(i-1)d+b}$ where $1\le b<x$ since $u$ has distance less than $x$ to them according to induction hypothesis. $w_2$ cannot be a vertex in $L_{2(i-1)d+b}$ where $b>x$, otherwise $(w_1,w_2)$ is not in $A_i$, also not in $A'$, which contradicts the fact that layers are constructed by shortest path tree on $E^T\backslash A'$. $w_2$ cannot be a vertex outside the tree because $s$ can reach $w_2$ in $E^T\backslash A'$. Finially, $w_2\in L_{2(i-1)d+x}$ and the induction hypothesis holds. 
\end{proof}

As mentioned in \Cref{sec:ub-overview}, Theorem~\ref{thm:upperbound} now follows combining the size of sets $F_1$ and $F_2$ from \Cref{lem:settlethickedges} and \Cref{lem:settlethinpairs} respectively.


 
