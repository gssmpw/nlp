\documentclass[11pt,letterpaper]{article}


\usepackage{hyperref}
\hypersetup{breaklinks, urlcolor=blue, colorlinks, citecolor=green!50!black, linkcolor=blue}
\usepackage[letterpaper, left=1in, right=1in, top=0.9in, bottom=0.9in]{geometry}
\usepackage[utf8]{inputenc}
\usepackage[american]{babel}
\usepackage[normalem]{ulem}
\usepackage{amsmath, amssymb, cases, amsthm}
\usepackage{thmtools}
\usepackage[shortlabels]{enumitem}
\usepackage{mdframed}
\usepackage{bbm}
\usepackage{bm}
\usepackage{microtype}
\usepackage{xcolor}
\usepackage{makecell}
\usepackage{mathtools}
\usepackage{float}
\usepackage{modletters}
\usepackage{comment}
\usepackage{multirow}
\usepackage[numbers]{natbib}
\usepackage[capitalize,noabbrev]{cleveref}
\usepackage{graphics}

\usepackage{footnotehyper}
\makesavenoteenv{tabular}

\usepackage{pifont}
\newcommand{\cmark}{\ding{51}}%
\newcommand{\xmark}{\ding{55}}%

\declaretheorem[numberwithin=section,refname={Theorem,Theorems},Refname={Theorem,Theorems}]{theorem}
\declaretheorem[numberwithin=section,refname={Theorem,Theorems},Refname={Theorem,Theorems}]{thm}
\declaretheorem[numberlike=theorem]{lemma}
\declaretheorem[numberlike=theorem]{proposition}
\declaretheorem[numberlike=theorem]{corollary}
\declaretheorem[numberlike=theorem,style=definition]{definition}
\declaretheorem[numberlike=theorem]{claim}
\declaretheorem[numberlike=theorem,style=remark]{remark}
\declaretheorem[numberlike=theorem,refname={Fact,Facts},Refname={Fact,Facts},name={Fact}]{fact}
\declaretheorem[numberlike=theorem, refname={Question,Questions},Refname={Question,Questions},name={Question}]{question}
\declaretheorem[numberlike=theorem, refname={Problem,Problems},Refname={Problem,Problems},name={Problem}]{problem}
\declaretheorem[numberlike=theorem, refname={Observation,Observations},Refname={Observation,Observations},name={Observation}]{observation}
\declaretheorem[numberlike=theorem, refname={Experiment,Experiments},Refname={Experiment,Experiments},name={Experiment}]{experiment}
\declaretheorem[numberlike=theorem,refname={Primitive,Primitives},Refname={Primitive,Primitives},name={Primitive}]{primitive}
\declaretheorem[numberlike=theorem,refname={Technique,Techniques},Refname={Technique,Techniques},name={Technique}]{technique}
\declaretheorem[numberlike=theorem,refname={Conjecture,Conjectures},Refname={Conjecture,Conjectures},name={Conjecture}]{conjecture}

\def\cameraready{1}  % set this to 1 to get the camera-ready version
\def\final{1}  % set this to 1 to get a comment-free version
\def\iflong{\iffalse}
\ifnum\final=0  %namely if we allow comments in the output
\newcommand{\parinya}[1]{{\color{magenta}[{\tiny Parinya: \bf #1}]\marginpar{\color{magenta}*}}}
\newcommand{\yonggang}[1]{{\color{blue}[{\tiny Yonggang: \bf #1}]\marginpar{*}}}
\newcommand{\danupon}[1]{{\color{red}[{\tiny Danupon: \bf #1}]\marginpar{\color{red}*}}}

\newcommand{\sagnik}[1]{{\color{green!50!black}[{\tiny Sagnik: \bf #1}]\marginpar{\color{green!50!black}*}}}
\newcommand{\todo}[1]{{\color{red}[{\tiny TODO: \bf #1}]\marginpar{\color{red}*}}}
\newcommand{\yuval}[1]{{\bf \color{red!50!black} YUVAL: #1}}
\newcommand{\jan}[1]{{\bf \color{green!50!black} Jan: #1}}
\newcommand{\blikstad}[1]{\textup{\color{magenta} [\textbf{Joakim}: #1]}}
\newcommand{\tawei}[1]{{\color{blue} [{\bf Ta-Wei:} #1]}}
\newcommand{\TODO}[1]{{\color{blue!50!black} [{\bf Todo:} #1]}}
\else % in this case [final=1] we don't want any comments to show
\newcommand{\yonggang}[1]{}
\newcommand{\danupon}[1]{}
\newcommand{\sagnik}[1]{}
\newcommand{\todo}[1]{}
\newcommand{\yuval}[1]{}
\newcommand{\jan}[1]{}
\newcommand{\blikstad}[1]{}
\newcommand{\tawei}[1]{}
\newcommand{\TODO}[1]{}
\fi  %ok, we're done with defining comment macros
\usepackage[ruled,vlined,linesnumbered]{algorithm2e}

\bibliographystyle{alpha}

\DeclarePairedDelimiter{\ceil}{\lceil}{\rceil}
\newcommand{\eps}{\varepsilon}

\newcommand{\mindeg}{\mathrm{mindeg}}
\newcommand{\dist}{\mathrm{dist}}
\newcommand{\nbh}{\mathrm{nbh}}
\newcommand{\spf}{\mathrm{spf}}
\newcommand{\direct}{MDCP }
\newcommand{\best}{\mathsf{best}}
\newcommand{\polylog}{\mathrm{polylog}}
\newcommand{\poly}{\mathrm{poly}}
\newcommand{\outdeg}{\mathrm{outdeg}}
\newcommand{\ones}{\mathrm{ones}}
\newcommand{\euler}{\mathrm{e}}
\newcommand{\var}{\mathrm{Var}}
\newcommand{\set}[2][ ]{\{#2 \ifthenelse{\equal{#1}{ }}{ }{~|~#1}\}}
\newcommand{\fvedge}{\mathsf{FindViolatingEdge}}
\newcommand{\XORQ}{\mathsf{XOR}}
\newcommand{\ANDQ}{\mathsf{AND}}

\newcommand{\cc}{\mathrm{cc}}
\newcommand{\Rcal}{\mathcal{R}}
\newcommand{\Pcal}{\mathcal{P}}
\newcommand{\Acal}{\mathcal{A}}
\newcommand{\Scal}{\mathcal{S}}
\newcommand{\Qcal}{\mathcal{Q}}
\newcommand{\pflag}{\mathrm{pflag}}
\newcommand{\R}{\mathbb{R}}
\newcommand{\N}{\mathbb{N}}
\newcommand{\Var}{\mathrm{Var}}
\newcommand{\one}{\mathbf{1}}
\newcommand{\inact}{\mathsf{Inactive}}
\newcommand{\act}{\mathsf{Active}}
\newcommand{\cut}{\mathrm{cut}}
\newcommand{\cutd}{\overrightarrow{\mathrm{cut}}}
\newcommand{\Ed}{\overrightarrow{E}}
\newcommand{\mincut}{\lambda}
\newcommand{\tOh}{\widetilde{O}}
\newcommand{\ith}{i^{\scriptsize \mbox{{\rm th}}}}
\newcommand{\jth}{j^{\scriptsize \mbox{{\rm th}}}}

%Yonggang's micro
\usepackage{caption2}
\usepackage{empheq}
\renewcommand{\O}[1]{O\left(#1\right)}
\newcommand{\The}[1]{\tilde{\Theta}\left(#1\right)}
\newcommand{\pset}{\mathcal{P}}
\newcommand{\aset}{\mathcal{A}}

\renewcommand{\L}{\mathcal{L}}
\newcommand{\I}{\mathcal{I}}
\renewcommand{\ss}{shortcut}
\newcommand{\ssss}[2]{$(#2,#1)$-shortcut}
\newcommand{\rd}{reduced diameter}
\newcommand{\labcov}{{\sf LabelCover}}
\newcommand{\labcovp}[1]{#1-LabelCover}
\newcommand{\mlab}{multilabeling}
\newcommand{\ba}[1]{#1-bi-criteria approximates}
\newcommand{\os}{{\sf MinShC}}
\newcommand{\oss}[2]{$(#2,#1)$-\os{}}
\newcommand{\tc}{TC spanner}
\newcommand{\TC}[2]{$(#2,#1)$-{\sf MinTC}}
\newcommand{\TCs}[2]{$(#2,#1)$-\tc{}}
\newcommand{\conj}{PGC}
\newcommand{\distt}[2]{\text{dist}_{#1}\left(#2\right)}
\newcommand{\opt}[1]{\dshortcut(#1)}
\newcommand{\optd}[1]{\sshortcut(#1)}
\newcommand{\AB}{\Delta}
\newcommand{\gadget}[2]{$#2$-{\sf MinStShC}$\mid_{#1}$}
\newcommand{\ga}{{\sf MinStShC}}
\newcommand{\lan}{{c_{lrs}}}
\newcommand{\bufs}{{s'}}
\newcommand{\bufN}{{N}}
\newcommand{\minrep}[1]{#1-LabelCover}
\newcommand{\minre}{LabelCover}
\newcommand{\as}[2]{#1-bi-criteria approximating #2-shortcut}
\newcommand{\fan}{copied graph part}
\newcommand{\blue}{\labcov{} part}
\newcommand{\orange}{star part}
\newcommand{\red}{shortcutting part}
\newcommand{\epsl}{\epsilon_{L}}
\renewcommand{\a}[1]{a^{(#1)}}
\renewcommand{\aa}[2]{a^{(#1)}_{#2}}
\newcommand{\aaa}[3]{\alpha^{(#1)}_{#2}[#3]}
\renewcommand{\b}[1]{b^{(#1)}}
\newcommand{\bb}[2]{b^{(#1)}_{#2}}
\newcommand{\bbb}[3]{\beta^{(#1)}_{#2}[#3]}
\newcommand{\dia}{\mathcal{\rho}}
\newcommand{\cs}{c_M}
\newcommand{\idx}{I}
\newcommand{\Idx}[1]{\idx{}[#1]}
\newcommand{\change}[2]{{\color{red}#1}{\color{blue}#2}}
%Yonggang's micro ends here

\newcommand{\intpot}[1]{Int_{pot}(#1)}


\newcommand{\ip}[1]{\left\langle #1\right\rangle }
\newcommand{\expt}[2]{\underset{#1}{\mathbb{E}}\left[#2\right]}
\newcommand{\A}{{\mathcal{A}}}
\newcommand{\BPM}{\mathsf{BPM}}
\newcommand{\BMM}{\mathsf{BMM}}
\newcommand{\UBPM}{\mathsf{UBPM}}

\newcommand{\T}{{\mathcal{T}}}

\newcommand{\U}{{\mathcal{U}}}

\newcommand{\bcc}[1]{\textbf{BCAST($#1$)}}
\newcommand{\dw}[1]{#1^{\downarrow}}
\SetKwComment{Comment}{/* }{ */}

\newcommand{\MM}{\mathcal{P}}
\newcommand{\ORQ}{\mathsf{OR}}
\newcommand{\ISQ}{\mathsf{IS}}
\newcommand{\vol}{\mathrm{vol}}
\newcommand{\volumelb}{\left(\tfrac{1}{20n}\right)^{2n}}

\newcommand{\rank}{\mathsf{rank}}
\newcommand{\sspan}{\mathsf{span}}



%% === Danupon's definitions ===

\newcommand{\stcspanner}{S^*_{TC}}
\newcommand{\sshortcut}{S^*_{Sh}}
\newcommand{\sboth}{S^*}
\newcommand{\dtcspanner}{D^*_{TC}}
\newcommand{\dshortcut}{D^*_{Sh}}
\newcommand{\dboth}{D^*}
\newcommand{\apxS}{\alpha_S}
\newcommand{\apxD}{\alpha_D}

%% === END: Danupon's definitions ===

\Crefname{algocf}{Algorithm}{Algorithms}


\usepackage[most]{tcolorbox}
\newcounter{myexample}
\usepackage{xparse}
\usepackage{lipsum}

\makeatletter
\newcommand\footnoteref[1]{\protected@xdef\@thefnmark{\ref{#1}}\@footnotemark}
\makeatother


\renewcommand{\paragraph}[1]{\medskip\noindent{\bf #1}\xspace}






\newcommand{\ground}{U}





\title{Shortcuts and Transitive-Closure Spanners Approximation}


\author{
Parinya Chalermsook \thanks{University of Sheffield, \texttt{chalermsook@gmail.com}}\and
Yonggang Jiang\thanks{MPI-INF, Germany, \texttt{yjiang@mpi-inf.mpg.de}} \and 
Sagnik Mukhopadhyay\thanks{University of Birmingham \texttt{s.mukhopadhyay@bham.ac.uk}} \and
Danupon Nanongkai\thanks{MPI-INF, Germany, \texttt{danupon@gmail.com}}
}
\date{}





\begin{document}
	
	\begin{titlepage}
		\maketitle \pagenumbering{roman}
		
		\begin{abstract}
Retrieval-Augmented Generation (RAG) is often used with Large Language Models (LLMs) to infuse domain knowledge or user-specific information. In RAG, given a user query, a retriever extracts chunks of relevant text from a knowledge base. These chunks are sent to an LLM as part of the input prompt. Typically, any given chunk is repeatedly retrieved across user questions. However, currently, for every question, attention-layers in LLMs fully compute the key values (KVs) repeatedly for the input chunks, as state-of-the-art methods cannot reuse KV-caches when chunks appear at arbitrary locations with arbitrary contexts. Naive reuse leads to output quality degradation.  This leads to potentially redundant computations on expensive GPUs and increases latency. In this work, we propose \sys, a system for managing and reusing precomputed KVs corresponding to the text chunks (we call \textit{chunk-caches}) in RAG-based systems. We present how to identify \hl{\textit{chunk-caches} that are reusable}, how to efficiently perform a small fraction of recomputation to \textit{fix} the cache to maintain output quality, and how to efficiently store and evict \textit{chunk-caches} in the hardware for maximizing reuse while masking any overheads. With real production workloads as well as synthetic datasets, we show that \sys reduces redundant computation by \textbf{51\%} over SOTA prefix-caching and \textbf{75\%} over full recomputation.
\hl{Additionally, with continuous batching on a real production workload, we get a \textbf{1.6$\times$} speedup in throughput and a \textbf{2$\times$} reduction in end-to-end response latency over prefix-caching while maintaining quality, for both the \llama-3-8B and \llama-3-70B models. 
}
\end{abstract}





		
		\setcounter{tocdepth}{3}
		\newpage
		\tableofcontents
		\newpage
	\end{titlepage}
	
	\newpage
	\pagenumbering{arabic}



\documentclass[../main.tex]{subfiles}
\graphicspath{{../images/}}
\makeatletter
\def\input@path{{../images/}}
\makeatother
\begin{document}
\section{Introduction}
\begin{figure}
\centering
\begin{tikzpicture}
\node[inner sep=0pt] (ws) at (0, 0) {
\includegraphics[height=.4\textwidth, trim={10cm 0 10cm 0},clip]{world_space.png}};
\node[inner sep=0pt] (cs) at (6,0) {\includegraphics[height=.4\textwidth, trim={10cm 1cm 10cm 4cm},clip]{conf_space.png}};
\end{tikzpicture}
\vspace{-5pt}
\label{fig:pbrm_intro}
\caption{\textbf{Left}: Shows world space obstacles as grey spheres. Robots start and goal configuration is colored red and green, respectively. Configurations along the computed path are colored transparent blue. \textbf{Right:} Mapped world space scenario to configuration space. Obstacle region is the grey mesh. Red spheres are collision-free regions computed by the neural SCDF. The optimized shortest path in the convex corridor is the blue curve.}
\vspace{-25pt}
\end{figure}
Motion planning is the problem of finding a collision-free trajectory that connects a given start and goal configuration. The planning takes place in the configuration space of the robot. For single body robots, like mobile robots or drones, the configuration space and the world space are usually the same. This simplifies the planning, since explicit obstacle representations are available which enables geometrical tools like separating hyperplanes, smallest distance to obstacles etc., to be used when designing motion planning algorithms. For multi-body robots like manipulators, the situation is completely different. The world space obstacles are usually mapped to non-convex regions, and to make the problem even harder, the mapping is usually not known. Forming explicit representations of the obstacle region in the configuration space is usually too expensive or intractable. Despite all of this, sampling based planners are used with great success, which mainly is due to their use of implicit representations of the obstacle region. The basic idea is to construct a graph in the configuration space that covers and connects the collision-free region. From this graph, a path can be extracted that connects a given start and goal configuration. The approach is computationally expensive, since the graph is constructed with the smallest geometrical building block available, points, which represents a collision-check. Furthermore, the extracted paths from the graph are non-smooth and jagged due to the stochastic nature of the approach. This adds an additional post-processing step to the process, where the paths are shortcutted and smoothened, before the path can be used for tracking. Clearly a lot of time is invested to form this graph and produce smooth paths. Thus, if the obstacles start to move, then all of this work is done in no use, since all points that make up this graph need to be re-verified, which is simply too time consuming to be done in real time.
\\\\
In this work, we want to address the existing drawbacks of the sampling based planners. Our main contribution is an improved motion planner where each vertex in the graph covers a collision-free region in the form of a sphere instead of a point and where the edges are formed with neighboring intersecting spheres. This representation has the advantage of instead of returning piecewise linear paths, returning a sequence of overlapping spheres, i.e. a convex corridor, that connects a given start and goal configuration, illustrated in Figure \ref{fig:pbrm_intro}. This convex corridor allows us to use convex optimization to produce smooth trajectories, instead of computationally expensive post-processing methods. The representation further allows us to estimate the coverage of the collision-free space, which gives us awareness and feedback in the offline roadmap construction phase. Finally, our representation is simple to adapt to moving obstacles, simply requery for the new radii and recheck for intersections. 
\\\\
The spherical collision-free regions are formed using a signed distance function (SDF), which is a function that returns the smallest distance from an arbitrary point to the boundary of an obstacle. As the name implies, the distance is signed, thus if the point is inside the obstacle it is negative otherwise positive. If the distance is positive, a sphere with radius equal to the distance is guaranteed to cover a collision-free region. Using an SDF in motion planning is not new, but what is novel about our approach is that we express the distance in the configuration space instead of the world space and by doing so allows us to form these convex collision-free regions. We refer to the resulting SDF as a signed configuration distance function (SCDF). Computing an SCDF analytically is non-trivial, our approach is therefore to parameterize the SCDF with a deep neural network and learn the mapping by supervised learning. Our resulting neural SCDF can compute distances for different parameter values of obstacle shapes and we also show how multiple distances can be combined, thus making our approach flexible.
\section{Related work}
Motion planning algorithms can roughly be divided into three families, grid-based, sampling based and optimization based methods. Grid-based methods (GBM) discretize the planning space from which a graph is then compiled. A standard search method is A$^\star$ \citep{a_star}, which is classified as an \textit{informed} search method, since it employs a heuristic function to speed up the search. A$^\star$ guarantees to return an optimal path at the level of discretization used. GBMs usually discretize the planning space by a regular lattice and this limits the GBMs to problems with low dimensionality due to the curse of dimensionality. Thus, GBMs are usually limited to single-body robots where the degrees of freedom (DOF) are low. To overcome the inherent scaling problem with the GBMs, stochastic methods are usually used for multi-body robots. These methods are termed as sampling-based methods (SBM) and core members within this family are the rapidly-exploring random trees (RRT) \citep{rrt} and the probabilistic roadmap (PRM) \citep{prm}. RRT grows a tree from the start configuration and explores the collision-free region in a rapid way until it is able to connect to the goal region. RRT is usually improved by bi-directional planning \citep{rrt_connect}, i.e. an additional tree is grown from the goal configuration and the trees are tested for connection after any tree has been expanded. RRT is a single-query method, thus it searches for a path from scratch each time it is queried. Contrary to this, PRM is a multi-query method, which solves for multiple queries without starting from scratch. PRM does this by creating a roadmap (graph) that covers the collision-free space as an offline step. The graph is then used to solve for multiple queries. PRMs are used in cases where the environment does not change since the extra offline step is too computationally costly and needs to be re-done if the environment is changed. In our work, we address this inherent issue by using a different roadmap representation. Our vertices in the graph cover a collision-free region in the form of spheres and we form the edges by checking for intersecting spheres. If something in the environment changes, we recompute the spheres radii and recheck the intersections, without relying on collision detection. We use a trained neural network to compute the sphere radius, therefore querying for the radius can be done fast, hence our representation enables the PRM for dynamic environments.
\\\\
In the recent decades, optimization based methods (OBM) \citep{chomp, schulman, itomp, stomp} have been introduced as an alternative to SBM for multi-body robots. Like the SBM, the OBMs scale well to higher dimensional problems and produce smoother motion. It is common to use a SDF in the optimization since it is a smooth function, thus enabling gradient-based methods. However, the standard way of expressing the SDF is in world space. The distance therefore needs to be mapped to the configuration space by the forward kinematics. This mapping makes the optimization problem a non-linear program (NLP), which is computationally expensive to solve. Recently, a different approach has been proposed. In \cite{mp_gcs} motion planning is formulated as a convex optimization problem by using the graph of convex sets framework \citep{gcs}. The underlying idea is to decompose the collision-free space into intersecting convex sets from which a convex optimization problem is formulated. In cases where an explicit representation of the obstacles in the configuration space exists, like for single-body robots, creating collision-free convex regions can be done fast \citep{iris}. For multi-body robots, this is non-trivial. Existing work does this successfully \citep{iris_nlp, iris_c} by an optimization based approach, but the methods are still too time consuming to be used in the presence of moving obstacles. Our approach is instead to use deep learning to learn an SDF expressed in the configuration space. With this, we can query for shortest distances to the collision boundary, which allows us to expand spherical regions which are collision-free. Our approach is fast and therefore enables our suggested roadmap planner to be used in dynamic environments.
\\\\
Recent research has focused on learning collision detection \citep{fk_kernel_distance, diffco, graphdistnet} by predicting the signed distance between the robot links and the surrounding obstacles in the world space. The learned SDF is used in trajectory optimization but since the distance is expressed in the world space, the problem becomes an NLP and therefore takes a long time to solve. We take a novel approach and suggest to instead express the signed distance in the configuration space. This allows us to improve the PRM at the same time as it enables convex optimization for trajectory optimization, which runs faster and is more reliable than NLP solvers. In \cite{cspf} a learned signed distance function in the configuration space is proposed similar to our approach. However, their approach is restricted to point cloud representations, while we propose to represent the obstacles as parameterized geometric shapes, e.g. spheres. Furthermore, we also show how to use our learned SCDF to improve an existing roadmap planner.
\section{Problem formulation}
A robot is located in the world space, $\W \subset \R^3 $. The unique location of the robot is given by its configuration $\q \in \C$, where $\C$ is the configuration space. The set of points covered by the robots bodies at a certain configuration is expressed as $\B(\q) \subset \W$. The robot is surrounded by $\NrObst$ obstacles $\O = \bigcup_{i=1}^{\NrObst} \O_i$, where  $\O_i \subset \W$. The representation of the obstacle in the configuration space is the set $\C\O_i = \{\q \in \C \: |\: \B(\q) \cap \O_i \neq \emptyset \}$. The obstacle space is formed as $\Co = \bigcup_{i=1}^{\NrObst} \C \O_i$. The complement is referred to as the free space, $\Cf = \C \setminus \Co$. The path planning problem is a tuple, ($\Cf$, $\qStart$, $\qGoal$), where we want to connect a query pair, consisting of a start, $\qStart$, and goal configuration, $\qGoal$, with a geometric path, $\q(s): [0, 1] \mapsto \Cf$, such that $\q(0)=\qStart$ and $\q(1)=\qGoal$, or report correctly when such a path does not exist.
\end{document}


\begin{figure*}[t]
\begin{center}
\includegraphics[width=.85\linewidth]{fig_overview_v3.pdf}
\end{center}
\caption{
FastAtlas Overview: In each frame, we compute charts spanning fully or partially visible triangles (a), determine texture space bounding boxes for the visible portions of the view-space projections of each chart, and tightly pack these boxes into atlases (b, here $2K \times 2K$). We simultaneously bijectively parameterize and shade the charts into their atlas boxes, obtaining high quality texture space shading (c), and use this shading to render the shaded frames (d).}
\label{fig:overview}
\label{fig:alg_overview}
\end{figure*}

\section{Overview}
\label{sec:overview}
Our work has two core contributions: a real-time, GPU-based algorithm for tight packing of general parameterized charts into compact atlases; and a real-time TSS method that
utilizes this packing.  

\paragraph*{FastAtlas Packing.}
FastAtlas runs entirely on the GPU as a series of compute shaders. It takes the bounding boxes of parameterized charts as input, and packs them into an atlas (Fig~\ref{fig:overview}b, Sec.~\ref{sec:pack}). As such, the only input it requires are the dimensions of the bounding boxes.
Its outputs are deterministic; identical input charts are packed into identical atlases. This is critical for TSS and similar applications, as it ensures that consecutive frames taken from the same camera view have the same shading. Even minute shading differences across such frames can cause sampling jitter, leading to undesirable flicker \cite{baker2012rock}. 
While prior methods such as \cite{mueller2018shading,hladky2019tessellated,hladky2021snakebinning,Neff2022MSA} cap the dimensions of the charts that can be packed as-is for a given atlas size, and scale down all charts that exceed these dimensions, we scale all charts by the same factor, and do so only when strictly necessary to achieve packing success (Figs~\ref{fig:atlas},~\ref{fig:sas_issues}). 

\paragraph*{TSS using FastAtlas.}
Our end-to-end TSS atlas generation method combines the packing method above with a novel approach for computing seamless per-frame charts. 
We define our charts as the connected components of the visible surfaces in each frame (Fig.~\ref{fig:overview}a), and efficiently compute them using a parallel union-find algorithm (Sec.~\ref{sec:visible}). Since the boundaries of these charts coincide with the contours of the rendered surface, they are {\em invisible} to the viewer. This approach 
eliminates the artifacts caused by shading discontinuities along visible seams (Fig.~\ref{fig:seams}). 

\begin{parWithWrapFigure}
\begin{wrapfigure}{l}{.27\columnwidth}%
\includegraphics[width=\linewidth]{fig_inset_view_plane.pdf}%
\end{wrapfigure}
We bijectively parametrize the {\em visible portions} of our charts by projecting them to view space (inset). This maps a constant number of texels to each pixel in the final rendered output, evenly distributing residual undersampling error across all image pixels. While conceptually straightforward, efficiently parameterizing charts containing partially visible triangles using viewspace projection is non-trivial, as the visible portions may no longer be triangular (e.g. green triangle in the inset); applying naive projection to triangles with vertices behind the camera may produce ill-posed results. Clipping triangles before projection is both computationally expensive and significantly complicates downstream operations. We avoid explicit clipping by observing that all that is required for atlas packing is the dimensions of, potentially conservative, bounding boxes of these projected visible portions. We compute such bounding boxes without explicit chart clipping by adapting a conservative screen coverage estimator \shortcite{Blinn:CalculatingScreenCoverage} (Sec.~\ref{sec:box}). We then pack the computed boxes using FastAtlas. 
\end{parWithWrapFigure}

Finally, we shade the visible portion of each chart into its corresponding atlas bounding box (Fig~\ref{fig:overview}c). 
The resulting texture is then used during rasterization as a standard texture map (Fig. ~\ref{fig:overview}d). 
Our framework is compatible with all existing approaches for texture space shading, including forward shading, raytraced illumination, or deferred shading in texture space \cite{baker:2016}. In the examples shown, we use the standard forward shading based rendering pipeline included in the G3D Innovation Engine \cite{G3D17}, a commercial grade renderer.


The full proofs are deferred to the appendix.

\appendix


\section{Preliminaries}\label{sec:preliminaries}



%We denote by $(\Ac(x_\Ac),\Bc(x_\Bc))(z)$ a random execution of $\pi$ with private inputs $(x_\Ac,y_\Ac)$, and common input $z$.

%\Jnote{Move to DP}
% At the end of such an execution, the protocol outputs a public transcript denoted by the random variable $\trans_\pi(x_\Ac,x_\Ac,z)$ we denotes the common as $\out(\trans_\pi(x_\Ac,x_\Ac,z)$, and each party $\Pc \in \set{\Ac,\Bc}$ obtains his view denoted $\view^\Pc_\pi(x_\Ac,x_\Bc,z)$, which may also contain a ``local output'' \Jnote{Local} $\out^\Pc(x_\Ac,x_\Bc,z)$ (if the protocol specifies such an output). \Jnote{Common output, and parties output}


\subsection{Distributions and Random Variables}\label{sec:prelim:dist}
The support of a distribution $P$ over a finite set $\cS$ is defined by $\Supp(P) \eqdef \set{x\in \cS: P(x)>0}$. For a distribution or a random variable $D$, let $d\from D$ denote that $d$ was sampled according to $D$. Similarly,  for a set $\cS$, let $x \from \cS$ denote that $x$ is drawn uniformly from $\cS$, and denote by $\cU_{\cS}$ the uniform distribution over $\cS$. For a finite set $\cX$ and a distribution $C_X$ over $\cX$, we use the capital letter $X$ to denote the random variable that takes values in $\cX$ and is sampled according to $C_X$. The {\sf statistical distance} (\aka {\sf~variation distance}) of two distributions $P$ and $Q$ over a discrete domain $\cX$ is defined by $\sdist{P}{Q} \eqdef \max_{\cS\subseteq \cX} \size{P(\cS)-Q(\cS)} = \frac{1}{2} \sum_{x \in \cS}\size{P(x)-Q(x)}$. 
For a vector $x = (x_1,\ldots,x_n)$ and index $i\in [n]$, we let $x_{-i} = (x_1,\ldots,x_{i-1},x_{i+1},\ldots,x_n)$ and $x^{(i)} = (x_1,\ldots,x_{i-1}, -x_i, x_{i+1},\ldots,x_n)$, for a set $\cS \subseteq [n]$ we let $x_{\cS} = (x_i)_{i \in \cS}$ and $x_{-\cS} = (x_i)_{i \in [n]\setminus \cS}$, and for a vector $r \in \zo^n$ we let $x_r = (x_i)_{\set{i \colon r_i = 1}}$ and $x_{-r} = (x_i)_{\set{i \colon r_i = 0}}$.

%For $n \in \N$ we let $U_n$ be the uniform distribution over $\oo^n$, and let $S_n$ be the distribution induces by the sum of $n$ i.i.d.\ random variables, each is distributed according to $U_1$. Let $\cN(0,1)$ be the standard normal distribution.
%For a distribution $\cD$ and a function $f$, we define by $f(\cD)$ the distribution that is induced by the output of $f(x)$ for $x \from \cD$. 





% \begin{theorem}[\cite{McGregorMPRTV10}]\label{thm:sv-extracotr}
% 	\Enote{Remove if not needed}
% 	There is a constant $c$ to make the following holds. Let $X$ be an $\alpha$-SV source on $\{0,1\}^n$, let $Y$ be a source on $\{0,1\}^n$ with min-entropy at least $\beta n$ (independent from $X$), and let $Z=\ip{X,Y}\mbox{mod m}$ for some $m\in\mathbb{N}$. Then for every $\delta\in[0,1]$, the random variable $(Y,Z)$ is $\delta$-close to $(Y,U)$ where $U$ is uniform on $\mathbb{Z}_m$ and independent of $Y$, provided that
% 	$$
% 	n\geq c\cdot\frac{m^2}{\alpha\beta}\cdot\log(\frac{m}{\beta})\cdot\log(\frac{m}{\delta}).
% 	$$
% \end{theorem}



\Enote{I removed the definition of DP since it already appears in the intro}
\remove{
\subsection{Differential Privacy}\label{sec:prelim:DP}
We use the following standard definition of (information theoretic) differential privacy, due to \citet{DMNS06}. For notational convenience, we focus on databases over $\oo$.
\begin{definition}[Differentially private mechanisms]\label{def:mech}
	A randomized function $f\colon\oo^n\mapsto \zs$ is an {\sf $n$-size, $(\eps,\delta)$-differentially private mechanism} (denoted $(\eps,\delta)$-\DP) if for every neighboring $w,w'\in \oo^n$ and every function $g\colon \zs\mapsto \zo$, it holds that 
	$$
	\pr{g(f(w))=1}\leq e^{\eps}\cdot \pr{g(f(w'))=1} +\delta.
	$$ 	
	If $\delta=0$, we omit it from the notation.
\end{definition}
}


\subsubsection{Computational Differential Privacy}
There are several ways for defining computational differential privacy (see \cref{sec:related-works}). We use the most relaxed version due to \cite{BNO08}. For notational convenience, we focus on databases over $\oo$.
\begin{definition}[Computational differentially private mechanisms]\label{def:ComMech}
	A randomized function ensemble $f=\set{f_\pk\colon\oo^{n(\pk)}\mapsto \zs}$ is an {\sf $n$-size, $(\eps,\delta)$-computationally differentially private} (denoted $(\eps,\delta)$-$\CDP$) if for every poly-size circuit family $\set{\Ac_\pk}_{\pk\in \N}$, the following holds for every large enough $\pk$ and every neighboring $w,w'\in\oo^{n(\pk)}$:
	$$
	\pr{\Ac_\pk(f_\pk(w))=1}\leq e^{\eps(\pk)}\cdot \pr{\Ac_\pk(f_\pk(w'))=1} +\delta(\pk).
	$$ 
	If $\delta(\pk) = \negl(\pk)$, we omit it from the notation. 
\end{definition}



\subsubsection{Two-Party Differential Privacy}\label{sec:DP}
In this section we formally define distributed differential privacy mechanism (\ie protocols). %For the ease of notation, we consider protocol with no common input.

\begin{definition}\label{def:DP}%\Nnote{fix security parameter}
	A two-party protocol $\Pi=(\Ac,\Bc)$ is {\sf $(\eps,\delta)$-differentially private}, denoted $(\eps,\delta)$-$\DP$, if the following holds for every algorithm $\Dc$: let $\V^\Pc(x,y)(\pk)$ be the view of party $\Pc$ in a random execution of $\Pi(x,y)(1^\pk)$. Then for every $\pk,n \in \N$, $x\in \oo^n$ and neighboring $y,y'\in\oo^n$:
	\begin{align*}
	\pr{\Dc(V^\Ac(x,y)(\pk))=1}\le e^{\eps(\pk)}\cdot \pr{\Dc(V^\Ac (x,y')(\pk))=1}+\delta(\pk),
	\end{align*} 
	and for every $y\in \oo^n$ and neighboring $x,x'\in\oo^{n}$:
	\begin{align*}
	\pr{\Dc(V^\Bc(x,y)(\pk))=1}\le e^{\eps(\pk)}\cdot \pr{\Dc(V^\Bc (x',y)(\pk))=1}+\delta(\pk).
	\end{align*} 	
	Protocol $\Pi$ is {\sf $(\eps,\delta)$-computational differentially private}, denoted $(\eps,\delta)$-$\CDP$, if the above inequalities only hold for a non-uniform \ppt $\Dc$ and large enough $\pk$. We omit $\delta = \negl(\pk)$ from the notation. \footnote{Note that define we give for two-party differentially private protocols is a semi-honest definition, in which we ask for the security to hold when the parties interact in an honest execution of the protocol. Since we are proving a lower bound, starting from this weaker guarantee (as opposed to security against malicious players), yields a stronger result.}
\end{definition}
%We omit $\delta$ from the notation if $\delta$ is a negligible function of $n$.

%\Enote{simulation-based}
\begin{remark}[The definition for computational differential privacy we use]\label{rem:comDPChannel} 
	An alternative, stronger definition of computational differential privacy, known as simulation-based computational differential privacy, requires that the distribution of each party’s view be computationally indistinguishable from a distribution that ensures privacy in an information-theoretic sense. \cref{def:DP} is a weaker notion in comparison. Consequently, establishing a lower bound for a protocol that satisfies this weaker guarantee (as we do in this work) yields a stronger result.%Actually, our lower bound only requires the privacy to hold against \emph{uniform} external observer.
	%\Nnote{Maybe add: When only interesting in \Dp against external observer, the two definitions can be achieve using key-agreement and (single-party) \Dp mechanism. }
\end{remark}




\subsection{Useful Claims}
\remove{
In this section, we state generic lemmas and propositions that we will use later in our proofs.

The following lemma which we prove in \cref{sec:missing-proofs:distance-I}, measures the distance between two uniform stings conditioned one a random index $i$ either being fixed to $0$ or to $1$.

\def\distanceILemma{
    Let $R \la \zo^n$. For any (randomized) function $f:\{0,1\}^n\rightarrow \{0,1\}$ and $\alpha > 0$, it holds that
    \begin{align}\label{eq:f-alpha}
        \ppr{i \la [n]}{\size{\:\ex{f(R) \mid R_i = 0}-\ex{f(R) \mid R_i = 1}\:}\geq \alpha} \leq \frac{2}{n \alpha^2},
    \end{align}
    where the expectations are taken over $R$ and the randomness of $f$.
}

\begin{lemma}\label{lem:distance-I}
    \distanceILemma
\end{lemma}
}

The following two propositions state that given the output of a differentially private function, it is not possible to predict well even a random index (even if all other indexes are leaked). The first proposition handles the information-theoretic case and the second handles the computation case. Both propositions are proven in \cref{sec:missing-proofs:hard-to-guess}. 

\def\propHardToGuessInf{
    Let $f\colon \oo^n \rightarrow \cY$ be an $(\eps,\delta)$-\DP function, let $g \colon [n] \times \oo^{n-1} \times \cY \rightarrow \set{-1,1,\bot}$ be a (randomized) function, and let $X = (X_1,\ldots,X_n) \la \oo^n$. Then the following holds for every $i \in [n]$ where $X_i^* = g(i,X_{-i},f(X_1,\ldots,X_n))$:
    \begin{align*}
        \pr{X_i^* = X_i} \leq e^{\eps}\cdot \pr{X_i^* = -X_i} + \delta.
    \end{align*}
}

\begin{proposition}\label{prop:hard-to-guess-inf}
    \propHardToGuessInf
\end{proposition}


\def\propHardToGuessComp{
    Let $f = \set{f_{\pk} \colon \oo^{n(\pk)} \rightarrow \zo^{m(\pk)}}_{\pk \in \bbN}$ be an $(\eps,\delta)$-\CDP function ensemble, and let $\set{g_{\pk}}_{\pk \in \bbN}$ be a poly-size circuit family. Then, for large enough $\pk$ and $X = (X_1,\ldots,X_{n(\pk)}) \la \oo^{n(\pk)}$, the following holds for every $i \in [n(\pk)]$ where $X_i^* = g_{\pk}(i,X_{-i},f_{\pk}(X_1,\ldots,X_n))$:
    \begin{align*}
        \pr{X_i^* = X_i} \leq e^{\eps(\pk)}\cdot \pr{X_i^* = -X_i} + \delta(\pk).
    \end{align*}
}

\begin{proposition}\label{prop:hard-to-guess-comp}
    \propHardToGuessComp
\end{proposition}





\remove{
\Enote{Chao's old statement:}
\begin{lemma}\label{lem:distance-I-old}
        Let $R \la \zo^n$. 
	For any function $f:\{0,1\}^n\rightarrow \{0,1\}$ and $\alpha<0.01$, it holds that
	$$
	\Pr_{i\la[n]}\left[\: \size{\:\mathbb{E}[f(R) \mid R_i = 0]-\mathbb{E}[f(R) \mid R_i = 1]\:}\geq \alpha\right]\leq \frac{2+2\log(\frac{1}{\alpha})}{n\alpha^2}.
	$$
\end{lemma}
\begin{proof}
	Define $S_1=\{r \in \zo^n \colon f(r)=1\}$. Then for any $i\in[n]$, we have
	$$
	\begin{array}{rl}
		\size{\mathbb{E}[f(R) \mid R_i = 0]-\mathbb{E}[f(R) \mid R_i = 1]}
		&=\size{\Pr[R\in S_1|R_i=0]-\Pr[R\in S_1|R_i=1]}\\
		&=\size{\frac{\Pr[R_i=0|R\in S_1]\cdot\Pr[R\in S_1]}{\Pr[R_i=0]}-\frac{\Pr[R_i=1|R\in S_1]\cdot\Pr[R\in S_1]}{\Pr[R_i=1]}}\\
		&=\frac{2\size{S_1}}{2^n}\size{\Pr[R_i=0|R\in S_1]-\Pr[R_i=1|R\in S_1]}
	\end{array}
	$$
	When $|S_1|\leq \alpha\cdot 2^{n-1}$, we have $\size{\mathbb{E}[f(R) \mid R_i = 0]-\mathbb{E}[f(R) \mid R_i = 1]}\leq\frac{2\size{S_1}}{2^n}\leq \alpha$ for any $i\in[n]$. Hence, in the following, we assume $|S_1|> \alpha\cdot 2^{n-1}$.

	%Define $I_{bad}=\{i|\size{\Pr[R_i=0|R\in S_1]-\Pr[R_i=1|R\in S_1]}>2\alpha\}$ and $k=\size{I_{bad}}$, then for any $i\notin I_{bad}$, we have 
    %$$
    %\begin{array}{rl}
    %    2\alpha&\geq \size{\Pr[R_i=0|R\in S_1]-\Pr[R_i=1|R\in S_1]}\\
    %    &=\size{\frac{\Pr[R\in S_1|R_i=0]\cdot\Pr[R_i=0]}{\Pr[R\in S_1]}-\frac{\Pr[R\in S_1|R_i=1]\cdot\Pr[R_i=1]}{\Pr[R\in S_1]}}\\
    %    &=\size{\Pr[R\in S_1|R_i=0]-\Pr[R\in S_1|R_i=1]}\cdot\frac{1}{2\Pr[R\in S_1]}\\
    %    &\geq \size{\mathbb{E}[f(R) \mid R_i = 0]-\mathbb{E}[f(R) \mid R_i = 1]}\cdot \frac{1}{2},
    %\end{array}
    %$$ 
    %where the last inequality is because $\Pr[R\in S_1]\leq 1$. So that $\size{\mathbb{E}}[f(R) \mid R_i = 0]-\mathbb{E}[f(R) \mid R_i = 1]\leq %4\alpha$.
    Define $I_{bad}=\{i \colon \size{\Pr[R_i=0|R\in S_1]-\Pr[R_i=1|R\in S_1]} \geq 2\alpha\}$ and $k=\size{I_{bad}}$, and denote $I_{bad}=\{i_1,\dots,i_k\}$. Define $(X_{i_1}, \ldots X_{i_k}) = (R_{i_1},\dots,R_{i_k})\mid_{R \in S_1}$. 
    Consider the min-entropy
	$$
	\begin{array}{rl}
		H_{min}(X_{i_1},\dots,X_{i_k})&\leq H(X_{i_1},\dots,X_{i_k})\\
		&\leq \sum_{j=1}^k H(X_{i_j})\\
		&\leq k\cdot \left(-(\frac{1}{2}+2\alpha)\cdot\log(\frac{1}{2}+2\alpha)-(\frac{1}{2}-2\alpha)\cdot\log(\frac{1}{2}-2\alpha)\right)\\
            &=k\cdot \left(-(\frac{1}{2}+2\alpha)\cdot(\log(1+4\alpha)-1)-(\frac{1}{2}-2\alpha)\cdot(\log(1-4\alpha)-1)\right)\\
            &=k\cdot \left(1-(\frac{1}{2}+2\alpha)\cdot\log(1+4\alpha)-(\frac{1}{2}-2\alpha)\cdot\log(1-4\alpha)\right),
		
	\end{array}
	$$
	where $H_{min}(Y)$ is the minimum entropy of $Y$ and $H(Y)$ is the Shannon entropy of $Y$.\Enote{add to preliminaries.}
        The third inequality holds since by the definition of $I_{bad}$, for every $j \in [k]$ it holds that $\size{\pr{X_{i_j} = 1}-\pr{X_{i_j} = 0}} > 2\alpha$, and therefore $H(X_{i_j}) \leq H(1/2 + 2\alpha)$\Enote{define}.
	
	Therefore, there exists $b_1,\dots,b_k\in\{0,1\}$, such that 
	
	\begin{align}\label{eq:min-entropy-result}
		\Pr\left[(R_{i_1},\ldots,R_{i_k}) = (b_1,\ldots,b_k) \mid R\in S_1\right]
		&= \pr{(X_{i_1},\ldots,X_{i_k}) = (b_1,\ldots,b_k)}\\
		&= 2^{-H_{min}(X_{i_1},\dots,X_{i_k})}\nonumber\\
		&\geq 2^{k\cdot \left(-1+(\frac{1}{2}+2\alpha)\cdot\log(1+4\alpha)+(\frac{1}{2}-2\alpha)\cdot\log(1-4\alpha)\right)}.\nonumber
	\end{align}
	
	Let $S_{bad}=\{r \in \zo^n  \colon \set{(r_{i_1},\ldots,r_{i_k}) = (b_1,\ldots,b_k)} \land \set{r\in S_1}\}$.
	It holds that
	\begin{align*}
		|S_{bad}|
		&= \size{S_1} \cdot \Pr\left[(R_{i_1},\ldots,R_{i_k}) = (b_1,\ldots,b_k) \mid R\in S_1\right]\\
		&\geq \alpha\cdot 2^{n-1}\cdot2^{k\cdot \left(-1+(\frac{1}{2}+2\alpha)\cdot\log(1+4\alpha)+(\frac{1}{2}-2\alpha)\cdot\log(1-4\alpha)\right)},
	\end{align*} 
	where the inequality holds by \cref{eq:min-entropy-result} and since $\size{S_1} \geq \alpha\cdot 2^{n-1}$.
	Notice that any string in $S_{bad}$ depends on at most $n-k$ bits. It implies that $|S_{bad}|\leq 2^{n-k}$. Therefore, we have
	$$
	\begin{array}{rl}
		&2^{n-k}\geq \alpha\cdot 2^{n-1}\cdot2^{k\cdot \left(-1+(\frac{1}{2}+2\alpha)\cdot\log(1+4\alpha)+(\frac{1}{2}-2\alpha)\cdot\log(1-4\alpha)\right)} \\
		\Rightarrow& n-k \geq \log \alpha+n-1+k\cdot \left(-1+(\frac{1}{2}+2\alpha)\cdot\log(1+4\alpha)+(\frac{1}{2}-2\alpha)\cdot\log(1-4\alpha)\right)\\
		\Rightarrow& 1-\log \alpha \geq k\cdot((\frac{1}{2}+2\alpha)\cdot\log(1+4\alpha)+(\frac{1}{2}-2\alpha)\cdot\log(1-4\alpha))\\
		\Rightarrow& 1-\log \alpha \geq k\cdot(4\alpha\cdot\log(1+4\alpha)+(\frac{1}{2}-2\alpha)\cdot\log(1-16\alpha^2))\\
        \Rightarrow& 1-\log\alpha \geq k\cdot(15.9\alpha^2-8\alpha^2+32\alpha^3)=k\cdot(7.9\alpha^2+32\alpha^3)>0.5k\alpha^2\\
		\Rightarrow& k\leq \frac{2-2\log \alpha}{\alpha^2} = \frac{2+2\log (1/\alpha)}{\alpha^2},
	\end{array}
	$$
	Where the third transition holds since 
	\begin{align*}
		\lefteqn{(\frac{1}{2}+2\alpha)\cdot\log(1+4\alpha)+(\frac{1}{2}-2\alpha)\cdot\log(1-4\alpha)}\\
		&= 4\alpha\cdot\log(1+4\alpha) + (\frac{1}{2}-2\alpha)\paren{\log(1+4\alpha)+\log(1-4\alpha)}\\
		&= 4\alpha\cdot\log(1+4\alpha)+(\frac{1}{2}-2\alpha)\cdot\log(1-16\alpha^2),
	\end{align*}
	and the forth transition holds since $4\alpha\cdot\log(1+4\alpha)+(\frac{1}{2}-2\alpha)\cdot\log(1-16\alpha^2) > 15.9\alpha^2-8\alpha^2+32\alpha^3$ for $\alpha < 0.01$.
	Thus, we conclude that 
	$$
	\Pr_{i\la[n]}\left[\size{\mathbb{E}[f(R) \mid R_i=0]-\mathbb{E}[f(R) \mid R_i = 1]}\geq \alpha\right]\leq \frac{k}{n}\leq \frac{2+2\log (1/\alpha)}{n\alpha^2}.
	$$
\end{proof}
}


\subsection{Channels and Two-Party Protocols}\label{sec:protocol}

\paragraph{Channels.}A channel is simply a distribution of a pair of tuples defined as follows. 
\begin{definition}[Channels]\label{def:channel} A {\sf channel} $C_{(X,U)(Y,V)}$ of size $\isize$ over alphabet $\Sigma$ is a probability distribution over $(\Sigma^\isize \times\zo^\ast) \times(\Sigma^\isize \times\zo^\ast)$. The ensemble $C_{(X,U)(Y,V)}= \set{C_{(X_\pk,U_\pk)(Y_\pk,V_\pk)}}_{\pk\in \N}$ is an $\isize$-size channel ensemble, if for every $\pk\in \N$, $C_{(X_\pk,U_\pk)(Y_\pk,V_\pk)}$ is an $\isize(\pk)$-size channel. %We denote a channel of size one by a \emph{single-bit} channel. 
We refer to $X$ and $Y$ as the {\sf local outputs}, and to $U$ and $V$ as the {\sf views}.	
\end{definition}

We view a  channel as the experiment in which there are two parties $\Ac$ and $\Bc$.  Party $\Ac$ receives ``output'' $X$ and ``view'' $U$, and party $\Bc$ receives ``output'' $Y$ and ``view'' $V$. Unless stated otherwise, the channels we consider are over the alphabet $\Sigma = \oo$. We naturally identify channels with the distribution that characterizes their output.








\subsubsection{Two-Party Protocols}

A two-party protocol $\Pi=(\Ac,\Bc)$ is \ppt if the running time of both parties is polynomial in their input length. We let $\Pi(x,y)(z)$ or $(\Ac(x),\Bc(y))(z)$ denote a random execution of $\Pi$ on a common input $z$, and private inputs $x,y$.%We assume \wlg that a protocol has a common output (part of its transcript).\Jnote{This is not really the case we consider in this paper..}

\begin{definition}[Oracle-aided protocols]\label{def:ChannelAidedProtocol}
	In a two-party protocol $\Pi$ with oracle access to a {\sf protocol} $\Psi$, denoted $\Pi^\Psi$, the parties make use of the \textit{next-message function} of $\Psi$.\footnote{The function that on a partial view of one of the parties, returns its next message.} In a two-party protocol $\Pi$ with oracle access to a {\sf channel} $C_{Z W}$, denoted $\Pi^C$, the parties can jointly invoke $C$ for several times. In each call, an independent pair $(z,w)$ is sampled according to $C_{Z W}$, one party gets $z$, the other gets $w$.
\end{definition}


\begin{definition}[The channel of a protocol]\label{def:ChannlOfProtocol}
	For a no-input two-party protocol $\Pi= (\Ac,\Bc)$, we associate the channel $C_\Pi$, defined by $\C_\Pi= C_{(X, U),(Y, V)}$, where $X$ and $Y$ are the local outputs of $\Ac$ and $\Bc$ (respectively) and
	$U$ and $V$ are the local views of $\Ac$ and $\Bc$ (respectively).
    
	For a two-party protocol $\Pi$ that gets a security parameter $1^\pk$ as its (only, common) input, we associate the channel ensemble $ \set{C_{\Pi(1^\pk)}}_{\pk\in \N}$. 
\end{definition}

\begin{definition}[$(\alpha,\gamma)$-Accurate channel]\label{def:accurate-func}
	A channel $C = C_{(X, U),(Y, V)}$ is {\sf $(\alpha,\gamma)$-accurate for the function $f$}, if $\ppr{C}{\size{\out(V)-f(X,Y)}\leq \alpha}\ge \gamma$, where $\out(V)$ is the designated output.
    A channel ensemble $C_{(X, U),(Y, V)}= \set{C_{(X_\pk, U_\pk),(Y_\pk, V_\pk)}}_{\pk\in \N}$ is  $(\alpha,\gamma)$-accurate for  $f$ if $C_{(X_\pk, U_\pk),(Y_\pk, V_\pk)}$ is $(\alpha(\pk),\gamma(\pk))$-accurate for $f$, for every $\pk \in \N$.
\end{definition}

\subsubsection{Differentially Private Channels}\label{sec:DPChannel}
Differentially private channels are naturally defined as follows:
\begin{definition}[Differentially private channels]\label{def:DPChannel}
	An $n$-size channel $C = C_{(X, U),(Y, V)}$ with $X, Y$ over $\oo^n$ 
	is {\sf$(\eps,\delta)$-differentially private} (denoted $(\eps,\delta)$-$\DP$) if for every $x \in \Supp(X)$ there exists an $n$-size $(\eps,\delta)$-$\DP$ mechanisms $\Mc_x$ such that $(X,Y,U) \equiv (X,Y,\Mc_X(Y))$, and for every $y \in \Supp(Y)$ there exists an $n$-size $(\eps,\delta)$-$\DP$ mechanisms $\Mc_y'$ such that $(X,Y,V) \equiv (X,Y,\Mc_Y'(X))$. In addition, we say that the channel is \emph{uniform} if $X$ and $Y$ are independent random variables uniformly distributed in $\oo^n$. 
\end{definition}

\begin{definition}[Computational differentially private channels]\label{def:CDPChannel}
	An $n$-size channel ensemble $C = \set{C_{(X_\pk, U_\pk),(Y_\pk, V_\pk)}}_{\pk\in\N}$ with $X_\pk, Y_\pk$ over $\oo^n$ 
	is {\sf$(\eps,\delta)$-computationally differentially private} (denoted $(\eps,\delta)$-$\CDP$) if for every ensemble $\set{x_\pk \in \Supp(X_\pk)}_{\pk\in\N}$ there exists an $n$-size $(\eps,\delta)$-\CDP mechanisms ensemble $\set{\Mc_{x_\pk}}_{\pk\in\N}$ such that $(X_\pk,Y_\pk,U_\pk) \equiv (X_\pk,Y_\pk,\Mc_{X_\pk}(Y_\pk))$, for every $\pk\in\N$, and for every ensemble $\set{y_\pk \in \Supp(Y_\pk)}_{\pk\in\N}$ there exists an $n$-size $(\eps,\delta)$-$\CDP$ mechanisms ensemble $\set{\Mc'_{y_\pk}}_{\pk\in\N}$ such that $(X_\pk,Y_\pk,V_\pk) \equiv (X_\pk,Y_\pk,\Mc_{Y_\pk}'(X_\pk))$ for every $\pk\in \N$. In addition, we say that the channel is \emph{uniform} if $X_\pk$ and $Y_\pk$ are independent random variables uniformly distributed in $\{\pm 1\}^n$ for all $\pk\in\N$.
\end{definition}




% \begin{lemma}~\label{lem:dp-sv-source}
% 	Let $P$ be an $\varepsilon$-DP randomized protocol. Let $X$ and $Y$ be independent random variables uniformly distributed in $\{\pm 1\}^n$ and let random variable $\Pi(X,Y)$ denote the transcript of running $P(X,y)$. Then for every $\pi\in Supp(\Pi)$, the random variables corresponding to the inputs conditioned on transcript $\pi$, $X_\pi$ and $Y_\pi$, are independent $e^{-\varepsilon}$-strong SV source.
% \end{lemma}





\subsubsection{Weak Erasure Channel (\WEC)}

\begin{definition}[\WEC]\label{def:WEC}
	A channel $((O_A,V_A), (O_B,V_B))$ with $O_A \in \set{0,1}$ and $O_B \in \set{0,1,\bot}$ is a {\sf weak erasure channel}, denoted $(\alpha,p,q)$-$\WEC$, if:
	\begin{itemize}
		%\item $O_A\in \set{-1,1}$ and $O_B\in \set{-1,1,\bot}$.
		\item Random erasure: $\pr{O_B = \perp} = 1/2$.
		
		\item Agreement: $\pr{O_A\ne O_B\mid O_B\ne \bot}\le \alpha$.
		
		\item Secrecy:
		
		\begin{enumerate}
			\item For every algorithm $\Dc$ it holds that\label{WEC:item:A}
			\begin{align*}
				%\size{\pr{\Ac(O_A,V_A) = 1 \mid O_B \neq \perp} - \pr{\Ac(O_A,V_A) = 1 \mid O_B = \perp}} \le p
				\size{\pr{\Dc(V_A) = 1 \mid O_B \neq \perp} - \pr{\Dc(V_A) = 1 \mid O_B = \perp}} \le p
			\end{align*}
			(Alice doesn't know if $O_B = \perp$.)
			
			\item For every algorithm $\Dc$ it holds that\label{WEC:item:B}
			\begin{align*}
				\pr{\Dc(V_B) = O_A \mid O_B=\bot} \leq \frac{1+q}{2}.
			\end{align*}
			(i.e., if $O_B=\bot$, Bob don't know what is the value of $O_A$).
			
			%\item $SD((O_A U|O_B=\bot),(O_A U|O_B\ne \bot))\le p$ (The sender don't know if $O_B=\bot$).
			
			%\item $SD(V O_A|O_B=\bot,V(-O_A)|O_B=\bot)\le q$ (If $O_B=\bot$, Bob don't know what the value of $O_A$).
		\end{enumerate}
	\end{itemize}
   We say that a channel ensemble $C=\set{C_\pk}_{\pk\in N}$ is a {\sf computational weak erasure channel}, denoted $(\alpha,p,q)$-\CompWEC, if for every \ppt algorithm $\Dc$ and every sufficiently large $\pk\in\N$, $C_\pk$ satisfies the properties stated in the items above, where the secrecy property holds with respect to a \ppt algorithm $\Dc$. A protocol $\Lambda$ is said to be $(\alpha,p,q)$-$\CompWEC$, if the ensemble induces by the protocol (that is, $C=\set{C_{\Lambda(\pk)}}_{\pk\in\N}$) is $(\alpha,p,q)$-$\CompWEC$.  
\end{definition}



\subsubsection{Approximate Weak Erasure Channel (\AWEC)}\label{sec:AWEC}

\begin{definition}[\AWEC]\label{def:AWEC}
	A channel $C = ((O_A,V_A), (O_B,V_B))$ over $([-n,n] \times \zo^*) \times (([-n,n] \cup \bot)  \times \zo^*)$ is an {\sf approximate weak erasure channel}, denoted $(\ell,\alpha,p,q)$-\AWEC if:
	\begin{itemize}
		
		\item Random erasure: $\pr{O_B = \perp} = 1/2$.
		
		\item Accuracy: $\pr{\size{O_A - O_B} > \ell \mid O_B \ne \bot}\le \alpha$.
		
		\item Secrecy:
		
		\begin{enumerate}
			\item For every algorithm $\Dc$ it holds that\label{AWEC:item:A}
			\begin{align*}
				%\size{\pr{\Ac(O_A,V_A) = 1 \mid O_B \neq \perp} - \pr{\Ac(O_A,V_A) = 1 \mid O_B = \perp}} \le p
				\size{\pr{\Dc(V_A) = 1 \mid O_B \neq \perp} - \pr{\Dc(V_A) = 1 \mid O_B = \perp}} \le p
			\end{align*}
			(Alice doesn't know if $O_B=\bot$).
			
			\item For every algorithm $\Dc$ it holds that\label{AWEC:item:B}
			\begin{align*}
				\pr{\size{\Dc(V_B) - O_A} \leq 1000 \ell \mid O_B=\bot} \leq q.
			\end{align*}
			(i.e., if $O_B=\bot$, Bob can't estimate the value of $O_A$ with error $\leq 1000 \ell$).
		\end{enumerate}
	\end{itemize}
     We say that a channel ensemble $C=\set{C_\pk}_{\pk\in N}$ is a {\sf computational approximate weak erasure channel}, denoted $(\ell,\alpha,p,q)$-\CompAWEC, if for every \ppt algorithm $\Dc$ and every sufficiently large $\pk\in\N$, $C_\pk$ satisfies the properties stated in the items above. A protocol $\Gamma$ is said to be $(\ell,\alpha,p,q)$-$\CompAWEC$, if the ensemble induced by the protocol (that is, $C=\set{C_{\Gamma(\pk)}}_{\pk\in\N}$) is $(\ell,\alpha,p,q)$-$\CompAWEC$.  
\end{definition}

We will make use of the following lemma, which shows that for some choices of the parameters, \AWEC implies \WEC. The lemma is proven in \cref{sec:AWEC-to-WEC}.

\begin{lemma}\label{lemma:AWEC-to-WEC}
	For every $\ell> 0$, there exists a \ppt protocol $\Lambda = (\Pc_1,\Pc_2)$ such that given an oracle access to an $(\ell,\alpha,p,q)$-\AWEC $C$, the channel $\tilde{C}$ induced by $\Lambda^C$ is $(\alpha'=\alpha+0.001,\: p' = p ,\:  q' = 1/2 + 2(q+0.01))$-\WEC.
	Furthermore, the proof is constructive in a black-box manner:
	\begin{enumerate}
		\item There exists an oracle-aided \ppt algorithm $\Ec_1$ such that for every channel $C = ((\OA,\VA), (\OB,\VB))$ and algorithm $\Dc$ violating the \WEC secrecy property~\ref{WEC:item:A} of $\tilde{C}$, algorithm $\Ec_1^{\Dc}$ violates the \AWEC secrecy property~\ref{AWEC:item:A} of $C$.
		
		\item There exists an oracle-aided \ppt algorithm $\Ec_2$ such that for every channel $C = ((\OA,\VA), (\OB,\VB))$ and algorithm $\Dc$ violating the \WEC secrecy property~\ref{WEC:item:B} of $\tilde{C}$, algorithm $\Ec_2^{\Dc}$ violates the \AWEC secrecy property~\ref{AWEC:item:B} of $C$.
	\end{enumerate}
\end{lemma}

Since \cref{lemma:AWEC-to-WEC} is constructive, the following is an immediate corollary.
\begin{corollary}\label{cor:CompAWEC to CompWEC}
There exists an oracle aided \ppt protocol $\Lambda$, such that given a protocol $\Gamma$ that induces $(\ell,\alpha,p,q)$-\CompAWEC, it holds that $\Lambda^\Gamma$ is $(\alpha'=\alpha+0.001,\: p' = p ,\:  q' = 1/2 + 2(q+0.01))$-\CompWEC.  
\end{corollary}
\begin{proof}[Proof of \ref{cor:CompAWEC to CompWEC}]
Let $\Lambda$ be the \ppt algorithm guaranteed  by Lemma \ref{lemma:AWEC-to-WEC}. Given an $(\ell,\alpha,p,q)$-\CompAWEC protocol $\Gamma$, we define $\Lambda(\pk)={\Lambda^{\Gamma(\pk)}(\pk)}$. Assume towards a contradiction that $\Lambda$ is not a $(\alpha',p',q')$-\CompWEC. It follows that there exists a \ppt $\Dc$ that for infinity many $\pk\in\N$ contradicts one of the \WEC secrecy properties of channel ensemble $\set{C_{\Lambda(\pk)}}_{\pk\in\N}$. Fix $\pk\in\N$ for which this holds. By Lemma \ref{lemma:AWEC-to-WEC}, there exists a \ppt $\Ec^\Dc$ that for every such $\pk$  contradicts one of the secrecy properties of the channel $C_{\Gamma(\pk)}$. This implies that for infinity many $\pk\in\N$, $\Ec^\Dc$  contradict the secrecy of the channel ensemble $\set{C_{\Gamma(\pk)}}_{\pk\in\N}$, which is a contradiction since this would means that $\Gamma$ is not a $(\ell,\alpha,p,q)$-\CompAWEC.       
\end{proof}



\subsection{Oblivious Transfer (\OT)}

\paragraph{Secure Computation.}
We use the standard notion of securely computing a functionality, \cf  \cite{Goldreich04}.
\begin{definition}[Secure computation]\label{def:SFE}
	A two-party protocol {\sf securely computes a functionality $f$}, if it does so according to the real/ideal paradigm.   We add the term perfectly/statistically/computationally/non-uniform computationally, if the simulator's output is  perfect/statistical/computationally indistinguishable/  non-uniformly indistinguishable from  the real distribution.  The protocol have the above notions of security {\sf against semi-honest  adversaries}, if its security only  guaranteed to holds against an adversary that follows the prescribed protocol.   Finally, for the case of perfectly secure computation, we naturally apply the above notion also to the non-asymptotic case: the protocol with no security parameter perfectly  compute a functionality $f$.
	
	A two-party protocol {\sf securely computes a functionality ensemble $f$ with oracle to a channel $C$}, if it does so according to the above definition when the parties have access to a trusted party computing $C$. All the above adjectives naturally extend to this setting.
\end{definition}

\paragraph{Oblivious Transfer.}
The (one-out-of-two) oblivious transfer functionality is defined as follows.
\begin{definition}[oblivious transfer functionality $f_{\OT}$]\label{def:OTfunc}
	The oblivious transfer functionality over $\zo \times (\zs)^2$ is defined by  $f_{\OT} (i,(\sigma_0,\sigma_1)) = (\perp,\sigma_i)$.
\end{definition}
A protocol is $\ast$ secure OT,   for \\$\ast\in \set{\text{semi-honest statistically/computationally/computationally non-uniform}}$, if it  compute the $f_{\OT}$  functionality with $\ast$ security.





% \begin{definition}[Computational oblivious transfer, semi-honest model]
% A protocol $\Pi=(\Ac,\Bc)$ is a semi-honest 1-out-of-2 computational oblivious transfer (comp-OT) protocol if the following holds. Given a common input $1^{\pk}$, the parties $\Ac$ and $\Bc$ run the protocol $\Pi(1^\pk)$ (in an honest manner) and    
% $\Ac$ outputs $X=(m_1,m_2)\in \zo\times\zo$ and has a view $U$ and $\Bc$ outputs $Y=(i,\hat{m})\in\zo\times\zo$ and has a view $V$, and the following properties are satisfied:
% \begin{enumerate}
%     \item \textbf{Correctness:} 
%     $\pr{\hat{m}\neq m_i}<\negl(\pk).$ 
    
%     \item \textbf{A's Privacy:} For every \ppt $\Dc$ and every sufficiently large $\pk$:
%     $\pr{\Dc(V)=m_{i-1}}<(1+\negl(\pk))/2$
    
%     \item \textbf{B's Privacy:} For every \ppt $\Dc$ and every sufficiently large $\pk$:
%     $\pr{\Dc(U)=i}<(1+\negl(\pk))/2$  
% \end{enumerate}
% \end{definition}

We make use of the following useful results by Wullschleger on oblivious transfer amplification from weak channels.
\begin{theorem}[\cite{Wullschleger09}, from \WEC to statistically secure \OT]\label{thm:WEC TO OT IT}
    There exists an oracle aided protocol $\Pi$ such that the following holds: Given a $(\alpha,p,q)$-\WEC $C$, if $44(\alpha+p)\le 1-q$ then $\Pi^{C}(1^\pk)$ is a semi-honest statistically secure \OT.
\end{theorem}

The following computational version of \cref{thm:WEC TO OT IT} is implicit in \cite{Wullschleger09} and is based on the computational proof explicitly stated in \cite{Wul07} (see Section 6 in \cite{Wullschleger09} for discussion).   

\begin{theorem}[\cite{Wullschleger09,   Wul07}, from \CompWEC to computinally secure \OT]\label{thm:WEC TO OT Comp}
    There exists an oracle aided protocol $\Pi$ such that the following holds: Given a $(\alpha,p,q)$-\CompWEC protocol $\Lambda$, if $44(\alpha+p)\le 1-q$ then $\Pi^{\Lambda}$ is a semi-honest computational secure \OT.
\end{theorem}



% \begin{definition}[Computational 1-out-of-2 Oblivious Transfer, semi-honest model]
% A protocol $\Pi=(\Ac,\Bc)$ is a semi-honest 1-out-of-2 $(\eps,\alpha,\beta)$-oblivious transfer (OT) protocol if the following holds. 

% The parties $\Ac$ and $\Bc$ run the protocol (in an honest manner) and    
% $\Ac$ outputs $X=(m_1,m_2)\in \zo\times\zo$ and has a view $U$ and $\Bc$ outputs $Y=(i,\hat{m})\in\zo\times\zo$ and has a view $V$, and following properties are satisfied:
% \begin{enumerate}
%     \item \textbf{Correctness:} 
%     $\pr{\hat{m}\neq m_i}<\eps.$ 
    
%     \item \textbf{A's Privacy:} For every adversary $\Dc$:
%     $\pr{\Dc(V)=m_{i-1}}<(1+\alpha)/2$
    
%     \item \textbf{B's Privacy:} For every adversary $\Dc$: $\pr{\Dc(U)=i}<(1+\beta)/2$  
% \end{enumerate}
% \end{definition}

	\section{Lower Bounds}\label{sec:lowerbounds}

This section contains three parts.

In~\cref{subsec:warmup}, we will prove that there is no polynomial algorithm solving \oss{n^\epsilon}{1}. The technique is similar to the construction used by Elkin and Peleg~\cite{ElkinP07} to prove the hardness of approximating directed spanner. We will also define \ga{}, as discussed in~\cref{sec:overview}, and prove the hardness for \ga{} when $s=o(m)$.

In~\cref{subsec:bicriterialower}, we will show how to prove the lower bound for \oss{n^\epsilon}{n^\epsilon}, using the lower bound for \ga{} while preserving the relative size of $s$ versus $m$. It will only prove that \oss{n^\epsilon}{n^\epsilon} is hard for some input $s=o(m)$ by combining with the result in~\cref{subsec:warmup}, which is still not what we want in~\cref{thm:main}.

In~\cref{subsec:largebicriteria}, we boost $s$ to be $\omega(m)$ in the lower bound proof of \ga{}. Combined with the lemma proved in~\cref{subsec:bicriterialower}, this will show us \oss{n^\epsilon}{n^\epsilon} is hard for some input $s=\Omega(m^{1+\epsilon})$, which is what we want in~\cref{thm:main}.


\subsection{Warm up: Lower Bounds when \texorpdfstring{$\apxD=1$}{apxD1}}\label{subsec:warmup}

This section will use a simple construction to prove the following lemma.

\begin{lemma}\label{lem:onecriteria}
	Assuming PGC (\cref{con:pgc}), there are no polynomial time algorithm solving \oss{n^{\epsilon}}{1} for some small constant $\epsilon$. 
	
\end{lemma}

The construction relies on the following \minre{} graph.

\begin{definition}[\labcov{} graph]\label{def:minrepgraph}
	Given a \labcov{} instance $\I=(A,B,E,\L,(\pi_e)_{e\in E})$ descirbed in~\cref{def:labcov} and a parameter $\rho$, we define the \labcov{} graph $G_{\I,\rho}$ as a directed graph defined as follows (see \cref{fig:minrepgraph}):
	\begin{itemize}
		\item Suppose $A=\{1,...,|A|\},B=\{1,...,|B|\},\L=\{1,2,...,|\L|\}$. In $G_{\I,\rho}$ we have vertices $\{\aa{i}{j},\bb{i}{j}\mid 1\le i\le |A|=|B|, 1\le j\le |\L|\}$ and edges $\{(\aa{i}{j},\bb{i'}{j'})\mid (i,i')\in E, (j,j')\in\pi_{(i,i')}\}$.
		
		\item In addition, $G_{\I,\rho}$ also contains vertices $\{\a{i},\b{i}\mid 1\le i,j\le |A|=|B|\}$, where $\a{i}$ has a directed path with length $\rho$ to each vertex in $\{\aa{i}{j}\mid 1\le j\le |\L|\}$ denoted by $\alpha^{(i)}_j$, the vertices along the path are $\{\aaa{i}{j}{k}\mid 1\le k\le \rho-1\}$; similarly, $\b{i}$ has a directed path with length $\rho$ from each vertex in $\{\bb{i}{j}\mid 1\le j\le |\L|\}$ denoted by $\beta^{(i)}_j$, the vertices along the path are $\{\bbb{i}{j}{k}\mid 1\le k\le \rho-1\}$. 
		
		\item Finally, we add edges $(\aaa{i}{j}{k},\a{i}),(\b{i},\bbb{i}{j}{k}),(\aa{i}{j},\a{i})$, $(\b{i},\bb{i}{j})$ for any possible $i,j,k$.
	\end{itemize}
	
\end{definition}

\begin{figure}[H]
	\centering
	\includegraphics[scale=1]{figures/minrepgraph.pdf}
	\setcaptionwidth{0.95\textwidth}
	\caption{\small In this example, $|A|=|B|=|\mathcal{L}|=\rho-1=3$. Suppose $A=\{1,2,3\},B=\{1,2,3\},\mathcal{L}=\{1,2,3\}$, then in this example we have $E=\{(2,2)\}$ and $\pi_{(2,2)}=\{(1,2),(2,3),(3,1)\}$, which corresponds to the three edges in the middle. }\label{fig:minrepgraph}	
\end{figure}


\begin{proof}[Proof of~\cref{lem:onecriteria}]
	Given a \labcov{} instance $\I=(A,B,E,\L,(\pi_e)_{e\in E})$, we use $\AB$ to denote the size of $A$ (and also $B$). We now describe how to distinguish between the case. Notice that $N$ is the input size of the \labcov{} instance $\I$.
	\begin{itemize}
		
		\item (Completeness:) There is a labeling that covers every edge. 
		
		\item (Soundness:) Any multilabeling of cost at most $N^{\epsl{}}(|A|+|B|)$ covers at most $N^{-\epsl{}}$ fraction of edges.             
	\end{itemize}
	According to~\cref{lem:pgc}, this violates PGC. 
	
	If $|E|\le 2\AB n^{\epsl}$, then it cannot be the Soundness case, so we assume $|E|>2\AB n^{\epsl}$. 
	
	First, we use polynomial time to compute the \labcov{} graph $G_{\I,\dia{}}$ with $n$ vertices, where $\dia{}$ is polynomial on $N$ and can be arbitrarily large. 
	Then We run the \oss{n^\epsilon}{1} algorithm with input $G_{\I,\dia{}}$ and parameters $s=2\AB,d=\dia{}+1$. We will prove the following two claims, which will lead to the solution to the \labcov{} problem.
	\begin{claim}\label{clai:caseI}
		If the \labcov{} instance is in case (completness), then $G_{\I,\dia{}}$ has a \ssss{2\AB }{\dia{}+1}. Moerover, for any $(i,j)\in E$, $(\a{i},\b{j})$ has distance at most $3$ after adding this \ss{}. 
	\end{claim}
	\begin{proof}
		Suppose the labeling is $\psi$. For any $i\in A$, we include the edge $(\a{i},\aa{i}{\psi(i)})$ in the \ss{}; for any $i\in B$, we include the edge $(\bb{i}{\psi(i)},\b{i})$ in the \ss{}. The \ss{} has size $2\AB $. Since $(i,j)$ is covered by $\psi$, there exists an edge $(\aa{i}{i'},\bb{j}{j'})$ such that $i'=\psi(i),j'=\psi(j)$. In that case, we have the length 3 path $(\a{i},\aa{i}{i'},\bb{j}{j'},\b{j})$ after adding the \ss{}. This proves the second statement of this lemma. Now we verify the distances between reachable pairs are at most $\dia{}+1$ one by one.
		\begin{enumerate}
			\item (start from $\a{i}$) $\a{i}$ can reach any $\aa{i}{j},\aaa{i}{j}{k}$ with distance at most $\dia{}$, $\a{i}$ can reach any $\b{j}$ with $(i,i')\in E$ with distance $3$, which means $\a{i}$ can reach any $\bbb{i'}{j}{k},\bb{i'}{j}$ with distance $4$.
			
			\item (start from $\aaa{i}{j}{k},\aa{i}{j}$) $\aaa{i}{j}{k}$ or $\aa{i}{j}$ has an edge to $\a{i}$, so they can reach any node that $\a{i}$ can reach with distance at most $\dia{}+1$. 
			
			\item (start from $\bbb{i}{j}{k},\bb{i}{j},\b{i}$) These nodes can reach any reachable nodes with distance at most $\dia{}+1$ in $G_{\I,\dia{}}$. 
		\end{enumerate}
		
	\end{proof}
	
	\begin{claim}\label{clai:caseII}
		If the \minre{} instance is in case (soundness), then $G_{\I,\dia{}}$ does not have a \ssss{2\AB\cdot n^{\epsilon}}{\dia{}+1} for sufficiently small constant $\epsilon$. Moreover, by adding any shortcut with size at most $2\AB\cdot n^{\epsilon}$, at most $o(1)$ fraction of pairs in $\{(a^{(i)},b^{(j)})\mid (i,j)\in E\}$ will have distance less than $\dia{}+1$. 
	\end{claim}
	\begin{proof}
		Suppose $G_{\I,\dia{}}$ has a shortcut $E'$ adding which reduces the diameter between $\Omega(1)$ fraction of pairs in $\{(a^{(i)},b^{(j)})\mid (i,j)\in E\}$ to less than $\dia{}+1$. We will try to get a contradiction. We first use $E'$ to build a \mlab{} $\psi$ in the following way: for any $1\le i\le |A|$, we let $\psi(i)$ to be the set of vertices among $\{\aa{i}{j}\mid 1\le j\le \L\}$ that $\a{i}$ has distance at most $\dia{}-1$ to after adding the \ss{}. 
		
		\paragraph{$\psi$ covers more than $N^{-\epsl}$ fraction of edges.}	We first argue that $\psi$ covers $\Omega(1)$ fraction of the edges. Write $A_i=\{\a{i},\aa{i}{j},\aaa{i}{j}{k}\mid 1\le j\le \L,1\le k\le \dia{}-1\}, B_i=\{\b{i},\bb{i}{j},\bbb{i}{j}{k}\mid 1\le j\le \L,1\le k\le \dia{}-1\}$. For an edges $(i,j)\in E$ (recall that $E$ is the edge set in the \labcov{} instance $\I$), we say it is \emph{crossed} if there is an edge $(u,v)\in E'$ with $u\in A_i,v\in B_j$ in $E'$. Remember that we assumed $|E|\ge \Delta\cdot N^{\epsl{}}$, otherwise, the case (soundness) can never happen. Therefore, the number of crossed edges is at most $2\AB\cdot n^{\epsilon}=o(|E|)$ for sufficiently small $\epsilon$. Now We prove that for any non-crossed edge $(i,j)\in E$ such that $(a^{(i)},b^{(j)})$ has distance less than $\dia{}+1$ after adding $E'$, $(i,j)$ is covered by $\psi$. If we can prove this, then at least $\Omega(1)$ fraction of edges in $E$ are covered. To prove this, notice that if $(i,j)$ is not covered, then consider the shortest path $p$ from $\a{i}$ to $\b{j}$ after adding the \ss{}, we write the part where $p$ is inside $A_i$ as $p_A$, and the part where $p$ is inside $B_j$ as $p_B$. if both $p_A,p_B$ has length at most $\dia{}-1$, then $(i,j)$ is covered. Thus, one of $p_A$ or $p_B$ has length at least $\dia{}$. Then we have $|p|=|p_A|+1+|p_B|\ge \dia{}+1+1\ge \dia{}+2$, which is a contradiction. 
		
		\paragraph{$\psi$ has cost at most $N^{\epsl}(|A|+|B|)$.} Then we argue that $\sum_{u\in A\cup B}|\psi(u)|\le |E'|\le 2\AB \cdot n^{\epsilon}$ (which will give us $\sum_{u\in A\cup B}|\psi(u)|\le N^{\epsl{}}(|A|+|B|)$ for sufficiently small $\epsilon$). For a label $j\in\psi(i)$, we have $\distt{(V,E\cup E')}{\a{i},\aa{i}{j}}<\dia{}$. Thus, there exists a path $p=(v_0,...,v_{\ell})$ from $\a{i}$ to $\aa{i}{j}$ with length at most $\dia{}-1$. Let $p'=(\a{i},\aaa{i}{j}{1},\aaa{i}{j}{2},...,\aa{i}{j})$. Let $k$ be the last index such that $v_k\not\in p'$ and $v_{k+1}\in p'$. Since $v_{\ell}$ in $p'$ and the length of $p'$ equals $\dia{}$, such $k$ must exist. Since $(v_k,v_{k+1})\not\in E$, we have $(v_k,v_{k+1})\in E'$. We call this edge $(v_k,v_{k+1})$ a \emph{critical edge} of $(i,j)$. For different $i$ and $j$ with $j\in\psi(i)$, they have different critical edges in $E'$ because the corresponding $v_{k+1}$ is always different. Thus, $\sum_{u\in A\cup B}|\psi(u)|\le |E'|\le 2\AB \cdot n^{\epsilon}$.
		
	\end{proof}
	
	If the \labcov{} instance is in case (completeness), by~\cref{clai:caseI}, $G_{\I,\dia{}}$ admits a \ssss{2\Delta}{\dia{}+1}, which means the output by the \oss{n^{\epsilon}}{1} algorithm will be a \ssss{2\Delta\cdot n^{\epsilon}}{\dia{}+1}; on the other hand, if the \labcov{} instance is in case (soundness), by~\cref{clai:caseII}, the output cannot be a \ssss{2\Delta\cdot n^{\epsilon}}{\dia{}+1} since $G_{\I,\dia{}}$ does not have one. Since checking whether the output is \ssss{2\Delta\cdot n^{\epsilon}}{\dia{}+1} or not is in polynomial time, \labcov{} is solved in polynomial time.
 
\end{proof}






In the following definition, we will abstract all the necessary properties of graph $G_{\I,d}$ in a black box that we are going to use to prove the hardness for \oss{n^{\epsilon}}{n^\epsilon}.

\begin{definition}\label{def:gadget}
	For $1>\epsilon,\cs{}>0$, the \gadget{\cs{}}{\epsilon} problem has inputs
	\begin{enumerate}
		\item a directed connected graph $G=(V,E)$ with $m$ edges, $n$ nodes and diameter polynomial on $m$, let the diameter be $d$,
		\item two sets $L,R\subseteq V$ with $L\cap R=\emptyset$, where $|L|$ is polynomial on $m$,
		\item a set of reachable vertex pairs $P\subseteq L\times R$,
		\item a positive integer $s=\Omega(m^{\cs{}})$.
	\end{enumerate}
	The problem asks to distinguish the following two types of graphs.
	\begin{description}
		\item[Type 1.] There exists a \ss{} $E'$ of $G$ with size $s$ such that all reachable pairs $(u,v)\in L\times R$ have distance $O(1)$ after adding $E'$ to $G$.
		\item[Type 2.] By adding any \ss{} with size $s\cdot n^\epsilon$, at most $o(1)$ fraction of the pairs in $P$ have distance at most $d/3$.
	\end{description}
	
\end{definition}

The following lemma shows that~\cref{fig:minrepgraph} is a hard instance for \gadget{\cs{}}{\epsilon}.
\begin{lemma}\label{lem:smallgadget}
	Under PGC (\cref{con:pgc}), there exist constants $\epsilon,\cs{}$ such that \gadget{\cs{}}{\epsilon} cannot be solved in polynomial time.
	
\end{lemma}
\begin{proof}
	Given any \labcov{} instance $\I=(A,B,E,\L,\pi)$ (we write $|A|=\AB$), we construct a \gadget{\cs{}}{\epsilon} instance with the following inputs.
	\begin{enumerate}
		\item The graph is $G_{\I,\dia{}}$ (see~\cref{def:minrepgraph}) for $\dia{}$ arbitrarily large such that $\dia{}$ is polynomial on $N$ (the bit length of instance $\I$). $G_{\I,d}$ has diameter $d=2\dia{}+1$.
		\item $L=\{a_i\mid 1\le i\le \AB\},R=\{b_i\mid 1\le i\le \AB\}$, clearly $|L|=|R|\le m$ and $|L|$ is polynomial on $m$.
		\item $P=\{(a_i,b_j)\mid (i,j)\in E\}$.
		\item $s=|A|+|B|$, it is polynomial on $m$.
	\end{enumerate}
	
	Now we prove that if we have a polynomial time algorithm to distinguish the two types of graphs as described in~\cref{def:gadget}, then we can solve \labcov{} in polynomial time.
	
	If $\I$ is in case (completeness), according to~\cref{clai:caseI}, by adding a \ss{} with size $s=(|A|+|B|)$, all reachable pairs $(a_i,b_j)$ (with $(i,j)\in E$) has distance at most 3. 
 
 If $\I$ is in case (soundness), according to~\cref{clai:caseII}, by adding a \ss{} with size less than $s\cdot n^{\epsilon}$ for sufficiently small $\epsilon$, at most $o(1)$ fraction of pairs in $P$ has distance less than $(2d+1)/3<\dia{}$. 
\end{proof}

\subsection{Lower Bound when \texorpdfstring{$\apxD>1$}{apxDge1} and \texorpdfstring{$s=o(m)$}{0<cs<1}}\label{subsec:bicriterialower}
In this section, we prove the following lemma, which shows how to use the hardness of \gadget{\cs{}}{\epsilon} to get lower bounds for $\apxD>1$.
\begin{lemma}\label{lem:usegadget}
	For any constant $1>\epsilon,\cs{}>0$, if there is no polynomial algorithm solving \gadget{\cs{}}{\epsilon}, then for sufficiently small constant $\gamma$, there is no polynomial algorithm solving \oss{n^{\gamma\epsilon}}{n^{\gamma\epsilon}} even if the input is restricted to $s=\Omega(m^{1+\gamma(\cs{}-1)})$.
	
\end{lemma}
By combining \cref{lem:smallgadget,lem:usegadget}, we can get the following corollary. Notice that this corollary is not the same as~\cref{thm:main}, since it does not restrict $s$ to be $\omega(m)$.

\begin{corollary}\label{cor:smallmain}
	Under \conj{} (\cref{con:pgc}), there is no polynomial algorithm solving \oss{n^{\epsilon}}{n^{\epsilon}} for sufficiently small constant $\epsilon$.
\end{corollary}

The following geometric graph~\cite{HuangP21} is a crucial part of our proof for~\cref{lem:usegadget}. We say a graph $G=(V,E)$ is a $k$-layered directed graph if $V=V_1\uplus V_2\uplus...\uplus V_k$ ($V$ is partitioned into $V_1,...,V_k$), such that for any edge $(u,v)\in E$, there exists $1\le i<k$ and $u\in V_i,v\in V_{i+1}$. $V_i$ is called the $i$-th layer of $G$.

\begin{lemma}[Section 2.2~\cite{HuangP21}]\label{lem:geograph}
	For any $\AB\in\mathbb{N}^+$, for arbitrary constant $0<\lan{}< 1$, we can compute in polynomial time a $\AB^\lan$ layered graph $G=(V,E)$, a set of pairs $P$ and an indexing $\idx{}$ satisfying the following properties.
 
	\begin{enumerate}
            \item Each layer of $G$ contains $\The{\AB^{10}}$ vertices and has a maximum in-degree and out-degree of at most $\AB$. Let the first layer be denoted as $V_1 = \{s_1, s_2, \dots, s_{K_1}\}$ and the last layer as $V_2 = \{t_1, t_2, \dots, t_{K_2}\}$. \label{geoitem1}

		\item $P \subseteq [K_1] \times [K_2]$. For every $i\in[K_1]$, there are $\Omega(\Delta^2)$ pairs $(i,j)\in P$. For every $(i,j)\in P$, there is a unique path from $s_i$ to $t_j$ in $G$, denoted by $p_{i,j}$. $p_{i,j}$ and $p_{i',j'}$ shares at most one edge for different pairs $(i,j)\not=(i',j')$. \label{geoitem3}
		\item The function $\idx{}$ assigns a value from $[\Delta]$ to each $(v, e)$ pair, where $e$ is either an in-edge or an out-edge of vertex $v$. For a given vertex $v$, $\idx{}$ assigns distinct values to its different in-edges, and similarly, $\idx{}$ assigns distinct values to its different out-edges.  For any $(x,y)\in[\AB]\times[\AB]$, there exists $\Omega(\Delta^{10})$ paths $p_{i,j}$ (where $(i,j)\in P$) such that for any edge $(u,v)$ on $p_{i,j}$,\label{geoitem4}
                \begin{itemize}
                    \item if $v$ is in the even layer, then $\Idx{v,(u,v)}=x$, 
                    \item if $u$ is in the even layer, then $\Idx{u,(u,v)}=y$.
                \end{itemize}
	\end{enumerate}
 
\end{lemma}
\begin{proof}[Proof of~\cref{lem:geograph}]
	The general idea is to use the graph described in Section 2.2~\cite{HuangP21} with max degree $\Delta$, and cut the first $\Delta^{\lan{}}$ layers to get our disired graph. Readers are recommanded to read the construction in Section 2.2~\cite{HuangP21}, and the construction for~\cref{lem:geograph} is followed straightforwardly. 
	
	Let the graph $G_1\otimes G_2$ described in Section 2.2~\cite{HuangP21} with parameters $d_1=d_2=2,r_1=\Delta^{3/2}$ be $G_{\Delta}$. $G_{\Delta}$ has $\Omega(\Delta)$ layers, where each layer has $\The{\Delta^{10}}$ vertices. The max in/out-degree is $\Delta$. We use the subgraph of $G_{\Delta}$ induced by the first $\Delta^{\lan{}}$ layers
	to get the desired graph described in~\cref{lem:geograph}, denoted by $G'_{\Delta}$. %
	We construct $P$ in the following way. According to Section 2.2~\cite{HuangP21}, there exist $\Omega(\Delta^{12})$ \emph{Critical Pairs} in $G_{\Delta}$ between the first and last laryer nodes where each of them has a unique path connecting them. We cut each unique path at the first $\Delta^{\lan{}}$ layers, resulting in $\Omega(\Delta^{12})$ pairs between the first and last layer in $G'_{\Delta}$. %
	Now we verify the properties of $P$ one by one.
	\begin{enumerate}
		\item In Section 2.2~\cite{HuangP21}, each critical pairs is $((a,b,0),(a+Dv,b+Dw,2D))$ for some $v,w\in\mathcal{V}_2(r_1)$. There are $\Delta$ possible choices for $v$ or $w$. Thus, for every $a$, there are $\Omega(\Delta^2)$ choices of $b$ such that $(a,b)\in P$.
		\item If the path between $(a,b)\in P$ is not unique, then the path between two critical pair in $G_{\Delta}$ is also not unique.
		\item Since every two path between two critical pairs in $G_\Delta$ share at most one edge according to Lemma 2.6~\cite{HuangP21}, our path segments in the first $\Delta^{\lan{}}$ layers also share at most one edge.
	\end{enumerate}

        Now we construct $\idx{}$. We fix an order for the set $\mathcal{V}_2(r_1)=\{w_1,w_2,...,w_\Delta\}$. Each node in the even layer $(a,b,i)$ has out-edges $\{(a+w_j,b,i+1)\left((a,b,i),(a+w_j,b,i+1)\right)\mid j\in[\Delta]\}$. $\idx{}$ assign index $j$ to the edge $\left((a,b,i),(a+w_j,b,i+1)\right)$. $(a,b,i)$ has in-edges $\{\left((a,b-w_j,i-1),(a,b,i)\right)\mid j\in[\Delta],b-w_j\in B_2(R_1+\left\lceil (i-1)/2\right\rceil r_1)\}$. $\idx{}$ assign $j$ to the edge $\left((a,b-w_j,i-1),(a,b,i)\right)$. Now for any $(x,y)\in[\Delta]\times[\Delta]$, we consider all the unique paths from $(a,b,0)$ to $(a+\Delta^{\lan{}}x,b+\Delta^{\lan{}}y,\Delta^{\lan{}}-1)$. All edges in this path is either $\left((a+ix,b+iy,2i),(a+(i+1)x,b+iy,2i+1)\right)$, which is the in-edge of $(a+(i+1)x,b+iy,2i+1)$ assigned as $x$, or $((a+(i+1)x,b+iy,2i+1)$,$(a+(i+1)x,b+(i+1)y,2i+2))$, which is the out-edge of $(a+(i+1)x,b+iy,2i+1)$ assigned as $y$.\footnote{To avoid confusion, notice that in Section 2.2~\cite{HuangP21}, the node $(x,y,i)$ is in the $(i+1)$-th layer, because $i$ is indexed from $0$. }
\end{proof}
Now we are ready to prove the main lemma in this section.
\begin{proof}[Proof of~\cref{lem:usegadget}]
	Suppose $\mathcal{A}$ is the polynomial time algorithm for \oss{n^{\gamma\epsilon}}{n^{\gamma\epsilon}} where input must satisfy $s=\Omega(m^{1+\gamma(\cs{}-1)})$. Given a \gadget{\cs{}}{\epsilon} instance $G_{inr},(L,R),P',\bufs$ (see~\cref{def:gadget}, we use $G_{inr},P',s'$ instead of $P,s$ to avoid conflicting of notations), we will show how to use $\mathcal{A}$ as an oracle to solve it in polynomial time.
	
	\paragraph{Definition of $G$.} Let $\Delta=|L|=|R|$, let $M$ denote the number of edges in $G_{inr}$ and let the diameter of $G_{inr}$ be $\rho{}$. We first construct a graph $G$ in the following way. Let the graph, pairs, and indexing described in~\cref{lem:geograph} with parameter $\AB$ and sufficiently small constant $\lan$ be $G_{geo},P,I$. Without loss of generality, we assume $\AB^\lan$ is odd. 
	$G$ contains the following parts. See~\cref{fig:origin} as an example.
	
	\begin{figure}
		\centering
		\includegraphics[scale=0.8]{figures/origin.pdf}
		\setcaptionwidth{0.95\textwidth}
		\caption{Suppose the graph above is the graph $G_{geo}$ with $\AB=2$ (notice that for ease of explanation, the graph does not satisfy properties described in~\cref{lem:geograph}). The graph below shows how we substitute each node $v$ by $G_v$. If $v$ is in the even layer, then $G_v$ is a copy of $G_{inr}$; otherwise, if $v$ is not in the first or last layer, $G_v$ is a star graph. }\label{fig:origin}	
	\end{figure}
	
	\begin{enumerate}
		\item (Substitute nodes in even layers by copies of $G_{geo}$) for each vertex $v$ in the even layers of $G_{geo}$, create a copy of graph $G_{inr}$, denoted as $G_v$ as part of $G$. Suppose $L=\{a_1,...,a_{\Delta}\},R=\{b_1,...,b_\Delta\}$ (remember that $L,R$ are inputs to \gadget{\cs{}}{\epsilon}). $a_i$ and $b_i$ are denoted by $a^v_i,b^v_i$ in $G_v$. 
		\item (Substitute nodes in odd layers by star graphs) for each vertex $v$ in the odd layers except the first and last layer of $G_{geo}$, create vertices $s_1,...,s_{\Delta},t_1,...,t_\Delta,v$, and create edges $(s_1,v),,...,(s_\Delta,v),(v,t_1),...,(v,t_\Delta)$.
		\item (Keep first and last layer) remember that $I$ is the input to \gadget{\cs{}}{\epsilon}. $G$ includes all nodes in the first and last layer of $G_{geo}$, denoted by $s_1,s_2,...,s_{K_1}$ and $t_1,t_2,...,t_{K_2}$.
		\item (Edges connected different parts) for each edge $e=(u,v)$ in $G_{geo}$, create an edge $(b^u_{\Idx{u,e}},a^v_{\Idx{v,e}})$ in $G$. If $u$ is in the first layer or $v$ is in the last layer, just create edge $(u,a^v_{\Idx{v,e}})$ or $(b^u_{\Idx{u,e}},v)$ instead.
	\end{enumerate}
	
	Remember that $G_{geo}$ has $\Delta^{\lan{}}$ layers, each with $\Delta^{10}$ vertices, and the max degree of $G_{geo}$ is $\Delta$. The number of edges $m$ in $G$ can be calculated as
	\[m=\The{\Delta^{10+\lan{}}\cdot M}+\The{\Delta^{10+\lan{}}\cdot \Delta}+O(\Delta^{10+\lan{}}\cdot \Delta)=\The{\Delta^{10+\lan{}}\cdot(\Delta+M)}\]
	
	\paragraph{Use $\mathcal{A}$ to solve \gadget{\cs{}}{\epsilon}.} After constructing $G$, we apply $\mathcal{A}$ on $G$ with parameter $s=\AB^{10+\lan}\cdot \bufs, d=C\rho$ for sufficiently large constant $C$. We first argue that $s=\Omega(m^{1+\gamma(\cs{}-1)})$ for sufficiently small $\gamma$ so this is a valid input. Remember that $\bufs$ is one of the inputs to \gadget{\cs{}}{\epsilon} such that $s'=\Omega(M^{\cs{}})$. Also remember in~\cref{def:gadget}, we require $|L|=|R|=\Delta\le M$ to be polynomial on $M$. Denote $M=\Delta^{c_{\Delta}}$. Then we have $m=\The{\Delta^{10+\lan{}+c_\Delta}}$ and $s=\Delta^{10+\lan{}+c_{\Delta}\cs{}}$. Now we have $s/m=\The{\Delta^{c_\Delta(\cs{}-1)}}$, where $\Delta^{c_\Delta}=m^{\gamma}$ for some constant $\gamma$. Thus, we have $s=\Omega(m^{1+\gamma(\cs{}-1)})$.
	
Remember that $\rho$ is the diameter of $G_{inr}$. According to~\cref{def:gadget}, $\rho{}$ is polynomial on $M$. Let $\rho{}= M^{c_D}$. %
Remember our goal is to distinguish the following two types of \gadget{\cs{}}{\epsilon} instances.

\begin{description}
	\item[Type 1.] There exists a \ss{} $E'$ of $G_{inr}$ with size $O(s')$ such that all reachable pairs $(u,v)\in L\times R$ have distance $O(1)$ after adding $E'$ to $G_{inr}$.
	\item[Type 2.] By adding any \ss{} with size $\bufs\cdot M^{\epsilon}$, at most $o(1)$ fraction of the pairs in $P'$ have distance at most $\rho{}/3$.
\end{description}

We prove the following two lemmas to show the output of $\mathcal{A}$ suffices to distinguish whether the \gadget{\cs{}}{\epsilon} instance is in type 1 or type 2. %

\begin{lemma}\label{lem:type1}
	If the \gadget{\cs{}}{\epsilon} instance is type 1, then $G$ has a \ssss{s=\AB^{10+\lan}\cdot \bufs}{O(\rho{})}.
\end{lemma}
\begin{proof}
	For each copy of $G_{inr}$ (denoted as $G_v$) in $G$, we add $\bufs$ edges to the \ss{} to make all reachable pairs $(a^v_{i},b^v_j)$ have distance $O(1)$. The \ss{} has size at most $\AB^{10+\lan}\cdot \bufs$. Consider a reachable $(u,v)$ in $G$, we first argue that they have distance $O(\rho{}+\AB^\lan)$ after adding the \ss{}. To see this, notice that a path connecting $u,v$ can only be in the form of $(u,p_1,p_2,...,p_{\ell},v)$ where $p_i$ is a path inside $G_v$ for some $v$, and $\ell\le\AB^{\lan}$. %
	Each $p_i$ with $1<i<\ell$ will be in the form $(a^v_i,...,b^v_j)$ for some $v,i,j$. Notice that $(a^v_i,b^v_j)$ has distance $O(1)$ after adding the shortcut if $v$ is in the even layer (if $v$ is in the odd layer then there is already an existing length 2 paths connecting $(a^v_i,b^v_j)$). Therefore, $u$ has distance $O(\rho{}+\AB^\lan)$ to $v$. Remember that $\rho{}=M^{c_D}=\Delta^{c_\Delta c_D}$. By setting $\lan{}<c_\Delta c_D$, the distance is at most $O(\rho{})$.
\end{proof}


\begin{lemma}\label{lem:type2}
	If the \gadget{\cs{}}{\epsilon} instance is type 2, then $G$ has no \ssss{n^{\gamma\epsilon}s}{n^{\gamma\epsilon}\rho{}}.
\end{lemma}
\begin{proof}
	
	Before we prove our lemma, we need to do some preparations. %
	Remember that $P'$ is the input to \gadget{\cs{}}{\epsilon} such that for any $(x,y)\in P'\subseteq[\AB]\times[\AB]$, $a^v_x$ can reach $b^v_y$ for any $v$. Also recall that according to~\cref{lem:geograph} item~\ref{geoitem4}, for every $(x,y)\in P'\subseteq [\AB]\times[\AB]$, in $G_{geo}$ there exists $\Omega(\Delta^{10})$ paths $p_{i,j}$ (which is the unique path connecting $s_i,t_j$) that all edges $(u,v)$ in this path is the in-edge indexed by $I$ as $x$ of $v$ if $v$ is in the even layer, or the out-edge indexed by $I$ as $y$ of $u$ if $u$ is in the even layer. That means in $G$, there is a path from $s_i$ to $t_j$ in the form of $p'_{i,j}=(s_i,a^{v_{2,q_2}}_{x},...,b^{v_{2,q_2}}_{y}, a^{v_{3,q_3}}_x,...,b^{v_{3,q_3}}_y,...,t_j)$. We say $p'_{i,j}$ covers vertices $v_{2,q_2},v_{3,q_3},...$ at $(x,y)$. Moreover, for $(i',j')\not=(i,j)$, we know from~\cref{lem:geograph} item~\ref{geoitem3} that $p_{i,j},p_{i',j'}$ shares at most one edge. Thus, $p'_{i,j}$ and $p'_{i',j'}$ cannot cover the same vertex $v$ at the same $(x,y)$, otherwise they share two edges: the in-edge indexed as $x$ of $v$ and the out-edge indexed as $y$ of $v$. In summary, we have $\Omega(\Delta^{10}|P'|)$ pairs $(i,j)$, which we call critical pairs, satisfying the following properties.
	\begin{enumerate}
		\item $s_i$ can reach $t_i$ in $G$, where all paths from $s_i$ to $t_i$ must be in the form $(s_i,a^{v_{2,q_2}}_{x},...,b^{v_{2,q_2}}_{y}, a^{v_{3,q_3}}_x,$ $...,b^{v_{3,q_3}}_y,...,t_j)$. 
		\item Any path from $s_i$ to $t_j$ cover one vertex $v$ in each layer at some pair $(x,y)\subseteq P'$. For different critical pairs $(i,j)\not=(i',j')$, they will not cover the same vertex $v$ at the same pair $(x,y)\subseteq P'$.
	\end{enumerate} 
	
	
	Now we are ready to prove \cref{lem:type2}. We will prove that $G$ does not have a \ssss{\AB^{10+\lan}\cdot \bufs\cdot M^{\epsilon/2}}{\rho{}\AB^{\lan}/9}. We first show why this implies~\cref{lem:type2}. Remember that $s=\AB^{10+\lan}\cdot \bufs$, thus, $\AB^{10+\lan}\cdot \bufs\cdot M^{\epsilon/2}>m^{\gamma\epsilon}s)$ for sufficiently small constant $\gamma$. Also remember that $m$ is polynomial on $\Delta$, thus, $\rho{}\Delta^{\lan{}}/9>m^{\gamma\epsilon}\rho{}$ for sufficiently small constant $\gamma$. 
	
	Suppose to the contrary, $G$ has a \ssss{\AB^{10+\lan}\cdot \bufs\cdot M^{\epsilon/2}}{\rho{}\AB^{\lan}/9}, we will make a contradiction. We first claim that $G$ also has a \ssss{\AB^{10+\lan}\cdot \bufs\cdot M^{\epsilon/2}\cdot \AB^{\lan}}{\rho{}\AB^{\lan}/8}, denoted by $E'$, such that both end points for every edge in $E'$ is in the same layer. We can achieve this by taking any edge $(u,v)$ in the original \ss{} that crosses different layers, finding the path from $u$ to $v$ denoted by $p$, and cut $p$ into at most $\AB^\lan$ segments $p_1,p_2,...,p_\ell$ where each segment is in the same layer, and replace $(u,v)\in E'$ by $\Delta^{\lan}$ new edges each connecting two end points of $p_i$ from $i=1$ to $i=\ell$. In this way, the \ss{} increase by a factor of at most $\AB^\lan$. The distance between two vertices increases by at most an additive factor of $O(\AB^\lan)$. Another property of $E'$ is that every edge in $E'$ has both endpoints in $G_v$ for some $v$. That is because different $G_v$ where $v$ in the same layer are not reachable from each other. 
	
	Now with $E'$ added to $G$, we denote the new graph by $G'$. We define an indicator variable $I_{v,i,j}$ to be $1$ if $\distt{G'}{a^v_i,b^v_j}\le \rho{}/3$. We know from the property of type 2 that for any $v$ and arbitrary small constant $\eta$, $\sum_{i,j\in P'}I_{v,i,j}>\eta|P'|$ implies $E'$ has at least $\bufs\cdot M^{\epsilon}$ edges in $G_v$. Since $|E'|\le \AB^{10+\lan}\cdot \bufs\cdot M^{\epsilon/2}\cdot \AB^{\lan}$, by setting $\lan<1/2$, at most $o(\AB^{10+\lan})$ vertices $v$ has the property that $\sum_{i,j\in P'}I_{v,i,j}>\eta|P'|$, which means $\sum_{v,(i,j)\in P'}I_{v,i,j}\le 2\eta\AB^{10+\lan}|P'|$. However, for each critical pair $(i,j)$, suppose $T_{i,j}=\{(v,x,y)\mid (i,j)\text{ covers }v\text{ at }(x,y)\}$, then we have $\sum_{(v,x,y)\in T_{i,j}}I_{v,x,y}\ge (1/2)\AB^\lan$, otherwise the distance $\distt{G'}{s_i,t_j}$ with be at least $(\AB^\lan)/2\cdot \rho{}/3>\rho{}\AB^\lan/6$. We also know that $T_{i,j}$ is disjoint for different $i,j$. Thus, we get $\sum_{v,i,j}I_{v,i,j}=\Omega(\Delta^{10}|P'|)\cdot \AB^\lan/2=\Omega(|P'|\AB^{10+\lan})$, contradicts the fact that $\sum_{v,i,j}I_{v,i,j}\le 2\eta\AB^{10+\lan}|P'|$ for sufficiently small constant $\eta$. 
	\yonggang{I think this proof is very hard to parse and I should try to refine it.}
	
\end{proof}

If the \gadget{\cs{}}{\epsilon} instance is type 1, then according to~\cref{lem:type1}, $G$ admits a \ssss{s=\AB^{10+\lan}\cdot \bufs}{O(\rho{})}, so $\mathcal{A}$ will output a \ssss{n^{\gamma\epsilon}s}{n^{\gamma\epsilon}\rho{}}; on the other hand, if the \gadget{\cs{}}{\epsilon} instance is type 2, then according to~\cref{lem:type1}, $G$ cannot output a \ssss{n^{\gamma\epsilon}s}{n^{\gamma\epsilon}\rho{}}. Since checking whether an edge set is a \ssss{n^{\gamma\epsilon}s}{n^{\gamma\epsilon}\rho{}} or not is in polynomial time, the output of $\mathcal{A}$ distinguish type 1 from type 2.

\end{proof}

\subsection{Lower Bound when \texorpdfstring{$\apxD>1$}{paxDge1} and \texorpdfstring{$s=\omega(m)$}{cs>1}}\label{subsec:largebicriteria}

In this section, we prove the following lemma.

\begin{lemma}\label{lem:larges}
	Assuming \conj{} (\cref{con:pgc}), there exists two constants $0<\delta,\epsilon<1$ such that \gadget{1+\delta}{\epsilon} cannot be solved in polynomial time.
\end{lemma}
by combining~\cref{lem:larges,lem:usegadget}, we can get the proof of~\cref{thm:main}.
\begin{proof}[Proof of~\cref{thm:main}]
	According to~\cref{lem:larges}, under \conj{}, there exists two constants $0<\delta,\epsilon<1$ such that \gadget{1+\delta}{\epsilon} cannot be solved in polynomial time. According to~\cref{lem:usegadget}, there are no polynomial algorithm solving \oss{m^{\gamma\epsilon}}{m^{\gamma\epsilon}} when input $s=\Omega(m^{1+\gamma(1+\delta-1)})=\Omega(m^{1+\gamma\delta})$ for some constant $\gamma$. 
\end{proof}

\begin{proof}[Proof of~\cref{lem:larges}]
	Given a \labcov{} instance $\I=(A,B,E,\L,(\pi_e)_{e\in E})$ described in~\cref{def:labcov}, \conj{} implies we cannot distinguish the following two cases in polynomial time for some small constant $\epsl{}$ according to to~\cref{lem:pgc}. Remember that $N$ is the input bit length of $\I$.
	
	\begin{itemize}
		\item (Completeness:) There is a labeling that covers every edge. 
		
		\item (Soundness:) Any multilabeling of cost at most $N^{\epsl{}}(|A|+|B|)$ covers at most $N^{-\epsl{}}$ fraction of edges.      	
	\end{itemize}
	
	
	Assuming there is a polynomial time algorithm $\mathcal{A}$ solving \gadget{1+\delta}{\epsilon} for sufficiently small constant $\epsilon$, we will show how to distinguish the above two cases in polynomial time, which will lead to a contradiction.
	
	
	\paragraph{Definition of $G$.} First, we construct a graph $G$ (see~\cref{fig:large}) according to the instance $\I$. Let $\Delta=N^{c_{\Delta}}$ for a sufficiently large constant $c_{\Delta}$. Denote the graph described by~\cref{lem:geograph} with parameter $\Delta$ as $G_{geo}$. Remember that $G_{geo}$ has $\Delta^{\lan{}}$ layers, each with size $\Theta(\Delta^{10})$, where the first layer is $S=\{s_1,s_2,...,s_{K_1}\}$, the last layer is $T=\{t_1,t_2,...,t_{K_2}\}$.
	\begin{enumerate}
		\item $G$ has two sides: $A$ side and $B$ side. Each side contains $K_1$ \emph{batches}, each batch is composed by $|A|=|B|$ \emph{fans}. Let the reversed graph of $G_{geo}$ (where we reverse all the edge directions in this graph) as $G^R_{geo}$. On the $A$ side, each fan is composed of $\L$ copies of graph $G^R_{geo}$; on the $B$ side, each is composed of $\L$ copies of a graph $G_{geo}$. All $G^R_{geo}$ or $G_{geo}$ in the same fan share the same $T$ vertex set. On the $A/B$ side, the $k$-th copied graph in the $i$-th batch, $j$-th fan is denoted by $G^{A/B}_{i,j,k}$, where the node $s_\ell$ in $G_{geo}$ is denoted by $s^{A/B}_{(i,j),\ell}$ for $1\le \ell\le K_1$ (recall that for different $k$, they share the same $s_\ell$), and the node $t_\ell$ in $G_{geo}$ is denoted by $t^{A/B}_{(i,j,k),\ell}$ for $1\le \ell\le K_2$. 
		\item $G$ has edges defined by the \labcov{} instance $\I$ from $A$ side to $B$ side described as follows. Remember that $\I=(A,B,E,\L,(\pi_e)_{e\in E})$ where $A=\{1,2,...,|A|\},B=\{1,2,...,|B|\},\L=\{1,2,...,|\L|\}$. For every $(j,j')\in E,(k,k')\in\pi_{(j,j')}$ and $i,\ell\in[K_1]$, there is an edge from $s^A_{(i,j,k),\ell}$ to $s^B_{(\ell,j',k'),i}$. 
		An intuition of why the edges are defined in this way is that, now if a path is from the $A$ side of the $i$-th batch to $B$ side of the $\ell$-th batch, the path must go through $s^A_{(i,j,k),\ell}$ to $s^B_{(\ell,j',k'),i}$ for some $(j,j')\in E,(k,k')\in\pi_{(j,j')}$.
		\item For every $\ell\in[K_2]$, create a node $p^A_\ell$ and edges $\{(t^A_{(i,j),\ell},p^A_\ell)\mid 1\le i\le K_1,1\le j\le |A|)\}$. For every $i\in[K_1]$, create a node $q^B_i$ and edges $\{(q^B_i,t^B_{(i,j),\ell})\mid 1\le j\le |A|,1\le \ell\le K_2\}$. Remember that in~\cref{lem:geograph}, $P$ is a set of pairs between first and last layer indexes of nodes in $G_{geo}$. For every $\ell\in[K_2],i\in[K_1]$ where $(i,\ell)\not\in P$, we create an edge $(p^A_{\ell},q^B_{i})$. 
		
		We also create a mirror of the above nodes and edges (\cref{fig:large} does not draw) by reversing $A$ and $B$. i.e., for every $\ell\in[K_2]$, create a node $p^B_\ell$ with edges $\{(p^B_\ell,t^B_{(i,j),\ell})\mid 1\le i\le K_1,1\le j\le |A|)\}$. For every $i\in[K_1]$, create a node $q^A_i$ with edges $\{(t^A_{(i,j),\ell},q^A_i)\mid 1\le j\le |A|,1\le \ell\le K_2\}$. For every $\ell\in[K_2],i\in[K_1]$ where $(i,\ell)\not\in P$, we create an edge $(q^A_{i},p^B_{\ell})$. 
	\end{enumerate} 
	
	Remember that the number of edges in $G_{geo}$ is $M=O(\Delta^{11+\lan{}})$. Let $n$ be the number of nodes in $G$. Let $m$ be the number of edges in $G$. $m$ can be calculated as
	\[m={\color{gray}M\cdot K_1|A||\L|}+{\color{blue}O(|\L|^2|A|^2)\cdot K_1^2}+{\color{orange}4K_1K_2|A|}+{\color{red}O(K_1K_2)}=\Theta(M\cdot K_1|A||\L|)\]
	The third and fourth terms are trivially less than the first term. The second term is less than the first term since we set $\Delta=N^{c_\Delta}$ for sufficiently large constant $c_\Delta$. 
	\begin{figure}
		
		\begin{center}
			
			\begin{tikzpicture}[x={(0.5cm,0.3cm)}, y={(-0.5cm,0.3cm)}, z={(0cm,0.5cm)}]
				
				\newcommand{\con}{3}
				
				\newcommand{\drawpiece}[4]{
					\coordinate (A) at (0 * #1 + #2, -0.3 + #3, 0 + #4);
					\coordinate (B) at (0 * #1 + #2, 2.3 + #3, 0 + #4);
					\coordinate (A1) at (3 * #1 + #2, 2 + #3, 1 + #4);
					\coordinate (B1) at (3 * #1 + #2, 0 + #3, 1 + #4);
					\coordinate (A2) at (3 * #1 + #2, 2 + #3, 0 + #4);
					\coordinate (B2) at (3 * #1 + #2, 0 + #3, 0 + #4);
					\coordinate (A3) at (3 * #1 + #2, 2 + #3, -1 + #4);
					\coordinate (B3) at (3 * #1 + #2, 0 + #3, -1 + #4);
					\draw[fill=gray!30] (A) -- (B) -- (A3) -- (B3) -- cycle;
					\draw[fill=gray!30] (A) -- (B) -- (A2) -- (B2) -- cycle;
					\draw[fill=gray!30] (A) -- (B) -- (A1) -- (B1) -- cycle;
					
				}
				\newcommand{\drawstack}[5]{
					\fill[gray] (1.5 * #1 + #2,1 + #3,-2 + #5  + #4) circle (1pt);
					\fill[gray] (1.5 * #1 + #2,1 + #3,-2.5 + #5  + #4) circle (1pt);
					\fill[gray] (1.5 * #1 + #2,1 + #3,-3 + #5  + #4) circle (1pt);
					\drawpiece{#1}{#2}{#3}{#4 + #5}
					\drawpiece{#1}{#2}{#3}{#4}
				}
				\newcommand{\drawbatch}[6]{
					\fill[gray] (1.5 * #1 + #2,7 + #3,-1.5) circle (1pt);
					\fill[gray] (1.5 * #1 + #2,7.5 + #3,-1.5) circle (1pt);
					\fill[gray] (1.5 * #1 + #2,8 + #3,-1.5) circle (1pt);
					\drawstack{#1}{#2}{#3 + #6}{#4}{#5}
					\drawstack{#1}{#2}{#3}{#4}{#5}
				}
				
				\coordinate (qB1) at (15, 1.5,-1.5);
				
				\draw  (qB1) node[circle, fill, inner sep=1pt, label={[label distance=-0.1cm, color = orange]right:\fontsize{10}{0}\selectfont$q^B_{1}$},  color=orange] {};
				
				
				\draw[->, >=stealth, color=orange, line width=0.5pt] (qB1) -- (11,-0.3,0);
				\draw[->, >=stealth, color=orange, line width=0.5pt] (qB1) -- (11,2.3,0);
				
				\draw[->, >=stealth, color=orange, line width=0.5pt] (qB1) -- (11,-0.3,-3);
				\draw[->, >=stealth, color=orange, line width=0.5pt] (qB1) -- (11,2.3,-3);
				\coordinate (qB1) at (15, 1.5,-1.5);
				
				\coordinate (qB2) at (15, 1.5 + 3,-1.5);
				\draw  (qB2) node[circle, fill, inner sep=1pt, label={[label distance=-0.1cm, color = orange]above:\fontsize{10}{0}\selectfont$q^B_{2}$},  color=orange] {};
				
				
				\draw[->, >=stealth, color=orange, line width=0.5pt] (qB2) -- (11,-0.3 + 3,0);
				\draw[->, >=stealth, color=orange, line width=0.5pt] (qB2) -- (11,2.3 +  3,0);
				
				\draw[->, >=stealth, color=orange, line width=0.5pt] (qB2) -- (11,-0.3 +  3,-3);
				\draw[->, >=stealth, color=orange, line width=0.5pt] (qB2) -- (11,2.3 +  3,-3);
				
				\drawbatch{-1}{11}{0}{0}{-3}{3}
				
				\draw[blue, dashed, dash pattern=on 3pt off 3pt] (3,0,-6) -- (3,0,1) -- (5 + \con,0,1) -- (5 + \con,0,-6);
				\draw[blue, dashed, dash pattern=on 3pt off 3pt] (3,0.5,-6) -- (3,0.5,1) -- (5 + \con,3,1) -- (5 + \con,3,-6);
				\draw[blue, dashed, dash pattern=on 3pt off 3pt] (3,3,-6) -- (3,3,1) -- (5 + \con,0.5,1) -- (5 + \con,0.5,-6);
				\draw[blue, dashed, dash pattern=on 3pt off 3pt] (3,3.5,-6) -- (3,3.5,1) -- (5 + \con,3.5,1) -- (5 + \con,3.5,-6);
				
				\drawbatch{1}{0}{0}{0}{-3}{3}
				
				\node at (1.5, 1, 0.5) {$G^A_{1,1,1}$};
				
				\node at (1.5, 4, 0.5) {$G^A_{2,1,1}$};
				
				\node at (1.5, 1, -2.5) {$G^A_{1,2,1}$};
				
				\node at (9.5, 1, 0.5) {$G^B_{1,1,1}$};
				
				\draw (3,0,1) node[circle, fill, inner sep=1pt, label={[label distance=-0.1cm, color = blue]right:\fontsize{5}{0}\selectfont$s^A_{(1,1,1),1}$},  color=blue] {};
				
				\draw (3,0.5,1) node[circle, fill, inner sep=1pt, label={[label distance=-0.1cm, color = blue]left:\fontsize{5}{0}\selectfont$s^A_{(1,1,1),2}$},  color=blue] {};
				
				\draw (3,0,0) node[circle, fill, inner sep=1pt, label={[label distance=-0.1cm, color = blue]right:\fontsize{5}{0}\selectfont$s^A_{(1,1,2),1}$},  color=blue] {};
				
				\draw (3,0,-2) node[circle, fill, inner sep=1pt, label={[label distance=-0.1cm, color = blue]right:\fontsize{5}{0}\selectfont$s^A_{(1,2,1),1}$},  color=blue] {};
				
				\draw (3,3,1) node[circle, fill, inner sep=1pt, label={[label distance=-0.1cm, color = blue]left:\fontsize{5}{0}\selectfont$s^A_{(2,1,1),1}$},  color=blue] {};
				
				\draw (0,-0.3,0) node[circle, fill, inner sep=1pt, label={[label distance=-0.1cm, color = blue]above:\fontsize{5}{0}\selectfont$t^A_{(1,1),1}$},  color=blue] {};
				
				
				\draw (8,0,1) node[circle, fill, inner sep=1pt, label={[label distance=-0.1cm, color = blue]right:\fontsize{5}{0}\selectfont$s^B_{(1,1,1),1}$},  color=blue] {};
				\draw (11,-0.3,0) node[circle, fill, inner sep=1pt, label={[label distance=-0.1cm, color = blue]right:\fontsize{5}{0}\selectfont$t^B_{(1,1,1),1}$},  color=blue] {};
				
				\coordinate (pA1) at (-4,0,-1.5);
				
				\draw  (pA1) node[circle, fill, inner sep=1pt, label={[label distance=-0.1cm, color = orange]left:\fontsize{10}{0}\selectfont$p^A_{1}$},  color=orange] {};
				
				
				\draw[->, >=stealth, color=orange, line width=0.5pt] (0,-0.3,0) -- (pA1);
				\draw[->, >=stealth, color=orange, line width=0.5pt] (0,2.7,0) -- (pA1);
				
				\draw[->, >=stealth, color=orange, line width=0.5pt] (0,-0.3,-3) -- (pA1);
				\draw[->, >=stealth, color=orange, line width=0.5pt] (0,2.7,-3) -- (pA1);
				
				
				\coordinate (pA2) at (-4,2,-1.5);
				
				\draw  (pA2) node[circle, fill, inner sep=1pt, label={[label distance=-0.1cm, color = orange]below:\fontsize{10}{0}\selectfont$p^A_{2}$},  color=orange] {};
				
				
				\draw[->, >=stealth, color=orange, line width=0.5pt] (0,0.2,0) -- (pA2);
				\draw[->, >=stealth, color=orange, line width=0.5pt] (0,3.2,0) -- (pA2);
				
				\draw[->, >=stealth, color=orange, line width=0.5pt] (0,0.2,-3) -- (pA2);
				\draw[->, >=stealth, color=orange, line width=0.5pt] (0,3.2,-3) -- (pA2);
				\fill[gray] (-4,3,-1.5) circle (1pt);
				\fill[gray] (-4,3.5,-1.5) circle (1pt);
				\fill[gray] (-4,4,-1.5) circle (1pt);
				
				
				\coordinate (start) at (-4,0,-1.5);
				\coordinate (end) at (15,4.5,-1.5);
				
				\coordinate (cp1) at (-3,-4,-1.5);
				\coordinate (cp2) at (12,-5,-1.5);
				\coordinate (cp6) at (16,0,-1.5);
				
				\draw[->, color=red, line width=0.5pt] plot[smooth, tension=0.5] coordinates{(start) (cp1) (cp2) (cp6) (end)};
				
				
				
			\end{tikzpicture}
		\end{center}
		\caption{$G^A_{i,j,k}$ is one copy of the graph described in~\cref{lem:geograph} in the $i$-th batch (from right to left), $j$-th fan (from top to bottom) and $k$-th piece. For different $k$ and fixed $i,j$, $G^A_{i,j,k}$ shares the same first layer nodes (which are $s^A_{(i,j),\ell}$), but have different last layer nodes (which are $t^A_{(i,j,k),\ell}$). By changing $A$ to $B$ we get another side of the graph. Each dashed rectangle specifies the graph similar to the middle part in~\cref{fig:minrepgraph} according to the input \labcov{} instance $\I$. Fix $j,k$, for any $i,\ell$, we have $t^A_{(i,j,k),\ell}$ and $t^B_{(\ell,j,k),i}$ in the same dashed rectangle. }\label{fig:large}
	\end{figure}
	
	
	\paragraph{Solve \labcov{} using \gadget{1+\delta}{\epsilon}} Now we run the \gadget{1+\delta}{\epsilon} algorithm $\mathcal{A}$ with the following inputs. We need to verify that the inputs satisfy the requirements specified by~\cref{def:gadget}.
	\begin{enumerate}
		\item Graph $G$ with $m$ edges. The diameter of $G$ is $d=2\Delta^{\lan{}}$, which is polynomial on $m$.
		\item Two sets $L=\{t^A_{(i,j),k}\mid 1\le i\le K_1,1\le j\le |A|,1\le k\le |\L|\}, R=\{t^B_{(i,j),k}\mid  1\le i\le K_1,1\le j\le |A|,1\le k\le |\L|\}$. The size of each set is $K_1|A||\L|$ which is polynomial on $m$.
		\item A set of reachable vertex pairs $P'\subseteq L\times R$ defined as follows. Recall that in the \labcov{} instance, we have $A=\{1,...,|A|\},B=\{1,...,|B|\}$, and $P$ is the set of pairs defined in~\cref{lem:geograph}. For every $i\in[K_1]$ let $I_{i}$ contain all indexes $x\in[K_2]$ such that $(i,x)\in P$. Let $I_{i}[\ell]$ be the $\ell$-th element in $I_{i}$. 
		Now we define our $P'$.
		\[P'=\left\{\left(t^A_{(i,j),I_{i'}[\ell]},t^B_{(i',j'),I_{i}[\ell]}\right)\mid 1\le i,i'\le K_2,(j,j')\in E,1\le\ell\le\min(|I_{i'}|,|I_{i}|)\right\}\]
		We need to argue that $P'$ only contains reachable pairs. Notice that $t^A_{(i,j),I_{i'}[\ell]}$ can reach $s^A_{(i,j,k),i'}$ for any $k$ since $(i',I_{i}[\ell])\in P'$ (recall that $G^A_{(i,j,k)}$ is the reversed graph $G^R_{geo}$); for the same reason $t^B_{(i',j'),I_{i}[\ell]}$ can be reached from $s^B_{(i',j',k'),i}$ for any $k'$. Now we only need to argue that there exists $k,k'$ such that $t^A_{(i,j,k),i'}$ can reach $s^B_{(i',j',k'),i}$. We can take the $(k,k')\in \pi_{(j,j')}$ where $\pi_{(j,j')}$ must be non-empty since $(j,j')\in E$.
	\end{enumerate}
	Remember that the output of $\mathcal{A}$ will distinguish the following two types of instances. 
	\begin{description}
		\item[Type 1.] There exists a \ss{} $E'$ of $G$ with size $O(m^{1+\delta})$ such that all reachable pairs $(u,v)\in L\times R$ have distance $O(1)$ after adding $E'$ to $G$. %
		\item[Type 2.] By adding any \ss{} with size $O(m^{1+\delta+\epsilon})$, at most $o(1)$ fraction of pairs in $P$ have distance at most $d/3$.  
	\end{description}
	
	Next we show the output can already distinguish the \labcov{} instance (completeness) from (soundness).
	
	\paragraph{(Completeness) implies Type 1.} %
	Suppose the \mlab{} covering all edges in (completeness) is $\psi$. Recall that $P'$ is the pair set described in~\cref{lem:geograph}. We create a \ss{} $E'$ defined as
	\begin{equation*}
		\begin{aligned}
			E' &= \{(t^A_{(i,j),\ell},s^A_{(i,j,k),\ell'}) \mid i\in[K_1],j\in[|A|],k\in\psi(j),(\ell',\ell)\in P'\} \\
			&\cup \{(s^B_{(i,j,k),\ell},t^B_{(i,j),\ell'}) \mid i\in[K_1],j\in[|B|],k\in\psi(j),(\ell,\ell')\in P'\}
		\end{aligned}
	\end{equation*}
	We have $|E'|=O(K_1|A|\Delta^{12})$. Remember that $m=\Theta(M\cdot K_1|A||\L|)$, we have $|E'|=\Theta(m^{1+\delta})$ for some constant $\delta$. Now we prove that all reachable pairs $(t^A_{(i,j),\ell},t^B_{(i',j'),\ell'})$ has distance $O(1)$ after adding $E'$. If $(i',\ell)\not\in P$ or $(i,\ell')\not\in P$, then $t^A_{(i,j),\ell}$ can reach $t^B_{(i',j'),\ell'}$ in $3$ steps. Thus, we only consider the case when $(i',\ell),(i,\ell')\in P$. If $(j,j')\not\in E$, then $(t^A_{(i,j),\ell},t^B_{(i',j'),\ell'})$ is not reachable. Suppose $(j,j')\in E$, let $k\in\psi(j),k'\in\psi(j')$, since $(j,j)$ is covered by $\psi$, there is a blue edge $(s^A_{(i,j,k),i'},s^B_{(i',j',k'),i})$. Besides, we have $(t^A_{(i,j),\ell},s^A_{(i,j,k),i'}),(s^B_{(i',j',k'),i},t^B_{(i',j'),\ell'})\in E'$ due to the fact that $(i',\ell),(i,\ell')\in P$. Thus, the distance between $(t^A_{(i,j),\ell},t^B_{(i',j'),\ell'})$ is $3$. 
	
	\paragraph{(Soundness) implies Type 2.} We will prove it by contradiction. Suppose there exists a \ss{} $E'$ with size $O(m^{1+\delta+\epsilon})=O(K_1|A|\Delta^{12}\cdot n^\epsilon)$, after adding which, not $o(1)$ fraction of pairs in $P$ have distance at most $d/3=(2/3)\Delta^{\lan{}}$. We first turn $E'$ into another \ss{} $E''$ where each shortcut in $E''$ is totally in a copy of graph $G_{geo}$ in the following way. Suppose $(u,v)\in E'$ where $u,v$ are not in the same copy of $G_{geo}$, a path from $u$ to $v$ must cross at most two copies of $G_{geo}$, one on thie $A$ size and another on the $B$ size. We split this path into the two copies, and creat two edges connected two end points of both path
	. $E''$ will have size twice of $E'$, and the distance after adding $E''$ will be at most $(2/3)\Delta^{\lan{}}+2$.
	
	Recall that for every $i\in[K_1]$, we defined $I_{i}$ as all indexes $x\in[K_2]$ such that $(i,x)\in P$. We create \mlab{} $\psi_{i,i',\ell}$ in the following way: for every $j\in A,j'\in B$, we let 
	\[\psi_{i,i',\ell}(j)=\left\{k\in\L\mid\distt{G+E''}{t^A_{(i,j),I_{i'}[\ell]},s^A_{(i,j,k),i'}}<\Delta^{\lan{}}\right\}\]
	\[\psi_{i,i',\ell}(j)=\left\{k\in\L\mid\distt{G+E''}{s^B_{(i',j,k),i},t^B_{(i',j),I_{i}[\ell]}}<\Delta^{\lan{}}\right\}\]
	
	One importance observation is, for every $\psi_{i,i',\ell}$ and $k\in\L$, there is a unique path from $t^A_{(i,j),I_{i'}[\ell]}$ to $t^A_{(i,j,k),i'}$ or from $s^B_{(i',j',k),I_i[\ell]}$ to $t^B_{(i',j'),\ell}$; any two of these unique paths shares at most one edge. The reason is $(i,I_i[\ell])\in P$ (see~\cref{lem:geograph}). As a result, each edge in $E''$ can make at most one pair of vertices $(t^A_{(i,j),I_{i'}[\ell]},s^A_{(i,j,k),i'})$ or $(s^B_{(i',j,k),i},t^B_{(i',j),I_{i}[\ell]})$ to have distance less that the original distance $\Delta^{\lan{}}$. Therefore,
	\[\sum_{i,i'\in[K_1],\ell\le \min(|I_{i'}|,|I_{i}|)}\left(|\psi_{i,i',\ell}(j)|+|\psi_{i,i',\ell}(j)|\right)\le |E''|=O(K_1|A|\Delta^{12}\cdot n^\epsilon)\]
	
	According to~\cref{lem:geograph}, we know $\min(|I_{i'}|,|I_{i}|)=\Omega(\Delta^2)$, which means the number of \mlab{} $\psi_{i,i',\ell}$ is $T=\Theta(K_1^2\Delta^2)=\Theta(K_1\Delta^{12})$. Thus, only $o(K_1\Delta^{12})$ of them will have size at least $N^{\epsl{}}(|A|+|B|)$ for sufficiently small constant $\epsilon$, where $N$ is the input size of $\I$ which is polynomial on $n$. Let $C(\psi_{i,i',\ell})$ be the number of edges covered by $\psi_{i,i',\ell}$ for instance $\I$. We have 
	\[\sum_{i,i',\ell}C(\psi_{i,i',\ell})\le o(K_1\Delta^{12})\cdot |E|+T\cdot o(|E|)\le o(T|E|)\]
	
	One can see that if $\psi_{i,i',\ell}$ does not cover an edge $(j,j')$, then the distance from $t^A_{(i,j),I_{i'}[\ell]}$ to $t^B_{(i',j),I_{i}[\ell]}$ is at least $\Delta^{\lan{}}>(2/3)n^{c_D}$, where $(t^A_{(i,j),I_{i'}[\ell]},t^B_{(i',j),I_{i}[\ell]})\in P'$. That is because the only path from $t^A_{(i,j),I_{i'}[\ell]}$ to $t^B_{(i',j),I_{i}[\ell]}$ must be the concatenation of paths between $t^A_{(i,j),I_{i'}[\ell]},s^A_{(i,j,k),i'}$ and between $s^B_{(i',j,k'),i},t^B_{(i',j),I_{i}[\ell]}$ for some $k,k'\in\L$, where at least one of them has length $\Delta^{\lan{}}$. Therefore, we have at least $(1-o(1))T|E|=(1-o(1))|P'|$ pairs in $P'$ that have distance at least $\Delta^{\lan{}}$, which is a contradiction.%
	
\yonggang{the notations in this proof are disasters, I need to refine it}
\end{proof}



\section{Upper Bounds}\label{sec:upperbound}
We use the algorithm idea of a previous spanner approximation algorithm~\cite{BermanBMRY13}. We explain our algorithm in the language of the shortcut problem. 

\subsection{Preliminaries}



\paragraph{Large diameter case:} If we aim for a relatively large diameter, i.e., $d \cdot \alpha_D \geq n^{0.34}$, we can simply use the known result. 

\begin{theorem}[\cite{KoganP22}]
There is an efficient algorithm that, given input graph $G$, computes a $(d,s)$-shortcut for $d = \tilde{O}(n^{1/3})$ and $s = \tilde{O}(n)$.    
\end{theorem}

Using the above theorem, if $\apxD{}d=\Omega(n^{0.34})$, we can easily achieve $(\alpha_D, \alpha_S)$-approximation for all $\alpha_S = \tilde{O}(1)$. 
Therefore, we make the following assumption throughout this section. 

\begin{remark}\label{rem:assumption}
We can assume w.l.o.g. that $\alpha_D d = O(n^{0.34})$. 
\end{remark}

\paragraph{Reduction to DAGs:} We argue that DAGs, in some sense, capture hard instances for our problems. This will allow us to focus on DAGs in the subsequent sections.
The formal statement is encapsulated in the following lemma. 

\begin{lemma}
If there exists an efficient $(\alpha_D,\alpha_S)$ approximation algorithm for DAGs, then there exists an efficient $(3\alpha_D, 3\alpha_S)$-approximation algorithm for all directed graphs.  
\end{lemma}
\begin{proof}
Assume that we are given an access to the algorithm $\aset(G,s,d)$ that produces $(\alpha_D,\alpha_S)$ approximation algorithm for the DAG case. 
Let $G$ be an input digraph, together with the input parameters $(d,s)$.  
Compute a collection ${\mathcal S}$ of strongly connected components (SCC) of $G$, and let $G'$ be the DAG obtained by contracting each SCC into a single node. Invoke the algorithm $\aset(G',s,d)$. Let $E' \subseteq E(G')$ be the shortcut edges so that $|E'| \leq \alpha_S \cdot s$ and the diameter of $G' \cup E'$ is at most $\alpha_D \cdot d$. 
These edges would be responsible for connecting the pairs whose endpoints are in distinct SCCs. 

For each SCC $C \in {\mathcal S}$, we pick an arbitrary ``center'' $u_C \in V(C)$ and connect it to every other vertex in $V(C)$. These edges are called $E''_C$. Define $E'' = \bigcup_{C \in {\mathcal S}} E''_C$. 
The final shortcut $F$ can be constructed by combining these two sets $E'$ and $E''$: Edges in $E''$ can be added into $F$ directly. For each edge that connects $C$ to $C'$ in $E'$, we create the corresponding edge $(u_C, u_C')$ connecting the centers. Notice that $|F| \leq |E'| + |E''| \leq \alpha_S \cdot s + 2n \leq 3\alpha_S \cdot s$ (here we used the assumption that $s \geq n$). Moreover, it is easy to verify that, for each reachable pair $(v,w)$ in $G$ where $v \in C$ and $w \in C'$, there is a path from $v$ to $w$ of length at most $3 \alpha_D \cdot d$.     
\end{proof}




\subsection{Overview} \label{sec:ub-overview}

Suppose $G=(V,E)$ is a directed acyclic graph and $G^T=(V,E^T)$ be its transitive closure, i.e., $(u,v)\in E^T$ if $u$ has a directed path with length at least $1$ to $v$ in $G$. Since $G$ is acyclic, $G^T$ contains no self loop. 

\begin{definition}[Local graphs]\label{def:localgraph}
For a pair $u,v \in V(G)$, we define the local graph $G^{u,v}$. 
 Let $G^{u,v}=(V^{u,v},E^{u,v})$ be the subgraph of $G^T$ induced by the vertices that can reach $v$ and can be reached from $u$ (i.e., these vertices lie on at least one path from $u$ to $v$). 
\end{definition}
\begin{definition}[Thick and thin pairs]\label{def:thickthinedge}
 Let $u,v \in V(G)$. 
 If $|V^{u,v}|\ge \beta$ ($1\le \beta\le n$ will be determined later), the corresponding edge $(s,t)$ is said to be $\beta$-thick, and otherwise, it is $\beta$-thin. When $\beta$ is clear from context, we will simply write thick and thin pairs respectively. 
\end{definition}

Denote by $\pset$ the set of pairs of vertices that are reachable in $G$. We can partition $\pset$ into $\pset_{thick} \cup \pset_{thin}$. 

\begin{definition}
	Let $d' \in {\mathbb N}$. A set $E'\subseteq E^T \setminus E$ is said to $d'$-settle a pair $(u,v)\in E$ if $(V,E \cup E')$ contains a path of length at most $d'$ from $u$ to $v$. 
\end{definition}


Our algorithm will find two edge sets $F_1,F_2\subseteq E^T \setminus E$ such that the set $F_1$ is responsbile for $(\apxD{} d)$-settling all thick pairs, while $F_2$ will $(\apxD{} d)$-settle all thin pairs. The final solution be $F_1\cup F_2$ (Notice that $F_1\cup F_2$ $(\apxD{} d)$-settles all the pairs.)  These are encapsulated in the following two lemmas: 

\begin{restatable}[thick pairs]{lemma}{settlethick}\label{lem:settlethickedges}
We can efficiently compute $F_1$ such that $|F_1| \le O\left(\frac{n^2\log^2 n}{\beta\apxD{}^2 d^2}+
n \log n
\right)$ and it $(\apxD{} d)$-settles all thick pairs w.h.p. 
\end{restatable}

\begin{lemma}[thin pairs] \label{lem:settlethinpairs}
We can efficiently compute $F_2$ such that $|F_2| \leq O\left(\frac{\beta \log^2 n s}{\alpha_D}\right)$ and $(\alpha_D d)$-settles all thin pairs with high probability.      
\end{lemma}

We will prove these two lemmas later. Meanwhile, we complete the proof of Theorem~\ref{thm:upperbound}. 
We minimize the sum $|F_1| + |F_2|$ by setting the value of $\beta=\frac{n}{d\sqrt{s\apxD{}}}$.\footnote{The only problem is that this value could be much less than $1$ when $d\sqrt{s\apxD{}}>n$, which leads to $(d\apxD{})^2s>n^2$. However, in that case,  we can use the known tradeoff~\cite{KoganP22} to construct a \ssss{s}{\apxD{}d} when $d\sqrt{s\apxD{}}>n$.} 
	Now we have 
 \[|F_1\cup F_2|=O\left(\frac{ns^{0.5}\log^{2}n}{d\apxD{}^{1.5}}+n\log n\right)\le s\cdot O\left(\frac{n\log^{2}n}{d\apxD{}^{1.5}s^{0.5}}\right).\]

\subsection{Settling the thick pairs}\label{subsec:thickedges}

In this Section, we prove Lemma~\ref{lem:settlethickedges}.  For convenience we assume $\apxD{}d=\omega(\log n)$. We will show the case when $\apxD{}d=O(\log n)$ later.
We will use the idea from~\cite{KoganP22} to construct $F_1$. First, the following lemma allows us to decompose a DAG into a collection of paths and independent sets. 

\begin{lemma}[Theorem 3.2 \cite{GrandoniILPU21}]
	There is a polynomial time algorithm given an $n$-vertex acyclic graph $G=(V,E)$ and an integer $k \in[1,n]$, partition $G$ into $k$ directed paths  $P_1,...,P_{k}$ and at most $2n/k$ independent set  $Q_1,...,Q_{2n/k}$ in $G^T$. In other words, $P_1,...,P_{k},Q_1,...,Q_{2n/k}$ are disjoint and $\left(\cup_{i\in[k]}P_i\right)\cup \left(\cup_{i\in[2n/k]}Q_i\right)=V$.
\end{lemma}

We first apply the following lemma with $k=n/(\apxD{} d)$ to get $P_1,...,P_{8n/(\apxD{} d)}$ and $Q_1,...,Q_{\apxD{} d/4}$. Notice that $\apxD d=\omega(\log n)$ and $\apxD d= O(n^{0.34})$, we can safely assume $8n/(\apxD{} d)$ and $\apxD{} d/4$ are integers without loss of generality.
We will use Lemma 1.1~\cite{Raskhodnikova10}.
\begin{lemma}[Lemma 1.1~\cite{Raskhodnikova10}]\label{lem:pathshortcut}
    For any integer $n\ge 3$, the directed path with length $\ell$ has a $2$-shortcut with at most $\ell\log \ell$ edges.
\end{lemma}

We first add $n\log n$ edges to $F_1$ and reduce the diameter of every path $P_i$ to $2$. To accomplish this, for each path, we use \cref{lem:pathshortcut}. Since different paths are disjoint regarding vertices, $n\log n$ edges suffice.


Next, let $R \subseteq V$ be obtained by sampling $\min\left((9\log n)\cdot n/\beta, n\right)$ vertices from $V$ uniformly at random. The algorithm also samples $\min\left((999\log n)\cdot n/(\apxD{} d)^2,n/(\apxD{} d)\right)$ paths from $P_1,...,$ $P_{n/(\apxD{} d)}$ uniformly at random; let ${\mathcal Q}$ denote the set of sampled paths. For each vertex $u\in R$ and path $p\in {\mathcal Q}$, add $(u,v_1)$ to $F_1$ where $v_1$ is the first node in $p$ that $u$ can reach (if it exists), and add $(v_2,u)$ to $F_1$ where $v_2$ is the last node in $p$ that can reach $u$ (if it exists). Observe that $\beta$ is between $1$ and $n$, $(\apxD d)$ is between $\omega(\log n)$ and $O(n^{0.34})$, so both $(9\log n)\cdot n/\beta$ and $(999\log n)\cdot n/(\apxD{} d)^2$ will not be too small, and we can safely assume they are integers without loss of generality.




\settlethick*
\begin{proof}
It is straightforward to verify that $|F_1| \le (999n^2\log^2 n)/(\beta\apxD{}^2 d^2)+n\log n$ by the construction.

Suppose $(s,t)\in E^T$ is a thick pair. Since $|V^{s,t}|\ge \beta$, a vertex $u\in R\cap V^{s,t}$ exists with high probability (w.h.p.). %
Let $G'$ be the graph obtained after adding all edges that reduce the diameter for each $P_i$. Denote the shortest path between $s$ and $t$ in $G'$ as $P_{s,t}$. 
The path $P_{s,t}$ intersects each path $P_i$ at no more than three nodes. Otherwise, there are four nodes in $P_i\cap P_{s,t}$ $v_1,v_2,v_3,v_4$ such that $v_1$ has distance at least $3$ to $v_4$ since $P_{s,t}$ is a shortest path, which contradict the fact that $P_i$ has diameter $2$. The path $P_{s,t}$ intersects each $Q_i$ at no more than one node since $Q_i$ is an independent set on $E^T$.
Therefore, if we disregard all nodes in $Q_i$ from $i=1$to $i=\apxD{} d/ 4$ (which incurs an additional $\apxD{} d/4$ steps) and examine the first $\apxD{} d/4$ vertices and the last $\apxD{} d/4$ vertices of $P_{s,t}$, both of them will intersect a path in ${\mathcal Q}$ w.h.p. since we sample $(999\log n)\cdot n/(\apxD{} d)^2$ paths into ${\mathcal Q}$ among all $8n/(\apxD d)$ paths. This implies that $s$ can first use a $\apxD{} d/4+\apxD{} d/9+1$ path to reach $u$, and then $u$ can use another $1+\apxD{} d/9+\apxD{} d/4$ path to reach $t$. 
\end{proof}

For the case when $\apxD{}d=O(\log n),\apxD{}d>1$, to get similar result as~\cref{lem:settlethickedges}, we want $|F_1|=O(n^2\log n/\beta+n)$. This is easy to construct by sampling $(n/\beta)\log n$ nodes, where w.h.p. one of them will be in $V^{s,t}$ for every thick pair $s,t$. For every sample node $u$, we just need to include $(v,u)$ to $F_1$ for every $v$ that can reach $u$, and $(u,v)$ to $F_1$ for every $v$ that $u$ can reach. Now every thick pair has distance $2$ in $F_1$. 

For the extreme case when $\apxD{}d=1$, their is a unique way of adding edges, which is connecting every reachable pair by one edge, thus, we ignore this situation.

\subsection{Settling the thin pairs}\label{subsec:thinedges}



\begin{definition}[Critical sets]\label{def:antispanner}
A set $A\subseteq E^T$ is a $k$-critical set of a pair $(u,v)\in E^T$ if $E^T\backslash A$ contains no path from $uv$ to $v$ with a length of at most $k$. If there does not exist an $A'\subset A$ such that $A'$ is also a $k$-critical, then we say $A$ is minimal.
\end{definition}


\begin{definition}[$\mathcal{A}_k$]\label{def:antispannerset}
Let $\mathcal{A}_k$ consist of all sets $A$ satisfying both (i) $A$ is a minimal critical set of some thin pair, and (ii) $A\cap E=\emptyset$.
\end{definition}

This gives an alternate characterization of a shortcut set. 

\begin{claim}[Adapted from \cite{BermanBMRY13}] 
    An edge set $E'$ is a $d$-shortcut set for all thin pairs if and only if $E' \cap A \neq \emptyset$  for all $A\in\mathcal{A}_{d}$. 
\end{claim}

\begin{proof}
We briefly sketch the proof of this claim below.
\begin{itemize}
    \item[($\Rightarrow$)] We first argue that $E'$ must intersect each $A\in\mathcal{A}_d$ with at least one edge. Suppose, on the contrary, that $E'$ is disjoint from a set $A\in\mathcal{A}_d$, which is a $d$-critical set of some pair $(u,v)$. Then, $E'\cup E\subseteq E^T \backslash A$, and the distance between $u$ and $v$ in $E'\cup E$ is larger than $d$, which contradicts our assumption. 
    
    \item[($\Leftarrow$)] On the other hand, assume that $E'$ intersects every $A\in\mathcal{A}_d$ with at least one edge then it must be a $d$-shortcut. Otherwise, there exists a pair $(u,v)\in E^T $ such that, in $E'\cup E$, $u$ has a distance larger than $d$ to $v$. This implies that $A= E^T\backslash(E'\cup E)$ is a $d$-critical set of $u,v$ and $A \cap E' = \emptyset$ (a contradiction). 
\end{itemize}
\end{proof}


We define the following polytope $\cP_{G,d}$. It includes at most $n^2$ variables: for each $e\in E^T$, there is a corresponding variable $x_e$. The polytope has an exponentially large number of constraints. However, it must be non-empty since we assume $G$ admits a \ssss{s}{d}.


\begin{align*}
	\text{Polytope }\cP_{G,d}: \qquad\sum_{e\in E^T\backslash E}x_e &\le s \\
	\sum_{e\in A}x_e &\ge 1\qquad \forall A\in \mathcal{A}_d\\
	x_e &\geq 0\qquad \forall e\in E^T
\end{align*}

The algorithm employs the cutting-plane method to attempt to find a point within the polytope. According to established analyses, the running time of the cutting-plane method is polynomial with respect to the number of variables, provided that a separation oracle with a polynomial running time exists based on the number of variables. A separation oracle either asserts that the point lies within the polytope or returns a violated constraint, which can function as a cutting plane. Since the constraints $\sum_{e\in E^T\backslash E}x_e\le s$ and $x_e\ge 0$ can be verified in polynomial time, identifying a cutting plane is straightforward if either of these constraints is violated. Consequently, we assume $\sum_{e\in E^T\backslash E}x_e\le s$ and $x_e\ge 0$.

In the subsequent algorithm, we accept a point as input and either return a non-trivial violated constraint (one of $\sum_{e\in A}x_e\ge 1$), which can act as a separation plane, or output a set $E_2$ that settles all thin pairs.




\begin{algorithm}[H]
	
	\caption{{\sc Cut-or-Round}$({\bf x})$}\label{alg:seperationoracle}
	
	\KwData{A vector ${\bf x} \in [0,1]^{E^T \setminus E}$ satisfying $\sum_{e\in E^T\backslash E}x_e\le s$ and $x_e\ge 0$.}
	\KwResult{A set $A \in  \mathcal{A}_d$ s.t. $\sum_{e\in A}x_e<1$, or a set $F_2\subset E^T \setminus E$ that $(\apxD{} d)$-settles all thin pairs.}
	Include  edge edge $e\in E^T$ independently into $F_2$ with probability $(500\log n)(\beta/\apxD{})x_e$\;
	
	\If{all thin pairs are $(\apxD{} d)$-settled}{\If{$|F_2\backslash E|\le (1000\log^2 n)(\beta/\apxD{})s$}{\Return{$F_2$}\;}\Else{\Return{fail}\;}}
	\Else{
		Use $F_2$ to find a $(\apxD{} d)$-critical set $A'\in\mathcal{A}_{\apxD{} d}$ with $F_2\cap A'=\emptyset$ using the algorithm from Claim 2.4 in~\cite{BermanBMRY13}\;
		\If{$\sum_{e\in A'}x_e<\apxD{}/9$}{
			Find a $d$-critical set $A\subseteq \mathcal{A}_d$  such that $\sum_{e\in A}x_e\ge 1$ using~\cref{lem:findantispanner}\;
			\Return violated constraint $\sum_{e\in A}x_e\ge 1$\;}
		\Else{\Return{fail}\;}
	}
\end{algorithm}

\begin{claim}\label{lem:notfail}
	The algorithm will fail with probability at most $\frac{1}{n^{\omega(1)}}$.
\end{claim}
\begin{proof}
	The first fail condition is $|F_2\backslash E|>(1000\log^2 n)(\beta/\apxD{})s$. Notice that each edge $e$ is included in $F_2$ independently with probability $(500\log n)(\beta/\apxD{})x_e$ where $\sum_{e\in E^T\backslash E}x_e\le s$. By chernoff bound, $|F_2\backslash E|>(1000\log^2 n)(\beta/\apxD{}) s$ happens with probability at most $e^{-(\log n)(\beta/\apxD{}) s}$. Remember that $s\ge n$ and $\apxD{}=O(n^{0.34})$, which means we have $e^{-(\log n)(\beta/\apxD{}) s}\le n^{-\omega(1)}$.
	
	Now we show that the second fail condition happens with small probability. We define event $\mathcal{E}$ as ``all $B\subseteq A_{\apxD{} d}$ with $\sum_{e\in B}x_e\ge \apxD{}/9$ satisfies $B\cap F_2\not=\emptyset$''. If the second fail is triggered, then there exists a set $A'\in\mathcal{A}_{\apxD{} d}$ with $\sum_{e\in A'}x_e\ge \apxD{}/9$ satisfies $A'\cap F_2=\emptyset$, which means $\mathcal{E}$ does not happen. Thus, the probability that the second fail is triggered is bounded by $1-\Pr[\mathcal{E}]$. For $B\subseteq A_{\apxD{} d}$ with $\sum_{e\in B}x_e\ge \apxD{}/9$, the probability that $B\cap F_2=\emptyset$ is bounded by $\exp\left(-9\beta\log n\right)$ according to Chernoff bound (remember that each edges is included in $F_2$ with probability $(500\log n)(\beta/\apxD{})x_e$). Now we count the size of $\mathcal{A}_{\apxD{} d}$. According to claim 2.5 in~\cite{BermanBMRY13}, we have $|\mathcal{A}_{\apxD{} d}|\le |E|\cdot \beta^\beta\le \exp\left(2\beta\log\beta\right)$. Finally, by using union bound, $\Pr[\mathcal{E}]\ge 1-\frac{1}{n^{\omega(1)}}$.
\end{proof}
\begin{lemma}[critical set decomposition]\label{lem:findantispanner}
	There exists a polynomial time algorithm that, given $A'\in\mathcal{A}_{\apxD{} d}$ with $\sum_{e\in A'}x_e< \apxD{}/9$, outputs $A\in\mathcal{A}_{d}$ such that $\sum_{e\in A}x_e< 1$. 
\end{lemma}
\begin{proof}
	The algorithm first use polynomial time to find $(s,t)\in E^T$ such that $A'$ is a $(\apxD{} d)$-critical set of $(s,t)$. Then the algorithm constructs a shortest path tree rooted at $s$ on the subgraph $E^T\backslash A'$, where all the nodes in the $i$-th laryer of the tree has distance $i$ from $s$. Denote the vertex set of the $i$-th layer as $L_i$. According to the definition of critical set, $t$ is on at least the $(\apxD{} d+1)$-th layer. The algorithm devides the first $\apxD{} d$ layers into at least $\apxD{}/3$ batches, where the $i$-th batch contains all vertices between layer $2(i-1)d$ to $2i\cdot d-1$. Denote the set of all the edges in $A'$ that has at least one end point in the $i$-th batch as $A_i$. Since each edge in $A'$ will be included in at most two $A_i$, it is easy to see that at least one of $A_i$ has the property $\sum_{e\in A_i}x_e<1$. 
	
	Now we prove that for any $i$, $A_i\in\mathcal{A}_d$. Since $A_i$ is a subset of $A'$, the second condition, i.e., $A_i\cap E=\emptyset$ is satisfied trivially. We only need to verify that $A_i$ is a $d$-critical set of some edge in $E^T$. The idea is to choose a vertex in $L_{2(i-1)d}$ and another vertex in $L_{2id-1}$, and argue that they are reachable and have distance more than $d$ in $E^T\backslash A_i$. Let $S$ contain all the vertices in $L_{2(i-1)d}$ that has at least one edge towards a vertex in $L_{2id-1}$. Notice that $G$ (and also $G^T$) is acyclic (\cref{rem:assumption}), which also means the induced subgraph $G^T[S]$ is acyclic. Thus, there exists a vertex $u\in S$ such that it has no edge in $E^T$ towards other vertices in $S$. Take an arbitrary vertex $v\in L_{2id-1}$ such that $(u,v)\in E^T$, now we argue that $u$ has distance more than $d$ to $v$ in $E^T\backslash A_i$.  
	
	We prove it by induction. Induction hypothesis: any vertex that $u$ has distance $x$ to in $E^T\backslash A_i$ is one of the following two types (i) a vertex that cannot reach $v$ in $G$ (ii) a vertex in $L_{2(i-1)d+x}$. If the induction hypothesis is correct for any $0\le x\le d+10$, then $u$ cannot have distance at most $d$ to $v$ in $E^T\backslash A_i$. Now we prove the induction hypothesis. When $x=0$, the hypothesis is correct. For $x>0$, suppose $w_2$ is a vertex with $\distt{E^T\backslash A_i}{u,w_2}=x$, then there exists an edge $(w_1,w_2)\in E^T\backslash A_i$ such that $\distt{E^T\backslash A_i}{u,w_1}=x-1$. According to induction hypothesis, we can assume $w_1$ is either type (i) or type (ii). If $w_1$ is type (i), then $w_2$ cannot reach $v$ in $G$, which means $w_2$ is also type (i). Now suppose $w_1$ is type (ii) and can reach $v$ in $G$. First of all, $w_2$ cannot be in the first $2(i-1)d$-th layer, otherwise $w_2$ has a path $p$ in $G$ to $v$ where $p$ must contain a vertex in $L_{2(i-1)d}$ (two consecutive vertices in $p$ cannot skip a layer since all edges in $p$ is in $E$, which is also in $E^T\backslash A'$), which means $u$ has an edge in $E^T$ to another vertex in $L_{2(i-1)d}$, leading to a contradiction. Then, $w_2$ cannot be a vertex in $L_{2(i-1)d+b}$ where $1\le b<x$ since $u$ has distance less than $x$ to them according to induction hypothesis. $w_2$ cannot be a vertex in $L_{2(i-1)d+b}$ where $b>x$, otherwise $(w_1,w_2)$ is not in $A_i$, also not in $A'$, which contradicts the fact that layers are constructed by shortest path tree on $E^T\backslash A'$. $w_2$ cannot be a vertex outside the tree because $s$ can reach $w_2$ in $E^T\backslash A'$. Finially, $w_2\in L_{2(i-1)d+x}$ and the induction hypothesis holds. 
\end{proof}

As mentioned in \Cref{sec:ub-overview}, Theorem~\ref{thm:upperbound} now follows combining the size of sets $F_1$ and $F_2$ from \Cref{lem:settlethickedges} and \Cref{lem:settlethinpairs} respectively.


 


\subsection{Research should be publicly available, and your software is part of it}\label{sec:openSource}

\paragraph{Background} Only openly available research software allows other researchers to reproduce your research results. Since research software is mainly publicly funded, it is also fair to make it publicly available. This also enables others to reuse the software, providing additional benefits like further testing of the software and contributing by creating extensions and compatible software. Besides making the software open source, it is also important to register it so other researchers can find it. These aspects are essential to make a research software \ac{FAIR}~\cite{barker2022introducing}. 

\paragraph{Recommendations} We recommend developing your research software open source. By directly starting the development as open source, challenges can be avoided when switching to open source later. Further, it allows other researchers to use your software and actively contribute as early as possible, improving your software's quality and encouraging cooperation. To develop the software open source, a suitable license is required. By choosing a license early each time another software is included, the compatibility of licenses can be checked directly. Some open source licenses are incompatible since they put certain conditions on the reuse of the code, which can conflict between licenses \cite{cui_empirical_2023}.
When choosing a license, you should check your institution's policy, which often already proposes a specific license. Also, various guides help to choose a license\footnote{e.g., \url{https://choosealicense.com/}, last access 2024-12-17.}.

Cooperation and joint research projects with industry are common in energy research. Sometimes, industry partners are critical of developing open source because they want to keep their intellectual property. Generally, we recommend discussing this topic as early as possible to find reasonable compromises. Often, certain parts can be developed open source while others remain closed source. Since open source is open to all researchers and all industries, it can also improve the exchange between industry and research. 

When you develop open source, there are specific approaches to make your code more easily reusable by others. First, the repository should follow programming-language-specific best practices. This can be achieved by starting with templates for the repository\footnote{e.g., cookie-cutter templates at \url{https://cookiecutter.io/templates}, last access 2024-12-17}. Additionally, citing your software can be made easy by including a \ac{CFF} file\footnote{\url{https://citation-file-format.github.io}, last access 2024-12-17.} in your repository
 \cite{druskat_citation_2021}.

Within the repository, it should be indicated if the software will be maintained, if support is available and if the developers are open to joining research projects, including the software. We recommend using the features of the software platform (GitHub/GitLab) to interact with potential users, e.g., by using issues.

Besides making the source code available, the software should also be findable. Therefore, we recommend registering the software in a domain-specific registry like the Open Energy Platform\footref{fn:oep}. Getting a DOI for the software is also helpful, e.g., by archiving versions of the software on Zenodo\footnote{\url{https://zenodo.org/}, last access 2024-12-17.} \cite{zenodo}. 

\par
Energy research is highly interdisciplinary \cite{tijssen_quantitative_1992}.
Therefore, we recommend adding a very general description to your software, allowing all researchers to understand its goals. This way, more researchers can identify whether the software is useful for them, increasing its reusability. GitHub also allows you to provide keywords to your repository, which again improves findability.


\par 
Platforms like GitHub and GitLab make it easy to publish your software under an open source license. Open source research software enables reproducibility and allows reusability, which can also improve your code quality. Therefore, we recommend: 

\recommendation{Develop open source \& make your software findable!}

\documentclass{MITstyle}

%\usepackage[table]{xcolor}
\usepackage{chngcntr}
\usepackage{hyperref}
\usepackage{microtype}

\title{A Lightweight and Extensible Cell Segmentation and Classification Model for Whole Slide Images}

\author{Nikita Shvetsov~$^{1, }$\footnote{Correspondence e-mail: nikita.shvetsov@uit.no}, Thomas K. Kilvaer~$^{2, 3}$, Masoud Tafavvoghi~$^{4}$, Anders Sildnes~$^{1}$, \\ Kajsa Møllersen~$^{4}$, Lill-Tove Rasmussen Busund~$^{5, 6}$, Lars Ailo Bongo~$^{1}$ \\
%
\vspace{1em} % Space between authors and afilliations
%
\normalfont{\small $^{1}$Department of Computer Science, UiT The Arctic University of Norway}\\
\normalfont{\small $^{2}$Department of Oncology, University Hospital of North Norway}\\
\normalfont{\small $^{3}$Department of Clinical Medicine, UiT The Arctic University of Norway}\\
\normalfont{\small $^{4}$Department of Community Medicine, UiT The Arctic University of Norway}\\
\normalfont{\small $^{5}$Department of Medical Biology, UiT The Arctic University of Norway} \\
\normalfont{\small $^{6}$Department of Clinical Pathology, University Hospital of North Norway} %\vspace{2em}
}

\begin{document}
\maketitle

\section*{Abstract}

% \begin{abstract}
% Developing clinically useful cell-level analysis tools in digital pathology remains challenging due to limitations in dataset granularity, inconsistent annotations, computational demands of advanced models, and difficulties in integrating new technologies into clinical workflows. To address these challenges, we propose a multi-faceted solution that enhances data quality, model performance, and usability to create a lightweight and extensible cell segmentation and classification model.

% First, we update data labels by employing a cross-relabeling process that refines the labels of two existing datasets, PanNuke and MoNuSAC, to create a new unified dataset with enhanced granularity, encompassing seven distinct cell types. Second, we leverage the H-Optimus foundation model as a fixed encoder to improve feature representation for simultaneous cell segmentation and classification tasks. Third, to address the computational demands of foundation models, we employ knowledge distillation to reduce model size and complexity while maintaining comparable performance. Finally, to facilitate integration into clinical workflows, we integrate the distilled model into the QuPath software, a widely used open-source platform in digital pathology.

% Our results demonstrate improvements in cell segmentation and classification performance using the H‑Optimus-based model compared to a CNN-based model. Specifically, the average $R^2$ improved from 0.575 to 0.871, and the average $PQ$ score improved from 0.450 to 0.492, indicating better alignment with actual cell counts and enhanced segmentation and classification quality. Furthermore, the distilled student model maintains performance comparable to the larger foundation model while reducing the parameter count by a factor of 48.
% Overall, by reducing computational complexity and integrating it into existing workflows, the proposed approach may significantly impact diagnostic processes, reduce the workload of pathologists, and contribute to improved patient outcomes. Though our approach shows potential enhancements in efficiency and usability of cell segmentation and classification models in digital pathology, extensive validation is needed to deploy these models in clinical practice.
% \end{abstract}

%%% shortened abstract
\begin{abstract}
Developing clinically useful cell-level analysis tools in digital pathology remains challenging due to limitations in dataset granularity, inconsistent annotations, high computational demands, and difficulties integrating new technologies into workflows. To address these issues, we propose a solution that enhances data quality, model performance, and usability by creating a lightweight, extensible cell segmentation and classification model. 

First, we update data labels through cross-relabeling to refine annotations of PanNuke and MoNuSAC, producing a unified dataset with seven distinct cell types. Second, we leverage the H-Optimus foundation model as a fixed encoder to improve feature representation for simultaneous segmentation and classification tasks. Third, to address foundation models' computational demands, we distill knowledge to reduce model size and complexity while maintaining comparable performance. Finally, we integrate the distilled model into QuPath, a widely used open-source digital pathology platform. 

Results demonstrate improved segmentation and classification performance using the H-Optimus-based model compared to a CNN-based model. Specifically, average $R^2$ improved from 0.575 to 0.871, and average $PQ$ score improved from 0.450 to 0.492, indicating better alignment with actual cell counts and enhanced segmentation quality. The distilled model maintains comparable performance while reducing parameter count by a factor of 48. By reducing computational complexity and integrating into workflows, this approach may significantly impact diagnostics, reduce pathologist workload, and improve outcomes. Although the method shows promise, extensive validation is necessary prior to clinical deployment.
\end{abstract}
\clearpage

\section{Introduction}
In digital pathology, accurate segmentation and classification of cells are crucial for many diagnostic, prognostic, and predictive analyses \cite{Jaber_Beziaeva_etal._2019,Lin_Pan_etal._2022,Park_Ock_etal._2022,Shen_Choi_etal._2024}. Nowadays, developments in computational pathology offer multiple solutions \cite{H._Qu_P._Wu_etal._2020,Javed_Mahmood_etal._2020} to utilize cell-level datasets to train machine learning models that solve these problems. The quality and specificity of training datasets are critical for robust and accurate models. Adhering to the principle of "garbage in, garbage out", it is essential to ensure that these datasets are extensively and accurately labeled with distinct classes that reflect the diverse biological characteristics of different cell types. Unfortunately, the number of open-source datasets comprising such high-quality annotations is limited. Existing cell segmentation datasets \cite{Gamper_Koohbanani_etal._2019,Graham_Vu_etal._2019,Verma_Kumar_etal._2021} may offer extensive annotations for certain cell types while providing more general labels for others. For example, in PanNuke, which is one of the largest open-source datasets comprising labeled cells, various types of morphologically and functionally different inflammatory cells like macrophages and lymphocytes are clustered in a broad "inflammatory" class. Consequently, these classes are frequently omitted from analyses or aggregated into broader meta-classes \cite{Gamper_Koohbanani_etal._2020} and likely interfere with other cell classes included in the dataset. This and similar inconsistencies in annotation granularity limit the ability of machine learning models to learn the comprehensive and nuanced features necessary for accurate cell segmentation and classification. To address these challenges, methods for refining and standardizing dataset annotations are essential to enhance the quality of training data.

A complementary approach to mitigate the absence of high-quality training data is the use of foundation models. Foundation models as encoders are defined as large-scale, versatile networks pre-trained on vast, diverse datasets using self-supervised learning, contrasting with convolutional neural network (CNN) pre-trained encoders that rely on supervised learning with labeled data. In practice, foundation models leverage enormous amounts of weakly or unlabeled data from millions of whole slide images (WSIs) and employ self-attention mechanisms to capture long-range dependencies and global context \cite{Chen_Ding_etal._2024,Saillard_Jenatton_etal._2024,Vorontsov_Bozkurt_etal._2024,Xu_Usuyama_etal._2024}. As a consequence, foundation models are able to produce transferable feature representations across different cell types and tissue environments. The feature representations can be leveraged by decoder networks to produce segmentation masks and pixel-level classifications. Because foundation models have comprehensive feature representations, they can be effectively fine-tuned using much smaller amounts of cell-level data compared to the large datasets needed to train models from scratch. Furthermore, foundation models incorporate adversarial training elements or contrastive learning \cite{Chen_Ding_etal._2024,Xu_Usuyama_etal._2024}, enhancing their resilience and adaptability by exposing them to challenging and varied scenarios during training. This may result in more generalizable models, often making them well-suited for diverse and complex tasks in digital pathology.

Despite the inherent advantages of foundation models, their deployment for practical use faces its own obstacles. In particular, they require substantial computational power, financial investments and rigorous testing to ensure reliability and efficacy for a given task \cite{Akkus_Dangott_etal._2022,Dragomir_Cocuz_etal._2022,Go_2022,Jafri_Farooqui_etal._2024}. Moreover, while foundation models enhance feature representation and performance, they depend on the quality of available annotations for decoder fine-tuning and, like any other model, cannot resolve existing inconsistencies or ambiguities in data labels. Therefore, there remains a critical need for solutions that address both data quality and practical deployment considerations.
Further, integrating new technologies into existing clinical workflows often encounters resistance, as it necessitates adjustments to established diagnostic processes. So, there is a need to develop solutions that could be integrated into current practices, minimizing the burden on medical professionals to adopt new tools \cite{King_Williams_etal._2023}.

Existing solutions \cite{Goldsborough_Philps_etal._2024,Hörst_Rempe_etal._2024}, while addressing some aspects of these challenges, fall short in providing a comprehensive approach. To address the data quality and clinical deployment issues, we propose a multi-faceted solution that encompasses data refinement, model optimization, and integration with existing pathology tools (\hyperref[fig:fig1]{Figure 1}). The outcome is a lightweight cell segmentation and classification model that can be integrated into digital pathology workflows for practical clinical use.

\begin{figure}[h!]
    \centering
    \includegraphics[width=\textwidth, height=0.82\textheight, keepaspectratio]{images/Figure_1.pdf}
    \caption{Overview of the proposed solution, including 1) Data refinement using cross-relabeling, 2) Teacher model development and fine tuning, 3) Student model optimization with knowledge distillation and 4) Student model and QuPath integration}
    \label{fig:fig1}
\end{figure}
\clearpage

Our approach begins with preparing the data for the fine-tuning and training of the machine learning models. We create a refined dataset, acquired via cross-relabeling two cell-level datasets, enhancing annotation specificity and consistency of the labeled data. Subsequently, we create a cell segmentation and classification model based on the foundation model. We leverage the foundation model as a fixed encoder and fine-tune a decoder using the refined dataset to improve generalization across diverse tissue- and cell types.
To ensure that the model remains lightweight and deployable in a possibly resource-constrained environment, we employ knowledge distillation to approximate the functionality of the foundation model. Finally, to facilitate the practical application of our model in digital pathology workflows, we integrate it with the QuPath \cite{Bankhead_Loughrey_etal._2017} application. Each methodological component contributes to the overarching goal of enhancing model performance, generalizability, and usability in clinical settings.

The primary contributions of this paper are:
\begin{enumerate}
    \item \textit{Data labels refinement through cross-relabeling:}
    
    We propose a new method for refining labels of cell-level datasets through cross-relabeling. This method employs classification models to re-label broad and ambiguous instances, resulting in a more diverse dataset. Our evaluation demonstrates that these classification models achieve high accuracy on test subsets, indicating the reliability of the method for label refinement.

    \item \textit{Enhanced model performance via foundation models:}
    
    We employ a foundation model as a feature extractor for the cell segmentation and classification task. In comparison with training a CNN model from scratch, the foundation model backbone only needs fine-tuning, which significantly reduces training time, computational resources and data requirements. We show that using a foundation model encoder leads to better performance in cell segmentation and classification networks than using a CNN-based encoder. This improvement may enable the model to generalize more effectively across various tissue types and imaging methods.
    
    \item \textit{Model optimization through knowledge distillation:}
    
    We show that a smaller student model trained using knowledge distillation on the refined dataset obtained via our cross-relabeling approach from a foundation model achieves comparable performance in cell segmentation and quantification tasks. As a result, this model is more suitable for deployment in environments without high-performance computing resources.
    
    \item \textit{Integration with QuPath:}
    
    We integrate the distilled cell segmentation and classification model into QuPath, a widely used open-source digital pathology platform, to accelerate clinical adaptation by enabling pathologists to more easily incorporate advanced computational tools into their existing workflows.
\end{enumerate}

Through these methodological steps, we aim to bridge the gap between advanced machine learning techniques and practical clinical applications, making accurate and efficient digital pathology accessible in a broader range of healthcare settings.

\section{Refining Existing Datasets Using Cross-Relabeling}
To address the limitations of sparse and ambiguous labeling of cell-level datasets, we propose a generalizable cross-relabeling strategy that can be applied to any dataset containing broadly categorized or imprecisely labeled cell types. This approach involves training and subsequently leveraging classification models to refine broad categories into more specific or biologically relevant classes.
When applied to cell-level data, the methodology includes extracting individual cell images from the dataset patches, preprocessing these images to standardize the size and accommodate partial cells, and then training deep learning classifiers capable of distinguishing between the finer cell subtypes within the coarser categories. 
To illustrate our approach, we focus on the PanNuke \cite{Gamper_Koohbanani_etal._2020, Gamper_Koohbanani_etal._2019} and MoNuSAC \cite{Verma_Kumar_etal._2021} datasets that we have used to train models for cell quantification in our previous works \cite{Shvetsov_Grønnesby_etal._2022,Shvetsov_Sildnes_etal._2024}. We find that for better cell differentiation we have to introduce more granular labels. PanNuke includes a broad classification of "inflammatory" cells, encompassing lymphocytes, macrophages, and neutrophils. Each cell type differs significantly in structure, function, and clinical relevance. Conversely, MoNuSAC uses the label "epithelial" for a class that comprises both benign epithelial cells and malignant neoplastic cells. This practice makes it challenging to differentiate between benign and malignant epithelial cells in the dataset, which is a critical distinction when identifying tumor areas within tissue samples. To address these issues, we implement a cross-relabeling strategy as shown in \hyperref[fig:fig2]{Figure 2}. The key components are two classification models: one is trained on singular cell images from PanNuke data to classify the epithelial meta-class into epithelial and neoplastic classes. The other is trained on MoNuSAC to refine the inflammatory class into lymphocytes, neutrophils, and macrophages.

\begin{figure}[h!]
    \centering
    \includegraphics[width=\textwidth]{images/Figure_2.pdf}
    \caption{Refined dataset generation via cross relabeling}
    \label{fig:fig2}
\end{figure}

The refining approach consists of three consecutive steps. The first is the preprocessing step, in which we extract individual cells from both datasets (\hyperref[fig:fig3]{Figure 3}). The specifics of PanNuke and MoNuSAC patch preparation before cell preprocessing are provided in \hyperref[chap:S1]{Appendix S1}.

\begin{figure}[h!]
    \centering
    \includegraphics[width=\textwidth]{images/Figure_3.pdf}
    \caption{Cell instances preprocessing including (1) cell map extraction, (2) bounding box delineation, (3) adjusting cell boxes and (4) cropping and resizing of cell images}
    \label{fig:fig3}
\end{figure}

During preprocessing, we extract cell type maps from the ground truth label mask and calculate bounding boxes around each cell instance. To accommodate partial cells at patch borders, a common issue in cropped patch images, we employ mirror padding and extend the field of view of the cell label by 15 pixels to capture adjacent cells. We then crop and resize the identified regions to $64 \times 64$ pixels using bicubic interpolation.

The preprocessed PanNuke dataset comprises 68,031 neoplastic and 23,207 epithelial cell images, while MoNuSAC comprises  33,104 lymphocytes, 1,252 neutrophils, and 1,695 macrophages, which we subsequently use in training cell classification models and classifying the cell image data \hyperref[fig:S2]{Appendix Figure S2 (1)}. 

The next step is to train two distinct ResNet50-based classifiers tailored to address the specific labeling challenges inherent in each dataset. We use ResNet50 for classification models due to its proven effectiveness for image classification tasks in histopathology \cite{pan2022reviewmachinelearningapproaches}, and its compatibility with small images. For the PanNuke dataset, we design the classifier, trained on MoNuSAC data, to disaggregate the heterogeneous "inflammatory" cell category into distinct subtypes: lymphocytes, macrophages, and neutrophils. Similarly, for the MoNuSAC dataset, the classifier is trained on PanNuke data and distinguishes between benign and malignant epithelial cells within the overarching "epithelial" label. By applying these targeted classifiers to their respective datasets, we assign more specific labels to individual cell instances, thus enabling us to create a unified dataset.
To ensure a balanced representation of classes, we train both models on datasets that had been equalized to match the size of the least represented class. Thus, we obtain datasets comprising 23,207 samples per class for PanNuke and 1,252 samples per class for MoNuSAC data. Next, we partition both of them into training (70\%), validation (20\%), and testing (10\%) subsets. To mitigate the risk of overfitting, we use a single dropout layer with a rate of p=0.5 in both models and data augmentation using randomized color perturbations, rotation, and horizontal and vertical flipping. We employ AdamW optimizer and the cross-entropy loss function for the training criterion.

To evaluate the two trained models, we measure the classification accuracy on the respective test subsets. The accuracies on the test subset for both classifiers are presented in \hyperref[tab:1]{Table 1}. The PanNuke model achieves an average accuracy of 93.57\%, with higher accuracy for neoplastic cells (96.06\%) compared to epithelial cells (86.26\%). The confusion matrix in Figure A3.1 shows that the model predominantly distinguishes accurately between epithelial and neoplastic tissues, with a substantial number of correct classifications and relatively few misclassifications. The MoNuSAC model demonstrates an average accuracy of 98.92\%, excelling in classifying lymphocytes (99.67\%) and macrophages (94.12\%), with lower performance for neutrophils (85.71\%). The confusion matrix in Figure A3.2 shows that the model identifies lymphocytes and performs reasonably well with macrophages and neutrophils.

\begin{table}[h!]
\renewcommand{\arraystretch}{1.5}
  \centering
  \caption{Cell classification results for PanNuke and MoNuSAC trained models (CI 95\%).}
  \label{tab:1}
  \begin{tabular}{|l|c|c|}
   \hline
   %\rowcolor{gray!30}
    Accuracy               & PanNuke model              & MoNuSAC model              \\
    \hline
    Average      & 0.936 (0.931--0.941)         & 0.989 (0.986--0.993)        \\
    \hline
    Neoplastic   & 0.961 (0.956--0.965)         & -                          \\
    \hline
    Epithelial   & 0.863 (0.849--0.877)         & -                          \\
    \hline
    Lymphocytes  & -                          & 0.997 (0.995--0.999)        \\
    \hline
    Neutrophils  & -                          & 0.857 (0.796--0.918)        \\
    \hline
    Macrophages  & -                          & 0.941 (0.906--0.976)        \\
    \hline
  \end{tabular}
\end{table}

Finally, during the last step, we use the model trained on PanNuke data for epithelial cells in MoNuSAC and the model trained on MoNuSAC for the inflammatory cells class in PanNuke. Specifically, we use classifier models to relabel epithelial cells in MoNuSAC and inflammatory cells in PanNuke data. Then we combine cells with refined labels and the rest of the cells in both datasets to create a refined dataset (\hyperref[fig:S2]{Appendix Figure S2 (2)}). The process of relabeling cells and visualizing them on a patch is shown in \hyperref[fig:fig4]{Figure 4}. The cell counts in the refined dataset are provided in \hyperref[tab:S4]{Appendix Table S4}.

\begin{figure}[h!]
    \centering
    \includegraphics[width=\textwidth, height=0.42\textheight, keepaspectratio]{images/Figure_4.pdf}
    \caption{Cell relabeling procedure for epithelial and inflammatory cell classes}
    \label{fig:fig4}
\end{figure}

%\hfill

Relabeling and combining datasets have been explored in a prior study \cite{Parulekar_Kanwat_etal._2023}, where consecutive fine-tuning on multiple datasets was employed to account for hierarchical class label structures. While the method presented in \cite{Parulekar_Kanwat_etal._2023} is intuitive, it often lacks consistency and requires multiple fine-tuning runs, which can be cumbersome and time-consuming. 
In contrast, cross-relabeling simplifies this process by using specialized classification models tailored to each dataset's specific labeling challenges. This approach provides better transparency and produces a unified dataset encompassing seven distinct cell types across multiple tissue samples, enhancing data diversity for further model training or fine-tuning.

Despite these improvements, cross-relabeling does not entirely resolve issues related to poor labeling quality or the amount of labeled data. Specifically, our results show lower accuracies persist for underrepresented classes, such as macrophages, which may stem from a limited sample availability and intrinsic challenges in distinguishing these cells based solely on H\&E staining. Furthermore, while our method enhances label specificity, it relies on the initial quality of the broad labels; thus, any fundamental inaccuracies in the original annotations can propagate through the relabeling process. Addressing the overall problem of limited data labels may require integrating additional data sources or utilizing complementary immunohistochemical staining methods.
Although the reported performance metrics are obtained from evaluations on the native test sets of each dataset, it is important to note that the primary application of these classifiers is to perform cross-relabeling, where a model trained on one dataset (e.g., PanNuke) is applied to another (e.g., MoNuSAC) and vice versa. We acknowledge that a more systematic evaluation of cross-dataset generalization is needed and could be performed in future work.

Overall, the refined dataset produced by our approach can enhance the supervised training or fine-tuning of cell segmentation and classification models, especially those that utilize pre-trained foundation models to improve feature extraction robustness. In addition, these models can detect nuanced classes that enable researchers to conduct more detailed analyses of biological processes in computational pathology.

\section{Foundation models for robust cell segmentation and classification}

Accurate cell segmentation and classification in digital pathology are hindered by limited labeled data and the fact that conventional CNNs are unable to capture global contextual information due to their local receptive field constraints \cite{Gheflati_Rivaz_2022,Yang_Marcus_etal.}. Traditional approaches in cell quantification have predominantly relied on CNN encoders, such as ResNet50, given their proven effectiveness in semantic segmentation tasks \cite{Deshmane_2023,Graham_Vu_etal._2019,Mukasheva_Koishiyeva_etal._2024,Stringer_Wang_etal._2021}. However, approaches that include fine-tuning of pretrained CNNs, data augmentation, and stain normalization to partially increase data variability and address staining differences often fail to achieve the necessary generalization and robustness across diverse tissue types and staining conditions \cite{G._Wang_W._Li_etal._2018,Gao_Bagci_etal._2018,Karim_El_Khoury_Martin_Fockedey_etal._2021}.

To overcome these challenges, we leverage an encoder-decoder network that uses a foundation model as the encoder and a CNN upsampling decoder (\hyperref[fig:fig5]{Figure 5}) for simultaneous cell segmentation and classification in 2D patches extracted from WSIs. Foundation models with transformer-based architectures are viable alternatives to CNN-based encoders \cite{Shamshad_Khan_etal._2023,Sourget_2023}. They enable the creation of more advanced architectures that can decode or transform learned features more effectively \cite{Chen_Duan_etal._2023,Cheng_Misra_etal._2022,Xie_Wang_etal._2021}.

\begin{figure}[h!]
    \centering
    \includegraphics[width=\textwidth]{images/Figure_5.pdf}
    \caption{UNETR-like model with foundational model as backbone}
    \label{fig:fig5}
\end{figure}

By utilizing a transformer-based encoder, we incorporate global contextual information into the feature extraction process, which is a key advantage of such architectures \cite{Chen_Lu_etal._2021}. This foundation model integration facilitates accurate pixel-wise segmentation and classification without the need for extensive encoder training, thereby potentially improving generalization across varied cellular structures and tissue types.
In our implementation, we employ a modified UNETR \cite{Hatamizadeh_Tang_etal._2021} architecture that combines a vision transformer (ViT) \cite{Dosovitskiy_Beyer_etal._2021} encoder with a CNN-based decoder. The encoder utilizes the pretrained H-Optimus foundation model, which contains 1.1 billion parameters and is trained on over 500,000 H\&E stained WSIs \cite{Saillard_Jenatton_etal._2024}. We extract outputs from four evenly spaced transformer blocks $Z_i$, where $i \in [1, 14, 26, 38]$, to serve as residual connections for the CNN decoder. We select these blocks based on our observation that features from non-adjacent levels of the encoder lead to better overall performance on the test subset.

The CNN decoder upsamples the feature representations, acquired from the transformer blocks, to generate an intermediate vector that is handled by two task-specific layers that generate cell segmentation and classification masks. The first task-specific layer is the ‘Cellpose head’,  which is used to delineate cell instances. The layer generates horizontal and vertical gradient maps to form vector fields that are refined through gradient tracking in a post-processing step using the Cellpose algorithm \cite{Stringer_Wang_etal._2021}, known for its efficacy in cell segmentation tasks and generalizability across multiple domains \cite{Pachitariu_Stringer_2022,Stringer_Pachitariu_2024}. The second task-specific layer is the "Cell type head", which assigns labels to individual pixels. In the post-processing step, we determine the output classification label of each segmented cell instance by majority voting over the labeled pixels that comprise the cell in the segmentation map.

To evaluate model performance and measure the impact of adding a foundation model as backbone, we compare it to a ResNet50-based model. ResNet50 is a widely used solution for encoders in segmentation architectures in the medical domain \cite{Deshmane_2023,Graham_Vu_etal._2019,Mukasheva_Koishiyeva_etal._2024,Stringer_Wang_etal._2021}. For the H-Optimus-based model, we utilize frozen weights for the encoder and only fine-tune the decoder to take advantage of the extensive pre-training of the foundation model. For the ResNet50-based model we start with ImageNet \cite{Deng_Dong_etal.} weights and train both encoder and decoder parts. Hyperparameters for the training step are set to be identical, where possible, for comparable evaluation. 
For this evaluation, we deliberately use the PanNuke dataset to provide a standardized and controlled comparison between the H‑Optimus and ResNet50-based models (\hyperref[fig:S2]{Appendix Figure S2 (3)}). Specifically, we use two of the default PanNuke dataset splits (66\%) for training and validation, and reserve the third split (33\%) for testing.

To address the challenge of cell class imbalance in the PanNuke dataset, which is a common characteristic in most cell-level H\&E patch datasets, both models’ training processes employ a weighted loss function comprising cross-entropy and focal loss \cite{Lin_Goyal_etal._2018}. The focal loss component is adjusted with coefficients derived from each cell class' instance frequency, emphasizing learning from underrepresented classes and enhancing the model's sensitivity to rare but significant cellular patterns. The cross-entropy loss is augmented with spectral decoupling regularization \cite{Pezeshki_Kaba_etal._2021,Pohjonen_Stürenberg_etal._2022} and spatially varying label smoothing \cite{Islam_Glocker_2021}, which potentially stabilizes training and improves generalization in case of complex tissue morphologies. For optimization, we employ the \textit{AdamW} \cite{Loshchilov_Hutter_2019} to counter unbalanced class scenarios, with cosine annealing learning rate scheduler.

We utilize the scikit-learn library \cite{Van_der_Walt_Schönberger_etal._2014} and HoVer-Net \cite{Graham_Vu_etal._2019} implementations of $R^2$ (the coefficient of determination) and $PQ$ (panoptic quality) to evaluate our experiments. Complete mathematical formulations and detailed explanations of these metrics are provided in \hyperref[chap:S5]{Appendix S5}. To compute confidence intervals, we use nonparametric bootstrapping, where after calculating the metric on the full sample, we generated 1000 bootstrap replicates by resampling with replacement and then determined the 95\% confidence intervals as the 2.5th and 97.5th percentiles of the resulting empirical distribution.

%\hfill

The model comparisons are summarized in \hyperref[tab:2]{Table 2}. The H‑Optimus-based model achieves higher $R^2$ across all cell classes compared to the ResNet50-based model, which means that its predictions are more closely aligned with the PanNuke cell counts, indicating a stronger correlation with the observed data. Notably, the improvement of $R^2_{dead}$ may be an indicator of better global contextual representations provided by the foundation model backbone. In terms of segmentation and classification quality combined, measured by the PQ score, the H‑Optimus-based model demonstrates notable improvements across most cell classes. Overall, the average $R^2$ improved from 0.575 to 0.871, while the average $PQ$ score improved from 0.450 to 0.492, demonstrating better performance of the H-Optimus-based model.

\begin{table}[h!]
\renewcommand{\arraystretch}{1.5}
  \centering
  \caption{Cell quantification metrics for baseline and proposed models (CI 95\%).}
  \label{tab:2}
  \begin{tabular}{|l|c|c|}
    \hline
    %\rowcolor{gray!30}
    Metric             & Resnet50-based            & H-optimus-based              \\
    \hline
    $R^2_{neoplastic}$    & 0.681 (0.576--0.769)       & \textbf{0.941 (0.917--0.960)} \\
    \hline
    $R^2_{inflammatory}$  & 0.863 (0.778--0.903)       & \textbf{0.949 (0.918--0.966)} \\
    \hline
    $R^2_{connective}$    & 0.600 (0.488--0.698)       & 0.609 (0.436--0.772)          \\
    \hline
    $R^2_{dead}$          & 0.097 (-11.389--0.669)     & 0.925 (0.404--0.982)          \\
    \hline
    $R^2_{epithelial}$    & 0.635 (0.490--0.747)       & \textbf{0.930 (0.886--0.964)} \\
    \hline
    $PQ_{neoplastic}$       & 0.517 (0.499--0.535)       & \textbf{0.589 (0.575--0.604)} \\
    \hline
    $PQ_{inflammatory}$     & 0.455 (0.429--0.482)       & \textbf{0.528 (0.507--0.549)} \\
    \hline
    $PQ_{connective}$       & 0.416 (0.400--0.431)       & \textbf{0.451 (0.436--0.465)} \\
    \hline
    $PQ_{dead}$             & 0.374 (0.342--0.408)       & 0.292 (0.209--0.365)          \\
    \hline
    $PQ_{epithelial}$       & 0.488 (0.460--0.519)       & \textbf{0.599 (0.579--0.618)} \\
    \hline
  \end{tabular}
\end{table}

Our results  show that integrating the H‑Optimus foundation model within the UNETR architecture enhances the model's ability to segment and classify cells across diverse tissues from PanNuke data. The pretrained transformer encoder provides robust feature representations, resulting in higher average $R^2$ and $PQ$ scores compared to the CNN-based model. This leads to more reliable cell quantification and more accurate downstream analysis. Additionally, the streamlined fine-tuning process reduces computational overhead and training time, making the model more adaptable for new data.

Despite these advancements, the foundation model-based approach does not fully resolve all challenges related to cell segmentation and classification. We observe lower metric scores for underrepresented classes in the training data. Furthermore, foundation models typically encompass billions of parameters, resulting in substantial computational and memory requirements. It therefore poses challenges for deployment in resource-constrained environments, limiting their practical applicability in certain clinical settings.

\section{Model optimization via Knowledge Distillation}

To address the limitations posed by the extensive size of foundation models, we implement knowledge distillation — a model compression technique that leverages the teacher-student paradigm \cite{Hinton_Vinyals_etal._2015}. By training a smaller, more efficient student model to replicate the output of a larger, pre-trained teacher model, we retain performance while significantly reducing the model's complexity and resource requirements (\hyperref[fig:fig6]{Figure 6}).

\begin{figure}[h!]
    \centering
    \includegraphics[width=\textwidth, height=0.45\textheight, keepaspectratio]{images/Figure_6.pdf}
    \caption{Knowledge distillation framework for training a student model using a pre-trained teacher}
    \label{fig:fig6}
\end{figure}

We employ knowledge distillation to compress the H‑Optimus-based teacher model into a more efficient student model. The teacher model is the modified UNETR architecture with the H‑Optimus foundation model described in the previous chapter. The student model is based on a UNet architecture augmented with residual connections and incorporates a smaller ViT encoder with 9 million parameters \cite{Steiner_Kolesnikov_etal._2022,Wightman_2019}. 

First, we fine-tune the teacher model using the refined dataset from the cross-relabeling procedure (Section 2). Initially we train the decoder of the teacher model while keeping the encoder weights frozen. We split the refined dataset into train (70\%), validation (20\%) and test (10\%) subsets (\hyperref[fig:S2]{Appendix Figure S2 (4)}). During fine-tuning, we use the train and validation subsets, while leaving the test subset for model evaluation. We set the training procedure and model hyperparameters to be identical to those that were used to demonstrate the utility of foundation models for the simultaneous cell segmentation and classification task.

Next, we perform knowledge distillation from teacher to student using the refined dataset used to fine-tune the teacher model. The student model is trained to replicate the teacher model's outputs. We utilize a specialized loss function that aligns the student's predicted probability distribution with the teacher's, incorporating the teacher's class probability distribution derived from the output. Following the methodology of Hinton et al. \cite{Hinton_Vinyals_etal._2015}, we experiment with various hyperparameter settings for the temperature ($T$) and the balancing coefficients ($\alpha$ and $\beta$) in the loss function. We vary $T$ from 1 to 20 and adjust $\alpha$ and $\beta$ to balance the distillation and student losses. Through iterative tuning and evaluation, we identify that setting $T=14$, $\alpha=0.3$, and $\beta=0.7$ yields a configuration that converges and closely approximates the teacher model's performance during training.

Finally, we assess the performance of both models using the $R^2$ and $PQ$ (defined in \hyperref[chap:S5]{Appendix S5}) on the test set of the refined dataset (\hyperref[tab:3]{Table 3}). We observe that the 95\% confidence intervals overlap for most cell types, so we cannot claim statistically significant performance differences between the teacher and student models. One exception appears in the neoplastic class. The teacher model produces an $R^2$ of 0.919, while the student model shows an $R^2$ of 0.852. In addition, the student model achieves higher $PQ$ values for the neoplastic and connective classes, though the confidence intervals show overlap.

\begin{table}[h!]
\renewcommand{\arraystretch}{1.5}
  \centering
  \caption{Cell quantification metrics for teacher and distilled student models (CI 95\%).}
  \label{tab:3}
  \begin{tabular}{|l|c|c|}
    \hline
    %\rowcolor{gray!30}
    Metric & Teacher & Student \\
    \hline
    $R^2_{neoplastic}$    & \textbf{0.919} (0.898--0.939) & 0.852 (0.800--0.891) \\
    \hline
    $R^2_{lymphocyte}$    & 0.969 (0.956--0.977)         & 0.969 (0.956--0.978) \\
    \hline
    $R^2_{connective}$    & 0.694 (0.548--0.809)         & 0.618 (0.469--0.741) \\
    \hline
    $R^2_{dead}$          & 0.755 (0.400--0.908)         & 0.424 (0.100--0.731) \\
    \hline
    $R^2_{epithelial}$    & 0.922 (0.870--0.958)         & 0.843 (0.738--0.917) \\
    \hline
    $R^2_{macrophage}$    & 0.384 (-0.369--0.724)        & 0.704 (0.352--0.859) \\
    \hline
    $R^2_{neutrofil}$     & 0.854 (0.578--0.929)         & 0.833 (0.502--0.925) \\
    \hline
    $PQ_{neoplastic}$       & 0.581 (0.569--0.593)         & 0.601 (0.588--0.613) \\
    \hline
    $PQ_{lymphocyte}$       & 0.536 (0.520--0.553)         & 0.563 (0.544--0.579) \\
    \hline
    $PQ_{connective}$       & 0.436 (0.421--0.451)         & 0.457 (0.441--0.474) \\
    \hline
    $PQ_{dead}$             & 0.272 (0.235--0.315)         & 0.279 (0.201--0.369) \\
    \hline
    $PQ_{epithelial}$       & 0.522 (0.500--0.545)         & 0.530 (0.506--0.555) \\
    \hline
    $PQ_{macrophage}$       & 0.524 (0.459--0.588)         & 0.474 (0.405--0.543) \\
    \hline
    $PQ_{neutrofil}$        & 0.541 (0.490--0.592)         & 0.565 (0.522--0.607) \\
    \hline
  \end{tabular}
\end{table}


We further decompose the $PQ$ metric into its $SQ$ and $DQ$ components (\hyperref[tab:S6]{Appendix Table S6}). Both models produce nearly identical $SQ$ values, which indicates that they predict instance boundaries with similar precision. Although the student model shows some improvement in $DQ$ scores for certain classes, the confidence intervals overlap and do not confirm a statistically significant difference.

We observe that the student and teacher models yield comparable detection performance despite the student model using a much smaller and simpler architecture. A model with fewer parameters reduces the risk of overfitting when training data are scarce relative to the model’s complexity \cite{Farias_Ludermir_etal._2022}. The knowledge distillation process also encourages the student model to focus on the most generalizable detection features learned from the teacher. These factors enable the student model to achieve similar detection performance across different cell types.

Additionally, considering the model sizes reported in \hyperref[tab:4]{Table 4}, the distilled model achieves a significant reduction compared to the teacher model, with a 48-fold decrease in parameter count and a 5.5-fold reduction in on-disk size. In inference mode, the teacher model requires 16 GB of VRAM for a batch size of 32, while the distilled model only needs 3 GB of VRAM for the same batch size. These reductions make the distilled model significantly more practical for fine-tuning and deployment in resource-constrained environments.

\begin{table}[h!]
\renewcommand{\arraystretch}{1.5}
  \centering
  \caption{Parameter counts and size of teacher and distilled model}
  \label{tab:4}
  \adjustbox{max width=\textwidth}{%
  \begin{tabular}{|l|c|c|c|}
    \hline
    %\rowcolor{gray!30}
    Metric & H-optimus-based (Teacher) & mobileViT-based (Student) & Magnitude of difference \\
    \hline
    Parameters count       & 1,158,917,906   & \textbf{24,093,393}   & \textbf{48x}  \\
    \hline
    Estimated Total Size (MB) & 87,912       & \textbf{15,935}    & \textbf{5.5x} \\
    \hline
  \end{tabular}%
}
\end{table}

%\hfill

With recent advancements in complex network architectures and the use of pretrained encoders to achieve state-of-the-art performance \cite{Baumann_Dislich_etal._2024,Hörst_Rempe_etal._2024} in cell segmentation and classification tasks, model size, computational complexity, and processing times have increased. This limits the scalability and accessibility of these models. As we demonstrate, this may be mitigated using knowledge distillation. Studies in the field of natural language processing have demonstrated the efficacy of knowledge distillation in retaining the capabilities of the teacher model while achieving significant reductions in size and complexity \cite{Huangpu_Gao_2024,Sun_Yu_etal.}. 

We demonstrate the feasibility of knowledge distillation in digital pathology, specifically for cell segmentation and classification tasks. Moreover, we achieve this performance while also significantly reducing the parameter count. In addressing the challenge of knowledge transfer, we found that distillation from a transformer-based model to a smaller transformer is more straightforward than attempting to map transformer features to CNN blocks. In our experiments, using a CNN-based network as a student results in worse cell quantification performance due to the structural constraints of CNN feature space dimensions. 

Although our primary approach relies on a transformer-based student model that performs well, it can be further optimized to incorporate advantages from CNN architectures. For example, employing alternative techniques such as using ViT adapters \cite{Chen_Duan_etal._2023} or $1 \times 1$ convolutions to adjust feature map sizes may be beneficial for harnessing CNN advantages like enhanced local feature extraction. Moreover, if additional performance improvements are desired, the process can be further enhanced by applying supplementary knowledge distillation techniques, such as self-distillation \cite{Zhang_Song_etal._2019} or online distillation \cite{Houyon_Cioppa_etal._2023}.

Despite these promising results, further validation on independent datasets is necessary to fully understand the model's limitations. Underrepresented classes may pose challenges when addressing complex cases. Pathologists need to validate these models to adopt them in clinical settings. While the distilled models are smaller and more deployable, a technological gap persists because pathologists traditionally rely on established methods for inspecting WSIs and diagnosing diseases. Addressing the complexities involved in deploying models for inference and supporting pathologists in adopting new tools is essential for integrating these models into clinical workflows.

\section{Model integration with QuPath}
Digital pathology tools with graphical user interfaces are essential for visualizing and analyzing WSIs. To make our student model useful in clinical pathology workflows, it needs to be integrated into a tool that enables inspecting regions, creating annotations, and providing quantitative analyses of biomarkers. Therefore, we integrate the trained student model from the previous chapter into the QuPath open‑source platform \cite{Bankhead_Loughrey_etal._2017}. QuPath provides the required annotation, visualization, and analysis tools to interpret complex histological data, including workflows for cell segmentation, classification, and quantification (\hyperref[fig:fig7]{Figure 7}). 

\begin{figure}[h!]
    \centering
    \includegraphics[width=\textwidth]{images/Figure_7.pdf}
    \caption{Visualization of model-generated cell quantification annotations (left) and the corresponding unannotated slide (right) in QuPath}
    \label{fig:fig7}
\end{figure}

To identify the regions in a WSI critical for prognosticating tumor development, such as specific tumor areas or border regions without overlapping healthy tissue, the pathologist uses QuPath to outline these regions. Then, the pathologist initiates a cell segmentation and classification script through the QuPath interface for the selected regions. The resulting annotations and quantified cell information are then directly overlaid onto the WSI in the QuPath interface. Additional design and implementation details are in \hyperref[chap:S7]{Appendix S7}. 

Two common approaches for integrating deep learning models into QuPath are Java‑based native QuPath extensions \cite{Goldsborough_Philps_etal._2024} and the execution of RESTful API requests to a model server coupled with handling the response via an extension, as demonstrated in the application of cell segmentation models applied to immunofluorescence images \cite{Sugawara_2023}. While the community is actively working on these integration strategies, there is currently no universal solution that fully addresses all integration and performance requirements.

Extensions may offer better integration with QuPath, allowing slightly improved performance and more widespread usage of the built-in QuPath models, but they lack the flexibility to customize models and modify their behavior. For example, the newest version of QuPath includes models such as StarDist \cite{Weigert_Schmidt} and InstanSeg \cite{Goldsborough_Philps_etal._2024} that can perform cell segmentation. Both models pose limitations when applied to simultaneous cell segmentation and classification. StarDist performs well only on convex, round shapes by design, whereas some neoplastic, inflammatory, and connective cells exhibit complex and non-convex shapes. InstanSeg provides only semantic segmentation without assigning classes to the segmented cells.

%\hfill

In contrast, our approach offers an alternative integration strategy. It utilizes the paquo library to directly interact with QuPath’s internal application programming interface from within Python. This enables data exchange and processing without the need for intermediate conversion steps and provides greater control over model customization, retraining, and the incorporation of custom processing steps.

The integration of our custom model with QuPath underscores its potential to significantly enhance the diagnostic process by reducing the time burden on pathologists and enabling them to focus on more complex interpretative tasks using familiar software. Leveraging a tool that is already well-established among pathologists increases the likelihood of its adoption into daily clinical workflows. The quantitative data generated through the automated workflow is critical for both clinical decision-making and research, facilitating more accurate biomarker analysis, enabling robust statistical evaluations, and supporting hypothesis generation and testing. Additionally, by streamlining cell segmentation and classification, the tool enhances the scalability and reproducibility of pathological assessments, ultimately contributing to improved diagnostic accuracy and patient outcomes.

\section{Conclusion and future work}

In this study, we address critical challenges in digital pathology and tackle the usability and deployment issues of the developed models in standard computing environments without the need for high-performance computing systems. Our multi-faceted approach encompasses data refinement through cross-relabeling, leveraging foundation models for robust cell segmentation and classification, optimizing model performance via knowledge distillation, and integrating the optimized model into the QuPath software for practical application. This approach is used to construct a capable, versatile, and adjustable model for cell segmentation and classification, with enhanced performance and usability.

\begin{sloppypar}
While our approach shows potential in the field of computational pathology, certain limitations persist. 
For example, our implementation currently exhibits lower performance in detecting macrophages. 
This serves as an instance of the broader challenge of accurately identifying complex cell types. In order to address this issue, extending our approach to incorporate additional data sources, exploring alternative modeling approaches, and integrating other imaging modalities such as immunohistochemical staining may help improve detection accuracy. Moreover, although the distilled model reduces computational demands, integrating advanced deep learning models into clinical practice requires addressing technological gaps and potential resistance to adopting new tools within established diagnostic processes.
\end{sloppypar}

Future work could focus on several key areas to refine the proposed approach and facilitate its adoption in clinical environments. Enhancing the cell-relabeling process with additional datasets \cite{Graham_Jahanifar_etal._2021} could improve the representation of underrepresented cell types and enhance overall model performance. Also, incorporating additional data sources, such as multi-modal imaging or complementary staining methods, may address limitations related to cell type differentiation and class imbalance. Exploring other foundation models \cite{Vorontsov_Bozkurt_etal._2024,Zimmermann_Vorontsov_etal._2024} or introducing additional modalities \cite{Ding_Wagner_etal._2024,Vaidya_Zhang_etal._2025} may provide alternative architectures better suited to specific tasks or offer improved efficiency. Implementing more complex knowledge distillation techniques \cite{Houyon_Cioppa_etal._2023,Zhang_Song_etal._2019} could further optimize the model's performance and adaptability. Additionally, deeper integration with QuPath or other digital pathology software could provide pathologists more control over cell quantification analysis directly within the QuPath interface, thereby increasing accessibility and usability. Such enhancements would not only refine model performance but also ensure greater adaptability and scalability within various clinical environments. Finally, extensive validation of the model by pathologists and benchmarking against independent datasets are essential steps toward establishing the model's reliability and fostering confidence in its clinical utility.

\section*{Acknowledgments} 
This work was funded in part by the Research Council of Norway grant no. 309439 SFI Visual Intelligence, and the North Norwegian Health Authority grant no. HNF1521-20.

\bibliographystyle{IEEEtran}
\begin{sloppypar}
\begin{thebibliography}{99}

\bibitem{chaplot2020neural} Chaplot, Devendra Singh, et al. "Neural topological slam for visual navigation." Proceedings of the IEEE/CVF conference on computer vision and pattern recognition. 2020.

\bibitem{maksymets2021thda} Maksymets, Oleksandr, et al. "Thda: Treasure hunt data augmentation for semantic navigation." Proceedings of the IEEE/CVF International Conference on Computer Vision. 2021.

\bibitem{mezghan2022memory} Mezghan, Lina, et al. "Memory-augmented reinforcement learning for image-goal navigation." 2022 IEEE/RSJ International Conference on Intelligent Robots and Systems (IROS). IEEE, 2022.

\bibitem{al2022zero} Al-Halah, Ziad, Santhosh Kumar Ramakrishnan, and Kristen Grauman. "Zero experience required: Plug \& play modular transfer learning for semantic visual navigation." Proceedings of the IEEE/CVF Conference on Computer Vision and Pattern Recognition. 2022.

\bibitem{ye2021auxiliary} Ye, Joel, et al. "Auxiliary tasks and exploration enable objectgoal navigation." Proceedings of the IEEE/CVF international conference on computer vision. 2021.

\bibitem{chaplot2020object} Chaplot, Devendra Singh, et al. "Object goal navigation using goal-oriented semantic exploration." Advances in Neural Information Processing Systems 33 (2020)

\bibitem{ramakrishnan2022poni} Ramakrishnan, Santhosh Kumar, et al. "Poni: Potential functions for objectgoal navigation with interaction-free learning." Proceedings of the IEEE/CVF Conference on Computer Vision and Pattern Recognition. 2022.

\bibitem{ramrakhya2022habitat} Ramrakhya, Ram, et al. "Habitat-web: Learning embodied object-search strategies from human demonstrations at scale." Proceedings of the IEEE/CVF Conference on Computer Vision and Pattern Recognition. 2022.

\bibitem{mousavian2019visual} Mousavian, Arsalan, et al. "Visual representations for semantic target driven navigation." 2019 International Conference on Robotics and Automation (ICRA). IEEE, 2019.

\bibitem{dhariwal2021diffusion} Dhariwal, Prafulla, and Alexander Nichol. "Diffusion models beat gans on image synthesis." Advances in neural information processing systems 34 (2021)

\bibitem{ho2022classifier} Ho, Jonathan, and Tim Salimans. "Classifier-free diffusion guidance." arXiv preprint arXiv:2207.12598 (2022).

\bibitem{nichol2021glide} Nichol, Alex, et al. "Glide: Towards photorealistic image generation and editing with text-guided diffusion models." arXiv preprint arXiv:2112.10741 (2021)

\bibitem{brooks2023instructpix2pix} Brooks, Tim, Aleksander Holynski, and Alexei A. Efros. "Instructpix2pix: Learning to follow image editing instructions." Proceedings of the IEEE/CVF Conference on Computer Vision and Pattern Recognition. 2023.

\bibitem{fu2023guiding} Fu, Tsu-Jui, et al. "Guiding instruction-based image editing via multimodal large language models." arXiv preprint arXiv:2309.17102 (2023).

\bibitem{geng2024instructdiffusion} Geng, Zigang, et al. "Instructdiffusion: A generalist modeling interface for vision tasks." Proceedings of the IEEE/CVF Conference on Computer Vision and Pattern Recognition. 2024.

\bibitem{zhou2024minedreamer} Zhou, Enshen, et al. "Minedreamer: Learning to follow instructions via chain-of-imagination for simulated-world control." arXiv preprint arXiv:2403.12037 (2024).

\bibitem{zhou2023esc} Zhou, Kaiwen, et al. "Esc: Exploration with soft commonsense constraints for zero-shot object navigation." International Conference on Machine Learning. PMLR, 2023.

\bibitem{yu2023l3mvn} Yu, Bangguo, Hamidreza Kasaei, and Ming Cao. "L3mvn: Leveraging large language models for visual target navigation." 2023 IEEE/RSJ International Conference on Intelligent Robots and Systems (IROS). IEEE, 2023.

\bibitem{gadre2023cows} Gadre, Samir Yitzhak, et al. "Cows on pasture: Baselines and benchmarks for language-driven zero-shot object navigation." Proceedings of the IEEE/CVF Conference on Computer Vision and Pattern Recognition. 2023.

\bibitem{shah2023navigation} Shah, Dhruv, et al. "Navigation with large language models: Semantic guesswork as a heuristic for planning." Conference on Robot Learning. PMLR, 2023.

\bibitem{cai2024bridging} Cai, Wenzhe, et al. "Bridging zero-shot object navigation and foundation models through pixel-guided navigation skill." 2024 IEEE International Conference on Robotics and Automation (ICRA). IEEE, 2024.

\bibitem{yu2023co} Yu, Bangguo, Hamidreza Kasaei, and Ming Cao. "Co-NavGPT: Multi-robot cooperative visual semantic navigation using large language models." arXiv preprint arXiv:2310.07937 (2023).

\bibitem{wu2024voronav} Wu, Pengying, et al. "Voronav: Voronoi-based zero-shot object navigation with large language model." arXiv preprint arXiv:2401.02695 (2024).

\bibitem{qin2023mp5} Qin, Yiran, et al. "Mp5: A multi-modal open-ended embodied system in minecraft via active perception." arXiv preprint arXiv:2312.07472 (2023).

\bibitem{du2024learning} Du, Yilun, et al. "Learning universal policies via text-guided video generation." Advances in Neural Information Processing Systems 36 (2024).

\bibitem{ajay2024compositional} Ajay, Anurag, et al. "Compositional foundation models for hierarchical planning." Advances in Neural Information Processing Systems 36 (2024).

\bibitem{liang2024skilldiffuser} Liang, Zhixuan, et al. "Skilldiffuser: Interpretable hierarchical planning via skill abstractions in diffusion-based task execution." Proceedings of the IEEE/CVF Conference on Computer Vision and Pattern Recognition. 2024.

\bibitem{heusel2017gans} Heusel, Martin, et al. "Gans trained by a two time-scale update rule converge to a local nash equilibrium." Advances in neural information processing systems 30 (2017).

\bibitem{zhang2018unreasonable} Zhang, Richard, et al. "The unreasonable effectiveness of deep features as a perceptual metric." Proceedings of the IEEE conference on computer vision and pattern recognition. 2018.

\bibitem{brown2020language} Brown, Tom B. "Language models are few-shot learners." arXiv preprint arXiv:2005.14165 (2020).

\bibitem{podell2023sdxl} Podell, Dustin, et al. "Sdxl: Improving latent diffusion models for high-resolution image synthesis." arXiv preprint arXiv:2307.01952 (2023).

\bibitem{brohan2022rt} Brohan, Anthony, et al. "Rt-1: Robotics transformer for real-world control at scale." arXiv preprint arXiv:2212.06817 (2022).

\bibitem{brohan2023rt} Brohan, Anthony, et al. "Rt-2: Vision-language-action models transfer web knowledge to robotic control." arXiv preprint arXiv:2307.15818 (2023).

\bibitem{li2024manipllm} Li, Xiaoqi, et al. "Manipllm: Embodied multimodal large language model for object-centric robotic manipulation." Proceedings of the IEEE/CVF Conference on Computer Vision and Pattern Recognition. 2024.

\bibitem{shah2023vint} Shah, Dhruv, et al. "ViNT: A foundation model for visual navigation." arXiv preprint arXiv:2306.14846 (2023).

\bibitem{liu2024visual} Liu, Haotian, et al. "Visual instruction tuning." Advances in neural information processing systems 36 (2024).

\bibitem{hu2021lora} Hu, Edward J., et al. "Lora: Low-rank adaptation of large language models." arXiv preprint arXiv:2106.09685 (2021).

\bibitem{qin2023supfusion} Qin, Yiran, et al. "SupFusion: Supervised LiDAR-camera fusion for 3D object detection." Proceedings of the IEEE/CVF International Conference on Computer Vision. 2023.

\bibitem{qin2024worldsimbench} Qin, Yiran, et al. "Worldsimbench: Towards video generation models as world simulators." arXiv preprint arXiv:2410.18072 (2024).

\bibitem{yu2025gamefactory} Yu, Jiwen, et al. "GameFactory: Creating New Games with Generative Interactive Videos." arXiv preprint arXiv:2501.08325 (2025).

\bibitem{zhou2024code} Zhou, Enshen, et al. "Code-as-Monitor: Constraint-aware Visual Programming for Reactive and Proactive Robotic Failure Detection." arXiv preprint arXiv:2412.04455 (2024).

\bibitem{zhang2024ad} Zhang, Zaibin, et al. "AD-H: Autonomous Driving with Hierarchical Agents." arXiv preprint arXiv:2406.03474 (2024).

\bibitem{wang2024toward} Wang, Chaoqun, et al. "Toward Accurate Camera-based 3D Object Detection via Cascade Depth Estimation and Calibration." arXiv preprint arXiv:2402.04883 (2024).

\bibitem{huang2024story3d} Huang, Yuzhou, et al. "Story3d-agent: Exploring 3d storytelling visualization with large language models." arXiv preprint arXiv:2408.11801 (2024).

\bibitem{savinov2018semi} Savinov, Nikolay, Alexey Dosovitskiy, and Vladlen Koltun. "Semi-parametric topological memory for navigation." arXiv preprint arXiv:1803.00653 (2018).

\bibitem{majumdar2022zson} Majumdar, Arjun, et al. "Zson: Zero-shot object-goal navigation using multimodal goal embeddings." Advances in Neural Information Processing Systems 35 (2022): 32340-32352.

\bibitem{yadav2023offline} Yadav, Karmesh, et al. "Offline visual representation learning for embodied navigation." Workshop on Reincarnating Reinforcement Learning at ICLR 2023. 2023.

\bibitem{yadav2023ovrl} Yadav, Karmesh, et al. "Ovrl-v2: A simple state-of-art baseline for imagenav and objectnav." arXiv preprint arXiv:2303.07798 (2023).

\bibitem{sun2024fgprompt} Sun, Xinyu, et al. "FGPrompt: fine-grained goal prompting for image-goal navigation." Advances in Neural Information Processing Systems 36 (2024).

\bibitem{zhu2017target} Zhu, Yuke, et al. "Target-driven visual navigation in indoor scenes using deep reinforcement learning." 2017 IEEE international conference on robotics and automation (ICRA). IEEE, 2017.

\bibitem{koh2024generating} Koh, Jing Yu, Daniel Fried, and Russ R. Salakhutdinov. "Generating images with multimodal language models." Advances in Neural Information Processing Systems 36 (2024).

\bibitem{krantz2022instance} Krantz, Jacob, et al. "Instance-specific image goal navigation: Training embodied agents to find object instances." arXiv preprint arXiv:2211.15876 (2022).

\bibitem{schulman2017proximal} Schulman, John, et al. "Proximal policy optimization algorithms." arXiv preprint arXiv:1707.06347 (2017).

\bibitem{anderson2018evaluation} Anderson, Peter, et al. "On evaluation of embodied navigation agents." arXiv preprint arXiv:1807.06757 (2018).

\bibitem{lin2024navcot} Lin, Bingqian, et al. "NavCoT: Boosting LLM-Based Vision-and-Language Navigation via Learning Disentangled Reasoning." arXiv preprint arXiv:2403.07376 (2024).

\bibitem{NavGPT} Zhou, Gengze, Yicong Hong, and Qi Wu. "Navgpt: Explicit reasoning in vision-and-language navigation with large language models." Proceedings of the AAAI Conference on Artificial Intelligence.

\bibitem{hahn2021no} Hahn, Meera, et al. "No rl, no simulation: Learning to navigate without navigating." Advances in Neural Information Processing Systems 34 (2021): 26661-26673.

\bibitem{li2025t2isafety} Li, Lijun, et al. "T2ISafety: Benchmark for Assessing Fairness, Toxicity, and Privacy in Image Generation." arXiv preprint arXiv:2501.12612 (2025).

\bibitem{an2024agfsync} An, Jingkun, et al. "AGFSync: Leveraging AI-Generated Feedback for Preference Optimization in Text-to-Image Generation." arXiv preprint arXiv:2403.13352 (2024).


\end{thebibliography}
\end{sloppypar}

\clearpage
\beginsupplement
\section*{Appendix}
\renewcommand{\thesubsection}{S\arabic{subsection}}

\subsection{\label{chap:S1}PanNuke and MoNuSAC preprocessing}
The PanNuke dataset comprises a set of 7,901 RGB patches, each with dimensions of $256 \times 256$ pixels, which we set as the standard patch size for our analysis. In contrast, the MoNuSAC dataset encompasses 294 images of heterogeneous dimensions. To standardize the MoNuSAC images with our experiments, we implement a standardization protocol. Specifically, for images exceeding the dimensions of $256 \times 256$ pixels, we segment them into equal-sized patches and apply mirror padding to the remaining portions to avoid information loss at the peripherals. Patches with dimensions less than $128 \times 128$ pixels are excluded from the dataset due to the insufficient resolution to capture relevant cellular details. For patches where either dimension falls between 128 and 256 pixels, we employ upsampling to achieve the standard patch size. As a result, we obtain a total of 2,823 RGB patches derived from the MoNuSAC dataset for subsequent analysis. For additional details on the MoNuSAC data preparation process, refer to the source code \cite{Shvetsov_2025a}.
\clearpage

\subsection{\label{chap:S2}Data usage for the methodology}

\counterwithin{figure}{subsection}
\renewcommand{\thefigure}{S\arabic{subsection}}

\begin{figure}[h!]
    \centering
    \includegraphics[width=\textwidth, height=0.85\textheight, keepaspectratio]{images/A2.pdf}
    \caption{Overview of the methodology for cross-labeling, dataset refinement, and model comparison. (1) Cross-relabeling - training and testing cell classification models, (2) Cross-relabeling - using cell classification models to create refined dataset, (3) Fine-tuning and training models for comparison, (4) Student knowledge distillation with refined dataset}
    \label{fig:S2}
\end{figure}
\clearpage

\subsection{\label{chap:S3}Confusion matrices for classification models}
\counterwithin{figure}{subsection}
\renewcommand{\thefigure}{S\arabic{subsection}.\arabic{figure}}

\begin{figure}[h!]
    \centering
    \includegraphics[width=\textwidth, height=0.4\textheight, keepaspectratio]{images/A3_1.pdf}
    \caption{Confusion matrix for PanNuke trained model}
    \label{fig:S3.1}
\end{figure}

\begin{figure}[h!]
    \centering
    \includegraphics[width=\textwidth, height=0.4\textheight, keepaspectratio]{images/A3_2.pdf}
    \caption{Confusion matrix for MoNuSAC trained model}
    \label{fig:S3.2}
\end{figure}

\clearpage

\subsection{\label{chap:S4}Datasets cell counts}

\counterwithin{table}{subsection}
\renewcommand{\thetable}{S\arabic{subsection}}

\begin{table}[h!]
\renewcommand{\arraystretch}{2.0}
\centering
\caption{\label{tab:S4}Cell counts for PanNuke, MoNuSAC and refined datasets. Numbers in parentheses indicate preprocessed cell counts for cell classifier models training and testing.}
%\adjustbox{max width=\textwidth}{%
\begin{tabular}{|l|c|c|c|}
\hline
%\rowcolor{gray!30}
Cell type & PanNuke & MoNuSAC & Refined \\
\hline
Neoplastic & 77,403 (68,031) & - & 105,451 \\
\hline
Epithelial & 26,572 (23,207) & - & 29,926 \\
\hline
Epithelial (benign and malignant) & - & 31,402 & - \\
\hline
Inflammatory & 32,276 & - & - \\
\hline
Lymphocytes & - & 37,045 (33,104) & 65,275 \\
\hline
Neutrophils & - & 1,355 (1,252) & 3,833 \\
\hline
Macrophage & - & 1,842 (1,695) & 3,410 \\
\hline
Dead & 2,908 & - & 2,908 \\
\hline
Connective & 50,585 & - & 50,585 \\
\hline
\end{tabular}
%
%}
\end{table}



\clearpage

\subsection{\label{chap:S5}Definition of validation metrics}
\counterwithin{equation}{subsection}
\renewcommand{\theequation}{\arabic{equation}}

\subsubsection{\label{chap:S5.1}R\textsuperscript{2}}
The coefficient of determination, denoted as $R^2$, is a statistical measure that represents the proportion of variance in the dependent variable that is predictable from the independent variables. In the context of cell quantification in pathology, $R^2$ is used to assess how well the predicted quantities of different cell types in a patch align with the actual quantities observed in the ground truth data, with higher values representing more accurate quantification. $R^2$ is defined as
\begin{equation*}
R^2 = 1 - \frac{\sum_{i=1}^n (y_i - \hat{y}_i)^2}{\sum_{i=1}^n (y_i - \bar{y})^2},
\end{equation*}
where $y_i$ represents the actual number of cells of a specific type in the $i$-th image, $\hat{y}_i$ represents the predicted number of cells of that type in the $i$-th image, $\bar{y}$ is the mean of the actual numbers across all images, and $n$ is the total number of images in the dataset.

The $R^2$ metric has a range of $(-\infty, 1]$. An $R^2$ of 1 indicates perfect prediction, where all predicted values exactly match the actual values. An $R^2$ of 0 suggests that the model explains none of the variability of the response data around its mean. If $R^2$ is negative, it indicates that the model performs worse than a model that simply predicts the mean of the actual values for all observations.

\subsubsection{\label{chap:S5.2}PQ}
Panoptic Quality ($PQ$) is a comprehensive metric used to evaluate the performance of segmentation models in tasks that require both instance segmentation and classification. $PQ$ provides a single score that encapsulates both the detection accuracy (i.e., how many objects were correctly identified) and the segmentation quality (i.e., how accurately the objects' boundaries were delineated). This metric is particularly useful in multiclass scenarios where each pixel is classified into distinct categories, such as different cell types in pathology images.

$PQ$ is calculated as the product of two terms: Detection Quality ($DQ$) and Segmentation Quality ($SQ$). It can be expressed as
\begin{equation*}
PQ = DQ \cdot SQ,
\end{equation*}
where
\begin{equation*}
DQ = \frac{TP}{TP + 0.5\, FP + 0.5\, FN},
\end{equation*}
\begin{equation*}
SQ = \frac{\sum_{(p, g) \in \mathcal{M}} IoU(p, g)}{TP}.
\end{equation*}
In these formulas, $TP$ denotes the number of correctly matched instances between ground truth and prediction, $FP$ denotes the predicted instances that have no corresponding ground truth, $FN$ denotes the ground truth instances that were not detected, $IoU(p, g)$ is the Intersection over Union for a pair of matched instances $p$ (prediction) and $g$ (ground truth), and $\mathcal{M}$ is the set of matched pairs.

The $PQ$ metric is calculated for each class and is averaged across classes to provide a global performance measure.

The $PQ$ score has a range of $[0, 1.0]$, where a higher score indicates better performance in both detecting and segmenting the instances correctly. A $PQ$ of 1 signifies perfect identification and segmentation of all instances, whereas a $PQ$ of 0 indicates that no instances were correctly identified and segmented.

\clearpage

\subsection{\label{chap:S6}Segmentation and Detection quality metrics for teacher and student models}

\begin{table}[h!]
\renewcommand{\arraystretch}{2.0}
\centering
\caption{Segmentation and detection quality for student and teacher models (CI 95\%)}
\label{tab:S6}
%\adjustbox{max width=\textwidth}{%
\begin{tabular}{|l|c|c|}
\hline
%\rowcolor{gray!30}
Metric & Teacher & Student \\
\hline
$SQ_{neoplastic}$ & 0.819 (0.815--0.823) & 0.824 (0.819--0.828) \\
\hline
$SQ_{lymphocyte}$ & 0.795 (0.788--0.802) & 0.790 (0.783--0.796) \\
\hline
$SQ_{connective}$ & 0.770 (0.762--0.776) & 0.780 (0.772--0.786) \\
\hline
$SQ_{dead}$ & 0.659 (0.623--0.688) & 0.657 (0.624--0.695) \\
\hline
$SQ_{epithelial}$ & 0.780 (0.770--0.790) & 0.788 (0.779--0.797) \\
\hline
$SQ_{macrophage}$ & 0.788 (0.760--0.810) & 0.757 (0.730--0.783) \\
\hline
$SQ_{neutrofil}$ & 0.782 (0.761--0.801) & 0.775 (0.759--0.792) \\
\hline
$DQ_{neoplastic}$ & 0.706 (0.692--0.719) & 0.727 (0.712--0.741) \\
\hline
$DQ_{lymphocyte}$ & 0.675 (0.656--0.698) & 0.713 (0.691--0.734) \\
\hline
$DQ_{connective}$ & 0.566 (0.546--0.584) & 0.583 (0.565--0.602) \\
\hline
$DQ_{dead}$ & 0.410 (0.361--0.465) & 0.435 (0.306--0.561) \\
\hline
$DQ_{epithelial}$ & 0.668 (0.639--0.694) & 0.673 (0.644--0.702) \\
\hline
$DQ_{macrophage}$ & 0.657 (0.583--0.727) & 0.615 (0.531--0.703) \\
\hline
$DQ_{neutrofil}$ & 0.691 (0.625--0.753) & 0.729 (0.679--0.778) \\
\hline
\end{tabular}
%
%}
\end{table}

\clearpage

\subsection{\label{chap:S7}QuPath integration method}
We adopt an integration strategy leveraging the paquo \cite{Bayer_AG} library, a Python package that enables direct interaction with QuPath’s internal API, thereby facilitating seamless data exchange without intermediate conversion steps. The data processing pipeline (\hyperref[fig:S7]{Appendix Figure S7}) begins with the acquisition of WSIs and their associated annotations from QuPath, which are represented as Shapely \cite{Gillies_Wel_etal._2024} polygons. Utilizing paquo, we directly read, create, and modify these annotations and detections within a QuPath project in the Python environment. Images are then cropped using these polygons and processed by cell segmentation and classification models employing standard vision processing toolkits such as OpenCV, pyvips, and PyTorch. Additionally, QuPath employs Groovy scripts to initiate a Python process that starts the entire pipeline from QuPath graphical interface: fetching polygons, extracting images from them, and running deep learning model inference on the cropped images. 
The results are returned to QuPath, leveraging paquo's Python bindings to manipulate QuPath data while minimizing the computational overhead typically associated with cross-environment communication.

\counterwithin{figure}{subsection}
\renewcommand{\thefigure}{S\arabic{subsection}}

\begin{figure}[h!]
    \centering
    \includegraphics[width=\textwidth]{images/A7.pdf}
    \caption{QuPath integration workflow using Python environment}
    \label{fig:S7}
\end{figure}

Compared to traditional workflows that involve exporting annotations as GeoJSON, classifying them in Python, and reimporting them into QuPath, our approach offers several advantages. We eliminate the need to switch between programming languages, providing a cohesive and streamlined development process entirely within QuPath software and removing the necessity to use other tools. Meanwhile, we avoid storing annotations as intermediate JSON files unless required for external use or archiving. By conducting the entire inference and post-processing workflow within the Python environment, we leverage the power and flexibility of Python libraries for image processing and machine learning. This approach also enables adjustments to any set of labels and models, thereby improving its applicability.

%\hfill

The distilled model and QuPath integration code are packaged into a Docker container, enabling streamlined execution with the Docker engine. Detailed integration code and deployment instructions can be found in the GitHub repository \cite{Shvetsov_2025b}.

Despite these benefits, we acknowledge that the paquo library is a proof‑of‑concept project in its early development stage and has not been tested across all versions of QuPath.

\clearpage

\subsection{\label{chap:S8}Data and code availability statement}
All datasets, models, and code used in this study are publicly available and can be obtained from the repositories listed below. 
The PanNuke \cite{Gamper_Koohbanani_etal._2019} and MoNuSAC \cite{Verma_Kumar_etal._2021} datasets are publicly accessible, and download information along with detailed descriptions can be found in their respective articles. Preprocessing scripts for PanNuke and MoNuSAC data, as well as individual cell extraction scripts, are available on GitHub \cite{Shvetsov_2025a}. The H-Optimus foundation model used in our experiments can be downloaded from the HuggingFace repository \cite{hoptimus2024}, and model information is available on GitHub \cite{Saillard_Jenatton_etal._2024}. In addition, the integration code for QuPath and the distilled model packaged in a Docker container are provided in the repository \cite{Shvetsov_2025b}, and paquo Python library is available from the authors GitHub repository \cite{Bayer_AG}.
\clearpage

\end{document}


\end{document}