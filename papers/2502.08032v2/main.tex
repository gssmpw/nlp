\documentclass[11pt,letterpaper]{article}


\usepackage{hyperref}
\hypersetup{breaklinks, urlcolor=blue, colorlinks, citecolor=green!50!black, linkcolor=blue}
\usepackage[letterpaper, left=1in, right=1in, top=0.9in, bottom=0.9in]{geometry}
\usepackage[utf8]{inputenc}
\usepackage[american]{babel}
\usepackage[normalem]{ulem}
\usepackage{amsmath, amssymb, cases, amsthm}
\usepackage{thmtools}
\usepackage[shortlabels]{enumitem}
\usepackage{mdframed}
\usepackage{bbm}
\usepackage{bm}
\usepackage{microtype}
\usepackage{xcolor}
\usepackage{makecell}
\usepackage{mathtools}
\usepackage{float}
\usepackage{modletters}
\usepackage{comment}
\usepackage{multirow}
\usepackage[numbers]{natbib}
\usepackage[capitalize,noabbrev]{cleveref}
\usepackage{graphics}

\usepackage{footnotehyper}
\makesavenoteenv{tabular}

\usepackage{pifont}
\newcommand{\cmark}{\ding{51}}%
\newcommand{\xmark}{\ding{55}}%

\declaretheorem[numberwithin=section,refname={Theorem,Theorems},Refname={Theorem,Theorems}]{theorem}
\declaretheorem[numberwithin=section,refname={Theorem,Theorems},Refname={Theorem,Theorems}]{thm}
\declaretheorem[numberlike=theorem]{lemma}
\declaretheorem[numberlike=theorem]{proposition}
\declaretheorem[numberlike=theorem]{corollary}
\declaretheorem[numberlike=theorem,style=definition]{definition}
\declaretheorem[numberlike=theorem]{claim}
\declaretheorem[numberlike=theorem,style=remark]{remark}
\declaretheorem[numberlike=theorem,refname={Fact,Facts},Refname={Fact,Facts},name={Fact}]{fact}
\declaretheorem[numberlike=theorem, refname={Question,Questions},Refname={Question,Questions},name={Question}]{question}
\declaretheorem[numberlike=theorem, refname={Problem,Problems},Refname={Problem,Problems},name={Problem}]{problem}
\declaretheorem[numberlike=theorem, refname={Observation,Observations},Refname={Observation,Observations},name={Observation}]{observation}
\declaretheorem[numberlike=theorem, refname={Experiment,Experiments},Refname={Experiment,Experiments},name={Experiment}]{experiment}
\declaretheorem[numberlike=theorem,refname={Primitive,Primitives},Refname={Primitive,Primitives},name={Primitive}]{primitive}
\declaretheorem[numberlike=theorem,refname={Technique,Techniques},Refname={Technique,Techniques},name={Technique}]{technique}
\declaretheorem[numberlike=theorem,refname={Conjecture,Conjectures},Refname={Conjecture,Conjectures},name={Conjecture}]{conjecture}

\def\cameraready{1}  % set this to 1 to get the camera-ready version
\def\final{1}  % set this to 1 to get a comment-free version
\def\iflong{\iffalse}
\ifnum\final=0  %namely if we allow comments in the output
\newcommand{\parinya}[1]{{\color{magenta}[{\tiny Parinya: \bf #1}]\marginpar{\color{magenta}*}}}
\newcommand{\yonggang}[1]{{\color{blue}[{\tiny Yonggang: \bf #1}]\marginpar{*}}}
\newcommand{\danupon}[1]{{\color{red}[{\tiny Danupon: \bf #1}]\marginpar{\color{red}*}}}

\newcommand{\sagnik}[1]{{\color{green!50!black}[{\tiny Sagnik: \bf #1}]\marginpar{\color{green!50!black}*}}}
\newcommand{\todo}[1]{{\color{red}[{\tiny TODO: \bf #1}]\marginpar{\color{red}*}}}
\newcommand{\yuval}[1]{{\bf \color{red!50!black} YUVAL: #1}}
\newcommand{\jan}[1]{{\bf \color{green!50!black} Jan: #1}}
\newcommand{\blikstad}[1]{\textup{\color{magenta} [\textbf{Joakim}: #1]}}
\newcommand{\tawei}[1]{{\color{blue} [{\bf Ta-Wei:} #1]}}
\newcommand{\TODO}[1]{{\color{blue!50!black} [{\bf Todo:} #1]}}
\else % in this case [final=1] we don't want any comments to show
\newcommand{\yonggang}[1]{}
\newcommand{\danupon}[1]{}
\newcommand{\sagnik}[1]{}
\newcommand{\todo}[1]{}
\newcommand{\yuval}[1]{}
\newcommand{\jan}[1]{}
\newcommand{\blikstad}[1]{}
\newcommand{\tawei}[1]{}
\newcommand{\TODO}[1]{}
\fi  %ok, we're done with defining comment macros
\usepackage[ruled,vlined,linesnumbered]{algorithm2e}

\bibliographystyle{alpha}

\DeclarePairedDelimiter{\ceil}{\lceil}{\rceil}
\newcommand{\eps}{\varepsilon}

\newcommand{\mindeg}{\mathrm{mindeg}}
\newcommand{\dist}{\mathrm{dist}}
\newcommand{\nbh}{\mathrm{nbh}}
\newcommand{\spf}{\mathrm{spf}}
\newcommand{\direct}{MDCP }
\newcommand{\best}{\mathsf{best}}
\newcommand{\polylog}{\mathrm{polylog}}
\newcommand{\poly}{\mathrm{poly}}
\newcommand{\outdeg}{\mathrm{outdeg}}
\newcommand{\ones}{\mathrm{ones}}
\newcommand{\euler}{\mathrm{e}}
\newcommand{\var}{\mathrm{Var}}
\newcommand{\set}[2][ ]{\{#2 \ifthenelse{\equal{#1}{ }}{ }{~|~#1}\}}
\newcommand{\fvedge}{\mathsf{FindViolatingEdge}}
\newcommand{\XORQ}{\mathsf{XOR}}
\newcommand{\ANDQ}{\mathsf{AND}}

\newcommand{\cc}{\mathrm{cc}}
\newcommand{\Rcal}{\mathcal{R}}
\newcommand{\Pcal}{\mathcal{P}}
\newcommand{\Acal}{\mathcal{A}}
\newcommand{\Scal}{\mathcal{S}}
\newcommand{\Qcal}{\mathcal{Q}}
\newcommand{\pflag}{\mathrm{pflag}}
\newcommand{\R}{\mathbb{R}}
\newcommand{\N}{\mathbb{N}}
\newcommand{\Var}{\mathrm{Var}}
\newcommand{\one}{\mathbf{1}}
\newcommand{\inact}{\mathsf{Inactive}}
\newcommand{\act}{\mathsf{Active}}
\newcommand{\cut}{\mathrm{cut}}
\newcommand{\cutd}{\overrightarrow{\mathrm{cut}}}
\newcommand{\Ed}{\overrightarrow{E}}
\newcommand{\mincut}{\lambda}
\newcommand{\tOh}{\widetilde{O}}
\newcommand{\ith}{i^{\scriptsize \mbox{{\rm th}}}}
\newcommand{\jth}{j^{\scriptsize \mbox{{\rm th}}}}

%Yonggang's micro
\usepackage{caption2}
\usepackage{empheq}
\renewcommand{\O}[1]{O\left(#1\right)}
\newcommand{\The}[1]{\tilde{\Theta}\left(#1\right)}
\newcommand{\pset}{\mathcal{P}}
\newcommand{\aset}{\mathcal{A}}

\renewcommand{\L}{\mathcal{L}}
\newcommand{\I}{\mathcal{I}}
\renewcommand{\ss}{shortcut}
\newcommand{\ssss}[2]{$(#2,#1)$-shortcut}
\newcommand{\rd}{reduced diameter}
\newcommand{\labcov}{{\sf LabelCover}}
\newcommand{\labcovp}[1]{#1-LabelCover}
\newcommand{\mlab}{multilabeling}
\newcommand{\ba}[1]{#1-bi-criteria approximates}
\newcommand{\os}{{\sf MinShC}}
\newcommand{\oss}[2]{$(#2,#1)$-\os{}}
\newcommand{\tc}{TC spanner}
\newcommand{\TC}[2]{$(#2,#1)$-{\sf MinTC}}
\newcommand{\TCs}[2]{$(#2,#1)$-\tc{}}
\newcommand{\conj}{PGC}
\newcommand{\distt}[2]{\text{dist}_{#1}\left(#2\right)}
\newcommand{\opt}[1]{\dshortcut(#1)}
\newcommand{\optd}[1]{\sshortcut(#1)}
\newcommand{\AB}{\Delta}
\newcommand{\gadget}[2]{$#2$-{\sf MinStShC}$\mid_{#1}$}
\newcommand{\ga}{{\sf MinStShC}}
\newcommand{\lan}{{c_{lrs}}}
\newcommand{\bufs}{{s'}}
\newcommand{\bufN}{{N}}
\newcommand{\minrep}[1]{#1-LabelCover}
\newcommand{\minre}{LabelCover}
\newcommand{\as}[2]{#1-bi-criteria approximating #2-shortcut}
\newcommand{\fan}{copied graph part}
\newcommand{\blue}{\labcov{} part}
\newcommand{\orange}{star part}
\newcommand{\red}{shortcutting part}
\newcommand{\epsl}{\epsilon_{L}}
\renewcommand{\a}[1]{a^{(#1)}}
\renewcommand{\aa}[2]{a^{(#1)}_{#2}}
\newcommand{\aaa}[3]{\alpha^{(#1)}_{#2}[#3]}
\renewcommand{\b}[1]{b^{(#1)}}
\newcommand{\bb}[2]{b^{(#1)}_{#2}}
\newcommand{\bbb}[3]{\beta^{(#1)}_{#2}[#3]}
\newcommand{\dia}{\mathcal{\rho}}
\newcommand{\cs}{c_M}
\newcommand{\idx}{I}
\newcommand{\Idx}[1]{\idx{}[#1]}
\newcommand{\change}[2]{{\color{red}#1}{\color{blue}#2}}
%Yonggang's micro ends here

\newcommand{\intpot}[1]{Int_{pot}(#1)}


\newcommand{\ip}[1]{\left\langle #1\right\rangle }
\newcommand{\expt}[2]{\underset{#1}{\mathbb{E}}\left[#2\right]}
\newcommand{\A}{{\mathcal{A}}}
\newcommand{\BPM}{\mathsf{BPM}}
\newcommand{\BMM}{\mathsf{BMM}}
\newcommand{\UBPM}{\mathsf{UBPM}}

\newcommand{\T}{{\mathcal{T}}}

\newcommand{\U}{{\mathcal{U}}}

\newcommand{\bcc}[1]{\textbf{BCAST($#1$)}}
\newcommand{\dw}[1]{#1^{\downarrow}}
\SetKwComment{Comment}{/* }{ */}

\newcommand{\MM}{\mathcal{P}}
\newcommand{\ORQ}{\mathsf{OR}}
\newcommand{\ISQ}{\mathsf{IS}}
\newcommand{\vol}{\mathrm{vol}}
\newcommand{\volumelb}{\left(\tfrac{1}{20n}\right)^{2n}}

\newcommand{\rank}{\mathsf{rank}}
\newcommand{\sspan}{\mathsf{span}}



%% === Danupon's definitions ===

\newcommand{\stcspanner}{S^*_{TC}}
\newcommand{\sshortcut}{S^*_{Sh}}
\newcommand{\sboth}{S^*}
\newcommand{\dtcspanner}{D^*_{TC}}
\newcommand{\dshortcut}{D^*_{Sh}}
\newcommand{\dboth}{D^*}
\newcommand{\apxS}{\alpha_S}
\newcommand{\apxD}{\alpha_D}

%% === END: Danupon's definitions ===

\Crefname{algocf}{Algorithm}{Algorithms}


\usepackage[most]{tcolorbox}
\newcounter{myexample}
\usepackage{xparse}
\usepackage{lipsum}

\makeatletter
\newcommand\footnoteref[1]{\protected@xdef\@thefnmark{\ref{#1}}\@footnotemark}
\makeatother


\renewcommand{\paragraph}[1]{\medskip\noindent{\bf #1}\xspace}






\newcommand{\ground}{U}





\title{Shortcuts and Transitive-Closure Spanners Approximation}


\author{
Parinya Chalermsook \thanks{University of Sheffield, \texttt{chalermsook@gmail.com}}\and
Yonggang Jiang\thanks{MPI-INF, Germany, \texttt{yjiang@mpi-inf.mpg.de}} \and 
Sagnik Mukhopadhyay\thanks{University of Birmingham \texttt{s.mukhopadhyay@bham.ac.uk}} \and
Danupon Nanongkai\thanks{MPI-INF, Germany, \texttt{danupon@gmail.com}}
}
\date{}





\begin{document}
	
	\begin{titlepage}
		\maketitle \pagenumbering{roman}
		
		\begin{abstract}  
Test time scaling is currently one of the most active research areas that shows promise after training time scaling has reached its limits.
Deep-thinking (DT) models are a class of recurrent models that can perform easy-to-hard generalization by assigning more compute to harder test samples.
However, due to their inability to determine the complexity of a test sample, DT models have to use a large amount of computation for both easy and hard test samples.
Excessive test time computation is wasteful and can cause the ``overthinking'' problem where more test time computation leads to worse results.
In this paper, we introduce a test time training method for determining the optimal amount of computation needed for each sample during test time.
We also propose Conv-LiGRU, a novel recurrent architecture for efficient and robust visual reasoning. 
Extensive experiments demonstrate that Conv-LiGRU is more stable than DT, effectively mitigates the ``overthinking'' phenomenon, and achieves superior accuracy.
\end{abstract}  
		
		\setcounter{tocdepth}{3}
		\newpage
		\tableofcontents
		\newpage
	\end{titlepage}
	
	\newpage
	\pagenumbering{arabic}



\section{Introduction}
\label{sec:introduction}
The business processes of organizations are experiencing ever-increasing complexity due to the large amount of data, high number of users, and high-tech devices involved \cite{martin2021pmopportunitieschallenges, beerepoot2023biggestbpmproblems}. This complexity may cause business processes to deviate from normal control flow due to unforeseen and disruptive anomalies \cite{adams2023proceddsriftdetection}. These control-flow anomalies manifest as unknown, skipped, and wrongly-ordered activities in the traces of event logs monitored from the execution of business processes \cite{ko2023adsystematicreview}. For the sake of clarity, let us consider an illustrative example of such anomalies. Figure \ref{FP_ANOMALIES} shows a so-called event log footprint, which captures the control flow relations of four activities of a hypothetical event log. In particular, this footprint captures the control-flow relations between activities \texttt{a}, \texttt{b}, \texttt{c} and \texttt{d}. These are the causal ($\rightarrow$) relation, concurrent ($\parallel$) relation, and other ($\#$) relations such as exclusivity or non-local dependency \cite{aalst2022pmhandbook}. In addition, on the right are six traces, of which five exhibit skipped, wrongly-ordered and unknown control-flow anomalies. For example, $\langle$\texttt{a b d}$\rangle$ has a skipped activity, which is \texttt{c}. Because of this skipped activity, the control-flow relation \texttt{b}$\,\#\,$\texttt{d} is violated, since \texttt{d} directly follows \texttt{b} in the anomalous trace.
\begin{figure}[!t]
\centering
\includegraphics[width=0.9\columnwidth]{images/FP_ANOMALIES.png}
\caption{An example event log footprint with six traces, of which five exhibit control-flow anomalies.}
\label{FP_ANOMALIES}
\end{figure}

\subsection{Control-flow anomaly detection}
Control-flow anomaly detection techniques aim to characterize the normal control flow from event logs and verify whether these deviations occur in new event logs \cite{ko2023adsystematicreview}. To develop control-flow anomaly detection techniques, \revision{process mining} has seen widespread adoption owing to process discovery and \revision{conformance checking}. On the one hand, process discovery is a set of algorithms that encode control-flow relations as a set of model elements and constraints according to a given modeling formalism \cite{aalst2022pmhandbook}; hereafter, we refer to the Petri net, a widespread modeling formalism. On the other hand, \revision{conformance checking} is an explainable set of algorithms that allows linking any deviations with the reference Petri net and providing the fitness measure, namely a measure of how much the Petri net fits the new event log \cite{aalst2022pmhandbook}. Many control-flow anomaly detection techniques based on \revision{conformance checking} (hereafter, \revision{conformance checking}-based techniques) use the fitness measure to determine whether an event log is anomalous \cite{bezerra2009pmad, bezerra2013adlogspais, myers2018icsadpm, pecchia2020applicationfailuresanalysispm}. 

The scientific literature also includes many \revision{conformance checking}-independent techniques for control-flow anomaly detection that combine specific types of trace encodings with machine/deep learning \cite{ko2023adsystematicreview, tavares2023pmtraceencoding}. Whereas these techniques are very effective, their explainability is challenging due to both the type of trace encoding employed and the machine/deep learning model used \cite{rawal2022trustworthyaiadvances,li2023explainablead}. Hence, in the following, we focus on the shortcomings of \revision{conformance checking}-based techniques to investigate whether it is possible to support the development of competitive control-flow anomaly detection techniques while maintaining the explainable nature of \revision{conformance checking}.
\begin{figure}[!t]
\centering
\includegraphics[width=\columnwidth]{images/HIGH_LEVEL_VIEW.png}
\caption{A high-level view of the proposed framework for combining \revision{process mining}-based feature extraction with dimensionality reduction for control-flow anomaly detection.}
\label{HIGH_LEVEL_VIEW}
\end{figure}

\subsection{Shortcomings of \revision{conformance checking}-based techniques}
Unfortunately, the detection effectiveness of \revision{conformance checking}-based techniques is affected by noisy data and low-quality Petri nets, which may be due to human errors in the modeling process or representational bias of process discovery algorithms \cite{bezerra2013adlogspais, pecchia2020applicationfailuresanalysispm, aalst2016pm}. Specifically, on the one hand, noisy data may introduce infrequent and deceptive control-flow relations that may result in inconsistent fitness measures, whereas, on the other hand, checking event logs against a low-quality Petri net could lead to an unreliable distribution of fitness measures. Nonetheless, such Petri nets can still be used as references to obtain insightful information for \revision{process mining}-based feature extraction, supporting the development of competitive and explainable \revision{conformance checking}-based techniques for control-flow anomaly detection despite the problems above. For example, a few works outline that token-based \revision{conformance checking} can be used for \revision{process mining}-based feature extraction to build tabular data and develop effective \revision{conformance checking}-based techniques for control-flow anomaly detection \cite{singh2022lapmsh, debenedictis2023dtadiiot}. However, to the best of our knowledge, the scientific literature lacks a structured proposal for \revision{process mining}-based feature extraction using the state-of-the-art \revision{conformance checking} variant, namely alignment-based \revision{conformance checking}.

\subsection{Contributions}
We propose a novel \revision{process mining}-based feature extraction approach with alignment-based \revision{conformance checking}. This variant aligns the deviating control flow with a reference Petri net; the resulting alignment can be inspected to extract additional statistics such as the number of times a given activity caused mismatches \cite{aalst2022pmhandbook}. We integrate this approach into a flexible and explainable framework for developing techniques for control-flow anomaly detection. The framework combines \revision{process mining}-based feature extraction and dimensionality reduction to handle high-dimensional feature sets, achieve detection effectiveness, and support explainability. Notably, in addition to our proposed \revision{process mining}-based feature extraction approach, the framework allows employing other approaches, enabling a fair comparison of multiple \revision{conformance checking}-based and \revision{conformance checking}-independent techniques for control-flow anomaly detection. Figure \ref{HIGH_LEVEL_VIEW} shows a high-level view of the framework. Business processes are monitored, and event logs obtained from the database of information systems. Subsequently, \revision{process mining}-based feature extraction is applied to these event logs and tabular data input to dimensionality reduction to identify control-flow anomalies. We apply several \revision{conformance checking}-based and \revision{conformance checking}-independent framework techniques to publicly available datasets, simulated data of a case study from railways, and real-world data of a case study from healthcare. We show that the framework techniques implementing our approach outperform the baseline \revision{conformance checking}-based techniques while maintaining the explainable nature of \revision{conformance checking}.

In summary, the contributions of this paper are as follows.
\begin{itemize}
    \item{
        A novel \revision{process mining}-based feature extraction approach to support the development of competitive and explainable \revision{conformance checking}-based techniques for control-flow anomaly detection.
    }
    \item{
        A flexible and explainable framework for developing techniques for control-flow anomaly detection using \revision{process mining}-based feature extraction and dimensionality reduction.
    }
    \item{
        Application to synthetic and real-world datasets of several \revision{conformance checking}-based and \revision{conformance checking}-independent framework techniques, evaluating their detection effectiveness and explainability.
    }
\end{itemize}

The rest of the paper is organized as follows.
\begin{itemize}
    \item Section \ref{sec:related_work} reviews the existing techniques for control-flow anomaly detection, categorizing them into \revision{conformance checking}-based and \revision{conformance checking}-independent techniques.
    \item Section \ref{sec:abccfe} provides the preliminaries of \revision{process mining} to establish the notation used throughout the paper, and delves into the details of the proposed \revision{process mining}-based feature extraction approach with alignment-based \revision{conformance checking}.
    \item Section \ref{sec:framework} describes the framework for developing \revision{conformance checking}-based and \revision{conformance checking}-independent techniques for control-flow anomaly detection that combine \revision{process mining}-based feature extraction and dimensionality reduction.
    \item Section \ref{sec:evaluation} presents the experiments conducted with multiple framework and baseline techniques using data from publicly available datasets and case studies.
    \item Section \ref{sec:conclusions} draws the conclusions and presents future work.
\end{itemize}

\section{Overview}

\revision{In this section, we first explain the foundational concept of Hausdorff distance-based penetration depth algorithms, which are essential for understanding our method (Sec.~\ref{sec:preliminary}).
We then provide a brief overview of our proposed RT-based penetration depth algorithm (Sec.~\ref{subsec:algo_overview}).}



\section{Preliminaries }
\label{sec:Preliminaries}

% Before we introduce our method, we first overview the important basics of 3D dynamic human modeling with Gaussian splatting. Then, we discuss the diffusion-based 3d generation techniques, and how they can be applied to human modeling.
% \ZY{I stopp here. TBC.}
% \subsection{Dynamic human modeling with Gaussian splatting}
\subsection{3D Gaussian Splatting}
3D Gaussian splatting~\cite{kerbl3Dgaussians} is an explicit scene representation that allows high-quality real-time rendering. The given scene is represented by a set of static 3D Gaussians, which are parameterized as follows: Gaussian center $x\in {\mathbb{R}^3}$, color $c\in {\mathbb{R}^3}$, opacity $\alpha\in {\mathbb{R}}$, spatial rotation in the form of quaternion $q\in {\mathbb{R}^4}$, and scaling factor $s\in {\mathbb{R}^3}$. Given these properties, the rendering process is represented as:
\begin{equation}
  I = Splatting(x, c, s, \alpha, q, r),
  \label{eq:splattingGA}
\end{equation}
where $I$ is the rendered image, $r$ is a set of query rays crossing the scene, and $Splatting(\cdot)$ is a differentiable rendering process. We refer readers to Kerbl et al.'s paper~\cite{kerbl3Dgaussians} for the details of Gaussian splatting. 



% \ZY{I would suggest move this part to the method part.}
% GaissianAvatar is a dynamic human generation model based on Gaussian splitting. Given a sequence of RGB images, this method utilizes fitted SMPLs and sampled points on its surface to obtain a pose-dependent feature map by a pose encoder. The pose-dependent features and a geometry feature are fed in a Gaussian decoder, which is employed to establish a functional mapping from the underlying geometry of the human form to diverse attributes of 3D Gaussians on the canonical surfaces. The parameter prediction process is articulated as follows:
% \begin{equation}
%   (\Delta x,c,s)=G_{\theta}(S+P),
%   \label{eq:gaussiandecoder}
% \end{equation}
%  where $G_{\theta}$ represents the Gaussian decoder, and $(S+P)$ is the multiplication of geometry feature S and pose feature P. Instead of optimizing all attributes of Gaussian, this decoder predicts 3D positional offset $\Delta{x} \in {\mathbb{R}^3}$, color $c\in\mathbb{R}^3$, and 3D scaling factor $ s\in\mathbb{R}^3$. To enhance geometry reconstruction accuracy, the opacity $\alpha$ and 3D rotation $q$ are set to fixed values of $1$ and $(1,0,0,0)$ respectively.
 
%  To render the canonical avatar in observation space, we seamlessly combine the Linear Blend Skinning function with the Gaussian Splatting~\cite{kerbl3Dgaussians} rendering process: 
% \begin{equation}
%   I_{\theta}=Splatting(x_o,Q,d),
%   \label{eq:splatting}
% \end{equation}
% \begin{equation}
%   x_o = T_{lbs}(x_c,p,w),
%   \label{eq:LBS}
% \end{equation}
% where $I_{\theta}$ represents the final rendered image, and the canonical Gaussian position $x_c$ is the sum of the initial position $x$ and the predicted offset $\Delta x$. The LBS function $T_{lbs}$ applies the SMPL skeleton pose $p$ and blending weights $w$ to deform $x_c$ into observation space as $x_o$. $Q$ denotes the remaining attributes of the Gaussians. With the rendering process, they can now reposition these canonical 3D Gaussians into the observation space.



\subsection{Score Distillation Sampling}
Score Distillation Sampling (SDS)~\cite{poole2022dreamfusion} builds a bridge between diffusion models and 3D representations. In SDS, the noised input is denoised in one time-step, and the difference between added noise and predicted noise is considered SDS loss, expressed as:

% \begin{equation}
%   \mathcal{L}_{SDS}(I_{\Phi}) \triangleq E_{t,\epsilon}[w(t)(\epsilon_{\phi}(z_t,y,t)-\epsilon)\frac{\partial I_{\Phi}}{\partial\Phi}],
%   \label{eq:SDSObserv}
% \end{equation}
\begin{equation}
    \mathcal{L}_{\text{SDS}}(I_{\Phi}) \triangleq \mathbb{E}_{t,\epsilon} \left[ w(t) \left( \epsilon_{\phi}(z_t, y, t) - \epsilon \right) \frac{\partial I_{\Phi}}{\partial \Phi} \right],
  \label{eq:SDSObservGA}
\end{equation}
where the input $I_{\Phi}$ represents a rendered image from a 3D representation, such as 3D Gaussians, with optimizable parameters $\Phi$. $\epsilon_{\phi}$ corresponds to the predicted noise of diffusion networks, which is produced by incorporating the noise image $z_t$ as input and conditioning it with a text or image $y$ at timestep $t$. The noise image $z_t$ is derived by introducing noise $\epsilon$ into $I_{\Phi}$ at timestep $t$. The loss is weighted by the diffusion scheduler $w(t)$. 
% \vspace{-3mm}

\subsection{Overview of the RTPD Algorithm}\label{subsec:algo_overview}
Fig.~\ref{fig:Overview} presents an overview of our RTPD algorithm.
It is grounded in the Hausdorff distance-based penetration depth calculation method (Sec.~\ref{sec:preliminary}).
%, similar to that of Tang et al.~\shortcite{SIG09HIST}.
The process consists of two primary phases: penetration surface extraction and Hausdorff distance calculation.
We leverage the RTX platform's capabilities to accelerate both of these steps.

\begin{figure*}[t]
    \centering
    \includegraphics[width=0.8\textwidth]{Image/overview.pdf}
    \caption{The overview of RT-based penetration depth calculation algorithm overview}
    \label{fig:Overview}
\end{figure*}

The penetration surface extraction phase focuses on identifying the overlapped region between two objects.
\revision{The penetration surface is defined as a set of polygons from one object, where at least one of its vertices lies within the other object. 
Note that in our work, we focus on triangles rather than general polygons, as they are processed most efficiently on the RTX platform.}
To facilitate this extraction, we introduce a ray-tracing-based \revision{Point-in-Polyhedron} test (RT-PIP), significantly accelerated through the use of RT cores (Sec.~\ref{sec:RT-PIP}).
This test capitalizes on the ray-surface intersection capabilities of the RTX platform.
%
Initially, a Geometry Acceleration Structure (GAS) is generated for each object, as required by the RTX platform.
The RT-PIP module takes the GAS of one object (e.g., $GAS_{A}$) and the point set of the other object (e.g., $P_{B}$).
It outputs a set of points (e.g., $P_{\partial B}$) representing the penetration region, indicating their location inside the opposing object.
Subsequently, a penetration surface (e.g., $\partial B$) is constructed using this point set (e.g., $P_{\partial B}$) (Sec.~\ref{subsec:surfaceGen}).
%
The generated penetration surfaces (e.g., $\partial A$ and $\partial B$) are then forwarded to the next step. 

The Hausdorff distance calculation phase utilizes the ray-surface intersection test of the RTX platform (Sec.~\ref{sec:RT-Hausdorff}) to compute the Hausdorff distance between two objects.
We introduce a novel Ray-Tracing-based Hausdorff DISTance algorithm, RT-HDIST.
It begins by generating GAS for the two penetration surfaces, $P_{\partial A}$ and $P_{\partial B}$, derived from the preceding step.
RT-HDIST processes the GAS of a penetration surface (e.g., $GAS_{\partial A}$) alongside the point set of the other penetration surface (e.g., $P_{\partial B}$) to compute the penetration depth between them.
The algorithm operates bidirectionally, considering both directions ($\partial A \to \partial B$ and $\partial B \to \partial A$).
The final penetration depth between the two objects, A and B, is determined by selecting the larger value from these two directional computations.

%In the Hausdorff distance calculation step, we compute the Hausdorff distance between given two objects using a ray-surface-intersection test. (Sec.~\ref{sec:RT-Hausdorff}) Initially, we construct the GAS for both $\partial A$ and $\partial B$ to utilize the RT-core effectively. The RT-based Hausdorff distance algorithms then determine the Hausdorff distance by processing the GAS of one object (e.g. $GAS_{\partial A}$) and set of the vertices of the other (e.g. $P_{\partial B}$). Following the Hausdorff distance definition (Eq.~\ref{equation:hausdorff_definition}), we compute the Hausdorff distance to both directions ($\partial A \to \partial B$) and ($\partial B \to \partial A$). As a result, the bigger one is the final Hausdorff distance, and also it is the penetration depth between input object $A$ and $B$.


%the proposed RT-based penetration depth calculation pipeline.
%Our proposed methods adopt Tang's Hausdorff-based penetration depth methods~\cite{SIG09HIST}. The pipeline is divided into the penetration surface extraction step and the Hausdorff distance calculation between the penetration surface steps. However, since Tang's approach is not suitable for the RT platform in detail, we modified and applied it with appropriate methods.

%The penetration surface extraction step is extracting overlapped surfaces on other objects. To utilize the RT core, we use the ray-intersection-based PIP(Point-In-Polygon) algorithms instead of collision detection between two objects which Tang et al.~\cite{SIG09HIST} used. (Sec.~\ref{sec:RT-PIP})
%RT core-based PIP test uses a ray-surface intersection test. For purpose this, we generate the GAS(Geometry Acceleration Structure) for each object. RT core-based PIP test takes the GAS of one object (e.g. $GAS_{A}$) and a set of vertex of another one (e.g. $P_{B}$). Then this computes the penetrated vertex set of another one (e.g. $P_{\partial B}$). To calculate the Hausdorff distance, these vertex sets change to objects constructed by penetrated surface (e.g. $\partial B$). Finally, the two generated overlapped surface objects $\partial A$ and $\partial B$ are used in the Hausdorff distance calculation step.

The full proofs are deferred to the appendix.

\appendix

% !TEX root =  ../main.tex
\section{Background on causality and abstraction}\label{sec:preliminaries}

This section provides the notation and key concepts related to causal modeling and abstraction theory.

\spara{Notation.} The set of integers from $1$ to $n$ is $[n]$.
The vectors of zeros and ones of size $n$ are $\zeros_n$ and $\ones_n$.
The identity matrix of size $n \times n$ is $\identity_n$. The Frobenius norm is $\frob{\mathbf{A}}$.
The set of positive definite matrices over $\reall^{n\times n}$ is $\pd^n$. The Hadamard product is $\odot$.
Function composition is $\circ$.
The domain of a function is $\dom{\cdot}$ and its kernel $\ker$.
Let $\mathcal{M}(\mathcal{X}^n)$ be the set of Borel measures over $\mathcal{X}^n \subseteq \reall^n$. Given a measure $\mu^n \in \mathcal{M}(\mathcal{X}^n)$ and a measurable map $\varphi^{\V}$, $\mathcal{X}^n \ni \mathbf{x} \overset{\varphi^{\V}}{\longmapsto} \V^\top \mathbf{x} \in \mathcal{X}^m$, we denote by $\varphi^{\V}_{\#}(\mu^n) \coloneqq \mu^n(\varphi^{\V^{-1}}(\mathbf{x}))$ the pushforward measure $\mu^m \in \mathcal{M}(\mathcal{X}^m)$. 


We now present the standard definition of SCM.

\begin{definition}[SCM, \citealp{pearl2009causality}]\label{def:SCM}
A (Markovian) structural causal model (SCM) $\scm^n$ is a tuple $\langle \myendogenous, \myexogenous, \myfunctional, \zeta^\myexogenous \rangle$, where \emph{(i)} $\myendogenous = \{X_1, \ldots, X_n\}$ is a set of $n$ endogenous random variables; \emph{(ii)} $\myexogenous =\{Z_1,\ldots,Z_n\}$ is a set of $n$ exogenous variables; \emph{(iii)} $\myfunctional$ is a set of $n$ functional assignments such that $X_i=f_i(\parents_i, Z_i)$, $\forall \; i \in [n]$, with $ \parents_i \subseteq \myendogenous \setminus \{ X_i\}$; \emph{(iv)} $\zeta^\myexogenous$ is a product probability measure over independent exogenous variables $\zeta^\myexogenous=\prod_{i \in [n]} \zeta^i$, where $\zeta^i=P(Z_i)$. 
\end{definition}
A Markovian SCM induces a directed acyclic graph (DAG) $\mathcal{G}_{\scm^n}$ where the nodes represent the variables $\myendogenous$ and the edges are determined by the structural functions $\myfunctional$; $ \parents_i$ constitutes then the parent set for $X_i$. Furthermore, we can recursively rewrite the set of structural function $\myfunctional$ as a set of mixing functions $\mymixing$ dependent only on the exogenous variables (cf. \cref{app:CA}). A key feature for studying causality is the possibility of defining interventions on the model:
\begin{definition}[Hard intervention, \citealp{pearl2009causality}]\label{def:intervention}
Given SCM $\scm^n = \langle \myendogenous, \myexogenous, \myfunctional, \zeta^\myexogenous \rangle$, a (hard) intervention $\iota = \operatorname{do}(\myendogenous^{\iota} = \mathbf{x}^{\iota})$, $\myendogenous^{\iota}\subseteq \myendogenous$,
is an operator that generates a new post-intervention SCM $\scm^n_\iota = \langle \myendogenous, \myexogenous, \myfunctional_\iota, \zeta^\myexogenous \rangle$ by replacing each function $f_i$ for $X_i\in\myendogenous^{\iota}$ with the constant $x_i^\iota\in \mathbf{x}^\iota$. 
Graphically, an intervention mutilates $\mathcal{G}_{\mathsf{M}^n}$ by removing all the incoming edges of the variables in $\myendogenous^{\iota}$.
\end{definition}

Given multiple SCMs describing the same system at different levels of granularity, CA provides the definition of an $\alpha$-abstraction map to relate these SCMs:
\begin{definition}[$\abst$-abstraction, \citealp{rischel2020category}]\label{def:abstraction}
Given low-level $\mathsf{M}^\ell$ and high-level $\mathsf{M}^h$ SCMs, an $\abst$-abstraction is a triple $\abst = \langle \Rset, \amap, \alphamap{} \rangle$, where \emph{(i)} $\Rset \subseteq \datalow$ is a subset of relevant variables in $\mathsf{M}^\ell$; \emph{(ii)} $\amap: \Rset \rightarrow \datahigh$ is a surjective function between the relevant variables of $\mathsf{M}^\ell$ and the endogenous variables of $\mathsf{M}^h$; \emph{(iii)} $\alphamap{}: \dom{\Rset} \rightarrow \dom{\datahigh}$ is a modular function $\alphamap{} = \bigotimes_{i\in[n]} \alphamap{X^h_i}$ made up by surjective functions $\alphamap{X^h_i}: \dom{\amap^{-1}(X^h_i)} \rightarrow \dom{X^h_i}$ from the outcome of low-level variables $\amap^{-1}(X^h_i) \in \datalow$ onto outcomes of the high-level variables $X^h_i \in \datahigh$.
\end{definition}
Notice that an $\abst$-abstraction simultaneously maps variables via the function $\amap$ and values through the function $\alphamap{}$. The definition itself does not place any constraint on these functions, although a common requirement in the literature is for the abstraction to satisfy \emph{interventional consistency} \cite{rubenstein2017causal,rischel2020category,beckers2019abstracting}. An important class of such well-behaved abstractions is \emph{constructive linear abstraction}, for which the following properties hold. By constructivity, \emph{(i)} $\abst$ is interventionally consistent; \emph{(ii)} all low-level variables are relevant $\Rset=\datalow$; \emph{(iii)} in addition to the map $\alphamap{}$ between endogenous variables, there exists a map ${\alphamap{}}_U$ between exogenous variables satisfying interventional consistency \cite{beckers2019abstracting,schooltink2024aligning}. By linearity, $\alphamap{} = \V^\top \in \reall^{h \times \ell}$ \cite{massidda2024learningcausalabstractionslinear}. \cref{app:CA} provides formal definitions for interventional consistency, linear and constructive abstraction.

	\section{Lower Bounds}\label{sec:lowerbounds}

This section contains three parts.

In~\cref{subsec:warmup}, we will prove that there is no polynomial algorithm solving \oss{n^\epsilon}{1}. The technique is similar to the construction used by Elkin and Peleg~\cite{ElkinP07} to prove the hardness of approximating directed spanner. We will also define \ga{}, as discussed in~\cref{sec:overview}, and prove the hardness for \ga{} when $s=o(m)$.

In~\cref{subsec:bicriterialower}, we will show how to prove the lower bound for \oss{n^\epsilon}{n^\epsilon}, using the lower bound for \ga{} while preserving the relative size of $s$ versus $m$. It will only prove that \oss{n^\epsilon}{n^\epsilon} is hard for some input $s=o(m)$ by combining with the result in~\cref{subsec:warmup}, which is still not what we want in~\cref{thm:main}.

In~\cref{subsec:largebicriteria}, we boost $s$ to be $\omega(m)$ in the lower bound proof of \ga{}. Combined with the lemma proved in~\cref{subsec:bicriterialower}, this will show us \oss{n^\epsilon}{n^\epsilon} is hard for some input $s=\Omega(m^{1+\epsilon})$, which is what we want in~\cref{thm:main}.


\subsection{Warm up: Lower Bounds when \texorpdfstring{$\apxD=1$}{apxD1}}\label{subsec:warmup}

This section will use a simple construction to prove the following lemma.

\begin{lemma}\label{lem:onecriteria}
	Assuming PGC (\cref{con:pgc}), there are no polynomial time algorithm solving \oss{n^{\epsilon}}{1} for some small constant $\epsilon$. 
	
\end{lemma}

The construction relies on the following \minre{} graph.

\begin{definition}[\labcov{} graph]\label{def:minrepgraph}
	Given a \labcov{} instance $\I=(A,B,E,\L,(\pi_e)_{e\in E})$ descirbed in~\cref{def:labcov} and a parameter $\rho$, we define the \labcov{} graph $G_{\I,\rho}$ as a directed graph defined as follows (see \cref{fig:minrepgraph}):
	\begin{itemize}
		\item Suppose $A=\{1,...,|A|\},B=\{1,...,|B|\},\L=\{1,2,...,|\L|\}$. In $G_{\I,\rho}$ we have vertices $\{\aa{i}{j},\bb{i}{j}\mid 1\le i\le |A|=|B|, 1\le j\le |\L|\}$ and edges $\{(\aa{i}{j},\bb{i'}{j'})\mid (i,i')\in E, (j,j')\in\pi_{(i,i')}\}$.
		
		\item In addition, $G_{\I,\rho}$ also contains vertices $\{\a{i},\b{i}\mid 1\le i,j\le |A|=|B|\}$, where $\a{i}$ has a directed path with length $\rho$ to each vertex in $\{\aa{i}{j}\mid 1\le j\le |\L|\}$ denoted by $\alpha^{(i)}_j$, the vertices along the path are $\{\aaa{i}{j}{k}\mid 1\le k\le \rho-1\}$; similarly, $\b{i}$ has a directed path with length $\rho$ from each vertex in $\{\bb{i}{j}\mid 1\le j\le |\L|\}$ denoted by $\beta^{(i)}_j$, the vertices along the path are $\{\bbb{i}{j}{k}\mid 1\le k\le \rho-1\}$. 
		
		\item Finally, we add edges $(\aaa{i}{j}{k},\a{i}),(\b{i},\bbb{i}{j}{k}),(\aa{i}{j},\a{i})$, $(\b{i},\bb{i}{j})$ for any possible $i,j,k$.
	\end{itemize}
	
\end{definition}

\begin{figure}[H]
	\centering
	\includegraphics[scale=1]{figures/minrepgraph.pdf}
	\setcaptionwidth{0.95\textwidth}
	\caption{\small In this example, $|A|=|B|=|\mathcal{L}|=\rho-1=3$. Suppose $A=\{1,2,3\},B=\{1,2,3\},\mathcal{L}=\{1,2,3\}$, then in this example we have $E=\{(2,2)\}$ and $\pi_{(2,2)}=\{(1,2),(2,3),(3,1)\}$, which corresponds to the three edges in the middle. }\label{fig:minrepgraph}	
\end{figure}


\begin{proof}[Proof of~\cref{lem:onecriteria}]
	Given a \labcov{} instance $\I=(A,B,E,\L,(\pi_e)_{e\in E})$, we use $\AB$ to denote the size of $A$ (and also $B$). We now describe how to distinguish between the case. Notice that $N$ is the input size of the \labcov{} instance $\I$.
	\begin{itemize}
		
		\item (Completeness:) There is a labeling that covers every edge. 
		
		\item (Soundness:) Any multilabeling of cost at most $N^{\epsl{}}(|A|+|B|)$ covers at most $N^{-\epsl{}}$ fraction of edges.             
	\end{itemize}
	According to~\cref{lem:pgc}, this violates PGC. 
	
	If $|E|\le 2\AB n^{\epsl}$, then it cannot be the Soundness case, so we assume $|E|>2\AB n^{\epsl}$. 
	
	First, we use polynomial time to compute the \labcov{} graph $G_{\I,\dia{}}$ with $n$ vertices, where $\dia{}$ is polynomial on $N$ and can be arbitrarily large. 
	Then We run the \oss{n^\epsilon}{1} algorithm with input $G_{\I,\dia{}}$ and parameters $s=2\AB,d=\dia{}+1$. We will prove the following two claims, which will lead to the solution to the \labcov{} problem.
	\begin{claim}\label{clai:caseI}
		If the \labcov{} instance is in case (completness), then $G_{\I,\dia{}}$ has a \ssss{2\AB }{\dia{}+1}. Moerover, for any $(i,j)\in E$, $(\a{i},\b{j})$ has distance at most $3$ after adding this \ss{}. 
	\end{claim}
	\begin{proof}
		Suppose the labeling is $\psi$. For any $i\in A$, we include the edge $(\a{i},\aa{i}{\psi(i)})$ in the \ss{}; for any $i\in B$, we include the edge $(\bb{i}{\psi(i)},\b{i})$ in the \ss{}. The \ss{} has size $2\AB $. Since $(i,j)$ is covered by $\psi$, there exists an edge $(\aa{i}{i'},\bb{j}{j'})$ such that $i'=\psi(i),j'=\psi(j)$. In that case, we have the length 3 path $(\a{i},\aa{i}{i'},\bb{j}{j'},\b{j})$ after adding the \ss{}. This proves the second statement of this lemma. Now we verify the distances between reachable pairs are at most $\dia{}+1$ one by one.
		\begin{enumerate}
			\item (start from $\a{i}$) $\a{i}$ can reach any $\aa{i}{j},\aaa{i}{j}{k}$ with distance at most $\dia{}$, $\a{i}$ can reach any $\b{j}$ with $(i,i')\in E$ with distance $3$, which means $\a{i}$ can reach any $\bbb{i'}{j}{k},\bb{i'}{j}$ with distance $4$.
			
			\item (start from $\aaa{i}{j}{k},\aa{i}{j}$) $\aaa{i}{j}{k}$ or $\aa{i}{j}$ has an edge to $\a{i}$, so they can reach any node that $\a{i}$ can reach with distance at most $\dia{}+1$. 
			
			\item (start from $\bbb{i}{j}{k},\bb{i}{j},\b{i}$) These nodes can reach any reachable nodes with distance at most $\dia{}+1$ in $G_{\I,\dia{}}$. 
		\end{enumerate}
		
	\end{proof}
	
	\begin{claim}\label{clai:caseII}
		If the \minre{} instance is in case (soundness), then $G_{\I,\dia{}}$ does not have a \ssss{2\AB\cdot n^{\epsilon}}{\dia{}+1} for sufficiently small constant $\epsilon$. Moreover, by adding any shortcut with size at most $2\AB\cdot n^{\epsilon}$, at most $o(1)$ fraction of pairs in $\{(a^{(i)},b^{(j)})\mid (i,j)\in E\}$ will have distance less than $\dia{}+1$. 
	\end{claim}
	\begin{proof}
		Suppose $G_{\I,\dia{}}$ has a shortcut $E'$ adding which reduces the diameter between $\Omega(1)$ fraction of pairs in $\{(a^{(i)},b^{(j)})\mid (i,j)\in E\}$ to less than $\dia{}+1$. We will try to get a contradiction. We first use $E'$ to build a \mlab{} $\psi$ in the following way: for any $1\le i\le |A|$, we let $\psi(i)$ to be the set of vertices among $\{\aa{i}{j}\mid 1\le j\le \L\}$ that $\a{i}$ has distance at most $\dia{}-1$ to after adding the \ss{}. 
		
		\paragraph{$\psi$ covers more than $N^{-\epsl}$ fraction of edges.}	We first argue that $\psi$ covers $\Omega(1)$ fraction of the edges. Write $A_i=\{\a{i},\aa{i}{j},\aaa{i}{j}{k}\mid 1\le j\le \L,1\le k\le \dia{}-1\}, B_i=\{\b{i},\bb{i}{j},\bbb{i}{j}{k}\mid 1\le j\le \L,1\le k\le \dia{}-1\}$. For an edges $(i,j)\in E$ (recall that $E$ is the edge set in the \labcov{} instance $\I$), we say it is \emph{crossed} if there is an edge $(u,v)\in E'$ with $u\in A_i,v\in B_j$ in $E'$. Remember that we assumed $|E|\ge \Delta\cdot N^{\epsl{}}$, otherwise, the case (soundness) can never happen. Therefore, the number of crossed edges is at most $2\AB\cdot n^{\epsilon}=o(|E|)$ for sufficiently small $\epsilon$. Now We prove that for any non-crossed edge $(i,j)\in E$ such that $(a^{(i)},b^{(j)})$ has distance less than $\dia{}+1$ after adding $E'$, $(i,j)$ is covered by $\psi$. If we can prove this, then at least $\Omega(1)$ fraction of edges in $E$ are covered. To prove this, notice that if $(i,j)$ is not covered, then consider the shortest path $p$ from $\a{i}$ to $\b{j}$ after adding the \ss{}, we write the part where $p$ is inside $A_i$ as $p_A$, and the part where $p$ is inside $B_j$ as $p_B$. if both $p_A,p_B$ has length at most $\dia{}-1$, then $(i,j)$ is covered. Thus, one of $p_A$ or $p_B$ has length at least $\dia{}$. Then we have $|p|=|p_A|+1+|p_B|\ge \dia{}+1+1\ge \dia{}+2$, which is a contradiction. 
		
		\paragraph{$\psi$ has cost at most $N^{\epsl}(|A|+|B|)$.} Then we argue that $\sum_{u\in A\cup B}|\psi(u)|\le |E'|\le 2\AB \cdot n^{\epsilon}$ (which will give us $\sum_{u\in A\cup B}|\psi(u)|\le N^{\epsl{}}(|A|+|B|)$ for sufficiently small $\epsilon$). For a label $j\in\psi(i)$, we have $\distt{(V,E\cup E')}{\a{i},\aa{i}{j}}<\dia{}$. Thus, there exists a path $p=(v_0,...,v_{\ell})$ from $\a{i}$ to $\aa{i}{j}$ with length at most $\dia{}-1$. Let $p'=(\a{i},\aaa{i}{j}{1},\aaa{i}{j}{2},...,\aa{i}{j})$. Let $k$ be the last index such that $v_k\not\in p'$ and $v_{k+1}\in p'$. Since $v_{\ell}$ in $p'$ and the length of $p'$ equals $\dia{}$, such $k$ must exist. Since $(v_k,v_{k+1})\not\in E$, we have $(v_k,v_{k+1})\in E'$. We call this edge $(v_k,v_{k+1})$ a \emph{critical edge} of $(i,j)$. For different $i$ and $j$ with $j\in\psi(i)$, they have different critical edges in $E'$ because the corresponding $v_{k+1}$ is always different. Thus, $\sum_{u\in A\cup B}|\psi(u)|\le |E'|\le 2\AB \cdot n^{\epsilon}$.
		
	\end{proof}
	
	If the \labcov{} instance is in case (completeness), by~\cref{clai:caseI}, $G_{\I,\dia{}}$ admits a \ssss{2\Delta}{\dia{}+1}, which means the output by the \oss{n^{\epsilon}}{1} algorithm will be a \ssss{2\Delta\cdot n^{\epsilon}}{\dia{}+1}; on the other hand, if the \labcov{} instance is in case (soundness), by~\cref{clai:caseII}, the output cannot be a \ssss{2\Delta\cdot n^{\epsilon}}{\dia{}+1} since $G_{\I,\dia{}}$ does not have one. Since checking whether the output is \ssss{2\Delta\cdot n^{\epsilon}}{\dia{}+1} or not is in polynomial time, \labcov{} is solved in polynomial time.
 
\end{proof}






In the following definition, we will abstract all the necessary properties of graph $G_{\I,d}$ in a black box that we are going to use to prove the hardness for \oss{n^{\epsilon}}{n^\epsilon}.

\begin{definition}\label{def:gadget}
	For $1>\epsilon,\cs{}>0$, the \gadget{\cs{}}{\epsilon} problem has inputs
	\begin{enumerate}
		\item a directed connected graph $G=(V,E)$ with $m$ edges, $n$ nodes and diameter polynomial on $m$, let the diameter be $d$,
		\item two sets $L,R\subseteq V$ with $L\cap R=\emptyset$, where $|L|$ is polynomial on $m$,
		\item a set of reachable vertex pairs $P\subseteq L\times R$,
		\item a positive integer $s=\Omega(m^{\cs{}})$.
	\end{enumerate}
	The problem asks to distinguish the following two types of graphs.
	\begin{description}
		\item[Type 1.] There exists a \ss{} $E'$ of $G$ with size $s$ such that all reachable pairs $(u,v)\in L\times R$ have distance $O(1)$ after adding $E'$ to $G$.
		\item[Type 2.] By adding any \ss{} with size $s\cdot n^\epsilon$, at most $o(1)$ fraction of the pairs in $P$ have distance at most $d/3$.
	\end{description}
	
\end{definition}

The following lemma shows that~\cref{fig:minrepgraph} is a hard instance for \gadget{\cs{}}{\epsilon}.
\begin{lemma}\label{lem:smallgadget}
	Under PGC (\cref{con:pgc}), there exist constants $\epsilon,\cs{}$ such that \gadget{\cs{}}{\epsilon} cannot be solved in polynomial time.
	
\end{lemma}
\begin{proof}
	Given any \labcov{} instance $\I=(A,B,E,\L,\pi)$ (we write $|A|=\AB$), we construct a \gadget{\cs{}}{\epsilon} instance with the following inputs.
	\begin{enumerate}
		\item The graph is $G_{\I,\dia{}}$ (see~\cref{def:minrepgraph}) for $\dia{}$ arbitrarily large such that $\dia{}$ is polynomial on $N$ (the bit length of instance $\I$). $G_{\I,d}$ has diameter $d=2\dia{}+1$.
		\item $L=\{a_i\mid 1\le i\le \AB\},R=\{b_i\mid 1\le i\le \AB\}$, clearly $|L|=|R|\le m$ and $|L|$ is polynomial on $m$.
		\item $P=\{(a_i,b_j)\mid (i,j)\in E\}$.
		\item $s=|A|+|B|$, it is polynomial on $m$.
	\end{enumerate}
	
	Now we prove that if we have a polynomial time algorithm to distinguish the two types of graphs as described in~\cref{def:gadget}, then we can solve \labcov{} in polynomial time.
	
	If $\I$ is in case (completeness), according to~\cref{clai:caseI}, by adding a \ss{} with size $s=(|A|+|B|)$, all reachable pairs $(a_i,b_j)$ (with $(i,j)\in E$) has distance at most 3. 
 
 If $\I$ is in case (soundness), according to~\cref{clai:caseII}, by adding a \ss{} with size less than $s\cdot n^{\epsilon}$ for sufficiently small $\epsilon$, at most $o(1)$ fraction of pairs in $P$ has distance less than $(2d+1)/3<\dia{}$. 
\end{proof}

\subsection{Lower Bound when \texorpdfstring{$\apxD>1$}{apxDge1} and \texorpdfstring{$s=o(m)$}{0<cs<1}}\label{subsec:bicriterialower}
In this section, we prove the following lemma, which shows how to use the hardness of \gadget{\cs{}}{\epsilon} to get lower bounds for $\apxD>1$.
\begin{lemma}\label{lem:usegadget}
	For any constant $1>\epsilon,\cs{}>0$, if there is no polynomial algorithm solving \gadget{\cs{}}{\epsilon}, then for sufficiently small constant $\gamma$, there is no polynomial algorithm solving \oss{n^{\gamma\epsilon}}{n^{\gamma\epsilon}} even if the input is restricted to $s=\Omega(m^{1+\gamma(\cs{}-1)})$.
	
\end{lemma}
By combining \cref{lem:smallgadget,lem:usegadget}, we can get the following corollary. Notice that this corollary is not the same as~\cref{thm:main}, since it does not restrict $s$ to be $\omega(m)$.

\begin{corollary}\label{cor:smallmain}
	Under \conj{} (\cref{con:pgc}), there is no polynomial algorithm solving \oss{n^{\epsilon}}{n^{\epsilon}} for sufficiently small constant $\epsilon$.
\end{corollary}

The following geometric graph~\cite{HuangP21} is a crucial part of our proof for~\cref{lem:usegadget}. We say a graph $G=(V,E)$ is a $k$-layered directed graph if $V=V_1\uplus V_2\uplus...\uplus V_k$ ($V$ is partitioned into $V_1,...,V_k$), such that for any edge $(u,v)\in E$, there exists $1\le i<k$ and $u\in V_i,v\in V_{i+1}$. $V_i$ is called the $i$-th layer of $G$.

\begin{lemma}[Section 2.2~\cite{HuangP21}]\label{lem:geograph}
	For any $\AB\in\mathbb{N}^+$, for arbitrary constant $0<\lan{}< 1$, we can compute in polynomial time a $\AB^\lan$ layered graph $G=(V,E)$, a set of pairs $P$ and an indexing $\idx{}$ satisfying the following properties.
 
	\begin{enumerate}
            \item Each layer of $G$ contains $\The{\AB^{10}}$ vertices and has a maximum in-degree and out-degree of at most $\AB$. Let the first layer be denoted as $V_1 = \{s_1, s_2, \dots, s_{K_1}\}$ and the last layer as $V_2 = \{t_1, t_2, \dots, t_{K_2}\}$. \label{geoitem1}

		\item $P \subseteq [K_1] \times [K_2]$. For every $i\in[K_1]$, there are $\Omega(\Delta^2)$ pairs $(i,j)\in P$. For every $(i,j)\in P$, there is a unique path from $s_i$ to $t_j$ in $G$, denoted by $p_{i,j}$. $p_{i,j}$ and $p_{i',j'}$ shares at most one edge for different pairs $(i,j)\not=(i',j')$. \label{geoitem3}
		\item The function $\idx{}$ assigns a value from $[\Delta]$ to each $(v, e)$ pair, where $e$ is either an in-edge or an out-edge of vertex $v$. For a given vertex $v$, $\idx{}$ assigns distinct values to its different in-edges, and similarly, $\idx{}$ assigns distinct values to its different out-edges.  For any $(x,y)\in[\AB]\times[\AB]$, there exists $\Omega(\Delta^{10})$ paths $p_{i,j}$ (where $(i,j)\in P$) such that for any edge $(u,v)$ on $p_{i,j}$,\label{geoitem4}
                \begin{itemize}
                    \item if $v$ is in the even layer, then $\Idx{v,(u,v)}=x$, 
                    \item if $u$ is in the even layer, then $\Idx{u,(u,v)}=y$.
                \end{itemize}
	\end{enumerate}
 
\end{lemma}
\begin{proof}[Proof of~\cref{lem:geograph}]
	The general idea is to use the graph described in Section 2.2~\cite{HuangP21} with max degree $\Delta$, and cut the first $\Delta^{\lan{}}$ layers to get our disired graph. Readers are recommanded to read the construction in Section 2.2~\cite{HuangP21}, and the construction for~\cref{lem:geograph} is followed straightforwardly. 
	
	Let the graph $G_1\otimes G_2$ described in Section 2.2~\cite{HuangP21} with parameters $d_1=d_2=2,r_1=\Delta^{3/2}$ be $G_{\Delta}$. $G_{\Delta}$ has $\Omega(\Delta)$ layers, where each layer has $\The{\Delta^{10}}$ vertices. The max in/out-degree is $\Delta$. We use the subgraph of $G_{\Delta}$ induced by the first $\Delta^{\lan{}}$ layers
	to get the desired graph described in~\cref{lem:geograph}, denoted by $G'_{\Delta}$. %
	We construct $P$ in the following way. According to Section 2.2~\cite{HuangP21}, there exist $\Omega(\Delta^{12})$ \emph{Critical Pairs} in $G_{\Delta}$ between the first and last laryer nodes where each of them has a unique path connecting them. We cut each unique path at the first $\Delta^{\lan{}}$ layers, resulting in $\Omega(\Delta^{12})$ pairs between the first and last layer in $G'_{\Delta}$. %
	Now we verify the properties of $P$ one by one.
	\begin{enumerate}
		\item In Section 2.2~\cite{HuangP21}, each critical pairs is $((a,b,0),(a+Dv,b+Dw,2D))$ for some $v,w\in\mathcal{V}_2(r_1)$. There are $\Delta$ possible choices for $v$ or $w$. Thus, for every $a$, there are $\Omega(\Delta^2)$ choices of $b$ such that $(a,b)\in P$.
		\item If the path between $(a,b)\in P$ is not unique, then the path between two critical pair in $G_{\Delta}$ is also not unique.
		\item Since every two path between two critical pairs in $G_\Delta$ share at most one edge according to Lemma 2.6~\cite{HuangP21}, our path segments in the first $\Delta^{\lan{}}$ layers also share at most one edge.
	\end{enumerate}

        Now we construct $\idx{}$. We fix an order for the set $\mathcal{V}_2(r_1)=\{w_1,w_2,...,w_\Delta\}$. Each node in the even layer $(a,b,i)$ has out-edges $\{(a+w_j,b,i+1)\left((a,b,i),(a+w_j,b,i+1)\right)\mid j\in[\Delta]\}$. $\idx{}$ assign index $j$ to the edge $\left((a,b,i),(a+w_j,b,i+1)\right)$. $(a,b,i)$ has in-edges $\{\left((a,b-w_j,i-1),(a,b,i)\right)\mid j\in[\Delta],b-w_j\in B_2(R_1+\left\lceil (i-1)/2\right\rceil r_1)\}$. $\idx{}$ assign $j$ to the edge $\left((a,b-w_j,i-1),(a,b,i)\right)$. Now for any $(x,y)\in[\Delta]\times[\Delta]$, we consider all the unique paths from $(a,b,0)$ to $(a+\Delta^{\lan{}}x,b+\Delta^{\lan{}}y,\Delta^{\lan{}}-1)$. All edges in this path is either $\left((a+ix,b+iy,2i),(a+(i+1)x,b+iy,2i+1)\right)$, which is the in-edge of $(a+(i+1)x,b+iy,2i+1)$ assigned as $x$, or $((a+(i+1)x,b+iy,2i+1)$,$(a+(i+1)x,b+(i+1)y,2i+2))$, which is the out-edge of $(a+(i+1)x,b+iy,2i+1)$ assigned as $y$.\footnote{To avoid confusion, notice that in Section 2.2~\cite{HuangP21}, the node $(x,y,i)$ is in the $(i+1)$-th layer, because $i$ is indexed from $0$. }
\end{proof}
Now we are ready to prove the main lemma in this section.
\begin{proof}[Proof of~\cref{lem:usegadget}]
	Suppose $\mathcal{A}$ is the polynomial time algorithm for \oss{n^{\gamma\epsilon}}{n^{\gamma\epsilon}} where input must satisfy $s=\Omega(m^{1+\gamma(\cs{}-1)})$. Given a \gadget{\cs{}}{\epsilon} instance $G_{inr},(L,R),P',\bufs$ (see~\cref{def:gadget}, we use $G_{inr},P',s'$ instead of $P,s$ to avoid conflicting of notations), we will show how to use $\mathcal{A}$ as an oracle to solve it in polynomial time.
	
	\paragraph{Definition of $G$.} Let $\Delta=|L|=|R|$, let $M$ denote the number of edges in $G_{inr}$ and let the diameter of $G_{inr}$ be $\rho{}$. We first construct a graph $G$ in the following way. Let the graph, pairs, and indexing described in~\cref{lem:geograph} with parameter $\AB$ and sufficiently small constant $\lan$ be $G_{geo},P,I$. Without loss of generality, we assume $\AB^\lan$ is odd. 
	$G$ contains the following parts. See~\cref{fig:origin} as an example.
	
	\begin{figure}
		\centering
		\includegraphics[scale=0.8]{figures/origin.pdf}
		\setcaptionwidth{0.95\textwidth}
		\caption{Suppose the graph above is the graph $G_{geo}$ with $\AB=2$ (notice that for ease of explanation, the graph does not satisfy properties described in~\cref{lem:geograph}). The graph below shows how we substitute each node $v$ by $G_v$. If $v$ is in the even layer, then $G_v$ is a copy of $G_{inr}$; otherwise, if $v$ is not in the first or last layer, $G_v$ is a star graph. }\label{fig:origin}	
	\end{figure}
	
	\begin{enumerate}
		\item (Substitute nodes in even layers by copies of $G_{geo}$) for each vertex $v$ in the even layers of $G_{geo}$, create a copy of graph $G_{inr}$, denoted as $G_v$ as part of $G$. Suppose $L=\{a_1,...,a_{\Delta}\},R=\{b_1,...,b_\Delta\}$ (remember that $L,R$ are inputs to \gadget{\cs{}}{\epsilon}). $a_i$ and $b_i$ are denoted by $a^v_i,b^v_i$ in $G_v$. 
		\item (Substitute nodes in odd layers by star graphs) for each vertex $v$ in the odd layers except the first and last layer of $G_{geo}$, create vertices $s_1,...,s_{\Delta},t_1,...,t_\Delta,v$, and create edges $(s_1,v),,...,(s_\Delta,v),(v,t_1),...,(v,t_\Delta)$.
		\item (Keep first and last layer) remember that $I$ is the input to \gadget{\cs{}}{\epsilon}. $G$ includes all nodes in the first and last layer of $G_{geo}$, denoted by $s_1,s_2,...,s_{K_1}$ and $t_1,t_2,...,t_{K_2}$.
		\item (Edges connected different parts) for each edge $e=(u,v)$ in $G_{geo}$, create an edge $(b^u_{\Idx{u,e}},a^v_{\Idx{v,e}})$ in $G$. If $u$ is in the first layer or $v$ is in the last layer, just create edge $(u,a^v_{\Idx{v,e}})$ or $(b^u_{\Idx{u,e}},v)$ instead.
	\end{enumerate}
	
	Remember that $G_{geo}$ has $\Delta^{\lan{}}$ layers, each with $\Delta^{10}$ vertices, and the max degree of $G_{geo}$ is $\Delta$. The number of edges $m$ in $G$ can be calculated as
	\[m=\The{\Delta^{10+\lan{}}\cdot M}+\The{\Delta^{10+\lan{}}\cdot \Delta}+O(\Delta^{10+\lan{}}\cdot \Delta)=\The{\Delta^{10+\lan{}}\cdot(\Delta+M)}\]
	
	\paragraph{Use $\mathcal{A}$ to solve \gadget{\cs{}}{\epsilon}.} After constructing $G$, we apply $\mathcal{A}$ on $G$ with parameter $s=\AB^{10+\lan}\cdot \bufs, d=C\rho$ for sufficiently large constant $C$. We first argue that $s=\Omega(m^{1+\gamma(\cs{}-1)})$ for sufficiently small $\gamma$ so this is a valid input. Remember that $\bufs$ is one of the inputs to \gadget{\cs{}}{\epsilon} such that $s'=\Omega(M^{\cs{}})$. Also remember in~\cref{def:gadget}, we require $|L|=|R|=\Delta\le M$ to be polynomial on $M$. Denote $M=\Delta^{c_{\Delta}}$. Then we have $m=\The{\Delta^{10+\lan{}+c_\Delta}}$ and $s=\Delta^{10+\lan{}+c_{\Delta}\cs{}}$. Now we have $s/m=\The{\Delta^{c_\Delta(\cs{}-1)}}$, where $\Delta^{c_\Delta}=m^{\gamma}$ for some constant $\gamma$. Thus, we have $s=\Omega(m^{1+\gamma(\cs{}-1)})$.
	
Remember that $\rho$ is the diameter of $G_{inr}$. According to~\cref{def:gadget}, $\rho{}$ is polynomial on $M$. Let $\rho{}= M^{c_D}$. %
Remember our goal is to distinguish the following two types of \gadget{\cs{}}{\epsilon} instances.

\begin{description}
	\item[Type 1.] There exists a \ss{} $E'$ of $G_{inr}$ with size $O(s')$ such that all reachable pairs $(u,v)\in L\times R$ have distance $O(1)$ after adding $E'$ to $G_{inr}$.
	\item[Type 2.] By adding any \ss{} with size $\bufs\cdot M^{\epsilon}$, at most $o(1)$ fraction of the pairs in $P'$ have distance at most $\rho{}/3$.
\end{description}

We prove the following two lemmas to show the output of $\mathcal{A}$ suffices to distinguish whether the \gadget{\cs{}}{\epsilon} instance is in type 1 or type 2. %

\begin{lemma}\label{lem:type1}
	If the \gadget{\cs{}}{\epsilon} instance is type 1, then $G$ has a \ssss{s=\AB^{10+\lan}\cdot \bufs}{O(\rho{})}.
\end{lemma}
\begin{proof}
	For each copy of $G_{inr}$ (denoted as $G_v$) in $G$, we add $\bufs$ edges to the \ss{} to make all reachable pairs $(a^v_{i},b^v_j)$ have distance $O(1)$. The \ss{} has size at most $\AB^{10+\lan}\cdot \bufs$. Consider a reachable $(u,v)$ in $G$, we first argue that they have distance $O(\rho{}+\AB^\lan)$ after adding the \ss{}. To see this, notice that a path connecting $u,v$ can only be in the form of $(u,p_1,p_2,...,p_{\ell},v)$ where $p_i$ is a path inside $G_v$ for some $v$, and $\ell\le\AB^{\lan}$. %
	Each $p_i$ with $1<i<\ell$ will be in the form $(a^v_i,...,b^v_j)$ for some $v,i,j$. Notice that $(a^v_i,b^v_j)$ has distance $O(1)$ after adding the shortcut if $v$ is in the even layer (if $v$ is in the odd layer then there is already an existing length 2 paths connecting $(a^v_i,b^v_j)$). Therefore, $u$ has distance $O(\rho{}+\AB^\lan)$ to $v$. Remember that $\rho{}=M^{c_D}=\Delta^{c_\Delta c_D}$. By setting $\lan{}<c_\Delta c_D$, the distance is at most $O(\rho{})$.
\end{proof}


\begin{lemma}\label{lem:type2}
	If the \gadget{\cs{}}{\epsilon} instance is type 2, then $G$ has no \ssss{n^{\gamma\epsilon}s}{n^{\gamma\epsilon}\rho{}}.
\end{lemma}
\begin{proof}
	
	Before we prove our lemma, we need to do some preparations. %
	Remember that $P'$ is the input to \gadget{\cs{}}{\epsilon} such that for any $(x,y)\in P'\subseteq[\AB]\times[\AB]$, $a^v_x$ can reach $b^v_y$ for any $v$. Also recall that according to~\cref{lem:geograph} item~\ref{geoitem4}, for every $(x,y)\in P'\subseteq [\AB]\times[\AB]$, in $G_{geo}$ there exists $\Omega(\Delta^{10})$ paths $p_{i,j}$ (which is the unique path connecting $s_i,t_j$) that all edges $(u,v)$ in this path is the in-edge indexed by $I$ as $x$ of $v$ if $v$ is in the even layer, or the out-edge indexed by $I$ as $y$ of $u$ if $u$ is in the even layer. That means in $G$, there is a path from $s_i$ to $t_j$ in the form of $p'_{i,j}=(s_i,a^{v_{2,q_2}}_{x},...,b^{v_{2,q_2}}_{y}, a^{v_{3,q_3}}_x,...,b^{v_{3,q_3}}_y,...,t_j)$. We say $p'_{i,j}$ covers vertices $v_{2,q_2},v_{3,q_3},...$ at $(x,y)$. Moreover, for $(i',j')\not=(i,j)$, we know from~\cref{lem:geograph} item~\ref{geoitem3} that $p_{i,j},p_{i',j'}$ shares at most one edge. Thus, $p'_{i,j}$ and $p'_{i',j'}$ cannot cover the same vertex $v$ at the same $(x,y)$, otherwise they share two edges: the in-edge indexed as $x$ of $v$ and the out-edge indexed as $y$ of $v$. In summary, we have $\Omega(\Delta^{10}|P'|)$ pairs $(i,j)$, which we call critical pairs, satisfying the following properties.
	\begin{enumerate}
		\item $s_i$ can reach $t_i$ in $G$, where all paths from $s_i$ to $t_i$ must be in the form $(s_i,a^{v_{2,q_2}}_{x},...,b^{v_{2,q_2}}_{y}, a^{v_{3,q_3}}_x,$ $...,b^{v_{3,q_3}}_y,...,t_j)$. 
		\item Any path from $s_i$ to $t_j$ cover one vertex $v$ in each layer at some pair $(x,y)\subseteq P'$. For different critical pairs $(i,j)\not=(i',j')$, they will not cover the same vertex $v$ at the same pair $(x,y)\subseteq P'$.
	\end{enumerate} 
	
	
	Now we are ready to prove \cref{lem:type2}. We will prove that $G$ does not have a \ssss{\AB^{10+\lan}\cdot \bufs\cdot M^{\epsilon/2}}{\rho{}\AB^{\lan}/9}. We first show why this implies~\cref{lem:type2}. Remember that $s=\AB^{10+\lan}\cdot \bufs$, thus, $\AB^{10+\lan}\cdot \bufs\cdot M^{\epsilon/2}>m^{\gamma\epsilon}s)$ for sufficiently small constant $\gamma$. Also remember that $m$ is polynomial on $\Delta$, thus, $\rho{}\Delta^{\lan{}}/9>m^{\gamma\epsilon}\rho{}$ for sufficiently small constant $\gamma$. 
	
	Suppose to the contrary, $G$ has a \ssss{\AB^{10+\lan}\cdot \bufs\cdot M^{\epsilon/2}}{\rho{}\AB^{\lan}/9}, we will make a contradiction. We first claim that $G$ also has a \ssss{\AB^{10+\lan}\cdot \bufs\cdot M^{\epsilon/2}\cdot \AB^{\lan}}{\rho{}\AB^{\lan}/8}, denoted by $E'$, such that both end points for every edge in $E'$ is in the same layer. We can achieve this by taking any edge $(u,v)$ in the original \ss{} that crosses different layers, finding the path from $u$ to $v$ denoted by $p$, and cut $p$ into at most $\AB^\lan$ segments $p_1,p_2,...,p_\ell$ where each segment is in the same layer, and replace $(u,v)\in E'$ by $\Delta^{\lan}$ new edges each connecting two end points of $p_i$ from $i=1$ to $i=\ell$. In this way, the \ss{} increase by a factor of at most $\AB^\lan$. The distance between two vertices increases by at most an additive factor of $O(\AB^\lan)$. Another property of $E'$ is that every edge in $E'$ has both endpoints in $G_v$ for some $v$. That is because different $G_v$ where $v$ in the same layer are not reachable from each other. 
	
	Now with $E'$ added to $G$, we denote the new graph by $G'$. We define an indicator variable $I_{v,i,j}$ to be $1$ if $\distt{G'}{a^v_i,b^v_j}\le \rho{}/3$. We know from the property of type 2 that for any $v$ and arbitrary small constant $\eta$, $\sum_{i,j\in P'}I_{v,i,j}>\eta|P'|$ implies $E'$ has at least $\bufs\cdot M^{\epsilon}$ edges in $G_v$. Since $|E'|\le \AB^{10+\lan}\cdot \bufs\cdot M^{\epsilon/2}\cdot \AB^{\lan}$, by setting $\lan<1/2$, at most $o(\AB^{10+\lan})$ vertices $v$ has the property that $\sum_{i,j\in P'}I_{v,i,j}>\eta|P'|$, which means $\sum_{v,(i,j)\in P'}I_{v,i,j}\le 2\eta\AB^{10+\lan}|P'|$. However, for each critical pair $(i,j)$, suppose $T_{i,j}=\{(v,x,y)\mid (i,j)\text{ covers }v\text{ at }(x,y)\}$, then we have $\sum_{(v,x,y)\in T_{i,j}}I_{v,x,y}\ge (1/2)\AB^\lan$, otherwise the distance $\distt{G'}{s_i,t_j}$ with be at least $(\AB^\lan)/2\cdot \rho{}/3>\rho{}\AB^\lan/6$. We also know that $T_{i,j}$ is disjoint for different $i,j$. Thus, we get $\sum_{v,i,j}I_{v,i,j}=\Omega(\Delta^{10}|P'|)\cdot \AB^\lan/2=\Omega(|P'|\AB^{10+\lan})$, contradicts the fact that $\sum_{v,i,j}I_{v,i,j}\le 2\eta\AB^{10+\lan}|P'|$ for sufficiently small constant $\eta$. 
	\yonggang{I think this proof is very hard to parse and I should try to refine it.}
	
\end{proof}

If the \gadget{\cs{}}{\epsilon} instance is type 1, then according to~\cref{lem:type1}, $G$ admits a \ssss{s=\AB^{10+\lan}\cdot \bufs}{O(\rho{})}, so $\mathcal{A}$ will output a \ssss{n^{\gamma\epsilon}s}{n^{\gamma\epsilon}\rho{}}; on the other hand, if the \gadget{\cs{}}{\epsilon} instance is type 2, then according to~\cref{lem:type1}, $G$ cannot output a \ssss{n^{\gamma\epsilon}s}{n^{\gamma\epsilon}\rho{}}. Since checking whether an edge set is a \ssss{n^{\gamma\epsilon}s}{n^{\gamma\epsilon}\rho{}} or not is in polynomial time, the output of $\mathcal{A}$ distinguish type 1 from type 2.

\end{proof}

\subsection{Lower Bound when \texorpdfstring{$\apxD>1$}{paxDge1} and \texorpdfstring{$s=\omega(m)$}{cs>1}}\label{subsec:largebicriteria}

In this section, we prove the following lemma.

\begin{lemma}\label{lem:larges}
	Assuming \conj{} (\cref{con:pgc}), there exists two constants $0<\delta,\epsilon<1$ such that \gadget{1+\delta}{\epsilon} cannot be solved in polynomial time.
\end{lemma}
by combining~\cref{lem:larges,lem:usegadget}, we can get the proof of~\cref{thm:main}.
\begin{proof}[Proof of~\cref{thm:main}]
	According to~\cref{lem:larges}, under \conj{}, there exists two constants $0<\delta,\epsilon<1$ such that \gadget{1+\delta}{\epsilon} cannot be solved in polynomial time. According to~\cref{lem:usegadget}, there are no polynomial algorithm solving \oss{m^{\gamma\epsilon}}{m^{\gamma\epsilon}} when input $s=\Omega(m^{1+\gamma(1+\delta-1)})=\Omega(m^{1+\gamma\delta})$ for some constant $\gamma$. 
\end{proof}

\begin{proof}[Proof of~\cref{lem:larges}]
	Given a \labcov{} instance $\I=(A,B,E,\L,(\pi_e)_{e\in E})$ described in~\cref{def:labcov}, \conj{} implies we cannot distinguish the following two cases in polynomial time for some small constant $\epsl{}$ according to to~\cref{lem:pgc}. Remember that $N$ is the input bit length of $\I$.
	
	\begin{itemize}
		\item (Completeness:) There is a labeling that covers every edge. 
		
		\item (Soundness:) Any multilabeling of cost at most $N^{\epsl{}}(|A|+|B|)$ covers at most $N^{-\epsl{}}$ fraction of edges.      	
	\end{itemize}
	
	
	Assuming there is a polynomial time algorithm $\mathcal{A}$ solving \gadget{1+\delta}{\epsilon} for sufficiently small constant $\epsilon$, we will show how to distinguish the above two cases in polynomial time, which will lead to a contradiction.
	
	
	\paragraph{Definition of $G$.} First, we construct a graph $G$ (see~\cref{fig:large}) according to the instance $\I$. Let $\Delta=N^{c_{\Delta}}$ for a sufficiently large constant $c_{\Delta}$. Denote the graph described by~\cref{lem:geograph} with parameter $\Delta$ as $G_{geo}$. Remember that $G_{geo}$ has $\Delta^{\lan{}}$ layers, each with size $\Theta(\Delta^{10})$, where the first layer is $S=\{s_1,s_2,...,s_{K_1}\}$, the last layer is $T=\{t_1,t_2,...,t_{K_2}\}$.
	\begin{enumerate}
		\item $G$ has two sides: $A$ side and $B$ side. Each side contains $K_1$ \emph{batches}, each batch is composed by $|A|=|B|$ \emph{fans}. Let the reversed graph of $G_{geo}$ (where we reverse all the edge directions in this graph) as $G^R_{geo}$. On the $A$ side, each fan is composed of $\L$ copies of graph $G^R_{geo}$; on the $B$ side, each is composed of $\L$ copies of a graph $G_{geo}$. All $G^R_{geo}$ or $G_{geo}$ in the same fan share the same $T$ vertex set. On the $A/B$ side, the $k$-th copied graph in the $i$-th batch, $j$-th fan is denoted by $G^{A/B}_{i,j,k}$, where the node $s_\ell$ in $G_{geo}$ is denoted by $s^{A/B}_{(i,j),\ell}$ for $1\le \ell\le K_1$ (recall that for different $k$, they share the same $s_\ell$), and the node $t_\ell$ in $G_{geo}$ is denoted by $t^{A/B}_{(i,j,k),\ell}$ for $1\le \ell\le K_2$. 
		\item $G$ has edges defined by the \labcov{} instance $\I$ from $A$ side to $B$ side described as follows. Remember that $\I=(A,B,E,\L,(\pi_e)_{e\in E})$ where $A=\{1,2,...,|A|\},B=\{1,2,...,|B|\},\L=\{1,2,...,|\L|\}$. For every $(j,j')\in E,(k,k')\in\pi_{(j,j')}$ and $i,\ell\in[K_1]$, there is an edge from $s^A_{(i,j,k),\ell}$ to $s^B_{(\ell,j',k'),i}$. 
		An intuition of why the edges are defined in this way is that, now if a path is from the $A$ side of the $i$-th batch to $B$ side of the $\ell$-th batch, the path must go through $s^A_{(i,j,k),\ell}$ to $s^B_{(\ell,j',k'),i}$ for some $(j,j')\in E,(k,k')\in\pi_{(j,j')}$.
		\item For every $\ell\in[K_2]$, create a node $p^A_\ell$ and edges $\{(t^A_{(i,j),\ell},p^A_\ell)\mid 1\le i\le K_1,1\le j\le |A|)\}$. For every $i\in[K_1]$, create a node $q^B_i$ and edges $\{(q^B_i,t^B_{(i,j),\ell})\mid 1\le j\le |A|,1\le \ell\le K_2\}$. Remember that in~\cref{lem:geograph}, $P$ is a set of pairs between first and last layer indexes of nodes in $G_{geo}$. For every $\ell\in[K_2],i\in[K_1]$ where $(i,\ell)\not\in P$, we create an edge $(p^A_{\ell},q^B_{i})$. 
		
		We also create a mirror of the above nodes and edges (\cref{fig:large} does not draw) by reversing $A$ and $B$. i.e., for every $\ell\in[K_2]$, create a node $p^B_\ell$ with edges $\{(p^B_\ell,t^B_{(i,j),\ell})\mid 1\le i\le K_1,1\le j\le |A|)\}$. For every $i\in[K_1]$, create a node $q^A_i$ with edges $\{(t^A_{(i,j),\ell},q^A_i)\mid 1\le j\le |A|,1\le \ell\le K_2\}$. For every $\ell\in[K_2],i\in[K_1]$ where $(i,\ell)\not\in P$, we create an edge $(q^A_{i},p^B_{\ell})$. 
	\end{enumerate} 
	
	Remember that the number of edges in $G_{geo}$ is $M=O(\Delta^{11+\lan{}})$. Let $n$ be the number of nodes in $G$. Let $m$ be the number of edges in $G$. $m$ can be calculated as
	\[m={\color{gray}M\cdot K_1|A||\L|}+{\color{blue}O(|\L|^2|A|^2)\cdot K_1^2}+{\color{orange}4K_1K_2|A|}+{\color{red}O(K_1K_2)}=\Theta(M\cdot K_1|A||\L|)\]
	The third and fourth terms are trivially less than the first term. The second term is less than the first term since we set $\Delta=N^{c_\Delta}$ for sufficiently large constant $c_\Delta$. 
	\begin{figure}
		
		\begin{center}
			
			\begin{tikzpicture}[x={(0.5cm,0.3cm)}, y={(-0.5cm,0.3cm)}, z={(0cm,0.5cm)}]
				
				\newcommand{\con}{3}
				
				\newcommand{\drawpiece}[4]{
					\coordinate (A) at (0 * #1 + #2, -0.3 + #3, 0 + #4);
					\coordinate (B) at (0 * #1 + #2, 2.3 + #3, 0 + #4);
					\coordinate (A1) at (3 * #1 + #2, 2 + #3, 1 + #4);
					\coordinate (B1) at (3 * #1 + #2, 0 + #3, 1 + #4);
					\coordinate (A2) at (3 * #1 + #2, 2 + #3, 0 + #4);
					\coordinate (B2) at (3 * #1 + #2, 0 + #3, 0 + #4);
					\coordinate (A3) at (3 * #1 + #2, 2 + #3, -1 + #4);
					\coordinate (B3) at (3 * #1 + #2, 0 + #3, -1 + #4);
					\draw[fill=gray!30] (A) -- (B) -- (A3) -- (B3) -- cycle;
					\draw[fill=gray!30] (A) -- (B) -- (A2) -- (B2) -- cycle;
					\draw[fill=gray!30] (A) -- (B) -- (A1) -- (B1) -- cycle;
					
				}
				\newcommand{\drawstack}[5]{
					\fill[gray] (1.5 * #1 + #2,1 + #3,-2 + #5  + #4) circle (1pt);
					\fill[gray] (1.5 * #1 + #2,1 + #3,-2.5 + #5  + #4) circle (1pt);
					\fill[gray] (1.5 * #1 + #2,1 + #3,-3 + #5  + #4) circle (1pt);
					\drawpiece{#1}{#2}{#3}{#4 + #5}
					\drawpiece{#1}{#2}{#3}{#4}
				}
				\newcommand{\drawbatch}[6]{
					\fill[gray] (1.5 * #1 + #2,7 + #3,-1.5) circle (1pt);
					\fill[gray] (1.5 * #1 + #2,7.5 + #3,-1.5) circle (1pt);
					\fill[gray] (1.5 * #1 + #2,8 + #3,-1.5) circle (1pt);
					\drawstack{#1}{#2}{#3 + #6}{#4}{#5}
					\drawstack{#1}{#2}{#3}{#4}{#5}
				}
				
				\coordinate (qB1) at (15, 1.5,-1.5);
				
				\draw  (qB1) node[circle, fill, inner sep=1pt, label={[label distance=-0.1cm, color = orange]right:\fontsize{10}{0}\selectfont$q^B_{1}$},  color=orange] {};
				
				
				\draw[->, >=stealth, color=orange, line width=0.5pt] (qB1) -- (11,-0.3,0);
				\draw[->, >=stealth, color=orange, line width=0.5pt] (qB1) -- (11,2.3,0);
				
				\draw[->, >=stealth, color=orange, line width=0.5pt] (qB1) -- (11,-0.3,-3);
				\draw[->, >=stealth, color=orange, line width=0.5pt] (qB1) -- (11,2.3,-3);
				\coordinate (qB1) at (15, 1.5,-1.5);
				
				\coordinate (qB2) at (15, 1.5 + 3,-1.5);
				\draw  (qB2) node[circle, fill, inner sep=1pt, label={[label distance=-0.1cm, color = orange]above:\fontsize{10}{0}\selectfont$q^B_{2}$},  color=orange] {};
				
				
				\draw[->, >=stealth, color=orange, line width=0.5pt] (qB2) -- (11,-0.3 + 3,0);
				\draw[->, >=stealth, color=orange, line width=0.5pt] (qB2) -- (11,2.3 +  3,0);
				
				\draw[->, >=stealth, color=orange, line width=0.5pt] (qB2) -- (11,-0.3 +  3,-3);
				\draw[->, >=stealth, color=orange, line width=0.5pt] (qB2) -- (11,2.3 +  3,-3);
				
				\drawbatch{-1}{11}{0}{0}{-3}{3}
				
				\draw[blue, dashed, dash pattern=on 3pt off 3pt] (3,0,-6) -- (3,0,1) -- (5 + \con,0,1) -- (5 + \con,0,-6);
				\draw[blue, dashed, dash pattern=on 3pt off 3pt] (3,0.5,-6) -- (3,0.5,1) -- (5 + \con,3,1) -- (5 + \con,3,-6);
				\draw[blue, dashed, dash pattern=on 3pt off 3pt] (3,3,-6) -- (3,3,1) -- (5 + \con,0.5,1) -- (5 + \con,0.5,-6);
				\draw[blue, dashed, dash pattern=on 3pt off 3pt] (3,3.5,-6) -- (3,3.5,1) -- (5 + \con,3.5,1) -- (5 + \con,3.5,-6);
				
				\drawbatch{1}{0}{0}{0}{-3}{3}
				
				\node at (1.5, 1, 0.5) {$G^A_{1,1,1}$};
				
				\node at (1.5, 4, 0.5) {$G^A_{2,1,1}$};
				
				\node at (1.5, 1, -2.5) {$G^A_{1,2,1}$};
				
				\node at (9.5, 1, 0.5) {$G^B_{1,1,1}$};
				
				\draw (3,0,1) node[circle, fill, inner sep=1pt, label={[label distance=-0.1cm, color = blue]right:\fontsize{5}{0}\selectfont$s^A_{(1,1,1),1}$},  color=blue] {};
				
				\draw (3,0.5,1) node[circle, fill, inner sep=1pt, label={[label distance=-0.1cm, color = blue]left:\fontsize{5}{0}\selectfont$s^A_{(1,1,1),2}$},  color=blue] {};
				
				\draw (3,0,0) node[circle, fill, inner sep=1pt, label={[label distance=-0.1cm, color = blue]right:\fontsize{5}{0}\selectfont$s^A_{(1,1,2),1}$},  color=blue] {};
				
				\draw (3,0,-2) node[circle, fill, inner sep=1pt, label={[label distance=-0.1cm, color = blue]right:\fontsize{5}{0}\selectfont$s^A_{(1,2,1),1}$},  color=blue] {};
				
				\draw (3,3,1) node[circle, fill, inner sep=1pt, label={[label distance=-0.1cm, color = blue]left:\fontsize{5}{0}\selectfont$s^A_{(2,1,1),1}$},  color=blue] {};
				
				\draw (0,-0.3,0) node[circle, fill, inner sep=1pt, label={[label distance=-0.1cm, color = blue]above:\fontsize{5}{0}\selectfont$t^A_{(1,1),1}$},  color=blue] {};
				
				
				\draw (8,0,1) node[circle, fill, inner sep=1pt, label={[label distance=-0.1cm, color = blue]right:\fontsize{5}{0}\selectfont$s^B_{(1,1,1),1}$},  color=blue] {};
				\draw (11,-0.3,0) node[circle, fill, inner sep=1pt, label={[label distance=-0.1cm, color = blue]right:\fontsize{5}{0}\selectfont$t^B_{(1,1,1),1}$},  color=blue] {};
				
				\coordinate (pA1) at (-4,0,-1.5);
				
				\draw  (pA1) node[circle, fill, inner sep=1pt, label={[label distance=-0.1cm, color = orange]left:\fontsize{10}{0}\selectfont$p^A_{1}$},  color=orange] {};
				
				
				\draw[->, >=stealth, color=orange, line width=0.5pt] (0,-0.3,0) -- (pA1);
				\draw[->, >=stealth, color=orange, line width=0.5pt] (0,2.7,0) -- (pA1);
				
				\draw[->, >=stealth, color=orange, line width=0.5pt] (0,-0.3,-3) -- (pA1);
				\draw[->, >=stealth, color=orange, line width=0.5pt] (0,2.7,-3) -- (pA1);
				
				
				\coordinate (pA2) at (-4,2,-1.5);
				
				\draw  (pA2) node[circle, fill, inner sep=1pt, label={[label distance=-0.1cm, color = orange]below:\fontsize{10}{0}\selectfont$p^A_{2}$},  color=orange] {};
				
				
				\draw[->, >=stealth, color=orange, line width=0.5pt] (0,0.2,0) -- (pA2);
				\draw[->, >=stealth, color=orange, line width=0.5pt] (0,3.2,0) -- (pA2);
				
				\draw[->, >=stealth, color=orange, line width=0.5pt] (0,0.2,-3) -- (pA2);
				\draw[->, >=stealth, color=orange, line width=0.5pt] (0,3.2,-3) -- (pA2);
				\fill[gray] (-4,3,-1.5) circle (1pt);
				\fill[gray] (-4,3.5,-1.5) circle (1pt);
				\fill[gray] (-4,4,-1.5) circle (1pt);
				
				
				\coordinate (start) at (-4,0,-1.5);
				\coordinate (end) at (15,4.5,-1.5);
				
				\coordinate (cp1) at (-3,-4,-1.5);
				\coordinate (cp2) at (12,-5,-1.5);
				\coordinate (cp6) at (16,0,-1.5);
				
				\draw[->, color=red, line width=0.5pt] plot[smooth, tension=0.5] coordinates{(start) (cp1) (cp2) (cp6) (end)};
				
				
				
			\end{tikzpicture}
		\end{center}
		\caption{$G^A_{i,j,k}$ is one copy of the graph described in~\cref{lem:geograph} in the $i$-th batch (from right to left), $j$-th fan (from top to bottom) and $k$-th piece. For different $k$ and fixed $i,j$, $G^A_{i,j,k}$ shares the same first layer nodes (which are $s^A_{(i,j),\ell}$), but have different last layer nodes (which are $t^A_{(i,j,k),\ell}$). By changing $A$ to $B$ we get another side of the graph. Each dashed rectangle specifies the graph similar to the middle part in~\cref{fig:minrepgraph} according to the input \labcov{} instance $\I$. Fix $j,k$, for any $i,\ell$, we have $t^A_{(i,j,k),\ell}$ and $t^B_{(\ell,j,k),i}$ in the same dashed rectangle. }\label{fig:large}
	\end{figure}
	
	
	\paragraph{Solve \labcov{} using \gadget{1+\delta}{\epsilon}} Now we run the \gadget{1+\delta}{\epsilon} algorithm $\mathcal{A}$ with the following inputs. We need to verify that the inputs satisfy the requirements specified by~\cref{def:gadget}.
	\begin{enumerate}
		\item Graph $G$ with $m$ edges. The diameter of $G$ is $d=2\Delta^{\lan{}}$, which is polynomial on $m$.
		\item Two sets $L=\{t^A_{(i,j),k}\mid 1\le i\le K_1,1\le j\le |A|,1\le k\le |\L|\}, R=\{t^B_{(i,j),k}\mid  1\le i\le K_1,1\le j\le |A|,1\le k\le |\L|\}$. The size of each set is $K_1|A||\L|$ which is polynomial on $m$.
		\item A set of reachable vertex pairs $P'\subseteq L\times R$ defined as follows. Recall that in the \labcov{} instance, we have $A=\{1,...,|A|\},B=\{1,...,|B|\}$, and $P$ is the set of pairs defined in~\cref{lem:geograph}. For every $i\in[K_1]$ let $I_{i}$ contain all indexes $x\in[K_2]$ such that $(i,x)\in P$. Let $I_{i}[\ell]$ be the $\ell$-th element in $I_{i}$. 
		Now we define our $P'$.
		\[P'=\left\{\left(t^A_{(i,j),I_{i'}[\ell]},t^B_{(i',j'),I_{i}[\ell]}\right)\mid 1\le i,i'\le K_2,(j,j')\in E,1\le\ell\le\min(|I_{i'}|,|I_{i}|)\right\}\]
		We need to argue that $P'$ only contains reachable pairs. Notice that $t^A_{(i,j),I_{i'}[\ell]}$ can reach $s^A_{(i,j,k),i'}$ for any $k$ since $(i',I_{i}[\ell])\in P'$ (recall that $G^A_{(i,j,k)}$ is the reversed graph $G^R_{geo}$); for the same reason $t^B_{(i',j'),I_{i}[\ell]}$ can be reached from $s^B_{(i',j',k'),i}$ for any $k'$. Now we only need to argue that there exists $k,k'$ such that $t^A_{(i,j,k),i'}$ can reach $s^B_{(i',j',k'),i}$. We can take the $(k,k')\in \pi_{(j,j')}$ where $\pi_{(j,j')}$ must be non-empty since $(j,j')\in E$.
	\end{enumerate}
	Remember that the output of $\mathcal{A}$ will distinguish the following two types of instances. 
	\begin{description}
		\item[Type 1.] There exists a \ss{} $E'$ of $G$ with size $O(m^{1+\delta})$ such that all reachable pairs $(u,v)\in L\times R$ have distance $O(1)$ after adding $E'$ to $G$. %
		\item[Type 2.] By adding any \ss{} with size $O(m^{1+\delta+\epsilon})$, at most $o(1)$ fraction of pairs in $P$ have distance at most $d/3$.  
	\end{description}
	
	Next we show the output can already distinguish the \labcov{} instance (completeness) from (soundness).
	
	\paragraph{(Completeness) implies Type 1.} %
	Suppose the \mlab{} covering all edges in (completeness) is $\psi$. Recall that $P'$ is the pair set described in~\cref{lem:geograph}. We create a \ss{} $E'$ defined as
	\begin{equation*}
		\begin{aligned}
			E' &= \{(t^A_{(i,j),\ell},s^A_{(i,j,k),\ell'}) \mid i\in[K_1],j\in[|A|],k\in\psi(j),(\ell',\ell)\in P'\} \\
			&\cup \{(s^B_{(i,j,k),\ell},t^B_{(i,j),\ell'}) \mid i\in[K_1],j\in[|B|],k\in\psi(j),(\ell,\ell')\in P'\}
		\end{aligned}
	\end{equation*}
	We have $|E'|=O(K_1|A|\Delta^{12})$. Remember that $m=\Theta(M\cdot K_1|A||\L|)$, we have $|E'|=\Theta(m^{1+\delta})$ for some constant $\delta$. Now we prove that all reachable pairs $(t^A_{(i,j),\ell},t^B_{(i',j'),\ell'})$ has distance $O(1)$ after adding $E'$. If $(i',\ell)\not\in P$ or $(i,\ell')\not\in P$, then $t^A_{(i,j),\ell}$ can reach $t^B_{(i',j'),\ell'}$ in $3$ steps. Thus, we only consider the case when $(i',\ell),(i,\ell')\in P$. If $(j,j')\not\in E$, then $(t^A_{(i,j),\ell},t^B_{(i',j'),\ell'})$ is not reachable. Suppose $(j,j')\in E$, let $k\in\psi(j),k'\in\psi(j')$, since $(j,j)$ is covered by $\psi$, there is a blue edge $(s^A_{(i,j,k),i'},s^B_{(i',j',k'),i})$. Besides, we have $(t^A_{(i,j),\ell},s^A_{(i,j,k),i'}),(s^B_{(i',j',k'),i},t^B_{(i',j'),\ell'})\in E'$ due to the fact that $(i',\ell),(i,\ell')\in P$. Thus, the distance between $(t^A_{(i,j),\ell},t^B_{(i',j'),\ell'})$ is $3$. 
	
	\paragraph{(Soundness) implies Type 2.} We will prove it by contradiction. Suppose there exists a \ss{} $E'$ with size $O(m^{1+\delta+\epsilon})=O(K_1|A|\Delta^{12}\cdot n^\epsilon)$, after adding which, not $o(1)$ fraction of pairs in $P$ have distance at most $d/3=(2/3)\Delta^{\lan{}}$. We first turn $E'$ into another \ss{} $E''$ where each shortcut in $E''$ is totally in a copy of graph $G_{geo}$ in the following way. Suppose $(u,v)\in E'$ where $u,v$ are not in the same copy of $G_{geo}$, a path from $u$ to $v$ must cross at most two copies of $G_{geo}$, one on thie $A$ size and another on the $B$ size. We split this path into the two copies, and creat two edges connected two end points of both path
	. $E''$ will have size twice of $E'$, and the distance after adding $E''$ will be at most $(2/3)\Delta^{\lan{}}+2$.
	
	Recall that for every $i\in[K_1]$, we defined $I_{i}$ as all indexes $x\in[K_2]$ such that $(i,x)\in P$. We create \mlab{} $\psi_{i,i',\ell}$ in the following way: for every $j\in A,j'\in B$, we let 
	\[\psi_{i,i',\ell}(j)=\left\{k\in\L\mid\distt{G+E''}{t^A_{(i,j),I_{i'}[\ell]},s^A_{(i,j,k),i'}}<\Delta^{\lan{}}\right\}\]
	\[\psi_{i,i',\ell}(j)=\left\{k\in\L\mid\distt{G+E''}{s^B_{(i',j,k),i},t^B_{(i',j),I_{i}[\ell]}}<\Delta^{\lan{}}\right\}\]
	
	One importance observation is, for every $\psi_{i,i',\ell}$ and $k\in\L$, there is a unique path from $t^A_{(i,j),I_{i'}[\ell]}$ to $t^A_{(i,j,k),i'}$ or from $s^B_{(i',j',k),I_i[\ell]}$ to $t^B_{(i',j'),\ell}$; any two of these unique paths shares at most one edge. The reason is $(i,I_i[\ell])\in P$ (see~\cref{lem:geograph}). As a result, each edge in $E''$ can make at most one pair of vertices $(t^A_{(i,j),I_{i'}[\ell]},s^A_{(i,j,k),i'})$ or $(s^B_{(i',j,k),i},t^B_{(i',j),I_{i}[\ell]})$ to have distance less that the original distance $\Delta^{\lan{}}$. Therefore,
	\[\sum_{i,i'\in[K_1],\ell\le \min(|I_{i'}|,|I_{i}|)}\left(|\psi_{i,i',\ell}(j)|+|\psi_{i,i',\ell}(j)|\right)\le |E''|=O(K_1|A|\Delta^{12}\cdot n^\epsilon)\]
	
	According to~\cref{lem:geograph}, we know $\min(|I_{i'}|,|I_{i}|)=\Omega(\Delta^2)$, which means the number of \mlab{} $\psi_{i,i',\ell}$ is $T=\Theta(K_1^2\Delta^2)=\Theta(K_1\Delta^{12})$. Thus, only $o(K_1\Delta^{12})$ of them will have size at least $N^{\epsl{}}(|A|+|B|)$ for sufficiently small constant $\epsilon$, where $N$ is the input size of $\I$ which is polynomial on $n$. Let $C(\psi_{i,i',\ell})$ be the number of edges covered by $\psi_{i,i',\ell}$ for instance $\I$. We have 
	\[\sum_{i,i',\ell}C(\psi_{i,i',\ell})\le o(K_1\Delta^{12})\cdot |E|+T\cdot o(|E|)\le o(T|E|)\]
	
	One can see that if $\psi_{i,i',\ell}$ does not cover an edge $(j,j')$, then the distance from $t^A_{(i,j),I_{i'}[\ell]}$ to $t^B_{(i',j),I_{i}[\ell]}$ is at least $\Delta^{\lan{}}>(2/3)n^{c_D}$, where $(t^A_{(i,j),I_{i'}[\ell]},t^B_{(i',j),I_{i}[\ell]})\in P'$. That is because the only path from $t^A_{(i,j),I_{i'}[\ell]}$ to $t^B_{(i',j),I_{i}[\ell]}$ must be the concatenation of paths between $t^A_{(i,j),I_{i'}[\ell]},s^A_{(i,j,k),i'}$ and between $s^B_{(i',j,k'),i},t^B_{(i',j),I_{i}[\ell]}$ for some $k,k'\in\L$, where at least one of them has length $\Delta^{\lan{}}$. Therefore, we have at least $(1-o(1))T|E|=(1-o(1))|P'|$ pairs in $P'$ that have distance at least $\Delta^{\lan{}}$, which is a contradiction.%
	
\yonggang{the notations in this proof are disasters, I need to refine it}
\end{proof}



\section{Upper Bounds}\label{sec:upperbound}
We use the algorithm idea of a previous spanner approximation algorithm~\cite{BermanBMRY13}. We explain our algorithm in the language of the shortcut problem. 

\subsection{Preliminaries}



\paragraph{Large diameter case:} If we aim for a relatively large diameter, i.e., $d \cdot \alpha_D \geq n^{0.34}$, we can simply use the known result. 

\begin{theorem}[\cite{KoganP22}]
There is an efficient algorithm that, given input graph $G$, computes a $(d,s)$-shortcut for $d = \tilde{O}(n^{1/3})$ and $s = \tilde{O}(n)$.    
\end{theorem}

Using the above theorem, if $\apxD{}d=\Omega(n^{0.34})$, we can easily achieve $(\alpha_D, \alpha_S)$-approximation for all $\alpha_S = \tilde{O}(1)$. 
Therefore, we make the following assumption throughout this section. 

\begin{remark}\label{rem:assumption}
We can assume w.l.o.g. that $\alpha_D d = O(n^{0.34})$. 
\end{remark}

\paragraph{Reduction to DAGs:} We argue that DAGs, in some sense, capture hard instances for our problems. This will allow us to focus on DAGs in the subsequent sections.
The formal statement is encapsulated in the following lemma. 

\begin{lemma}
If there exists an efficient $(\alpha_D,\alpha_S)$ approximation algorithm for DAGs, then there exists an efficient $(3\alpha_D, 3\alpha_S)$-approximation algorithm for all directed graphs.  
\end{lemma}
\begin{proof}
Assume that we are given an access to the algorithm $\aset(G,s,d)$ that produces $(\alpha_D,\alpha_S)$ approximation algorithm for the DAG case. 
Let $G$ be an input digraph, together with the input parameters $(d,s)$.  
Compute a collection ${\mathcal S}$ of strongly connected components (SCC) of $G$, and let $G'$ be the DAG obtained by contracting each SCC into a single node. Invoke the algorithm $\aset(G',s,d)$. Let $E' \subseteq E(G')$ be the shortcut edges so that $|E'| \leq \alpha_S \cdot s$ and the diameter of $G' \cup E'$ is at most $\alpha_D \cdot d$. 
These edges would be responsible for connecting the pairs whose endpoints are in distinct SCCs. 

For each SCC $C \in {\mathcal S}$, we pick an arbitrary ``center'' $u_C \in V(C)$ and connect it to every other vertex in $V(C)$. These edges are called $E''_C$. Define $E'' = \bigcup_{C \in {\mathcal S}} E''_C$. 
The final shortcut $F$ can be constructed by combining these two sets $E'$ and $E''$: Edges in $E''$ can be added into $F$ directly. For each edge that connects $C$ to $C'$ in $E'$, we create the corresponding edge $(u_C, u_C')$ connecting the centers. Notice that $|F| \leq |E'| + |E''| \leq \alpha_S \cdot s + 2n \leq 3\alpha_S \cdot s$ (here we used the assumption that $s \geq n$). Moreover, it is easy to verify that, for each reachable pair $(v,w)$ in $G$ where $v \in C$ and $w \in C'$, there is a path from $v$ to $w$ of length at most $3 \alpha_D \cdot d$.     
\end{proof}




\subsection{Overview} \label{sec:ub-overview}

Suppose $G=(V,E)$ is a directed acyclic graph and $G^T=(V,E^T)$ be its transitive closure, i.e., $(u,v)\in E^T$ if $u$ has a directed path with length at least $1$ to $v$ in $G$. Since $G$ is acyclic, $G^T$ contains no self loop. 

\begin{definition}[Local graphs]\label{def:localgraph}
For a pair $u,v \in V(G)$, we define the local graph $G^{u,v}$. 
 Let $G^{u,v}=(V^{u,v},E^{u,v})$ be the subgraph of $G^T$ induced by the vertices that can reach $v$ and can be reached from $u$ (i.e., these vertices lie on at least one path from $u$ to $v$). 
\end{definition}
\begin{definition}[Thick and thin pairs]\label{def:thickthinedge}
 Let $u,v \in V(G)$. 
 If $|V^{u,v}|\ge \beta$ ($1\le \beta\le n$ will be determined later), the corresponding edge $(s,t)$ is said to be $\beta$-thick, and otherwise, it is $\beta$-thin. When $\beta$ is clear from context, we will simply write thick and thin pairs respectively. 
\end{definition}

Denote by $\pset$ the set of pairs of vertices that are reachable in $G$. We can partition $\pset$ into $\pset_{thick} \cup \pset_{thin}$. 

\begin{definition}
	Let $d' \in {\mathbb N}$. A set $E'\subseteq E^T \setminus E$ is said to $d'$-settle a pair $(u,v)\in E$ if $(V,E \cup E')$ contains a path of length at most $d'$ from $u$ to $v$. 
\end{definition}


Our algorithm will find two edge sets $F_1,F_2\subseteq E^T \setminus E$ such that the set $F_1$ is responsbile for $(\apxD{} d)$-settling all thick pairs, while $F_2$ will $(\apxD{} d)$-settle all thin pairs. The final solution be $F_1\cup F_2$ (Notice that $F_1\cup F_2$ $(\apxD{} d)$-settles all the pairs.)  These are encapsulated in the following two lemmas: 

\begin{restatable}[thick pairs]{lemma}{settlethick}\label{lem:settlethickedges}
We can efficiently compute $F_1$ such that $|F_1| \le O\left(\frac{n^2\log^2 n}{\beta\apxD{}^2 d^2}+
n \log n
\right)$ and it $(\apxD{} d)$-settles all thick pairs w.h.p. 
\end{restatable}

\begin{lemma}[thin pairs] \label{lem:settlethinpairs}
We can efficiently compute $F_2$ such that $|F_2| \leq O\left(\frac{\beta \log^2 n s}{\alpha_D}\right)$ and $(\alpha_D d)$-settles all thin pairs with high probability.      
\end{lemma}

We will prove these two lemmas later. Meanwhile, we complete the proof of Theorem~\ref{thm:upperbound}. 
We minimize the sum $|F_1| + |F_2|$ by setting the value of $\beta=\frac{n}{d\sqrt{s\apxD{}}}$.\footnote{The only problem is that this value could be much less than $1$ when $d\sqrt{s\apxD{}}>n$, which leads to $(d\apxD{})^2s>n^2$. However, in that case,  we can use the known tradeoff~\cite{KoganP22} to construct a \ssss{s}{\apxD{}d} when $d\sqrt{s\apxD{}}>n$.} 
	Now we have 
 \[|F_1\cup F_2|=O\left(\frac{ns^{0.5}\log^{2}n}{d\apxD{}^{1.5}}+n\log n\right)\le s\cdot O\left(\frac{n\log^{2}n}{d\apxD{}^{1.5}s^{0.5}}\right).\]

\subsection{Settling the thick pairs}\label{subsec:thickedges}

In this Section, we prove Lemma~\ref{lem:settlethickedges}.  For convenience we assume $\apxD{}d=\omega(\log n)$. We will show the case when $\apxD{}d=O(\log n)$ later.
We will use the idea from~\cite{KoganP22} to construct $F_1$. First, the following lemma allows us to decompose a DAG into a collection of paths and independent sets. 

\begin{lemma}[Theorem 3.2 \cite{GrandoniILPU21}]
	There is a polynomial time algorithm given an $n$-vertex acyclic graph $G=(V,E)$ and an integer $k \in[1,n]$, partition $G$ into $k$ directed paths  $P_1,...,P_{k}$ and at most $2n/k$ independent set  $Q_1,...,Q_{2n/k}$ in $G^T$. In other words, $P_1,...,P_{k},Q_1,...,Q_{2n/k}$ are disjoint and $\left(\cup_{i\in[k]}P_i\right)\cup \left(\cup_{i\in[2n/k]}Q_i\right)=V$.
\end{lemma}

We first apply the following lemma with $k=n/(\apxD{} d)$ to get $P_1,...,P_{8n/(\apxD{} d)}$ and $Q_1,...,Q_{\apxD{} d/4}$. Notice that $\apxD d=\omega(\log n)$ and $\apxD d= O(n^{0.34})$, we can safely assume $8n/(\apxD{} d)$ and $\apxD{} d/4$ are integers without loss of generality.
We will use Lemma 1.1~\cite{Raskhodnikova10}.
\begin{lemma}[Lemma 1.1~\cite{Raskhodnikova10}]\label{lem:pathshortcut}
    For any integer $n\ge 3$, the directed path with length $\ell$ has a $2$-shortcut with at most $\ell\log \ell$ edges.
\end{lemma}

We first add $n\log n$ edges to $F_1$ and reduce the diameter of every path $P_i$ to $2$. To accomplish this, for each path, we use \cref{lem:pathshortcut}. Since different paths are disjoint regarding vertices, $n\log n$ edges suffice.


Next, let $R \subseteq V$ be obtained by sampling $\min\left((9\log n)\cdot n/\beta, n\right)$ vertices from $V$ uniformly at random. The algorithm also samples $\min\left((999\log n)\cdot n/(\apxD{} d)^2,n/(\apxD{} d)\right)$ paths from $P_1,...,$ $P_{n/(\apxD{} d)}$ uniformly at random; let ${\mathcal Q}$ denote the set of sampled paths. For each vertex $u\in R$ and path $p\in {\mathcal Q}$, add $(u,v_1)$ to $F_1$ where $v_1$ is the first node in $p$ that $u$ can reach (if it exists), and add $(v_2,u)$ to $F_1$ where $v_2$ is the last node in $p$ that can reach $u$ (if it exists). Observe that $\beta$ is between $1$ and $n$, $(\apxD d)$ is between $\omega(\log n)$ and $O(n^{0.34})$, so both $(9\log n)\cdot n/\beta$ and $(999\log n)\cdot n/(\apxD{} d)^2$ will not be too small, and we can safely assume they are integers without loss of generality.




\settlethick*
\begin{proof}
It is straightforward to verify that $|F_1| \le (999n^2\log^2 n)/(\beta\apxD{}^2 d^2)+n\log n$ by the construction.

Suppose $(s,t)\in E^T$ is a thick pair. Since $|V^{s,t}|\ge \beta$, a vertex $u\in R\cap V^{s,t}$ exists with high probability (w.h.p.). %
Let $G'$ be the graph obtained after adding all edges that reduce the diameter for each $P_i$. Denote the shortest path between $s$ and $t$ in $G'$ as $P_{s,t}$. 
The path $P_{s,t}$ intersects each path $P_i$ at no more than three nodes. Otherwise, there are four nodes in $P_i\cap P_{s,t}$ $v_1,v_2,v_3,v_4$ such that $v_1$ has distance at least $3$ to $v_4$ since $P_{s,t}$ is a shortest path, which contradict the fact that $P_i$ has diameter $2$. The path $P_{s,t}$ intersects each $Q_i$ at no more than one node since $Q_i$ is an independent set on $E^T$.
Therefore, if we disregard all nodes in $Q_i$ from $i=1$to $i=\apxD{} d/ 4$ (which incurs an additional $\apxD{} d/4$ steps) and examine the first $\apxD{} d/4$ vertices and the last $\apxD{} d/4$ vertices of $P_{s,t}$, both of them will intersect a path in ${\mathcal Q}$ w.h.p. since we sample $(999\log n)\cdot n/(\apxD{} d)^2$ paths into ${\mathcal Q}$ among all $8n/(\apxD d)$ paths. This implies that $s$ can first use a $\apxD{} d/4+\apxD{} d/9+1$ path to reach $u$, and then $u$ can use another $1+\apxD{} d/9+\apxD{} d/4$ path to reach $t$. 
\end{proof}

For the case when $\apxD{}d=O(\log n),\apxD{}d>1$, to get similar result as~\cref{lem:settlethickedges}, we want $|F_1|=O(n^2\log n/\beta+n)$. This is easy to construct by sampling $(n/\beta)\log n$ nodes, where w.h.p. one of them will be in $V^{s,t}$ for every thick pair $s,t$. For every sample node $u$, we just need to include $(v,u)$ to $F_1$ for every $v$ that can reach $u$, and $(u,v)$ to $F_1$ for every $v$ that $u$ can reach. Now every thick pair has distance $2$ in $F_1$. 

For the extreme case when $\apxD{}d=1$, their is a unique way of adding edges, which is connecting every reachable pair by one edge, thus, we ignore this situation.

\subsection{Settling the thin pairs}\label{subsec:thinedges}



\begin{definition}[Critical sets]\label{def:antispanner}
A set $A\subseteq E^T$ is a $k$-critical set of a pair $(u,v)\in E^T$ if $E^T\backslash A$ contains no path from $uv$ to $v$ with a length of at most $k$. If there does not exist an $A'\subset A$ such that $A'$ is also a $k$-critical, then we say $A$ is minimal.
\end{definition}


\begin{definition}[$\mathcal{A}_k$]\label{def:antispannerset}
Let $\mathcal{A}_k$ consist of all sets $A$ satisfying both (i) $A$ is a minimal critical set of some thin pair, and (ii) $A\cap E=\emptyset$.
\end{definition}

This gives an alternate characterization of a shortcut set. 

\begin{claim}[Adapted from \cite{BermanBMRY13}] 
    An edge set $E'$ is a $d$-shortcut set for all thin pairs if and only if $E' \cap A \neq \emptyset$  for all $A\in\mathcal{A}_{d}$. 
\end{claim}

\begin{proof}
We briefly sketch the proof of this claim below.
\begin{itemize}
    \item[($\Rightarrow$)] We first argue that $E'$ must intersect each $A\in\mathcal{A}_d$ with at least one edge. Suppose, on the contrary, that $E'$ is disjoint from a set $A\in\mathcal{A}_d$, which is a $d$-critical set of some pair $(u,v)$. Then, $E'\cup E\subseteq E^T \backslash A$, and the distance between $u$ and $v$ in $E'\cup E$ is larger than $d$, which contradicts our assumption. 
    
    \item[($\Leftarrow$)] On the other hand, assume that $E'$ intersects every $A\in\mathcal{A}_d$ with at least one edge then it must be a $d$-shortcut. Otherwise, there exists a pair $(u,v)\in E^T $ such that, in $E'\cup E$, $u$ has a distance larger than $d$ to $v$. This implies that $A= E^T\backslash(E'\cup E)$ is a $d$-critical set of $u,v$ and $A \cap E' = \emptyset$ (a contradiction). 
\end{itemize}
\end{proof}


We define the following polytope $\cP_{G,d}$. It includes at most $n^2$ variables: for each $e\in E^T$, there is a corresponding variable $x_e$. The polytope has an exponentially large number of constraints. However, it must be non-empty since we assume $G$ admits a \ssss{s}{d}.


\begin{align*}
	\text{Polytope }\cP_{G,d}: \qquad\sum_{e\in E^T\backslash E}x_e &\le s \\
	\sum_{e\in A}x_e &\ge 1\qquad \forall A\in \mathcal{A}_d\\
	x_e &\geq 0\qquad \forall e\in E^T
\end{align*}

The algorithm employs the cutting-plane method to attempt to find a point within the polytope. According to established analyses, the running time of the cutting-plane method is polynomial with respect to the number of variables, provided that a separation oracle with a polynomial running time exists based on the number of variables. A separation oracle either asserts that the point lies within the polytope or returns a violated constraint, which can function as a cutting plane. Since the constraints $\sum_{e\in E^T\backslash E}x_e\le s$ and $x_e\ge 0$ can be verified in polynomial time, identifying a cutting plane is straightforward if either of these constraints is violated. Consequently, we assume $\sum_{e\in E^T\backslash E}x_e\le s$ and $x_e\ge 0$.

In the subsequent algorithm, we accept a point as input and either return a non-trivial violated constraint (one of $\sum_{e\in A}x_e\ge 1$), which can act as a separation plane, or output a set $E_2$ that settles all thin pairs.




\begin{algorithm}[H]
	
	\caption{{\sc Cut-or-Round}$({\bf x})$}\label{alg:seperationoracle}
	
	\KwData{A vector ${\bf x} \in [0,1]^{E^T \setminus E}$ satisfying $\sum_{e\in E^T\backslash E}x_e\le s$ and $x_e\ge 0$.}
	\KwResult{A set $A \in  \mathcal{A}_d$ s.t. $\sum_{e\in A}x_e<1$, or a set $F_2\subset E^T \setminus E$ that $(\apxD{} d)$-settles all thin pairs.}
	Include  edge edge $e\in E^T$ independently into $F_2$ with probability $(500\log n)(\beta/\apxD{})x_e$\;
	
	\If{all thin pairs are $(\apxD{} d)$-settled}{\If{$|F_2\backslash E|\le (1000\log^2 n)(\beta/\apxD{})s$}{\Return{$F_2$}\;}\Else{\Return{fail}\;}}
	\Else{
		Use $F_2$ to find a $(\apxD{} d)$-critical set $A'\in\mathcal{A}_{\apxD{} d}$ with $F_2\cap A'=\emptyset$ using the algorithm from Claim 2.4 in~\cite{BermanBMRY13}\;
		\If{$\sum_{e\in A'}x_e<\apxD{}/9$}{
			Find a $d$-critical set $A\subseteq \mathcal{A}_d$  such that $\sum_{e\in A}x_e\ge 1$ using~\cref{lem:findantispanner}\;
			\Return violated constraint $\sum_{e\in A}x_e\ge 1$\;}
		\Else{\Return{fail}\;}
	}
\end{algorithm}

\begin{claim}\label{lem:notfail}
	The algorithm will fail with probability at most $\frac{1}{n^{\omega(1)}}$.
\end{claim}
\begin{proof}
	The first fail condition is $|F_2\backslash E|>(1000\log^2 n)(\beta/\apxD{})s$. Notice that each edge $e$ is included in $F_2$ independently with probability $(500\log n)(\beta/\apxD{})x_e$ where $\sum_{e\in E^T\backslash E}x_e\le s$. By chernoff bound, $|F_2\backslash E|>(1000\log^2 n)(\beta/\apxD{}) s$ happens with probability at most $e^{-(\log n)(\beta/\apxD{}) s}$. Remember that $s\ge n$ and $\apxD{}=O(n^{0.34})$, which means we have $e^{-(\log n)(\beta/\apxD{}) s}\le n^{-\omega(1)}$.
	
	Now we show that the second fail condition happens with small probability. We define event $\mathcal{E}$ as ``all $B\subseteq A_{\apxD{} d}$ with $\sum_{e\in B}x_e\ge \apxD{}/9$ satisfies $B\cap F_2\not=\emptyset$''. If the second fail is triggered, then there exists a set $A'\in\mathcal{A}_{\apxD{} d}$ with $\sum_{e\in A'}x_e\ge \apxD{}/9$ satisfies $A'\cap F_2=\emptyset$, which means $\mathcal{E}$ does not happen. Thus, the probability that the second fail is triggered is bounded by $1-\Pr[\mathcal{E}]$. For $B\subseteq A_{\apxD{} d}$ with $\sum_{e\in B}x_e\ge \apxD{}/9$, the probability that $B\cap F_2=\emptyset$ is bounded by $\exp\left(-9\beta\log n\right)$ according to Chernoff bound (remember that each edges is included in $F_2$ with probability $(500\log n)(\beta/\apxD{})x_e$). Now we count the size of $\mathcal{A}_{\apxD{} d}$. According to claim 2.5 in~\cite{BermanBMRY13}, we have $|\mathcal{A}_{\apxD{} d}|\le |E|\cdot \beta^\beta\le \exp\left(2\beta\log\beta\right)$. Finally, by using union bound, $\Pr[\mathcal{E}]\ge 1-\frac{1}{n^{\omega(1)}}$.
\end{proof}
\begin{lemma}[critical set decomposition]\label{lem:findantispanner}
	There exists a polynomial time algorithm that, given $A'\in\mathcal{A}_{\apxD{} d}$ with $\sum_{e\in A'}x_e< \apxD{}/9$, outputs $A\in\mathcal{A}_{d}$ such that $\sum_{e\in A}x_e< 1$. 
\end{lemma}
\begin{proof}
	The algorithm first use polynomial time to find $(s,t)\in E^T$ such that $A'$ is a $(\apxD{} d)$-critical set of $(s,t)$. Then the algorithm constructs a shortest path tree rooted at $s$ on the subgraph $E^T\backslash A'$, where all the nodes in the $i$-th laryer of the tree has distance $i$ from $s$. Denote the vertex set of the $i$-th layer as $L_i$. According to the definition of critical set, $t$ is on at least the $(\apxD{} d+1)$-th layer. The algorithm devides the first $\apxD{} d$ layers into at least $\apxD{}/3$ batches, where the $i$-th batch contains all vertices between layer $2(i-1)d$ to $2i\cdot d-1$. Denote the set of all the edges in $A'$ that has at least one end point in the $i$-th batch as $A_i$. Since each edge in $A'$ will be included in at most two $A_i$, it is easy to see that at least one of $A_i$ has the property $\sum_{e\in A_i}x_e<1$. 
	
	Now we prove that for any $i$, $A_i\in\mathcal{A}_d$. Since $A_i$ is a subset of $A'$, the second condition, i.e., $A_i\cap E=\emptyset$ is satisfied trivially. We only need to verify that $A_i$ is a $d$-critical set of some edge in $E^T$. The idea is to choose a vertex in $L_{2(i-1)d}$ and another vertex in $L_{2id-1}$, and argue that they are reachable and have distance more than $d$ in $E^T\backslash A_i$. Let $S$ contain all the vertices in $L_{2(i-1)d}$ that has at least one edge towards a vertex in $L_{2id-1}$. Notice that $G$ (and also $G^T$) is acyclic (\cref{rem:assumption}), which also means the induced subgraph $G^T[S]$ is acyclic. Thus, there exists a vertex $u\in S$ such that it has no edge in $E^T$ towards other vertices in $S$. Take an arbitrary vertex $v\in L_{2id-1}$ such that $(u,v)\in E^T$, now we argue that $u$ has distance more than $d$ to $v$ in $E^T\backslash A_i$.  
	
	We prove it by induction. Induction hypothesis: any vertex that $u$ has distance $x$ to in $E^T\backslash A_i$ is one of the following two types (i) a vertex that cannot reach $v$ in $G$ (ii) a vertex in $L_{2(i-1)d+x}$. If the induction hypothesis is correct for any $0\le x\le d+10$, then $u$ cannot have distance at most $d$ to $v$ in $E^T\backslash A_i$. Now we prove the induction hypothesis. When $x=0$, the hypothesis is correct. For $x>0$, suppose $w_2$ is a vertex with $\distt{E^T\backslash A_i}{u,w_2}=x$, then there exists an edge $(w_1,w_2)\in E^T\backslash A_i$ such that $\distt{E^T\backslash A_i}{u,w_1}=x-1$. According to induction hypothesis, we can assume $w_1$ is either type (i) or type (ii). If $w_1$ is type (i), then $w_2$ cannot reach $v$ in $G$, which means $w_2$ is also type (i). Now suppose $w_1$ is type (ii) and can reach $v$ in $G$. First of all, $w_2$ cannot be in the first $2(i-1)d$-th layer, otherwise $w_2$ has a path $p$ in $G$ to $v$ where $p$ must contain a vertex in $L_{2(i-1)d}$ (two consecutive vertices in $p$ cannot skip a layer since all edges in $p$ is in $E$, which is also in $E^T\backslash A'$), which means $u$ has an edge in $E^T$ to another vertex in $L_{2(i-1)d}$, leading to a contradiction. Then, $w_2$ cannot be a vertex in $L_{2(i-1)d+b}$ where $1\le b<x$ since $u$ has distance less than $x$ to them according to induction hypothesis. $w_2$ cannot be a vertex in $L_{2(i-1)d+b}$ where $b>x$, otherwise $(w_1,w_2)$ is not in $A_i$, also not in $A'$, which contradicts the fact that layers are constructed by shortest path tree on $E^T\backslash A'$. $w_2$ cannot be a vertex outside the tree because $s$ can reach $w_2$ in $E^T\backslash A'$. Finially, $w_2\in L_{2(i-1)d+x}$ and the induction hypothesis holds. 
\end{proof}

As mentioned in \Cref{sec:ub-overview}, Theorem~\ref{thm:upperbound} now follows combining the size of sets $F_1$ and $F_2$ from \Cref{lem:settlethickedges} and \Cref{lem:settlethinpairs} respectively.


 


\subsection{Research should be publicly available, and your software is part of it}\label{sec:openSource}

\paragraph{Background} Only openly available research software allows other researchers to reproduce your research results. Since research software is mainly publicly funded, it is also fair to make it publicly available. This also enables others to reuse the software, providing additional benefits like further testing of the software and contributing by creating extensions and compatible software. Besides making the software open source, it is also important to register it so other researchers can find it. These aspects are essential to make a research software \ac{FAIR}~\cite{barker2022introducing}. 

\paragraph{Recommendations} We recommend developing your research software open source. By directly starting the development as open source, challenges can be avoided when switching to open source later. Further, it allows other researchers to use your software and actively contribute as early as possible, improving your software's quality and encouraging cooperation. To develop the software open source, a suitable license is required. By choosing a license early each time another software is included, the compatibility of licenses can be checked directly. Some open source licenses are incompatible since they put certain conditions on the reuse of the code, which can conflict between licenses \cite{cui_empirical_2023}.
When choosing a license, you should check your institution's policy, which often already proposes a specific license. Also, various guides help to choose a license\footnote{e.g., \url{https://choosealicense.com/}, last access 2024-12-17.}.

Cooperation and joint research projects with industry are common in energy research. Sometimes, industry partners are critical of developing open source because they want to keep their intellectual property. Generally, we recommend discussing this topic as early as possible to find reasonable compromises. Often, certain parts can be developed open source while others remain closed source. Since open source is open to all researchers and all industries, it can also improve the exchange between industry and research. 

When you develop open source, there are specific approaches to make your code more easily reusable by others. First, the repository should follow programming-language-specific best practices. This can be achieved by starting with templates for the repository\footnote{e.g., cookie-cutter templates at \url{https://cookiecutter.io/templates}, last access 2024-12-17}. Additionally, citing your software can be made easy by including a \ac{CFF} file\footnote{\url{https://citation-file-format.github.io}, last access 2024-12-17.} in your repository
 \cite{druskat_citation_2021}.

Within the repository, it should be indicated if the software will be maintained, if support is available and if the developers are open to joining research projects, including the software. We recommend using the features of the software platform (GitHub/GitLab) to interact with potential users, e.g., by using issues.

Besides making the source code available, the software should also be findable. Therefore, we recommend registering the software in a domain-specific registry like the Open Energy Platform\footref{fn:oep}. Getting a DOI for the software is also helpful, e.g., by archiving versions of the software on Zenodo\footnote{\url{https://zenodo.org/}, last access 2024-12-17.} \cite{zenodo}. 

\par
Energy research is highly interdisciplinary \cite{tijssen_quantitative_1992}.
Therefore, we recommend adding a very general description to your software, allowing all researchers to understand its goals. This way, more researchers can identify whether the software is useful for them, increasing its reusability. GitHub also allows you to provide keywords to your repository, which again improves findability.


\par 
Platforms like GitHub and GitLab make it easy to publish your software under an open source license. Open source research software enables reproducibility and allows reusability, which can also improve your code quality. Therefore, we recommend: 

\recommendation{Develop open source \& make your software findable!}

% This must be in the first 5 lines to tell arXiv to use pdfLaTeX, which is strongly recommended.
\pdfoutput=1
% In particular, the hyperref package requires pdfLaTeX in order to break URLs across lines.

\documentclass[11pt]{article}

% Change "review" to "final" to generate the final (sometimes called camera-ready) version.
% Change to "preprint" to generate a non-anonymous version with page numbers.
\usepackage{acl}

% Standard package includes
\usepackage{times}
\usepackage{latexsym}

% Draw tables
\usepackage{booktabs}
\usepackage{multirow}
\usepackage{xcolor}
\usepackage{colortbl}
\usepackage{array} 
\usepackage{amsmath}

\newcolumntype{C}{>{\centering\arraybackslash}p{0.07\textwidth}}
% For proper rendering and hyphenation of words containing Latin characters (including in bib files)
\usepackage[T1]{fontenc}
% For Vietnamese characters
% \usepackage[T5]{fontenc}
% See https://www.latex-project.org/help/documentation/encguide.pdf for other character sets
% This assumes your files are encoded as UTF8
\usepackage[utf8]{inputenc}

% This is not strictly necessary, and may be commented out,
% but it will improve the layout of the manuscript,
% and will typically save some space.
\usepackage{microtype}
\DeclareMathOperator*{\argmax}{arg\,max}
% This is also not strictly necessary, and may be commented out.
% However, it will improve the aesthetics of text in
% the typewriter font.
\usepackage{inconsolata}

%Including images in your LaTeX document requires adding
%additional package(s)
\usepackage{graphicx}
% If the title and author information does not fit in the area allocated, uncomment the following
%
%\setlength\titlebox{<dim>}
%
% and set <dim> to something 5cm or larger.

\title{Wi-Chat: Large Language Model Powered Wi-Fi Sensing}

% Author information can be set in various styles:
% For several authors from the same institution:
% \author{Author 1 \and ... \and Author n \\
%         Address line \\ ... \\ Address line}
% if the names do not fit well on one line use
%         Author 1 \\ {\bf Author 2} \\ ... \\ {\bf Author n} \\
% For authors from different institutions:
% \author{Author 1 \\ Address line \\  ... \\ Address line
%         \And  ... \And
%         Author n \\ Address line \\ ... \\ Address line}
% To start a separate ``row'' of authors use \AND, as in
% \author{Author 1 \\ Address line \\  ... \\ Address line
%         \AND
%         Author 2 \\ Address line \\ ... \\ Address line \And
%         Author 3 \\ Address line \\ ... \\ Address line}

% \author{First Author \\
%   Affiliation / Address line 1 \\
%   Affiliation / Address line 2 \\
%   Affiliation / Address line 3 \\
%   \texttt{email@domain} \\\And
%   Second Author \\
%   Affiliation / Address line 1 \\
%   Affiliation / Address line 2 \\
%   Affiliation / Address line 3 \\
%   \texttt{email@domain} \\}
% \author{Haohan Yuan \qquad Haopeng Zhang\thanks{corresponding author} \\ 
%   ALOHA Lab, University of Hawaii at Manoa \\
%   % Affiliation / Address line 2 \\
%   % Affiliation / Address line 3 \\
%   \texttt{\{haohany,haopengz\}@hawaii.edu}}
  
\author{
{Haopeng Zhang$\dag$\thanks{These authors contributed equally to this work.}, Yili Ren$\ddagger$\footnotemark[1], Haohan Yuan$\dag$, Jingzhe Zhang$\ddagger$, Yitong Shen$\ddagger$} \\
ALOHA Lab, University of Hawaii at Manoa$\dag$, University of South Florida$\ddagger$ \\
\{haopengz, haohany\}@hawaii.edu\\
\{yiliren, jingzhe, shen202\}@usf.edu\\}



  
%\author{
%  \textbf{First Author\textsuperscript{1}},
%  \textbf{Second Author\textsuperscript{1,2}},
%  \textbf{Third T. Author\textsuperscript{1}},
%  \textbf{Fourth Author\textsuperscript{1}},
%\\
%  \textbf{Fifth Author\textsuperscript{1,2}},
%  \textbf{Sixth Author\textsuperscript{1}},
%  \textbf{Seventh Author\textsuperscript{1}},
%  \textbf{Eighth Author \textsuperscript{1,2,3,4}},
%\\
%  \textbf{Ninth Author\textsuperscript{1}},
%  \textbf{Tenth Author\textsuperscript{1}},
%  \textbf{Eleventh E. Author\textsuperscript{1,2,3,4,5}},
%  \textbf{Twelfth Author\textsuperscript{1}},
%\\
%  \textbf{Thirteenth Author\textsuperscript{3}},
%  \textbf{Fourteenth F. Author\textsuperscript{2,4}},
%  \textbf{Fifteenth Author\textsuperscript{1}},
%  \textbf{Sixteenth Author\textsuperscript{1}},
%\\
%  \textbf{Seventeenth S. Author\textsuperscript{4,5}},
%  \textbf{Eighteenth Author\textsuperscript{3,4}},
%  \textbf{Nineteenth N. Author\textsuperscript{2,5}},
%  \textbf{Twentieth Author\textsuperscript{1}}
%\\
%\\
%  \textsuperscript{1}Affiliation 1,
%  \textsuperscript{2}Affiliation 2,
%  \textsuperscript{3}Affiliation 3,
%  \textsuperscript{4}Affiliation 4,
%  \textsuperscript{5}Affiliation 5
%\\
%  \small{
%    \textbf{Correspondence:} \href{mailto:email@domain}{email@domain}
%  }
%}

\begin{document}
\maketitle
\begin{abstract}
Recent advancements in Large Language Models (LLMs) have demonstrated remarkable capabilities across diverse tasks. However, their potential to integrate physical model knowledge for real-world signal interpretation remains largely unexplored. In this work, we introduce Wi-Chat, the first LLM-powered Wi-Fi-based human activity recognition system. We demonstrate that LLMs can process raw Wi-Fi signals and infer human activities by incorporating Wi-Fi sensing principles into prompts. Our approach leverages physical model insights to guide LLMs in interpreting Channel State Information (CSI) data without traditional signal processing techniques. Through experiments on real-world Wi-Fi datasets, we show that LLMs exhibit strong reasoning capabilities, achieving zero-shot activity recognition. These findings highlight a new paradigm for Wi-Fi sensing, expanding LLM applications beyond conventional language tasks and enhancing the accessibility of wireless sensing for real-world deployments.
\end{abstract}

\section{Introduction}

In today’s rapidly evolving digital landscape, the transformative power of web technologies has redefined not only how services are delivered but also how complex tasks are approached. Web-based systems have become increasingly prevalent in risk control across various domains. This widespread adoption is due their accessibility, scalability, and ability to remotely connect various types of users. For example, these systems are used for process safety management in industry~\cite{kannan2016web}, safety risk early warning in urban construction~\cite{ding2013development}, and safe monitoring of infrastructural systems~\cite{repetto2018web}. Within these web-based risk management systems, the source search problem presents a huge challenge. Source search refers to the task of identifying the origin of a risky event, such as a gas leak and the emission point of toxic substances. This source search capability is crucial for effective risk management and decision-making.

Traditional approaches to implementing source search capabilities into the web systems often rely on solely algorithmic solutions~\cite{ristic2016study}. These methods, while relatively straightforward to implement, often struggle to achieve acceptable performances due to algorithmic local optima and complex unknown environments~\cite{zhao2020searching}. More recently, web crowdsourcing has emerged as a promising alternative for tackling the source search problem by incorporating human efforts in these web systems on-the-fly~\cite{zhao2024user}. This approach outsources the task of addressing issues encountered during the source search process to human workers, leveraging their capabilities to enhance system performance.

These solutions often employ a human-AI collaborative way~\cite{zhao2023leveraging} where algorithms handle exploration-exploitation and report the encountered problems while human workers resolve complex decision-making bottlenecks to help the algorithms getting rid of local deadlocks~\cite{zhao2022crowd}. Although effective, this paradigm suffers from two inherent limitations: increased operational costs from continuous human intervention, and slow response times of human workers due to sequential decision-making. These challenges motivate our investigation into developing autonomous systems that preserve human-like reasoning capabilities while reducing dependency on massive crowdsourced labor.

Furthermore, recent advancements in large language models (LLMs)~\cite{chang2024survey} and multi-modal LLMs (MLLMs)~\cite{huang2023chatgpt} have unveiled promising avenues for addressing these challenges. One clear opportunity involves the seamless integration of visual understanding and linguistic reasoning for robust decision-making in search tasks. However, whether large models-assisted source search is really effective and efficient for improving the current source search algorithms~\cite{ji2022source} remains unknown. \textit{To address the research gap, we are particularly interested in answering the following two research questions in this work:}

\textbf{\textit{RQ1: }}How can source search capabilities be integrated into web-based systems to support decision-making in time-sensitive risk management scenarios? 
% \sq{I mention ``time-sensitive'' here because I feel like we shall say something about the response time -- LLM has to be faster than humans}

\textbf{\textit{RQ2: }}How can MLLMs and LLMs enhance the effectiveness and efficiency of existing source search algorithms? 

% \textit{\textbf{RQ2:}} To what extent does the performance of large models-assisted search align with or approach the effectiveness of human-AI collaborative search? 

To answer the research questions, we propose a novel framework called Auto-\
S$^2$earch (\textbf{Auto}nomous \textbf{S}ource \textbf{Search}) and implement a prototype system that leverages advanced web technologies to simulate real-world conditions for zero-shot source search. Unlike traditional methods that rely on pre-defined heuristics or extensive human intervention, AutoS$^2$earch employs a carefully designed prompt that encapsulates human rationales, thereby guiding the MLLM to generate coherent and accurate scene descriptions from visual inputs about four directional choices. Based on these language-based descriptions, the LLM is enabled to determine the optimal directional choice through chain-of-thought (CoT) reasoning. Comprehensive empirical validation demonstrates that AutoS$^2$-\ 
earch achieves a success rate of 95–98\%, closely approaching the performance of human-AI collaborative search across 20 benchmark scenarios~\cite{zhao2023leveraging}. 

Our work indicates that the role of humans in future web crowdsourcing tasks may evolve from executors to validators or supervisors. Furthermore, incorporating explanations of LLM decisions into web-based system interfaces has the potential to help humans enhance task performance in risk control.






\section{Related Work}
\label{sec:relatedworks}

% \begin{table*}[t]
% \centering 
% \renewcommand\arraystretch{0.98}
% \fontsize{8}{10}\selectfont \setlength{\tabcolsep}{0.4em}
% \begin{tabular}{@{}lc|cc|cc|cc@{}}
% \toprule
% \textbf{Methods}           & \begin{tabular}[c]{@{}c@{}}\textbf{Training}\\ \textbf{Paradigm}\end{tabular} & \begin{tabular}[c]{@{}c@{}}\textbf{$\#$ PT Data}\\ \textbf{(Tokens)}\end{tabular} & \begin{tabular}[c]{@{}c@{}}\textbf{$\#$ IFT Data}\\ \textbf{(Samples)}\end{tabular} & \textbf{Code}  & \begin{tabular}[c]{@{}c@{}}\textbf{Natural}\\ \textbf{Language}\end{tabular} & \begin{tabular}[c]{@{}c@{}}\textbf{Action}\\ \textbf{Trajectories}\end{tabular} & \begin{tabular}[c]{@{}c@{}}\textbf{API}\\ \textbf{Documentation}\end{tabular}\\ \midrule 
% NexusRaven~\citep{srinivasan2023nexusraven} & IFT & - & - & \textcolor{green}{\CheckmarkBold} & \textcolor{green}{\CheckmarkBold} &\textcolor{red}{\XSolidBrush}&\textcolor{red}{\XSolidBrush}\\
% AgentInstruct~\citep{zeng2023agenttuning} & IFT & - & 2k & \textcolor{green}{\CheckmarkBold} & \textcolor{green}{\CheckmarkBold} &\textcolor{red}{\XSolidBrush}&\textcolor{red}{\XSolidBrush} \\
% AgentEvol~\citep{xi2024agentgym} & IFT & - & 14.5k & \textcolor{green}{\CheckmarkBold} & \textcolor{green}{\CheckmarkBold} &\textcolor{green}{\CheckmarkBold}&\textcolor{red}{\XSolidBrush} \\
% Gorilla~\citep{patil2023gorilla}& IFT & - & 16k & \textcolor{green}{\CheckmarkBold} & \textcolor{green}{\CheckmarkBold} &\textcolor{red}{\XSolidBrush}&\textcolor{green}{\CheckmarkBold}\\
% OpenFunctions-v2~\citep{patil2023gorilla} & IFT & - & 65k & \textcolor{green}{\CheckmarkBold} & \textcolor{green}{\CheckmarkBold} &\textcolor{red}{\XSolidBrush}&\textcolor{green}{\CheckmarkBold}\\
% LAM~\citep{zhang2024agentohana} & IFT & - & 42.6k & \textcolor{green}{\CheckmarkBold} & \textcolor{green}{\CheckmarkBold} &\textcolor{green}{\CheckmarkBold}&\textcolor{red}{\XSolidBrush} \\
% xLAM~\citep{liu2024apigen} & IFT & - & 60k & \textcolor{green}{\CheckmarkBold} & \textcolor{green}{\CheckmarkBold} &\textcolor{green}{\CheckmarkBold}&\textcolor{red}{\XSolidBrush} \\\midrule
% LEMUR~\citep{xu2024lemur} & PT & 90B & 300k & \textcolor{green}{\CheckmarkBold} & \textcolor{green}{\CheckmarkBold} &\textcolor{green}{\CheckmarkBold}&\textcolor{red}{\XSolidBrush}\\
% \rowcolor{teal!12} \method & PT & 103B & 95k & \textcolor{green}{\CheckmarkBold} & \textcolor{green}{\CheckmarkBold} & \textcolor{green}{\CheckmarkBold} & \textcolor{green}{\CheckmarkBold} \\
% \bottomrule
% \end{tabular}
% \caption{Summary of existing tuning- and pretraining-based LLM agents with their training sample sizes. "PT" and "IFT" denote "Pre-Training" and "Instruction Fine-Tuning", respectively. }
% \label{tab:related}
% \end{table*}

\begin{table*}[ht]
\begin{threeparttable}
\centering 
\renewcommand\arraystretch{0.98}
\fontsize{7}{9}\selectfont \setlength{\tabcolsep}{0.2em}
\begin{tabular}{@{}l|c|c|ccc|cc|cc|cccc@{}}
\toprule
\textbf{Methods} & \textbf{Datasets}           & \begin{tabular}[c]{@{}c@{}}\textbf{Training}\\ \textbf{Paradigm}\end{tabular} & \begin{tabular}[c]{@{}c@{}}\textbf{\# PT Data}\\ \textbf{(Tokens)}\end{tabular} & \begin{tabular}[c]{@{}c@{}}\textbf{\# IFT Data}\\ \textbf{(Samples)}\end{tabular} & \textbf{\# APIs} & \textbf{Code}  & \begin{tabular}[c]{@{}c@{}}\textbf{Nat.}\\ \textbf{Lang.}\end{tabular} & \begin{tabular}[c]{@{}c@{}}\textbf{Action}\\ \textbf{Traj.}\end{tabular} & \begin{tabular}[c]{@{}c@{}}\textbf{API}\\ \textbf{Doc.}\end{tabular} & \begin{tabular}[c]{@{}c@{}}\textbf{Func.}\\ \textbf{Call}\end{tabular} & \begin{tabular}[c]{@{}c@{}}\textbf{Multi.}\\ \textbf{Step}\end{tabular}  & \begin{tabular}[c]{@{}c@{}}\textbf{Plan}\\ \textbf{Refine}\end{tabular}  & \begin{tabular}[c]{@{}c@{}}\textbf{Multi.}\\ \textbf{Turn}\end{tabular}\\ \midrule 
\multicolumn{13}{l}{\emph{Instruction Finetuning-based LLM Agents for Intrinsic Reasoning}}  \\ \midrule
FireAct~\cite{chen2023fireact} & FireAct & IFT & - & 2.1K & 10 & \textcolor{red}{\XSolidBrush} &\textcolor{green}{\CheckmarkBold} &\textcolor{green}{\CheckmarkBold}  & \textcolor{red}{\XSolidBrush} &\textcolor{green}{\CheckmarkBold} & \textcolor{red}{\XSolidBrush} &\textcolor{green}{\CheckmarkBold} & \textcolor{red}{\XSolidBrush} \\
ToolAlpaca~\cite{tang2023toolalpaca} & ToolAlpaca & IFT & - & 4.0K & 400 & \textcolor{red}{\XSolidBrush} &\textcolor{green}{\CheckmarkBold} &\textcolor{green}{\CheckmarkBold} & \textcolor{red}{\XSolidBrush} &\textcolor{green}{\CheckmarkBold} & \textcolor{red}{\XSolidBrush}  &\textcolor{green}{\CheckmarkBold} & \textcolor{red}{\XSolidBrush}  \\
ToolLLaMA~\cite{qin2023toolllm} & ToolBench & IFT & - & 12.7K & 16,464 & \textcolor{red}{\XSolidBrush} &\textcolor{green}{\CheckmarkBold} &\textcolor{green}{\CheckmarkBold} &\textcolor{red}{\XSolidBrush} &\textcolor{green}{\CheckmarkBold}&\textcolor{green}{\CheckmarkBold}&\textcolor{green}{\CheckmarkBold} &\textcolor{green}{\CheckmarkBold}\\
AgentEvol~\citep{xi2024agentgym} & AgentTraj-L & IFT & - & 14.5K & 24 &\textcolor{red}{\XSolidBrush} & \textcolor{green}{\CheckmarkBold} &\textcolor{green}{\CheckmarkBold}&\textcolor{red}{\XSolidBrush} &\textcolor{green}{\CheckmarkBold}&\textcolor{red}{\XSolidBrush} &\textcolor{red}{\XSolidBrush} &\textcolor{green}{\CheckmarkBold}\\
Lumos~\cite{yin2024agent} & Lumos & IFT  & - & 20.0K & 16 &\textcolor{red}{\XSolidBrush} & \textcolor{green}{\CheckmarkBold} & \textcolor{green}{\CheckmarkBold} &\textcolor{red}{\XSolidBrush} & \textcolor{green}{\CheckmarkBold} & \textcolor{green}{\CheckmarkBold} &\textcolor{red}{\XSolidBrush} & \textcolor{green}{\CheckmarkBold}\\
Agent-FLAN~\cite{chen2024agent} & Agent-FLAN & IFT & - & 24.7K & 20 &\textcolor{red}{\XSolidBrush} & \textcolor{green}{\CheckmarkBold} & \textcolor{green}{\CheckmarkBold} &\textcolor{red}{\XSolidBrush} & \textcolor{green}{\CheckmarkBold}& \textcolor{green}{\CheckmarkBold}&\textcolor{red}{\XSolidBrush} & \textcolor{green}{\CheckmarkBold}\\
AgentTuning~\citep{zeng2023agenttuning} & AgentInstruct & IFT & - & 35.0K & - &\textcolor{red}{\XSolidBrush} & \textcolor{green}{\CheckmarkBold} & \textcolor{green}{\CheckmarkBold} &\textcolor{red}{\XSolidBrush} & \textcolor{green}{\CheckmarkBold} &\textcolor{red}{\XSolidBrush} &\textcolor{red}{\XSolidBrush} & \textcolor{green}{\CheckmarkBold}\\\midrule
\multicolumn{13}{l}{\emph{Instruction Finetuning-based LLM Agents for Function Calling}} \\\midrule
NexusRaven~\citep{srinivasan2023nexusraven} & NexusRaven & IFT & - & - & 116 & \textcolor{green}{\CheckmarkBold} & \textcolor{green}{\CheckmarkBold}  & \textcolor{green}{\CheckmarkBold} &\textcolor{red}{\XSolidBrush} & \textcolor{green}{\CheckmarkBold} &\textcolor{red}{\XSolidBrush} &\textcolor{red}{\XSolidBrush}&\textcolor{red}{\XSolidBrush}\\
Gorilla~\citep{patil2023gorilla} & Gorilla & IFT & - & 16.0K & 1,645 & \textcolor{green}{\CheckmarkBold} &\textcolor{red}{\XSolidBrush} &\textcolor{red}{\XSolidBrush}&\textcolor{green}{\CheckmarkBold} &\textcolor{green}{\CheckmarkBold} &\textcolor{red}{\XSolidBrush} &\textcolor{red}{\XSolidBrush} &\textcolor{red}{\XSolidBrush}\\
OpenFunctions-v2~\citep{patil2023gorilla} & OpenFunctions-v2 & IFT & - & 65.0K & - & \textcolor{green}{\CheckmarkBold} & \textcolor{green}{\CheckmarkBold} &\textcolor{red}{\XSolidBrush} &\textcolor{green}{\CheckmarkBold} &\textcolor{green}{\CheckmarkBold} &\textcolor{red}{\XSolidBrush} &\textcolor{red}{\XSolidBrush} &\textcolor{red}{\XSolidBrush}\\
API Pack~\cite{guo2024api} & API Pack & IFT & - & 1.1M & 11,213 &\textcolor{green}{\CheckmarkBold} &\textcolor{red}{\XSolidBrush} &\textcolor{green}{\CheckmarkBold} &\textcolor{red}{\XSolidBrush} &\textcolor{green}{\CheckmarkBold} &\textcolor{red}{\XSolidBrush}&\textcolor{red}{\XSolidBrush}&\textcolor{red}{\XSolidBrush}\\ 
LAM~\citep{zhang2024agentohana} & AgentOhana & IFT & - & 42.6K & - & \textcolor{green}{\CheckmarkBold} & \textcolor{green}{\CheckmarkBold} &\textcolor{green}{\CheckmarkBold}&\textcolor{red}{\XSolidBrush} &\textcolor{green}{\CheckmarkBold}&\textcolor{red}{\XSolidBrush}&\textcolor{green}{\CheckmarkBold}&\textcolor{green}{\CheckmarkBold}\\
xLAM~\citep{liu2024apigen} & APIGen & IFT & - & 60.0K & 3,673 & \textcolor{green}{\CheckmarkBold} & \textcolor{green}{\CheckmarkBold} &\textcolor{green}{\CheckmarkBold}&\textcolor{red}{\XSolidBrush} &\textcolor{green}{\CheckmarkBold}&\textcolor{red}{\XSolidBrush}&\textcolor{green}{\CheckmarkBold}&\textcolor{green}{\CheckmarkBold}\\\midrule
\multicolumn{13}{l}{\emph{Pretraining-based LLM Agents}}  \\\midrule
% LEMUR~\citep{xu2024lemur} & PT & 90B & 300.0K & - & \textcolor{green}{\CheckmarkBold} & \textcolor{green}{\CheckmarkBold} &\textcolor{green}{\CheckmarkBold}&\textcolor{red}{\XSolidBrush} & \textcolor{red}{\XSolidBrush} &\textcolor{green}{\CheckmarkBold} &\textcolor{red}{\XSolidBrush}&\textcolor{red}{\XSolidBrush}\\
\rowcolor{teal!12} \method & \dataset & PT & 103B & 95.0K  & 76,537  & \textcolor{green}{\CheckmarkBold} & \textcolor{green}{\CheckmarkBold} & \textcolor{green}{\CheckmarkBold} & \textcolor{green}{\CheckmarkBold} & \textcolor{green}{\CheckmarkBold} & \textcolor{green}{\CheckmarkBold} & \textcolor{green}{\CheckmarkBold} & \textcolor{green}{\CheckmarkBold}\\
\bottomrule
\end{tabular}
% \begin{tablenotes}
%     \item $^*$ In addition, the StarCoder-API can offer 4.77M more APIs.
% \end{tablenotes}
\caption{Summary of existing instruction finetuning-based LLM agents for intrinsic reasoning and function calling, along with their training resources and sample sizes. "PT" and "IFT" denote "Pre-Training" and "Instruction Fine-Tuning", respectively.}
\vspace{-2ex}
\label{tab:related}
\end{threeparttable}
\end{table*}

\noindent \textbf{Prompting-based LLM Agents.} Due to the lack of agent-specific pre-training corpus, existing LLM agents rely on either prompt engineering~\cite{hsieh2023tool,lu2024chameleon,yao2022react,wang2023voyager} or instruction fine-tuning~\cite{chen2023fireact,zeng2023agenttuning} to understand human instructions, decompose high-level tasks, generate grounded plans, and execute multi-step actions. 
However, prompting-based methods mainly depend on the capabilities of backbone LLMs (usually commercial LLMs), failing to introduce new knowledge and struggling to generalize to unseen tasks~\cite{sun2024adaplanner,zhuang2023toolchain}. 

\noindent \textbf{Instruction Finetuning-based LLM Agents.} Considering the extensive diversity of APIs and the complexity of multi-tool instructions, tool learning inherently presents greater challenges than natural language tasks, such as text generation~\cite{qin2023toolllm}.
Post-training techniques focus more on instruction following and aligning output with specific formats~\cite{patil2023gorilla,hao2024toolkengpt,qin2023toolllm,schick2024toolformer}, rather than fundamentally improving model knowledge or capabilities. 
Moreover, heavy fine-tuning can hinder generalization or even degrade performance in non-agent use cases, potentially suppressing the original base model capabilities~\cite{ghosh2024a}.

\noindent \textbf{Pretraining-based LLM Agents.} While pre-training serves as an essential alternative, prior works~\cite{nijkamp2023codegen,roziere2023code,xu2024lemur,patil2023gorilla} have primarily focused on improving task-specific capabilities (\eg, code generation) instead of general-domain LLM agents, due to single-source, uni-type, small-scale, and poor-quality pre-training data. 
Existing tool documentation data for agent training either lacks diverse real-world APIs~\cite{patil2023gorilla, tang2023toolalpaca} or is constrained to single-tool or single-round tool execution. 
Furthermore, trajectory data mostly imitate expert behavior or follow function-calling rules with inferior planning and reasoning, failing to fully elicit LLMs' capabilities and handle complex instructions~\cite{qin2023toolllm}. 
Given a wide range of candidate API functions, each comprising various function names and parameters available at every planning step, identifying globally optimal solutions and generalizing across tasks remains highly challenging.



\section{Preliminaries}
\label{Preliminaries}
\begin{figure*}[t]
    \centering
    \includegraphics[width=0.95\linewidth]{fig/HealthGPT_Framework.png}
    \caption{The \ourmethod{} architecture integrates hierarchical visual perception and H-LoRA, employing a task-specific hard router to select visual features and H-LoRA plugins, ultimately generating outputs with an autoregressive manner.}
    \label{fig:architecture}
\end{figure*}
\noindent\textbf{Large Vision-Language Models.} 
The input to a LVLM typically consists of an image $x^{\text{img}}$ and a discrete text sequence $x^{\text{txt}}$. The visual encoder $\mathcal{E}^{\text{img}}$ converts the input image $x^{\text{img}}$ into a sequence of visual tokens $\mathcal{V} = [v_i]_{i=1}^{N_v}$, while the text sequence $x^{\text{txt}}$ is mapped into a sequence of text tokens $\mathcal{T} = [t_i]_{i=1}^{N_t}$ using an embedding function $\mathcal{E}^{\text{txt}}$. The LLM $\mathcal{M_\text{LLM}}(\cdot|\theta)$ models the joint probability of the token sequence $\mathcal{U} = \{\mathcal{V},\mathcal{T}\}$, which is expressed as:
\begin{equation}
    P_\theta(R | \mathcal{U}) = \prod_{i=1}^{N_r} P_\theta(r_i | \{\mathcal{U}, r_{<i}\}),
\end{equation}
where $R = [r_i]_{i=1}^{N_r}$ is the text response sequence. The LVLM iteratively generates the next token $r_i$ based on $r_{<i}$. The optimization objective is to minimize the cross-entropy loss of the response $\mathcal{R}$.
% \begin{equation}
%     \mathcal{L}_{\text{VLM}} = \mathbb{E}_{R|\mathcal{U}}\left[-\log P_\theta(R | \mathcal{U})\right]
% \end{equation}
It is worth noting that most LVLMs adopt a design paradigm based on ViT, alignment adapters, and pre-trained LLMs\cite{liu2023llava,liu2024improved}, enabling quick adaptation to downstream tasks.


\noindent\textbf{VQGAN.}
VQGAN~\cite{esser2021taming} employs latent space compression and indexing mechanisms to effectively learn a complete discrete representation of images. VQGAN first maps the input image $x^{\text{img}}$ to a latent representation $z = \mathcal{E}(x)$ through a encoder $\mathcal{E}$. Then, the latent representation is quantized using a codebook $\mathcal{Z} = \{z_k\}_{k=1}^K$, generating a discrete index sequence $\mathcal{I} = [i_m]_{m=1}^N$, where $i_m \in \mathcal{Z}$ represents the quantized code index:
\begin{equation}
    \mathcal{I} = \text{Quantize}(z|\mathcal{Z}) = \arg\min_{z_k \in \mathcal{Z}} \| z - z_k \|_2.
\end{equation}
In our approach, the discrete index sequence $\mathcal{I}$ serves as a supervisory signal for the generation task, enabling the model to predict the index sequence $\hat{\mathcal{I}}$ from input conditions such as text or other modality signals.  
Finally, the predicted index sequence $\hat{\mathcal{I}}$ is upsampled by the VQGAN decoder $G$, generating the high-quality image $\hat{x}^\text{img} = G(\hat{\mathcal{I}})$.



\noindent\textbf{Low Rank Adaptation.} 
LoRA\cite{hu2021lora} effectively captures the characteristics of downstream tasks by introducing low-rank adapters. The core idea is to decompose the bypass weight matrix $\Delta W\in\mathbb{R}^{d^{\text{in}} \times d^{\text{out}}}$ into two low-rank matrices $ \{A \in \mathbb{R}^{d^{\text{in}} \times r}, B \in \mathbb{R}^{r \times d^{\text{out}}} \}$, where $ r \ll \min\{d^{\text{in}}, d^{\text{out}}\} $, significantly reducing learnable parameters. The output with the LoRA adapter for the input $x$ is then given by:
\begin{equation}
    h = x W_0 + \alpha x \Delta W/r = x W_0 + \alpha xAB/r,
\end{equation}
where matrix $ A $ is initialized with a Gaussian distribution, while the matrix $ B $ is initialized as a zero matrix. The scaling factor $ \alpha/r $ controls the impact of $ \Delta W $ on the model.

\section{HealthGPT}
\label{Method}


\subsection{Unified Autoregressive Generation.}  
% As shown in Figure~\ref{fig:architecture}, 
\ourmethod{} (Figure~\ref{fig:architecture}) utilizes a discrete token representation that covers both text and visual outputs, unifying visual comprehension and generation as an autoregressive task. 
For comprehension, $\mathcal{M}_\text{llm}$ receives the input joint sequence $\mathcal{U}$ and outputs a series of text token $\mathcal{R} = [r_1, r_2, \dots, r_{N_r}]$, where $r_i \in \mathcal{V}_{\text{txt}}$, and $\mathcal{V}_{\text{txt}}$ represents the LLM's vocabulary:
\begin{equation}
    P_\theta(\mathcal{R} \mid \mathcal{U}) = \prod_{i=1}^{N_r} P_\theta(r_i \mid \mathcal{U}, r_{<i}).
\end{equation}
For generation, $\mathcal{M}_\text{llm}$ first receives a special start token $\langle \text{START\_IMG} \rangle$, then generates a series of tokens corresponding to the VQGAN indices $\mathcal{I} = [i_1, i_2, \dots, i_{N_i}]$, where $i_j \in \mathcal{V}_{\text{vq}}$, and $\mathcal{V}_{\text{vq}}$ represents the index range of VQGAN. Upon completion of generation, the LLM outputs an end token $\langle \text{END\_IMG} \rangle$:
\begin{equation}
    P_\theta(\mathcal{I} \mid \mathcal{U}) = \prod_{j=1}^{N_i} P_\theta(i_j \mid \mathcal{U}, i_{<j}).
\end{equation}
Finally, the generated index sequence $\mathcal{I}$ is fed into the decoder $G$, which reconstructs the target image $\hat{x}^{\text{img}} = G(\mathcal{I})$.

\subsection{Hierarchical Visual Perception}  
Given the differences in visual perception between comprehension and generation tasks—where the former focuses on abstract semantics and the latter emphasizes complete semantics—we employ ViT to compress the image into discrete visual tokens at multiple hierarchical levels.
Specifically, the image is converted into a series of features $\{f_1, f_2, \dots, f_L\}$ as it passes through $L$ ViT blocks.

To address the needs of various tasks, the hidden states are divided into two types: (i) \textit{Concrete-grained features} $\mathcal{F}^{\text{Con}} = \{f_1, f_2, \dots, f_k\}, k < L$, derived from the shallower layers of ViT, containing sufficient global features, suitable for generation tasks; 
(ii) \textit{Abstract-grained features} $\mathcal{F}^{\text{Abs}} = \{f_{k+1}, f_{k+2}, \dots, f_L\}$, derived from the deeper layers of ViT, which contain abstract semantic information closer to the text space, suitable for comprehension tasks.

The task type $T$ (comprehension or generation) determines which set of features is selected as the input for the downstream large language model:
\begin{equation}
    \mathcal{F}^{\text{img}}_T =
    \begin{cases}
        \mathcal{F}^{\text{Con}}, & \text{if } T = \text{generation task} \\
        \mathcal{F}^{\text{Abs}}, & \text{if } T = \text{comprehension task}
    \end{cases}
\end{equation}
We integrate the image features $\mathcal{F}^{\text{img}}_T$ and text features $\mathcal{T}$ into a joint sequence through simple concatenation, which is then fed into the LLM $\mathcal{M}_{\text{llm}}$ for autoregressive generation.
% :
% \begin{equation}
%     \mathcal{R} = \mathcal{M}_{\text{llm}}(\mathcal{U}|\theta), \quad \mathcal{U} = [\mathcal{F}^{\text{img}}_T; \mathcal{T}]
% \end{equation}
\subsection{Heterogeneous Knowledge Adaptation}
We devise H-LoRA, which stores heterogeneous knowledge from comprehension and generation tasks in separate modules and dynamically routes to extract task-relevant knowledge from these modules. 
At the task level, for each task type $ T $, we dynamically assign a dedicated H-LoRA submodule $ \theta^T $, which is expressed as:
\begin{equation}
    \mathcal{R} = \mathcal{M}_\text{LLM}(\mathcal{U}|\theta, \theta^T), \quad \theta^T = \{A^T, B^T, \mathcal{R}^T_\text{outer}\}.
\end{equation}
At the feature level for a single task, H-LoRA integrates the idea of Mixture of Experts (MoE)~\cite{masoudnia2014mixture} and designs an efficient matrix merging and routing weight allocation mechanism, thus avoiding the significant computational delay introduced by matrix splitting in existing MoELoRA~\cite{luo2024moelora}. Specifically, we first merge the low-rank matrices (rank = r) of $ k $ LoRA experts into a unified matrix:
\begin{equation}
    \mathbf{A}^{\text{merged}}, \mathbf{B}^{\text{merged}} = \text{Concat}(\{A_i\}_1^k), \text{Concat}(\{B_i\}_1^k),
\end{equation}
where $ \mathbf{A}^{\text{merged}} \in \mathbb{R}^{d^\text{in} \times rk} $ and $ \mathbf{B}^{\text{merged}} \in \mathbb{R}^{rk \times d^\text{out}} $. The $k$-dimension routing layer generates expert weights $ \mathcal{W} \in \mathbb{R}^{\text{token\_num} \times k} $ based on the input hidden state $ x $, and these are expanded to $ \mathbb{R}^{\text{token\_num} \times rk} $ as follows:
\begin{equation}
    \mathcal{W}^\text{expanded} = \alpha k \mathcal{W} / r \otimes \mathbf{1}_r,
\end{equation}
where $ \otimes $ denotes the replication operation.
The overall output of H-LoRA is computed as:
\begin{equation}
    \mathcal{O}^\text{H-LoRA} = (x \mathbf{A}^{\text{merged}} \odot \mathcal{W}^\text{expanded}) \mathbf{B}^{\text{merged}},
\end{equation}
where $ \odot $ represents element-wise multiplication. Finally, the output of H-LoRA is added to the frozen pre-trained weights to produce the final output:
\begin{equation}
    \mathcal{O} = x W_0 + \mathcal{O}^\text{H-LoRA}.
\end{equation}
% In summary, H-LoRA is a task-based dynamic PEFT method that achieves high efficiency in single-task fine-tuning.

\subsection{Training Pipeline}

\begin{figure}[t]
    \centering
    \hspace{-4mm}
    \includegraphics[width=0.94\linewidth]{fig/data.pdf}
    \caption{Data statistics of \texttt{VL-Health}. }
    \label{fig:data}
\end{figure}
\noindent \textbf{1st Stage: Multi-modal Alignment.} 
In the first stage, we design separate visual adapters and H-LoRA submodules for medical unified tasks. For the medical comprehension task, we train abstract-grained visual adapters using high-quality image-text pairs to align visual embeddings with textual embeddings, thereby enabling the model to accurately describe medical visual content. During this process, the pre-trained LLM and its corresponding H-LoRA submodules remain frozen. In contrast, the medical generation task requires training concrete-grained adapters and H-LoRA submodules while keeping the LLM frozen. Meanwhile, we extend the textual vocabulary to include multimodal tokens, enabling the support of additional VQGAN vector quantization indices. The model trains on image-VQ pairs, endowing the pre-trained LLM with the capability for image reconstruction. This design ensures pixel-level consistency of pre- and post-LVLM. The processes establish the initial alignment between the LLM’s outputs and the visual inputs.

\noindent \textbf{2nd Stage: Heterogeneous H-LoRA Plugin Adaptation.}  
The submodules of H-LoRA share the word embedding layer and output head but may encounter issues such as bias and scale inconsistencies during training across different tasks. To ensure that the multiple H-LoRA plugins seamlessly interface with the LLMs and form a unified base, we fine-tune the word embedding layer and output head using a small amount of mixed data to maintain consistency in the model weights. Specifically, during this stage, all H-LoRA submodules for different tasks are kept frozen, with only the word embedding layer and output head being optimized. Through this stage, the model accumulates foundational knowledge for unified tasks by adapting H-LoRA plugins.

\begin{table*}[!t]
\centering
\caption{Comparison of \ourmethod{} with other LVLMs and unified multi-modal models on medical visual comprehension tasks. \textbf{Bold} and \underline{underlined} text indicates the best performance and second-best performance, respectively.}
\resizebox{\textwidth}{!}{
\begin{tabular}{c|lcc|cccccccc|c}
\toprule
\rowcolor[HTML]{E9F3FE} &  &  &  & \multicolumn{2}{c}{\textbf{VQA-RAD \textuparrow}} & \multicolumn{2}{c}{\textbf{SLAKE \textuparrow}} & \multicolumn{2}{c}{\textbf{PathVQA \textuparrow}} &  &  &  \\ 
\cline{5-10}
\rowcolor[HTML]{E9F3FE}\multirow{-2}{*}{\textbf{Type}} & \multirow{-2}{*}{\textbf{Model}} & \multirow{-2}{*}{\textbf{\# Params}} & \multirow{-2}{*}{\makecell{\textbf{Medical} \\ \textbf{LVLM}}} & \textbf{close} & \textbf{all} & \textbf{close} & \textbf{all} & \textbf{close} & \textbf{all} & \multirow{-2}{*}{\makecell{\textbf{MMMU} \\ \textbf{-Med}}\textuparrow} & \multirow{-2}{*}{\textbf{OMVQA}\textuparrow} & \multirow{-2}{*}{\textbf{Avg. \textuparrow}} \\ 
\midrule \midrule
\multirow{9}{*}{\textbf{Comp. Only}} 
& Med-Flamingo & 8.3B & \Large \ding{51} & 58.6 & 43.0 & 47.0 & 25.5 & 61.9 & 31.3 & 28.7 & 34.9 & 41.4 \\
& LLaVA-Med & 7B & \Large \ding{51} & 60.2 & 48.1 & 58.4 & 44.8 & 62.3 & 35.7 & 30.0 & 41.3 & 47.6 \\
& HuatuoGPT-Vision & 7B & \Large \ding{51} & 66.9 & 53.0 & 59.8 & 49.1 & 52.9 & 32.0 & 42.0 & 50.0 & 50.7 \\
& BLIP-2 & 6.7B & \Large \ding{55} & 43.4 & 36.8 & 41.6 & 35.3 & 48.5 & 28.8 & 27.3 & 26.9 & 36.1 \\
& LLaVA-v1.5 & 7B & \Large \ding{55} & 51.8 & 42.8 & 37.1 & 37.7 & 53.5 & 31.4 & 32.7 & 44.7 & 41.5 \\
& InstructBLIP & 7B & \Large \ding{55} & 61.0 & 44.8 & 66.8 & 43.3 & 56.0 & 32.3 & 25.3 & 29.0 & 44.8 \\
& Yi-VL & 6B & \Large \ding{55} & 52.6 & 42.1 & 52.4 & 38.4 & 54.9 & 30.9 & 38.0 & 50.2 & 44.9 \\
& InternVL2 & 8B & \Large \ding{55} & 64.9 & 49.0 & 66.6 & 50.1 & 60.0 & 31.9 & \underline{43.3} & 54.5 & 52.5\\
& Llama-3.2 & 11B & \Large \ding{55} & 68.9 & 45.5 & 72.4 & 52.1 & 62.8 & 33.6 & 39.3 & 63.2 & 54.7 \\
\midrule
\multirow{5}{*}{\textbf{Comp. \& Gen.}} 
& Show-o & 1.3B & \Large \ding{55} & 50.6 & 33.9 & 31.5 & 17.9 & 52.9 & 28.2 & 22.7 & 45.7 & 42.6 \\
& Unified-IO 2 & 7B & \Large \ding{55} & 46.2 & 32.6 & 35.9 & 21.9 & 52.5 & 27.0 & 25.3 & 33.0 & 33.8 \\
& Janus & 1.3B & \Large \ding{55} & 70.9 & 52.8 & 34.7 & 26.9 & 51.9 & 27.9 & 30.0 & 26.8 & 33.5 \\
& \cellcolor[HTML]{DAE0FB}HealthGPT-M3 & \cellcolor[HTML]{DAE0FB}3.8B & \cellcolor[HTML]{DAE0FB}\Large \ding{51} & \cellcolor[HTML]{DAE0FB}\underline{73.7} & \cellcolor[HTML]{DAE0FB}\underline{55.9} & \cellcolor[HTML]{DAE0FB}\underline{74.6} & \cellcolor[HTML]{DAE0FB}\underline{56.4} & \cellcolor[HTML]{DAE0FB}\underline{78.7} & \cellcolor[HTML]{DAE0FB}\underline{39.7} & \cellcolor[HTML]{DAE0FB}\underline{43.3} & \cellcolor[HTML]{DAE0FB}\underline{68.5} & \cellcolor[HTML]{DAE0FB}\underline{61.3} \\
& \cellcolor[HTML]{DAE0FB}HealthGPT-L14 & \cellcolor[HTML]{DAE0FB}14B & \cellcolor[HTML]{DAE0FB}\Large \ding{51} & \cellcolor[HTML]{DAE0FB}\textbf{77.7} & \cellcolor[HTML]{DAE0FB}\textbf{58.3} & \cellcolor[HTML]{DAE0FB}\textbf{76.4} & \cellcolor[HTML]{DAE0FB}\textbf{64.5} & \cellcolor[HTML]{DAE0FB}\textbf{85.9} & \cellcolor[HTML]{DAE0FB}\textbf{44.4} & \cellcolor[HTML]{DAE0FB}\textbf{49.2} & \cellcolor[HTML]{DAE0FB}\textbf{74.4} & \cellcolor[HTML]{DAE0FB}\textbf{66.4} \\
\bottomrule
\end{tabular}
}
\label{tab:results}
\end{table*}
\begin{table*}[ht]
    \centering
    \caption{The experimental results for the four modality conversion tasks.}
    \resizebox{\textwidth}{!}{
    \begin{tabular}{l|ccc|ccc|ccc|ccc}
        \toprule
        \rowcolor[HTML]{E9F3FE} & \multicolumn{3}{c}{\textbf{CT to MRI (Brain)}} & \multicolumn{3}{c}{\textbf{CT to MRI (Pelvis)}} & \multicolumn{3}{c}{\textbf{MRI to CT (Brain)}} & \multicolumn{3}{c}{\textbf{MRI to CT (Pelvis)}} \\
        \cline{2-13}
        \rowcolor[HTML]{E9F3FE}\multirow{-2}{*}{\textbf{Model}}& \textbf{SSIM $\uparrow$} & \textbf{PSNR $\uparrow$} & \textbf{MSE $\downarrow$} & \textbf{SSIM $\uparrow$} & \textbf{PSNR $\uparrow$} & \textbf{MSE $\downarrow$} & \textbf{SSIM $\uparrow$} & \textbf{PSNR $\uparrow$} & \textbf{MSE $\downarrow$} & \textbf{SSIM $\uparrow$} & \textbf{PSNR $\uparrow$} & \textbf{MSE $\downarrow$} \\
        \midrule \midrule
        pix2pix & 71.09 & 32.65 & 36.85 & 59.17 & 31.02 & 51.91 & 78.79 & 33.85 & 28.33 & 72.31 & 32.98 & 36.19 \\
        CycleGAN & 54.76 & 32.23 & 40.56 & 54.54 & 30.77 & 55.00 & 63.75 & 31.02 & 52.78 & 50.54 & 29.89 & 67.78 \\
        BBDM & {71.69} & {32.91} & {34.44} & 57.37 & 31.37 & 48.06 & \textbf{86.40} & 34.12 & 26.61 & {79.26} & 33.15 & 33.60 \\
        Vmanba & 69.54 & 32.67 & 36.42 & {63.01} & {31.47} & {46.99} & 79.63 & 34.12 & 26.49 & 77.45 & 33.53 & 31.85 \\
        DiffMa & 71.47 & 32.74 & 35.77 & 62.56 & 31.43 & 47.38 & 79.00 & {34.13} & {26.45} & 78.53 & {33.68} & {30.51} \\
        \rowcolor[HTML]{DAE0FB}HealthGPT-M3 & \underline{79.38} & \underline{33.03} & \underline{33.48} & \underline{71.81} & \underline{31.83} & \underline{43.45} & {85.06} & \textbf{34.40} & \textbf{25.49} & \underline{84.23} & \textbf{34.29} & \textbf{27.99} \\
        \rowcolor[HTML]{DAE0FB}HealthGPT-L14 & \textbf{79.73} & \textbf{33.10} & \textbf{32.96} & \textbf{71.92} & \textbf{31.87} & \textbf{43.09} & \underline{85.31} & \underline{34.29} & \underline{26.20} & \textbf{84.96} & \underline{34.14} & \underline{28.13} \\
        \bottomrule
    \end{tabular}
    }
    \label{tab:conversion}
\end{table*}

\noindent \textbf{3rd Stage: Visual Instruction Fine-Tuning.}  
In the third stage, we introduce additional task-specific data to further optimize the model and enhance its adaptability to downstream tasks such as medical visual comprehension (e.g., medical QA, medical dialogues, and report generation) or generation tasks (e.g., super-resolution, denoising, and modality conversion). Notably, by this stage, the word embedding layer and output head have been fine-tuned, only the H-LoRA modules and adapter modules need to be trained. This strategy significantly improves the model's adaptability and flexibility across different tasks.


\section{Experiment}
\label{s:experiment}

\subsection{Data Description}
We evaluate our method on FI~\cite{you2016building}, Twitter\_LDL~\cite{yang2017learning} and Artphoto~\cite{machajdik2010affective}.
FI is a public dataset built from Flickr and Instagram, with 23,308 images and eight emotion categories, namely \textit{amusement}, \textit{anger}, \textit{awe},  \textit{contentment}, \textit{disgust}, \textit{excitement},  \textit{fear}, and \textit{sadness}. 
% Since images in FI are all copyrighted by law, some images are corrupted now, so we remove these samples and retain 21,828 images.
% T4SA contains images from Twitter, which are classified into three categories: \textit{positive}, \textit{neutral}, and \textit{negative}. In this paper, we adopt the base version of B-T4SA, which contains 470,586 images and provides text descriptions of the corresponding tweets.
Twitter\_LDL contains 10,045 images from Twitter, with the same eight categories as the FI dataset.
% 。
For these two datasets, they are randomly split into 80\%
training and 20\% testing set.
Artphoto contains 806 artistic photos from the DeviantArt website, which we use to further evaluate the zero-shot capability of our model.
% on the small-scale dataset.
% We construct and publicly release the first image sentiment analysis dataset containing metadata.
% 。

% Based on these datasets, we are the first to construct and publicly release metadata-enhanced image sentiment analysis datasets. These datasets include scenes, tags, descriptions, and corresponding confidence scores, and are available at this link for future research purposes.


% 
\begin{table}[t]
\centering
% \begin{center}
\caption{Overall performance of different models on FI and Twitter\_LDL datasets.}
\label{tab:cap1}
% \resizebox{\linewidth}{!}
{
\begin{tabular}{l|c|c|c|c}
\hline
\multirow{2}{*}{\textbf{Model}} & \multicolumn{2}{c|}{\textbf{FI}}  & \multicolumn{2}{c}{\textbf{Twitter\_LDL}} \\ \cline{2-5} 
  & \textbf{Accuracy} & \textbf{F1} & \textbf{Accuracy} & \textbf{F1}  \\ \hline
% (\rownumber)~AlexNet~\cite{krizhevsky2017imagenet}  & 58.13\% & 56.35\%  & 56.24\%& 55.02\%  \\ 
% (\rownumber)~VGG16~\cite{simonyan2014very}  & 63.75\%& 63.08\%  & 59.34\%& 59.02\%  \\ 
(\rownumber)~ResNet101~\cite{he2016deep} & 66.16\%& 65.56\%  & 62.02\% & 61.34\%  \\ 
(\rownumber)~CDA~\cite{han2023boosting} & 66.71\%& 65.37\%  & 64.14\% & 62.85\%  \\ 
(\rownumber)~CECCN~\cite{ruan2024color} & 67.96\%& 66.74\%  & 64.59\%& 64.72\% \\ 
(\rownumber)~EmoVIT~\cite{xie2024emovit} & 68.09\%& 67.45\%  & 63.12\% & 61.97\%  \\ 
(\rownumber)~ComLDL~\cite{zhang2022compound} & 68.83\%& 67.28\%  & 65.29\% & 63.12\%  \\ 
(\rownumber)~WSDEN~\cite{li2023weakly} & 69.78\%& 69.61\%  & 67.04\% & 65.49\% \\ 
(\rownumber)~ECWA~\cite{deng2021emotion} & 70.87\%& 69.08\%  & 67.81\% & 66.87\%  \\ 
(\rownumber)~EECon~\cite{yang2023exploiting} & 71.13\%& 68.34\%  & 64.27\%& 63.16\%  \\ 
(\rownumber)~MAM~\cite{zhang2024affective} & 71.44\%  & 70.83\% & 67.18\%  & 65.01\%\\ 
(\rownumber)~TGCA-PVT~\cite{chen2024tgca}   & 73.05\%  & 71.46\% & 69.87\%  & 68.32\% \\ 
(\rownumber)~OEAN~\cite{zhang2024object}   & 73.40\%  & 72.63\% & 70.52\%  & 69.47\% \\ \hline
(\rownumber)~\shortname  & \textbf{79.48\%} & \textbf{79.22\%} & \textbf{74.12\%} & \textbf{73.09\%} \\ \hline
\end{tabular}
}
\vspace{-6mm}
% \end{center}
\end{table}
% 

\subsection{Experiment Setting}
% \subsubsection{Model Setting.}
% 
\textbf{Model Setting:}
For feature representation, we set $k=10$ to select object tags, and adopt clip-vit-base-patch32 as the pre-trained model for unified feature representation.
Moreover, we empirically set $(d_e, d_h, d_k, d_s) = (512, 128, 16, 64)$, and set the classification class $L$ to 8.

% 

\textbf{Training Setting:}
To initialize the model, we set all weights such as $\boldsymbol{W}$ following the truncated normal distribution, and use AdamW optimizer with the learning rate of $1 \times 10^{-4}$.
% warmup scheduler of cosine, warmup steps of 2000.
Furthermore, we set the batch size to 32 and the epoch of the training process to 200.
During the implementation, we utilize \textit{PyTorch} to build our entire model.
% , and our project codes are publicly available at https://github.com/zzmyrep/MESN.
% Our project codes as well as data are all publicly available on GitHub\footnote{https://github.com/zzmyrep/KBCEN}.
% Code is available at \href{https://github.com/zzmyrep/KBCEN}{https://github.com/zzmyrep/KBCEN}.

\textbf{Evaluation Metrics:}
Following~\cite{zhang2024affective, chen2024tgca, zhang2024object}, we adopt \textit{accuracy} and \textit{F1} as our evaluation metrics to measure the performance of different methods for image sentiment analysis. 



\subsection{Experiment Result}
% We compare our model against the following baselines: AlexNet~\cite{krizhevsky2017imagenet}, VGG16~\cite{simonyan2014very}, ResNet101~\cite{he2016deep}, CECCN~\cite{ruan2024color}, EmoVIT~\cite{xie2024emovit}, WSCNet~\cite{yang2018weakly}, ECWA~\cite{deng2021emotion}, EECon~\cite{yang2023exploiting}, MAM~\cite{zhang2024affective} and TGCA-PVT~\cite{chen2024tgca}, and the overall results are summarized in Table~\ref{tab:cap1}.
We compare our model against several baselines, and the overall results are summarized in Table~\ref{tab:cap1}.
We observe that our model achieves the best performance in both accuracy and F1 metrics, significantly outperforming the previous models. 
This superior performance is mainly attributed to our effective utilization of metadata to enhance image sentiment analysis, as well as the exceptional capability of the unified sentiment transformer framework we developed. These results strongly demonstrate that our proposed method can bring encouraging performance for image sentiment analysis.

\setcounter{magicrownumbers}{0} 
\begin{table}[t]
\begin{center}
\caption{Ablation study of~\shortname~on FI dataset.} 
% \vspace{1mm}
\label{tab:cap2}
\resizebox{.9\linewidth}{!}
{
\begin{tabular}{lcc}
  \hline
  \textbf{Model} & \textbf{Accuracy} & \textbf{F1} \\
  \hline
  (\rownumber)~Ours (w/o vision) & 65.72\% & 64.54\% \\
  (\rownumber)~Ours (w/o text description) & 74.05\% & 72.58\% \\
  (\rownumber)~Ours (w/o object tag) & 77.45\% & 76.84\% \\
  (\rownumber)~Ours (w/o scene tag) & 78.47\% & 78.21\% \\
  \hline
  (\rownumber)~Ours (w/o unified embedding) & 76.41\% & 76.23\% \\
  (\rownumber)~Ours (w/o adaptive learning) & 76.83\% & 76.56\% \\
  (\rownumber)~Ours (w/o cross-modal fusion) & 76.85\% & 76.49\% \\
  \hline
  (\rownumber)~Ours  & \textbf{79.48\%} & \textbf{79.22\%} \\
  \hline
\end{tabular}
}
\end{center}
\vspace{-5mm}
\end{table}


\begin{figure}[t]
\centering
% \vspace{-2mm}
\includegraphics[width=0.42\textwidth]{fig/2dvisual-linux4-paper2.pdf}
\caption{Visualization of feature distribution on eight categories before (left) and after (right) model processing.}
% 
\label{fig:visualization}
\vspace{-5mm}
\end{figure}

\subsection{Ablation Performance}
In this subsection, we conduct an ablation study to examine which component is really important for performance improvement. The results are reported in Table~\ref{tab:cap2}.

For information utilization, we observe a significant decline in model performance when visual features are removed. Additionally, the performance of \shortname~decreases when different metadata are removed separately, which means that text description, object tag, and scene tag are all critical for image sentiment analysis.
Recalling the model architecture, we separately remove transformer layers of the unified representation module, the adaptive learning module, and the cross-modal fusion module, replacing them with MLPs of the same parameter scale.
In this way, we can observe varying degrees of decline in model performance, indicating that these modules are indispensable for our model to achieve better performance.

\subsection{Visualization}
% 


% % 开始使用minipage进行左右排列
% \begin{minipage}[t]{0.45\textwidth}  % 子图1宽度为45%
%     \centering
%     \includegraphics[width=\textwidth]{2dvisual.pdf}  % 插入图片
%     \captionof{figure}{Visualization of feature distribution.}  % 使用captionof添加图片标题
%     \label{fig:visualization}
% \end{minipage}


% \begin{figure}[t]
% \centering
% \vspace{-2mm}
% \includegraphics[width=0.45\textwidth]{fig/2dvisual.pdf}
% \caption{Visualization of feature distribution.}
% \label{fig:visualization}
% % \vspace{-4mm}
% \end{figure}

% \begin{figure}[t]
% \centering
% \vspace{-2mm}
% \includegraphics[width=0.45\textwidth]{fig/2dvisual-linux3-paper.pdf}
% \caption{Visualization of feature distribution.}
% \label{fig:visualization}
% % \vspace{-4mm}
% \end{figure}



\begin{figure}[tbp]   
\vspace{-4mm}
  \centering            
  \subfloat[Depth of adaptive learning layers]   
  {
    \label{fig:subfig1}\includegraphics[width=0.22\textwidth]{fig/fig_sensitivity-a5}
  }
  \subfloat[Depth of fusion layers]
  {
    % \label{fig:subfig2}\includegraphics[width=0.22\textwidth]{fig/fig_sensitivity-b2}
    \label{fig:subfig2}\includegraphics[width=0.22\textwidth]{fig/fig_sensitivity-b2-num.pdf}
  }
  \caption{Sensitivity study of \shortname~on different depth. }   
  \label{fig:fig_sensitivity}  
\vspace{-2mm}
\end{figure}

% \begin{figure}[htbp]
% \centerline{\includegraphics{2dvisual.pdf}}
% \caption{Visualization of feature distribution.}
% \label{fig:visualization}
% \end{figure}

% In Fig.~\ref{fig:visualization}, we use t-SNE~\cite{van2008visualizing} to reduce the dimension of data features for visualization, Figure in left represents the metadata features before model processing, the features are obtained by embedding through the CLIP model, and figure in right shows the features of the data after model processing, it can be observed that after the model processing, the data with different label categories fall in different regions in the space, therefore, we can conclude that the Therefore, we can conclude that the model can effectively utilize the information contained in the metadata and use it to guide the model for classification.

In Fig.~\ref{fig:visualization}, we use t-SNE~\cite{van2008visualizing} to reduce the dimension of data features for visualization.
The left figure shows metadata features before being processed by our model (\textit{i.e.}, embedded by CLIP), while the right shows the distribution of features after being processed by our model.
We can observe that after the model processing, data with the same label are closer to each other, while others are farther away.
Therefore, it shows that the model can effectively utilize the information contained in the metadata and use it to guide the classification process.

\subsection{Sensitivity Analysis}
% 
In this subsection, we conduct a sensitivity analysis to figure out the effect of different depth settings of adaptive learning layers and fusion layers. 
% In this subsection, we conduct a sensitivity analysis to figure out the effect of different depth settings on the model. 
% Fig.~\ref{fig:fig_sensitivity} presents the effect of different depth settings of adaptive learning layers and fusion layers. 
Taking Fig.~\ref{fig:fig_sensitivity} (a) as an example, the model performance improves with increasing depth, reaching the best performance at a depth of 4.
% Taking Fig.~\ref{fig:fig_sensitivity} (a) as an example, the performance of \shortname~improves with the increase of depth at first, reaching the best performance at a depth of 4.
When the depth continues to increase, the accuracy decreases to varying degrees.
Similar results can be observed in Fig.~\ref{fig:fig_sensitivity} (b).
Therefore, we set their depths to 4 and 6 respectively to achieve the best results.

% Through our experiments, we can observe that the effect of modifying these hyperparameters on the results of the experiments is very weak, and the surface model is not sensitive to the hyperparameters.


\subsection{Zero-shot Capability}
% 

% (1)~GCH~\cite{2010Analyzing} & 21.78\% & (5)~RA-DLNet~\cite{2020A} & 34.01\% \\ \hline
% (2)~WSCNet~\cite{2019WSCNet}  & 30.25\% & (6)~CECCN~\cite{ruan2024color} & 43.83\% \\ \hline
% (3)~PCNN~\cite{2015Robust} & 31.68\%  & (7)~EmoVIT~\cite{xie2024emovit} & 44.90\% \\ \hline
% (4)~AR~\cite{2018Visual} & 32.67\% & (8)~Ours (Zero-shot) & 47.83\% \\ \hline


\begin{table}[t]
\centering
\caption{Zero-shot capability of \shortname.}
\label{tab:cap3}
\resizebox{1\linewidth}{!}
{
\begin{tabular}{lc|lc}
\hline
\textbf{Model} & \textbf{Accuracy} & \textbf{Model} & \textbf{Accuracy} \\ \hline
(1)~WSCNet~\cite{2019WSCNet}  & 30.25\% & (5)~MAM~\cite{zhang2024affective} & 39.56\%  \\ \hline
(2)~AR~\cite{2018Visual} & 32.67\% & (6)~CECCN~\cite{ruan2024color} & 43.83\% \\ \hline
(3)~RA-DLNet~\cite{2020A} & 34.01\%  & (7)~EmoVIT~\cite{xie2024emovit} & 44.90\% \\ \hline
(4)~CDA~\cite{han2023boosting} & 38.64\% & (8)~Ours (Zero-shot) & 47.83\% \\ \hline
\end{tabular}
}
\vspace{-5mm}
\end{table}

% We use the model trained on the FI dataset to test on the artphoto dataset to verify the model's generalization ability as well as robustness to other distributed datasets.
% We can observe that the MESN model shows strong competitiveness in terms of accuracy when compared to other trained models, which suggests that the model has a good generalization ability in the OOD task.

To validate the model's generalization ability and robustness to other distributed datasets, we directly test the model trained on the FI dataset, without training on Artphoto. 
% As observed in Table 3, compared to other models trained on Artphoto, we achieve highly competitive zero-shot performance, indicating that the model has good generalization ability in out-of-distribution tasks.
From Table~\ref{tab:cap3}, we can observe that compared with other models trained on Artphoto, we achieve competitive zero-shot performance, which shows that the model has good generalization ability in out-of-distribution tasks.


%%%%%%%%%%%%
%  E2E     %
%%%%%%%%%%%%


\section{Conclusion}
In this paper, we introduced Wi-Chat, the first LLM-powered Wi-Fi-based human activity recognition system that integrates the reasoning capabilities of large language models with the sensing potential of wireless signals. Our experimental results on a self-collected Wi-Fi CSI dataset demonstrate the promising potential of LLMs in enabling zero-shot Wi-Fi sensing. These findings suggest a new paradigm for human activity recognition that does not rely on extensive labeled data. We hope future research will build upon this direction, further exploring the applications of LLMs in signal processing domains such as IoT, mobile sensing, and radar-based systems.

\section*{Limitations}
While our work represents the first attempt to leverage LLMs for processing Wi-Fi signals, it is a preliminary study focused on a relatively simple task: Wi-Fi-based human activity recognition. This choice allows us to explore the feasibility of LLMs in wireless sensing but also comes with certain limitations.

Our approach primarily evaluates zero-shot performance, which, while promising, may still lag behind traditional supervised learning methods in highly complex or fine-grained recognition tasks. Besides, our study is limited to a controlled environment with a self-collected dataset, and the generalizability of LLMs to diverse real-world scenarios with varying Wi-Fi conditions, environmental interference, and device heterogeneity remains an open question.

Additionally, we have yet to explore the full potential of LLMs in more advanced Wi-Fi sensing applications, such as fine-grained gesture recognition, occupancy detection, and passive health monitoring. Future work should investigate the scalability of LLM-based approaches, their robustness to domain shifts, and their integration with multimodal sensing techniques in broader IoT applications.


% Bibliography entries for the entire Anthology, followed by custom entries
%\bibliography{anthology,custom}
% Custom bibliography entries only
\bibliography{main}
\newpage
\appendix

\section{Experiment prompts}
\label{sec:prompt}
The prompts used in the LLM experiments are shown in the following Table~\ref{tab:prompts}.

\definecolor{titlecolor}{rgb}{0.9, 0.5, 0.1}
\definecolor{anscolor}{rgb}{0.2, 0.5, 0.8}
\definecolor{labelcolor}{HTML}{48a07e}
\begin{table*}[h]
	\centering
	
 % \vspace{-0.2cm}
	
	\begin{center}
		\begin{tikzpicture}[
				chatbox_inner/.style={rectangle, rounded corners, opacity=0, text opacity=1, font=\sffamily\scriptsize, text width=5in, text height=9pt, inner xsep=6pt, inner ysep=6pt},
				chatbox_prompt_inner/.style={chatbox_inner, align=flush left, xshift=0pt, text height=11pt},
				chatbox_user_inner/.style={chatbox_inner, align=flush left, xshift=0pt},
				chatbox_gpt_inner/.style={chatbox_inner, align=flush left, xshift=0pt},
				chatbox/.style={chatbox_inner, draw=black!25, fill=gray!7, opacity=1, text opacity=0},
				chatbox_prompt/.style={chatbox, align=flush left, fill=gray!1.5, draw=black!30, text height=10pt},
				chatbox_user/.style={chatbox, align=flush left},
				chatbox_gpt/.style={chatbox, align=flush left},
				chatbox2/.style={chatbox_gpt, fill=green!25},
				chatbox3/.style={chatbox_gpt, fill=red!20, draw=black!20},
				chatbox4/.style={chatbox_gpt, fill=yellow!30},
				labelbox/.style={rectangle, rounded corners, draw=black!50, font=\sffamily\scriptsize\bfseries, fill=gray!5, inner sep=3pt},
			]
											
			\node[chatbox_user] (q1) {
				\textbf{System prompt}
				\newline
				\newline
				You are a helpful and precise assistant for segmenting and labeling sentences. We would like to request your help on curating a dataset for entity-level hallucination detection.
				\newline \newline
                We will give you a machine generated biography and a list of checked facts about the biography. Each fact consists of a sentence and a label (True/False). Please do the following process. First, breaking down the biography into words. Second, by referring to the provided list of facts, merging some broken down words in the previous step to form meaningful entities. For example, ``strategic thinking'' should be one entity instead of two. Third, according to the labels in the list of facts, labeling each entity as True or False. Specifically, for facts that share a similar sentence structure (\eg, \textit{``He was born on Mach 9, 1941.''} (\texttt{True}) and \textit{``He was born in Ramos Mejia.''} (\texttt{False})), please first assign labels to entities that differ across atomic facts. For example, first labeling ``Mach 9, 1941'' (\texttt{True}) and ``Ramos Mejia'' (\texttt{False}) in the above case. For those entities that are the same across atomic facts (\eg, ``was born'') or are neutral (\eg, ``he,'' ``in,'' and ``on''), please label them as \texttt{True}. For the cases that there is no atomic fact that shares the same sentence structure, please identify the most informative entities in the sentence and label them with the same label as the atomic fact while treating the rest of the entities as \texttt{True}. In the end, output the entities and labels in the following format:
                \begin{itemize}[nosep]
                    \item Entity 1 (Label 1)
                    \item Entity 2 (Label 2)
                    \item ...
                    \item Entity N (Label N)
                \end{itemize}
                % \newline \newline
                Here are two examples:
                \newline\newline
                \textbf{[Example 1]}
                \newline
                [The start of the biography]
                \newline
                \textcolor{titlecolor}{Marianne McAndrew is an American actress and singer, born on November 21, 1942, in Cleveland, Ohio. She began her acting career in the late 1960s, appearing in various television shows and films.}
                \newline
                [The end of the biography]
                \newline \newline
                [The start of the list of checked facts]
                \newline
                \textcolor{anscolor}{[Marianne McAndrew is an American. (False); Marianne McAndrew is an actress. (True); Marianne McAndrew is a singer. (False); Marianne McAndrew was born on November 21, 1942. (False); Marianne McAndrew was born in Cleveland, Ohio. (False); She began her acting career in the late 1960s. (True); She has appeared in various television shows. (True); She has appeared in various films. (True)]}
                \newline
                [The end of the list of checked facts]
                \newline \newline
                [The start of the ideal output]
                \newline
                \textcolor{labelcolor}{[Marianne McAndrew (True); is (True); an (True); American (False); actress (True); and (True); singer (False); , (True); born (True); on (True); November 21, 1942 (False); , (True); in (True); Cleveland, Ohio (False); . (True); She (True); began (True); her (True); acting career (True); in (True); the late 1960s (True); , (True); appearing (True); in (True); various (True); television shows (True); and (True); films (True); . (True)]}
                \newline
                [The end of the ideal output]
				\newline \newline
                \textbf{[Example 2]}
                \newline
                [The start of the biography]
                \newline
                \textcolor{titlecolor}{Doug Sheehan is an American actor who was born on April 27, 1949, in Santa Monica, California. He is best known for his roles in soap operas, including his portrayal of Joe Kelly on ``General Hospital'' and Ben Gibson on ``Knots Landing.''}
                \newline
                [The end of the biography]
                \newline \newline
                [The start of the list of checked facts]
                \newline
                \textcolor{anscolor}{[Doug Sheehan is an American. (True); Doug Sheehan is an actor. (True); Doug Sheehan was born on April 27, 1949. (True); Doug Sheehan was born in Santa Monica, California. (False); He is best known for his roles in soap operas. (True); He portrayed Joe Kelly. (True); Joe Kelly was in General Hospital. (True); General Hospital is a soap opera. (True); He portrayed Ben Gibson. (True); Ben Gibson was in Knots Landing. (True); Knots Landing is a soap opera. (True)]}
                \newline
                [The end of the list of checked facts]
                \newline \newline
                [The start of the ideal output]
                \newline
                \textcolor{labelcolor}{[Doug Sheehan (True); is (True); an (True); American (True); actor (True); who (True); was born (True); on (True); April 27, 1949 (True); in (True); Santa Monica, California (False); . (True); He (True); is (True); best known (True); for (True); his roles in soap operas (True); , (True); including (True); in (True); his portrayal (True); of (True); Joe Kelly (True); on (True); ``General Hospital'' (True); and (True); Ben Gibson (True); on (True); ``Knots Landing.'' (True)]}
                \newline
                [The end of the ideal output]
				\newline \newline
				\textbf{User prompt}
				\newline
				\newline
				[The start of the biography]
				\newline
				\textcolor{magenta}{\texttt{\{BIOGRAPHY\}}}
				\newline
				[The ebd of the biography]
				\newline \newline
				[The start of the list of checked facts]
				\newline
				\textcolor{magenta}{\texttt{\{LIST OF CHECKED FACTS\}}}
				\newline
				[The end of the list of checked facts]
			};
			\node[chatbox_user_inner] (q1_text) at (q1) {
				\textbf{System prompt}
				\newline
				\newline
				You are a helpful and precise assistant for segmenting and labeling sentences. We would like to request your help on curating a dataset for entity-level hallucination detection.
				\newline \newline
                We will give you a machine generated biography and a list of checked facts about the biography. Each fact consists of a sentence and a label (True/False). Please do the following process. First, breaking down the biography into words. Second, by referring to the provided list of facts, merging some broken down words in the previous step to form meaningful entities. For example, ``strategic thinking'' should be one entity instead of two. Third, according to the labels in the list of facts, labeling each entity as True or False. Specifically, for facts that share a similar sentence structure (\eg, \textit{``He was born on Mach 9, 1941.''} (\texttt{True}) and \textit{``He was born in Ramos Mejia.''} (\texttt{False})), please first assign labels to entities that differ across atomic facts. For example, first labeling ``Mach 9, 1941'' (\texttt{True}) and ``Ramos Mejia'' (\texttt{False}) in the above case. For those entities that are the same across atomic facts (\eg, ``was born'') or are neutral (\eg, ``he,'' ``in,'' and ``on''), please label them as \texttt{True}. For the cases that there is no atomic fact that shares the same sentence structure, please identify the most informative entities in the sentence and label them with the same label as the atomic fact while treating the rest of the entities as \texttt{True}. In the end, output the entities and labels in the following format:
                \begin{itemize}[nosep]
                    \item Entity 1 (Label 1)
                    \item Entity 2 (Label 2)
                    \item ...
                    \item Entity N (Label N)
                \end{itemize}
                % \newline \newline
                Here are two examples:
                \newline\newline
                \textbf{[Example 1]}
                \newline
                [The start of the biography]
                \newline
                \textcolor{titlecolor}{Marianne McAndrew is an American actress and singer, born on November 21, 1942, in Cleveland, Ohio. She began her acting career in the late 1960s, appearing in various television shows and films.}
                \newline
                [The end of the biography]
                \newline \newline
                [The start of the list of checked facts]
                \newline
                \textcolor{anscolor}{[Marianne McAndrew is an American. (False); Marianne McAndrew is an actress. (True); Marianne McAndrew is a singer. (False); Marianne McAndrew was born on November 21, 1942. (False); Marianne McAndrew was born in Cleveland, Ohio. (False); She began her acting career in the late 1960s. (True); She has appeared in various television shows. (True); She has appeared in various films. (True)]}
                \newline
                [The end of the list of checked facts]
                \newline \newline
                [The start of the ideal output]
                \newline
                \textcolor{labelcolor}{[Marianne McAndrew (True); is (True); an (True); American (False); actress (True); and (True); singer (False); , (True); born (True); on (True); November 21, 1942 (False); , (True); in (True); Cleveland, Ohio (False); . (True); She (True); began (True); her (True); acting career (True); in (True); the late 1960s (True); , (True); appearing (True); in (True); various (True); television shows (True); and (True); films (True); . (True)]}
                \newline
                [The end of the ideal output]
				\newline \newline
                \textbf{[Example 2]}
                \newline
                [The start of the biography]
                \newline
                \textcolor{titlecolor}{Doug Sheehan is an American actor who was born on April 27, 1949, in Santa Monica, California. He is best known for his roles in soap operas, including his portrayal of Joe Kelly on ``General Hospital'' and Ben Gibson on ``Knots Landing.''}
                \newline
                [The end of the biography]
                \newline \newline
                [The start of the list of checked facts]
                \newline
                \textcolor{anscolor}{[Doug Sheehan is an American. (True); Doug Sheehan is an actor. (True); Doug Sheehan was born on April 27, 1949. (True); Doug Sheehan was born in Santa Monica, California. (False); He is best known for his roles in soap operas. (True); He portrayed Joe Kelly. (True); Joe Kelly was in General Hospital. (True); General Hospital is a soap opera. (True); He portrayed Ben Gibson. (True); Ben Gibson was in Knots Landing. (True); Knots Landing is a soap opera. (True)]}
                \newline
                [The end of the list of checked facts]
                \newline \newline
                [The start of the ideal output]
                \newline
                \textcolor{labelcolor}{[Doug Sheehan (True); is (True); an (True); American (True); actor (True); who (True); was born (True); on (True); April 27, 1949 (True); in (True); Santa Monica, California (False); . (True); He (True); is (True); best known (True); for (True); his roles in soap operas (True); , (True); including (True); in (True); his portrayal (True); of (True); Joe Kelly (True); on (True); ``General Hospital'' (True); and (True); Ben Gibson (True); on (True); ``Knots Landing.'' (True)]}
                \newline
                [The end of the ideal output]
				\newline \newline
				\textbf{User prompt}
				\newline
				\newline
				[The start of the biography]
				\newline
				\textcolor{magenta}{\texttt{\{BIOGRAPHY\}}}
				\newline
				[The ebd of the biography]
				\newline \newline
				[The start of the list of checked facts]
				\newline
				\textcolor{magenta}{\texttt{\{LIST OF CHECKED FACTS\}}}
				\newline
				[The end of the list of checked facts]
			};
		\end{tikzpicture}
        \caption{GPT-4o prompt for labeling hallucinated entities.}\label{tb:gpt-4-prompt}
	\end{center}
\vspace{-0cm}
\end{table*}
% \section{Full Experiment Results}
% \begin{table*}[th]
    \centering
    \small
    \caption{Classification Results}
    \begin{tabular}{lcccc}
        \toprule
        \textbf{Method} & \textbf{Accuracy} & \textbf{Precision} & \textbf{Recall} & \textbf{F1-score} \\
        \midrule
        \multicolumn{5}{c}{\textbf{Zero Shot}} \\
                Zero-shot E-eyes & 0.26 & 0.26 & 0.27 & 0.26 \\
        Zero-shot CARM & 0.24 & 0.24 & 0.24 & 0.24 \\
                Zero-shot SVM & 0.27 & 0.28 & 0.28 & 0.27 \\
        Zero-shot CNN & 0.23 & 0.24 & 0.23 & 0.23 \\
        Zero-shot RNN & 0.26 & 0.26 & 0.26 & 0.26 \\
DeepSeek-0shot & 0.54 & 0.61 & 0.54 & 0.52 \\
DeepSeek-0shot-COT & 0.33 & 0.24 & 0.33 & 0.23 \\
DeepSeek-0shot-Knowledge & 0.45 & 0.46 & 0.45 & 0.44 \\
Gemma2-0shot & 0.35 & 0.22 & 0.38 & 0.27 \\
Gemma2-0shot-COT & 0.36 & 0.22 & 0.36 & 0.27 \\
Gemma2-0shot-Knowledge & 0.32 & 0.18 & 0.34 & 0.20 \\
GPT-4o-mini-0shot & 0.48 & 0.53 & 0.48 & 0.41 \\
GPT-4o-mini-0shot-COT & 0.33 & 0.50 & 0.33 & 0.38 \\
GPT-4o-mini-0shot-Knowledge & 0.49 & 0.31 & 0.49 & 0.36 \\
GPT-4o-0shot & 0.62 & 0.62 & 0.47 & 0.42 \\
GPT-4o-0shot-COT & 0.29 & 0.45 & 0.29 & 0.21 \\
GPT-4o-0shot-Knowledge & 0.44 & 0.52 & 0.44 & 0.39 \\
LLaMA-0shot & 0.32 & 0.25 & 0.32 & 0.24 \\
LLaMA-0shot-COT & 0.12 & 0.25 & 0.12 & 0.09 \\
LLaMA-0shot-Knowledge & 0.32 & 0.25 & 0.32 & 0.28 \\
Mistral-0shot & 0.19 & 0.23 & 0.19 & 0.10 \\
Mistral-0shot-Knowledge & 0.21 & 0.40 & 0.21 & 0.11 \\
        \midrule
        \multicolumn{5}{c}{\textbf{4 Shot}} \\
GPT-4o-mini-4shot & 0.58 & 0.59 & 0.58 & 0.53 \\
GPT-4o-mini-4shot-COT & 0.57 & 0.53 & 0.57 & 0.50 \\
GPT-4o-mini-4shot-Knowledge & 0.56 & 0.51 & 0.56 & 0.47 \\
GPT-4o-4shot & 0.77 & 0.84 & 0.77 & 0.73 \\
GPT-4o-4shot-COT & 0.63 & 0.76 & 0.63 & 0.53 \\
GPT-4o-4shot-Knowledge & 0.72 & 0.82 & 0.71 & 0.66 \\
LLaMA-4shot & 0.29 & 0.24 & 0.29 & 0.21 \\
LLaMA-4shot-COT & 0.20 & 0.30 & 0.20 & 0.13 \\
LLaMA-4shot-Knowledge & 0.15 & 0.23 & 0.13 & 0.13 \\
Mistral-4shot & 0.02 & 0.02 & 0.02 & 0.02 \\
Mistral-4shot-Knowledge & 0.21 & 0.27 & 0.21 & 0.20 \\
        \midrule
        
        \multicolumn{5}{c}{\textbf{Suprevised}} \\
        SVM & 0.94 & 0.92 & 0.91 & 0.91 \\
        CNN & 0.98 & 0.98 & 0.97 & 0.97 \\
        RNN & 0.99 & 0.99 & 0.99 & 0.99 \\
        % \midrule
        % \multicolumn{5}{c}{\textbf{Conventional Wi-Fi-based Human Activity Recognition Systems}} \\
        E-eyes & 1.00 & 1.00 & 1.00 & 1.00 \\
        CARM & 0.98 & 0.98 & 0.98 & 0.98 \\
\midrule
 \multicolumn{5}{c}{\textbf{Vision Models}} \\
           Zero-shot SVM & 0.26 & 0.25 & 0.25 & 0.25 \\
        Zero-shot CNN & 0.26 & 0.25 & 0.26 & 0.26 \\
        Zero-shot RNN & 0.28 & 0.28 & 0.29 & 0.28 \\
        SVM & 0.99 & 0.99 & 0.99 & 0.99 \\
        CNN & 0.98 & 0.99 & 0.98 & 0.98 \\
        RNN & 0.98 & 0.99 & 0.98 & 0.98 \\
GPT-4o-mini-Vision & 0.84 & 0.85 & 0.84 & 0.84 \\
GPT-4o-mini-Vision-COT & 0.90 & 0.91 & 0.90 & 0.90 \\
GPT-4o-Vision & 0.74 & 0.82 & 0.74 & 0.73 \\
GPT-4o-Vision-COT & 0.70 & 0.83 & 0.70 & 0.68 \\
LLaMA-Vision & 0.20 & 0.23 & 0.20 & 0.09 \\
LLaMA-Vision-Knowledge & 0.22 & 0.05 & 0.22 & 0.08 \\

        \bottomrule
    \end{tabular}
    \label{full}
\end{table*}




\end{document}


\end{document}