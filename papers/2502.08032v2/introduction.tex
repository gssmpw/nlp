
\section{Introduction} \label{sec:intro}

\yonggang{
There are two points I want to address in the introduction.
1. We improve exponentially from the previous lower bound by d-TC spanners [2009], this is a concrete improvement that is not overselling.

2. There is a possible way to get universally optimal result for reachability in parallel and CONGEST: first compute the best shortcut for the graph (with bi-criteria), then do reachability (because currently all reachability algorithms are based on shortcut). 

We showed evidence that this cannot be done because bi-criteria for shortcut is hard in sequential model. (We believe our technique gives insight to deny the algorithm in parallel congest model)

Moreover, if we restrict the running time for parallel and congest, then we already deny such an algorithm.

3. remove the overselling part where we say our upper bound gives hope to close the $n^{1/3}$ and $n^{1/4}$ gap for existential shortcut.

}

\danupon{I thought I've tried writing (2), but I cannot find where it is. My concern is that it's hard for the readers to get this point; so it's hard to highlight (and thus can hardly be a point that makes the paper accepted). I'm ok with other comments.}





This paper focuses on polynomial-time approximation algorithms for two important formalizations of diameter reduction techniques called {\em shortcuts} and {\em transitive-closure spanners (TC spanners)}. We discuss the contexts briefly below. More details are discussed after we state our results. 









\paragraph{Shortcut and TC spanner approximation.}
For a directed unweighted graph $G=(V, E)$ and vertices $u,v\in V$, let $\dist_G(u,v)$ be the length of the shortest path from $u$ to $v$ in $G$.
For an integer $d$, a set of edges $E'\subseteq V\times V$ is called a {\em $d$-TC spanner} of $G$ if the graph $H:=(V, E')$ has
{\bf (i)} the same transitive-closure as $G$ (i.e. for any $u,v\in V$, $\dist_G(u,v)\neq \infty\iff\dist_H(u,v)\neq \infty$) and 
{\bf (ii)} diameter at most $d.$ 
A set $E''\subseteq V\times V$ is a {\em $d$-shortcut} of $G$ if $E\cup E''$ is a $d$-TC spanner of $G$. 


TC spanners and shortcuts have been extensively studied.  The general goal is to find $d$-shortcuts and $d$-TC spanners of small size. 
There are two main research directions:
{\bf (1)} 
Motivated by solving reachability problems with small worst-case guarantees in various computational models, 
the first direction concerns the {\em globally optimal} TC spanners and shortcuts over a class of graphs;
for example, a recent breakthrough of Kogan and Parter \cite{KoganP22} showed that every $n$-vertex graph admits an $n^{1/3}$-shortcut of size $\tilde O(n)$ which admits a fast algorithm \cite{KoganP-icalp22}. (For earlier work on planar graphs, trees, efficient computation, etc. see e.g.~\cite{Fineman20,Thorup97,Chazelle87a,Yao82}.) 
{\bf (2)} This paper focuses on making progress in a very different, but complementary,  direction that witnessed much less progress in recent years.
It concerns {\em instance optimal} shortcuts and TC-spanners. 
Roughly, the problem is: 
\begin{quote}
{\em     
Given a directed graph $G$ and integers $d$ and $s$ such that $G$ admits a $d$-shortcut of size $s$, find a $(d\apxD)$-shortcut with $s\apxS$ edges, for as small $\apxS$ and $\apxD$ as possible. 
The problem is defined similarly for $d$-TC spanners.
} (See \Cref{sec:prelim} for a formal definition.) 
\end{quote}
We call algorithms for finding such shortcuts and TC spanners {\em $(\apxD,\apxS)$-approximation algorithms.}
We are mainly interested in the case where $s\geq n$ and $d\gg \log(n)$ as it suffices in most applications (e.g., Kogan and Parter's result also has $s=\tilde{\Theta}(n)$). 



An ultimate goal of this direction is $(\apxD,\apxS)$-approximation algorithms that are efficient in various computational settings (e.g. parallel, distributed, two-party communication and streaming settings) for some small $\apxD$ and $\apxS$ (e.g. $\apxD,\apxS=\polylog(n)$ or even $\apxD,\apxS=O(n^{0.01})$). 
Such algorithms would provide  {\em beyond-worst-case  structure-oblivious performance guarantees}---they would solve reachability faster on input graphs that admit better shortcuts or TC spanners (e.g. planar graphs) without the need to know the input structures.  
As discussed more in \Cref{sec:implications}, lower bounds for the second direction can be an intermediate step for proving strong worst-case lower bounds for reachability and other problems, which would be a major breakthough in graph algorithms.  
Unfortunately, 
the progress towards this ultimate goal has stalled and we still know very little about how to design instance-optimal algorithms or argue lower bounds in {\em any} settings. 

\paragraph{Polynomial-time sequential algorithms.}
Since efficient algorithms in many settings are typically adaptations of efficient sequential algorithms, a major step forward would be to better understand the sequential setting. 
In this setting, the main obstacle is that not much was known even when we relax to {\em polynomial-time} algorithms: 
{\bf (a)} On the lower bound side, super-constant lower bounds were known only for $d$-TC spanners with constant $d$ and $\apxD=1$ \cite{BhattacharyyaGJRW12}. The case of $d=\omega(1)$ or even $d=\omega(\log n)$ is crucial because algorithms' performances are typically bad when $d$ is large. 
Lower bounds for super-constant $d$ were previously known only for a more general case of {\em directed spanners} \cite{ElkinP07}. 
{\bf (b)}
On the upper bound side, the current best approximation factors only need na\"{i}ve techniques in the sense that they can be achieved by one of the following methods:
{\bf (i)} By using the $O(\sqrt{n}\ln n)$-approximation algorithm \cite{BermanBMRY11} for the directed spanner problem which generalizes both shortcut and TC spanner, we get a $(\apxD=1, \apxS=O(\sqrt{n}\ln n))$-approximation factor. 
{\bf (ii)} By returning the aforementioned globally-optimal shortcut of \cite{KoganP22} and, since we assume $s\geq n$, we get  $(\apxD=n^{1/3}, \apxS=\polylog(n))$-approximation factor\footnote{Recall that we assume $s\geq n$ throughout.}; extending this argument to exploit the trade-off provided by  \cite{KoganP22} gives $(n^{\gamma_D}, n^{\gamma_S})$-approximation factor for all $3\gamma_D + \gamma_S>1$ (e.g. an $(n^{1/4+o(1)}, n^{1/4+o(1)})$-approximation factor).
(We discuss previous results more in \Cref{sec:intro:spanner}.)




The state-of-the-art results above illustrate how little we know about the instance-optimal case, not only in the sequential setting but in general.
Even simple questions remain unanswered. 
For example, is it possible to get a $(\apxD=1, \apxS=\polylog(n))$-approximation ratio for super-constant $d$, or even just $(n^{o(1)}, n^{o(1)})$-approximation ratio (which is still very good)? Are there techniques to design approximation algorithms beyond the naive arguments above? Having no answers to these questions, it looks hopeless to reach the ultimate goal.
This also reflects the barriers in making progress on approximating other types of spanners (for more detail, we refer the readers to~\cite{BermanBMRY13, DinitzZ16, KortsarzP94, DinitzKR16, AhmedBSHJKS20}). 











\paragraph{Our results.} 
We present new lower and upper bounds that answer the aforementioned questions. 
Note that, when $s\geq n$, a polynomial-time $(\apxD,\apxS)$-approximation algorithm for finding a shortcut implies a polynomial-time $(\apxD,\apxS+1)$-approximation algorithm for finding a TC spanner (see~\cref{lem:redTCtoSh}).\footnote{
The reduction is as follows. Given a directed graph $G$ and parameters $s\geq n$ and $d$, where we would like to approximately find a $d$-TC spanner of size $s$, find a miminum-carinality subgraph $G'$ that preserves the transitive closure of $G$. $G'$ can be found in polynomial time by repeatedly removing any edge $e$ whose removal does not change the transitive closure of $G$~\cite{AhoGU72}.
Then, call the $(\apxD, \apxS)$-approximation algorithm to find a shortcut in $G'$ with the same input parameters $s$ and $d$. Return the shortcut together with $G'$. 
Note that the same reduction shows that finding a shortcut is in fact equivalent to the TC spanner problem when we allow edge costs to be $0$ and $1$ (to do this we only need to modify the first step to include all cost edges of cost 0). See~\cref{lem:redTCtoSh} for more detail.}
Thus, we need to prove an upper bound only for shortcuts and prove a lower bound only for TC spanners. The proof overview is in \Cref{sec:overview}. See \Cref{subsec:mainresults} for detailed statements of the results.


A simplified statement of our result under the Projection Games Conjecture (\conj{}) \cite{Moshkovitz15} is as follows. 


\begin{theorem}[Lower bound; informal]\label{thm:intro:lowerbound}
		Under \conj{}, there exists a small constant $\epsilon>0$, such that no polynomial-time $(n^{\epsilon},n^{\epsilon})$-approximation algorithm exists for finding $d$-shortcuts as well as $d$-TC spanners of size $s$. 
\end{theorem}



Our result improves upon and extends the existing hardness result in two ways: (i) This is the first super-constant lower bound for the important case of $d = \Omega(\log n)$ (our result holds up to polynomial value of $d$). The range of $d$ for which the hardness result applies is exponentially larger. Similar lower bounds for super-constant $d$ were previously known only for a more general case of directed spanner \cite{ElkinP07}. (ii) Our result gives a bicriteria hardness while the previous does not.  

Given this hardness result, which holds even in our setting of interest $s=\Omega(n)$, it is unlikely that these problems admit $(n^{o(1)}, n^{o(1)})$-approximation algorithms. 
Therefore, it is natural to shift attention to designing a bicriteria $(n^{\gamma_D}, n^{\gamma_S})$-approximation for some constants $\gamma_D$ and $\gamma_S$.   As a side result, we complement the above with the following upper bound. 



\begin{theorem}[Upper bound; informal]\label{thm:intro:upperbound}

  There is a polynomial time randomized (Las Vegas) algorithm that, given a directed $n$-vertex graph $G$ and integers $s\geq n$ and $d$ such that $G$ admits a $d$-shortcut (respectively $d$-TC spanner) of size $s$, the algorithm can find a $(d n^{\gamma_D})$-shortcut (resp. $(d n^{\gamma_D})$-TC spanner) with $s n^{\gamma_S}$ edges 
  for all $3\gamma_D + 2\gamma_S >1$.
  

  
	\end{theorem}

 
\noindent

In comparison to the previous result, when $\gamma_D$ is fixed, our algorithm provides $n^{(1-3\gamma_D)/2}$-approximation on the size, while the existing bound is $n^{1-3\gamma_S}$. This is a quadratic improvement. 
We remark that our assumption $s \geq n$ is a natural assumption for studying both upper and lower bounds: For applications such as parallel algorithms, $s=\tilde O(m)$ would be ideal. For other applications, such as distributed, streaming, and communication complexity, we need a smaller $s=\Theta(n)$. We are not aware of any applications where smaller value of $s=o(n)$ is needed. 

While bicriteria approximation  algorithms have not been studied in the context of spanners, it is a natural and well-studied approach for many problems such as graph partitioning~\cite{MakarychevM14}, CSPs~\cite{MakarychevM17}, and clustering~\cite{MakarychevMSW16, AlamdariS17,fox2019embedding}. For our problems in particular, obtaining $(n^{0.01}, n^{0.01})$-approximation would  be very interesting and would suffice for applications. 


Below we discuss our results in broader contexts.





\subsection{Implications of our results}
\label{sec:implications}

Our problems have  connections to other fundamental problems that arise in various models of computation. Here we highlight those connections and discuss implications of our results. 



\paragraph{Reachability computation:} The $st$-reachability problem aims at answering whether there is a directed path from vertex $s$ to vertex $t$. Despite being solvable in the sequential setting easily via breadth-first search (BFS), this basic problem has been one of the most challenging goals in parallel, streaming, dynamic, and other settings. Shortcuts and spanners has been the only tool in developing efficient algorithms in these settings  (e.g. \cite{KleinS97,HenzingerKN14,HenzingerKN15,ForsterN18,LiuJS19,Fineman18,GutenbergW20,BernsteinGW20,CaoFR20,CaoFR21,KarczmarzS21}). 
Indeed, a major goal has been to solve reachability as fast as the best shortcuts and TC spanners would allow, e.g., $st$-reachability can be solved in near-linear work and $\tilde O(\sqrt{n})$ depth in the parallel setting due to~\cite{LiuJS19}, and the depth would be further improved to $\tilde O(n^{1/3})$ if we could efficiently compute the better $\tilde O(n^{1/3})$-shortcut of size $\tilde O(n)$ by \cite{KoganP22}. 

The concept of instance-optimal shortcuts studied in this paper would serve as an ``oblivious'' method to tackle the reachability problem, i.e., a non-constructive improvement on the sizes and diameters of shortcuts  would, when combined with instance-optimal algorithms, immediately lead to improved algorithms in various models of computation. As discussed earlier, efficient algorithms in various settings were typically adapted from fast sequential algorithms. Thus, understanding the power of polynomial-time $(\apxD,\apxS)$-approximation algorithms is a stepping stone to achieving beyond-worst-case guarantees.
While our lower bounds suggest that there is little hope to achieve $\apxD,\apxS=n^{o(1)}$ in these settings (and no hope for efficient parallel algorithms assuming \conj{}), our upper bounds leave some hope for slightly bigger $\apxD$ and $\apxS$, e.g., for the input instances where the optimal shortcut is small (e.g., $d=n^{o(1)}$), our $(n^{1/5}, n^{1/5})$-approximation would already perform better than the $n^{1/3}$ upper bound that holds universally for every instance.\footnote{It is known that the optimal diameter lies in the range $[n^{1/6}, n^{1/3}]$ when we have $s=\tilde{O}(n)$ \cite{KoganP22, Hesse03, HuangP21}.} 


\paragraph{Proving strong lower bounds:}
Studying approximation algorithms also serves as an intermediate step to prove strong lower bounds for reachability. 
One major challenge of reachability computation is the {\em lack of lower-bound techniques}. For example, despite a number of recent breakthrough lower bound results \cite{CKPS0Y21,AssadiR20, AssadiCK19,GuruswamiO16}, we still cannot rule out the possibility that $st$-reachability can be solved in the streaming settings using $\tilde O(n)$ space and (say) $\tilde O(n^{0.49})$ passes. (We know that $\tilde O(n^{0.5})$ passes are possible.) 
Similar situations hold for, e.g., the parallel and two-party communication settings. 
Observe that if we could rule out this possibility, then we would also rule out $(\apxD,\apxS)$-approximation algorithms that are efficient in the corresponding settings when, e.g., $\apxD,\apxS=\polylog(n)$ (because every graph admits an $n^{1/3}$-shortcut of size $
\tilde O(n)$ \cite{KoganP22}). 
Thus, developing techniques for ruling out efficient $(\polylog(n),\polylog(n))$-approximation algorithms in {\em any} computational model is an intermediate step for proving strong reachability lower bounds.
This paper (\Cref{thm:intro:lowerbound}) provides techniques for ruling out some polynomial-time approximation algorithms. This also immediately rules out efficient parallel algorithms. Extending these techniques to prove similar results in other settings can shed some light on the techniques required to prove strong reachability lower bounds, which in turn imply lower bounds for many other problems such as strong connectivity, shortest path, st-flow, matrix inverse, determinant, and rank \cite{AbboudD16,BrandNS19}.

\paragraph{Spanners and sparsification:} Graph sparsification plays a key role in solving many fundamental graph problems such as min-cut, maxflow and vertex connectivity~\cite{ChenKLPGS22,Cen0NPSQ21,Karger00}. One classic graph sparsification technique, introduced by \cite{PelegS89}, is {\em spanners}: Given a weighted directed graph $G$, a subgraph $H$ of $G$ is a $k$-spanner of $G$ if $\dist_{H}(u,v)\leq k\cdot\dist_G(u,v)$ for every pair of vertices $(u,v).$ Directed spanners have received a lot of attention in the past decades \cite{BhattacharyyaGJRW12,BermanBMRY11,ElkinP07,KoganP22,DinitzK11,DinitzKR16}. 
The TC-spanner problem  corresponds to the directed spanner problem in \textit{transitive closure graphs}: For a graph $G$, its transitive-closure graph $G^T$ contains an edge $(u,v)$ if and only if $u$ can reach $v$ in $G$. When  all edges in $G^T$ have unit cost and length, a $d$-spanner of $G^T$ is exactly a $d$-TC spanner of $G$. When the cost of all edges in $G$ is zero and other edges in $G^T$ have cost one, a $d$-spanner of $G^T$ of cost $s$ corresponds to a $d$-shortcut of $G$ of size $s$. 
So our work can be seen as studying a natural special case of spanners, and therefore our techniques may find further use in that context. 







\subsection{Open problems}
We hope that our results serve as a stepping stone toward the ultimate goals of proving a tight lower bound and designing algorithms with beyond-worst-case guarantees for reachability across different computational models. These goals are still far to reach. Improving our bounds would be an obvious step closer to these goals. Besides this, we believe that answering the following questions is the next important step toward these goals.

\begin{enumerate}
\item Can we achieve similar lower bounds in other settings? Some of the most interesting next steps include (i) the $\polylog(n)$-pass $\tilde O(n)$-space streaming setting and (ii) the $\tilde O(\sqrt{n})$-round distributed (CONGEST) setting.\danupon{CITE something for relevant work?} Are there some reasonable conjectures in these settings that would help in arguing lower bounds?
\item Can we improve the approximation factor to (say) $(\apxD=n^{0.01}, \apxS=n^{0.01})$? Additionally, we consider it a major breakthrough if this can be achieved with an efficient algorithm such as a nearly-linear polylogarithmic-depth parallel algorithm, a $\polylog(n)$-pass $\tilde O(n)$-space streaming algorithm or a $\tilde O(\sqrt{n})$-round distributed (CONGEST) algorithm. 

\item Does the graph spanner problem admit bicriteria approximation algorithms with the same ratio? In particular, is there $(n^{\gamma_D}, n^{\gamma_S})$-approximation for $3\gamma_D + 2\gamma_S >1$? Notice that this goal would roughly interpolate between $(n^{1/3}, n^{o(1)})$-approximation and $(n^{o(1)},\sqrt{n})$-approximation. 
\end{enumerate}


\subsection{Further related work}


































\label{sec:intro:spanner}
We provide further related works on spanner approximation. 
For polynomial-time algorithms, the main question is finding a spanner of approximately minimum cost. This problem has been extensively studied since its introduction in 1990s \cite{KortsarzP94}. 
For the rather general case of directed graphs with edge costs in $\{0, 1, \ldots, O(1)\}$ and arbitrary edge lengths, the best approximation factor is $O(\sqrt{n}\ln{n})$ (due to Berman, Bhattacharyya, Makarychev, Raskhodnikova, and Yaroslavtsev \cite{BermanBMRY11} which improved from the first sublinear approximation factor by Dinitz and Krauthgamer \cite{DinitzK11}).\footnote{In \cite{BermanBMRY11}, edge costs were not considered. Their techniques can be easily adapted to work with the case where costs are in $\{0, 1, \ldots, O(1)\}$ and $s=\Omega(n)$.} 


Given the importance of the problem and the difficulties of the general case, many special cases have been extensively studied. 
One special case is when $k$ is small; e.g. for $k=4$ and directed graphs with unit length and cost, the approximation ratio can be improved to $\tilde O(n^{1/3})$ \cite{BermanBMRY13, DinitzZ16, KortsarzP94, KortsarzP98}.
Another natural special case is when we restrict the input graph class. 
One of the most basic classes is {\em undirected graphs}, where an $O(n^{2/(k+1)})$-approximation factor exists; in particular, this factor is $O(\log n)$ when $k=\Omega(\log n)$. 
This approximation factor is achieved in a rather naive way: by using a well-known fact that 
{\em every} $n$-vertex connected undirected graph contains a $k$-spanner with $O(n^{1+\frac{2}{k+1}})$ edges. Since every spanner requires $n-1$ edges to keep the graph connected, the $O(n^{2/(k+1)})$ approximation factor follows. Interestingly, this naive approximation ratio is fairly tight \cite{DinitzKR16}. 
A more interesting graph class is {\em planar graphs}, where $(1+\epsilon)$-spanner with $O(1/\epsilon)$ approximate cost can be found in polynomial time \cite{AlthoferDDJS93}. This is a building block in the polynomial-time approximation scheme for the traveling salesperson problem on weighted planar graphs \cite{AroraGKKW98}. 
While there are sophisticated algorithms for many variants of spanners (see \cite{AhmedBSHJKS20} for a comprehensive survey), we are not aware of any nontrivial algorithm for any special graph classes besides that for planar graphs.

For transitive closure graphs, we only have naive techniques to approximate spanners for graphs in this class, and our knowledge about hardness is also very limited: On the one hand, the best approximation factors for both problems are achieved by either (i) using the $O(\sqrt{n}\ln n)$-approximation algorithm for the general case \cite{BermanBMRY11}
or (ii) adapting (in a similar manner as in the case of undirected graphs) the fact that, for every $p\geq 1$, every $n$-vertex graph admits a $(n^{1/3}p)$-shortcut of size $\tilde O(np^3)$ \cite{KoganP22};
these give, e.g., the following approximation factors of TC spanners when $s\geq n$: (i) $\apxD=1$ and $\apxS=O(\sqrt{n}\ln n)$ and (ii) $\apxS=\tilde O(n^2/(sd^3\apxD^3))$ (which implies, e.g., $\apxS=\apxD=n^{1/4}$).\footnote{\cite{KoganP22} only claimed approximation factors $\apxD=1$ and $\apxS=n/d^3$. The argument (from \cite{AhoGU72}) can be generalized to the approximation factors we mentioned.} 
On the other hand, super-constant lower bounds are known only for $d$-TC spanners with constant $d \geq 2$ and $\apxD=1$: Assuming $P \neq NP$, an $\apxS = \Omega(\log n)$ inapproximability result is known for $d = 2$ and, furthermore, assuming $NP\subsetneq DTIME(2^{\polylog(n)})$, an $\apxS= \Omega(2^{\log^{1-\eps}n})$ inapproximability result is known for a constant $d \geq 3$ and any constant $0<\eps < 1$.
































\endinput
























































