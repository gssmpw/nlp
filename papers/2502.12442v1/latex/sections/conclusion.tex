\section{Conclusion}
\label{sec:conclusion}
In this paper, we introduced HopRAG, a novel RAG system with a logic-aware retrieval mechanism. HopRAG connects related passages through pseudo-queries, which allows identifying truly relevant passages within multi-hop neighborhoods of indirectly relevant ones, significantly enhancing both the precision and recall of retrieval.

Extensive experiments on multi-hop QA benchmarks, i.e. MuSiQue, 2WikiMultiHopQA, and HotpotQA, demonstrate that HopRAG outperforms conventional RAG systems and state-of-the-art baselines. Specifically, HopRAG achieved over 76.78\% higher answer accuracy and 65.07\% improved retrieval F1 score compared to dense retrievers. It highlights the effectiveness of integrating logical reasoning into the retrieval module. Moreover, ablation studies provide insights into the sensitivity of hyperparameters such as top-k and hop number, revealing trade-offs between retrieval performance and computational costs. 

HopRAG paves the way toward reasoning-driven knowledge retrieval. \textbf{Future work} involves scaling HopRAG to broader domains beyond QA tasks; optimizing the indexing and traversal strategies to handle even more complex scenarios with lower computation costs.



\section{Limitations}
Despite the benefits of HopRAG, the current evaluation focuses on multi-hop or multi-document QA tasks. In order to mitigate the risk of performance fluctuations when applying HopRAG to other datasets, we should explore its generalization capabilities across a broader range of domains. Besides, more sophisticated query simulation and edge merging strategies may lead to further improvements. Finally, though we were inspired by the theories of six degrees of separation and small-world networks, the degree distribution of our passage graph vertices does not exhibit the power-law characteristic. On the one hand, these theories only serve as the motivation and intuitive analogy; on the other hand, exploring more appropriate degree distribution strategies is an interesting research topic. We leave these research problems for future work.

\section{Ethics Impact}
In our work, we acknowledge two key ethical considerations. First, we utilized AI assistant to enhance the writing process of our paper and code. We ensure that the AI assistant was used as a tool to improve clarity and conciseness, while the final content and ideas were developed and reviewed by human authors. Second, we employed multiple open source datasets and one open source tool Neo4j Community Edition\citep{10.1145/2384716.2384777} under GPL v3 license in our experiments. We are transparent about their origin and limitations, and we respect data ownership and user privacy. 
% This section synthesizes the key findings and contributions discussed throughout the paper, emphasizing the advancements made by HopRAG in addressing the challenges of imperfect retrieval in RAG systems. It concludes with potential directions for future research to build upon the foundation laid by this work.


% Retrieval-augmented generation (RAG) systems often struggle with imperfect retrieval, where lexically or semantically similar passages lack logical relevance to user queries. To address this, we introduced \textbf{HopRAG}, a novel graph-structured RAG framework that integrates logical reasoning into retrieval through a graph index and multi-hop traversal. By connecting passages via pseudo queries and employing a retrieve-reason-prune mechanism, HopRAG transforms indirectly relevant passages into stepping stones for identifying truly relevant information.


% In this paper, we notice that the logical relations among different passages can be used to index a multi-hop reasoning dataset and improve the suboptimal performance of similarity retrieval for knowledge intensive question answering task. Specifically, we introduce HopRAG, an innovative graph-structured RAG approach for logic-aware retrieval-augmented generation, which also delivers a consistent improvement over state-of-the-art RAG baselines across several multi-hop QA benchmarks.


% Our experiments demonstrated HopRAG’s superiority over conventional RAG methods, achieving improvements of 76.78% in answer accuracy and 65.07% in retrieval F1 score. The system’s ability to model cross-document logical relations while avoiding the complexity of predefined knowledge graphs offers a practical advancement for knowledge-intensive tasks. Ablation studies further validated the importance of pseudo query generation and reasoning-guided traversal.

% HopRAG’s success highlights the potential of logic-aware retrieval paradigms. Future work could explore dynamic graph updates, optimized traversal algorithms, and integration with advanced reasoning models. By bridging the gap between retrieval and reasoning, HopRAG paves the way for more robust, interpretable, and contextually grounded RAG systems.


% We found that while increasing `topk` and `nhop` generally improves retrieval quality, it also introduces redundancy which can negatively impact the model's ability to efficiently identify useful information. Notably, an optimal setting of `nhop=4` was identified as balancing performance gains with computational efficiency.

% The design of HopRAG offers several advantages over existing RAG systems: it supports flexible logical modeling across documents, facilitates efficient construction and updating of the knowledge index, and avoids the introduction of additional summary or proposition aggregate nodes. These features make HopRAG particularly well-suited for downstream tasks requiring precise and logically coherent retrieval of information.

