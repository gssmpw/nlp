\section{Introduction}
\label{sec:Intro}

\noindent
A critical computational domain for hardware accelerators is the area of solving combinatorial optimization problems (COPs) that are NP-complete or NP-hard -- e.g., the traveling salesman, satisfiability, and knapsack problems.  Today, such problems are solved on classical computers using heuristics with no optimality guarantees, or approximation algorithms with loose optimality bounds.

Ising computation is a promising emerging computational model for solving COPs.  Ising machines are inspired by the work of Ernst Ising, who proposed a formulation based on binary states called {\em spins}, with allowable values of $+1$ and $-1$, to explain ferromagnetism. Such systems have a natural tendency to find a {\em ground state} with a configuration of spins that minimizes energy.  Ising computation maps discrete combinatorial optimization problems to this paradigm.  Under a linear transformation, Boolean quadratic unconstrained binary optimization (QUBO) problems can be formulated as two-body Ising interactions.   Karp's list of 21 NP-complete problems are shown to have an Ising formulation~\cite{Lucas_Ising_Frontiers14}, and many other problems can also be formulated in this way. 

Recently, there has been great interest in building Ising hardware accelerators, realizing spins using superconducting loops in D-Wave machines~\cite{Johnson2011, Bian2014}; optical parametric oscillators in coherent Ising machines~\cite{Inagaki2016,Yamamoto2017}, ring oscillators (ROs) in CMOS-based Ising machines~\cite{wang2019matlab,moy20221,Lo2023},
and memory cells in SRAM-based engines~\cite{Yamaoka16}.   
Oscillator-based methods use the phenomenon of {\em synchronization}, whereby a system of coupled oscillators with similar frequencies, converge to a common frequency and fixed phase difference through injection locking.
The dynamics of coupled oscillators have been studied as early as 1663, when Huygens noticed the synchronization of pendulums connected to a common bar~\cite{Willms17}. Adler derived a closed-form expression for locking in LC oscillators~\cite{Adler}, while Winfree explored weak interactions of periodic behavior in biological rhythms~\cite{WINFREE196715}. 

Kuramoto's analysis~\cite{Kuramoto1984-hj} studied chemical oscillations under sinusoidal interactions. These works simulate synchronization behavior 
through differential equations that relate the rate of change of each oscillator phase to the phases and frequencies of other oscillators. 

Unlike platforms using exotic futuristic technologies, CMOS RO-based Ising machines use a mainstream semiconductor technology that is scalable, compact, economically and reliably mass-manufacturable today, and can operate at room temperature instead of requiring expensive high-power mK-level refrigeration schemes. The synchronization of RO-based Ising machines can be simulated using HSPICE, but this is computationally intensive and does not scale well. Simulators for oscillator-based Ising machines are based on analytical solutions to the generalized Adler equation~\cite{Bhansali2009} and the generalized Kuramoto equation~\cite{wang2019matlab}.  A prior event-driven approach~\cite{sreedhara23} fast-forwards through multiple RO cycles until the phase difference between some pair of ROs crosses an integer multiple of $\pi$: this is considered to be an event. If the phase difference remains close to an integer multiple of $\pi$ for some iterations, the associated coupling is removed from the system and a phase merging scheme is used to lock the phases of these oscillators henceforth. A hardware realization of a generalized Kuramoto equation solver has also been demonstrated~\cite{sreedhara_date23}.

Prior methods have several limitations.  First, 
they represent the phase of each oscillator by the phase of a single reference stage. However, the phase differences at specific coupling sites between two oscillators may differ from the differences in their reference phases. 
Second, methods that use phase merging~\cite{sreedhara23} can be misleading: the phase of an RO can diverge even after it appears to come close to another RO phase. 
This work proposes DROID (Discrete-Time Simulation for Ring-Oscillator-Based Ising Design), a method for simulating RO-based Ising machines, that overcomes the above limitations. Its contributions are as follows:
\begin{itemize}[noitemsep,topsep=-1pt,leftmargin=*]
\item 
We show that for coupled RO systems, prior continuous-time (CT) simulation abstractions, such as the generalized Adler formulation~\cite{Bhansali2009}, are abstractions of a discrete-event simulation, operating under restrictive assumptions that allow closed-form solutions, including assumptions of infinitesimal changes 
(Section~\ref{sec:Adler_relation}).  Our approach removes these restrictions and uses lookup-table-based functions, characterized using HSPICE. 

\item 
Unlike prior methods that work in the continuous domain, we develop a discrete-time event-driven simulation methodology (Section~\ref{sec:Simulation}) to predict the behavior of coupled RO systems; this method is inspired by timing analysis methods that are widely used for digital circuits, which achieve acceptable accuracy at a fraction of the runtime of HSPICE. Our approach is event-driven, where an event is defined with fine granularity, associated with a coupled transition between two oscillators.

\item 
Our approach is 125$\times$--7441$\times$ faster than HSPICE at similar accuracy, with larger speedups for larger systems. We match the distribution of our solutions, across 250 problems of various oscillator coupling densities, 100 samples per problem, and multiple initial conditions, against a CMOS RO-based Ising hardware solver~\cite{Lo2023}, and show that the distance between distributions, is small. 

\end{itemize}

The paper is organized as follows. Section~\ref{sec:Background} summarizes the concepts that guide this work. Section~\ref{sec:A2A} describes an all-to-all-connected Ising hardware accelerator that serves as our hardware testcase. Sections~\ref{sec:Adler_relation} and~\ref{sec:limitations_of_ct_approx} then analyze the relationship between discrete-time and continuous-time simulation of coupled oscillator systems.  We describe our event-driven simulation scheme for the Ising hardware in Section~\ref{sec:Simulation} and show our simulator results in Section~\ref{sec:Results}, finally concluding the paper in Section~\ref{sec:Conclusion}.