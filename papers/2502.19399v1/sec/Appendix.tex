\section{}
\label{app:appendix}

\subsection{Pseudocode for the \textproc{process\_event} function}
\label{app:process_event}

\noindent
The pseudocode for the function \textproc{process\_event} is listed in Algorithm~\ref{alg:process_event}. The function receives an event E, the queue Q, and the maps Net2Event and PendingTrigger, and generates output event(s) which it inserts into Q. The major steps involved are as follows:

\noindent 
\textbf{Step PE1: Find the cell type of the cell that receives E as input} The netlist and the net name property of the event object E are used to determine the cell that receives the event at its input.  If the cell is an enable cell, the output event can be calculated directly (lines~\ref{alg:enable_start}--\ref{alg:enable_end}), and the event is processed. Otherwise, in case of a coupling or a shorting cell, we proceed to Step PE2. 

\noindent 
\textbf{Step PE2: Find the direction of the path that E lies on} If the event E occurs on a net on the reverse path (on pin $h^r_{in}$ or $v^r_{in}$ of a cell), there is no coupling interaction and the output events are calculated directly (lines~\ref{alg:backward_start}--\ref{alg:backward_end}). Otherwise, for an event on a forward path (on pin $h^f_{in}$ or $v^f_{in}$), we proceed to Step PE3.

\begin{algorithm}[H]
    {
    \small
    \caption{\textproc{process\_event}: Processing an event in the event queue}
    \label{alg:process_event}
    \algrenewcommand{\algorithmicindent}{0.5em}
        \begin{algorithmic}[1]
        \Function{process\_event}{E, Q, Net2Event, PendingTrigger}
            \State \textit{cell} =  the cell instance that receives event E at its input
            \State remove\_event = list() \textit{// List for processed events}
            \State NE = list() \textit{// List for output events}
            \State \textit{// Step PE1: Find the cell type of the cell that receives E as input.} 
            \If {\textit{cell} is a coupling or shorting cell}
                \State \textit{pin} = the pin of \textit{cell} that event E occurs at
                \State \textit{// Step PE2: Find the direction of the path that E lies on.}
                \If {\textit{pin} is $h_{in}^{f}$ or $v_{in}^{f}$}
                    \State \textit{// Step PE3: Find an interacting event for the forward path.}
                    \If {event E' exists on other input} \label{alg:forward_start}
                        \If {E and E' are within a window of W}
                            \State NE $\xleftarrow{}$ Calculate output events from E and E'
                            \State remove\_event.add([E, E'])
                        \Else
                            \State NE $\xleftarrow{}$ Calculate output event from E
                            \State remove\_event.add(E)
                        \EndIf \label{alg:forward_end}
                    \Else
                        \State \textit{// Step PE4: Look back to find events that might result in interaction.}
                        \State \textit{net} = the net connected to other input of cell
                        \State (left, right) = (E.arrival\_time - W, E.arrival\_time + W)
                        \State status = \Call{look\_back}{\textit{net}, (left, right), left, Net2Event}
                        \If {status}
                            \State PendingTrigger[\textit{net}] = E
                        \Else
                            \State NE $\xleftarrow{}$ Calculate output event from E
                            \State remove\_event.add(E)
                        \EndIf
                    \EndIf
                \Else
                    \State \textit{// Calculate events on the backward path.} \label{alg:backward_start}
                    \State NE $\xleftarrow{}$ Calculate output event from E
                    \State remove\_event.add(E) \label{alg:backward_end}
                \EndIf
            \Else 
                \State \textit{// enable cell} \label{alg:enable_start}
                \State NE $\xleftarrow{}$ Calculate output event from E
                \State remove\_event.add(E) \label{alg:enable_end}
            \EndIf
            \State \textit{// Step PE5: Check if events in NE are triggers to a pending event}
            \If {NE not empty} 
                \For {new\_event $\in$ NE}
                    \State Q.add(new\_event)
                    \State Net2Event[new\_event.netname] = new\_event
                    \If {new\_event.netname \textbf{in} PendingTrigger}
                        \State Q.add(PendingTrigger[new\_event.netname])
                        \State PendingTrigger.pop(new\_event.netname)
                    \EndIf
                \EndFor
            \EndIf
            \State \textit{// Remove consumed events}
            \For {used\_event $\in$ remove\_events}
                \State Net2Event.pop(used\_event.netname)
            \EndFor
        \EndFunction
        \end{algorithmic}
    }
\end{algorithm}

\noindent 
\textbf{Step PE3: Find an interacting event for the forward path} If the event E occurs on $h_{in}^{f}$, then only an event at $v_{in}^{f}$ of the same cell can interact with E and vice-versa. The map Net2Event can be queried with the net name of the other input to look for an interacting event E'. Even if an event at the other input is not found in the map, it is possible that such an event has not yet been generated from a predecessor cell, and we proceed to Step PE4 to look for events that might still result in interaction with E. Note that the interaction window lies in a range of $\pm W$ around the arrival time of event E, and therefore this process should identify any event within this window -- one that precedes or succeeds E.   

If found, the switching information for events E and E' is sufficient to determine whether they interact or not, and the output events may be calculated (lines~\ref{alg:forward_start}--\ref{alg:forward_end}). Interacting events E and E' generate a pair of output events, while a noninteracting event generates one output event. 

\noindent
\textbf{Step PE4: Look back to find events that might result in interaction} When an event is on the other input not found in the map Net2Event, the function \textproc{look\_back}, described in detail in Appendix~\ref{app:appendix}-\ref{app:look_back}, is invoked.  As described at the end of the example, for an event E at a cell C, intuitively, this procedure looks into the predecessors of C to determine whether any incoming, but as yet unprocessed, event might result in another event that interacts with E. If such an event is found, the event E is added to the PendingTrigger map with the net name of the other input of the cell as its trigger. If not, we generate the output event from E assuming no interaction. 

\noindent
\textbf{Step PE5: Check if output events are triggers to a pending event} If E generates an event on a net that is a trigger to a pending event in the PendingTrigger map, then the pending event is ready to be processed and is added to the queue.


\subsection{Pseudocode for the \textproc{look\_back} function}
\label{app:look_back}

\begin{algorithm}[H]
    {
    \small
    \caption{\textproc{look\_back}: Identification of events that might interact with an event being processed}
    \label{alg:look_back}
    \algrenewcommand{\algorithmicindent}{0.5em} 
    \begin{algorithmic}[1]
    \Function{look\_back}{\textit{event\_net}, (left, right), \textit{threshold}, Net2Event}
        \State \textit{// looks back recursively to find events that might interact}
        \If {right $<$ \textit{threshold}} 
            \State \textit{// Base case: looked back far enough}
            \State \Return False 
        \EndIf
        \If {\textit{event\_net} is in Net2Event}
            \State \textit{// Base cases: Interacting or not}
            \If {the event will arrive in (left, right)}
                \State \Return True 
            \ElsIf {the event will not arrive in (left, right)}
                \State \Return False 
            \EndIf
        \Else
            \State \textit{// Recursive step: need to look back further}
            \State \textit{preceding\_net} = the net that can cause an event at \textit{event\_net}
            \State \textit{preceding\_cell} = the instance for which event\_net is an output
            \State (nleft, nright) = (left - \textit{preceding\_cell}.$d_{max}$, right - \textit{preceding\_cell}.$d_{min}$)
            \State \Return \Call{look\_back}{\textit{preceding\_net}, (nleft, nright), \textit{threshold}, Net2Event}
        \EndIf
    \EndFunction
    \end{algorithmic}
    }
\end{algorithm}
\noindent
The pseudocode for the function \textproc{look\_back} is provided in Algorithm~\ref{alg:look_back}. The inputs to the function are a net \textit{event\_net}, a window of arrival time (left, right), a threshold arrival time \textit{threshold}, and the map Net2Event. \textproc{look\_back} is a recursive function that terminates when it encounters one of three base cases:

\begin{itemize}[noitemsep,topsep=-1pt,leftmargin=*]
    \item The latest arrival time of an event at \textit{event\_net} is less than \textit{threshold}. The function returns false. 
    \item An event arrives at \textit{event\_net} within the window and can result in an interaction. The function returns true.
    \item An event arrives at \textit{event\_net} outside the window and will not result in an interaction. The function returns false.
\end{itemize}


The first base case prevents us from looking at too many predecessors by comparing the latest arrival time to \textit{threshold}. The \textit{threshold} is assigned a value in \textproc{process\_event} which corresponds to the earliest arrival time for an interaction with the event that invoked \textproc{look\_back}. 
The second and third cases require looking at the map Net2Event for an event with the key \textit{event\_net} to decide if it will arrive within the arrival time window. When none of the base cases are encountered, the function moves to the recursive step.

The predecessor of \textit{event\_net}, which we call \textit{preceding\_net} is found when an event at \textit{event\_net} does not exist in Net2Event. The cell that \textit{preceding\_net} is an input of is called \textit{preceding\_cell}. The minimum and maximum delays ($d_{min}$ and $d_{max}$, respectively) for \textit{preceding\_cell} are used to calculate a window of arrival for \textit{preceding\_net}. A recursive call is made at \textit{preceding\_net} with this new window, with the recursion concluding when the interaction window has been exceeded.