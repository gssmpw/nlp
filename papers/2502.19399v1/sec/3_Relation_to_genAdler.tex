\section{Discrete-time vs. continuous-time simulation of coupled-RO systems}
\label{sec:Adler_relation}

\noindent
\underline{Traditional continuous-time formulations.}
The behavior of an LC oscillator, when injected with a sinusoidal signal, was described by Adler's equation~\cite{Adler}; a slightly different equation is used by Kuramoto~\cite{Kuramoto1984-hj}. To extend this beyond a single coupling and sinusoidal signals, the generalized Adler (GenAdler) equation for a network of $N$ coupled oscillators was shown~\cite{Bhansali2009} to have the form:
\begin{equation} \label{eq:gen-adler}
    \frac{d \phi_{i}(t)}{dt} = (\omega_i - \omega^*) + \omega_i \textstyle \sum_{j=1,j \neq i}^{N}c_{ij} (\phase{ij}{}(t)) 
\end{equation}
Here, $\phi_{ij}(t) = \phi_{i}(t) - \phi_{j}(t)$ is the difference between the phases of oscillators $i$ and $j$, $\omega_i$ is the frequency of the $i^{\rm th}$ oscillator, $\omega^*$ is the central frequency of the network, and for oscillators $i$ and $j$, $c_{ij}(.)$ is a $2\pi$-periodic function that represents the coupling-induced delay shift in each RO cycle. Prior methods~\cite{Bhansali2009} abstract $c_{ij}$ as a well-behaved function of $\phi_{ij}$, the phase difference of the coupled ring oscillators; Fig.~\ref{fig:char_function} shows an example HSPICE-characterized function showing the RO period shift against the phase difference.

\begin{figure}[htb]
    \centering
    \hspace*{-3mm}
    \subfigure[]{\includegraphics[width=0.45\linewidth]{img/charplot.pdf}
    \label{fig:char_function}
    }
    \hspace{-2mm}
    \subfigure[]{\includegraphics[width=0.45\linewidth]{img/HSPICE_vs_tanh_f.pdf}
    \label{fig:genAdler_vs_HSPICE}
    }
    \vspace{-3mm}
    \caption{(a)~The results for an example characterization setup, with $J_{ij}=1$, showing the delay shift $f_{J_{ij}}$ as a function of $\phi_{ij}$. (b)~HSPICE characterization of $f_{J_{ij}}$ and its tanh approximation as a function of phase difference $\phi_{ij}$.}
    \vspace{-2mm}
\end{figure}

\noindent
\underline{Discrete-time formulation.}
The continuous formulation models a discrete-event system composed of a sequence of coupling events.  We now examine the limitations of the continuous-time GenAdler model, as well as those of the phase delay shift model.

Fig.~\ref{fig:char_function} shows a model for the delay shift, $f_{J_{ij}}(\phi_{ij})$ of RO$_i$ and RO$_j$ at an example coupling strength $J_{ij} = 1$, where $\phi_{ij}$ is the phase difference between the two ROs. The delay shifts of RO$_i$ and RO$_j$ are shown by the red and green dotted lines, and the relative phase shift, i.e., the difference between these delay shifts, is shown by the solid blue line. It is seen that when the edges align to be in-phase or out-of-phase (i.e., at 0 or $T/2)$, the relative phase shift is zero.

We model the coupled system using a sequence of discrete events that add up to a shift in an RO clock period at the end of a cycle.  We denote the phase, period, and frequency as $\phase{i}{k}$, $\period{i}{k}$, and $\freq{i}{k}$, respectively, for RO$_i$ at the end of the $k^{\rm th}$ cycle. We use a datum oscillator frequency, $\omega^*$, which may be the frequency corresponding to the period $T$ of each uncoupled oscillator.

The phase shift of each oscillator from $\phase{i}{k}$ to $\phase{i}{k+1}$ during the $(k+1)^{\rm th}$ cycle is caused by two factors:\\
(1) frequency drift with respect to the reference oscillator:
\begin{align}
    \Delta \phase[1]{i}{k+1} = \phase[1]{i}{k+1} - \phase[1]{i}{k} 
                             = (\freq{i}{k} - \omega^*) \period{i}{k} 
    \label{eq:delta_phase_i_1} 
\end{align}
(2) phase/frequency drift due to coupling to other ROs:
\begin{align}
    \Delta \phase[2]{i}{k+1} = \phase[2]{i}{k+1} - \phase[2]{i}{k} 
                             = \textstyle \sum_{(i,j)\in E} f_{J_{ij}} (\phase{ij}{k}) 
    \label{eq:delta_phase_i_2}
\end{align}
where $E$ is the set of edges in the coupling graph (Section~\ref{sec:Background}-\ref{subsec:ro_ising}).  The net phase shift in the $(k+1)^{\rm th}$ cycle is
\begin{align}
\Delta \phase{i}{k+1} 
    =\Delta \phase[1]{i}{k+1} + \Delta \phase[2]{i}{k+1} 
    =(\freq{i}{k} - \omega^*) \period{i}{k} + \textstyle \sum_{(i,j)\in E} f_{J_{ij}} (\phase{ij}{k})
    \label{eq:delta_phase_i}
\end{align}
At synchronization, the clock frequency, $\omega_i^k$ is the same for all oscillators. Thus, in writing the relative phase difference between coupled oscillators RO$_i$ and RO$_j$ (note that $\phi_{ji} = -\phi_{ij}$), their corresponding first terms in~\eqref{eq:delta_phase_i} cancel, and we have:
\begin{align}
\Delta \phase{i}{k+1} - \Delta \phase{j}{k+1} = \textstyle \sum_{(i,j)\in E} f_{J_{ij}} (\phase{ij}{k}) - f_{J_{ij}}(-\phase{ij}{k}) = 0
\end{align}
The last equality arises because when the phases are locked, the difference between the delay shifts of locked oscillators is zero.

\noindent
\underline{Relation between the continuous- and discrete-time formulations.}
Unlike coupled sinusoidal oscillators, as modeled by the GenAdler formulation, where the signal value changes throughout a cycle, for an RO system, the coupling component of $d\phi_{i}(t)/dt$ changes during signal transitions but not when signals are stable at logic 0 or logic 1. 
Under infinitesimal phase changes per cycle, the derivative can be approximated if the net phase change over $m$ cycles is small. If the period of RO$_i$ in cycle $k$ is $\period{i}{k}$,  
\begin{align}
\frac{d\phi_{i}(t)}{dt} \approx \frac{1}{m} \frac{\sum_{k=1}^m \Delta\phase{i}{k}}{\Delta t},
\mbox{  where  } \Delta t = \textstyle \sum_{k=1}^m \period{i}{k}
\end{align}
From~\eqref{eq:delta_phase_i}, the phase change in time $\Delta t$ ($m$ cycles of RO$_i$) is 
\begin{align}
    \Delta \phi_{i} = \textstyle \sum_{k=1}^{m} \Delta \phase{i}{k} 
    =\textstyle \sum_{k=1}^m \left(\freq{i}{k} - \omega^* \right) T_i^k
        + \textstyle \sum_{(i,j)\in E} \textstyle \sum_{k=1}^{m} f_{J_{ij}} (\phase{ij}{k})         
    \label{eq:delta_phase_i_m_cycles}
\end{align}
The delay shifts over the $m$ cycles must be assumed to be small, i.e., $\freq{i}{k} \approx \omega_i \; \forall \; k = 1, \cdots, m$. Under this assumption, 
\begin{align}
\sum_{k=1}^{m} (\freq{i}{k} \period{i}{k} - \omega^*\period{i}{k}) 
    \approx (\omega_{i} - \omega^*) \sum_{k=1}^{m} \period{i}{k} 
    = (\omega_{i} - \omega^*) \Delta t
    \label{eq:summed_first_term}
\end{align}
Let $T_i$ be the average value of $\period{i}{k}$. Then  
\begin{align}
T_i =  \frac{1}{m} {\textstyle \sum_{k=1}^{m} \period{i}{k}}= \frac{\Delta t}{m}
\Rightarrow
m = \frac{\Delta t}{T_i} = \left( \frac{\omega_i}{2\pi} \right) \Delta t \label{eq:omega_to_m}
\end{align}
where $\omega_i \stackrel{\Delta}{=} 2\pi/T_i$.
Under small phase changes over $m$ cycles,  for $l = 1, 2, \cdots, k$, we write
$f_{J_{ij}} (\phase{ij}{l}) \approx f_{J_{ij}} (\phase{ij}{1}) + S_{ij}(\phase{ij}{l}-\phase{ij}{1})$ as a linear approximation,
where the sensitivity $S_{ij} = \left . \left ( \partial f_{J_{ij}}/\partial \phi_{ij} \right ) \right|_{\phase{ij}{1}}$. Therefore,
\begin{align}
\textstyle \sum_{k=1}^{m} f_{J_{ij}} (\phase{ij}{l}) 
    &= m \left[ f_{J_{ij}} (\phase{ij}{1}) + S_{ij} \left(\frac{\textstyle \sum_{k=1}^{m}\phase{ij}{k}}{m}  - \phase{ij}{1} \right) \right]
    \\
    &= \omega_i \left[ \frac{f_{J_{ij}} (\phi_{ij})}{2\pi}  \right] \Delta t \mbox{\hspace{4mm} (using \eqref{eq:omega_to_m})} \label{eq:summed_second_term}
\end{align}
where $\phi_{ij} = \sum_{k=1}^{m}\phase{ij}{k}/m$ is the mean value of $\phase{ij}{k}$ over $m$ cycles. 
From~\eqref{eq:delta_phase_i_m_cycles},~\eqref{eq:summed_first_term}, and~\eqref{eq:summed_second_term},
since $\Delta t$ is assumed to be small,
\begin{align}
    \frac{\partial \phi_i}{\partial t} &\approx \frac{\Delta \phi_i }{\Delta t} = (\omega_{i} - \omega^*) + \omega_i \sum_{(i,j)\in E}\frac{f_{J_{ij}} (\phi_{ij})}{2\pi} 
    \label{eq:final}
\end{align}
Setting $c_{ij} = f_{J_{ij}} (\phi_{ij})/(2\pi)$, this is the GenAdler equation,~\eqref{eq:gen-adler}.

\section{Limitations of the continuous-time approximation}
\label{sec:limitations_of_ct_approx}

\noindent
The continuous-time approach is effective in matching a coupled CMOS RO system when: 
\begin{enumerate}[label=(\alph*),noitemsep,topsep=-1pt,leftmargin=*]
    \item the phase difference between any pair of oscillators is independent of the location of coupling, and
    \item the phase difference between any pair of oscillators within a cycle is independent of their coupling to other oscillators. 
\end{enumerate}
From Section~\ref{sec:Adler_relation}, the function $c_{ij}(.)$ is crucial to the correctness of the model.  For CMOS ROs, coupling is expressed through complex MOS models (e.g., BSIM4, BSIM-CMG), the mapping from the system to the coefficient is nontrivial. 

\begin{figure}[htb]
    \centering
    \subfigure[]{\includegraphics[width=0.5\linewidth]{img/cplng_row.pdf}
    \label{fig:coupling_row}}
    
    \subfigure[]{\includegraphics[width=0.47\linewidth]{img/cplng_effect_base.pdf}
    \label{fig:coupling_effect_base}}
    \subfigure[]{\includegraphics[width=0.47\linewidth]{img/cplng_effect_shifted.pdf}
    \label{fig:coupling_effect_shifted}}
    \vspace{-3mm}
    \caption{(a)~A section of the A2A array with three coupling cells. (b)~Phase difference at coupling cells depends on the delay difference at the inputs of the coupling cell and varies with location. (c)~The delay introduced by previous coupling cells affects the phase differences at later stages within the same cycle. 
    }
    \label{fig:coupling_example}
    \vspace{-4mm}
\end{figure}

To describe the impact of assumption~(a), we present an example that shows that the phase difference at a coupling cell in an A2A array depends on the location of the cell within the array and the magnitude of coupling in previous stages of the ROs. Our example in Fig.~\ref{fig:coupling_row} shows a section of an A2A array, where the labeled nets are inputs to a set of coupled cells in the array. The horizontal RO, RO$_Z$, runs through the lines $z$, $z'$, and $z''$. This oscillator couples with the vertical oscillators, RO$_W$, RO$_X$, and RO$_Y$, at the stages with inputs $w$, $x$, and $y$, respectively,  through tiles that implement the coupling coefficients, $J_{ZW} = 6$, $J_{ZX} = 6$, and $J_{ZY} = -6$. 
The locations $w$, $x$, $y$, and $z$ correspond to reference phases of respective ROs. Given a set of initial reference phases for each RO, we show the transitions at various locations in Fig.~\ref{fig:coupling_effect_base}. As mentioned in Section~\ref{sec:Background}-\ref{sec:practical_considerations}, the phase difference at the coupling cell sites may differ from the phase difference at the reference stages of their oscillators. For instance, the phase difference at $J_{ZX}$ is 2.6ps, corresponding to the difference in arrival times of transitions at $z'$ and $x$, which is different from the difference or 107.4ps in the arrival times of transitions at $z$ and $x$. As can be seen from Fig.~\ref{fig:transitions}, an inaccuracy of this magnitude (comparable to window $W$) will impact delay calculation for a coupling cell, also affecting the phase difference at the next cell.

Next, we consider the impact of assumption (b) alone, using corrected arrival times to eliminate the contribution of errors from assumption (a).  
As discussed in Section~\ref{sec:Background}-\ref{sec:practical_considerations}, RO delays can vary with the arrival time difference, and these delays are also subtly impacted by changes in the signal transition time at the RO input.
We determine the coupling between oscillators in Fig.~\ref{fig:coupling_effect_shifted} for a case where the arrival times of transitions at $w$, $x$, and $y$ are identical, but the transition at $z$ is slightly delayed, compared to the corresponding values in Fig.~\ref{fig:coupling_effect_base}.  This small change has the effect of changing the coupling delay and transition time at $z'$, with a ripple effect on the timing of transitions at $z'$ and $z''$, caused by the coupling between RO$_Z$ and other oscillators: it can be seen that the small shift of $<$20ps at $z$ shifts the transition at $z'$ by $>$50ps, and at $z''$ by $>$100ps.
These magnified shifts arise because the arrival time at $z'$ depends on the magnitude of coupling, $J_{ZW}$, and that at $z''$ depends on both $J_{ZW}$ and $J_{ZY}$. Thus, it is not just the number of stages from the reference to the coupling stage that affects the delay shift; the magnitude of coupling in previous stages, and the precise timing relationship between waveforms in those stages, also affect phase differences at a later stage.

The GenAdler formulation in~\eqref{eq:gen-adler} makes assumptions (a) and (b), and simply uses the phase difference $\phi_{ij}$ between the reference stages of oscillators $i$ and $j$.  In Section~\ref{sec:Simulation}, we present a fine-grained event-driven approach to overcome these limitations. 

We know of only one prior event-driven approach~\cite{sreedhara23}, but it uses a fundamentally different definition of events from ours, and speeds up the generalized Kuramoto simulation. This method inherits the assumptions as GenAdler, and hence, limitations (a) and (b).  Its speedup mechanism determines whether, at any time, two or more ROs achieve a phase difference that is an integral multiple of $\pi$ radians: if so, it assumes that these oscillators will remain permanently phase-locked from that time onwards. 
Through this assumption, the number of variables is reduced as the simulation proceeds, reducing its computational cost.

We show a counterexample, based on HSPICE simulations, that illustrates that such preliminary phase-locking assumptions~\cite{sreedhara23} can be incorrect.  Consider a system comprising four ROs, denoted as RO$_A$, RO$_B$, RO$_C$, and RO$_D$. Fig.~\ref{fig:diverge} shows the phase differences in radians between coupled RO pairs as the system evolves in time. The phase difference $\phi_{AB}$ (red) remains close to 0 from 20ns to 400ns and $\phi_{BC}$ (blue) remains close to $-\pi$ from 20ns to 400ns, as shown in the highlighted green box. In the interval [20ns, 400ns], it appears as if RO$_A$ and RO$_B$ are locked in-phase while RO$_B$ and RO$_C$ are locked out-of-phase, but oscillator RO$_D$ is not yet settled as $\phi_{AD}$ (green) continues to change. As shown in the example in Fig.~\ref{fig:coupling_example}, couplings in earlier stages affect delays within the same cycle, and it is a result of this effect that the changes in $\phi_{AD}$ become more dramatic around $t=$ 400ns, it causes phase differences at other coupling cells to change. The net effect of these changes on RO$_A$, RO$_B$, and RO$_C$ is that they leave their seemingly phase-locked relationships as the system evolves, and settle to a different equilibrium at $t=$ 900ns, where RO$_A$ is out-of-phase with RO$_B$, and RO$_D$, and in-phase with RO$_C$. If the phases of RO$_A$ and RO$_B$ (or RO$_B$ and RO$_C$) were merged into a single phase based on their behavior between 20ns and 400ns, this equilibrium stage would not be captured by the simulation.

\begin{figure}[htb]
    \centering
    \includegraphics[width=0.95\linewidth]{img/phase_merging_violated.pdf}
    \vspace{-4mm}
    \caption{The phase differences across various edges are shown above. The phase differences $\phi_{AB}$ (red) and $\phi_{BC}$ (blue) stay close to $0$ and $\pi$ between 20ns to 400ns giving the impression of being phase-locked. The phase differences deviate from these seemingly phase-locked positions beyond 400ns. 
    }
    \label{fig:diverge}
    \vspace{-8mm}
\end{figure}

