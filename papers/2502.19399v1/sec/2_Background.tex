\section{CMOS-based Coupled-oscillator Systems}
\label{sec:Background}

\noindent
\subsection{The Ising model}
\label{subsec:Ising}

\noindent
The Ising formulation of a COP minimizes the following objective function, referred to as a {\em Hamiltonian}:
\begin{equation}
    H(\mathbf{s}) = \textstyle -\sum_{i = 1}^{N} \sum_{j=1}^{N} J_{ij} s_i s_j - \sum_{i=1}^{N} h_i s_i,
    \label{eq:ham}
\end{equation}
In the magnetics domain, this models the energy of a system of $N$ spins; spin is an intrinsic property associated with a subatomic particle, atom, or molecule, and can take on a value of $+1$ or $-1$.
The Hamiltonian is the energy of a system of spins as a function of their interactions ($J_{ij} s_i s_j$) and the effect of external magnetic fields on individual spins ($h_i s_i$). A physical Ising machine settles to a ground state of low-energy states favored by nature, thus minimizing the Hamiltonian. Therefore, by suitably mapping a COP to the weights $J_{ij}$ and $h_i$, an Ising machine can solve a COP formulated as a Hamiltonian.

Two spins $s_i$ and $s_j$ are in-phase if $s_i = s_j$, and out-of-phase otherwise. From~\eqref{eq:ham}, a positive [negative] $J_{ij}$ encourages $s_i$ and $s_j$ to be in-phase [out-of-phase], and a positive [negative] $h_i$ pushes $s_i$ to be $+1$ [$-1$].

\subsection{CMOS-Ring-Oscillator-Based Ising Machines}
\label{subsec:ro_ising}

\vspace{-4mm}
\begin{figure}[htb]
    \centering
    \includegraphics[width=0.8\linewidth]{img/coupled_rosc.pdf}
    \vspace{-4mm}
    \caption{Three five-stage ROs, with a positive and a negative coupling.  The green stage is the reference, and odd (even) stages are shown in red (blue).}
    \label{fig:RO_coupling}
    \vspace{-2mm}
\end{figure}

\noindent
\underline{Core principle.}
Fig.~\ref{fig:RO_coupling} shows a CMOS-based Ising machine with three coupled ROs. Each RO has an identical number of stages; each stage in the $i^{\rm th}$ oscillator, RO$_i$, is an inverter, except for one NAND stage with an enable signal, en$_i$, to start the oscillator. 

We denote the $k^{\rm th}$ stage of RO$_i$ as RO$_{i}^{k}$, with a phase, $\phi_i^k$, given by the arrival time of the rising edge at the stage output. The phase $\phi_i$ of an oscillator is defined as the phase, $\phi_i^0$, at the output of its zeroth stage (shown in green in the figure). The phase at stage RO$_i^k$ is found by adding to $\phi_i$ the sum of the stage delays from RO$_i^0$ to RO$_i^k$.  The time between two consecutive rising edges at the output of a stage is the period of the RO. We denote the nominal period, i.e., the period of each uncoupled, freely oscillating RO, as $T$. When the oscillators are coupled together, they synchronize to the same period, which may be different from $T$.

We designate one of the oscillators as a reference oscillator, always setting its spin to $+1$; without loss of generality, we refer to this as RO$_0$. To assign a spin value to an oscillator, its phase is compared with that of the reference oscillator: an oscillator that is in-phase with RO$_0$ is said to have a spin of $+1$, and one that is out-of-phase has a spin of $-1$.

Ising machines employ weak coupling~\cite{Kuramoto1984-hj}, where the delay change due to coupling is small compared to the nominal period $T$. A non-zero coupling coefficient, $J_{ij}$, in the Ising model is realized by coupling RO$_i$ and RO$_j$. One way to couple a pair of ROs is by connecting the outputs of two corresponding stages in each RO by a resistor; other coupling schemes may also be used~\cite{cilasun2024}. For resistive coupling, the coupling strength $J_{ij}$ is determined by the values of the resistors between inverters in RO$_i$ and RO$_j$; Section~\ref{sec:A2A} will describe a circuit that implements multiple programmable $J_{ij}$ values.  

We refer to inverters that are at an even-parity [odd-parity] distance from the reference stage as even [odd] stages, shown in blue [red] in Fig.~\ref{fig:RO_coupling}. Coupling between two same-parity stages in different ROs is referred to as {\em positive coupling} (RO$_0$ and RO$_1$ in Fig.~\ref{fig:RO_coupling}), while coupling between opposite-parity stages is termed {\em negative coupling} (RO$_1$ and RO$_2$ in Fig.~\ref{fig:RO_coupling}). Positive coupling encourages the ROs to be in-phase, while negative coupling encourages the ROs to be out-of-phase.
In Fig.~\ref{fig:RO_coupling}, the net $a$ driven by stage RO$_0^1$ is coupled to net $b$ at the output of RO$_1^1$ by a resistor; this causes the stage delays to change from their nominal values:
\begin{itemize}[noitemsep,topsep=-1pt,leftmargin=*]
\item 
When $a$ rises, if $b$ is low (i.e., yet to rise), it opposes the rise transition and the delay of $a$ is increased by $\delta_1^a$ relative to the uncoupled case.  On the other hand, if $b$ is high (i.e., it has already risen), it aids the rise and reduces the delay of $a$ by $\delta_2^a$.  In both cases, the rising edge of $a$ is brought closer to the rising edge of $b$, reducing the phase difference between the signals. 
\item
Similarly, when $a$ falls, if $b$ is high (i.e., yet to fall), the delay is increased by $\delta_3^a$; if $b$ is low (i.e., has already fallen), its delay reduces by $\delta_4^a$. In each case, the falling edges of $a$ and $b$ are brought closer, bringing the oscillators closer to phase-locking.
\end{itemize} 
Net $b$ behaves analogously, with delay shifts of $\delta_1^b, \cdots , \delta_4^b$.   

\begin{figure}[htb]
\centering
\subfigure[]{\includegraphics[width=0.45\textwidth]{img/coupled_period_waveform_v2.pdf}
\label{fig:coupled_waveform}}

\vspace{-4mm}
\subfigure[]{\includegraphics[width=0.35\textwidth]{img/effect_of_coupling_v2.pdf}
\label{fig:effect_of_coupling}}
\vspace{-3mm}
\caption{(a)~Waveforms for nets $a$ and $b$ in two ROs, when uncoupled, and with coupling enabled at $t=0$. (b)~Detailed view of the green box in (a), showing the effect of coupling. The period of waveform at $a$ increases while that at $b$ decreases, reducing the phase difference.}
\label{fig_sim}
\vspace{-6mm}
\end{figure}

Consider the coupled oscillator system of Fig.~\ref{fig:RO_coupling}, but with the coupling between RO$_1$ and RO$_2$ removed, i.e., only RO$_0^1$ and RO$_1^1$ are coupled.  Fig.~\ref{fig:coupled_waveform} shows the effect of coupling over multiple cycles, where the waveforms at $a$ and $b$ begin with a phase difference, $\phi_{12}$. In the uncoupled case, this phase difference is unchanged, but under coupling, as $a$ slows down and $b$ speeds up, the phase difference decreases and the edges align. The inset in Fig.~\ref{fig:effect_of_coupling} shows the first cycle, i.e., the green region in Fig.~\ref{fig:coupled_waveform}. Due to coupling, the first rising edge of $a$ is delayed by $\delta_1^a$ while that of $b$ arrives earlier by $\delta_2^b$.  Similarly, the falling edge at $a$ is delayed by an additional $\delta_3^a$ while that at $b$ is sped up by an additional $\delta_4^b$. Thus, the phase difference is reduced by $(\delta_1^a + \delta_2^b + \delta_3^a + \delta_4^b)$ after the first cycle. Subsequently, as long as transitions in $a$ are completed before those in $b$, $a$ continues to be delayed by $\delta_1^a+\delta_3^a$ and $b$ continues to speed up by $\delta_2^b+\delta_4^b$ every cycle, and the phase difference at the end of $k$ cycles reduces to $\phi_{12} - k(\delta_1^a + \delta_2^b + \delta_3^a + \delta_4^b)$ until the phases align.

\noindent
\underline{Coupled oscillator systems.}
We abstract a general coupled oscillator system with a graph $G=(V,E)$, where the vertex set $V$ is the set of oscillators, and each element $e_{ij}$ of the edge set $E$ corresponds to a pair of oscillators RO$_i$ and RO$_j$ with an edge weight corresponding to the coupling strength $J_{ij}$. We denote the change in delay of the coupled stage of RO$_i$ in each cycle, under a phase difference of $\phi_{ij}$, by $f_{J_{ij}}(\phi_{ij})$; for the example in Fig.~\ref{fig:effect_of_coupling}, $f_{J_{12}}(\phi_{12}) = (\delta_1^a + \delta_3^a)$ and $f_{J_{12}}(\phi_{21}) = (\delta_2^b + \delta_4^b)$. The net change in the period of RO$_i$ in the $k^{\rm th}$ cycle, $\mathcal{D}_i^k$, is the sum of changes in the delay of each stage, i.e., $\mathcal{D}_i^k = \textstyle \sum_{(i,j) \in E} f_{J_{ij}}(\phase{ij}{k})$.

\noindent
\underline{Synchronization.} A pair of coupled oscillators that have the same period will have a constant phase difference. Conversely, when all phase differences are constant, it implies that all pairs of coupled oscillators have the same period, a phenomenon referred to as \emph{synchronization}. In Fig.~\ref{fig:coupled_waveform}, since the coupling is purely positive, all phases align upon synchronization. As a result, all stage delays go back to their nominal uncoupled values and the period reverts to the nominal period, $T$.

\subsection{Practical considerations for CMOS-based coupled oscillator systems}
\label{sec:practical_considerations}

\noindent
\underline{Delay change as a function of coupling location}
At a given coupling location between two ROs, the delays from the reference stage of each RO to their coupling stages may be different. As a result, the phase difference at the coupling site may be different from the phase difference of the reference ROs. For example, in Fig.~\ref{fig:RO_coupling}, the path from the reference stage to the coupling site for $J_{12}$ involves two inverter delays in RO$_1$, but three in RO$_2$.  Therefore, the phase difference at this coupling site is not the same as $\phi_{12}$, the phase difference between their reference stages. Prior simulators have not considered this issue. 

Moreover, the stage delay depends on the presence or absence of coupling: ROs with more couplings may have different delays to a coupling site than ROs with fewer couplings. Thus, as the phase changes along an RO, delay shifts from earlier stages affect the phase difference at later stages, and using an identical $\phi_{i}$ at all coupling stages leads to inaccuracies. Many events within a cycle interact subtly to produce a total delay shift for each RO. Such interactions within a cycle are considered in our framework.

\begin{figure}[htb]
    \centering
    \includegraphics[width=0.95\linewidth]{img/delay_variation.pdf} 
    \caption{The relative arrival times on nets $a$ (green waveform) and $b$ (blue waveform), top, impact the transition delay on net $c$ (middle). The transition delay as a function of phase difference, and the interaction window $W$ are illustrated at the bottom.}
    \label{fig:transitions}
\end{figure}

\noindent
\underline{Delay change as a function of arrival time difference}
The speedup or slowdown in stage delay due to coupling depends on the arrival time differences between the signals on the coupled nets. Fig.~\ref{fig:transitions} illustrates this trend at five different relative signal arrival times, (P1, $\cdots$, P5), on nets $a$ (blue) and $b$ (green) in Fig.~\ref{fig:RO_coupling}. Near P1 and P5, the difference in arrival times is large enough that the transition on $a$ does not overlap with a transition on $b$, and the change in delays is constant as $a$ is effectively stable throughout the rise of $b$. At P2 and P4, when the edges are closer, the opposing or assisting 
transistor in the other RO is no longer completely on and it sees a reduced gate-source voltage, which reduces its effect on the delay. If two rising edges with identical transition times are exactly aligned, at P3, no current flows through the resistor, and the transitions do not affect each other. Thus, we see that there is a window around each edge beyond which the arrival of another edge causes a constant stage delay shift, but within this window, the delay shift is a function of the phase difference. This window of width $W$ is called the \emph{interaction window} and extends on either side of a transition as seen by the highlighted orange box in Fig.~\ref{fig:transitions}.

The delay at the output of a stage is also a function of the transition time of the signal at its input~\cite{Sapatnekar04}.  The transition time at net $b$ is also seen to show a similar trend as shown in the bottom panel of Fig.~\ref{fig:transitions}, which has an effect on the delays of subsequent stages.

\section{A silicon-proven all-to-all coupled-RO Ising machine}
\label{sec:A2A}

\begin{figure}[htb]
    \centering
    \includegraphics[width=0.55\textwidth]{img/array_simplified_ports.pdf}
    \vspace{-6mm}
    \caption{An illustration of the A2A concept through a small three-RO structure, showing shorting cells $S$ on the diagonal, programmable off-diagonal coupling cells $C$, and enable cells $EN$.}
    \label{fig:array_schm}
    \vspace{-4mm}
\end{figure}

\noindent
The design of a coupled-RO Ising machine is particularly tricky because of the need to ensure the uniformity of coupling weights between each pair of ROs. The shift in each transition time in the RO depends on RC parasitics associated with the coupling mechanism, which depend on the precise layout. It has been shown that through regular matched layouts, an all-to-all (A2A) coupled RO array with uniform coupling coefficients can be designed~\cite{Lo2023}.  This A2A design is silicon-proven and will be used in our evaluations.  Note that other designs with planar (hexagonal~\cite{Ahmed2021} and King's graph~\cite{moy20221}) coupling have been proposed, but we focus on the A2A testcase because of its compactness and greater flexibility: in particular, an A2A design with $N$ coupled ROs is equivalent to a planar hexagonal/King's graph array with $\sim N^2$ coupled ROs~\cite{Tabi21, Lucas2019} due to the need for planar arrays to replicate spins during minor embedding~\cite{Lo2023}. This family of A2A arrays has been applied to solve problems ranging from max-cut~\cite{Lo2023} to maximal independent set~\cite{cilasun2024} to satisfiability~\cite{cilasun20243sat}.

For illustration, a simplified schematic, with $J_{ij} \in \{-1,0,+1\}$, for a three-RO system is presented in Fig.~\ref{fig:array_schm}. Each RO is a combination of a vertical and a horizontal RO that are strongly coupled so as to implement the same spin, and have \textit{enable cells} $EN$ (similar to $RO_i^0$ in Fig.~\ref{fig:RO_coupling}) outside the array. Strong couplings are implemented as shorts (upper inset) in the \textit{shorting cells} $S$, placed at the diagonals $(i,i)$. Each \textit{coupling cell} $C$ at off-diagonal location $(i,j)$ has programmable coupling between RO$_i$ and RO$_j$.  A coupling cell has two switches (lower inset) to enable either a positive or a negative coupling. The simplified figure shows possible couplings of $\{-1, 0, +1\}$, but coupling of $\pm 7$ is demonstrated in silicon~\cite{Lo2023}.  Using couplings at both $(i,j)$ and $(j,i)$, each programmable to integer values up to $\pm C_{max}$, the coupling coefficient of $s_i s_j$ can implement integer $J_{ij} \in [-2C_{max}, +2C_{max}]$. 

\vspace{-4mm}