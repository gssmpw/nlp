Finally, we investigate the impact of different languages and scripts on the effectiveness of our tasks.
Our main study featured English-centric URLs in the Latin script---the most common script for URLs.
We ask ourselves: do phishing susceptibility and the usefulness of our tasks differ for non-native English speakers and for non-native Latin script readers?
Is reviewing URLs in a different script more challenging, and do the interactions we propose help users more or less in these cases?
To answer these questions, we had our study fully localized by native speakers, with the exception of the URLs, and ran it on 300 German participants (non-native English speakers but native Latin script readers) and 300 Japanese participants (non-native English speakers and non-native Latin script readers).
We picked these large countries as they have access to the same Internet services as the initial U.S.-based study (e.g., Google, PayPal) and have good availability of crowdsourcing platforms: Prolific for German speakers and Lancers for Japanese speakers.
We now present in the following a comparison with the U.S. study.

\begin{table}[t]
	\renewcommand{\arraystretch}{1.1}
	\footnotesize
	\centering
	\caption{\textbf{Different scripts:} phishing rates per group.}
	\label{tab:performance_languages}
	\rowcolors{2}{white}{gray!10}
	\begin{tabularx}{\linewidth}{@{}l|XXX@{}}
		\toprule
		& U.S. & Germany & Japan \\
		\midrule
		\textbf{Control} & 74.5\% & 53.2\% & 82.2\% \\
		\textbf{Passive} & 57.2\% & 38.4\% & 60.7\% \\
		\textbf{Active} & 61.0\% & 43.4\% & 67.3\% \\
		\textbf{Inspection Tasks} & 35.0\% & 25.9\% & 57.1\% \\
		\bottomrule
	\end{tabularx}
\end{table}

\paragraph{Demographics}
We briefly report on the demographics of the German and Japanese participants.
The German participants were similarly distributed to the U.S. participants in terms of all recorded variables---importantly, age, education, reported level of technology usage, and phishing awareness.
The Japanese participants instead had a very different age distribution: participants aged 31-40 were 27\%, 41-50 were 40.5\%, and 51-60 were 19.1\%---a significantly older population than the U.S. and German studies.
They also had more varied (and lower) reported levels of technology usage.

\paragraph{Study performance}
We report the high-level results for the German and Japanese studies in Table~\ref{tab:performance_languages}.
For German participants, we observed an overall lower phishing susceptibility compared to the U.S. and Japanese participants, including, most notably, the control group. 
However, the results are in line with the U.S. study: while both baselines helped participants, the proposed mechanism was the most effective, halving the phishing rate of the control group.
Results for the Japanese participants are more nuanced: while their control group and baselines exhibit slightly more susceptibility than their U.S. counterparts, our inspection tasks were less effective, with marginal improvements over the baselines.

\begin{table}[t]
	\renewcommand{\arraystretch}{1.1}
	\footnotesize
	\centering
	\caption{\textbf{Different scripts:} phishing rates per URL type.}
	\label{tab:performance_language_url}
	\rowcolors{2}{white}{gray!10}
	\begin{tabularx}{\linewidth}{@{}r|XXXXX@{}}
		\toprule
		& \textsc{sub} & \textsc{first} & \textsc{last} & \textsc{path} & \textsc{squat} \\ \midrule

        % \textbf{Germany} & & & & & \\
        \textbf{Germany} - Click & 26.3\% & 28.8\% & 24.2\% & 29.8\% & - \\
        Highlight & 25.7\% & - & 38.6\% & 36.2\% & - \\
        Type & 21.2\% & 35.3\% & 20.0\% & 44.4\% & 7.7\% \\ \midrule

        % \textbf{Japan} & & & & & \\
        \textbf{Japan} - Click & 48.7\% & 71.4\% & 76.7\% & 80.8\% & - \\
        Highlight & 53.8\% & - & 52.8\% & 51.9\% & - \\
        Type & 57.7\% & 72.5\% & 53.3\% & 52.8\% & 41.3\% \\

		\bottomrule
	\end{tabularx}
\end{table}

\paragraph{Task performance}
We report the performance of the inspection tasks per URL type for the German and Japanese studies in Table~\ref{tab:performance_language_url}.
This data gives us further insights into the poor performance of the Japanese participants: we observe that the clicking task was highly ineffective for impersonation at the beginning and end of the domain.
We reflect that in these cases, the task proposed both the legitimate domain (e.g., \texttt{example.com}) and the phishing one (\texttt{example-login.com})---it is possible that these participants interpreted the task with excessive compliance and selected the (correct) phishing domain instead of reporting it.
This suspicion is further supported by the high phishing rates on the typing task for impersonations at the beginning of the domain---one that requires also pre-existing knowledge of the correct domain to be solved correctly, as discussed in Section~\ref{sec:design.tasks}.
One further observation is the demographic imbalance of the Japanese study, featuring older participants.
\new{While in our US experiment we did not observe correlations between phishing falling and age, older users derived limited benefits from our mechanisms (see Section~\ref{sec:results.demographics}) and are known in the literature to be more susceptible to phishing~\cite{sheng2010falls}.}
Overall, our results suggest that the effectiveness of our mechanism is influenced by familiarity with the script of the URLs; further, they highlight the importance of wording and framing in the design of these types of countermeasures~\cite{franz2021sok,bauer2013warning}.