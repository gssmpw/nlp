In this paper, we presented \textit{URL inspection tasks}, a novel approach to help users detect phishing URLs in emails.
\new{Our active approach, recommended as a sporadic countermeasure in more sensitive environments such as corporate settings, reduced the victimization rate of participants in the study from 75\% to 25\% and providing strong protection against hard-to-spot typosquatting URLs.}
The effectiveness of our approach comes from a design that follows the guidelines and best practices in warning and security UIs~\cite{schaub2015design,franz2021sok,bauer2013warning} employing contextual, active tasks that help users pay attention, combined with the intention verification aspect.

Our results also offer insights into why users are susceptible to phishing URLs.
The difference in victimization rates between the control group, which includes users with a standard browser-based email client, and the group that interacted with our tasks, highlights the need for better presentation.
This indicates that URLs should be displayed more prominently, as participants often recognized deception while completing the tasks. 
Additionally, we show that up to 50\% of the participants who were initially unsuccessful in solving the tasks were aided by the notification of intention mismatch, demonstrating the need for better education regarding URL structures.
Finally, our design and study highlight that there still exists a gap between technical indicators and users' intentions (e.g., the domain \texttt{example.com} and whether it identifies the intended ``Example'' online service) that needs to be investigated further.
One possible direction is to explore whether our tasks can be simplified and abstracted away from technical indicators and whether this approach might impact security.

Potential future research directions are investigating more user-friendly interventions and better URL education methods; further examining non-native Latin script readers; and adapting our tasks to mobile platforms.