Phishing is a widespread problem, with attackers using increasingly sophisticated techniques to deceive users~\cite{lin2022phish}  
-- a situation only exacerbated by the COVID-19 pandemic with its shift to remote work and digital communication~\cite{al2022covid}, 
as well as by increasingly accessible and sophisticated AI tools~\cite{mink2022deepphish} that can
generate highly realistic, yet deceptive, content.

A common goal of phishing attacks is to deliver a \textit{payload} to its victim, usually malicious attachments or URLs pointing
to malicious content (e.g., websites that harvest credentials or install malware) via email~\cite{cofense2023}. 
In the case of malicious attachments, many technical and user-interface countermeasures have been widely 
studied~\cite{aslan2020comprehensive,bravo2010bridging} and deployed. This is further aided by better user education: 
informed users learn to avoid opening attachments from unknown sources~\cite{vishwanath2018suspicion}.

However, for URLs, the situation is different: technical countermeasures (e.g., URL blacklists, machine 
learning techniques) lag behind attackers' increasing sophistication~\cite{oest2020phishtime,sheng2009empirical}.
Mainstream user interfaces offer little help to users besides showing the full URL in address bars and small tool-tips.
Unsurprisingly, URLs are currently the main vector for phishing attacks, more so than attachments~\cite{cofense2023}, 
for the purposes of harvesting credentials and malware delivery~\cite{cofense2023}. It seems that users struggle with 
the current state-of-the-art anti-phishing methods which fail to support their decision-making in: (i) paying attention 
to the URL which they click; (ii) understanding its structure (e.g., what is the domain-name component and what it means); 
and (iii) deciding whether the clicked URL will take them to the website they expect. Recent studies show that 
user inattention is among the main contributors to the success of phishing attacks~\cite{greene2018user,vishwanath2011people}.

Motivated by this, we design and evaluate several \textit{URL inspection tasks}: small challenges 
served to users (when they click on links contained in emails) that must be solved correctly
before they can continue to their (intended) destination.
Solving these challenges requires interaction with the URL: they focus users' attention on the URL they are about to 
visit and require a basic understanding of its structure to be solved. They also help users check if the URL they are 
about to visit matches their intent by (indirectly) making them solve the challenge incorrectly in case of a misunderstanding.
Since solving these challenges requires users to determine where {\em they think they are going}, those who are confused 
by the common impersonation tactics of phishing URLs (e.g., containing the name of a reputable domain to inspire trustworthiness) 
answer incorrectly, thus triggering a warning.

We implemented three types of inspection tasks using three basic HCI mechanisms: \textit{click-selection} among a list of 
candidate URLs, 
\textit{highlighting} the domain by selecting it, and \textit{re-typing} the domain that the user thinks they want to visit.
We evaluated these tasks in a large (\numTotalParticipants{} participants) on-line study, designed as a realistic role-play 
experiment wherein participants pretend to be employees of a fictitious company who routinely manage email in a custom mailbox.
Our results show that inspection tasks prevented participants from falling for phishing attacks, 
compared to a control group that reflected the typical current experience of Internet users, as the rate of successful 
phishing emails fell from 74\% to 35\%.

The studied tasks also outperformed a passive baseline (57\% phishing success rate) where the URL is shown again though
the user only has to confirm their intention to proceed. This testifies to the effectiveness and importance of 
active engagement with the task and its prevention of habituation. 
They also outperformed a \textit{semi-active} baseline (61\% phishing success rate) where participants drag-and-drop parts of the 
URL back in place (thus presenting an active task component) and can only be solved correctly. This approach does not help users 
understand whether the destination matches their intent, while our results demonstrate the importance of this last step.
The difficulty of detecting different types of phishing URLs varies: while our tasks outperformed the baselines for all types 
of URLs that impersonate the victim's domain, they were especially effective against typo-squat URLs 
(that participants struggled with otherwise), decreasing the successful phishing rate from 79\% to 17\%.

\new{Active approaches such as ours should be used sporadically, similar to how CAPTCHAs are used today, and are recommended in scenarios requiring higher security, such as corporate environments.
Indeed, this approach trades off increased vigilance and detection against a moderate increase in user burden and false positives. Our study also aims to understand this tradeoff: our tasks slowed users by 7-10 seconds on average and somewhat increased annoyance compared to a regular email workflow, but provided a higher level of protection.}


The contributions of this work are:
\begin{compactitem}
    \item The concept of \textit{inspection tasks} upon clicking on links in emails to help focus users' attention, 
    and verify whether the URL they are about to visit matches their intent.
    \item Assessment of three types of inspection tasks grounded in basic HCI mechanisms: clicking, highlighting, and re-typing. 
    We tested them on a wide range of phishing URLs as part of a large on-line study with \numTotalParticipants{} participants 
    from the United States, Germany, and Japan.
    \item Results of the study show significant improvement in lowering the fraction of users who succumb to phishing attacks.
    The tasks are especially effective against sophisticated typo-squatting URLs. This effectiveness is due to both (1) 
    active user engagement with the task and (2) helping users check whether a given URL matches their intent.
\end{compactitem}