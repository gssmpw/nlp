\paragraph{Motivation}
% 
Phishing by email generally uses two attack vectors: URLs and attachments.
One major reason why URL-based email phishing succeeds is due to the difficulty of parsing the complicated 
structure of embedded links by users~\cite{albakry2020url}, and inattentiveness. This is especially the case when malicious 
URLs impersonate legitimate services~\cite{reynolds2020measuring}.
Attachment-based phishing is currently less effective since many modern email clients, browsers, and OSes 
implement defense mechanisms, e.g., via blocking downloads or explicit warnings. Also, 
users have become increasingly aware of the perils of opening unknown attachments~\cite{vishwanath2018suspicion}.

It is thus surprising that for URLs, users are left on their own: both standalone and browser-based 
email clients do not provide much help besides showing the destination of links on small tooltips upon mouse hover. Browsers help users by highlighting URLs in the address bar or hiding their path. However, these countermeasures do not seem to assuage users' 
struggles~\cite{lin2011does,xiong2017domain,dhamija2006phishing}.
Furthermore, users who already made up their minds about a given email~\cite{lain2021phishing} 
might ignore the URL when it is displayed in the browser~\cite{lin2011does}.

Therefore, research has focused on improving email clients and browsers by helping users 
understand links and URLs. This has been done by providing tooltips with information about the 
URLs~\cite{althobaiti2021don}, introducing delays before opening the link~\cite{volkamer2017user}, 
or forcing users to click on it again~\cite{petelka2019put}. Another popular countermeasure employed by 
many online services is a warning page displayed upon clicking on a URL and asking the user to 
confirm that they wish to visit it. 

\begin{figure*}[t]
    \centering
    \includegraphics[width=.8\textwidth]{overview.pdf}
    \caption{\textbf{Overview of active tasks for URL inspection.} Upon clicking on a link in an 
    email, the user is presented with a task to be solved on the clicked URL, forcing attention and helping to understand where it is taking them.}
    \label{fig:overview}
\end{figure*}

\paragraph{Limitations of Prior Approaches}
We focus on a concrete class of phishing attempts: consider an email containing at least one URL that 
impersonates (resembles) a legitimate website such as
\texttt{example.com.scam.com}, hosted on the attacker-owned \texttt{scam.com} domain, and attempting to impersonate the legitimate website \texttt{example.com}
This is a popular form of URL impersonation~\cite{reynolds2020measuring,zeng2021winding} that aims to deceive 
the user into thinking the bogus URL leads to a legitimate website.
To make an informed decision about the legitimacy of this URL, several things need to happen:

First, a user must take the time to pay attention instead of simply clicking it, which means visually parsing
the URL. An artificial slow-down of user interaction~\cite{volkamer2017user,petelka2019put} here makes sense, since phishing susceptibility is 
often based on quick decisions~\cite{purkait2014empirical}. 

Second, a user needs to {\bf understand} that the URL leads to \texttt{scam.com}. However, just providing additional 
contextual information~\cite{volkamer2017user,nicholson2017can} can be easily ignored~\cite{neupane2015multi}.
Meanwhile, an additional cognitive effort imposed on users makes them less vulnerable to 
phishing~\cite{wang2012research,purkait2014empirical}. 

Third, a user needs help understanding that clicking will NOT lead to the expected website \texttt{example.com}.

\new{Finally, the user needs to know that \texttt{example.com} is the correct domain of their desired service ``Example'', and that they wanted to visit this website instead.}

\subsection{Overview}
Ideally, an effective anti-phishing technique must employ best practices of security interface design: 
(1) prevent habituation and desensitization~\cite{krol2012don}, and (2) require user 
interaction~\cite{bravo2013your} while (3) providing actionable information to help users
make a decision~\cite{li2007usability,schaub2015design,bauer2013warning}. Prior techniques
do not satisfy this.

Our work uses all three aforementioned elements. It involves active challenges that need to be solved 
by interacting with the URL, thus alerting users and directing their attention. It
requires basic understanding of the URL structure to make users understand where a URL would
lead if they were to click it, thus helping users in making a informed decision.
%
Furthermore, challenges can be designed to (indirectly) help users answer the question \textit{``Where would this URL take you?''}
and notify them in case there is a mismatch between their stated intention and the URL. 
In our example above, a challenge would result in the answer: \texttt{example.com}, 
and warn the user that the URL would in fact bring them to \texttt{scam.com}.

We overview our approach in Figure~\ref{fig:overview}: upon clicking on a link in an email 
(denoted as \one), the user is immediately presented with an attention-enhancing task (denoted as 
\two) that motivates them to inspect and understand the URL, e.g., in a tooltip or a page 
on their browser. There, the URL is presented in a way that is easy to read and understand~\cite{franz2021sok}.
The user has to solve the task correctly in order to proceed to website \four. If 
they make a mistake, an error is shown (\three) by, e.g., presenting both 
the original domain and the user's answer. The user is then asked to confirm whether 
they want to proceed.

\new{Note that our approach alone does not help with user knowledge of the domain for any expected service: we discuss the implications of this gap further in Section~\ref{sec:discussion.approach}.}