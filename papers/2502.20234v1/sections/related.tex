\paragraph{Design of warnings and security UIs}
Warning design is an active area of research, both for physical products~\cite{wogalter2002based} and digital interfaces~\cite{franz2021sok,schaub2015design}.
Design principles for this special type of communication derive from theoretical models of human communication and information processing~\cite{wogalter2018communication}, mental models~\cite{bravo2010bridging}, or from empirical studies of how users interact with warnings, e.g., for SSL warnings in web browsers~\cite{akhawe2013alice,sunshine2009crying,reeder2018experience}, or of privacy notices~\cite{mcdonald2009comparative}.
This research lead to the creation of guidelines for security warnings and UIs~\cite{schaub2015design,franz2021sok,bauer2013warning}.
Effective warnings should be salient~\cite{egelman2008you,reeder2018experience}, concise and accurate~\cite{aneke2019designing,bauer2013warning}, contextual to what triggered them~\cite{petelka2019put}, and attract attention both through design elements~\cite{franz2021sok} and through requiring user interaction to proceed~\cite{felt2015improving,akhawe2013alice,aneke2019designing,bravo2013your}, as users otherwise tend to spend little time on security-relevant indicators~\cite{harrison2016individual,neupane2015multi}.
Habituation and desensitization due to excessive exposure and predictability of the warnings are also a major concern to address~\cite{sunshine2009crying,krol2012don}.

\paragraph{Teaching URLs to users}
Users are generally not very proficient at parsing modern URLs~\cite{albakry2020url,reynolds2020measuring}.
This is especially true for obfuscated and long URLs, where users struggle to understand their structure, and for URLs that impersonate familiar brands by placing their names in some parts of it~\cite{reynolds2020measuring}.
Therefore, several works have proposed tools and games to teach users how URLs are composed and to recognize phishing URLs~\cite{althobaiti2018faheem,canova2015learn,kumaraguru2010teaching}, uncovering features that are most helpful to users~\cite{althobaiti2021don}.
These proposals leverage presentation elements to explain how to divide a URL into its parts~\cite{althobaiti2018faheem,canova2015learn}, as users of all levels of technical proficiency otherwise struggle with reading URLs without help~\cite{albakry2020url}.
They also focus on providing tips and heuristics to recognize phishing URLs~\cite{kumaraguru2010teaching}, and use gamification to make the learning process more engaging~\cite{kumaraguru2010teaching,canova2015learn}.
The main drawback of these support UIs is that they struggle to give users transferable knowledge, as performance can drop after the UI is not available anymore~\cite{althobaiti2018faheem}.

\paragraph{Related UIs}
Domain highlighting is one of the main techniques used to help users understand URLs, by showing the domain part of the URL in a different color or font weight~\cite{volkamer2017user,lin2011does}; however, it is only effective for users with good technical knowledge~\cite{lin2011does}.
Another approach is augmenting existing interfaces to show more indicators, e.g., the sender's name and time of sending~\cite{nicholson2017can}, or the URL's age and popularity~\cite{althobaiti2021don}.
However, all these approaches are passive and thus easy to ignore~\cite{felt2015improving,dhamija2006phishing}; furthermore, increasing the amount of information in passive warnings does not improve phishing detection~\cite{lain2021phishing,zheng2022presenting}.
To help users focus on the URL, studies investigated inhibitive warnings by enforcing delays while a tooltip presents more information about the clicked link~\cite{volkamer2017user}, or requiring to click on it again ~\cite{petelka2019put}, but these can still be prone to habituation as users can click through these warnings without paying attention.
\new{Tasks similar to ours have been successfully explored in the context of untrusted applications~\cite{bravo2013your}, where users were required to retype the name of the application publisher to detect impersonation attacks.
Therefore, it is worth investigating whether these tasks translate to URLs, as they have richer semantics (e.g., components), phishers employ different types of deception, and users have different understanding and mental models.}
