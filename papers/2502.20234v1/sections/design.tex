We face several challenges in creating concrete actionable tasks.
First, we need to understand which tasks can help users and how, plus analyze inherent 
trade-offs between ease-of-use, solving speed, and effectiveness, similar to challenges faced by 
CAPTCHA mechanisms~\cite{searles2023empirical}. Second, it is unclear how to design tasks that 
help users understand URLs, especially phishing URLs, as well as how to understand user intent 
and trigger an error in case of a mismatch. To tackle these challenges, we begin by exploring  
the ecosystem of phishing URLs and then propose a set of appropriate tasks.

\subsection{Types of Phishing URLs}
\label{sec:design.url}
%
It is important to identify common phishing URL types, since tasks should help users understand 
the URL structure and (hopefully) capture their intentions by triggering an error in case of a 
misunderstanding. Following taxonomies from the 
literature~\cite{reynolds2020measuring,zeng2021winding,tupsamudre2019everything,aung2019survey,canova2015nophish}, 
we observe that there are two main families of phishing URLs: (i) URLs that have no relationship with what they 
are impersonating, e.g., the domain name refers to a compromised domain, a random name, or an IP address; and 
(ii) URLs that somehow refer to what they are impersonating. Type-(i) can be spotted by a user by simply re-reading 
the URL to realize that it does not correspond to their intent. However, type-(ii) is more deceiving since the URL
contains a literal, near-literal (e.g., a typosquat), or partial name of the domain is impersonates, making it more  
difficult to understand~\cite{albakry2020url}. We thus focus on type-(ii).

The impersonated domain can appear in different parts of a phishing URL~\cite{reynolds2020measuring,zeng2021winding,canova2015nophish}.
Suppose that an adversary wants to impersonate \textit{example.com}. The impersonation can occur in the 
following parts (actual domain is underlined):
%
\begin{compactitem}
    \item \textbf{Subdomains:} \texttt{example.\uline{com-login.com}}.
    \item \textbf{Beginning of Domain:} \texttt{\uline{example-login.com}}.
    \item \textbf{End of Domain:} \texttt{\uline{login-example.com}}.
    \item \textbf{In Path:} \texttt{\uline{login.com}/example.com}.
    \item \textbf{Typosquat:} \texttt{\uline{exampie.com}} that substitutes the character \texttt{l} with the similar-looking \texttt{i}.
\end{compactitem}
%
In the next section, we discuss the potential impact of each task over the different URL types.

\begin{figure*}[!t]
    \centering
    \begin{subfigure}{0.33\textwidth}
        \centering
        \includegraphics[width=\textwidth]{figures/click.png}
        \caption{Clicking Task.}
        \label{fig:tasks.clicking}
    \end{subfigure}
    \begin{subfigure}{0.3\textwidth}
        \centering
        \includegraphics[width=\textwidth]{figures/highlight3.png}
        \caption{Highlighting Task.}
        \label{fig:tasks.highlighting}
    \end{subfigure}
    \begin{subfigure}{0.3\textwidth}
        \centering
        \includegraphics[width=\textwidth]{figures/retype2.png}
        \caption{Typing Task.}
        \label{fig:tasks.typing}
    \end{subfigure}
    \caption{Three selected tasks for active URL inspection after clicking on a link.}
    \label{fig:tasks}
\end{figure*}

\subsection{URL Tasks}
\label{sec:design.tasks}

We selected three tasks based on three basic human-computer interaction (HCI) actions: clicking, 
dragging with the mouse, and typing. These are similar to most common CAPTCHA interactions~\cite{searles2023empirical}.
These tasks, shown in Figure~\ref{fig:tasks}, were designed so that performing them would force the user to re-read the URL and help them understand 
where it leads. This is in contrast with prior 
approaches~\cite{volkamer2017user,lin2011does,althobaiti2021don,petelka2019put}.
The tasks require the user to identify the domain portion of the URL. To do so, the user needs to (i) understand what a 
domain name is and how it identifies a specific website within a URL. However, as introduced, they also need to (ii) know the domain of the intended website: we discuss how users' knowledge affects the three selected tasks below; Section~\ref{sec:discussion} provides further details. 

\paragraph{Clicking Task}
Asking users to click on the domain itself~\cite{volkamer2017user} might lead to a simple form of habituation -- the domain would be 
presented roughly in the middle of a URL and users might click on it without paying attention.
Instead, our clicking task involves subdomains: we list the 
domain and subdomains in random order, \new{selected for example with heuristics on keywords}, and ask the user to click on the domain (Figure~\ref{fig:tasks.clicking}).

The main idea here is that this task should alert a user to a URL that contains a deceptive string in its subdomains or path.
In other words, a user would click on the domain they intend to visit. For example, given a choice between 
\texttt{example.\uline{com-login.com}} and \texttt{example.com} a user would click the latter.
This task can also help against deceptions within the domain itself, e.g., it can detect the presence of keywords within it with heuristics and propose the legitimate domain among the list.
Here, however, success would ultimately depend on a user's understanding that \texttt{example-login.com} is not the correct domain: we discuss this further in Section~\ref{sec:discussion}.
Finally, for typosquats, this task does not provide any specific help other than making a user re-read the URL.

\paragraph{Highlighting Task}
In this task, a user is asked to highlight the domain-name component of the URL (by clicking and dragging a mouse over it)
and then confirm by pressing a button, as shown in Figure~\ref{fig:tasks.highlighting}.

This task aims to capture a user's real intent for URLs that contain impersonations in subdomains, e.g., presented
with \texttt{example.\uline{com-login.com}}, a user would highlight \texttt{example.com} rather than \texttt{com-login.com}. 
It also aims to do the same for impersonations at the end of the domain (e.g., \texttt{\uline{login-example.com}}) 
and for ones contained in the URL fragment.
However, recognizing impersonations at the beginning of the domain requires a user to know that 
\texttt{\uline{example-login.com}} is not their intended (i.e., spoofed) domain, as they would (correctly, but without preventing the attack) highlight the whole domain otherwise.
Finally, for typosquats, this task helps users review the characters one at a time, potentially helping spot the deception.

Indeed, the modern URL structure presents a trade-off.
On the one hand, the relevant part of the URL for a given user might be the second-level subdomain, e.g., {\tt ``drive''} in 
\texttt{drive.google.com}, where the service name is embedded).
To accommodate this, the task must allow users to highlight subdomains.
On the other hand, this flexibility might lead users to highlight subdomains that contain impersonations 
of the service name, successfully completing the task while evading the mechanism's intended purpose.

\paragraph{Typing Task}
This task requires a user to re-type the domain in a text box, as shown in Figure~\ref{fig:tasks.typing}.
Its goal is to mitigate all types of impersonations since it allows the user to freely express their intent 
by entering the domain they intend to visit.
Simple techniques need to be employed to prevent users from copy-pasting the URL or dragging it into the textbox.

This task seems especially beneficial against subdomain and path 
impersonation. It is also effective against typosquats, since the user has to re-type the domain.
However, the same issue of knowledge of the correct domain remains for some types of URLs, e.g., when impersonations are at the beginning of the domain.
The main downside of this task is its user burden of having to type the (potentially long) domain character by character, 
This results in longer solving time, higher false positive rate, and increased user frustration.
\new{More advanced design, e.g., parsing natural language answers, could be considered to mitigate these issues.}


\subsubsection{Communicating Mistakes}

Different types of phishing URLs, tasks, and user errors require distinct feedback approaches.
One effective strategy is to alert users when the URL's domain does not match their response, asking if they still want to proceed.
Another approach is to provide more specific feedback for certain impersonations, such as \texttt{example-login.com}, by highlighting the discrepancy between the target URL's domain and the user's answer when showing the alert.
Meanwhile, for typing tasks, feedback might take the form of visually highlighting the difference between the user-typed URL and the actual URL to alert a user to typosquatting.
