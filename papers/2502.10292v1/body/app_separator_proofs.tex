

\subsection{Proof of \Cref{prop:singular_value}}\label{app:singular_value}

This follows from a standard argument in linear algebra.  Indeed, abusing notation, let $f, f'$ be a pair of functions in $\cF$ projected onto $\cZ$ and note that there exist basis vectors $u, u' \in \rr^{m}$ such that $f -f' = A (u - u')$.  By definition of the minimal singular value, it holds that
\begin{align}
    \norm{f -f'}_2 = \norm{A (u - u')}_2 \geq \sigma \cdot \norm{u - u'}_2 = \sigma \cdot \sqrt{2}.
\end{align} 
The result follows by renormalizing.







\subsection{Proof of \Cref{lem:separating_measure_sample}} 
\label{sec:separator_proofs} 

    
    In this section we state and prove a more general version of \Cref{lem:separating_measure_sample}, which applies to arbitrary real-valued function classes.  We first recall two more notions of function class complexity: fat-shattering dimension \citep{bartlett1994fat} and covering numbers.
    \begin{definition}
        Let $\cF: \cX \to \rr$ be a function class.  We say that $\cF$ is shattered at scale $\alpha > 0$ by points $z_1, \dots, z_m$ if there exists some witness $s \in \rr^m$ such that for all $\epsilon \in \left\{ \pm 1 \right\}^m$ there is an $f_\epsilon \in \cF$ satisfying $\epsilon_i (f_\epsilon(z_i) - s_i) \geq \alpha / 2$.  The fat-shattering dimension at scale $\alpha$, $\fat_\alpha(\cF)$ is the maximal number of points that can be shattered at scale $\alpha$.
    \end{definition}

    \begin{definition}
        Let $\cF: \cX \to \rr$ be a function class.  The $\epsilon$-covering number of $\cF$ with respect to a measure $\mu$, denoted by $N(\epsilon, \cF, \mu)$ is the smallest cardinality of $\mathcal{F'} \subseteq \mathcal{F} $ such that for all $f \in \mathcal{F}$, there exists a $f' \in \mathcal{F}' $ such that $ \norm{ f - f' }_{L^2(\mu)} <  \epsilon $.
    \end{definition}


    The following fundamental result controls covering numbers by fat-shattering dimension.
    \begin{theorem}[\citet{mendelson2003entropy}]\label{thm:mendelson}
        Let $\cF: \cX \to \rr$ be a function class.  There exist universal constants $c, C > 0$ such that for all $\epsilon$ and all $\mu$, it holds that
        \begin{align}
            N(\epsilon, \cF, \mu) \leq \left( \frac 2\epsilon \right)^{C\cdot \fat_{c \epsilon}(\cF)}.
        \end{align}
    \end{theorem}
    In the special case where $\cF$ is binary-valued, $\fat_0(\cF)$ becomes the VC dimension and the classical Dudley extraction result \citep{dudley1969speed} follows.  We have the following result.

    \begin{lemma}
        Let $\cF: \cX \to \rr$ be a function class and suppose that $\mu$ is a $\rho$-separating measure.  Then there exist constants $c, C > 0$ such that
        \begin{align}
            \abs{\cF} \leq \left( \frac 2 \rho \right)^{C \cdot \fat_{c \rho}(\cF)}.
        \end{align}
        Moreover, there exists a $\rho$-separating set of size $m = \bigO\left( \fat_{c \rho}(\cF) / \rho^2 \right)$.  In particular, if $\cF$ is a binary-valued function class of VC dimension $d$, then $\abs{\cF} \leq \left( \frac{10} \rho \right)^{d}$ and there exists a separator set of size $m = \bigO\left( d / \rho^2 \right)$.
    \end{lemma}
    \begin{proof}
        We first note that if $\cF$ has a $\rho$-separating measure $\mu$, then $\abs{\cF} \leq N\left( \rho, \cF, \mu \right)$.  This is because if $\mathcal{F}'$ is a $\rho$ covering of $\mathcal{F}$, we must have $\mathcal{F} = \mathcal{F}'$. 
        To see this note that, from $\rho$ separation, the set $ \left\{ f \in  \mathcal{F} : \norm{f - f'}_{\mu} < \rho    \right\} = \{ f'\}    $. 
        Since, we have $\mathcal{F}'$ is a $\rho$-covering, by singletons, we must have $\mathcal{F} = \mathcal{F}'$.   The first statement follows by applying \Cref{thm:mendelson}. 
        
        
        On the other hand, by \Cref{lem:separator_set_2} and the probabilistic method, it holds that there exists some set of size $m = 2 \log( N( \rho , \mathcal{F} , \mu )  ) / \rho^{2}$ that $\rho$-separates $\cF$.  Again applying \Cref{thm:mendelson} gives the second statement.  The binary-valued $\cF$ case is proved similarly.
    \end{proof}





        

\begin{lemma}
    \label{lem:separator_set_2}
    Let $\cF$ be a function class such that $\mu$ is a $\rho$-separating measure for $\cF$.
    Then, with probability $1 - \delta$, a sample $S = \left\{ x_i \right\}$  of size $m = ( \log ( N( \rho , \mathcal{F} , \mu )  ) + \log(1/ \delta)) \rho^{-2}$ 
    \begin{align}
        \norm{f - g}_m^2 = \frac{1}{m} \sum_i (f(x_i) - g(x_i))^2  \geq \frac{\rho}{2}. 
    \end{align} 
    Further, this set is a separator set for $\cF$.
    \end{lemma}
    \begin{proof}
        The proof follows from standard uniform convergence arguments for finite classes \citep{shai}.
    \end{proof}
    
    We note here that in the presence of $\rho$-separation for $\mu$, we get uniform convergence in terms of the covering number for $\mu$ as opposed to the standard result in terms of the uniform convering numbers $ \sup_{\nu} N( \rho , \mathcal{F} , \nu )  $.   
    



  






    
    
    
    
    
    


    
    
   

   
