%% bare_jrnl.tex
%% V1.4b
%% 2015/08/26
%% by Michael Shell
%% see http://www.michaelshell.org/
%% for current contact information.
%%
%% This is a skeleton file demonstrating the use of IEEEtran.cls
%% (requires IEEEtran.cls version 1.8b or later) with an IEEE
%% journal paper.
%%
%% Support sites:
%% http://www.michaelshell.org/tex/ieeetran/
%% http://www.ctan.org/pkg/ieeetran
%% and
%% http://www.ieee.org/

%%*************************************************************************
%% Legal Notice:
%% This code is offered as-is without any warranty either expressed or
%% implied; without even the implied warranty of MERCHANTABILITY or
%% FITNESS FOR A PARTICULAR PURPOSE! 
%% User assumes all risk.
%% In no event shall the IEEE or any contributor to this code be liable for
%% any damages or losses, including, but not limited to, incidental,
%% consequential, or any other damages, resulting from the use or misuse
%% of any information contained here.
%%
%% All comments are the opinions of their respective authors and are not
%% necessarily endorsed by the IEEE.
%%
%% This work is distributed under the LaTeX Project Public License (LPPL)
%% ( http://www.latex-project.org/ ) version 1.3, and may be freely used,
%% distributed and modified. A copy of the LPPL, version 1.3, is included
%% in the base LaTeX documentation of all distributions of LaTeX released
%% 2003/12/01 or later.
%% Retain all contribution notices and credits.
%% ** Modified files should be clearly indicated as such, including  **
%% ** renaming them and changing author support contact information. **
%%*************************************************************************


% *** Authors should verify (and, if needed, correct) their LaTeX system  ***
% *** with the testflow diagnostic prior to trusting their LaTeX platform ***
% *** with production work. The IEEE's font choices and paper sizes can   ***
% *** trigger bugs that do not appear when using other class files.       ***                          ***
% The testflow support page is at:
% http://www.michaelshell.org/tex/testflow/



\documentclass[journal]{IEEEtran}
%
% If IEEEtran.cls has not been installed into the LaTeX system files,
% manually specify the path to it like:
% \documentclass[journal]{../sty/IEEEtran}


\usepackage{wasysym}        % For clock symbol
\usepackage{fontawesome5}    % For gear and scales symbols
\usepackage{amssymb}         % For diamond symbol
\usepackage{pifont}          % For lightning symbol

\usepackage{subcaption} 
% Some very useful LaTeX packages include:
% (uncomment the ones you want to load)
%\setlength{\abovecaptionskip}{0pt} % Space above caption
%\setlength{\belowcaptionskip}{0pt} % Space below caption

\setlength{\floatsep}{0pt}     % Space between floats
\setlength{\textfloatsep}{0pt} % Space between text and floats
\setlength{\intextsep}{0pt}    % Space between in-text floats
\setlength{\dbltextfloatsep}{0pt} % Space for two-column layout
\setlength{\dblfloatsep}{0pt}     % Space between two-column floats

% *** MISC UTILITY PACKAGES ***
%
%\usepackage{ifpdf}
% Heiko Oberdiek's ifpdf.sty is very useful if you need conditional
% compilation based on whether the output is pdf or dvi.
% usage:
% \ifpdf
%   % pdf code
% \else
%   % dvi code
% \fi
% The latest version of ifpdf.sty can be obtained from:
% http://www.ctan.org/pkg/ifpdf
% Also, note that IEEEtran.cls V1.7 and later provides a builtin
% \ifCLASSINFOpdf conditional that works the same way.
% When switching from latex to pdflatex and vice-versa, the compiler may
% have to be run twice to clear warning/error messages.






% *** CITATION PACKAGES ***
%
%\usepackage{cite}
% cite.sty was written by Donald Arseneau
% V1.6 and later of IEEEtran pre-defines the format of the cite.sty package
% \cite{} output to follow that of the IEEE. Loading the cite package will
% result in citation numbers being automatically sorted and properly
% "compressed/ranged". e.g., [1], [9], [2], [7], [5], [6] without using
% cite.sty will become [1], [2], [5]--[7], [9] using cite.sty. cite.sty's
% \cite will automatically add leading space, if needed. Use cite.sty's
% noadjust option (cite.sty V3.8 and later) if you want to turn this off
% such as if a citation ever needs to be enclosed in parenthesis.
% cite.sty is already installed on most LaTeX systems. Be sure and use
% version 5.0 (2009-03-20) and later if using hyperref.sty.
% The latest version can be obtained at:
% http://www.ctan.org/pkg/cite
% The documentation is contained in the cite.sty file itself.






% *** GRAPHICS RELATED PACKAGES ***
%
\ifCLASSINFOpdf
  % \usepackage[pdftex]{graphicx}
  % declare the path(s) where your graphic files are
  % \graphicspath{{../pdf/}{../jpeg/}}
  % and their extensions so you won't have to specify these with
  % every instance of \includegraphics
  % \DeclareGraphicsExtensions{.pdf,.jpeg,.png}
\else
  % or other class option (dvipsone, dvipdf, if not using dvips). graphicx
  % will default to the driver specified in the system graphics.cfg if no
  % driver is specified.
  % \usepackage[dvips]{graphicx}
  % declare the path(s) where your graphic files are
  % \graphicspath{{../eps/}}
  % and their extensions so you won't have to specify these with
  % every instance of \includegraphics
  % \DeclareGraphicsExtensions{.eps}
\fi
% graphicx was written by David Carlisle and Sebastian Rahtz. It is
% required if you want graphics, photos, etc. graphicx.sty is already
% installed on most LaTeX systems. The latest version and documentation
% can be obtained at: 
% http://www.ctan.org/pkg/graphicx
% Another good source of documentation is "Using Imported Graphics in
% LaTeX2e" by Keith Reckdahl which can be found at:
% http://www.ctan.org/pkg/epslatex
%
% latex, and pdflatex in dvi mode, support graphics in encapsulated
% postscript (.eps) format. pdflatex in pdf mode supports graphics
% in .pdf, .jpeg, .png and .mps (metapost) formats. Users should ensure
% that all non-photo figures use a vector format (.eps, .pdf, .mps) and
% not a bitmapped formats (.jpeg, .png). The IEEE frowns on bitmapped formats
% which can result in "jaggedy"/blurry rendering of lines and letters as
% well as large increases in file sizes.
%
% You can find documentation about the pdfTeX application at:
% http://www.tug.org/applications/pdftex





% *** MATH PACKAGES ***
%
%\usepackage{amsmath}
% A popular package from the American Mathematical Society that provides
% many useful and powerful commands for dealing with mathematics.
%
% Note that the amsmath package sets \interdisplaylinepenalty to 10000
% thus preventing page breaks from occurring within multiline equations. Use:
%\interdisplaylinepenalty=2500
% after loading amsmath to restore such page breaks as IEEEtran.cls normally
% does. amsmath.sty is already installed on most LaTeX systems. The latest
% version and documentation can be obtained at:
% http://www.ctan.org/pkg/amsmath





% *** SPECIALIZED LIST PACKAGES ***
%
%\usepackage{algorithmic}
% algorithmic.sty was written by Peter Williams and Rogerio Brito.
% This package provides an algorithmic environment fo describing algorithms.
% You can use the algorithmic environment in-text or within a figure
% environment to provide for a floating algorithm. Do NOT use the algorithm
% floating environment provided by algorithm.sty (by the same authors) or
% algorithm2e.sty (by Christophe Fiorio) as the IEEE does not use dedicated
% algorithm float types and packages that provide these will not provide
% correct IEEE style captions. The latest version and documentation of
% algorithmic.sty can be obtained at:
% http://www.ctan.org/pkg/algorithms
% Also of interest may be the (relatively newer and more customizable)
% algorithmicx.sty package by Szasz Janos:
% http://www.ctan.org/pkg/algorithmicx




% *** ALIGNMENT PACKAGES ***
%
%\usepackage{array}
% Frank Mittelbach's and David Carlisle's array.sty patches and improves
% the standard LaTeX2e array and tabular environments to provide better
% appearance and additional user controls. As the default LaTeX2e table
% generation code is lacking to the point of almost being broken with
% respect to the quality of the end results, all users are strongly
% advised to use an enhanced (at the very least that provided by array.sty)
% set of table tools. array.sty is already installed on most systems. The
% latest version and documentation can be obtained at:
% http://www.ctan.org/pkg/array


% IEEEtran contains the IEEEeqnarray family of commands that can be used to
% generate multiline equations as well as matrices, tables, etc., of high
% quality.




% *** SUBFIGURE PACKAGES ***
%\ifCLASSOPTIONcompsoc
%  \usepackage[caption=false,font=normalsize,labelfont=sf,textfont=sf]{subfig}
%\else
%  \usepackage[caption=false,font=footnotesize]{subfig}
%\fi
% subfig.sty, written by Steven Douglas Cochran, is the modern replacement
% for subfigure.sty, the latter of which is no longer maintained and is
% incompatible with some LaTeX packages including fixltx2e. However,
% subfig.sty requires and automatically loads Axel Sommerfeldt's caption.sty
% which will override IEEEtran.cls' handling of captions and this will result
% in non-IEEE style figure/table captions. To prevent this problem, be sure
% and invoke subfig.sty's "caption=false" package option (available since
% subfig.sty version 1.3, 2005/06/28) as this is will preserve IEEEtran.cls
% handling of captions.
% Note that the Computer Society format requires a larger sans serif font
% than the serif footnote size font used in traditional IEEE formatting
% and thus the need to invoke different subfig.sty package options depending
% on whether compsoc mode has been enabled.
%
% The latest version and documentation of subfig.sty can be obtained at:
% http://www.ctan.org/pkg/subfig




% *** FLOAT PACKAGES ***
%
%\usepackage{fixltx2e}
% fixltx2e, the successor to the earlier fix2col.sty, was written by
% Frank Mittelbach and David Carlisle. This package corrects a few problems
% in the LaTeX2e kernel, the most notable of which is that in current
% LaTeX2e releases, the ordering of single and double column floats is not
% guaranteed to be preserved. Thus, an unpatched LaTeX2e can allow a
% single column figure to be placed prior to an earlier double column
% figure.
% Be aware that LaTeX2e kernels dated 2015 and later have fixltx2e.sty's
% corrections already built into the system in which case a warning will
% be issued if an attempt is made to load fixltx2e.sty as it is no longer
% needed.
% The latest version and documentation can be found at:
% http://www.ctan.org/pkg/fixltx2e


%\usepackage{stfloats}
% stfloats.sty was written by Sigitas Tolusis. This package gives LaTeX2e
% the ability to do double column floats at the bottom of the page as well
% as the top. (e.g., "\begin{figure*}[!b]" is not normally possible in
% LaTeX2e). It also provides a command:
%\fnbelowfloat
% to enable the placement of footnotes below bottom floats (the standard
% LaTeX2e kernel puts them above bottom floats). This is an invasive package
% which rewrites many portions of the LaTeX2e float routines. It may not work
% with other packages that modify the LaTeX2e float routines. The latest
% version and documentation can be obtained at:
% http://www.ctan.org/pkg/stfloats
% Do not use the stfloats baselinefloat ability as the IEEE does not allow
% \baselineskip to stretch. Authors submitting work to the IEEE should note
% that the IEEE rarely uses double column equations and that authors should try
% to avoid such use. Do not be tempted to use the cuted.sty or midfloat.sty
% packages (also by Sigitas Tolusis) as the IEEE does not format its papers in
% such ways.
% Do not attempt to use stfloats with fixltx2e as they are incompatible.
% Instead, use Morten Hogholm'a dblfloatfix which combines the features
% of both fixltx2e and stfloats:
%
% \usepackage{dblfloatfix}
% The latest version can be found at:
% http://www.ctan.org/pkg/dblfloatfix




%\ifCLASSOPTIONcaptionsoff
%  \usepackage[nomarkers]{endfloat}
% \let\MYoriglatexcaption\caption
% \renewcommand{\caption}[2][\relax]{\MYoriglatexcaption[#2]{#2}}
%\fi
% endfloat.sty was written by James Darrell McCauley, Jeff Goldberg and 
% Axel Sommerfeldt. This package may be useful when used in conjunction with 
% IEEEtran.cls'  captionsoff option. Some IEEE journals/societies require that
% submissions have lists of figures/tables at the end of the paper and that
% figures/tables without any captions are placed on a page by themselves at
% the end of the document. If needed, the draftcls IEEEtran class option or
% \CLASSINPUTbaselinestretch interface can be used to increase the line
% spacing as well. Be sure and use the nomarkers option of endfloat to
% prevent endfloat from "marking" where the figures would have been placed
% in the text. The two hack lines of code above are a slight modification of
% that suggested by in the endfloat docs (section 8.4.1) to ensure that
% the full captions always appear in the list of figures/tables - even if
% the user used the short optional argument of \caption[]{}.
% IEEE papers do not typically make use of \caption[]'s optional argument,
% so this should not be an issue. A similar trick can be used to disable
% captions of packages such as subfig.sty that lack options to turn off
% the subcaptions:
% For subfig.sty:
% \let\MYorigsubfloat\subfloat
% \renewcommand{\subfloat}[2][\relax]{\MYorigsubfloat[]{#2}}
% However, the above trick will not work if both optional arguments of
% the \subfloat command are used. Furthermore, there needs to be a
% description of each subfigure *somewhere* and endfloat does not add
% subfigure captions to its list of figures. Thus, the best approach is to
% avoid the use of subfigure captions (many IEEE journals avoid them anyway)
% and instead reference/explain all the subfigures within the main caption.
% The latest version of endfloat.sty and its documentation can obtained at:
% http://www.ctan.org/pkg/endfloat
%
% The IEEEtran \ifCLASSOPTIONcaptionsoff conditional can also be used
% later in the document, say, to conditionally put the References on a 
% page by themselves.




% *** PDF, URL AND HYPERLINK PACKAGES ***
%
%\usepackage{url}
% url.sty was written by Donald Arseneau. It provides better support for
% handling and breaking URLs. url.sty is already installed on most LaTeX
% systems. The latest version and documentation can be obtained at:
% http://www.ctan.org/pkg/url
% Basically, \url{my_url_here}.




% *** Do not adjust lengths that control margins, column widths, etc. ***
% *** Do not use packages that alter fonts (such as pslatex).         ***
% There should be no need to do such things with IEEEtran.cls V1.6 and later.
% (Unless specifically asked to do so by the journal or conference you plan
% to submit to, of course. )

\usepackage{graphicx}
%\usepackage{float}  
%\usepackage{amsmath} 
% correct bad hyphenation here
\hyphenation{op-tical net-works semi-conduc-tor}


\begin{document}
%
% paper title
% Titles are generally capitalized except for words such as a, an, and, as,
% at, but, by, for, in, nor, of, on, or, the, to and up, which are usually
% not capitalized unless they are the first or last word of the title.
% Linebreaks \\ can be used within to get better formatting as desired.
% Do not put math or special symbols in the title.
\title{Enhancing 5G O-RAN Communication Efficiency Through AI-Based Latency Forecasting}
%
%
% author names and IEEE memberships
% note positions of commas and nonbreaking spaces ( ~ ) LaTeX will not break
% a structure at a ~ so this keeps an author's name from being broken across
% two lines.
% use \thanks{} to gain access to the first footnote area
% a separate \thanks must be used for each paragraph as LaTeX2e's \thanks
% was not built to handle multiple paragraphs
%

\author{Raúl Parada,~Ebrahim Abu-Helalah,~%~\IEEEmembership{Member,~IEEE,}
        Jordi Serra,~\IEEEmembership{Senior Member,~IEEE,}~
        Anton Aguilar,~
        and~Paolo~Dini,~\IEEEmembership{Senior Member,~IEEE}% <-this % stops a space
        \vspace{-6mm}
\thanks{The authors are researcher at the SAI research unit within the Technological Telecommunications Centre of Catalonia (CTTC/CERCA)e-mail: \{rparada, aebrahim, jserra, aaguilar, pdini,\}@cttc.es.}}
%\vspace{-20mm}
% <-this % stops a space
%\thanks{Marco Oliveria  }% <-this % stops a space
%\thanks{J. Doe and J. Doe are with Anonymous University.}% <-this % stops a space
%\thanks{Manuscript received April 19, 2005; revised August 26, 2015.}}

% note the % following the last \IEEEmembership and also \thanks - 
% these prevent an unwanted space from occurring between the last author name
% and the end of the author line. i.e., if you had this:
% 
% \author{....lastname \thanks{...} \thanks{...} }
%                     ^------------^------------^----Do not want these spaces!
%
% a space would be appended to the last name and could cause every name on that
% line to be shifted left slightly. This is one of those "LaTeX things". For
% instance, "\textbf{A} \textbf{B}" will typeset as "A B" not "AB". To get
% "AB" then you have to do: "\textbf{A}\textbf{B}"
% \thanks is no different in this regard, so shield the last } of each \thanks
% that ends a line with a % and do not let a space in before the next \thanks.
% Spaces after \IEEEmembership other than the last one are OK (and needed) as
% you are supposed to have spaces between the names. For what it is worth,
% this is a minor point as most people would not even notice if the said evil
% space somehow managed to creep in.



% The paper headers
%\markboth{Journal of \LaTeX\ Class Files,~Vol.~14, No.~8, August~2015}%
%{Shell \MakeLowercase{\textit{et al.}}: Bare Demo of IEEEtran.cls for IEEE Journals}
% The only time the second header will appear is for the odd numbered pages
% after the title page when using the twoside option.
% 
% *** Note that you probably will NOT want to include the author's ***
% *** name in the headers of peer review papers.                   ***
% You can use \ifCLASSOPTIONpeerreview for conditional compilation here if
% you desire.




% If you want to put a publisher's ID mark on the page you can do it like
% this:
%\IEEEpubid{0000--0000/00\$00.00~\copyright~2015 IEEE}
% Remember, if you use this you must call \IEEEpubidadjcol in the second
% column for its text to clear the IEEEpubid mark.



% use for special paper notices
%\IEEEspecialpapernotice{(Invited Paper)}




% make the title area
\maketitle

% As a general rule, do not put math, special symbols or citations
% in the abstract or keywords.
\begin{abstract}
The increasing complexity and dynamic nature of 5G open radio access networks (O-RAN) pose significant challenges to maintaining low latency, high throughput, and resource efficiency. While existing methods leverage machine learning for latency prediction and resource management, they often lack real-world scalability and hardware validation. This paper addresses these limitations by presenting an artificial intelligence-driven latency forecasting system integrated into a functional O-RAN prototype. The system uses a bidirectional long short-term memory model to predict latency in real time within a scalable, open-source framework built with FlexRIC. Experimental results demonstrate the model's efficacy, achieving a loss metric below 0.04, thus validating its applicability in dynamic 5G environments. %Furthermore, we demonstrate how these latency predictions directly contribute to optimizing communication efficiency through real-time network adjustments and resource allocation.
\end{abstract}

% Note that keywords are not normally used for peerreview papers.
\begin{IEEEkeywords}
O-RAN, latency forecasting, artificial intelligence, 5G optimization, resource management
\end{IEEEkeywords}
\vspace{-2mm}




\IEEEpeerreviewmaketitle


%\vspace{-1.5mm}
\section{Introduction}
\vspace{-1.5mm}
\IEEEPARstart{O}{ptimizing} 5G open radio access networks (O-RAN) are critical to achieving low latency, high throughput, and reliable communication in dynamic wireless environments. Traditional methods, which are based on static models, do not adapt to these rapidly changing conditions. Recent advances in machine learning (ML) show promise in addressing these challenges, particularly for traffic forecasting and resource management. However, much of the existing research lacks real-world implementation and validation of scalability.
This work addresses this gap by introducing a latency forecasting system integrated into a fully functional O-RAN prototype. The system leverages artificial intelligence (AI)-driven xApps and the FlexRIC framework \cite{flexric} for real-time decision making, enabling programmable and scalable optimization. Our key contributions are as follows:
\begin{itemize} \setlength{\itemsep}{-0.3em} 
\item To design a bidirectional long short-term memory (LSTM) model for accurate latency prediction in dynamic 5G O-RANs.
\item To validate the system as a functional O-RAN prototype using open-source tools.
\item To develope a containerized framework for modular and reproducible O-RAN deployment.
\end{itemize} 

In existing research, ML techniques have been applied to optimize O-RAN, with a focus on traffic forecasting, latency reduction, and resource management. 
Habib et al. \cite{Habib2024} use LSTM models for traffic forecasting and dynamic scaling in 5G O-RAN, but lack hyperparameter optimization. Kavehmadavani et al. \cite{Kavehmadavani2023} employ time-series models like autoregressive integrated moving average (ARIMA) and LSTM for latency reduction, but do not evaluate model configurations or implement a hardware prototype. Khalid et al. \cite{Khalid2023} compare ML models for network optimization, but their work is limited to simulations. Khalid and Manar \cite{Khalid2024} use reinforcement learning for dynamic resource management, focusing on policy learning rather than real-time latency forecasting. Perveen et al. \cite{Perveen2023} address the integration of traffic prediction with network optimization but emphasize standard models over scalability techniques or latency trade-offs. In contrast, our work implements a real-time latency predictor for further optimal transmission decision, validated through a hardware-based O-RAN prototype. This approach combines scalability and practical deployment, addressing gaps in prior research \cite{Habib2024, Khalid2023}. Table \ref{tab:comparison} compares our work with the above-mentioned papers:
\begin{table}[h]
    \centering
    \renewcommand{\arraystretch}{0.8} % Adjust this value to change row height
    \caption{\small Comparison of our contributions to related works based on Latency Prediction (LP), Prototype Implementation (PI), Model Complexity (MC), Real-Time Processing (RTP), and Latency Trade-off (LTo).}
    \begin{tabular}{|l|c|c|c|c|c|}
        \hline
        \textbf{Work} & \textbf{LP} & \textbf{PI} & \textbf{MC} & \textbf{RTP} & \textbf{LTo} \\
        \hline
        \textbf{Habib et al.\cite{Habib2024}}         & $\checkmark$     & --               & $\rightarrow$    & --               & $\rightarrow$    \\
        \hline
        \textbf{Kavehmadavani et al.\cite{Kavehmadavani2023}} & $\checkmark$     & --               & $\rightarrow$    & $\rightarrow$     & $\rightarrow$    \\
        \hline
        \textbf{Khalid et al.\cite{Khalid2023}}       & $\checkmark$     & --               & $\rightarrow$    & --               & $\downarrow$     \\
        \hline
        \textbf{Khalid \& Manar\cite{Khalid2024}}     & --               & --               & $\uparrow$       & $\checkmark$     & $\uparrow$       \\
        \hline
        \textbf{Perveen et al.\cite{Perveen2023}}     & $\checkmark$     & --               & $\downarrow$     & $\rightarrow$     & $\rightarrow$    \\
        \hline
        \textbf{Our Work}              & $\checkmark$     & $\checkmark$    & $\uparrow$       & $\checkmark$     & $\uparrow$       \\
        \hline
    \end{tabular}
    \label{tab:comparison}
\end{table}\newline
This paper uniquely bridges the scalability and real-world validation gap by deploying a fully operational hardware prototype with AI-driven latency forecasting capabilities, demonstrating significant performance improvements over existing solutions.
\vspace{-4.5mm}
\section{Latency forecasting architecture}
\vspace{-1.5mm}
This section describes the composition of the entire latency forecasting architecture. The top image in Figure \ref{fig:arch} illustrates the complete architecture. The hardware elements of this setup are: a GPU-based workstation (Debian 12), a software defined radio device (Ettus USRP B210), an Nvidia Jetson Nano (Ubuntu 18) and a 5G modem (Quectel RMU500EK). To enhance flexibility and scalability, we deployed Linux containers (LXCs) on the workstations. All the tools used are open source, ensuring reproducibility. The LXCs are: \textbf{A)} srsRAN Project (Ubuntu 22): The O-RAN-native centralized unit (CU) / distributed unit (DU) developed by SRS that acts as a stand-alone gNB, \textbf{B)} Open5GS (Ubuntu 22): A 4G/5G core network implementation, \textbf{C)} Flexric (Ubuntu 24): A RAN intelligent control framework for programmable, real-time control and optimization of 5G RANs, \textbf{D)} Kafka (Ubuntu 24): A consumer instance that stores the key performance metrics (KPMs) generated by Flexric, \textbf{E)} xApp (Ubuntu 20): An edge service instance that trains the selected KPMs from Kafka consumer and infers latency forecast measurements and \textbf{F)} iPerf server (Ubuntu 20): A network testing tool that listens for incoming iPerf connections from iPerf clients to generate traffic and provide valid KPM values. 
\begin{figure}[!ht]
    \centering
    \begin{subfigure}[t]{\linewidth}
        \centering        \includegraphics[width=1.0\linewidth]{schemev3.JPG} % Replace with the first image filename
    \end{subfigure}
    \vspace{-2mm} % Adjust spacing between the images
    \begin{subfigure}[t]{\linewidth}
        \centering        \includegraphics[width=1.0\linewidth]{demo_cropv3.png} % Replace with the second image filename
    \end{subfigure}
    \caption{\small The top image illustrates the latency forecasting architecture and the bottom image shows the real demo.}
    \label{fig:arch}
\end{figure}
\vspace{-2.5mm} 
\section{Experimental setup \& results} \label{sec:res}
\vspace{-2mm}
This section provides the steps on how to set up the latency forecasting architecture described in the previous section and some preliminary experimental results. The bottom image from Figure \ref{fig:arch} shows the demonstration setup in our facilities' laboratory. The demo runs as follows (assuming you have created all the LXCs mentioned above): Execute \textit{lxc list} to display the created LXCs, their state should appear stopped. First, we launch containers A, B, C and D to run the O-RAN environment with \textit{lxc start gnb o5gs ric kafka}. The 5G Core instance might start automatically in the background. This is done to reduce the number of terminals to check. We enter the gNB container and execute \texttt{./run} to start the base station. Then, inside the Jetson Nano, we start the user equipment (UE) instance to make the O-RAN environment run. Once the O-RAN environment is running, we can start both the Kafka and FlexRIC instances to prepare the environment for KPM acquisition. Afterward, we start the xApp script, which allows training the acquired KPMs for a further inference period from the trained model. Since the xApp requests data from the Kafka consumer instance, we need to start it to connect to FlexRIC. Finally, because the system requires traffic flow throughout the system, we start the iPerf server waiting for an iPerf client to execute on the Nvidia Jetson Nano. The xApp forecasts latency in real time and displays a plot comparing actual and predicted values.
The system tracks several FlexRIC KPMs to evaluate and optimize network performance. Key metrics include the number of connected UE devices to assess network load, past latency values for real-time performance, and physical resource blocks (available and total) uplink for resource capacity. Transmission reliability is monitored via uplink packet success rate, while uplink throughput is reflected in transmitted service data unit volumes and throughput metrics. Air interface delay captures transmission latency, and signal quality is measured by signal-to-noise and channel quality indicator, enabling adaptive optimizations such as modulation and coding adjustments.
Since the data are streamed in time-series format, we have chosen the LSTM algorithm to train a prediction model to infer the latency forecasting, we have used the python library TensorFlow due to its hardware adaptability for process performance optimization. The training is performed offline with a preliminary dataset composed of 56k rows. The model is configured as bidirectional LSTM with 100 units and ReLU activation, leveraging a 60-step lookback period. The model includes dropout (0.2) to prevent overfitting and outputs a single value using a dense layer. It is compiled with mean squared error loss and the Adam optimizer (learning rate = 1e-5). Training employs early stopping (patience = 10) and saves the best model, using validation loss as the metric. Both training and validation loss metric achieved were below 0.04 indicating that the model is not overfitting, and it generalizes well to new, unseen data. The latency prediction contributes to a more efficient communication by deciding whether to transmit or not based on the link quality, hence, reducing the number of possible unsuccessful transmissions. A real-time prediction can be observed and described in the video uploaded in https://gitlab.cttc.es/supercom/LatencyForecasting.
\vspace{-2.5mm}
\section{Conclusion}
This work demonstrates the feasibility and effectiveness of integrating AI-driven latency forecasting into 5G O-RAN using open-source tools and hardware prototypes. Using a bidirectional LSTM model for real-time prediction, the system optimizes network performance through dynamic resource management and adaptive decision making. The results achieved, including accurate latency prediction and scalable deployment, address critical gaps in existing approaches. This implementation paves the way for advanced O-RAN optimizations, improving the reliability and efficiency of next-generation wireless networks. Future work includes exploring state-of-the-art models such as extended LSTM and temporal Kolmogorov-Arnold networks, measuring the system's CO2 emission reduction, and scaling the demo with multiple UEs by uniquely connecting additional 5G modems.
\vspace{-3.5mm}
\section*{Acknowledgment}
This work has been funded by the ”Ministerio de Asuntos Económicos y Transformación Digital” and the European Union-NextGenerationEU in the frameworks of the ”Plan de Recuperación, Transformación y Resiliencia” and of the ”Mecanismo de Recuperación y Resiliencia” under references TSI-063000-2021-18/24/77. The authors thank Marco Oliveira for designing the original demo in the SAI research unit at the CTTC.


% Can use something like this to put references on a page
% by themselves when using endfloat and the captionsoff option.
\ifCLASSOPTIONcaptionsoff
  \newpage
\fi



% trigger a \newpage just before the given reference
% number - used to balance the columns on the last page
% adjust value as needed - may need to be readjusted if
% the document is modified later
%\IEEEtriggeratref{8}
% The "triggered" command can be changed if desired:
%\IEEEtriggercmd{\enlargethispage{-5in}}

% references section

% can use a bibliography generated by BibTeX as a .bbl file
% BibTeX documentation can be easily obtained at:
% http://mirror.ctan.org/biblio/bibtex/contrib/doc/
% The IEEEtran BibTeX style support page is at:
% http://www.michaelshell.org/tex/ieeetran/bibtex/
\vspace{-4mm}
\bibliographystyle{IEEEtran}
% argument is your BibTeX string definitions and bibliography database(s)
\bibliography{bibtex/bib/IEEEexample}
%
% <OR> manually copy in the resultant .bbl file
% set second argument of \begin to the number of references
% (used to reserve space for the reference number labels box)
%\begin{thebibliography}{1}

%\bibitem{IEEEhowto:kopka}
%H.~Kopka and P.~W. Daly, \emph{A Guide to \LaTeX}, 3rd~ed.\hskip 1em plus
%  0.5em minus 0.4em\relax Harlow, England: Addison-Wesley, 1999.

%\end{thebibliography}

% biography section
% 
% If you have an EPS/PDF photo (graphicx package needed) extra braces are
% needed around the contents of the optional argument to biography to prevent
% the LaTeX parser from getting confused when it sees the complicated
% \includegraphics command within an optional argument. (You could create
% your own custom macro containing the \includegraphics command to make things
% simpler here.)
%\begin{IEEEbiography}[{\includegraphics[width=1in,height=1.25in,clip,keepaspectratio]{mshell}}]{Michael Shell}
% or if you just want to reserve a space for a photo:

%\begin{IEEEbiography}{Michael Shell}
%Biography text here.
%\end{IEEEbiography}

% if you will not have a photo at all:
%\begin{IEEEbiographynophoto}{John Doe}
%Biography text here.
%\end{IEEEbiographynophoto}

% insert where needed to balance the two columns on the last page with
% biographies
%\newpage

%\begin{IEEEbiographynophoto}{Jane Doe}
%Biography text here.
%\end{IEEEbiographynophoto}

% You can push biographies down or up by placing
% a \vfill before or after them. The appropriate
% use of \vfill depends on what kind of text is
% on the last page and whether or not the columns
% are being equalized.

%\vfill

% Can be used to pull up biographies so that the bottom of the last one
% is flush with the other column.
%\enlargethispage{-5in}



% that's all folks
\end{document}


