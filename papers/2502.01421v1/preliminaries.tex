\section{Preliminaries}

A hypergraph \( \hypergraph{H} = (V, E, \vect{w}) \) with a vertex set \( V \), a hyperedge set \( E \) and a weight vector \( \vect{w} \in \mathbb R _+ ^{E} \) defined in \( E \), is a generalization of the standard definition of graphs in which each hyperedge consists of a subset of vertices.
We define \( m = |E| \) and \( n = |V| \) and usually refer to \hypergraph{H} as an \( m \)-hyperedge \( n \)-vertex hypergraph.
For an integer \( r \geq 2 \), we say that \hypergraph{H} is of rank \( r \) if \( |e| \leq r \) for every hyperedge \( e \in E \).
We denote by \( E(H) \) the set of hyperedges of \( H \).

For a hyperedge \( e \in E \), \w{e} is the coordinate in \vect{w} corresponding to \( e \).
Similarly, for a vector \( \vect{x} \in \mathbb R ^{V} \) defined on \( V \) and a vertex \( u \in V \), \x{u} is the coordinate in \vect{x} corresponding to \( u \).

\paragraph{Spectral Hypergraph Sparsification.}
For any input vector \( \vect{x} \in \mathbb R ^{V} \), the \textit{energy} of hypergraph \( \hypergraph{H} = (V, E, \vect{w}) \) on \vect{x} is defined as
\begin{equation*}
Q _\hypergraph{H} (\vect{x}) = \sum_{e \in E} \w{e} \cdot \max_{u, v \in e} \left( \x{u} - \x{v} \right)^2.
\end{equation*}

We use \( Q _\hypergraph{H} (\vect{x}) \) to define the notion of a spectral hypergraph sparsifier for \hypergraph{H}.
For sake of brevity, we refer to such sparsifier as a \textit{spectral hypersparsifier} of \hypergraph{H}.
\begin{definition}
Let \hypergraph{H = (V, E, \vect{w})} be a hypergraph and let \( \hypergraph{\sparsifier{H}} = (V, \tilde{E} , \tilde{\vect{w}}) \) be a sub-hypergraph of \hypergraph{H}, i.e., \( \tilde{E} \subseteq E \).
For \( 0 < \varepsilon < 1 \), we say that \hypergraph{\sparsifier{H}} is a 
\SpectralHypersparsifier{} of \( H \) if, for every vector \( \vect{x} \in \mathbb R ^{V} \) defined on \( V \),
\begin{equation} \label{eq:spectral_hypergraph}
(1 - \varepsilon) Q _\hypergraph{\sparsifier{H}} (\vect{x}) \leq Q _\hypergraph{H} (\vect{x}) \leq (1 + \varepsilon) Q _\hypergraph{\sparsifier{H}} (\vect{x}).
\end{equation}
\end{definition}

Note that for graphs (i.e., hypergraphs of rank \( 2 \)), this definition is equivalent to the definition of a \( (1 \pm \varepsilon) \)-spectral sparsifier; for a graph \( G = (V, E, \vect{w}) \) and a vector \( \vect{x} \in \mathbb R ^{V} \), it follows that
\begin{equation} \label{eq:quadratic}
Q _G (\vect{x}) = \sum_{ uv \in E} \w{uv} \left( \x{u} - \x{v} \right)^2 = \vect{x}^T \mathcal L _G \vect{x},
\end{equation}
where \( \mathcal L _G
\) is the Laplacian matrix of \( G \) defined as the \( n \times n \) matrix with non-diagonal coordinate \( uv \) equal to the negated weight \( - w_{uv} \) for every edge \( uv \in E \), and diagonal coordinate \( uu \) equal to the weighted degree \( \sum _{uv \in E} w_{uv} \) of vertex \( u \).


\paragraph{Effective Resistance}
A common approach to studying the spectral properties of a graph \( G = (V, E, \vect{w}) \) is to view it as an electrical network, where each edge \( e \in E \) acts as a resistor with resistance \( \rr{e} = 1 / \w{e} \), and each node has a potential.
Using this, \Cref{eq:quadratic} is the energy of the electrical flow when the potential vector \vect{x} is applied to the nodes. 

The \textit{effective resistance \( R_G(u, v) \)} of a pair of vertices \( u, v \in V \) is defined as the potential difference between \( u \) and \( v \) when inducing one unit of current in the network.
Alternatively, it can be defined as 
\begin{equation*}
R_G(u, v) = \max_{\vect{x} \in \mathbb R ^{V}} \frac{\left( \x{u} - \x{v} \right)^2}{\vect{x}^T \mathcal L_G \vect{x}},
\end{equation*}
and thus, bounding the effective resistance \( R_G(u, v) \) bounds the energy \( Q_G(\vect{x}) \), and vice versa. 

To compute \( R_G(u, v) \) in an electrical network (i.e., the graph obtained from \( G \) by setting \( \rr{e} = 1 / \w{e} \) for every \( e \in E \)), we will utilize the well-known rules for combining the series and parallel resistors summarized in the following facts.


\begin{fact}
A series of resistors from \( u \) to \( v \) with resistances \( r_1, r_2, \dots, r_k \) can be replaced by a single resistor with endpoint \( u \) and \( v \) and an equivalent resistance of \( \sum_{i = 1} ^k r_i \).
\end{fact}

\begin{fact}
A set of parallel resistors with endpoints \( u \) and \( v \) and resistances \( r_1, r_2, \dots, r_k \) can be replaced by a single resistor with endpoints \( u \) and \( v \) and an equivalent resistance of \( 1 / \left( \sum_{i = 1} ^k 1/r_i \right) \).
\end{fact}

Since our analysis relies on computing \( R_G(\cdot, \cdot) \), in the rest of the text, we consider \( 1 / \w{e} \) as the \textit{length} of the edge \( e \). 



\paragraph{Spanners and Hyperspanners.}
We first discuss the concept of spanners in graphs and then consider its generalization to hypergraphs.
For a graph \( G = (V, E, \vect{w}) \) and a pair of vertices \( u, v \in V \), let \( d_G(u , v) \) denote the length of a shortest \( (u, v) \)-path in \( G \).
As discussed before, the length of an edge \( e \in E \) is equal to \( 1/\w{e} \).
Thus,
\begin{equation*}
d_G(u, v) = \min _{\text{\( (u, v) \)-path \( P \) in \( G \)}} \left( \sum_{e \in P} \frac{1}{\w{e}} \right).
\end{equation*}
\begin{definition}
For a graph \( G = (V, E, \vect{w}) \) and \( \alpha \geq 1 \), a subgraph \( G' = (V, E', \vect{w}') \) of \( G \) is called an \( \alpha \)-spanner of \( G \) if \( d_G(u, v) \leq d_{G'}(u, v) \leq \alpha d_G(u, v) \) for every pair of vertices \( u, v \in V \).
\end{definition}

To discuss the notion of hyperspanners, we first define the concept of hyperpaths.
In a hypergraph \hypergraph{H = (V, E, \vect{w})}, a subset of hyperedges \( e_1, e_2, \dots, e_k \) is called a \( (u, v) \)-hyperpath if (1) \( u \in e_1 \), (2) \( v \in e_k \), and (3) for every \( 1 \leq i \leq k - 1 \), \( e_i \cap e_{i + 1} \neq \emptyset \).
Analogous to graphs, the length of the hyperpath is defined as \( \sum_{i = 1} ^k 1/\w{e_i} \).
We denote by \( d_\hypergraph{H} (u, v) \) the length of the shortest \( (u, v) \)-hyperpath in \hypergraph{H}.

\begin{definition}
For a hypergraph \( \hypergraph{H} = (V, E, \vect{w}) \) and \( \alpha \geq 1 \), a sub-hypergraph \( \hypergraph{H'} = (V, E', \vect{w}') \) of \hypergraph{H} is called an \( \alpha \)-hyperspanner of \hypergraph{H} if \( d_\hypergraph{H}(u, v) \leq d_\hypergraph{H'}(u, v) \leq \alpha d_\hypergraph{H}(u, v) \) for every pair of vertices \( u, v \in V \).
\end{definition}

\paragraph{The Associated Graph.}
The associated graph of a hypergraph \hypergraph{H = (V, E, \vect{w})} is a graph \associated{\hypergraph{H} = (V, E_H, \vect{w}_H)}, defined as follows: for every hyperedge \( e \in E \) in the hypergraph, we create a clique \( C(e) \) in \associated{\hypergraph{H}} on the vertices of \( e \), with all edges in \( C(e) \) assigned a weight equal to \w{e}.
We define a function \( f \colon E_H \to E \), which maps each edge \( e_H \in E_H  \) of the associated graph \associated{\hypergraph{H}} to the hyperedge \( e \in E \) where \( e_H \in C(e) \).

Note that \associated{\hypergraph{H}} may have parallel edges, as a pair of vertices can appear in several hyperedges.













\paragraph{\( t \)-Bundle \( \alpha \)-spanners and \( t \)-Bundle \( \alpha \)-Hyperspanners.}

Given a graph \( G = (V, E, \vect{w}) \), a \( t \)-bundle \( \alpha \)-spanner \( B \) of \( G \) is a union \( T_1 \cup T_2 \cup \dots \cup T_t \) of \( \alpha \)-spanners, where \( T_i \) is an \( \alpha \)-spanner of \( G \setminus \cup_{j = 1} ^{i-1} T_j \).
In other words, \( T_1 \) is an \( \alpha \)-spanner of \( G \), \( T_2 \) is an \( \alpha \)-spanner of \( G \setminus T_1 \), and so on.

The existence of a \( t \)-bundle \( \alpha \)-spanner \( B \) of \( G \) guarantees that for each edge \( uv \in E \setminus B \), there are at least \( t \) paths with length at most \( \alpha \cdot d_G(u,v) \).
We will use this property later to bound the effective resistance \( R_{G_H}(u, v) \) of the associated graph \associated{\hypergraph{H}} of a hypergraph \hypergraph{H}.



Similarly, given a hypergraph \( \hypergraph{H} = (V, E, \vect{w}) \), a \( t \)-bundle \( \alpha \)-hyperspanner \hypergraph{B} of \hypergraph{H} is a union \(  \hypergraph{T_1} \cup \hypergraph{T_2} \cup \dots \cup \hypergraph{T_t} \) of \( \alpha \)-hyperspanners, where \hypergraph{T_i} is an \( \alpha \)-hyperspanner of \( \hypergraph{H} \setminus \cup_{j = 1} ^{i-1} \hypergraph{T_j} \).
In other words, \hypergraph{T_1} is an \( \alpha \)-hyperspanner of \hypergraph{H}, \hypergraph{T_2} is a an \( \alpha \)-hyperspanner of \( \hypergraph{H} \setminus \hypergraph{T_1} \), and so on.


\paragraph{Chernoff Bound \cite{Chernoff:1952aa}.}
We will use the multiplicative version of the Chernoff bound.
Let \( X_1, X_2, \dots, X_k \) be independent random variables with values \( 0 \) or \( 1 \); i.e., for each \( 1 \leq i \leq k \), \( X_i = 1 \) with probability \( p_i \) and \( X_i = 0 \) with probability \( 1 - p_i \).
Let \( X = \sum_{i = 1} ^k X_i \), and let \( \mu = \mathbb E \left[ X \right] = \sum_{i = 1} ^k p_i \).
Then, for every \( \delta \geq 0 \),
\begin{equation} \label{eq:Chernoff}
\mathbb P \left[ X \geq (1 + \delta) \mu \right] \leq \exp\left( - \frac{\delta ^2 \mu}{2 + \delta}  \right).
\end{equation}


