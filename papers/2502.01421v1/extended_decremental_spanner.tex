\section{Decremental Monotone \( t \)-Bundle Spanners}\label{app:abraham}

We explain the algorithm of \cite{Abraham:2016aa} for maintaining a monotone \( t \)-bundle \atmost{\log n}-spanner.

\abraham*

The algorithm is explained in \Cref{subsec:app_tbundle}, which is built upon another algorithm for maintaining a decremental monotone \( \alpha \)-spanner, as explained in \Cref{subsec:app_spanner}. 

\begin{remark} \label{remark:weighted}
We explain the algorithms assuming the graphs are unweighted.
To generalize them to weighted graphs, we use the standard batching technique as follows.
Given a graph \( G = (V, E, \vect{w}) \) with a weight ratio \( W = \max _{i,j} w_i / w_j \), we partition \( G \) into subgraphs \( G_1, G_2, \dots, G_{\log_{} W + 1} \), where \( G_i \) contains the edges of \( G \) with weights in the range of \( [2^{i-1}, 2^{i} ) \).
Assuming each \( G_i \) is unweighted, we can employ the algorithm for unweighted graphs and compute an \( \alpha \)-spanner \( F_i \) of \( G_i \) of size \( S(m, n, \alpha) \).
It is easy to see that \( F = \cup _i F_i \) is a \( 2 \alpha \)-spanner of \( G \) of size \atmost{S(m, n, \alpha) \log_{} W}.
\end{remark}







\subsection{Decremental Monotone \( \alpha \)-Spanner} \label{subsec:app_spanner}
Given a decremental unweighted graph \( G = (V, E) \), we maintain an \( \alpha \)-spanner \( F \) of \( G \) with the guarantees given in the following lemma.

\begin{lemma}[\cite{Abraham:2016aa}] \label{lem:decremental_spanner}
Given \( 0 < \varepsilon \leq 1 \) and an integer \( k \geq 2 \), and an \( n \)-vertex graph \( G = (V, E) \) with \( m \) initial edges, there is a decremental monotone algorithm against an oblivious adversary that maintains a \( (2k - 1) \)-spanner \( F \) of \( G \) with an expected size of \atmost{k^2 n^{1 + 1/k} \log n } and an expected total update time of \atmost{k^2 m \log n}.
\end{lemma}

The algorithm is based on \cite{Baswana:2012aa}, and is adjusted by \cite{Abraham:2016aa} to support the monotonicity property.
The spanner is computed using clusters on \( G \), as defined below.
Each cluster consists of selected inter-cluster edges and intra-cluster edges.
The goal is to maintain the clusters and the selected edges after each edge deletion.

\begin{definition}[Clusters] \label{def:clusters}
Given an unweighted graph \( G = (V, E) \), a set \( S \subseteq V \) of centers, a random permutation \( \sigma \) of \( S \), and a positive integer \( i \), 
the cluster \( \cluster{S}{i}{[s]} \subseteq V \) of radius \( i \) centered at \( s \in S \) consists of a vertex \( v \) if 
\begin{enumerate}
\item \( d(v, s) \leq i \), and
\item if \( d(v, s) = i \), then \( \sigma(s) < \sigma(s') \) for every other center \( s' \) satisfying \( d(v, s') = i \).
\end{enumerate}
We define \cluster{S}{i}{} to be the clustering that consists of all the clusters of radius \( i \) centered at \( S \).
\end{definition}




For each cluster \cluster{S}{i}{[s]}, we define the tree \tree{s} consisting of a shortest path from each vertex  \( v \in \cluster{S}{i}{[s]} \) to the center \( s \) as follows.
The parent of \( v \) is the neighbor with the shortest distance from \( s \) (i.e., a neighbor in \cluster{S}{i}{[s]} with distance \( d(v, s) - 1 \) from \( s \)) and the smallest possible ordering w.r.t.\ \( \sigma \).
We define the forest \( \forest{S}{i}{} = \cup_{s \in S} T_s \).

The main difference between the algorithm of \cite{Baswana:2012aa} and that of \cite{Abraham:2016aa} is that \cite{Abraham:2016aa} takes into account the random ordering \( \sigma \) (as explained above) when computing the tree \tree{s}.
Using a straightforward modification of the algorithm of \citeauthor{Even:1981aa} \cite{Even:1981aa}, we can decrementally maintain the clustering \cluster{S}{i}{} and the forest \forest{S}{i}{} on the multi-graph \( G \).
The guarantee of the algorithm is summarized in the theorem below.



\begin{theorem}[\cite{Baswana:2012aa}] \label{th:intra_cluster}
Given an \( n \)-vertex graph \( G = (V, E) \) with \( m \) initial edges, a set \( S \subseteq V \) of centers, a random permutation \( \sigma \), and a positive integer \( i \), there is a decremental data structure that maintains the clustering \cluster{S}{i}{} and the forest 
 \forest{S}{i}{} 
in expected total time \atmost{ m i \log n}.
\end{theorem}




We now define the spanner \( F \) of \( G \) by selecting its edges from a sequence of clusterings, referred to as the \textit{clustering hierarchy} on \( G \).

\begin{definition}[Clustering hierarchy]
Given a graph \( G = (V, E) \), a random permutation \( \sigma \), and an integer \( k \geq 2 \), let \( S_0 \supseteq S_1 \supseteq \dots \supseteq S_k \) be subsets of \( V \), where \( S_0 = V \) and \( S_k = \emptyset \), such that \( S_{i+1} \) is obtained by sampling each vertex of \( S_i \) with probability \( n^{-1/k} \).

For every \( 1 \leq i \leq k \), we define the clustering \( C_i = \cluster{S_i}{i}{} \) on \( G \) with centers \( S_i \) and radius \( i \).
We denote by \( V_i \) the set of all vertices covered by \( C_i \).
i.e., \( V_i = \{ v \in V \mid d(v, S_i) \leq i \} \).
Moreover, we define the forest \( F_i = \forest{S}{i}{} \) on \( C_i \) as defined before.

We denote by \( C_1, C_2, \dots, C_k \) the clustering hierarchy on \( G \).
\end{definition}

\begin{definition}[Spanner \( F \)] \label{def:spanner_F}
Given a graph \( G = (V, E) \), a random permutation \( \sigma \), and an integer \( k \geq 2 \), let \( C_1, C_2, \dots, C_k \) be a clustering hierarchy on \( G \) with covered vertices \( V_1, V_2, \dots, V_k \) and forests \( F_1, F_2, \dots, F_k \). 
The spanner \( F \) consists of the following edges:
\begin{enumerate}
\item 
Intra-cluster Edges: For every \( 1 \leq i \leq k \), \( F \) consists of all the edges in the forest \( F_i \) defined on the clustering \( C_i \).

\item 
Inter-cluster Edges: For every \( 1 \leq i \leq k - 1 \) and for each \( v \in V_{i} \setminus V_{i + 1} \) belonging to the cluster \cluster{S_i}{i}{[s]} of \( C_i \) centered at \( s \), if \( v \) has some neighbors in the cluster \cluster{S_i}{i}{[s']} with \( s' \neq s \), then \( F \) consists of the edge \( (v, v') \) where \( v' \) is the neighbor of \( v \) in \cluster{S_i}{i}{[s']} with the smallest possible ordering w.r.t.\ \( \sigma \).
\end{enumerate}
After an update in \( G \), an edge in \( F \) may no longer correspond to items 1 and 2 above.
In such a case, we keep the edge in \( F \) until it is removed from \( G \), while adding new edges to \( F \) that satisfy items 1 or 2. 
\end{definition}



\begin{remark} \label{remark:inter_cluster}
\cite{Abraham:2016aa} lets the inter-cluster edge \( (v, v') \) to be arbitrarily chosen, but we choose \( v' \) to be the neighbor with the smallest possible ordering w.r.t.\ \( \sigma \).
Although this does not change the asymptotic analysis on the running time and the size of \( F \), it reduces the size of the spanner in implementation.
Recall that every selected edge remains in \( F \) until it is removed from \( G \).
By choosing \( v' \) w.r.t.\ \( \sigma \), when \( v \) changes its cluster, the edge connecting \( v \) to its new parent has already been added to \( F \) as an inter-cluster edge.
This eliminates the need to add  a new edge to \( F \) connecting \( v \) to its new parent.
\end{remark}


\begin{remark}
The choices for intra-cluster and inter-cluster edges in \( F \) are not unique when \( G \) is a multi-graph.
In this case, we have a set of parallel edges adjacent to the vertex with the least possible order.
Here, we arbitrarily choose one of these edges to be in \( F \). 
It is straightforward to see this approach does not change the guarantees compared to when \( G \) is simple.
\end{remark}


To prove the update time for maintaining the spanner \( F \), \cite{Baswana:2012aa} first computes the expected number of times a vertex changes its cluster in the clustering \( C_i \).

\begin{lemma}[\cite{Baswana:2012aa}] \label{lem:cluster_change}
For a vertex \( v \in V \), the expected number of times \( v \) changes its cluster in the clustering \( C_i \) is \atmost{i \log n}.
\end{lemma}

Since a change of cluster for \( v \) costs \atmost{\deg(v)} time to find the new inter-cluster and the intra-cluster edges adjacent to \( v \), the total expected update time for maintaining the edges of \( F \) that belong to \( C_i \) is \( \atmost{\sum _{v \in V} i \deg(v) \log n} = \atmost{i m \log n} \).

Since we support the monotonicity property, we keep more edges in \( F \) than the spanner of \cite{Baswana:2012aa}: due to the updates, \( v \) may change its parent in the forest \( F_i \) from \( v' \) to \( v'' \) as follows.
If the neighbor \( v'' \) of \( v \) becomes part of the cluster containing \( v \) and \( \sigma(v'') < \sigma(v') \), then \( v'' \) becomes the parent of \( v \) in \( F_i \), and we add \( (v, v'') \) to \( F \).
To support the monotonicity, we  keep the edge \( (v, v') \) in \( F \) while adding \( (v, v'') \) to it as the new edge connecting \( v \) to its parent in \( F_i \).
Consequently, to obtain the guarantees for the monotone algorithm maintaining \( F \), we need to be more elaborate and compute the expected number of times \( v \) changes its \textit{parent}.


\begin{lemma}[\cite{Abraham:2016aa}] \label{lem:parent_change}
For a vertex \( v \in V \), the expected number of times \( v \) changes its parent in the forest \( F_i \) is \atmost{i \log n}.
\end{lemma}

\begin{proof}

Since parallel edges have the same endpoints, the expected number of parent changes in \( G \) is upper bounded by the expected number of parent changes in the underlying simple graph of \( G \).
Therefore, we assume in this proof that \( G \) is a simple graph.

Suppose that \( G \) has gone under an arbitrary sequence of edge deletions.
Let \( V_j \) be the centers that had distance \( j \) from \( v \) at some point, and let \( (s, v') \) be the configuration in which \( v \) was assigned to the cluster centered at \( s \), \( d(v, s) = j \), and \( v' \) was the parent of \( v \) in \( F_i \).

We first prove that while keeping its distance equal \( j \) from \( S_i \), \( v \) cannot go back to the configuration \( (s, v') \) after moving to another configuration due to an update.
There are two possible cases.
\begin{enumerate}
\item 
If, after an edge deletion, \( v \) changes its configuration but stays in the same cluster.
i.e., \( v \) changes its configuration from \( (s, v') \) to \( (s, v'') \).
Since \( v \) has not changed its cluster, it follows that \( d(v', s) \leq d(v'', s) = j \) \textit{before} the update.
However, note that \( \sigma ( v' ) \leq \sigma ( v'' ) \) as \( v' \) had chosen before \( v'' \) to be the parent of \( v \).
Thus, if \( v \) chooses \( v'' \) as its parent only if \( d( v, s ) > j \) after the update.
As \( d(v, s) \) is monotonic, \( v \) cannot come back to \( (s, v') \) while keeping its distance form \( S_i \) equal to \( j \).

\item
If \( v \) changes its configuration to \( (s', v'') \), where \( s' \neq s \).
Since \( v \) keeps its distance from \( S_i \) to be equal \( j \), it follows that \( d(v, s) = d(v, s') = j \) \textit{before} the update.
By construction, \( \sigma ( s ) \leq \sigma ( s' ) \) as \( v \) had been assigned to the cluster centered at \( s \) before the update.
However, since \( v \) changed its cluster, \( d( v, s ) > j \) after the update.
As \( d(v, s) \) is monotonic, \( v \) cannot come back to \( (s, v') \) while keeping its distance form \( S_i \) equal to \( j \).
\end{enumerate}



We now bound the expected number of times \( v \) changes its parent while keeping \( d(v, S_i) = j \).
We only need to compute the expected number of configurations that appear throughout any arbitrary sequence of edge deletions.
Let \( k \) be the number of configurations \( v \) visits while \( d(v, S_i) = j \).
Note that \( k \leq n^2 \).
Since \( \sigma \) is a random permutation, each configuration has an equal probability to happen first, which is equal to \( 1/k \).
Similarly, after visiting the \( l \)th configuration, the probability of moving to an unvisited configuration is \( 1/(k - l + 1) \).
Thus, the expected number of configuration change is
\[
\sum_{l = 1} ^{k} \frac{1}{k - l + 1 } = \atmost{\log k} = \atmost{\log n},
\]
which also upper bounds the expected number of parent changes while \( d(v, S_i) = j \), as   several consecutive configurations might assign \( v \) to different clusters with the same parent.

To upper bound the number of times \( v \) changes its parent in \( F_i \), first note that once \( d(v, S_i) \) is increased, it cannot be decreased later as \( d(v, S_i) \) is monotonic.
Since in \( F_i \), \( 1 \leq d(v, S_i) \leq i \) and the expected number of times that \( v \) changes its parent while \( d(v, S_i) = j \) is \atmost{\log n}, it follows that the number of parent changes in \( F_i \) is \atmost{i \log n}.
\end{proof}






We are now ready to obtain the guarantees on the quality and the size of \( F \).


\begin{lemma}[\cite{Baswana:2012aa}] \label{lem:app_spanner}
\( F \) is a \( (2k - 1) \)-spanner of \( G \).
\end{lemma}
\begin{proof}
Suppose that \( (u, v) \) is an edge in \( G \setminus F \) and let \( j \) be the smallest integer such that \( u \) or \( v \) belongs to \( V_j \setminus V_{j+1} \).
Thus, \( u, v \in V_j \), which means that \( u \) and \( v \) are covered by the clusters \cluster{S_j}{j}{[s]} and \cluster{S_j}{j}{[s']} in \( C_j \), centered at \( s, s' \in S_j \), respectively.
Without loss of generality, suppose that \( u \in V_j \setminus V_{j+1} \).
There are two possibilities to consider:
\begin{enumerate}
\item 
If \( s = s' \), then \( u \) and \( v \) are connected in \( F \) by the path \( v \to s \to u \) which exists in the cluster \cluster{S_j}{j}{[s]} centered at \( s \).
Since the radius of the cluster is \( j \leq k - 1 \), the length of the path is at most \( 2k - 2 \).

\item
If \( s \neq s' \), then \( u \) and \( v \) are covered by different clusters in \( C_j \).
Since \( u \in \cluster{S_j}{j}{[s]} \) is connected to \( v \in \cluster{S_j}{j}{[s']} \) but \( (u, v) \) does not belong to \( F \), by item 2 of \Cref{def:spanner_F}, \( u \) is connected to a vertex \( v' \in \cluster{S_j}{j}{[s']} \) such that \( (v, v') \) belongs to \( F \).
Thus, the path \( v' \to s' \to v  \) in \( \cluster{S_j}{j}{[s']} \) concatenated with the edge \( (u, v') \) forms a path in \( F \) from \( u \) to \( v \) with a length of at most \( 1 + 2(k - 1) = 2k - 1 \).
\end{enumerate}
The statement follows accordingly.
\end{proof}


\begin{lemma}[\cite{Abraham:2016aa}] \label{lem:app_size}
The expected number of edges of \( F \) is \atmost{k^2 n^{1 + 1/k} \log n }.
\end{lemma}
\begin{proof}
Since each forest \( F_i \) has \atmost{n} edges, the total number of intra-cluster edges present in \( F \) at the beginning of the algorithm is \atmost{kn}.
As discussed in \Cref{remark:inter_cluster}, other edges to be added in \( F \) will be counted as inter-cluster edges.

To bound the number of inter-cluster edges, suppose that \( v \in V_i \setminus V_{i + 1} \).
Recall that the set \( S_{i + 1} \) of centers of the clustering \( C_{i + 1} \) is obtained by sampling each vertex of \( S_i \) with probability \( n ^{-1/n} \).
Thus, the expected number of removed centers from \( S_i \) to form \( S_{i + 1} \) is \atmost{n^{1/n}}, meaning that \( v \notin V_{i + 1} \) because \( v \) has lost its connections to every cluster in \( C_i \) after the removal of \atmost{n^{1/n}} centers.
Therefore, the expected number of inter-cluster edges in \( C_i \) touching \( v \) that are counted in \( F \) is \atmost{n^{1/n}}.
As \( C_i \) is decremental, \( v \) may change its parent, where by \Cref{remark:inter_cluster}, the new parent is connected by an edge already chosen as inter-cluster edge in the previous updates.
To support the monotonicity property, these edges remain in \( F \), and new inter-cluster edges touching \( v \) are added to \( F \), which are \atmost{n^{1/n}} as discussed before.
By \Cref{lem:parent_change}, the expected number of times \( v \) changes its parent in \( C_i \) is \atmost{i \log n}, which gives us \atmost{i n^{1/n} \log n } as the total expected number of inter-cluster edges touching \( v \) in \( C_i \) that appeared in \( F \).
Thus, the total number of inter-cluster edges in \( F \) is \( \sum _{v \in V} \sum _{i = 1} ^ k \atmost{i n^{1/n} \log n } = \atmost{k^2 n^{1 + 1/n} \log n } \).
\end{proof}


We conclude this section by proving \Cref{lem:decremental_spanner}.

\begin{proof}[Proof of \Cref{lem:decremental_spanner}]
\( F \) being a spanner and the guarantee on its size is proved by \Cref{lem:app_spanner,lem:app_size}.
By \Cref{def:spanner_F}, the edges stay in \( F \) until they are removed from \( G \), and so \( F \) supports the monotonicity property.
We prove the update time.


\underline{Update time:}
by \Cref{th:intra_cluster}, the clustering \( C_i \) and the forest \( F_i \) are maintained in \atmost{i m \log n} total update time.
By \Cref{lem:parent_change}, the number of times \( v \) changes its parent in \( C_i \) is \atmost{i \log n}.
Since a change of parent for \( v \) costs \atmost{\deg(v)} time to find the new inter-cluster  edges touching \( v \), the total update time to maintain all inter-cluster edges in \( C_i \) is \( \atmost{\sum _{v \in V} i \deg(v) \log n} = \atmost{i m \log n} \).
Therefore, the total update time for maintaining all the clusterings and forests is \( \sum_{i = 1} ^ k \atmost{m i \log n} = \atmost{m k^2 \log n} \)
.
\end{proof}




\subsection{Decremental Monotone \( t \)-Bundle \atmost{\log n}-Spanner} \label{subsec:app_tbundle}

We use the decremental monotone \( (2k -1) \)-spanner of 
\Cref{lem:decremental_spanner} where \( k = \lceil \log n \rceil \) to maintain the spanner \( T_1 \) of \( G \), spanner \( T_2 \) of \( G \setminus T_1 \) and so on.
Since \( \alpha = \atmost{\log n} \), \( B = \cup_{i = 1} ^t T_i \) is the resulting \( t \)-bundle \atmost{\log n}-spanner.




















\begin{proof}[Proof of \Cref{lem:Abraham_decremental_t-bundle}]
We first assume that \( G \) is unweighted.
The fact that \( B \) is a \( t \)-bundle \atmost{\log n}-spanner follows immediately from its construction.
We prove the update time and that \( B \) supports the monotonicity property, and then generalize the algorithm to weighted graphs.

\underline{Monotonicity:}
we show that \( \cup_{i = 1} ^j T_i \) supports the monotonicity property by induction on \( j \).
For \( j = 1 \), \( T_1 \) is a monotone spanner of \( G \) by \Cref{lem:decremental_spanner}.
For \( j \geq 2 \), suppose that \( T_1 \cup T_2, \dots \cup T_j \) is a monotone spanner of \( G \).
i.e., an edge \( e \) is removed from \( T_1, T_2, \dots, T_j \) only if \( e \) is removed from \( G \), and so \( e \) will not be added to \( G \setminus \cup_{i = 1} ^{k} T_i \) for every \( 1 \leq k \leq j \).
Now, note that \( T_1 \cup T_2 \cup \dots \cup T_{j + 1} \) is a partition of edges, which means that an edge \( e \) is removed from \( \cup_{i = 1} ^{j+1} T_i \) iff \( e \) is removed from \( T_l \) for an integer \( 1 \leq l \leq j + 1 \).
If \( l \leq j \), then by assumption, \( e \) will not be added to \( G \setminus \cup _{i = 1} ^j T_i \).
If \( l = j + 1 \), then by \Cref{lem:decremental_spanner}, \( T_{j + 1} \) is a monotone spanner of \( G \setminus \cup_{i = 1} ^{j} T_i \), meaning that \( e \) will not be added to \( G \setminus \cup _{i = 1} ^j T_i \) for every \( 1 \leq k \leq j \).

\underline{Update time:}
we set \( k = \lceil \log n \rceil \) and by \Cref{lem:decremental_spanner}, each \( T_i \) has an expected size of \atmost{n \log ^3 n } and is maintained in an expected total update time of \atmost{m \log ^3 n}.
Thus, \( B \) has an expected size of \atmost{t n \log ^3 n } and is maintained in an expected total update time of \atmost{t m \log ^3 n}.

\underline{Generalization to weighted graphs:}
we use \Cref{remark:weighted}, which results in an \atmost{\log W} overhead in the size of the spanner.
Since the batching technique partitions the edges of the graph into subgraphs, the total update time will not change: \( \sum _i \atmost{t m_i \log ^3 n} = \atmost{t m \log ^3 n} \).
The monotonicity clearly follows, as the algorithm maintains a monotone spanner on each subgraph in the batch. 
\end{proof}









