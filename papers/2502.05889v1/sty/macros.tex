
%----- Algorithm Environment ----------------------------------
\RequirePackage{algpseudocode}
%Header for Algorithms/Functionalities
\newcommand{\algoHead}[1]{\vspace{0.2em} \underline{\textbf{#1}} \vspace{0.3em}}
\newcommand{\algoHeadExt}[2]{\vspace{0.2em} \underline{\textbf{#1} #2} \vspace{0.3em}}
%Multiline Algo-States
\makeatletter
\algnewcommand{\ExtendedState}[1]{\State
\parbox[t]{\dimexpr\linewidth-\ALG@thistlm}{\hangindent=\algorithmicindent\strut\hangafter=3#1\strut}}
\makeatother
%Algorithms States
\algnewcommand\algorithmicinput{\textbf{Input:}}
\algnewcommand\Input{\item[\algorithmicinput]}
\renewcommand{\algorithmicensure}{\textbf{Output:}}
%Algo Comments
\algrenewcommand{\algorithmiccomment}[1]{{\color{gray}// #1}}
%Inline ifs
\algnewcommand{\IIf}[1]{\State\algorithmicif\ #1\ \algorithmicthen}
\algnewcommand{\EndIIf}{\unskip\ \algorithmicend\ \algorithmicif}

% Define "upon" keyword in algpseudocode.
\algdef{SE}% flags used internally to indicate we're defining a new block statement
[UPON]% new block type, not to be confused with loops or if-statements
{Upon}
{EndUpon}
[1]% There is one argument.
{Upon {#1}} % How begin looks.
{\textbf{end upon}} % How end looks.
\algtext*{EndUpon} % Remove end, forcefully.
% The proper way to remove them based on package flags.
% \makeatletter
% \ifthenelse{\equal{\ALG@noend}{t}}%
%   %{}
%   {\algtext*{EndUpon}}
%   {}%
% \makeatother

% Define "when" keyword in algpseudocode.
\algdef{SE}% flags used internally to indicate we're defining a new block statement
[WHEN]% new block type, not to be confused with loops or if-statements
{When}
{EndWhen}
[1]% There is one argument.
{When {#1}} % How begin looks.
{\textbf{end when}} % How end looks.
\algtext*{EndWhen} % Remove end, forcefully.
%\algtext*{EndIf} % This was here before for some reason.
% The proper way to remove them based on package flags.
% \makeatletter
% \ifthenelse{\equal{\ALG@noend}{t}}%
%   %{}
%   {\algtext*{EndWhen}}
%   {}%
% \makeatother

%----- Box Environment -----------------------------------------
% v 2019.1.DT small skips
%---
 \RequirePackage{varwidth}
 \RequirePackage{color}
 \RequirePackage[most]{tcolorbox}%with most option


%----- Protocol Boxes
 \newtcolorbox{titlebox}[5]{enhanced,center,colframe=black,colback=white,boxrule={#3},arc={#2},auto outer arc,%
 breakable,pad at break*=5pt,vfill before first,before={%\par\smallskip\noindent
 },after={\par\smallskip},top=12pt,left=4pt,%
 enlarge top by=2pt,%enlarge bottom by=7pt,%
 fontupper=\small,
 title={\rule[-.3\baselineskip]{0pt}{\baselineskip}\normalsize\sffamily\bfseries #1}, varwidth boxed title*=-30pt, 
 attach boxed title to top left={yshift=-10pt,xshift=10pt}, coltitle=black,
 boxed title style={colback=white,boxrule={#5},arc={#4},auto outer arc}
 }

 \newenvironment{systembox}[1]
 {\vspace{\baselineskip}\begin{titlebox}{Functionality \normalfont #1}{2.5pt}{1pt}{3.5pt}{1pt}}
 {\end{titlebox}}

 \newenvironment{protocolbox}[1]
 {\begin{titlebox}{Protocol \normalfont #1}{0.5pt}{0.5pt}{1pt}{0.75pt}}
 {\end{titlebox}}


 \newenvironment{adaptedprotocolbox}[1]
 {\begin{titlebox}{Adapted protocol \normalfont #1}{0.5pt}{0.5pt}{1pt}{0.75pt}}
 {\end{titlebox}}
 
 
  \newenvironment{simulatorbox}[1]
 {\begin{titlebox}{Simulator \normalfont #1}{0.5pt}{0.5pt}{2pt}{0.75pt}}
 {\end{titlebox}}
 
  \newenvironment{processbox}[1]
 {\begin{titlebox}{Process \normalfont #1}{0.5pt}{0.5pt}{1pt}{0.75pt}}
 {\end{titlebox}}

 \newenvironment{algobox}[1]
 {\begin{titlebox}{Algorithm \normalfont #1}{0.5pt}{0.5pt}{1pt}{0.75pt}}
 {\end{titlebox}}
 
 \newenvironment{funcbox}[1]
 {\begin{titlebox}{Function \normalfont #1}{0.5pt}{0.5pt}{1pt}{0.75pt}}
 {\end{titlebox}}

 \newenvironment{dianabox}[1]
 {\begin{titlebox}{\normalfont #1}{0.5pt}{0.5pt}{1pt}{0.75pt}}
 {\end{titlebox}}