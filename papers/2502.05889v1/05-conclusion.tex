\section{Conclusion}

We investigated whether stable matching can be achieved in a synchronous network where some of the parties involved may be byzantine. We analyzed this problem under various network topologies, both with and without cryptographic assumptions. For each setting, we gave necessary and sufficient conditions,  assuming that each party holds as input a complete ranking of the parties on the other side.


Our work highlights multiple promising directions for further research. A first direction could be generalizing our results to the \emph{stable roommate} problem. Instead of assuming that the parties to be matched are in two disjoint sets, the stable roommate problem seeks a stable matching within the same set. Note that our necessary conditions also apply to a byzantine variant of the stable roommate problem, even though there is no longer a distinction between byzantine parties on the two sides. However, the stable matching problem comes with the guarantee that a stable matching always exists, while the stable roommate problem does not. Hence, definitions and properties need to be refined to account for this.

Another interesting direction would be to extend our question to the asynchronous model. Using our current definitions, one can prove that even if only one party known in advance can be byzantine, stable matching is not solvable. Therefore, the properties required for the stable matching would have to be relaxed for this problem to be of interest.

Finally, while our work has provided a complete characterization in terms of solvability, there are multiple aspects in which our feasibility results could be improved. This includes improvements in terms of efficiency (i.e., communication complexity), but also improvements in terms of guarantees, such as providing some degree of privacy. 




