\begin{abstract}
We investigate implications of the \emph{(extended) low-degree conjecture} (recently formalized in \cite{moitra2023precise}) in the context of the symmetric stochastic block model. Assuming the conjecture holds, we establish that no polynomial-time algorithm can \emph{weakly recover} community labels below the \emph{Kesten-Stigum (KS) threshold}. In particular, we rule out polynomial-time estimators that, with constant probability, achieve correlation with the true communities that is significantly better than random. Whereas, above the KS threshold, polynomial-time algorithms are known to achieve constant correlation with the true communities with high probability  \cite{massoulie2014community,abbe2015community}. 

To our knowledge, we provide the first rigorous evidence for the sharp transition in recovery rate for polynomial-time algorithms at the KS threshold. Notably, under a stronger version of the low-degree conjecture, our lower bound remains valid even when the number of blocks diverges. Furthermore, our results provide evidence of a computational-to-statistical gap in learning the parameters of stochastic block models.

In contrast to prior work, which either (i) rules out polynomial-time algorithms for hypothesis testing with $1 - o(1)$ success probability \cite{Hopkins18, bandeira2021spectral} under the low-degree conjecture, or (ii) rules out low-degree polynomials for learning the edge connection probability matrix \cite{luo2023computational}, our approach provides stronger lower bounds on the recovery  and learning problem.

Our proof combines low-degree lower bounds from \cite{Hopkins18, bandeira2021spectral} with graph splitting and cross-validation techniques. In order to rule out general recovery algorithms, we employ the correlation preserving projection method developed in \cite{Hopkins17}.
\end{abstract}