\RequirePackage[l2tabu,orthodox]{nag}
\documentclass
[letterpaper,12pt,]
{article}


\usepackage{etex}
\usepackage{verbatim}
\usepackage{xspace,enumerate}
\usepackage[dvipsnames]{xcolor}
\usepackage[T1]{fontenc}
\usepackage[full]{textcomp}
\usepackage[american]{babel}
\usepackage{mathtools}
\newcommand\hmmax{0} %
\usepackage{amsthm}
\usepackage[
letterpaper,
top=1in,
bottom=1in,
left=1in,
right=1in]{geometry}
\usepackage{newpxtext} %
\usepackage{textcomp} %
\usepackage[varg,bigdelims]{newpxmath}
\usepackage[scr=rsfso]{mathalfa}%
\usepackage{bm} %
\linespread{1.1}%
\let\mathbb\varmathbb
\usepackage{microtype}
\usepackage[pagebackref,colorlinks=true,urlcolor=blue,linkcolor=blue,citecolor=OliveGreen]{hyperref}
\usepackage[capitalise,nameinlink]{cleveref}
\crefname{lemma}{Lemma}{Lemmas}
\crefname{fact}{Fact}{Facts}
\crefname{theorem}{Theorem}{Theorems}
\crefname{corollary}{Corollary}{Corollaries}
\crefname{claim}{Claim}{Claims}
\crefname{example}{Example}{Examples}
\crefname{algorithm}{Algorithm}{Algorithms}
\crefname{problem}{Problem}{Problems}
\crefname{definition}{Definition}{Definitions}
\crefname{exercise}{Exercise}{Exercises}
\crefname{model}{Model}{Models}
\usepackage{amsthm}
\newtheorem{exercise}{Exercise}[section]
\newtheorem{theorem}{Theorem}[section]
\newtheorem*{theorem*}{Theorem}
\newtheorem{lemma}[theorem]{Lemma}
\newtheorem*{lemma*}{Lemma}
\newtheorem{fact}[theorem]{Fact}
\newtheorem*{fact*}{Fact}
\newtheorem{proposition}[theorem]{Proposition}
\newtheorem*{proposition*}{Proposition}
\newtheorem{corollary}[theorem]{Corollary}
\newtheorem*{corollary*}{Corollary}
\newtheorem{hypothesis}[theorem]{Hypothesis}
\newtheorem*{hypothesis*}{Hypothesis}
\newtheorem{conjecture}[theorem]{Conjecture}
\newtheorem*{conjecture*}{Conjecture}
\theoremstyle{definition}
\newtheorem{definition}[theorem]{Definition}
\newtheorem*{definition*}{Definition}
\newtheorem{construction}[theorem]{Construction}
\newtheorem*{construction*}{Construction}
\newtheorem{example}[theorem]{Example}
\newtheorem*{example*}{Example}
\newtheorem{question}[theorem]{Question}
\newtheorem*{question*}{Question}
\newtheorem{algorithm}[theorem]{Algorithm}
\newtheorem*{algorithm*}{Algorithm}
\newtheorem{assumption}[theorem]{Assumption}
\newtheorem*{assumption*}{Assumption}
\newtheorem{problem}[theorem]{Problem}
\newtheorem*{problem*}{Problem}
\newtheorem{openquestion}[theorem]{Open Question}
\newtheorem*{openquestion*}{Open Question}
\theoremstyle{remark}
\newtheorem{claim}[theorem]{Claim}
\newtheorem*{claim*}{Claim}
\newtheorem{remark}[theorem]{Remark}
\newtheorem*{remark*}{Remark}
\newtheorem{observation}[theorem]{Observation}
\newtheorem*{observation*}{Observation}
\theoremstyle{model}
\newtheorem{model}[theorem]{Model}
\newtheorem*{model*}{Model}
\usepackage{paralist}
\frenchspacing
\let\originalleft\left
\let\originalright\right
\renewcommand{\left}{\mathopen{}\mathclose\bgroup\originalleft}
\renewcommand{\right}{\aftergroup\egroup\originalright}
\usepackage{turnstile}
\usepackage{mdframed}
\usepackage{tikz}
\usetikzlibrary{shapes,arrows}
\usetikzlibrary{positioning}
\usetikzlibrary{decorations.pathreplacing}
\usepackage{caption}
\DeclareCaptionType{Algorithm}
\usepackage{newfloat}
\usepackage{array}
\usepackage{subfig}
\usepackage{xparse}
\usepackage{amsthm} %
\makeatletter
\let\latexparagraph\paragraph
\RenewDocumentCommand{\paragraph}{som}{%
  \IfBooleanTF{#1}
    {\latexparagraph*{#3}}
    {\IfNoValueTF{#2}
       {\latexparagraph{\maybe@addperiod{#3}}}
       {\latexparagraph[#2]{\maybe@addperiod{#3}}}%
  }%
}
\newcommand{\maybe@addperiod}[1]{%
  #1\@addpunct{.}%
}
\makeatother


\newcommand{\Authornote}[2]{{\sffamily\small\color{red}{[#1: #2]}}}
\newcommand{\Authornotecolored}[3]{{\sffamily\small\color{#1}{[#2: #3]}}}
\newcommand{\Authorcomment}[2]{{\sffamily\small\color{gray}{[#1: #2]}}}
\newcommand{\Authorfnote}[2]{\footnote{\color{red}{#1: #2}}}
\newcommand{\Jnote}{\Authornote{J}}
\newcommand{\Jcomment}{\Authorcomment{J}}
\newcommand{\Jfnote}{\Authorfnote{J}}
\newcommand{\Rnote}{\Authornote{R}}
\newcommand{\Rcomment}{\Authorcomment{R}}
\newcommand{\Rfnote}{\Authorfnote{R}}
\newcommand{\Dnote}{\Authornote{D}}
\newcommand{\Dcomment}{\Authorcomment{D}}
\newcommand{\Dfnote}{\Authorfnote{D}}
\newcommand{\Ynote}{\Authornote{Y}}
\newcommand{\Ycomment}{\Authorcomment{Y}}
\newcommand{\Yfnote}{\Authorfnote{Y}}
\usepackage{boxedminipage}
\newcommand{\paren}[1]{(#1)}
\newcommand{\Paren}[1]{\left(#1\right)}
\newcommand{\bigparen}[1]{\big(#1\big)}
\newcommand{\Bigparen}[1]{\Big(#1\Big)}
\newcommand{\brac}[1]{[#1]}
\newcommand{\Brac}[1]{\left[#1\right]}
\newcommand{\bigbrac}[1]{\big[#1\big]}
\newcommand{\Bigbrac}[1]{\Big[#1\Big]}
\newcommand{\Biggbrac}[1]{\Bigg[#1\Bigg]}
\newcommand{\abs}[1]{\lvert#1\rvert}
\newcommand{\Abs}[1]{\left\lvert#1\right\rvert}
\newcommand{\bigabs}[1]{\big\lvert#1\big\rvert}
\newcommand{\Bigabs}[1]{\Big\lvert#1\Big\rvert}
\newcommand{\card}[1]{\lvert#1\rvert}
\newcommand{\Card}[1]{\left\lvert#1\right\rvert}
\newcommand{\bigcard}[1]{\big\lvert#1\big\rvert}
\newcommand{\Bigcard}[1]{\Big\lvert#1\Big\rvert}
\newcommand{\set}[1]{\{#1\}}
\newcommand{\Set}[1]{\left\{#1\right\}}
\newcommand{\bigset}[1]{\big\{#1\big\}}
\newcommand{\Bigset}[1]{\Big\{#1\Big\}}
\newcommand{\norm}[1]{\lVert#1\rVert}
\newcommand{\Norm}[1]{\left\lVert#1\right\rVert}
\newcommand{\bignorm}[1]{\big\lVert#1\big\rVert}
\newcommand{\Bignorm}[1]{\Big\lVert#1\Big\rVert}
\newcommand{\Biggnorm}[1]{\Bigg\lVert#1\Bigg\rVert}
\newcommand{\normt}[1]{\norm{#1}_2}
\newcommand{\Normt}[1]{\Norm{#1}_2}
\newcommand{\bignormt}[1]{\bignorm{#1}_2}
\newcommand{\Bignormt}[1]{\Bignorm{#1}_2}
\newcommand{\snormt}[1]{\norm{#1}^2_2}
\newcommand{\Snormt}[1]{\Norm{#1}^2_2}
\newcommand{\bigsnormt}[1]{\bignorm{#1}^2_2}
\newcommand{\Bigsnormt}[1]{\Bignorm{#1}^2_2}
\newcommand{\snorm}[1]{\norm{#1}^2}
\newcommand{\Snorm}[1]{\Norm{#1}^2}
\newcommand{\bigsnorm}[1]{\bignorm{#1}^2}
\newcommand{\Bigsnorm}[1]{\Bignorm{#1}^2}
\newcommand{\normo}[1]{\norm{#1}_1}
\newcommand{\Normo}[1]{\Norm{#1}_1}
\newcommand{\bignormo}[1]{\bignorm{#1}_1}
\newcommand{\Bignormo}[1]{\Bignorm{#1}_1}
\newcommand{\normi}[1]{\norm{#1}_\infty}
\newcommand{\Normi}[1]{\Norm{#1}_\infty}
\newcommand{\bignormi}[1]{\bignorm{#1}_\infty}
\newcommand{\Bignormi}[1]{\Bignorm{#1}_\infty}
\newcommand{\iprod}[1]{\langle#1\rangle}
\newcommand{\Iprod}[1]{\left\langle#1\right\rangle}
\newcommand{\bigiprod}[1]{\big\langle#1\big\rangle}
\newcommand{\Bigiprod}[1]{\Big\langle#1\Big\rangle}
\newcommand{\Esymb}{\mathbb{E}}
\newcommand{\Psymb}{\mathbb{P}}
\newcommand{\Vsymb}{\mathbb{V}}
\DeclareMathOperator*{\E}{\Esymb}
\DeclareMathOperator*{\Var}{\Vsymb}
\DeclareMathOperator*{\ProbOp}{\Psymb}
\renewcommand{\Pr}{\ProbOp}
\newcommand{\given}{\mathrel{}\middle\vert\mathrel{}}
\newcommand{\suchthat}{\;\middle\vert\;}
\newcommand{\tensor}{\otimes}
\newcommand{\where}{\text{where}}
\newcommand{\textparen}[1]{\text{(#1)}}
\newcommand{\using}[1]{\textparen{using #1}}
\newcommand{\smallusing}[1]{\text{(\small using #1)}}
\newcommand{\by}[1]{\textparen{by #1}}
\newcommand{\sge}{\succeq}
\newcommand{\sle}{\preceq}
\newcommand{\lmin}{\lambda_{\min}}
\newcommand{\lmax}{\lambda_{\max}}
\newcommand{\signs}{\set{1,-1}}
\newcommand{\varsigns}{\set{\pm 1}}
\newcommand{\maximize}{\mathop{\textrm{maximize}}}
\newcommand{\minimize}{\mathop{\textrm{minimize}}}
\newcommand{\subjectto}{\mathop{\textrm{subject to}}}
\renewcommand{\ij}{{ij}}
\newcommand{\symdiff}{\Delta}
\newcommand{\varsymdiff}{\bigtriangleup}
\newcommand{\bits}{\{0,1\}}
\newcommand{\sbits}{\{\pm1\}}
\newcommand{\listoptions}{\labelsep0mm\topsep-0mm\itemindent-6mm\itemsep0mm}
\newcommand{\displayoptions}[1]{\abovedisplayshortskip#1mm\belowdisplayshortskip#1mm\abovedisplayskip#1mm\belowdisplayskip#1mm}
\newcommand{\super}[2]{#1^{\paren{#2}}}
\newcommand{\varsuper}[2]{#1^{\scriptscriptstyle\paren{#2}}}
\newcommand{\tensorpower}[2]{#1^{\tensor #2}}
\newcommand{\inv}[1]{{#1^{-1}}}
\newcommand{\dual}[1]{{#1^*}}
\newcommand{\vbig}{\vphantom{\bigoplus}}
\newcommand{\sm}{\setminus}
\newcommand{\defeq}{\stackrel{\mathrm{def}}=}
\newcommand{\seteq}{\mathrel{\mathop:}=}
\newcommand{\from}{\colon}
\newcommand{\bigmid}{~\big|~}
\newcommand{\Bigmid}{~\Big|~}
\newcommand{\Mid}{\nonscript\;\middle\vert\nonscript\;}
\newcommand{\bvec}[1]{\bar{\vec{#1}}}
\newcommand{\pvec}[1]{\vec{#1}'}
\newcommand{\tvec}[1]{{\tilde{\vec{#1}}}}
\newcommand{\mper}{\,.}
\newcommand{\mcom}{\,,}
\newcommand{\bbG}{\mathbb{G}}
\newcommand\bdot\bullet
\newcommand{\trsp}[1]{{#1}^\dagger}
\DeclareMathOperator{\Ind}{\mathbf 1}
\DeclareMathOperator{\Inf}{Inf}
\DeclareMathOperator{\Tr}{Tr}
\DeclareMathOperator{\SDP}{SDP}
\DeclareMathOperator{\sdp}{sdp}
\DeclareMathOperator{\val}{val}
\DeclareMathOperator{\LP}{LP}
\DeclareMathOperator{\OPT}{OPT}
\DeclareMathOperator{\opt}{opt}
\DeclareMathOperator{\vol}{vol}
\DeclareMathOperator{\poly}{poly}
\DeclareMathOperator{\qpoly}{qpoly}
\DeclareMathOperator{\qpolylog}{qpolylog}
\DeclareMathOperator{\qqpoly}{qqpoly}
\DeclareMathOperator{\argmax}{argmax}
\DeclareMathOperator{\polylog}{polylog}
\DeclareMathOperator{\supp}{supp}
\DeclareMathOperator{\dist}{dist}
\DeclareMathOperator{\sign}{sign}
\DeclareMathOperator{\conv}{conv}
\DeclareMathOperator{\Conv}{Conv}
\DeclareMathOperator{\rank}{rank}

\DeclareMathOperator*{\argmin}{arg\,min}
\DeclareMathOperator*{\median}{median}
\DeclareMathOperator*{\Median}{Median}
\DeclareMathOperator*{\varsum}{{\textstyle \sum}}
\DeclareMathOperator{\tsum}{{\textstyle \sum}}
\let\varprod\undefined
\DeclareMathOperator*{\varprod}{{\textstyle \prod}}
\DeclareMathOperator{\tprod}{{\textstyle \prod}}
\newcommand{\ie}{i.e.,\xspace}
\newcommand{\eg}{e.g.,\xspace}
\newcommand{\Eg}{E.g.,\xspace}
\newcommand{\phd}{Ph.\,D.\xspace}
\newcommand{\msc}{M.\,S.\xspace}
\newcommand{\bsc}{B.\,S.\xspace}
\newcommand{\etal}{et al.\xspace}
\newcommand{\iid}{i.i.d.\xspace}
\newcommand\naive{na\"{\i}ve\xspace}
\newcommand\Naive{Na\"{\i}ve\xspace}
\newcommand\naively{na\"{\i}vely\xspace}
\newcommand\Naively{Na\"{\i}vely\xspace}
\newcommand{\Erdos}{Erd\H{o}s\xspace}
\newcommand{\Renyi}{R\'enyi\xspace}
\newcommand{\Lovasz}{Lov\'asz\xspace}
\newcommand{\Juhasz}{Juh\'asz\xspace}
\newcommand{\Bollobas}{Bollob\'as\xspace}
\newcommand{\Furedi}{F\"uredi\xspace}
\newcommand{\Komlos}{Koml\'os\xspace}
\newcommand{\Luczak}{\L uczak\xspace}
\newcommand{\Kucera}{Ku\v{c}era\xspace}
\newcommand{\Szemeredi}{Szemer\'edi\xspace}
\newcommand{\Hastad}{H{\aa}stad\xspace}
\newcommand{\Hoelder}{H\"{o}lder\xspace}
\newcommand{\Holder}{\Hoelder}
\newcommand{\Brandao}{Brand\~ao\xspace}
\newcommand{\Z}{\mathbb Z}
\newcommand{\N}{\mathbb N}
\newcommand{\R}{\mathbb R}
\newcommand{\C}{\mathbb C}
\newcommand{\Rnn}{\R_+}
\newcommand{\varR}{\Re}
\newcommand{\varRnn}{\varR_+}
\newcommand{\varvarRnn}{\R_{\ge 0}}
\newcommand{\problemmacro}[1]{\texorpdfstring{\textup{\textsc{#1}}}{#1}\xspace}
\newcommand{\pnum}[1]{{\footnotesize #1}}
\newcommand{\uniquegames}{\problemmacro{unique games}}
\newcommand{\maxcut}{\problemmacro{max cut}}
\newcommand{\multicut}{\problemmacro{multi cut}}
\newcommand{\vertexcover}{\problemmacro{vertex cover}}
\newcommand{\balancedseparator}{\problemmacro{balanced separator}}
\newcommand{\maxtwosat}{\problemmacro{max \pnum{3}-sat}}
\newcommand{\maxthreesat}{\problemmacro{max \pnum{3}-sat}}
\newcommand{\maxthreelin}{\problemmacro{max \pnum{3}-lin}}
\newcommand{\threesat}{\problemmacro{\pnum{3}-sat}}
\newcommand{\labelcover}{\problemmacro{label cover}}
\newcommand{\setcover}{\problemmacro{set cover}}
\newcommand{\maxksat}{\problemmacro{max $k$-sat}}
\newcommand{\mas}{\problemmacro{maximum acyclic subgraph}}
\newcommand{\kwaycut}{\problemmacro{$k$-way cut}}
\newcommand{\sparsestcut}{\problemmacro{sparsest cut}}
\newcommand{\betweenness}{\problemmacro{betweenness}}
\newcommand{\uniformsparsestcut}{\problemmacro{uniform sparsest cut}}
\newcommand{\grothendieckproblem}{\problemmacro{Grothendieck problem}}
\newcommand{\maxfoursat}{\problemmacro{max \pnum{4}-sat}}
\newcommand{\maxkcsp}{\problemmacro{max $k$-csp}}
\newcommand{\maxdicut}{\problemmacro{max dicut}}
\newcommand{\maxcutgain}{\problemmacro{max cut gain}}
\newcommand{\smallsetexpansion}{\problemmacro{small-set expansion}}
\newcommand{\minbisection}{\problemmacro{min bisection}}
\newcommand{\minimumlineararrangement}{\problemmacro{minimum linear arrangement}}
\newcommand{\maxtwolin}{\problemmacro{max \pnum{2}-lin}}
\newcommand{\gammamaxlin}{\problemmacro{$\Gamma$-max \pnum{2}-lin}}
\newcommand{\basicsdp}{\problemmacro{basic sdp}}
\newcommand{\dgames}{\problemmacro{$d$-to-1 games}}
\newcommand{\maxclique}{\problemmacro{max clique}}
\newcommand{\densestksubgraph}{\problemmacro{densest $k$-subgraph}}
\newcommand{\cA}{\mathcal A}
\newcommand{\cB}{\mathcal B}
\newcommand{\cC}{\mathcal C}
\newcommand{\cD}{\mathcal D}
\newcommand{\cE}{\mathcal E}
\newcommand{\cF}{\mathcal F}
\newcommand{\cG}{\mathcal G}
\newcommand{\cH}{\mathcal H}
\newcommand{\cI}{\mathcal I}
\newcommand{\cJ}{\mathcal J}
\newcommand{\cK}{\mathcal K}
\newcommand{\cL}{\mathcal L}
\newcommand{\cM}{\mathcal M}
\newcommand{\cN}{\mathcal N}
\newcommand{\cO}{\mathcal O}
\newcommand{\cP}{\mathcal P}
\newcommand{\cQ}{\mathcal Q}
\newcommand{\cR}{\mathcal R}
\newcommand{\cS}{\mathcal S}
\newcommand{\cT}{\mathcal T}
\newcommand{\cU}{\mathcal U}
\newcommand{\cV}{\mathcal V}
\newcommand{\cW}{\mathcal W}
\newcommand{\cX}{\mathcal X}
\newcommand{\cY}{\mathcal Y}
\newcommand{\cZ}{\mathcal Z}
\newcommand{\scrA}{\mathscr A}
\newcommand{\scrB}{\mathscr B}
\newcommand{\scrC}{\mathscr C}
\newcommand{\scrD}{\mathscr D}
\newcommand{\scrE}{\mathscr E}
\newcommand{\scrF}{\mathscr F}
\newcommand{\scrG}{\mathscr G}
\newcommand{\scrH}{\mathscr H}
\newcommand{\scrI}{\mathscr I}
\newcommand{\scrJ}{\mathscr J}
\newcommand{\scrK}{\mathscr K}
\newcommand{\scrL}{\mathscr L}
\newcommand{\scrM}{\mathscr M}
\newcommand{\scrN}{\mathscr N}
\newcommand{\scrO}{\mathscr O}
\newcommand{\scrP}{\mathscr P}
\newcommand{\scrQ}{\mathscr Q}
\newcommand{\scrR}{\mathscr R}
\newcommand{\scrS}{\mathscr S}
\newcommand{\scrT}{\mathscr T}
\newcommand{\scrU}{\mathscr U}
\newcommand{\scrV}{\mathscr V}
\newcommand{\scrW}{\mathscr W}
\newcommand{\scrX}{\mathscr X}
\newcommand{\scrY}{\mathscr Y}
\newcommand{\scrZ}{\mathscr Z}
\newcommand{\bbB}{\mathbb B}
\newcommand{\bbS}{\mathbb S}
\newcommand{\bbR}{\mathbb R}
\newcommand{\bbZ}{\mathbb Z}
\newcommand{\bbI}{\mathbb I}
\newcommand{\bbQ}{\mathbb Q}
\newcommand{\bbP}{\mathbb P}
\newcommand{\bbE}{\mathbb E}
\newcommand{\bbN}{\mathbb N}
\newcommand{\sfE}{\mathsf E}
\newcommand{\RPQ}{R_{P,Q}}

\renewcommand{\leq}{\leqslant}
\renewcommand{\le}{\leqslant}
\renewcommand{\geq}{\geqslant}
\renewcommand{\ge}{\geqslant}
\let\epsilon=\varepsilon
\numberwithin{equation}{section}
\newcommand\MYcurrentlabel{xxx}
\newcommand{\MYstore}[2]{%
  \global\expandafter \def \csname MYMEMORY #1 \endcsname{#2}%
}
\newcommand{\MYload}[1]{%
  \csname MYMEMORY #1 \endcsname%
}
\newcommand{\MYnewlabel}[1]{%
  \renewcommand\MYcurrentlabel{#1}%
  \MYoldlabel{#1}%
}
\newcommand{\MYdummylabel}[1]{}
\newcommand{\torestate}[1]{%
  \let\MYoldlabel\label%
  \let\label\MYnewlabel%
  #1%
  \MYstore{\MYcurrentlabel}{#1}%
  \let\label\MYoldlabel%
}
\newcommand{\restatetheorem}[1]{%
  \let\MYoldlabel\label
  \let\label\MYdummylabel
  \begin{theorem*}[Restatement of \cref{#1}]
    \MYload{#1}
  \end{theorem*}
  \let\label\MYoldlabel
}
\newcommand{\restatelemma}[1]{%
  \let\MYoldlabel\label
  \let\label\MYdummylabel
  \begin{lemma*}[Restatement of \cref{#1}]
    \MYload{#1}
  \end{lemma*}
  \let\label\MYoldlabel
}
\newcommand{\restateprop}[1]{%
  \let\MYoldlabel\label
  \let\label\MYdummylabel
  \begin{proposition*}[Restatement of \cref{#1}]
    \MYload{#1}
  \end{proposition*}
  \let\label\MYoldlabel
}
\newcommand{\restatefact}[1]{%
  \let\MYoldlabel\label
  \let\label\MYdummylabel
  \begin{fact*}[Restatement of \cref{#1}]
    \MYload{#1}
  \end{fact*}
  \let\label\MYoldlabel
}
\newcommand{\restate}[1]{%
  \let\MYoldlabel\label
  \let\label\MYdummylabel
  \MYload{#1}
  \let\label\MYoldlabel
}
\newcommand{\la}{\leftarrow}
\newcommand{\sse}{\subseteq}
\newcommand{\ra}{\rightarrow}
\newcommand{\e}{\epsilon}
\newcommand{\eps}{\epsilon}
\newcommand{\eset}{\emptyset}
\allowdisplaybreaks
\sloppy
\newcommand*{\Id}{\mathrm{Id}}
\newcommand*{\Lowner}{L\"owner\xspace}
\newcommand*{\normop}[1]{\norm{#1}_{\mathrm{op}}}
\newcommand*{\Normop}[1]{\Norm{#1}_{\mathrm{op}}}
\newcommand*{\Normtv}[1]{\Norm{#1}_{\mathrm{TV}}}
\newcommand*{\normtv}[1]{\norm{#1}_{\mathrm{TV}}}
\newcommand*{\normf}[1]{\norm{#1}_{\mathrm{F}}}
\newcommand*{\Normf}[1]{\Norm{#1}_{\mathrm{F}}}
\newcommand{\bsaw}[2]{\text{BSAW}_{#1,#2}}
\newcommand{\normn}[1]{\norm{#1}_\textnormal{nuc}}
\newcommand{\Normn}[1]{\Norm{#1}_\textnormal{nuc}}
\newcommand{\bignormn}[1]{\bignorm{#1}_\textnormal{nuc}}
\newcommand{\Bignormn}[1]{\Bignorm{#1}_\textnormal{nuc}}
\newcommand{\normm}[1]{\norm{#1}_\textnormal{max}}
\newcommand{\Normm}[1]{\Norm{#1}_\textnormal{max}}
\newcommand{\bignormm}[1]{\bignorm{#1}_\textnormal{max}}
\newcommand{\Bignormm}[1]{\Bignorm{#1}_\textnormal{max}}
\newcommand{\spike}[1]{\alpha_{\textnormal{sp}}\Paren{#1} }
\newcommand{\ind}[1]{\mathbf{1}_{\Brac{#1}}}
\newcommand{\one}{\mathbb{1}}
\newcommand{\zero}{\mathbb{0}}
\newcommand{\bbM}{\mathbb M}
\newcommand{\sskip}{\smallskip}
\newcommand{\bskip}{\bigskip}

\newenvironment{algorithmbox}{\begin{mdframed}[nobreak=true]
\begin{algorithm}}{\end{algorithm}\end{mdframed}}
\providecommand{\Yij}{Y_{ij}}
\providecommand{\Wij}{W_{ij}}
\providecommand{\Zij}{Z_{ij}}
\providecommand{\Xij}{X_{ij}}
\providecommand{\yij}{y_{ij}}
\providecommand{\wij}{w_{ij}}
\providecommand{\zij}{z_{ij}}
\providecommand{\xij}{x_{ij}}
\providecommand{\todo}{{\color{red}{\textbf{TODO }}}}
\providecommand{\tochange}{{\color{green}{\textbf{TOCHANGE }}}}
\providecommand{\toexpand}{{\color{blue}{\textbf{TOEXPAND }}}}
\providecommand{\nulld}{\textit{nl}}
\providecommand{\plantedd}{\textit{pl}}
\providecommand{\Ep}{\E_{\plantedd}}
\providecommand{\En}{\E_{\nulld}}
\newcommand{\saw}[2]{\text{SAW}^{#2}_{#1}}
\newcommand{\nbw}[2]{\text{NBW}^{#2}_{#1}}
\newcommand{\sbm}{\text{sbm}_{d,\eps}}
\newcommand{\SBM}{\mathsf{SBM}}
\newcommand{\Esbm}{\E_{\text{sbm}}}
\newcommand{\Q}{Q^{(s)}}
\renewcommand{\normo}[1]{\norm{#1}_{\text{sum}}}
\renewcommand{\Normo}[1]{\Norm{#1}_{\text{sum}}}
\renewcommand{\bignormo}[1]{\bignorm{#1}__{\text{sum}}}
\renewcommand{\Bignormo}[1]{\Bignorm{#1}_{\text{sum}}}
\renewcommand{\normi}[1]{\norm{#1}_{\max}}
\renewcommand{\Normi}[1]{\Norm{#1}_{\max}}
\renewcommand{\bignormi}[1]{\bignorm{#1}_{\max}}
\renewcommand{\Bignormi}[1]{\Bignorm{#1}_{\max}}
\DeclareMathOperator{\diag}{diag}
\DeclareMathOperator{\oan}{\overline{\text{an}}}
\DeclareMathOperator{\opa}{\overline{\text{pa}}}
\DeclareMathOperator{\pE}{\tilde{\mathbb{E}}}
\DeclareMathOperator{\Span}{Span}
\DeclareMathOperator{\sspan}{Span}
\DeclarePairedDelimiter\ceil{\lceil}{\rceil}
\DeclarePairedDelimiter\floor{\lfloor}{\rfloor}
\newcommand*{\transpose}[1]{{#1}{}^{\mkern-1.5mu\mathsf{T}}}
\newcommand*{\dyad}[1]{#1#1{}^{\mkern-1.5mu\mathsf{T}}}

\newcommand{\SSBM}{\text{SSBM}}
\newcommand{\Bnull}{B^\circ}
\newcommand{\thetanull}{\theta^\circ}
\newcommand{\Wnull}{W^\circ}
\newcommand{\GW}{\text{GW}}

\def\layersep{3cm}
\def\nodeinlayersep{1.5cm}

\title{Low degree conjecture implies sharp computational thresholds in stochastic block model\thanks{This work is supported by funding from the European Research Council (ERC) under the European Union’s Horizon 2020 research and innovation programme (grant agreement No 815464).}}


\author{
Jingqiu Ding\thanks{ETH Z\"urich.}
\and 
Yiding Hua\footnotemark[2]
\and
Lucas Slot\footnotemark[2]
\and
David Steurer\footnotemark[2]
}

\begin{document}

\pagestyle{empty}


\maketitle




\begin{abstract}  
Test time scaling is currently one of the most active research areas that shows promise after training time scaling has reached its limits.
Deep-thinking (DT) models are a class of recurrent models that can perform easy-to-hard generalization by assigning more compute to harder test samples.
However, due to their inability to determine the complexity of a test sample, DT models have to use a large amount of computation for both easy and hard test samples.
Excessive test time computation is wasteful and can cause the ``overthinking'' problem where more test time computation leads to worse results.
In this paper, we introduce a test time training method for determining the optimal amount of computation needed for each sample during test time.
We also propose Conv-LiGRU, a novel recurrent architecture for efficient and robust visual reasoning. 
Extensive experiments demonstrate that Conv-LiGRU is more stable than DT, effectively mitigates the ``overthinking'' phenomenon, and achieves superior accuracy.
\end{abstract}  




\microtypesetup{protrusion=false}
\tableofcontents{}
\microtypesetup{protrusion=true}


\clearpage

\pagestyle{plain}
\setcounter{page}{1}


\section{Introduction}
\label{sec:introduction}
The business processes of organizations are experiencing ever-increasing complexity due to the large amount of data, high number of users, and high-tech devices involved \cite{martin2021pmopportunitieschallenges, beerepoot2023biggestbpmproblems}. This complexity may cause business processes to deviate from normal control flow due to unforeseen and disruptive anomalies \cite{adams2023proceddsriftdetection}. These control-flow anomalies manifest as unknown, skipped, and wrongly-ordered activities in the traces of event logs monitored from the execution of business processes \cite{ko2023adsystematicreview}. For the sake of clarity, let us consider an illustrative example of such anomalies. Figure \ref{FP_ANOMALIES} shows a so-called event log footprint, which captures the control flow relations of four activities of a hypothetical event log. In particular, this footprint captures the control-flow relations between activities \texttt{a}, \texttt{b}, \texttt{c} and \texttt{d}. These are the causal ($\rightarrow$) relation, concurrent ($\parallel$) relation, and other ($\#$) relations such as exclusivity or non-local dependency \cite{aalst2022pmhandbook}. In addition, on the right are six traces, of which five exhibit skipped, wrongly-ordered and unknown control-flow anomalies. For example, $\langle$\texttt{a b d}$\rangle$ has a skipped activity, which is \texttt{c}. Because of this skipped activity, the control-flow relation \texttt{b}$\,\#\,$\texttt{d} is violated, since \texttt{d} directly follows \texttt{b} in the anomalous trace.
\begin{figure}[!t]
\centering
\includegraphics[width=0.9\columnwidth]{images/FP_ANOMALIES.png}
\caption{An example event log footprint with six traces, of which five exhibit control-flow anomalies.}
\label{FP_ANOMALIES}
\end{figure}

\subsection{Control-flow anomaly detection}
Control-flow anomaly detection techniques aim to characterize the normal control flow from event logs and verify whether these deviations occur in new event logs \cite{ko2023adsystematicreview}. To develop control-flow anomaly detection techniques, \revision{process mining} has seen widespread adoption owing to process discovery and \revision{conformance checking}. On the one hand, process discovery is a set of algorithms that encode control-flow relations as a set of model elements and constraints according to a given modeling formalism \cite{aalst2022pmhandbook}; hereafter, we refer to the Petri net, a widespread modeling formalism. On the other hand, \revision{conformance checking} is an explainable set of algorithms that allows linking any deviations with the reference Petri net and providing the fitness measure, namely a measure of how much the Petri net fits the new event log \cite{aalst2022pmhandbook}. Many control-flow anomaly detection techniques based on \revision{conformance checking} (hereafter, \revision{conformance checking}-based techniques) use the fitness measure to determine whether an event log is anomalous \cite{bezerra2009pmad, bezerra2013adlogspais, myers2018icsadpm, pecchia2020applicationfailuresanalysispm}. 

The scientific literature also includes many \revision{conformance checking}-independent techniques for control-flow anomaly detection that combine specific types of trace encodings with machine/deep learning \cite{ko2023adsystematicreview, tavares2023pmtraceencoding}. Whereas these techniques are very effective, their explainability is challenging due to both the type of trace encoding employed and the machine/deep learning model used \cite{rawal2022trustworthyaiadvances,li2023explainablead}. Hence, in the following, we focus on the shortcomings of \revision{conformance checking}-based techniques to investigate whether it is possible to support the development of competitive control-flow anomaly detection techniques while maintaining the explainable nature of \revision{conformance checking}.
\begin{figure}[!t]
\centering
\includegraphics[width=\columnwidth]{images/HIGH_LEVEL_VIEW.png}
\caption{A high-level view of the proposed framework for combining \revision{process mining}-based feature extraction with dimensionality reduction for control-flow anomaly detection.}
\label{HIGH_LEVEL_VIEW}
\end{figure}

\subsection{Shortcomings of \revision{conformance checking}-based techniques}
Unfortunately, the detection effectiveness of \revision{conformance checking}-based techniques is affected by noisy data and low-quality Petri nets, which may be due to human errors in the modeling process or representational bias of process discovery algorithms \cite{bezerra2013adlogspais, pecchia2020applicationfailuresanalysispm, aalst2016pm}. Specifically, on the one hand, noisy data may introduce infrequent and deceptive control-flow relations that may result in inconsistent fitness measures, whereas, on the other hand, checking event logs against a low-quality Petri net could lead to an unreliable distribution of fitness measures. Nonetheless, such Petri nets can still be used as references to obtain insightful information for \revision{process mining}-based feature extraction, supporting the development of competitive and explainable \revision{conformance checking}-based techniques for control-flow anomaly detection despite the problems above. For example, a few works outline that token-based \revision{conformance checking} can be used for \revision{process mining}-based feature extraction to build tabular data and develop effective \revision{conformance checking}-based techniques for control-flow anomaly detection \cite{singh2022lapmsh, debenedictis2023dtadiiot}. However, to the best of our knowledge, the scientific literature lacks a structured proposal for \revision{process mining}-based feature extraction using the state-of-the-art \revision{conformance checking} variant, namely alignment-based \revision{conformance checking}.

\subsection{Contributions}
We propose a novel \revision{process mining}-based feature extraction approach with alignment-based \revision{conformance checking}. This variant aligns the deviating control flow with a reference Petri net; the resulting alignment can be inspected to extract additional statistics such as the number of times a given activity caused mismatches \cite{aalst2022pmhandbook}. We integrate this approach into a flexible and explainable framework for developing techniques for control-flow anomaly detection. The framework combines \revision{process mining}-based feature extraction and dimensionality reduction to handle high-dimensional feature sets, achieve detection effectiveness, and support explainability. Notably, in addition to our proposed \revision{process mining}-based feature extraction approach, the framework allows employing other approaches, enabling a fair comparison of multiple \revision{conformance checking}-based and \revision{conformance checking}-independent techniques for control-flow anomaly detection. Figure \ref{HIGH_LEVEL_VIEW} shows a high-level view of the framework. Business processes are monitored, and event logs obtained from the database of information systems. Subsequently, \revision{process mining}-based feature extraction is applied to these event logs and tabular data input to dimensionality reduction to identify control-flow anomalies. We apply several \revision{conformance checking}-based and \revision{conformance checking}-independent framework techniques to publicly available datasets, simulated data of a case study from railways, and real-world data of a case study from healthcare. We show that the framework techniques implementing our approach outperform the baseline \revision{conformance checking}-based techniques while maintaining the explainable nature of \revision{conformance checking}.

In summary, the contributions of this paper are as follows.
\begin{itemize}
    \item{
        A novel \revision{process mining}-based feature extraction approach to support the development of competitive and explainable \revision{conformance checking}-based techniques for control-flow anomaly detection.
    }
    \item{
        A flexible and explainable framework for developing techniques for control-flow anomaly detection using \revision{process mining}-based feature extraction and dimensionality reduction.
    }
    \item{
        Application to synthetic and real-world datasets of several \revision{conformance checking}-based and \revision{conformance checking}-independent framework techniques, evaluating their detection effectiveness and explainability.
    }
\end{itemize}

The rest of the paper is organized as follows.
\begin{itemize}
    \item Section \ref{sec:related_work} reviews the existing techniques for control-flow anomaly detection, categorizing them into \revision{conformance checking}-based and \revision{conformance checking}-independent techniques.
    \item Section \ref{sec:abccfe} provides the preliminaries of \revision{process mining} to establish the notation used throughout the paper, and delves into the details of the proposed \revision{process mining}-based feature extraction approach with alignment-based \revision{conformance checking}.
    \item Section \ref{sec:framework} describes the framework for developing \revision{conformance checking}-based and \revision{conformance checking}-independent techniques for control-flow anomaly detection that combine \revision{process mining}-based feature extraction and dimensionality reduction.
    \item Section \ref{sec:evaluation} presents the experiments conducted with multiple framework and baseline techniques using data from publicly available datasets and case studies.
    \item Section \ref{sec:conclusions} draws the conclusions and presents future work.
\end{itemize}
\section{Result} \label{sec:result}

\subsection{Setup}

In this section, we evaluate VB-Com across the following perspectives:
\begin{itemize}
    \item Under what conditions does VB-Com demonstrate superior performance compared to using a single-policy approach?
    \item How does VB-Com outperforms baseline methods in those scenarios?
    \item How well does the proposed return estimator contribute to the composition system?
\end{itemize}

\begin{figure}[h]
\centering{\includegraphics[width=0.5\textwidth]{figures/noise.png}}
\caption{We present four types of perception noises and implement them on heightmaps during evaluation: gaussian noise, \textcolor{red}{forward shifting noise}, \textcolor{green}{lateral shifting noise} and \textcolor{blue}{floating noise}.}
\label{noise}
\end{figure}

\subsubsection{Evaluation Noise}
To simulate situations where the robot encounters perception outliers not present in the simulation, we introduce a quantitative curriculum noise designed to mimic varying levels of perception deficiency. As shown in Fig. \ref{noise}, we focus on four types of noise: (1) \textbf{Gaussian noise}: noise points sampled from a Gaussian distribution, to the original heightmap. The noise level is scaled from 0.0 to 1.0, where the training noise level corresponds to a 0.1 noise level in this scenario. (2) \textbf{Shifting noise}: replacing points in the original heightmap with noise sampled from a Gaussian distribution. The range of replacement points is controlled by the noise level, where a $100\%$ noise level results in a fully noisy heightmap. The shifting direction can either be along the heading direction (red box) or sideways (green box). (3) \textbf{Floating noise}: The heightmap is displaced vertically, either upwards or downwards, the floating noise simulates variations in terrain height. (blue box).

\begin{table}[!ht]
\caption{Terrain Size Scales (m)}
\label{tab:terrains}
\begin{center}
\renewcommand\arraystretch{1.25}
\begin{tabular}{lcccc}
\toprule[1.0pt]
Terrain & Length & Width & Heights\\
\midrule[0.8pt]

Gaps        & $(0.6, 1.2)$ & $(\bm{0.6}, \bm{0.8})$ & $(-1.8, -1.5)$\\  
Hurdles     & $(0.8, 1.0)$ & $(0.1, 0.2)$ & $(\bm{0.2}, \bm{0.4})$\\  
Obstacles   & $(\bm{0.2}, \bm{0.4})$ & $(0.2, 0.4)$ & $(1.4,1.8)$\\  

\bottomrule[1.0pt]
\end{tabular}
\end{center}
\end{table}

\subsubsection{Experiments Setup}
In simulation, we conduct $10 \times 3$ experiments for each method across three types of terrain, replicating the experiments three times to calculate the variance. Each episode involves the robot navigating through 8 goal points, with each goal paired with a corresponding challenging terrain or obstacle. The size of the terrains is set to the maximum curriculum terrain level, as shown in Table \ref{tab:terrains}. The bolded values indicate the primary factors that contribute to the difficulty for the terrain.

\subsubsection{Baselines}
We primarily compare VB-Com with the vision and blind policies operating independently. Additionally, as previous works have shown that robust perceptive locomotion can be learned by incorporating various perception noises during training \cite{miki2022learning}, we add a \textbf{Noisy Perceptive policy baseline} trained using the same noises implemented in the evaluation. This allows us to examine how well the proposed VB-Com policy performs compared to policies that have already seen the evaluation noises. The evaluation noises are introduced to the Noisy Perceptive policy in a curriculum format during training, which evolves with the terrain level.

\begin{figure*}[h]
\centering{\includegraphics[width=\textwidth]{figures/returnsim.png}}
\caption{Illustrations of the variation in estimated return and action phases(0 for $a_b$ and 1 for $a_v$) across three concerned terrains.}
\label{return}
\end{figure*}

\subsection{Example Case}
First, we illustrate how VB-Com operates, specifically when the composition switches to $\pi_b$ and how it effectively controls the robot to traverse the terrain against deficient perception (Fig. \ref{return}). We demonstrate $3$ seconds of the estimated returns, along with the policy composition phase, as the robot walking through the challenging terrain during the simulation experiments at the noise level of $100\%$. Before the robot encounters challenging terrains, we observe that the estimated return $G^e_{\pi_v}(s_t)$ consistently exceeds $G^e_{\pi_b}(s_t)$, as the robot is walking on flat ground with relatively stable motion. This observation aligns with the discussion in Section \ref{subsec:vb-com}, where it was explained that $\pi_v$ benefits from the external state observation and results in a higher return $G_t$. This characteraistic ensures the robot operates at $\pi_b$ while stable motion. 

Once the deficient perception reaches the $100\%$ noise level, the robot will not be aware of the incoming challenging terrains until it collides with them. At this point, we observe that $G^e_{\pi}(s_t)$ drops sharply within several control steps, prompting the switch to the blind policy. This switch allows the robot to respond to the terrain, and once the motion stabilizes, $G^e_{\pi}(s_t)$ returns to a normal level, at which point the vision policy regains control. These cases demonstrate the effectiveness of VB-Com, which responds quickly to deficient perception, but avoids unnecessary switches to the blind policy when it is not needed.


\begin{table*}[!h]
\caption{VB-Com Evaluations}
\label{tab:VB-Com}
\begin{center}
\renewcommand\arraystretch{1.25}
\begin{tabular}{lccccccc}
\toprule[1.0pt]
Noise Level &Method & Goals Completed($\%$) & Rewards & Average Velocity & Fail Rate & Collision Steps($\%$) & Reach Steps\\
\midrule[0.8pt]

% \multirow{4}{*}{Prop Advisor}&0.25& $0.7560$& $0.7964$& $0.7001$ & \multirow{4}{*}{$0.8600$}\\

\multirow{2}{*}{0\% noise} & VB-Com & $84.05 \pm 2.28$ & \bm{$142.07 \pm 4.19$} & $0.71 \pm 0.01$ & \bm{$0.29 \pm 0.01$} & $1.50 \pm 0.14$ & $177.29 \pm 4.66$\\  
                              & Vision & $73.57 \pm 4.97$ & $118.07 \pm 10.42$ & $0.73 \pm 0.01$ & $0.42 \pm 0.07$ & \bm{$1.39 \pm 0.53$} & $204.82 \pm 28.91$\\  \midrule
\multirow{2}{*}{30\% noise} & VB-Com & $82.24 \pm 6.6$ & $132.81 \pm 7.64$ & $0.71 \pm 0.01$ & $0.34 \pm 0.10$ & $2.09 \pm 0.13$ & $178.13 \pm 4.13$\\  
                              & Vision & $72.76 \pm 2.29$ & $115.20 \pm 2.43$ & $0.75 \pm 0.02$ & $0.43 \pm 0.05$ & $2.52 \pm 0.32$ & $195.58 \pm 21.98$\\  \midrule
\multirow{2}{*}{70\% noise} & VB-Com & $82.48 \pm 1.20$ & $132.44 \pm 6.17$ & $0.70 \pm 0.02$ & $0.33 \pm 0.03$ & $2.12 \pm 0.11$ & $184.81 \pm 4.47$\\  
                              & Vision & $55.38 \pm 3.33$ & $58.24 \pm 13.97$ & $0.73 \pm 0.03$ & $0.67 \pm 0.07$ & $6.08 \pm 0.82$ & $190.50 \pm 18.28$\\  \midrule
\multirow{3}{*}{100\% noise} & VB-Com & \bm{$84.81 \pm 6.45$} & $129.99 \pm 9.84$ & $0.72 \pm 0.02$ & \bm{$0.29 \pm 0.08$} & $2.60 \pm 0.68$ & $182.29 \pm 11.47$\\  
                              & Vision & $48.71 \pm 5.60$ & $47.53 \pm 17.55$ & $0.70 \pm 0.06$ & $0.69 \pm 0.06$ & $6.92 \pm 1.36$ & $268.40 \pm 57.11$\\  
                              & Noisy Perceptive & $80.52 \pm 0.91$ & $116.94 \pm 4.07$ & \bm{$0.76 \pm 0.02$} & $0.32 \pm 0.04$ & $3.49 \pm 0.38$ & \bm{$154.98 \pm 4.41$}\\ \midrule
& Blind & $83.76 \pm 1.35$ & $131.29 \pm 3.48$ & $0.70 \pm 0.01$ & $0.33 \pm 0.05$ & $2.57 \pm 0.27$ & $184.08 \pm 1.85$\\  

% Perceptive  & $0.00 \pm 0.00$ & $0.00 \pm 0.00$ & $0.00 \pm 0.00$ & $0.00 \pm 0.00$ & $0.00 \pm 0.00$\\  
% Blind  & $0.00 \pm 0.00$ & $0.00 \pm 0.00$ & $0.00 \pm 0.00$ & $0.00 \pm 0.00$ & $0.00 \pm 0.00$\\  
% Noisy Perceptive & $0.00 \pm 0.00$ & $0.00 \pm 0.00$ & $0.00 \pm 0.00$ & $0.00 \pm 0.00$ & $0.00 \pm 0.00$\\  

\bottomrule[1.0pt]
\end{tabular}
\end{center}
\end{table*}

\subsection{Evaluations on Different Noise Levels}
\textbf{VB-Com achieves robust locomotion performance under different levels of perception deficiency.} As shown in Tab \ref{tab:VB-Com}, performance of the vision policy declines shaprly with the arise of noise level. In addition, since the evaluation experiments set the terrain curriculum to the maximum level, the vision policy struggles even at a $0\%$ noise level: It only achieves around $73\%$ goal-reaching success, with a termination rate exceeding $40\%$. This poor performance is likely due to the severe challenge terrains, such as the farthest range of the heightmap $(0.85m)$ is only $0.05m$ wider than the width of the gaps$(0.8m)$. In contrast, VB-Com achieves a stable higher goal-reaching success against different levels of perception deficiency. In contrast, VB-Com achieves consistently higher goal-reaching success across varying levels of perception deficiency, including both noise and perception range limitations.

Despite the high goal-reaching success, we also include additional metrics to further analyze the performance. The reward values recorded throughout each episode indicate the proposed method’s ability to achieve both goal completion and collision avoidance. These rewards strongly correlate with the robot’s success in reaching the target while minimizing collisions. For instance, VB-Com at the $0\%$ noise level achieves the highest rewards$(142.07)$, although the goal completion rate$(84.05)$ is slightly lower compared to the trail in $100\%$ noise level $(84.81)$. This is because VB-Com switches to the blind policy more often in  $100\%$  noise level, resulting in more frequent collisions and lower rewards obtained. 

The reach steps metrics indicates the smoothness of the policy in overcoming challenging obstacles. Since the switching mechanism requires several steps to respond effectively, VB-Com results in a higher number of reach steps as the noise level increases. This is because, under higher noise conditions, the system needs additional time to transition from the vision policy to the blind policy, which leads to more gradual and controlled responses to terrain challenges.
\begin{figure}[h]
\centering{\includegraphics[width=0.5\textwidth]{figures/noiseevalueate.png}}
\caption{We compare the collision and goal-reaching performances under different noise levels. VB-Com achieves low collisions and high success rates with accurate perception, and its success rate remains high under deficient perception.}
\label{noiseevalueate}
\end{figure}

\begin{figure}[h]
\centering{\includegraphics[width=0.5\textwidth]{figures/terraineval.png}}
\caption{Comparisons between the Noisy Perceptive policy and VB-Com in navigating gaps and hurdles separately.}
\label{terraineval}
\end{figure}


\subsection{Comparisons with Blind Policy}
\textbf{VB-Com achieves less collision than the blind policy when perception becomes less dificient.} As shown in Tab \ref{tab:VB-Com}, the blind policy achieves a relatively high Goals Completed rate $(83.76\%)$, as its performance is unaffected by deficient perception. Therefore, we include an evaluation of the collision performance between VB-Com and the blind policy to further highlight the advantage of the proposed framework. In our evaluations, "Collision Steps" is defined as the ratio of the number of steps during which the robot collision model (Fig \ref{robot}) makes illegal contact with the terrain or obstacles, relative to the total number of steps within an episode.

We can observe from Tab \ref{tab:VB-Com} that the collision steps increase with the noise level for VB-Com. Fig \ref{noiseevalueate} provides a more intuitive illustration: as perception becomes more comprehensive, VB-Com achieves both fewer collisions and better goal-reaching performance. In contrast, the blind policy maintains a high goal-reaching rate but results in more collisions, while the vision policy performs better in avoiding collisions when the perception is accurate and comprehensive. As the noise level increases, the performance of VB-Com begins to resemble that of the blind policy. These results demonstrate the effectiveness of the composition system, which benefits from both sub-policies to achieve better performance in terms of both goal-reaching and minimizing collisions.

\subsection{Comparisons with Noisy Perceptive Training}
\textbf{Compared to policies trained with noisy priors, VB-Com achieves equivalent performance without prior knowledge of the noise, while also demonstrating better training efficiency and the ability to handle more challenging terrain difficulties.} The comparisons (Tab \ref{tab:VB-Com}) with Noisy Perceptive policy show that the Noisy Perceptive policy achieves a relatively high goal completion rate $(80.52\%)$ but exhibits a higher collision step rate $(3.49\%)$. It can be concluded that, as severe noise is introduced during evaluation, the heightmap quickly becomes random noise with the increasing noise level. In response, the Noisy Perceptive policy begins to exhibit behavior similar to that of the blind policy—making contact with obstacles and reacting when the noisy signals overwhelm the external observations.

To further investigate the conditions under which the Noisy Perceptive policy fails to surpass the performance of VB-Com, we evaluate goal-reaching performance under different terrains (Fig. \ref{terraineval}). The results show that VB-Com outperforms the Noisy Perceptive policy in gap terrains, while the Noisy Perceptive policy performs better in hurdle situations, achieving a higher success rate in preventing the robot from being tripped by hurdles. However, recovering from missed gaps requires a quicker response, or the robot risks falling. These results demonstrate that the single-policy method fails to handle such dynamic challenges effectively, highlighting the advantages of the composition in VB-Com.

\begin{figure}[h]
\centering{\includegraphics[width=0.5\textwidth]{figures/trainplot.png}}
\caption{Training curves for terrain levels and the return estimation loss.}
\label{train}
\end{figure}

Moreover, the terrain level rises slowly for the Noisy Perceptive policy (Fig. \ref{train}-(a)), and it fails to reach the maximum level achieved by the vision and blind policies. This is because the policy struggles with the trade-off of whether to trust the external perception, which requires the addition of an extra module to address the challenge. This slow progression highlights the difficulty of handling high levels of perception deficiency, whereas VB-Com can efficiently navigate such situations by leveraging the strengths of both the vision and blind policies.

\begin{table}[!ht]
\caption{Return Estimation Evaluations}
\label{tab:RE}
\begin{center}
\renewcommand\arraystretch{1.25}
\begin{tabular}{lcccc}
\toprule[1.0pt]
Method & Goals Completed($\%$) & Collisions & Reach Steps\\
\midrule[0.8pt]

100-steps) & $78.24 \pm 1.86$ & \bm{$2.49 \pm 0.04$} & $193.7 \pm 3.2$\\  
RE(50-steps)  & \bm{$81.90 \pm 2.81$} & $2.75 \pm 0.17$ & $184.6 \pm 1.4$\\ 
Re(5-steps)   & $69.90 \pm 7.34$ & $5.23 \pm 0.59$ & $192.6 \pm 3.3$\\  
Re(1-step)    & $59.57 \pm 2.00$ & $4.78 \pm 0.16$ & \bm{$167.4 \pm 5.0$}\\  
MC-based      & $74.14 \pm 2.69$ & $4.26 \pm 0.56$ & $192.8 \pm 11.8$\\  

\bottomrule[1.0pt]
\end{tabular}
\end{center}
\end{table}

\subsection{Return Estimator Evaluations}
\textbf{The proposed return estimator achieves accurate and efficient return estimation with accessible states observations.} Since we update the return estimator using temporal difference, we compare it with the Monte Carlo-based search return estimator that estimate the furtuen expected returns with the following regression loss directly: $\mathbb{E}_t[\hat{G}_{\pi_i}^e(s_t) - \sum_{t} ^ {t+T} \gamma^t r(s_t, a_t)]$. As shown in Fig. \ref{train}-(a), the MC-based estimator struggles to converge due to the accumulation of noise. In contrast, the proposed TD-based return estimator within the vision policy convergent stably as it updates alongside the locomotion policy. The results in Tab \ref{tab:RE} further highlight the ineffectiveness of the MC-based return estimator in providing accurate estimations to guide the policy composition. Specifically, the MC-based estimator struggles to respond promptly to collisions with obstacles, this delay in response leads to larger collisions and longer reach steps, as the policy cannot effectively adjust its actions in real-time. 

\textbf{We also evaluate the impact of different switch periods (T), which define the expected return duration during return estimator updates.} While training performance remains consistent across varying periods, we observe that excessively short switch periods can negatively impact system performance. In such cases, the two policies may conflict, resulting in incomplete motion trajectories when traversing the challenging terrains and failures.

\textbf{We observe that training effectiveness is highly dependent on data variance.} For instance, the estimator within vision policy converges the fastest due to its access to more accurate and comprehensive state observations, leading to fewer low-return instances. In contrast, the estimator within Noisy Perceptive and blind policies encounter more collisions and lower returns, causing their loss to degrade more slowly.

\textbf{We demonstrate that the estimated return threhold $G_{th}$ is crucial to the performance of VB-Com.} Tab \ref{tab:TH} evaluates the system's performance under different values of $\alpha$, as well as without $G_{th}$. The results demonstrate that $G_{th}$ is critical for mitigating miscorrection during motion abnormalities, and that a value of $\alpha < 1.0$ ensures a sensitive response to the states that could lead to motion failures.

\begin{table}[!ht]
\caption{Estimated Return Threhold Evaluations}
\label{tab:TH}
\begin{center}
\renewcommand\arraystretch{1.25}
\begin{tabular}{lcccc}
\toprule[1.0pt]
Method & Goals Completed($\%$) & Collisions & Reach Steps\\
\midrule[0.8pt]
 
$\alpha = 2.0$   & $77.10 \pm 4.71$ & $2.63 \pm 0.68$ & $185.11 \pm 7.17$\\ 
$\alpha = 0.5$   & \bm{$85.76 \pm 2.88$} & $2.29 \pm 0.17$ & $186.96 \pm 3.83$\\  
$\alpha = 0.1$   & $84.43 \pm 1.23$ & \bm{$2.10 \pm 0.25$} & $\bm{184.35 \pm 6.27}$\\  
w/o $G_{th}$     & $48.48 \pm 1.28$ & $6.24 \pm 0.41$ & $261.96 \pm 35.63$\\  

\bottomrule[1.0pt]
\end{tabular}
\end{center}
\end{table}



\subsection{Real-World Experiments}

We deploy the proposed system on both the Unitree G1 and Unitree H1 robots and evaluate the performance of the proposed VB-Com method. 
\begin{figure*}[h]
\centering{\includegraphics[width= \textwidth]{figures/hardwarecurve.png}}
\caption{Illustrations of the variation in estimated return under static/dynamic obstacles in hardware experiments.}
\label{hardwarecurve}
\end{figure*}

\subsubsection{Hardware Return Estimations}

We illustrate how VB-Com operates on real robots by plotting $4$ seconds of the estimated return while the robot avoids static (left) and dynamic (right) obstacles (Fig \ref{hardwarecurve}). The results demonstrate that, for static obstacles (a standing person), the elevation map can accurately perceive the obstacle, allowing the robot to plan motions in advance and avoid collisions. Corresponding to this behavior, we observe that the estimated return on the G1 stays a high value, with $\hat{G}^e_{\pi_b}$ slightly lower than $\hat{G}^e_{\pi_v}$, consistent with the scenario where the vision policy continues to operate within VB-Com.

On the other hand, when a person moves towards the robot at high speed, the perception module fails to detect the obstacle, causing a collision, both $\hat{G}^e_{\pi_b}$ and $\hat{G}^e_{\pi_v}$ decline sharply upon collision. However, VB-Com quickly switches to $\pi_b$ to avoid the person, demonstrating the  \textbf{rapid response to collision provided by the proposed return estimation and the successful obstacle avoidance capability of VB-Com under perceptual deficiency}.


\begin{figure}[h]
\centering{\includegraphics[width=0.5\textwidth]{figures/g1avoid.png}}
\caption{ Real-world comparisons of VB-Com, vision, and blind policies in obstacle avoidance on the G1.}
\label{avoid}
\end{figure}

\subsubsection{Avoid Obstacles}
In this section, we make comparisons between VB-Com along with the vision policy and blind policy on G1 (Fig \ref{avoid}), to demonstrate the superior performance of VB-Com in hardware compared with signle policies. In the evaluation scenario, G1 encounters two consecutive obstacles along its path. The second dynamic obstacle obstructs the robot's direction before the elevation map can perceive it. VB-Com enables the robot to avoid the static obstacle without collision and subsequently avoid the dynamic obstacle after it collides with the suddenly appearing obstacle.

In contrast, for the baseline policies, the blind policy makes unnecessary contact with the static obstacles before avoiding them, which damages the environment. As for the vision policy, the robot collides with the obstacle and is unable to avoid it until the newly added obstacle is detected and integrated into the map.

\begin{figure}[h]
\centering{\includegraphics[width=0.5\textwidth]{figures/hurdlegap.png}}
\caption{Hardware demonstrations on the robots traversing gaps and hurldes given deficient perception with VB-Com.}
\label{hurdlegap}
\end{figure}

\subsubsection{Performance Against Deficient Perception}
In this section, we demonstrate the ability of VB-Com to traverse challenging terrains given deficient perception (Fig. \ref{hurdlegap}). We provide zero inputs for the heightmaps to evaluate the performance of VB-Com under perceptual deficiency. We introduce two consecutive hurdles, and the robot successfully recovers after colliding with them by switching to $\pi_b$. Additionally, we demonstrate that VB-Com enables recovery from a missed step on an unobserved gap. In this case, VB-Com saves the robot by performing a larger forward step to traverse the gap without perception, as the blind policy has learned during simulation.



% Consider a lasso optimization procedure with potentially distinct regularization penalties:
% \begin{align}
%     \hat{\beta} = \arg\min_{\beta}\{\|y-X\beta\|^2_2+\sum_{i=1}^{N}\lambda_i|\beta_i|\}.
% \end{align}
\subsection{Supervised Data-Driven Learning}\label{subsec:supervised}
We consider a generic data-driven supervised learning procedure. Given a dataset \( \mathcal{D} \) consisting of \( n \) data points \( (x_i, y_i) \in \mathcal{X} \times \mathcal{Y} \) drawn from an underlying distribution \( p(\cdot|\theta) \), our goal is to estimate parameters \( \theta \in \Theta \) through a learning procedure, defined as \( f: (\mathcal{X} \times \mathcal{Y})^n \rightarrow \Theta \) 
that minimizes the predictive error on observed data. 
Specifically, the learning objective is defined as follows:
\begin{align}
\hat{\theta}_f := f(\mathcal{D}) = \arg\min_{\theta} \mathcal{L}(\theta, \mathcal{D}),
\end{align}
where \( \mathcal{L}(\cdot,\mathcal{D}) := \sum_{i=1}^{n} \mathcal{L}(\cdot, (x_i, y_i))\), and $\mathcal{L}$ is a loss function quantifying the error between predictions and true outcomes. 
Here, $\hat{\theta}_f$ is the parameter that best explains the observed data pairs \( (x_i, y_i) \) according to the chosen loss function \( \mathcal{L} (\cdot) \).

\paragraph{Feature Selection.}
Feature selection aims to improve model \( f \)'s predictive performance while minimizing redundancy. 
%Formally, given data \( X \), response \( y \), feature set \( \mathcal{F} \), loss function \( \mathcal{L}(\cdot) \), and a feature limit \( k \), the objective is:
% \begin{align}
% \mathcal{S}^* = \arg \min_{\mathcal{S} \subseteq \mathcal{F}, |\mathcal{S}| \leq k} \mathcal{L}(y, f(X_\mathcal{S})) + \lambda R(\mathcal{S}),
% \end{align}
% where \( X_\mathcal{S} \) is the submatrix of \( X \) for selected features \( \mathcal{S} \), \( \lambda \) is a regularization parameter, and \( R(\mathcal{S}) \) penalizes feature redundancy.
 State-of-the-art techniques fall into four categories: (i) filter methods, which rank features based on statistical properties like Fisher score \citep{duda2001pattern,song2012feature}; (ii) wrapper methods, which evaluate model performance on different feature subsets \citep{kohavi1997wrappers}; (iii) embedded methods, which integrate feature selection into the learning process using techniques like regularization \citep{tibshirani1996LASSO,lemhadri2021lassonet}; and (iv) hybrid methods, which combine elements of (i)-(iii) \citep{SINGH2021104396,li2022micq}. This paper focuses on embedded methods via Lasso, benchmarking against approaches from (i)-(iii).

\subsection{Language Modeling}
% The objective of language modeling is to learn a probability distribution \( p_{LM}(x) \) over sequences of text \( x = (X_1, \ldots, X_{|x|}) \), such that \( p_{LM}(x) \approx p_{text}(x) \), where \( p_{text}(x) \) represents the true distribution of natural language. This process involves estimating the likelihood of token sequences across variable lengths and diverse linguistic structures.
% Modern large language models (LLMs) are trained on vast datasets spanning encyclopedias, news, social media, books, and scientific papers \cite{gao2020pile}. This broad training enables them to generalize across domains, learn contextual knowledge, and perform zero-shot learning—tackling new tasks using only task descriptions without fine-tuning \cite{brown2020gpt3}.
Language modeling aims to approximate the true distribution of natural language \( p_{\text{text}}(x) \) by learning \( p_{\text{LM}}(x) \), a probability distribution over text sequences \( x = (X_1, \ldots, X_{|x|}) \). Modern large language models, trained on diverse datasets \citep{gao2020pile}, exhibit strong generalization across domains, acquire contextual knowledge, and perform zero-shot learning—solving new tasks using only task descriptions—or few-shot learning by leveraging a small number of demonstrations \citep{brown2020gpt3}.
\paragraph{Retrieval-Augmented Generation (RAG).} Retrieval-Augmented Generation (RAG) enhances the performance of generative language models by  integrating a domain-specific information retrieval process  \citep{lewis2020retrieval}. The RAG framework comprises two main components: \textit{retrieval}, which extracts relevant information from external knowledge sources, and \textit{generation}, where an LLM generates context-aware responses using the prompt combined with the retrieved context. Documents are indexed through various databases, such as relational, graph, or vector databases \citep{khattab2020colbert, douze2024faiss, peng2024graphretrievalaugmentedgenerationsurvey}, enabling efficient organization and retrieval via algorithms like semantic similarity search to match the prompt with relevant documents in the knowledge base. RAG has gained much traction recently due to its demonstrated ability to reduce incidence of hallucinations and boost LLMs' reliability as well as performance \citep{huang2023hallucination, zhang2023merging}. 
 
% image source: https://medium.com/@bindurani_22/retrieval-augmented-generation-815c1ae438d8
\begin{figure}
    \centering
\includegraphics[width=1.03\linewidth]{fig/fig1.pdf}
\vspace{-0.6cm}
\scriptsize 
    \caption{Retrieval Augmented Generation (RAG) based $\ell_1$-norm weights (penalty factors) for Lasso. Only feature names---no training data--- are included in LLM prompt.} 
    \label{fig:rag}
\end{figure}
% However, for the RAG model to be effective given the input token constraints of the LLM model used, we need to effectively process the retrieval documents through a procedure known as \textit{chunking}.

\subsection{Task-Specific Data-Driven Learning}
LLM-Lasso aims to bridge the gap between data-driven supervised learning and the predictive capabilities of LLMs trained on rich metadata. This fusion not only enhances traditional data-driven methods by incorporating key task-relevant contextual information often overlooked by such models, but can also be especially valuable in low-data regimes, where the learning algorithm $f:\mathcal{D}\rightarrow\Theta$ (seen as a map from datasets $\mathcal{D}$ to the space of decisions $\Theta$) is susceptible to overfitting.

The task-specific data-driven learning model $\tilde{f}:\mathcal{D}\times\mathcal{D}_\text{meta}\rightarrow\Theta$ can be described as a metadata-augmented version of $f$, where a link function $h(\cdot)$ integrates metadata (i.e. $\mathcal{D}_\text{meta}$) to refine the original learning process. This can be expressed as:
\[
\tilde{f}(\mathcal{D}, \mathcal{D}_\text{meta}) := \mathcal{T}(f(\mathcal{D}),  h(\mathcal{D}_{\text{meta}})),
\]
where the functional $\mathcal{T}$ takes the original learning algorithm $f(\mathcal{D})$ and transforms it into a task-specific learning algorithm $\tilde{f}(\mathcal{D}, \mathcal{D}_\text{meta})$ by incorporating the metadata $\mathcal{D}_\text{meta}$. 
% In particular, the link function $h(\mathcal{D}_{\text{meta}})$ provides a structured mechanism summarizing the contextual knowledge.

There are multiple approaches to formulate $\mathcal{T}$ and $h$.
%to ``inform" the data-driven model $f$ of %meta knowledge. 
For instance, LMPriors \citep{choi2022lmpriorspretrainedlanguagemodels} designed $h$ and $\mathcal{T}$ such that $h(\mathcal{D}_{\text{meta}})$ first specifies which features to retain (based on a probabilistic prior framework), and then $\mathcal{T}$ keeps the selected features and removes all the others from the original learning objective of $f$. 
Note that this approach inherently is restricted as it selects important features solely based on $\mathcal{D}_\text{meta}$ without seeing $\mathcal{D}$.

In contrast, we directly embed task-specific knowledge into the optimization landscape through regularization by introducing a structured inductive bias. This bias guides the learning process toward solutions that are consistent with metadata-informed insights, without relying on explicit probabilistic modeling. Abstractly, this can be expressed as:
\begin{align}
    \!\!\!\!\!\hat{\theta}_{\tilde{f}} := \tilde{f}(\mathcal{D},\mathcal{D}
    _\text{meta})= \arg\min_{\theta} \mathcal{L}(\theta, \mathcal{D}) + \lambda R(\theta, \mathcal{D}_{\text{meta}}),
\end{align}
where \( \lambda \) is a regularization parameter, \( R(\cdot) \) is a regularizer, and $\theta$ is the prediction parameter.
%We explain our framework with more details in the following section.


% Our research diverges from both aforementioned approaches by positioning the LLM not as a standalone feature selector but as an enhancement to data-driven models through an embedded feature selection method, L-LASSO. L-LASSO incorporates domain expertise—auxiliary natural language metadata about the task—via the LLM-informed LASSO penalty, which is then used in statistical models to enhance predictive performance. This method integrates the rich, context-sensitive insights of LLMs with the rigor and transparency of statistical modeling, bridging the gap between data-driven and knowledge-driven feature selection approaches. To approach this task, we need to tackle two key components: (i). train an LLM that is expert in the task-specific knowledge; (ii). inform data-driven feature selector LASSO with LLM knowledge.

% In practice, this involves combining techniques like prompt engineering and data engineering to develop an effective framework for integrating metadata into existing data-driven models. We will go through this in detail in Section \ref{mthd} and \ref{experiment}.


\section{Semantic Equivalence Based Program Clustering}
\label{sec:symexclustering}

The NLG techniques proposed by Kuhn~\etal~\cite{kuhnsemantic} and Abbasi~\etal~\cite{abbasi2024believe} rely on semantic clustering, where semantically equivalent programs are grouped together. 
Achieving this requires an effective method for assessing program equivalence. Kuhn~\etal employ the DeBERTa-large model~\cite{he2020deberta} for this task, while Abbasi~\etal determine equivalence using an F1 score based on token inclusion~\cite{DBLP:journals/corr/JoshiCWZ17}.

In the domain of code generation, program equivalence has a precise definition: two programs are considered equivalent if they produce identical behavior for all possible inputs. 
Consequently, a domain-specific equivalence check is required.
In this paper, we base the semantic equivalence check on \emph{symbolic execution}, where, instead of executing a program with concrete inputs, \emph{symbolic variables} are used to represent inputs, generating constraints that describe the program's behavior across all possible input values~\cite{symex_klee}.

The particular flavor of symbolic execution we use in this work is inspired by the lightweight \emph{peer architecture} described in Bruni~\etal~\cite{Bruni2011APA}. 
Unlike traditional approaches that require building a standalone symbolic interpreter, this architecture embeds the symbolic execution engine as a lightweight library operating alongside the target program. 
Their design is based on the insight that languages that provide the ability to dynamically dispatch primitive operations (\eg Python) allow symbolic values to behave as native values and be tracked at runtime.

Symbolic execution typically traverses the program's control flow graph, maintaining a symbolic state consisting of \emph{path constraints} (\ie logical conditions that must be satisfied for a given execution path to be feasible) and \emph{symbolic expressions} (\ie representations of program variables as functions of the symbolic inputs). 

\paragraph{Equivalence check.} Given two code snippets \(s^{(1)}\) and \(s^{(2)}\), we check semantic equivalence between them by comparing their symbolic traces. 
One such symbolic trace, \eg \(T(s^{(1)})\), consists of the corresponding path constraint and symbolic expressions denoting all the variables encountered on the corresponding execution path. 
Intuitively, for two code snippets to be semantically equivalent, all their corresponding symbolic traces must align. 
Specifically, for each path constraint, the traces produced by both snippets must be identical meaning that there is no concrete counterexample input for which the execution of the two snippets diverges.

%\CD{I'm not sure whether it's worth formalising this a bit.}

%Semantic equivalence between two code snippets \(s^{(i)}\) and \(s^{(j)}\) is determined by comparing their symbolic traces, \(T(s^{(i)})\) and \(T(s^{(j)})\), respectively:
%\begin{multline}
%    T(s^{(i)}) \equiv T(s^{(j)}) \iff \text{Path Constraints and Symbolic Expressions of } \\
%    s^{(i)} \text{ and } s^{(j)} \text{ are identical.}
%\end{multline}

Since program equivalence is undecidable in general, we perform a bounded equivalence check. 
This approach verifies that no counterexample input exists when exploring traces up to a given depth.

%Exact equivalence is often \emph{undecidable} due to the complexity of symbolic traces. 
%Instead, we employ \emph{subsumption}, where one trace subsumes another if all behaviours of the latter are captured by the former. 
%This allows us to approximate equivalence effectively.

%By using this lightweight symbolic execution approach, our clustering methodology emphasizes functional semantics, avoiding overfitting to syntactic similarities. 
%This methodology strikes a balance between precision and efficiency, leveraging the extensibility and simplicity of the peer architecture to scale across diverse programming scenarios.


\begin{algorithm}[ht!]
    \caption{Clustering with Symbolic Execution}
    \label{alg:clustering}
    \begin{algorithmic}[1]
    \Require Set of generated code snippets $\{s^{(1)}, \ldots, s^{(M)}\}$
    \Ensure Clusters of semantically equivalent snippets $C = \{c_1, c_2, \ldots, c_k\}$
    
    \State Initialize an empty cluster set $C \gets \emptyset$, and an equivalence map $E \gets \emptyset$ \label{alg:clustering:init}
    
    \For{each snippet $s^{(i)}$}
        \If{$s^{(i)}$ is invalid}
            \State $E[s^{(i)}] \gets \{\,s^{(i)}\}$ 
            \Comment{Assign invalid snippet to its own equivalence class}
        \EndIf
    \EndFor
    
    \For{each pair of valid snippets $(s^{(i)}, s^{(j)})$} \label{alg:clustering:pairwise}
        \State Perform symbolic execution on $s^{(i)}$ and $s^{(j)}$ to extract traces $T(s^{(i)})$ and $T(s^{(j)})$ \label{alg:clustering:trace}
        \If{$T(s^{(i)}) \equiv T(s^{(j)})$} \label{alg:clustering:check}
            \State $E[s^{(i)}] \gets E[s^{(i)}] \cup \{\,s^{(j)}\}$
            \State $E[s^{(j)}] \gets E[s^{(j)}] \cup \{\,s^{(i)}\}$ \label{alg:clustering:update}
        \EndIf
        \State \Comment{Enforce transitivity of equivalences}
        \If{$s^{(i)} \sim s^{(j)}$ and $s^{(j)} \sim s^{(k)}$ for some $s^{(k)}$}
            \State $E[s^{(i)}] \gets E[s^{(i)}] \cup \{\,s^{(k)}\}$
            \State $E[s^{(k)}] \gets E[s^{(k)}] \cup \{\,s^{(i)}\}$
        \EndIf
    \EndFor
    
    \State Identify equivalence classes in $E$ to form final clusters $C$ \label{alg:clustering:extract}
    
    \State \Return $C$ \label{alg:clustering:return}
    \end{algorithmic}
    \end{algorithm}

%\CD{we need to modify the alg so that it's obvious that we are talking about sets of traces at line 8.}

\paragraph{Semantic clustering.}
Algorithm~\ref{alg:clustering} illustrates how to cluster code snippets based on their functional semantics, with an additional check for invalid snippets. 
We first create empty structures for storing the final clusters ($C$) and an equivalence map ($E$) to track relationships (line~\ref{alg:clustering:init}). 

Next, in the \emph{invalid snippet handling phase}, each code snippet $s^{(i)}$ is examined and if it is detected to be invalid, it is immediately placed in its own equivalence class in $E$ and is thus isolated from further consideration. 

In the \emph{pairwise comparison phase} (line~\ref{alg:clustering:pairwise}), each pair of \emph{valid} snippets $(s^{(i)}, s^{(j)})$ is symbolically executed to produce traces $T(s^{(i)})$ and $T(s^{(j)})$ (line~\ref{alg:clustering:trace}). 
If the traces are equivalent (line~\ref{alg:clustering:check}), indicating identical functional behavior, both snippets are added to each other's equivalence classes (line~\ref{alg:clustering:update}). 
In reality, $T(s^{(i)})$ and $T(s^{(j)})$ actually denote sets of traces, and the equivalence check involves comparing individual traces from each set that share the same path constraint.
For brevity, in Algorithm~\ref{alg:clustering}, we represent this as $T(s^{(i)}) \equiv T(s^{(j)})$.
To maintain consistency, transitivity is enforced: if $s^{(i)}$ is equivalent to $s^{(j)}$, and $s^{(j)}$ is equivalent to $s^{(k)}$, then $s^{(i)}$ must also be in the same equivalence class as $s^{(k)}$. 

Finally, the equivalence map $E$ is processed to derive the clusters themselves (line~\ref{alg:clustering:extract}), and the resulting set of clusters is returned (line~\ref{alg:clustering:return}). 
% By isolating invalid snippets in single-item clusters, the algorithm cleanly separates non-functional or syntactically invalid code from semantically consistent groups. 
% This ensures that the final clustering reflects the functional semantics of valid snippets while transparently segregating invalid code. 

\section{Estimating the Probability Distribution of LLM Responses}
\label{sec:probcomp}

Both NLG techniques we adapt for code generation, at certain points, query the LLM, sample responses along with the log-probabilities of their tokens, and apply a softmax-style normalization to interpret them as a valid probability distribution.
However, since the LLM responses in this setting are programs, they are longer than the natural language responses used in the original studies---while the evaluation for Kuhn~\etal and Abbasi~\etal considered question-answer datasets typically involving one word answers, the programs produced in this work are around 200 tokens per response. 

The probability of a response is represented as the joint probability of its tokens, meaning that it decreases exponentially with length, often leading to \emph{numerical underflow}. 
This ultimately compromises the effectiveness of the technique. 
For instance, if the probabilities for all response programs underflow, then softmax returns NaN, which then propagates through the computation.

To address the issue of exponentially decaying probabilities, we propose two methods for approximating the probability distribution of LLM responses, as outlined below.

\paragraph{Length normalization.}

%The techniques we propose require computing the probability of responses generated by language models, where the probability of a response is represented as the joint probability of its tokens. 
%However, for longer responses, this probability decreases exponentially with length, which adversely impacts our estimation of uncertainty.

%In NLG tasks such as those addressed by prior works~\cite{kuhnsemantic,abbasi2024believe}, this issue is less pronounced because the goal in their problem domains (\eg TriviaQA) is to exactly match short reference answers. 
%In contrast, for code generation, the outputs are often longer. 
%While the evaluation for Kuhn~\etal~\cite{kuhnsemantic} and Abbasi~\etal~\cite{abbasi2024believe} considered question-answer datasets while typically involve one word answers, the program snippets produced in this work were typically around 200 tokens per response. 
% \CD{For our experiments, the average number of tokens in the results generated by the LLM is ...}. 
%Notably, while it is true that accuracy tends to decrease with length, existing research demonstrates that LLMs can generate high-quality code snippets, even up to 100 lines~\cite{codetranslation2}. %Consequently, the drastic reduction in probability with length disproportionately affects these scenarios. 

One solution is to use \emph{length normalization}~\cite{DBLP:conf/wmt/MurrayC18,DBLP:conf/aclnmt/KoehnK17}, more precisely length-normalizing the log probability of a program, a technique  used by other existing works, \eg to compute length-normalized predictive entropy~\cite{DBLP:conf/iclr/MalininG21}. %This also allows compar uncertainties of sequences of different length

More concretely, to compute a length-normalized probability from log probabilities, we begin by calculating the sum of the log probabilities. 
Let \(\ell_1, \ell_2, \dots, \ell_n\) denote the log probabilities associated with each token in the sequence that forms the response.
The log probability of the response is given by:
$S = \sum_{i=1}^{n} \ell_i$.
%
Next, the log probability is normalized by the sequence length \(L\) and the normalizing factor \(\gamma\). 
The normalized log probability is computed as:
$\ell_{\text{norm}} = \frac{S}{L \cdot \gamma}$.
%
Finally, the normalized probability \(P\) of the response is obtained by exponentiating the normalized log probability:
$P = e^{\ell_{\text{norm}}}$.

Intuitively, when using length-normalization in the context of uncertainty computation, probabilities remain comparable across responses of different lengths, whereas uncertainty is linked to the semantic differences between responses. %In our experimental evaluation, we compute uncertainty measures both with length-normalization and without.

\paragraph{Uniform distribution of LLM-generated responses.}
Intuitively, when multiple LLM responses are semantically equivalent, it indicates a higher degree of certainty in the (semantics of the) generated output.
To test this intuition, we propose disregarding the log-probabilities reported by the LLM and instead assuming a uniform distribution over the sampled responses. 
Specifically, if we sample $n$ responses, each response is assigned an equal probability of $1/n$.

Our experimental results demonstrated that using the semantic equivalence-based approach (described in Section~\ref{sec:symex}) in conjunction with this distribution reveals a negative correlation between the LLM's uncertainty and correctness---see Section~\ref{sec:results-discussion}. 
Furthermore, applying an uncertainty threshold derived from this technique to filter LLM responses—allowing only those above a specified correctness score (measured as the percentage of passed unit tests)—leads to high accuracy---see Section~\ref{sec:usability}.


%This approach ensures that the probabilities are appropriately scaled with respect to the sequence length.


%%For illustration, consider below as an example response produced by \gptturbo for the prompt used in Figure~\ref{fig:sampleproblem} from \S\ref{sec:motivating}:
%% % ``\texttt{Write a Python function that counts how many people older than 60 appear in a data list.}''

%% \begin{lstlisting}[language=Python]
%%     def candidate1(details):
%%      count = 0
%%         for detail in details:
%%             age = int(detail[11:13])
%%             if age > 60:
%%                 count += 1
%%         return count
%% \end{lstlisting}

%% For illustration, let us consider a code snippet generated by the LLM of $N=20$ tokens. ,  each with an individual token probability of $0.7$. 

%% \CD{Can we actually find out the number of tokens and log probabilities?}
%% When working with language models, each token in a generated completion has an associated log probability.  
%% 

%% Then if we simply sum the log probabilities of each token in a generated response as shown earlier:
%% \[
%%    \sum_{i=1}^{20} \log(0.7)
%%    \;=\;
%%    20 \,\log(0.7)
%%    \;\approx\;
%%    -7.1335,
%% \]
%% and exponentiate this sum yields:
%% \[
%%    \exp(-7.1335)
%%    \;\approx\;
%%    0.0008.
%% \]


%% Moreover, The probabilities of individual responses tend to decay exponentially with length, which can lead to disproportionately low values for valid but lengthy outputs.

%% If we merely sum these log probabilities over $N$ tokens,



%% Hence, although $0.7$ per token is fairly high, the \emph{total} probability from multiplying all 20 token probabilities becomes quite small ($0.0008$). 
%% A shorter completion, having fewer tokens, might end up with a larger total probability even if its average token confidence is slightly lower. 
%% This demonstrates how \emph{lengthier responses} can be unfairly penalized if we only sum or multiply all token probabilities, hence motivating \textbf{length-based normalisation}.


%% Length-normalization also helps when the responses generated for a query have different lengths.




% To address this, we perform a length-based normalisation to ensure that probabilities remain comparable across responses of different lengths. Specifically, for a generated snippet \(s\), its normalized probability is defined as:
% \begin{equation}
%     \tilde{p}(s \mid x) = \frac{p(s \mid x)}{|s|^\alpha},
% \end{equation}
% where \(|s|\) is the length of the snippet \(s\), and \(\alpha\) is a hyperparameter that controls the degree of normalisation. This adjustment prevents the probabilities of longer responses from dominating or vanishing entirely, ensuring a fair representation in subsequent entropy computations.

%% When working with language models, each token in a generated completion has an associated log probability.  
%% If we merely sum these log probabilities over $N$ tokens,
%% \[
%%    \text{sum\_logprob} \;=\; \sum_{i=1}^{N} \log \bigl(p(\mathrm{token}_i)\bigr),
%% \]
%% longer completions tend to accumulate more negative values simply due to having more tokens. 
%% This can make them appear less likely, even if each token is reasonably probable.

%% To address this, we perform a length-based normalisation to ensure that probabilities remain comparable across responses of different lengths. Specifically, for a generated snippet \(s\), its normalized probability is defined as:
%% \begin{equation}
%%     \tilde{p}(s \mid x) = \frac{p(s \mid x)}{|s|^\alpha},
%% \end{equation}
%% where \(|s|\) is the length of the snippet \(s\), and \(\alpha\) is a hyperparameter that controls the degree of normalisation. 
%% This adjustment prevents the probabilities of longer responses from dominating or vanishing entirely, ensuring a fair representation in subsequent entropy computations.

% To mitigate this effect, we use \emph{length-based normalisation}.
% Instead of summing the log probabilities, we compute the \textbf{average} log probability per token:
% \[
%    \text{avg\_logprob} \;=\; \frac{1}{N} \; \sum_{i=1}^{N} \log \bigl(p(\mathrm{token}_i)\bigr).
% \]
% We then exponentiate this average to obtain a \textbf{length-normalized probability}:
% \[
%    \text{length\_normalized\_prob} \;=\;
%    \exp\!\bigl(\text{avg\_logprob}\bigr).
% \]
% Although this value is not a ``true'' probability for the entire sequence, it is 
% a fairer score for comparing completions of different lengths, 
% because a longer completion is not automatically penalized by virtue of having more tokens.


%% Hence, instead of relying on the total product of probabilities, we use the 
%% \textbf{average log probability} per token:
%% \[
%%    \text{avg\_logprob} 
%%    \;=\;
%%    \frac{1}{N} 
%%    \sum_{i=1}^{N} \log\bigl(p(\mathrm{token}_i)\bigr)
%%    \;=\;
%%    \log(0.7),
%% \]
%% for $N=20$ tokens in this simplified scenario. 
%% Exponentiating that average log probability gives:
%% \[
%%    \exp\!\bigl(\text{avg\_logprob}\bigr)
%%    \;=\;
%%    \exp(\log(0.7))
%%    \;=\;
%%    0.7.
%% \]

%% Thus, while the raw product \(\prod_{i=1}^{20} p(\mathrm{token}_i)\) is about $0.0008$, the \emph{length-normalized} probability is $0.7$, reflecting the fairer notion that each token has a $70\%$ likelihood on average. This avoids unfairly penalizing longer responses merely due to having more tokens multiplied together. 
%% Length-based normalisation is therefore crucial for comparing or ranking completions of different lengths.

% A different completion with more lines of code and docstrings might sum to a lower total log probability (due to more tokens), but its average log probability could be similar (say, $0.95$). When using length-based normalisation, these two scores ($0.97$ vs.\ $0.95$) are directly comparable as per-token likelihoods, rather than an apples-to-oranges comparison of total log probabilities.


% If you want to interpret these normalized scores as a \emph{distribution} over completions, 
% you can renormalize across all candidates so that they sum to 1:
% \[
%   \hat{p}_i \;=\; 
%   \frac{\exp\!\bigl(\text{avg\_logprob}_i\bigr)}{\sum_{j=1}^{k} \exp\!\bigl(\text{avg\_logprob}_j\bigr)},
% \]
% where $k$ is the total number of candidate completions. 
% This transforms the length-normalized scores into valid probabilities for comparing or sampling.

\section{Semantic Uncertainty via Symbolic Clustering}
\label{sec:symex}
This section presents our adaptation of the semantic entropy-based approach by Kuhn~\etal~\cite{kuhnsemantic} for code generation. 
We follow the main steps from the original work while diverging in two key aspects: the way we estimate the distribution of generated LLM responses and the clustering methodology.

%This section presents our methodology for assessing semantic uncertainty in code generation by leveraging clustering based on symbolic execution. 

\subsubsection{Generation}
The first step involves sampling $M$ code snippets (using the same hyperparameters as Kuhn~\etal~\cite{kuhnsemantic}), $\{s^{(1)}, \ldots, s^{(M)}\}$, from the LLM's output distribution $p(s \mid x)$ for a given prompt $x$. 
%Given a code generation prompt, the model generates $M$ samples, $\{s^{(1)}, \ldots, s^{(M)}\}$, from its predictive distribution $p(s \mid x)$.
% Sampling is carried out using multinomial techniques, with hyperparameters such as temperature and nucleus sampling selected based on those used by Kuhn~\etal~\cite{kuhnsemantic}.
The probabilities of the collected samples are processed using a softmax-style normalization function, ensuring that the resulting values can be interpreted as a valid probability distribution.

As explained in Section~\ref{sec:probcomp}, this process can lead to numerical underflows. 
To mitigate this, we approximate the probability distribution of the LLM responses using either length-normalization or a uniform distribution.
%
Following this approximation, let \(\tilde{p}(s \mid x)\) denote the probability of a snippet \(s\) according to the adjusted distribution.

%softmax normalized
%Then, we have: 
 %\[
 %\tilde{p}(s \mid x) = \frac{p(s \mid x)}{|s|^\alpha},
 %\]
 %where \(|s|\) is the length of the snippet \(s\), and \(\alpha\) is a hyperparameter controlling the degree of normalization. 

%\CD{fix softmax normalization}

%This is similar to the approach taken by Kuhn~\etal~\cite{kuhnsemantic} and Abbasi~\etal~\cite{abbasi2024believe} and is a necessary step to prevent the probability computations from becoming invalid within the various formulae.
% \CD{Do we use these optional techniques?}

\subsubsection{Clustering via Symbolic Execution}
The second step works by grouping the aforementioned snippets into clusters based on semantic equivalence. %, determined through symbolic execution traces.
%To determine semantic equivalence, we employ symbolic execution, a program analysis technique that computes execution paths and constraints for a given snippet. 
%Two snippets $s^{(i)}$ and $s^{(j)}$ are deemed functionally equivalent if their symbolic execution traces are identical or exhibit subsumption.
%\CD{We need details on the semantic equivalence. This is, in principle, undecidable, so we need to explain a bit more on how this works.}
%
This process, as shown in Algorithm~\ref{alg:clustering} from Section~\ref{sec:symexclustering}, is based on symbolic execution. %ensures that clustering is driven by functional, rather than syntactic or lexical, similarities, aligning with the stricter requirements of code quality evaluation.
%It is important to note that two syntactically different responses can still end up in the same cluster if they are semantically equivalent.
% This is due to our adapted symbolic execution based clustering algorithm . 

\subsubsection{Entropy Estimation}
The final step computes uncertainty as the semantic entropy over clusters, reflecting the diversity of functional behaviors.

%Semantic entropy quantifies the uncertainty in functional behaviour by measuring the probability distribution over clusters.
%
First, the probability associated with a cluster \(c\) is calculated as:
\begin{equation*}
    \tilde{p}(c \mid x) = \sum_{s \in c} \tilde{p}(s \mid x),
    %p(c \mid x) = \sum_{s \in c} p(s \mid x),    
\end{equation*}
where \(s \in c\) indicates that the snippet \(s\) belongs to the cluster \(c\).
%
Then, the entropy \(H(C \mid x)\) over the set of clusters \(C\) is defined as:
\begin{equation*}
    H(C \mid x) = -\sum_{c \in C} \log \tilde{p}(c \mid x),
    %H(C \mid x) = -\sum_{c \in C} \log p(c \mid x),    
\end{equation*}
where \(C\) denotes all semantic clusters obtained from Algorithm~\ref{alg:clustering} in Section~\ref{sec:symexclustering}. 
%
%This formulation captures both the diversity and confidence of the model's outputs while accounting for the length-based normalization, offering a self-contained metric independent of external validation.
A higher entropy indicates greater semantic diversity and hence higher uncertainty in the functional behavior captured by the clusters. 
Conversely, a lower entropy suggests that the model's outputs are concentrated around a few semantically equivalent behaviors, reflecting higher confidence.

\paragraph{Motivating example revisited for \textnormal{\gptturbo}.} To illustrate our uncertainty computation, we'll go back to the motivating example from Section~\ref{sec:motivating}.

As discussed there, Figure~\ref{fig:good-llm-snippets} (Listings~\ref{lst:good1}, \ref{lst:good2}, and \ref{lst:good3}) contains code snippets generated by \gptturbo. % that are semantically equivalent. %, all three snippets exhibit the same behaviour.
%while Figure~\ref{fig:bad-llm-snippets} (Listings~\ref{lst:bad1}, \ref{lst:bad2}, and \ref{lst:bad3} from \salesforce/\codegenmonoC) contains code snippets that are \textbf{semantically distinct}, none of the snippets share the same functional behaviour.
We denote these snippets by $s^{(1)}, s^{(2)}, s^{(3)}$, and, according to Algorithm~\ref{alg:clustering} from Section~\ref{sec:symexclustering}, they are all grouped in the same functional cluster $c_1$ as they are semantically equivalent.
%Now for the \gptturbo case, we have three generated snippets (Listings~\ref{lst:good1}, \ref{lst:good2}, and \ref{lst:good3} from Figure~\ref{fig:good-llm-snippets}), denoted $s^{(1)}, s^{(2)}, s^{(3)}$, all found by using the Algorithm~\ref{alg:clustering} from \S\ref{sec:symexclustering}  to be in one functional cluster $c_1$.
In other words, $C = \{ c_1 \}$ with $c_1 = \{ s^{(1)}, s^{(2)}, s^{(3)} \}$.

%Given the very small probabilities reported by the LLM as shown in Section~\ref{sec:motivating} (which would cause underflows in our implementation), here we use length normalized log-probabilties according to the normalization formula in Section~\ref{sec:probcomp}.
Following softmax normalization, we obtain the following probabilities for the three snippets: %, where we use $\tilde{p}$ to denote the probability based on length-normalization: 
%For these snippets, following are the normalized probabilities based on the token-level \texttt{logprobs} data obtained from \gptturbo:
\(\tilde{p}(s^{(1)} \mid x) = \GPTsnipNormProbA,\; \tilde{p}(s^{(2)} \mid x) = \GPTsnipNormProbB,\;
\tilde{p}(s^{(3)} \mid x) = \GPTsnipNormProbC.\)
Since all responses belong to the single cluster $c_1$, its cluster probability is:
\[
   \tilde{p}(c_1 \mid x)
   \;=\;
   \sum_{s \in c_1} \tilde{p}(s \mid x)
   \;=\;
   \GPTsnipNormProbA \;+\; \GPTsnipNormProbB \;+\; \GPTsnipNormProbC
   \;=\;
   1.0
\]
Thus the distribution over clusters is \(\tilde{p}(c_1 \mid x) = 1,\) and the entropy of clusters is:
\[
   H(C \mid x)
   \;=\;
   -\sum_{c \in C} \log \tilde{p}(c \mid x)
   \;=\;
   -\log(1.0)
   \;=\;
   0.
\]
A \emph{zero} semantic entropy indicates high confidence in the model's response for this prompt.

\paragraph{Motivating example revisited for \textnormal{\salesforce}.} Let's now consider the three snippets $s^{(1)}, s^{(2)}, s^{(3)}$ from Figure~\ref{fig:bad-llm-snippets} generated by the \salesforce/\codegenmonoC model. 
Their respective probabilities are:
$\tilde{p}(s^{(1)}\!\mid x) = \SFsnipNormProbA,\;
 \tilde{p}(s^{(2)}\!\mid x) = \SFsnipNormProbB,\;
 \tilde{p}(s^{(3)}\!\mid x) = \SFsnipNormProbC$.
These snippets get categorized in three distinct semantic clusters
$C = \{ c_1, c_2, c_3 \}$, with $c_1 = \{ s^{(1)} \}$,
$c_2 = \{ s^{(2)} \}$, and $c_3 = \{ s^{(3)} \}.$
 Because each snippet resides in its own cluster, the cluster probabilities are:
   $\tilde{p}(c_1 \mid x) = \SFsnipNormProbA, \tilde{p}(c_2 \mid x) = \SFsnipNormProbB, \tilde{p}(c_3 \mid x) = \SFsnipNormProbC$.
%\[
%   \tilde{p}(c_1 \mid x) = \SFsnipNormProbA,\quad
%   \tilde{p}(c_2 \mid x) = \SFsnipNormProbB,\quad
%   \tilde{p}(c_3 \mid x) = \SFsnipNormProbC.
%\]
%
The entropy then is:
\[
\begin{aligned}
   H(C \mid x)
   &=
   -\!\sum_{c \in \{c_1,c_2,c_3\}}
   \log \tilde{p}(c \mid x)
   \\
   &=
   -\,\Bigl(
      \log(\SFsnipNormProbA)\;+\;\log(\SFsnipNormProbB)\;+\;\log(\SFsnipNormProbC) 
   \Bigr) \approx \SFSE.
\end{aligned}
\]
%Numerically, this is approximately \SFSE.
%A \emph{higher entropy} in this example indicates more disagreement or diversity in the model's functional outputs: the \salesforce/\codegenmonoC model produced three \textbf{distinctly incorrect} solutions, each forming its own cluster.


Intuitively, the uncertainty estimate reflects the strength of our belief in the LLM's prediction. 
Based on this, an \textit{abstention policy} can be implemented, whereby the system abstains from making a prediction if the entropy exceeds a predefined \emph{uncertainty threshold}. 
This approach minimizes the likelihood of committing to incorrect or suboptimal solutions. The abstention threshold is empirically determined by analyzing the entropy distribution. 
The methodology for computing this threshold will be detailed in Section~\ref{sec:eval}.

%Thus, these two scenarios exemplify how \emph{semantic clustering} and the corresponding \emph{entropy measure} can capture both the diversity (or uniformity) of model generations \emph{and} the model's confidence in those generations' functional behaviour.


\section{Mutual Information Estimation via Symbolic Clustering}
\label{sec:mi}
% \AS{TODO:Redo this section, their is a sizeable gap between the theory of the paper and its practical implementation. Make sure that these gaps are explained in this section.
% Algorithm 2 and 3 of the paper work quite differently, so ensure that the explanation matches the implementation, where needed.}
This section presents an adaptation of the mutual information-based approach for quantifying epistemic uncertainty by Abbasi~\etal~\cite{abbasi2024believe} to the domain of code generation.
We follow the steps from the original work: iterative prompting for generating LLM responses, clustering, and mutual information estimation. 
However, similar to Section~\ref{sec:symex}, we diverge with respect to the methodology for clustering responses and the way we estimate the distribution of generated LLM responses.

\subsubsection{Iterative Prompting for Code Generation}
Iterative prompting is used for generating multiple responses from the LLM and consequently in constructing a pseudo joint distribution of outputs. 

% Beginning with the original prompt \(F_0(x) = x\) for \(i = 1, 2, \ldots, n\) we get a \emph{family of prompts} where the \(i\)-th response prompt in the family would be:
% \[
%   F_i(x, s_1, \ldots, s_i) = \text{``Original prompt: } x \text{. Previous responses: } s_1, \ldots, s_i \text{."}
% \]
% where $s_i$ is the \(i\)-th response from the LLM. 

More precisely, the LLM is sampled to produce $n$ responses while also getting their respective probabilities, $\mu(X_j)$ for \(j = 1, 2, \ldots, n\).
These responses are first used to construct iterative prompts by appending the response to the original prompt and asking the LLM to produce more responses. 
This step then makes use of softmax-style normalization to obtain values that can then be treated as probabilities which are used in the subsequent steps.  

As explained in Section~\ref{sec:probcomp} and Section~\ref{sec:symex}, this process can lead to numerical underflows. 
To mitigate this, we use length-normalization.
As opposed to the approach in Section~\ref{sec:symex}, here we did not use the uniform distribution approximation, as the actual LLM-reported probabilities are needed to distinguish between aleatoric and epistemic uncertainties.
%

Following length normalization, we compute conditional probabilities,  $\mu(X_m|X_n)$ for \(m,n = 1, 2, \ldots, n\), by looking at the response probabilities received from the LLM when subjected to the aforementioned iterative prompts.


% \CD{Are these $X_i$ rather than $s_i$ or that's just for clusters? Also, where do we get the probabilities $\mu(X_i)$ and $\mu(X_j \mid X_i)$ that are used below?}

\subsubsection{Clustering via Symbolic Execution}
To handle functional diversity, the generated program snippets are clustered based on their semantic equivalence using Algorithm~\ref{alg:clustering}. 
%symbolic execution. 
%Symbolic execution analyses each program to compute execution paths and constraints. 
%Two programs \(s^{(i)}\) and \(s^{(j)}\) are considered functionally equivalent if their symbolic execution traces \(T(s^{(i)})\) and \(T(s^{(j)})\) satisfy:
%\[
%T(s^{(i)}) \equiv T(s^{(j)}) \quad \text{or} \quad T(s^{(i)}) \subseteq T(s^{(j)}).
%\]

% Let \(C = \{c_1, c_2, \ldots, c_k\}\) denote the clusters formed, where each cluster \(c_j\) groups semantically equivalent programs. 
%The clustering procedure ensures that semantically redundant responses are grouped together, focusing on functional equivalence rather than syntactic similarity.
%The algorithm used is same as Algorithm~\ref{alg:clustering} from the previous section.

\subsubsection{Mutual Information Estimation}
% Once clustering is complete, mutual information is computed over the resulting clusters to quantify epistemic uncertainty. 
% The pseudo joint distribution is defined as:
% \[
% \tilde{p}(s_1, \ldots, s_n \mid x) = p(s_1 \mid F_0(x)) \prod_{i=2}^n p(s_i \mid F_{i-1}(x, s_1, \ldots, s_{i-1})).
% \]
% The marginal distribution is then:
% \[
% \tilde{p}^\otimes(s_1, \ldots, s_n) = \prod_{i=1}^n p(s_i \mid F_0(x)).
% \]

% The mutual information is then computed as:
% \[
% I(\tilde{p}) = D_{KL}(\tilde{p} \| \tilde{p}^\otimes).
% \]

Once clustering is complete, mutual information is computed over the resulting clusters to quantify epistemic uncertainty. 

The aggregated probabilities are defined as:
\[
\mu_1'(X_i) = \sum_{j \in D(i)} \mu(X_j), \quad
\mu_2'(X_t \mid X_i) = \sum_{j \in D(t)} \mu(X_j \mid X_i),
\]
where \( X_i \) and \( X_t \) are clusters, \( \mu(X_j) \) represents the probability of the output \( X_j \), and \( \mu(X_j \mid X_i) \) is the conditional probability of \( X_j \) given \( X_i \). The set \( D(i) \) contains all outputs assigned to the cluster \( X_i \).

The normalized empirical distributions are:
\[
\hat{\mu}_1(X_i) = \frac{\mu_1'(X_i)}{Z}, \quad \text{where} \quad Z = \sum_{j \in S} \mu_1'(X_j),
\]
\[
\hat{\mu}_2(X_t \mid X_i) = \frac{\mu_2'(X_t \mid X_i)}{Z_i}, \quad \text{where} \quad Z_i = \sum_{j \in S} \mu_2'(X_j \mid X_i).
\]
Here, \( \hat{\mu}_1(X_i) \) is the normalized marginal distribution for cluster \( X_i \), and \( \hat{\mu}_2(X_t \mid X_i) \) is the normalized conditional distribution for \( X_t \) given \( X_i \). The terms \( Z \) and \( Z_i \) are normalization constants to ensure that the distributions sum to 1.

The joint and pseudo-joint distributions are defined as:
\[
\hat{\mu}(X_i, X_t) = \hat{\mu}_1(X_i) \hat{\mu}_2(X_t \mid X_i), \quad
\hat{\mu}^\otimes(X_i, X_t) = \hat{\mu}_1(X_i) \sum_{j \in S} \hat{\mu}_1(X_j) \hat{\mu}_2(X_t \mid X_j).
\]
The joint distribution \( \hat{\mu}(X_i, X_t) \) combines the marginal and conditional distributions, while the pseudo-joint distribution \( \hat{\mu}^\otimes(X_i, X_t) \) assumes independence between clusters.

Finally, the mutual information is computed as:
\[
\hat{I}(\gamma_1, \gamma_2) = \sum_{i, t \in S} \hat{\mu}(X_i, X_t) \ln \left( \frac{\hat{\mu}(X_i, X_t) + \gamma_1}{\hat{\mu}^\otimes(X_i, X_t) + \gamma_2} \right).
\]
Here, \( \gamma_1 \) and \( \gamma_2 \) are small stabilization parameters to prevent division by zero, and \( S \) is the set of clusters.

% \subsubsection{Finite-Sample Estimation with Clusters}
% To estimate mutual information from the finite sample of clusters \(C = \{c_1, \ldots, c_k\}\), the probability of a cluster \(c\) is computed as:
% \[
% p(c \mid x) = \sum_{s \in c} p(s \mid x).
% \]
% The empirical mutual information is then:
% \[
% \hat{I}_k = \sum_{c \in C} \hat{p}(c \mid x) \ln \left( \frac{\hat{p}(c \mid x)}{\prod_{i=1}^n \hat{p}(c_i \mid x)} \right),
% \]
% where \(\hat{p}(c \mid x)\) is the observed cluster probability. 
% The empirical mutual information is then:
% \[
% \hat{I}_k(\gamma_1, \gamma_2) = \sum_{i, t \in S} p(X_i, X_t) \ln \left( \frac{p(X_i, X_t) + \gamma_1}{p^\otimes(X_i, X_t) + \gamma_2} \right),
% \]
% where $\hat{I}_k(\gamma_1, \gamma_2)$ is the estimated mutual information, $p(X_i, X_t)$ represents the joint empirical distribution over clusters $X_i$ and $X_t$, and $p^\otimes(X_i, X_t)$ denotes the product of their marginal probabilities for cluster pairs. 
% The stabilization parameters $\gamma_1$ and $\gamma_2$ are included to handle cases where $p(X_i, X_t)$ or $p^\otimes(X_i, X_t)$ might be zero, preventing undefined logarithmic terms. 
% This formulation allows for a robust estimation of mutual information by quantifying the dependencies between cluster distributions while addressing numerical instabilities.

% Entropy regularization is then applied for stability:
% \[
% \hat{I}_k(\gamma) = \sum_{c \in C} \hat{p}(c \mid x) \ln \left( \frac{\hat{p}(c \mid x) + \gamma}{\prod_{i=1}^n (\hat{p}(c_i \mid x) + \gamma)} \right).
% \]

This mutual information score serves as a proxy for epistemic uncertainty. 
High \(\hat{I}\) values signal significant uncertainty.
% \[
% a_\lambda(x) =
% \begin{cases} 
% 1 & \text{if } \hat{I}_k \geq \lambda, \\
% 0 & \text{otherwise.}
% \end{cases}
% \]

\paragraph{Motivating example revisited for \textnormal{\gptturbo}.}
We now illustrate the MI computation for our motivating example from Section~\ref{sec:motivating}. 
We use 3 samples with an iteration prompt length of 2.
All responses from \gptturbo fall in the same cluster and hence following the earlier formula for MI we get:

\begin{align*}
    \hat{I}(\gamma_1, \gamma_2) = \hat{\mu}(X_1, X_1) \ln \left( \frac{\hat{\mu}(X_1, X_1) + \gamma_1}{\hat{\mu}^\otimes(X_1, X_1) + \gamma_2} \right) 
    = 1.0 \ln \left( \frac{1.0 + \gamma_1}{1.0 + \gamma_2} \right).
\end{align*}

Since $\gamma_1$ and $\gamma_2$ are zero as per Abbasi~\etal~\cite{abbasi2024believe}:
%\begin{align*}
   $\ln \left( \frac{1.0 + \gamma_1}{1.0 + \gamma_2} \right) = \ln(1.0) = 0$.
%\end{align*}
Therefore, $\hat{I} = 1.0 \cdot 0 = 0$.
%\begin{align*}
%    \hat{I} &= 1.0 \cdot 0 = 0.
%\end{align*}
%
% EXPLANATION: Zero MI is expected as the paper says this "For the S.E. and M.I. methods, the responses for a large number of queries can be clustered into a single group, and therefore the semantic entropy and mutual information scores are zero."
%
A \emph{zero} MI indicates very low uncertainty in the model's responses, suggesting a high likelihood of correctness as we will show later in the paper.

\paragraph{Motivating example revisited for \textnormal{\salesforce}.}
For \codegenmonoC, all three responses fall in their own separate clusters \ie  $R_1 \in X_1$, $R_2 \in X_2$, $R_3 \in X_3$ and hence,
%
%\begin{align*}
    $P(X_1) = 0.9995, P(X_2) = 0.0004, P(X_3) = 0.0001$.
%\end{align*}

For brevity, we omit the computation of the marginals which was carried out using the same formula used by Abbasi~\etal~\cite{abbasi2024believe} which for this example looks like:

\[
\hat{I}(\gamma_1, \gamma_2)
= \sum_{i=1}^3 \sum_{j=1}^3
  \hat{\mu}(X_i, X_j)
  \ln\!\Biggl(\frac{\hat{\mu}(X_i, X_j) + \gamma_1}
                   {\hat{\mu}^\otimes(X_i, X_j) + \gamma_2}\Biggr).
\]

Computing the terms individually then looks like:

\[
\begin{aligned}
    (X_1, X_1) &: 0.800 \ln \left( \frac{0.800}{0.9990} \right) = -0.1788, & (X_1, X_2) &: 0.120 \ln \left( \frac{0.120}{0.0004} \right) = 0.6845, \\
    (X_1, X_3) &: 0.040 \ln \left( \frac{0.040}{0.0001} \right) = 0.2397, & (X_2, X_1) &: 0.050 \ln \left( \frac{0.050}{0.0004} \right) = 0.5050, \\
    (X_2, X_2) &: 0.010 \ln \left( \frac{0.010}{0.00000016} \right) = 0.1104, & (X_2, X_3) &: 0.020 \ln \left( \frac{0.020}{0.00000004} \right) = 0.2624, \\
    (X_3, X_1) &: 0.020 \ln \left( \frac{0.020}{0.0001} \right) = 0.1609, & (X_3, X_2) &: 0.020 \ln \left( \frac{0.020}{0.00000004} \right) = 0.2624, \\
    (X_3, X_3) &: 0.010 \ln \left( \frac{0.010}{0.00000001} \right) = 0.1382.
\end{aligned}
\]

Then total MI then is:
%\begin{align*}
    $\hat{I} = -0.1788 + 0.6845 + 0.2397 + 0.5050 + 0.1104 + 0.2624 + 0.1609 + 0.2624 + 0.1382 = 2.1847$.
%\end{align*}

%\begin{align*}
%    \hat{I} &= -0.1788 + 0.6845 + 0.2397 + 0.5050 + 0.1104 \\
%    &\quad + 0.2624 + 0.1609 + 0.2624 + 0.1382 = 2.1847.
%\end{align*}


While lower uncertainty is intuitively desirable, as discussed in Section~\ref{sec:symex}, the uncertainty estimate is assessed against an abstention threshold to derive meaningful conclusions (see Section~\ref{sec:usability}).

%This shows \emph{moderately} high uncertainty, due to all responses being in their own respective clusters. 

%By integrating symbolic execution-based clustering with iterative prompting and mutual information estimation, this methodology captures both functional diversity and epistemic uncertainty in program generation. 




\section{Discussion of Assumptions}\label{sec:discussion}
In this paper, we have made several assumptions for the sake of clarity and simplicity. In this section, we discuss the rationale behind these assumptions, the extent to which these assumptions hold in practice, and the consequences for our protocol when these assumptions hold.

\subsection{Assumptions on the Demand}

There are two simplifying assumptions we make about the demand. First, we assume the demand at any time is relatively small compared to the channel capacities. Second, we take the demand to be constant over time. We elaborate upon both these points below.

\paragraph{Small demands} The assumption that demands are small relative to channel capacities is made precise in \eqref{eq:large_capacity_assumption}. This assumption simplifies two major aspects of our protocol. First, it largely removes congestion from consideration. In \eqref{eq:primal_problem}, there is no constraint ensuring that total flow in both directions stays below capacity--this is always met. Consequently, there is no Lagrange multiplier for congestion and no congestion pricing; only imbalance penalties apply. In contrast, protocols in \cite{sivaraman2020high, varma2021throughput, wang2024fence} include congestion fees due to explicit congestion constraints. Second, the bound \eqref{eq:large_capacity_assumption} ensures that as long as channels remain balanced, the network can always meet demand, no matter how the demand is routed. Since channels can rebalance when necessary, they never drop transactions. This allows prices and flows to adjust as per the equations in \eqref{eq:algorithm}, which makes it easier to prove the protocol's convergence guarantees. This also preserves the key property that a channel's price remains proportional to net money flow through it.

In practice, payment channel networks are used most often for micro-payments, for which on-chain transactions are prohibitively expensive; large transactions typically take place directly on the blockchain. For example, according to \cite{river2023lightning}, the average channel capacity is roughly $0.1$ BTC ($5,000$ BTC distributed over $50,000$ channels), while the average transaction amount is less than $0.0004$ BTC ($44.7k$ satoshis). Thus, the small demand assumption is not too unrealistic. Additionally, the occasional large transaction can be treated as a sequence of smaller transactions by breaking it into packets and executing each packet serially (as done by \cite{sivaraman2020high}).
Lastly, a good path discovery process that favors large capacity channels over small capacity ones can help ensure that the bound in \eqref{eq:large_capacity_assumption} holds.

\paragraph{Constant demands} 
In this work, we assume that any transacting pair of nodes have a steady transaction demand between them (see Section \ref{sec:transaction_requests}). Making this assumption is necessary to obtain the kind of guarantees that we have presented in this paper. Unless the demand is steady, it is unreasonable to expect that the flows converge to a steady value. Weaker assumptions on the demand lead to weaker guarantees. For example, with the more general setting of stochastic, but i.i.d. demand between any two nodes, \cite{varma2021throughput} shows that the channel queue lengths are bounded in expectation. If the demand can be arbitrary, then it is very hard to get any meaningful performance guarantees; \cite{wang2024fence} shows that even for a single bidirectional channel, the competitive ratio is infinite. Indeed, because a PCN is a decentralized system and decisions must be made based on local information alone, it is difficult for the network to find the optimal detailed balance flow at every time step with a time-varying demand.  With a steady demand, the network can discover the optimal flows in a reasonably short time, as our work shows.

We view the constant demand assumption as an approximation for a more general demand process that could be piece-wise constant, stochastic, or both (see simulations in Figure \ref{fig:five_nodes_variable_demand}).
We believe it should be possible to merge ideas from our work and \cite{varma2021throughput} to provide guarantees in a setting with random demands with arbitrary means. We leave this for future work. In addition, our work suggests that a reasonable method of handling stochastic demands is to queue the transaction requests \textit{at the source node} itself. This queuing action should be viewed in conjunction with flow-control. Indeed, a temporarily high unidirectional demand would raise prices for the sender, incentivizing the sender to stop sending the transactions. If the sender queues the transactions, they can send them later when prices drop. This form of queuing does not require any overhaul of the basic PCN infrastructure and is therefore simpler to implement than per-channel queues as suggested by \cite{sivaraman2020high} and \cite{varma2021throughput}.

\subsection{The Incentive of Channels}
The actions of the channels as prescribed by the DEBT control protocol can be summarized as follows. Channels adjust their prices in proportion to the net flow through them. They rebalance themselves whenever necessary and execute any transaction request that has been made of them. We discuss both these aspects below.

\paragraph{On Prices}
In this work, the exclusive role of channel prices is to ensure that the flows through each channel remains balanced. In practice, it would be important to include other components in a channel's price/fee as well: a congestion price  and an incentive price. The congestion price, as suggested by \cite{varma2021throughput}, would depend on the total flow of transactions through the channel, and would incentivize nodes to balance the load over different paths. The incentive price, which is commonly used in practice \cite{river2023lightning}, is necessary to provide channels with an incentive to serve as an intermediary for different channels. In practice, we expect both these components to be smaller than the imbalance price. Consequently, we expect the behavior of our protocol to be similar to our theoretical results even with these additional prices.

A key aspect of our protocol is that channel fees are allowed to be negative. Although the original Lightning network whitepaper \cite{poon2016bitcoin} suggests that negative channel prices may be a good solution to promote rebalancing, the idea of negative prices in not very popular in the literature. To our knowledge, the only prior work with this feature is \cite{varma2021throughput}. Indeed, in papers such as \cite{van2021merchant} and \cite{wang2024fence}, the price function is explicitly modified such that the channel price is never negative. The results of our paper show the benefits of negative prices. For one, in steady state, equal flows in both directions ensure that a channel doesn't loose any money (the other price components mentioned above ensure that the channel will only gain money). More importantly, negative prices are important to ensure that the protocol selectively stifles acyclic flows while allowing circulations to flow. Indeed, in the example of Section \ref{sec:flow_control_example}, the flows between nodes $A$ and $C$ are left on only because the large positive price over one channel is canceled by the corresponding negative price over the other channel, leading to a net zero price.

Lastly, observe that in the DEBT control protocol, the price charged by a channel does not depend on its capacity. This is a natural consequence of the price being the Lagrange multiplier for the net-zero flow constraint, which also does not depend on the channel capacity. In contrast, in many other works, the imbalance price is normalized by the channel capacity \cite{ren2018optimal, lin2020funds, wang2024fence}; this is shown to work well in practice. The rationale for such a price structure is explained well in \cite{wang2024fence}, where this fee is derived with the aim of always maintaining some balance (liquidity) at each end of every channel. This is a reasonable aim if a channel is to never rebalance itself; the experiments of the aforementioned papers are conducted in such a regime. In this work, however, we allow the channels to rebalance themselves a few times in order to settle on a detailed balance flow. This is because our focus is on the long-term steady state performance of the protocol. This difference in perspective also shows up in how the price depends on the channel imbalance. \cite{lin2020funds} and \cite{wang2024fence} advocate for strictly convex prices whereas this work and \cite{varma2021throughput} propose linear prices.

\paragraph{On Rebalancing} 
Recall that the DEBT control protocol ensures that the flows in the network converge to a detailed balance flow, which can be sustained perpetually without any rebalancing. However, during the transient phase (before convergence), channels may have to perform on-chain rebalancing a few times. Since rebalancing is an expensive operation, it is worthwhile discussing methods by which channels can reduce the extent of rebalancing. One option for the channels to reduce the extent of rebalancing is to increase their capacity; however, this comes at the cost of locking in more capital. Each channel can decide for itself the optimum amount of capital to lock in. Another option, which we discuss in Section \ref{sec:five_node}, is for channels to increase the rate $\gamma$ at which they adjust prices. 

Ultimately, whether or not it is beneficial for a channel to rebalance depends on the time-horizon under consideration. Our protocol is based on the assumption that the demand remains steady for a long period of time. If this is indeed the case, it would be worthwhile for a channel to rebalance itself as it can make up this cost through the incentive fees gained from the flow of transactions through it in steady state. If a channel chooses not to rebalance itself, however, there is a risk of being trapped in a deadlock, which is suboptimal for not only the nodes but also the channel.

\section{Conclusion}
This work presents DEBT control: a protocol for payment channel networks that uses source routing and flow control based on channel prices. The protocol is derived by posing a network utility maximization problem and analyzing its dual minimization. It is shown that under steady demands, the protocol guides the network to an optimal, sustainable point. Simulations show its robustness to demand variations. The work demonstrates that simple protocols with strong theoretical guarantees are possible for PCNs and we hope it inspires further theoretical research in this direction.
\section*{Acknowledgment}
This work was supported by the National Natural Science Foundation of China (62441239,~U23A20319,~62172056,~62472394,~62192784, \\ U22B2038) as well as the 8th Young Elite Scientists Sponsorship Program by CAST (2022QNRC001).



\phantomsection
\addcontentsline{toc}{section}{References}
\bibliographystyle{amsalpha}
\bibliography{bib/scholar}

\appendix

\crefalias{section}{appendix} %
\section{Computational lower bound for recovery}\label{sec:lb-weak-recovery}
In this section, we prove \cref{thm:main-theorem-weak-recovery} by showing that there exists an efficient algorithm that reduces testing to weak recovery in SBM. We will show that there exists a efficiently computable testing function (shown in \cref{alg:reduction-test-recovery}) that is large with constant probability if the input is sampled from $\SSBM(n,\frac{d}{n},\e,k)$ and is small with high probability if the input is sampled from $\bbG(n, d/n)$. This will lead to a contradiction with low-degree lower bounds of testing if we assume Conjecture \ref{conj:low-degree}.

Before describing the algorithm, we restate \cref{thm:main-theorem-weak-recovery} here for completeness.

\begin{theorem}[Full version of \cref{thm:main-theorem-weak-recovery}]\label{thm:full-main-theorem-weak-recovery}
    Let $k,d\in \N^+$ be such that $k\leq O(1), d\leq o(n)$.
    Assume that for any $d' \in \N^+$ such that $0.999 d\leq d'\leq d$, Conjecture \ref{conj:low-degree} holds for distribution $P = \SSBM(n,\frac{d'}{n},\e,k)$ and distribution $Q=\bbG(n, \frac{d'}{n})$.
    Then for any small constants $\delta_1,\delta_2$, no $\exp\Paren{n^{0.99}}$ time algorithm can achieve recovery rate $n^{-0.5+\delta_1}$ in the $k$-stochastic block model when $\epsilon^2 d\leq (1-\delta_2) k^2$.
\end{theorem}

The reduction that we consider is the following.

\begin{algorithmbox}[Reduction from testing to weak recovery]
    \label{alg:reduction-test-recovery}
    \mbox{}\\
    \textbf{Input:} A random graph $G$ with equal probability sampled from \Erdos-\Renyi model or stochastic block model, and target recovery rate $\delta$, parameters $\e,k,d$. \\
    \textbf{Output:} Testing statistics $g(Y)\in \R$, where $Y$ is the adjacency matrix.\\
    \textbf{Algorithm:} 
    \begin{enumerate}[1.]
        \item Let $\eta=0.001\delta_2$, where $\delta_2=1-\e^2 d/k^2$. Obtain subgraph $G_1$ by subsampling each edge with probability $1-\eta$, and let $G_2= G\setminus G_1$. 
        \item Obtain estimator $\hat{M}_0$ by running weak recovery algorithm on graph $G_1$.
        \item Obtain $\hat{M}$ by applying correlation preserving projection (see \cref{thm:correlation-preserving-projection}) on $\hat{M}_0$ to the set $\cK=\Set{M\in [-1/\delta,1/\delta]^{n\times n}: M+\frac{1}{k\delta} \Ind \Ind^{\top} \succeq 0 \,, \Tr(M + \frac{1}{k\delta} \Ind \Ind^{\top}) \leq n/\delta}$. 
        \item Return testing statistics $g(Y)=\iprod{\hat{M},Y_2-\frac{\eta d}{n} \one \one^{\top}}$, where $Y_2$ is the adjacency matrix for the graph $G_2$.
    \end{enumerate}
\end{algorithmbox}

To prove \cref{thm:full-main-theorem-weak-recovery}, we will show that the testing statistics $g(Y)$ from \cref{alg:reduction-test-recovery} satisfies the following two lemmas.

\begin{lemma}
\label[lemma]{lem:lb_sbm}
    Let $Y$ be the adjacency matrix of the graph sampled from the symmetric $k$-stochastic block model $\SSBM(n,\frac{d}{n},\e,k)$ and $M^\circ\in \Set{-1/k,1-1/k}^{n\times n}$ be the corresponding community membership matrix.
    Suppose that $\iprod{\hat{M}_0,M^{\circ}}\geq  n^{-0.5+\delta_1} \normf{\hat{M}_0} \normf{M^{\circ}}$ and $\normf{\hat{M}} = \Theta(\normf{M^{\circ}})$.
    Then \cref{alg:reduction-test-recovery} outputs testing statistics $g(Y)\in \R$ such that $g(Y)\geq \Omega\Paren{n^{0.5(1+\delta_1)}}$.
\end{lemma}

\begin{lemma}
\label[lemma]{lem:ub_ER}
    Let $Y$ be the adjacency matrix of the graph sampled from \Erdos-\Renyi random graph $\bbG(n, d/n)$. 
    With probability at least $1-\exp(-n^{0.001\delta_1})$, \cref{alg:reduction-test-recovery} outputs $g(Y) \leq O(n^{0.5+\delta_1/3})$ in polynomial time.
\end{lemma}

Combining \cref{lem:lb_sbm} and \cref{lem:ub_ER}, \cref{thm:main-theorem-weak-recovery} follows as a corollary.

\begin{proof}[Proof of \cref{thm:main-theorem-weak-recovery}]
Suppose that there is a $\exp\Paren{n^{0.99}}$ time algorithm which outputs estimator $\hat{M}_0$ such that $\iprod{\hat{M}_0,M^{\circ}}\geq  n^{-0.5+\delta_1} \normf{\hat{M}_0} \normf{M^{\circ}}$.
Let $f(Y)=\mathbf{1}_{g(Y)\geq 0.001n^{0.5+\delta_1/2}}$.
When $\e^2 d \geq \Omega(k^2)$, combining \cref{lem:lb_sbm} and \cref{lem:ub_ER}, we have
    \begin{equation*}
        \frac{\E_P f(Y)}{\sqrt{\text{Var}_Q(f(Y))}} \geq \exp(n^{0.001\delta_1})\,.
    \end{equation*}
    By the low-degree likelihood ratio upper bound \cref{thm:ldlr-sbm}, when $\e^2 d\leq (1-\delta_2)k^2$, we have 
    \begin{equation*}
       \max_{\text{deg}(f)\leq n^{0.01}}\frac{\E f(Y)}{\sqrt{\text{Var}_Q(f(Y))}}\leq \exp(k^2) \,.
    \end{equation*}
    
    Since $f(Y)$ can be evaluated in $O(\exp\Paren{n^{0.99}})$ time, assuming Conjecture ~\ref{conj:low-degree}, we then have 
   \begin{equation*}
    \frac{\E f(Y)}{\sqrt{\text{Var}_Q(f(Y))}} \lesssim \max_{\text{deg}(f)\leq n^{0.01}}\frac{\E f(Y)}{\sqrt{\text{Var}_Q(f(Y))}}\leq O(1)\,,
   \end{equation*}
which leads to a contradiction.
As a result, assuming Conjecture ~\ref{conj:low-degree}, we cannot achieve weak recovery in $\exp\Paren{n^{0.99}}$ time when $\epsilon^2 d\leq (1-\delta_2)k^2$. 
\end{proof}

\subsection{Correlation preserving projection}

In this part, we prove that we can project the estimator into the set of matrices with bounded entries and bounded nuclear norm, while preserving correlation.
\begin{lemma}\label[lemma]{lem:corr-preserve-proj}
Let $M^{\circ}\in \{-1/k,1-1/k\}^{n\times n}$ be a symmetric matrix with rank-$(k+1)$.
For any $\delta\leq O(1)$, given matrix $\hat{M}_0$ such that $\iprod{\hat{M}_0,M^{\circ}}\geq \delta\normf{\hat{M}_0} \normf{M^{\circ}}$, there is a polynomial time algorithm which outputs $\hat{M} \in \cK$ such that 
$\iprod{\hat{M},M^{\circ}}\geq \Omega(1)\cdot \delta\normf{\hat{M}} \normf{M^{\circ}}$ and $\normf{\hat{M}}\geq \Omega(\normf{M^{\circ}})$, where
\begin{equation*}
    \cK=\Set{M\in [-1/\delta,1/\delta]^{n\times n}: M+\frac{1}{k\delta} \Ind \Ind^{\top} \succeq 0 \,, \Tr(M + \frac{1}{k\delta} \Ind \Ind^{\top}) \leq n/\delta} \,.
\end{equation*}
\end{lemma}
\begin{proof}
    We apply the correlation preserving projection from \cite{Hopkins17} (restated in \cref{thm:correlation-preserving-projection}).
    By definition, $M^{\circ} = X^{\circ} (X^{\circ})^{\top} - \frac{1}{k} \Ind \Ind^{\top}$ is in $\cK$.
    Let $N$ be the matrix that minimizes $\normf{N}$ subject to $N\in \cK^\prime$ and $\iprod{N,\hat{M}_0}\geq \delta \normf{M^{\circ}} \normf{\hat{M}_0}$, where
    \begin{equation*}
    \cK^\prime=\Set{M\in [-1,1]^{n\times n}: M+\frac{1}{k} \Ind \Ind^{\top} \succeq 0 \,, \Tr(M + \frac{1}{k} \Ind \Ind^{\top}) \leq n} \,.
    \end{equation*}
    Using ellipsoid method, this semidefinite program can be solved in polynomial time.
    By \cref{thm:correlation-preserving-projection}, we have $\iprod{N,M^{\circ}}\geq \Omega(1)\cdot \delta\normf{N} \normf{M^{\circ}}$ and $\normf{N} \geq \delta\normf{M^{\circ}}$.
    We let $\hat{M}=\frac{\normf{M^\circ}}{\normf{N}}\cdot N$.
    Then it follows that $\hat{M}\in \cK$, $\normf{\hat{M}}=\normf{M^\circ}$ and $\iprod{\hat{M},M^{\circ}}\geq \Omega(\delta) \normf{\hat{M}}\cdot \normf{M^\circ}$.
\end{proof}

\subsection{Proof of \cref{lem:lb_sbm}}

In this section, we prove \cref{lem:lb_sbm}.
\begin{proof}[Proof of \cref{lem:lb_sbm}]
We consider the decomposition that $Y_2-\frac{\eta d}{n} \one \one^{\top}= \frac{\epsilon \eta d}{n}M^{\circ}+W_2$ where $W_2$ is a symmetric random matrix with independent and zero mean entries.
By \cref{lem:decoupling}, there exists an i.i.d zero mean symmetric matrix $\tilde{W}_2$ that is independent with $Y_1$, and satisfies that the entries in $\tilde{W}_2-W_2$ are independent with zero mean and have variance bounded by $O(d^3/n^3)$, conditioning on the subsampled graph $Y_1$ and community matrix $M^\circ$.
As result, we have
\begin{equation*}
    \iprod{Y_2-\frac{\eta d}{n} \one \one^{\top},\hat{M}}= \iprod{\frac{\epsilon \eta d}{n}M^{\circ}, \hat{M}}+ \iprod{W_2-\tilde{W}_2, \hat{M}}+\iprod{\tilde{W}_2, \hat{M}}\,.
\end{equation*}

For the first term $\iprod{\frac{\epsilon \eta d}{n}M^{\circ}, \hat{M}}$, it follows from \cref{lem:corr-preserve-proj} that
\begin{equation*}
\begin{split}
\iprod{M^{\circ}, \hat{M}}
& \geq \Omega \Paren{\frac{\epsilon \eta d}{n}} \delta\normf{\hat{M}} \normf{M^{\circ}} \\
& \geq \Omega \Paren{\frac{\delta\epsilon \eta d}{n}} \normf{M^{\circ}}^2\,.
\end{split}
\end{equation*}
As with probability at least $1-\exp(-n^{0.001})$, we have $\normf{M^{\circ}}^2\geq \Omega(n^2)$, and as result $\iprod{M^\circ,\hat{M}}\geq \Omega(n\delta \e d)$.

For bounding the second term $\iprod{W_2-\tilde{W}_2, \hat{M}}$, we condition on the subsampled graph $Y_1$ and the community matrix $M^\circ$. 
With probability at least $1-\exp(-n^{\delta_1})$, we have 
\begin{align*}
    |\iprod{W_2-\tilde{W}_2, \hat{M}}|\leq \normf{W_2-\tilde{W}_2} \cdot \normf{\hat{M}}\lesssim \sqrt{\frac{d^3}{n^3} \cdot n^{2+\delta_1} \cdot n^2}= \sqrt{n^{1+\delta_1}d^3}\,. 
\end{align*}

For the third term $\iprod{\tilde{W}_2, \hat{M}}$, we again conditional on the subsampled graph $Y_1$ and the community matrix $M^\circ$.
We note that it can be written as the summation of independent zero-mean random variables
\begin{equation*}
    \iprod{\tilde{W}_2, \hat{M}}= \sum_{i,j} \tilde{W}_2(i,j)\hat{M}(i,j)\,. 
\end{equation*}
where $\tilde{W}_2(i,j)\hat{M}(i,j)$ are independent zero mean variables bounded by $O(1/\delta)$ for all $i\leq j$. 
Moreover, we have
\begin{equation*}
    \sum_{i,j} \hat{M}(i,j)^2 \E\Brac{\tilde{W}_2(i,j)^2}\lesssim \frac{d}{n}\sum_{i,j} \hat{M}(i,j)^2 \leq O(n^2\cdot \frac{d}{n})=O(nd)\,.
\end{equation*}
By Bernstein inequality, we have
\begin{equation*}
    \Pr\Brac{\Abs{\sum_{i,j} \tilde{W}_2(i,j)\hat{M}(i,j)}\geq 100 t}\leq \exp\Paren{-t^2/(nd+t/\delta)} \,. 
\end{equation*}
Taking $t=n^{(1+\delta_1)/2}\sqrt{d}$ and $\delta\geq n^{-0.5+\delta_1}$, we have
\begin{equation*}
    \Pr\Brac{\Abs{\sum_{i,j} \tilde{W}_2(i,j)\hat{M}(i,j)}\geq n^{0.5(1+\delta_1)}\sqrt{d}}\leq \exp\Paren{-n^{\delta_1/2}}\,.
\end{equation*}

As a result, when $d\leq n^{o(1)},k\leq n^{o(1)},\delta\geq n^{-0.5+\delta_1}, \e=\Theta(1/\sqrt{d})$, with constant probability, we have
\begin{equation*}
    \iprod{Y_2-\frac{\eta d}{n} \one \one^{\top},\hat{M}}
    \geq \Omega(n^{0.5+\delta_1}\sqrt{d})\,.
\end{equation*}
\end{proof}

\subsection{Proof of \cref{lem:ub_ER}}

In this section, we prove \cref{lem:ub_ER}.
\begin{proof}[Proof of \cref{lem:ub_ER}]
We will use the fact that $Y_2-\frac{\eta d}{n} \one \one^{\top}$ and $\hat{M}$ are approximately independent. 
More precisely, let $W_2=Y_2-\frac{\eta d}{n} \one \one^{\top}$, by \cref{lem:decoupling}, there exists symmetric zero mean matrix $\tilde{W}_2$ with independent entries  such that each entry in $\tilde{W}_2-W_2$ has zero mean variance bounded by $O(d^3/n^3)$ conditioning on $\hat{M}$. 
By triangle inequality, we have
\begin{equation*}
   g(Y)=\Abs{\iprod{Y_2-\frac{\eta d}{n}\one \one^{\top},\hat{M}} }\leq \Abs{\iprod{W_2-\tilde{W}_2,\hat{M}}}+\Abs{\iprod{\tilde{W}_2,\hat{M}}}\,. 
\end{equation*}
For bounding the first term $\iprod{W_2-\tilde{W}_2, \hat{M}}$, we condition on the subsampled graph $Y_1$. 
With probability at least $1-\exp(-n^{\delta_1/3})$, we have 
\begin{align*}
    |\iprod{W_2-\tilde{W}_2, \hat{M}}|\leq \normf{W_2-\tilde{W}_2} \cdot \normf{\hat{M}}\lesssim \sqrt{\frac{d^3}{n^3} \cdot n^{2+\delta_1/3} \cdot n^2}= \sqrt{d^3 n^{1+\delta_1/3}}\,. 
\end{align*}
For the second term, we note that $\iprod{\tilde{W}_2, \hat{M}}$ can be written as the summation of independent zero-mean random variables
\begin{equation*}
    \iprod{\tilde{W}_2, \hat{M}}= \sum_{i,j} \tilde{W}_2(i,j)\hat{M}(i,j)\,. 
\end{equation*}
where $\tilde{W}_2(i,j)\hat{M}(i,j)$ are independent zero mean variables bounded by $O(1/\delta)$ for $i\leq j$. 
Moreover, we have
\begin{equation*}
    \sum_{i,j} \hat{M}(i,j)^2 \E\Brac{\tilde{W}_2(i,j)^2}\lesssim \frac{d}{n}\sum_{i,j} \hat{M}(i,j)^2 \leq O(n^2\cdot \frac{d}{n})=O(nd)\,.
\end{equation*}
By Bernstein inequality, we have
\begin{equation*}
    \Pr\Brac{\Abs{\sum_{i,j} \tilde{W}_2(i,j)\hat{M}(i,j)}\geq 100 t}\geq \exp\Paren{-t^2/(nd+t/\delta)} \,. 
\end{equation*}
Taking $t=n^{0.5+\delta_1/3}\sqrt{d}$ and $\delta\geq n^{-0.5+\delta_1}$, we have
\begin{equation*}
    \Pr\Brac{\Abs{\sum_{i,j} \tilde{W}_2(i,j)\hat{M}(i,j)}\geq n^{0.5+\delta_1/3}}\leq \exp\Paren{-n^{0.001\delta_1}}\,.
\end{equation*}
\end{proof}

\subsection{Proof of \cref{thm:main-theorem-super-constant-blocks}}

In this part, we give the proof of \cref{thm:main-theorem-super-constant-blocks}, which is the same as the proof of \cref{thm:main-theorem-weak-recovery} except that we assume stronger low-degree conjecture.
\begin{proof}[Proof of \cref{thm:main-theorem-super-constant-blocks}]
Suppose that there is a polynomial time algorithm which outputs estimator $\hat{M}_0$ such that $\iprod{\hat{M}_0,M^{\circ}}\geq  n^{-0.5+\delta_1} \normf{\hat{M}_0} \normf{M^{\circ}}$.
Let $f(Y)=\mathbf{1}_{g(Y)\geq 0.001n^{0.5+\delta_1/2}}$.
When $0.001k^2\leq \e^2 d \leq (1-\delta_2)k^2$, combining \cref{lem:lb_sbm} and \cref{lem:ub_ER}, we have
    \begin{equation*}
        \frac{\E_P f(Y)}{\sqrt{\text{Var}_Q(f(Y))}} \geq \exp(n^{0.001})\,.
    \end{equation*}
    Since $f(Y)$ can be evaluated in $O(\exp(n^{0.001}))$ time, assuming Conjecture ~\ref{conj:low-degree}, by \cite{Hopkins18}(stated in \cref{thm:ldlr-sbm}), we have 
   \begin{equation*}
    \frac{\E f(Y)}{\sqrt{\text{Var}_Q(f(Y))}} \lesssim \max_{\text{deg}(f)\leq n^{0.99}}\frac{\E f(Y)}{\sqrt{\text{Var}_Q(f(Y))}}\leq \exp(k^2)\,.
   \end{equation*}
When $k^2\leq n^{0.001}$, this leads to a contradiction.
As a result, assuming Conjecture ~\ref{conj:eldlr}, we cannot achieve recovery rate $n^{-0.5+\delta_1}$ in polynomial time when $\epsilon^2 d\leq (1-\Omega(1))k^2$. 
\end{proof}



\section{Computational lower bound for learning stochastic block model}\label{sec:lb-learning}

\subsection{Computational lower bound for learning the edge connection probability matrix}

In this section, we prove \cref{thm:lb-edge-probability} by showing that there exists an efficient algorithm that reduces testing to learning in SBM. 
The reduction of algorithm \cref{alg:reduction-test-learning} is similar to that of \cref{alg:reduction-test-recovery}. The proof of \cref{thm:lb-edge-probability} is also a similar proof by contradiction to the proof of \cref{thm:main-theorem-weak-recovery}.

Before describing the algorithm, we restate \cref{thm:lb-edge-probability} here for completeness.
\begin{theorem}[Restatement of \cref{thm:lb-edge-probability}]
\label{thm:lb-edge-probability-restatement}
    Let $k,d\in \N^+$ be such that $k\leq n^{o(1)}, d\leq o(n)$.
    Assume that for any $d'\in \N^+$ such that $0.999 d\leq d'\leq d$, Conjecture \ref{conj:eldlr} holds with distribution $P$ given by $\SSBM(n,\frac{d'}{n},\e,k)$ and distribution $Q$ given by \Erdos-\Renyi graph model $\bbG(n, \frac{d'}{n})$. 
    Then given graph $G\sim \SSBM(n,\frac{d}{n},\e,k)$, no $\exp\Paren{n^{0.99}}$ time algorithm can output $\theta\in [0,1]^{n\times n}$ achieving error rate $\normf{\theta-\thetanull}^2\leq 0.99kd/4$ with constant probability, where $\thetanull$ is the ground truth sampled edge connection probability matrix.
\end{theorem}

The reduction that we consider is the following.

\begin{algorithmbox}[Reduction from testing to learning]
    \label{alg:reduction-test-learning}
    \mbox{}\\
    \textbf{Input:} A random graph $G$ with equal probability sampled from \Erdos-\Renyi model or stochastic block model. \\
    \textbf{Output:} Testing statistics $g(Y)\in \R$, where $Y$ is the centered adjacency matrix\\
    \textbf{Algorithm:} 
    \begin{enumerate}[1.]
        \item Obtain subgraph $G_1$ by subsampling each edge with probability $1-\eta=0.999$, and let $G_2= G\setminus G_1$. 
        \item Run learning algorithm on $G_1$, and obtain estimator $\hat{\theta}\in \R^{n\times n}$
        \item Obtain $\hat{M}$ by running correlation preserving projection on $\hat{\theta}-\frac{d}{n}\Ind \Ind^{\top}$ to the set $\cK=\Set{M\in [-1,1]^{n\times n}: M+\frac{1}{k} \Ind \Ind^{\top} \succeq 0 \,, \Tr(M + \frac{1}{k} \Ind \Ind^{\top}) \leq n}$. 
        \item Construct the testing statistics $g(Y)=\iprod{\hat{M},Y_2-\frac{\eta d}{n}\Ind \Ind^{\top}}$, where $Y_2$ is the adjacency matrix for the graph $G_2$.
    \end{enumerate}
\end{algorithmbox}

Before proving \cref{thm:lb-edge-probability}, we first show the relationship between learning edge connection probability and weak recovery.
 \begin{lemma}\label[lemma]{lem:reduction-learning-recovery}
     Consider the distribution of $\SSBM(n,\frac{d}{n},\e,k)$ with $d\le n^{o(1)}$. 
     Suppose give graph $Y\sim \SSBM(n,\frac{d}{n},\e,k)$, the estimator $\hat{\theta}\in \R^{n\times n}$ achieves error rate $\normf{\hat{\theta}- \thetanull}\leq \frac{1}{2}\sqrt{0.99kd}$ with constant probability, then $\hat{\theta}-d/n$ achieves weak recovery when $\e^2 d\geq 0.99k^2$.
 \end{lemma}
\begin{proof}
By the relation between edge connection probability matrix $\thetanull$ and the community matrix $M^\circ$, We have
    \begin{equation*}
        \iprod{\hat{\theta}-\frac{d}{n}\Ind \Ind^\top,M^\circ}=\iprod{\hat{\theta}-\theta^\circ,M^\circ}+\iprod{\theta^\circ-\frac{d}{n}\Ind \Ind^\top,M^\circ}=\iprod{\hat{\theta}-\theta^\circ,M^\circ}+\iprod{\frac{\e d}{n}M^\circ,M^\circ}\,.
    \end{equation*}
    For the first term, since with constant probability, $\normf{\hat{\theta}-\theta^\circ}\leq \sqrt{0.99kd}$, we have
    \begin{equation*}
      \Abs{\iprod{\hat{\theta}-\theta^\circ,M^\circ}}\leq \normf{M^\circ}\normf{\hat{\theta}-\theta^\circ}\leq 
        \normf{M^\circ} \sqrt{0.99kd}\,.
    \end{equation*}
    For the second term, since with overwhelming high probability, $\normf{M^\circ}\geq \frac{n}{\sqrt{k}}(1-\frac{1}{k})$, we have
    \begin{equation*}
        \iprod{\frac{\e d}{n}M^\circ,M^\circ}=\frac{\e d}{n}\normf{M^\circ}^2\geq \frac{\e d }{2\sqrt{k}} \normf{M^\circ}\,.
    \end{equation*}
    Therefore, when $\e^2 d> 0.999 k^2$, we have 
    \begin{equation*}
        \iprod{\hat{\theta}-\frac{d}{n}\Ind \Ind^{\top},M^\circ}\geq \frac{\e d }{2\sqrt{k}} \normf{M^\circ}-\normf{M^\circ} \frac{\sqrt{0.99kd}}{2}\geq \Omega\Paren{\frac{\e d \normf{M^\circ}}{\sqrt{k}}} \,.
    \end{equation*}
    On the other hand, by triangle inequality
    \begin{equation*}
        \Normf{\hat{\theta}-\frac{d}{n}\Ind \Ind^{\top}}\leq  \Normf{\hat{\theta}-\theta^\circ}+ \Normf{\theta^\circ-\frac{d}{n}\Ind \Ind^{\top}}\leq O(\sqrt{kd}+\frac{\e d}{\sqrt{k}}) \leq O\Paren{\e d/\sqrt{k}}\,,
    \end{equation*}
Therefore we have 
\begin{equation*}
    \iprod{\hat{\theta}-\frac{d}{n}\Ind \Ind^{\top},M^\circ}\geq \Omega(\normf{M^\circ}\cdot \normf{\hat{\theta}-\frac{d}{n}\Ind \Ind^{\top}})\,.
\end{equation*}
    We thus conclude that with constant probability, $\hat{\theta}-\frac{d}{n}\Ind \Ind^\top$ achieves weak recovery when $\e^2 d\geq 0.99k^2$.
\end{proof}
With \cref{lem:reduction-learning-recovery}, the proof of lower bound for learning the edge connection probability matrix of stochastic block model follows as a corollary.
\begin{proof}[Proof of \cref{thm:lb-edge-probability}]
    By \cref{lem:reduction-learning-recovery}, suppose an $\exp\Paren{n^{0.99}}$ time algorithm achieves error rate less than $0.99\sqrt{kd}$ in estimating the edge connection probability matrix, then in \cref{alg:reduction-test-learning}, $\hat{\theta}-\frac{d}{n}$ achieves weak recovery when $\e^2 d=0.99k^2$.
    We let $f(Y)=\mathbf{1}_{g(Y)\geq 0.001 \e^2 d^2/k}$. 

    We show that with constant probability under $P$, we have $f(Y)=1$.    
    We essentially follow the proof of \cref{lem:lb_sbm} with $\delta$ taken as a constant, except that we take a different strategy for bounding
    $\iprod{W_2-\tilde{W}_2, \hat{M}}$.
    By \cref{lem:spectral-concentration-sbm}, we have, with probability at least $1-o(1)$, the following spectral radius bounds on the symmetric random matrices
\begin{equation*}
    \normop{W_2-\tilde{W}_2}\leq O\Paren{\sqrt{d\log(n)}\cdot \sqrt{\frac{d}{n}}}\,.
\end{equation*}
Therefore, by Trace inequality, we have
\begin{equation*}
\begin{split}
|\iprod{W_2-\tilde{W}_2, \hat{M}}|
& = |\iprod{W_2-\tilde{W}_2, \hat{M}+\frac{1}{k\delta}\Ind \Ind^{\top}} - \iprod{W_2-\tilde{W}_2, \frac{1}{k\delta}\Ind \Ind^{\top}}| \\
& \leq |\iprod{W_2-\tilde{W}_2, \hat{M}+\frac{1}{k\delta}\Ind \Ind^{\top}}| + |\iprod{W_2-\tilde{W}_2, \frac{1}{k\delta}\Ind \Ind^{\top}}| \\
& \leq \normop{W_2-\tilde{W}_2} \Tr(\hat{M}+\frac{1}{k\delta}\Ind \Ind^{\top}) + \normop{W_2-\tilde{W}_2} \Tr(\frac{1}{k\delta}\Ind \Ind^{\top}) \\
& \leq O\Paren{\sqrt{d\log(n)}\cdot \sqrt{\frac{d}{n}} (1+\frac{1}{k})\frac{n}{\delta}}\\
& = O\Paren{(d+\frac{d}{k})\frac{\sqrt{n\log(n)}}{\delta}} \,.
\end{split}
\end{equation*}

    With the same reasoning, by \cref{lem:ub_ER}, with probability at least $1-\exp(-n^{0.001})$ under distribution $Q$, we have $f(Y)=0$. 
    Therefore, we have $\RPQ(f)\geq \exp(n^{0.001})$. 
    Since $f(A)$ can be evaluated in $O\Paren{\exp\Paren{n^{0.99}}}$ time, assuming conjecture \ref{conj:low-degree} we have
   \begin{equation*}
       R_{P,Q}(f)\coloneqq \frac{\E f(A)}{\sqrt{\text{Var}_Q(f(A))}} \lesssim \max_{\text{deg}(f)\leq n^{0.99}}\frac{\E f(A)}{\sqrt{\text{Var}_Q(f(A))}}\,.
   \end{equation*}
    On the other hand, by low-degree lower bound stated in \cref{thm:ldlr-sbm}, we have 
    \begin{equation*}
       \max_{\text{deg}(f)\leq n^{0.99}}\frac{\E f(A)}{\sqrt{\text{Var}_Q(f(A))}}\leq \exp(k^2)\,. 
    \end{equation*}
Since we have $\exp(n^{0.001})\gg\exp(k^2)$ when $k\leq n^{o(1)}$, this leads to a contradiction. 
\end{proof}

\subsection{Computational lower bound for learning graphon}
In this part, we give formal proof of \cref{thm:lb-learning-graphon}. 

\begin{theorem}[Restatement of \cref{thm:lb-learning-graphon}]
    Let $k,d\in \N^+$ be such that $k\leq O(1), d\leq o(n)$.
    Assume that Conjecture \ref{conj:low-degree} holds with distribution $P$ given by $\SSBM(n,\frac{d}{n},\e,k)$ and distribution $Q$ given by \Erdos-\Renyi graph model $\bbG(n, \frac{d}{n})$. 
    Then no $\exp\Paren{n^{0.99}}$ time algorithm can output a $\poly(n)$-block graphon function $\hat{W}:[0,1]\times [0,1]\to [0,1]$ such that $\GW(\hat{W},\Wnull) \leq \frac{d}{3n}\sqrt{\frac{k}{d}}$  with $1-o(1)$ probability under distribution $P$ and distribution $Q$(where $\Wnull$ is the underlying graphon of the corresponding distribution).
\end{theorem}
\begin{proof}
Let $W_0$ be the graphon function underlying the distribution $\bbG(n,\frac{d}{n})$ and $W_1$ be the graphon function underlying the distribution $\SSBM(n,\frac{d}{n},\e,k)$, we have $\GW(W_0,W_1)\geq \frac{d}{n}\sqrt{\frac{0.99k}{d}}$ when $\e^2 d\geq 0.99k^2$. 

Now suppose there is a polynomial time algorithm, which given random graph $G$ sampled from an arbitrary symmetric $k$-stochastic block model, outputs an $n$-block graphon function $\hat{W}:[0,1]\times [0,1]\to [0,1]$ achieving error $\frac{d}{3n}\sqrt{\frac{k}{d}}$ with probability $1-o(1)$.
Then one can construct the testing statistics by taking
\begin{equation*}
f(Y) =
\begin{cases}
    1, & \text{if } \GW(\hat{W}, W_0) \leq \frac{d}{3n} \sqrt{\frac{k}{d}} \\
    0, & \text{otherwise}
\end{cases}
\end{equation*}
We have $f(Y)=1$ with probability $1-o(1)$ under the distribution of symmetric stochastic block model $\SSBM(n,\frac{d}{n},\e,k)$.
By triangle inequality, we have $f(Y)=0$ with probability $1-o(1)$ under the distribution $\bbG(n,\frac{d}{n})$. 
Therefore we have $\RPQ(f)\geq \omega(1)$.

Now since the function $\hat{W}$ can be represented as a symmetric matrix with $\poly(n)$ number of rows and columns, and moreove since $W_0$ is a constant function,
\begin{equation*}
    \GW(\hat{W},W_0)= \int_0^1 \int_0^1 (\hat{W}(x,y)-W_0(x,y))^2 dx dy\,.
\end{equation*}
Therefore, the function $f(\cdot )$ can be evaluated in polynomial time. 
This contradicts the low-degree lower bound (\cref{thm:ldlr-sbm}) assuming \cref{conj:low-degree}.
\end{proof}

















\section{Low-degree recovery lower bound for learning dense stochastic block model}
\label{sec:low-degree-reduction}
In this part, we give unconditional lower bound against low-degree polynomial estimators for the edge connection probability matrix in stochastic block model, via implementing reduction from hypothesis testing to weak recovery using low-degree polynomials.
For simplicity, we focus on the dense graph.
\begin{theorem}[Low-degree lower bound for learning]\label{thm:low-degree-graphon}
Let $n\in \N^+$ and $\ell\leq n^{0.001}$.
Let $d=\Theta(n)$.
    Let $\cF_{n,\ell}$ be the set of degree-$\ell$ polynomials mapping from $n\times n$ symmetric matrices to $n\times n$ symmetric matrices.
    Suppose $\thetanull\in [0,1]^{n\times n}, Y\in \Set{0,1}$ are edge connection probability matrix and adjacency matrix sampled from symmetric stochastic block model $\SSBM(n,\frac{d}{n},\e,k)$.
    Then for $k\leq n^{0.001}$, we have 
    \begin{equation*}
       \min_{f\in \cF_{n,\ell}}\max_{\e\in [0,1]} \E_{(Y,\thetanull)\sim \SSBM(n,\frac{d}{n},\e,k)} \normf{f(Y)-\thetanull}^2\geq \Omega(k\cdot n)\,.
    \end{equation*}
\end{theorem}

\subsection{Construction of the low-degree polynomial}
For simplicity, we define the community matrix of symmetric stochastic block model.
\begin{definition}[Community matrix for stochastic block model]\label{def:community-matrix}
    Under symmetric stochastic block model $\SSBM(n,\frac{d}{n},\e,k)$, we define the community matrix $X^\circ\in \Set{\pm 1}$ as following: $X^\circ(i,j)=1$ if vertex $i,j$ have the same community label and $X^\circ(i,j)=0$ otherwise.
\end{definition}

Given the polynomial function $f:\R^{n\times n}\to \R^{n\times n}$. We consider a graph with $2n$ nodes and randomly partition the nodes into two equal-sized sets $S_1$ and $S_2$.   
Let $X=\frac{n}{\e d} \Paren{f(Y_1)-\frac{d}{n}}$ where $Y_1$ is the subgraph induced by vertices in $S_1$.
We construct the polynomial function $g: \R^{n\times n}\to \R$ as following:
\begin{equation}\label{eq:testing-polynomial}
    g(Y)=\Iprod{\Paren{Y_{12}-\frac{d}{n}}X\Paren{Y_{12}-\frac{d}{n}}, Y_{2}-\frac{d}{n}}\,,
\end{equation}
where $Y_{12}\in \R^{n\times n}$ is the adjacency matrix of the bipartite graph between vertices in $S_1$ and $S_2$, and $Y_2$ is the adjacency matrix of the induced subgraph supported on $S_2$.

We show the lower bound of this polynomial under the symmetric stochastic block model, and the upper bound of this polynomial under the \Erdos-\Renyi graph model.
\begin{lemma}\label[lemma]{lem:expectation-planted}
    Let $\thetanull, Y$ be the edge connection probability matrix and adjacency matrix sampled from the planted distribution $\SSBM(n,\frac{d}{n},\e,k)$. 
    Let $X_1^\circ$ be the community matrix of the subgraph induced by vertices in $S_1$.
    Suppose in \cref{eq:testing-polynomial}, $\E\normf{X-X^\circ}^2\leq o(n^2)$, then we have $\E g(Y)\geq \Paren{\frac{\epsilon d}{n}}^3 n^4$. 
\end{lemma}


\begin{lemma}\label[lemma]{lem:variance-null}
    When the graph is sampled from the null distribution $\bbG(n,\frac{d}{n})$, we have
    $\E g(Y)=0$ and $\sqrt{\text{Var}(g(Y))}\leq d^{3/2}\cdot n^{1-\Omega(1)}$. 
\end{lemma}

Combining \cref{lem:expectation-planted} and \cref{lem:variance-null}, \cref{thm:low-degree-graphon} follows as a corollary: 
\begin{proof}[Proof of \cref{thm:low-degree-graphon}]
Suppose there is a degree-$n^{0.001}$ polynomial $f:\R^{n\times n}\to \R^{n\times n}$ which gives error rate $o(n\cdot k)$. 
Let $X=\frac{n}{\e d}\Paren{f(Y)-\frac{d}{n}}$.
Then we have 
\begin{equation*}
    \normf{X-X^\circ}=\frac{n^2}{\e^2 d^2}\normf{f(Y)-\thetanull}^2\leq o\Paren{\frac{n^2}{\e^2 d^2} kn}\leq o\Paren{\frac{k}{\e^2 d}\cdot \frac{n}{d} \cdot n^2}\,.
\end{equation*}

When $\epsilon^2 d \geq 0.001k^2$ and $d=\Theta(n)$, 
we have $\E\normf{X-X^\circ}^2\leq o(n^2)$. 
combining \cref{lem:expectation-planted} and \cref{lem:variance-null}, we have
    \begin{equation*}
        \frac{\E g(Y)}{\sqrt{\text{Var}(g(Y))}} \geq n^{0.001}\,.
    \end{equation*}
    Since $g(Y)$ is a degree-$\ell$ polynomial with $\ell\leq n^{0.01}$, by \cite{Hopkins18}, we have 
   \begin{equation*}
       \frac{\E g(Y)}{\sqrt{\text{Var}(g(Y))}} \leq \exp(k^2)\,.
   \end{equation*}
When $\exp(k^2)\leq n^{0.001}$, this leads to a contradiction.
As result, we conclude that no degree-$n^{0.001}$ polynomial can achieve error rate $o(nk)$.
\end{proof}


\subsection{Proof of \cref{lem:expectation-planted}}

In this section, we analyze the property of the polynomial in \cref{eq:testing-polynomial} under the $k$-symmetric stochastic block model, and give a proof for \cref{lem:expectation-planted}. 
\begin{proof}[Proof of \cref{lem:expectation-planted}] 
    Let $X^\circ_{12}\in \{\pm 1\}^{n\times n}$ be the community matrix for the bipartite graph between $S_1$ and $S_2$, i.e for $i\in S_1$ and $j\in S_2$, we have $X^\circ_{12}(i,j)=1$ if $i,j$ belongs to the same community and $X^\circ_{12}(i,j)=-1$ if $i,j$ belongs to different communities. 
    Moreover, we let $X^\circ_1$ be the community matrix for the induced subgraph on $S_1$ and let $X^\circ_2$ be the community matrix for the induced subgraph on $S_2$. 
    Then we have $Y_{12}=\frac{\e d}{n}X^\circ_{12}+W_{12}$, $Y_1=\frac{\e d}{n}X^\circ_1+W_1$ and $Y_2=\frac{\e d}{n}X^\circ_2+W_2$, where 
    \begin{itemize}
        \item $W_{12}, W_1,W_2$ are independent,
        \item $(W_{12},W_1,W_2)$ is independent with $(X^\circ_{12},X^\circ_1,X^\circ_2)$,
        \item every entry in $W_{12},W_1,W_2$ has zero mean.
    \end{itemize}

    
    Then we have
    \begin{align*}
         \E g(Y) & =\E\Iprod{(Y_{12}-\frac{d}{n})X(Y_{12}-\frac{d}{n}),Y_2-\frac{d}{n}}\\
         &= \Paren{\frac{\e d}{n}}^3 \E\Iprod{X^\circ_{12} XX^\circ_{12},X^\circ_2}+ \E \Iprod{W_{12}XW_{12},W_2}+\frac{2\e d}{n}\E\Iprod{W_{12}XX^\circ_{12},W_2}\\
         &= \Paren{\frac{\e d}{n}}^3 \E\Iprod{X^\circ_{12} XX^\circ_{12},X^\circ_2}\,. 
    \end{align*}
    Since $\E\Iprod{X^\circ_{12} X_1^\circ X^\circ_{12},X^\circ_2}\geq \Omega(n^4)$ and
    \begin{equation*}
        \E\Iprod{Y^\circ_{12} (X_1-X_1^\circ) X^\circ_{12},X^\circ_2}\leq \sqrt{\E\normf{X_1-X_1^\circ}^2\cdot \E\normf{X^\circ_{12}X^\circ_2 X^\circ_{12}}^2}\leq o(n^4)\,.
    \end{equation*}
    Therefore, we have $\E\Iprod{X^\circ_{12} XX^\circ_{12},X^\circ_2}\geq \Omega(n^4)$.
    and the claim follows. 
\end{proof}

\subsection{Proof of \cref{lem:variance-null}}

In this section, we analyze the property of the polynomial defined in \cref{eq:testing-polynomial}, under the \Erdos-\Renyi graph distribution, and give a proof for \cref{lem:variance-null}. 

\begin{proof}[Proof of \cref{lem:variance-null}]
Under the \Erdos-\Renyi graph distribution, the entries in $Y_2-\frac{d}{n}$ are i.i.d zero mean random variables, independent with $Y_{12}$ and $Y_2$(i.e the rest of the graph). 
As result, we have $\E g(Y)=0$.

It remains to bound the variance of the polynomial under the \Erdos-\Renyi graph distribution, which is to say, we bound 
\begin{equation*}
    \E g(Y)^2= \E \Iprod{\Paren{Y_{12}-\frac{d}{n}}X\Paren{Y_{12}-\frac{d}{n}}, Y_2-\frac{d}{n}}^2\,. 
\end{equation*}
Let $W_{12}=Y_{12}-\frac{d}{n}$ and $W_2=Y_2-\frac{d}{n}$.
The main observation is that $X,W_{12},W_2$ are all independent. 
As result, let $Z=W_{12}XW_{12}$, we have
\begin{align*}
    \E \Iprod{W_{12}XW_{12}, W_2}^2=\sum_{ij}\Paren{\E W_2(i,j)Z(i,j)}^2
    = \sum_{ij} \E W_2^2(i,j) \E Z(i,j)^2 
    = \frac{d}{n} \E \normf{Z}^2\,.
\end{align*}
As $\normf{X}\leq O(n)$ without loss of generality, and $\norm{W_{12}}\leq \sqrt{d}\log(n)$ with overwhelming high probability, we have
\begin{equation*}
    \E \normf{Z}^2\leq O\Paren{n^2 d^2\log^4(n)}\,.
\end{equation*}
Therefore we have
\begin{equation*}
    \E \Iprod{W_{12}XW_{12}, W_2}^2\leq O\Paren{n d^3\log^4(n)}\,.
\end{equation*}
By taking the square root, we conclude the proof. 
\end{proof}
\begin{figure}[!t]
    \raggedright
    \includegraphics[width=\linewidth]{figs/modserve-overview.pdf}
    \caption{Overview of the \sysname{} architecture.}
    \label{fig:overview}
\end{figure}
\section{Useful algorithmic results}\label{sec:algo-results}
In this section, we provide two algorithmic results from previous work that will be useful in our paper.

\subsection{Correlation preserving projection}
Given a vector $P$ that has constant correlation with an unknown vector $Y$, \cite{Hopkins17} shows that one can project the vector $P$ into a convex set containing $Y$, and preserve the constant correlation with $Y$. 
\begin{theorem}[Correlation preserving projection, theorem 2.3 in \cite{Hopkins17}]
  \label{thm:correlation-preserving-projection}
  Let $\delta\in \R^+$
  Let $\cC$ be a convex set and $Y\in \cC$.
  Let $P$ be a vector with $\iprod{P,Y}\ge \delta \cdot \norm{P}\cdot \norm{Y}$.
  Then, if we let $Q$ be the vector that minimizes $\norm{Q}$ subject to $Q\in \cC$ and $\iprod{P,Q}\ge \delta \cdot \norm{P}\cdot \norm{Y}$, we have
  \begin{equation}
    \iprod{Q,Y}\ge \delta/2 \cdot \norm{Q}\cdot \norm{Y}\,.
  \end{equation}
  Furthermore, $Q$ satisfies $\norm{Q}\ge \delta \norm{Y}$.
\end{theorem}

We include their proof here for completeness.
\begin{proof}
    By construction, $Q$ is the Euclidean project of $0$ into the set $\Set{Q\in \cC|\iprod{P,Q}\geq \delta \norm{P}\norm{Y}}$. 
    By Pythagorean inequality, the Euclidean projection into a set decreases distances to points into the set.
    Therefore, $\norm{Y-Q}^2\le \norm{Y-0}^2$, which implies that $\iprod{Y,Q}\geq \norm{Q}^2/2$.
    Moreover, $\iprod{P,Q}\geq \delta \norm{P}\norm{Y}$, which implies $\norm{Q}\geq \delta \norm{Y}$ by Cauchy-Schwartz. 
    Thus, we can conclude that $\iprod{Y,Q}\geq \delta/2\cdot \norm{Y}\cdot \norm{Q}$.
\end{proof}

\subsection{Learning edge connection probability matrix via SVD}
\label{sec:learning_algorithm}

\begin{theorem}\label{thm:algorithm-learning-sbm}
    When $d\geq \log(n)$, there is a polynomial time algorithm which given the adjacency matrix of a graph sampled from symmetric $k$-stochastic block model $\SSBM(n,\frac{d}{n},\e,k)$, returns an estimator $\hat{\theta}\in [0,1]^{n\times n}$ such that $\normf{\thetanull-\theta}^2\leq kd$ with high probability.
\end{theorem}
\begin{proof}
    We take $\hat{\theta}$ as the best rank-$k$ approximation for the adjacency matrix. 
    Then since $\normop{A-\theta^\circ}\leq \sqrt{kd}$ with high probability, we have $\normop{\hat{\theta}-A}\leq \sqrt{kd}$ with high probability.
    By triangle inequality, we have $\normop{\hat{\theta}-\theta^\circ}\leq 2\sqrt{d}$.
    As result, we have
\( \normf{\hat{\theta}-\theta^\circ}\leq 2\sqrt{kd}\)\,.
\end{proof}

\section{Low-degree lower bound beyond constant number of blocks}\label{sec:beyond-constant-ldlr}

By extending the result of \cite{pmlr-v134-bandeira21a}, we can get a more general bound (with respect to $k$) on low degree likelihood ratio of k-SBM. The proof of the extended result follows trivially from the proof of Theorem 2.20 of \cite{pmlr-v134-bandeira21a}. Therefore, we only provide a proof sketch by pointing out the simple modifications that we need from the original proof.

\begin{theorem}[Restatement of \cref{thm:ldlr-sbm}]\torestate{
Let $d=o(n),k=o(\log(n))$ and $\e\in [0,1]$. 
Let $\mu:\{0,1\}^{n \times n} \rightarrow \mathbb{R}$ be the relative density of SBM $(n, d, \varepsilon, k)$ with respect to $G\left(n, \frac{d}{n}\right)$. 
Let $\mu^{\leqslant \ell}$ be the projection of $\mu$ to the degree-$\ell$
polynomials with respect to the norm induced by $G\left(n, \frac{d}{n}\right)$ For any constant $\delta>0$,
\begin{align*}
\left\|\mu^{\leqslant \ell}\right\| \text{ is } \begin{cases}
    \geqslant n^{\Omega(1)}, & \quad \text{if } \varepsilon^2 d > (1+\delta) k^2, \quad \ell \geqslant O(\log n) \\[8pt]
    \leqslant O_{\delta, \varepsilon, d} \left(\exp(k^2)\right), & \quad \text{if } \varepsilon^2 d < (1-\delta) k^2, \quad \ell < n^{0.99}
\end{cases}
\end{align*}
}
\end{theorem}

\begin{proof}
    In this proof, we stick to the notations of \cite{pmlr-v134-bandeira21a}. The only modification we need is that the size of the $\delta$-net of the unit sphere in $\R^k$, denoted by $C(\delta, k)$, is equal to $\exp(O_\delta(k))$. The size of the $\delta$-net $C(\delta, k)$ is crucial in Proposition 6.4 and Proposition 6.5 of \cite{pmlr-v134-bandeira21a} and is treated as constant in the proof of Theorem 2.16 and Theorem 2.20 of \cite{pmlr-v134-bandeira21a}.
    
    By plugging $C(\delta, k)=O_\delta(\exp(k))$ into the upper bound of the small deviation term $L_1$ in the proof of Theorem 2.16 of \cite{pmlr-v134-bandeira21a}, it follows that we have $\Norm{L^{\leq D}} = O_\delta(\exp(k))$ for the likelihood ratio $L$ and $D \leq o(n/\log(n))$. Then, the bound on low-degree likelihood ratio of k-SBM follows from the same reduction as in proof of Theorem 2.20 of \cite{pmlr-v134-bandeira21a}.
\end{proof}

\end{document}
