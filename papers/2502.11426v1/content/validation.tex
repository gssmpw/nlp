\section{Real-World Validation}
\label{sec::validation}
To validate the Verti-Bench evaluation results, we deploy one representative mobility system from each of the three classes, i.e., PID for classical, ACL for end-to-end, and TNT for hybrid, on a physical 1/10th scale open-source Verti-4-Wheeler robot~\cite{datar2024toward} on an off-road mobility testbed constructed by rocks, foam, grass, and wood, presenting different geometry, semantics, and obstacle features (Fig.~\ref{fig::real}). The testbed is constructed in two different configurations to include low and high elevation in order to validate the cross-elevation evaluation results from Verti-Bench.  

\begin{figure}[ht]
    \centering
    \includegraphics[width=\columnwidth]{figure/Exp.png}
    \caption{Physical Off-Road Testbed Similar to Verti-Bench.}
    \label{fig::real}
\end{figure}

Table \ref{tab::results} shows the physical validation results of the three mobility systems on both low and high elevation testbeds. In general, the trend of the physical experiment results matches with that of Verti-Bench evaluation results: On the low elevation testbed, both PID and TNT are able to finish all five trials, while ACL still suffers from poor generalization. PID is still the fastest due to a lack of consideration of elevation. The roll and pitch angles are small and stable for all systems due to the lower and smoother terrain elevation; 
On the high elevation testbed, the difference between TNT and PID starts to emerge. TNT succeeds all five trials, while PID fails two. TNT and PID also exhibit smaller roll and pitch angle respectively. PID is still the fastest. ACL, similar to all previous cases, fails three trials and experiences the largest roll and pitch angles. Notice that due to the difficulty in conducting physical experiments, we only limit to physically evaluating three systems and five trials each, totaling 30 trials. This contrast against our 10000 trials (ten systems, 1000 trials each) in Verti-Bench, which can achieve much better statistical significance, further suggests the utility of Verti-Bench to evaluate off-road mobility. 

\begin{table}
\caption{\textbf{Physical Validation of PID, ACL, and TNT:} Success Rate, Traversal Time, Roll, and Pitch.}
% Best results are shown in bold.} 
\centering
\resizebox{\columnwidth}{!}{%
\small
\setlength{\tabcolsep}{4pt}
\begin{tabular}{ccccccccc}
\toprule
                                          % & \multicolumn{3}{c}{\footnotesize {V4W}}                        \\
%\cmidrule(rl){1-2} \cmidrule(rl){3-4} \cmidrule(rl){5-6}
% \cmidrule(l){2-4} 
Low Elevation                    & PID & ACL & TNT \\
\midrule
{Success Rate $\uparrow$}      & {\textbf{5/5}} & {3/5} & {\textbf{5/5}}\\
{Traversal Time $\downarrow$}      & {\textbf{6.64s$\pm$1.09s}} & {7.03s}$\pm$0.47s & {8.90s}$\pm$0.76s\\
{Roll $\downarrow$}      & {6.45}\textdegree$\pm$5.20\textdegree & {6.20\textdegree$\pm$4.13\textdegree} & \textbf{6.02\textdegree$\pm$4.92\textdegree}\\
{Pitch $\downarrow$}      & {5.45\textdegree$\pm$3.37}\textdegree & {6.34}\textdegree$\pm$3.32\textdegree & \textbf{{4.74\textdegree$\pm$3.23\textdegree}}\\
% {$\Delta$ Roll $\downarrow$}      & {\textbf{0.51\textdegree$\pm$1.03\textdegree}} & {0.63\textdegree$\pm$6.07\textdegree} & {0.97\textdegree$\pm$1.6\textdegree}\\
% {$\Delta$ Pitch $\downarrow$}      & {\textbf{0.53\textdegree$\pm$0.68\textdegree}} & {0.080\textdegree$\pm$1.3\textdegree} & {0.57\textdegree$\pm$1.08\textdegree}\\
% {$\Delta$ Throttle $\downarrow$}      & {\textbf{0.042$\pm$0.15}} & {0.065}$\pm$0.34 & {0.053$\pm$0.31}\\
% {$\Delta$ Steering $\downarrow$}      & {\textbf{0.071$\pm$0.24}} & {0.153$\pm$0.38} & {0.087$\pm$0.23}\\
\midrule
High Elevation                   & PID & ACL & TNT  \\
\midrule
{Success Rate $\uparrow$}      & {3/5} & {2/5} & {\textbf{5/5}} \\
{Traversal Time $\downarrow$}     & \textbf{{11.00s}}$\pm$1.00s  & {16.00s}$\pm$0.72s & {17.50s$\pm$1.95s} \\
{Roll $\downarrow$}   & {10.14\textdegree$\pm$8.96\textdegree}   & {13.30\textdegree$\pm$12.72\textdegree}  & \textbf{{7.12}\textdegree$\pm$6.65\textdegree}\\
{Pitch $\downarrow$}   & \textbf{{7.61\textdegree$\pm$5.46\textdegree}}  & {9.78}\textdegree$\pm$6.76\textdegree  & {9.26\textdegree$\pm$8.41}\textdegree \\
\bottomrule
\end{tabular}%
}
\label{tab::results}
\end{table}
