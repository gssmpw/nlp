\section{Evaluation and Discussions}
\label{sec::evaluation}

\begin{figure*}[t]
    \centering
    \includegraphics[width=0.49\textwidth]{figure/success_rate.pdf}
    \includegraphics[width=0.49\textwidth]{figure/traversal_time.pdf}
    \includegraphics[width=0.49\textwidth]{figure/roll.pdf}
    \includegraphics[width=0.49\textwidth]{figure/pitch.pdf}
    \includegraphics[width=1\textwidth]{figure/legend.pdf}
    \caption{Success Rate, Traversal Time, Roll, and Pitch of Ten Off-Road Mobility Systems on 1000 Navigation Tasks. }
    \label{fig::evaluation}
\end{figure*}

We evaluate ten off-road mobility systems using Verti-Bench, ranging from purely classical, end-to-end learning, and hybrid systems. We present and discuss our evaluation results and point out future research directions. 

\subsection{Off-Road Mobility Systems for Evaluation}
The three classical off-road mobility systems include
\begin{itemize}
    \item PID: A controller that takes a local goal 10 m away from the robot on the global path and minimizes the error angle between the desired and vehicle heading by regulating the steering and maintaining a 3 m/s speed;
    \item Elevation Heuristics (EH): A controller that splits the elevation map in front of the current robot pose to five regions and drives toward the region with the most similar mean to the current terrain patch and lowest variance; 
    \item MPPI: An MPPI-based~\cite{williams2017model} planner that uses a 2D bicycle model for trajectory rollout and obstacle avoidance. 
\end{itemize}

The three systems based on end-to-end learning include
\begin{itemize}
    \item RL: A RL policy learned from trial and error~\cite{xu2024reinforcement}; 
    \item MCL: A RL policy learned from a manually designed curriculum~\cite{xu2024reinforcement};
    \item ACL: A RL policy learned using Automatic Curriculum Learning~\cite{xu2024verti}. 
\end{itemize}

The four hybrid (classical and learning) systems include

\begin{itemize}
    \item WMVCT: A planner based on a decomposed 6-DoF kinodynamic model (bicycle model for x, y, and yaw, elevation map for z, and neural network prediction for roll and pitch)~\cite{datar2024learning};
    \item MPPI-6: An MPPI-based planner with a learned full 6-DoF kinodynamic model for trajectory rollout~\cite{lee2023learning};
    \item TAL: An MPPI-based planner with a 6-DoF kinodynamic model that learns to attend to specific terrain patches~\cite{datar2024terrain};
    \item TNT: An MPPI-based planner that samples based on traversability and then unfolds 6-DoF kinodynamics~\cite{pan2024traverse}.
\end{itemize}
We reach out to the authors of the original papers for their implementations of their mobility systems. Considering Verti-Bench's focus on off-road mobility evaluation, we make minimal modifications to their implementations to interface with Verti-Bench so that their mobility systems are no longer dependent on any perception system. For example, visual odometry inputs are replaced with ground truth vehicle states from Verti-Bench; Real-world elevation mapping systems are skipped by directly providing their systems with ground truth Verti-Bench elevation maps. 

\subsection{Evaluation Results and Discussions}

The evaluation results are shown in Fig.~\ref{fig::evaluation}, including percentage of Succuss Rate and mean and variance of Traversal Time, Roll, and Pitch. We also divide the evaluation results with respect to the three elevation levels. 

In general, navigation performance significantly declines with increasing elevation levels, including reduced Success Rate and increased Traversal Time, Roll, and Pitch. In addition to mean, the variance of Roll and Pitch also drastically increases, indicating much less stable vehicle chassis when traversing high elevation environments. Among the three categories, end-to-end learned mobility systems achieve the worst performance, while hybrid systems outperform the other two in general. While ACL is expected to outperform MCL, the results suggest otherwise. Such results indicate that end-to-end learning methods, trained from other sources, still have much room for improvement in terms of generalization in Verti-Bench. For hybrid systems, TAL and TNT are the two top performing planners among all systems overall, achieving the highest Success Rate in all cases and lowest Roll and Pitch angle in most cases. WMVCT and MPPI-6 achieve good Success Rate in high elevation environments, but with large Roll and Pitch. Classical planners perform in between their end-to-end and hybrid counterparts. PID, due to its simplicity and robustness, performs very well in low elevation environments, with EH catching up on Success Rate when facing higher elevation.  MPPI does not perform well in most cases and only outperforms PID in terms of Success Rate in high elevation environments. 
Notice that Traversal Time is only averaged over successful trials and thus only indicates how fast a mobility system is given navigation success. 

Our evaluation results indicate the potential of hybrid mobility systems to tackle vertically challenging terrain by combining the best of both worlds of classical and learning approaches. The overall success of TAL and TNT indicates the importance of an accurate 6-DoF kinodynamic model enabled by sophisticated learning techniques in conjunction with a sampling-based motion planner. MPPI-6, with a 6-DoF kinodynamic model based on a simplistic neural network, underperforms TAL and TNT, while the inaccuracies introduced by WMVCT's efficient 6-DoF decomposition lead to the worst mobility performance among hybrid systems. The degraded performance of all hybrid systems in high elevation environments motivates further research, potentially to both increase the kinodynamic modeling accuracy and improve the sampling-based motion planner. On the other hand, it is surprising to see the superior performance of the simple PID planner in low elevation environments compared to the more sophisticated EH, whose advantage only starts to slightly emerge in high elevation environments. This observation reveals a tradeoff between system complexity and performance when facing simple environments. One potential future research direction is to develop off-road mobility systems composed of multiple planners with different complexities and specialties to fit different environments~\cite{choudhury2015planner}. Lastly, research of end-to-end learning approaches, despite their recent success in relatively benign indoor or on-road enviornments, still needs to focus on robustly generalizing to out-of-distribution scenarios, which are very common to encounter in off-road enviornments. 



