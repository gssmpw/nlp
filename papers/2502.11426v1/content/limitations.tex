\section{Limitations}
\label{sec::limitations}
Despite being a general and scalable benchmark, Verti-Bench still has a few limitations. 
Due to the high requirement to compute high-fidelity physics and a lack of GPU acceleration, Verti-Bench can only achieve near-real-time simulation speed, with real time factor ranging between 0.4 and 1.5 (faster and slower than real time respectively) depending on simulation complexity. Integrating with GPU accelerators to increase benchmarking efficiency is an important next step. 
Verti-Bench aims to evaluate off-road mobility and assumes ground truth perception is available to the mobility system. However, such an assumption does not hold in the real world. Future work will add realistic perception noises and test the robustness of mobility systems when facing imperfect vehicle state estimation, elevation and semantics mapping, and obstacle detection. Another direction of expanding the current Verti-Bench is to create more complex real-world counterparts than the current small-scale physical testbed so that the sim-to-real gap can be more extensively studied to further validate the efficacy and improve the fidelity of Verti-Bench evaluation. 