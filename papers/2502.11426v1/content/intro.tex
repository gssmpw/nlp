\section{Introduction}
\label{sec::intro}

Off-road autonomous mobile robots present unique opportunities in search and rescue, environment monitoring, scientific exploration, and other application domains.  
However, off-road autonomy presents unique challenges to both robot perception and vehicle mobility that distinguish from structured on-road environments~\cite{jackel2006darpa, naranjo2016autonomous, price2024expanding}, such as variable geometry, deformable surfaces, and natural obstacles. Considering recent advancement, standardized benchmarks for off-road autonomy are necessary to objectively and quantitatively evaluate and compare the progress of the off-road robotics community. 

Unlike off-road perception, evaluating off-road mobility is more difficult. A plethora of off-road perception datasets~\cite{wigness2019rugd, jiang2021rellis, triest2022tartandrive, min2022orfd, mortimer2024goose, knights2023wild, sharma2022cat, liu2024botanicgarden} are available to provide static ground truth labels to evaluate against perception systems' outputs, since the robot actions can be recorded and fed into the perception systems, not generated by them  (if actions are necessary at all). However, mobility systems produce new actions to drive robots to different states than those collected in the dataset, i.e., distribution shift, and thus cannot be evaluated by comparing against a static dataset. Therefore, vehicle-terrain interactions need to be unfolded during off-road mobility evaluation, which creates difficulty in standardization across research groups. 

Considering a lack of standard off-road mobility evaluation, researchers currently develop their own benchmarks to evaluate their mobility systems and re-implement previous systems to compare against. Such a practice, however, leads to ad-hoc evaluation in a few aspects: The evaluation platforms, either a physical or a simulated robot, vary in terms of robot size, weight, actuation mechanism, and/or levels of off-road physics simulation fidelity; The evaluation environments range from handcrafted off-road simulation features~\cite{rana2024towards, yu2024adaptive, young2020unreal, xu2024reinforcement, cai2025pietra, so2022sim}, small-scale indoor testbeds~\cite{xu2023efficient, datar2024toward, xiao2018review, xiao2015locomotive}, enclosed outdoor tracks~\cite{goldfain2019autorally, xiao2021learning, atreya2022high, karnan2022vi, pan2020imitation}, and large-scale real-world testing facilities~\cite{triest2022tartandrive, han2023model, jiang2021rellis}; Re-implementation of previous approaches on one's own robot is not only laborious, but also subject to misinterpretation of implementation details. Therefore, a standard off-road mobility benchmark which is general and scalable to all these aspects is desired for the off-road mobility research community. 

In this work, we present Verti-Bench (Fig.~\ref{fig:cover}), a general and scalable off-road mobility benchmark that focuses on extremely rugged, vertically challenging terrain with a variety of unstructured off-road features. Based on a high-fidelity multi-physics dynamics simulator, Chrono~\cite{tasora2016chrono}, Verti-Bench encapsulates variations in four orthogonal dimensions: Using the Sliced Wasserstein Autoencoder (SWAE)~\cite{kolouri2018sliced} and real-world off-road terrain data, off-road geometry is represented as a diverse set of 2.5D elevation maps; Ten terrain semantics classes, including seven rigid and three deformable, are designed with different distributions of physics parameters, e.g., friction coefficient, cohesive effect, and soil stiffness; Different types of natural obstacles, e.g., boulders and vegetations, are randomly distributed based on different densities; A set of off-road vehicles with different scales (from 1/10th to full scale) and actuation mechanisms (4-, 6-, and 8-wheeled and 2-tracked chassis, single- and double-wishbone, multilink, toebar leaf-spring, and special tensioning suspensions, as well as pitman-arm, rack-and-pinion, toebar, bellcrank/rotary arm, and differential steering) are provided, with the possibility of adding customized vehicles. Using Verti-Bench, we also provide datasets from expert demonstration, random exploration, failure cases (rolling over and getting stuck), as well as a gym-like interface for Reinforcement Learning (RL). We use Verti-Bench to benchmark ten off-road mobility systems, present our findings, and identify future off-road mobility research directions. To summarize, our contributions are: 
\begin{itemize}
    \item A general off-road mobility benchmark on vertically challenging terrain with 100 off-road environments and 1000 navigation tasks scalable to various vehicle types;
    \item Millions of off-road terrain features including geometry, semantics (rigid and deformable), and obstacles;
    \item Various datasets and a RL interface to facilitate data-driven off-road mobility;
    \item Findings and future research directions based on benchmark results of various off-road mobility systems.  
\end{itemize}

