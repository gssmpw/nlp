\section{Verti-Bench}
\label{sec::approach}

We present Verti-Bench's core high-fidelity multi-physics dynamics engine, diverse set of off-road features, including wide-ranging geometry, physics-grounded semantics, and natural obstacles, scalability to a variety of vehicle platforms, and standardized metrics to quantify off-road mobility performance. We also discuss various datasets we collect using Verti-Bench to complement real-world off-road mobility data to develop data-driven systems.

\subsection{Simulation}
Verti-Bench is based on Project Chrono~\cite{tasora2016chrono}, a high-fidelity multi-physics dynamics engine with a platform-independent open-source design implemented in C++ with a Python version, PyChrono. Compared to other commonly used robotics simulators (e.g., Gazebo, Unreal, Unity, PyBullet, MuJoCo, and IsaacGym with well-known physics limitations especially for differential-drive mobile robots~\cite{zifan_isaac}), Chrono is especially suitable to simulate complex off-road vehicle-terrain interactions involving suspension, tire, track, and terrain deformation, varying terrain contact friction, vehicle weight distribution and momentum, motor, powertrain, transmission, and wheel torque characteristics, aggressive vehicle poses with all six Degrees of Freedom (DoFs), etc. 
In Chrono, vehicle systems and terrain properties are made of rigid and flexible/compliant parts with constraints, motors and contacts, along with three-dimensional shapes for collision detection. 

One point worth noting is that Verti-Bench's choice of Chrono as its core simulator is primarily due to its high-fidelity multi-physics dynamics, a vital aspect for off-road mobility evaluation. However, Chrono is not the best simulator for photorealism, one focus of off-road perception simulation, which is out of scope of Verti-Bench. For the perception components, Verti-Bench has standard interfaces to provide ground truth elevation and semantics maps and obstacle occupancy grids. Another point is that Chrono is not yet GPU-accelerated. Combined with the high computation required for high physics fidelity, Chrono can only provide slightly faster-than-real-time simulation, depending on the complexity of the simulated environments (e.g., areas of deformable terrain and number, size, and number of mesh vertices of obstacles). Therefore, despite its intended efficient usage in off-road mobility evaluation, learning off-road mobility is expected to take a significant amount of training time with Verti-Bench (e.g., using our provided gym-like RL interface). 

In Chrono, each of the 100 Verti-Bench full-scale environments is constructed as a 129m$\times$129m world, with a resolution of 1m per pixel. Each environment can be down-scaled to cater vehicles of different sizes, e.g., 1/6th or 1/10th scale. Each pixel contains geometry, semantics, and obstacle information (details below). For each of the 100 environments, ten pairs of start and goal locations separated by 120 m are distributed in a circular manner, leading to a total of 1000 navigation tasks. 

\subsection{Geometry}
Real-world off-road terrain is characterized by various geometry in terms of elevation changes, e.g., slopes, hills, ditches, gullies, ravines, and other form of undulations. Some of such terrain can be traversed by certain types of off-road vehicles, while others cannot. Autonomous off-road mobility systems need to decide which of them can be attempted with what vehicle maneuvers. For example, a steep slope with low friction cannot be traversed at low speeds, but large vehicle momentum by high speeds at the bottom can help the vehicle ascend the top; Approaching a deep ditch quickly may get the vehicle stuck due to extensive suspension depression at the bottom, but slowly negotiating through is possible to mitigate suspension travel in order to maximize clearance.  

Therefore, the geometry of Verti-Bench environments is represented as 2.5D elevation maps created by SWAE~\cite{kolouri2018sliced} and real-world elevation data. To be specific, we physically construct vertically challenging terrain with boulders and rocks and use a Microsoft Azure Kinect RGB-D camera to create elevation maps of different real-world terrain surfaces~\cite{mikielevation2022}. We then use SWAE~\cite{kolouri2018sliced}, a scalable generative model that captures the rich and often nonlinear distribution of high-dimensional data, as a feature extractor to reduce the dimension of the real-world elevation maps while preserving the original elevation information in a latent space, from which samples can be drawn to generate new elevation maps that resemble real-world vertically challenging terrain. To further introduce diversity and quantification of Verti-Bench geometry, we scale the output of the trained SWAE to 30\%, 60\%, and 100\% and denote them as low, medium, and high elevation level (Fig.~\ref{fig::elevation} top). Each Verti-Bench environment is generated with 1/3 probability of each elevation level. Fig.~\ref{fig::elevation} bottom shows the histogram of elevation values of all three levels of terrain geometry. High elevation environments also have the largest variance (most rugged terrain), while low elevation environments are smoother. 

\begin{figure}[ht]
    \centering
    \includegraphics[width=\columnwidth]{figure/Elevations.png}
    \caption{Top: Low, Medium, and High Elevation Maps; Bottom: Elevation Histograms across Three Elevation Levels. }
    \label{fig::elevation}
\end{figure}


\subsection{Semantics}
In addition to geometry, off-road terrain also presents challenges in terms of semantics and its associated physical vehicle-terrain contact features, such as friction, slip, and deformability. 
For example, an autonomous off-road mobility system should be aware that when driving through an icy or sandy laterally inclined slope with low friction or high deformability, sideway sliding downhill or wheel sinkage due to imbalanced load and then rollover is possible, respectively.  

Therefore, we also add ten different semantics classes to the terrain elevation. We design seven rigid and three deformable semantics classes with different textures and distributions of physics parameters (Fig.~\ref{fig::semantics}). To be specific, the seven rigid semantics classes, i.e., grass, wood, gravel, dirt, clay, rock, and concrete, associate with a normal distribution of friction coefficient. When a pixel is sampled to be a certain terrain type, its friction coefficient is sampled from the corresponding distribution. We fix the restitution coefficient to 0.01 for all rigid semantics classes. For the three deformable terrain classes, i.e., snow, mud, and sand, we adopt the deformable Soil Contact Model (SCM) based on the Bekker-Wong model~\cite{laughery1990bekker} to simulate terrain deformation after wheel interaction: SCM presents the underlying terrain by a 2D grid and assumes each cell can only be displaced vertically and does not maintain any history other than the current vertical displacement. We hard-code three sets of physics parameters, including cohesive effect, soil stiffness, and hardening effect, for three different deformability levels, i.e., soft, medium, and hard. 
Verti-Bench also provides terrain with granular materials. But due to the slow simulation speed when simulating thousands of particles, it is only reserved for special evaluation circumstances where granular materials must be simulated and simulation speed is not of concern. All statistics of the ten terrain semantics classes can be found in Fig.~\ref{fig::semantics}. 

\begin{figure}[h]
    \centering
    \includegraphics[width=\columnwidth]{figure/TexturePie.png}
    \caption{Seven Rigid (percentage, mean and variance of friction coefficient, and texture) and Three Deformable (percentage, deformability, and texture) Terrain Semantics.} 
    \label{fig::semantics}
\end{figure}

To create various terrain semantics while maintaining simulation efficiency, each 129$\times$129 Verti-Bench environment is first partitioned into a 16$\times$16 grid, with each grid cell as a 9$\times$9 patch (one overlapping pixel between every pair of patches to assure connectivity).  
% each annotated with metadata including spatial coordinates and a unique identifier. 
To emulate real-world continuous terrain patches with same semantics and similar physical properties, we employ a cluster-based approach, where cluster centers are sampled from the environment. 
% from a Gaussian distribution centered on the mean patch coordinates. 
% This sampling strategy ensures spatial coherence in terrain distribution. 
Each patch is then associated with its nearest cluster center using Euclidean distance. For all patches associated with a cluster center, the same semantics class is sampled and corresponding texture assigned, with each patch's physical property sampled from a predetermined distribution (Fig.~\ref{fig::semantics}). 
This approach creates natural physical variations within every region of the same semantics class while maintaining semantics diversity across regions. 



% \begin{table}
% \centering
% \begin{tabular}{|c|c|c|c|c|}
% \hline
% \begin{tabular}[c]{@{}c@{}} Wood\\ 7.66\%\\ 0.45, 0.66\\ \includegraphics[width=0.14\columnwidth]{figure/wood1.jpg}\end{tabular} & 
% \begin{tabular}[c]{@{}c@{}} Grass\\ 8.27\%\\ 0.36, 0.07\\ \includegraphics[width=0.14\columnwidth]{figure/grass3.jpg}\end{tabular} & 
% \begin{tabular}[c]{@{}c@{}} Gravel\\ 8.50\%\\ 0.4, 0.03\\ \includegraphics[width=0.14\columnwidth]{figure/gravel2.jpg}\end{tabular} & 
% \begin{tabular}[c]{@{}c@{}} Dirt\\ 9.89\%\\ 0.4, 0.03\\ \includegraphics[width=0.14\columnwidth]{figure/dirt2.jpg}\end{tabular} & 
% \begin{tabular}[c]{@{}c@{}} Clay\\ 9.95\%\\ 0.4, 0.03\\ \includegraphics[width=0.14\columnwidth]{figure/clay2.jpg}\end{tabular}\\
% \hline
% \begin{tabular}[c]{@{}c@{}} Concrete\\ 10.87\%\\ 0.4, 0.03\\ \includegraphics[width=0.14\columnwidth]{figure/concrete1.jpg}\end{tabular} & 
% \begin{tabular}[c]{@{}c@{}} Rock\\ 11.45\%\\ 0.4, 0.03\\ \includegraphics[width=0.14\columnwidth]{figure/rock2.jpg}\end{tabular} & 
% \cellcolor{gray!40}\begin{tabular}[c]{@{}c@{}} Snow\\ 11.46\%\\ Soft\\ \includegraphics[width=0.14\columnwidth]{figure/snow3.jpg}\end{tabular} & 
% \cellcolor{gray!40}\begin{tabular}[c]{@{}c@{}} Mud\\ 10.27\%\\ Medium\\ \includegraphics[width=0.14\columnwidth]{figure/mud3.jpg}\end{tabular} & 
% \cellcolor{gray!40}\begin{tabular}[c]{@{}c@{}} Sand\\ 11.67\%\\ Hard\\ \includegraphics[width=0.14\columnwidth]{figure/sand2.jpg}\end{tabular}\\
% \hline
% \end{tabular}
% \caption{Seven Rigid (percentage, mean and variance of friction coefficient, and texture) and Three Deformable (percentage, deformability, and texture) Terrain Semantics.}
% \label{tab::semantics}
% \end{table}




\subsection{Obstacles}
Undulating geometry and varying semantics require off-road mobility systems to understand fine-grained vehicle-terrain interactions when driving on them. Off-road obstacles, like large boulders or trees, exist in real-world off-road environments, which are simply beyond vehicles' mechanical capabilities and hence need to be avoided. We also include natural obstacles in Verti-Bench to pose challenges to obstacle avoidance systems. For example, a large boulder triple the size of the vehicle is completely non-traversable, while a steep hill as part of the terrain may or may not be ascended with the right maneuver. We add natural obstacles as instances of the former. To further promote variation, we randomly sample the locations and types (different sizes of boulders or trees) of 10, 20, and 40 obstacles to place on each 129$\times$129 Verti-Bench environment, denoted as sparse, medium, and dense for obstacle distribution. We resample a new obstacle if the old one is within 10 m of another obstacle, a start, or a goal. Assuming a holonomic point-mass vehicle, we also provide pre-planned global paths leading from start to goal locations and avoiding obstacles. Fig.~\ref{fig::obstacles} shows three examples of sparse, medium, and dense obstacle distributions and their corresponding global paths. 

\begin{figure}[ht]
    \centering
    % \includegraphics[width=0.32\columnwidth]{figure/Sparse.png}
    % \includegraphics[width=0.32\columnwidth]{figure/Medium.png}
    \includegraphics[width=\columnwidth]{figure/ob2.png}
    \caption{Top: Sparse, Medium, and Dense Obstacles (black) and Global Paths (red) between Start and Goal (green). Bottom: Corresponding simulation scenario in Verti-Bench (elevation and semantics are removed for obstacle clarity).}
    \label{fig::obstacles}
\end{figure}


\subsection{Vehicles}
We also provide a set of vehicle platforms in Verti-Bench, with the possibility of adding new customized ones in the future, so that different off-road mobility systems can be evaluated on standardized vehicles. Compared to simplified vehicles in existing simulators, the Verti-Bench vehicles are more sophisticated and articulated, including engine/motor, drivetrain, transmission, suspension, steering mechanism, and wheel torque, whose responses to complex terrain interactions are simulated. To be specific, Verti-Bench  provides nine types of off-road vehicles, which are sourced from Project Chrono~\cite{tasora2016chrono}, open-source real and simulated research platforms~\cite{elmquist2022art}, and custom-created vehicles using 3D scanning and modeling (with a Creality CR-Scan Raptor 3D scanner) of real-world scaled vehicles (Fig.~\ref{fig::vehicles}). Those vehicles vary in terms of scale (1/10th, 1/6th, and full scale), chassis (4-, 6-, and 8-wheeled and 2-tracked), suspension (single- and double-wishbone, multilink, toebar leaf-spring, and special tensioning), steering (pitman-arm, rack-and-pinion, toebar, bellcrank/rotary arm,
and differential), and tires (rigid and handling, excluding FEA-based models due to significantly reduced simulation speed). All vehicles, regardless of their sources, are implemented as native C++ classes in Chrono's C++ framework. The C++ implementations are then compiled and exposed to Python through SWIG-generated bindings.  

% \begin{figure*}[h]
%     \centering
%     \includegraphics[width=2\columnwidth]{figure/full_scale.png}
%     % \includegraphics[height=0.26\columnwidth]{figure/one_tenth.png}
%     \caption{Different Verti-Bench Vehicle Types: Full (left), 1/10th, and 1/6th (Right) Scale Vehicles.}
%     \label{fig::vehicles}
% \end{figure*}

% \begin{figure*}[ht]
%     \centering
%     \includegraphics[width=2\columnwidth]{figure/Vehicles2.png}
%     % \includegraphics[height=0.26\columnwidth]{figure/one_tenth.png}
%     \caption{Verti-Bench Vehicles with Different Scale (1/10th, 1/6th, and full scale), Chassis (4-, 6-, and 8-wheeled and 2-tracked), Steering (pitman-arm, rack-and-pinion,
% toebar, bellcrank/rotary arm, and differential), and Tires (rigid
% and handling). }
%     \label{fig::vehicles}
% \end{figure*}
\begin{figure*}[ht]
    \centering
    \includegraphics[width=2\columnwidth]{figure/Vehicles.png}
    % \includegraphics[height=0.26\columnwidth]{figure/one_tenth.png}
    \caption{Verti-Bench Vehicles with Different Scale (1/10th, 1/6th, and full scale), Chassis (4-, 6-, and 8-wheeled and 2-tracked), Steering (pitman-arm, rack-and-pinion,
toebar, bellcrank/rotary arm, and differential), and Tires (rigid
and handling). }
    \label{fig::vehicles}
\end{figure*}

\subsection{Metrics}
Verti-Bench automatically computes a set of standard metrics to quantify off-road mobility performance, while additional metrics can be customized as needed: Success Rate captures the percentage of successful trials over total number of attempts; Traversal Time indicates how long it takes to finish a successful traversal; Roll and Pitch describes the stability of the vehicle during traversal, whose raw and absolute values can be used to derive mean, variance, and maximum; Vehicle actions, such as throttle and steering, can be used as metrics to quantify energy consumption, planning confidence, and path smoothness, along with other metrics. Verti-Bench provides infrastructure to save raw vehicle-terrain interaction data as well as to compute performance metrics. 

\subsection{Datasets}
Using Verti-Bench, we also collect a few datasets to facilitate future data-driven off-road mobility research. Current datasets are collected on the High Mobility Multipurpose Wheeled Vehicle (HMMWV, Fig.~\ref{fig:cover} right), while future data collection can expand to different vehicles. 

\subsubsection{Expert Demonstration} 
A team of four human operators collect 4 hours of expert demonstration of successfully driving the off-road vehicle in different Verti-Bench environments. We filter out all failure cases, e.g., vehicle rollover and getting-stuck, to maintain high demonstration quality. 

\subsubsection{Random Exploration}
To facilitate off-road kinodynamics learning~\cite{triest2022tartandrive}, we collect a random exploration dataset on different off-road terrain by driving the off-road vehicle with sinusoidal steering and 2 m/s speed commands for ten hours. Each data collection trial terminates if the vehicle rolls over, gets stuck, or reaches the environment boundaries.

\subsubsection{Failure Cases}
To enable future data-driven off-road mobility by preventing vehicle failures, we also curate a dataset of failure cases by providing the last ten seconds of trajectory before the vehicle rolls over or gets stuck. The failure cases can be used to learn high-cost regions to be avoided in data-driven mobility systems. 

\subsubsection{RL Interface}
Although not being a main purpose of Verti-Bench, we also provide a gym-like RL interface so that vehicles can learn off-road mobility through trial-and-error experiences in Verti-Bench. Vehicle states and actions, as well as reward functions, can be customized by our interface, which communicates with existing RL algorithm implementations. 