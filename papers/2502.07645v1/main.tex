\documentclass[journal]{IEEEtran}
\usepackage{times}

% numbers option provides compact numerical references in the text.
\let\labelindent\relax
\usepackage{enumitem}
\usepackage{jabbrv}
\usepackage[numbers]{natbib}
\usepackage{multicol}
\usepackage[bookmarks=true]
{hyperref}

\usepackage{bm}
\usepackage{amsmath}
\usepackage{amssymb}
\usepackage{mathtools} % Provides \coloneqq
\usepackage{amsfonts}
\usepackage{comment}
\usepackage{graphicx}
\usepackage[dvipsnames]{xcolor}

\usepackage{gensymb}
\usepackage{algorithm}
\usepackage{algpseudocode}
% \usepackage{todonotes}
\usepackage{booktabs}
\usepackage{makecell}  % for \Xhline
\usepackage{tabularx} % For controlling table width
\usepackage{cases}  % provides the numcases environment

\usepackage{pifont} % for cross symbol

% \pdfinfo{
%    /Author (Homer Simpson)
%    /Title  (Robots: Our new overlords)
%    /CreationDate (D:20101201120000)
%    /Subject (Robots)
%    /Keywords (Robots;Overlords)
% }

\newtheorem{proposition}{Proposition}
\usepackage{multirow}

% make the caption of the table not in captical letters
\usepackage{caption}
\captionsetup{labelfont=bf, textfont=normal}
\captionsetup[table]{aboveskip=0pt}  % reduce the distance between a table and its above caption
\captionsetup[table]{belowskip=-5pt}  % Adjust 5pt as needed


\usepackage{svg}  % for svg type image

% to use \diagdown
\usepackage{amssymb}
% \usepackage{bbm} % for mathbb 1
% \usepackage{cite}

\newcommand{\issue}[1]{\vspace{0.1em}\noindent \textbf{#1 \hspace{0.2em}}} % for rebuttal.tex

\usepackage{booktabs} % for pandas-generated table
\usepackage{dblfloatfix}

% \usepackage[colorlinks=false, pdfborder={0 0 0}, hypertexnames=false]{hyperref}

\setlength{\textfloatsep}{3pt}
\setlength{\abovedisplayskip}{2pt}
\setlength{\belowdisplayskip}{2pt}
\setlength{\abovedisplayshortskip}{2pt}
\setlength{\belowdisplayshortskip}{2pt}

\renewcommand*{\bibfont}{\footnotesize} % decrease the fontsize of bibliography

\begin{document}


% paper title
% \title{Direct Policy Shaping via Human Correction Feedback: a policy contrastive approach}
% \title{A General Approach for Policy Shaping via Action Contrasting}

% \title{
% Policy Shaping Through Interactive Action Contrasting: A Contrastive Approach in Action Space
% }

% \title{
% Policy Shaping Through Action Contrasting: A General Approach to Optimal Action Estimation
% }

% \title{
% Contrastive Policy Learning via Interactive Correction: A General Approach to Optimal Action Estimation
% }

% \title{Contrastive Policy Learning from Interactive Correction for Optimal Action Estimation}

% \title{Contrastive Policy Learning from Interactive Corrections via Optimal Action Estimation}


% \title{Contrastive Policy Learning from Interactive Corrections via Iterative  Undesired Actions Pruning}

% \title{Contrastive Policy Learning from Interactive Absolute and Relative Corrections}

% \title{Policy Shaping through Human Feedback via Contrastive Learning}

% \title{Policy Shaping through Human Feedback: \\A Contrastive Learning Perspective}

\title{Beyond Behavior Cloning: Robustness through Interactive Imitation and Contrastive Learning}

% \title{Beyond Behavior Cloning: Interactive Imitation and Contrastive Learning Enable Robust, Efficient Learning}

% \title{Beyond Behavior Cloning: Robustness and  Efficiency with Interactive Imitation and Contrastive Learning}

% \title{Beyond Behavior Cloning: \\ Achieving Robustness Efficiently with \\Interactive Imitation and Contrastive Learning}


% \title{Contrastive Policy Learning from Interactive Corrections via Contracting the Desired Action Space}


% You will get a Paper-ID when submitting a pdf file to the conference system
% \author{Author Names Omitted for Anonymous Review. Paper-ID 119}
\author{
    \IEEEauthorblockN{Zhaoting Li, Rodrigo P{\'e}rez-Dattari, Robert Babuska, Cosimo Della Santina, Jens Kober} \\
    \IEEEauthorblockA{Delft University of Technology, \{ z.li-23, r.j.perezdattari, r.babuska, c.dellasantina, j.kober \}@tudelft.nl}
    \href{https://clic-webpage.github.io }{https://clic-webpage.github.io}
}

% \author{\authorblockN{Zhaoting Li}
% \authorblockA{Department of Cognitive Robotics\\ Delft University of Technology\\ The
% Netherlands\\
% Email: Z.Li-23@tudelft.nl}
% \and
% \authorblockN{Rodrigo P{\'e}rez-Dattari}
% \authorblockA{Department of Cognitive Robotics\\ Delft University of Technology\\ The
% Netherlands\\
% Email: Z.Li-23@tudelft.nl}
% \and
% \authorblockN{ Robert Babuska}
% \authorblockA{Department of Cognitive Robotics\\ Delft University of Technology\\ The
% Netherlands\\
% Email: Z.Li-23@tudelft.nl}
% \and 
% \authorblockN{Cosimo Della Santina}
% \authorblockA{Department of Cognitive Robotics\\ Delft University of Technology\\ The
% Netherlands\\
% Email: Z.Li-23@tudelft.nl}
% \and
% \authorblockN{Jens Kober}
% \authorblockA{Department of Cognitive Robotics\\ Delft University of Technology\\ The
% Netherlands\\
% Email: Z.Li-23@tudelft.nl}}


% avoiding spaces at the end of the author lines is not a problem with
% conference papers because we don't use \thanks or \IEEEmembership


% for over three affiliations, or if they all won't fit within the width
% of the page, use this alternative format:
% 
%\author{\authorblockN{Michael Shell\authorrefmark{1},
%Homer Simpson\authorrefmark{2},
%James Kirk\authorrefmark{3}, 
%Montgomery Scott\authorrefmark{3} and
%Eldon Tyrell\authorrefmark{4}}
%\authorblockA{\authorrefmark{1}School of Electrical and Computer Engineering\\
%Georgia Institute of Technology,
%Atlanta, Georgia 30332--0250\\ Email: mshell@ece.gatech.edu}
%\authorblockA{\authorrefmark{2}Twentieth Century Fox, Springfield, USA\\
%Email: homer@thesimpsons.com}
%\authorblockA{\authorrefmark{3}Starfleet Academy, San Francisco, California 96678-2391\\
%Telephone: (800) 555--1212, Fax: (888) 555--1212}
%\authorblockA{\authorrefmark{4}Tyrell Inc., 123 Replicant Street, Los Angeles, California 90210--4321}}


\IEEEpeerreviewmaketitle
% \maketitle
\twocolumn[{%
	\renewcommand\twocolumn[1][]{#1}%
	\maketitle
        \vspace{-6mm}
	\begin{center}
		\includegraphics[width=0.99\textwidth]{figs/Fig_3_cover_Figure_3.pdf}
% \includesvg[width=0.99\textwidth,inkscapelatex=false]{figs/Fig_3_cover_Figure_3.svg}
 \captionof{figure}{\small
 A: Our method operates in an Interactive Imitation Learning framework. The robot's policy outputs a robot action $\bm a^r$, which interacts with the environment. The human teacher provides corrective feedback occasionally if the robot action is suboptimal.
 In (a1), such feedback stored in data buffer $\mathcal{ D}$ defines multiple desired action spaces. 
These regions collectively define an overall desired action space $\hat{\mathcal{A}}^{\mathcal{D}}$.
 In (a2), the policy, modeled as an energy-based model (EBM), is trained to generate actions within $\hat{\mathcal{A}}^{\mathcal{D}}$.
B: Examples of the learned EBMs in a 2D action space. Implicit BC \cite{2022_implicit_BC} overfits each action label, while our method estimates the optimal action without overfitting.
% The main difference is that, instead of lowering energy for each action label and raising it for others, our loss reduces energy for actions within a desired action space. 
% which is defined by the data and shown as the gray-shaded area (b1).
\label{fig:framework}}
	\end{center}
}]
% \begin{abstract} Behavior cloning (BC) traditionally relies on demonstration feedback, assuming the demonstrated actions are optimal. This assumption can lead to overfitting, particularly with expressive models like the energy-based model utilized in Implicit BC. To address this, we reformulate behavior cloning as an optimal action estimation problem and introduce \textit{Contrastive policy Learning from Interactive Corrections (CLIC)}. CLIC leverages human corrections—both absolute and relative—to construct a set of desired actions and optimizes the policy to select actions from this set. We provide theoretical guarantees for the convergence of the desired action set to optimal actions in both single and multiple optimal action cases. Extensive simulation and real-robot experiments validate CLIC's advantages over state-of-the-art BC methods, including stable training of energy-based models, robustness to feedback noise, and adaptability to diverse feedback beyond demonstrations.
% To our best knowledge, this work offers a fresh perspective on policy learning by focusing on iteratively estimating optimal actions rather than directly imitating them. 
% The code will be publicly available. 
% \end{abstract}
\begin{abstract} Behavior cloning (BC) traditionally relies on demonstration data, assuming the demonstrated actions are optimal. This can lead to overfitting under noisy data, particularly when expressive models are used (e.g., the energy-based model in Implicit BC). To address this, we extend behavior cloning into an iterative process of optimal action estimation within the Interactive Imitation Learning framework. Specifically, we introduce \textit{Contrastive policy Learning from Interactive Corrections (CLIC)}. CLIC leverages human corrections to estimate a set of desired actions and optimizes the policy to select actions from this set. We provide theoretical guarantees for the convergence of the desired action set to optimal actions in both single and multiple optimal action cases. Extensive simulation and real-robot experiments validate CLIC's advantages over existing state-of-the-art methods, including stable training of energy-based models, robustness to feedback noise, and adaptability to diverse feedback types beyond demonstrations. Our code will be publicly available soon.
% \textcolor{blue}{Our code will be publicly available upon acceptance.} 
\end{abstract}

% \begin{keywords}
%     Interactive Imitation Learning, Corrective feedback, Contrastive Learning, Learning from Demonstration, Energy-based Models
% \end{keywords}

\begin{IEEEkeywords}
Interactive Imitation Learning, Corrective feedback, Contrastive Learning, Learning from Demonstration, Energy-based Models
\end{IEEEkeywords}




\section{Introduction}
\label{section:introduction}

% redirection is unique and important in VR
Virtual Reality (VR) systems enable users to embody virtual avatars by mirroring their physical movements and aligning their perspective with virtual avatars' in real time. 
As the head-mounted displays (HMDs) block direct visual access to the physical world, users primarily rely on visual feedback from the virtual environment and integrate it with proprioceptive cues to control the avatar’s movements and interact within the VR space.
Since human perception is heavily influenced by visual input~\cite{gibson1933adaptation}, 
VR systems have the unique capability to control users' perception of the virtual environment and avatars by manipulating the visual information presented to them.
Leveraging this, various redirection techniques have been proposed to enable novel VR interactions, 
such as redirecting users' walking paths~\cite{razzaque2005redirected, suma2012impossible, steinicke2009estimation},
modifying reaching movements~\cite{gonzalez2022model, azmandian2016haptic, cheng2017sparse, feick2021visuo},
and conveying haptic information through visual feedback to create pseudo-haptic effects~\cite{samad2019pseudo, dominjon2005influence, lecuyer2009simulating}.
Such redirection techniques enable these interactions by manipulating the alignment between users' physical movements and their virtual avatar's actions.

% % what is hand/arm redirection, motivation of study arm-offset
% \change{\yj{i don't understand the purpose of this paragraph}
% These illusion-based techniques provide users with unique experiences in virtual environments that differ from the physical world yet maintain an immersive experience. 
% A key example is hand redirection, which shifts the virtual hand’s position away from the real hand as the user moves to enhance ergonomics during interaction~\cite{feuchtner2018ownershift, wentzel2020improving} and improve interaction performance~\cite{montano2017erg, poupyrev1996go}. 
% To increase the realism of virtual movements and strengthen the user’s sense of embodiment, hand redirection techniques often incorporate a complete virtual arm or full body alongside the redirected virtual hand, using inverse kinematics~\cite{hartfill2021analysis, ponton2024stretch} or adjustments to the virtual arm's movement as well~\cite{li2022modeling, feick2024impact}.
% }

% noticeability, motivation of predicting a probability, not a classification
However, these redirection techniques are most effective when the manipulation remains undetected~\cite{gonzalez2017model, li2022modeling}. 
If the redirection becomes too large, the user may not mitigate the conflict between the visual sensory input (redirected virtual movement) and their proprioception (actual physical movement), potentially leading to a loss of embodiment with the virtual avatar and making it difficult for the user to accurately control virtual movements to complete interaction tasks~\cite{li2022modeling, wentzel2020improving, feuchtner2018ownershift}. 
While proprioception is not absolute, users only have a general sense of their physical movements and the likelihood that they notice the redirection is probabilistic. 
This probability of detecting the redirection is referred to as \textbf{noticeability}~\cite{li2022modeling, zenner2024beyond, zenner2023detectability} and is typically estimated based on the frequency with which users detect the manipulation across multiple trials.

% version B
% Prior research has explored factors influencing the noticeability of redirected motion, including the redirection's magnitude~\cite{wentzel2020improving, poupyrev1996go}, direction~\cite{li2022modeling, feuchtner2018ownershift}, and the visual characteristics of the virtual avatar~\cite{ogawa2020effect, feick2024impact}.
% While these factors focus on the avatars, the surrounding virtual environment can also influence the users' behavior and in turn affect the noticeability of redirection.
% One such prominent external influence is through the visual channel - the users' visual attention is constantly distracted by complex visual effects and events in practical VR scenarios.
% Although some prior studies have explored how to leverage user blindness caused by visual distractions to redirect users' virtual hand~\cite{zenner2023detectability}, there remains a gap in understanding how to quantify the noticeability of redirection under visual distractions.

% visual stimuli and gaze behavior
Prior research has explored factors influencing the noticeability of redirected motion, including the redirection's magnitude~\cite{wentzel2020improving, poupyrev1996go}, direction~\cite{li2022modeling, feuchtner2018ownershift}, and the visual characteristics of the virtual avatar~\cite{ogawa2020effect, feick2024impact}.
While these factors focus on the avatars, the surrounding virtual environment can also influence the users' behavior and in turn affect the noticeability of redirection.
This, however, remains underexplored.
One such prominent external influence is through the visual channel - the users' visual attention is constantly distracted by complex visual effects and events in practical VR scenarios.
We thus want to investigate how \textbf{visual stimuli in the virtual environment} affect the noticeability of redirection.
With this, we hope to complement existing works that focus on avatars by incorporating environmental visual influences to enable more accurate control over the noticeability of redirected motions in practical VR scenarios.
% However, in realistic VR applications, the virtual environment often contains complex visual effects beyond the virtual avatar itself. 
% We argue that these visual effects can \textbf{distract users’ visual attention and thus affect the noticeability of redirection offsets}, while current research has yet taken into account.
% For instance, in a VR boxing scenario, a user’s visual attention is likely focused on their opponent rather than on their virtual body, leading to a lower noticeability of redirection offsets on their virtual movements. 
% Conversely, when reaching for an object in the center of their field of view, the user’s attention is more concentrated on the virtual hand’s movement and position to ensure successful interaction, resulting in a higher noticeability of offsets.

Since each visual event is a complex choreography of many underlying factors (type of visual effect, location, duration, etc.), it is extremely difficult to quantify or parameterize visual stimuli.
Furthermore, individuals respond differently to even the same visual events.
Prior neuroscience studies revealed that factors like age, gender, and personality can influence how quickly someone reacts to visual events~\cite{gillon2024responses, gale1997human}. 
Therefore, aiming to model visual stimuli in a way that is generalizable and applicable to different stimuli and users, we propose to use users' \textbf{gaze behavior} as an indicator of how they respond to visual stimuli.
In this paper, we used various gaze behaviors, including gaze location, saccades~\cite{krejtz2018eye}, fixations~\cite{perkhofer2019using}, and the Index of Pupil Activity (IPA)~\cite{duchowski2018index}.
These behaviors indicate both where users are looking and their cognitive activity, as looking at something does not necessarily mean they are attending to it.
Our goal is to investigate how these gaze behaviors stimulated by various visual stimuli relate to the noticeability of redirection.
With this, we contribute a model that allows designers and content creators to adjust the redirection in real-time responding to dynamic visual events in VR.

To achieve this, we conducted user studies to collect users' noticeability of redirection under various visual stimuli.
To simulate realistic VR scenarios, we adopted a dual-task design in which the participants performed redirected movements while monitoring the visual stimuli.
Specifically, participants' primary task was to report if they noticed an offset between the avatar's movement and their own, while their secondary task was to monitor and report the visual stimuli.
As realistic virtual environments often contain complex visual effects, we started with simple and controlled visual stimulus to manage the influencing factors.

% first user study, confirmation study
% collect data under no visual stimuli, different basic visual stimuli
We first conducted a confirmation study (N=16) to test whether applying visual stimuli (opacity-based) actually affects their noticeability of redirection. 
The results showed that participants were significantly less likely to detect the redirection when visual stimuli was presented $(F_{(1,15)}=5.90,~p=0.03)$.
Furthermore, by analyzing the collected gaze data, results revealed a correlation between the proposed gaze behaviors and the noticeability results $(r=-0.43)$, confirming that the gaze behaviors could be leveraged to compute the noticeability.

% data collection study
We then conducted a data collection study to obtain more accurate noticeability results through repeated measurements to better model the relationship between visual stimuli-triggered gaze behaviors and noticeability of redirection.
With the collected data, we analyzed various numerical features from the gaze behaviors to identify the most effective ones. 
We tested combinations of these features to determine the most effective one for predicting noticeability under visual stimuli.
Using the selected features, our regression model achieved a mean squared error (MSE) of 0.011 through leave-one-user-out cross-validation. 
Furthermore, we developed both a binary and a three-class classification model to categorize noticeability, which achieved an accuracy of 91.74\% and 85.62\%, respectively.

% evaluation study
To evaluate the generalizability of the regression model, we conducted an evaluation study (N=24) to test whether the model could accurately predict noticeability with new visual stimuli (color- and scale-based animations).
Specifically, we evaluated whether the model's predictions aligned with participants' responses under these unseen stimuli.
The results showed that our model accurately estimated the noticeability, achieving mean squared errors (MSE) of 0.014 and 0.012 for the color- and scale-based visual stimili, respectively, compared to participants' responses.
Since the tested visual stimuli data were not included in the training, the results suggested that the extracted gaze behavior features capture a generalizable pattern and can effectively indicate the corresponding impact on the noticeability of redirection.

% application
Based on our model, we implemented an adaptive redirection technique and demonstrated it through two applications: adaptive VR action game and opportunistic rendering.
We conducted a proof-of-concept user study (N=8) to compare our adaptive redirection technique with a static redirection, evaluating the usability and benefits of our adaptive redirection technique.
The results indicated that participants experienced less physical demand and stronger sense of embodiment and agency when using the adaptive redirection technique. 
These results demonstrated the effectiveness and usability of our model.

In summary, we make the following contributions.
% 
\begin{itemize}
    \item 
    We propose to use users' gaze behavior as a medium to quantify how visual stimuli influences the noticebility of redirection. 
    Through two user studies, we confirm that visual stimuli significantly influences noticeability and identify key gaze behavior features that are closely related to this impact.
    \item 
    We build a regression model that takes the user's gaze behavioral data as input, then computes the noticeability of redirection.
    Through an evaluation study, we verify that our model can estimate the noticeability with new participants under unseen visual stimuli.
    These findings suggest that the extracted gaze behavior features effectively capture the influence of visual stimuli on noticeability and can generalize across different users and visual stimuli.
    \item 
    We develop an adaptive redirection technique based on our regression model and implement two applications with it.
    With a proof-of-concept study, we demonstrate the effectiveness and potential usability of our regression model on real-world use cases.

\end{itemize}

% \delete{
% Virtual Reality (VR) allows the user to embody a virtual avatar by mirroring their physical movements through the avatar.
% As the user's visual access to the physical world is blocked in tasks involving motion control, they heavily rely on the visual representation of the avatar's motions to guide their proprioception.
% Similar to real-world experiences, the user is able to resolve conflicts between different sensory inputs (e.g., vision and motor control) through multisensory integration, which is essential for mitigating the sensory noise that commonly arises.
% However, it also enables unique manipulations in VR, as the system can intentionally modify the avatar's movements in relation to the user's motions to achieve specific functional outcomes,
% for example, 
% % the manipulations on the avatar's movements can 
% enabling novel interaction techniques of redirected walking~\cite{razzaque2005redirected}, redirected reaching~\cite{gonzalez2022model}, and pseudo haptics~\cite{samad2019pseudo}.
% With small adjustments to the avatar's movements, the user can maintain their sense of embodiment, due to their ability to resolve the perceptual differences.
% % However, a large mismatch between the user and avatar's movements can result in the user losing their sense of embodiment, due to an inability to resolve the perceptual differences.
% }

% \delete{
% However, multisensory integration can break when the manipulation is so intense that the user is aware of the existence of the motion offset and no longer maintains the sense of embodiment.
% Prior research studied the intensity threshold of the offset applied on the avatar's hand, beyond which the embodiment will break~\cite{li2022modeling}. 
% Studies also investigated the user's sensitivity to the offsets over time~\cite{kohm2022sensitivity}.
% Based on the findings, we argue that one crucial factor that affects to what extent the user notices the offset (i.e., \textit{noticeability}) that remains under-explored is whether the user directs their visual attention towards or away from the virtual avatar.
% Related work (e.g., Mise-unseen~\cite{marwecki2019mise}) has showcased applications where adjustments in the environment can be made in an unnoticeable manner when they happen in the area out of the user's visual field.
% We hypothesize that directing the user's visual attention away from the avatar's body, while still partially keeping the avatar within the user's field-of-view, can reduce the noticeability of the offset.
% Therefore, we conduct two user studies and implement a regression model to systematically investigate this effect.
% }

% \delete{
% In the first user study (N = 16), we test whether drawing the user's visual attention away from their body impacts the possibility of them noticing an offset that we apply to their arm motion in VR.
% We adopt a dual-task design to enable the alteration of the user's visual attention and a yes/no paradigm to measure the noticeability of motion offset. 
% The primary task for the user is to perform an arm motion and report when they perceive an offset between the avatar's virtual arm and their real arm.
% In the secondary task, we randomly render a visual animation of a ball turning from transparent to red and becoming transparent again and ask them to monitor and report when it appears.
% We control the strength of the visual stimuli by changing the duration and location of the animation.
% % By changing the time duration and location of the visual animation, we control the strengths of attraction to the users.
% As a result, we found significant differences in the noticeability of the offsets $(F_{(1,15)}=5.90,~p=0.03)$ between conditions with and without visual stimuli.
% Based on further analysis, we also identified the behavioral patterns of the user's gaze (including pupil dilation, fixations, and saccades) to be correlated with the noticeability results $(r=-0.43)$ and they may potentially serve as indicators of noticeability.
% }

% \delete{
% To further investigate how visual attention influences the noticeability, we conduct a data collection study (N = 12) and build a regression model based on the data.
% The regression model is able to calculate the noticeability of the offset applied on the user's arm under various visual stimuli based on their gaze behaviors.
% Our leave-one-out cross-validation results show that the proposed method was able to achieve a mean-squared error (MSE) of 0.012 in the probability regression task.
% }

% \delete{
% To verify the feasibility and extendability of the regression model, we conduct an evaluation study where we test new visual animations based on adjustments on scale and color and invite 24 new participants to attend the study.
% Results show that the proposed method can accurately estimate the noticeability with an MSE of 0.014 and 0.012 in the conditions of the color- and scale-based visual effects.
% Since these animations were not included in the dataset that the regression model was built on, the study demonstrates that the gaze behavioral features we extracted from the data capture a generalizable pattern of the user's visual attention and can indicate the corresponding impact on the noticeability of the offset.
% }

% \delete{
% Finally, we demonstrate applications that can benefit from the noticeability prediction model, including adaptive motion offsets and opportunistic rendering, considering the user's visual attention. 
% We conclude with discussions of our work's limitations and future research directions.
% }

% \delete{
% In summary, we make the following contributions.
% }
% % 
% \begin{itemize}
%     \item 
%     \delete{
%     We quantify the effects of the user's visual attention directed away by stimuli on their noticeability of an offset applied to the avatar's arm motion with respect to the user's physical arm. 
%     Through two user studies, we identified gaze behavioral features that are indicative of the changes in noticeability.
%     }
%     \item 
%     \delete{We build a regression model that takes the user's gaze behavioral data and the offset applied to the arm motion as input, then computes the probability of the user noticing the offset.
%     Through an evaluation study, we verified that the model needs no information about the source attracting the user's visual attention and can be generalizable in different scenarios.
%     }
%     \item 
%     \delete{We demonstrate two applications that potentially benefit from the regression model, including adaptive motion offsets and opportunistic rendering.
%     }

% \end{itemize}

\begin{comment}
However, users will lose the sense of embodiment to the virtual avatars if they notice the offset between the virtual and physical movements.
To address this, researchers have been exploring the noticing threshold of offsets with various magnitudes and proposing various redirection techniques that maintain the sense of embodiment~\cite{}.

However, when users embody virtual avatars to explore virtual environments, they encounter various visual effects and content that can attract their attention~\cite{}.
During this, the user may notice an offset when he observes the virtual movement carefully while ignoring it when the virtual contents attract his attention from the movements.
Therefore, static offset thresholds are not appropriate in dynamic scenarios.

Past research has proposed dynamic mapping techniques that adapted to users' state, such as hand moving speed~\cite{frees2007prism} or ergonomically comfortable poses~\cite{montano2017erg}, but not considering the influence of virtual content.
More specifically, PRISM~\cite{frees2007prism} proposed adjusting the C/D ratio with a non-linear mapping according to users' hand moving speed, but it might not be optimal for various virtual scenarios.
While Erg-O~\cite{montano2017erg} redirected users' virtual hands according to the virtual target's relative position to reduce physical fatigue, neglecting the change of virtual environments. 

Therefore, how to design redirection techniques in various scenarios with different visual attractions remains unknown.
To address this, we investigate how visual attention affects the noticing probability of movement offsets.
Based on our experiments, we implement a computational model that automatically computes the noticing probability of offsets under certain visual attractions.
VR application designers and developers can easily leverage our model to design redirection techniques maintaining the sense of embodiment adapt to the user's visual attention.
We implement a dynamic redirection technique with our model and demonstrate that it effectively reduces the target reaching time without reducing the sense of embodiment compared to static redirection techniques.

% Need to be refined
This paper offers the following contributions.
\begin{itemize}
    \item We investigate how visual attractions affect the noticing probability of redirection offsets.
    \item We construct a computational model to predict the noticing probability of an offset with a given visual background.
    \item We implement a dynamic redirection technique adapting to the visual background. We evaluate the technique and develop three applications to demonstrate the benefits. 
\end{itemize}



First, we conducted a controlled experiment to understand how users perceived the movement offset while subjected to various distractions.
Since hand redirection is one of the most frequently used redirections in VR interactions, we focused on the dynamic arm movements and manually added angular offsets to the' elbow joint~\cite{li2022modeling, gonzalez2022model, zenner2019estimating}. 
We employed flashing spheres in the user's field of view as distractions to attract users' visual attention.
Participants were instructed to report the appearing location of the spheres while simultaneously performing the arm movements and reporting if they perceived an offset during the movement. 
(\zhipeng{Add the results of data collection. Analyze the influence of the distance between the gaze map and the offset.}
We measured the visual attraction's magnitude with the gaze distribution on it.
Results showed that stronger distractions made it harder for users to notice the offset.)
\zhipeng{Need to rewrite. Not sure to use gaze distribution or a metric obtained from the visual content.}
Secondly, we constructed a computational model to predict the noticing probability of offsets with given visual content.
We analyzed the data from the user studies to measure the influence of visual attractions on the noticing probability of offsets.
We built a statistical model to predict the offset's noticing probability with a given visual content.
Based on the model, we implement a dynamic redirection technique to adjust the redirection offset adapted to the user's current field of view.
We evaluated the technique in a target selection task compared to no hand redirection and static hand redirection.
\zhipeng{Add the results of the evaluation.}
Results showed that the dynamic hand redirection technique significantly reduced the target selection time with similar accuracy and a comparable sense of embodiment.
Finally, we implemented three applications to demonstrate the potential benefits of the visual attention adapted dynamic redirection technique.
\end{comment}

% This one modifies arm length, not redirection
% \citeauthor{mcintosh2020iteratively} proposed an adaptation method to iteratively change the virtual avatar arm's length based on the primary tasks' performance~\cite{mcintosh2020iteratively}.



% \zhipeng{TO ADD: what is redirection}
% Redirection enables novel interactions in Virtual Reality, including redirected walking, haptic redirection, and pseudo haptics by introducing an offset to users' movement.
% \zhipeng{TO ADD: extend this sentence}
% The price of this is that users' immersiveness and embodiment in VR can be compromised when they notice the offset and perceive the virtual movement not as theirs~\cite{}.
% \zhipeng{TO ADD: extend this sentence, elaborate how the virtual environment attracts users' attention}
% Meanwhile, the visual content in the virtual environment is abundant and consistently captures users' attention, making it harder to notice the offset~\cite{}.
% While previous studies explored the noticing threshold of the offsets and optimized the redirection techniques to maintain the sense of embodiment~\cite{}, the influence of visual content on the probability of perceiving offsets remains unknown.  
% Therefore, we propose to investigate how users perceive the redirection offset when they are facing various visual attractions.


% We conducted a user study to understand how users notice the shift with visual attractions.
% We used a color-changing ball to attract the user's attention while instructing users to perform different poses with their arms and observe it meanwhile.
% \zhipeng{(Which one should be the primary task? Observe the ball should be the primary one, but if the primary task is too simple, users might allocate more attention on the secondary task and this makes the secondary task primary.)}
% \zhipeng{(We need a good and reasonable dual-task design in which users care about both their pose and the visual content, at least in the evaluation study. And we need to be able to control the visual content's magnitude and saliency maybe?)}
% We controlled the shift magnitude and direction, the user's pose, the ball's size, and the color range.
% We set the ball's color-changing interval as the independent factor.
% We collect the user's response to each shift and the color-changing times.
% Based on the collected data, we constructed a statistical model to describe the influence of visual attraction on the noticing probability.
% \zhipeng{(Are we actually controlling the attention allocation? How do we measure the attracting effect? We need uniform metrics, otherwise it is also hard for others to use our knowledge.)}
% \zhipeng{(Try to use eye gaze? The eye gaze distribution in the last five seconds to decide the attention allocation? Basically constructing a model with eye gaze distribution and noticing probability. But the user's head is moving, so the eye gaze distribution is not aligned well with the current view.)}

% \zhipeng{Saliency and EMD}
% \zhipeng{Gaze is more than just a point: Rethinking visual attention
% analysis using peripheral vision-based gaze mapping}

% Evaluation study(ideal case): based on the visual content, adjusting the redirection magnitude dynamically.

% \zhipeng{(The risk is our model's effect is trivial.)}

% Applications:
% Playing Lego while watching demo videos, we can accelerate the reaching process of bricks, and forbid the redirection during the manipulation.

% Beat saber again: but not make a lot of sense? Difficult game has complicated visual effects, while allows larger shift, but do not need large shift with high difficulty



\section{Related Work}
\label{sec:related_work}
In this section, we provide an overview of the related work on policy learning from various feedback types, such as demonstration, relative correction, and preference.
We also summarize methods for training policies represented by EBMs.
%  The energy-based model can be trained in many different approaches, depending on the assumption of the type of data (demonstration, correction, reward, etc).
% Besides, other types of models, such as diffusion, have been applied to policy representation, which outperforms its EBM counterpart (IBC). 
% In this section, we briefly overview the related works and the IBC's training instability issue. At the end, we draw connection between IBC and our method, which utlize the corrective feedback for EBM traning. 

\subsection{Learning from Demonstration}
%1 introduce big picture of learning from Demonstration, deep generative model, diffusion, EBM
Learning from demonstration aims to teach robot behavior models from demonstration data, which provides the robot with examples of desired actions. 
Traditional methods often struggled with capturing complex data distributions, especially when multiple optimal actions exist for a given state, such as the task of pushing a T-shape object to a goal configuration \cite{2023_diffusionpolicy, 2024_RSS_pushT_Traj_optimization}.
Deep generative models, including EBMs \cite{2019_EBM_Du_Yilun, 2021_how_to_train_EBM},  diffusion models \cite{2015_diffusion, 2020_diffusion}, have been introduced to better capture such multi-modal data distributions \cite{2024_survey_deep_generative_model_in_robotics}. EBMs learn unnormalized energy values for inputs and have been applied to learn the energy of the entire state-action space via IBC \cite{2022_implicit_BC}.  
Diffusion models, which learn to denoise noise-corrupted data \cite{2015_diffusion, 2020_diffusion}, have also been utilized to represent robot policies, resulting in diffusion policies \cite{2023_diffusionpolicy, 2023_score_diffusion_policy}. These policies effectively learn the gradient (score) of the EBM \cite{2020_Score_based_diffusion} and offer improved training stability \cite{2023_diffusionpolicy}. Implicit models, such as EBMs and diffusion policies, have demonstrated superior capability in handling long-horizon, multi-modal tasks compared to explicit policies and have been successfully applied to various real-world applications \cite{2024_EBM_planning_air_hockey_application, 2024_IBC_RL_planning, 2024_survey_deep_generative_model_in_robotics}.
These implicit models have been extended into an online interactive imitation learning (IIL) framework 
\cite{2023_IIFL_implicit_interactive_BC, 2024_Diffusion_dagger}.  
However, as mentioned in the introduction section, the powerful encoding capability of these models can also cause overfitting behavior with behavior cloning loss, especially when the demonstration data deviates from the optimal action. 
To address this issue while still leveraging their encoding capability, 
we propose a new perspective on iteratively estimating optimal actions in the IIL framework. 
Instead of relying on behavior cloning loss, we introduce a novel loss function that updates the policy to align with desired action spaces. 


% \cite{2024_IBC_RL_planning}

\subsection{Learning from Preference Feedback}
Preference-based feedback involves the comparison of different robot trajectory segments.
% , which can be treated as a relative version of the evaluative feedback.
From this,
a reward/objective model is usually estimated and employed to obtain a policy \cite{2017_DRL_preference, 2015_IJRR_Preference_traj_ranking, 2021_Pebble_preference_RL, 2020_openai_RLHF, 2024_comparative_language}.
% The sub-optimal demonstration data can also be transformed into preference data to learn a high-quality reward model \cite{2019_IRL_ranked_demonstration}.
Some approaches enable directly learning a policy by the contrastive learning approach proposed by \cite{2023_Contrastive_Prefence_Learning, zhao2022calibrating} or direct preference optimization approach \cite{2023_DPO} to improve efficiency.
Moreover, sub-optimal demonstrations can be transformed into preference data to learn a high-quality reward model  \cite{2019_IRL_ranked_demonstration}. 
% Although this feedback modality is promising, it is not very data-efficient, requiring more data and time from the teacher to learn a good policy.
Although this feedback modality is promising, it is not very data-efficient. This is because the feedback is given over complete trajectories, requiring the learner to infer per-state behavior and requiring more data and teacher effort.
While there are works to make it data efficient via active learning \cite{2024batch_acitve_learning_Preference}  or utilizing prior knowledge of state \cite{2024hindsight_preference}, our paper focuses on human feedback in the state-action space.

\subsection{Learning from Relative Corrective Feedback}
% Relative corrective feedback provides incremental information on how to improve an action, balancing information richness and simplicity for the teacher \cite{2022_IIL_survey}. 
% % The data consists of pre-correction and post-correction actions.
% This correction feedback can be transferred into preference data with trajectory pairs and
% the objective function can be learned from the preference data, as in \cite{2015_IJRR_Preference_traj_ranking, 2017_pHRI_correction_learning_objective, 2018_uncertainty_correction_objective}.
% Alternatively, \cite{2022_TRO_correction_objective_function} proposed directly inferring the objective function without preference transformation.
% However, these objective functions are linear combinations of features, which may struggle with complex tasks.
% Another line of work is the COACH-based framework (Corrective Advice Communicated by Humans), which directly learns a policy from relative corrections
% \cite{2018_D_COACH, 2019_Carlos_COACH, 2021_BDCOACH}.
% This framework has been extended to 
%  utilize the feedback from the state space instead of the action space 
% \cite{2020_COACH_state_Space} and combined with reinforcement learning to increase the RL efficiency\cite{2019_Carlos_IJRR}.
% However, as we mentioned in Section~\ref{sec:introduction}, the previous COACH-based framework fails to utilize the history data, making it inefficient compared with demonstration-based methods. 
% Instead, our CLIC method can utilize the history data as it will not harm the policy under our assumption of desired action space. 


Relative corrections provide incremental information on how to improve an action, balancing information richness and simplicity for the teacher \cite{2022_IIL_survey}. 
% The data consists of pre-correction and post-correction actions.
This correction feedback can be transferred into preference data with trajectory pairs and
objective functions can be learned from the preference data, as in \cite{2015_IJRR_Preference_traj_ranking, 2017_pHRI_correction_learning_objective, 2018_uncertainty_correction_objective}.
Alternatively, \cite{2022_TRO_correction_objective_function} proposed directly inferring objective functions without preference transformation.
However, these objective functions are linear combinations of features, which may struggle with complex tasks.

Another line of work is the COACH-based framework (Corrective Advice Communicated by Humans), which directly learns a policy from relative corrections
\cite{2018_D_COACH, 2019_Carlos_COACH, 2021_BDCOACH}.
This framework has been extended to 
 utilize the feedback from the state space instead of the action space 
\cite{2020_COACH_state_Space} and combined with reinforcement learning to increase the RL efficiency\cite{2019_Carlos_IJRR}.
However, COACH-based methods rely on the over-optimistic assumption that the action labels derived from relative corrections are optimal, allowing the policy to be refined by imitating them via the BC loss \cite{2019_Rodrigo_D_COACH, 2019_Carlos_IJRR, 2019_Carlos_COACH}. 
This assumption becomes a critical limitation when feedback is aggregated into a reply buffer.  
As the robot's policy continuously improves, previous feedback may no longer be valid, causing incorrect policy updates \cite{2021_BDCOACH}. 
As a result, the buffer size is limited to being small, ensuring it contains only recent corrections. This leads to policies that tend to overfit data collected from recently visited trajectories, making it inefficient compared with demonstration-based methods. 
In contrast, our CLIC method can utilize the history data, as the desired action spaces created from it will not mislead the policy. 

The COACH-based framework utilizes explicit policies\cite{2019_Carlos_IJRR, 2019_Rodrigo_D_COACH}, limiting its ability to handle multi-modal tasks. 
Implicit policies, encoded by
EBMs, can also be learned using methods like Proxy Value Propagation (PVP)\cite{2023_NIPS_PVP}.
PVP uses a loss function that only considers the energy values of recorded robot and human actions. As a result, the loss provides limited information and fails to train an EBM effectively.
In contrast, our approach generates action samples from the EBM and classifies them into desired and undesired actions based on the desired action space. These classified samples are then used to compute the loss, which can effectively train EBMs.

% using desired action spaces to clarify action samples from the EBM into desired and undesired, which are then used to calculate the loss. 
% proposing that each correction data defines a desired and undesired action set, which are used to train an EBM and enable a more efficient learning process compared to PVP.
% Our method then trains an EBM by pushing down the energy within the desired action set and pushing up the energy within the undesired action set, which is more effective than the PVP method.  





% 2 transition, talk about EBM's advantage and applications
\subsection{Learning Policies Represented as Energy-Based Models}
 % The energy-based model can be trained in many different approaches, depending on the assumption of the type of data (demonstration, correction, reward, etc).
EBMs have been widely used across different types of feedback data.
In reinforcement learning, where data is typically scalar rewards, the energy function is used to encode the action-value function Q, with the relationship $ Q_{soft}(\bm s, \bm a) = - \alpha E(\bm s, \bm a) $ \cite{2017_soft_Q_learning, 2018_SAC, 2024_RAL_imperfect_demon}. 
Reward-conditioned policies can also be learned through Bayesian approaches \cite{2023_Bayesian_reprameterized_RCRL}. For preference feedback, EBMs can be aligned with human preferences via inverse preference learning \cite{2023_Inverse_Preference_learning}. 
In scenarios involving corrective feedback, where both the robot and the human actions are given, methods such as PVP have been proposed to learn EBMs that assign low energy values to human actions and high energy values to robot actions \cite{2023_NIPS_PVP}.
For demonstration or absolute correction data, EBMs can be trained directly using the objective that demonstrated actions have lower energy than other actions  \cite{2020_RSS_expert_interventio_learning, 2022_implicit_BC}.
In discrete action spaces, EBMs can be straightforward to train \cite{2020_RSS_expert_interventio_learning}, though the discrete nature of the action space limits the scope of the method. 
For continuous action spaces, EBMs can be trained
via the InfoNCE loss in IBC \cite{2022_implicit_BC}, through which the energy of human actions is decreased, and the energy of other actions is increased. 

Although IBC achieves better performance than explicit policies \cite{2022_implicit_BC}, it is known to suffer from training instability. The process of training EBMs involves selecting counterexample actions, and the quality of these counterexamples significantly impacts the learning outcomes \cite{2021_how_to_train_EBM,2020_flow_constrastive_estimation_EBM}. Empirically, counterexamples that are near data labels are often preferred \cite{2020_hard_negative_mixing_contrastive}, but these selections may contribute to instability during training \cite{2022_arxiv_IBC_gaps, 2023_diffusionpolicy}. Our method addresses this issue by relaxing the BC assumption and using the proposed desired action space to train EBMs, leading to a stable training process.



\section{Preliminaries}
\label{sec:Preliminaries}

\subsection{Interactive Imitation Learning}
\label{sec:Preliminaries:IIL}

In a typical Interactive IL (IIL) problem, a Markov Decision Process (MDP) is used to model the decision-making of an agent taught by a human teacher. 
An MDP is a 5-tuple \((\mathcal  S, \mathcal A, T, R, \gamma)\), where \(\mathcal  S\) represents the set of all possible states in the environment, \(\mathcal  A\) denotes the set of actions the agent can take, \(T(\bm s' |\bm s, \bm a)\) is the transition probability, \(R(\bm s, \bm s', \bm a)\) is the reward function, giving the reward received for transitioning from state \(\bm s\) to state \(\bm s'\) via action \(\bm a\), and \(\gamma \in [0, 1]\) is the discount factor.
The reward typically reflects the desirability of the state transition.
A policy in an MDP defines the agent's behavior at a given state, denoted by \(\pi\).
In general, $\pi$ can be represented by the conditional probability $\pi(\bm a|\bm s)$ of the density function \(\pi: \mathcal S \times \mathcal A \rightarrow [0, 1]\). Consequently, given a state, $\pi$ is employed to select an action. The objective in an MDP is to find the optimal policy \(\pi^*\) that maximizes the expected sum of discounted future rewards $J(\pi) = \mathbb{E} \left[ \sum_{t=0}^\infty \gamma^{t} R(\bm s_t, \bm a_t) \right]$. 
% In this work, we model $\pi$ using a Deep Neural Network (DNN). To achieve this, we consider a Gaussian policy with fixed covariance $\bm \Sigma$ and with mean $\bm \mu_{\theta}(\bm s)$, represented by the DNN. Hence, we obtain the parameterized policy $\pi_{\bm \theta}(\bm a | \bm s) \sim \mathcal{N}\left (\bm \mu_{\theta}(\bm s), \bm \Sigma\right)$, where $\bm \theta$ denotes the DNN's parameter vector.

In IIL, a human instructor, known as the \emph{teacher}, aims to improve the behavior of the learning agent, referred to as the \emph{learner}, by providing feedback on the learner's actions. 
IIL does not rely on a predefined reward function for specific tasks, thanks to the direct guidance provided by the teacher \cite{2022_IIL_survey}.
In this paper, we consider the teacher feedback to act in the state-action space, which is described by the function \( \bm h = {H}(\bm s, \bm a)\).
%%% Explain the teacher's feedback H
The feedback $\bm h$ can be defined according to the feedback type.
For instance, in demonstration/intervention feedback, $\bm h$ represents the action the learner should execute at a given state. In contrast, for relative corrective feedback, 
% $\bm h$ is a directional signal guiding the learner's action towards a more favorable one.
 $\bm h$ is a normalized vector indicating the direction in which
the learner's action should be modified, i.e., 
${\bm h \in \mathcal{H} = \{ \bm d \in \mathcal{A} \mid  ||\bm d|| = 1\}}$.
% ${\bm h \in \mathcal{H} = \{ \bm d = \bm a_2 - \bm a_1, \bm a_1\in \mathcal{A}, \bm a_2 \in \mathcal{A} \mid  ||\bm d|| = 1\}}$.
In our context, the reward function is unknown; therefore, we cannot directly maximize the expected accumulated future rewards $J(\pi)$.
% Instead,
% an observable surrogate loss $l_{\pi}( \bm s)$ can be formulated as $l_{\pi}( \bm s) = \mathbb{E}_{\bm a \sim \pi(\bm s)} \left[ l_{\pi}(\bm s, \bm a, \mathcal{H}(\bm s, \bm a)) \right]$ 
% and the minimization of 
% $l_{\pi}( \bm s)$ 
% Instead, 
% an observable surrogate loss $l_{\pi}( \bm s)$ can be defined to indicate how well the learner's policy $\pi$ follows the teacher's feedback $\bm h$
% and the minimization of   $l_{\pi}( \bm s)$ 
% indirectly minimizes $J(\pi)$ \cite{2011_DAgger}.
Instead, an observable surrogate loss, \(\ell_{\pi}(\bm{s})\), is formulated. This loss measures the alignment of the learner's policy \(\pi\) with the teacher's feedback. In IIL, it is assumed that minimizing \(\ell_{\pi}(\bm{s})\) indirectly maximizes \(J(\pi)\) \cite{2011_DAgger,2022_IIL_survey}.
% IIL aims to find a learner's policy $\pi^l$ by solving 
As a result, the objective is to determine an optimal learner's policy $\pi^{l*}$ by solving the following equation:
\begin{equation}
    \pi^{l*} = \underset{\pi\in\Pi}{\arg\min} \mathbb{E}_{\bm s\sim d_{\pi}(\bm s)} \left[ \ell_{\pi}( \bm s) \right]
    \label{eq:IIL_formulation}
\end{equation}
%%% explain the data distribution?
where $d_{\pi}(\bm s)$ is the state distribution induced by the policy $\pi$.
In practice, the expected value of the surrogate loss in Eq.~\eqref{eq:IIL_formulation} is approximated using the data collected by a policy that interacts with the environment and the teacher.



\subsection{Implicit Behavior Cloning}

% Implicit BC \cite{2022_implicit_BC} reformulates the traditional behavioral cloning approach by using implicit models, specifically energy-based models (EBM), to represent policies. Instead of directly mapping observations to actions with an explicit function, 
Implicit BC (IBC) \cite{2022_implicit_BC} defines the policy through an energy function \( E_\theta(s, a) \) that takes state \( s \) and action \( a \) as inputs and outputs a scalar energy value. 
This formulation allows IBC to handle complex, multi-modal, and discontinuous behaviors more effectively than explicit models.
The energy function is trained using maximum likelihood estimation by minimizing the InfoNCE loss \cite{2018_InfoNCE_representation_learning, 2024_revisting_IBC}:
\begin{align}
\label{eq:ibc_info_NCE}
\ell_{\text{InfoNCE}}(\bm s, \bm a, \mathbb{A}^{neg})
\!\! = \!\! - \! \log \! \left[ \! \frac{e^{-E_\theta(\bm s, \bm a)}}{e^{-\!E_\theta(\bm s, \bm a)} \!\! + \!\!\sum_{j=1}^{N_{\text{neg}}} e^{-\!E_\theta(\bm s, \tilde{\bm a}_j)}} \!\right] \!\!,\!\!
\end{align}
where the action label \(\bm a\) is the teacher action,  \(\tilde{\bm a}_j \in \mathbb{A}^{neg} (j = 1, \dots, N_{\text{neg}}) \) are negative samples, and $ \mathbb{A}^{neg}$ is the set that includes negative samples. 
The InfoNCE loss encourages the model to assign low energy to action labels and high energy to negative samples. 
To ensure the EBM learns an accurate data distribution, the negative samples should be close to the action label, avoiding overly obvious distinctions that hinder effective learning \cite{2020_flow_constrastive_estimation_EBM}. 
% This can be achieved by obtaining
%  negative samples by sampling from the current EBM \cite{2019_EBM_Du_Yilun}, which can be achieved via MCMC (Markov Chain Monte Carlo) sampling with stochastic gradient Langevin dynamics \cite{2011_Langevin_dynamics}:
This can be achieved by generating negative samples from the current EBM using MCMC sampling with stochastic gradient Langevin dynamics \cite{2011_Langevin_dynamics, 2019_EBM_Du_Yilun}:
\begin{equation}
\tilde{\bm a}_j^i = \tilde{\bm a}_j^{i-1} - \lambda \nabla_{\bm a} E_\theta(\bm s, \tilde{\bm a}_j^{i-1}) + \sqrt{2\lambda}   \omega^i,    
\label{eq:mcmc_sampling}
\end{equation}
where $\{ \tilde{\bm a}_j^0 \}$ is initialized using the uniform distribution and $\omega^i$ is the standard normal distribution.
For each $\tilde{\bm a}_j^0 $, we run $N_{\text{MCMC}}$ steps of the MCMC chain, with $i = 0, \dots, N_{\text{MCMC}}$ denoting the step index. The step size $\lambda > 0$ can be adjusted using a polynomially decaying schedule.

% During inference, the optimal action \(\hat{\bm a}^*\) is estimated as the action with the lowest energy value and is determined by minimizing the energy function among samples from  Langevin MCMC:
During inference, the estimated optimal action \(\hat{\bm a}^*\) is obtained by minimizing the energy function, and can be approximated through Langevin MCMC:
\[
\hat{\bm a}^* = \underset{\bm a}{ \arg\min} E_\theta(\bm s, \bm a).
\]


One core assumption of IBC is that the action label is optimal and all other actions are not \cite{2022_implicit_BC}.
This assumption simplifies the classification of action samples into positive and negative categories. 
Specifically, 
the action label is considered positive, and all other sampled actions are considered negative.
However, actions considered as negative may still be valid and should not be overly penalized.
This makes selecting appropriate negative samples challenging and introduces instability during the IBC's training process.
In contrast, our CLIC method addresses this issue by labeling sampled actions within the desired action space as positive and those outside it as negative, ensuring a more stable and effective training process.
 % limiting Implicit BC's ability to learn a consistent action representation across different trails.



\subsection{Problem Formulation}

% Breifly introduce how our policy is modeled, how action is sampled (using IBC's method), how the feedback looks like

% The objective is to estimate the optimal action $\bm a^*$ for every state $\bm s \sim d_\pi(\bm s)$ via an occasional feedback signal $\bm h$.
The objective is to learn a policy $\pi$ that accurately estimates the optimal action $\bm a^*$ for every state $\bm s \sim d_\pi(\bm s)$, using occasional corrective feedback $\bm h$.
This feedback is provided by a human teacher in the robot's state-action space, placing the problem in the context of IIL.
% in the action space provided by the human teacher in the IIL framework. 
The feedback $\bm h$ can be either absolute or relative corrections, as defined in Section~\ref{sec:Preliminaries:IIL}.
% Although $\bm h$ has different meanings for absolute or relative corrections, 
% the pair of actions can be defined, which are the robot action $\bm a^{r}$ and human action $\bm a^{h}$.  
Accordingly, the \textit{observed action pair} $(\bm a^r, \bm a^h)$ can be defined, where $\bm a^r$ denotes the robot action and $\bm a^h$ denotes the human feedback action, referred to as human action for simplicity. 
For absolute correction, we have that $\bm a^h = \bm h$. 
In contrast, for relative correction, we have that $\bm a^h = \bm a^r + e \bm h$, where the magnitude hyperparameter $e$ is set to a small value.
% compared to the difference between $\bm a^r$ and the optimal action $\bm a^*$.
% % briefly show different types of feedback
% In addition to accurate absolute and relative corrections, Table \ref{tab:feedback-definitions} summarizes other common types of feedback humans provide, which are also used to evaluate the algorithms in the experiment section.

% \begin{table}[h!]
% \caption{Various feedback in the action space}
% \centering
% % \begin{tabular}{@{}ll@{}}
% \begin{tabularx}{0.49\textwidth}{@{}lX@{}}
% \toprule
% \textbf{Type of Feedback Data}      & \textbf{Definition} \\ \midrule
% Accurate absolute correction              & \( \bm a^h = \bm h, \bm h = \bm a^* \) \\
% Gaussian noise             & \( \bm a^h \sim \mathcal{N}(\bm a^*, ||\bm a^* - \bm a^r|| I) \) \\
% Partial feedback           & \( \bm a^h \in \{[\bm a^*_{r1}, \bm a_{r2}], [\bm a_{r1}, \bm a_{r2}^*]\} \) \\ 
% \hline
% Accurate relative correction              & \(\bm  a^h = \bm a^r + e\bm h^*, \bm h^*= \frac{\bm a^* - \bm a^r}{||\bm a^* - \bm a^r||}  \) \\
% Direction noise            & \( \bm a^h = \bm a^r + e \bm h_r \), \( \angle (\bm h_r, \bm h^*) = \beta \in [0, 90^\circ) \) \\
% \bottomrule
% \end{tabularx}
% \label{tab:feedback-definitions}
% \end{table}


% briefly show what is the goal of our paper, and the corresponding general formulation
Although the optimal action may not be directly extracted from human feedback for a given state, it includes information on which subset of $\mathcal{A}$ is more likely to contain optimal actions. 
Based on this insight, assuming the Euclidean action space $\mathcal{A}$, we introduce the \textit{desired action space} $\hat{\mathcal{A}}(\bm a^r, \bm a^h)$ as the set of desired actions, defined as a function of the observed action pair $(\bm a^r, \bm a^h)$.
Formally, ${\hat{\mathcal{A}}:\mathcal{A} \times \mathcal{A} \twoheadrightarrow\mathcal{A}}$, where the symbol $\twoheadrightarrow$ defines a set-valued function, i.e., elements of $\mathcal{A} \times \mathcal{A} $ are mapped to subsets of $\mathcal{A}$.
Actions within the desired action space are more likely to be optimal than those outside it. The construction of this set is detailed in Section~\ref{sec:Desired_action_space}.  
% The desired action set should include $\bm a^h$ and exclude $\bm a^r$; it also includes other desired actions implied by $(\bm a^r, \bm a^h)$.
% Instead of assuming $\bm a^* = \bm a^h$, we assume that the desired action space includes at least one optimal action. 
% And there exists a function $ I(\bm a, \bm a^r, \bm a^h) = \mathbf{1}_{\bm a \in \bar{\mathcal{A}}(\bm a^r, \bm a^h)}$ that determines whether an action belongs to this set.
% Then we can have
% \[
% \bar{\mathcal{A}}(\bm a^r, \bm a^h) = \{ \bm a \in \mathcal{A} \mid I(\bm a, \bm a^r, \bm a^h) = 1 \}.
% \]

% To learn a policy from the set $\bar{\mathcal{A}}(\bm a^r, \bm a^h)$, there are three important aspects tackled in this paper:
% (1) how to construct this set from feedback,
% (2) how to design a loss function to encourage policy to generate actions within the desired action set,  
% and (3) how to guarantee that the policy can generate optimal actions with sufficient feedback data.
% For question (1) and (3), we tackle it in in Section~\ref{sec:Desired_action_space}, which detail
% the definition and property of $\mathcal{A}(\bm a^r, \bm a^h)$.
% For question (2), we tackle it in Section~\ref{sec:Policy_shaping}, where we first give a probablisitc formulation the desired action set, then design the loss function to train the policy. 

% The desired action space can shape the policy to generate actions within this space, which we will introduce in the Section~\ref{sec:Policy_shaping}.
% We first introduce the policy model 
In this work, we model the policy $\pi$ using a Deep Neural Network (DNN). 
To achieve this, following IBC \cite{2022_implicit_BC}, for multi-modal tasks with multiple optimal actions, our policy is defined by the energy function 
\[\pi_{\theta}(\bm a | \bm s) = \frac{\exp(-E_{\theta}(\bm s, \bm a))}{Z},
\]
where $Z$ is a normalizing constant and can be approximated by $Z = \sum_{j=1}^{N_{\text{MCMC}}} \exp(-E_{\theta}(\bm s, \bm a'_j))  $.
The samples $\{\bm a'_j\}$ are obtained via Langevin MCMC sampling (see Eq.~\eqref{eq:mcmc_sampling}), and $\bm \theta$ denotes the DNN's parameter vector.

For simpler tasks with a single optimal action for every state, we can consider an explicit Gaussian policy with fixed covariance $\bm \Sigma$ and with mean $\bm \mu_{\theta}(\bm s)$, which we model using a DNN. Hence, we obtain the parameterized policy $\pi_{\bm \theta}(\bm a | \bm s) \sim \mathcal{N}\left (\bm \mu_{\theta}(\bm s), \bm \Sigma\right)$.
Both explicit and implicit policies can be estimated using our method based on the desired action space, and we will introduce them in Section~\ref{sec:Policy_shaping}.

% \textcolor{red}{Put this framework at the end of intro}

% This paper addresses three key questions for learning a policy from $\hat{\mathcal{A}}(\bm a^r, \bm a^h)$:
% \begin{enumerate}
%     \item \textbf{Constructing the set}: How to derive $\hat{\mathcal{A}}(\bm a^r, \bm a^h)$ from feedback.
%     \item \textbf{Designing the loss}: How to design a loss function that encourages the policy to generate actions within $\hat{\mathcal{A}}(\bm a^r, \bm a^h)$.
%     \item \textbf{Guaranteeing optimality}: How to ensure the policy generates optimal actions with sufficient feedback.
% \end{enumerate}

% Questions (1) and (3) are addressed in Section~\ref{sec:Desired_action_space}, detailing the definition and properties of $\hat{\mathcal{A}}(\bm a^r, \bm a^h)$. Question (2) is covered in Section~\ref{sec:Policy_shaping}, which introduces a probabilistic formulation of the desired action set and a corresponding loss function for policy training.




% We denote the data buffer of corrective data as $\mathcal{D}_{t-1} = \{[\bm s_i, \bm a_i, e \bm h_i], i = 1, \dots, k \}$ that contains all the received corrective feedback, where $k$ is the total number of the data tuples. 
% Our policy is defined by the energy function $\pi_{\theta}(\bm a | \bm s) = \frac{\exp(-E_{\theta}(\bm s, \bm a))}{Z}$, where $Z$ is the normalizing constant and is approximated by $Z = \sum_{j=1}^{N} \exp(-E_{\theta}(\bm s, \bm a'_j))  $, where the samples $\bm a'_j$ are obtained via Langevin sampling in Eq.~\eqref{eq:mcmc_sampling}.






\section{Methodology}

After annotating the dataset, we split the dataset into training, validation, and test sets. We used 184,719 images ($\sim$80\%) to train our object detection models and 23,090 images ($\sim$10\%) to validate the model during training time. The rest of the 23,090 images ($\sim$10\%) are held out to test the trained model’s performance. In this study, we employed two advanced deep-learning models for weed detection and classification: RetinaNet with a ResNeXt-101 backbone and Detection Transformer (DETR) with a ResNet-50 backbone. These models were tasked with classifying weed species and their respective growth stages (in weeks), while simultaneously localizing them within the images via bounding box predictions. We configured and trained these models using PyTorch and mmDetection on an NVIDIA RTX 3090 GPU.

\subsection{Detection Transformer with ResNet-50}

The Detection Transformer (DETR) model is an end-to-end object detection architecture that combines a convolutional backbone with a transformer encoder-decoder \cite{carion2020end}. This approach effectively addresses the complexities of identifying weeds in agricultural images. The backbone of our model ResNet-50 is a convolutional neural network, pre-trained on ImageNet (\texttt{open-mmlab://resnet50}). This 50-layer network, organized into four stages, serves as a powerful feature extractor. We utilize the output from the final stage (out\_indices=(3,)) and freeze the initial stages during training to preserve pre-learned features. The backbone's output can be represented as:
\vspace{-0.2cm}
\begin{equation}
F_{\text{resnet}} = \text{ResNet50}(I)
\end{equation}
where \(I\) is the input image. A Channel Mapper follows the backbone, transforming ResNet-50's 2048-channel output into a 256-channel feature map suitable for the transformer. This dimensionality reduction is achieved through a 1x1 convolution:
\vspace{-0.2cm}
\begin{equation}
F_{\text{neck}} = \text{Conv1x1}(F_{\text{resnet}})
\end{equation}

The core of our DETR model is the transformer module, comprising a 6-layer encoder and decoder. Each encoder layer incorporates a self-attention mechanism with 8 heads, followed by a feed-forward network (FFN) with ReLU activation. The model's bounding box head processes the decoder's output to predict class labels and bounding boxes. We employ cross-entropy loss for classification and a combination of L1 and Generalized IoU losses for bounding box regression. The overall loss function \cite{yin2019context} is defined as:
\vspace{-0.2cm}
\begin{equation}
L = \alpha \cdot L_{\text{cls}} + \beta \cdot L_{\text{bbox}} + \gamma \cdot L_{\text{iou}}
\end{equation}
where \(\alpha\), \(\beta\), and \(\gamma\) are weight coefficients. \( L_{\text{cls}} \) represents the classification loss, which in this case is the cross-entropy loss. \( L_{\text{bbox}} \) represents the bounding box regression loss, which is a combination of L1 loss and Generalized IoU loss, and \( L_{\text{iou}} \) 	represents the IoU loss, which is specifically aimed at improving the localization accuracy by penalizing the model based on the intersection over union between the predicted and ground truth bounding boxes. During training, we utilize the Hungarian algorithm \cite{ye2020cost} for bipartite matching, ensuring a one-to-one correspondence between predicted and ground-truth boxes. This approach optimizes the model's ability to accurately locate and classify weeds within agricultural images. By integrating the robust feature extraction capabilities of ResNet-50 with the DETR architecture's powerful attention mechanisms, our model achieves good performance in weed detection with 174 classes.

\subsection{RetinaNet with ResNeXt-101}

RetinaNet is a single-stage object detection model designed to address the extreme foreground-background class imbalance encountered during training \cite{li2019light}. The architecture comprises three main components: a backbone network for feature extraction, a neck (FPN) for generating multi-scale feature maps, and a detection head for predicting bounding boxes and class probabilities. We utilized ResNeXt-101 as the backbone, a variant of the ResNet architecture that employs grouped convolutions for improved efficiency and performance. The ResNeXt-101 backbone consists of 101 layers organized into four stages, with 32 groups and a base width of 4 channels per group. We initialized the backbone with weights pretrained on ImageNet (\texttt{open-mmlab://resnext101\_32x4d}) to leverage transfer learning. Batch normalization is applied after each convolutional layer to stabilize the learning process.

The Feature Pyramid Network (FPN) enhances the backbone's feature maps by combining high-level semantic features with low-level detailed features, enabling the detection of objects at various scales. The FPN generates multiple feature maps of different resolutions, which are then fed into the detection head. The detection head of RetinaNet comprises two subnetworks: a classification subnetwork for predicting object presence probabilities and a regression subnetwork for predicting bounding box coordinates. Each subnetwork consists of four convolutional layers, followed by a final convolutional layer that produces the desired outputs. To handle class imbalance, we employed the focal loss function \cite{lin2017focal} for training the classification subnetwork:
\vspace{-0.2cm}
\begin{equation}
\text{FL}(p_t) = -\alpha_t (1 - p_t)^\gamma \log(p_t)
\end{equation}

where \(p_t\) is the predicted probability, \(\alpha_t\) is a balancing factor, and \(\gamma\) is the focusing parameter.

We trained our model using an epoch-based training loop with the AdamW optimizer (learning rate \(lr = 0.0001\), weight decay \(wd = 0.0001\)). The learning rate schedule incorporated a linear warmup over the first 1000 iterations. We trained for 12 epochs with a batch size of 16, employing automatic learning rate scaling to accommodate potential batch size changes.


% \begin{algorithm}
% \caption{Model Training Process}\label{alg:training}
% \KwData{Training dataset, validation dataset, initial model parameters}
% \KwResult{Trained model}
% Initialize model parameters $\theta$\;
% \For{epoch = 1 to max\_epochs}{
%     \For{each mini-batch $(X, y)$ in training dataset}{
%         Compute predictions $\hat{y} = f_\theta(X)$\;
%         Compute classification loss $L_{\text{cls}}$ and regression loss $L_{\text{reg}}$\;
%         Compute total loss $L = L_{\text{cls}} + L_{\text{reg}}$\;
%         Backpropagate to compute gradients $\nabla_\theta L$\;
%         Update parameters $\theta$ using AdamW optimizer\;
%     }
%     \If{epoch \% val\_interval == 0}{
%         Evaluate model on validation dataset\;
%         Save model checkpoint if performance improves\;
%     }
% }
% \end{algorithm}

% The algorithm \cite{ilyas2022datamodels} outlines a model training process where the goal is to optimize the model parameters, denoted by \(\theta\). Initially, the model parameters are set to their initial values. The training process runs for a specified number of epochs, iterating over mini-batches of the training dataset in each epoch. For each mini-batch, the model makes predictions \(\hat{y}\), and the classification and regression losses are computed. These losses are summed to obtain the total loss, which is then used to calculate the gradients through backpropagation. The model parameters are updated using the AdamW optimizer. At regular intervals, defined by \texttt{val\_interval}, the model's performance is evaluated on the validation dataset, and the model checkpoint is saved if there is an improvement in performance.


% \vspace{-0.4cm}

\subsection{Evaluation Metrics}

To assess the performance of our weed detection models, we employ a comprehensive set of metrics that capture both the accuracy and robustness of the detections. Our primary metrics are Average Precision (AP), Average Recall (AR), and Mean Average Precision (mAP) evaluated across various Intersection over Union (IoU) thresholds.

AP provides a single-value summary of the precision-recall curve, effectively balancing the trade-off between precision and recall. Precision (P) is defined as the ratio of true positive detections to the sum of true positive and false positive detections: $\text{} P = \frac{TP}{TP + FP}$ and Recall (R) is the ratio of true positive detections to the sum of true positive and false negative detections: $\text{} (R) = \frac{TP}{TP + FN}$.

In this research, a true positive is a detected bounding box that correctly identifies a weed species and has an IoU above a specified threshold (e.g., 0.50) with the ground truth bounding box. A false positive is a detection that either does not sufficiently overlap with any ground truth box or incorrectly identifies the weed species. A false negative occurs when a ground truth weed instance is not detected by the model. AP \cite{robertson2008new} is calculated by integrating the precision over the recall range and it can be defined as:
\vspace{-0.2cm}
\begin{equation}
\text{AP} = \int_0^1 P(R) \, dR
\end{equation}

AR \cite{zhu2004recall} measures the model's ability to detect all relevant objects. It is computed as the average of maximum recalls at specified IoU thresholds:
\vspace{-0.2cm}
\begin{equation}
\text{AR} = \frac{1}{N} \sum_{i=1}^N R_{\text{max}}(IoU_i)
\end{equation}

mAP is the mean of AP values across different classes and is a common metric for evaluating object detection models. It provides a balanced measure of precision and recall across various IoU thresholds. It can be defined as:
\vspace{-0.2cm}
\begin{equation}
\text{mAP} = \frac{1}{C} \sum_{c=1}^C AP_c
\end{equation}

where $AP_c$ is the Average Precision for class $c$, and $C$ is the total number of classes.
We evaluate these metrics at various IoU thresholds. This multi-faceted evaluation approach allows us to comprehensively analyze our models' capabilities in detecting and classifying weeds across various scenarios, providing insights into their precision, recall, and overall detection performance.

\section{Experiment}

This section evaluates IRIS's capability to create demonstrations, focusing on the efficiency and intuitiveness of data collection.
It is followed by a performance profile analysis.

\subsection{Data Collection Evaluation in Simulation}

% IRIS is promising to collect demonstration in the simualtion,
% To verify the data collection efficiency of IRIS,
% a study was conducted to evaluate it.
% LIBERO was selected and 4 tasks is picked which represent four different movement.
% which includes
% \textit{close the microwave},
% \textit{turn off the stove},
% \textit{pick up the book in the middle and place it on the cabinet shelf},
% and
% \textit{turn on the stove and put the frying pan on it}.


% To make a fair comparison, we use the task from LIBERO and the data rather than designing other task.
% The baselines are two control interfaces from LIBERO (keyboards and 3D mouse),
% From the prior study from \cite{jiang2024comprehensive} \textcolor{red}{find another one support research} and our try,
% the hand tracking is not stable comparing to motion controller. so we choose KT and MC to conduct the user study.

% There are 8 participants in this user study, we evaluate each interface by using objective metrics ans subjective metrics.
% The objective metrics includes success rate and time consumed in each tasks,
% and the subjective metrics are conducted by questionnaires includes 
% \textit{Experience}, \textit{Usefulness}, \textit{Intuitiveness}, and \textit{Efficiency}.
% Every participant will use each interface to collect demonstration five times for each task,
% they will give a score in these four dimensions from 1 to 7.

% 
\begin{figure}[h]
    \centering
    \setlength{\tabcolsep}{2pt} % Reduce space between columns
    \renewcommand{\arraystretch}{.5} % Adjust row spacing
    \begin{tabular}{cccc}
        \includegraphics[width=0.115\textwidth]{image/user_study_task/t1.png} & 
        \includegraphics[width=0.115\textwidth]{image/user_study_task/t2.png} & 
        \includegraphics[width=0.115\textwidth]{image/user_study_task/t3.png} & 
        \includegraphics[width=0.115\textwidth]{image/user_study_task/t4.png} \\
        \scriptsize (a1) & 
        \scriptsize (b1) & 
        \scriptsize (c1) & 
        \scriptsize (d1) \\
        \includegraphics[width=0.115\textwidth]{image/user_study_task/libt1.png} & 
        \includegraphics[width=0.115\textwidth]{image/user_study_task/libt2.png} & 
        \includegraphics[width=0.115\textwidth]{image/user_study_task/libt3.png} & 
        \includegraphics[width=0.115\textwidth]{image/user_study_task/libt4.png} \\
        \scriptsize (a2) & 
        \scriptsize (b2) & 
        \scriptsize (c2) & 
        \scriptsize (d2) \\
    \end{tabular}
    \caption{
    Four tasks from LIBERO in simulation (top row: a1, b1, c1, d1) and corresponding view from Meta Quest 3 (bottom row: a2, b2, c2, d2): (a) Close the microwave, (b) Turn off the stove, (c) Pick up the book and place it on the shelf, (d) Turn on the stove and place the frying pan on it.}
    \label{fig:images_grid}
\end{figure}


To assess the data collection efficiency of IRIS, a user study was conducted by using tasks from the LIBERO benchmark \cite{liu2024libero}, which represents diverse types of movements.
Four representative tasks (see Figure~\ref{fig:images_grid}) were selected in the dimension of translation, rotation, and compound movement:
\begin{itemize}
\item[-] Task 1: \textit{close the microwave}
\item[-] Task 2: \textit{turn off the stove}
\item[-] Task 3: \textit{pick up the book in the middle and place it on the cabinet shelf}
\item[-] Task 4: \textit{turn on the stove and put the frying pan on it}.
\end{itemize}


\begin{figure}[h]
    \centering
    \setlength{\tabcolsep}{2pt} % Reduce space between columns
    \renewcommand{\arraystretch}{.5} % Adjust row spacing
    \begin{tabular}{cccc}
        \includegraphics[width=0.115\textwidth]{image/user_study_task/t1.png} & 
        \includegraphics[width=0.115\textwidth]{image/user_study_task/t2.png} & 
        \includegraphics[width=0.115\textwidth]{image/user_study_task/t3.png} & 
        \includegraphics[width=0.115\textwidth]{image/user_study_task/t4.png} \\
        \scriptsize (a1) & 
        \scriptsize (b1) & 
        \scriptsize (c1) & 
        \scriptsize (d1) \\
        \includegraphics[width=0.115\textwidth]{image/user_study_task/libt1.png} & 
        \includegraphics[width=0.115\textwidth]{image/user_study_task/libt2.png} & 
        \includegraphics[width=0.115\textwidth]{image/user_study_task/libt3.png} & 
        \includegraphics[width=0.115\textwidth]{image/user_study_task/libt4.png} \\
        \scriptsize (a2) & 
        \scriptsize (b2) & 
        \scriptsize (c2) & 
        \scriptsize (d2) \\
    \end{tabular}
    \caption{
    Four tasks from LIBERO in simulation (top row: a1, b1, c1, d1) and corresponding view from Meta Quest 3 (bottom row: a2, b2, c2, d2): (a) Close the microwave, (b) Turn off the stove, (c) Pick up the book and place it on the shelf, (d) Turn on the stove and place the frying pan on it.}
    \label{fig:images_grid}
\end{figure}



To ensure a fair comparison, the tasks and data were taken directly from LIBERO \cite{liu2024libero} without any modification.
The baselines for this study were two standard control interfaces provided by LIBERO: the Keyboard (KB) and the 3D Mouse (3M).
Based on findings from prior research \cite{jiang2024comprehensive},
hand tracking was found to be less stable than motion controllers. Therefore, we selected Kinesthetic Teaching and Motion Controller as the interfaces for the user study.

The study involved eight participants who evaluated the efficiency and intuitiveness of each interface for collecting demonstrations using both objective and subjective metrics.
The objective metrics included the success rate and the average time taken per task.
To ensure successful demonstrations, 
participants executed tasks at a very slow pace, which diluted efficiency measurements (as all interfaces appeared efficient when tasks were performed slowly), which biased participants against later interfaces \cite{jiang2024comprehensive}. 
To mitigate these biases and ensure high-quality data collection, a time limit per task was introduced,
and the time limits for four tasks are 20s, 20s, 30s, and 40s, which are quite enough for finishing the task.
The subjective metrics were assessed through a questionnaire evaluating four dimensions: \textit{Experience}, \textit{Usefulness}, \textit{Intuitiveness}, and \textit{Efficiency}.
Each participant performed each task five times using all interfaces. After finishing using one interface, they provided ratings on the subjective dimensions using a 7-point Likert scale.


\begin{table*}[!h]
\caption{Mean success rate and standard deviation per platform from the data in Fig.\ref{fig:box success rates}}
\centering
\begin{tabular}{c|c c c c}
\toprule
\textbf{Platform} & \textbf{Procedural} & \textbf{Diffusion}& \textbf{Diffusion} & \textbf{Diffusion} \\
\textbf{Heights (cm)} & \textbf{Generator} & & \textbf{VF-Offline} & \textbf{VF-Online}\\
\hline
0.10 & 0.88 $\pm$ 0.06 & 0.97 $\pm$ 0.04 & 0.98 $\pm$ 0.02 & 0.97 $\pm$ 0.02 \\
0.15 & 0.90 $\pm$ 0.07 & 0.96 $\pm$ 0.02 & 0.99 $\pm$ 0.02 & 0.99 $\pm$ 0.02 \\
0.20 & 0.91 $\pm$ 0.02 & 0.93 $\pm$ 0.05 & 0.98 $\pm$ 0.02 & 0.98 $\pm$ 0.04 \\
0.25 & 0.81 $\pm$ 0.11 & 0.93 $\pm$ 0.07 & 0.95 $\pm$ 0.08 & 0.96 $\pm$ 0.04 \\
0.30 & 0.67 $\pm$ 0.08 & 0.83 $\pm$ 0.09 & 0.86 $\pm$ 0.06 & 0.95 $\pm$ 0.03 \\
0.35 & 0.55 $\pm$ 0.13 & 0.80 $\pm$ 0.08 & 0.95 $\pm$ 0.04 & 0.93 $\pm$ 0.04 \\
0.40 & 0.62 $\pm$ 0.07 & 0.87 $\pm$ 0.09 & 0.94 $\pm$ 0.02 & 0.95 $\pm$ 0.03 \\
0.45 & 0.41 $\pm$ 0.10 & 0.79 $\pm$ 0.06 & 0.95 $\pm$ 0.05 & 0.96 $\pm$ 0.04 \\
0.50 & 0.36 $\pm$ 0.10 & 0.81 $\pm$ 0.09 & 0.97 $\pm$ 0.04 & 0.97 $\pm$ 0.04 \\
0.55 & 0.33 $\pm$ 0.05 & 0.76 $\pm$ 0.07 & 0.93 $\pm$ 0.04 & 0.98 $\pm$ 0.04 \\
0.60 & 0.18 $\pm$ 0.02 & 0.64 $\pm$ 0.05 & 0.82 $\pm$ 0.07 & 0.88 $\pm$ 0.04 \\
0.65 & 0.05 $\pm$ 0.03 & 0.21 $\pm$ 0.05 & 0.39 $\pm$ 0.06 & 0.68 $\pm$ 0.07 \\
\bottomrule
\end{tabular}
\label{tab:merged_success_rates}
\end{table*}


\begin{figure}[h!]
    \centering
    % \includegraphics[width=0.45\textwidth]{image/objective_study.pdf}
    \includegraphics[width=\linewidth]{image/objective_study_new.pdf}
    \caption{This graph shows the average task completion time (in seconds) for each interface across tasks. The KT and MC interfaces consistently perform more efficiently, while the Keyboard and 3D Mouse interfaces result in longer completion times, particularly on more complex tasks.}
    \label{fig:objective_study}
\end{figure}
For the objective metrics, Table \ref{tab:success_rate} presents the success rates for each interface across the tasks. 
A trial was considered unsuccessful if the participant failed to complete the task or exceeded the time limit. The time limits were set based on task difficulty: 20 seconds for Task 1 and Task 2, 30 seconds for Task 3, and 40 seconds for Task 4.
The data shows a success rate of over $90\%$ across all four tasks when using the KT and MC interfaces from IRIS. In contrast, the Keyboard and 3D Mouse methods from LIBERO often resulted in failures. For example, the 3D Mouse interface achieved only a $37.5\%$ success rate on task 3.
Figure \ref{fig:objective_study} shows the average time consumed for each task across four interfaces. The KT and MC interfaces consistently demonstrate lower task completion times, indicating higher efficiency. In contrast, the Keyboard and 3D Mouse interfaces show significantly higher completion times, particularly for Task 3 and Task 4, where the 3D Mouse method approaches or exceeds the task's time limit. These results align with the observed lower success rates for these interfaces, highlighting their inefficiency in time-critical tasks.


\begin{figure}[h!]
    \centering
    % \includegraphics[width=0.45\textwidth]{image/subjective_study.pdf}
    \includegraphics[width=\linewidth]{image/subjective_study_new.pdf}
    \caption{Subjective evaluation scores for usefulness, experience, intuitiveness, and efficiency across four interfaces. The KT and MC interfaces perform favorably in all categories, while the Keyboard and 3D Mouse interfaces receive lower ratings, particularly in intuitiveness and efficiency.}
    \label{fig:subjective_study}
\end{figure}

Figure \ref{fig:subjective_study} compares subjective scores for four interfaces based on usefulness, experience, intuitiveness, and efficiency.
The KT and MC interfaces consistently receive high scores across all criteria, indicating positive user perception and ease of use. 
In contrast, the Keyboard and 3D Mouse interfaces receive significantly lower ratings, particularly in intuitiveness and efficiency, reflecting the participants' difficulties in using these methods to control the robot.

The results of this user study show that IRIS received higher scores than the baseline interfaces across both objective and subjective metrics.
This indicates that the system offers a more intuitive and efficient approach for data collection.

\subsection{Performance Analysis}

The performance analysis focuses on two key aspects: network latency and headset FPS (frames per second).

Since IRIS uses asynchronous bidirectional data transfer, network latency is low, averaging around 20-30 ms. Transmission bandwidth depends on Wi-Fi capacity. Even for large scenes from RoboCasa, which contain over 200 MB of compressed assets, it takes no more than 5 seconds to transfer and generate a full scene with more than 300 objects. For real robot teleoperation, the system handles point cloud data efficiently, achieving a transmission speed of 10,000 points at 60 Hz—exceeding the camera’s frame rate.

The FPS performance depends on the hardware. On the Meta Quest 3, a scene with one robot runs at approximately 70 FPS, while on HoloLens 2, it runs around 40 FPS. As the scene size increases, FPS gradually decreases. With around 200 objects, the Meta Quest 3 headsets struggle to keep up with head movement, causing virtual objects to lag or become stuck. However, performance is additionally influenced by the complexity of the meshes and textures, as these require significant computational resources from the XR headset.

In our experiments by using Meta Quest 3, IRIS successfully handled all benchmark scenarios listed in the paper, except for some scenes from RoboCasa \cite{nasiriany2024robocasa}. These scenes have over 700 MB of assets in one single instance, which is closed to the Meta Quest 3 RAM limit. Nonetheless, IRIS was able to manage most of scenes from RoboCasa without any significant performance issues.
For real robot data collection, the optimal point cloud size is around 10,000 points, which achieves a balance between point cloud quality and FPS, maintaining a frame rate of approximately 40 FPS.


% \textcolor{red}{should be a table of performance here}



% \section{ Task Generalization Beyond i.i.d. Sampling and Parity Functions
}\label{sec:Discussion}
% Discussion: From Theory to Beyond
% \misha{what is beyond?}
% \amir{we mean two things: in the first subsection beyond i.i.d subsampling of parity tasks and in the second subsection beyond parity task}
% \misha{it has to be beyond something, otherwise it is not clear what it is about} \hz{this is suggested by GPT..., maybe can be interpreted as from theory to beyond theory. We can do explicit like Discussion: Beyond i.i.d. task sampling and the Parity Task}
% \misha{ why is "discussion" in the title?}\amir{Because it is a discussion, it is not like separate concrete explnation about why these thing happens or when they happen, they just discuss some interesting scenraios how it relates to our theory.   } \misha{it is not really a discussion -- there is a bunch of experiments}

In this section, we extend our experiments beyond i.i.d. task sampling and parity functions. We show an adversarial example where biased task selection substantially hinders task generalization for sparse parity problem. In addition, we demonstrate that exponential task scaling extends to a non-parity tasks including arithmetic and multi-step language translation.

% In this section, we extend our experiments beyond i.i.d. task sampling and parity functions. On the one hand, we find that biased task selection can significantly degrade task generalization; on the other hand, we show that exponential task scaling generalizes to broader scenarios.
% \misha{we should add a sentence or two giving more detail}


% 1. beyond i.i.d tasks sampling
% 2. beyond parity -> language, arithmetic -> task dependency + implicit bias of transformer (cannot implement this algorithm for arithmatic)



% In this section, we emphasize the challenge of quantifying the level of out-of-distribution (OOD) differences between training tasks and testing tasks, even for a simple parity task. To illustrate this, we present two scenarios where tasks differ between training and testing. For each scenario, we invite the reader to assess, before examining the experimental results, which cases might appear “more” OOD. All scenarios consider \( d = 10 \). \kaiyue{this sentence should be put into 5.1}






% for parity problem




% \begin{table*}[th!]
%     \centering
%     \caption{Generalization Results for Scenarios 1 and 2 for $d=10$.}
%     \begin{tabular}{|c|c|c|c|}
%         \hline
%         \textbf{Scenario} & \textbf{Type/Variation} & \textbf{Coordinates} & \textbf{Generalization accuracy} \\
%         \hline
%         \multirow{3}{*}{Generalization with Missing Pair} & Type 1 & \( c_1 = 4, c_2 = 6 \) & 47.8\%\\ 
%         & Type 2 & \( c_1 = 4, c_2 = 6 \) & 96.1\%\\ 
%         & Type 3 & \( c_1 = 4, c_2 = 6 \) & 99.5\%\\ 
%         \hline
%         \multirow{3}{*}{Generalization with Missing Pair} & Type 1 &  \( c_1 = 8, c_2 = 9 \) & 40.4\%\\ 
%         & Type 2 & \( c_1 = 8, c_2 = 9 \) & 84.6\% \\ 
%         & Type 3 & \( c_1 = 8, c_2 = 9 \) & 99.1\%\\ 
%         \hline
%         \multirow{1}{*}{Generalization with Missing Coordinate} & --- & \( c_1 = 5 \) & 45.6\% \\ 
%         \hline
%     \end{tabular}
%     \label{tab:generalization_results}
% \end{table*}

\subsection{Task Generalization Beyond i.i.d. Task Sampling }\label{sec: Experiment beyond iid sampling}

% \begin{table*}[ht!]
%     \centering
%     \caption{Generalization Results for Scenarios 1 and 2 for $d=10, k=3$.}
%     \begin{tabular}{|c|c|c|}
%         \hline
%         \textbf{Scenario}  & \textbf{Tasks excluded from training} & \textbf{Generalization accuracy} \\
%         \hline
%         \multirow{1}{*}{Generalization with Missing Pair}
%         & $\{4,6\} \subseteq \{s_1, s_2, s_3\}$ & 96.2\%\\ 
%         \hline
%         \multirow{1}{*}{Generalization with Missing Coordinate}
%         & \( s_2 = 5 \) & 45.6\% \\ 
%         \hline
%     \end{tabular}
%     \label{tab:generalization_results}
% \end{table*}




In previous sections, we focused on \textit{i.i.d. settings}, where the set of training tasks $\mathcal{F}_{train}$ were sampled uniformly at random from the entire class $\mathcal{F}$. Here, we explore scenarios that deliberately break this uniformity to examine the effect of task selection on out-of-distribution (OOD) generalization.\\

\textit{How does the selection of training tasks influence a model’s ability to generalize to unseen tasks? Can we predict which setups are more prone to failure?}\\

\noindent To investigate this, we consider two cases parity problems with \( d = 10 \) and \( k = 3 \), where each task is represented by its tuple of secret indices \( (s_1, s_2, s_3) \):

\begin{enumerate}[leftmargin=0.4 cm]
    \item \textbf{Generalization with a Missing Coordinate.} In this setup, we exclude all training tasks where the second coordinate takes the value \( s_2 = 5 \), such as \( (1,5,7) \). At test time, we evaluate whether the model can generalize to unseen tasks where \( s_2 = 5 \) appears.
    \item \textbf{Generalization with Missing Pair.} Here, we remove all training tasks that contain both \( 4 \) \textit{and} \( 6 \) in the tuple \( (s_1, s_2, s_3) \), such as \( (2,4,6) \) and \( (4,5,6) \). At test time, we assess whether the model can generalize to tasks where both \( 4 \) and \( 6 \) appear together.
\end{enumerate}

% \textbf{Before proceeding, consider the following question:} 
\noindent \textbf{If you had to guess.} Which scenario is more challenging for generalization to unseen tasks? We provide the experimental result in Table~\ref{tab:generalization_results}.

 % while the model struggles for one of them while as it generalizes almost perfectly in the other one. 

% in the first scenario, it generalizes almost perfectly in the second. This highlights how exposure to partial task structures can enhance generalization, even when certain combinations are entirely absent from the training set. 

In the first scenario, despite being trained on all tasks except those where \( s_2 = 5 \), which is of size $O(\d^T)$, the model struggles to generalize to these excluded cases, with prediction at chance level. This is intriguing as one may expect model to generalize across position. The failure  suggests that positional diversity plays a crucial role in the task generalization of Transformers. 

In contrast, in the second scenario, though the model has never seen tasks with both \( 4 \) \textit{and} \( 6 \) together, it has encountered individual instances where \( 4 \) appears in the second position (e.g., \( (1,4,5) \)) or where \( 6 \) appears in the third position (e.g., \( (2,3,6) \)). This exposure appears to facilitate generalization to test cases where both \( 4 \) \textit{and} \( 6 \) are present. 



\begin{table*}[t!]
    \centering
    \caption{Generalization Results for Scenarios 1 and 2 for $d=10, k=3$.}
    \resizebox{\textwidth}{!}{  % Scale to full width
        \begin{tabular}{|c|c|c|}
            \hline
            \textbf{Scenario}  & \textbf{Tasks excluded from training} & \textbf{Generalization accuracy} \\
            \hline
            Generalization with Missing Pair & $\{4,6\} \subseteq \{s_1, s_2, s_3\}$ & 96.2\%\\ 
            \hline
            Generalization with Missing Coordinate & \( s_2 = 5 \) & 45.6\% \\ 
            \hline
        \end{tabular}
    }
    \label{tab:generalization_results}
\end{table*}

As a result, when the training tasks are not i.i.d, an adversarial selection such as exclusion of specific positional configurations may lead to failure to unseen task generalization even though the size of $\mathcal{F}_{train}$ is exponentially large. 


% \paragraph{\textbf{Key Takeaways}}
% \begin{itemize}
%     \item Out-of-distribution generalization in the parity problem is highly sensitive to the diversity and positional coverage of training tasks.
%     \item Adversarial exclusion of specific pairs or positional configurations can lead to systematic failures, even when most tasks are observed during training.
% \end{itemize}




%################ previous veriosn down
% \textit{How does the choice of training tasks affect the ability of a model to generalize to unseen tasks? Can we predict which setups are likely to lead to failure?}

% To explore these questions, we crafted specific training and test task splits to investigate what makes one setup appear “more” OOD than another.

% \paragraph{Generalization with Missing Pair.}

% Imagine we have tasks constructed from subsets of \(k=3\) elements out of a larger set of \(d\) coordinates. What happens if certain pairs of coordinates are adversarially excluded during training? For example, suppose \(d=5\) and two specific coordinates, \(c_1 = 1\) and \(c_2 = 2\), are excluded. The remaining tasks are formed from subsets of the other coordinates. How would a model perform when tested on tasks involving the excluded pair \( (c_1, c_2) \)? 

% To probe this, we devised three variations in how training tasks are constructed:
%     \begin{enumerate}
%         \item \textbf{Type 1:} The training set includes all tasks except those containing both \( c_1 = 1 \) and \( c_2 = 2 \). 
%         For this example, the training set includes only $\{(3,4,5)\}$. The test set consists of all tasks containing the rest of tuples.

%         \item \textbf{Type 2:} Similar to Type 1, but the training set additionally includes half of the tasks containing either \( c_1 = 1 \) \textit{or} \( c_2 = 2 \) (but not both). 
%         For the example, the training set includes all tasks from Type 1 and adds tasks like \(\{(1, 3, 4), (2, 3, 5)\}\) (half of those containing \( c_1 = 1 \) or \( c_2 = 2 \)).

%         \item \textbf{Type 3:} Similar to Type 2, but the training set also includes half of the tasks containing both \( c_1 = 1 \) \textit{and} \( c_2 = 2 \). 
%         For the example, the training set includes all tasks from Type 2 and adds, for instance, \(\{(1, 2, 5)\}\) (half of the tasks containing both \( c_1 \) and \( c_2 \)).
%     \end{enumerate}

% By systematically increasing the diversity of training tasks in a controlled way, while ensuring no overlap between training and test configurations, we observe an improvement in OOD generalization. 

% % \textit{However, the question is this improvement similar across all coordinate pairs, or does it depend on the specific choices of \(c_1\) and \(c_2\) in the tasks?} 

% \textbf{Before proceeding, consider the following question:} Is the observed improvement consistent across all coordinate pairs, or does it depend on the specific choices of \(c_1\) and \(c_2\) in the tasks? 

% For instance, consider two cases for \(d = 10, k = 3\): (i) \(c_1 = 4, c_2 = 6\) and (ii) \(c_1 = 8, c_2 = 9\). Would you expect similar OOD generalization behavior for these two cases across the three training setups we discussed?



% \paragraph{Answer to the Question.} for both cases of \( c_1, c_2 \), we observe that generalization fails in Type 1, suggesting that the position of the tasks the model has been trained on significantly impacts its generalization capability. For Type 2, we find that \( c_1 = 4, c_2 = 6 \) performs significantly better than \( c_1 = 8, c_2 = 9 \). 

% Upon examining the tasks where the transformer fails for \( c_1 = 8, c_2 = 9 \), we see that the model only fails at tasks of the form \((*, 8, 9)\) while perfectly generalizing to the rest. This indicates that the model has never encountered the value \( 8 \) in the second position during training, which likely explains its failure to generalize. In contrast, for \( c_1 = 4, c_2 = 6 \), while the model has not seen tasks of the form \((*, 4, 6)\), it has encountered tasks where \( 4 \) appears in the second position, such as \((1, 4, 5)\), and tasks where \( 6 \) appears in the third position, such as \((2, 3, 6)\). This difference may explain why the model generalizes almost perfectly in Type 2 for \( c_1 = 4, c_2 = 6 \), but not for \( c_1 = 8, c_2 = 9 \).



% \paragraph{Generalization with Missing Coordinates.}
% Next, we investigate whether a model can generalize to tasks where a specific coordinate appears in an unseen position during training. For instance, consider \( c_1 = 5 \), and exclude all tasks where \( c_1 \) appears in the second position. Despite being trained on all other tasks, the model fails to generalize to these excluded cases, highlighting the importance of positional diversity in training tasks.



% \paragraph{Key Takeaways.}
% \begin{itemize}
%     \item OOD generalization depends heavily on the diversity and positional coverage of training tasks for the parity problem.
%     \item adversarial exclusion of specific pairs or positional configurations in the parity problem can lead to failure, even when the majority of tasks are observed during training.
% \end{itemize}


%################ previous veriosn up

% \paragraph{Key Takeaways} These findings highlight the complexity of OOD generalization, even in seemingly simple tasks like parity. They also underscore the importance of task design: the diversity of training tasks can significantly influence a model’s ability to generalize to unseen tasks. By better understanding these dynamics, we can design more robust training regimes that foster generalization across a wider range of scenarios.


% #############


% Upon examining the tasks where the transformer fails for \( c_1 = 8, c_2 = 9 \), we see that the model only fails at tasks of the form \((*, 8, 9)\) while perfectly generalizing to the rest. This indicates that the model has never encountered the value \( 8 \) in the second position during training, which likely explains its failure to generalize. In contrast, for \( c_1 = 4, c_2 = 6 \), while the model has not seen tasks of the form \((*, 4, 6)\), it has encountered tasks where \( 4 \) appears in the second position, such as \((1, 4, 5)\), and tasks where \( 6 \) appears in the third position, such as \((2, 3, 6)\). This difference may explain why the model generalizes almost perfectly in Type 2 for \( c_1 = 4, c_2 = 6 \), but not for \( c_1 = 8, c_2 = 9 \).

% we observe a striking pattern: generalization fails entirely in Type 1, regardless of the coordinate pair (\(c_1, c_2\)). However, in Type 2, generalization varies: \(c_1 = 4, c_2 = 6\) achieves 96\% accuracy, while \(c_1 = 8, c_2 = 9\) lags behind at 70\%. Why? Upon closer inspection, the model struggles specifically with tasks like \((*, 8, 9)\), where the combination \(c_1 = 8\) and \(c_2 = 9\) is entirely novel. In contrast, for \(c_1 = 4, c_2 = 6\), the model benefits from having seen tasks where \(4\) appears in the second position or \(6\) in the third. This suggests that positional exposure during training plays a key role in generalization.

% To test whether task structure influences generalization, we consider two variations:
% \begin{enumerate}
%     \item \textbf{Sorted Tuples:} Tasks are always sorted in ascending order.
%     \item \textbf{Unsorted Tuples:} Tasks can appear in any order.
% \end{enumerate}

% If the model struggles with generalizing to the excluded position, does introducing variability through unsorted tuples help mitigate this limitation?

% \paragraph{Discussion of Results}

% In \textbf{Generalization with Missing Pairs}, we observe a striking pattern: generalization fails entirely in Type 1, regardless of the coordinate pair (\(c_1, c_2\)). However, in Type 2, generalization varies: \(c_1 = 4, c_2 = 6\) achieves 96\% accuracy, while \(c_1 = 8, c_2 = 9\) lags behind at 70\%. Why? Upon closer inspection, the model struggles specifically with tasks like \((*, 8, 9)\), where the combination \(c_1 = 8\) and \(c_2 = 9\) is entirely novel. In contrast, for \(c_1 = 4, c_2 = 6\), the model benefits from having seen tasks where \(4\) appears in the second position or \(6\) in the third. This suggests that positional exposure during training plays a key role in generalization.

% In \textbf{Generalization with Missing Coordinates}, the results confirm this hypothesis. When \(c_1 = 5\) is excluded from the second position during training, the model fails to generalize to such tasks in the sorted case. However, allowing unsorted tuples introduces positional diversity, leading to near-perfect generalization. This raises an intriguing question: does the model inherently overfit to positional patterns, and can task variability help break this tendency?




% In this subsection, we show that the selection of training tasks can affect the quality of the unseen task generalization significantly in practice. To illustrate this, we present two scenarios where tasks differ between training and testing. For each scenario, we invite the reader to assess, before examining the experimental results, which cases might appear “more” OOD. 

% % \amir{add examples, }

% \kaiyue{I think the name of scenarios here are not very clear}
% \begin{itemize}
%     \item \textbf{Scenario 1:  Generalization Across Excluded Coordinate Pairs (\( k = 3 \))} \\
%     In this scenario, we select two coordinates \( c_1 \) and \( c_2 \) out of \( d \) and construct three types of training sets. 

%     Suppose \( d = 5 \), \( c_1 = 1 \), and \( c_2 = 2 \). The tuples are all possible subsets of \( \{1, 2, 3, 4, 5\} \) with \( k = 3 \):
%     \[
%     \begin{aligned}
%     \big\{ & (1, 2, 3), (1, 2, 4), (1, 2, 5), (1, 3, 4), (1, 3, 5), \\
%            & (1, 4, 5), (2, 3, 4), (2, 3, 5), (2, 4, 5), (3, 4, 5) \big\}.
%     \end{aligned}
%     \]

%     \begin{enumerate}
%         \item \textbf{Type 1:} The training set includes all tuples except those containing both \( c_1 = 1 \) and \( c_2 = 2 \). 
%         For this example, the training set includes only $\{(3,4,5)\}$ tuple. The test set consists of tuples containing the rest of tuples.

%         \item \textbf{Type 2:} Similar to Type 1, but the training set additionally includes half of the tuples containing either \( c_1 = 1 \) \textit{or} \( c_2 = 2 \) (but not both). 
%         For the example, the training set includes all tuples from Type 1 and adds tuples like \(\{(1, 3, 4), (2, 3, 5)\}\) (half of those containing \( c_1 = 1 \) or \( c_2 = 2 \)).

%         \item \textbf{Type 3:} Similar to Type 2, but the training set also includes half of the tuples containing both \( c_1 = 1 \) \textit{and} \( c_2 = 2 \). 
%         For the example, the training set includes all tuples from Type 2 and adds, for instance, \(\{(1, 2, 5)\}\) (half of the tuples containing both \( c_1 \) and \( c_2 \)).
%     \end{enumerate}

% % \begin{itemize}
% %     \item \textbf{Type 1:} The training set includes tuples \(\{1, 3, 4\}, \{2, 3, 4\}\) (excluding tuples with both \( c_1 \) and \( c_2 \): \(\{1, 2, 3\}, \{1, 2, 4\}\)). The test set contains the excluded tuples.
% %     \item \textbf{Type 2:} The training set includes all tuples in Type 1 plus half of the tuples containing either \( c_1 = 1 \) or \( c_2 = 2 \) (e.g., \(\{1, 2, 3\}\)).
% %     \item \textbf{Type 3:} The training set includes all tuples in Type 2 plus half of the tuples containing both \( c_1 = 1 \) and \( c_2 = 2 \) (e.g., \(\{1, 2, 4\}\)).
% % \end{itemize}
    
%     \item \textbf{Scenario 2: Scenario 2: Generalization Across a Fixed Coordinate (\( k = 3 \))} \\
%     In this scenario, we select one coordinate \( c_1 \) out of \( d \) (\( c_1 = 5 \)). The training set includes all task tuples except those where \( c_1 \) is the second coordinate of the tuple. For this scenario, we examine two variations:
%     \begin{enumerate}
%         \item \textbf{Sorted Tuples:} Task tuples are always sorted (e.g., \( (x_1, x_2, x_3) \) with \( x_1 \leq x_2 \leq x_3 \)).
%         \item \textbf{Unsorted Tuples:} Task tuples can appear in any order.
%     \end{enumerate}
% \end{itemize}




% \paragraph{Discussion of Results.} In the first scenario, for both cases of \( c_1, c_2 \), we observe that generalization fails in Type 1, suggesting that the position of the tasks the model has been trained on significantly impacts its generalization capability. For Type 2, we find that \( c_1 = 4, c_2 = 6 \) performs significantly better than \( c_1 = 8, c_2 = 9 \). 

% Upon examining the tasks where the transformer fails for \( c_1 = 8, c_2 = 9 \), we see that the model only fails at tasks of the form \((*, 8, 9)\) while perfectly generalizing to the rest. This indicates that the model has never encountered the value \( 8 \) in the second position during training, which likely explains its failure to generalize. In contrast, for \( c_1 = 4, c_2 = 6 \), while the model has not seen tasks of the form \((*, 4, 6)\), it has encountered tasks where \( 4 \) appears in the second position, such as \((1, 4, 5)\), and tasks where \( 6 \) appears in the third position, such as \((2, 3, 6)\). This difference may explain why the model generalizes almost perfectly in Type 2 for \( c_1 = 4, c_2 = 6 \), but not for \( c_1 = 8, c_2 = 9 \).

% This position-based explanation appears compelling, so in the second scenario, we focus on a single position to investigate further. Here, we find that the transformer fails to generalize to tasks where \( 5 \) appears in the second position, provided it has never seen any such tasks during training. However, when we allow for more task diversity in the unsorted case, the model achieves near-perfect generalization. 

% This raises an important question: does the transformer have a tendency to overfit to positional patterns, and does introducing more task variability, as in the unsorted case, prevent this overfitting and enable generalization to unseen positional configurations?

% These findings highlight that even in a simple task like parity, it is remarkably challenging to understand and quantify the sources and levels of OOD behavior. This motivates further investigation into the nuances of task design and its impact on model generalization.


\subsection{Task Generalization Beyond Parity Problems}

% \begin{figure}[t!]
%     \centering
%     \includegraphics[width=0.45\textwidth]{Figures/arithmetic_v1.png}
%     \vspace{-0.3cm}
%     \caption{Task generalization for arithmetic task with CoT, it has $\d =2$ and $T = d-1$ as the ambient dimension, hence $D\ln(DT) = 2\ln(2T)$. We show that the empirical scaling closely follows the theoretical scaling.}
%     \label{fig:arithmetic}
% \end{figure}



% \begin{wrapfigure}{r}{0.4\textwidth}  % 'r' for right, 'l' for left
%     \centering
%     \includegraphics[width=0.4\textwidth]{Figures/arithmetic_v1.png}
%     \vspace{-0.3cm}
%     \caption{Task generalization for the arithmetic task with CoT. It has $d =2$ and $T = d-1$ as the ambient dimension, hence $D\ln(DT) = 2\ln(2T)$. We show that the empirical scaling closely follows the theoretical scaling.}
%     \label{fig:arithmetic}
% \end{wrapfigure}

\subsubsection{Arithmetic Task}\label{subsec:arithmetic}











We introduce the family of \textit{Arithmetic} task that, like the sparse parity problem, operates on 
\( d \) binary inputs \( b_1, b_2, \dots, b_d \). The task involves computing a structured arithmetic expression over these inputs using a sequence of addition and multiplication operations.
\newcommand{\op}{\textrm{op}}

Formally, we define the function:
\[
\text{Arithmetic}_{S} \colon \{0,1\}^d \to \{0,1,\dots,d\},
\]
where \( S = (\op_1, \op_2, \dots, \op_{d-1}) \) is a sequence of \( d-1 \) operations, each \( \op_k \) chosen from \( \{+, \times\} \). The function evaluates the expression by applying the operations sequentially from left-to-right order: for example, if \( S = (+, \times, +) \), then the arithmetic function would compute:
\[
\text{Arithmetic}_{S}(b_1, b_2, b_3, b_4) = ((b_1 + b_2) \times b_3) + b_4.
\]
% Thus, the sequence of operations \( S \) defines how the binary inputs are combined to produce an integer output between \( 0 \) and \( d \).
% \[
% \text{Arithmetic}_{S} 
% (b_1,\,b_2,\,\dots,b_d)
% =
% \Bigl(\dots\bigl(\,(b_1 \;\op_1\; b_2)\;\op_2\; b_3\bigr)\,\dots\Bigr) 
% \;\op_{d-1}\; b_d.
% \]
% We now introduce an \emph{Arithmetic} task that, like the sparse parity problem, operates on $d$ binary inputs $b_1, b_2, \dots, b_d$. Specifically, we define an arithmetic function
% \[
% \text{Arithmetic}_{S}\colon \{0,1\}^d \;\to\; \{0,1,\dots,d\},
% \]
% where $S = (i_1, i_2, \dots, i_{d-1})$ is a sequence of $d-1$ operations, each $i_k \in \{+,\,\times\}$. The value of $\text{Arithmetic}_{S}$ is obtained by applying the prescribed addition and multiplication operations in order, namely:
% \[
% \text{Arithmetic}_{S}(b_1,\,b_2,\,\dots,b_d)
% \;=\;
% \Bigl(\dots\bigl(\,(b_1 \;i_1\; b_2)\;i_2\; b_3\bigr)\,\dots\Bigr) 
% \;i_{d-1}\; b_d.
% \]

% This is an example of our framework where $T = d-1$ and $|\Theta_t| = 2$ with total $2^d$ possible tasks. 




By introducing a step-by-step CoT, arithmetic class belongs to $ARC(2, d-1)$: this is because at every step, there is only $\d = |\Theta_t| = 2$ choices (either $+$ or $\times$) while the length is  $T = d-1$, resulting a total number of $2^{d-1}$ tasks. 


\begin{minipage}{0.5\textwidth}  % Left: Text
    Task generalization for the arithmetic task with CoT. It has $d =2$ and $T = d-1$ as the ambient dimension, hence $D\ln(DT) = 2\ln(2T)$. We show that the empirical scaling closely follows the theoretical scaling.
\end{minipage}
\hfill
\begin{minipage}{0.4\textwidth}  % Right: Image
    \centering
    \includegraphics[width=\textwidth]{Figures/arithmetic_v1.png}
    \refstepcounter{figure}  % Manually advances the figure counter
    \label{fig:arithmetic}  % Now this label correctly refers to the figure
\end{minipage}

Notably, when scaling with \( T \), we observe in the figure above that the task scaling closely follow the theoretical $O(D\log(DT))$ dependency. Given that the function class grows exponentially as \( 2^T \), it is truly remarkable that training on only a few hundred tasks enables generalization to an exponentially larger space—on the order of \( 2^{25} > 33 \) Million tasks. This exponential scaling highlights the efficiency of structured learning, where a modest number of training examples can yield vast generalization capability.





% Our theory suggests that only $\Tilde{O}(\ln(T))$ i.i.d training tasks is enough to generalize to the rest of unseen tasks. However, we show in Figure \ref{fig:arithmetic} that transformer is not able to match  that. The transformer out-of distribution generalization behavior is not consistent across different dimensions when we scale the number of training tasks with $\ln(T)$. \hongzhou{implicit bias, optimization, etc}
 






% \subsection{Task generalization Beyond parity problem}

% \subsection{Arithmetic} In this setting, we still use the set-up we introduced in \ref{subsec:parity_exmaple}, the input is still a set of $d$ binary variable, $b_1, b_2,\dots,b_d$ and ${Arithmatic_{S}}:\{0,1\}\rightarrow \{0, 1, \dots, d\}$, where $S = (i_1,i_2,\dots,i_{d-1})$ is a tuple of size $d-1$ where each coordinate is either add($+
% $) or multiplication ($\times$). The function is as following,

% \begin{align*}
%     Arithmatic_{S}(b_1, b_2,\dots,b_d) = (\dots(b1(i1)b2)(i3)b3\dots)(i{d-1})
% \end{align*}
    


\subsubsection{Multi-Step Language Translation Task}

 \begin{figure*}[h!]
    \centering
    \includegraphics[width=0.9\textwidth]{Figures/combined_plot_horiz.png}
    \vspace{-0.2cm}
    \caption{Task generalization for language translation task: $\d$ is the number of languages and $T$ is the length of steps.}
    \vspace{-2mm}
    \label{fig:language}
\end{figure*}
% \vspace{-2mm}

In this task, we study a sequential translation process across multiple languages~\cite{garg2022can}. Given a set of \( D \) languages, we construct a translation chain by randomly sampling a sequence of \( T \) languages \textbf{with replacement}:  \(L_1, L_2, \dots, L_T,\)
where each \( L_t \) is a sampled language. Starting with a word, we iteratively translate it through the sequence:
\vspace{-2mm}
\[
L_1 \to L_2 \to L_3 \to \dots \to L_T.
\]
For example, if the sampled sequence is EN → FR → DE → FR, translating the word "butterfly" follows:
\vspace{-1mm}
\[
\text{butterfly} \to \text{papillon} \to \text{schmetterling} \to \text{papillon}.
\]
This task follows an \textit{AutoRegressive Compositional} structure by itself, specifically \( ARC(D, T-1) \), where at each step, the conditional generation only depends on the target language, making \( D \) as the number of languages and the total number of possible tasks is \( D^{T-1} \). This example illustrates that autoregressive compositional structures naturally arise in real-world languages, even without explicit CoT. 

We examine task scaling along \( D \) (number of languages) and \( T \) (sequence length). As shown in Figure~\ref{fig:language}, empirical  \( D \)-scaling closely follows the theoretical \( O(D \ln D T) \). However, in the \( T \)-scaling case, we observe a linear dependency on \( T \) rather than the logarithmic dependency \(O(\ln T) \). A possible explanation is error accumulation across sequential steps—longer sequences require higher precision in intermediate steps to maintain accuracy. This contrasts with our theoretical analysis, which focuses on asymptotic scaling and does not explicitly account for compounding errors in finite-sample settings.

% We examine task scaling along \( D \) (number of languages) and \( T \) (sequence length). As shown in Figure~\ref{fig:language}, empirical scaling closely follows the theoretical \( O(D \ln D T) \) trend, with slight exceptions at $ T=10 \text{ and } 3$ in Panel B. One possible explanation for this deviation could be error accumulation across sequential steps—longer sequences require each intermediate translation to be approximated with higher precision to maintain test accuracy. This contrasts with our theoretical analysis, which primarily focuses on asymptotic scaling and does not explicitly account for compounding errors in finite-sample settings.

Despite this, the task scaling is still remarkable — training on a few hundred tasks enables generalization to   $4^{10} \approx 10^6$ tasks!






% , this case, we are in a regime where \( D \ll T \), we observe  that the task complexity empirically scales as \( T \log T \) rather than \( D \log T \). 


% the model generalizes to an exponentially larger space of \( 2^T \) unseen tasks. In case $T=25$, this is $2^{25} > 33$ Million tasks. This remarkable exponential generalization demonstrates the power of structured task composition in enabling efficient generalization.


% In the case of parity tasks, introducing CoT effectively decomposes the problem from \( ARC(D^T, 1) \) to \( ARC(D, T) \), significantly improving task generalization.

% Again, in the regime scaling $T$, we again observe a $T\log T$ dependency. Knowing that the function class is scaling as $D^T$, it is remarkable that training on a few hundreds tasks can generalize to $4^{10} \approx 1M$ tasks. 





% We further performed a preliminary investigation on a semi-synthetic word-level translation task to show that (1) task generalization via composition structure is feasible beyond parity and (2) understanding the fine-grained mechanism leading to this generalization is still challenging. 
% \noindent
% \noindent
% \begin{minipage}[t]{\columnwidth}
%     \centering
%     \textbf{\scriptsize In-context examples:}
%     \[
%     \begin{array}{rl}
%         \textbf{Input} & \hspace{1.5em} \textbf{Output} \\
%         \hline
%         \textcolor{blue}{car}   & \hspace{1.5em} \textcolor{red}{voiture \;,\; coche} \\
%         \textcolor{blue}{house} & \hspace{1.5em} \textcolor{red}{maison \;,\; casa} \\
%         \textcolor{blue}{dog}   & \hspace{1.5em} \textcolor{red}{chien \;,\; perro} 
%     \end{array}
%     \]
%     \textbf{\scriptsize Query:}
%     \[
%     \begin{array}{rl}
%         \textbf{Input} & \textbf{Output} \\
%         \hline
%         \textcolor{blue}{cat} & \hspace{1.5em} \textcolor{red}{?} \\
%     \end{array}
%     \]
% \end{minipage}



% \begin{figure}[h!]
%     \centering
%     \includegraphics[width=0.45\textwidth]{Figures/translation_scale_d.png}
%     \vspace{-0.2cm}
%     \caption{Task generalization behavior for word translation task.}
%     \label{fig:arithmetic}
% \end{figure}


\vspace{-1mm}
\section{Conclusions}
% \misha{is it conclusion of the section or of the whole paper?}    
% \amir{The whole paper. It is very short, do we need a separate section?}
% \misha{it should not be a subsection if it is the conclusion the whole thing. We can just remove it , it does not look informative} \hz{let's do it in a whole section, just to conclude and end the paper, even though it is not informative}
%     \kaiyue{Proposal: Talk about the implication of this result on theory development. For example, it calls for more fine-grained theoretical study in this space.  }

% \huaqing{Please feel free to edit it if you have better wording or suggestions.}

% In this work, we propose a theoretical framework to quantitatively investigate task generalization with compositional autoregressive tasks. We show that task to $D^T$ task is theoretically achievable by training on only $O (D\log DT)$ tasks, and empirically observe that transformers trained on parity problem indeed achieves such task generalization. However, for other tasks beyond parity, transformers seem to fail to achieve this bond. This calls for more fine-grained theoretical study the phenomenon of task generalization specific to transformer model. It may also be interesting to study task generalization beyond the setting of in-context learning. 
% \misha{what does this add?} \amir{It does not, i dont have any particular opinion to keep it. @Hongzhou if you want to add here?}\hz{While it may not introduce anything new, we are following a good practice to have a short conclusion. It provides a clear closing statement, reinforces key takeaways, and helps the reader leave with a well-framed understanding of our contributions. }
% In this work, we quantitatively investigate task generalization under autoregressive compositional structure. We demonstrate that task generalization to $D^T$ tasks is theoretically achievable by training on only $\tilde O(D)$ tasks. Empirically, we observe that transformers trained indeed achieve such exponential task generalization on problems such as parity, arithmetic and multi-step language translation. We believe our analysis opens up a new angle to understand the remarkable generalization ability of Transformer in practice. 

% However, for tasks beyond the parity problem, transformers appear to fail to reach this bound. This highlights the need for a more fine-grained theoretical exploration of task generalization, especially for transformer models. Additionally, it may be valuable to investigate task generalization beyond the scope of in-context learning.



In this work, we quantitatively investigated task generalization under the autoregressive compositional structure, demonstrating both theoretically and empirically that exponential task generalization to $D^T$ tasks can be achieved with training on only $\tilde{O}(D)$ tasks. %Our theoretical results establish a fundamental scaling law for task generalization, while our experiments validate these insights across problems such as parity, arithmetic, and multi-step language translation. The remarkable ability of transformers to generalize exponentially highlights the power of structured learning and provides a new perspective on how large language models extend their capabilities beyond seen tasks. 
We recap our key contributions  as follows:
\begin{itemize}
    \item \textbf{Theoretical Framework for Task Generalization.} We introduced the \emph{AutoRegressive Compositional} (ARC) framework to model structured task learning, demonstrating that a model trained on only $\tilde{O}(D)$ tasks can generalize to an exponentially large space of $D^T$ tasks.
    
    \item \textbf{Formal Sample Complexity Bound.} We established a fundamental scaling law that quantifies the number of tasks required for generalization, proving that exponential generalization is theoretically achievable with only a logarithmic increase in training samples.
    
    \item \textbf{Empirical Validation on Parity Functions.} We showed that Transformers struggle with standard in-context learning (ICL) on parity tasks but achieve exponential generalization when Chain-of-Thought (CoT) reasoning is introduced. Our results provide the first empirical demonstration of structured learning enabling efficient generalization in this setting.
    
    \item \textbf{Scaling Laws in Arithmetic and Language Translation.} Extending beyond parity functions, we demonstrated that the same compositional principles hold for arithmetic operations and multi-step language translation, confirming that structured learning significantly reduces the task complexity required for generalization.
    
    \item \textbf{Impact of Training Task Selection.} We analyzed how different task selection strategies affect generalization, showing that adversarially chosen training tasks can hinder generalization, while diverse training distributions promote robust learning across unseen tasks.
\end{itemize}



We introduce a framework for studying the role of compositionality in learning tasks and how this structure can significantly enhance generalization to unseen tasks. Additionally, we provide empirical evidence on learning tasks, such as the parity problem, demonstrating that transformers follow the scaling behavior predicted by our compositionality-based theory. Future research will  explore how these principles extend to real-world applications such as program synthesis, mathematical reasoning, and decision-making tasks. 


By establishing a principled framework for task generalization, our work advances the understanding of how models can learn efficiently beyond supervised training and adapt to new task distributions. We hope these insights will inspire further research into the mechanisms underlying task generalization and compositional generalization.

\section*{Acknowledgements}
We acknowledge support from the National Science Foundation (NSF) and the Simons Foundation for the Collaboration on the Theoretical Foundations of Deep Learning through awards DMS-2031883 and \#814639 as well as the  TILOS institute (NSF CCF-2112665) and the Office of Naval Research (ONR N000142412631). 
This work used the programs (1) XSEDE (Extreme science and engineering discovery environment)  which is supported by NSF grant numbers ACI-1548562, and (2) ACCESS (Advanced cyberinfrastructure coordination ecosystem: services \& support) which is supported by NSF grants numbers \#2138259, \#2138286, \#2138307, \#2137603, and \#2138296. Specifically, we used the resources from SDSC Expanse GPU compute nodes, and NCSA Delta system, via allocations TG-CIS220009. 

\vspace{-3.5mm}
\section{Conclusion}
\vspace{-2mm}

In summary, we introduce a first-order proximal algorithm to solve the perspective relaxation of cardinality-constrained GLM problems.
By leveraging the problem’s unique mathematical structure, we design a customized PAVA to efficiently evaluate the proximal operator, ensuring scalability to high-dimensional settings.
Further acceleration is achieved through an efficient value-based restart strategy and compatibility with GPUs, which collectively enhance convergence rates and computational speed.
Extensive empirical results demonstrate that our method outperforms state-of-the-art solvers by 1-2 orders of magnitude, establishing it as a practical, high-performance component for integration into next-generation MIP solvers.

\section*{Acknowledgments}
Omitted for Anonymous Review.

%% Use plainnat to work nicely with natbib. 
\appendix

\section{Ice-breaker questions}
\label{appendix:ice-breaker}
Please go through these questions together, you do not need to answer them all.
\begin{itemize}
    \item Share [University Name] Introductions 
    \item What's the last TV show or movie you watched and enjoyed?
    \item Do you have any pets, and if not, what kind of pet would you like to have?
    \item If you could travel anywhere in the world, where would you go and why?
    \item What was your dream job as a kid?
    \item What's your favorite type of food, and why do you love it?
    \item What's a hobby or activity you enjoy doing in your free time?
    \item What kind of music do you like to listen to, and do you have any favorite artists?
    \item What's a skill or talent you wish you had, and why?
    \item What's the best piece of advice you've ever received, and did you follow it?
\end{itemize}










% % \bibliographystyle{plainnat}
% \bibliographystyle{unsrt} % Ensures citations appear in order
\bibliographystyle{jabbrv_ieeetr}
\bibliography{main}

% \bibliographystyle{./IEEEtranBST/IEEEtran}
% \bibliography{./IEEEtranBST/IEEEabrv, references}

% \documentclass[10pt,twocolumn,letterpaper]{article}
\usepackage[rebuttal]{cvpr}

% Include other packages here, before hyperref.
\usepackage{graphicx}
\usepackage{amsmath}
\usepackage{amssymb}
\usepackage{booktabs}
\usepackage{color}
\usepackage{colortbl}
% \usepackage{ulem}
% \useunder{\uline}{\ul}{}
% Import additional packages in the preamble file, before hyperref
%
% --- inline annotations
%
\newcommand{\red}[1]{{\color{red}#1}}
\newcommand{\todo}[1]{{\color{red}#1}}
\newcommand{\TODO}[1]{\textbf{\color{red}[TODO: #1]}}
% --- disable by uncommenting  
% \renewcommand{\TODO}[1]{}
% \renewcommand{\todo}[1]{#1}



\newcommand{\VLM}{LVLM\xspace} 
\newcommand{\ours}{PeKit\xspace}
\newcommand{\yollava}{Yo’LLaVA\xspace}

\newcommand{\thisismy}{This-Is-My-Img\xspace}
\newcommand{\myparagraph}[1]{\noindent\textbf{#1}}
\newcommand{\vdoro}[1]{{\color[rgb]{0.4, 0.18, 0.78} {[V] #1}}}
% --- disable by uncommenting  
% \renewcommand{\TODO}[1]{}
% \renewcommand{\todo}[1]{#1}
\usepackage{slashbox}
% Vectors
\newcommand{\bB}{\mathcal{B}}
\newcommand{\bw}{\mathbf{w}}
\newcommand{\bs}{\mathbf{s}}
\newcommand{\bo}{\mathbf{o}}
\newcommand{\bn}{\mathbf{n}}
\newcommand{\bc}{\mathbf{c}}
\newcommand{\bp}{\mathbf{p}}
\newcommand{\bS}{\mathbf{S}}
\newcommand{\bk}{\mathbf{k}}
\newcommand{\bmu}{\boldsymbol{\mu}}
\newcommand{\bx}{\mathbf{x}}
\newcommand{\bg}{\mathbf{g}}
\newcommand{\be}{\mathbf{e}}
\newcommand{\bX}{\mathbf{X}}
\newcommand{\by}{\mathbf{y}}
\newcommand{\bv}{\mathbf{v}}
\newcommand{\bz}{\mathbf{z}}
\newcommand{\bq}{\mathbf{q}}
\newcommand{\bff}{\mathbf{f}}
\newcommand{\bu}{\mathbf{u}}
\newcommand{\bh}{\mathbf{h}}
\newcommand{\bb}{\mathbf{b}}

\newcommand{\rone}{\textcolor{green}{R1}}
\newcommand{\rtwo}{\textcolor{orange}{R2}}
\newcommand{\rthree}{\textcolor{red}{R3}}
\usepackage{amsmath}
%\usepackage{arydshln}
\DeclareMathOperator{\similarity}{sim}
\DeclareMathOperator{\AvgPool}{AvgPool}

\newcommand{\argmax}{\mathop{\mathrm{argmax}}}     



% If you comment hyperref and then uncomment it, you should delete
% egpaper.aux before re-running latex.  (Or just hit 'q' on the first latex
% run, let it finish, and you should be clear).
\definecolor{cvprblue}{rgb}{0.21,0.49,0.74}
\definecolor{mygray}{gray}{.9}
\usepackage[pagebackref,breaklinks,colorlinks,citecolor=cvprblue]{hyperref}


\newcommand{\re}[2]{\textcolor{#1}{{\bf #2}}}

% Support for easy cross-referencing
\usepackage[capitalize]{cleveref}
\crefname{section}{Sec.}{Secs.}
\Crefname{section}{Section}{Sections}
\Crefname{table}{Table}{Tables}
\crefname{table}{Tab.}{Tabs.}

% If you wish to avoid re-using figure, table, and equation numbers from
% the main paper, please uncomment the following and change the numbers
% appropriately.
%\setcounter{figure}{2}
\setcounter{table}{0}
\renewcommand\thetable{Rb\arabic{table}}
%\setcounter{equation}{2}

% If you wish to avoid re-using reference numbers from the main paper,
% please uncomment the following and change the counter for `enumiv' to
% the number of references you have in the main paper (here, 6).
%\let\oldthebibliography=\thebibliography
%\let\oldendthebibliography=\endthebibliography
%\renewenvironment{thebibliography}[1]{%
%     \oldthebibliography{#1}%
%     \setcounter{enumiv}{6}%
%}{\oldendthebibliography}


%%%%%%%%% PAPER ID  - PLEASE UPDATE
\def\paperID{2514} % *** Enter the Paper ID here
\def\confName{CVPR}
\def\confYear{2025}

\begin{document}

%%%%%%%%% TITLE - PLEASE UPDATE
\title{Category-Level Object Pose Estimation via Causal Learning and Knowledge Distillation}  % **** Enter the paper title here

\maketitle
\thispagestyle{empty}
\appendix

%%%%%%%%% BODY TEXT - ENTER YOUR RESPONSE BELOW
% \section{Introduction}
% \re{red}{R1} \re{blue}{R2} \re{green}{R3}
\noindent
We appreciate reviewers for their valuable feedback, acknowledging the \emph{“clarity and novelty”} (\re{green}{rYxx@R3}) of our core idea, as well as the \emph{“detailed formulation provided”} (\re{blue}{qCQv@R2}). We are encouraged they recognize our approach \emph{“well-organized and easy to follow”} (\re{blue}{R2}, \re{green}{R3}), evaluated with \emph{“extensive experiments”} (\re{green}{R3}), and \emph{“achieving {\bf \emph{SOTA}} performance”} in multiple benchmarks (\re{red}{QHjZ@R1}, \re{blue}{R2}, \re{green}{R3}). We sincerely thank the reviewers for their diligent work and hope our response meets their approval.

\noindent
{\bf $\triangleright$ Responses to individual questions for each reviewer.}

\noindent
\re{red}{@R1 \#w1:} {\bf The Clarity of Paper.}  Thanks for your feedback! Key terms such as \emph{“confounders”} and \emph{“front-door path”} are formally introduced in Sec.3.2 (L201-214), and the causal modeling is illustrated in Fig.2. To improve clarity, we will further enhance the explanation of these key concepts by providing more concrete examples from the task domain and clarifying their roles in the methodology. Additionally, we will confirm that the causal modeling process is presented more intuitively to make it easier to follow.

\noindent
\re{red}{@R1 \#w2:} {\bf The Novelty of Method.} We appreciate your concerns. The key novelty and contribution of our work lies in integrating causal theoretical foundation [36, 37] into COPE models. As Reviewer 3 noted: \emph{“this idea is novel and not straightforward”}. To the best of our knowledge, we are the first to introduce causal modeling to enhance COPE models and achieve significant improvements. Therefore, our work is not simply a combination of existing methods, which involves careful exploration and design that addresses specific challenges mitigating confounding effects.


\noindent
\re{red}{@R1 \#w3:} {\bf Fairness of Comparison.} We would address your comment in two aspects: First, one of the contributions in our work is to explore how to better utilize 3D foundation models to enhance generalization, which is closely related to the overall design of our method, rather than introducing “extra knowledge” that benefits the comparison. Second, 3D foundation model only provides supervision during training, meaning it would not increase any computational burden during inference.
% as shown in \cref{tab:ablation_main} (\#1, \#2).
As for the comparison, we followed the domain consensus to report the metric accuracy.
In response to your suggestions, we have added more terms (\eg encoder type, inference latency) in \cref{tab:ablation_main} for comprehensive comparison. Due to space limitations, detailed comparisons and explanations will be included in revision.

\noindent
\re{blue}{@R2 \#w1:} {\bf Efficacy of Causal Learning.} In fact, as shown in Tab. \textcolor{red}{4}, introducing causal inference can already achieves SOTA results in the rigorous metric of 5°2\emph{cm}, 5°5\emph{cm} and 10°2\emph{cm}, surpassing the baseline by 2.7\%, 1.9\% and 2.8\%. Additionally, as confirmed in \cref{tab:ablation_main}, the front-door adjustment only increases the number of parameters by 10\% (246M \vs 223M), while the running time remains nearly unchanged (33 \vs 35 in FPS).
To ensure a fair comparison, we also replace the front-door module with MLPs that have the same number of parameters (\#3). The results further demonstrate the superior effectiveness of causal learning.


\noindent
\re{blue}{@R2 \#w2:} {\bf Comparison with Feature Concat.} In response to your suggestion, we have concatenated $\mathcal{F}^{ULIP}_{P}$, $\mathcal{F}_{I}$ and $\mathcal{F}_{P}$ for comparison in \cref{tab:ablation_main} (\#4 \vs \#2). The results indicate that the proposed knowledge distillation approach is more effective than feature concatenation.

\noindent
\re{blue}{@R2 \#w3:} {\bf Type of ViT and Additional Comparison.} ViT-S/14 (DINOv2). Following your suggestion, we have added the encoder backbone in \cref{tab:ablation_main}, and included additional results with ResNet18 (\#6). Specifically, our method still outperforms AG-Pose with ResNet18 setting (\#6 \vs \#5), further supporting the efficacy of our approach.

\noindent
\re{green}{@R3 \#w1:} {\bf Reflect of Limitation.} We sincerely acknowledge your valuable feedback! Regarding this limitation, since our method selects PointNet++ as 3D encoder, we hypothesize that when the teacher and student models share similar architectures (\eg, both use PointNet++), the distillation may mislead the student model to focus on feature structure similarity rather than transferring category knowledge. We will add the analysis and detailed difference among three backbones in main text or appendix materials.

\noindent
\re{green}{@R3 \#w2:} {\bf Pose Loss.} L1 loss. We will address it.

\noindent
\re{green}{@R3 \#w4:} {\bf Effect of $N_{s}$.} We speculate that a larger sample size may introduce noise and redundant information that affects key features in causal inference. An appropriate sample size can balance valid and redundant information, prompting the model to focus on learning more representative causal correlations. This will be added in appendix.

\noindent
\re{green}{@R3 \#w5:} {\bf Effect of Sampling.} We have conducted additional experiments with 6 different random seeds, as shown in \cref{tab:inference_samp}. The computed variances $\sigma^{2}$ for metrics demonstrate stable performance across different random seeds, indicating the robustness and reliability of our method.
% Further discussion will be added in revision.

\noindent
\re{blue}{@R2 \#w4}, \re{green}{@R3 \#w3:} {\bf Proofreading.} Thanks for reviewers' reminders and corrections. We will add related works' results and address the caption in revision.
\vspace{-0.3cm}

\begin{table}[htbp]
    \small
    \centering
    \setlength\tabcolsep{4pt}%2pt 列宽
    % \renewcommand\arraystretch{0.9} % 行高
    \begin{tabular}{c|l|c|c|c|c|c}
    % \toprule%[1.2pt]
    \hline
   ID & Method & Encoder & Param.$\downarrow$ & Distill. & 5°2\emph{cm}$\uparrow$  & FPS$\uparrow$\\
    % \midrule%[1pt]
    \hline
    1        &AG-Pose      &  ViT-S/14     &\textbf{223M}          & -    &57.0  &\textbf{35}   \\
    \rowcolor{mygray}
    2         &ours     &ViT-S/14        & \underline{246M}        &Default    &	\textbf{61.5} &\underline{33}   \\
    % \hline
    3        &ours$^*$      &  ViT-S/14      & \underline{246M}          &Default    &59.4 &\underline{33}   \\
    4         &ours     &ViT-S/14        & \underline{246M}        &Concat    &59.8 &31   \\
    \hline
    5         & AG-Pose        & resnet18      & \textbf{220M}          & -   &56.2 &\textbf{35}   \\
    6         & ours        & resnet18      & \underline{243M}          & Default   &\underline{60.3} &\underline{33}   \\
    % \bottomrule%[1.2pt]
    \hline
    \end{tabular}
    \vspace{-0.3cm}
    \caption{Additional results. $*$ denotes replacement of causal module with MLPs of the same number of parameters.
    }
    \label{tab:ablation_main}
\end{table}

\vspace{-0.6cm}

\begin{table}[htbp]
    \small
    \centering
    \setlength\tabcolsep{5pt}%2pt 列宽
    % \renewcommand\arraystretch{0.9} % 行高
    \begin{tabular}{c|cccccc|c}
    % \toprule%[1.2pt]
    \hline
   Seed & 1 & 42 & 500 & 1k & 1w  & 10w& $\sigma^{2}\downarrow$ \\
    % \midrule%[1pt]
    \hline
    5°2\emph{cm}              &  61.4     &61.5         & 61.7    &61.4  &61.3 &61.7&0.03   \\
    5°5\emph{cm}     &67.2        & 67.3        &67.5    &	67.2 &67.1 &67.6&0.04  \\
    % \hline
    % 10°2\emph{cm}              &  78.1      & 78.3          &78.0    &78.0 &78.0  &78.0 &0.01 \\
    % \bottomrule%[1.2pt]
    \hline
    \end{tabular}
    \vspace{-0.2cm}
    \caption{Effect of sampling during inference.
    }
    \label{tab:inference_samp}
\end{table}

% \begin{table}[htbp]
%     \small
%     \centering
%     \setlength\tabcolsep{5pt}%2pt 列宽
%     % \renewcommand\arraystretch{1.2} % 行高
%     \begin{tabular}{c|cccccc|c}
%     \toprule%[1.2pt]
%    $N_{s}$(Infer.) & 6 & 12 & 18 & 24 & 48  & 80& $\sigma^2$ \\
%     % \midrule%[1pt]
%     \hline
%     5°2\emph{cm}$\uparrow$              &  61.1     &\textbf{61.5}         & \underline{61.4}    &60.9  &60.1 &59.9&0.03   \\
%     5°5\emph{cm}$\uparrow$     &66.9        & \textbf{67.4}        &\underline{67.2}    &	66.5 &66.0 &65.8&0.03  \\
%     % \hline
%     10°2\emph{cm}$\uparrow$              &  78.0      & \underline{78.3}          &\textbf{78.5}    &77.8 &77.6  &76.7&0.02 \\
%     \bottomrule%[1.2pt]
%     \end{tabular}
%     \vspace{-0.2cm}
%     \caption{Effect of causal learning and knowledge distillation.
%     }
%     \vspace{-0.2cm}
%     \label{tab:inference_Ns}
% \end{table}



\end{document}


\end{document}


