\documentclass[journal]{IEEEtran}
\usepackage{times}

% numbers option provides compact numerical references in the text.
\let\labelindent\relax
\usepackage{enumitem}
\usepackage{jabbrv}
\usepackage[numbers]{natbib}
\usepackage{multicol}
\usepackage[bookmarks=true]
{hyperref}

\usepackage{bm}
\usepackage{amsmath}
\usepackage{amssymb}
\usepackage{mathtools} % Provides \coloneqq
\usepackage{amsfonts}
\usepackage{comment}
\usepackage{graphicx}
\usepackage[dvipsnames]{xcolor}

\usepackage{gensymb}
\usepackage{algorithm}
\usepackage{algpseudocode}
% \usepackage{todonotes}
\usepackage{booktabs}
\usepackage{makecell}  % for \Xhline
\usepackage{tabularx} % For controlling table width
\usepackage{cases}  % provides the numcases environment

\usepackage{pifont} % for cross symbol

% \pdfinfo{
%    /Author (Homer Simpson)
%    /Title  (Robots: Our new overlords)
%    /CreationDate (D:20101201120000)
%    /Subject (Robots)
%    /Keywords (Robots;Overlords)
% }

\newtheorem{proposition}{Proposition}
\usepackage{multirow}

% make the caption of the table not in captical letters
\usepackage{caption}
\captionsetup{labelfont=bf, textfont=normal}
\captionsetup[table]{aboveskip=0pt}  % reduce the distance between a table and its above caption
\captionsetup[table]{belowskip=-5pt}  % Adjust 5pt as needed


\usepackage{svg}  % for svg type image

% to use \diagdown
\usepackage{amssymb}
% \usepackage{bbm} % for mathbb 1
% \usepackage{cite}

\newcommand{\issue}[1]{\vspace{0.1em}\noindent \textbf{#1 \hspace{0.2em}}} % for rebuttal.tex

\usepackage{booktabs} % for pandas-generated table
\usepackage{dblfloatfix}

% \usepackage[colorlinks=false, pdfborder={0 0 0}, hypertexnames=false]{hyperref}

\setlength{\textfloatsep}{3pt}
\setlength{\abovedisplayskip}{2pt}
\setlength{\belowdisplayskip}{2pt}
\setlength{\abovedisplayshortskip}{2pt}
\setlength{\belowdisplayshortskip}{2pt}

\renewcommand*{\bibfont}{\footnotesize} % decrease the fontsize of bibliography

\begin{document}


% paper title
% \title{Direct Policy Shaping via Human Correction Feedback: a policy contrastive approach}
% \title{A General Approach for Policy Shaping via Action Contrasting}

% \title{
% Policy Shaping Through Interactive Action Contrasting: A Contrastive Approach in Action Space
% }

% \title{
% Policy Shaping Through Action Contrasting: A General Approach to Optimal Action Estimation
% }

% \title{
% Contrastive Policy Learning via Interactive Correction: A General Approach to Optimal Action Estimation
% }

% \title{Contrastive Policy Learning from Interactive Correction for Optimal Action Estimation}

% \title{Contrastive Policy Learning from Interactive Corrections via Optimal Action Estimation}


% \title{Contrastive Policy Learning from Interactive Corrections via Iterative  Undesired Actions Pruning}

% \title{Contrastive Policy Learning from Interactive Absolute and Relative Corrections}

% \title{Policy Shaping through Human Feedback via Contrastive Learning}

% \title{Policy Shaping through Human Feedback: \\A Contrastive Learning Perspective}

\title{Beyond Behavior Cloning: Robustness through Interactive Imitation and Contrastive Learning}

% \title{Beyond Behavior Cloning: Interactive Imitation and Contrastive Learning Enable Robust, Efficient Learning}

% \title{Beyond Behavior Cloning: Robustness and  Efficiency with Interactive Imitation and Contrastive Learning}

% \title{Beyond Behavior Cloning: \\ Achieving Robustness Efficiently with \\Interactive Imitation and Contrastive Learning}


% \title{Contrastive Policy Learning from Interactive Corrections via Contracting the Desired Action Space}


% You will get a Paper-ID when submitting a pdf file to the conference system
% \author{Author Names Omitted for Anonymous Review. Paper-ID 119}
\author{
    \IEEEauthorblockN{Zhaoting Li, Rodrigo P{\'e}rez-Dattari, Robert Babuska, Cosimo Della Santina, Jens Kober} \\
    \IEEEauthorblockA{Delft University of Technology, \{ z.li-23, r.j.perezdattari, r.babuska, c.dellasantina, j.kober \}@tudelft.nl}
    \href{https://clic-webpage.github.io }{https://clic-webpage.github.io}
}

% \author{\authorblockN{Zhaoting Li}
% \authorblockA{Department of Cognitive Robotics\\ Delft University of Technology\\ The
% Netherlands\\
% Email: Z.Li-23@tudelft.nl}
% \and
% \authorblockN{Rodrigo P{\'e}rez-Dattari}
% \authorblockA{Department of Cognitive Robotics\\ Delft University of Technology\\ The
% Netherlands\\
% Email: Z.Li-23@tudelft.nl}
% \and
% \authorblockN{ Robert Babuska}
% \authorblockA{Department of Cognitive Robotics\\ Delft University of Technology\\ The
% Netherlands\\
% Email: Z.Li-23@tudelft.nl}
% \and 
% \authorblockN{Cosimo Della Santina}
% \authorblockA{Department of Cognitive Robotics\\ Delft University of Technology\\ The
% Netherlands\\
% Email: Z.Li-23@tudelft.nl}
% \and
% \authorblockN{Jens Kober}
% \authorblockA{Department of Cognitive Robotics\\ Delft University of Technology\\ The
% Netherlands\\
% Email: Z.Li-23@tudelft.nl}}


% avoiding spaces at the end of the author lines is not a problem with
% conference papers because we don't use \thanks or \IEEEmembership


% for over three affiliations, or if they all won't fit within the width
% of the page, use this alternative format:
% 
%\author{\authorblockN{Michael Shell\authorrefmark{1},
%Homer Simpson\authorrefmark{2},
%James Kirk\authorrefmark{3}, 
%Montgomery Scott\authorrefmark{3} and
%Eldon Tyrell\authorrefmark{4}}
%\authorblockA{\authorrefmark{1}School of Electrical and Computer Engineering\\
%Georgia Institute of Technology,
%Atlanta, Georgia 30332--0250\\ Email: mshell@ece.gatech.edu}
%\authorblockA{\authorrefmark{2}Twentieth Century Fox, Springfield, USA\\
%Email: homer@thesimpsons.com}
%\authorblockA{\authorrefmark{3}Starfleet Academy, San Francisco, California 96678-2391\\
%Telephone: (800) 555--1212, Fax: (888) 555--1212}
%\authorblockA{\authorrefmark{4}Tyrell Inc., 123 Replicant Street, Los Angeles, California 90210--4321}}


\IEEEpeerreviewmaketitle
% \maketitle
\twocolumn[{%
	\renewcommand\twocolumn[1][]{#1}%
	\maketitle
        \vspace{-6mm}
	\begin{center}
		\includegraphics[width=0.99\textwidth]{figs/Fig_3_cover_Figure_3.pdf}
% \includesvg[width=0.99\textwidth,inkscapelatex=false]{figs/Fig_3_cover_Figure_3.svg}
 \captionof{figure}{\small
 A: Our method operates in an Interactive Imitation Learning framework. The robot's policy outputs a robot action $\bm a^r$, which interacts with the environment. The human teacher provides corrective feedback occasionally if the robot action is suboptimal.
 In (a1), such feedback stored in data buffer $\mathcal{ D}$ defines multiple desired action spaces. 
These regions collectively define an overall desired action space $\hat{\mathcal{A}}^{\mathcal{D}}$.
 In (a2), the policy, modeled as an energy-based model (EBM), is trained to generate actions within $\hat{\mathcal{A}}^{\mathcal{D}}$.
B: Examples of the learned EBMs in a 2D action space. Implicit BC \cite{2022_implicit_BC} overfits each action label, while our method estimates the optimal action without overfitting.
% The main difference is that, instead of lowering energy for each action label and raising it for others, our loss reduces energy for actions within a desired action space. 
% which is defined by the data and shown as the gray-shaded area (b1).
\label{fig:framework}}
	\end{center}
}]
% \begin{abstract} Behavior cloning (BC) traditionally relies on demonstration feedback, assuming the demonstrated actions are optimal. This assumption can lead to overfitting, particularly with expressive models like the energy-based model utilized in Implicit BC. To address this, we reformulate behavior cloning as an optimal action estimation problem and introduce \textit{Contrastive policy Learning from Interactive Corrections (CLIC)}. CLIC leverages human corrections—both absolute and relative—to construct a set of desired actions and optimizes the policy to select actions from this set. We provide theoretical guarantees for the convergence of the desired action set to optimal actions in both single and multiple optimal action cases. Extensive simulation and real-robot experiments validate CLIC's advantages over state-of-the-art BC methods, including stable training of energy-based models, robustness to feedback noise, and adaptability to diverse feedback beyond demonstrations.
% To our best knowledge, this work offers a fresh perspective on policy learning by focusing on iteratively estimating optimal actions rather than directly imitating them. 
% The code will be publicly available. 
% \end{abstract}
\begin{abstract} Behavior cloning (BC) traditionally relies on demonstration data, assuming the demonstrated actions are optimal. This can lead to overfitting under noisy data, particularly when expressive models are used (e.g., the energy-based model in Implicit BC). To address this, we extend behavior cloning into an iterative process of optimal action estimation within the Interactive Imitation Learning framework. Specifically, we introduce \textit{Contrastive policy Learning from Interactive Corrections (CLIC)}. CLIC leverages human corrections to estimate a set of desired actions and optimizes the policy to select actions from this set. We provide theoretical guarantees for the convergence of the desired action set to optimal actions in both single and multiple optimal action cases. Extensive simulation and real-robot experiments validate CLIC's advantages over existing state-of-the-art methods, including stable training of energy-based models, robustness to feedback noise, and adaptability to diverse feedback types beyond demonstrations. Our code will be publicly available soon.
% \textcolor{blue}{Our code will be publicly available upon acceptance.} 
\end{abstract}

% \begin{keywords}
%     Interactive Imitation Learning, Corrective feedback, Contrastive Learning, Learning from Demonstration, Energy-based Models
% \end{keywords}

\begin{IEEEkeywords}
Interactive Imitation Learning, Corrective feedback, Contrastive Learning, Learning from Demonstration, Energy-based Models
\end{IEEEkeywords}




\begin{figure}[ht]
    \centering
    \includegraphics[width=0.8\linewidth]{graphs/greater_than_naive.pdf}
    \vspace{0.5cm}
    \includegraphics[width=0.8\linewidth]{graphs/p1_bottom.png}
    \vspace{-5pt}
    \caption{\textcolor{positional}{Positional} vs.\ \textcolor{nonpositional}{non-positional} circuits. In a \textcolor{nonpositional}{non-positional} circuit, the same edges must be included at all positions. A \textcolor{positional}{positional} circuit can distinguish between the same edge at different positions. This specificity yields better trade-offs between circuit size and faithfulness. It can also increase both precision and recall.}
    \label{fig:p1}
    \vspace{-5pt}
\end{figure}

\section{Introduction}

\looseness=-1
A primary goal of interpretability research is to characterize the internal mechanisms in language models (LMs) and other NLP models. 
A core approach in this area is \textbf{circuit discovery}---identifying the minimal subgraph within the model's computation graph that performs a specific task \citep{olah2021framework,olah-mech}.
Typically, the nodes of a circuit represent model components (e.g., attention heads, neurons, or layers).
While manual circuit discovery methods can yield position-specific insights \citep{wanginterpretability,goldowskydill2023localizingmodelbehaviorpath}, \emph{automatic methods often overlook positional information}, treating components as uniformly relevant across all input token positions \citep{conmytowards,syed2023attribution}. 
For instance, if an attention head is included in a circuit, it is assumed to contribute equally to the computation for every position in the input sequence.
The assumption that circuits are position-invariant ignores the fact that different positions often require distinct computations.
By ignoring positions, current methods limit their ability to capture mechanisms that operate across positions, such as interactions between attention heads across positions.

In this study, we start by demonstrating that positional agnosticism is a significant limitation (\S\ref{sec:motivating}). Then, to address these limitations, we introduce a new approach: position-aware edge attribution patching (PEAP; \S\ref{sec:full_circ_discovery}; Figure~\ref{fig:p1}). Current approaches  assume that if an edge is in a circuit, then the same edge will be in the circuit at all positions, thus leading to low precision. It is also assumed that an edge's importance should be aggregated across positions before deciding whether it should be included in the circuit; this can lead to cancellation effects, and thus low recall. PEAP instead allows us to compute the importance of cross-positional edges, and separately evaluates edge importance at each position. We show that this leads to smaller and more accurate circuits; see Figure~\ref{fig:p1}.

Incorporating positional information into circuit discovery is straightforward when inputs have the same length and structure across examples.

However, realistic datasets are not nearly this templatic.
How, then, can we incorporate positional information into automatic circuit discovery?
To address this challenge, we propose \textbf{schemas} (\S\ref{sec:schema}). 
Schemas assign semantic labels to spans of tokens, enabling information aggregation across examples even when the spans differ in length.

For example, in the input ``The \textcolor{positional}{war} lasted from 1453 to 14\underline{\hspace{1em}},'' the span ``\textcolor{positional}{war}'' could be labeled as ``\emph{Subject}''.
This enables handling spans with varying lengths: the phrase ``\textcolor{positional}{Black Plague}'' in another example can be treated as a single positional span with the same role as ``\textcolor{positional}{war}''.
In experiments with two LMs and three tasks, we find that circuits discovered using schemas achieve a better trade-off between circuit size and faithfulness to the model's behavior than position-agnostic circuits.
Importantly, position-aware circuits offer a more precise representation of the underlying mechanisms, providing a more concise foundation for mechanistic explanations.

We also present a fully automated pipeline for schema generation and application (\S\ref{sec:schema-generation}) using large language models (LLMs). 
We evaluate the quality of the generated schemas and their utility in discovering position-aware circuits (\S\ref{sec:schema-eval}).
Notably, circuits derived using automatically generated and applied schemas achieve comparable faithfulness scores to circuits discovered with human-designed and manually applied schemas.

We summarize our contributions as follows:
\begin{itemize}[noitemsep,leftmargin=*,topsep=1pt,parsep=1pt]
    \item Introduce a position-aware circuit discovery method, which obtains better faithfulness than position-agnostic discovery.  
    \item Introduce dataset schemas,  facilitating positional circuit discovery in more naturalistic settings. 
    \item Develop an automated schema generation and application pipeline with LLMs, yielding schemas that are comparable to manually-annotated ones.
\end{itemize}

\section{Related Work}
\label{sec:related_work}
In this section, we provide an overview of the related work on policy learning from various feedback types, such as demonstration, relative correction, and preference.
We also summarize methods for training policies represented by EBMs.
%  The energy-based model can be trained in many different approaches, depending on the assumption of the type of data (demonstration, correction, reward, etc).
% Besides, other types of models, such as diffusion, have been applied to policy representation, which outperforms its EBM counterpart (IBC). 
% In this section, we briefly overview the related works and the IBC's training instability issue. At the end, we draw connection between IBC and our method, which utlize the corrective feedback for EBM traning. 

\subsection{Learning from Demonstration}
%1 introduce big picture of learning from Demonstration, deep generative model, diffusion, EBM
Learning from demonstration aims to teach robot behavior models from demonstration data, which provides the robot with examples of desired actions. 
Traditional methods often struggled with capturing complex data distributions, especially when multiple optimal actions exist for a given state, such as the task of pushing a T-shape object to a goal configuration \cite{2023_diffusionpolicy, 2024_RSS_pushT_Traj_optimization}.
Deep generative models, including EBMs \cite{2019_EBM_Du_Yilun, 2021_how_to_train_EBM},  diffusion models \cite{2015_diffusion, 2020_diffusion}, have been introduced to better capture such multi-modal data distributions \cite{2024_survey_deep_generative_model_in_robotics}. EBMs learn unnormalized energy values for inputs and have been applied to learn the energy of the entire state-action space via IBC \cite{2022_implicit_BC}.  
Diffusion models, which learn to denoise noise-corrupted data \cite{2015_diffusion, 2020_diffusion}, have also been utilized to represent robot policies, resulting in diffusion policies \cite{2023_diffusionpolicy, 2023_score_diffusion_policy}. These policies effectively learn the gradient (score) of the EBM \cite{2020_Score_based_diffusion} and offer improved training stability \cite{2023_diffusionpolicy}. Implicit models, such as EBMs and diffusion policies, have demonstrated superior capability in handling long-horizon, multi-modal tasks compared to explicit policies and have been successfully applied to various real-world applications \cite{2024_EBM_planning_air_hockey_application, 2024_IBC_RL_planning, 2024_survey_deep_generative_model_in_robotics}.
These implicit models have been extended into an online interactive imitation learning (IIL) framework 
\cite{2023_IIFL_implicit_interactive_BC, 2024_Diffusion_dagger}.  
However, as mentioned in the introduction section, the powerful encoding capability of these models can also cause overfitting behavior with behavior cloning loss, especially when the demonstration data deviates from the optimal action. 
To address this issue while still leveraging their encoding capability, 
we propose a new perspective on iteratively estimating optimal actions in the IIL framework. 
Instead of relying on behavior cloning loss, we introduce a novel loss function that updates the policy to align with desired action spaces. 


% \cite{2024_IBC_RL_planning}

\subsection{Learning from Preference Feedback}
Preference-based feedback involves the comparison of different robot trajectory segments.
% , which can be treated as a relative version of the evaluative feedback.
From this,
a reward/objective model is usually estimated and employed to obtain a policy \cite{2017_DRL_preference, 2015_IJRR_Preference_traj_ranking, 2021_Pebble_preference_RL, 2020_openai_RLHF, 2024_comparative_language}.
% The sub-optimal demonstration data can also be transformed into preference data to learn a high-quality reward model \cite{2019_IRL_ranked_demonstration}.
Some approaches enable directly learning a policy by the contrastive learning approach proposed by \cite{2023_Contrastive_Prefence_Learning, zhao2022calibrating} or direct preference optimization approach \cite{2023_DPO} to improve efficiency.
Moreover, sub-optimal demonstrations can be transformed into preference data to learn a high-quality reward model  \cite{2019_IRL_ranked_demonstration}. 
% Although this feedback modality is promising, it is not very data-efficient, requiring more data and time from the teacher to learn a good policy.
Although this feedback modality is promising, it is not very data-efficient. This is because the feedback is given over complete trajectories, requiring the learner to infer per-state behavior and requiring more data and teacher effort.
While there are works to make it data efficient via active learning \cite{2024batch_acitve_learning_Preference}  or utilizing prior knowledge of state \cite{2024hindsight_preference}, our paper focuses on human feedback in the state-action space.

\subsection{Learning from Relative Corrective Feedback}
% Relative corrective feedback provides incremental information on how to improve an action, balancing information richness and simplicity for the teacher \cite{2022_IIL_survey}. 
% % The data consists of pre-correction and post-correction actions.
% This correction feedback can be transferred into preference data with trajectory pairs and
% the objective function can be learned from the preference data, as in \cite{2015_IJRR_Preference_traj_ranking, 2017_pHRI_correction_learning_objective, 2018_uncertainty_correction_objective}.
% Alternatively, \cite{2022_TRO_correction_objective_function} proposed directly inferring the objective function without preference transformation.
% However, these objective functions are linear combinations of features, which may struggle with complex tasks.
% Another line of work is the COACH-based framework (Corrective Advice Communicated by Humans), which directly learns a policy from relative corrections
% \cite{2018_D_COACH, 2019_Carlos_COACH, 2021_BDCOACH}.
% This framework has been extended to 
%  utilize the feedback from the state space instead of the action space 
% \cite{2020_COACH_state_Space} and combined with reinforcement learning to increase the RL efficiency\cite{2019_Carlos_IJRR}.
% However, as we mentioned in Section~\ref{sec:introduction}, the previous COACH-based framework fails to utilize the history data, making it inefficient compared with demonstration-based methods. 
% Instead, our CLIC method can utilize the history data as it will not harm the policy under our assumption of desired action space. 


Relative corrections provide incremental information on how to improve an action, balancing information richness and simplicity for the teacher \cite{2022_IIL_survey}. 
% The data consists of pre-correction and post-correction actions.
This correction feedback can be transferred into preference data with trajectory pairs and
objective functions can be learned from the preference data, as in \cite{2015_IJRR_Preference_traj_ranking, 2017_pHRI_correction_learning_objective, 2018_uncertainty_correction_objective}.
Alternatively, \cite{2022_TRO_correction_objective_function} proposed directly inferring objective functions without preference transformation.
However, these objective functions are linear combinations of features, which may struggle with complex tasks.

Another line of work is the COACH-based framework (Corrective Advice Communicated by Humans), which directly learns a policy from relative corrections
\cite{2018_D_COACH, 2019_Carlos_COACH, 2021_BDCOACH}.
This framework has been extended to 
 utilize the feedback from the state space instead of the action space 
\cite{2020_COACH_state_Space} and combined with reinforcement learning to increase the RL efficiency\cite{2019_Carlos_IJRR}.
However, COACH-based methods rely on the over-optimistic assumption that the action labels derived from relative corrections are optimal, allowing the policy to be refined by imitating them via the BC loss \cite{2019_Rodrigo_D_COACH, 2019_Carlos_IJRR, 2019_Carlos_COACH}. 
This assumption becomes a critical limitation when feedback is aggregated into a reply buffer.  
As the robot's policy continuously improves, previous feedback may no longer be valid, causing incorrect policy updates \cite{2021_BDCOACH}. 
As a result, the buffer size is limited to being small, ensuring it contains only recent corrections. This leads to policies that tend to overfit data collected from recently visited trajectories, making it inefficient compared with demonstration-based methods. 
In contrast, our CLIC method can utilize the history data, as the desired action spaces created from it will not mislead the policy. 

The COACH-based framework utilizes explicit policies\cite{2019_Carlos_IJRR, 2019_Rodrigo_D_COACH}, limiting its ability to handle multi-modal tasks. 
Implicit policies, encoded by
EBMs, can also be learned using methods like Proxy Value Propagation (PVP)\cite{2023_NIPS_PVP}.
PVP uses a loss function that only considers the energy values of recorded robot and human actions. As a result, the loss provides limited information and fails to train an EBM effectively.
In contrast, our approach generates action samples from the EBM and classifies them into desired and undesired actions based on the desired action space. These classified samples are then used to compute the loss, which can effectively train EBMs.

% using desired action spaces to clarify action samples from the EBM into desired and undesired, which are then used to calculate the loss. 
% proposing that each correction data defines a desired and undesired action set, which are used to train an EBM and enable a more efficient learning process compared to PVP.
% Our method then trains an EBM by pushing down the energy within the desired action set and pushing up the energy within the undesired action set, which is more effective than the PVP method.  





% 2 transition, talk about EBM's advantage and applications
\subsection{Learning Policies Represented as Energy-Based Models}
 % The energy-based model can be trained in many different approaches, depending on the assumption of the type of data (demonstration, correction, reward, etc).
EBMs have been widely used across different types of feedback data.
In reinforcement learning, where data is typically scalar rewards, the energy function is used to encode the action-value function Q, with the relationship $ Q_{soft}(\bm s, \bm a) = - \alpha E(\bm s, \bm a) $ \cite{2017_soft_Q_learning, 2018_SAC, 2024_RAL_imperfect_demon}. 
Reward-conditioned policies can also be learned through Bayesian approaches \cite{2023_Bayesian_reprameterized_RCRL}. For preference feedback, EBMs can be aligned with human preferences via inverse preference learning \cite{2023_Inverse_Preference_learning}. 
In scenarios involving corrective feedback, where both the robot and the human actions are given, methods such as PVP have been proposed to learn EBMs that assign low energy values to human actions and high energy values to robot actions \cite{2023_NIPS_PVP}.
For demonstration or absolute correction data, EBMs can be trained directly using the objective that demonstrated actions have lower energy than other actions  \cite{2020_RSS_expert_interventio_learning, 2022_implicit_BC}.
In discrete action spaces, EBMs can be straightforward to train \cite{2020_RSS_expert_interventio_learning}, though the discrete nature of the action space limits the scope of the method. 
For continuous action spaces, EBMs can be trained
via the InfoNCE loss in IBC \cite{2022_implicit_BC}, through which the energy of human actions is decreased, and the energy of other actions is increased. 

Although IBC achieves better performance than explicit policies \cite{2022_implicit_BC}, it is known to suffer from training instability. The process of training EBMs involves selecting counterexample actions, and the quality of these counterexamples significantly impacts the learning outcomes \cite{2021_how_to_train_EBM,2020_flow_constrastive_estimation_EBM}. Empirically, counterexamples that are near data labels are often preferred \cite{2020_hard_negative_mixing_contrastive}, but these selections may contribute to instability during training \cite{2022_arxiv_IBC_gaps, 2023_diffusionpolicy}. Our method addresses this issue by relaxing the BC assumption and using the proposed desired action space to train EBMs, leading to a stable training process.



\section{Preliminaries}
\label{sec:Preliminaries}

\subsection{Interactive Imitation Learning}
\label{sec:Preliminaries:IIL}

In a typical Interactive IL (IIL) problem, a Markov Decision Process (MDP) is used to model the decision-making of an agent taught by a human teacher. 
An MDP is a 5-tuple \((\mathcal  S, \mathcal A, T, R, \gamma)\), where \(\mathcal  S\) represents the set of all possible states in the environment, \(\mathcal  A\) denotes the set of actions the agent can take, \(T(\bm s' |\bm s, \bm a)\) is the transition probability, \(R(\bm s, \bm s', \bm a)\) is the reward function, giving the reward received for transitioning from state \(\bm s\) to state \(\bm s'\) via action \(\bm a\), and \(\gamma \in [0, 1]\) is the discount factor.
The reward typically reflects the desirability of the state transition.
A policy in an MDP defines the agent's behavior at a given state, denoted by \(\pi\).
In general, $\pi$ can be represented by the conditional probability $\pi(\bm a|\bm s)$ of the density function \(\pi: \mathcal S \times \mathcal A \rightarrow [0, 1]\). Consequently, given a state, $\pi$ is employed to select an action. The objective in an MDP is to find the optimal policy \(\pi^*\) that maximizes the expected sum of discounted future rewards $J(\pi) = \mathbb{E} \left[ \sum_{t=0}^\infty \gamma^{t} R(\bm s_t, \bm a_t) \right]$. 
% In this work, we model $\pi$ using a Deep Neural Network (DNN). To achieve this, we consider a Gaussian policy with fixed covariance $\bm \Sigma$ and with mean $\bm \mu_{\theta}(\bm s)$, represented by the DNN. Hence, we obtain the parameterized policy $\pi_{\bm \theta}(\bm a | \bm s) \sim \mathcal{N}\left (\bm \mu_{\theta}(\bm s), \bm \Sigma\right)$, where $\bm \theta$ denotes the DNN's parameter vector.

In IIL, a human instructor, known as the \emph{teacher}, aims to improve the behavior of the learning agent, referred to as the \emph{learner}, by providing feedback on the learner's actions. 
IIL does not rely on a predefined reward function for specific tasks, thanks to the direct guidance provided by the teacher \cite{2022_IIL_survey}.
In this paper, we consider the teacher feedback to act in the state-action space, which is described by the function \( \bm h = {H}(\bm s, \bm a)\).
%%% Explain the teacher's feedback H
The feedback $\bm h$ can be defined according to the feedback type.
For instance, in demonstration/intervention feedback, $\bm h$ represents the action the learner should execute at a given state. In contrast, for relative corrective feedback, 
% $\bm h$ is a directional signal guiding the learner's action towards a more favorable one.
 $\bm h$ is a normalized vector indicating the direction in which
the learner's action should be modified, i.e., 
${\bm h \in \mathcal{H} = \{ \bm d \in \mathcal{A} \mid  ||\bm d|| = 1\}}$.
% ${\bm h \in \mathcal{H} = \{ \bm d = \bm a_2 - \bm a_1, \bm a_1\in \mathcal{A}, \bm a_2 \in \mathcal{A} \mid  ||\bm d|| = 1\}}$.
In our context, the reward function is unknown; therefore, we cannot directly maximize the expected accumulated future rewards $J(\pi)$.
% Instead,
% an observable surrogate loss $l_{\pi}( \bm s)$ can be formulated as $l_{\pi}( \bm s) = \mathbb{E}_{\bm a \sim \pi(\bm s)} \left[ l_{\pi}(\bm s, \bm a, \mathcal{H}(\bm s, \bm a)) \right]$ 
% and the minimization of 
% $l_{\pi}( \bm s)$ 
% Instead, 
% an observable surrogate loss $l_{\pi}( \bm s)$ can be defined to indicate how well the learner's policy $\pi$ follows the teacher's feedback $\bm h$
% and the minimization of   $l_{\pi}( \bm s)$ 
% indirectly minimizes $J(\pi)$ \cite{2011_DAgger}.
Instead, an observable surrogate loss, \(\ell_{\pi}(\bm{s})\), is formulated. This loss measures the alignment of the learner's policy \(\pi\) with the teacher's feedback. In IIL, it is assumed that minimizing \(\ell_{\pi}(\bm{s})\) indirectly maximizes \(J(\pi)\) \cite{2011_DAgger,2022_IIL_survey}.
% IIL aims to find a learner's policy $\pi^l$ by solving 
As a result, the objective is to determine an optimal learner's policy $\pi^{l*}$ by solving the following equation:
\begin{equation}
    \pi^{l*} = \underset{\pi\in\Pi}{\arg\min} \mathbb{E}_{\bm s\sim d_{\pi}(\bm s)} \left[ \ell_{\pi}( \bm s) \right]
    \label{eq:IIL_formulation}
\end{equation}
%%% explain the data distribution?
where $d_{\pi}(\bm s)$ is the state distribution induced by the policy $\pi$.
In practice, the expected value of the surrogate loss in Eq.~\eqref{eq:IIL_formulation} is approximated using the data collected by a policy that interacts with the environment and the teacher.



\subsection{Implicit Behavior Cloning}

% Implicit BC \cite{2022_implicit_BC} reformulates the traditional behavioral cloning approach by using implicit models, specifically energy-based models (EBM), to represent policies. Instead of directly mapping observations to actions with an explicit function, 
Implicit BC (IBC) \cite{2022_implicit_BC} defines the policy through an energy function \( E_\theta(s, a) \) that takes state \( s \) and action \( a \) as inputs and outputs a scalar energy value. 
This formulation allows IBC to handle complex, multi-modal, and discontinuous behaviors more effectively than explicit models.
The energy function is trained using maximum likelihood estimation by minimizing the InfoNCE loss \cite{2018_InfoNCE_representation_learning, 2024_revisting_IBC}:
\begin{align}
\label{eq:ibc_info_NCE}
\ell_{\text{InfoNCE}}(\bm s, \bm a, \mathbb{A}^{neg})
\!\! = \!\! - \! \log \! \left[ \! \frac{e^{-E_\theta(\bm s, \bm a)}}{e^{-\!E_\theta(\bm s, \bm a)} \!\! + \!\!\sum_{j=1}^{N_{\text{neg}}} e^{-\!E_\theta(\bm s, \tilde{\bm a}_j)}} \!\right] \!\!,\!\!
\end{align}
where the action label \(\bm a\) is the teacher action,  \(\tilde{\bm a}_j \in \mathbb{A}^{neg} (j = 1, \dots, N_{\text{neg}}) \) are negative samples, and $ \mathbb{A}^{neg}$ is the set that includes negative samples. 
The InfoNCE loss encourages the model to assign low energy to action labels and high energy to negative samples. 
To ensure the EBM learns an accurate data distribution, the negative samples should be close to the action label, avoiding overly obvious distinctions that hinder effective learning \cite{2020_flow_constrastive_estimation_EBM}. 
% This can be achieved by obtaining
%  negative samples by sampling from the current EBM \cite{2019_EBM_Du_Yilun}, which can be achieved via MCMC (Markov Chain Monte Carlo) sampling with stochastic gradient Langevin dynamics \cite{2011_Langevin_dynamics}:
This can be achieved by generating negative samples from the current EBM using MCMC sampling with stochastic gradient Langevin dynamics \cite{2011_Langevin_dynamics, 2019_EBM_Du_Yilun}:
\begin{equation}
\tilde{\bm a}_j^i = \tilde{\bm a}_j^{i-1} - \lambda \nabla_{\bm a} E_\theta(\bm s, \tilde{\bm a}_j^{i-1}) + \sqrt{2\lambda}   \omega^i,    
\label{eq:mcmc_sampling}
\end{equation}
where $\{ \tilde{\bm a}_j^0 \}$ is initialized using the uniform distribution and $\omega^i$ is the standard normal distribution.
For each $\tilde{\bm a}_j^0 $, we run $N_{\text{MCMC}}$ steps of the MCMC chain, with $i = 0, \dots, N_{\text{MCMC}}$ denoting the step index. The step size $\lambda > 0$ can be adjusted using a polynomially decaying schedule.

% During inference, the optimal action \(\hat{\bm a}^*\) is estimated as the action with the lowest energy value and is determined by minimizing the energy function among samples from  Langevin MCMC:
During inference, the estimated optimal action \(\hat{\bm a}^*\) is obtained by minimizing the energy function, and can be approximated through Langevin MCMC:
\[
\hat{\bm a}^* = \underset{\bm a}{ \arg\min} E_\theta(\bm s, \bm a).
\]


One core assumption of IBC is that the action label is optimal and all other actions are not \cite{2022_implicit_BC}.
This assumption simplifies the classification of action samples into positive and negative categories. 
Specifically, 
the action label is considered positive, and all other sampled actions are considered negative.
However, actions considered as negative may still be valid and should not be overly penalized.
This makes selecting appropriate negative samples challenging and introduces instability during the IBC's training process.
In contrast, our CLIC method addresses this issue by labeling sampled actions within the desired action space as positive and those outside it as negative, ensuring a more stable and effective training process.
 % limiting Implicit BC's ability to learn a consistent action representation across different trails.



\subsection{Problem Formulation}

% Breifly introduce how our policy is modeled, how action is sampled (using IBC's method), how the feedback looks like

% The objective is to estimate the optimal action $\bm a^*$ for every state $\bm s \sim d_\pi(\bm s)$ via an occasional feedback signal $\bm h$.
The objective is to learn a policy $\pi$ that accurately estimates the optimal action $\bm a^*$ for every state $\bm s \sim d_\pi(\bm s)$, using occasional corrective feedback $\bm h$.
This feedback is provided by a human teacher in the robot's state-action space, placing the problem in the context of IIL.
% in the action space provided by the human teacher in the IIL framework. 
The feedback $\bm h$ can be either absolute or relative corrections, as defined in Section~\ref{sec:Preliminaries:IIL}.
% Although $\bm h$ has different meanings for absolute or relative corrections, 
% the pair of actions can be defined, which are the robot action $\bm a^{r}$ and human action $\bm a^{h}$.  
Accordingly, the \textit{observed action pair} $(\bm a^r, \bm a^h)$ can be defined, where $\bm a^r$ denotes the robot action and $\bm a^h$ denotes the human feedback action, referred to as human action for simplicity. 
For absolute correction, we have that $\bm a^h = \bm h$. 
In contrast, for relative correction, we have that $\bm a^h = \bm a^r + e \bm h$, where the magnitude hyperparameter $e$ is set to a small value.
% compared to the difference between $\bm a^r$ and the optimal action $\bm a^*$.
% % briefly show different types of feedback
% In addition to accurate absolute and relative corrections, Table \ref{tab:feedback-definitions} summarizes other common types of feedback humans provide, which are also used to evaluate the algorithms in the experiment section.

% \begin{table}[h!]
% \caption{Various feedback in the action space}
% \centering
% % \begin{tabular}{@{}ll@{}}
% \begin{tabularx}{0.49\textwidth}{@{}lX@{}}
% \toprule
% \textbf{Type of Feedback Data}      & \textbf{Definition} \\ \midrule
% Accurate absolute correction              & \( \bm a^h = \bm h, \bm h = \bm a^* \) \\
% Gaussian noise             & \( \bm a^h \sim \mathcal{N}(\bm a^*, ||\bm a^* - \bm a^r|| I) \) \\
% Partial feedback           & \( \bm a^h \in \{[\bm a^*_{r1}, \bm a_{r2}], [\bm a_{r1}, \bm a_{r2}^*]\} \) \\ 
% \hline
% Accurate relative correction              & \(\bm  a^h = \bm a^r + e\bm h^*, \bm h^*= \frac{\bm a^* - \bm a^r}{||\bm a^* - \bm a^r||}  \) \\
% Direction noise            & \( \bm a^h = \bm a^r + e \bm h_r \), \( \angle (\bm h_r, \bm h^*) = \beta \in [0, 90^\circ) \) \\
% \bottomrule
% \end{tabularx}
% \label{tab:feedback-definitions}
% \end{table}


% briefly show what is the goal of our paper, and the corresponding general formulation
Although the optimal action may not be directly extracted from human feedback for a given state, it includes information on which subset of $\mathcal{A}$ is more likely to contain optimal actions. 
Based on this insight, assuming the Euclidean action space $\mathcal{A}$, we introduce the \textit{desired action space} $\hat{\mathcal{A}}(\bm a^r, \bm a^h)$ as the set of desired actions, defined as a function of the observed action pair $(\bm a^r, \bm a^h)$.
Formally, ${\hat{\mathcal{A}}:\mathcal{A} \times \mathcal{A} \twoheadrightarrow\mathcal{A}}$, where the symbol $\twoheadrightarrow$ defines a set-valued function, i.e., elements of $\mathcal{A} \times \mathcal{A} $ are mapped to subsets of $\mathcal{A}$.
Actions within the desired action space are more likely to be optimal than those outside it. The construction of this set is detailed in Section~\ref{sec:Desired_action_space}.  
% The desired action set should include $\bm a^h$ and exclude $\bm a^r$; it also includes other desired actions implied by $(\bm a^r, \bm a^h)$.
% Instead of assuming $\bm a^* = \bm a^h$, we assume that the desired action space includes at least one optimal action. 
% And there exists a function $ I(\bm a, \bm a^r, \bm a^h) = \mathbf{1}_{\bm a \in \bar{\mathcal{A}}(\bm a^r, \bm a^h)}$ that determines whether an action belongs to this set.
% Then we can have
% \[
% \bar{\mathcal{A}}(\bm a^r, \bm a^h) = \{ \bm a \in \mathcal{A} \mid I(\bm a, \bm a^r, \bm a^h) = 1 \}.
% \]

% To learn a policy from the set $\bar{\mathcal{A}}(\bm a^r, \bm a^h)$, there are three important aspects tackled in this paper:
% (1) how to construct this set from feedback,
% (2) how to design a loss function to encourage policy to generate actions within the desired action set,  
% and (3) how to guarantee that the policy can generate optimal actions with sufficient feedback data.
% For question (1) and (3), we tackle it in in Section~\ref{sec:Desired_action_space}, which detail
% the definition and property of $\mathcal{A}(\bm a^r, \bm a^h)$.
% For question (2), we tackle it in Section~\ref{sec:Policy_shaping}, where we first give a probablisitc formulation the desired action set, then design the loss function to train the policy. 

% The desired action space can shape the policy to generate actions within this space, which we will introduce in the Section~\ref{sec:Policy_shaping}.
% We first introduce the policy model 
In this work, we model the policy $\pi$ using a Deep Neural Network (DNN). 
To achieve this, following IBC \cite{2022_implicit_BC}, for multi-modal tasks with multiple optimal actions, our policy is defined by the energy function 
\[\pi_{\theta}(\bm a | \bm s) = \frac{\exp(-E_{\theta}(\bm s, \bm a))}{Z},
\]
where $Z$ is a normalizing constant and can be approximated by $Z = \sum_{j=1}^{N_{\text{MCMC}}} \exp(-E_{\theta}(\bm s, \bm a'_j))  $.
The samples $\{\bm a'_j\}$ are obtained via Langevin MCMC sampling (see Eq.~\eqref{eq:mcmc_sampling}), and $\bm \theta$ denotes the DNN's parameter vector.

For simpler tasks with a single optimal action for every state, we can consider an explicit Gaussian policy with fixed covariance $\bm \Sigma$ and with mean $\bm \mu_{\theta}(\bm s)$, which we model using a DNN. Hence, we obtain the parameterized policy $\pi_{\bm \theta}(\bm a | \bm s) \sim \mathcal{N}\left (\bm \mu_{\theta}(\bm s), \bm \Sigma\right)$.
Both explicit and implicit policies can be estimated using our method based on the desired action space, and we will introduce them in Section~\ref{sec:Policy_shaping}.

% \textcolor{red}{Put this framework at the end of intro}

% This paper addresses three key questions for learning a policy from $\hat{\mathcal{A}}(\bm a^r, \bm a^h)$:
% \begin{enumerate}
%     \item \textbf{Constructing the set}: How to derive $\hat{\mathcal{A}}(\bm a^r, \bm a^h)$ from feedback.
%     \item \textbf{Designing the loss}: How to design a loss function that encourages the policy to generate actions within $\hat{\mathcal{A}}(\bm a^r, \bm a^h)$.
%     \item \textbf{Guaranteeing optimality}: How to ensure the policy generates optimal actions with sufficient feedback.
% \end{enumerate}

% Questions (1) and (3) are addressed in Section~\ref{sec:Desired_action_space}, detailing the definition and properties of $\hat{\mathcal{A}}(\bm a^r, \bm a^h)$. Question (2) is covered in Section~\ref{sec:Policy_shaping}, which introduces a probabilistic formulation of the desired action set and a corresponding loss function for policy training.




% We denote the data buffer of corrective data as $\mathcal{D}_{t-1} = \{[\bm s_i, \bm a_i, e \bm h_i], i = 1, \dots, k \}$ that contains all the received corrective feedback, where $k$ is the total number of the data tuples. 
% Our policy is defined by the energy function $\pi_{\theta}(\bm a | \bm s) = \frac{\exp(-E_{\theta}(\bm s, \bm a))}{Z}$, where $Z$ is the normalizing constant and is approximated by $Z = \sum_{j=1}^{N} \exp(-E_{\theta}(\bm s, \bm a'_j))  $, where the samples $\bm a'_j$ are obtained via Langevin sampling in Eq.~\eqref{eq:mcmc_sampling}.






\section{Methodology}

After annotating the dataset, we split the dataset into training, validation, and test sets. We used 184,719 images ($\sim$80\%) to train our object detection models and 23,090 images ($\sim$10\%) to validate the model during training time. The rest of the 23,090 images ($\sim$10\%) are held out to test the trained model’s performance. In this study, we employed two advanced deep-learning models for weed detection and classification: RetinaNet with a ResNeXt-101 backbone and Detection Transformer (DETR) with a ResNet-50 backbone. These models were tasked with classifying weed species and their respective growth stages (in weeks), while simultaneously localizing them within the images via bounding box predictions. We configured and trained these models using PyTorch and mmDetection on an NVIDIA RTX 3090 GPU.

\subsection{Detection Transformer with ResNet-50}

The Detection Transformer (DETR) model is an end-to-end object detection architecture that combines a convolutional backbone with a transformer encoder-decoder \cite{carion2020end}. This approach effectively addresses the complexities of identifying weeds in agricultural images. The backbone of our model ResNet-50 is a convolutional neural network, pre-trained on ImageNet (\texttt{open-mmlab://resnet50}). This 50-layer network, organized into four stages, serves as a powerful feature extractor. We utilize the output from the final stage (out\_indices=(3,)) and freeze the initial stages during training to preserve pre-learned features. The backbone's output can be represented as:
\vspace{-0.2cm}
\begin{equation}
F_{\text{resnet}} = \text{ResNet50}(I)
\end{equation}
where \(I\) is the input image. A Channel Mapper follows the backbone, transforming ResNet-50's 2048-channel output into a 256-channel feature map suitable for the transformer. This dimensionality reduction is achieved through a 1x1 convolution:
\vspace{-0.2cm}
\begin{equation}
F_{\text{neck}} = \text{Conv1x1}(F_{\text{resnet}})
\end{equation}

The core of our DETR model is the transformer module, comprising a 6-layer encoder and decoder. Each encoder layer incorporates a self-attention mechanism with 8 heads, followed by a feed-forward network (FFN) with ReLU activation. The model's bounding box head processes the decoder's output to predict class labels and bounding boxes. We employ cross-entropy loss for classification and a combination of L1 and Generalized IoU losses for bounding box regression. The overall loss function \cite{yin2019context} is defined as:
\vspace{-0.2cm}
\begin{equation}
L = \alpha \cdot L_{\text{cls}} + \beta \cdot L_{\text{bbox}} + \gamma \cdot L_{\text{iou}}
\end{equation}
where \(\alpha\), \(\beta\), and \(\gamma\) are weight coefficients. \( L_{\text{cls}} \) represents the classification loss, which in this case is the cross-entropy loss. \( L_{\text{bbox}} \) represents the bounding box regression loss, which is a combination of L1 loss and Generalized IoU loss, and \( L_{\text{iou}} \) 	represents the IoU loss, which is specifically aimed at improving the localization accuracy by penalizing the model based on the intersection over union between the predicted and ground truth bounding boxes. During training, we utilize the Hungarian algorithm \cite{ye2020cost} for bipartite matching, ensuring a one-to-one correspondence between predicted and ground-truth boxes. This approach optimizes the model's ability to accurately locate and classify weeds within agricultural images. By integrating the robust feature extraction capabilities of ResNet-50 with the DETR architecture's powerful attention mechanisms, our model achieves good performance in weed detection with 174 classes.

\subsection{RetinaNet with ResNeXt-101}

RetinaNet is a single-stage object detection model designed to address the extreme foreground-background class imbalance encountered during training \cite{li2019light}. The architecture comprises three main components: a backbone network for feature extraction, a neck (FPN) for generating multi-scale feature maps, and a detection head for predicting bounding boxes and class probabilities. We utilized ResNeXt-101 as the backbone, a variant of the ResNet architecture that employs grouped convolutions for improved efficiency and performance. The ResNeXt-101 backbone consists of 101 layers organized into four stages, with 32 groups and a base width of 4 channels per group. We initialized the backbone with weights pretrained on ImageNet (\texttt{open-mmlab://resnext101\_32x4d}) to leverage transfer learning. Batch normalization is applied after each convolutional layer to stabilize the learning process.

The Feature Pyramid Network (FPN) enhances the backbone's feature maps by combining high-level semantic features with low-level detailed features, enabling the detection of objects at various scales. The FPN generates multiple feature maps of different resolutions, which are then fed into the detection head. The detection head of RetinaNet comprises two subnetworks: a classification subnetwork for predicting object presence probabilities and a regression subnetwork for predicting bounding box coordinates. Each subnetwork consists of four convolutional layers, followed by a final convolutional layer that produces the desired outputs. To handle class imbalance, we employed the focal loss function \cite{lin2017focal} for training the classification subnetwork:
\vspace{-0.2cm}
\begin{equation}
\text{FL}(p_t) = -\alpha_t (1 - p_t)^\gamma \log(p_t)
\end{equation}

where \(p_t\) is the predicted probability, \(\alpha_t\) is a balancing factor, and \(\gamma\) is the focusing parameter.

We trained our model using an epoch-based training loop with the AdamW optimizer (learning rate \(lr = 0.0001\), weight decay \(wd = 0.0001\)). The learning rate schedule incorporated a linear warmup over the first 1000 iterations. We trained for 12 epochs with a batch size of 16, employing automatic learning rate scaling to accommodate potential batch size changes.


% \begin{algorithm}
% \caption{Model Training Process}\label{alg:training}
% \KwData{Training dataset, validation dataset, initial model parameters}
% \KwResult{Trained model}
% Initialize model parameters $\theta$\;
% \For{epoch = 1 to max\_epochs}{
%     \For{each mini-batch $(X, y)$ in training dataset}{
%         Compute predictions $\hat{y} = f_\theta(X)$\;
%         Compute classification loss $L_{\text{cls}}$ and regression loss $L_{\text{reg}}$\;
%         Compute total loss $L = L_{\text{cls}} + L_{\text{reg}}$\;
%         Backpropagate to compute gradients $\nabla_\theta L$\;
%         Update parameters $\theta$ using AdamW optimizer\;
%     }
%     \If{epoch \% val\_interval == 0}{
%         Evaluate model on validation dataset\;
%         Save model checkpoint if performance improves\;
%     }
% }
% \end{algorithm}

% The algorithm \cite{ilyas2022datamodels} outlines a model training process where the goal is to optimize the model parameters, denoted by \(\theta\). Initially, the model parameters are set to their initial values. The training process runs for a specified number of epochs, iterating over mini-batches of the training dataset in each epoch. For each mini-batch, the model makes predictions \(\hat{y}\), and the classification and regression losses are computed. These losses are summed to obtain the total loss, which is then used to calculate the gradients through backpropagation. The model parameters are updated using the AdamW optimizer. At regular intervals, defined by \texttt{val\_interval}, the model's performance is evaluated on the validation dataset, and the model checkpoint is saved if there is an improvement in performance.


% \vspace{-0.4cm}

\subsection{Evaluation Metrics}

To assess the performance of our weed detection models, we employ a comprehensive set of metrics that capture both the accuracy and robustness of the detections. Our primary metrics are Average Precision (AP), Average Recall (AR), and Mean Average Precision (mAP) evaluated across various Intersection over Union (IoU) thresholds.

AP provides a single-value summary of the precision-recall curve, effectively balancing the trade-off between precision and recall. Precision (P) is defined as the ratio of true positive detections to the sum of true positive and false positive detections: $\text{} P = \frac{TP}{TP + FP}$ and Recall (R) is the ratio of true positive detections to the sum of true positive and false negative detections: $\text{} (R) = \frac{TP}{TP + FN}$.

In this research, a true positive is a detected bounding box that correctly identifies a weed species and has an IoU above a specified threshold (e.g., 0.50) with the ground truth bounding box. A false positive is a detection that either does not sufficiently overlap with any ground truth box or incorrectly identifies the weed species. A false negative occurs when a ground truth weed instance is not detected by the model. AP \cite{robertson2008new} is calculated by integrating the precision over the recall range and it can be defined as:
\vspace{-0.2cm}
\begin{equation}
\text{AP} = \int_0^1 P(R) \, dR
\end{equation}

AR \cite{zhu2004recall} measures the model's ability to detect all relevant objects. It is computed as the average of maximum recalls at specified IoU thresholds:
\vspace{-0.2cm}
\begin{equation}
\text{AR} = \frac{1}{N} \sum_{i=1}^N R_{\text{max}}(IoU_i)
\end{equation}

mAP is the mean of AP values across different classes and is a common metric for evaluating object detection models. It provides a balanced measure of precision and recall across various IoU thresholds. It can be defined as:
\vspace{-0.2cm}
\begin{equation}
\text{mAP} = \frac{1}{C} \sum_{c=1}^C AP_c
\end{equation}

where $AP_c$ is the Average Precision for class $c$, and $C$ is the total number of classes.
We evaluate these metrics at various IoU thresholds. This multi-faceted evaluation approach allows us to comprehensively analyze our models' capabilities in detecting and classifying weeds across various scenarios, providing insights into their precision, recall, and overall detection performance.

% Tips: for each results section, when you write each paragraph, try to follow this:
% (1) motivation, why are you doing this?
% (2) preset result (pure data)
% (3) give interpretation


\begin{figure}[t]
	\centering
	\includegraphics[width=0.49\textwidth]{figs/Fig5_exp_platforms.pdf}
    % \includesvg[width=0.49\textwidth, inkscapelatex=false]{figs/Fig5_exp_platforms.svg} 
	\caption{Tasks for the simulation experiments. Each task is tested with various feedback types, including accurate demonstrations, noisy demonstrations, and relative corrective feedback. For the TwoArm-Lift task, partial feedback is also tested by applying feedback only to one of the robots.}
	\label{fig:tasks}
\end{figure}



% \begin{table*}
% \footnotesize
% % \setlength{\tabcolsep}{10pt}
% \caption{Experimental results in simulation under various types of feedback data. SR indicates the success rate, and CT represents the convergence timestep. A ‘$\diagdown$’ symbol denotes that the algorithm did not converge. For calculating CT, $\diagdown$ entries are replaced with the maximum allowable timestep.}
% \label{tab:data_folder_presence}
% \begin{center}
% \begin{tabular}{lcccccccccccc}
% \Xhline{0.75pt}
% Method & \multicolumn{2}{c}{CLIC-Half } & \multicolumn{2}{c}{Diffusion} & \multicolumn{2}{c}{Implicit BC} & \multicolumn{2}{c}{PVP} & \multicolumn{2}{c}{CLIC-Simplified} & \multicolumn{2}{c}{HG-DAgger/D-COACH } \\
% \hline
%  \textbf{Accurate data}& SR & CT & SR & CT & SR & CT & SR & CT &SR & CT & SR & CT \vspace{2.5pt}\\
% PushT & \textbf{0.931} & 25.6 & 0.915 & 24.1 & 0.890 & 31.8 & 0.440 & 28.5 &  0.765 & 35.6 & 0.710 & 38.6  \\
% Square & 0.930 & 43.0 & \textbf{0.953} & 48.4 & 0.732 & 54.6 &  0.000 & $\diagdown$   &0.634 & 65.9 & 0.420 & 67.0 \\
% PickCan & 0.983 & 37.8 & 0.963 & 36.1 & 0.688 & 44.2 &  0.000 & $\diagdown$    & 0.995 & 42.5 & 0.990 & 35.7 \\
% TwoArmLift & 0.970 & 34.9 & 0.990 & 23.0 & 0.000 & 2.1 &  0.000 & $\diagdown$    & 0.902 & 14.0 & 0.982 & 14.9  \\
% \hline
% \textbf{Gaussian noise} & SR & CT & SR & CT & SR & CT & SR & CT &
% SR & CT & SR & CT\vspace{2.5pt}\\ 
% PushT & 0.880 & 35.6 & \textbf{0.893} & 49.0 & 0.735 & 42.1 & 0.155 & 45.8 & 0.663 & 41.4 & 0.598 & 41.0 \\
% Square & \textbf{0.925} & 64.5 & 0.000 & 2.1 & 0.000 & 2.1 &   0.000 & $\diagdown$    & 0.238 & 71.0 & 0.060 & 77.2 \\
% PickCan & \textbf{0.973} & 37.8 & 0.467 & 68.2 & 0.070 & 70.4 &   0.000 & $\diagdown$    & 0.800 & 69.5 & 0.028 & 23.1 \\
% TwoArmLift & \textbf{0.847} & 68.0 & 0.000 & 2.1 &   0.000 & $\diagdown$    &  0.000 & $\diagdown$    & 0.433 & 39.3 & 0.008 & 63.8 \\
% \hline
% \textbf{Partial feedback} & SR & CT & SR & CT & SR & CT & SR & CT &SR & CT & SR & CT\vspace{2.5pt}\\
% TwoArmLift & \textbf{0.990} & 26.9 & 0.897 & 29.7 & 0.000 & $\diagdown$  & 0.000 & $\diagdown$ & 0.780 & 19.1 & 0.687 & 25.7 \\
% \hline
% \textbf{Relative correction}  & SR & CT & SR & CT & SR & CT & SR & CT & SR & CT & SR & CT \vspace{2.5pt}\\
% PushT & \textbf{0.853} & 40.8 & 0.060 & 72.0 & 0.400 & 58.8 &   0.110	& 50.4   & 0.733 & 43.5 & 0.520 & 49.0 \\
% Square & \textbf{0.817} & 69.0 & 0.000 & 2.1 & 0.005 & 56.3 &   0.000 & $\diagdown$    & 0.065 & 66.1 & 0.243 & 79.7 \\
% PickCan & 0.870 & 40.8 & 0.000 & 2.0 & 0.310 & 81.7 &   0.000 & $\diagdown$    & \textbf{0.890} & 67.2 & 0.693 & 62.8 \\
% TwoArmLift & \textbf{0.860} & 31.1    &  0.000 & $\diagdown$   & 0.000  & $\diagdown$  &   0.000 & $\diagdown$   & 0.613 & 18.9 & 0.115 & 64.7 \\
% \hline
% \textbf{Direction noise}  & SR & CT & SR & CT & SR & CT & SR & CT & SR & CT & SR & CT \vspace{2.5pt}\\
% PushT & 0.700 & 48.1 & 0.187 & 67.9 & 0.574 & 55.5 & NaN & NaN & NaN & NaN & NaN & NaN \\
% Square & 0.870 & 63.6 & 0.125 & 75.8 & 0.230 & 70.4 & NaN & NaN & NaN & NaN & NaN & NaN \\
% PickCan & 0.850 & 33.6 & NaN & NaN & 0.482 & 81.4 & NaN & NaN & NaN & NaN & NaN & NaN \\
% TwoArmLift & 0.907 & 20.8 & 0.885 & 46.6 & NaN & NaN & NaN & NaN & NaN & NaN & NaN & NaN \\
% \hline
% Average & \textbf{0.928} & \textbf{42.1} &  0.675 & 46.3 & 0.346  & 48.9  & 0.066 & 62.7 & & & & \\
% \Xhline{0.75pt}
% \end{tabular}
% \end{center}
% \end{table*}


% \begin{table*}
% \footnotesize
% % \setlength{\tabcolsep}{10pt}
% \caption{Experimental results in simulation under various types of feedback data. SR indicates the success rate, and CT represents the convergence timestep. A ‘$\diagdown$’ symbol denotes that the algorithm did not converge. For calculating CT, $\diagdown$ entries are replaced with the maximum allowable timestep.}
% \label{tab:data_folder_presence}
% \begin{center}
% \begin{tabular}{lcccccccc|cccc}
% \Xhline{0.75pt}
% Method & \multicolumn{2}{c}{CLIC-Half } & \multicolumn{2}{c}{Diffusion} & \multicolumn{2}{c}{Implicit BC} & \multicolumn{2}{c}{PVP} & \multicolumn{2}{c}{CLIC-Simplified} & \multicolumn{2}{c}{HG-DAgger/D-COACH } \\
%  & SR & CT & SR & CT & SR & CT & SR & CT &SR & CT & SR & CT \\
%  \hline  \textbf{Accurate data}  && & & & & & & & && & \\
% PushT & \textbf{0.931} & 25.6 & 0.915 & 24.1 & 0.890 & 31.8 & 0.440 & 28.5 &  0.765 & 35.6 & 0.710 & 38.6  \\
% Square & 0.930 & 43.0 & \textbf{0.953} & 48.4 & 0.732 & 54.6 &  0.000 & $\diagdown$   &0.634 & 65.9 & 0.420 & 67.0 \\
% PickCan & 0.983 & 37.8 & 0.963 & 36.1 & 0.688 & 44.2 &  0.000 & $\diagdown$    & 0.995 & 42.5 & 0.990 & 35.7 \\
% TwoArmLift & 0.970 & 34.9 & 0.990 & 23.0 & 0.000 & 2.1 &  0.000 & $\diagdown$    & 0.902 & 14.0 & 0.982 & 14.9  \\
% \hline
% \textbf{Gaussian noise } && & & & & & & & && &\\ 
% PushT & 0.880 & 35.6 & \textbf{0.893} & 49.0 & 0.735 & 42.1 & 0.155 & 45.8 & 0.663 & 41.4 & 0.598 & 41.0 \\
% Square & \textbf{0.925} & 64.5 & 0.000 & 2.1 & 0.000 & 2.1 &   0.000 & $\diagdown$    & 0.238 & 71.0 & 0.060 & 77.2 \\
% PickCan & \textbf{0.973} & 37.8 & 0.467 & 68.2 & 0.070 & 70.4 &   0.000 & $\diagdown$    & 0.800 & 69.5 & 0.028 & 23.1 \\
% TwoArmLift & \textbf{0.847} & 68.0 & 0.000 & 2.1 &   0.000 & $\diagdown$    &  0.000 & $\diagdown$    & 0.433 & 39.3 & 0.008 & 63.8 \\
% \hline
% \textbf{Partial feedback } && & & & & & & & && & \\
% TwoArmLift & \textbf{0.990} & 26.9 & 0.897 & 29.7 & 0.000 & $\diagdown$  & 0.000 & $\diagdown$ & 0.780 & 19.1 & 0.687 & 25.7 \\
% \hline
% \textbf{Relative correction} && & & & & & & & && &  \\
% PushT & \textbf{0.853} & 40.8 & 0.060 & 72.0 & 0.400 & 58.8 &   0.110	& 50.4   & 0.733 & 43.5 & 0.520 & 49.0 \\
% Square & \textbf{0.817} & 69.0 & 0.000 & 2.1 & 0.005 & 56.3 &   0.000 & $\diagdown$    & 0.065 & 66.1 & 0.243 & 79.7 \\
% PickCan & 0.870 & 40.8 & 0.000 & 2.0 & 0.310 & 81.7 &   0.000 & $\diagdown$    & \textbf{0.890} & 67.2 & 0.693 & 62.8 \\
% TwoArmLift & \textbf{0.860} & 31.1    &  0.000 & $\diagdown$   & 0.000  & $\diagdown$  &   0.000 & $\diagdown$   & 0.613 & 18.9 & 0.115 & 64.7 \\
% \hline
% \textbf{Direction noise }   && & & & & & & & && & \\
% PushT & 0.700 & 48.1 & 0.187 & 67.9 & 0.574 & 55.5 & NaN & NaN & NaN & NaN & NaN & NaN \\
% Square & 0.870 & 63.6 & 0.125 & 75.8 & 0.230 & 70.4 & NaN & NaN & NaN & NaN & NaN & NaN \\
% PickCan & 0.850 & 33.6 & NaN & NaN & 0.482 & 81.4 & NaN & NaN & NaN & NaN & NaN & NaN \\
% TwoArmLift & 0.907 & 20.8 & 0.885 & 46.6 & NaN & NaN & NaN & NaN & NaN & NaN & NaN & NaN \\
% \hline
% Average & \textbf{0.928} & \textbf{42.1} &  0.675 & 46.3 & 0.346  & 48.9  & 0.066 & 62.7 & & & & \\
% \Xhline{0.75pt}
% \end{tabular}
% \end{center}
% \end{table*}


\begin{table*}[t!]
\footnotesize
% \setlength{\tabcolsep}{10pt}
\caption{Experimental results in simulation under accurate feedback data. SR indicates the success rate, and CT represents the convergence timestep ($\times 10^3$). A ‘$\diagdown$’ symbol denotes that the algorithm did not converge. 
% For calculating CT, $\diagdown$ entries are replaced with the maximum allowable timestep.
}
\label{tab:sim_exp_accurate}
\begin{center}
\begin{tabular}{lcccccccccc|cccc}
\Xhline{0.75pt}
Method & \multicolumn{2}{c}{CLIC-Half (ours) } &  \multicolumn{2}{c}{CLIC-Circular (ours)} & \multicolumn{2}{c}{Diffusion Policy} & \multicolumn{2}{c}{Implicit BC} & \multicolumn{2}{c}{PVP} & \multicolumn{2}{c}{CLIC-Explicit (ours)} & \multicolumn{2}{c}{HG-DAgger} \\
 & SR & CT & SR & CT & SR & CT & SR & CT & SR & CT &SR & CT & SR & CT \\ \hline
 %  \textbf{Accurate data}  && & & & & & & & & & && & \\
Push-T & 0.931 & 25.6  & \textbf{0.955} & 28.3 & 0.915 & 24.1 & 0.890 & 31.8 & 0.440 & 28.5 &  0.765 & 35.6 & 0.710 & 38.6  \\
Square & 0.930 & 43.0 & \textbf{0.960} & 55.5 & 0.953 & 48.4 & 0.732 & 54.6 &  0.000 & $\diagdown$   &0.634 & 65.9 & 0.420 & 67.0 \\
Pick-Can & 0.983 & 37.8 & \textbf{0.990} &  38.4 & 0.980 & 36.1 & 0.688 & 44.2 &  0.000 & $\diagdown$    & 0.995 & 42.5 & 0.990 & 35.7 \\
TwoArm-Lift & 0.970 & 18.5 &  \textbf{0.990} & 12.6 & \textbf{0.990} & 23.0 & 0.000 & $\diagdown$ &  0.000 & $\diagdown$    & 0.932 & 14.7 & 0.982 & 14.9  \\
\hline
Average & {0.954} & 31.2 & \textbf{0.974} & 33.7 &  0.960 & 32.9 & 0.578  & $\diagdown$  & 0.066 & $\diagdown$ & 0.836 & 39.7 & 0.776 & 39.1 \\
\Xhline{0.75pt}
\end{tabular}
\end{center}
\end{table*}

% \vspace{-20pt}

\section{Experiments}
\label{sec:experiments}

In the experiment section, we demonstrate the effectiveness of our CLIC method through a series of simulations and real-world experiments. In Section \ref{sec:exp:simulation}, we compare CLIC with state-of-the-art methods under various types of feedback in the robot action space. Section \ref{sec:exp:ablation} presents an ablation study, analyzing the impact of key parameters and design choices. In Section \ref{sec:exp:toy_exp}, a 2D toy experiment highlights how CLIC prevents overfitting. Finally, Section \ref{sec:exp:real_rotbo} showcases the performance of CLIC in real-robot experiments.


\subsection{Simulation Experiments}
\label{sec:exp:simulation}
\textbf{Baselines} We compare CLIC with multiple baselines.
For explicit policies, we consider HG-DAgger \cite{2019_HG_DAgger} and D-COACH \cite{2019_Rodrigo_D_COACH}, which are IIL algorithms that learn from demonstration data and relative correction data, respectively. These two methods are refined for better performance, as reported in Appendix \ref{appendix:baselines}.
For implicit policies, the baselines include IBC \cite{2022_implicit_BC}, PVP \cite{2023_NIPS_PVP}, and Diffusion Policy \cite{2023_diffusionpolicy}. 
As IBC and Diffusion Policy are originally offline IL methods, to make fair comparisons, we adapt them to the IIL framework to ensure fair comparisons. 
Within this IIL framework, all methods share the same structure, differing only in their specific policy update methods.
PVP \cite{2023_NIPS_PVP} employs a loss function to assign low energy values to human actions $\bm a^h$ and high energy values to robot actions $\bm a^r$, given an observed action pair $(\bm a^r, \bm a^h)$ at state $\bm s$.
IBC, detailed in Section \ref{sec:Preliminaries}, and PVP are closely related to our method because they both involve training energy-based models. 
The Diffusion Policy is a counterpart method to IBC when learning from demonstrations. It outperforms IBC because of the improved training stability offered by the diffusion model. 
To ensure fair comparisons, we use a consistent velocity control scheme across all methods. Specifically, each method outputs a velocity command for the robot's end-effector and, if applicable, the gripper as the action at each time step.

% \textbf{Baselines} We compare CLIC with multiple baselines.
% For implicit policies, the baselines include Implicit BC (IBC) \cite{2022_implicit_BC}, PVP \cite{2023_NIPS_PVP}, and Diffusion Policy \cite{2023_diffusionpolicy}. For explicit policies, the baselines include HG-DAgger \cite{2019_HG_DAgger} and D-COACH \cite{2019_Rodrigo_D_COACH}.
% HG-DAgger and D-COACH are Interactive IL algorithms that learn from demonstration data and relative corrective data, respectively, both assuming an explicit Gaussian policy\footnote{The methods are refined for better performance, reported in Appendix \ref{appendix:HGDAgger}.}.
% When adapting offline IL baselines to the Interactive IL framework, we modify the policy update process while retaining the overall IIL structure.
% The Proxy Value Propagation (PVP) \cite{2023_NIPS_PVP} method utilizes a Proxy Value loss to assign high Q-values to human actions $\bm a^h$ and low Q-values to robot actions $\bm a^r$ that trigger human intervention.
% The IBC method, detailed in Section \ref{sec:Preliminaries}, and PVP are closely related to our method, as they both involve training energy-based models. 
% The Diffusion Policy is a counterpart method to IBC when learning from demonstrations, which outperforms IBC because of improved training stability offered by the diffusion model. 
% To ensure fair comparisons, we use a consistent velocity control scheme across all methods. Specifically, each method outputs a velocity command for the robot's end-effector as the action at each time step.



\textbf{Tasks and metrics} We compared these methods across four simulated tasks, including a Push-T task introduced in \cite{2023_diffusionpolicy} and three manipulation tasks from the robosuite benchmark \citep{2020_robosuite}, as illustrated in Fig. \ref{fig:tasks} and described in Appendix \ref{appendix:simulated_experiments_task details}.
The agent is trained by each method using an IIL framework, where the agent interacts with the environment and receives feedback from a \textbf{simulated teacher}.
This simulated teacher is employed to guarantee repeatability and fairness in training, because human teachers may not provide consistent feedback across different experimental trials. 
This is an expert policy that compares its actions with those of the learner every $n$ time steps. If the distance between these actions exceeds a threshold (set at 0.2 for all tasks), the simulated teacher provides feedback to the learner.  For all the simulation tasks, we set $n=2$.
 Each method was run for 160 episodes in every experiment, with this entire procedure repeated 3 times to calculate average final success rates and convergence time steps. 
Specifically, for each individual experiment, we calculated the final success rate by averaging the success rates of the last 8 episodes, each determined by evaluating the learned policy 10 times at the end of that episode.
We defined the convergence time step as the earliest time step when the success rate exceeded 90\% of the final success rate.

\begin{table}[t!]
\caption{Various feedback in the action space}
\centering
% \begin{tabular}{@{}ll@{}}
\begin{tabularx}{0.49\textwidth}{@{}lX@{}}
\toprule
\textbf{Type of Feedback Data}      & \textbf{Definition} \\ \midrule
Accurate absolute correction              & \( \bm a^h = \bm a^* \) \\
Gaussian noise             & \( \bm a^h = \bm a^* + \bm \omega, \bm \omega \sim \mathcal{N}(\bm0, ||\bm a^* - \bm a^r||) \) \\
Partial feedback           & \( \bm a^h \in \{[\bm a^*_{r1}, \bm a_{r2}], [\bm a_{r1}, \bm a_{r2}^*]\} \) \\ 
\hline
Accurate relative correction              & \(\bm  a^h = \bm a^r + e\bm h^*, \bm h^*= \frac{\bm a^* - \bm a^r}{||\bm a^* - \bm a^r||}  \) \\
Direction noise            & \( \bm a^h = \bm a^r + e \bm h_r \), \( \angle (\bm h_r, \bm h^*) = \beta \in [0, 90^\circ) \) \\
\bottomrule
\end{tabularx}
\label{tab:feedback-definitions}
\end{table}
% briefly show different types of feedback
\textbf{Feedback types}
In addition to accurate absolute and relative corrections, Table \ref{tab:feedback-definitions} summarizes other common types of feedback humans provide. These feedback types are also utilized in the simulation experiments. Here, the optimal action $\bm a^*$ is the original action taken by the simulated teacher. The partial feedback is utilized in the TwoArm-Lift task, where $\bm a_{r\,i}, i\in\{1, 2\}$ denotes each robot's action, and $\bm a_{ri}^*$ denotes its optimal action. 


% \begin{table*}
% \footnotesize
% % \setlength{\tabcolsep}{10pt}
% \caption{Experimental results in simulation under various types of feedback data. SR indicates the success rate, and CT represents the convergence timestep. A ‘$\diagdown$’ symbol denotes that the algorithm did not converge. For calculating CT, $\diagdown$ entries are replaced with the maximum allowable timestep.}
% \label{tab:data_folder_presence}
% \begin{center}
% \begin{tabular}{lcccccccccccc}
% \Xhline{0.75pt}
% Method & \multicolumn{2}{c}{CLIC-Half } & \multicolumn{2}{c}{Diffusion} & \multicolumn{2}{c}{Implicit BC} & \multicolumn{2}{c}{PVP} & \multicolumn{2}{c}{CLIC-Simplified} & \multicolumn{2}{c}{HG-DAgger/D-COACH } \\
% \hline
%  \textbf{Accurate data}& SR & CT & SR & CT & SR & CT & SR & CT &SR & CT & SR & CT \vspace{2.5pt}\\
% PushT & \textbf{0.931} & 25.6 & 0.915 & 24.1 & 0.890 & 31.8 & 0.440 & 28.5 &  0.765 & 35.6 & 0.710 & 38.6  \\
% Square & 0.930 & 43.0 & \textbf{0.953} & 48.4 & 0.732 & 54.6 &  0.000 & $\diagdown$   &0.634 & 65.9 & 0.420 & 67.0 \\
% PickCan & 0.983 & 37.8 & 0.963 & 36.1 & 0.688 & 44.2 &  0.000 & $\diagdown$    & 0.995 & 42.5 & 0.990 & 35.7 \\
% TwoArmLift & 0.970 & 34.9 & 0.990 & 23.0 & 0.000 & 2.1 &  0.000 & $\diagdown$    & 0.902 & 14.0 & 0.982 & 14.9  \\
% \hline
% \textbf{Gaussian noise} & SR & CT & SR & CT & SR & CT & SR & CT &
% SR & CT & SR & CT\vspace{2.5pt}\\ 
% PushT & 0.880 & 35.6 & \textbf{0.893} & 49.0 & 0.735 & 42.1 & 0.155 & 45.8 & 0.663 & 41.4 & 0.598 & 41.0 \\
% Square & \textbf{0.925} & 64.5 & 0.000 & 2.1 & 0.000 & 2.1 &   0.000 & $\diagdown$    & 0.238 & 71.0 & 0.060 & 77.2 \\
% PickCan & \textbf{0.973} & 37.8 & 0.467 & 68.2 & 0.070 & 70.4 &   0.000 & $\diagdown$    & 0.800 & 69.5 & 0.028 & 23.1 \\
% TwoArmLift & \textbf{0.847} & 68.0 & 0.000 & 2.1 &   0.000 & $\diagdown$    &  0.000 & $\diagdown$    & 0.433 & 39.3 & 0.008 & 63.8 \\
% \hline
% \textbf{Partial feedback} & SR & CT & SR & CT & SR & CT & SR & CT &SR & CT & SR & CT\vspace{2.5pt}\\
% TwoArmLift & \textbf{0.990} & 26.9 & 0.897 & 29.7 & 0.000 & $\diagdown$  & 0.000 & $\diagdown$ & 0.780 & 19.1 & 0.687 & 25.7 \\
% \hline
% \textbf{Relative correction}  & SR & CT & SR & CT & SR & CT & SR & CT & SR & CT & SR & CT \vspace{2.5pt}\\
% PushT & \textbf{0.853} & 40.8 & 0.060 & 72.0 & 0.400 & 58.8 &   0.110	& 50.4   & 0.733 & 43.5 & 0.520 & 49.0 \\
% Square & \textbf{0.817} & 69.0 & 0.000 & 2.1 & 0.005 & 56.3 &   0.000 & $\diagdown$    & 0.065 & 66.1 & 0.243 & 79.7 \\
% PickCan & 0.870 & 40.8 & 0.000 & 2.0 & 0.310 & 81.7 &   0.000 & $\diagdown$    & \textbf{0.890} & 67.2 & 0.693 & 62.8 \\
% TwoArmLift & \textbf{0.860} & 31.1    &  0.000 & $\diagdown$   & 0.000  & $\diagdown$  &   0.000 & $\diagdown$   & 0.613 & 18.9 & 0.115 & 64.7 \\
% \hline
% \textbf{Direction noise}  & SR & CT & SR & CT & SR & CT & SR & CT & SR & CT & SR & CT \vspace{2.5pt}\\
% PushT & 0.700 & 48.1 & 0.187 & 67.9 & 0.574 & 55.5 &   &   &   &   &   &   \\
% Square & 0.870 & 63.6 & 0.125 & 75.8 & 0.230 & 70.4 &   &   &   &   &   &   \\
% PickCan & 0.850 & 33.6 &   &   & 0.482 & 81.4 &   &   &   &   &   &   \\
% TwoArmLift & 0.907 & 20.8 & 0.885 & 46.6 &   &   &   &   &   &   &   &   \\
% \hline
% Average & \textbf{0.928} & \textbf{42.1} &  0.675 & 46.3 & 0.346  & 48.9  & 0.066 & 62.7 & & & & \\
% \Xhline{0.75pt}
% \end{tabular}
% \end{center}
% \end{table*}


% \begin{table*}
% \footnotesize
% % \setlength{\tabcolsep}{10pt}
% \caption{Experimental results in simulation under various types of feedback data. SR indicates the success rate, and CT represents the convergence timestep. A ‘$\diagdown$’ symbol denotes that the algorithm did not converge. For calculating CT, $\diagdown$ entries are replaced with the maximum allowable timestep.}
% \label{tab:data_folder_presence}
% \begin{center}
% \begin{tabular}{lcccccccc|cccc}
% \Xhline{0.75pt}
% Method & \multicolumn{2}{c}{CLIC-Half } & \multicolumn{2}{c}{Diffusion} & \multicolumn{2}{c}{Implicit BC} & \multicolumn{2}{c}{PVP} & \multicolumn{2}{c}{CLIC-Simplified} & \multicolumn{2}{c}{HG-DAgger/D-COACH } \\
%  & SR & CT & SR & CT & SR & CT & SR & CT &SR & CT & SR & CT \\
%  \hline  \textbf{Accurate data}  && & & & & & & & && & \\
% PushT & \textbf{0.931} & 25.6 & 0.915 & 24.1 & 0.890 & 31.8 & 0.440 & 28.5 &  0.765 & 35.6 & 0.710 & 38.6  \\
% Square & 0.930 & 43.0 & \textbf{0.953} & 48.4 & 0.732 & 54.6 &  0.000 & $\diagdown$   &0.634 & 65.9 & 0.420 & 67.0 \\
% PickCan & 0.983 & 37.8 & 0.963 & 36.1 & 0.688 & 44.2 &  0.000 & $\diagdown$    & 0.995 & 42.5 & 0.990 & 35.7 \\
% TwoArmLift & 0.970 & 34.9 & 0.990 & 23.0 & 0.000 & 2.1 &  0.000 & $\diagdown$    & 0.902 & 14.0 & 0.982 & 14.9  \\
% \hline
% \textbf{Gaussian noise } && & & & & & & & && &\\ 
% PushT & 0.880 & 35.6 & \textbf{0.893} & 49.0 & 0.735 & 42.1 & 0.155 & 45.8 & 0.663 & 41.4 & 0.598 & 41.0 \\
% Square & \textbf{0.925} & 64.5 & 0.000 & 2.1 & 0.000 & 2.1 &   0.000 & $\diagdown$    & 0.238 & 71.0 & 0.060 & 77.2 \\
% PickCan & \textbf{0.973} & 37.8 & 0.467 & 68.2 & 0.070 & 70.4 &   0.000 & $\diagdown$    & 0.800 & 69.5 & 0.028 & 23.1 \\
% TwoArmLift & \textbf{0.847} & 68.0 & 0.000 & 2.1 &   0.000 & $\diagdown$    &  0.000 & $\diagdown$    & 0.433 & 39.3 & 0.008 & 63.8 \\
% \hline
% \textbf{Partial feedback } && & & & & & & & && & \\
% TwoArmLift & \textbf{0.990} & 26.9 & 0.897 & 29.7 & 0.000 & $\diagdown$  & 0.000 & $\diagdown$ & 0.780 & 19.1 & 0.687 & 25.7 \\
% \hline
% \textbf{Relative correction} && & & & & & & & && &  \\
% PushT & \textbf{0.853} & 40.8 & 0.060 & 72.0 & 0.400 & 58.8 &   0.110	& 50.4   & 0.733 & 43.5 & 0.520 & 49.0 \\
% Square & \textbf{0.817} & 69.0 & 0.000 & 2.1 & 0.005 & 56.3 &   0.000 & $\diagdown$    & 0.065 & 66.1 & 0.243 & 79.7 \\
% PickCan & 0.870 & 40.8 & 0.000 & 2.0 & 0.310 & 81.7 &   0.000 & $\diagdown$    & \textbf{0.890} & 67.2 & 0.693 & 62.8 \\
% TwoArmLift & \textbf{0.860} & 31.1    &  0.000 & $\diagdown$   & 0.000  & $\diagdown$  &   0.000 & $\diagdown$   & 0.613 & 18.9 & 0.115 & 64.7 \\
% \hline
% \textbf{Direction noise }   && & & & & & & & && & \\
% PushT & 0.700 & 48.1 & 0.187 & 67.9 & 0.574 & 55.5 &   &   &   &   &   &   \\
% Square & 0.870 & 63.6 & 0.125 & 75.8 & 0.230 & 70.4 &   &   &   &   &   &   \\
% PickCan & 0.850 & 33.6 &   &   & 0.482 & 81.4 &   &   &   &   &   &   \\
% TwoArmLift & 0.907 & 20.8 & 0.885 & 46.6 &   &   &   &   &   &   &   &   \\
% \hline
% Average & \textbf{0.928} & \textbf{42.1} &  0.675 & 46.3 & 0.346  & 48.9  & 0.066 & 62.7 & & & & \\
% \Xhline{0.75pt}
% \end{tabular}
% \end{center}
% \end{table*}


\begin{table*}[t!]
\footnotesize
% \setlength{\tabcolsep}{10pt}
\caption{Simulation results under noisy demonstration data. SR: success rate, CT: convergence timestep ($\times 10^3$). }
\label{tab:sim_exp_noise}
\begin{center}
\begin{tabular}{lcccccccccc|cccc}
\Xhline{0.75pt}
Method & \multicolumn{2}{c}{CLIC-Half  } & \multicolumn{2}{c}{CLIC-Circular } & \multicolumn{2}{c}{Diffusion Policy} & \multicolumn{2}{c}{Implicit BC} & \multicolumn{2}{c}{PVP} & \multicolumn{2}{c}{CLIC-Explicit} & \multicolumn{2}{c}{HG-DAgger} \\
 & SR & CT & SR & CT & SR & CT & SR & CT &SR & CT & SR & CT & SR & CT \\
%  \hline  \textbf{Accurate data}  && & & & & & & & && & \\
% PushT & \textbf{0.931} & 25.6 & 0.915 & 24.1 & 0.890 & 31.8 & 0.440 & 28.5 &  0.765 & 35.6 & 0.710 & 38.6  \\
% Square & 0.930 & 43.0 & \textbf{0.953} & 48.4 & 0.732 & 54.6 &  0.000 & $\diagdown$   &0.634 & 65.9 & 0.420 & 67.0 \\
% PickCan & 0.983 & 37.8 & 0.963 & 36.1 & 0.688 & 44.2 &  0.000 & $\diagdown$    & 0.995 & 42.5 & 0.990 & 35.7 \\
% TwoArmLift & 0.970 & 34.9 & 0.990 & 23.0 & 0.000 & 2.1 &  0.000 & $\diagdown$    & 0.902 & 14.0 & 0.982 & 14.9  \\
\hline
\textbf{Gaussian noise } && & & & & & & & & & && &\\ 
Push-T & 0.880 & 35.6 &  \textbf{0.960} & 29.2 & 0.893 & 49.0 & 0.735 & 42.1 & 0.155 & 45.8 & 0.663 & 41.4 & 0.598 & 41.0 \\
Square & \textbf{0.925} & 64.5 & 0.855 & 63.9 & 0.000 &  $\diagdown$ & 0.000 &  $\diagdown$ &   0.000 & $\diagdown$    & 0.238 & 71.0 & 0.060 & 77.2 \\
Pick-Can & 0.973 & 37.8 & \textbf{1.000} & 42.8 & 0.467 & 68.2 & 0.070 & 70.4 &   0.000 & $\diagdown$    & 0.800 & 69.5 & 0.028 & 23.1 \\
TwoArm-Lift & 0.847 & 47.0 & \textbf{0.945} &  19.2& 0.000 & $\diagdown$ &   0.000 & $\diagdown$    &  0.000 & $\diagdown$    & 0.433 & 39.3 & 0.008 & 63.8 \\
\hline
% \textbf{Partial feedback } && & & & & & & & && & \\
% TwoArmLift & \textbf{0.990} & 26.9 & 0.897 & 29.7 & 0.000 & $\diagdown$  & 0.000 & $\diagdown$ & 0.780 & 19.1 & 0.687 & 25.7 \\
% \hline
% \textbf{Relative correction} && & & & & & & & && &  \\
% PushT & \textbf{0.853} & 40.8 & 0.060 & 72.0 & 0.400 & 58.8 &   0.110	& 50.4   & 0.733 & 43.5 & 0.520 & 49.0 \\
% Square & \textbf{0.817} & 69.0 & 0.000 & 2.1 & 0.005 & 56.3 &   0.000 & $\diagdown$    & 0.065 & 66.1 & 0.243 & 79.7 \\
% PickCan & 0.870 & 40.8 & 0.000 & 2.0 & 0.310 & 81.7 &   0.000 & $\diagdown$    & \textbf{0.890} & 67.2 & 0.693 & 62.8 \\
% TwoArmLift & \textbf{0.860} & 31.1    &  0.000 & $\diagdown$   & 0.000  & $\diagdown$  &   0.000 & $\diagdown$   & 0.613 & 18.9 & 0.115 & 64.7 \\
% \hline
% \hline
Average & 0.906 & 46.2 & \textbf{0.933} &  42.0 &  0.340 & $\diagdown$ & 0.268  & $\diagdown$  & 0.039 & $\diagdown$ & 0.566 & 53.3 & 0.172& 35.7 \\
\hline
\textbf{Direction noise }   && & & & & & & & & & && & \\
Push-T & 0.700 & 48.1 &  \textbf{0.950} & 27.3  & 0.187 & 67.9 & 0.574 & 55.5 & 0.000  &  $\diagdown$ & 0.638 & 44.6 & 0.473 & 43.6 \\
Square & 0.870 & 63.6 &  \textbf{0.910} & 58.9 & 0.125 & 75.8 & 0.230 & 70.4 & 0.000  &  $\diagdown$ & 0.161 & 66.9 & 0.128 & 71.6 \\
Pick-Can & \textbf{1.000} & 43.1 & \textbf{1.000} &  39.0 &   0.000  & $\diagdown$   & 0.482 & 81.4 & 0.000  & $\diagdown$  & 0.867 & 55.4 & 0.342 & 72.0 \\
TwoArm-Lift & 0.965 & 18.7 & \textbf{0.980} & 16.5  & 0.885 & 46.6 &   0.000 & $\diagdown$   &   0.000  & $\diagdown$   & 0.807 & 21.0 & 0.157 & 31.4 \\
\hline
Average & 0.884 & 43.4 &    \textbf{0.960} & 35.4 & 0.399  & $\diagdown$  & 0.429 & $\diagdown$ & 0.000 & $\diagdown$ &  0.618& 47.0 & 0.275 &  54.7 \\
\Xhline{0.75pt}
\end{tabular}
\end{center}
\end{table*}

\subsubsection{Experiments with accurate feedback}
\label{sec:exp:accurate_feedback}
% briefly introduce the tasks (or in appendix)

Table \ref{tab:sim_exp_accurate} shows the results when the teacher's feedback has no noise. 

\textbf{CLIC-Half outperforms IBC, PVP, and performs on par with Diffusion Policy} 
The results shown in Table \ref{tab:sim_exp_accurate} indicate that CLIC-Half constantly outperforms IBC and PVP in terms of success rate and convergence timesteps. 
PVP fails at the robosuite tasks because its loss function only considers the energy value of observed action pairs and cannot effectively shape the EBM. 
The optimal action assumption of IBC and Diffusion Policy is valid when the teacher's demonstration feedback is noise-free.
Under such ideal conditions, this assumption should provide more informative guidance than the assumption used by CLIC-Half.
However, IBC performance decreases as the action dimension of the task increases, with zero success rate in the TwoArm-Lift task. 
Notably, CLIC-Half achieves a $37.6 \%$ higher average success rate compared to IBC. 
Besides, CLIC-Half achieves a similar performance to that of Diffusion Policy.
These results highlight the effectiveness of training EBMs using half-space desired action spaces.

% IBC performance decreases as the action dimension of the task increases, with zero success rate in the TwoArm-Lift task. 
% CLIC-Half achieves a $37.6 \%$ higher average success rate compared to IBC. 
% Besides, CLIC-Half achieves a similar performance to that of Diffusion Policy.
% These results are remarkable because the optimal action assumption of IBC and Diffusion Policy holds true when the teacher's demonstration feedback is noise-free.
% In such ideal conditions, this assumption should provide more informative guidance than the assumption used by CLIC-Half.


 
% Therefore, this result further highlights a promising alternative approach to training policies through the estimation of the optimal actions using desired action spaces.


% motivation, why are you doing this?
% preset result (pure data)
% give interpretation
\textbf{CLIC-Circular outperforms all the baselines} 
% motivation, why are you doing this?
CLIC-Circular is the version of CLIC that mostly closely resembles IBC, as it utilizes a circular desired action space under the assumption of demonstration data. 
It reduces to IBC if all the three following conditions are met: (1) a very small radius defines the circular desired action space, (2) the temperature of the sigmoid function goes to zero, and (3) the uniform Bayes loss is used instead of policy-weighted Bayes loss.
However, CLIC-Circular outperforms IBC by a large margin. Notably,  CLIC-Circular can achieve a 99$\%$ success rate in the TwoArm-Lift task while IBC achieves zero.
CLIC-Circular also outperforms CLIC-Half as it has a more strict assumption. 
Moreover, the fact that CLIC-Circular outperforms Diffusion Policy underscores the capacity of policies represented by EBMs to surpass their diffusion-based counterparts.
Therefore, while previous studies \cite{2022_arxiv_IBC_gaps, 2023_diffusionpolicy} highlight the challenges of training EBMs, our results indicate that EBM-based policies can be trained reliably using demonstration data, by leveraging the concept of the desired action space.






\begin{table*}
\footnotesize
% \setlength{\tabcolsep}{10pt}
\caption{Simulation results under partial and relative feedback data. SR: success rate, CT: convergence timestep ($\times 10^3$). 
% A ‘$\diagdown$’ symbol denotes that the algorithm did not converge. For calculating CT, $\diagdown$ entries are replaced with the maximum allowable timestep.
}
\label{tab:sim_exp_relative_partial}
\begin{center}
\begin{tabular}{lcccccccccc|cccc}
\Xhline{0.75pt}
Method & \multicolumn{2}{c}{CLIC-Half } & \multicolumn{2}{c}{CLIC-Circular } & \multicolumn{2}{c}{Diffusion Policy} & \multicolumn{2}{c}{Implicit BC} & \multicolumn{2}{c}{PVP} & \multicolumn{2}{c}{CLIC-Explicit} & \multicolumn{2}{c}{D-COACH} \\
 & SR & CT & SR & CT & SR & CT & SR & CT &SR & CT & SR & CT & SR & CT \\\hline
\textbf{Partial feedback } && & & & & & & & && & \\
TwoArm-Lift & \textbf{0.990} & 26.9 & 0.920 &  17.8   & 0.897 & 29.7 & 0.000 & $\diagdown$  & 0.000 & $\diagdown$ & 0.863 & 18.1 & 0.687 & 25.7 \\
\hline
\textbf{Relative correction} && & &  & & & & & & & && &  \\
Push-T & \textbf{0.853} & 40.8  & 0.000 & $\diagdown$    & 0.060 & 72.0 & 0.400 & 58.8 &   0.110	& 50.4   & 0.733 & 43.5 & 0.520 & 49.0 \\
Square & \textbf{0.940} & 65.6 & 0.000 & $\diagdown$    & 0.000 & $\diagdown$ & 0.005 & 56.3 &   0.000 & $\diagdown$    & 0.065 & 66.1 & 0.243 & 79.7 \\
Pick-Can & \textbf{0.983} & 41.9 &  0.000 & $\diagdown$    & 0.000 & $\diagdown$ & 0.310 & 81.7 &   0.000 & $\diagdown$    & 0.890 & 67.2 & 0.693 & 62.8 \\
TwoArm-Lift & \textbf{0.955} & 25.3    &  0.000 & $\diagdown$    & 0.000 & $\diagdown$   & 0.000  & $\diagdown$  &   0.000 & $\diagdown$   & 0.920 & 16.8 & 0.115 & 64.7 \\
\hline
Average & \textbf{0.933} & \textbf{43.4} &  0.000 & $\diagdown$    & 0.015 & $\diagdown$ & 0.346  & $\diagdown$ & 0.066 & $\diagdown$ & 0.652 & 48.4 & 0.393 & 64.1 \\
\Xhline{0.75pt}
\end{tabular}
\end{center}
\end{table*}

\textbf{CLIC-Explicit achieves good results in uni-modal tasks, whereas PVP performs poorly}
The losses used by PVP and CLIC-Explicit share a critical similarity: they both increase the probability of human actions and decrease that of robot actions.
We denote this loss type as \textit{point-based loss} as it calculates the loss only on observed action pairs. In contrast, the losses for CLIC-Half, CLIC-Circular, and IBC are termed \textit{set-based loss}.
These losses utilize not only observed action pairs but also additional actions sampled from the EBM for loss calculation. 
Notably, PVP has zero success rate at robosuite tasks, whereas CLIC-Explicit performs well in Pick-Can and TwoArm-Lift—tasks that are both unimodal and align with the Gaussian policy assumption. 
Since PVP employs an EBM as its policy and CLIC-Explicit uses a Gaussian-parametrized policy, 
the result suggests that point-based loss is only effective for policies with simple forms, such as those based on Gaussian distributions, and fails to shape complex policies like EBMs. 
In contrast, the set-based loss provides richer information and is more effective for training EBMs. 





\subsubsection{Experiments with noisy demonstration feedback}
% motivation, why are you doing this?
% preset result (pure data)
% give interpretation

Noise is common when human teachers provide feedback to robots, either for absolute or relative corrections.
This can arise due to factors such as human fatigue or the limitations of teleoperation devices. Therefore, it is essential for algorithms to account for such noise.
To evaluate the ability of CLIC and baseline methods to learn from noisy feedback, we implemented two types of noise in simulation, as defined in Table \ref{tab:feedback-definitions}. 
For absolute corrections, we use a Gaussian distribution to model the noise, which is added to the original demonstration data. 
For relative corrections, the feedback is derived from absolute correction with a known magnitude. To introduce noise, we perturb the original direction signal by $45^\circ$ while maintaining its magnitude.
% we add direction noise to the original direction signal, resulting in a noisy direction signal with the same magnitude but has a $ 45 ^ \circ$ angle between the accurate signal. 
% To make the direction noise less challenging for the baselines, the relative correction is derived from absolute correction with known magnitude. 

% \textcolor{red}{detail more}

% \textbf{Implicit CLIC also outperforms Diffusion Policy when feedback is noisy or partial} 
\textbf{CLIC remains robust while baselines degrade under noisy feedback}
The results in Table \ref{tab:sim_exp_noise} show that, as feedback transitions from accurate to noisy, CLIC-Half and CLIC-Circular experience much smaller performance drops compared to the other baselines.
This can be observed by comparing Table \ref{tab:sim_exp_noise} with Table \ref{tab:sim_exp_accurate}.
In contrast, methods like Diffusion Policy and IBC perform significantly worse under noisy conditions.
This difference arises because these baselines depend on the strict assumption of having accurate demonstrations, making them unable to handle noisy feedback effectively. In comparison, CLIC allows adjusting the desired action space through hyperparameters, ensuring that the true optimal action remains within the desired action space even under noisy feedback.
This capability helps maintain robust performance under noisy conditions. 

\subsubsection{Experiments with relative or partial feedback}
\label{sec:exp:simulation_relative_partial}
% motivation, why are you doing this?
% Previous behavior cloning methods rely on accurate, high-quality demonstrations, limiting their use to scenarios with experts and precise teleoperation devices.  
When providing demonstrations is not possible, humans can provide feedback in more flexible ways, offering valuable information to guide improvement.
One such scenario involves partial feedback, where limitations in the control interface or a large action space make it challenging to provide complete demonstrations. In this case, we evaluated all methods on the TwoArm-Lift task, in which the teacher provides demonstration feedback to only one robot at a time.
Another scenario involves relative corrective feedback.
This feedback type is easier to provide than demonstration, because it does not require the teacher to know precisely which action should be taken; instead, it only necessitates an understanding of the general behavior the robot should exhibit.
To assess how effectively the methods handle this feedback type, we conducted experiments across four simulation tasks.


\textbf{CLIC-Half and CLIC-Explicit effectively learn from partial feedback}
The results of partial feedback experiments are shown in Table \ref{tab:sim_exp_relative_partial}.
In these experiments, human actions consist of two parts: actions on the \textit{feedback dimensions} (where human feedback is provided) and actions on the \textit{non-feedback dimensions} (which may be suboptimal).
While CLIC-Half maintains its success rate when transitioning from accurate demonstration to partial feedback, Diffusion Policy suffers from lower success rates and longer convergence times.  
This difference arises because the BC loss in Diffusion Policy attempts to imitate the entire teacher action, including non-feedback dimensions, potentially leading to suboptimal behaviors.
In contrast, CLIC-Half imitates a desired action space rather than a single action label.
It focuses on improving actions on the feedback dimensions while leaving the non-feedback dimensions unconstrained. 
As a result, CLIC-Half is robust to partial feedback as long as the optimal action lies within the desired action space.
The same reasoning explains the results of CLIC-Explicit outperforming HG-DAgger.
On the other hand, CLIC-Circular's performance drops because its circular desired action space might not include the optimal action. 
This result highlights the importance of ensuring the assumption of CLIC aligns with the data. 
% By using an intersected half-space that includes the optimal action, CLIC-Half avoids such issues, demonstrating consistent performance in partial feedback scenarios.

\textbf{CLIC-Half and CLIC-Explicit can learn from relative corrective feedback}
The results of relative corrective feedback are reported in Table  \ref{tab:sim_exp_relative_partial}. 
In these experiments, human actions improve upon robot actions but are not optimal.
CLIC-Half and CLIC-Explicit show only small performance drops compared to results in absolute corrections in Table \ref{tab:sim_exp_accurate}, because the correction magnitude information is unknown in relative correction.
In contrast, all demonstration-based baselines fail completely, achieving near-zero success rates. 
This failure occurs because the BC loss can mislead policy updates, especially when the current policy’s actions are better than the human actions in the dataset. 
Meanwhile, CLIC-Half and CLIC-Explicit construct desired action spaces to update the policy; as the policy improves, these spaces constructed from the dataset do not conflict with it and are still useful for policy improvement. 
Additionally, CLIC-Circular fails with a zero success rate, as its circular desired action space fails to include the optimal action.
Overall, the ability of CLIC-Half and CLIC-Explicit to learn from relative corrective feedback highlights their distinct advantages in such scenarios.
% As previously discussed in the analysis of partial feedback failures, the same limitations of behavior cloning methods contribute to their inability to handle relative corrective feedback effectively. CLIC’s ability to adapt highlights its distinct advantage in such scenarios.


\begin{figure}[t]
    \centering
    \includegraphics[width = 0.49\textwidth]{figs/Fig10_effects_of_alpha.pdf}
    % \includesvg[width=0.49\textwidth, inkscapelatex=false]{figs/Fig10_effects_of_alpha.svg} 
	\caption{Hyperparameter analysis of the directional certainty parameter 
$\alpha$ for CLIC-Half. The right figure visualizes how different values of 
$\alpha$ adjust the desired action space in 3D.}
 \label{fig:Fig10_effects_of_alpha}
\end{figure}


\subsection{Ablation Study}
\label{sec:exp:ablation}
In this section, we analyze the impact of various hyperparameters and loss design choices on the performance of the CLIC method.  Specifically, we focus on the directional certainty parameter $\alpha$, the temperature $T$ used in the sigmoid function of the observation model, and the different assumptions regarding the prior probability $p(\bm{a}|\bm{s})$.
During these experiments, CLIC-Half is utilized.


\subsubsection{Effects of directional certainty $\alpha$ }
The angle $\alpha$ is the main parameter utilized to control the shape of the desired action space for CLIC-Half, as shown in the right part of Fig. \ref{fig:Fig10_effects_of_alpha}.
In this experiment, we carried out experiments in the Square task. Two feedback types were considered, one with accurate feedback and another with direction noise (noise angle $\beta=45^{\circ}$). The results are reported in the left part of Fig. \ref{fig:Fig10_effects_of_alpha}.
For accurate feedback cases, the success rate decreases when $\alpha$ is larger than $120^\circ$. 
 This occurs because increasing $\alpha$ expands the desired action space to include more undesired actions, thereby providing less useful information for updating the EBM.
For direction noise, the success rate decreases for $\alpha < 2 \beta = 90^\circ$. 
This is because for any given feedback with $\alpha < 2 \beta$, the desired action space fails to include the optimal action and misguides the EBM in its update process.
These findings highlight the importance of carefully selecting $\alpha$ to balance the trade-off: maintaining an informative desired action space and ensuring that it includes the optimal action.

\begin{figure}[t]
    \centering
    \includegraphics[width = 0.49\textwidth]{figs/Fig13_exp_ablation_2.pdf}
    % \includesvg[width=0.49\textwidth, inkscapelatex=false]{figs/Fig13_exp_ablation_2.svg}
	\caption{Ablation study: (1) effects of the temperature parameter $T$. (2) Policy-weighted Bayes loss vs uniform Bayes loss.}
 \label{fig:Fig13_exp_ablation_2}
\end{figure}

\begin{figure*}[t]
	\centering
	% \includegraphics[width=\textwidth]{figs/Fig1_Illustration_generalization_across_state_2.png}
    \includegraphics[width=\textwidth]{figs/Fig1_Illustration_generalization_across_state_3.pdf}
    % \includesvg[width=\textwidth, inkscapelatex=false]{figs/Fig1_Illustration_generalization_across_state_3.svg} 
	\caption{ \textbf{Learned EBM landscapes across different trials}. The figure compares the energy landscapes learned by CLIC, PVP, and IBC after training in a 2D action space. Each row corresponds to the resulting EBMs of each trial. 
    In the middle part, we visualize the process of how CLIC-Circular reduces to IBC as $\varepsilon$ increases.
    CLIC-Circular ( with $\varepsilon=0.5$) effectively trains EBM across different trials, leading to consistent minima close to the true optimal action. In contrast, IBC overfits human actions and fails to estimate the true optimal action. Three evaluation metrics are shown in the right part of the figure.}
 \label{Fig1_Illustration_generalization_across_state}
\end{figure*}


\subsubsection{Effects of temperature $T$}
In Section \ref{section:sub:prob_desired_action_space}, the temperature $T$ is utilized to control the sharpness of the probability distribution $\text{Pr} [\bm a \in \mathcal{A} {(\bm a^r, \bm a^h)} | \bm a , \bm s] $.
We study the effects of $T$ in this experiment on the performance of CLIC, where four different values of $T$ are tested in the Square task. 
The results are presented in the left side of Fig. \ref{fig:Fig13_exp_ablation_2}.
When $T$ is very small ($\log_{10} T = -3$), the success rate drops sharply. At this extreme, the observation model becomes binary (0/1), creating a sharp boundary that is difficult for the neural network to learn. Conversely, when $T$ is too large ($T = 1$), the success rate also declines. In this case, the probabilities of actions belonging or not belonging to $\mathcal{A} {(\bm a^r, \bm a^h)}$ become nearly indistinguishable, offering limited information for policy improvement.
$T = 0.1$ proves to be a good balance between these extremes and is selected across all experiments for CLIC-Half. 


% motivation, why are you doing this?
% preset result (pure data)
% give interpretation
\subsubsection{Policy-weighted Bayes loss vs Uniform Bayes loss}
As described in Section \ref{sec:sub:sub:uniform_bayes_loss}, the uniform Bayes loss treats all actions within the desired action space equally, whereas the policy-weighted Bayes loss prioritizes actions closer to the current robot's policy.
To evaluate the effectiveness of the policy-weighted Bayes loss, we compared it against the uniform Bayes loss.  We implement the uniform variant of CLIC and evaluate it across four simulation tasks with accurate demonstration feedback. The results, shown in the right side of Fig. \ref{fig:Fig13_exp_ablation_2}, demonstrate that the uniform Bayes loss leads to significantly poorer performance compared to the policy-weighted Bayes loss.
This highlights the importance of incremental policy updates. Since the desired action space may include some undesired actions, staying close to the current policy helps avoid imitating unintended behaviors, resulting in a more stable training process.

\subsection{Toy Experiments on Noisy Feedback}
\label{sec:exp:toy_exp}




In this section, we present an example to illustrate the improved performance of CLIC over IBC. The toy task in this example consists of a single constant state with a 2D action space, where the optimal action is set to $\bm{0}$ (see Fig. \ref{Fig1_Illustration_generalization_across_state}). 
The objective is to estimate the optimal action through multiple corrective feedback.
We carried out experiments over 10 trials. In each trial, we generated a randomly sampled dataset consisting of 6 or 7 data points $(\bm s, \bm a^r, \bm a^h)$, where human actions were drawn from a Gaussian distribution centered at the optimal action. 
Each method was trained for 1,000 steps in an offline IL setting, and we visualized the trained EBMs for each method for the first two trials in Fig. \ref{Fig1_Illustration_generalization_across_state}.

To evaluate the methods, we introduced three metrics: (1) the mean square error (MSE) to optimal action: this measures MSE between each local minimum action of the EBM and the optimal action. (2) MSE to human action: this calculates the average MSE between each local minimum action of the EBM and its nearest human action. A smaller value indicates that the EBM is overfitting to the human action. (3) Variance across trials: this evaluates the variance of the EBM values over the entire action space across ten different trials.
These metrics are computed by averaging the results over the 10 trials and are reported in the right side of Fig. \ref{Fig1_Illustration_generalization_across_state}.



\textbf{CLIC learns consistent EBM landscapes across different trials}
% \cite{2017_NIPS_understanding_noise_generalization}
 The PVP-trained EBM tends to be over-optimistic about favorable actions by outputting low energy values for a large region of actions that are not present in the dataset, as shown in the left side of Fig. \ref{Fig1_Illustration_generalization_across_state}.
 This occurs because PVP calculates its loss using only observed action pairs, leaving the energy values of other actions in the action space uncontrolled. 
  On the other hand, IBC's loss function encourages human actions and discourages all other actions, even actions that are very similar to human actions. Consequently, the IBC-trained EBM overfits the data, learning minima corresponding to individual human actions. 
  This overfitting results in EBM landscapes with high variance across different trials, as shown in the bottom-right figure of Fig. \ref{Fig1_Illustration_generalization_across_state}.
 In contrast, the CLIC-trained EBM maintains consistent landscapes, with minima close to the true optimal action and low variance across different trials.
 This explains the superior performance of  CLIC over IBC and PVP, as observed in Section \ref{sec:exp:accurate_feedback}.
 


\textbf{CLIC-Circular reduces to IBC under stricter assumptions}
To illustrate how CLIC-Circular reduces to IBC, we progressively decrease the radius of the circular desired action spaces by increasing the value of $\varepsilon$ (see the middle part of Fig. \ref{Fig1_Illustration_generalization_across_state}). 
As $\varepsilon$ increases, the CLIC-trained EBM starts to split into several clusters and overfit data labels with smaller MSE to human actions, as reported in the right part of Fig. \ref{Fig1_Illustration_generalization_across_state}.
 This overfitting also leads to a larger MSE to the optimal action, indicating that the EBM becomes less effective at identifying the optimal action.
Eventually, as $\varepsilon \rightarrow 1$,  the EBM landscape closely resembles the one trained using IBC.
When the radius becomes nearly zero, imitating a circular desired action space reduces to imitating a human action label, leading to overfitting and performance drops. 
This observation highlights the key distinction between CLIC-Circular and IBC: imitating a circular desired action space rather than a single action. This distinction is crucial for training EBMs stably.
% We demonstrate the transition from CLIC-Circular to IBC by progressively adjusting its hyperparameters and analyzing their impact on the energy-based model.
% (1) For CLIC-Circular with $\varepsilon = 0.5$, by changing the policy-weighted Bayes loss to uniform Bayes loss, the learned EBM tends to imitate the whole desired action space and leads to the EBM landscape quite flat and over-optimistic.





\begin{figure*}[t]
    \centering
    \includegraphics[width = 1.0\textwidth]{figs/Fig7_InsertT_exp_results_traj_3.pdf}
    % \includesvg[width=1.0\textwidth, inkscapelatex=false]{figs/Fig7_InsertT_exp_results_traj_2.svg}
	\caption{Examples of CLIC-Half policy rollout for the Insert-T task after training. At each step, the transparent figure shows the initial state, and the orange arrow indicates the end-effector’s trajectory. The solid figure illustrates the resulting end state, which becomes the initial state for the next step.}
 \label{fig:Fig7_InsertT_exp_results_traj}
\end{figure*}

\begin{figure}[t]
    \centering
    \includegraphics[width = 0.49\textwidth]{figs/Fig7_InsertT_exp_results.pdf}
    % \includesvg[width=0.49\textwidth, inkscapelatex=false]{figs/Fig7_InsertT_exp_results.svg}
	\caption{Experiment results for the Insert-T task, categorized by difficulty levels (easy, medium, and hard). Each column shows the performance metrics for a given difficulty level, along with examples of initial states for that level.
    ``CLIC-Half (offline)'' denotes results for CLIC-Half trained offline.}
 \label{fig:Fig7_InsertT_exp_results}
\end{figure}

\subsection{Real-robot Validations}
\label{sec:exp:real_rotbo}

% \textcolor{red}{explain why we use CLIC-simplified }

Here, we use three tasks to demonstrate the practical applicability of CLIC.
The experiments include a long-horizon multi-modal Insert-T task, a dynamic ball-catching task, and a water-pouring task that necessitates precise control of the robot's end effector position and orientation. 
For the Insert-T task, we employ CLIC-Half and compare its performance against IBC and Diffusion Policy.
For the ball-catching and water-pouring tasks, we use CLIC-Explicit because it performs well in uni-modal tasks, as demonstrated in Section \ref{sec:exp:simulation}, and is more time-efficient compared to CLIC with an EBM policy (Details in Appendix \ref{appendix:time_efficiency_comparision}.).
% \footnote{Details on the time efficiency comparison between CLIC-Explicit and CLIC with an EBM policy are provided in the Appendix \ref{appendix:time_efficiency_comparision}.}
The experiments were carried out using a 7-DoF KUKA iiwa manipulator. When required, an underactuated robotic hand (1-dimensional action space) was attached to its end effector. A 6D space mouse was employed to provide feedback on the pose of the robot's end effector. Furthermore, in the ball-catching task, a keyboard provided feedback on the gripper's actuation. 
The setup of each task is detailed in Appendix \ref{appendix:real_robot_experiments_task details}, and the time durations used are reported in Appendix \ref{appendix:time_duration}. 
% The learned policy is evaluated every 5 episodes for the water-pouring task, every 10 episodes for the ball-catching task, and every 20 episodes for the Insert-T task.
% Results in Fig. \ref{fig:real_exp_figs_combined_all} show that the success rate for all tasks
% exhibits an overall improving trend as the timestep increases, demonstrating the practical applicability of CLIC. 
The experiment results are reported as follows:



\subsubsection{Insert-T—a comparison between state-of-the-art methods}
The Insert-T task requires the robot to insert a T-shaped object into a U-shaped object by pushing to adjust their positions and orientations.
Compared to the Push-T task in simulation experiments, Insert-T is more complex due to two factors: (1) it involves two objects, introducing multi-modal decisions about which object to manipulate first; and (2) it has an increased task horizon. This makes Insert-T a valuable benchmark for evaluating CLIC's performance in long-horizon, multi-modal tasks compared to state-of-the-art methods. 
To better analyze the performance of each method, we categorize the task into three difficulty levels based on the initial state of the objects and the number of contact changes required to complete the task (according to the teacher’s policy).  Examples of these categories are shown in the upper part of Fig. \ref{fig:Fig7_InsertT_exp_results}.
Tasks requiring fewer than 1 contact change are classified as ``easy”, fewer than 5 as ``medium”, and 5 or more as ``hard”. 
During each evaluation, 10 different initial states are tested for each category. To ensure a fair comparison, all methods are evaluated using the same set of initial states.
In the experiment, human-provided demonstration feedback is used to train CLIC-Half within the IIL framework. The collected data is also used to train baseline methods (Diffusion and IBC) offline. For a fair comparison, CLIC is additionally trained offline on the same dataset as the baselines.

Results for different difficulty levels are shown in Fig. \ref{fig:Fig7_InsertT_exp_results}. For easy tasks, baseline methods perform similarly to CLIC but converge more slowly. For medium and hard tasks, CLIC achieves significantly higher success rates. This is particularly evident for hard tasks, where CLIC achieves 80$\%$ success compared to 30$\%$ for Diffusion and 10$\%$ for IBC. 
The results demonstrate CLIC's ability to handle complex multi-modal tasks, thanks to the powerful encoding capabilities of EBMs and CLIC's stable EBM training. 
Furthermore, as the task difficulty increases, CLIC outperforms Diffusion and IBC by a large margin. This suggests that, for training policies, using a desired action space is more robust and efficient in real-robot tasks than relying on a single action label.
 Examples of post-training policy rollouts for CLIC in hard tasks are shown in Fig. \ref{fig:Fig7_InsertT_exp_results_traj}.

Note that CLIC is an interactive learning method, whereas Diffusion and IBC are offline methods in this experiment. Figure \ref{fig:Fig7_InsertT_exp_results} also includes results for CLIC trained with offline data, showing a similar final success rate to the online version. This indicates that CLIC can be employed to learn from offline data as well.
While CLIC is primarily based on the IIL framework, the core ideas proposed here could also benefit offline methods. We believe exploring offline training of CLIC is a promising direction for future work.



\subsubsection{Ball catching—quick coordination and partial feedback}
 The ball-catching task is challenging because of its highly dynamic nature. 
This complexity makes it difficult to provide successful demonstrations of the task to the robot, thus ruling out demonstration-based IIL methods for solving it\footnote{This limitation could be overcome with a highly reactive and precise teleoperation device. However, this also makes the solution more expensive.}.
Instead, relative corrective feedback is more intuitive and easier for this task as humans can provide direction signals occasionally to improve the robot's policy \cite{2020_DCOACH_temporal}.  
Besides, for a successful grasp, the robot must coordinate precisely the ball's motion, the end-effector's motion, along with the gripper's actuation.
This requirement makes it challenging to provide feedback on the complete action space at any given moment, and makes partial feedback suitable for this task.
With partial feedback, relative corrections can be independently provided for either the end-effector's motion or the gripper's actuation.
Therefore, this task enables testing CLIC's ability to learn effective policies from relative corrective feedback that is also partial. 
% The robot is also expected to react after an unsuccessful attempt by trying to grasp the ball again. 
% As a result, this task is particularly interesting for evaluating CLIC's performance, as it allows for the analysis of its ability to quickly coordinate multiple variables in a problem.
% Additionally, the necessary coordination between end-effector motion and gripper actuation makes it challenging to provide feedback on the complete action space at any given moment. 
% Therefore, this enables testing CLIC's ability to learn effective policies from partial feedback in real-world problems, where feedback is independently provided for either the end-effector's motion or the gripper's actuation.

% \textcolor{red}{add results analysis.}
 \begin{figure}
    \centering
    \includegraphics[width = 0.46\textwidth]{figs/Fig8_CatchBall02.pdf}
    % \includesvg[width=0.49\textwidth, inkscapelatex=false]{figs/Fig8_CatchBall02.svg}
	\caption{Experiment results for the ball-catching task.}
 \label{fig:Fig8_CatchBall}
\end{figure}

\begin{figure}
    \centering
    \includegraphics[width = 0.45\textwidth]{figs/Fig9_waterpouring.pdf}
    % \includesvg[width=0.49\textwidth, inkscapelatex=false]{figs/Fig9_waterpouring.svg}
	\caption{Experiment results for the water-pouring task.}
 \label{fig:Fig9_waterpouring}
\end{figure}

Fig. \ref{fig:Fig8_CatchBall} shows the experiment results of the ball-catching task, reporting the success rate of catching the ball within one, two, and three attempts. 
By the end of training, the robot achieves a 1.0 success rate for catching the ball within two attempts, and its first-attempt success rate continues to improve to 0.4.
One post-training policy rollout of a successful first-attempt catch is shown in Fig. \ref{fig:Fig8_CatchBall}, where the ball is caught within 1.5 seconds, an impressive result given the actuation delay of the robot hand.
This experiment demonstrates that CLIC can leverage both the relative corrective feedback and partial feedback effectively to learn challenging high-frequency tasks. 




% \begin{figure}
%     \centering
%     \includegraphics[width = 0.49\textwidth]{figs/Fig7_InsertT_exp_results_traj.png}
% 	\caption{Example of one CLIC-implicit policy rollout for the InsertT task. At each step, The transparent figure represents the initial state, the green arrow visualizes the trajectory of the end-effector, and the solid figure shows the corresponding end state, which becomes the initial state for the next step.}
%  \label{fig:Fig7_InsertT_exp_results_traj}
% \end{figure}



\subsubsection{Water pouring—learning full pose control with CLIC}
% In this experiment, the CLIC-Absolute and CLIC-Relative methods are combined to teach the robot, allowing the human teacher to select (by pressing a button) which teaching mode (absolute or relative correction) to use. 
% This approach showcases the flexibility of CLIC, which can learn from relative or absolute corrections depending on which is more appropriate at each moment. 
The water-pouring task requires the robot to control the pose of a bottle to precisely pour liquid (represented with marbles) into a small bowl. 
CLIC-Explicit is utilized in this experiment. 
The human teacher can provide either absolute or relative corrective feedback and has the flexibility to switch between these modes by pressing a specific keyboard button. 
Initially, absolute feedback was preferred as the policy was learned from scratch, and it was easier to intervene in a 6D action space. As the policy improved, relative corrections made it easier to refine the policy in specific regions of the state space.
% Furthermore, since CLIC allows partial feedback, the corrective signal could be provided in three ways: (1) position only, (2) rotation only, and (3) both position and rotation. This is useful when one part of the robot's action is correct while the other still needs improvement.
% An example
% of the learned policy and the results are shown in Fig. \ref{fig:Fig9_waterpouring}.

The experimental data is shown in Fig. \ref{fig:Fig9_waterpouring}. From Episode $1$ to Episode $16$, the teacher's feedback is provided in an absolute correction format. From Episode $16$ onward, the teacher's feedback is given as relative corrections to make small adjustments to the robot's policy. The success rate exhibits an overall improving trend, consistently increasing from 0.6 in Episode 26 to 0.9 by Episode 41. An example of the policy rollouts after training is illustrated in Fig. \ref{fig:Fig9_waterpouring}.
This experiment demonstrates the effectiveness of CLIC for learning precise control over position and orientation.














% \begin{figure*}[t]
%  \captionsetup{skip=0.2pt} % Adjust skip parameter for this figure
% 	\centering
% 	\includegraphics[width = 0.99\textwidth]{figs/real_exp_figs_combined_all.jpg}
%  % \vspace{2pt}
% 	\caption{Experiment results of the three real-world tasks, with success rate evolution in time and timesteps.
%  In the ball-catching task, success rates are recorded over different numbers of attempts.
%  }	\label{fig:real_exp_figs_combined_all}
% \end{figure*}

 





% \section{Discussion and Future Work}\label{sec:discussion}
This paper pioneers the novel approach of selective response, showing that withholding responses can be a powerful tool for GenAI systems. By opting not to answer every query as accurately as it can---particularly when new or complex topics emerge---GenAI can encourage user participation on community-driven platforms and thereby generate more high-quality data for future training. This mechanism ultimately enhances GenAI's long-term performance and revenue. From a welfare perspective, our results indicate that such selective engagement can also benefit users, leading to better solutions and increased overall satisfaction. Since this work is the first to address selective response strategies for GenAI, numerous promising directions remain for future research; we highlight some of them below. 

First, from a technical standpoint, all of the results in this paper rely on Assumption~\ref{assumption: data lip}, involving the lipshitz condition of the accuracy function and the sensitivity parameter $\beta$. Future work could seek to relax this assumption. Furthermore, our constrained optimization approach in Subsection~\ref{sec: welfare constrained revenue maximization} could be extended to approximate the optimal (continuous) strategy instead of the optimal discrete strategy.

Second, our stylized model adopts the simplifying---though unrealistic---assumption that only a single GenAI platform exists. Admittedly, this makes it easier to focus on the idea of selective responses, and indeed, this assumption is pivotal in keeping our analysis tractable. Future research could explore scenarios with multiple GenAI platforms and human-centered forums. In such settings, one platform's selective response might redirect users not only to forums but also to competing GenAI platforms, leading to the tragedy of the commons \cite{hardin1968tragedy}: Although all GenAI platforms benefit from fresh data generation, none may choose to respond selectively if it means losing users to competitors. 

Third, we assumed Forum behaves non-strategically. In reality, human-centered platforms often monetize their data by selling it to GenAI platforms, adding a further layer of strategic interaction for GenAI. Moreover, data transfer between the platforms can form the basis for collaboration: GenAI could employ selective response to bolster Forum content creation, and Forum could, in turn, attribute that content to GenAI for subsequent use in retraining.


%Third, we make the (again) simplifying assumption that Forum is non-strategic. However, in practice, human-centered platforms can sell their data to GenAI platforms. This adds additional considerations for GenAI. Furthermore, data transmission between the platforms can also become the basis for collaboration: GenAI can use selective response to ensure enough content is generated in Forum, and Forum could provide the data attributed to this mechanism back to GenAI. 


%Second, this paper makes the simplifying yet unrealistic assumption of the existence of one GenAI platform. Indeed, this simplifies many aspects and allows us to analyze selective responses. Future work could address the data generation process with more than one GenAI platform and possibly several human-centered forums. In such a case, selective response of one GenAI platform can either drive users to forums or to other GenAI platforms; thus, we might face a tragedy of the commons situation~\ref{hardin1968tragedy}, where all GenAI platforms are interested in fresh data generation but none volunteer to selectively respond and lose users. 

%This paper examines the competition between a generative AI platform and human-based platforms, challenging the assumption that always providing answers is optimal. We analyzed the impact of withholding answers on GenAI's revenue and developed an efficient approximately optimal algorithm for this purpose. We further explored how withholding affects users, showing that it can lead to better outcomes compared to always answering. Specifically, we demonstrated that withholding can Pareto-dominate this strategy and derived the necessary and sufficient conditions for that. Finally, we proposed a second approximately optimal algorithm that maximizes GenAI's revenue while ensuring users are better off than when GenAI answers all queries.

%On a more conceptual level, our model assumes that GenAI’s data comes solely from the competing platform (Forum). Future research could explore a scenario where GenAI can purchase additional data from a third party. This extension could provide valuable insights into the interplay between withholding answers and data purchasing, and whether these two strategies can complement each other or must be traded off.
\section{Conclusion}\label{sec:conclusion}
%This work explores the impact of grid-connected and wireless measurement setups on capacitive human body communication, revealing significant differences in both channel \revise{gain} and frequency behavior. 
While conventional data acquisition setups are effective for quantifying the forward path loss, which depends on the conductive properties of the human body, they substantially alter the return path behavior by artificially modifying the capacitive coupling to earth ground.
Therefore, a wireless, wearable-sized data acquisition system is essential for quantitatively evaluating the full \ac{HBC} communication channel in a realistic environment with minimal measurement interference. 
To address this challenge, this work introduces \textit{BodySense}, an evaluation platform for human body communication that is fully wireless, compact enough for wearable applications, and designed for extendability.
To validate the proposed system, the measured channel gains of a classical, grid-connected setup and a wireless setup have been determined for distances of \qty{10}{\centi\meter}, \qty{30}{\centi\meter}, and \qty{50}{\centi\meter} between transmitter and receiver for a frequency range between \qty{4}{\mega\hertz} and \qty{64}{\mega\hertz}.
A comparison between the two scenarios yields an average overestimation of \qty{18.15}{\db} over all investigated distances for the classical case, highlighting the importance of evaluating capacitive \ac{HBC} in realistic conditions.
When comparing the energy consumption of capacitive \ac{HBC} with \ac{BLE}, we achieved results comparable to state-of-the-art \ac{BLE} frontends. 
This demonstrates its potential as a promising alternative to conventional \ac{RF} links, offering opportunities to further enhance the overall energy efficiency of wearable devices and move closer to the realization of battery-free, body-worn sensor nodes.



%This paper proposes \textit{Bodysense}, a fully wireless, wearable-sized system designed to accurately evaluate capacitive human body communication. Experimental evaluation has revealed significant differences in both channel loss and frequency behavior. This paper demonstrated that while conventional data acquisition setups are effective for quantifying the forward path loss, which depends on the conductive properties of the human body, they substantially alter the return path behavior by artificially modifying the capacitive coupling to earth ground. Thus, the proposed wearable-sized data acquisition system is essential for quantitatively evaluating the full \ac{HBC} communication channel in a realistic environment with minimal measurement interference. 
%To address this issue, this paper presents \textit{Bodysense}, a fully wireless, wearable-sized, and extendable evaluation platform for human body communication.
%To validate the proposed system, the measured channel gains of a classical, grid-connected setup and a wireless setup have been determined for distances of \qty{10}{\centi\meter}, \qty{30}{\centi\meter}, and \qty{50}{\centi\meter} between transmitter and receiver for a frequency range between \qty{4}{\mega\hertz} and \qty{64}{\mega\hertz}.
%A comparison between the two scenarios yields an average overestimation of \qty{18.15}{\db} over all investigated distances for the classical case, highlighting the importance of evaluating capacitive \ac{HBC} with a measurement setup that is similar or ideally identical to the envisaged use case.


\section*{Acknowledgments}
Omitted for Anonymous Review.

%% Use plainnat to work nicely with natbib. 
\appendix
\onecolumn

\part{Appendix} 

\newcommand{\appendixnumberline}[1]{Appendix\space}

\renewcommand{\appendixname}{Appendix}
\renewcommand{\thesection}{\appendixname~\Alph{section}}
\renewcommand{\thesubsection}{\Alph{section}.\arabic{subsection}}

\section{Proofs}
\label{appendix_sec:proofs}
This section contains all omitted proofs in the paper.

\subsection{Proof of Lemma~\ref{lemma:equivalence_between_perspective_relaxation_and_convexification}}

\begin{namedlemma}
    [~\ref{lemma:equivalence_between_perspective_relaxation_and_convexification}]
    The closed convex hull of the set
    \begin{align*}
        \textstyle \left\{ (\tau, \bbeta, \bz) \middle|
        \| \bbeta \|_\infty \leq M, \, \bz \in \{0, 1\}^p, \, \mathbf{1}^\top \bz \leq k, \, \beta_j ( 1 - z_j) = 0 ~~ \forall j \in [p], \, \sum_{j \in [p]} \beta_j^2 \leq \tau \right\}
    \end{align*}
    is given by the set
    \begin{align*}
        \textstyle \left\{ (\tau, \bbeta, \bz)  \;\middle|\; -M z_j\leq \bbeta_j \leq M z_j ~ \forall j \in [p], \, \bz \in [0, 1]^p, \, \mathbf{1}^\top \bz \leq k, \, \sum_{j \in [p]} \beta_j^2 / z_j \leq \tau \right\}.
    \end{align*}
\end{namedlemma}

\begin{proof}
    Let $\mathcal T$ represent the first set mentioned in the statement of the lemma. Using the definition of the perspective function and applying the big-M formulation technique, we have
    \begin{align*}
        \textstyle \mathcal T = \left\{ (\tau, \bbeta, \bz)  \;\middle|\; -M z_j\leq \bbeta_j \leq M z_j ~ \forall j \in [p], \, \bz \in \{0, 1\}^p, \, \mathbf{1}^\top \bz \leq k, \, \sum_{j \in [p]} \beta_j^2 / z_j \leq \tau \right\}.
    \end{align*}
    As the epigraph of a perspective function constitutes a cone \citep[Lemma~1 \& 2]{shafiee2024constrained}, we may write $\mathcal T = \mathrm{Proj}_{(\tau, \bbeta, \bz)}(\overline {\mathcal T})$, where 
    \begin{align*}
        \textstyle \overline {\mathcal T} = \left\{ (\tau, \bbeta, \bt, \bz) \;\middle|\; \bm 1^\top \bt = \tau, \, \bz \in \{0, 1\}^p, \, \mathbf{1}^\top \bz \leq k, \, \bm A_j \begin{bmatrix} t_j \\ \beta_j \end{bmatrix} + \bm B_j z_j \in \mathbb K_j ~ \forall j \in [p] \right\}
    \end{align*}
    admits a mixed-binary conic representation with
    \begin{align*}
        \bm A = \begin{bmatrix} 1 & 0 \\ 0 & 1 \\ 0 & 0 \\ 0 & 1 \\ 0 & -1 \end{bmatrix}, \,
        \bm B = \begin{bmatrix} 0 \\ 0 \\ 0 \\ M \\ M \end{bmatrix}, \,
        \mathbb K_j = \mathbb L_+ \times \R_+ \times \R_+ \qquad \forall j \in [p].
    \end{align*}
    Here, $\mathbb L_+ \in \R^3$ denotes the rotated second order cone, that is, $\mathbb L_+ = \{ (t, \beta, z) \in \R_+ \times \R \times \R_+: \beta^2 \leq t z  \}$.
    Thus, using \citep[Lemma~4]{shafiee2024constrained}, the set $\overline{\mathcal T}$ satisfies all the requirements of \citep[Theorem~1]{shafiee2024constrained}, and therefore, its continuous relaxation gives the closed convex hull of $\overline{\mathcal T}$, that is,
    \begin{align*}
        \textstyle \cl \conv(\overline {\mathcal T}) = \left\{ (\tau, \bbeta, \bt, \bz) \;\middle|\; \bm 1^\top \bt = \tau, \, \bz \in [0, 1]^p, \, \mathbf{1}^\top \bz \leq k, \, \bm A_j \begin{bmatrix} t_j \\ \beta_j \end{bmatrix} + \bm B_j z_j \in \mathbb K_j ~ \forall j \in [p] \right\}.
    \end{align*}
    The prove concludes by applying Fourier-Motzkin elimination method to project out the variable $\bt$.
\end{proof} 

\begin{namedlemma}
    [~\ref{lemma:fenchel_conjugate_of_g_closed_form_expression}]
    The conjugate of $g$ is given by
    \begin{equation*}
        g^*(\balpha) = \TopSum_k({\bf H}_M(\balpha)).
    \end{equation*}
\end{namedlemma}

\begin{proof}
    Using the definition of the implicit function $g$ in~\eqref{eq:function_g_definition}, we have
    \begin{align}
        \label{eq:max:g*}
        g^*(\balpha) = \left\{
        \begin{array}{cl}
            \max & \balpha^\top \bbeta -  \frac{1}{2} \sum_{j \in [p]} {\beta_j^2}/{z_j} \\[1ex]
            \st & \bbeta \in \R^p, \, \bz \in [0, 1]^p, \, \bm 1^\top \bz \leq k, \\[1ex]
            & -M z_j \leq \beta_j \leq M z_j ~ \forall j \in [p]
        \end{array}
        \right.
    \end{align}
    For any fixed feasible $\bz$, the maximization problem over $\bbeta$ is a simple constrained quadratic problem, that can be solved analytically by the vector $\beta^\star$ whose $j$'th element is given by
    $\beta_j^\star = \sgn(\alpha_j) \min(\vert{\alpha_j}, M) z_j.$
    Substituting the optimizer $\beta^\star$, the objective function of the maximization problem in~\eqref{eq:max:g*} simplifies to
    \begin{align*}
        \balpha^\top \bbeta^\star - \frac{1}{2} \sum_{j \in [p]} {\beta_j^\star}^2 / z_j 
        &= \sum_{j \in [p]} \alpha_j \cdot \sgn(\alpha_j) \min(\vert{\alpha_j}, M) z_j - \frac{\left( \sgn\left( \alpha_j \right) \min\left(\vert{\alpha_j}, M \right) z_j \right)^2}{2z_j} \\
        &= \sum_{j \in [p]} ( \vert{\alpha_j} \min(\vert{\alpha_j}, M) - \frac{1}{2} \min(\alpha_j^2, M^2) ) z_j %\\
        % &= \begin{cases} \frac{1}{2} \alpha_j^2 z_j & \text{if } \vert{\alpha_j} \leq M  \\ \left( M \vert{\alpha_j} - \frac{1}{2} M^2 \right) z_j & \text{if } \vert{\alpha_j} > M
        % \end{cases} \\
        = H_M(\alpha_j) z_j,
    \end{align*}
    where the second equality holds as $\bz$ is a binary vector, and the last equality follows from the definition of the Huber loss function. We thus arrive at
    \begin{align*}
        g^*(\balpha) = \max_{\bz \in [0,1]^p} \left\{ \textstyle \sum_{j \in [p]} H_M (\alpha_j) z_j: \bm 1^\top \bz \leq k \right\} = \TopSum_k ({\mathbf{H}}_M(\balpha)).
    \end{align*}
    This completes the proof.
\end{proof}

\subsection{Proof of Lemma~\ref{lemma:equivalence_between_proximal_operator_and_huber_isotonic_regression}}

\begin{namedlemma}
    [~\ref{lemma:equivalence_between_proximal_operator_and_huber_isotonic_regression}]
    For any $\bmu \in \R^p$, we have 
    $$\prox_{\rho g^*}(\bmu) = \sgn(\bmu) \odot \bnu^\star, $$ 
    where $\odot$ denotes the Hadamard (element-wise) product, $\bnu^\star$ is the unique solution of the following optimization problem
    \begin{align}
        \label{A:obj:KyFan_Huber_isotonic_regression}
        \begin{array}{cl}
            \min\limits_{\bnu \in \R^p} & \frac{1}{2} \sum_{j \in [p]} (\nu_j - \vert{\mu_j})^2 + \rho \sum_{j \in \calJ} H_M (\nu_j) \\[2ex]
            \st & \quad \nu_j \geq \nu_l \; \text{ if } \; \vert{\mu_j} \geq \vert{\mu_l} ~~ \forall j, l \in [p],
        \end{array} 
    \end{align}
    and $\calJ$ is the set of indices of the top $k$ largest elements of~$ \vert{\mu_j}, j \in [p]$. 
\end{namedlemma}

\begin{proof}
    For simplicity, let $\balpha^\star = \prox_{\rho g^*}(\bmu)$, that is,
    \begin{align}
        \label{eq:alpha:star}
        \balpha^\star = \argmin_{\bm \alpha \in \R^p} ~ \frac{1}{2} \Vert{\bm \alpha - \bm \mu}_2^2 + \rho g^*(\bm \alpha).
    \end{align}
    We first show that $\sgn(\balpha^\star) = \sgn(\bmu)$ (step 1) and then establish that for every $j, l \in [p]$ with $\vert{\mu_j} \geq \vert{\mu_l}$, we have $\vert{\alpha_j^\star} \geq \vert{\alpha_l^\star}$ (step 2). We then conclude the proof using these observations.

    \begin{itemize}[label=$\diamond$,leftmargin=*]
        \item \textbf{Step 1.} We prove the sign-preserving property through a proof by contradiction. For the sake of contradiction, suppose that there exists some $j \in [p]$ such that $\sgn(\alpha_j^\star) \neq \sgn(\mu_j)$.
        Hence, we can construct a new $\balpha'$ by flipping the sign of $\alpha_j^\star$, i.e., $\alpha_j' = -\alpha_j^\star$, and keeping the rest of the elements the same as $\balpha^\star$.
        Now under the assumption that $\sgn(\alpha_j^\star) \neq \sgn(\mu_j)$, we have $\left\lvert{\alpha_j^\star - \mu_j}\right\rvert > \left\lvert{\lvert{\alpha_j^\star}\rvert - \lvert{\mu_j}\rvert}\right\rvert = \left\lvert{\alpha_j' - \mu_j}\right\rvert$, so the $j$-th term in the first summation of the objective function will decrease while everything else remains the same.
        This leads to a smaller objective value for $\balpha'$ than $\balpha^\star$, which contradicts the optimality of $\balpha^\star$.
        Thus, the claim follows.
        
        \item \textbf{Step 2.} We prove the relative magnitude-preserving property through a proof by contradiction. For the sake of contradiction, suppose that there exists some $j, l \in [p]$ such that $\vert{\mu_j} \geq \vert{\mu_l}$ but $\vert{\alpha_j^\star} < \vert{\alpha_l^\star}$.
        Then, we can construct a new $\balpha'$ by swapping $\alpha_j^\star$ and $\alpha_l^\star$, i.e., $\alpha_j' = \alpha_l^\star$ and $\alpha_l' = \alpha_j^\star$, and keeping the rest of the elements the same as $\balpha^\star$.
        Under the assumption that $\vert{\mu_j} \geq \vert{\mu_l}$ but $\vert{\alpha_j^\star} < \vert{\alpha_l^\star}$, we have $\left\lvert{\alpha_j^\star - \mu_j}\right\rvert + \left\lvert{\alpha_l^\star - \mu_l}\right\rvert > \left\lvert{\alpha_l^\star - \mu_j}\right\rvert + \left\lvert{\alpha_j^\star - \mu_l}\right\rvert =
        \left\lvert{\alpha_j' - \mu_j}\right\rvert + \left\lvert{\alpha_l' - \mu_l}\right\rvert$, so the sum of the $j$-th and $l$-th terms in the first summation of the objective function will decrease while everything else remains the same.
        This leads to a smaller objective value for $\balpha'$ than $\balpha^\star$, which contradicts the optimality of $\balpha^\star$. Thus, the claim~follows.
    \end{itemize}
    Using these two observations, we are ready to prove that $\balpha^\star = \sgn(\bmu) \odot \bnu^\star$.
    We first reparametrize the minimization problem~\eqref{eq:alpha:star} by substituting the decision variable $\balpha$ with a new variable $\bnu \in \R_+^p$ satisfying $\balpha = \sgn(\bmu) \odot \bnu$. By the sign-preserving property in step 1, it is easy to show the equivalence between the optimization problem in~\eqref{eq:alpha:star} and the following optimization problem
    \begin{align*}
        \min_{\bnu \in \R^p_+} ~ \textstyle \frac{1}{2} \sum_{j \in [p]} (\nu_j - \vert{\mu_j})^2 + \rho \TopSum_k \left( \mathbf{H}_M ( \bnu ) \right).
    \end{align*}
    By the relative magnitude-preserving property in step 2, we can further set the equivalence between the minimization problem in~\eqref{eq:alpha:star} and the following optimization problem
    \begin{align*}
        \begin{array}{cl}
            \displaystyle \min_{\bnu \in \R_+^p} & \frac{1}{2} \sum_{j \in [p]} (\nu_j - \vert{\mu_j})^2 + \rho \sum_{j \in \calJ} H_M (\nu_j), \\ 
            \st & \quad \nu_j \geq \nu_l \; \text{ if } \; \vert{\mu_j} \geq \vert{\mu_l}.
        \end{array} 
    \end{align*}
    Lastly, the nonnegative constraint on $\bnu$ can be removed as the second summation term in the objective function implies that $\nu_j \geq 0$. Thus, we have shown that any feasible point $\balpha$ in the minimization problem~\eqref{eq:alpha:star} can be reconstructed by any feasible point $\bnu$ in the minimization problem in the statement of lemma, while maintaining the same objective value. Hence, we may conclude that $\balpha^\star = \sgn(\bmu) \odot \bnu^\star$, as required.
\end{proof}

\subsection{Proof of Lemma~\ref{lemma:PAVA_algorithm_exact_solution}}






%Assuming that the input vector $\bmu$ has already been sorted so that the elements are in nonincreasing order in terms of their absolute values, the algorithm runs in linear time complexity, $O(p)$, where $p$ is the number of elements in the input vector $\bmu$.

\begin{namedlemma}
    [~\ref{lemma:PAVA_algorithm_exact_solution}]
    The vector $\hat \bnu$ in Algorithm~\ref{alg:PAVA_algorithm} solves~\eqref{obj:KyFan_Huber_isotonic_regression} exactly.
\end{namedlemma}

\begin{proof}
    The minimization problem~\eqref{obj:KyFan_Huber_isotonic_regression} is an instance of a generalized isotonic regression problem taking the form
    \begin{align}
        \label{obj:KyFan_Huber_isotonic_regression_rewritten_as_generalized_isotonic_regression}
        \min_{\bnu} \sum_{j=1}^{p} h_j(\nu_j) \quad \st \quad \nu_1 \geq \nu_2 \geq \cdots \geq \nu_J,
    \end{align}
    where $h_j(\nu) = \frac{1}{2} (\nu - \mu_j)^2 + \rho_j H_M(\nu)$, $\rho_j = \rho$ if $j \in \calJ$ and $\rho_j = 0$ otherwise, and the set $\calJ$ is the set of indices of top k largest elements of $\vert{\mu_j}$, as defined in the statement of Lemma~\ref{lemma:equivalence_between_proximal_operator_and_huber_isotonic_regression}.
    Thanks to~\cite{best2000minimizing,ahuja2001fast}, the optimizer of~\eqref{obj:KyFan_Huber_isotonic_regression_rewritten_as_generalized_isotonic_regression} satisfies two key properties: 
    \begin{itemize}[label=$\diamond$,leftmargin=*]
        \item \textbf{Property 1: Optimal solution for a merged block is single-valued.} 
        Suppose we have two adjacent blocks $[a_1, a_2]$ and $[a_2+1, a_3]$ such that the optimal solution of each block is single-valued, that is, the minimization problems
        \begin{align*}
            \left\{
            \begin{array}{cl}
                \min\limits_{\bnu_{a_1:a_2}} & \sum_{j=a_1}^{a_2} h_j(\nu_j) \\
                \st & \nu_{a_1} \geq \cdots \geq \nu_{a_2}
            \end{array}
            \right. \quad \text{and} \quad
            \left\{
            \begin{array}{cl}
                \min\limits_{\bnu_{a_2+1:a_3}} & \sum_{j=a_2+1}^{a_3} h_j(\nu_j) \\
                \st & \nu_{a_2+1} \geq \cdots \geq \nu_{a_3} \\
            \end{array}
            \right.
        \end{align*}
        are solved by $\bnu_{a_1:a_2}^\star$ and $\bnu_{a_2+1:a_3}^\star$ with $\nu_{a_1}^\star = \cdots = \nu_{a_2}^\star$ and $\nu_{a_2+1}^\star = \cdots = \nu_{a_3}^\star$, respectively.
        If $\nu_{a_1}^\star \leq \nu_{a_2+1}^\star$, then the optimal solution for the merged block $[a_1, a_3]$ is single-valued, that is, the minimization problem
        \begin{align*}
            \left\{
            \begin{array}{cl}
                \min\limits_{\bnu_{a_1:a_3}} & \sum_{j=a_1}^{a_3} h_j(\nu_j) \\
                \st & \nu_{a_1} \geq \cdots \geq \nu_{a_3}
            \end{array}
            \right.
        \end{align*}
        is solved by $\bnu_{a_1:a_3}^\star$ with $\nu_{a_1}^\star = \cdots = \nu_{a_3}^\star$.

        \item \textbf{Property 2: No isotonic constraint violation between single-valued blocks implies the solution is optimal.} Suppose that we have $s$ blocks $[a_1, a_2], [a_2+1, a_3], \ldots, [a_{s}+1, a_{s+1}]$ (with $a_1=1$ and $a_{s+1}=p$) such that the optimal solution for each block is single-valued, that is, $\nu^\star_{a_l+1} = \dots = \nu^\star_{a_{l+1}}$ for all $l \in [s]$. Then, if $\hat{\nu}_{a_1} \geq \hat{\nu}_{a_2+1} \geq \ldots \hat{\nu}_{a_{s}}$, then $\hat{\bnu}$ is the optimal solution to~\eqref{obj:KyFan_Huber_isotonic_regression_rewritten_as_generalized_isotonic_regression}.
    \end{itemize}
    
    Using these two properties, it is now easy to see why Algorithm~\ref{alg:PAVA_algorithm} returns the optimal solution. 
    We start by constructing blocks which have length 1.
    The initial value restricted to each block is optimal.
    Then, we iteratively merge adjacent blocks and update the values of $\nu_j$'s whenever there is a violation of the isotonic constraint.
    By the first property, the optimal solution for the merged block is single-valued.
    Therefore, we can compute the optimal solution for the merged block by solving a univariate optimization problem.
    We keep merging blocks until there is no isotonic constraint violation.
    When this happens, by construction, the solution for each block is single-valued and optimal.
    By the second property, the final vector $\hat{\bnu}$ is the optimal solution to~\eqref{obj:KyFan_Huber_isotonic_regression_rewritten_as_generalized_isotonic_regression}, as required.
\end{proof}

\subsection{Proof of Lemma~\ref{lemma:PAVA_merging_linear_time_complexity}}

\begin{namedlemma}
    [~\ref{lemma:PAVA_merging_linear_time_complexity}]
    The merging step (lines 11-14) in Algorithm~\ref{alg:PAVA_algorithm} can be performed in linear time complexity $\mathcal O(p)$.
\end{namedlemma}

\begin{proof}
A detailed implementation of line 11-14 (Step 3) of the PAVA Algorithm~\ref{alg:PAVA_algorithm} that achieves a linear time complexity is presented in Algorithm~\ref{alg:up_and_down_block_algorithm_for_merging_in_PAVA}. In the following, we first show that Algorithm~\ref{alg:up_and_down_block_algorithm_for_merging_in_PAVA} accomplishes the objective in lines 11-14 of Algorithm~\ref{alg:PAVA_algorithm}. We then establish that Algorithm~\ref{alg:up_and_down_block_algorithm_for_merging_in_PAVA} runs in linear time complexity.

\begin{algorithm}[hb]
    \caption{Up and Down Block Algorithm for Merging in PAVA}
    \label{alg:up_and_down_block_algorithm_for_merging_in_PAVA}
    \begin{flushleft}
    \textbf{Input:} vector $\bmu \in \mathbb{R}^p$, nonnegative weights $\brho \in \mathbb{R}_{+}^p$ ($\rho_{[1:k]}=\rho, \rho_{k+1:p}=0$), vector $\hat{\bnu}$ ($\hat{\nu}_j = \text{prox}_{\rho_j H_M}(\vert{\mu_j})$), integer $k \in \mathbb{N}$ (first $k$ elements subject to Huber penalty), and threshold $M > 0$ for the Huber loss function. %\\
    %\textbf{Output:} vector $\hat{\bnu} \in \mathbb{R}^p$, which is the optimal solution to Problem~\eqref{obj:KyFan_Huber_isotonic_regression}.\\
    \end{flushleft}
    \begin{algorithmic}[1]
        \STATE \COMMENT{Initialization for the first block}
        \STATE Initialize $b=1$, $P_1 = \rho_1$, $S_1 = \vert{\mu_1}$, $N_b=1$, $\nu_1$, $r_1 = 1$.
        \STATE $\nu_{\text{prev}} = \hat{\nu}_1$, $j=2$
        \WHILE{$j \leq n$}
            \STATE $b = b + 1$
            \STATE $P_b = \rho_j$, $S_b = \vert{\mu_j}$, $N_b=1$, $\nu = \hat{\nu}_j$
            \STATE \COMMENT{If the value for the current singleton block is greater that of the previous block (isotonic violation), merge the current block with the previous block}
            \IF{$\nu > v_{\text{prev}}$}
                \STATE $b = b - 1$
                \STATE $P_b = P_b + \rho_j$, \, $S_b = S_b + \vert{\mu_j}$, \, $N_b = N_b + 1$, \, $\nu = \text{prox}_{\frac{P_b}{N_b} H_{M}}(\frac{S_b}{N_b})$
                \STATE \COMMENT{Look forward: keep merging the current block with the next block if the isotonic violation persists}
                \WHILE{$j < n$ \AND $\nu \leq \hat{\nu}_j$}
                    \STATE $j = j + 1$
                    \STATE $P_b = P_b + \rho_j$, \, $S_b = S_b + \vert{\mu_j}$, \, $N_b = N_b + 1$, \, $\nu = \text{prox}_{\frac{P_b}{N_b} H_{M}}(\frac{S_b}{N_b})$
                \ENDWHILE
                \STATE \COMMENT{Look backward: keep merging the current block with the previous block if the isotonic violation persists}
                \WHILE{$b > 1$ \AND $\nu_{b-1} < \nu$}
                    \STATE $b = b - 1$
                    \STATE $P_b = P_b + P_{b+1}$, \, $S_b = S_b + S_{b+1}$, \, $N_b = N_b + N_{b+1}$, \, $\nu = \text{prox}_{\frac{P_b}{N_b} H_{M}}(\frac{S_b}{N_b})$
                \ENDWHILE
            \ENDIF
            \STATE \COMMENT{Save the current block's value and the index of the last element in the block}
            \STATE $\nu_b = \nu$, $r_b = j$
            \STATE \COMMENT{Start fresh on the next element}
            \STATE $\nu_{\text{prev}} = \nu$, $j = j + 1$
        \ENDWHILE
        \STATE \COMMENT{Modify the output vector to have the same new value for all elements in each block}
        \FOR{$l = 1, ..., b$}
            \STATE $\hat{\nu}_{[r_{l-1}+1:r_l]} = \nu_l$
        \ENDFOR
        \STATE \textbf{return} $\hat{\bnu}$
    \end{algorithmic}
\end{algorithm}

% \begin{algorithm}[H]
%     \caption{Modified PAVA with Huber Penalty for Nonincreasing Isotonic Regression}
%     \label{alg:up_and_down_block_algorithm_for_merging_in_PAVA}
%     \begin{flushleft}
%     \textbf{Input:} vector $\bmu \in \mathbb{R}^n$ (observations), nonnegative weights $w \in \mathbb{R}_{\ge 0}^n$, integer $k \in \mathbb{N}$ (first $k$ elements subject to Huber penalty), scalar $\rho > 0$ (Huber penalty coefficient), and threshold $M > 0$.\\
%     \textbf{Output:} vector $x \in \mathbb{R}^n$ (monotone, nonincreasing sequence approximating $y$).\\
%     \end{flushleft}
%     \begin{algorithmic}[1]
%         \STATE $M, P$
%         \STATE $y_1 \gets y,\; y_2 \gets y,\; w_1 \gets w,\; w_2 \gets w$
%         \FOR{$j = 0$ to $k-1$}
%             \STATE $w_1[j] \gets w[j] + \frac{\rho}{2}$
%             \STATE $y_1[j] \gets y[j] \cdot \frac{w[j]}{w_1[j]}$
%             \STATE $y_2[j] \gets y[j] - \frac{\rho M}{2w_2[j]}$
%         \ENDFOR
%         \STATE Initialize boolean array $\text{use\_y1}$ of length $n$:
%         \FOR{$j = 0$ to $n-1$}
%             \STATE $\text{use\_y1}[j] \gets (y_1[j] \leq M)$
%         \ENDFOR
    
%         \STATE Allocate arrays $x_1^{\text{block}}, w_1^{\text{block}}, x_2^{\text{block}}, w_2^{\text{block}}$, and $\text{use\_x1\_block}$ of length $n$
%         \STATE Allocate array $r$ of length $n+1$
%         \STATE $r[0] \gets -1,\; r[1] \gets 0$
%         \STATE $b \gets 1$ \COMMENT{Number of blocks}
    
%         \STATE $x_1^{\text{block}}[0] \gets y_1[0],\; w_1^{\text{block}}[0] \gets w_1[0]$
%         \STATE $x_2^{\text{block}}[0] \gets y_2[0],\; w_2^{\text{block}}[0] \gets w_2[0]$
%         \STATE $\text{use\_x1\_block}[0] \gets \text{use\_y1}[0]$
    
%         \STATE $j \gets 1$
%         \WHILE{$j < n$}
%             \STATE $b \gets b + 1$
    
%             \STATE \textbf{Compute current values:}
%             \IF{$\text{use\_y1}[j] = \text{True}$}
%                 \STATE $x_{\text{curr}} \gets y_1[j],\; w_{\text{curr}} \gets w_1[j]$
%             \ELSE
%                 \STATE $x_{\text{curr}} \gets \max(y_2[j], M),\; w_{\text{curr}} \gets w_2[j]$
%             \ENDIF
    
%             \STATE \textbf{Compute previous block values:}
%             \STATE $\ell \gets b - 2$ \COMMENT{Index of previous block}
%             \IF{$\text{use\_x1\_block}[\ell] = \text{True}$}
%                 \STATE $x_{\text{prev}} \gets x_1^{\text{block}}[\ell],\; w_{\text{prev}} \gets w_1^{\text{block}}[\ell]$
%             \ELSE
%                 \STATE $x_{\text{prev}} \gets \max(x_2^{\text{block}}[\ell], M),\; w_{\text{prev}} \gets w_2^{\text{block}}[\ell]$
%             \ENDIF
    
%             \STATE \textbf{Check for nonincreasing violation:} 
%             \IF{$x_{\text{prev}} < x_{\text{curr}}$}
%                 \STATE $b \gets b - 1$
%                 \STATE Merge current element with previous block:
    
%                 \STATE $S_1 \gets (w_1^{\text{block}}[b-1] \cdot x_1^{\text{block}}[b-1]) + (w_1[j] \cdot y_1[j])$
%                 \STATE $W_1 \gets w_1^{\text{block}}[b-1] + w_1[j]$
%                 \STATE $x_{1,\text{merged}} \gets S_1 / W_1$
    
%                 \STATE $S_2 \gets (w_2^{\text{block}}[b-1] \cdot x_2^{\text{block}}[b-1]) + (w_2[j] \cdot y_2[j])$
%                 \STATE $W_2 \gets w_2^{\text{block}}[b-1] + w_2[j]$
%                 \STATE $x_{2,\text{merged}} \gets S_2 / W_2$
    
%                 \STATE $\text{use\_x1\_merged} \gets (x_{1,\text{merged}} \leq M)$
    
%                 \COMMENT{k-up step: merge forward if violation persists}
%                 \WHILE{$j < n-1$ \AND $\bigl(x_{1,\text{merged}} \cdot \text{use\_x1\_merged} + (1-\text{use\_x1\_merged}) \cdot \max(x_{2,\text{merged}}, M)\bigr) \leq \bigl(y_1[j+1] \cdot \text{use\_y1}[j+1] + (1-\text{use\_y1}[j+1]) \cdot \max(y_2[j+1], M)\bigr)$}
%                     \STATE $j \gets j + 1$
%                     \STATE $S_1 \gets S_1 + w_1[j] \cdot y_1[j],\; W_1 \gets W_1 + w_1[j],\; x_{1,\text{merged}} \gets S_1 / W_1$
%                     \STATE $S_2 \gets S_2 + w_2[j] \cdot y_2[j],\; W_2 \gets W_2 + w_2[j],\; x_{2,\text{merged}} \gets S_2 / W_2$
%                     \STATE $\text{use\_x1\_merged} \gets (x_{1,\text{merged}} \leq M)$
%                 \ENDWHILE
    
%                 \COMMENT{k-down step: merge backward if violation persists}
%                 \WHILE{$b > 1$ \AND $\bigl(x_1^{\text{block}}[b-2] \cdot \text{use\_x1\_block}[b-2] + (1-\text{use\_x1\_block}[b-2]) \cdot \max(x_2^{\text{block}}[b-2], M)\bigr) < \bigl(x_{1,\text{merged}} \cdot \text{use\_x1\_merged} + (1-\text{use\_x1\_merged}) \cdot \max(x_{2,\text{merged}}, M)\bigr)$}
%                     \STATE $b \gets b - 1$
%                     \STATE $S_1 \gets S_1 + w_1^{\text{block}}[b-1] \cdot x_1^{\text{block}}[b-1],\; W_1 \gets W_1 + w_1^{\text{block}}[b-1],\; x_{1,\text{merged}} \gets S_1 / W_1$
%                     \STATE $S_2 \gets S_2 + w_2^{\text{block}}[b-1] \cdot x_2^{\text{block}}[b-1],\; W_2 \gets W_2 + w_2^{\text{block}}[b-1],\; x_{2,\text{merged}} \gets S_2 / W_2$
%                     \STATE $\text{use\_x1\_merged} \gets (x_{1,\text{merged}} \leq M)$
%                 \ENDWHILE
    
%                 \STATE $x_1^{\text{block}}[b-1] \gets x_{1,\text{merged}},\; w_1^{\text{block}}[b-1] \gets W_1$
%                 \STATE $x_2^{\text{block}}[b-1] \gets x_{2,\text{merged}},\; w_2^{\text{block}}[b-1] \gets W_2$
%                 \STATE $\text{use\_x1\_block}[b-1] \gets \text{use\_x1\_merged}$
%                 \STATE \textit{No violation}
%             \ELSE
%                 \STATE \COMMENT{No violation}
%                 \STATE $x_1^{\text{block}}[b-1] \gets y_1[j],\; w_1^{\text{block}}[b-1] \gets w_1[j]$
%                 \STATE $x_2^{\text{block}}[b-1] \gets y_2[j],\; w_2^{\text{block}}[b-1] \gets w_2[j]$
%                 \STATE $\text{use\_x1\_block}[b-1] \gets \text{use\_y1}[j]$
%             \ENDIF
    
%             \STATE $r[b] \gets j$
%             \STATE $j \gets j + 1$
%         \ENDWHILE
    
%         \COMMENT{Expand blocks to form final $x$}
%         \STATE $x \gets$ empty array of length $n$
%         \STATE $f \gets n-1$
    
%         \FOR{$\ell = b$ down to $1$}
%             \STATE $start\_idx \gets r[\ell-1] + 1$
%             \STATE $end\_idx \gets r[\ell]$
%             \IF{$\text{use\_x1\_block}[\ell-1] = \text{True}$}
%                 \STATE $block\_value \gets x_1^{\text{block}}[\ell-1]$
%             \ELSE
%                 \STATE $block\_value \gets \max(x_2^{\text{block}}[\ell-1], M)$
%             \ENDIF
%             \FOR{$idx = end\_idx$ down to $start\_idx$}
%                 \STATE $x[idx] \gets block\_value$
%             \ENDFOR
%             \STATE $f \gets start\_idx - 1$
%         \ENDFOR
    
%         \STATE \textbf{return} $x$
%     \end{algorithmic}
% \end{algorithm}


To prove the first claim, we show that the parameters $P_b, S_b,$ and $\nu_b$ amount to
\begin{align*}
    \textstyle
    P_b = \sum_{j \in \calB(b)} \rho_j, ~ 
    S_b = \sum_{j \in \calB(b)} \vert{\mu_j}, ~ 
    \nu_b = \prox_{\sum_{j \in \calB(b)} \rho_j H_M}(|\mu_j|)
\end{align*}
for each block index $b$, where $\calB(b)$ denoting the set of indices in the $b$'th block. It is easy to verify that Algorithm~\ref{alg:up_and_down_block_algorithm_for_merging_in_PAVA} recursively computes $P_b$ and $S_b$. Thus, we will focus on $\nu_b$.
Note that the computation of the proximal operator in $\nu_b$ is reduced to solving a univariate optimization problem for each $b$ and satisfies
\begin{align*}
    \nu_b =& \argmin_{v \in \R} \sum_{j \in \calB(b)} \left( \frac{1}{2} (v - \vert{\mu_j})^2 + \rho_j H_M(v) \right) \\
    %= & \argmin_{v} \sum_{j \in \calB(b)} \left( \frac{1}{2} (v^2 - 2v\vert{\mu_j} + \mu_j^2) + \rho_j H_M(v) \right) \\
    = & \argmin_{v} \sum_{j \in \calB(b)} \left( \frac{1}{2} v^2 - v\vert{\mu_j} + \rho_j H_M(v) \right) \\
    %= & \argmin_{v} \left( \sum_{j \in \calB(b)} \frac{1}{2} v^2 - \sum_{j \in \calB(b)} v\vert{\mu_j} + \sum_{j \in \calB(b)} \rho_j H_M(v) \right) \\
    %= & \argmin_{v} \left( N_b \frac{1}{2} v^2 - S_b \vert{\mu_j} + P_b H_M(v) \right) \\
    = & \argmin_{v} \left( \frac{1}{2} v^2 - \frac{S_b}{N_b} \vert{\mu_j} + \frac{P_b}{N_b} H_M(v) \right) 
    = \argmin_{v} \left( \frac{1}{2} \left( v - \frac{S_b}{N_b} \right)^2 + \frac{P_b}{N_b} H_M(v) \right) 
    = \prox_{\frac{P_b}{N_b} H_{M}}(\frac{S_b}{N_b}).
\end{align*}

Thus, Algorithm~\ref{alg:up_and_down_block_algorithm_for_merging_in_PAVA} merges two adjacent blocks if the isotonic violation persists, and the output of the proximal operator is the minimizer of the univariate function in the merged block.
This is exactly the same as the objective in lines 11-14 of Algorithm~\ref{alg:PAVA_algorithm}. Hence, the first claim follows.

To show that the algorithm runs in linear time, notice that in the while loop $j \leq p$ in Algorithm~\ref{alg:up_and_down_block_algorithm_for_merging_in_PAVA}, the variable $j$ is incremented by $1$ in each iteration, and the loop terminates when $j = p$.
Although there are two while loops inside the main while loop, the total number of iterations in the two inner while loops is at most $p$.
This is because we start with $p$ blocks, and each iteration of the inner while loops either merges two blocks forward or merges two blocks backward.
The total number of merging operations is at most $p-1$.
Thus, the total number of iterations in the while loop $j \leq p$ is at most $p$.
Lastly, since we can evaluate the proximal operator of the Huber loss function, $\mathbf{H}_M$, in constant time complexity, the total time complexity of Algorithm~\ref{alg:up_and_down_block_algorithm_for_merging_in_PAVA} is $O(p)$.
\end{proof}

\subsection{Proof of Theorem~\ref{theorem:pava_algorithm_linear_time_complexity_and_exact_solution}}

\begin{namedtheorem}
    [~\ref{theorem:pava_algorithm_linear_time_complexity_and_exact_solution}]
    For any $\bmu \in \R^p$, Algorithm~\ref{alg:PAVA_algorithm} returns the \textit{exact} evaluation of $\prox_{\rho g^*}(\bmu)$ in $\tilde {\mathcal O}(p)$.
\end{namedtheorem}

\begin{proof}
    By Lemmas~\ref{lemma:equivalence_between_proximal_operator_and_huber_isotonic_regression} and~\ref{lemma:PAVA_algorithm_exact_solution}, the output of Algorithm~\ref{alg:PAVA_algorithm} computes $\prox_{\rho g^*}$ exactly. 
    The linear time complexity statement also holds thanks to Lemma~\ref{lemma:PAVA_merging_linear_time_complexity}.
\end{proof}

\subsection{Proof of Theorem~\ref{theorem:compute_g_value_algorithm_correctness}}

\begin{namedtheorem}
    [~\ref{theorem:compute_g_value_algorithm_correctness}]
        For any $\bbeta \in \R^p$, Algorithm~\ref{alg:compute_g_value_algorithm} computes the exact value of $g(\bbeta)$, defined in~\eqref{eq:function_g_definition}, in $\mathcal O(p + p \log k)$.
\end{namedtheorem}

\begin{proof}
% Add proof content here

We first show that the algorithm correctly computes the value of $g(\bbeta)$ and then analyze its computational complexity. Define the mixed-binary set
\begin{align*}
    \calS_0 = \left\{ (t, \bbeta) \;\middle|\; \textstyle \frac{1}{2} \sum_{j \in [p]} \beta_j^2 \leq t, \, \|\bbeta \|_\infty \leq M, \, \|\bbeta \|_0 \leq k \right\}.
\end{align*}
Using the perspective and big-M reformulation techniques, the set $\calS_0$ admits the equivalent representation
\begin{align*}
    \calS_0 = \left\{ (t, \bbeta) \;\middle|\; \exists \bz \in \{0,1\}^p ~ \st ~ \textstyle \frac{1}{2} \sum_{j \in [p]} \beta_j^2 / z_j \leq t, \, \bm 1^\top \bz \leq k, \, -M z_j \leq \beta_j \leq M z_j ~~ \forall j \in [p] \right\}.
\end{align*}
Following the proof of Lemma~\ref{lemma:equivalence_between_perspective_relaxation_and_convexification}, one can show that the closed convex hull of $\calS_0$ is given by 
\begin{align*}
    \cl \conv(\calS_0) = \left\{ (t, \bbeta) \;\middle|\; \exists \bz \in [0,1]^p ~ \st ~ \textstyle \frac{1}{2} \sum_{j \in [p]} \beta_j^2 / z_j \leq t, \, \bm 1^\top \bz \leq k, \, -M z_j \leq \beta_j \leq M z_j ~~ \forall j \in [p] \right\}.
\end{align*}
Therefore, the implicit function $g$ can be written as the evaluation of the support function of $\cl\conv(\calS_0)$ at $(1, \bm 0)$, that is,
\begin{align}
    \label{eq:g:S0}
    g(\bbeta) = \min  \{ t : (t, \bbeta) \in \cl\conv(\calS_0) \}.
\end{align}
Notice that the set $\calS_0$ is sign- and permutation-invariants. Hence, by ~\citep[Theorem~4]{kim2022convexification}, its closed convex hull admits the following (different) lifted represenation
\begin{align}
    \label{eq:diff:conv}
    \cl \conv(\calS_0) = \left\{ (t, \bbeta) \;\middle|\; \exists \bphi \in \R^p ~ \st ~
    \begin{array}{l}
        \frac{1}{2} \sum_{j \in [p]} \phi_j^2 \leq t, \, \vert{\bbeta} \preceq_m \bphi, \\
        0 \leq \phi_k \leq \ldots \leq \phi_1 \leq M, \\
        \phi_{k+1} = \phi_{k+2} = \ldots = \phi_n = 0 
    \end{array}
    \right\},
\end{align}
where the absolute value operator $\vert{\cdot}$ is applied to a vector in an element-wise fashion, and the constraint $\vert{\bbeta} \preceq_m \bphi$ denotes that $\bphi$ majorizes $\vert{\bbeta}$, that is,
\begin{align*}
    \textstyle \vert{\bbeta} \preceq_m \bphi  \quad \iff \quad  \sum_{j \in [l]} \vert{\beta_j} \leq \sum_{j \in [l]} \phi_j \quad \forall l \in [p-1] \quad \text{and} \quad \sum_{j \in [p]} \phi_j = \sum_{j \in [p]} \vert{\beta_j}.
\end{align*}
Using this alternative convex hull description of $\calS_0$ in~\eqref{eq:diff:conv} and the implicit formulation~\eqref{eq:g:S0}, we may conclude that
\begin{align}
    g(\bbeta) = \min\limits_{\bphi \in \R^p}
    \textstyle \left\{ \frac{1}{2} \sum_{j \in [p]} \phi_j^2 :  \vert{\bbeta} \preceq_m \bphi, \, 0 \leq \phi_k \leq \ldots \leq \phi_1 \leq M, \, \phi_{k+1} = \phi_{k+2} = \ldots = \phi_n = 0
    \right\}. \label{appendix_obj:compute_g_value_majorization_formulation}
\end{align}
In the following we show that Algorithm~\ref{alg:compute_g_value_algorithm} can efficiently solve the minimization problem in~\eqref{appendix_obj:compute_g_value_majorization_formulation}. At the first iteration $j=1$ of the algorithm, we have
\begin{align*}
    \textstyle k \phi_1 \geq \sum_{j \in [k]} \phi_j = \sum_{j \in [p]} \phi_j \geq \sum_{j \in [p]} \vert{\beta_j} \quad \Rightarrow \quad \phi_1 \geq \frac{1}{k} \sum_{j \in [p]} \vert{\beta_j}.
\end{align*}
At the same time, we also need to satisfy $\vert{\beta_1} \leq \phi_1$ from the first majorization constraint. We now discuss two cases
\begin{itemize}[label=$\diamond$,leftmargin=*]
    \item \textbf{Case 1:} If $\frac{1}{k} \sum_{j \in [p]} \vert{\beta_j} \geq \vert{\beta_1}$, in order to solve the minimization problem in~\eqref{appendix_obj:compute_g_value_majorization_formulation}, we set $\phi_1 = \frac{1}{k} \sum_{j=1}^n \vert{\beta_j}$. Notice that $\phi_1 \leq M$ is automatically satisfied because $\phi_1 = \frac{1}{k} \sum_{j \in [p]} \vert{\beta_j} = \frac{1}{k} \sum_{j \in [p]} M z_j \leq M$. This leads to $\phi_2 = \ldots = \phi_k = \frac{1}{k} \sum_{j \in [p]} \vert{\beta_j}$.
    To see this, for the sake of contradition, assume that $\exists j \in \{2, \ldots, k\}$ such that $\phi_j < \frac{1}{k} \sum_{j \in [p]} \vert{\beta_j}$. 
    Since $\phi_j \leq \phi_1 = \frac{1}{k} \sum_{j \in [p]} \vert{\beta_j}$, we have $\sum_{j \in [k]} \phi_j < \sum_{j \in [k]} \frac{1}{k} \sum_{j \in [p]} \vert{\beta_j} = \sum_{j \in [p]} \vert{\beta_j}$, which contradicts the majorization constraint.

    \item \textbf{Case 2:} If $\frac{1}{k} \sum_{j \in [n]} \vert{\beta_j} < \vert{\beta_1}$, we can set $\phi_1 = \vert{\beta_1}$. Notice that $\phi_1 \leq M$ is automatically satisfied because $\vert{\beta_1} \leq M z_1 \leq M$.
    Then we are left with $k-1$ coefficients to set, and we can follow the same argument as we did for $j=1$ with slight difference that the majorization constraints are changed to
    \begin{align*}
        \textstyle
        \sum_{j=2}^l \phi_j \geq \sum_{j=2}^l \vert{\beta_j} \quad \forall l \in \{2, \ldots, p-1\} \quad \text{and} \quad \sum_{j=2}^p \phi_j = \sum_{j=2}^p \vert{\beta_j}.
    \end{align*}
\end{itemize}
We repeat this process until we set all $k$ coefficients $\phi_1, \ldots, \phi_k$, as implemented by Algorithm~\ref{alg:compute_g_value_algorithm}.
The output of the algorithm coincides with the optimal value of the minimization problem in~\eqref{appendix_obj:compute_g_value_majorization_formulation}. Hence, the first claim follows.

As for the complexity claim, it is easy to see that Algorithm~\ref{alg:compute_g_value_algorithm}.
only requires partial sorting step on Line 2, which has a complexity of $\mathcal O(p \log k)$. The summation step on Line 3 has a complexity of $\mathcal O(p)$. The for-loop step on Line 4-8 has a complexity of $\mathcal O(k)$, so does the final summation step on Line 9. Therefore, the overall computational complexity of Algorithm~\ref{alg:compute_g_value_algorithm} is $\mathcal O(p + p \log k)$. This concludes the proof.
\end{proof}

\newpage
\section{Experimental Setup Details}
\label{appendix:experimental_setup}

\subsection{Setup for Evaluating Proximal Operators}
\label{appendix:setup_for_evaluating_proximal_operators}
The synthetic data generation process is as follows.
We sample the input vector $\bgamma \in \bbR^p$ from the standard multivariate Gaussian distribution, $\bgamma \sim \calN(\mathbf{0}, \bI_p)$, where $\bI_p$ denotes the identity matrix with dimension $p$.
We vary the dimension $p \in \{2^0, 2^1, ..., 2^{10}\} \times 10^2$ and set the cardinality $k$ to be $10$, the box constraint $M$ to be $1.0$, and the weight parameter $\rho$  to be $1.0$.
We report the running time for evaluating these proximal operators.
To obtain the mean and standard deviation of the running time, we repeat each setting 5 times, each with a different random seed.

\subsection{Setup for Solving the Perspective Relaxation}
\label{appendix:setup_for_solving_the_perspective_relaxation}

We generate our synthetic datasets in the following procedure.
First, we sample each feature vector $\bx_i \in \bbR^p $ from a Gaussian distribution, $\bx_i \sim \calN(\mathbf{0}, \bSigma)$, where the covariance matrix has entries $\Sigma_{jl} = \sigma^{\vert{j-l}}$.
The variable $\sigma \in (0, 1)$ controls the features correlation: if we increase $\sigma$, feature columns in the design matrix $\bX$ become more correlated.
Throughout the experimental section, we set $\sigma=0.5$.
Next, we create the sparse coefficient vector $\bbeta^*$ with $k$ equally spaced nonzero entries, where $\beta^*_j = 1$ if $j \text{ mod } (p/k) = 0$ and $\beta^*_j = 0$ otherwise.
After these two steps, we build the prediction vector $\by$.
If our loss function is squared error loss (regression task), we set $y_i = \bx_i^T \bbeta^* + \epsilon_i$, where $\epsilon_i$ is a Gaussian random noise with $\epsilon_i \sim \calN(0, \frac{\Vert{\bX \bbeta^*}}{\text{SNR}})$, and $\text{SNR}$ stands for the signal-to-noise ratio.
In all our experiments, we choose $\text{SNR}=5$.
If our loss function is logistic loss (classification task), we set $y_i \sim Bern(\bx_i^T \bbeta^* + \epsilon_i)$, where $Bern(P)$ is a Bernoulli random variable with $\bbP(y_i = 1) = P$ and $\bbP(y_i = -1) = 1 - P$.
For this experiment, we vary the feature dimension $p \in \{1000, 2000, 4000, 8000, 16000\}$.
We control the sample size by using a parameter called $n$-to-$p$ ratio, or sample to feature ratio.
For the results in the main paper, we set $n$-to-$p$ ratio to be $1.0$, the box constraint $M$ to be $2$, the number of nonzero coefficients k (also the cardinality constraint) to be $10$, and $\ell_2$ regularization coefficient $\lambda_2$ to be $1.0$.
Again, we report and compare the running times, with means and standard deviations calculated based on 5 repeated simulations with different random seeds.

\subsection{Setup for Certifying Optimality}
\label{appendix:setup_for_certifying_optimality}

\paragraph{Datasets and Preprocessing}
We run on both synthetic and real-world datasets.
For the synthetic datasets, we run on the largest synthetic instances ($n=16000$ and $p=16000$).
For the real-world datasets, we use the dataset cancer drug response~\cite{liu2020deepcdr} for linear regression and DOROTHEA~\cite{asuncion2007uci} for logistic regression.

The cancer drug response dataset has 822 samples and orginally has 34674 features.
However, many feature only has a single value, so we prune all these features, which result in 2200 features.
The DOROTHEA dataset has 1950 samples and 100000 features.
After pruning redundant features, we have 91598 features.

For both the cancer drug response and DOROTHEA dataset, we center each feature to have mean $0$ and norm equal to $1$.

\paragraph{Choice of Hyperparameters}
For the cardinality constraint $k$, we set $k=10$ for both synthetic datasets.
For the cancer drug response dataset, we set $k=5$.
For DOROTHEA, we set $k=15$.
In practice, this choice can be made more judiciously by doing 5 fold cross validation with a heuristic sparse learning algorithm first.
However, since our emphasis here is simply to compare certification speed, we just pick a variety of $k$'s.

For the $\ell_2$ regularization coefficient, we set $\lambda_2=1$.
For the box constraint, we set $M=2$ for the synthetic datasets and DOROTHEA.
The infinity norm of the final optimal solution less than this value.
For the cancer drug response dataset, we set $M=5$, which is also bigger than the infinity norm of the final optimal solution.

\paragraph{Branch and Bound}
For our method, we write a customized branch-and-bound (BnB) framework.
We use Algorithm~\ref{alg:main_algorithm} to solve the relaxation at each node and use Equation~\eqref{eq:fenchel_duality_theorem_F_y(Ax)+G(x)} to calculate the safe lower bound to prune the search space.
To find feasible solutions, we use an effective approach called beamsearch~\cite{liu2022fasterrisk} from the existing literature.
For branching, we branch on the feature based on the best feasible solution found by the beamsearch algorithm at each node.
For the nonzero coefficients of this solution, we branch on the variable which would lead to the largest loss increase if the coefficient to $0$.
The intuition is that such a variable is important and should be branched early in the BnB framework.

\subsection{Computing Platforms}
When investigating how much GPU can accelerate our computation, we run the experiments with both CPU and GPU implementations on the Nvidia RTXA5000s.
For everything else, we run the experiments with the CPU implementation on AMD Milan with CPU speed 2.45 Ghz and 8 cores.

\section{Additional Numerical Results}
\label{appendix:numerical}

% \subsection{Perturbation Study regarding Proximal Operators}
% \label{appendix:numerical_proximal_operators}

% \subsubsection{Perturbation Study on $k$ Values}

% \subsubsection{Perturbation Study on $M$ Values}

% \subsubsection{Perturbation Study on $\rho$ Values}

\subsection{Perturbation Study regarding Solving the Perspective Relaxation}
\label{appendix:numerical_solve_convex_relaxation}

\subsubsection{Perturbation Study on $M$ Values}

\begin{figure*}[!ht]
    \centering
    \includegraphics[width=0.9\textwidth]{sections/Plots/big_M_perturbation/convex_relaxation_comparison_n_p_ratio_1.0_M_1.2.png}
    \caption{Solve the perspective relaxation in Problem~\eqref{obj:original_sparse_problem_perspective_formulation_convex_relaxation}.
    We set $M=1.2$, $\lambda_2=1.0$, $n$-to-$p$ ratio to be 1.}
    \label{fig:solve_convex_relaxation_M_1.2_lambda2_1.0_n_p_ratio_1.0}
\end{figure*}

\begin{figure*}[!ht]
    \centering
    \includegraphics[width=0.9\textwidth]{sections/Plots/big_M_perturbation/convex_relaxation_comparison_n_p_ratio_1.0_M_1.5.png}
    \caption{Solve the perspective relaxation in Problem~\eqref{obj:original_sparse_problem_perspective_formulation_convex_relaxation}.
    We set $M=1.5$, $\lambda_2=1.0$, $n$-to-$p$ ratio to be 1.}
    \label{fig:solve_convex_relaxation_M_1.5_lambda2_1.0_n_p_ratio_1.0}
\end{figure*}

\begin{figure*}[!ht]
    \centering
    \includegraphics[width=0.9\textwidth]{sections/Plots/big_M_perturbation/convex_relaxation_comparison_n_p_ratio_1.0_M_3.0.png}
    \caption{Solve the perspective relaxation in Problem~\eqref{obj:original_sparse_problem_perspective_formulation_convex_relaxation}.
    We set $M=3.0$, $\lambda_2=1.0$, $n$-to-$p$ ratio to be 1.}
    \label{fig:solve_convex_relaxation_M_3.0_lambda2_1.0_n_p_ratio_1.0}
\end{figure*}

\begin{figure*}[!ht]
    \centering
    \includegraphics[width=0.9\textwidth]{sections/Plots/big_M_perturbation/convex_relaxation_comparison_n_p_ratio_1.0_M_5.0.png}
    \caption{Solve the perspective relaxation in Problem~\eqref{obj:original_sparse_problem_perspective_formulation_convex_relaxation}.
    We set $M=5.0$, $\lambda_2=1.0$, $n$-to-$p$ ratio to be 1.}
    \label{fig:solve_convex_relaxation_M_5.0_lambda2_1.0_n_p_ratio_1.0}
\end{figure*}

\begin{figure*}[!ht]
    \centering
    \includegraphics[width=0.9\textwidth]{sections/Plots/big_M_perturbation/convex_relaxation_comparison_n_p_ratio_1.0_M_10.0.png}
    \caption{Solve the perspective relaxation in Problem~\eqref{obj:original_sparse_problem_perspective_formulation_convex_relaxation}.
    We set $M=10.0$, $\lambda_2=1.0$, $n$-to-$p$ ratio to be 1.}
    \label{fig:solve_convex_relaxation_M_10.0_lambda2_1.0_n_p_ratio_1.0}
\end{figure*}

\newpage

\subsubsection{Perturbation Study on $\lambda_2$ Values}


\begin{figure*}[!ht]
    \centering
    \includegraphics[width=0.9\textwidth]{sections/Plots/lambda2_perturbation/convex_relaxation_comparison_lambda2_0.1.png}
    \caption{Solve the perspective relaxation in Problem~\eqref{obj:original_sparse_problem_perspective_formulation_convex_relaxation}.
    We set $M=2.0$, $\lambda_2=0.1$, $n$-to-$p$ ratio to be 1.}
    \label{fig:solve_convex_relaxation_M_2.0_lambda2_0.1_n_p_ratio_1.0}
\end{figure*}

\begin{figure*}[!ht]
    \centering
    \includegraphics[width=0.9\textwidth]{sections/Plots/lambda2_perturbation/convex_relaxation_comparison_lambda2_10.0.png}
    \caption{Solve the perspective relaxation in Problem~\eqref{obj:original_sparse_problem_perspective_formulation_convex_relaxation}.
    We set $M=2.0$, $\lambda_2=10.0$, $n$-to-$p$ ratio to be 1.}
    \label{fig:solve_convex_relaxation_M_2.0_lambda2_10.0_n_p_ratio_1.0}
\end{figure*}

\newpage

\subsubsection{Perturbation Study on $n$-to-$p$ Ratios}


\begin{figure*}[!ht]
    \centering
    \includegraphics[width=0.9\textwidth]{sections/Plots/n_p_ratio_perturbation/convex_relaxation_comparison_n_p_ratio_0.1_M_2.0.png}
    \caption{Solve the perspective relaxation in Problem~\eqref{obj:original_sparse_problem_perspective_formulation_convex_relaxation}.
    We set $M=2.0$, $\lambda_2=1.0$, $n$-to-$p$ ratio to be 10.0.}
    \label{fig:solve_convex_relaxation_M_2.0_lambda2_1.0_n_p_ratio_10.0}
\end{figure*}


\begin{figure*}[!ht]
    \centering
    \includegraphics[width=0.9\textwidth]{sections/Plots/n_p_ratio_perturbation/convex_relaxation_comparison_n_p_ratio_10.0_M_2.0.png}
    \caption{Solve the perspective relaxation in Problem~\eqref{obj:original_sparse_problem_perspective_formulation_convex_relaxation}.
    We set $M=2.0$, $\lambda_2=1.0$, $n$-to-$p$ ratio to be 0.1.}
    \label{fig:solve_convex_relaxation_M_2.0_lambda2_1.0_n_p_ratio_0.1}
\end{figure*}

\newpage

\section{Additional Discussions}
We first provide common calculus rules for conjugate functions, whose proof can be found in standard optimization textbooks such as~\citep{beck2017first}.

\begin{itemize}[label=$\diamond$,leftmargin=*]
    \item \textbf{Separable Sum Rule:} Let $f(\bx) = \sum_{j \in [p]} f_j(x_j)$, where $f_j: \R \rightarrow \R$ is convex for all $j \in [p]$. Then, the conjugate of $f$ is given by $f^*(\bmu) = \sum_{j \in [p]} f_j^*(\mu_j)$.
    
    \item \textbf{Scalar Multiplication Rule:}  Let $g : \R^p \to \R$ be convex and $\alpha > 0$ be a scalar. Then, the conjugate of $f(\bx) = \alpha g(\bx)$ is given by $f^*(\bmu) = \alpha g^*(\bmu/\alpha)$.
    
    \item \textbf{Addition to Affine Function Rule:} Let $g : \R^p \to \R$ be convex and $\ba, \bb \in \mathbb{R}^p$ be two vectors. Then, the conjugate of $f(\bx) = g(\bx) + \ba^\top\bx + b$ is given by $f^*(\bmu) = g^*(\bmu - \ba) - \bb$.
    
    \item \textbf{Composition with Invertible Linear Mapping Rule:} Let $g : \R^p \to \R$ be convex and $\bA \in \mathbb{R}^{p \times p}$ be an invertible matrix. Then, the convex conjugate of $f(\bx) = g(\bA \bx)$ is given by $f^*(\bmu) = g^*(\bA^{-\top} \bmu)$.
    
    \item \textbf{Infimal Convolution Rule:} Let $g, h : \R^p \to \R$ be convex. Then, the convex conjugate of $f(\bx) = \inf_{by} ~ g(\by) + h(\bx - \by)$ is given by $f^*(\bmu) = g^*(\bmu) + h^*(\bmu)$.
\end{itemize}
These rules are useful for discussions in~\ref{appendix_sec:convex_conjugate_for_GLM_loss_functions} and~\ref{appendix_sec:safe_lower_bound_more_discussions}.


\subsection{Convex Conjugate for GLM Loss Functions}
\label{appendix_sec:convex_conjugate_for_GLM_loss_functions}

The convex conjugates of some of GLM loss functions are summarized bellow.
\begin{itemize}[label=$\diamond$,leftmargin=*]
    \item \textbf{Linear Regression:} 
    $$F(\bX \bbeta) = \Vert{\bX \bbeta - \by}_2^2 \quad \& \quad F^*(-\bzeta) = \frac{1}{4} \Vert{\bzeta}_2^2 - \by^T \bzeta.$$
    \item \textbf{Logistic Regression:} 
    $$F(\bX \bbeta) = \sum_{i \in [n]} \log(1 + \exp(-y_i (\bX \bbeta)_i)) \quad \& \quad F^*(-\bzeta) = \sum_{i \in [n]} \left( 1- \frac{\zeta_i}{y_i} \right) \log \left( 1-\frac{\zeta_i}{y_i} \right) + \frac{\zeta_i}{y_i} \log \left( \frac{\zeta_i}{y_i} \right).$$ 
    \item \textbf{Poisson Regression:} 
    $$F(\bX \bbeta) = \sum_{i \in [n]} \left( \exp(\bX \bbeta)_i - y_i (\bX \bbeta)_i \right) \quad \& \quad F^*(-\bzeta) = \sum_{i \in [n]} h(-\zeta_i + y_i), $$
    where $h(z) = z \log(z) - z$ if $z > 0$ and $h(z)=0$ if $z = 0$.
    \item \textbf{Gamma Regression:}
    $$F(\bX \bbeta) = \sum_{i \in [n]} \left( y_i \exp(-(\bX \bbeta)_i) + (\bX \bbeta)_i\right) \quad \& \quad F^*(-\bzeta) = \sum_{i \in [n]} y_i h(\frac{1-\zeta_i}{y_i}), $$
    where $h(z) = z \log(z) - z$ if $z > 0$ and $h(z)=0$ if $z = 0$.
    \item \textbf{Squared Hinge Loss:}
    For binary classification with labels $y_i \in \{-1, +1\}$,
    $$F(\bX \bbeta) = \sum_{i \in [n]} \max(0, 1-y_i (\bX \bbeta)_i)^2 \quad \& \quad F^*(-\bzeta) = \sum_{i \in [n]}  h(- y_i \zeta_i),$$
    where $h(z) = z + \frac{z^2}{4}$ if $z \leq 0$ and $h(z)=\infty$ if $z > 0$.
    % \item \textbf{Multinomial Logistic Regression:}
    % For multiclass classification with $K$ classes with coefficients $\bbeta \in \mathbb{R}^{p \times K}$, let $y_{ik}$ be a binary indicator such that $y_{ik}=1$ if the $i$-th sample belongs to class $k$, and $y_{ik}=0$ otherwise.
    % $$F(\bX \bbeta) = \sum_{i \in [n]} \left( \log\left( \sum_{j=1}^K \exp((\bX \bbeta)_{ij}) \right) - \sum_{k=1}^K y_{ik} (\bX \bbeta)_{ik} \right) $$
    % $$F^*(-\bzeta) = \sum_{i \in [n]} \begin{cases}
    %     \sum_{k=1}^K (y_{ik} - \zeta_{ik}) \log(y_{ik} - \zeta_{ik}) & \text{if } \sum_{k=1}^K \zeta_{ik} = 0, \zeta_{ik} \le y_{ik} \text{ for all } k \\
    %     +\infty & \text{otherwise} \end{cases} $$
\end{itemize}




\subsection{Safe Lower Bound}
\label{appendix_sec:safe_lower_bound_more_discussions}

The linear regression problem with eigen-perspective relaxation is formulated as
\begin{align*}
    P^\star_{\text{eig-conv}} = \min_{\bbeta \in \R^p} \bbeta^\top \bQ_{\text{eig}} \bbeta - 2\by^\top \bX \bbeta +  2 \lambda_{\text{eig}} g(\bbeta),
\end{align*}
where $\bQ_{\text{eig}} = \bX^\top \bX - \lambda_{\text{min}}(\bX^\top \bX) \bI$, $\lambda_{\text{eig}} = \lambda_2 + \lambda_{\text{min}}(\bX^\top \bX)$, and $\lambda_{\text{min}}(\cdot)$ denotes minimum eigenvalue of the input matrix.
Using the standard version of weak duality theorem, we have
\begin{align*}
    P_{\text{MIP}}^\star \geq P_{\text{eig-conv}}^\star \geq - F^*(-\hat{\bzeta}) - G^*(\hat{\bzeta}),
\end{align*}
where $F(\bbeta)= \bbeta^\top \bQ_{\text{eig}} \bbeta$, $G(\bbeta) = -2\by^\top \bX \bbeta + 2 \lambda_{\text{eig}} g(\bbeta)$, and $\hat{\bzeta} = -\nabla F(\hat{\bbeta}) = -2\bQ_{\text{eig}} \hat{\bbeta}$.
The conjugate functions admit the following closed form expressions
\begin{align*}
    F^*(-\hat{\bzeta}) &= \frac{1}{4} \hat{\bzeta}^\top \bQ_{\text{eig}}^{\dagger} \hat{\bzeta} = \hat{\bbeta} \bQ_{\text{eig}} \hat{\bbeta} \quad \& \quad G^*(\hat{\bzeta}) = 2\lambda_{\text{eig}} \, g^* \left(\frac{-\bQ_{\text{eig}}\hat{\bbeta} +  \bX^\top \by}{\lambda_{\text{eig}}} \right), 
\end{align*}
where we use $(\cdot)^{\dagger}$ to denote the pseudo-inverse of a matrix. We may conclude that
\begin{align*}
    P_{\text{MIP}}^\star & \geq \hat{\bbeta} \bQ_{\text{eig}} \hat{\bbeta} + 2\lambda_{\text{eig}} \, g^* \left(\frac{-\bQ_{\text{eig}}\hat{\bbeta} +  \bX^\top \by}{\lambda_{\text{eig}}}\right).
\end{align*}
The above lower bound can be viewed as a generalization of the safe lower bound formula from~\citep[Theorem~3.1]{liu2024okridge}. Specifically, as $M$ approaches $\infty$, the above lower bound matches the lower bound in in~\citep[Theorem~3.1]{liu2024okridge}. 
Furthermore, Our proof uses a simple weak duality argument and is concise, in contrast to the lengthy two-page algebraic proof of~\citep[Theorem~3.1]{liu2024okridge}.


% % \bibliographystyle{plainnat}
% \bibliographystyle{unsrt} % Ensures citations appear in order
\bibliographystyle{jabbrv_ieeetr}
\bibliography{main}

% \bibliographystyle{./IEEEtranBST/IEEEtran}
% \bibliography{./IEEEtranBST/IEEEabrv, references}

% \documentclass[10pt,twocolumn,letterpaper]{article}
\usepackage[rebuttal]{cvpr}

% Include other packages here, before hyperref.
\usepackage{graphicx}
\usepackage{amsmath}
\usepackage{amssymb}
\usepackage{booktabs}

% Import additional packages in the preamble file, before hyperref
\newcommand{\CG}{\mathcal{G}\xspace}
\newcommand{\CV}{\mathcal{V}\xspace}
\newcommand{\CE}{\mathcal{E}\xspace}
\newcommand{\CA}{\mathcal{A}\xspace}
\newcommand{\CF}{\mathcal{F}\xspace}
\newcommand{\CR}{\mathcal{R}\xspace}
\newcommand{\CB}{\mathcal{B}\xspace}
\newcommand{\CX}{\mathcal{X}\xspace}
\newcommand{\CK}{\mathcal{K}\xspace}
\newcommand{\CM}{\mathcal{M}\xspace}
\newcommand{\CC}{\mathcal{C}\xspace}
\newcommand{\CL}{\mathcal{L}\xspace}
\newcommand{\CI}{\mathcal{I}\xspace}
\newcommand{\CQ}{\mathcal{Q}\xspace}
\newcommand{\CO}{\mathcal{O}\xspace}
\newcommand{\CP}{\mathcal{P}\xspace}
\newcommand{\CS}{\mathcal{S}\xspace}
\newcommand{\CT}{\mathcal{T}\xspace}
\newcommand{\CJ}{\mathcal{J}\xspace}
\usepackage[para]{footmisc}
\usepackage{subfig}
% \usepackage{subcaption}
% \usepackage{array}
% \usepackage{colortbl}



% If you comment hyperref and then uncomment it, you should delete
% egpaper.aux before re-running latex.  (Or just hit 'q' on the first latex
% run, let it finish, and you should be clear).
\definecolor{cvprblue}{rgb}{0.21,0.49,0.74}
\usepackage[pagebackref,breaklinks,colorlinks,citecolor=cvprblue]{hyperref}

% Support for easy cross-referencing
\usepackage[capitalize]{cleveref}
\crefname{section}{Sec.}{Secs.}
\Crefname{section}{Section}{Sections}
\Crefname{table}{Table}{Tables}
\crefname{table}{Tab.}{Tabs.}

% If you wish to avoid re-using figure, table, and equation numbers from
% the main paper, please uncomment the following and change the numbers
% appropriately.
%\setcounter{figure}{2}
%\setcounter{table}{1}
%\setcounter{equation}{2}

% If you wish to avoid re-using reference numbers from the main paper,
% please uncomment the following and change the counter for `enumiv' to
% the number of references you have in the main paper (here, 6).
%\let\oldthebibliography=\thebibliography
%\let\oldendthebibliography=\endthebibliography
%\renewenvironment{thebibliography}[1]{%
%     \oldthebibliography{#1}%
%     \setcounter{enumiv}{6}%
%}{\oldendthebibliography}


%%%%%%%%% PAPER ID  - PLEASE UPDATE
\def\paperID{*****} % *** Enter the Paper ID here
\def\confName{CVPR}
\def\confYear{2023}
\newcommand{\Ours}{\textsc{GraphGPT-o}\xspace}
\begin{document}

%%%%%%%%% TITLE - PLEASE UPDATE
\title{\Ours: Synergistic Multimodal Comprehension and Generation on Graphs}

\maketitle
\thispagestyle{empty}
\appendix

%%%%%%%%% BODY TEXT - ENTER YOUR RESPONSE BELOW
\section{To Reviewer 66KC}
\textbf{\textit{Question 1:}} PageRank is a classic algorithm for graphs, but the paper could explore more modern techniques, such as graph neural networks, for graph token extraction.

\noindent
\textbf{\textit{Answer 1:}} In \Ours, the PageRank method is used for neighbor selection. Surely it can be refined by introducing some more advanced methods, such as sampling based on textual or visual similarity. We leave this part for future work to make the process more accurate and more controllable. As for graph neural networks, we did some experiments replacing our hierarchical q-former with it, and the result is shown below. \\



\noindent
\textbf{\textit{Question 2:}} In Table 1, several image-only results outperform other methods. This outcome needs further explanation.

\noindent
\textbf{\textit{Answer 2:}} Thank you for pointing this out—it’s indeed a fascinating result. There are two main reasons for this observation: \textbf{First,} it occurs in the Beauty dataset, where the textual information often appears in forms like \textbf{\textit{"Victoria's Secret Dream Angels Heavenly Body Mist 8.4 Oz (250 ML)"}}, which may confuse the MLLM. \textbf{Secondly,} the original MLLM backbone may have limitations in effectively processing long sequences. 
\noindent
\newline
This raises an interesting research question for future work: for each node, how can we adaptively sample or select the most suitable modality for the task at hand? \\

\noindent
\textbf{\textit{Question 3\&4:}} The paper ID is missing. The paper’s overall formatting could be improved for better readability, such as the placement of Figure 4 and Figure 5, which are too far from their corresponding text.

\noindent
\textbf{\textit{Answer 3\&4:}} We are sorry to miss the ID part the formatting. We will refine these in the later version. \\

\section{To Reviewer CKyo}
\noindent
\textbf{\textit{Question 1:}} Intrinsically, it is still an visual conditional MLLM. A key difference is that this work sample visual conditions from graph data and via a certain sampling method. Would the sampling method significantly effect the generation performance?

\noindent
\textbf{\textit{Answer 1:}} Yes, the sampling method is rather important. We compared the results of different sampling strategies in Figure 3. Moreover, the sampling method could be important to make graph for generation more controllable.\\

\noindent
\textbf{\textit{Question 2:}} Beside of sampling method, any novelty in MLLM? Would be the MLLM part replaced by any SOTA MLLMs? 

\noindent
\textbf{\textit{Answer 2:}} Our goal is to introduce a plug-and-play component that seamlessly integrates with all SOTA MLLMs. The novelty lies in the hierarchical tokenization approach, which aligns semantic information across text, image, and graph modalities, enabling richer and more cohesive representations. This component is designed to be both easy to implement and train, making it a versatile addition to existing MLLMs.\\

\noindent
\textbf{\textit{Question 3:}} A node will carry multiple images/captions? The token length for a node will vary a lot?

\noindent
\textbf{\textit{Answer 3:}} A node is associated with only one image and one textual description. The token length within one dataset will not vary a lot. \\

\noindent
\textbf{\textit{Question 4:}} For neighbors, are you only using first nearest neighbors? What if including second or ever further neightbors?

\noindent
\textbf{\textit{Answer 4:}} For neighbors sampling, we used PageRank to sample neighbors. Some 2-hop or multi-hop neighbors might be sampled.

\section{To Reviewer X3yz}
\noindent
\textbf{\textit{Question 1:}} The ablation study includes only the hierarchical aligner module and the Personalized PageRank sampling method. The ablation studies for different approaches to graph linearization and inference strategies are missing.

\noindent
\textbf{\textit{Answer 1:}} We demonstrate the results of graph linearization in Table 1. And based on the results, we choose to input both text and image modalities in the order of text-first and also text-first during inference. \\

\noindent
\textbf{\textit{Question 2:}} It appears that there are too few methods compared in the study.

\noindent
\textbf{\textit{Answer 2:}} We had added two more baselines and the results are shown below.\\

\noindent
\textbf{\textit{Question 3:}} The qualitative results are insufficient.


\noindent
\textbf{\textit{Answer 3:}}


\end{document}


\end{document}


