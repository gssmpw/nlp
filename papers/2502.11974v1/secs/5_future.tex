\section{Open Problems and Promising Directions} \label{sec:experiment}
\subsection{Open Problems}
\nibf{Trading-off high-fidelity reconstruction and controllable editing.} 
GAN/Diffusion inversion often suffers from ``overfitting" the latent space when pursuing pixel-level high-fidelity reconstruction, which sacrifices the ability for subsequent controllable editing, and vice versa. There is a need to define more reasonable latent space constraints while exploring more expressive and generalized large-scale generative models trained on large datasets. 

\nibf{Computational efficiency bottleneck.} 
Although GAN inversion via encoders is faster, it often sacrifices precision. In contrast, diffusion model inversion achieves better results but is computationally expensive due to iterative optimization, especially for high-resolution generation. This increases training/inference costs and limits applications in resource-constrained scenarios.

\nibf{Challenges in multimodal and cross-domain inversion}
Existing inversion methods are primarily focused on single modalities and struggle to generalize to multimodal (\textit{e.g.}, image-to-text) or cross-domain tasks (\textit{e.g.}, medical image to natural image). Developing universal inversion frameworks capable of handling multimodal inputs while maintaining semantic consistency is critical.

\nibf{Data bias and fairness issues}
Generative model inversion may produce biased or unfair results due to data imbalances, such as under- or over-representing specific racial or gender attributes. Introducing fairness constraints into the inversion process to mitigate bias is an urgent problem to address.



\subsection{Promising Directions}

\nibf{Multi-scale latent space representation learning.} 
By introducing multi-scale latent space representations (\textit{e.g.}, combining local and global encodings), a better balance between high-fidelity reconstruction and controllable editing can be achieved. Further research on defining semantically consistent latent space representations will be crucial.

\nibf{Efficient optimization and inference algorithms.}
To address the iterative optimization bottleneck, lightweight optimization modules or efficient algorithms based on approximate inference (\textit{e.g.}, explicit latent space mapping or hierarchical optimization methods) can significantly reduce computational costs and improve efficiency in high-resolution scenarios.

\nibf{Multimodal consistency inversion techniques.}
Developing inversion frameworks that support multimodal inputs, such as jointly optimizing the latent spaces of images and texts, can enhance performance in cross-modal tasks. Ensuring semantic consistency in multimodal generation tasks (\textit{e.g.}, image-to-text-to-image) will be a key focus.

\nibf{Autoregressive image generation inversion framework.}
Recent autoregressive-based image and video generation has gained attention. Compared to diffusion models, autoregressive models offer higher efficiency and are expected to become one of next-generation foundational models.

\nibf{Inversion research based on Chain-of-Thought (CoT).}
CoT technology significantly enhances the reasoning capabilities of large language models. The incorporation of reinforcement learning enables the generation of results that better align with human thought processes. Recently, this technology has also been extended to the field of image generation. Inversion schemes based on CoT can further improve the quality and controllability of generated images. 

