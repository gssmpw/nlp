\begin{abstract}
    Image inversion is a fundamental task in generative models, aiming to map images back to their latent representations to enable downstream applications such as editing, restoration, and style transfer. This paper provides a comprehensive review of the latest advancements in image inversion techniques, focusing on two main paradigms: Generative Adversarial Network (GAN) inversion and diffusion model inversion. We categorize these techniques based on their optimization methods. For GAN inversion, we systematically classify existing methods into encoder-based approaches, latent optimization approaches, and hybrid approaches, analyzing their theoretical foundations, technical innovations, and practical trade-offs. For diffusion model inversion, we explore training-free strategies, fine-tuning methods, and the design of additional trainable modules, highlighting their unique advantages and limitations. Additionally, we discuss several popular downstream applications and emerging applications beyond image tasks, identifying current challenges and future research directions. By synthesizing the latest developments, this paper aims to provide researchers and practitioners with a valuable reference resource, promoting further advancements in the field of image inversion. 
    We keep track of the latest works at \href{https://github.com/RyanChenYN/ImageInversion}{\textcolor{magenta}{https://github.com/RyanChenYN/ImageInversion}}.
\end{abstract}

