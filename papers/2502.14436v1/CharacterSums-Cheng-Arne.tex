% ----------------------------------------------------------------
% AMS-LaTeX Paper ************************************************
% **** -----------------------------------------------------------
\documentclass[10pt,reqno]{amsart}
\usepackage{bbm}
\usepackage{tabu}
\usepackage{amsmath}
\usepackage{mathrsfs}
\usepackage{bm}
\usepackage{enumerate}
\usepackage{amsfonts} %%% i.e. use 12pt type
\textwidth=13.5cm %%% in the preamble; this will require
%%% adjusting the layouot of some wide formulas
\baselineskip=17pt %%% after \begin{document}
%\documentclass{amsart}
\usepackage{graphicx,latexsym,bm,amsmath,amssymb,verbatim,multicol,lscape}
\usepackage{enumitem}
% ----------------------------------------------------------------
\vfuzz2pt % Don't report over-full v-boxes if over-edge is small
\hfuzz2pt % Don't report over-full h-boxes if over-edge is small

% THEOREMS -------------------------------------------------------
\newtheorem{thm}{Theorem} [section]
\newtheorem{prob}[thm]{Problem}
\newtheorem{cor}[thm]{Corollary}
\newtheorem{lem}[thm]{Lemma}
\newtheorem{prop}[thm]{Proposition}
\newtheorem{fac}[thm]{Fact}
\theoremstyle{definition}
\newtheorem{defn}[thm]{Definition}
\theoremstyle{remark}
\newtheorem{rem}[thm]{Remark}
\newtheorem{exa}[thm]{Example}
\newtheorem{con}[thm]{Conjecture}


\numberwithin{equation}{section}
% MATH -----------------------------------------------------------
\newcommand{\norm}[1]{\left\Vert#1\right\Vert}
\newcommand{\abs}[1]{\left\vert#1\right\vert}
\newcommand{\set}[1]{\left\{#1\right\}}
\newcommand{\Real}{\mathbb R}
\newcommand{\eps}{\varepsilon}
\newcommand{\To}{\longrightarrow}
\newcommand{\BX}{\mathbf{B}(X)}
\newcommand{\A}{\mathcal{A}}
\newcommand{\SSS}{\stackrel}
%\renewcommand{\baselinestretch}{3}


\newcommand{\fq}{{\mathbb F}_{q}}
\newcommand{\fqr}{{\mathbb F}_{q^r}}
\newcommand{\fqk}{{\mathbb F}_{q^k}}
\newcommand{\fp}{{\mathbb F}_{p}}
\newcommand{\fpn}{{\mathbb F}_{p^n}}
\newcommand{\fqn}{{\mathbb F}_{q^n}}
\newcommand{\fqd}{{\mathbb F}_{q^d}}
\newcommand{\fqq}{{\mathbb F}_{q^2}}
\newcommand{\fpp}{{\mathbb F}_{p^2}}
\newcommand{\ftwo}{{\mathbb F}_{2}}
\newcommand{\fth}{{\mathbb F}_{3}}
\newcommand{\ftwon}{{\mathbb F}_{2^n}}
\newcommand{\ftwom}{{\mathbb F}_{2^m}}
\newcommand{\fntwo}{{\mathbb F}^{\, n}_{2}}
\newcommand{\Fbar}{\overline{\mathbb F}}
\newcommand{\barF}{\overline{\mathbb F}}
\newcommand{\Gbar}{\overline{G}}
\newcommand{\diff}{\partial}

\newcommand{\fy}{{\mathbb F}[y]}
\newcommand{\fxy}{{\mathbb F}[x,y]}
\newcommand{\qxy}{{\mathbb Q}[x,y]}
\newcommand{\fbxy}{{\overline{\mathbb F}}[x,y]}
\newcommand{\dx}{\partial x}
\newcommand{\dy}{\partial y}

\newcommand{\N}{{\mathbb N}}
\newcommand{\G}{{\mathcal G}}
\newcommand{\M}{{\mathcal M}}
\newcommand{\B}{{\mathcal B}}
\newcommand{\Z}{{\mathbb Z}}
\newcommand{\R}{{\mathbb R}}
\newcommand{\F}{{\mathcal F}}
\newcommand{\C}{{\mathbb C}}
\newcommand{\K}{{\mathbb K}}
\newcommand{\LL}{{\mathbb L}}
\newcommand{\Q}{{\mathbb Q}}

\newcommand{\fx}{{\mathbb F}[x]}
\newcommand{\qx}{{\mathbb{Q}[x]}}
\newcommand{\fqx}{{\mathbb F}_{q}[x]}
\newcommand{\fqxy}{{\mathbb F}_{q}[x,y]}
\newcommand{\fpoly}{{\mathbb F}[x_1, \dots, x_n]}
\newcommand{\ftwopoly}{{\mathbb F}_{2}[x_1, \dots, x_n]}
\newcommand{\fnpoly}{{\mathbb F}[x_1, \dots, x_{n-1}]}

\newcommand{\bR}{{\mathbf R}}
\newcommand{\bI}{{\mathbf I}}
\newcommand{\bJ}{{\mathbf J}}
\newcommand{\bx}{{\mathbf x}}
\newcommand{\bi}{{\mathbf i}}
\newcommand{\bj}{{\mathbf j}}
\newcommand{\bu}{{\mathbf u}}
\newcommand{\bv}{{\mathbf v}}
\newcommand{\bw}{{\mathbf w}}

\newcommand{\dft}{\mbox{DFT}}
\newcommand{\idft}{\mbox{iDFT}}
\newcommand{\Char}{\mbox{Char}}
\newcommand{\Oh}{{\mathcal O}}
\newcommand{\lt}{\mbox{lt}}
\newcommand{\lm}{\mbox{lm}}
\newcommand{\lc}{\mbox{lc}}
\newcommand{\lcm}{\mbox{lcm}}
\newcommand{\Ker}{\mbox{Ker}}
\newcommand{\mdeg}{\mbox{mdeg}}
\newcommand{\bdeg}{\mbox{bdeg}}
\newcommand{\Span}{\mbox{Span}}
\newcommand{\supp}{\mbox{supp}}
\newcommand{\Supp}{\mbox{supp}}
\newcommand{\prb}{\mbox{Prob}}
\newcommand{\Res}{\mbox{Res}}
\newcommand{\Tr}{\mbox{Tr}}
\newcommand{\wt}{\mbox{wt}}
\newcommand{\taylor}{\mbox{Taylor}}

\newcommand{\bIbar}{\overline{{\mathbf I}}} %Nate Black added this
\newcommand{\fbarpoly}{\overline{\mathbb F}[x_1, \dots, x_n]} % Nate Black added this

\newcommand{\classP}{\mathcal{P}}
\newcommand{\classNP}{\mathcal{NP}}
\newcommand{\classCONP}{\mbox{co-}\mathcal{NP}}
\newcommand{\classNPC}{\mathcal{NP}\textrm{-complete}}
\newcommand{\classRP}{\mathcal{RP}}
\newcommand{\perm}{\textrm{Perm}}

\newcommand{\pf}{{\em Proof.\ }}
\newcommand{\rmv}[1]{}

\def\<{\left\langle}
\def\>{\right\rangle}
\def\lT{\vartriangleleft}
\def\tbe{\textbf{e}}
\def\sk{\textbf{s}_k}



% ----------------------------------------------------------------
\begin{document}

\title[New estimates for character sums over sparse elements of finite fields]
{New estimates for character sums over sparse elements of finite fields}%
\author{Kaimin Cheng}
%Address of record for the research reported here
\address{School of Mathematics and Information, China West Normal University, Nanchong, 637002, P. R. China}
\email{ckm20@126.com,kcheng1020@gmail.com}
\author{Arne Winterhof}
%    Address of record for the research reported here
\address{Johann Radon Institute for Computational and Applied Mathematics, Austrian Academy of Science, Linz 4040, Austria}
\email{arne.winterhof@oeaw.ac.at}
%\thanks{This work was supported partially by the Natural Science Foundation of China (No. 12226335) and the China Scholarship Council Foundation(No. 202301010002).}
\subjclass{Primary 11T23}%
\keywords{Finite fields, Character sums, Sparse elements.}
%\subjclass[2010]{Primary 11P05}
\date{\today}%zxzsasszxsmkht5rrfc  vasv
%\dedicatory{}%
%\commby{}%
% ----------------------------------------------------------------
\begin{abstract}
Let $q$ be a prime power and $r$ a positive integer. Let $\fq$ be the finite field with $q$ elements, and let $\fqr$ be its extension field of degree $r$. Let $\chi$ be a nontrivial multiplicative character of $\fqr$. In this paper, we provide new estimates for the character sums
$\sum_{g \in\G}\chi(f(g))$,
where $\G$ is a given sparse subsets of $\fqr$ and $f(X)$ is a polynomial over $\fqr$ of certain type. Specifically, by extending a sum over sparse subsets to subfields, rather than to general linear spaces, we obtain significant improvements of previous estimates.
\end{abstract}



\maketitle
\section{Introduction}
Character sums are fundamental tools in number theory and finite field theory. In pure mathematics, character sums over sparse subsets of finite fields are of particular interest. Unlike the entire field, sparse subsets often lack well-behaved algebraic structures and properties, and their size and structural characteristics present unique analytical challenges, making the task of obtaining sharp estimates significantly more difficult. 

Let $q$ be a power of a prime $p$ and $\fq$ be the finite field with $q$ elements. Let $r$ be a positive integer and $\fqr$ be the extension field over $\fq$ of degree $r$. Let $\chi$ be a nontrivial multiplicative character of $\fqr$, $\G$ be a certain subset of sparse elements of $\fqr$ and $f(X)$ be a monic polynomial of $\fqr[X]$. Let  $S(\mathcal{G},\chi,f)$ be the character sum defined by 
$$S(\mathcal{G},\chi,f)=\sum_{\gamma\in\mathcal{G}}\chi(f(\gamma)),$$
and set $S(\mathcal{G},\chi)=S(\mathcal{G},\chi,f)|_{f(X)=X}$. Recently, Grzywaczyk and Winterhof \cite{[GW]} provided an estimate for $S(\G_1,\chi)$, where $\G_1$ is a set of elements in $\fqr$ avoiding affine hyperplanes, thereby resolving a problem posed by Fernandes and Reis \cite{[FR]}. M\'{e}rai, Shparlinski, and Winterhof \cite{[MSW]} presented a bound for $S(\G_2,\chi,f)$, where $\G_2$ is a subset of $\fqr$ with a fixed weight. For a more detailed discussion on character sums over finite fields, we refer the reader to the works such as \cite{[AS],[FR]}, as well as \cite{[IS],[Wan],[Win]}.

In this paper, we focus on estimating the character sum  $S(\G, \chi, f)$ for two classes of sparse subsets $\G$ of $\fqr$, where $f(X)$ is a polynomial of a certain form over $\fqr$. Moreover, by applying Wan’s bound on character sums we provide significant improvements over previous results on character sums over sparse subsets of finite fields by leveraging subfields rather than general linear spaces. Our contributions can be summarized as follows.
\begin{itemize}[left=1.5em] 
\item $\G$ with Restricted Coordinates (Section 3):
We study the first class of subsets consisting of elements whose coordinates are restricted with respect to a given basis of $\fqr$ over $\fq$. For these subsets, we derive a general estimate for the character sum. A direct application of this result leads to a substantial improvement of the work by Fernandes and Reis \cite{[FR]}, Iyer and Shparlinski \cite{[IS]}, and Grzywaczyk and Winterhof \cite{[GW]}. Specifically, when $r$ is even, we strengthen their results by proving that the finite field $\fqr$ avoiding affine hyperplanes  must contain a primitive element, even in the case when $q=3$, where their original result falls short.
\item $\G$ with $s$-sparse elements (Section 4): We study the second class of subsets consisting of elements with $s$-sparse representations with respect to a given basis of $\fqr$ over $\fq$. For even integer $r$, we improve the bound of M\'{e}rai, Shparlinski, and Winterhof \cite{[MSW]} by providing sharper estimates, particularly for the range  $0.13 < \rho < 0.32$. Our improvements are evident in this interval, where our estimates outperform both the trivial bound and previous estimates.
\end{itemize}
\section{Preliminaries}
In this section, we introduce some preliminary concepts and key results that will be used throughout the paper. Let $q$ be a power of a prime $p$ and  $\fq$ denote the finite field with $q$ elements. Let  $d\geq 2$ be a fixed positive integer, and let $r$ be a positive integer such that $d$ divides $r$. Set $k=\frac{r}{d}$. We work with the extension fields $\fqr$ and  $\fqk$, where $\fqk$ is a subfield of $\fqr$. Naturally,  $\fqk$ is also a subspace of  $\fqr$ over $\fq$.

Throughout the paper, we will assume that $\alpha_1,\ldots,\alpha_k$ is a basis of $\fqk$ over $\fq$, and $\alpha_1,\ldots,\alpha_r$ a completion of $\alpha_1,\ldots,\alpha_k$ to a basis of $\fqr$ over $\fq$. We begin by presenting a result of Wan \cite{[Wan]}, which will serve as a foundation for further estimates.
\begin{lem}\label{lem1.1} \cite[Corollary 2.4]{[Wan]}
Let $\chi_1,\dots,\chi_n$ be nontrivial multiplicative characters of $\fqr$. Let $f_1(X),\ldots,f_n(X)$ be pairwise prime polynomials over $\fqr$. Let  $D$ be the degree of the largest squarefree divisor of $\prod_{i=1}^nf_i(X)$. Suppose that for some $1\le i\le n$, there is a root $\xi_i$ with multiplicity $m_i$ of $f_i(X)$ such that the character $\chi_i^{m_i}$ is nontrival on the set $Norm_{\fqr[\xi_i]/\fqr}(\fq[\xi_i])$. Then, we have the estimate
$$\Big|\sum_{a\in\fq}\chi_1(f_1(a))\cdots\chi_n(f_n(a))\Big|\le (rD-1)q^{\frac{1}{2}}.$$
\end{lem}
We now derive a corollary from Lemma \ref{lem1.1}, which will play a key role in our estimates.
\begin{cor}\label{cor1.2}
Let $\chi$ be a nontrivial multiplicative character of $\fqr$, and $\overline{\chi}$ be the conjugated character of $\chi$. Let $f(X)$ be a monic polynomial over $\fqr$ having a simple root in $\fqr$.  Let $\beta_1$ and $\beta_2\in\fqr$ satisfy that $\gcd(f(X+\beta_1),f(X+\beta_2))=1$, and for a simple root $\xi$ of $f(X)$ in $\fqr$, either $\xi-\beta_1$ or $\xi-\beta_2$ is a defining element of $\fqr$ over $\fq$. Then, we have the estimate
$$\Big|\sum_{a\in\fq}\chi(f(a+\beta_1))\overline{\chi}(f(a+\beta_2))\Big|\le(2rD-1)q^{\frac{1}{2}},$$
where $D$ is the degree of the largest squarefree divisor of $f(X)$.
\end{cor}
\begin{proof}
Let $f_1(X)=f(X+\beta_1)$, $f_2(X)=f(X+\beta_2)$, so that $f_1(X)$ and $f_2(X)$ are coprime. since $\xi-\beta_i$ is a simple root of $f_i(X+\beta_i)$ for some $i$, and 
${\rm Norm}_{\fqr[\xi-\beta_i]/\fqr}(\fq[\xi-\beta_i])={\rm Norm}_{\fqr/\fqr}(\fqr)=\fqr$, the character $\chi$ is nontrivial on this set. Therefore, applying \ref{lem1.1} yields the desired result.
\end{proof}
To make better use of Corollary \ref{cor1.2}, we need to carefully analyze the relationship between the parameters  $\beta_1, \beta_2$ , and the polynomial $f(X)$. Specifically, given a polynomial $f(X)\in\fqr[X]$ and a divisor $k$ of $r$, we aim to identify all pairs $\beta_1,\beta_2\in\fqr$ that satisfy the conditions of Corollary \ref{cor1.2}. 

For a given $f(X)$ and a divisor $k$ of $r$, we restate the conditions that $\beta_1$ and $\beta_2$ must satisfy, which we will refer to as Condition $A_{k,r}(f)$, as follows:
\begin{itemize}[left=1.5em] 
\item $\gcd(f(X+\beta_1), f(X+\beta_2)) = 1$; and
\item there exists a simple root $\xi$ of $f(X)$ in $\fqr$ such that at least one of $\xi-\beta_1$ or $\xi-\beta_2$ is a defining element of $\fqr$ over $\fqk$.
 \end{itemize}
 Next, let $\F_{k,r}$ denote the set of monic polynomials $f(X)$ over $\fqk$ that satisfy the following conditions:
\begin{itemize}[left=1.5em] 
\item $f(X)$ has a simple root $\xi\in\fqk$; 
\item all the roots in $\fqr$ of $f(X)$ belong to $\fqk$; and
\item the degrees of all nonlinear irreducible factors of  $f(X)$ are distinct and none of them is divisible by $p$. 
 \end{itemize}
We can then identify a specific class of polynomials $f(X)$ for which we can determine all $\beta_1$ and $\beta_2$ that satisfy Condition $A_{k, r}(f)$.
 \begin{lem}\label{lem1.3}
 Let $r$ be an even number, $k=\frac{r}{2}$ and $f(X)\in\F_{k,r}$. For any $\beta_1,\beta_2\in\fqr\setminus\fqk^*$, the $\beta_1,\beta_2$ satisfy Condition $A_{k,r}(f)$ if and only if $\beta_1\ne\beta_2$.
 \end{lem}
 \begin{proof}
The necessity is straightforward. For the sufficiency, consider $\B:=\fqr\setminus\fqk^*$, and $\beta_1,\beta_2\in \B$ with $\beta_1\ne \beta_2$. We can assume without loss generality that $\beta_1\ne 0$. It implies that $\xi-\beta_1\not\in\fqk$, as $\xi\in\fqk$. Since $\xi-\beta_1\in\fqr$ and $[\fqr:\fqk]=2$. It follows that $\xi-\beta_1$ is of degree two over $\fqk$, and therefore it is a defining element of $\fqr$ over $\fqk$. 

To show that $\gcd(f(X+\beta_1),f(X+\beta_2))=1$, let
$$f(X)=(X-\xi)(X-c_1)^{h_1}\cdots(X-c_s)^{h_s}p_1^{e_1}(X)\cdots p_t^{e_t}(X),$$
where $\xi,c_1,\ldots,c_s$ are distinct elements of $\fqk$, and $p_i(x)$ are nonlinear irreducible factors of $f(X)$, all with distinct degrees not divisible by $p$. Let $c_0=\xi$. Then, since $\beta_1-\beta_2\in \B\setminus\{0\}$ and $c_i-c_j\in\fqk$ for any $0\le i,j\le s$, it follows that $\beta_1-\beta_2\ne c_i-c_j\in\fqk$ for any $0\le i,j\le s$. Thus, $f(X+\beta_1)$ and $f(X+\beta_2)$ share no common linear factor. 

Now, suppose there exists a common factor of $f(X+\beta_1)$ and $f(X+\beta_2)$ of degree greater than one. Let $p_i(X+\beta_1)$ be an irreducible divisor of this common factor of degree $n_i$, where $1\le i\le t$. Then $p_i(X+\beta_1)=p_j(X+\beta_2)$ for some $1\le j\le t$, so $p_i(X+\beta_1)=p_i(X+\beta_2)$ since $\deg(p_i(X))\ne\deg(p_j(X))$ for any $j\ne i$. Let $\alpha$ be a root of $g_i(X)$ in the extension field $\mathbb{F}_{q^{rn_i}}$. Then $\alpha-\beta_1$ and $\alpha-\beta_2$ are roots of $g_1(X+\beta_1)$ and $g_i(X+\beta_2)$, respectively. It follows that ${\rm Tr}(\alpha-\beta_1)={\rm Tr}(\alpha-\beta_2)$, where ${\rm Tr}$ is the trace function from $\mathbb{F}_{q^{rn_i}}$ onto $\fqr$. Therefore,  ${\rm Tr}(\beta_1-\beta_2)=0$, but since $\beta_1-\beta_2\in\fqr^*$ and $p\nmid n_i$, we conclude that ${\rm Tr}(\beta_1-\beta_2)=n_i(\beta_1-\beta_2)\ne 0$, a contradiction. Hence, $\gcd(f(X+\beta_1),f(X+\beta_2))=1$, and the proof of Lemma \ref{lem1.3} is complete.
 \end{proof}

\section{$\mathcal{G}$ being a subset of elements with restricted coordinates}
In this section, we focus on the first class of subsets, which consists of elements of $\fqr$ with restricted coordinates relative to a given basis of  $\fqr$ over $\fq$. We derive a general estimate for the character sum over these subsets, and we demonstrate its usefulness through a direct application, showing significant improvements over previous results.

Let $d\ge 2$ be a fixed integer and let $r$ be a positive integer such that $r$ is a multiple of $d$. Set $k=\frac{r}{d}$. Let $\A=\{\mathcal{A}_i\}_{i=1}^r$ be a given family of subsets of $\fq$. In this section, we define the subset $\mathcal{G}_{\A}$ as follows
$$\mathcal{G}_{\A}=\{a_1\alpha_1+\cdots+a_r\alpha_r: a_1\in\mathcal{A}_1,\ldots, a_r\in\mathcal{A}_r \}.$$
We can split this set into two components
$$V=\{a_1\alpha_1+\cdots+a_k\alpha_k: a_1\in\mathcal{A}_1,\ldots, a_k\in\mathcal{A}_k \}$$
and 
$$W=\{a_{k+1}\alpha_{k+1}+\cdots+a_r\alpha_r: a_{k+1}\in\mathcal{A}_{k+1},\ldots, a_r\in\mathcal{A}_r \}.$$
Thus, the set $\G_{\A}$ can be expressed as the direct sum $\G_{\A}=V\oplus W$.

Now, we present the first result, which gives an upper bound on the character sum over the subset $\G_{\A}$.
\begin{prop}\label{prop3.1}
Let $d\ge 2$ be a fixed integer, and let $r$ be a multiple of $d$ with $k=\frac{r}{d}$. Let $\G_{\A}$, $V$, $W$ be defined as above. Let $\chi$ be a nontrivial multiplicative character of $\fqr$, and let $f(X)$ be a polynomial over $\fqr$ of positive degree. Let $D$ be the degree of the largest squarefree divisor of $f(X)$. Then, we have the following bound on the character sum:
$$|S(\G_{\A},\chi,f)|\le(\#V)^{\frac{1}{2}}q^{\frac{r}{4d}}\left(N_f(W)(2dD-1)+((\#W)^2-N_f(W))q^{\frac{r}{2d}}\right)^{\frac{1}{2}},$$
where $N_f(W)$ denotes the number of pairs $(w_1,w_2)\in W^2$ satisfying Condition $A_{k,r}(f)$.
\end{prop}
\begin{proof}
First, we express the character sum $S(\G_{\A},\chi,f)$ as follows:
$$S(\G_{\A},\chi,f)=\sum_{\gamma\in\G}\chi(f(\gamma))=\sum_{v\in V}\sum_{w\in W}\chi(f(v+w)).$$
Making use of the Cauchy-Schwarz inequality, we obtain that
\begin{align*}
|S(\G_{\A},\chi,f)|^2&\le\# V\sum_{v\in V}\Big|\sum_{w\in W}\chi(f(v+w))\Big|^2\\
&\le\# V\sum_{v\in\fqk}\Big|\sum_{w\in W}\chi(f(v+w))\Big|^2\\
&\le\# V\sum_{w_1,w_2\in W}\Big|\sum_{v\in\fqk}\chi(f(v+w_1))\overline{\chi(f(v+w_2))}\Big|.
\end{align*}
Next, we bound the inner sum. For each pair $(w_1,w_2)\in W^2$ satisfying Condition $A_{k,r}(f)$, we use Corollary \ref{cor1.2} to bound the sum. Otherwise, we use the trivial bound $q^k$. Therefore, we have
$$|S(\G_{\A},\chi,f)|^2\le \#V\left(N_f(W)(2dD-1)q^{\frac{r}{2d}}+((\#W)^2-N_f(W))q^{\frac{r}{d}}\right).$$
Taking the square root on both sides completes the proof of Proposition \ref{prop3.1}.
\end{proof}

\begin{thm}\label{thm3.2}
Let $r$ be an even number and $k=\frac{r}{2}$. Let $\G_{\A}$, $V$, $W$ be defined as before. Let $\chi$ be a nontrivial multiplicative character of $\fqr$, and $f(X)\in\F_{k,r}$. Then, we have
$$|S(\G_{\A},\chi,f)|\le(\#\G_{\A})^{\frac{1}{2}}q^{\frac{r}{8}}\left((\#W-1)(4D-1)+q^{\frac{r}{4}}\right)^{\frac{1}{2}}.$$
Consequently, if $\#\mathcal{A}_i>q^{\frac{1}{2}}$ for each $\frac{r}{2}+1\le i\le r$, then
$$|S(\G_{\A},\chi,f)|\le2(\#\G_{\A})^{\frac{1}{2}}q^{\frac{r}{8}}D^{\frac{1}{2}}(\#W-1)^{\frac{1}{2}}.$$
Here, $D$ is the degree of the largest squarefree divisor of $f(X)$. 
\end{thm}
\begin{proof}
This follows immediately from Proposition \ref{prop3.1} and Lemma \ref{lem1.3}. 
\end{proof}
Clearly, the estimate of Theorem \ref{thm3.2} is nontrivial if $\#\mathcal{A}_i>q^{\frac{1}{2}}$ for any $1\le i\le r$. As a direct application of Theorem \ref{thm3.2}, we obtain the following nontrivial bound.
\begin{cor}\label{cor2.3}
Let $q=3$ and $r$ be an even number. Let $\A=\{\A_i\}_{i=1}^r$ be a sequence of subsets with cardinality two of $\fq$, and $\G_{\A}$ be defined as before. Then for $f(X)\in\F_{k,r}$ we have 
$$|S(\G_{\A},\chi,f)|<2\sqrt{D}\cdot2^{\delta r},$$
where $D$ is the degree of the largest squarefree divisor of $f(X)$, and
$$\delta=\frac{1}{8}\log_2192=0.94812\cdots.$$
\end{cor}
\begin{rem}\label{rem2.4}
Let $q=3$, and let $\mathcal{A}_i=\{0,2\}$ for any $1\le i\le r$. Let $\G_{\A}=\{\sum_{i=1}^ra_i\alpha_i: a_i\in\A_i\}$, $\chi$ be a multiplicative character of order $d\ge 2$ and $f(X)\in\fqr[X]$ satisfy that $f(X)$ is not a $d$th power of a polynomial over $\fqr$. In the previous result, Iyer and Shparlinski \cite[Corollary 3.3]{[IS]} derived a nontrivial bound for $|S(\G,\chi,f)|$ as follows:
$$|S(\G_{\A},\chi,f)|\le c\cdot 2^{\gamma r},$$
where $c>0$ is a constant and $\gamma=1-\kappa(\frac{\log3}{\log2})=0.99128\cdots$ with $\kappa(x)=\frac{10x^2-4x-1}{100x^2+20x}$. Comparing our result from Corollary \ref{cor2.3}, it provides a much sharper upper bound for the character sum $S(\G_{\A},\chi,f)$ under a certain restriction.
\end{rem}

Recently, Grzywaczyk and Winterhof (see \cite[Theorem 2.1]{[GW]}) provided a new estimate as follows.
\begin{lem}\label{lem2.3}
Let $c_1,c_2\ldots,c_r$ be elements of $\fq$, and let $\A=\{\A_i\}_{i=1}^r$ with $\mathcal{A}_i=\fq\setminus\{c_i\}$ for any $1\le i\le r$. Then for $\G_{\A}=\{\sum_{i=1}^ra_i\alpha_i: a_i\in\mathcal{A}_i\ \text{for\ each}\ i\}$ and a nontrivial multiplicative character $\chi$ of $\fqr$, one has
$$|S(\G_{\A},\chi)|\le \sqrt{3}(q-1)^{\frac{r}{2}}q^{\lceil 3r/4\rceil/2}.$$
\end{lem}
Let $\G_{\A}$ be the set as given in Lemma \ref{lem2.3}. Using Theorem \ref{thm3.2} and taking $f(X)=X$, we obtain a sharper upper bound for the character sum $S(\G_{\A},\chi)$:
\begin{align}\label{c2-3}
|S(\G_{\A},\chi)|<2(q-1)^{\frac{3r}{4}}q^{\frac{r}{8}},
\end{align}
where $r$ is an even number. This bound improves upon the result of Lemma \ref{lem2.3} for $q\ge 3$ and holds asymptotically as $r$ increases.

Grzywaczyk and Winterhof appplied Lemma \ref{lem2.3} to prove that the set $\G_{\A}$ contains a primitive element of $\fqr$ when $q=4$ or $q=5$, provided $r$ is large enough. However, their result is not applicable for the case $q=3$. Our improved upper bound of (\ref{c2-3}) can be used to demonstrate that $\G_{\A}$ contains a primitive element of $\fqr$ for $q\ge 3$, assuming $r$ is a sufficiently large even integer. 
\begin{thm}\label{thm2.4}
Let $\G_{\A}$ be defined as in Lemma \ref{lem2.3}, and let $N(\G_{\A})$ denote the number of primitive elements in $\G_{\A}$. Then, for even $r$, we have the following lower bound: 
$$N(\G_{\A})>\frac{\phi(q^r-1)(q-1)^{\frac{3r}{4}}}{q^r-1}\Big((q-1)^{\frac{1}{4}}-2q^{\frac{1}{8}}W(q^r-1)\Big),$$
where $\phi$ is the Euler totient function, and $W(q^r-1)$ is the number of squarefree divisors of $q^2-1$.
\end{thm}
\begin{proof}
By applying the well known Vinogradov formula (see \cite[Exercise 5.14]{[LN]}), we derive 
$$
\frac{q^r-1}{\phi(q^r-1)}N(\G_{\A})=\sum_{e\mid (q^r-1)}\frac{\mu(e)}{\phi(e)}\sum_{\chi\in\Lambda_e}\sum_{w\in\G_{\A}}\chi(w),$$
where $\mu$ is the Mobius function and $\Lambda_e$ is the set of all multiplicative characters of $\fqr$ of order $e$. Using the upper bound of (\ref{c2-3}), we get
\begin{align*}
\frac{q^r-1}{\phi(q^r-1)}N(\G_{\A})&\ge (q-1)^r-1-\sum_{2\le e\mid (q^r-1)\atop \mu(e)\ne 0}\frac{1}{\phi(e)}\sum_{\chi\in\Lambda_e}2(q-1)^{\frac{3r}{4}}q^{\frac{r}{8}}\\
&=(q-1)^r-1-2(q-1)^{\frac{3r}{4}}q^{\frac{r}{8}}\sum_{2\le e\mid (q^r-1)\atop \mu(e)\ne 0}1\\
&>(q-1)^r-2(q-1)^{\frac{3r}{4}}q^{\frac{r}{8}}W(q^r-1),
\end{align*}
which leads the desired inequality. The proof of Theorem \ref{thm2.4} is done.
\end{proof}
\begin{lem}\label{lem2.5}\cite{[CST]}
Let $W(t)$ represent the number of squarefree divisors of $t$. Then for $t\ge 3$ one has that
$$W(t-1)<t^{\frac{0.96}{\log\log t}}.$$
\end{lem}
Thus, from Theorem \ref{thm2.4} and Lemma \ref{lem2.5}, we obtain the following result.
\begin{cor}\label{cor3.8}
Let $c_1,\ldots,c_r$ be elements of $\fq$, and let $\A=\{\A_i\}_{i=1}^r$ with $\mathcal{A}_i=\fq\setminus\{c_i\}$ for each $i$. Then for any $q\ge 3$ the set $\G_{\A}$ contains a primitive element of $\fqr$ if $r$ is a large enough even integer. Precisely, for some small $q$ and even $r$, $\G_{\A}$ contains a primitive element of $\fqr$ provided one of the following holds:
\begin{enumerate}[label=(\alph*)]
    \item $q=9$ and $r\ge 2504$;
    \item $q=8$ and $r\ge 3256$;
    \item $q=7$ and $r\ge 4754$;
    \item $q=5$ and $r\ge 25662$;
    \item $q=4$ and $r\ge 363184$;
    \item $q=3$ and $r\ge 7951\times 10^8$.
\end{enumerate}
\end{cor}
\begin{proof}
Theorem \ref{thm2.4} and Lemma \ref{lem2.5} provide a sufficient condition for the existence of a primitive element in \( \G_{\A} \):
\begin{align}\label{c3-3}
\left( 2q^{\frac{0.96r}{\log\log q^r}} \right)^{\frac{1}{r}} \leq \left( \frac{(q-1)^2}{q} \right)^{\frac{1}{8}}.
\end{align}
Note that the term \( \left( 2q^{\frac{0.96r}{\log\log q^r}} \right)^{\frac{1}{r}} \) is monotonically decreasing and tends to 1 as \( r \to \infty \), while \( \left( \frac{(q-1)^2}{q} \right)^{\frac{1}{8}} > 1 \) for all \( q \geq 3 \). Thus, the inequality holds for sufficiently large \( r \). Moreover, for \( q = 4, 5, 7, 8, \) or \( 9 \), a computer-assisted check verifies that inequality \eqref{c3-3} holds when \( q = 9 \) and \( r \geq 2504 \), \( q = 8 \) and \( r \geq 3256 \), \( q = 7 \) and \( r \geq 4754 \), \( q = 5 \) and \( r \geq 25662 \), \( q = 4 \) and \( r \geq 363184 \), and $q=3$ and $r\ge 7951\times 10^8$. Hence, the result follows.
 \end{proof}
 The following table (see TABLE \ref{table1}) displays the performance of three estimates in establishing the existence of primitive elements in the finite field $\fqr$ avoiding affine hyperplanes.
 \begin{table}[h]
\centering
\caption{Values of $(q,r)$ for which the three estimates are applicable}
\label{table1}
\begin{tabular}{|c|c|c|c|}
\hline
\( q \) & Our Estimate  & G. and W. Estimate & F. and R. Estimate  \\
\hline
9 & $r\ge 2504$ & $r\ge 8040$ & $r$ large enough   \\
8& $r\ge 3256$ & $r\ge 14831$ & $r$ large enough\\
7 & $r\ge 4754$ & $r\ge 39247$ & $r$ large enough \\
5& $r\ge 25662$ & $r\ge 1.92\times 10^7$ & not applicable  \\
4& $r\ge 363184$ & $r$ large enough & not applicable\\
3& $r\ge7.96\times 10^{11}$ & not applicable & not applicable\\
\hline
\end{tabular}
\end{table}
Therefore, the lower bound we obtained for $N(\G_{\A})$ is significantly better than the results of Fernandes and Reis \cite{[FR]} and Grzywaczyk and Winterhof \cite{[GW]} when $q$ is small. However, our results are subject to two constraints: \( r \) must be an even integer, and the basis \(\alpha_1, \ldots, \alpha_r\) of \(\mathbb{F}_{q^r}\) must be extended from a basis \(\alpha_1, \ldots, \alpha_{r/2}\) of its subfield \(\mathbb{F}_{q^{r/2}}\). Can we relax the latter restriction? 

Now, let \( r \) be even, and let \(\beta_1,\ldots,\beta_r\) be any given basis of \(\mathbb{F}_{q^r}\) over \(\mathbb{F}_q\). Let \(\mathcal{B} = \{\mathcal{B}_i\}_{i=1}^r\) be a family of subsets of \(\mathbb{F}_q\), and define \(\G_{\mathcal{B}} = \left\{\sum_{i=1}^r b_i \beta_i : b_i \in \mathcal{B}_i\right\}\). We claim that the results of Corollary \ref{cor2.3}, Eq. (\ref{c2-3}), Theorem \ref{thm2.4}, and Corollary \ref{cor3.8} are still valid if we replace \(\G_{\A}\) by \(\G_{\mathcal{B}}\). 

In fact, let
$$
(\beta_1, \ldots, \beta_r) = (\alpha_1, \ldots, \alpha_r) M
$$
with \( M = (m_{ij}) \) being an invertible matrix in \(\mathbb{F}_q^{r \times r}\). For any \( x = \sum_{i=1}^r b_i \beta_i \in \G_{\mathcal{B}} \) with each \( b_i \in \mathcal{B}_i \), we have
$$
x = (\beta_1, \ldots, \beta_r) (b_1, \ldots, b_r)^T = (\alpha_1, \ldots, \alpha_r) M (b_1, \ldots, b_r)^T = \sum_{i=1}^r a_i \alpha_i,
$$
where \( a_i \in \A_i := \left\{\sum_{j=1}^r m_{ij} b_j : b_j \in \mathcal{B}_j \right\} \) for each \( 1 \le i \le r \), which implies that \(\G_{\mathcal{B}} = \G_{\A}\), so that \( S(\G_{\mathcal{B}}, \chi, f) = S(\G_{\A}, \chi, f) \).

Since \( M \) is invertible, we have
$$
\prod_{i=1}^r \#\A_i = \prod_{i=1}^r \#\mathcal{B}_i, \quad \text{and} \quad \#\A_i \geq \min_{1 \le j \le r} \#\mathcal{B}_j \quad \text{for all} \quad 1 \le i \le r,
$$
and therefore, \( \#\A_1 = \cdots = \#\A_r \) if \( \#\mathcal{B}_1 = \cdots = \#\mathcal{B}_r \). The claim follows.

\section{$\G$ being a subset of $s$-sparse elements}
Let $r$ be an even number and $k=\frac{r}{2}$. Let $\alpha_1,\ldots,\alpha_k$ is a basis of $\fqk$ over $\fq$, and $\alpha_1,\ldots,\alpha_r$ be a completion of $\alpha_1,\ldots,\alpha_k$ to a basis of $\fqr$ over $\fq$. For $x\in\fqr$, we define the \textit{weight} of $x$ by $wt(x):=\#\{1\le i\le r: a_i\ne 0\}$ if $x=a_1\alpha_1+\cdots+a_r\alpha_r$ with each $a_i\in\fq$. For a given positive integer $s<r$, let 
\begin{align}\label{c4-0}
\mathcal{G}_s:=\{x\in\fqr: wt(x)=s\},
\end{align}
and we call the $\G_s$ a set with $s$-sparse elements of $\fqr$. In this section, we are going to estimate the sum $S(\G_s,\chi,f)$, where $\chi$ is a nontrivial multiplicative character of $\fqr$ and $f(X)$ is a polynomial over $\fqr$; and we focus the investigation on the natural case $q=2$. 

Let us begin by conducting some preliminaries. Let $H(x)$ be the \textit{entropy} function defined by 
$$H(x)=-x\log_2x-(1-x)\log_2(1-x),\ 0<x<1,$$
and $H(0)=H(1)=0$. We also define 
$$H^*(x)=\begin{cases}
H(x),&\text{if}\ 0\le x\le \frac{1}{2},\\
1,&\text{if}\ x>\frac{1}{2}.
\end{cases}$$
The following lemma will be used to bound cardinalities of some sets.
\begin{lem}\label{lem4.1} \cite[Chapter 10, Corollary 9]{[MS]}
For any positive integer $n$ and a real number $0<\gamma\le \frac{1}{2}$, we have
$$\sum_{0\le m\le\gamma n}\binom{n}{m}\le 2^{nH(\gamma)}.$$
Consequently, 
$$\binom{n}{m}\le2^{nH^*(\gamma)}$$
for any positive integers $m$ and $n$ with $m\le n$.
\end{lem}
For quantities $A$ and $B$, we use $A\ll B$ to represent $|A|\le c\cdot B$ for some constant $c>0$, and $A=o(B)$ to represent $|A|\le\epsilon\cdot B$ for any $\epsilon>0$. Let $\mathrm{f, g, h}$ be three functions in terms of the variables $\rho$ and $\lambda$ with $0<\lambda\le\rho<1$ by
\begin{align}\label{c4-1-1}
&\mathrm{f}(\rho,\lambda):=\frac{1}{2}H(2\lambda)+\frac{1}{2}H^*(2\rho-2\lambda),\\\label{c4-1-2}
&\mathrm{g}(\rho,\lambda):=\frac{1}{8}+\frac{1}{2}H(\rho)+\frac{1}{4}H^*(2\rho-2\lambda),\\\label{c4-1-3}
&\mathrm{h}(\rho,\lambda):=\frac{1}{2}H(\rho)+\frac{1}{4}.
\end{align}

Now we can state the main result of this section
as follows.
\begin{thm}\label{thm4.2}
Let $r$ be an even number and $k=\frac{r}{2}$. Let $q=2$ and $f(X)\in\F_{k,r}$. For any $s$ with $1\le s\le k$, let $\G_s$ be defined as in (\ref{c4-0}), and put $\rho:=\frac{s}{r}$. Then we have
$$S(\G_s,\chi,f)\le 2^{\eta’(\rho)r+o(r)},$$
where
$$\eta’(\rho)=\min_{0<\lambda\le\frac{1}{2}\rho}\max\{\mathrm{f}(\rho,\lambda),\mathrm{g}(\rho,\lambda),\mathrm{h}(\rho,\lambda)\}.$$ 
\end{thm}
\begin{proof}
First, let
$$U_1=\{a_1\alpha_1+\cdots+a_k\alpha_k: a_i\in\ftwo\ \text{for\ any}\ 1\le i\le k \}$$
and 
$$U_2=\{a_{k+1}\alpha_{k+1}+\cdots+a_r\alpha_r: a_i\in\ftwo\ \text{for\ any}\ k+1\le i\le r\},$$
then we can decompose the set $\G_s$ into the disjoint union of $s+1$ subsets as 
$$\G_s=\bigcup_{i=0}^{s}\left\{u_1+u_2: u_1\in U_1^{(i)}\ \text{and}\ u_2\in U_2^{(s-i)}\right\},$$
where $U_j^{(i)}=\{u\in U_j: wt(u)=i\}$ for any $j=1,2$ and $0\le i\le s$. It follows that
$$S(\G_s,\chi,f)=\sum_{i=0}^{s}\M_i,$$
where 
$$\M_i=\sum_{u_1\in U_1^{(i)}}\sum_{u_2\in U_2^{(s-i)}}\chi(f(u_1+u_2)).$$
Let $t=\lfloor\lambda r \rfloor$, where $0<\lambda\le\rho$ is a variable. Now we estimate each $\M_i$. For $0\le i<t$, one has  the trivial bound
$$|\M_i|\le \#U_{1}^{(i)}\#U_{2}^{(s-i)}=\binom{k}{i}\binom{r-k}{s-i}.$$
Then by Lemma \ref{lem4.1}, we obtain that 
$$\M_i\le 2^{\frac{1}{2}(H^*(2\tau)+H^*(2\rho-2\tau))r},$$
where $\tau:=\frac{i}{r}$. If let $\tilde{\mathrm{f}}(\tau):=\frac{1}{2}(H^*(2\tau)+H^*(2\rho-2\tau))$, one then easily sees that 
$\tilde{\mathrm{f}}(\tau)$ can attain its maximum at $\tau=\min\{\lambda, \frac{\rho}{2}\}$ when $\tau$ runs over $0<\tau\le \lambda$. So one may let $\lambda\le \frac{\rho}{2}$, $\mathrm{f}(\rho,\lambda):=\tilde{\mathrm{f}}(\lambda)=\frac{1}{2}(H^*(2\lambda)+H^*(2\rho-2\lambda))$, and then we get that
\begin{align}\label{c4-4}
\sum_{i=0}^{t-1}|\M_i|\le t\cdot 2^{\mathrm{f}(\rho,\lambda)r}=2^{\mathrm{f}(\rho,\lambda)r+o(r)}.
\end{align}

Next, we proceed to estimate $\M_i$ for the case $i\ge t$. Note that
\begin{align}\label{c4-1}
|\M_i|\le\sum_{u_1\in U_1^{(i)}}1\times \Big|\sum_{u_2\in U_2^{(s-i)}}\chi(f(u_1+u_2))\Big|.
\end{align}
By employing the Cauchy-Schwarz inequality into the right-hand side of (\ref{c4-1}), we get that
\begin{align}\label{c4-2}
|\M_i|^2\le\#U_1^{(i)}\sum_{u_1\in U_1^{(i)}}\Big|\sum_{u_2\in U_2^{(s-i)}}\chi(f(u_1+u_2))\Big|^2.
\end{align}
For the sums in the right-hand side of (\ref{c4-2}), we extend the first sum over $U_1^{(i)}$ to a sum over $U_1=\fqk$, and derive
\begin{align*}
\sum_{u_1\in U_1^{(i)}}\Big|\sum_{u_2\in U_2^{(s-i)}}\chi(f(u_1+u_2))\Big|^2&\le\sum_{u_1\in\fqk}\Big|\sum_{u_2\in U_2^{(s-i)}}\chi(f(u_1+u_2))\Big|^2\\&=\sum_{u^{\prime}_2,u^{\prime\prime}_2\in U_2^{(s-i)}}\Big|\sum_{u_1\in\fqk}\chi(f(u_1+u^{\prime}_2))\overline{\chi}(f(u_1+u^{\prime\prime}_2))\Big|.
\end{align*}
For the last inner sum, we bound it by the result of Corollary \ref{cor1.2} if $u^{\prime}_2,u^{\prime\prime}_2\in U_2^{(s-i)}$ satisfy Condition $A_{k,r}(f)$; otherwise, it is bounded by the trivial bound $q^k$. Thus,
\begin{align*}
&\sum_{u_1\in U_1^{(i)}}\Big|\sum_{u_2\in U_2^{(s-i)}}\chi(f(u_1+u_2))\Big|^2\\
&\le N_f(U_2^{(s-i)})(4D-1)2^{\frac{r}{4}}+((\#U_2^{(s-i)})^2-N_f(U_2^{(s-i)}))2^{\frac{r}{2}},
\end{align*}
where $N_f(U_2^{(s-i)})$ represents the number of pairs $(u^{\prime}_2,u^{\prime\prime}_2)\in U_2^{(s-i)}\times U_2^{(s-i)}$ satisfying Condition $A_{k,r}(f)$. Note that $f(X)\in\F_{k,r}$, It then follows from (\ref{c4-2}) that
$$|\M_i|^2\ll\#U_1^{(i)}(\#U_2^{(s-i)})^22^{\frac{r}{4}}+\#U_1^{(i)}\#U_2^{(s-i)}2^{\frac{r}{2}}.$$
This together with a trivial bound $\#U_1^{(i)}\#U_2^{(s-i)}\le\#\G_s$ give that
$$|\M_i|^2\ll\#\G_s\#U_2^{(s-i)}2^{\frac{r}{4}}+\#\G_s2^{\frac{r}{2}}.$$
Finally, by using Lemma \ref{lem4.1}, we obtain that
$$|\M_i|\ll(2^{\mathrm{g}(\rho,\lambda)r}+2^{\mathrm{h}(\rho,\lambda)r})2^{o(r)},$$
where $ \mathrm{g}(\rho,\lambda)$ and $\mathrm{h}(\rho,\lambda)$ are defined as (\ref{c4-1-2}) and (\ref{c4-1-3}), and then we have
\begin{align}\label{c4-7}
\sum_{i=t}^{s}|\M_i|\le (s-t+1)(2^{\mathrm{g}(\rho,\lambda)r}+2^{\mathrm{h}(\rho,\lambda)r})2^{o(r)}\le(2^{\mathrm{g}(\rho,\lambda)r}+2^{\mathrm{h}(\rho,\lambda)r})2^{o(r)}.
\end{align}
Therefore, combing (\ref{c4-4}) and (\ref{c4-7}) together, we arrive at
the final result. 
\end{proof}
It is noted from Lemma \ref{lem4.1} that we have a trivial estimate
$$|S(\G_s,\chi,f)|\le 2^{H(\rho)r+o(r)}.$$
Thus, when $\eta’(\rho)<H(\rho)$ in Theorem \ref{thm4.2}, our estimate becomes non-trivial.  Additionally, M\'{e}rai, Shparlinski, and Winterhof recently obtained an estimate for $S(\G_s,\chi,f)$ that
$$|S(\G_s,\chi,f)|\le 2^{\eta(\rho)r+o(r)},$$
where $\eta(\rho)$ is a function of $\rho$, and one can refer to \cite[Theorem 2.5]{[MSW]} for more details. To facilitate comparison between our estimate, the trivial estimate, and the estimate obtained by M\'{e}rai, Shparlinski, and Winterhof, we present a plot of these three estimates below (see Figure \ref{fig1}). 
\begin{figure}[htbp]
    \centering
    \includegraphics[width=0.6\textwidth]{bounds.png}  % 修改example-image为你的图片文件名
    \caption{Comparison of $\eta(\rho)$, $\eta’(\rho)$ and $H(\rho)$}
    \label{fig1}
\end{figure}
It is clear that our estimate performs better. More precisely, with the help of computer, we obtain 
$$\eta^{\prime}(0.13) \approx 0.55680, \eta(0.13) \approx 0.62751, H(0.13) \approx 0.55743,$$
and
$$\eta^{\prime}(0.32) \approx 0.82690, \eta(0.32) \approx 0.82719, H(0.32) \approx 0.90438,$$
Furthermore, it can be easily observed that for $\rho \in (\rho_1, \rho_2)$, $\eta^{\prime}(\rho) < \min\{\eta(\rho), H(\rho)\}$, for some $\rho_1 < 0.13$ and $\rho_2 > 0.32$.







% use section* for acknowledgment
%\section*{Acknowledgment}
%The author would like to thank the anonymous referees for their helpful comments which improved the presentation of the paper.

\begin{thebibliography}{99}
\bibitem{[AS]} A. Adolphson and S. Sperber, \textit{Character sums in finite fields}, Compositio Mathematica {\bf 52}(1984), 325-354.

\bibitem{[CST]} S. Cohen, T. Oliveira e Silva and T. Trudgian, \textit{On consecutive primitive elements in a finite field}, Bull. Lond. Math. Soc.  {\bf 47}(2015), 418-426.

\bibitem{[FR]} A. Fernandes and Lucas Reis, \textit{On primitive elements of finite fields avoiding affine hyperplanes}, Finite Fields Appl. {\bf 76}(2021), 101911.

\bibitem{[GW]} P. Grzywaczyk and A. Winterhof, \textit{Primitive elements of finite fields $\fqr$ avoiding affine hyperplanes for $q=4$ and $q=3$}, Finite Fields Appl. {\bf 96}(2024), 102416.

\bibitem{[IS]} S. Iyer and I. Shparlinski, \textit{Character sums over elements of extensions of finite fields with restricted coordinates}, Finite Fields Appl. {\bf 93}(2024), 102335.

\bibitem{[LN]} R. Lidl and H. Niederreiter, Finite Fields, Cambridge University Press, Cambridge, 1997.

\bibitem{[MSW]} L. M\'{e}rai, I. Shparlinski and A. Winterhof, \textit{Character sums over sparse elements of finite fields}, Bull. London Math. Soc. {\bf 56}(2024), 1488-1510.

\bibitem{[MS]} F.  MacWilliams and N. Sloane, The theory of error-correcting codes, North-Holland Mathematical Library, vol.16, North-Holland Publishing Co., Amsterdam-New York-Oxford, 1977.

\bibitem{[Wan]} D. Wan, \textit{Generators and irreducible polynomials over finite fields}, Math. Comp. {\bf 66}(1997),  1195-1212.

\bibitem{[Win]} A. Winterhof, \textit{Character sums, primitive elements, and powers in finite fields}, J. Number Theory {\bf 91}(2001),  153-163.

\end{thebibliography}

\end{document}
