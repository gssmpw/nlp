\documentclass[a4paper,12pt]{article}

\usepackage[utf8]{inputenc}
\usepackage{fullpage}
\usepackage{graphicx}
\usepackage[usenames,dvipsnames]{xcolor}
\usepackage{amsmath}
\usepackage{amsthm}
\usepackage{amssymb}
\usepackage{breakcites}
\usepackage[left]{lineno}
\usepackage{blindtext}
\usepackage{subfig}
\usepackage{hyperref}
\usepackage{lineno}
\usepackage{url}
\usepackage{multicol}

\theoremstyle{definition}
\newtheorem{definition}{Definition}%[section]

%there's no option "example" for theoremstyle, there's plain, definition and remark

\newtheorem{plain}{Plain}%[section]

\theoremstyle{remark}
\newtheorem{remark}{Remark}%[section]

\usepackage{changes}
\colorlet{cMat}{Orchid}
\definechangesauthor[name={Mateusz}, color=cMat]{Mateusz}
\newcommand{\madd}[2][]{\added[id=Mateusz,comment=#1]{#2}}
\newcommand{\mdel}[2][]{\deleted[id=Mateusz,comment=#1]{#2}}
\newcommand{\mat}[1]{\todo[inline,color=cMat]{\color{black}#1}}

\colorlet{cal}{olive}%Dandelion}
\definechangesauthor[name={Alex}, color=cal]{Alex}
\newcommand{\aadd}[2][]{\added[id=Alex,comment=#1]{#2}}
\newcommand{\adel}[2][]{\deleted[id=Alex,comment=#1]{#2}}
\newcommand{\al}[1]{\todo[inline,color=cal]{\color{black}#1}}

\colorlet{cbar}{ForestGreen}
\definechangesauthor[name={Bartek}, color=cbar]{Bartek}
\newcommand{\badd}[2][]{\added[id=Bartek,comment=#1]{#2}}
\newcommand{\bdel}[2][]{\deleted[id=Bartek,comment=#1]{#2}}
\newcommand{\bart}[1]{\todo[inline,color=cbar]{\color{black}#1}}

\colorlet{cmwa}{Maroon}%BurntOrange}
\definechangesauthor[name={Mwawi}, color=cmwa]{Mwawi}
\newcommand{\mwadd}[2][]{\added[id=Mwawi,comment=#1]{#2}}
\newcommand{\mwdel}[2][]{\deleted[id=Mwawi,comment=#1]{#2}}
\newcommand{\mwa}[1]{\todo[inline,color=cmwa]{\color{black}#1}}

\colorlet{cgok}{RoyalBlue}
\definechangesauthor[name={Gokhan}, color=cgok]{Gokhan}
\newcommand{\gadd}[2][]{\added[id=Gokhan,comment=#1]{#2}}
\newcommand{\gdel}[2][]{\deleted[id=Gokhan,comment=#1]{#2}}
\newcommand{\gok}[1]{\todo[inline,color=cgok]{\color{black}#1}}

\colorlet{adb}{RoyalBlue}
\definechangesauthor[name={ADab}, color=adb]{Gokhan}
\newcommand{\adadd}[2][]{\added[id=Gokhan,comment=#1]{#2}}
\newcommand{\adbdel}[2][]{\deleted[id=Gokhan,comment=#1]{#2}}
\newcommand{\adbcom}[1]{\todo[inline,color=adb]{\color{black}#1}}

\newcommand{\HH}{\mathcal{H}}
\newcommand{\bHH}{\overline{\mathcal{H}}}
\newcommand{\tHH}{\tilde{\mathcal{H}}}

\newcommand{\V}{\mathcal{V}}
\newcommand{\mG}{\mathcal{G}}
\newcommand{\mH}{\mathcal{H}}
\newcommand{\mU}{\mathcal{U}}
\newcommand{\bV}{\overline{\mathcal{V}}}
\newcommand{\tV}{\tilde{\mathcal{V}}}
\newcommand{\E}{\mathcal{E}}
\newcommand{\bE}{\overline{\mathcal{E}}}
\newcommand{\tE}{\tilde{\mathcal{E}}}
\newcommand{\tphi}{\tilde{\phi}}

\newcommand{\EG}{E_{\mathcal{G}}}
\newcommand{\EH}{E_{\mathcal{H}}}
\newcommand{\EU}{E_{\mathcal{U}}}

\newcommand{\tin}{\mathrm{in}} %like text in
\newcommand{\out}{\mathrm{out}}
\newcommand{\mI}{\mathcal{I}}
\newcommand{\inci}{\mathcal{I}^{\tin}}
\newcommand{\inco}{\mathcal{I}^{\out}}
\newcommand{\adji}{\mathcal{A}^{\tin}}
\newcommand{\adjo}{\mathcal{A}^{\out}}

\newcommand{\vtx}{\text{v}}
\newcommand{\Ph}{\vtx_\text{P}}
\newcommand{\A}{\vtx_\text{A}}
\newcommand{\Sc}{\vtx_\text{S}}
\newcommand{\B}{\vtx_\text{B}}

\newcommand{\RR}{\mathbb{R}}
\newcommand{\NN}{\mathbb{N}}
\newcommand{\ZZ}{\mathbb{Z}}


\title{Ubergraphs as models of higher-order interactions in ecosystems}
%\author{Mateusz Iskrzyński, Bartłomiej Morawski, G\"okhan M\"otlu, \\ Mwawi Nyirenda, Aleksandra Puchalska} %an alphabetic placeholder, the order will be input-based 


\begin{document}




\appendix\section{Literature review}

Relevant articles:
\begin{itemize}

\item \cite{Cervantes_2021} presents an analysis of experimental data on behavior of four pollinator species in controlled conditions. The main focus is the influence of resource availability and presence of pesticide on pollinator-pollinator interactions, both between species and within one. Statistical tools show that out of the three presented possible models, one accounting for these environmental conditions is by far the best. Their influence differs between species but is clearly present. The authors stress that the environmental context in which pollinators interact with plants and with each other is the crucial part of the interactions. \\
I think this is a good example of an ecological paper describing a higher-order interaction, since the influence of other species is impacted by the environment (available amount of flowers and pesticide presence).
\begin{enumerate}
    \item adding effects from a distinct type of objects calls for a multilayer network formulation more than for a hypergraph 
    \item binary/discrete case, like presence/absence of a pesticide can be modelled through a multiplex network (with two/more copies of the underlying bipartite graph plant-pollinator)
    \item continuous case (continuous resource $R$ availability in formula (11)): a multilayer network with one layer of the bipartite graph, and another with the vertices influencing continuously the interactions in the bipartite graph
\end{enumerate}

\item \cite{Grilli2017} considers higher-order competition models. Their research stems from the lacking stability of of simple models as well as neutral models. They focus on large communities where competitors interact. They note that the impact of higher-order interactions of species is crucial on the existence of other species. They consider both deterministic and stochastic models.

Their narrative is based on a forest with $ m $ species. Given that a tree of species $ i $ dies, there is then a competition to fill up the gap created. There are two outputs,
\begin{enumerate}
    \item Deterministic model: An equivalence is shown where the competition is done in a sequence (where interactions are still done pairwise at a time, but with the winner of a game proceeding to compete with the next competitor) and a simultaneous competition (where trees from all $ m $ species compete at the same time).
    \item Stochastic model: Higher-order interactions leads to longer periods of coexistence of species.
\end{enumerate}

    \item \cite{Arruda} 
    has a good intro on dynamics on hypergraphs and linear stability analysis
    
    In the paper the hypergraph $\HH =\left\{\V,\E\right\}$ is defined in the following way. $\V=\left\{ v_i:\,\,i\in I\right\}$ is a set of verticies, $N=|\V|$ and $\E=\left\{e_j\right\}$ is a set of hyperedges, and $e_j\in 2^{\V}$ has a cardinality $|e_j|$. By $\E_i$ we also denote the set of hyperedges that contain the vertex $v_i$. The weighted adjacency matrix can be defined as

\[
A_{ij}=\sum_{e_j\in \E;\,\, i,k\in e_j;\,\,i\neq k }\frac{w_{ij}(e_j)}{|e_j|-1}
\]

where $w_{ik}(e_j)\in \RR$ is a weight.

    Let us consider a system

    \begin{equation}\label{eq:main}
\frac{dx_i}{dt}=f_i(x_i)+\sum_{e_j\in \E_i}g_j(x_{\left\{e_j\right\}}),
    \end{equation}
    where $x_i$ is the state of the vertex $v_i$, $f_i:\RR \rightarrow \RR$ is function that depends only on the state $x_i$ and $g_j(x_{\left\{e_j\right\}}):\RR^{|e_j|}\rightarrow \RR$ that takes all the states of the vertices on the hyperedge $e_j$, here denoted as $x_{\left\{e_j\right\}}$, and compute its contribution to $x_i$.

    Its linear stability in the neighbourhood of fixed point $x^*$ depends on the spectrum of matrix $M=F(x^*)+G(x^*)$ where 
    \[
    G(x^*)=\sum_{m=1}^{\max{|e_j|}}G^m(x^*), \qquad G^m_{ik}(x^*)=\sum_{e_j\in \E_i;\,\,|e_j|=m}\partial_{x_k}g_j(x_{\left\{e_j\right\}})|_{x=x^*}.
    \]

   %Two particular examples of system \eqref{eq:main} are considered.

    %\begin{enumerate}
    %\item diffusion-like process on hypergraph
    %\[
    %\frac{dx_i}{dt}=f_i(x_i)+\frac{1}{|e_j|-1}\sum_{k\in e_j;\,\, k\neq i} (g(xi)-%g(x_k))
    %\]
    %\item 
    %\end{enumerate}
    %\item \cite{Carletti_2020} 
    %discusses a small subclass of dynamical systems on unweighted hypergraphs though

    \item \cite{Bairey2016} is the most typical example of use of a dynamical model with higher-order (non-bilinear) interactions in mathematical ecology. This is what we might want to cover - how to translate it to a weighted hypergraph properly?
    
    They used a replicator equation with non-linear fitness, see Fig.~\ref{Bairey_eq_ex} (for our reference). The model parameters are given by random two-, three- and four-dimensional tensors with entries drawn from a Gaussian distribution (mean 0, variance 1). It describes evolving fractional abundances of species whose total population is constant. The initial fractional abundances are equal ($\tfrac{1}{N}$, where $N$ is the number of species/variables).
    \begin{figure}[h!]
	\begin{center}
        \subfloat{\includegraphics[width=0.49\linewidth]{./fig/Bairey_1_equation.jpg}
		}
        \subfloat{\includegraphics[width=0.49\linewidth]{./fig/Bairey_2_example.jpg}}
        \caption{The model of \cite{Bairey2016} (left) and their examples of higher-order ecological interactions of microorganisms (right).} 
    	\label{Bairey_eq_ex}
    \end{center}
    \end{figure}
    To compare the impacts of various terms, they have run numerical simulations separately for only pairwise ($B\equiv 0$, $C\equiv 0$), then only three-way, and only four-way interactions.
    They defined a community as feasible if all the species existed at the end of the simulation with an abundance above $10^{-5}\tfrac{1}{N}$. They defined the \emph{critical strength of interactions} as corresponding to  of the communities exhibiting extinctions. 
    They found that the interaction strength at which $5\%$ of the random systems exhibit extinctions is the lower the more species there are for pairwise interactions and can be higher the more species there are for four-way interactions (Fig.~\ref{Bairey_res})

    \begin{figure}[h!]
	\begin{center}
        \subfloat{\includegraphics[width=0.49\linewidth]{./fig/Bairey_3_result.jpg}
		}
        \subfloat{\includegraphics[width=0.49\linewidth]{./fig/Bairey_4_interpret.jpg}}
        \caption{The results of \cite{Bairey2016} (left) and their description (right).} 
    	\label{Bairey_res}
    \end{center}
    \end{figure}
    

They remark, that local stability analysis models exclusively pairwise interactions, since the equation is linearised around a point and the high-order interactions are therefore embedded in the coefficients of the effective pairwise interactions obtained in the linearisation.

    \item 
    \cite{Young2021}
    introduces a method (a Bayesian generative model) to reconstruct possible higher-order interactions (represented using hypergraphs) from information about pariwise interactions, since actual data is often limited to those. Authors claim their method introduces hyperedges only with enough enough statistical evidence. The possible hypergraphs are evaluated using the formula \(P(H|G) = \frac{P(G|H)P(H)}{P(G)}\). \(P(H|G)\) is the probability of the hypergraph $H$ being a certain way for a given network $G$. $P(G|H)$ is called a projection component and is by analogy the probability of the network $G$ being a certain way given a hypergraph $G$. $P(G)$ is not important and serves as a normalisation constant. $P(H)$ is called a hypergraph prior. \\ 
    $P(G|H)$ follows a simple distribution, being $1$ iff two vertices appear jointly in any of the hyperedges of $H$ and $0$ otherwise. Authors remark that checking if $G$ is equal to the projection of the hypergraph $H$ is not a problem from a numerical point of view.
    $P(H)$ is based on an existing model, Poisson Random Hypergraphs Model (PRHM). The desired properties which the model fulfills are: the size of the interactions varies, not all vertices are connected by a hyperedge and some of the interactions are repeated (I'm not sure why it's good actually). 
    In this model the number of hyperedges connecting a set of vertices is a random variable following a Poisson distribution, with mean $\lambda_k$ dependent on the size of the set. The number of hyperedges connecting a set of $k$ vertices is equal to $A$ with probabily $P(A|\lambda_k) = \frac{\lambda_k^A}{A!}e^{-\lambda_k}$. All the hyperedges are modeled independently. 
    From this authors derive the formula for the probability of any hypergraph (assuming also some maximal hyperedge size), density of which is controlled by the parameters $\lambda_k$. They use what they call `a hierarchical empirical Bayes approach' and treat $\lambda_k$ as unknows drawn from prior distributions, use I believe some information theory and consider the problem of finding appropriate $\lambda_k$s solved. \\ 
    The authors prove two noteworthy properties of their model: firstly, it favors hypergraphs without repeated hyperedges, even though PRHM allows for duplicates. Secondly, the model favors sparser hypergraphs and introduces hyperedges only when it is justified. Those two properties allow the authors to conclude that such minimal hypergraphs are high-quality local maxima of $P(H|G)$. When it comes to the numerical point of view, the task is quite demanding, therefore some agorithm is proposed. In the conclusion authors say they would like to see a better one. \\
    Authors give several examples with both empirical and artificial data used, with the most attention given to an example of 613 American football games between 115 teams. Teams may play each other more often because of various reasons, mainly conferences (groups of team which all play each other) and geographical proximity. Some of the results are presented in the figures below (Fig.~\ref{Young_football_1}, ~\ref{Young_football_2}). There is also some focus on bipartite networks and the obtainted results show that more complex approach allowing higher-order interactions proved to be better in multiple cases (Fig. ~\ref{Young_bipartite}).

    \begin{figure}[h!]
	\begin{center}
        \includegraphics[width=0.9\linewidth]{./fig/Young_Football_1.png}
        \caption{Results of \cite{Young2021} about american football teams} 
    	\label{Young_football_1}
    \end{center}
    \end{figure}

    \begin{figure}[h!]
	\begin{center}
        \includegraphics[width=0.9\linewidth]{./fig/Young_Football_2.png}
        \caption{More results of \cite{Young2021} about american football teams} 
    	\label{Young_football_2}
    \end{center}
    \end{figure}

    \begin{figure}[h!]
	\begin{center}
        \includegraphics[width=0.5\linewidth]{./fig/Young_bipartite.png}
        \caption{Results of \cite{Young2021} about empirical bipartite networks} 
    	\label{Young_bipartite}
    \end{center}
    \end{figure}

\item [complex hypergraphs are equivalent to either an ubergraph or a multilayer network of two digraphs] \cite{Vazquez2022} introduces a concept of complex hyperhraphs (chygraphs). The main idea is to generalize the concept of hypergraphs beyond ubergraphs. Please note that the paper is not yet reviewed or published.
\begin{definition}[Complex hypergraph]
    A complex hypergraph (chygraph) $\chi(A,C)$ is a set of vertices $A$ and hypergraphs $C$ with vertex sets in $A \cup C$.
\end{definition}
The author gives several examples. For a graph the hypergraphs are said to be edges (which I understand as hypergraphs containing only one pairwise edge each, since they are meant to be actual hyperpgraphs?).
An ubergraph is a chygraph where the hypergraphs contain only one edge. \\
Another example is the system of scientific publications, which is represented by a chygraph $\chi(A, \{ \mathcal{H}_i(A_i \cup R_i, \{A_i, R_i\})\}) $. A publication is represented by a single hypergraph $\mathcal{H}_i$ with two edges $A_i$ (authors) and $R_i$ (references), which do not overlap. Here the difference between a chygraph and an ubergraph is visible: for an ubergraph the way to represent both references and authors would probably be similar, with two edges containing references and authors respectively. However the connection between those edges is important and the chygraph structure allows to include them in one hypergraph, which also has a clear meaning, as it represents a single publication. \\
\begin{figure}[h!]
	\begin{center}
        \includegraphics[width=0.5\linewidth]{./fig/Vazquez_example.png}
        \caption{Illustration of an example from \cite{Vazquez2022}} 
    	\label{Vazquez_example}
    \end{center}
    \end{figure}
The author also defines a matrix for chygraphs, which he decides to call the chy-adjacency matrix (although is seems more like an incidence matrix if anything):
\begin{definition}[Chy-adjacency matrix]
    Let $\chi(A, C=\{ \mathcal{H}_i(V_i, E_i)\})$ be a chygraph. Then the chy-adjacency matrix $\alpha$ is a  $|A\cup C| \times |A \cup C|$ matrix with 
    \begin{equation*}
        \alpha_{ij}=\begin{cases}
            1, \quad \text{if} \ i \in V_j ,& \\
            0, \quad \text{otherwise}
        \end{cases}
    \end{equation*}
    for $i,j \in A \cup C$.
    \end{definition}
    (to me this does not seem correct for several reasons, but maybe I'm missing something) \\
    There is also a definition of the length of the chygraph:
    \begin{definition}[Chygraph length]
        Let $\chi(A, C=\{ \mathcal{H}_i(V_i, E_i)\})$ be a chygraph and let $\Pi = \Pi_1 \cup \dots \cup \Pi_l$ be a partition of $C$ such that $\Pi_i \cap \Pi_j = \emptyset \text{ for } i \neq j$ and if $\mathcal{H}_i \in \Pi_j$, then $V_i \subset A \cup (\bigcup_{k \leq j} \Pi_k)$. The chygraph length $L(\chi)$ is the maximum $l$ between such partitions. 
        \end{definition}
    The second condition means the partition is hierarchical. The author then proceeds to analyze the mean component size. \\ 
    I think the paper seems to contains inaccuracies and does not provide much theory about chygraphs (and does not justify the need to define them). 



\end{itemize}

\section{Definitions and methods}
\subsection{Hypergraphs}
Hypergraphs enable us to model systems of different nature (e.g. in social systems, neuroscience, ecology and biology) in which some connections and relationships are described by higher order interactions rather than pairwise interactions. They provide the most general and flexible way of modeling such systems involving higher order interactions. As an example of systems where higher order interactions take place we can consider a complex ecosystem in which three or more species compete for food and/or territory \cite{Benson2018} or social mechanisms, such as peer-pressure or collaborations, where the connections involve three or more individuals \cite{levine2017}. 
 
A hypergraph $H$ consists of a set $V$ of vertices representing the elements or individuals in the system and a set $E$ of hyperedges representing the higher-order interactions (interactions in groups of more than two elements) within the system and describing which vertices are involved in them. 

\begin{definition}[Hypergraph]
A hypergraph $H$ is a tuple $H=(V,E)$ where $V=\{v_i : i\in I \}$ is the set of vertices (vertices) $v_i$ and $E=\{e_j : j \in J \}$ is the set of hyperedges $e_j$ where each hyperedge $e_j$ is a nonempty subset of $V$.
\end{definition}

 We assume that a hypergraph has a finite, non-empty vertex and hyperedge sets i.e. nonempty and nontrivial hypergraph. We also assume that the hypergraph does not have a repeated hyperedge which correspond to multiple edges in a graph. \al{But there are multiple 'edges' in the ubergaph in the different sense. Example: $V=\{v_1,v_2\}$, $E=\{\{v_1,v_2\}, \{\{v_1\},\{v_2\}\}\}$.} The order of the hypergraph is $ \lvert V \rvert $ and the size of the hypergraph is $\lvert E \rvert $.  

A hyperedge $e$ with $ \lvert e \rvert =1 $ is a (self-)loop and $ \lvert e \rvert =2 $ is an edge in the classical sense. Two vertices $v_i$ and $v_j$ are adjacent if there is a hyperedge $e$ which contains both vertices, i.e. $ \{ v_i , v_j \} \subset e $. In particular, if $e=\{v \} $ is an hyperedge then $v$ is adjacent to itself. Two hyperedges in a hypergraph are incident if they share some common vertices i.e. their intersection is not empty. A hypergraph is uniform or in particular $k$-uniform if all hyperedges have the same cardinality $k$. 

Pairwise graph measures and metrics can be generalised to hypergraphs. Let $ H $ be a hypergraph. Let us consider the notion of degree of a vertex for hypergraph setting. In a graph, degree of a vertex $v$ is defined as the number of adjacent vertices to $v$ or equivalently the number of edges containing $v$. However, for a hypergraph, these two definitions are not equivalent as one hyperedge may contain more than one adjacent vertices to $v$. Therefore, the notion of degree of a vertex can be defined in two different ways. Namely, we distinguish these two definitions as incidence degree \aadd{$deg_i(v)$} which is defined as the number of hyperedges containing the given vertex \aadd{$v$} and adjacency degree \aadd{$deg_a(v)$} which is defined as the number of adjacent vertices to the given vertex, namely
\begin{equation}
\aadd{deg_i(v):=|\{e\in E:\,\,v\in e\}|,\qquad deg_a(v):=|\{w\in V:\,\,\exists_{e\in E} \,v,w\in e\}|}
\end{equation}

Consider a hypergraph $H=(V,E)$ with $V=\{ v_1 , \ldots, v_n \}$ and $E=\{ e_1 , \ldots, e_m \}$ such that there is no isolated vertex. An incidence graph or a Levi graph \cite{Joslyn2017UbergraphsAD} is a bipartite graph $ G = (V \cup E, E') $ which decomposes a hypergraph into the vertex set and edge set where $ (v_i,e_j) \in E'$ if and only if $v_i \in e_j $. The incidence matrix of the hypergraph $ H $ is defined as $n \times m$ matrix $I=(I_{ij} )$ where $I_{ij} =1$ if $v_i \in e_j$ and $I_{ij} =0$ otherwise. Unlike incidence matrices of graphs, in incidence matrices of hypergraphs, the sum of the elements in a column can sum up to a value larger than 2. In fact, this value is the size of the relevant hyperedge. The transition between incidence graph and the incidence matrix is apparent, there exists an edge $ (v_i,e_j) \in E' $ if $ I_{ij} = 1$.

\textcolor{blue}{In a graph, a walk of length $ k $ is a sequence of vertices $ v_1,v_2,\ldots, v_{k+1} $ not necessarily all distinct such that for all $ i \in \{ 1,2,\ldots, k \}$ there exists $e_j \in E $ with $v_i, v_{i+1} \in e_j $ \cite{Estrada_2006}. \cite{Aksoy2020} however points out the need to distinguish between a vertex based walk and an edge based walk. With hypergraphs, there is introduced the notion of "width" of an edge in the sense of the number of vertices belonging to that edge.}

\textcolor{blue}{\begin{definition}
    An \textit{s-walk} of length $ k $ in a hypergraph $ H = (V,E) $ is a sequence of overlapping edges $e_0, e_1, \ldots, e_k$ such that $ |e_i\cap e_j | \ge s$, for all $ i \ne j$, $ i,j \in \{0,1,\ldots, k\} $.
\end{definition}}

\textcolor{blue}{Using the s-walk, other graph metrics are also defined such that connectedness, distance,eccentricity, closeness centrality. However, these are defined in terms of edges.}

Using the adjancency matrix, the centrality measure of a hypergraph can be defined. Let $A$ be an adjacency matrix of a hypergraph. We know $ A $ is a symmetric matrix and hence, there exists an orthogonal matrix $ U $ that diagonalises $ A$,
\[A = UDU^T \]
where $ D = diag(\lambda_1, \lambda2, \ldots, \lambda_n) $ and the columns of $ U $ are the corresponding eigenvectors to the eigenvalues in $D$. The number of walks of length $ k $ from vertex $ v_i $ to vertex $ v_j $ is,
\[ \mu_k (ij) = (A^k)_{ij} = \sum_{l=1}^n(UD^k)_{il}(U^T)_{lj} = \sum_{l=1}^n(UD^k)_{il}u_{jl} = \sum_{l=1}^n(u_{il}u_{jl}\lambda_l^k). \]
The number of walks of length $ k $, $W_k $, is,
\[ W_k = \sum_{ij} \mu_{ij} = \sum_{l=1}^n \left(\sum_{i=1}^n u_{il}u_{jl}\right)^2 \lambda_l^k. \]

The subgraph centrality \cite{Estrada05} of vertex $v_i$ is,
\[C_S(v_i) = \sum_{k\ge 0 } \frac{(A^k)_{ii}}{k!}\]

\subsection{Representations of higher-order interactions}
There are various ways to represent networks that involve higher-order interactions. These representations can be grouped into two categories: graph based representations and explicit higher-order representations. Graph based representations encodes all interactions with considering only edges (i.e. pairwise interactions). These include graphs, bipartite graphs (where interactions and interaction vertices are encoded as vertices in two different layers), motifs (small subgraphs with specific connectivity structures) and cliques (special type of cliques) (see Figure \ref{highorder}). 

However, in these representations some information about the higher-order interactions are not reflected completely and therefore some data is lost. For this reason, some explicit higher-order representations that are constructed using non-pairwise building blocks (e.g. simplices and hyperedges) are needed. Explicit higher order representations consist of simplicial complexes and hypergraphs (see Figure \ref{highorder}). Simplicial complexes are collection of simplices that allow us to reflect the exact nature of higher-order interactions. However, their drawback is that given a simplex all subsimplices are required in a simplical complex. This requirement results in complications in the representation. In hypergraphs, this requirement is removed and hence they provide the most efficient and general way to represent higher-order interactions.   

\subsection{Directed hypergraph}
%The notion of a hypergraph was introduced in order to capture non-pairwaise interactions between considered entities. Application of undirected hypernetwork indicates groups of entites that are in relation but abandon both the information about the impact of their relation and its quantitive aspect. In order to include both information into consideratons we introduce the notion of directed, weighted hypergraph. There can be many approaches to generalise a hypergraph. 

Let us define our notion of weighted dihypergraph that, unlike its classical counterpart, allows to capture long distance interactions. We start with a classical definition of dihypergraph that can be found also here\newline

\url{https://www.pks.mpg.de/~mapcon12/Slides/Ostermeier_Mapcon12.pdf}

\begin{definition}[Classical dihypergraph]
By the notion of classical dihypergraph $\HH$ we understand a pair $\HH =\left(\V,\E\right)$ such that $\V=\left\{ v_i:\,\,i\in I\right\}$ is a set of verticies, $N=|\V|$,  and $\E=\left\{e_j\right\}$ is a set of directed hyperedges, $e_j=(\V_j^{in},\V_j^{out})$, $\V_j^{in},\V_j^{out}\subset \V$.
\end{definition}
The elements of $\V_j^{in}$ (resp. $\V_j^{out}$) we call heads (resp. tails) of an edge $e_j$. By $\E_i^{in}$, (resp. $ \E_i^{out}$) we denote the set of hyperedges that contain the vertex $v_i$ as a head (resp. tail).

\begin{definition}[Substrate digraph]
\end{definition}

\begin{definition}[Classical sub-dihypergraph]\label{def:clas_sub-dhg}
By the sub-dihypergraph $\bHH$ of $\HH$ we understand a dihypergraph $\bHH =\left(\bV,\bE\right)$ such that $\V=\bV$, $\bV_j^{x}\subset \V_k^{x}$, $x=\text{in},\text{out}$; and  
\[
\forall_{\overline{e}_j=\left(\bV_j^{in},\bV_j^{out}\right)\in \bV}\quad\exists_{e_k=\left(\V_k^{in},\V_k^{out}\right)\in \V}\qquad \bV_j^{in}\cap \V_k^{in}\neq \emptyset \neq \bV_j^{out}\cap \V_k^{out}. 
\]
\end{definition}
\begin{remark}
\begin{enumerate}
\item Note that according to definition \ref{def:clas_sub-dhg}, the condition $\bE\subset\E$ does not have to hold. Indeed, if $(\left\{v_1\right\},\left\{v_2,v_3\right\})\in \E$ then $(\left\{v_1\right\},\left\{v_2\right\})\in \bE$. In the meantime $(\left\{v_1\right\},\left\{v_2\right\})\notin \E$.

\item The inclusion holds in the sense of substrate digraphs. Namely, if $S(\HH)=\left(S(\V),S(\E)\right)$ is a substrate digraph of dihypergraph $\HH$, then $S(\bE)\subset S(\E)$.
\end{enumerate}
\end{remark}

\subsection{Weighted dihypergraph - our first approach}

\begin{definition}[Generalised weighted dihypergraph]
We say that $\tHH$ is a generalised weighted dihypergraph of a dihypergraph $\HH$ if $\tHH=\left(\tV, \tE, \tphi\right)$ where $\V=\tV$, 
\[
\tE=\left\{\left(\bHH^{out}_j,\bHH^{in}_j\right):\,\,j\in J\right\}, \qquad  \bHH^{out}_j,\bHH^{in}_j\text{ are sub-dihypergraphs of}\,\, \HH,
\]
and $\tphi:\left\{\bHH^{in}_j:\,\, j\in J\right\}\rightarrow \RR$.
\end{definition}

\subsection{Ubergraph}
An ubergraph \cite{Joslyn2017UbergraphsAD} is a generalisation of a hypergraph in which hyperedges are allowed to contain not only vertices but other edges as vertices as well. In other words, in an ubergraph, an edge (called uberedge) can consist of some vertices and some other edges. In some sense, some edges have two roles: they are edges of the graph and they can also serve as a vertex. For the formal definition of ubergraphs we need the following notation.  

Let $X$ be a finite set. We denote 
    \begin{equation*}
        \mathcal{P}(X)^k := \mathcal{P} \left( \bigcup_{i=0}^{k} P_i \right),
    \end{equation*}
where $\mathcal{P}(X)$ denotes the family of all subsets of $X$ and
    \begin{equation*}
        P_0 = X, \quad P_i =\mathcal{P} \left( \bigcup_{j=0}^{i-1} P_j \right), \quad i\geq 1. 
    \end{equation*}
As a consequence of this iterative process, we have $\mathcal{P}(X)^0 =\mathcal{P}(X)$ and $\mathcal{P}(X)^1 =\mathcal{P}(X\cup \mathcal{P}(X) )$ and so on. 
\begin{definition}(\cite{Joslyn2017UbergraphsAD})
    A depth $k$ ubergraph is a pair $U=(V,E)$ where 
    \begin{itemize}
        \item[1] $V$ is a non-empty set of fundamental vertices,
        \item[2] $E \subseteq \mathcal{P}(V)^k$ is a finite set of uberedges,
        \item[3] if $s$ belongs to an uberedge and $s \notin V$, then $s$ is itself an edge. 
    \end{itemize}  
\end{definition}
\begin{remark}
Note that the condition 3 requires that uberedges can only contain fundamental vertices and other edges. Consequently, in an ubergraph, there are some edges that are also vertices. To distinguish such vertices from fundamental vertices, we will call the elements of $V$ as fundamental vertices and elements of $V \cup E$ as vertices.
\end{remark}
\begin{plain}
    Every hypergraph $H=(V,E)$ is a depth-0 ubergraph. In this case, set of fundamental vertices is $V$, set of uberedges is set of hyperedges $E$ and the requirement 3 is fulfilled since all hyperedges can only contain vertices.    
\end{plain}
\begin{plain}\label{exuber}(\cite{Joslyn2017UbergraphsAD})
    Consider the ubergraph $U=(V,E)$ with fundamental vertex set $V=\{ 1,2,3 \}$ and let $e_1 =\{ 1 \} , \ e_2 =\{ 1,3\} $. The set of uberedges is given by $$E=\{ e_1 ,e_2, \{1,3, e_1\} , \{2, e_2 \}, \{ 1, \{2, e_2 \} \}  \} .$$
    Here, the set of vertices is $ \{ 1,2,3, e_1 , e_2 , e_3, e_4 , e_5 \} $. It is clear that the uberedge $e_1$ is a vertex in the uberedge $e_3 := \{1,3, e_1\}$. Similarly, the uberedge $e_2$ is contained as a vertex in the uberedge $ \{2, e_2 \}$ and $e_4 := \{2, e_2 \}$ is contained as a vertex in the uberedge $e_5 := \{ 1, \{2, e_2 \} \}=\{1, e_4 \}$. Note that $U $ is a depth 2 ubergraph. 
\end{plain}
\begin{definition}(\cite{Joslyn2017UbergraphsAD})
    The incidence matrix of an ubergraph can be defined in a similar manner to hypergraphs. Namely, the incidence matrix of an ubergraph $U=(V,E)$ is a $ (\lvert V \rvert +\lvert E \rvert) \times \lvert E \rvert $ matrix with the entries 
    \begin{equation*}
        I_{ij}=\begin{cases}
            1, \quad \text{if} \ i \in j ,& \\
            0, \quad \text{otherwise}
        \end{cases}
    \end{equation*}
    \gdel{The degree of a vertex $x \in V \cup E$ can be defined as the sum of the entries in the corresponding row of incidence matrix. Namely, the degree of $x$ is the number of uberedges containing $x$.}  
\end{definition}
\begin{plain}
    Let us construct the incidence matrix of the ubergraph presented in Example \ref{exuber}. Since the set of vertices is $ \{ 1,2,3, e_1 , e_2 , e_3, e_4 , e_5 \} $ and the set of uberedges is $\{ e_1 ,e_2, e_3 ,e_4, e_5 \} $, the incidence matrix can be given
    \begin{equation*}
        I=\begin{pmatrix}
    1 & 1 & 1 & 0 &1 \\
    0 & 0 & 0 & 1 & 0 \\
    0  &  1 & 1  & 0 & 0 \\
     0  & 0  & 1  & 0 & 0 \\
     0  & 0  & 0  & 1 & 0 \\
     0  & 0  & 0  & 0 & 0 \\
     0  & 0  & 0  & 0 & 1 \\
     0  & 0  & 0  & 0 & 0
    \end{pmatrix}.
    \end{equation*}
\end{plain}
\begin{definition}(\cite{Joslyn2017UbergraphsAD})
    Let $U=(V,E)$ be an ubergraph with fundamental set of vertices $V$ and set of uberedges $E$ such that $\lvert V \rvert =m $, $\lvert E \rvert =n $. Two vertices $u,v \in V \cup E$ are adjacent if there is an uberedge $e$ such that $u,v \in e$. For example, the vertices $1$ and $e_1 $ are adjacent in Example \ref{exuber} since they both lie in the uberedge $e_3 := \{1,3, e_1\}$. 
    
    The adjacency matrix of an ubergraph can be defined analogously to hypergraphs. Explicitly, the adjacency matrix $A=(A_{ij} )$ of ubergraph $U=(V,E)$ is a $ (n +m) \times (n +m) $ matrix such that $A_{ii} =0$ and for $i \neq j$, $A_{ij} $ is the number of uberedges that contain both $i$ and $j$. 
\end{definition}
\begin{plain}
    Let us construct the adjacency matrix of the ubergraph presented in Example \ref{exuber}. Since the set of vertices is $ \{ 1,2,3, e_1 , e_2 , e_3, e_4 , e_5 \} $, the adjacency matrix can be given
    \begin{equation*}
        A=\begin{pmatrix}
    0 & 0 & 2 & 1 &0 & 0 & 1 & 0 \\
     0 & 0 & 0 & 0 &1 & 0 & 0 & 0 \\
     2 & 0 & 0 & 1 &0 & 0 & 0 & 0 \\
     1 & 0 & 1 & 0 &0 & 0 & 0 & 0 \\
     0 & 1 & 0 & 0 &0 & 0 & 0 & 0 \\
      0 & 0 & 0 & 0 &0 & 0 & 0 & 0 \\
      1 & 0 & 0 & 0 &0 & 0 & 0 & 0 \\
     0 & 0 & 0 & 0 &0 & 0 & 0 & 0
    \end{pmatrix}.
    \end{equation*}
\end{plain}


\subsection{Metabolic graph}
 Metabolic graphs~\cite{metabolic_network} are defined to allow modelling networks of reactions in which every reaction can combine multiple vertices and some vertices can influence a reaction (think of enzymes inhibiting or stimulating a reaction).
 
\begin{definition}[Metabolic graph]
A metabolic graph is an ordered quintuplet $G=(V, H, U, \Psi_H, \Psi_U)$ where $V$ is the set of vertices (vertices), $H$ is the set of directed hyperedges, $\Psi_H$ is the function assigning weights to hyperedges, $U$ is the set of signed uberedges and $\Psi_U: U \rightarrow \{+,-\}$
\end{definition}

\subsection{Petri net}
Petri nets~\cite{Petri_thesis, Petri_Peterson_book} represent chemical reactions and other processes, such as population dynamics in ecology. They are bipartite graphs representing compounds (species, places) and reactions (transitions) as separate sets of vertices. Importantly, they were invented as an exact representation of reactions in the spirit of Markov chains, with weights being only natural numbers. Their literal translation to ecology used explicit analogs - e.g. a wolf and a rabbit enter a predatory reaction, with two rabbits leaving it. The definitions of such objects come in various flavours, as well as the names 'Petri nets' and 'reaction networks'. 

\begin{definition}[Reaction network]
    A reaction network is a tuple $N = ( S , T , s, t )$, where $P$ and $T$ are disjoint finite sets of species and transitions, respectively. Maps $s,t: S \rightarrow \mathbb{R}^{\|S\|}$ define the weights with which each species participate in a transition as a substrate or a product, respectively.
\end{definition}
A reaction network/Petri does not include a representations of the influence of vertices on edges/reactions.

\begin{definition}[Open Petri net]
    An open reaction network~\cite{Baez_open_petri_2017} is a reaction network $R$ with a list of inputs $X \in  \mathbb{R}^{\|S\|}$ and outputs $Y \in  \mathbb{R}^{\|S\|}$ from outside the system.
\end{definition}

Petri nets are equivalent to directed hypergraphs, which redefine transitions as hyperedges. 

\begin{figure}
    \centering
    \includegraphics{fig/higher order repr.jpg}
    \caption{Representations of higher-order interactions. This figure is taken from \cite{BATTISTON20201}.}
    \label{highorder}
\end{figure}

\subsection{Examples of hypergraphs}
Let us present some concrete examples to hypergraphs.
\begin{plain}\label{ex21}
The simplest example of a hypergraph is a graph which consists of a vertex set $V$ and an edge set $E$ where edges are connecting two vertices, hence only pairwise interactions are taken into account.
\end{plain}

\begin{plain}\label{ex22}
Let us present an abstract example. Consider the hypergraph $H=(V,E)$ with $V=\{a,b,c,d,e\}$ and $E=\{ e_1 =\{a\},e_2 = \{a,b\},e_3 = \{a,b,c\}, e_4 = \{b,c,d,e\} \}   $. Here the hyperedges $\{a,b,c\}$ and $\{b,c,d,e\}$ represent higher-order (non-pairwise) interactions, $\{a,b\}$ is an edge in the classical sense and $\{a\}$ is a loop. The degree of the vertex for example $a$ is three since it is $H(a)=\{ \{a\}, \{a,b\}, \{a,b,c\} \} $ i.e. there are three hyperedges containing $a$.
\end{plain}

\subsection{Matrix representations of hypergraphs}
Consider a hypergraph $H=(V,E)$ with $V=\{ v_1 , \ldots, v_n \}$ and $E=\{ e_1 , \ldots, e_m \}$ such that there is no isolated vertex. Then, the incidence matrix of the hypergraph $H$ is defined as an $n \times m$ matrix $I=(I_{ij} )$ where $I_{ij} =1$ if $v_i \in e_j$ and $I_{ij} =0$ otherwise. Unlike incidence matrices of graphs, in incidence matrices of hypergraphs, the sum of the elements in a column can sum up to a value larger than 2. In fact, this value is the size of the relevant hyperedge. Moreover, similar to graphs, the degree of a vertex $v_i$ is equal to the sum of the inputs on the $i$-th row of the incidence matrix.  

For example, the incidence matrix of the hypergraph in Example \ref{ex22} is
\begin{equation*}
    I=\begin{pmatrix}
    1 & 1 & 1 & 0 \\
    0 & 1 & 1 & 1 \\
    0  &  0 & 1  & 1 \\
     0  & 0  & 0  & 1 \\
     0  & 0  & 0  & 1
    \end{pmatrix},
\end{equation*}
where $v_1 =a, \ v_2 =b, \ v_3 =c, \ v_4 =d,\ v_5 =e $.

Adjacency matrix of a graph or a hypergraph completely encodes the connectivity of the graph or hypergraph. Namely, adjacency matrix of a hypergraph is an $n \times n$ matrix $A=(A_{ij} )$ where $A_{ii} =0$ and for $i \neq j$, $A_{ij} $ is the number of hyperedges that contain both $v_i$ and $v_j$. Moreover, it can be represented in terms of the incidence matrix as
\begin{equation*}
    A=II^T -D,
\end{equation*}
where $D$ is the diagonal matrix whose diagonal entries are the number of hyperedges a vertex belongs to. 

As an example, let us construct the adjacency matrix of the hypergraph in Example \ref{ex22} that can be given by 
\begin{equation*}
    A=\begin{pmatrix}
    0 & 2 & 1 & 0 & 0 \\
    2 & 0 & 2 & 1 & 1 \\
    1  &  2 & 0  & 1 & 1 \\
     0  & 1  & 1  & 0 & 1 \\
     0  & 1  & 1  & 1 & 0
    \end{pmatrix}.
\end{equation*}
As for hypergraphs in which hyperedges are weighted, the adjacency matrix can be introduced as 
\begin{equation*}
    A=IWI^T -D, 
\end{equation*}
where $I$ is the incidence matrix, $W$ is the diagonal matrix with the weights of the hyperedges along the diagonal, and $D$ is the diagonal matrix whose diagonal entries are the number of hyperedges a vertex belongs to.

%One possible definition is presented in \cite{Arruda}. 



\subsection{Questions}
\begin{itemize}
    \item does weighted have a unique equivalent weighted network (as in 'Dynamical systems on hypergraphs' by Carletti et al.)? Is it useful just for Laplacians or for other purposes too?
\end{itemize}







\section{Ecological examples}
\subsection{Competition}
A study of the competition between plants~\cite{Mayfield2017} provided empirical evidence that "higher-order interactions strongly influence species’ performance in natural plant communities, with variation in seed production (as a proxy for per capita fitness) explained
dramatically better when at least some higher-order interactions are considered."

The model (see Fig.~\ref{fig:Mayfield_1}) predicts a plant fecundity through an exponential dependence on the total abundance of plants in the area as well as on products of abundances of two plants. As these two species can be different from the influenced species in question, it constitutes a higher-order interaction. 

In ubergraph picture this higher-order interaction would contain two vertices influencing a self-loop of a vertex.

The predictive power of the model has gained significantly when such higher-order interactions were present (see Fig.~\ref{fig:Mayfield_2}).


\begin{figure}[h!]
	\begin{center}
        \subfloat{\includegraphics[width=0.6\linewidth]{./fig/Mayfield_1.png}}
        \subfloat{\includegraphics[width=0.4\linewidth]{./fig/Mayfield_1_1.png}}
        \caption{Impact on fecundity (number of offspring) of plants from the abundances of other plants - in \cite{Mayfield2017}.} 
    	\label{fig:Mayfield_1}
    \end{center}
    \end{figure}  

\begin{figure}[h!]
	\begin{center}
        \includegraphics[width=\linewidth]{./fig/Mayfield_2.png}
        \caption{The goodness-of-fit results of the nonlinear model of plant competition in \cite{Mayfield2017}.} 
    	\label{fig:Mayfield_2}
    \end{center}
    \end{figure}


Other competitive examples include e.g. secretion of chemical compounds affecting many other species (antibiotics). [TO DO]




\subsection{Scavengers}
A 2014 review~\cite{Moleon_scavenging} of ecological literature describing scavenging highlights the intertwined interactions between the live prey, its predators and scavengers. In particular, scavenging indirectly affects the population dynamics of consumed organisms. The familiar model in which vertices exercise influence on edges in a digraph (see Fig.~\ref{Moleon}) shows a clear case for a hypergraph application.

\begin{figure}[h!]
	\begin{center}
        \includegraphics[width=0.7\linewidth]{./fig/Moleon_1_model.jpg}
        \caption{A model of four-way live prey-predator-carrion-scavenger interactions in \cite{Moleon_scavenging}, adapted from \cite{Getz_2011}.} 
    	\label{Moleon}
    \end{center}
    \end{figure}

  A recent model of the  scavenger-predator-prey-carrion interaction~\cite{Mellard2021} employs non-polynomial four-way interaction terms between them (see Fig.\ref{Mellard_fig}). Predators and scavengers affect each other by changing the time each spends handling its prey. In short, the abundances of prey $R$, predator $P$, carrion $C$ and scavenger $S$ change according to
\begin{align}
    \dot{P}&=P[-m_P + R k_P(R,C,S) +C q_P(R,C,S)] \\
    \dot{S}&=P[-m_S + R k_S(R,C,P) +C q_S(R,C,P)] \\
    \dot{R}&=g(R) - R[P k_P(R,C,S) - S k_S(R,C,S)] \\
    \dot{C}&=P k'_P(R,C,S) + S k'_S(R,C,P) - P q'_P(R,C,S) -S q'_S(R,C,P) - C\delta ]
\end{align}
where $k_i, q_i,k'_i, q'_i$ with $i \in {P,S}$ are rational functions of indicated variables, resulting from the functional responses of predators attacking prey to the abundance of prey and predators~\cite{holling_1959}. All other symbols: $m_P, m_S, \delta$ refer to constants.

\begin{figure}[h!]
	\begin{center}
        \subfloat{\includegraphics[width=0.49\linewidth]{./fig/Mellard_1_model.jpg}}
        \subfloat{\includegraphics[width=0.49\linewidth]{./fig/Mellard_2_results.jpg}}
        \caption{Left: an explicit dynamical model of four-way live prey-predator-carrion-scavenger interactions in \cite{Mellard2021}. Right: the effects of scavengers on predator kill rates as observed in real ecosystems.} 
    	\label{Mellard_fig}
    \end{center}
    \end{figure}  

 
\subsection{Lichens and other composite organisms}

\section{Other notes}

\section{Previous notes from Section 3}


\subsection{Graph theory toolbox}
In the ecological modelling a network serves as a basic tool to represent a food web. In our considerations we start with weighted multilayer digraph, which in classical theory not only bears the message on trophic relationship but allows to indicate e.g. the trophic level by the position in the multistructure. In this research we argue that multilayer structure can be interpreeted in the context of long-distance interection between entities in the food web. 

Let us change classical definition of multilayer weighted digraph $\mathcal{G} = (V,E,L,\phi)$, \cite[Sec.~2.1]{KivArena2014}, into the one that seem far way more complicated at the first glance, but allows for easy genaralisation.

\begin{definition}%[Weighted digraph]
    \label{def: WeightedDigraph}
     \textbf{A weighted multilayer digraph} is an ordered triple $\mathcal{G} = (V,E, \phi)$ where
    \begin{enumerate}
        \item $V = \{v_i\; | \; i \in I\}$, $|I|=n$, is a set of elements, called vertices;
        \item $E = \{e_k\; | \; k \in K\}\subseteq V \times V$, $|K|=M$,   is a set of ordered pairs of elements from V, called \textbf{edges};
        \item function $\phi: E \xrightarrow{} \mathbb{R}$ assigns a weight to each edge.
    \end{enumerate} 
\end{definition}



\subsection{Weighted directed hypergraph}

Hypergraphs generalise graphs by allowing edges to contain any number of vertices. To adequately describe ecosystems and generalise weighted digraphs we introduce weighted dihypergraphs~\cite{Bretto2013} straight away. They distinguish points of entry (tails) and exit (heads) among the vertices belonging to a hyperedge.
\begin{definition}%[Hypergraph]
\textbf{A weighted dihypergraph} $H$ is an ordered triple $H=(V,E,\phi)$, where
\begin{enumerate}
    \item $V=\{v_i : i\in I \}$ is the set of vertices (vertices) $v_i$,
    \item $E=\{(e_j^{\mathrm{in}},e_j^{\mathrm{out}}) : j\in J \}$ is the set of directed hyperedges (hyperarcs), where each hyperedge $e$ consists of nonempty subsets of $V$ called tails and heads;
    \item $\phi: E \rightarrow \mathbb{R}$ asigns weights to hyperarcs.
\end{enumerate} 
\end{definition}

\subsection{Diubergraph}

An ubergraph~\cite{Joslyn2017UbergraphsAD} is a generalisation of a hypergraph in which hyperedges are allowed to contain not only vertices but other edges as vertices as well. In other words, in an ubergraph, an edge (called uberedge) can consist of \mdel{some }vertices and \mdel{some }other edges. \madd{This way, some edges play a second role, analogous to a digraph vertex.} For the formal definition of ubergraphs we need the following notation.  

Let $X$ be a finite set. We denote 
    \begin{equation}
        \mathcal{P}(X)^k := \mathcal{P} \left( \bigcup_{i=0}^{k} P_i \right),
    \end{equation}
where $\mathcal{P}(X)$ denotes the family of all subsets of $X$ and
    \begin{equation}
        P_0 = X, \quad P_i =\mathcal{P} \left( \bigcup_{j=0}^{i-1} P_j \right), \quad i\geq 1. 
    \end{equation}
As a consequence of this iterative process, we have $\mathcal{P}(X)^0 =\mathcal{P}(X)$ and $\mathcal{P}(X)^1 =\mathcal{P}(X\cup \mathcal{P}(X) )$ and so on. 

\begin{definition}
    A depth $k$ ubergraph is a pair $U=(V,E)$ where 
    \begin{itemize}
        \item[1] $V$ is a non-empty set of fundamental vertices,
        \item[2] $E \subseteq \mathcal{P}(V)^k$ is a finite set of \gadd{nonempty} uberedges,
        \item[3] if $s$ belongs to an uberedge and $s \notin V$, then $s$ is itself an \gadd{uber}edge. 
    \end{itemize}  
     We say that elements of $V\cup E$ are vertices, while for every $e \subset E$ elements of $e \setminus V$ are called edges. 
\end{definition}
\madd{The third condition ensures that edges participating in another uberedge already exist.}
\mat{An alternative definition introducing less symbols and consistent with layers of uberedges:}
\begin{definition}
    For a finite set $X$ we denote the family of all ordered subsets of $X$ by $\mathcal{P}(X)$. We introduce a series $P_k(X)$,
\begin{equation}
        P_0(X) = X, \quad P_1(X) = \mathcal{P}(X), \quad P_{k+1}(X) =\mathcal{P} \left( \bigcup_{i=0}^{k} P_i(X) \right), \quad i\geq 1.
\end{equation}
Elements of $P_k(X)$ are called \textbf{depth-$k$ uberedges}\footnote{We correct the definition of \cite{Joslyn2017UbergraphsAD} where $P_0$ was inconsistent with the recursive formula.}.
\end{definition}
\madd{Digraph edges are depth-one uberedges.}

\begin{definition}
    For $k\geq 1$, a \textbf{depth-$k$ diubergraph} is a triple $U=(V,E, \phi)$ where 
    \begin{itemize}
        \item[1] $V$ is a non-empty set of fundamental vertices,
        \item[2] $E \subseteq P_k(V)$ is a finite set of nonempty uberedges,
        \item[3] if $s$ belongs to an uberedge and $s \notin V$, then $s$ is itself an uberedge,
        \item[4] $\phi:E\rightarrow \mathbb{R}$ assigns weights to uberedges. 
    \end{itemize}  
     We say that elements of $V\cup E$ are vertices, while for every $e \subset E$ elements of $e \setminus V$ are called edges.   
\end{definition} 
\mat{Do we find this definition of edges useful or should we rather call uberedges edges interchangeably?}
\madd{A digraph is a depth-one diubergraph.}


We note that each uberedge can be explicitly written as a nested set of posibly ordered sets of fundamental vertices from $V$. We call it a \emph{fundamental form of an uberedge}, and denote as $e(V)$. For example, let $V=\{v_1, v_2,v_3\}$, $e_1=(v_1,v_2)$ and $e_2=(v_3,e_1)$.  Then,
\begin{equation}\label{Fuedge}
    e_2(V)=(v_3, (v_1, v_2)).
\end{equation}


\subsection{Incidence graph representation of an ubergraph}
The ubergraph concept is a natural generalisation of a graph. It is also the structure we immediately imagine when we think of vertices impacting processes (edges) rather than vertices themselves. It has been explicitly drawn in numerous applied studies~\cite{Bairey2016, Moleon_scavenging}. It can also be represented through its digraph incidence (Levi \footnote{vertices from V plus U}) graph.
The incidence matrix of an ubergraph can be interpreted as an adjacency matrix of a directed graph consisting of the joint sets of vertices and uberedges.
They are connected by the relation of being a part of another.
\begin{definition}\label{uber_levi_graph}
    A digraph incidence representation of an ubergraph $\mathcal{U}=(V,E)$ is a digraph $\mathcal{G}=(V \cup E,E')$ where for $v, w \in (V \cup E)$ 
    $$(v,w) \in E' \iff v \in w. $$
 
\end{definition}

Fig.\ref{fig:PASB_ubergraph_incidence_graph} shows an example of an ubergraph and its digraph incidence representation. The fundamental vertices $V$ can be identified as the only ones having zero in-degree.
\begin{figure}[h!]
	\begin{center}
        \subfloat{\includegraphics[width=0.42\linewidth]{./fig/Azteca_example/4_node_Azteca_v_labels.png}
		}
        \subfloat{\includegraphics[width=0.5\linewidth]{./fig/Azteca_example/incidence_4_node_Azteca.png}}
        \caption{A 4-node subgraph of the coffee agroecosystem model of \cite{GOLUBSKI2016344} (left) and its digraph incidence representation (right). Colours map the depth of an uberedge, and match those used in the original paper. Fundamental vertices (depth-zero) are green, depth-one edges are black, depth-two blue and depth-three edge red.} 
    	\label{fig:PASB_ubergraph_incidence_graph_appendix}
    \end{center}
    \end{figure}




 \madd{Paths from fundamental vertices to an uberedge define their role in it. Operationally, passing through a vertex on the path adds a bracket around objects pointing to this vertex in the fundamental form of an uberedge.} Physically, these layers represent also the interaction order and its natural strength, estimated by dimensional analysis.

Such a representation is more complicated than the straightforward ubergraph definition. However, it might be used to translate any analysis or algorithm defined for ubergraphs to one performed on digraphs.

\subsection{Classes of hyper- and uberedges}\label{sec:classes_of_hyper_uber}
Hyperedges of differing cardinality can be expected to be of different physical and causal nature. The same applies to uberedges which in addition differ by the way they are composed of fundamental vertices. This leads us to define classes of hyper-/uberedges $\mathcal{\tE}_e$ as the sets of existing edges that could be obtained from a given edge $e$ by renaming the fundamental vertices. 

\begin{definition}\label{def:classes_of_hyper_uber}
Edges $e$ and $e'$ belong to the same class $\mathcal{E}_e$ $\iff \exists$ a \madd{map} $\Pi: V\rightarrow V$, such that $e(V)=e'(\Pi(V))$.
\end{definition}
In a way, an uberedge defines its class through the the way the brackets are placed in its fundamental form. Def.\ref{def:classes_of_hyper_uber} partitions the set of uberedges into equivalence classes.
\begin{plain}
	Let us consider an ubergraph $U=(\{u,v,w\},\{e_1=(u,(v,w)), e_2=((u,v),w),$ $e_3=(w,(u,v)),e_4=(u,(v,v))\})$. The edge set consists of two classes:
	$\mathcal{E}_1=\{e_1,e_3,e_4\}$ and $\mathcal{E}_2=\{e_2\}$.
\end{plain}


\subsection{Ecograph - flows of matter and their functional dependence}
\begin{definition}%[Food web]
\label{def: FoodWeb}
\textbf{A food web} is a connected weighted digraph $\mathcal{G} = (V,E,\phi, x)$ with flows $\phi: E \rightarrow \mathbb{R}_+ $ and biomasses $x: V \rightarrow \mathbb{R}_+ $. Vertices $V$ are subdivided into living, and non-living (detrital). 
\begin{equation*}
    V=\{v_i: i\in I_L\} \cup \{v_i: i\in I_{nL}\}, 
\end{equation*}
where $I_L=\left\{1,\ldots,l\right\}$, $I_{nL}=\left\{l+1,\ldots,n \right\}$ and $I=I_L\cup I_{nL}$.
\end{definition}

\madd{We propose an explicit structure to represent ecosystem interactions, which we call an ecograph. Ubergraphs can model all the examples from ecological literature without loss of information. However, a less general object suffices to represent them.  
All the reported ecological examples contain food web flows (or abundance changes) and interactions in which one vertex influences other interactions.}
\begin{definition}[Alternative Definition]
An ecograph is an ordered triple $\mathcal{G} = (V,U, \phi)$ where $V=\{v_i : i \in I \}$ is the set of fundamental vertices, $U=\{ u_j : j \in J\}$ is the set of uberedges such that an uberedge $u_j$ can consist of either any number of fundamental vertices (i.e. $ u_j = \{v_i, \ldots, v_j \}$) or some fundamental vertices and only one uberedge (i.e. $ u_j = \{v_i, \ldots, v_j, u_k \}$). The weights of the uberedges are assigned via the function $\phi$. 
\end{definition}

\badd{A vertex in a food web represents a single species participating in trophic relationships or a resource supplying energy or matter to the ecosystem. However, representing higher-order interactions might necessitate the inclusion of other vertices which do not serve either of these roles, although they do influence them. Such vertices could represent both biotic and abiotic external factors, e.g. luminosity, temperature or pesticide presence~\cite{Cervantes_2021, Polatto2014}. }
%TO DO: example of influence of another species as an external factor
%Could we not also have several fundamental vertices influencing an uberedge?

\badd{Since the interacitons involving these vertices are of different kind, they could be represented in a separate way. An approach similar to multilayer networks could be considered, as one of their common purposes is representation of multiple interaction types, including non-trophic ones~\cite{Kefi2016}. A simpler approach would be to incorporate the external factors as vertices on a par with those involved in trophic relationships. Any of these special vertices could be involved in multiple uberedges, although each of them would consist only of said vertex and a single edge representing the influenced interaction. This means the depth of any uberedge involving this vertex would be equal to at least $2$. 
}
\section{Backup}\label{backup}
\subsubsection{Weighted multilayer digraph}
Let us change the classical definition of weighted multilayer  digraph $\mG = (V,V_L,E,\phi)$, \cite[Sec.~2.1]{KivArena2014}, into one that is slightly more complicated, but allows for easy generalisation. Let us denote by $\mathcal{P}^*(\cdot)$ a power set in which an empty set is excluded.
%\mat{I have a feeling that it risks confusing the reader at the very beginning - at worst losing his interest in reading further. If we would like to keep this insightful version of digraph definition, we could start from the gently built diubergraph definition and then show that digraphs are a special case, satisfying the equation (1). After we agree on the general order of presentation I will also try making it all easier for general readers.}
%\begin{definition}%[Weighted digraph]
%    \label{def:WeightedDigraph}
%     \aadd{\textbf{A weighted multilayer digraph} is an ordered tuple $\mG= (V,L,V_L,\EG,\phi)$ where}
 %   \begin{enumerate}
 %       \item \aadd{$V = \{v_i\; | \; i \in I\}$, $|I|=n$, is a set of vertices;}
 %       \item \aadd{$L=\{L_j\; | \; j \in J\}$, $|J|=d$, is a set of layers;}
 %       \item \aadd{$V_L \subset L\times V$ is a set of layer-vertex pairs; }
 %       \item \aadd{$E = \{e_k\; | \; k \in K\}\subseteq \mathcal{P}^*(V_L) \times \mathcal{P}^*(V_L)$, $|K|=m$,   such that}
 %   \begin{equation}\label{eq:edge}
 %       \aadd{\forall_{(E_{in},E_{out})\in E}\,\,|E_{in}|=|E_{out}|=1,}
 %       \end{equation}
 %       \aadd{is a set of edges;}
 %       \item \aadd{function $\phi: E \xrightarrow{} \mathbb{R}$ assigns a weight to each edge.}
 %$   \end{enumerate} 
%\end{definition}

\begin{definition}%[Weighted digraph]
    \label{def:WeightedDigraph}
     \textbf{A weighted multilayer digraph} is an ordered tuple $\mG= (V,V_L,E,\phi)$ where
    \begin{enumerate}
        \item $V = \{v_i\; | \; i \in I\}$, $|I|=n$, is a set of vertices;
        \item $L=\{L_j\subset V\; | \; j \in J\}$, $|J|=d$, is a set of layers;
        \item $E = \{(e_k^{\tin},e_k^{\out})\; | \; k \in K\}\subseteq \mathcal{P}^*(V) \times \mathcal{P}^*(V)$, $|K|=m$,   such that for any $e=(e^{\tin},e^{\out})\in E$
    \begin{equation}\label{eq:edge}
        |e^{\tin}|=|e^{\out}|=1,
        \end{equation}
        is a set of edges;
        \item function $\phi: E \xrightarrow{} \mathbb{R}$ assigns a weight to each edge.
    \end{enumerate} 
\end{definition}

%\madd{In analogy to vector coordinates we will use superscripts "in" and "out" to denote the first and second element of an ordered set, respectively. }
%\mat{Can it be written better? The aim is to simplify notation - if we write $x_i$ for the $i$-th coordinate, we want to write $e^{in}$ for the set of tails.}

\aadd{We can easily note that in this approach one vertex can belong to arbitrary many layers. If $V_L$ is a partition of $V$ into nonempty subsets then following the standard nomenclature we say that $\mathcal{G}$ is \textbf{layer-disjoint}. If there exists exactly one layer we call it a \textbf{weighted digraph} and denote by $\mG= (V,\EG,\phi)$.}

\madd{An edge $(e^{\tin},e^{\out})\in E$ represents a connection from its tail $e^{\tin}$ to its head $e^{\out}$.} Following standard definitions edges that stay within a single layer are called \textbf{intra-layer}, while those which cross layers are called \textbf{inter-layer} edges. \textcolor{red}{Consequently we define an intra-layer graph, an inter-layer graph and a coupling graph. }
\al{To be added.}

\madd{Let us summarise a few standard matrices that characterise digraphs. 
The interactions in a multilayer digraph can be encoded in its incoming and outgoing adjacency tensors $\adji, \adjo$ with indices $i,j\in I$ and $l,k\in J$.}
\begin{multicols}{2}
    \begin{equation}
        \mathbb{A}^{\tin}_{ij;lk}=\phi( (v_i^l, v_j^{k}) ),
    \end{equation}
\break
    \begin{equation}
        \mathbb{A}^{\out}_{ij;lk}=\phi( (v_j^l, v_i^{k}) ),
    \end{equation}
\end{multicols}
They describe the incoming and outgoing interactions of vertices $v_i^l, v_j^k\in V$ belonging to the respective layers $L_l$ and $L_k$, with $l,k \in J$.
\madd{Supra-adjacency matrices flatten the layer dimension, reducing the number of indices. They treat the multilayer digraph $(V,V_L,E,\phi)$ as a digraph $(V \times V_L, E,\phi)$. If $d_i$ denotes the number of vertices in layers below layer $i$,
\begin{multicols}{2}
    \begin{equation}
        \adji_{d_i+l\, d_j+k}=\mathbb{A}^{\tin}_{ij;lk},
    \end{equation}
\break
    \begin{equation}
        \adjo_{d_i+l\, d_j+k}=\mathbb{A}^{\out}_{ij;lk},
    \end{equation}
\end{multicols}



Let $D=\sum_{i=1}^d|V_i|$ and $D_j=\sum_{i=1}^{j-1}|V_i|$. $ D\times D$ matrix $\mathbb{A}=(\mathbb{A}^{ij})_{i,j\in J}$ is supra-adjacency matrix if it is a block matrix where for any fixed $i,j\in J$ $\mathbb{A}^{ij}$ is $|V_i|\times |V_j|$ matrix such that 
$$
\mathbb{A}=A_{ij}^{lk}
$$
\al{To be corrected.}
}

\madd{A supra-adjacency matrix describes a multilayer digraph as if it were an ordinary digraph, storing the layer information in vertex order. In the rest of the article we drop the layer indices.}

\madd{Incidence matrices $\inci, \inco: V \times E \rightarrow {0,1}$ offer an alternative definition of digraph interactions, by stating which vertices are tails or heads of which edges.}
\begin{multicols}{2}
\begin{equation}
    \inci_{ij}=\begin{cases}
        1, \quad \text{if} \quad v_j \in e_{i}^{\mathrm{in}}\\
        0, \quad \text{otherwise}.
    \end{cases}
\end{equation}
\break
\begin{equation}
    \inco_{ij}=\begin{cases}
        1, \quad \text{if} \quad v_j \in e_{i}^{\mathrm{out}} \\
        0, \quad \text{otherwise}.
    \end{cases}
\end{equation}
\end{multicols}
\madd{We store the information about hyperedge weights in a weight matrix $\Phi: E \times E \rightarrow \mathbb{R}$,}
\begin{equation}\label{eq:weight_matrix}
    \Phi(e_i, e_j)=\begin{cases}
    \phi(e_i), \quad \text{if } i = j  \quad\text{and} \quad e_i \in E \\
    1, \quad \text{if } i = j \quad \text{and} \quad e_i \in V.
    \end{cases}
\end{equation}
\madd{The adjacency matrix can then be computed as $(\inco)^T \Phi \inci$.}

\al{We have to choose a notation $I, I^{\HH}$?}
\mat{We have $I=I^{\HH}$, while for $I^U$ index i runs over $V\cup \EU$. $\leftarrow$ that was a good premonition of incidence matrix issue}



%\mwa{What is $D_j=\sum_{i=1}^{j-1}|V_i|$ and $ \mathbb{A}_{d_i+l\, d_j+k}\mathbb{A}_{ij}^{lk} $?}
%\al{It is not finished. Please give me some time. :)}

\section{Additional copy}
\subsubsection{Weighted multilayer dihypergraph}\label{sec:H_graph}
One can also allow each edge to have more than one head and one tail. A \textbf{weighted multilayer dihypergraph} $\mH = (V,V_L,\EH,\phi)$ satisfies all conditions of Definition \ref{def:WeightedDigraph} except \eqref{eq:edge}. In order to distinguish elements from a set $\EG$ from those in $\EH$ we call the latter hyperedges. This notion has been already used in ecological context in the undirected version, see \cite{Bretto2013}.

Let us summarise standard objects defined for dihypergraphs that will be generalised in the next subsection. Unlike in a digraph, there may be several different hyperedges all containing the same vertex $v_i$ as a head and the same $v_j$ as a tail. Thus, there is no one-to-one correspondence between a dihypergraph and its adjacency matrix $A=(A_{ij})_{i,j\in I}$~\cite{BATTISTON20201},
    \begin{equation}
        A_{ij}=\sum_{\left\{e:\,\, v_i \in e^{\mathrm{out}}, \, v_j \in e^{\mathrm{in}}\right\}} \phi(e).
    \end{equation}
\aadd{In the theory of hypergraphs there are some generalisations know as adjacency tensor, degree normalized k-adjacency tensor, eigenvalues normalized k-adjacency tensor etc. To avoid this ambiguity it the paper we use the notation related to incidence matrix.}     
\al{We have two topics to understand.\newline
1. What is the relation between incidence matrix ($\mathcal{I}$), adjacency matrix ($A$) and Laplacian $\mathcal{L}$ in the theory of dihypergraphs (with positive weights).
2. How to combine all notions with arbitrary sign weights.}

\al{In the unnormalised case (unnormalised Laplacian) we have, \cite{Mugnolo}:
\begin{equation}
\mathcal{L}=\mathcal{I}\mathcal{I}^T=D-A
\end{equation}
In the normalised graph case we have, \cite{Li2012Dilaplacian}:
\begin{equation}
\mathcal{L}=D^{-\frac{1}{2}}(D-A)D^{\-\frac{1}{2}}.
\end{equation}
In the unnormalised digraph case we have, \cite{Boley2016}
\begin{equation}
\mathcal{L}=D^{out}-A
\end{equation}
In the normalised digraph case we have [broken Latex equation, commented out], 
\cite{Li2012Dilaplacian}: }
%\begin{equation}
%\mathcal{L}=\Phi^{\frac{1}{2}}(I-D^{out}^{-1}A)\Phi^{-\frac{1}{2}}, \quad \text{where %}\Phi^T(D^{out})^{-1}A=\Phi. 
%\end{equation}




On the contrary, the \emph{incidence matrix} $I^{\HH}=(I_{ij}^{\HH})_{i\in I\,j\in K}$ uniquely determines the hypergraph,
\begin{equation}\label{eq:hypergraph_incidence}
    I_{ij}^{\HH}=\begin{cases}
        \phi(e_{j}), \quad \text{if} \quad v_i \in e_{j}^{\mathrm{in}} \\
        -\phi(e_{j}), \quad \text{if} \quad v_i \in e_{j}^{\mathrm{out}} \\
        0, \quad \text{otherwise}.
    \end{cases}
\end{equation}
\mat{when checking direction convention, we go with $A_{ij}$, from $j$ to $i$, I had a problem with incidence like above. Formulas like $I^T I$ do not produce anything reasonable for directed hypergraphs - you could go from tail to tail, or head to head.
It seems we do need two incidence matrices: 'in' and 'out':
$I_{in}=I[\text{entries}>0]$,$I_{out}=I[\text{entries}<0]$, then $A=-I_{out}^T I_{in}$}

\al{The matrix that you define below is called signless incidence matrix ref: Mugnolo eq. (2.3) but for $V\times V$. In my opinion $I^{\mathcal{U}}$ should be defined on $V_{{\mathcal{U}}}$. So for hypergraph we obtain classical def.}

\begin{figure}[h!]
	\begin{center}
        \includegraphics[width=\linewidth]{./fig/Laying_out_motivation.jpg}
        \caption{Laying out our motivation} 
    	\label{motivation}
    \end{center}
    \end{figure}

    \begin{figure}[h!]
	\begin{center}
        \includegraphics[width=\linewidth]{./fig/Narrative.jpg}
        \caption{Narrative ideas} 
    	\label{motivation}
    \end{center}
    \end{figure}


    \end{document}