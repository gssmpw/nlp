\section{Literature review}
Relevant articles:
\begin{itemize}

\item \cite{Cervantes_2021} presents an analysis of experimental data on behavior of four pollinator species in controlled conditions. The main focus is the influence of resource availability and presence of pesticide on pollinator-pollinator interactions, both between species and within one. Statistical tools show that out of the three presented possible models, one accounting for these environmental conditions is by far the best. Their influence differs between species but is clearly present. The authors stress that the environmental context in which pollinators interact with plants and with each other is the crucial part of the interactions. \\
I think this is a good example of an ecological paper describing a higher-order interaction, since the influence of other species is impacted by the environment (available amount of flowers and pesticide presence).
\begin{enumerate}
    \item adding effects from a distinct type of objects calls for a multilayer network formulation more than for a hypergraph 
    \item binary/discrete case, like presence/absence of a pesticide can be modelled through a multiplex network (with two/more copies of the underlying bipartite graph plant-pollinator)
    \item continuous case (continuous resource $R$ availability in formula (11)): a multilayer network with one layer of the bipartite graph, and another with the vertices influencing continuously the interactions in the bipartite graph
\end{enumerate}

\item \cite{Grilli2017} considers higher-order competition models. Their research stems from the lacking stability of of simple models as well as neutral models. They focus on large communities where competitors interact. They note that the impact of higher-order interactions of species is crucial on the existence of other species. They consider both deterministic and stochastic models.

Their narrative is based on a forest with $ m $ species. Given that a tree of species $ i $ dies, there is then a competition to fill up the gap created. There are two outputs,
\begin{enumerate}
    \item Deterministic model: An equivalence is shown where the competition is done in a sequence (where interactions are still done pairwise at a time, but with the winner of a game proceeding to compete with the next competitor) and a simultaneous competition (where trees from all $ m $ species compete at the same time).
    \item Stochastic model: Higher-order interactions leads to longer periods of coexistence of species.
\end{enumerate}

    \item \cite{Arruda} 
    has a good intro on dynamics on hypergraphs and linear stability analysis
    
    In the paper the hypergraph $\HH =\left\{\V,\E\right\}$ is defined in the following way. $\V=\left\{ v_i:\,\,i\in I\right\}$ is a set of verticies, $N=|\V|$ and $\E=\left\{e_j\right\}$ is a set of hyperedges, and $e_j\in 2^{\V}$ has a cardinality $|e_j|$. By $\E_i$ we also denote the set of hyperedges that contain the vertex $v_i$. The weighted adjacency matrix can be defined as

\[
A_{ij}=\sum_{e_j\in \E;\,\, i,k\in e_j;\,\,i\neq k }\frac{w_{ij}(e_j)}{|e_j|-1}
\]

where $w_{ik}(e_j)\in \RR$ is a weight.

    Let us consider a system

    \begin{equation}\label{eq:main}
\frac{dx_i}{dt}=f_i(x_i)+\sum_{e_j\in \E_i}g_j(x_{\left\{e_j\right\}}),
    \end{equation}
    where $x_i$ is the state of the vertex $v_i$, $f_i:\RR \rightarrow \RR$ is function that depends only on the state $x_i$ and $g_j(x_{\left\{e_j\right\}}):\RR^{|e_j|}\rightarrow \RR$ that takes all the states of the vertices on the hyperedge $e_j$, here denoted as $x_{\left\{e_j\right\}}$, and compute its contribution to $x_i$.

    Its linear stability in the neighbourhood of fixed point $x^*$ depends on the spectrum of matrix $M=F(x^*)+G(x^*)$ where 
    \[
    G(x^*)=\sum_{m=1}^{\max{|e_j|}}G^m(x^*), \qquad G^m_{ik}(x^*)=\sum_{e_j\in \E_i;\,\,|e_j|=m}\partial_{x_k}g_j(x_{\left\{e_j\right\}})|_{x=x^*}.
    \]

   %Two particular examples of system \eqref{eq:main} are considered.

    %\begin{enumerate}
    %\item diffusion-like process on hypergraph
    %\[
    %\frac{dx_i}{dt}=f_i(x_i)+\frac{1}{|e_j|-1}\sum_{k\in e_j;\,\, k\neq i} (g(xi)-%g(x_k))
    %\]
    %\item 
    %\end{enumerate}
    %\item \cite{Carletti_2020} 
    %discusses a small subclass of dynamical systems on unweighted hypergraphs though

    \item \cite{Bairey2016} is the most typical example of use of a dynamical model with higher-order (non-bilinear) interactions in mathematical ecology. This is what we might want to cover - how to translate it to a weighted hypergraph properly?
    
    They used a replicator equation with non-linear fitness, see Fig.~\ref{Bairey_eq_ex} (for our reference). The model parameters are given by random two-, three- and four-dimensional tensors with entries drawn from a Gaussian distribution (mean 0, variance 1). It describes evolving fractional abundances of species whose total population is constant. The initial fractional abundances are equal ($\tfrac{1}{N}$, where $N$ is the number of species/variables).
    \begin{figure}[h!]
	\begin{center}
        \subfloat{\includegraphics[width=0.49\linewidth]{./fig/Bairey_1_equation.jpg}
		}
        \subfloat{\includegraphics[width=0.49\linewidth]{./fig/Bairey_2_example.jpg}}
        \caption{The model of \cite{Bairey2016} (left) and their examples of higher-order ecological interactions of microorganisms (right).} 
    	\label{Bairey_eq_ex}
    \end{center}
    \end{figure}
    To compare the impacts of various terms, they have run numerical simulations separately for only pairwise ($B\equiv 0$, $C\equiv 0$), then only three-way, and only four-way interactions.
    They defined a community as feasible if all the species existed at the end of the simulation with an abundance above $10^{-5}\tfrac{1}{N}$. They defined the \emph{critical strength of interactions} as corresponding to  of the communities exhibiting extinctions. 
    They found that the interaction strength at which $5\%$ of the random systems exhibit extinctions is the lower the more species there are for pairwise interactions and can be higher the more species there are for four-way interactions (Fig.~\ref{Bairey_res})

    \begin{figure}[h!]
	\begin{center}
        \subfloat{\includegraphics[width=0.49\linewidth]{./fig/Bairey_3_result.jpg}
		}
        \subfloat{\includegraphics[width=0.49\linewidth]{./fig/Bairey_4_interpret.jpg}}
        \caption{The results of \cite{Bairey2016} (left) and their description (right).} 
    	\label{Bairey_res}
    \end{center}
    \end{figure}
    

They remark, that local stability analysis models exclusively pairwise interactions, since the equation is linearised around a point and the high-order interactions are therefore embedded in the coefficients of the effective pairwise interactions obtained in the linearisation.

    \item 
    \cite{Young2021}
    introduces a method (a Bayesian generative model) to reconstruct possible higher-order interactions (represented using hypergraphs) from information about pariwise interactions, since actual data is often limited to those. Authors claim their method introduces hyperedges only with enough enough statistical evidence. The possible hypergraphs are evaluated using the formula \(P(H|G) = \frac{P(G|H)P(H)}{P(G)}\). \(P(H|G)\) is the probability of the hypergraph $H$ being a certain way for a given network $G$. $P(G|H)$ is called a projection component and is by analogy the probability of the network $G$ being a certain way given a hypergraph $G$. $P(G)$ is not important and serves as a normalisation constant. $P(H)$ is called a hypergraph prior. \\ 
    $P(G|H)$ follows a simple distribution, being $1$ iff two vertices appear jointly in any of the hyperedges of $H$ and $0$ otherwise. Authors remark that checking if $G$ is equal to the projection of the hypergraph $H$ is not a problem from a numerical point of view.
    $P(H)$ is based on an existing model, Poisson Random Hypergraphs Model (PRHM). The desired properties which the model fulfills are: the size of the interactions varies, not all vertices are connected by a hyperedge and some of the interactions are repeated (I'm not sure why it's good actually). 
    In this model the number of hyperedges connecting a set of vertices is a random variable following a Poisson distribution, with mean $\lambda_k$ dependent on the size of the set. The number of hyperedges connecting a set of $k$ vertices is equal to $A$ with probabily $P(A|\lambda_k) = \frac{\lambda_k^A}{A!}e^{-\lambda_k}$. All the hyperedges are modeled independently. 
    From this authors derive the formula for the probability of any hypergraph (assuming also some maximal hyperedge size), density of which is controlled by the parameters $\lambda_k$. They use what they call `a hierarchical empirical Bayes approach' and treat $\lambda_k$ as unknows drawn from prior distributions, use I believe some information theory and consider the problem of finding appropriate $\lambda_k$s solved. \\ 
    The authors prove two noteworthy properties of their model: firstly, it favors hypergraphs without repeated hyperedges, even though PRHM allows for duplicates. Secondly, the model favors sparser hypergraphs and introduces hyperedges only when it is justified. Those two properties allow the authors to conclude that such minimal hypergraphs are high-quality local maxima of $P(H|G)$. When it comes to the numerical point of view, the task is quite demanding, therefore some agorithm is proposed. In the conclusion authors say they would like to see a better one. \\
    Authors give several examples with both empirical and artificial data used, with the most attention given to an example of 613 American football games between 115 teams. Teams may play each other more often because of various reasons, mainly conferences (groups of team which all play each other) and geographical proximity. Some of the results are presented in the figures below (Fig.~\ref{Young_football_1}, ~\ref{Young_football_2}). There is also some focus on bipartite networks and the obtainted results show that more complex approach allowing higher-order interactions proved to be better in multiple cases (Fig. ~\ref{Young_bipartite}).

    \begin{figure}[h!]
	\begin{center}
        \includegraphics[width=0.9\linewidth]{./fig/Young_Football_1.png}
        \caption{Results of \cite{Young2021} about american football teams} 
    	\label{Young_football_1}
    \end{center}
    \end{figure}

    \begin{figure}[h!]
	\begin{center}
        \includegraphics[width=0.9\linewidth]{./fig/Young_Football_2.png}
        \caption{More results of \cite{Young2021} about american football teams} 
    	\label{Young_football_2}
    \end{center}
    \end{figure}

    \begin{figure}[h!]
	\begin{center}
        \includegraphics[width=0.5\linewidth]{./fig/Young_bipartite.png}
        \caption{Results of \cite{Young2021} about empirical bipartite networks} 
    	\label{Young_bipartite}
    \end{center}
    \end{figure}

\item [complex hypergraphs are equivalent to either an ubergraph or a multilayer network of two digraphs] \cite{Vazquez2022} introduces a concept of complex hyperhraphs (chygraphs). The main idea is to generalize the concept of hypergraphs beyond ubergraphs. Please note that the paper is not yet reviewed or published.
\begin{definition}[Complex hypergraph]
    A complex hypergraph (chygraph) $\chi(A,C)$ is a set of vertices $A$ and hypergraphs $C$ with vertex sets in $A \cup C$.
\end{definition}
The author gives several examples. For a graph the hypergraphs are said to be edges (which I understand as hypergraphs containing only one pairwise edge each, since they are meant to be actual hyperpgraphs?).
An ubergraph is a chygraph where the hypergraphs contain only one edge. \\
Another example is the system of scientific publications, which is represented by a chygraph $\chi(A, \{ \mathcal{H}_i(A_i \cup R_i, \{A_i, R_i\})\}) $. A publication is represented by a single hypergraph $\mathcal{H}_i$ with two edges $A_i$ (authors) and $R_i$ (references), which do not overlap. Here the difference between a chygraph and an ubergraph is visible: for an ubergraph the way to represent both references and authors would probably be similar, with two edges containing references and authors respectively. However the connection between those edges is important and the chygraph structure allows to include them in one hypergraph, which also has a clear meaning, as it represents a single publication. \\
\begin{figure}[h!]
	\begin{center}
        \includegraphics[width=0.5\linewidth]{./fig/Vazquez_example.png}
        \caption{Illustration of an example from \cite{Vazquez2022}} 
    	\label{Vazquez_example}
    \end{center}
    \end{figure}
The author also defines a matrix for chygraphs, which he decides to call the chy-adjacency matrix (although is seems more like an incidence matrix if anything):
\begin{definition}[Chy-adjacency matrix]
    Let $\chi(A, C=\{ \mathcal{H}_i(V_i, E_i)\})$ be a chygraph. Then the chy-adjacency matrix $\alpha$ is a  $|A\cup C| \times |A \cup C|$ matrix with 
    \begin{equation*}
        \alpha_{ij}=\begin{cases}
            1, \quad \text{if} \ i \in V_j ,& \\
            0, \quad \text{otherwise}
        \end{cases}
    \end{equation*}
    for $i,j \in A \cup C$.
    \end{definition}
    (to me this does not seem correct for several reasons, but maybe I'm missing something) \\
    There is also a definition of the length of the chygraph:
    \begin{definition}[Chygraph length]
        Let $\chi(A, C=\{ \mathcal{H}_i(V_i, E_i)\})$ be a chygraph and let $\Pi = \Pi_1 \cup \dots \cup \Pi_l$ be a partition of $C$ such that $\Pi_i \cap \Pi_j = \emptyset \text{ for } i \neq j$ and if $\mathcal{H}_i \in \Pi_j$, then $V_i \subset A \cup (\bigcup_{k \leq j} \Pi_k)$. The chygraph length $L(\chi)$ is the maximum $l$ between such partitions. 
        \end{definition}
    The second condition means the partition is hierarchical. The author then proceeds to analyze the mean component size. \\ 
    I think the paper seems to contains inaccuracies and does not provide much theory about chygraphs (and does not justify the need to define them). 



\end{itemize}