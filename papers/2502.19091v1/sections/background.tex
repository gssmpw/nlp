This section provides a concise overview of how MASs have evolved. We begin by summarizing their origins and pinpointing the key milestones that have propelled modern MASs into one of the most promising approaches for achieving advanced integrated intelligence. However, it is worth noting that many recent advances in LLM-based MASs stem from leveraging LLMs as a practical means to interface multiple agents, rather than from a direct extension of heuristics-based MAS methods. In other words, recent developments in the field have largely occurred in parallel with\textemdash or even independently of\textemdash traditional MAS research.

\subsection{The Origin of Multi-Agent Systems}
As already mentioned earlier, a MAS architecture, depicted in Figure~\ref{fig:mas-architecture}-a, comprises multiple agents capable of perceiving their environment, reasoning about both local states and shared objectives, and executing actions in parallel to achieve a common goal. By distributing tasks among agents, a MAS leverages specialized capabilities and diverse perspectives, often yielding more robust solutions~\cite{durfee1999distributed}. Early conceptualizations of MASs emerged during the study of distributed problem-solving in the 1980s, spurred by the notion that coordinated groups of autonomous entities can achieve more efficient and reliable outcomes than individuals working alone. Foundational work by Minsky~\cite{minsky1988society}, Wooldridge and Jennings~\cite{wooldridge1995intelligent}, and Stone and Veloso~\cite{stone2000multiagent} established the core principles of agent autonomy, collaboration, and decentralized decision-making. Over the ensuing decades, researchers addressed fundamental challenges in agent-to-agent communication, exploring topics such as task allocation~\cite{sandholm1998contract}, negotiation protocols~\cite{jennings2001automated,shoham1997emergence}, and conflict resolution mechanisms~\cite{rosenschein1994rules}.

\begin{figure}[h]
    \centering
    \includegraphics[width=\linewidth]{images/mas}
    \caption{Evolution of Multi-Agent System Architectures: a) {\em Traditional MAS Architecture}, where agents interact with their environment through observations and actions; b) {\em ReAct Architecture}, an innovative agent design that incorporates advanced reasoning capabilities; and c) {\em LLM-Based MAS Architecture}, a cutting-edge approach leveraging LLMs for reasoning and decision-making.}
    \label{fig:mas-architecture}
\end{figure}

\subsection{LLM-Enhanced Multi-Agent Systems}
Recent advances in LLMs have reignited interest in MASs by enabling more sophisticated reasoning, natural language communication, and advanced planning. While traditional MASs often relied on symbolic or rule-based methods for coordination and decision-making~\cite{genesereth1994software,genesereth1997agent}, modern LLMs can interpret complex instructions, generate contextually relevant responses, and adapt naturally to diverse communication protocols. This significantly reduces the need for manually crafted dialogue policies and negotiation strategies. Several works have already demonstrated that LLMs can serve as the backbone of MASs. For instance, Park et al.\cite{park2023generative} illustrate how LLM-powered agents simulate dynamic social interactions by reasoning about internal goals and social norms. These agents can generate messages to coordinate with others, interpret feedback, and refine their plans in real time, aligning more seamlessly with human-like communication standards\cite{andreas2022language}. Consequently, they are inherently better equipped to tackle complex tasks requiring contextual understanding, creative reasoning, or dynamic problem-solving.

\subsection{ReAct: Reasoning and Action}
In parallel with the evolution of MASs and LLMs, researchers have sought to render each agent's reasoning process more explicit and adaptable. One notable approach is the {\em ReAct} paradigm~\cite{yao2022react} (short for ``Reasoning + Act''), originally introduced for single-agent systems, as depicted in Figure~\ref{fig:mas-architecture}-b. ReAct structures an agent's decision-making into an iterative cycle that goes as follows: ($i$) {\bf observe}, where the agent receives new information from its environment or other agents; ($ii$) {\bf reason}, where the agent produces a {\em chain-of-thought} (CoT), often internally or in a hidden state, to determine the next step; and ($iii$) {\bf act}, where the agent executes a specific action, such as calling a tool, sending a message, or updating its state. This cycle continues until the task is complete. By explicitly separating reasoning from action, ReAct enhances transparency and adaptability, enabling agents to dynamically revise their approaches as contexts evolve~\cite{wei2022chain}.

\subsection{Next-Generation MAS Architectures}
Although ReAct was originally conceived for single-agent scenarios, its design naturally extends to multi-agent systems, where each agent is supported by an LLM (see Figure~\ref{fig:mas-architecture}-c). In this setting, each agent follows a ReAct-style loop to process observations, perform internal reasoning, and act, such as by communicating with other agents or invoking external tools. This integration yields powerful synergies, including:

\setlist{nolistsep}
\begin{itemize}[noitemsep]
\item {\bf Enhanced Coordination}: LLM-based agents can communicate in natural language to negotiate plans, share partial solutions, or request assistance.
\item {\bf Iterative Reasoning and Action}: The ReAct cycle ensures that each agent's CoT remains flexible, context-aware, and up-to-date as it receives new inputs from the environment or from other agents.
\item {\bf Meta-Cognitive Techniques}: Approaches such as reflection~\cite{yao2022react}, task decomposition~\cite{wei2022chain,yao2024tree}, and dynamic tool creation~\cite{qin2023toolllm} can be layered on top of the ReAct loop to enable deeper analysis, more systematic planning, and specialized behaviors.
\end{itemize}

These developments clearly point towards a future where MASs, enhanced by LLMs and meta-cognitive processes like ReAct, can handle sophisticated teamwork and autonomous problem-solving at scales once deemed intractable for traditional MAS approaches. Indeed, recent work has demonstrated that multi-agent setups are particularly effective for tasks such as GUI automation~\cite{agashe2024agent,tan2024cradle,zhang2024ufo} and automatic code debugging~\cite{sami2024aivril,sami2024eda,zhao2024mage}, just to name a few, illustrating a rapidly evolving landscape of possibilities.

\subsection{Modern MAS Frameworks}
These advancements have led to the emergence of various toolkits and frameworks aimed at simplifying the design and deployment of agentic workflows. Projects such as AutoGPT\footnote{AutoGPT introduced a multi-agent paradigm in its most recent release.}\cite{autogpt} and HuggingGPT\cite{shen2024hugginggpt} offer automated pipelines for task decomposition and tool usage. However, they predominantly rely on single-agent paradigms with modular sub-routine execution rather than on fully decentralized, multi-agent collaboration. Other open-source initiatives, including LangGraph~\cite{langgraph}, AutoGen~\cite{wu2023autogen}, crewAI~\cite{crewai}, Dynamiq~\cite{dynamiq}, Magentic-One~\cite{fourney2024magentic}, and Haystack~\cite{haystack}, provide more customizable infrastructures for building multi-agent systems, although they typically require significant coding expertise. By contrast, commercial offerings sometimes feature no-code interfaces, yet often lack transparent integration paths for broader automation.

In addressing this gap, the proposed Nexus framework offers a twofold solution. First, it streamlines the creation and prototyping of complex agentic structures by means of YAML files, significantly reducing the programming expertise required to design architectures for complex problems. Second, it provides straightforward integration with software automation environments\textemdash typically built on top of shell interfaces\textemdash thereby supporting end-to-end automated workflows out-of-the-box. Designed with maximum flexibility in mind, Nexus accommodates multiple supervisors (or orchestrators) within its hierarchical architecture, facilitating the management of highly complex tasks while enhancing decentralization and collaborative problem-solving. All these features distinguish Nexus from existing frameworks and open new avenues for scalable, automated design and deployment across diverse domains.