% SIAM Article Template
 
\documentclass{siamart220329}

%%===========================================================================
\usepackage{amsfonts}
\usepackage{graphicx}
\usepackage{epstopdf}
\usepackage{todonotes}
\usepackage{mathabx}
%% add my own
\usepackage{algorithmic}
\usepackage{algorithm}
\usepackage{enumerate}
\usepackage{bm}
\usepackage{tikz}
\usepackage{diagbox}
\usetikzlibrary{decorations.pathreplacing,calligraphy}
\usetikzlibrary{matrix}
\usetikzlibrary{shapes,patterns,arrows,snakes,decorations.shapes}
\usetikzlibrary{positioning,fit,calc}
\usetikzlibrary{plotmarks}
\usepackage{multirow}
\newtheorem{example}{Example}[section]
\newtheorem{assumption}{Assumption}[section]

\ifpdf
  \DeclareGraphicsExtensions{.eps,.pdf,.png,.jpg}
\else
  \DeclareGraphicsExtensions{.eps}
\fi

% Add a serial/Oxford comma by default.
\newcommand{\creflastconjunction}{, and~}

% Used for creating new theorem and remark environments
\newsiamremark{remark}{Remark}
\newsiamremark{hypothesis}{Hypothesis}
\crefname{hypothesis}{Hypothesis}{Hypotheses}
\newsiamthm{claim}{Claim}
 
% Sets running headers as well as PDF title and authors
\headers{Characterization of NGMRES on Linear Systems}{C. Greif and Y. He}

% Title. If the supplement option is on, then "Supplementary Material"
% is automatically inserted before the title.
\title{A Characterization of the Behavior of Nonlinear GMRES on Linear Systems\thanks{Submitted to the editors DATE.}}

% Authors: full names plus addresses.
\author{Chen Greif\thanks{Department of Computer Science, The University of British Columbia, Vancouver, BC, Canada (\email{greif@cs.ubc.ca}). The work of the first author was supported in part by Discovery Grant RGPIN-2023-05244 of the Natural Sciences and Engineering Research Council of Canada.}
  \and Yunhui He\thanks{Department of Mathematics, University of Houston, 3551 Cullen Blvd, Room 641, Houston, Texas 77204-3008, USA (\email{yhe43@central.uh.edu}).}}


\usepackage{amsopn}
\DeclareMathOperator{\diag}{diag}

%============================================================================


 
% Optional PDF information
\ifpdf
\hypersetup{
	pdftitle={A Characterization of the Behavior of Nonlinear GMRES Applied to Linear Systems},
	pdfauthor={Chen Greif, Yunhui He}
}
\fi


\begin{document}

\maketitle

% REQUIRED
\begin{abstract}
 The Nonlinear GMRES (NGMRES) proposed by Oosterlee and Washio [SIAM Journal on Scientific Computing, 2000, 21(5):1670-–1690] is an acceleration method for fixed point iterations. It has been demonstrated to be effective, but its convergence properties have not been extensively studied in the literature so far, even for linear systems. In this work we aim to close some of this gap. We offer a convergence analysis for NGMRES applied to linear systems. A central part of our analysis focuses on identifying equivalences between NGMRES and the classical Krylov subspace GMRES method. 
 
\end{abstract}
% REQUIRED
\begin{keywords}
 GMRES, nonlinear GMRES, Anderson acceleration, orthogonality, fixed-point iteration  
\end{keywords}

% REQUIRED
\begin{AMS}
65F10,  	% Iterative numerical methods for linear systems 
65F20    	% Numerical solutions to overdetermined systems, pseudoinverses
\end{AMS}
 
%===============================================================================
 \section{Introduction} \label{sec:intro}
	
	Consider solving the system of equations
	\begin{equation}
		g(x)=0,
	\end{equation}
	with a fixed-point iteration: given an initial guess, $x_0$, 
	\begin{equation}\label{eq:FP}
		x_{k+1}=q(x_k), \qquad k=0,1,2,\dots
	\end{equation}
	In practice,  \eqref{eq:FP} may slowly converge or even diverge, depending on the derivative of $q(x)$ and the proximity to a fixed point. Methods have been developed to accelerate \eqref{eq:FP}; examples here are nonlinear GMRES (NGMRES) \cite{oosterlee2000krylov,sterck2012nonlinear} and Anderson Acceleration (AA) \cite{walker2011anderson,anderson1965iterative}, among other instances. 
    
     NGMRES was originally proposed by Oosterlee and Washio~\cite{oosterlee2000krylov} as an accelerator for multigrid applied to nonlinear partial differential equations. Later on, it was studied for other nonlinear problems and applications, such as the alternating least-squares method for solving tensor decomposition problems \cite{sterck2012nonlinear,sterck2013steepest,sterck2021asymptotic}. But its convergence properties have not been broadly explored, despite the enormous popularity of the closely-related (classical) GMRES method for linear systems \cite{saad1986gmres,loe2022polynomial}. It has been observed in \cite{sterck2021asymptotic} that the performance of NGMRES is competitive with Anderson Acceleration, which has been studied more extensively in the literature \cite{ni2009anderson,tang2024anderson,anderson2019comments,de2024anderson,de2022linear,pollock2019anderson,higham2016anderson,potra2013characterization,yang2022anderson,pratapa2016anderson,suryanarayana2019alternating,chen2022composite,ji2023improved}.

Let us provide a brief introduction to NGMRES and AA. Based on \eqref{eq:FP}, we define the residual at the $k$th iterate as
	\begin{equation}
		r(x_k)=x_k-q(x_k)=g(x_k).
	\end{equation}
Then, NGMRES is given in Algorithm \ref{alg:NGMRES}, and AA is given in Algorithm~\ref{alg:AA}. We denote the windowed versions of these iterative schemes with window size $m$ as NGMRES($m$) and AA($m$), respectively, and accordingly, their infinite window versions ($m=\infty$) as  NGMRES($\infty$) and AA($\infty$).  

	\begin{algorithm}[t!]
		\caption{Nonlinear GMRES: NGMRES($m$)} 
            \label{alg:NGMRES}
		\begin{algorithmic}[1]
			\REQUIRE $x_0$  and $m\geq0$
			\FOR {$k=0,1,\cdots$ until convergence}
				\STATE compute 
				\begin{equation}\label{eq:xkp1}
					x_{k+1} = q(x_k) + \sum_{i=0}^{m_k}\beta_i^{(k)} \left(q(x_k)-x_{k-i} \right),
				\end{equation}
				where $m_k=\min\{k,m\}$ and $\beta_i^{(k)}$ is obtained by solving the  least-squares problem
				\begin{equation}\label{eq:min}
					\min_{\big(\beta_0^{(k)},\beta_1^{(k)},\cdots, \beta_{m_k}^{(k)}\big)} \left\|g(q(x_k))+\sum_{i=0}^{m_k} \beta_i^{(k)} \left(g(q(x_k))-g(x_{k-i}) \right) \right\|_2^2.
				\end{equation}
			\ENDFOR
		\end{algorithmic}
	\end{algorithm}

We note with regard to Algorithm \ref{alg:NGMRES} that since $g(x)=r(x)$, we can rewrite \eqref{eq:min} as
	\begin{equation*}
		\min_{\big(\beta_0^{(k)},\beta_1^{(k)},\cdots, \beta_{m_k}^{(k)}\big)} \left\|r(q(x_k))+\sum_{i=0}^{m_k} \beta_i^{(k)} \left(r(q(x_k))- r(x_{k-i}) \right) \right\|_2^2.
	\end{equation*}

    
	\begin{algorithm}[h!]
		\caption{Anderson Acceleration:  AA($m$)} \label{alg:AA}
		\begin{algorithmic}[1] 
			\REQUIRE $x_0$  and $m\geq0$
			\FOR {$k=0,1,\cdots$ until convergence }
				\STATE compute 
				\begin{equation}\label{eq:xkp1-AA}
					x_{k+1} = q(x_k) + \sum_{i=0}^{m_k}\gamma_i^{(k)} \left(q(x_k)-q(x_{k-i}) \right),
				\end{equation}
				where $m_k=\min\{k,m\}$ and $\gamma_i^{(k)}$ is obtained by solving the  least-squares problem
				\begin{equation}\label{eq:min-AA}
					\min_{\big(\gamma_0^{(k)},\gamma_1^{(k)},\cdots, \gamma_{m_k}^{(k)}\big)} \left\|r(x_k)+\sum_{i=0}^{m_k} \gamma_i^{(k)} \left(r(x_k)-r(x_{k-i}) \right) \right\|_2^2.
				\end{equation}
			\ENDFOR
		\end{algorithmic}
	\end{algorithm}

In this work, we aim to close a gap in the understanding of NGMRES by providing a detailed study of the performance of this method when applied to linear systems. 

        Consider  the linear system 
	\begin{equation}\label{eq:linear-system}
		Ax=b,
	\end{equation}
	where $A\in \mathbb{R}^{n\times n}$, $b \in \mathbb{R}^n$, and denote its solution by $x^* \in \mathbb{R}^n$.
    We will study the fixed-point iteration  
	\begin{equation}\label{eq:linear-q}
		q(x)=Mx+b,
	\end{equation}
	where 
    \begin{equation}\label{eq:M}
    M=I-A.
    \end{equation}
    Note that $r(x)=x-q(x)=x-(Mx+b)=Ax-b$. For the solution $x^*$ we have $r(x^*)=0,$ 
        and given that $r(x)=g(x)=0=Ax-b$, we also have $g(x^*)=0$. 
        
    \begin{remark}
        As evident by step 2 of Algorithm \ref{alg:NGMRES}, our definition of the $k$-step NGMRES  starts from $k=0$ and generates $x_{k+1}$. In contrast,  $k$-step classical GMRES starts with $x_0$ and computes $x_k$. So, there is an index shift here, which we will account for in our analysis.
    \end{remark}
     \begin{remark} In this work we denote $r_k=r(x_k)=Ax_k-b$ as the residual for all methods considered, even though standard practice for classical GMRES in the literature is to refer to the negated quantity as the residual.   
        \end{remark}

    It has been shown that  AA with an infinite window size (namely, AA($\infty$))  applied to a linear system is equivalent to classical GMRES \cite{saad2003iterative} under certain conditions \cite{walker2011anderson,ni2009anderson}. 
	As for NGMRES, in \cite{sterck2012nonlinear} the author discussed similarities between the two methods, but the question whether NGMRES applied to linear systems is identical to GMRES was not fully addressed. Indeed, in the literature there seems to be no detailed characterization of the relationship between NGMRES($m$) and classical GMRES even for solving linear systems, neither theoretically nor experimentally. We note, though, that our study is not intended to promote the use of NGMRES($m$) as a comparable linear solver to GMRES, but rather to understand its properties.
	
    The main contributions of this work are as follows. Assuming that the 2-norm of the residual of GMRES is strictly decreasing, then:
	\begin{enumerate}[(i)]
		\item  we prove that the residuals of NGMRES($\infty$) are the same as these of GMRES, and if $A$ is invertible, the iterates generated by the two methods are identical;
		\item we derive some orthogonality properties for NGMRES;
        \item for the finite-window version, when $A$ defined in \eqref{eq:linear-system} is either symmetric or shifted skew-symmetric, we prove that NGMRES($1$) is the same as  GMRES and we show that NGMRES($1$) has a  simpler way of updating the solution approximations.
		\item we derive upper bounds on the convergence rate of NGMRES for certain cases.
	\end{enumerate}



	The remainder of this paper is organized as follows. In section \ref{sec:comparison}, we prove (i). In section \ref{sec:NGMRESm}, we establish (ii) and (iii). In section \ref{sec:conv}, we provide upper bounds  as per (iv).  In section \ref{sec:num}, we briefly discuss a few implementation details and present some numerical experiments to validate our theoretical findings.  Finally, we  draw some conclusions in section \ref{sec:con}.
	
	\section{Comparison of AA, GMRES, and NGMRES for solving linear systems}\label{sec:comparison}
 
      For NGMRES, let $$\boldsymbol{\beta}^{(k)}=\big(\beta_0^{(k)},\beta_1^{(k)},\cdots, \beta_{m_k}^{(k)}\big).$$  
    Given the linear system \eqref{eq:linear-system} and fixed-point iteration \eqref{eq:linear-q}, we claim that 
	\begin{equation}\label{GMRES-LSP-rk}
		\min_{\boldsymbol{\beta}^{(k)}}  \left\|g(q(x_k))+\sum_{i=0}^{m_k} \beta_i^{(k)} \left(g(q(x_k))-g(x_{k-i}) \right) \right\|_2^2= \min_{\boldsymbol{\beta}^{(k)}} \left\| r_{k+1}  \right\|_2^2.
	\end{equation}
	This means that NGMRES in each step minimizes the residual.
	By standard calculations, we have
	\begin{align*}
		&  \min_{\boldsymbol{\beta}^{(k)}}  \left\|g(q(x_k))+\sum_{i=0}^{m_k} \beta_i^{(k)} \left(g(q(x_k))-g(x_{k-i}) \right) \right\|_2^2 \\
		= &  \min_{\boldsymbol{\beta}^{(k)}} \left\|A q(x_k)-b+\sum_{i=0}^{m_k} \beta_i^{(k)} \left(A q(x_k)-b-A x_{k-i}+b) \right) \right\|_2^2 \\
		= &  \min_{\boldsymbol{\beta}^{(k)}} \left\|A q(x_k)-b+\sum_{i=0}^{m_k} \beta_i^{(k)} \left(A q(x_k)-A x_{k-i}) \right) \right\|_2^2 \\
		= &  \min_{\boldsymbol{\beta}^{(k)}}  \left\| A \left( q(x_k)+ \sum_{i=0}^{m_k} \beta_i^{(k)} \left( q(x_k)- x_{k-i} \right) \right) -b \right\|_2^2 \\
		= &  \min_{\boldsymbol{\beta}^{(k)}}  \left\| A x_{k+1} -b \right\|_2^2 \\
		= &  \min_{\boldsymbol{\beta}^{(k)}} \left\| r_{k+1}  \right\|_2^2.
	\end{align*}

We start by presenting recurrence relations that the residuals satisfy among themselves.
	\begin{theorem}
		The residuals of NGMRES applied to fixed-point iteration \eqref{eq:linear-q} satisfy the recurrence relation
		\begin{equation} r_{k+1} = \left(1+\sum_{i=0}^{m_k} \beta_i^{(k)} \right) M r_k - \sum_{i=0}^{m_k}  \beta_i^{(k)} r_{k-i}. 
			\label{eq:rkp1}
		\end{equation}
	\end{theorem}
	
	\begin{proof}
		Using $r_k=A x_k-b,$ by \eqref{eq:xkp1} we have
		\begin{align*}
			r_{k+1}  = &A q(x_k) + \sum_{i=0}^{m_k} \beta_i^{(k)} \left(A q(x_k)-A x_{k-i} \right) - b\\
			= & A \left[\left(1+\sum_{i=0}^{m_k} \beta_i^{(k)} \right) M x_k
			+\left(1+\sum_{i=0}^{m_k} \beta_i^{(k)} \right) b  - \sum_{i=0}^{m_k}\beta_i^{(k)}x_{k-i} \right] - b\\
			= & M \left(1+\sum_{i=0}^{m_k} \beta_i^{(k)} \right)   A x_k
			+\left(1+\sum_{i=0}^{m_k} \beta_i^{(k)} \right) Ab - \beta_0^{(k)} A x_k - \beta_1^{(k)} A x_{k-1}\\
			&-\cdots-
			\beta_{m_k}^{(k)} Ax_{k-m_k}  - b\\
			= & M \left(1+\sum_{i=0}^{m_k} \beta_i^{(k)} \right)   (r_k+b)
			+\left(1+\sum_{i=0}^{m_k} \beta_i^{(k)} \right) Ab - \sum_{i=0}^{m_k}\beta_i^{(k)} (r_{k-i}+b)  - b\\
			= &  \left(1+\sum_{i=0}^{m_k} \beta_i^{(k)} \right)  M r_k
			+\left(1+\sum_{i=0}^{m_k} \beta_i^{(k)} \right) b - \beta_0^{(k)} (r_k+b) - \beta_1^{(k)} (r_{k-1}+b)\\
			&-\cdots-
			\beta_{m_k}^{(k)} (r_{k-m_k}+b)  - b\\
			= &  \left(1+\sum_{i=0}^{m_k} \beta_i^{(k)} \right)  M r_k
			-\sum_{i=0}^{m_k} \beta_i^{(k)}   r_{k-i}.
		\end{align*}
	\end{proof}
	
	\begin{definition}
    We denote the Krylov subspace solver associated with the iterations of the methods we consider by
	\begin{equation}
		\mathcal{K}_s={\rm span}\{r_0, Ar_0, A^2r_0,\cdots, A^{s-1}r_0\}.
	\end{equation}
    \end{definition}
	
	Let us denote $x_j^A, x_j^{NG}$ and $x_j^G$ as the iterates generated by   AA($\infty$), NGMRES($\infty$), and GMRES, respectively, and analogously their residuals: $r_j^A, r_j^{NG}$ and $r_j^G$.

    
	\begin{lemma}\cite[Lemma 2.4]{walker2011anderson} \label{lem:GMRES-mono}
		Suppose that GMRES is applied to linear system \eqref{eq:linear-system}. If $\|r_{k-1}^G\|_2 >\|r_k^G\|_2$ for some $k>0$, then $r_k^G\notin \mathcal{K}_k$.\footnote{In the statement of \cite[Lemma 2.4]{walker2011anderson} $A$ was required to be nonsingular, but this was not used in the proof.  So, we do not include this condition in the statement of our Lemma \ref{lem:GMRES-mono}. }  
	\end{lemma}
	
	
	In \cite{walker2011anderson}, the authors have shown that the iterates obtained from AA($\infty$) have a direct relationship with the iterates of GMRES. We state the result below.
	\begin{lemma}\cite[Theorem 2.2]{walker2011anderson}\label{lem:AA-GMRES}
		Assume that $A$ is invertible, and GMRES and AA($\infty$) use the same initial guess $x_0\neq x^*$. For some $k_0> 0$,
		$r_{k_0-1}^{G}\neq 0$ and $\|r_{k-1}^G\|_2 >\|r_k^G\|_2$ for each $k$ such that $0 < k < k_0$. Then, $x_{j+1}^A=q(x_j^G)=x_j^G-r_j^G$, where $0\leq j \leq k_0$, and $r_j^G=Ax_j^G-b$. 
	\end{lemma}
	
	
	Next, we derive a relationship among AA($\infty$), NGMRES($\infty$) and GMRES.
	
	\begin{theorem}\label{thm:relation-infinity}
		Assume that AA($\infty$), NGMRES($\infty$) and classical GMRES use the same initial guess $x_0\neq x^*$ Furthermore, for some $k_0> 0$,  $r_{k_0-1}^{G}\neq 0$ and $\|r_{k-1}^{G}\|_2 >\|r_k^{G}\|_2$ for each $k$ such that $0 < k< k_0$. Then, 
		\begin{enumerate}[(i)]
			\item $r_j^{NG}=r_j^G$, where $0\leq j \leq k_0$;
			\item if $A$ is invertible,  $x_j^{NG}=x_j^G$ and   $x_{j+1}^A=q(x_j^{NG})=x_j^{NG}-r_j^{NG}$, where $0\leq j \leq k_0$.
		\end{enumerate} 
	\end{theorem}
	\begin{proof}
		First, we define 
		\begin{equation}\label{eq:basis1}
			z_i=x_{i-1}^{NG}-x_0, \quad  i=1,2,\cdots, k_0-1, 
	     \end{equation}
			and
		\begin{equation}\label{eq:basis2}
		z_{k_0}=x_{k_0-1}^{NG}-x_0-r_{k_0-1}^{NG}.
		\end{equation}
		To prove that $r_j^{NG}=r_j^G$ (if $A$ is invertible, $x_j^{NG}=x_j^G$), it is sufficient to prove the following two Claims:
		\begin{enumerate}
			\item For $1\leq j\leq k_0$, if $\{z_1,\cdots, z_j\}$ is a basis for $\mathcal{K}_j$, then $r_j^{NG}=r_j^G$. Moreover, if $A$ is invertible, $x_j^{NG}=x_j^G$.
			\item For $1\leq j\leq k_0$,  $\{z_1,\cdots, z_j\}$ is a basis for $\mathcal{K}_j$.
		\end{enumerate}
		{\em Proof of Claim 1.} We do so by induction.   NGMRES($\infty$) and GMRES use the same initial guess $x_0$, and we show  later, in subsection \ref{subsec:NGMRES1-GMRES},  that $r_1^{NG}=r_1^G$ and $x_1^{NG}=x_1^G$. 
        
        Suppose that for $1\leq j<k_0$, $r_j^{NG}=r_j^G$ (if $A$ is invertible, then $x_j^{NG}=x_j^G$). 
        We now show that  $r_{k_0}^{NG}=r_{k_0}^G$ (if $A$ is invertible, $x_{k_0}^{NG}=x_{k_0}^G$). Let $r_0=Ax_0-b$. From \eqref{eq:basis1} and \eqref{eq:basis2}, we have
		\begin{equation*}
		Az_i=r_{i-1}^{NG}-r_0, \quad  i=1,2,\cdots, k_0-1, 
		\end{equation*}
		and 
		\begin{equation*}
			Az_{k_0}=Mr_{k_0-1}^{NG}-r_0. 
		\end{equation*}
		Then, using \eqref{eq:rkp1}  and the above two equations, for the $(k_0-1)$-step  NGMRES($\infty$), where $m_k=\min(\infty,k_0-1)=k_0-1$, we have
		\begin{align*}
	r_{k_0}^{NG}&=A x_{k_0}^{NG} -b\\
			    &= \left(1+\sum_{i=0}^{k_0-1} \beta_i^{(k_0-1)} \right)  M r_{k_0-1}^{NG}
			    -\sum_{i=0}^{k_0-1} \beta_i^{(k_0-1)}  r_{k_0-1-i}^{NG}\\
			    &=\left(1+\sum_{i=0}^{k_0-1} \beta_i^{(k_0-1)} \right)(Az_{k_0}+r_0)
			    -\sum_{i=0}^{k_0-2} \beta_i^{(k_0-1)} (Az_{k_0-1-i}+r_0)-\beta_{k_0-1}r_0\\
			    &=r_0+A \left(1+\sum_{i=0}^{k_0-1} \beta_i^{(k_0-1)} \right)z_{k_0}
			    - A\left(\sum_{i=0}^{k_0-2} \beta_i^{(k_0-1)} z_{k_0-1-i}\right)	\\
			    &=r_0-A\left(\sum_{i=1}^{k_0} \alpha_i z_i\right), 
		\end{align*}
		where $\alpha_i=\beta_{k_0-1-i}^{(k_0-1)}$ for $i=1, 2, \cdots, k_0-1$,  and $\alpha_{k_0}=-(1+\sum_{i=0}^{k_0-1} \beta_i^{(k_0-1)}) $. We emphasize that there is no constraint on $\alpha_i$ for $i=1, 2, \cdots, k_0$. Let $\boldsymbol{\alpha}_i=(\alpha_1,\cdots,\alpha_{k_0})^T$.
		Recall that $z_{k_0}=x_{k_0-1}^{NG}-x_0-r_{k_0-1}^{NG} \in \mathcal{K}_{k_0}$ and $z_{k_0}=x_{k_0-1}^{NG}-x_0-r_{k_0-1} \notin \mathcal{K}_{k_0-1}$. We know that the $(k_0-1)$-step NGMRES($\infty$) solves the problem
		\begin{equation}\label{eq:LSPrksame}
			\min_{\boldsymbol{\beta}^{(k_0-1)}}  \left\| A x_{k_0}^{NG} -b \right\|_2^2
			=\min_{\boldsymbol{\alpha}_i}\left\| r_0-A(\sum_{i=1}^{k_0} \alpha_i z_i)\right\|_2^2=
			\min_{u\in \mathcal{K}_{k_0}} \left\| r_0-Au\|_2^2, \right.
		\end{equation}
		which is exactly what $k_0$-step GMRES does. Thus, $r_{k_0}^{NG}=r_{k_0}^G$. If $A$ is invertible, $x_{k_0}^{NG}=x_{k_0}^G$.
		
		{\em Proof of Claim 2.} From upcoming subsection \ref{subsec:NGMRES1-GMRES}, we know that $x_1^{G}=x_1^{NG}$. We have $z_1=x_1^{NG}-x_0=\beta_0^{(0)}r_0\neq 0$. Thus, $z_1$ is a basis for $\mathcal{K}_1$. Suppose that $\{z_1,\cdots, z_j\}$ is a basis for $\mathcal{K}_j$ for all $j$ such that $1\leq j< k_0$. Next, we prove that $\{z_1,\cdots, z_{k_0}\}$ is a basis for $\mathcal{K}_{k_0}$. According to the definition,
		\begin{align*}
			z_{k_0}&=x_{k_0-1}^{NG}-x_0-r_{k_0-1}^{NG}.
		\end{align*}
		Since we know $\{z_1,\cdots, z_{k_0-1}\}$ is a basis for $\mathcal{K}_{k_0-1}$, from Claim 1 we  have $r_{k_0-1}^{NG}=r_{k_0-1}^G$. Since $x_{k_0-1}^{NG}-x_0 \in \mathcal{K}_{k_0-1}$ and  $r_{k_0-1}^G \in \mathcal{K}_{k_0}$, we have $z_{k_0}\in \mathcal{K}_{k_0}$. Moreover, since $\|r_{k_0-2}^G\|_2 >\|r_{k_0-1}^G\|_2$ by assumption, Lemma \ref{lem:GMRES-mono} states $r_{k_0-1}^G\notin \mathcal{K}_{k_0-1}$, i.e., $z_{k_0}\notin \mathcal{K}_{k_0-1}$. As a result, $\{z_1,\cdots, z_{k_0}\}$ is a basis for $\mathcal{K}_{k_0}$. From  Lemma \ref{lem:AA-GMRES}, we have the desired result for NGMRES and AA.  
	\end{proof}
	
	\begin{remark}		 
		In Theorem \ref{thm:relation-infinity}, if we assume that when $A$ is not invertible NGMRES and GMRES return the minimum-norm solution (we know this is unique) of the least-squares problem \eqref{eq:LSPrksame}, then $x_j^{NG}=x_j^G$. In practice,  the least-squares problems might be solved differently in GMRES and NGMRES.  
	\end{remark}
 
	
 
	\begin{corollary}
		Suppose that the assumptions of Theorem \ref{thm:relation-infinity} hold and that $r_{k_0}^G=r_{k_0-1}^G\neq 0$. Then, $r_{k_0}^{NG}=r_{k_0-1}^{NG}$.
	\end{corollary}


    \begin{remark}[Stagnation]
In Theorem \ref{thm:relation-infinity}, in some situations $k_0$ might not exist. Indeed, we know that GMRES may stagnate. 
When stagnation occurs, the behavior of GMRES is complicated \cite{nachtigal1992fast}  and it is harder to establish convergence properties.  The same goes for NGMRES; there seems to be no general way to describe the behavior of the residuals $r_j^G$ and $r_j^{NG}$ for $j>k_0$.  See Example \ref{ex:circ-staga} for an instance of this. 
\end{remark}

Theorem \ref{thm:relation-infinity}  extends to the preconditioned case, as follows.
	\begin{corollary}
		Let $A = P-N$, where $P$ is nonsingular, and consider
		GMRES applied to the left-preconditioned system  $P^{-1}Ax = P^{-1}b$. Define $r_{j,pr}=P^{-1}Ax_j-P^{-1}b$. Let	 NGMRES($\infty$) be applied to accelerate the fixed-point iteration $q_{pr}(x)=(I-P^{-1}A)x+P^{-1}b$.
		Assume that AA($\infty$),  NGMRES($\infty$) and classical GMRES use the same initial guess $x_0\neq x^*$. Moreover, for some $k_0> 0$,  $r_{k_0-1,p}^{NG}\neq 0$ and $\|r_{k-1,pr}^{NG}\|_2 >\|r_{k,pr}^{NG}\|_2$ for each $k$ such that $0 < k < k_0$. Then,   $r_j^{NG}=r_j^G$, where $0\leq j \leq k_0$. Furthermore, if $A$ is invertible, $x_j^{NG}=x_j^G$, and  $x_{j+1}^A=q(x_j^{NG})=x_j^{NG}-r_j^{NG}$ where $0\leq j \leq k_0$.
	\end{corollary}


	
	
	
	\section{Equivalence of NGMRES($m$) and GMRES}\label{sec:NGMRESm}
	In this section, we explore some properties of windowed NGMRES. We start with a polynomial relation for the residuals.
	
	\begin{theorem}\label{thm:NGMRESm-poly}
		For NGMRES($m$) applied to \eqref{eq:linear-q},  the residuals satisfy 
		$$ r_{k+1} = p_{k+1} (M) r_0, \quad k\ge 0,$$
		where $p_{k+1}(\lambda)$ is a polynomial of degree at most $k + 1$ with $p_{k+1}(0)=0$ and $p_{k+1}(1)=1$, satisfying the following recurrence relation: 
		\begin{align*}
			p_{k+1}(\lambda) & = \left(1+\sum_{i=0}^{m_k} \beta_i^{(k)} \right) \lambda p_k(\lambda) - \sum_{i=0}^{m_k} \beta_i^{(k)} p_{k-i}(\lambda)  \\
			&=\left( \left(1+\sum_{i=0}^{m_k} \beta_i^{(k)} \right) \lambda +\beta_0^{(k)}\right) p_k(\lambda) - \sum_{i=1}^{m_k} \beta_i^{(k)} p_{k-i}(\lambda).
		\end{align*}
	\end{theorem}
	\begin{proof}
		The result can be derived easily; thus, we omit it.
	\end{proof}
	We point out that Theorem \ref{thm:NGMRESm-poly} holds for $m=\infty$.
	\subsection{Orthogonality properties}
	Next, we study orthogonality properties of NGMRES. We first mention some orthogonality properties of GMRES, which might be known or can be easily inferred but are useful to present here for the purpose of comparison with NGMRES. 

    Recall that GMRES is founded upon requiring
	\begin{equation}\label{eq:GMRES-orth-space}
		r_k^G \perp A\mathcal{K}_k.
	\end{equation}
	 The following orthogonality properties of GMERS are useful for our later proof of Theorem \ref{thm:gmres1=gmres}.  
	\begin{lemma}\label{lem:GMRES-ortho}
		The residuals  of GMRES satisfy 
		\begin{equation}\label{eq:GMRES-pro}
			(r_{k+1}^G)^TAr_k^G =0,\quad   (r_{k+1}^G)^T(r_j^G-r_i^G) =0,  \quad 0 \leq  i\leq j\leq k+1.
		\end{equation}
	\end{lemma}
	\begin{proof}
		We know that $r_k^G\in \mathcal{K}_{k+1}$. From \eqref{eq:GMRES-orth-space}, we have $(r_{k+1}^G)^TAr_k^G =0$. Notice that $r_j^G-r_i^G=A(x_j^G-x_i^G)=A(x_j^G-x_0^G+x_0^G-x_i^G)$, as well as $x_j^G-x_0^G \in \mathcal{K}_j$ and $x_i^G-x_0^G \in \mathcal{K}_i$.  It follows that $x_j^G-x_i^G\in \mathcal{K}_j$ and $r_j^G-r_i^G \in A\mathcal{K}_j$. Since $0\leq i\leq j\leq k+1$ and $\mathcal{K}_j\subset \mathcal{K}_{k+1}$, we obtain the desired result.
	\end{proof}
        \begin{remark}
	If we take $j=k+1$ and $i=k$ in \eqref{eq:GMRES-pro}, this property indicates that the norm of residuals of GMRES are nonincreasing, as expected from a residual-minimization process.
    \end{remark}
	For notational convenience,  for AA let $\boldsymbol{\gamma}^{(k)}=\big(\gamma_0^{(k)},\gamma_1^{(k)},\cdots, \gamma_{m_k}^{(k)}\big)$. One can show that if $M$ is invertible, the least-squares problem \eqref{eq:min-AA} in AA can be rewritten as
	\begin{equation}\label{eq:AA-min}
		\min_{\boldsymbol{\gamma}^{(k)}} \left\| M^{-1} r^A_{k+1}  \right\|_2^2,
	\end{equation}
	where $r^A_{k+1}=Ax^A_{k+1}-b$. For AA($\infty$), using Lemma \ref{lem:AA-GMRES}, we have 
	\begin{equation*}
		r_{k+1}^A=Ax_{k+1}^A-b=A(x_k^G-r_k^G)-b=r_k^G-Ar_k^G=Mr_k^G. 
	\end{equation*}
	
	Thus, we have the following result.
	\begin{theorem}
		Assume that the conditions in Lemma \ref{lem:AA-GMRES} hold and $M$ is invertible. For AA($\infty$),  
		\begin{equation}\label{eq:AA-min-G}
			\min_{\boldsymbol{\gamma}^{(k)}} \left\| M^{-1} r^A_{k+1}  \right\|_2^2=\min\|r_k^G\|_2,
		\end{equation}
		and 
		\begin{equation}\label{eq:AA-orth-space}
			M^{-1}r_{k+1}^A \perp A\mathcal{K}_k.
		\end{equation}  
		Moreover, if $M$ is symmetric, then 
		\begin{equation}\label{eq:AA-orth-space-sym}
			r_{k+1}^A \perp  M^{-1}A\mathcal{K}_k.
		\end{equation} 
	\end{theorem}
    \begin{proof}
    The proof is straightforward; thus we omit it.
    \end{proof}
	Let $R_k$ be the coefficient matrix of the least-squares problem \eqref{eq:min-AA} in AA given by 
	\begin{equation*}
		R_k=\begin{bmatrix} r_k^A-r_{k-1}^A & r_k^A-r_{k-2}^A, & \cdots, & r_k^A-r_{k-m_k}^A \end{bmatrix}.
	\end{equation*}
	In \cite[Proposition 5]{de2024anderson} it is shown that the residuals of AA($m$) satisfy 
	\begin{equation*}
		R_k^TM^{-1}r_{k+1}^A=0,   
	\end{equation*}
	 which leads to
	\begin{equation}\label{eq:M-orth-AA}
		(M^{-1}r_{k+1}^A)^T(r_{k-j}^A-r_{k-i}^A)=0, \quad i,j=0, 1,2,\cdots, m_k.
	\end{equation}
 This indicates that  the $(k+1)$st residual of AA($m$) is $M^{-1}$-orthogonal to the difference of two residuals that belong to $\{ r_k^A, \cdots, r_{k-m_k}^A\}$. 	When $m_k=k$ (i.e., AA($\infty$)), the above result is equivalent to \eqref{eq:AA-orth-space}.  

 Moreover, for the residuals of AA (see \cite{de2024anderson}) we have
\begin{equation}\label{eq:r_k-AA-recur}
    r_{k+1}^A = r_{k}^A + \sum_{i=0}^{m_k} \gamma_i^{(k)}M(r_k^A-  r_{k-i}^A).
\end{equation}
Using \eqref{eq:M-orth-AA} and \eqref{eq:r_k-AA-recur}, we have
\begin{equation}
    (M^{-1}r_{k+1}^A)^T (M^{-1}r_{k+1}^A)=(M^{-1}r_{k+1}^A)^T(M^{-1}r_k^A),
\end{equation}
which means that either $r_{k+1}^A=r_k^A$ or $\|M^{-1}r_k^A\|>\|M^{-1}r_{k+1}^A\|$. For a general $M$, there is no guarantee that the norms of residuals of AA are nonincreasing, which is a significant difference compared to NGMRES and GMRES.
 
	
	We now derive an analogous orthogonality property for NGMRES. Recall that $m_k={\rm min}(m,k)$.  We drop the superscript, NG, for NGMRES for simplicity.  The following results hold for both NGMRES($\infty$) and NGMRES($m$).  Based on our discussion so far, the minimization problem \eqref{eq:min} can be rewritten as
	\begin{equation}\label{eq:LSP-linear}
		\min_{\boldsymbol{\beta}^{(k)}} \left\| r_{k+1}  \right\|_2^2= \min_{\beta_i^{(k)}}  \left\| M r_k+ \sum_{i=0}^{m_k} \beta_i^{(k)} M r_k -\sum_{i=0}^{m_k} \beta_i^{(k)}  r_{k-i}\right\|_2^2.
	\end{equation}
	Let us define
	\begin{equation}\label{eq:definition-Wk}
		W_k=  \begin{bmatrix}
			r_k-Mr_k&  r_{k-1}-Mr_k &\cdots & r_{k-m_k}-Mr_k.
		\end{bmatrix}
	\end{equation}
	The  normal equations for the least-squares problem \eqref{eq:LSP-linear} are given by
	\begin{equation}\label{eq:normal-equ}
		(W^T_k W_k)	\boldsymbol{\beta}^{(k)}= W^T_k Mr_k
	\end{equation}
	and if $W_k$ has full rank, the solution is 
	\begin{equation}\label{eq:beta-form}
		\boldsymbol{\beta}^{(k)}= (W^T_k W_k)^{-1} W^T_k Mr_k.
	\end{equation}
	
    
    From this it follows that NGMRES has a similar orthogonality property to that of AA, as follows.
	\begin{theorem}\label{thm:orthogonality}
		The residuals of NGMRES  satisfy  
		$$ r_{k+1}^T W_k=0,$$
		that is, 
		\begin{equation}\label{eq:orthogonality-dif}
			r_{k+1}^TAr_k=0, \quad r_{k+1}^T(r_{k-j}-r_{k-i})=0, \quad i,j=0, 1,2,\cdots, m_k.
		\end{equation}
		Moreover,
		\begin{equation}\label{eq:NGMRESm-decrease}
			r_{k+1}^T(r_{k+1}-r_k)=0,
		\end{equation}
		which means that either $r_{k+1}=r_k$ or $\|r_k\|>\|r_{k+1}\|$. 
	\end{theorem}
	Before we give the proof, we point out that for $m=\infty$, i.e., $m_k=k$, the results in Theorem \ref{thm:orthogonality} are the same as these in Lemma \ref{lem:GMRES-ortho} for GMRES, which is  consistent with the fact that NGMRES($\infty$) and GMRES coincide.  The orthogonality property \eqref{eq:NGMRESm-decrease} indicates that for both windowed versions of NGMRES, i.e., NGMRES($m$) and the infinity-windowed version, the norms of the residuals of NGMRES are nonincreasing. For windowed NGMRES, i.e., $m_k=m$,   Theorem \ref{thm:orthogonality} indicates that the ($k$+1)st residual of NGMRES($m$) is orthogonal to the difference of two previous residuals that belong to $\{r_{k+1}, r_k, \cdots, r_{k-m_k}\}$.
	
	\begin{proof}
		  The minimization problem \eqref{eq:LSP-linear} of NGMRES can be rewritten as 
		\begin{align*}
			\min_{\boldsymbol{\beta}^{(k)}} \left\| r_{k+1}  \right\|_2^2= \min_{\beta_i^{(k)}}  \left\| M r_k-W_k\bm{\beta}^{(k)}\right\|_2^2.
		\end{align*}
		Thus,
		$$ W_k^T (M r_k-W_k\boldsymbol{\beta}^{(k)})=W_k^T r_{k+1}=0,$$
		which gives the desired result. The above equality also indicates that
		\begin{equation}\label{eq:temp-result-orth}
			r_{k+1}^T(r_{k-i}-Mr_k)=0, \quad i=0, 1,\cdots, m_k.
		\end{equation}
		However, using $M=I-A$, it follows that  $r_{k-i} -M r_k=r_{k-i} - r_k+Ar_k$. For $i=0$, \eqref{eq:temp-result-orth} gives 
		\begin{equation*}
			r_{k+1}^TAr_k=0.
		\end{equation*}
		For $ i=1,\cdots, m_k$, \eqref{eq:temp-result-orth} gives $r_{k+1}^T(r_k-r_{k-i})=0$. For $i,j=0,1,\cdots, m_k$, it follows that 
		\begin{equation*}
			r_{k+1}^T(r_{k-j}-r_{k-i})=r_{k+1}^T(r_k-r_{k-i})-r_{k+1}^T(r_k-r_{k-j})=0,
		\end{equation*}
		which gives the desired result.
		
		Next, we prove \eqref{eq:NGMRESm-decrease}. From \eqref{eq:rkp1}, we have 
		\begin{align*}
			r_{k+1} &= \left(1+\sum_{i=0}^{m_k} \beta_i^{(k)} \right) (I-A) r_k - \sum_{i=0}^{m_k}  \beta_i^{(k)} r_{k-i}\\
			&=r_k -\left(1+\sum_{i=0}^{m_k} \beta_i^{(k)} \right) A r_k - \sum_{i=0}^{m_k}  \beta_i^{(k)} (r_k-r_{k-i}). 
		\end{align*}
		Using the previous orthogonality property $r_{k+1}^TAr_k=0$ and $r_{k+1}^T(r_k-r_{k-i})=0$, we have
		\begin{equation}
			r_{k+1}^Tr_{k+1}=r_{k+1}^Tr_k,
		\end{equation}
		that is, $r_{k+1}^T(r_{k+1}-r_k)=0$, which is the desired result. 
	\end{proof}
	
	\subsection{NGMRES($m$) vs. GMRES}\label{subsec:NGMRES1-GMRES}
	In this subsection, we explore the relationship between NGMRES($m$) and GMRES. We start with NGMRES(0).
	
	For $m=0$, let us denote the (single) coefficient, given $k$, as $\beta^{(k)}$. Then, one can show that $x_{k+1}=x_k-(1+\beta^{(k)})r_k$.  By standard calculations, we have
	\begin{equation*}
		r(q(x_k))=Mr_k \quad \text{and} \quad r(q(x_k))-r(x_k)=-Ar_k.
	\end{equation*}
	Thus,
	\begin{equation*}
		\beta^{(k)} =-\frac{r(q(x_k))^T (r(q(x_k))-r(x_k) )}{\|r(q(x_k))-r(x_k) \|^2}=\frac{r_k^TAr_k}{r_k^TA^TAr_k}-1.
	\end{equation*}
	Let $\alpha_k=\frac{r_k^TAr_k}{r_k^TA^TAr_k}$. It follows that $x_{k+1}=x_k-\alpha_k r_k.$
	From Theorem \ref{thm:orthogonality}, we have $r_{k+1}^TAr_k=0$ for all $k$.  Note the first iteration of  GMRES is $x_1 = x_0 -\frac{r_0^T A^T r_0}{r_0^T A^T A r_0} r_0$, which is the same as $x_1$ of NGMRES(0). However, from the next iteration, NGMRES(0) generates different iterates. 
	
	From the above discussion, we can rewrite NGMRES(0) as Algorithm \ref{alg:NGMRES0}, which is exactly the  Minimal Residual (MR) Iteration or GMRES(1), see \cite[Chapter 5.3.2]{saad2003iterative}. 
	\begin{algorithm}[H]
		\caption{NGMRES(0) or Minimal Residual (MR) Iteration or GMRES(1)} \label{alg:NGMRES0}
		\begin{algorithmic}[1] 
			\STATE Given $x_0$, let $r_0=Ax_0-b$
			\STATE  For $k=0,1, \cdots$ until convergence Do:
			\begin{itemize}
				\item compute $\alpha_k=\frac{r_k^TAr_k}{r_k^TA^TAr_k}$
				\item compute $x_{k+1}=x_k-\alpha_k r_k$ 
				\item compute $r_{k+1}=r_k-\alpha_kAr_k$
			\end{itemize}
			EndDo
		\end{algorithmic}
	\end{algorithm}
	
	
	In the following, we provide an interesting result for NGMRES($m$), which establishes the relationship among NGMRES($m$), AA($\infty$), and GMRES. We denote the iterates and residuals of NGMRES($m$) as $x_j^{NG(m)}$ and $r_j^{NG(m)}$, respectively. In practice, we are interested in $A$ being invertible, so we will focus on this situation.
	\begin{theorem}\label{thm:gmres1=gmres}
		Assume that  NGMRES($m$) with $m\geq 1$, AA($\infty$), and classical GMRES use the same initial guess $x_0\neq x^*$,  $A$ is invertible, and $A$ is either symmetric or shifted skew-symmetric of the form $A=\alpha I+S$ where $S$ is skew-symmetric. Moreover, for some $k_0> 0$,  $r_{k_0-1}^{G}\neq 0$ and $\|r_{k-1}^{G}\|_2 >\|r_k^{G}\|_2$ for each $k$ such that $0 < k < k_0$. Then,  for $0\leq j \leq k_0$,  we have 
		\begin{equation}
			x_j^{NG(m)}=x_j^G, \quad \forall m \in \mathbb{Z}^{+},
		\end{equation}	
		and
		\begin{equation}
			x_{j+1}^A=q(x_j^{NG(m)})=x_j^{NG(m)}-r_j^{NG(m)}, \quad \forall m \in \mathbb{Z}^{+}.
		\end{equation}
		
	\end{theorem}
	Theorem \ref{thm:gmres1=gmres} tells us that under certain conditions, we can generate the iterates obtained from GMRES by NGMRES(1), which is very simple because in each step one only needs to solve a $2\times 2$ linear system that gives the coefficients $ \beta_0^{(k)} $ and $ \beta_1^{(k)} $, to update the iterates $x_{k+1}$. 
	
	Before giving the proof, we first establish some notation for later use.
	Let us write $W_k$ defined in \eqref{eq:definition-Wk} as
	\begin{equation}\label{eq:simply-notation}
		W_k=[w_1, w_2,\cdots, w_{m_k+1}] \in \mathbb{R}^{n\times (m_k+1)}.
	\end{equation}
	Then,
	\begin{equation}\label{eq:Cmk}
		C_{m_k,k}=W_k^TW_k=
		\begin{bmatrix}
			w_1^Tw_1 &  w_1^Tw_2 & \cdots & w_1^Tw_{m_k+1}\\
			w_2^Tw_1 &  w_2^Tw_2 & \cdots & w_2^Tw_{m_k+1}\\
			\vdots   &  \vdots   & \vdots  & \vdots \\
			w_{m_k}^Tw_1 &  w_{m_k}^Tw_2 & \cdots& w_{m_k}^Tw_{m_k+1}\\
			w_{m_k+1}^Tw_1 &  w_{m_k+1}^Tw_2 & \cdots& w_{m_k+1}^Tw_{m_k+1}
		\end{bmatrix},
	\end{equation}
	and 
	\begin{equation}\label{eq:fmk}
		f_{m_k,k}=W_k^TMr_k=
		\begin{bmatrix}
			w_1^TMr_k \\
			w_2^TMr_k\\
			\vdots  \\
			w_{m_k}^TMr_k \\
			w_{m_k+1}^TMr_k
		\end{bmatrix}\in\mathbb{R}^{(m_k+1)}.
	\end{equation}
	For $\boldsymbol{\beta}^{(k)}$ in \eqref{eq:beta-form}, we reset as $\boldsymbol{\beta}^{(k)}_{m_k}$. Then, \eqref{eq:beta-form} can be rewritten as
	\begin{equation}
		C_{m_k,k}\boldsymbol{\beta}^{(k)}_{m_k} =f_{m_k,k}.
	\end{equation}
	We point out that  $w_j$ in $W_k$  defined in \eqref{eq:simply-notation} and $W_{\ell}$ are different. 
    
    We first state a lemma that will be used in our proof of Theorem \ref{thm:gmres1=gmres}.
	\begin{lemma}\label{lem:xA^Tyz}
		Assume $A$ is either symmetric or shifted skew-symmetric of the form $A=\alpha I +S$, where $S$ is skew-symmetric. For $x , y ,z\in \mathbb{R}^n$ satisfying $x^TA(y-z)=0$ and $x^T(y-z)=0$, we have
		\begin{equation*}
			x^TA^T(y-z) =0.
		\end{equation*}   
	\end{lemma}
	\begin{proof}
		 If $A$ is symmetric, then $A^T=A$ and it follows trivially that $x^TA^T(y-z) =x^TA(y-z)=0$. If $A$ is shifted skew-symmetric, then $x^TA^T(y-z)=x^T(\alpha I+S)^T(y-z)=0+x^T(-S)(y-z)=0$.  
	\end{proof}
	
	
	Given a matrix $B$, let $B(1 : s,1 : s)$ be the top-left $s\times s$ block of $B$. We define the $k$-step iterate in \eqref{eq:xkp1} of NGMRES($m$) as $x_{k+1}^{NG(m)}$. When the iterates for NGMRES and GMRES are the same, we drop the superscript for simplicity. The main idea now is that using Theorem \ref{thm:relation-infinity}, we have $r_{k+1}^{NG(\ell)}=x_{k+1}^G$ for any $ \ell\geq k$, and then we only need to prove that $x_{k+1}^{NG(\ell_1)}=x_{k+1}^{NG(k)}$ for $\ell_1=1, 2, \cdots, k-1$. 
    
    We are now ready to present the proof of Theorem \ref{thm:gmres1=gmres}.
	\begin{proof}
		We consider NMGRES($m$) with $m\geq 1$ and the $k$-step update $x_{k+1}$, where $k\leq k_0-1$.
		
		\begin{enumerate}[(i)] 
        \item When $k=0$ and $\forall m\in\mathbb{Z}^+$,  $\min\{k, m\}=0$. From Theorem \ref{thm:relation-infinity} we have $x_1^{NG(m)}=x_1^G$.
		
	\item When $k=1$ and $\forall m\in\mathbb{Z}^+$,  $\min\{k, m\}=1$.  From Theorem \ref{thm:relation-infinity} we have $x_2^{NG(m)}=x_2^G$.
		
	\item  When $k=2$ and $\ell\geq 2$,  $\min\{k, \ell\}=2$.  From Theorem \ref{thm:relation-infinity} we have  $x_3^{NG(\ell)}=x_3^G$ for $ \ell\geq 2$. Now, we only need to show that $x_3^{NG(1)}=x_3^{NG(2)}$. For NGMRES(2),  $m_k=\min\{k, m\}=2$ and
		\begin{equation*}
			W_2=[r_2-Mr_2, r_1-Mr_2, r_0-Mr_2]:=[w_1,w_2,w_3].
		\end{equation*}
        \end{enumerate}
		Next, we show that the corresponding $C_{m_k,k}$ and $f_{m_k,k}$  in \eqref{eq:Cmk} and \eqref{eq:fmk}, respectively, have a special structure:
		\begin{equation}\label{eq:c22}
			C_{2,2}=
			\begin{bmatrix}
				w_1^Tw_1 &  w_1^Tw_2  & w_1^Tw_3\\
				w_2^Tw_1 &  w_2^Tw_2  & w_2^Tw_3\\
				w_{3}^Tw_1 &  w_{3}^Tw_2 & w_{3}^Tw_3
			\end{bmatrix}
			=\begin{bmatrix}
				w_1^Tw_1 &  w_1^Tw_2  & w_1^Tw_2\\
				w_1^Tw_2 &  w_2^Tw_2  &  w_2^Tw_2\\
				w_1^Tw_2 &  w_2^Tw_2 & w_{3}^Tw_3
			\end{bmatrix},
		\end{equation}
		and 
		\begin{equation}\label{eq:f22}
			f_{2,2}=W_2^TMr_2=
			\begin{bmatrix}
				w_1^TMr_2 \\
				w_2^TMr_2\\
				w_2^TMr_2
			\end{bmatrix}.
		\end{equation}
		Equations \eqref{eq:c22} and \eqref{eq:f22} are equivalent to the following three conditions:
		\begin{align*}
			w_1^Tw_2 &= w_1^Tw_3 \quad \Longleftrightarrow (Ar_2)^T(r_1-r_0)=0;\\
			w_2^Tw_2  &= w_2^Tw_3 \quad \Longleftrightarrow (r_1-Mr_2)^T(r_1-r_0)=0;\\	
			w_2^TMr_2  &= w_3^TMr_2 \quad \Longleftrightarrow (Mr_2)^T(r_1-r_0)=0.
		\end{align*}
		Using \eqref{eq:orthogonality-dif} with $k=1, j=0, i=1$, we have $r_2^T(r_1-r_0)=0$  and $r_2Ar_1=0$. Recall $r_1^T(r_1-r_0)=0$.  We only need the following condition to make the above equalities hold true:
		\begin{equation}\label{eq:r2Ar1r0}
			r_2^TA^T(r_1-r_0)=0.
		\end{equation}
		Using \eqref{eq:GMRES-orth-space} for $k=2$, we have $r_2^TAr_0$. It follows that $r_2^TA(r_1-r_0)=0$. From Lemma \ref{lem:xA^Tyz}, Equation \eqref{eq:r2Ar1r0} holds.
		
		For NGMRES(1),  we have	$C_{1,2}= C_{2,2}(1 : 2,1 : 2)$ and $f_{1,2}=f_{2,2}(1 : 2)$. 
		Let $\boldsymbol{\beta}_{1,*}^{(2)}$ be a solution of $C_{1,2}\boldsymbol{\beta}_1^{(2)}=f_{1,2}$. It is obvious that $[\boldsymbol{\beta}_{1,*}^{(2)};0]$ is a solution of $C_{2,2}\boldsymbol{\beta}_2^{(2)}=f_{2,2}$. Thus, $r_3^{NG(1)}=r_3^{NG(2)}$. However, $r_3^{NG(1)}=Ax_3^{NG(1)}-b$, $r_3^{NG(2)}=Ax_3^{NG(2)}-b$ and $A$ is invertible. It follows that $x_3^{NG(1)}=x_3^{NG(2)}$.
		
		We proceed by an induction:
		
		\noindent$\bullet$ We assume that for $k\leq k_*<k_0-1$, it holds $x_{k+1}^{NG(m)}=x_{k+1}^G$ for $\forall m\in \mathbb{Z}^+$.
		
		\noindent$\bullet$ We now prove that for $k= k_*+1:=p$, it holds  $x_{k+1}^{NG(m)}=x_{k+1}^G$ $\forall m\in\mathbb{Z}^+$. Using Theorem \ref{thm:relation-infinity}, we have $x_{k+1}^{NG(\ell)}=x_{k+1}^G$ for any $\ell\geq k_*+1$. Next, we show that for $\ell_1=1, 2, \cdots, p-1$ it holds $x_{k+1}^{NG(\ell_1)}=x_{k+1}^{NG(p)}$. Then, we have $x_{k+1}^{NG(m)}=x_{k+1}^G$ for $\forall m\in\mathbb{Z}^+$.
		
		 For NGMRES($p$),  $m_k=\min\{k, p\}=p$ and
		\begin{equation*}
			W_{p,p}=[r_p-Mr_p, r_{p-1}-Mr_p,\cdots, r_0-Mr_p]:=[w_1,w_2,\cdots,w_{p+1}].
		\end{equation*}
		Next, we prove that the corresponding $C_{m_k,k}$ and $f_{m_k,k}$ in \eqref{eq:Cmk} and \eqref{eq:fmk} have the following forms, for $t=2, 3, \cdots, p+1$:
		\begin{equation}\label{eq:property-Cmk-fmk}
			C_{m_k,k}(t,1 : 2)=[w_2^Tw_1, w_2^Tw_2]\quad  \text{and}\quad f_{m_k,k}(t)=w_2^TMr_p,
		\end{equation}
		that is,   
		\begin{align*}
			C_{p,p}&=
			\begin{bmatrix}
				w_1^Tw_1 &  w_1^Tw_2 & \cdots & w_1^Tw_{p+1}\\
				w_2^Tw_1 &  w_2^Tw_2 & \cdots & w_2^Tw_{p+1}\\
				\vdots   &  \vdots   & \vdots  & \vdots \\
				w_{p}^Tw_1 &  w_{p}^Tw_2 & \cdots& w_{p}^Tw_{p+1}\\
				w_{p+1}^Tw_1 &  w_{p+1}^Tw_2 & \cdots& w_{p+1}^Tw_{p+1}
			\end{bmatrix}=\begin{bmatrix}
				w_1^Tw_1 &  w_1^Tw_2 & \cdots & w_1^Tw_2\\
				w_2^Tw_1 &  w_2^Tw_2 & \cdots & w_2^Tw_2\\
				\vdots   &  \vdots   & \vdots  & \vdots \\
				w_2^Tw_1 &  w_2^Tw_2 & \cdots& w_{p}^Tw_{p+1}\\
				w_2^Tw_1 &  w_2^Tw_2 & \cdots& w_{p+1}^Tw_{p+1}
			\end{bmatrix},
		\end{align*}
		and 
		\begin{equation*}
			f_{p,p}=W_{p}^TMr_{p}=
			\begin{bmatrix}
				w_1^TMr_p \\
				w_2^TMr_p\\
				\vdots  \\
				w_p^TMr_p \\
				w_{p+1}^TMr_p
			\end{bmatrix}=	\begin{bmatrix}
				w_1^TMr_p \\
				w_2^TMr_p\\
				\vdots  \\
				w_2^TMr_p \\
				w_2^TMr_p
			\end{bmatrix}.
		\end{equation*}
We are only interested in the first two columns of $C_{p,p}$.	Proving \eqref{eq:property-Cmk-fmk}  is equivalent to proving that for $t=3, 4, \cdots, p+1$ the following three requirements hold:
	\begin{subequations}
    \label{eq:req}
        \begin{align}
			w_1^Tw_2 &= w_1^Tw_t \quad \Longleftrightarrow (Ar_p)^T(r_{p-1}-r_{p+1-t})=0; \label{eq:cond1}\\
			w_2^Tw_2  &= w_2^Tw_t \quad \Longleftrightarrow (r_{p-1}-Mr_p)^T(r_{p-1}-r_{p+1-t})=0; \label{eq:cond2}\\	
			w_2^TMr_2  &= w_t^TMr_2 \quad \Longleftrightarrow (Mr_p)^T(r_{p-1}-r_{p+1-t})=0.\label{eq:cond3}
		\end{align}
        \end{subequations}
		Below, we prove the requirements of \eqref{eq:req}. 
		Since $k=k_*+1=p$, we have $k-1=k_*=p-1$. For the $k_*$-step NGMRES($k_*$), using \eqref{eq:orthogonality-dif}, i.e., $r_{k_*+1}^TAr_{k_*}=0$ and $r_{k_*+1}^T(r_{k_*-j}-r_{k_*-i})=0$ with $ j=0, i=t-2$, we have
		\begin{equation}\label{eq:orth-rpArp}
			r_p^TAr_{p-1}=0 \quad \text{and} \quad r_{p}^T(r_{p-1}-r_{p+1-t})=0.	
		\end{equation} 
		Recall the assumption that for $j\leq k_*$,  NGMRES($m$) is the same as GMRES for 
        $\forall m\in \mathbb{Z}^+$. Since $r_{p+1-t}\in \mathcal{K}_{p+1-t+1}$ and $p+2-t\leq p-1$,  using  \eqref{eq:GMRES-orth-space} gives
		\begin{equation}\label{eq:orth-rpArpt}
			r_p^T A r_{p+1-t}=0.
		\end{equation}
		Moreover, using the second equality in \eqref{eq:GMRES-pro}, i.e., $(r_{k+1}^G)^T(r_j^G-r_i^G) =0$ with $k=p-2, j=p-1, i=p+1-t$, we obtain 
		\begin{equation}\label{eq:orth-rpm1Arpt}
			r_{p-1}^T(r_{p-1}-r_{p+1-t})=0.	
		\end{equation}
		Using \eqref{eq:orth-rpArp} and \eqref{eq:orth-rpArpt}, we have 
		\begin{equation}\label{eq:orth-rpArpmpt}
			r_p^TA(r_{p-1}-r_{p+1-t})=0	.
		\end{equation}
		Using  \eqref{eq:orth-rpm1Arpt} and \eqref{eq:orth-rpArpmpt},  Lemma \ref{lem:xA^Tyz} gives us
		\begin{equation}\label{eq:orth-rpATrpmpt}
			r_p^TA^T(r_{p-1}-r_{p+1-t})=0.
		\end{equation}
		Thus, using the second equality in \eqref{eq:orth-rpArp}, \eqref{eq:orth-rpm1Arpt} and \eqref{eq:orth-rpATrpmpt} gives  \eqref{eq:cond1}, \eqref{eq:cond2}, and \eqref{eq:cond3}.
		
		Next, we show $r_{k+1}^{NG(\ell_1)}=r_{k+1}^{NG(p)}$, where $\ell_1= 1, 2, \cdots, p-1$. For NGMRES($\ell_1$), the coefficient matrix of the normal equation \eqref{eq:normal-equ} is $C_{\ell_1,k}=C_{p,p}(1:\ell_1+1,1:\ell_1+1)$, and $f_{\ell_1,k}=f_{p,p}(1:\ell_1+1)$. For NGMRES(1), let $\boldsymbol{\beta}_{1,*}^{(k)}$ be a solution of $C_{1,k}\boldsymbol{\beta}_1^{(k)}=f_{1,k}$. Then, $\boldsymbol{\beta}_{\ell_1,*}^{(k)}=[\boldsymbol{\beta}_{1,*}^{(k)};0_{\ell_1-1}]$, where $0_{\ell_1-1}$ is a column zero vector with length $\ell_1-1$,  is a solution of 
		$C_{\ell_1,k}\boldsymbol{\beta}_{\ell_1}^{(k)}=f_{\ell_1,k}$, i.e., a solution of the least-squares problem, \eqref{eq:LSP-linear}. It follows that $r_{k+1}^{NG(\ell_1)}=r_{k+1}^{NG(p)}$. Since $A$ is invertible, we have $x_{k+1}^{NG(\ell_1)}=x_{k+1}^{NG(p)}=x_{k+1}^{G}$ for $\ell_1= 1, 2, \cdots, p-1$.  Since $x_{k+1}^{NG(\ell)}=x_{k+1}^G $ for any $\ell\geq k_*+1=p$, we have $x_{k+1}^{NG(m)}=x_{k+1}^G$ for $m\in\mathbb{Z}^+$.
	\end{proof}
	
	\begin{remark}
		If we remove the requirement that $A$ is invertible, we can obtain $r_j^{NG(m)}=r_j^G, \quad \forall m \in \mathbb{Z}^{+}$, which can be easily seen from the above proof.  
	\end{remark}
	
	\subsection{Three-term recurrence for NGMRES(1)}
	For $m=1$ and given $x_0$, NGMRES(1) gives
	\begin{align}
		x_1 &= x_0 -\frac{r_0^T A^T r_0}{r_0^T A^T A r_0} r_0,\\ 
		x_{k+1} &= q(x_k) + \beta_0^{(k)} \left(q(x_k)-x_{k} \right) + \beta_1^{(k)} \left(q(x_k)-x_{k-1} \right), \quad k\geq 1 \label{eq:xk1-gmres1},
	\end{align}
	where the coefficients $ \beta_0^{(k)} $ and $ \beta_1^{(k)} $ are given by \eqref{eq:beta-form}. 
	We thus have
	\begin{equation*}
		W_{1,k}=[r_k-Mr_k, r_{k-1}-Mr_k]=[Ar_k, Ar_k+r_{k-1}-r_k]:=[w_1,w_2],
	\end{equation*}
	and the matrix and right-hand side of the least-squares problem are given, respectively,  by
	\begin{equation}\label{eq:ngmres1WkWk}
		C_{1,k}=
		\begin{bmatrix}
			w_1^Tw_1 &  w_1^Tw_2 \\
			w_2^Tw_1 &  w_2^Tw_2
		\end{bmatrix}
	\end{equation}
	and 
	\begin{equation}\label{eq:ngmres1f1k}
		f_{1,k}=W_2^TMr_k=
		\begin{bmatrix}
			w_1^TMr_k \\
			w_2^TMr_k
		\end{bmatrix}.
	\end{equation}
	We rewrite \eqref{eq:xk1-gmres1} as
	\begin{equation}
		x_{k+1}=x_k-(1+\beta_0^{(k)}+\beta_1^{(k)})r_k + \beta_1^{(k)}(x_k-x_{k-1}).
	\end{equation}
	The iterative procedure for NGMRES(1) is thus as given in Algorithm \ref{alg:NGMRES1} . When $A$ is symmetric, $w_1^TMr_2=-w_1^Tw_2$ because $(Ar_k)^Tr_{k-1}=0$. In this case, NGMRES(1) is GMRES, which in turn can be rewritten as the Conjugate Residual Method (CR)  \cite[Chapter 6.8]{saad2003iterative}. Thus, NGMRES(1) is equivalent to CR. However, CR updates $x_k$ in a different way. Moreover, in Algorithm \ref{alg:NGMRES1} if we replace $r_k$ by $Ax_k-b$, then the update $x_{k+1}$ is a two-term recurrence. 
	\begin{algorithm}[H]
		\caption{NGMRES(1): three-term recurrence} \label{alg:NGMRES1}
		\begin{algorithmic}[1] 
			\STATE Given $x_0$, let $r_0=Ax_0-b$
			\STATE Compute:  $x_1=x_0-\frac{r_0^TAr_0}{r_0^TA^TAr_0} r_0$
			\STATE  For $k=1,2,\cdots$ until convergence Do:
			\begin{itemize}
				\item compute $r_k=Ax_k-b$
				\item solve $C_{1,k}\boldsymbol{\beta}_1^{(k)}=f_{1,k}$, where are given in \eqref{eq:ngmres1WkWk} and \eqref{eq:ngmres1f1k}, and $\boldsymbol{\beta}_1^{(k)}=[\beta_0^{(k)}; \beta_1^{(k)}]$
				\item compute $x_{k+1}=x_k-(1+\beta_0^{(k)}+\beta_1^{(k)})r_k + \beta_1^{(k)}(x_k-x_{k-1})$
			\end{itemize}
			EndDo
		\end{algorithmic}
	\end{algorithm}
	\begin{remark}
	In Theorem \ref{thm:gmres1=gmres}, when $A=\alpha I+ S$, where $\alpha=0$ and $S$ is skew-symmetric,  no such $k_0$ exists, because in Algorithm \ref{alg:NGMRES1}, $x_1=x_0$ for both GMRES and NGMRES.  In this situation, NGMRES($m$) will make no progress, i.e., $x_k=x_0$ for $k>0$.  We note, however, that special short-recurrence versions of GMRES exist for purely skew-symmetric linear systems; see, for example, \cite{greif2016numerical, greif2009iterative} and the references therein.
	\end{remark}
	\section{Convergence analysis}\label{sec:conv}
	In this section, we conduct convergence analysis for NGMRES. From Theorem \ref{thm:orthogonality}, we know that the 2-norm of the residuals generated by NGMRES are  nonincreasing. Thus, the sequence $\{r_k\}$ always converges. If $\{r_k\}$  converges to zero, then  $\{x_k\}$ converges to the exact solution $x^*$. However, if $\{r_k\}$ converges to a nonzero vector, then $\{x_k\}$ does not converge to $x^*$. In the following, we will present a concrete convergence analysis for NGMRES. 
	
	
	We call $A$ a real positive definite matrix (or positive real)  if $A+A^T$ is symmetric positive definite.
    
	From \eqref{GMRES-LSP-rk}, we know that NGMRES($m$) minimizes the current residuals $r_{k+1}$, and NGMRES(0) is the same as MR.  Thus, the result of MR offers an upper bound on the convergence rate of NGMRES($m$). 
	
	\begin{theorem}\label{thm:NGMRES-converge-pd} 
		Let $A$ be a real positive definite matrix, and $\mu=\lambda_{\rm min}(A+A^T)/2$, $\nu=\lambda_{\rm min}(A^{-1}+(A^{-1})^T)/2$ and $\sigma=\|A\|_2$. Then, NGMRES($m$) converges for any initial guess $x_0$ and the corresponding residuals satisfy the relation
		\begin{equation}\label{eq:NGMRESm-bd1}
			\|r_{k+1}\|_2\leq \sqrt{1-\frac{\mu^2}{\sigma^2}}\,\|r_k\|_2,
		\end{equation}
		\begin{equation}\label{eq:NGMRESm-bd2}
			\|r_{k+1}\|_2\leq \sqrt{1- \mu \nu}\,\|r_k\|_2,
		\end{equation}
	\end{theorem}
	\begin{proof}
		When $m=0$, NGMRES(0) is MR, and  the above results can be directly inferred from \cite[Theorem 5.10]{saad2003iterative}.
		When $m>0$,  using 
		\begin{align*}
			\|r_{k+1}\|_2^2&=\min_{\boldsymbol{\beta}^{(k)}}\|\left(1+\sum_{i=0}^{m_k} \beta_i^{(k)} \right)  M r_k
			-\sum_{i=0}^{m_k} \beta_i^{(k)}   r_{k-i}\|^2_2 \\
			&\leq\min_{\beta_0^{(k)}}\|(1+\beta_0^{(k)})Mr_k-\beta_0^{(k)}r_k\|^2_2 \\
			&=\|(I-\alpha_k A)r_k\|^2_2,
		\end{align*}
		we obtain the desired result.
	\end{proof}
	
	
	For $m=0$,  AA($m$) is just the fixed-point iteration, $q(x_k)=Mx_k+b$. Then, AA($m$) converges if and only if $\|M\|<1$. In contrast, Theorem \ref{thm:NGMRES-converge-pd} tells us that for NGMRES(0) we only need to require that $A$ is a real positive definite matrix to guarantee convergence. In this situation, NGMRES(0) indeed accelerates the original fixed-point iteration, which is AA(0).
	
	We comment that the upper bounds \eqref{eq:NGMRESm-bd1} and \eqref{eq:NGMRESm-bd2} might not be sharp. Further investigation is needed; this is beyond the scope of this work.
	
	When $M=I-A$ is either symmetric or skew-symmetric, NGMRES(0) (which is identical to GMRES(1)) convergence has been studied in \cite{he2025gmres1}. Again, the result can offer an upper bound on the convergence of NGMRES($m$). We state it below.
	\begin{theorem}\cite{he2025gmres1}\label{thm:NGMRESm-sym-skew}
		When $A$ is a symmetric (positive or negative) definite matrix, NGMRES($m$) converges for any initial guess $x_0$ and the corresponding residuals satisfy the relation
		\begin{equation}\label{eq:NGMRES0-defi}
			\|r_{k+1}\|_2\leq \frac{|\lambda_{\max}(A)-\lambda_{\min}(A)|}{|\lambda_{\max}(A)+\lambda_{\min}(A)|}\|r_k\|_2.
		\end{equation} 
		When $M=I-A$ is skew-symmetric, then NMGRES($m$) converges for any initial guess $x_0$ and the corresponding residuals satisfy the relation
		\begin{equation}\label{eq:NGMRES0-skew}
			\|r_{k+1}\|_2\leq \frac{\rho(M)}{\sqrt{1+\rho(M)}}\|r_k\|_2,
		\end{equation} 
		where $\rho(M)$ is the spectral radius of $M$. 
	\end{theorem}
	\begin{proof}
		Using the fact that
		\begin{align*}
			\|r_{k+1}\|_2&=\min_{\boldsymbol{\beta}^{(k)}}\|\left(1+\sum_{i=0}^{m_k} \beta_i^{(k)} \right)  M r_k
			-\sum_{i=0}^{m_k} \beta_i^{(k)}   r_{k-i}\|_2 \\
			&\leq\min_{\beta_0^{(k)}}\|(1+\beta_0^{(k)})Mr_k-\beta_0^{(k)}r_k\|_2 \\
			&=\|(I-\alpha_k A)r_k\|_2,
		\end{align*}
		and the result for NMGRES(0) leads to the desired result.
	\end{proof}
	
	We comment that  bounds \eqref{eq:NGMRES0-defi} and \eqref{eq:NGMRES0-skew} are the worst-case root-convergence factor of NGMRES(0), i.e., GMRES(1), and can be attained; see \cite{he2025gmres1}. The convergence rate of GMRES(1) depends on the initial guess.  In \cite{he2025gmres1}, we have  shown that if $A$ is indefinite, the performance of GMRES(1) is highly dependent on the initial guess, and an initial guesses for which it may diverge can be constructed.  
	
	We  comment that the result in Theorem \ref{thm:NGMRESm-sym-skew} might not be sharp. However, it indicates that NGMRES($m$) ($m>0$) will not be worse than NGMRES(0).
	
	When $A$ is real symmetric positive definite, GMRES and  NGMRES(1) are the same. 
	\begin{corollary}\label{corollary:NGMRESm-r-converge-PSD} 
		Let $A$ be a real symmetric definite matrix. Then, NGMRES(1) converges for any initial guess $x_0$ and the corresponding residuals satisfy the relation
		\begin{equation}
			\|r_{k+1}\|_2\leq 2\left( \frac{\sqrt{\kappa}-1}{\sqrt{\kappa}+1}\right)^{k+1}\,\|r_0\|_2,
		\end{equation}
		where $\kappa=\frac{\lambda_{\rm max}(A)}{\lambda_{\rm min}(A)}$ if $A$ is symmetric positive definite matrix and $\kappa=\left|\frac{\lambda_{\rm min}(A)}{\lambda_{\rm max}(A)}\right|$ if $A$ is symmetric negative definite matrix. 
	\end{corollary}
	\begin{proof}
		This can be directly derived using Chebyshev polynomials (see, e.g.,  \cite[Chapter 3]{greenbaum1997iterative}).  
	\end{proof}
	
	\section{Numerical experiments}\label{sec:num}
	 
	In this section, we present some numerical results that validate our theoretical findings. We keep our numerical illustrations brief.
  
From \eqref{eq:definition-Wk} it can be observed that when $W_k$ is updated in the next iteration, an inexpensive procedure is required:
		$W_k=  D_k- E_k,
	$ where 
    $$D_k = \begin{bmatrix}
			r_k&  r_{k-1} &\cdots & r_{k-m_k}
		\end{bmatrix}$$
    and $E_k$ is a rank-one matrix whose columns are all identical to $M r_k$: 
    $$ E_k = F \otimes M r_k,$$ where $F$ is a column vector of all ones.  When $k \to k+1$, to obtain $W_{k+1}=D_{k+1}-E_{k+1}$, all columns but the first column of $D_k$  are just shifted, and the last column is discarded: $D_{k+1}$ will contain a new first column, $r_{k+1}$, and its next $m_k$ columns will be equal to the first through $(m_k-1)$st column of $D_k$, which are nothing but the last residuals. The matrix $E_{k+1}$  is trivial to construct, being rank-one. Therefore, the update of $W_k$ in every iteration amounts to an insertion of one column as the first new column, the deletion of the last column, and a rank-one correction with respect to the previous iteration. Fast techniques such as the ones discussed in \cite{daniel1976reorthogonalization} for the QR factorization may then be applied. 

\begin{example} \label{ex:conv-diff}
We consider the finite difference discretization of a convection-diffusion equation in two dimensions with constant convective coefficients.
 	\begin{equation}
 		-\Delta u+ \sigma   u_x+ \tau u_y=f.
 	\end{equation}
 Here, $\sigma$ and $\tau$ are the convective coefficients, and $f=f(x,y)$ is a known function. We discretize the equation on a uniform mesh whose size is $h$, and assume homogeneous Dirichlet boundary conditions. The discretization results in a system of $n^2 \times n^2$ linear equations, where $n$ is directly connected to the mesh size: $h = 1 / (n+1).$ Denoting the matrix of the linear system by $K$, we observe that for any nonzero $\sigma$ or $\tau$, $K$ is nonsymmetric. We denote the mesh Reynolds numbers by $\gamma_1 = \frac{\sigma h}{2}$ and
 $\gamma_2 = \frac{\tau h}{2}$. 
% 	
We consider $\gamma_1=\gamma_2=0.5$ and experiment with a linear system $Ax=b$ for two scenarios: (i) we take $A$ to be $K$; (ii) we denote $M=\frac{K-K^T}{2}$ and take $A=I-M$. The right-hand side vectors are set so that our solution is a vector of all 1s, and we start with a random initial guess. We note that there is an insignificant difference in the iteration counts between different choices of the initial guess or the right-hand side. 

Figure~\ref{fig:conv_diff} shows results for the two experiments with $n=32$ (matrix dimensions $1,024 \times 1,024$).
The left-hand side figure is consistent with our theoretical observation that NGMRES(1) and GMRES are different than one another for nonsymmetric linear systems.
The right-hand side figure confirms our theoretical analysis for a shifted skew-symmetric system (namely, that NGMRES(1) and GMRES are identical). 

\begin{figure}[H]
		\centering
		\includegraphics[width=6cm]{conv-diff2.pdf}
		\includegraphics[width=6cm]{conv-diff.pdf}
		\caption{Example \ref{ex:conv-diff}. Convergence history (residual norms) for NGMRES(1) vs GRMES. The left-hand side graph is for a nonsymmetric system. The right-hand side graph is for a shifted skew-symmetric system. }\label{fig:conv_diff}
	\end{figure}

 \end{example}

    
\begin{example}\label{ex:circ-staga}
We consider  $Ax=b$ taken from \cite[Chapter 6, Problem P-6.8]{saad2003iterative},  where 
\begin{equation}
   A= \begin{bmatrix}
     0 & 0 & 0 &0 & 1\\
     1&0 & 0 &0 &0 \\
     0& 1& 0& 0 &0 \\
    0& 0& 1& 0 &0\\ 
     0& 0& 0& 1 &0
    \end{bmatrix}, \quad 
    b=\begin{bmatrix}
     1\\0\\0\\0\\0
    \end{bmatrix},
\end{equation}
with inital guess $x_0=[0, 0, 0, 0,0]^T$.
\end{example}
The matrix $A$ is invertible and the exact solution is $x^*=[0, 0, 0, 0, 1]^T$. For this example, we derive the theoretical iterates  rather than run our numerical code. It can be shown that the iterates of GMRES are
\begin{equation}
x_4^G=x_3^G=x_2^G=x_1^G=x_0, \quad x_5^G=x^*.
\end{equation}
Thus, no $k_0$ exists in Theorem \ref{thm:relation-infinity},  and the theorem  does not tell us anything about the relationship between GMRES and NGMRES.

For NGMRES($\infty$), $ x_1^{NG}=x_0$. It follows that $W_1$ is rank deficient. We assume that NGMRES($\infty$) returns the minimum-norm solution  of the least-squares problem \eqref{eq:LSPrksame}. It can be theoretically shown that
\begin{equation}
    x_k^{NG}=x_0, \,\,  \forall k>0.
\end{equation}

When NGMRES(0) applies to Example \ref{ex:circ-staga}, it makes no progress too, i.e., $x_k^{NG}=x_0$ for $k>0$. 

We point out that if we consider a different initial guess, $x_0=[1, 1, 1, 1 ,1]^T$. Then, NGMRES($\infty$) and GMRES generate the same iterates and converges to the exact soltuon $x_5=x^*$. We also note that $W_k$ is full rank before NGMRES($\infty$)  conveges to $x^*$.
 

  In the next example, we compare the performance of NGMRES($m$) and GMRES for nonsymmetric $A$.  
\begin{example}\label{ex:cir-n50}
We extend Example \ref{ex:circ-staga} to a matrix $A$ with size $n=50$. We set $x_0$  to be a vector of all 1s, and run  NGMRES(10). Figure \ref{fig:A50} shows its performance  compared with GMRES. It indicates that when $A$ is nonsymmetric, the performance of NGMRES($m$) is not as good as that of GMRES, because NGMRES($m$) does not have a finite convergence property. The first eleven iterations are identical to machine precision (using double precision) but later iterations differ between the two methods, with GMRES having slightly lower residual norms throughout the first 49 iterations and converging to the exact solution at the last iteration.
\end{example}
	\begin{figure}[H]
		\centering
		%\includegraphics[width=6cm]{ngmres2}
		\includegraphics[width=6cm]{ngmres10}
		\caption{Example \ref{ex:cir-n50}. NGMRES(10) vs GRMES.}\label{fig:A50}
	\end{figure}
 
	
	\section{Concluding remarks}\label{sec:con}
	
	We have studied the properties of NGMRES applied to linear systems, which we summarized in the introduction and proved throughout the paper. 	Several questions remain open. For example, can the convergence behavior of NGMRES be characterized more precisely when the coefficient matrix is not real positive definite, and what can be said about convergence for specific scenarios of the spectral distribution of the matrix. The development of new efficient variants of NGMRES in combination with preconditioning is critical in practical applications and may provide a rich basis for further exploration, both theoretically and in terms of a practical efficient implementation. The implementation or development of efficient and stable solvers for the least-squares problems involved in NGMRES is an interesting topic, especially when the matrix of the least-squares problem is rank-deficient.  Finally, the investigation of situations of stagnation is a challenging and important issue. 
 

 
\bibliographystyle{siam}
\bibliography{NGMRESbib}

 


  

\end{document}
