\subsubsection{Inductive Proof}
\label{sec: probability bounds induction}
By induction, we will prove, given any MDP $\mathcal{M}$, that if we can assign $\theta$ such that the constraints of the optimisation problem \eqref{proofeq:interventional constraint}-\eqref{proofeq:valid prob2} are satisfied for all $|\mathcal{S}| \times |\mathcal{A}|$ state-action pairs in $\mathcal{M}$ separately, we know these can be combined to form a single valid $\theta$ across all $|\mathcal{S}| \times |\mathcal{A}|$ state-action pairs in $\mathcal{M}$.

Take an arbitrary MDP $\mathcal{M}$ with state space $|\mathcal{S}|$ and action space $|\mathcal{A}|$. In total, there are $N = |\mathcal{S}| \times |\mathcal{A}|$ state-action pairs in the MDP, meaning there are $|U_t| = |\mathcal{S}|^{|\mathcal{S}| \times |\mathcal{A}|}$ possible unique structural equation mechanisms. At each stage of the inductive proof, we will add a new state-action pair value from the MDP $\mathcal{M}$ to the causal model, and adjust $\theta$ to satisfy the constraints for this new state-action pair without affecting whether the constraints hold for all state-action pair values already considered in the causal model. When we have considered $k$ state-action pairs, this results in $\theta \in \mathbb{R}^{|\mathcal{S}|^k}$. Therefore, once we have added all $|\mathcal{S}| \times |\mathcal{A}|$ state-action pairs, we will have $\theta \in \mathbb{R}^{|U_t|}$, as required by the optimisation problem.

\paragraph{Base Case 1} Assume $(s_t, a_t)$ (the observed state-action pair) is the only state-action pair value we want to consider in the causal model (this always exists, as we always have an observed transition). We require a structural equation mechanism for each possible next state (of which there are $|\mathcal{S}|$). This leads to an assignment of $\theta^1 \in \mathbb{R}^{|\mathcal{S}|}$\footnote{The superscript of each $\theta^k$ indicates the number of state-action pair values we have considered in the causal model. When we have all $|\mathcal{S}| \times |\mathcal{A}|$ state-action pairs, we will have our final $\theta = \theta^{|\mathcal{S}| \times |\mathcal{A}|}$} as follows, which is the only possible assignment that satisfies all the constraints \eqref{proofeq:interventional constraint}-\eqref{proofeq:valid prob2}:

\begin{table}[h]
\centering
\begin{tabular}{c|c|c|c}
$\theta^1_{u_1}$                                  & $\theta^1_{u_2}$                                 & ... & $\theta^1_{u_{|\mathcal{S}|}}$\\
\hline
$P(s_1 \mid s_t, a_t)$ & $P(s_2 \mid s_t, a_t)$ & ... & $P(s_{|\mathcal{S}|} \mid s_t, a_t)$
\end{tabular}
\end{table}

$\theta^1 = [P(s_1 \mid s_t, a_t), P(s_2 \mid s_t, a_t), ..., P(s_{|\mathcal{S}|} \mid s_t, a_t)]$. Given $\theta^1$, the counterfactual transition probability for each transition $s_t, a_t \rightarrow s'$ can be calculated with:

\[
\tilde{P}_t(s' \mid s_t, a_t) = \dfrac{\sum_{u_t = 1}^{|U_t|} \mu_{s_t, a_t, u_t, s'} \cdot \mu_{s_{t}, a_{t}, u_t, s_{t+1}} \cdot \theta_{u_t}}{P(s_{t+1} \mid s_t, a_t)}
\]

Therefore, given $\theta^1$, $\tilde{P}_t(s_{t+1} \mid s_t, a_t) = 1$, and $\forall s' \in \mathcal{S}\setminus\{s_{t+1}\}$, $\tilde{P}_t(s' \mid s_t, a_t) = 0$. These are the only possible counterfactual probabilities, so these probabilities are produced by the closed-form bounds in Section \ref{sec: probability bounds cases}. These counterfactual probabilities satisfy the monotonicity and counterfactual stability constraints of the optimisation problem, as follows:

\begin{itemize}
    \item Because $\dfrac{P(s_{t+1} \mid s_t, a_t)}{P(s_{t+1} \mid s_t, a_t)} = 1$ and $\forall s' \in \mathcal{S}, \dfrac{P(s' \mid s_t, a_t)}{P(s' \mid s_t, a_t)} = 1$,

    \[
    \forall s' \in \mathcal{S}, \dfrac{P(s_{t+1} \mid s_t, a_t)}{P(s_{t+1} \mid s_t, a_t)} = \dfrac{P(s' \mid s_t, a_t)}{P(s' \mid s_t, a_t)}
    \]

    so CS \eqref{proofeq:counterfactual stability} is always vacuously true for all transitions from $(s_t, a_t)$.\\

    \item Because $\tilde{P}_{t}(s_{t+1} \mid s_t, a_t) = 1 \geq P(s_{t+1} \mid s_t, a_t)$, this satisfies Mon1 \eqref{proofeq:monotonicity1}.
    
    \item Because $\forall s' \in \mathcal{S}\setminus \{s_{t+1}\}, \tilde{P}_{t}(s' \mid s_t, a_t) = 0 \leq P(s' \mid s_t, a_t)$, this satisfies Mon2 \eqref{proofeq:monotonicity2}.
\end{itemize}

\paragraph{Base Case 2} Let us assume that we only have the observed state-action pair in the causal model, and assume we have a valid $\theta^1$ with $|\mathcal{S}|$ structural equation mechanisms. 

Now, take the state-action pair for which we wish to calculate the probability bounds, $(s, a)$, to add to the causal model. We now require $|\mathcal{S}|^2$ structural equation mechanisms, one for each possible combination of transitions from the two state-action pairs. We can view this as separating each of the existing structural equation mechanisms from Base Case 1 into $|\mathcal{S}|$ new structural equation mechanisms, one for every possible transition from the new state-action pair, $(s, a)$:

\begin{figure}[!ht]
\centering
\resizebox{0.5\textwidth}{!}{%
\begin{circuitikz}
\tikzstyle{every node}=[font=\LARGE]
\node [font=\LARGE] at (3,16.25) {$u_1$};
\node [font=\LARGE] at (6.0,16.25) {$...$};
\node [font=\LARGE] at (8.75,16.25) {$u_{|S|}$};
\draw [short] (3,15.75) -- (1.25,13.5);
\draw [short] (3,15.75) -- (5,13.5);
\draw [short] (3,15.75) -- (3,13.5);
\node [font=\LARGE] at (1,13) {$u_{1,1}$};
\node [font=\LARGE] at (5,13) {$u_{1,|S|}$};
\node [font=\LARGE] at (3,13) {$u_{1,2}$};
\node [font=\LARGE] at (4.0,13) {$...$};
\node [font=\LARGE] at (9.7,13) {$...$};

\draw [short] (8.75,15.75) -- (7,13.5);
\draw [short] (8.75,15.75) -- (10.75,13.5);
\draw [short] (8.75,15.75) -- (8.75,13.5);
\node [font=\LARGE] at (6.75,13) {$u_{|S|,1}$};
\node [font=\LARGE] at (10.75,13) {$u_{|S|,|S|}$};
\node [font=\LARGE] at (8.75,13) {$u_{|S|,2}$};
\end{circuitikz}
}%
\end{figure}

where, for $1 \leq i \leq |\mathcal{S}|, 1 \leq n \leq |\mathcal{S}|$, $u_{i, n}$ leads to the same next state for $(s_t, a_t)$ as $u_i$, and produces the transition $s, a \rightarrow s_n$. \\

In the same way, we can split each $\theta^1_{u_1}, ..., \theta^1_{u_{|\mathcal{S}|}}$ across these new structural equation mechanisms to obtain $\theta^2 \in \mathbb{R}^{|\mathcal{S}|^2}$:

\begin{figure}[!ht]
\centering
\resizebox{0.5\textwidth}{!}{%
\begin{circuitikz}
\tikzstyle{every node}=[font=\LARGE]
\node [font=\LARGE] at (3,16.25) {$\theta^1_{u_1}$};
\node [font=\LARGE] at (6.0,16.25) {$...$};
\node [font=\LARGE] at (8.75,16.25) {$\theta^1_{u_{|S|}}$};
\draw [short] (3,15.75) -- (1.25,13.5);
\draw [short] (3,15.75) -- (5,13.5);
\draw [short] (3,15.75) -- (3,13.5);
\node [font=\LARGE] at (1,13) {$\theta^2_{u_{1,1}}$};
\node [font=\LARGE] at (5,13) {$\theta^2_{u_{1,|S|}}$};
\node [font=\LARGE] at (3,13) {$\theta^2_{u_{1,2}}$};
\node [font=\LARGE] at (3.9,13) {$...$};
\node [font=\LARGE] at (9.6,13) {$...$};
\draw [short] (8.75,15.75) -- (7,13.5);
\draw [short] (8.75,15.75) -- (10.75,13.5);
\draw [short] (8.75,15.75) -- (8.75,13.5);
\node [font=\LARGE] at (6.75,13) {$\theta^2_{u_{|S|,1}}$};
\node [font=\LARGE] at (10.75,13) {$\theta^2_{u_{|S|,|S|}}$};
\node [font=\LARGE] at (8.75,13) {$\theta^2_{u_{|S|,2}}$};
\end{circuitikz}
}%
\end{figure}

By splitting $\theta^1$ in this way, we guarantee that the total probability across each set of mechanisms (where each set produces a different transition from $(s_t, a_t)$) remains the same, i.e., \[\forall i \in \{1, .., |\mathcal{S}|\}, \sum_{n=1}^{|\mathcal{S}|}\theta^2_{u_{i,n}} = \theta_{u_i}^1\]

This means that the counterfactual probabilities of all transitions from $(s_t, a_t)$ will be exactly the same when using $\theta^2$ as with $\theta^1$. Because we know all the constraints were satisfied when using $\theta^1$, this guarantees that all the constraints of the optimisation problem will continue to hold for $(s_t, a_t)$ with $\theta^2$.\\

Now, we only need to ensure we assign $\theta^2$ such that all the constraints hold for the transitions from $(s, a)$. Firstly, we need $\sum_{i=1}^{|\mathcal{S}|}{\theta^2_{u_{i, n}}} = P(s_n \mid s, a)$ for every possible next state $s_n \in \mathcal{S}$, to satisfy the interventional constraint \eqref{proofeq:interventional constraint}. For each transition $s, a \rightarrow s_n$, there is a $\theta^2_{u_{i, n}}$ for every $u_i$, and for every $\theta^2_{u_{i, n}}$, $0 \leq \theta^2_{u_{i, n}} \leq \theta^1_{u_n}$. Therefore, $0 \leq \sum_{i=1}^{|\mathcal{S}|}{\theta^2_{u_{i, n}}} \leq 1, \forall s_n \in \mathcal{S}$. We also know that:

\begin{align*}
\sum_{n=1}^{|\mathcal{S}|}\sum_{i=1}^{|\mathcal{S}|}\theta^2_{u_i,n} &= \sum_{i=1}^{|\mathcal{S}|}\theta^1_{u_i} \text{ (because we split the probabilities of each $\theta^1_{u_i}$)}\\
&= \sum_{i=1}^{|\mathcal{S}|}P(s_i|s_t, a_t) \\&= 1
\end{align*}

and 

\[
\sum_{n=1}^{|\mathcal{S}|}\sum_{i=1}^{|\mathcal{S}|}{\theta^2_{u_{i, n}}} = 
\sum_{n=1}^{|\mathcal{S}|}P(s_n|s,a) = 1
\]

Therefore, no matter the assignment of $\theta^1$, we can satisfy the interventional probability constraints \eqref{proofeq:interventional constraint} for all transitions from $(s, a)$. In Section \ref{sec: probability bounds cases}, we prove valid assignments of $\theta$ that would minimise and maximise the counterfactual probability for some transition from $(s, a)$, and satisfy all the constraints \eqref{proofeq:interventional constraint}-\eqref{proofeq:valid prob2}. Therefore, we can pick the assignment of $\theta^2$ from these cases to minimise/maximise the counterfactual probability of the transition in question.

\paragraph{Inductive Case} Let us assume that we now have $k$ state-action pair values from the MDP $\mathcal{M}$ in our causal model, and we have found a valid assignment $\theta^k \in \mathbb{R}^{|\mathcal{S}|^k}$ that satisfies all the constraints for these $k$ state-action pairs, including $(s_t, a_t)$. Because there are $k$ state-action pairs, there are $|\mathcal{S}|^{k}$ possible unique structural equation mechanisms for these state-action pairs\footnote{Note we have changed the indexing of each $\theta^k_{i, n}$ to $\theta^k_{j}$, where $j = (i-1)\cdot|\mathcal{S}| + n$.}, with probabilities as follows:

\begin{table}[h]
\centering
\begin{tabular}{lllllll}
$\theta^k_{u_1}$ & ... & $\theta^k_{u_{|\mathcal{S}|}}$ & ... & $\theta^k_{u_{|\mathcal{S}|^2}}$ & ... & $\theta^k_{u_{|\mathcal{S}|^{k}}}$ \\
\end{tabular}
\end{table}

Now, we wish to add the $k+1^{th}$ state-action pair from the MDP to the causal model, resulting in $|\mathcal{S}|^{k+1}$ possible structural equation mechanisms. Let the $k+1^{th}$ state-action pair be $(s, a)$ arbitrarily. We can view this as separating the existing structural equation mechanisms each into $|\mathcal{S}|$ new structural equation mechanisms, one for every possible transition from the $k+1^{th}$ state-action pair:

\begin{figure}[!ht]
\centering
\resizebox{1\textwidth}{!}{%
\begin{circuitikz}
\tikzstyle{every node}=[font=\LARGE]
\node [font=\LARGE] at (3.0,16.25) {$u_1$};
\node [font=\LARGE] at (6.0,16.25) {$...$};
\node [font=\LARGE] at (8.75,16.25) {$u_{|S|}$};
\node [font=\LARGE] at (12,16.25) {$...$};
\node [font=\LARGE] at (14.75,16.25) {$u_{|S|^2}$};
\node [font=\LARGE] at (20.75,16.25) {$u_{|S|^k}$};
\node [font=\LARGE] at (18,16.25) {$...$};
\draw [short] (3,15.75) -- (1.25,13.5);
\draw [short] (3,15.75) -- (5,13.5);
\draw [short] (3,15.75) -- (3,13.5);
\node [font=\LARGE] at (1,13) {$u_{1,1}$};
\node [font=\LARGE] at (5,13) {$u_{1,|S|}$};
\node [font=\LARGE] at (3,13) {$u_{1,2}$};
\node [font=\LARGE] at (3.9,13) {$...$};
\draw [short] (8.75,15.75) -- (7,13.5);
\draw [short] (8.75,15.75) -- (10.75,13.5);
\draw [short] (8.75,15.75) -- (8.75,13.5);
\node [font=\LARGE] at (6.75,13) {$u_{|S|,1}$};
\node [font=\LARGE] at (10.75,13) {$u_{|S|,|S|}$};
\node [font=\LARGE] at (8.75,13) {$u_{|S|,2}$};
\node [font=\LARGE] at (9.7,13) {$...$};
\draw [short] (14.75,15.75) -- (13,13.5);
\draw [short] (14.75,15.75) -- (16.75,13.5);
\draw [short] (14.75,15.75) -- (14.75,13.5);
\node [font=\LARGE] at (12.75,13) {$u_{|S|^2,1}$};
\node [font=\LARGE] at (17,13) {$u_{|S|^2,|S|}$};
\node [font=\LARGE] at (14.75,13) {$u_{|S|^2,2}$};
\node [font=\LARGE] at (15.8,13) {$...$};
\draw [short] (20.75,15.75) -- (19,13.5);
\draw [short] (20.75,15.75) -- (22.75,13.5);
\draw [short] (20.75,15.75) -- (20.75,13.5);
\node [font=\LARGE] at (18.75,13) {$u_{|S|^{k},1}$};
\node [font=\LARGE] at (23,13) {$u_{|S|^{k},|S|}$};
\node [font=\LARGE] at (20.75,13) {$u_{|S|^{k}, 2}$};
\node [font=\LARGE] at (21.8,13) {$...$};
\end{circuitikz}
}%
\end{figure}
where each $u_{i, n}$ produces the same next states for all of the first $k$ state-action pairs as $u_i$, and produces the transition $s, a \rightarrow s_n$.

In the same way, we can split each $\theta^k_{u_1}, ..., \theta^k_{u_{|\mathcal{S}|}}, ...$ across these new structural equation mechanisms to find $\theta^{k+1} \in \mathbb{R}^{|\mathcal{S}|^{k+1}}$.

\begin{figure}[!ht]
\centering
\resizebox{1\textwidth}{!}{%
\begin{circuitikz}
\tikzstyle{every node}=[font=\LARGE]
\node [font=\LARGE] at (3,16.25) {$\theta^k_{u_1}$};
\node [font=\LARGE] at (6.0,16.25) {$...$};
\node [font=\LARGE] at (8.75,16.25) {$\theta^k_{u_{|S|}}$};
\node [font=\LARGE] at (11.8,16.25) {$...$};
\node [font=\LARGE] at (14.75,16.25) {$\theta^k_{u_{|S|^2}}$};
\node [font=\LARGE] at (20.75,16.25) {$\theta^k_{u_{|S|^k}}$};
\node [font=\LARGE] at (17.8,16.25) {$...$};
\draw [short] (3,15.75) -- (1.25,13.5);
\draw [short] (3,15.75) -- (5,13.5);
\draw [short] (3,15.75) -- (3,13.5);
\node [font=\LARGE] at (1,13) {$\theta^{k+1}_{u_{1,1}}$};
\node [font=\LARGE] at (5,13) {$\theta^{k+1}_{u_{1,|S|}}$};
\node [font=\LARGE] at (3,13) {$\theta^{k+1}_{u_{1,2}}$};
\node [font=\LARGE] at (3.9,13) {$...$};
\draw [short] (8.75,15.75) -- (7,13.5);
\draw [short] (8.75,15.75) -- (10.75,13.5);
\draw [short] (8.75,15.75) -- (8.75,13.5);
\node [font=\LARGE] at (6.75,13) {$\theta^{k+1}_{u_{|S|,1}}$};
\node [font=\LARGE] at (10.9,13) {$\theta^{k+1}_{u_{|S|,|S|}}$};
\node [font=\LARGE] at (8.75,13) {$\theta^{k+1}_{u_{|S|,2}}$};
\node [font=\LARGE] at (9.7,13) {$...$};
\draw [short] (14.75,15.75) -- (13,13.5);
\draw [short] (14.75,15.75) -- (16.75,13.5);
\draw [short] (14.75,15.75) -- (14.75,13.5);
\node [font=\LARGE] at (12.75,13) {$\theta^{k+1}_{u_{|S|^2,1}}$};
\node [font=\LARGE] at (17,13) {$\theta^{k+1}_{u_{|S|^2,|S|}}$};
\node [font=\LARGE] at (14.75,13) {$\theta^{k+1}_{u_{|S|^2,2}}$};
\node [font=\LARGE] at (15.75,13) {$...$};
\draw [short] (20.75,15.75) -- (19,13.5);
\draw [short] (20.75,15.75) -- (22.75,13.5);
\draw [short] (20.75,15.75) -- (20.75,13.5);
\node [font=\LARGE] at (18.75,13) {$\theta^{k+1}_{u_{|S|^{k},1}}$};
\node [font=\LARGE] at (23,13) {$\theta^{k+1}_{u_{|S|^{k},|S|}}$};
\node [font=\LARGE] at (20.75,13) {$\theta^{k+1}_{u_{|S|^{k}, 2}}$};
\node [font=\LARGE] at (21.75,13) {$...$};
\end{circuitikz}
}%
\end{figure}

By splitting $\theta^k$ in this way, we have $\forall i \in \{1, .., |\mathcal{S}|\}, \sum_{n=1}^{|\mathcal{S}|}\theta^{k+1}_{u_{i,n}} = \theta_{u_i}^k$. Each $\theta^{k+1}_{u_{i, n}}$ is only different to $\theta^k_{u_i}$ in the transition from $(s, a)$. Therefore, this guarantees that for every state-action pair $(s', a') \neq (s, a)$ in the first $k$ pairs added to the causal model, the total probability across each set of mechanisms (where each set produces a different transition from $(s', a')$) remains the same in $\theta^{k+1}$ as in $\theta^k$. As a result, the counterfactual probabilities of all transitions from each $(s', a')$ will be exactly the same when using $\theta^{k+1}$ as with $\theta^k$. Because we know all the constraints were satisfied when using $\theta^k$, this guarantees that all the constraints of the optimisation problem will continue to hold for $(s', a')$ with $\theta^{k+1}$.\\

Now, we only need to make sure that we can assign $\theta^{k+1}$ such that the constraints also hold for all the transitions from $(s, a)$. We need $\sum_{i=1}^{|\mathcal{S}|^k}{\theta^{k+1}_{u_{i, n}}} = P(s_n \mid s, a)$ for every possible next state $s_n \in \mathcal{S}$, to satisfy the interventional probability constraints \eqref{proofeq:interventional constraint}. For each transition $s, a \rightarrow s_n$, there is a $\theta^{k+1}_{u_{i, n}}$ for every $u_i$, and for every $\theta^{k+1}_{u_{i, n}}$, $0 \leq \theta^{k+1}_{u_{i, n}} \leq \theta^k_{u_n}$, therefore $0 \leq \sum_{i=1}^{|\mathcal{S}|^k}{\theta^{k+1}_{u_{i, n}}} \leq 1, \forall s_n \in \mathcal{S}$. We also know that:

\begin{align*}
\sum_{n=1}^{|\mathcal{S}|}\sum_{i=1}^{|\mathcal{S}|^k}\theta^{k+1}_{u_i,n} &= \sum_{i=1}^{|\mathcal{S}|^k}\theta^k_{u_i} \text{ (because we split the probabilities of each $\theta^k_{u_i}$)}\\&= 1
\end{align*}

and 

\[
\sum_{n=1}^{|\mathcal{S}|}\sum_{i=1}^{|\mathcal{S}|^k}{\theta^2_{u_{i, n}}} = 
\sum_{n=1}^{|\mathcal{S}|}P(s_n|s,a) = 1
\]

Therefore, no matter the assignment of $\theta^k$, we can satisfy the interventional probability constraints \eqref{proofeq:interventional constraint} for all transitions from $(s, a)$. In Section \ref{sec: probability bounds cases}, we show that we can assign $\theta^{k+1}$ such that it would minimise and maximise the counterfactual probability for some transition from $(s, a)$, and satisfy all the constraints \eqref{proofeq:interventional constraint}-\eqref{proofeq:valid prob2}. Therefore, we know we can assign $\theta^{k+1}$ such that it satisfies the constraints for all $k+1$ state-action pairs in the causal model.

\paragraph{Conclusion} By induction, we have proven that as long as we can satisfy the constraints for each state-action pair separately, we can combine these into a single valid $\theta$ that satisfies the constraints for all state-action pairs, and minimises/maximises the counterfactual probability for a particular transition in question. Because the state-action pairs are added in an arbitrary order in this inductive proof (other than the observed state-action pair and the state-action pair for the transition in question, which both must always exist), this works for any MDP.