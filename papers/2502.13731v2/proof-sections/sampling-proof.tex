\pagebreak
\subsection{Any CFMDP Entailed by an Interval CFMDP is a Valid CFMDP}
\label{sec: sampled interval cfmdp proof}
In our experiments, we evaluate the policies by sampling example CFMDPs from the interval CFMDPs. But, even if $\theta$s exist that would produce each sampled CF transition probability separately, this does not necessarily guarantee that there exists a single $\theta$ that produces all of the CF transition probabilities in the CFMDP. We can prove that, given any CFMDP $\tilde{\mathcal{M}}_t$, there exists a $\theta$ that produces all the counterfactual probabilities in $\tilde{\mathcal{M}}_t$, as follows.

\begin{theorem}
    Given any arbitrary interval counterfactual MDP, any counterfactual MDP $\tilde{\mathcal{M}}_t$ entailed by this interval counterfactual MDP will be a valid counterfactual MDP.
\end{theorem}

\begin{proof}
Let us assume that we have an interval CFMDP, and an arbitrary CFMDP $\tilde{\mathcal{M}_t}$ entailed by this interval CFMDP for each time-step $t$ from this interval CFMDP. First, we need to prove that, for every transition $s, a \rightarrow s'$ in $\tilde{\mathcal{M}_t}$, there exists a $\theta$ that produces its counterfactual probability $\tilde{P}_t(s' \mid s, a)$. Next, we need to show that, for all state-action pairs $(s, a)$, there exists a $\theta$ that satisfies the constraints of the optimisation problem \eqref{proofeq:interventional constraint}-\eqref{proofeq:valid prob2} and produces the entailed CF probability distribution $\tilde{P}_t(\cdot \mid s, a)$. Finally, similarly to the proof in Section \ref{sec: probability bounds proof}, we can prove by induction that these assignments of $\theta$ for each state-action pair can be combined to form a single $\theta$ that produces $\tilde{\mathcal{M}_t}$.

\subsubsection{Existence of $\theta$ for Each $\tilde{P}_t(s' \mid s, a)$}
The counterfactual transition probability bounds are produced by a convex optimisation problem (or using the analytical bounds, which we have proven produce the same exact bounds as the optimisation problem). Because the optimisation problem is convex, the feasible set for each of the counterfactual transition probabilities is also convex (i.e., there are no gaps in the probability bounds). This means there is guaranteed to be at least one $\theta$ that can produce each of the entailed counterfactual transition probabilities in $\tilde{\mathcal{M}_t}$.

\subsubsection{Existence of $\theta$ for Each $\tilde{P}_t(\cdot \mid s, a)$}
\label{sec: theta for sampled distribution}
Next, we need to show that, for all state-action pairs $(s, a)$, there exists a $\theta$ that satisfies the constraints of the optimisation problem \eqref{proofeq:interventional constraint}-\eqref{proofeq:valid prob2} and can produce the entailed CF probability distribution $\tilde{P}_t(\cdot \mid s, a)$. Take any state-action pair $(s, a)$ in $\tilde{\mathcal{M}_t}$ arbitrarily. Each CF transition probability $\tilde{P}_t(\cdot \mid s, a)$ arises from the probability over the mechanisms that produce each transition $s, a \rightarrow s'$ and observed transition $s_t, a_t \rightarrow s_{t+1}$ vs. the total probability assigned to $P(s_{t+1} \mid s_t, a_t)$. Therefore, we need to show that the total probability assigned across all mechanisms containing each transition and the observed transition sums to $P(s_{t+1} \mid s_t, a_t)$. Because $\tilde{P}_t(\cdot \mid s, a)$ is a valid probability distribution, we know

\[
\sum_{s' \in \mathcal{S}}\tilde{P}_t(s' \mid s, a) = 1
\]

This means

\[
\sum_{s' \in \mathcal{S}}\dfrac{\sum_{u_t = 1}^{|U_t|} \mu_{s, a, u_t, s'} \cdot \mu_{s_{t}, a_{t}, u_t, s_{t+1}} \cdot \theta_{u_t}}{P(s_{t+1} \mid s_t, a_t)} = 1
\]

and 

\[
\sum_{s' \in \mathcal{S}}\sum_{u_t = 1}^{|U_t|} \mu_{s, a, u_t, s'} \cdot \mu_{s_{t}, a_{t}, u_t, s_{t+1}} \cdot \theta_{u_t} = {P(s_{t+1} \mid s_t, a_t)}
\]

Because $\mu_{s, a, u_t, s'} = 1$ for exactly one next state, $s'$, and $0$ for the remaining states, we must have

\[
\sum_{u_t = 1}^{|U_t|} \mu_{s_{t}, a_{t}, u_t, s_{t+1}} \cdot \theta_{u_t} = {P(s_{t+1} \mid s_t, a_t)}
\]

so the total probability assigned across all mechanisms containing each transition and the observed transition sums to $P(s_{t+1} \mid s_t, a_t)$.

Next, we need to show that there exists an assignment of $\theta$ which satisfies the interventional probability constraints \eqref{proofeq:interventional constraint} of the optimisation problem. We know that the interventional probabilities of all transitions from $(s, a)$ must sum to $1$:

\[
\sum_{s' \in \mathcal{S}}P(s' \mid s, a) = 1
\]

Because $0 \leq \theta_{u_t} \leq 1, \forall u_t$ \eqref{proofeq:valid prob1} and $\sum_{u_t=1}^{|U_t|}\theta_{u_t} = 1$ \eqref{proofeq:valid prob2} it must be possible to assign $\theta$ such that 

\[
\sum_{u_t = 1}^{|U_t|} \mu_{s, a, u_t, s'} \cdot \theta_{u_t} = P(s' \mid s, a)
\]

and

\[
\forall s' \in \mathcal{S}, \sum_{u_t = 1}^{|U_t|} \mu_{s, a, u_t, s'} \cdot \mu_{s_{t}, a_{t}, u_t, s_{t+1}} \cdot \theta_{u_t} = \tilde{P}_t(s' \mid s, a) \cdot P(s_{t+1} \mid s_t, a_t)
\]

and 

\[
\forall s' \in \mathcal{S}, \sum_{\tilde{s}' \neq s_{t+1}}\sum_{u_t = 1}^{|U_t|} \mu_{s, a, u_t, s'} \cdot \mu_{s_{t}, a_{t}, u_t, \tilde{s}'} \cdot \theta_{u_t} = P(s' \mid s, a) - \tilde{P}_t(s' \mid s, a) \cdot P(s_{t+1} \mid s_t, a_t)
\]

Therefore, this assignment satisfies
\[
\sum_{u_t = 1}^{|U_t|} \mu_{s, a, u_t, s'} \cdot \theta_{u_t} = P(s' \mid s, a)
\]

\[
\sum_{u_t = 1}^{|U_t|} \mu_{s_{t}, a_{t}, u_t, s_{t+1}} \cdot \theta_{u_t} = {P(s_{t+1} \mid s_t, a_t)}
\]

and 

\[
\sum_{s' \in \mathcal{S}\setminus\{s_{t+1}\}}\sum_{u_t = 1}^{|U_t|} \mu_{s_{t}, a_{t}, u_t, s'} \cdot \theta_{u_t} = 1 - P(s_{t+1} \mid s_t, a_t)
\]

which satisfies the interventional probability constraints \eqref{proofeq:interventional constraint}. Finally, because each counterfactual transition probability was sampled from its bounds, we know that the entailed counterfactual transition probabilities satisfy the Mon1 \eqref{proofeq:monotonicity1}, Mon \eqref{proofeq:monotonicity2} and CS \eqref{proofeq:counterfactual stability} constraints as well.

\subsubsection{Existence of $\theta$ for $\tilde{\mathcal{M}_t}$}

We have shown that there exists a $\theta$ that meets the constraints of the optimisation problem for each state-action pair separately in the MDP. Now, 
we need to prove that these assignments can be combined to form a valid $\theta$ for all state-action pairs. Let us prove inductively that adding a new state-action pair from the MDP to the causal model (and changing $\theta$ to meet the constraints for this new state-action pair) does not affect whether the constraints of the other state-action pairs are satisfied. If this is the case, we can consider how $\theta$ can be assigned to satisfy the constraints for each state-action pair separately, as we know these can be combined to form a valid assignment of $\theta$ over all of the state-action pairs. 

\paragraph{Base Case 1} Assume $(s_t, a_t)$ (the observed state-action pair) is the only state-action pair currently considered in the causal model (this always exists, as we always have an observed transition). We require a structural equation mechanism for each possible next state (of which there are $|\mathcal{S}|$). This leads to an assignment of $\theta^1$ as follows, which is the only possible assignment that satisfies the constraints \eqref{proofeq:interventional constraint}-\eqref{proofeq:valid prob2}:

\begin{table}[h]
\centering
\begin{tabular}{c|c|c|c}
$\theta^1_{u_1}$                                  & $\theta^1_{u_2}$                                 & ... & $\theta^1_{u_{|\mathcal{S}|}}$\\
\hline
$P(s_1 \mid s_t, a_t)$ & $P(s_2 \mid s_t, a_t)$ & ... & $P(s_{|\mathcal{S}|} \mid s_t, a_t)$
\end{tabular}
\end{table}

$\theta^1 = [P(s_1 \mid s_t, a_t), P(s_2 \mid s_t, a_t), ..., P(s_{|\mathcal{S}|} \mid s_t, a_t)]$. Given $\theta^1$, the counterfactual transition probability for each transition $s_t, a_t \rightarrow s'$ can be calculated with:

\[
\tilde{P}_t(s' \mid s_t, a_t) = \dfrac{\sum_{u_t = 1}^{|U_t|} \mu_{s_t, a_t, u_t, s'} \cdot \mu_{s_{t}, a_{t}, u_t, s_{t+1}} \cdot \theta_{u_t}}{P(s_{t+1} \mid s_t, a_t)}, \forall s'
\]

Therefore, given $\theta^1$, $\tilde{P}_t(s_{t+1} \mid s_t, a_t) = 1$, and $\forall s' \in \mathcal{S}\setminus\{s_{t+1}\}$, $\tilde{P}_t(s' \mid s_t, a_t) = 0$. This is the only possible counterfactual probability distribution for $(s_t, a_t)$, so this is guaranteed to be the distribution entailed from the interval CFMDP. These counterfactual probabilities also satisfy the monotonicity and counterfactual stability constraints (6-8) of the optimisation problem, as follows:

\begin{itemize}
    \item Because $\dfrac{P(s_{t+1} \mid s_t, a_t)}{P(s_{t+1} \mid s_t, a_t)} = 1$ and $\forall s' \in \mathcal{S}, \dfrac{P(s' \mid s_t, a_t)}{P(s' \mid s_t, a_t)} = 1$,

    \[
    \forall s' \in \mathcal{S}, \dfrac{P(s_{t+1} \mid s_t, a_t)}{P(s_{t+1} \mid s_t, a_t)} = \dfrac{P(s' \mid s_t, a_t)}{P(s' \mid s_t, a_t)}
    \]

    so CS \eqref{proofeq:counterfactual stability} is always vacuously true for all transitions from $(s_t, a_t)$.\\

    \item Because $\tilde{P}_{t}(s_{t+1} \mid s_t, a_t) = 1 \geq P(s_{t+1} \mid s_t, a_t)$, this satisfies Mon1 \eqref{proofeq:monotonicity1}.
    
    \item Because $\forall s' \in \mathcal{S}\setminus \{s_{t+1}\}, \tilde{P}_{t}(s' \mid s_t, a_t) = 0 \leq P(s' \mid s_t, a_t)$, this satisfies Mon2 \eqref{proofeq:monotonicity2}.
\end{itemize} 

\paragraph{Base Case 2} Let us assume that we only have the observed state-action pair in the causal model, and assume we have a valid $\theta^1$ with $|\mathcal{S}|$ structural equation mechanisms. 

Now, take another state-action pair $(s, a)$ arbitrarily from the MDP to add to the causal model. We now require $|\mathcal{S}|^2$ structural equation mechanisms, one for each possible combination of transitions from the two state-action pairs. We can view this as separating each of the existing structural equation mechanisms from Base Case 1 into $|\mathcal{S}|$ new structural equation mechanisms, one for every possible transition from the new state-action pair, $(s, a)$:

\begin{figure}[!ht]
\centering
\resizebox{0.5\textwidth}{!}{%
\begin{circuitikz}
\tikzstyle{every node}=[font=\LARGE]
\node [font=\LARGE] at (3,16.25) {$u_1$};
\node [font=\LARGE] at (6.0,16.25) {$...$};
\node [font=\LARGE] at (8.75,16.25) {$u_{|S|}$};
\draw [short] (3,15.75) -- (1.25,13.5);
\draw [short] (3,15.75) -- (5,13.5);
\draw [short] (3,15.75) -- (3,13.5);
\node [font=\LARGE] at (1,13) {$u_{1,1}$};
\node [font=\LARGE] at (5,13) {$u_{1,|S|}$};
\node [font=\LARGE] at (3,13) {$u_{1,2}$};
\node [font=\LARGE] at (4.0,13) {$...$};
\node [font=\LARGE] at (9.7,13) {$...$};

\draw [short] (8.75,15.75) -- (7,13.5);
\draw [short] (8.75,15.75) -- (10.75,13.5);
\draw [short] (8.75,15.75) -- (8.75,13.5);
\node [font=\LARGE] at (6.75,13) {$u_{|S|,1}$};
\node [font=\LARGE] at (10.75,13) {$u_{|S|,|S|}$};
\node [font=\LARGE] at (8.75,13) {$u_{|S|,2}$};
\end{circuitikz}
}%
\end{figure}

where, for $1 \leq i \leq |\mathcal{S}|, 1 \leq n \leq |\mathcal{S}|$, $u_{i, n}$ leads to the same next state for $(s_t, a_t)$ as $u_i$, and produces the transition $s, a \rightarrow s_n$.\\

In the same way, we can split each $\theta^1_{u_1}, ..., \theta^1_{u_{|\mathcal{S}|}}$ across these new structural equation mechanisms to obtain $\theta^2$:

\begin{figure}[!ht]
\centering
\resizebox{0.5\textwidth}{!}{%
\begin{circuitikz}
\tikzstyle{every node}=[font=\LARGE]
\node [font=\LARGE] at (3,16.25) {$\theta^1_{u_1}$};
\node [font=\LARGE] at (6.0,16.25) {$...$};
\node [font=\LARGE] at (8.75,16.25) {$\theta^1_{u_{|S|}}$};
\draw [short] (3,15.75) -- (1.25,13.5);
\draw [short] (3,15.75) -- (5,13.5);
\draw [short] (3,15.75) -- (3,13.5);
\node [font=\LARGE] at (1,13) {$\theta^2_{u_{1,1}}$};
\node [font=\LARGE] at (5,13) {$\theta^2_{u_{1,|S|}}$};
\node [font=\LARGE] at (3,13) {$\theta^2_{u_{1,2}}$};
\node [font=\LARGE] at (3.9,13) {$...$};
\node [font=\LARGE] at (9.6,13) {$...$};
\draw [short] (8.75,15.75) -- (7,13.5);
\draw [short] (8.75,15.75) -- (10.75,13.5);
\draw [short] (8.75,15.75) -- (8.75,13.5);
\node [font=\LARGE] at (6.75,13) {$\theta^2_{u_{|S|,1}}$};
\node [font=\LARGE] at (10.75,13) {$\theta^2_{u_{|S|,|S|}}$};
\node [font=\LARGE] at (8.75,13) {$\theta^2_{u_{|S|,2}}$};
\end{circuitikz}
}%
\end{figure}
By splitting $\theta^1$ in this way, we guarantee that the total probability across each set of mechanisms (where each set produces a different transition from $(s_t, a_t)$) remains the same, i.e., \[\forall i \in \{1, .., |\mathcal{S}|\}, \sum_{n=1}^{|\mathcal{S}|}\theta^2_{u_{i,n}} = \theta_{u_i}^1\]

This means that the counterfactual probabilities of all transitions from $(s_t, a_t)$ will be exactly the same when using $\theta^2$ as with $\theta^1$. Because we assume all the constraints were satisfied when using $\theta^1$, this guarantees that all the constraints of the optimisation problem will continue to hold for $(s_t, a_t)$ with $\theta^2$.\\

Now, we only need to ensure we assign $\theta^2$ such that all the constraints hold for the transitions from $(s, a)$. Firstly, we need $\sum_{i=1}^{|\mathcal{S}|}{\theta^2_{u_{i, n}}} = P(s_n \mid s, a)$ for every possible next state $s_n \in \mathcal{S}$, to satisfy the interventional constraint \eqref{proofeq:interventional constraint}. For each transition $s, a \rightarrow s_n$, there is a $\theta^2_{u_{i, n}}$ for every $u_i$, and for every $\theta^2_{u_{i, n}}$, $0 \leq \theta^2_{u_{i, n}} \leq \theta^1_{u_n}$. Therefore, $0 \leq \sum_{i=1}^{|\mathcal{S}|}{\theta^2_{u_{i, n}}} \leq 1, \forall s_n \in \mathcal{S}$. We also know that:

\begin{align*}
\sum_{n=1}^{|\mathcal{S}|}\sum_{i=1}^{|\mathcal{S}|}\theta^2_{u_i,n} &= \sum_{i=1}^{|\mathcal{S}|}\theta^1_{u_i} \text{ (because we split the probabilities of each $\theta^1_{u_i}$)}\\
&= \sum_{i=1}^{|\mathcal{S}|}P(s_i|s_t, a_t) \\&= 1
\end{align*}

and 

\[
\sum_{n=1}^{|\mathcal{S}|}\sum_{i=1}^{|\mathcal{S}|}{\theta^2_{u_{i, n}}} = 
\sum_{i=1}^{|\mathcal{S}|}P(s_i|s,a) = 1
\]

Therefore, no matter the assignment of $\theta^1$, we can satisfy all the constraints \eqref{proofeq:interventional constraint}-\eqref{proofeq:valid prob2} for all transitions from $(s, a)$. In Section \ref{sec: theta for sampled distribution}, we prove that, for any $(s, a)$, we can assign $\theta^2$ such that it produces the entailed counterfactual probability distribution.

\paragraph{Inductive Case} Let us assume that we now have $k$ state-action pair values from the MDP $\mathcal{M}$ in our causal model, and we have found a valid assignment $\theta^k$ that satisfies all the constraints for $k$ state-action pairs from the MDP $\mathcal{M}$, including $(s_t, a_t)$. Because there are $k$ state-action pairs, there are $|\mathcal{S}|^{k}$ possible unique structural equation mechanisms for these state-action pairs \footnote{Note we have changed the indexing of each $\theta^k_{i, n}$ to $\theta^k_{j}$, where $j = (i-1)\cdot|\mathcal{S}| + n$}, with probabilities as follows:

\begin{table}[h]
\centering
\begin{tabular}{lllllll}
$\theta^k_{u_1}$ & ... & $\theta^k_{u_{|\mathcal{S}|}}$ & ... & $\theta^k_{u_{|\mathcal{S}|^2}}$ & ... & $\theta^k_{u_{|\mathcal{S}|^{k}}}$ \\
\end{tabular}
\end{table}

Now, we wish to add the $k+1^{th}$ state-action pair from the MDP to the causal model, resulting in $|\mathcal{S}|^{k+1}$ possible structural equation mechanisms. Let the $k+1^{th}$ state-action pair be $(s, a)$ arbitrarily. We can view this as separating the existing structural equation mechanisms each into $|\mathcal{S}|$ new structural equation mechanisms, one for every possible transition from the $k+1^{th}$ state-action pair:

\begin{figure}[!ht]
\centering
\resizebox{1\textwidth}{!}{%
\begin{circuitikz}
\tikzstyle{every node}=[font=\LARGE]
\node [font=\LARGE] at (3.0,16.25) {$u_1$};
\node [font=\LARGE] at (6.0,16.25) {$...$};
\node [font=\LARGE] at (8.75,16.25) {$u_{|S|}$};
\node [font=\LARGE] at (12,16.25) {$...$};
\node [font=\LARGE] at (14.75,16.25) {$u_{|S|^2}$};
\node [font=\LARGE] at (20.75,16.25) {$u_{|S|^k}$};
\node [font=\LARGE] at (18,16.25) {$...$};
\draw [short] (3,15.75) -- (1.25,13.5);
\draw [short] (3,15.75) -- (5,13.5);
\draw [short] (3,15.75) -- (3,13.5);
\node [font=\LARGE] at (1,13) {$u_{1,1}$};
\node [font=\LARGE] at (5,13) {$u_{1,|S|}$};
\node [font=\LARGE] at (3,13) {$u_{1,2}$};
\node [font=\LARGE] at (3.9,13) {$...$};
\draw [short] (8.75,15.75) -- (7,13.5);
\draw [short] (8.75,15.75) -- (10.75,13.5);
\draw [short] (8.75,15.75) -- (8.75,13.5);
\node [font=\LARGE] at (6.75,13) {$u_{|S|,1}$};
\node [font=\LARGE] at (10.75,13) {$u_{|S|,|S|}$};
\node [font=\LARGE] at (8.75,13) {$u_{|S|,2}$};
\node [font=\LARGE] at (9.7,13) {$...$};
\draw [short] (14.75,15.75) -- (13,13.5);
\draw [short] (14.75,15.75) -- (16.75,13.5);
\draw [short] (14.75,15.75) -- (14.75,13.5);
\node [font=\LARGE] at (12.75,13) {$u_{|S|^2,1}$};
\node [font=\LARGE] at (17,13) {$u_{|S|^2,|S|}$};
\node [font=\LARGE] at (14.75,13) {$u_{|S|^2,2}$};
\node [font=\LARGE] at (15.8,13) {$...$};
\draw [short] (20.75,15.75) -- (19,13.5);
\draw [short] (20.75,15.75) -- (22.75,13.5);
\draw [short] (20.75,15.75) -- (20.75,13.5);
\node [font=\LARGE] at (18.75,13) {$u_{|S|^{k},1}$};
\node [font=\LARGE] at (23,13) {$u_{|S|^{k},|S|}$};
\node [font=\LARGE] at (20.75,13) {$u_{|S|^{k}, 2}$};
\node [font=\LARGE] at (21.8,13) {$...$};
\end{circuitikz}
}%
\end{figure}

where each $u_{i, n}$ produces the same next states for all of the first $k$ state-action pairs as $u_i$, and produces the transition $s, a \rightarrow s_n$.

In the same way, we can split each $\theta^k_{u_1}, ..., \theta^k_{u_{|\mathcal{S}|}}, ...$ across these new structural equation mechanisms to find $\theta^{k+1}$.

\pagebreak
\begin{figure}[!ht]
\centering
\resizebox{1\textwidth}{!}{%
\begin{circuitikz}
\tikzstyle{every node}=[font=\LARGE]
\node [font=\LARGE] at (3,16.25) {$\theta^k_{u_1}$};
\node [font=\LARGE] at (6.0,16.25) {$...$};
\node [font=\LARGE] at (8.75,16.25) {$\theta^k_{u_{|S|}}$};
\node [font=\LARGE] at (11.8,16.25) {$...$};
\node [font=\LARGE] at (14.75,16.25) {$\theta^k_{u_{|S|^2}}$};
\node [font=\LARGE] at (20.75,16.25) {$\theta^k_{u_{|S|^k}}$};
\node [font=\LARGE] at (17.8,16.25) {$...$};
\draw [short] (3,15.75) -- (1.25,13.5);
\draw [short] (3,15.75) -- (5,13.5);
\draw [short] (3,15.75) -- (3,13.5);
\node [font=\LARGE] at (1,13) {$\theta^{k+1}_{u_{1,1}}$};
\node [font=\LARGE] at (5,13) {$\theta^{k+1}_{u_{1,|S|}}$};
\node [font=\LARGE] at (3,13) {$\theta^{k+1}_{u_{1,2}}$};
\node [font=\LARGE] at (3.9,13) {$...$};
\draw [short] (8.75,15.75) -- (7,13.5);
\draw [short] (8.75,15.75) -- (10.75,13.5);
\draw [short] (8.75,15.75) -- (8.75,13.5);
\node [font=\LARGE] at (6.75,13) {$\theta^{k+1}_{u_{|S|,1}}$};
\node [font=\LARGE] at (10.9,13) {$\theta^{k+1}_{u_{|S|,|S|}}$};
\node [font=\LARGE] at (8.75,13) {$\theta^{k+1}_{u_{|S|,2}}$};
\node [font=\LARGE] at (9.7,13) {$...$};
\draw [short] (14.75,15.75) -- (13,13.5);
\draw [short] (14.75,15.75) -- (16.75,13.5);
\draw [short] (14.75,15.75) -- (14.75,13.5);
\node [font=\LARGE] at (12.75,13) {$\theta^{k+1}_{u_{|S|^2,1}}$};
\node [font=\LARGE] at (17,13) {$\theta^{k+1}_{u_{|S|^2,|S|}}$};
\node [font=\LARGE] at (14.75,13) {$\theta^{k+1}_{u_{|S|^2,2}}$};
\node [font=\LARGE] at (15.75,13) {$...$};
\draw [short] (20.75,15.75) -- (19,13.5);
\draw [short] (20.75,15.75) -- (22.75,13.5);
\draw [short] (20.75,15.75) -- (20.75,13.5);
\node [font=\LARGE] at (18.75,13) {$\theta^{k+1}_{u_{|S|^{k},1}}$};
\node [font=\LARGE] at (23,13) {$\theta^{k+1}_{u_{|S|^{k},|S|}}$};
\node [font=\LARGE] at (20.75,13) {$\theta^{k+1}_{u_{|S|^{k}, 2}}$};
\node [font=\LARGE] at (21.75,13) {$...$};
\end{circuitikz}
}%
\end{figure}

By splitting $\theta^k$ in this way, we have $\forall i \in \{1, .., |\mathcal{S}|\}, \sum_{n=1}^{|\mathcal{S}|}\theta^{k+1}_{u_{i,n}} = \theta_{u_i}^k$. Each $\theta^{k+1}_{u_{i, n}}$ is only different to $\theta^k_{u_i}$ in the transition from $(s, a)$. Therefore, this guarantees that for every state-action pair $(s', a') \neq (s, a)$ in the first $k$ pairs added to the causal model, the total probability across each set of mechanisms (where each set produces a different transition from $(s', a')$) remains the same in $\theta^{k+1}$ as in $\theta^k$. As a result, the counterfactual probabilities of all transitions from each $(s', a')$ will be exactly the same when using $\theta^{k+1}$ as with $\theta^k$. Because we assume all the constraints were satisfied when using $\theta^k$, this guarantees that all the constraints of the optimisation problem will continue to hold for $(s', a')$ with $\theta^{k+1}$.\\

Now, we only need to make sure that we can assign $\theta^{k+1}$ such that the constraints also hold for all the transitions from $(s, a)$. We need $\sum_{i=1}^{|\mathcal{S}|^k}{\theta^{k+1}_{u_{i, n}}} = P(s_n \mid s, a)$ for every possible next state $s_n \in \mathcal{S}$, to satisfy the interventional probability constraints \eqref{proofeq:interventional constraint}. For each transition $s, a \rightarrow s_n$, there is a $\theta^{k+1}_{u_{i, n}}$ for every $u_i$, and for every $\theta^{k+1}_{u_{i, n}}$, $0 \leq \theta^{k+1}_{u_{i, n}} \leq \theta^k_{u_n}$, therefore $0 \leq \sum_{i=1}^{|\mathcal{S}|^k}{\theta^{k+1}_{u_{i, n}}} \leq 1, \forall s_n \in \mathcal{S}$. We also know that:

\begin{align*}
\sum_{n=1}^{|\mathcal{S}|}\sum_{i=1}^{|\mathcal{S}|^k}\theta^{k+1}_{u_i,n} &= \sum_{i=1}^{|\mathcal{S}|^k}\theta^k_{u_i} \text{ (because we split the probabilities of each $\theta^k_{u_i}$)}\\&= 1
\end{align*}

and 

\[
\sum_{n=1}^{|\mathcal{S}|}\sum_{i=1}^{|\mathcal{S}|^k}{\theta^2_{u_{i, n}}} = 
\sum_{n=1}^{|\mathcal{S}|}P(s_n|s,a) = 1
\]

Therefore, no matter the assignment of $\theta^k$, we can satisfy the interventional probability constraints \eqref{proofeq:interventional constraint} for all transitions from $(s, a)$. In Section \ref{sec: theta for sampled distribution}, we prove that, for any $(s, a)$, we can assign $\theta^{k+1}$ such that it produces the entailed counterfactual probability distribution (which we prove always satisfies the constraints of the optimisation problem). Therefore, we know we can assign $\theta^{k+1}$ such that it produces the counterfactual probability distributions for all $k+1$ state-action pairs in the causal model.

\paragraph{Conclusion} By induction, we have proven that as long as we can find a $\theta$ that produces the entailed counterfactual probability distribution for each state-action pair separately, we can combine these into a single valid $\theta$ that produces the entailed CFMDP $\tilde{\mathcal{M}_t}$.
\end{proof}
