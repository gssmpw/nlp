\section{Background}
In this section, we provide background on counterfactual inference and an overview of related work on counterfactual inference in MDPs and partial counterfactual inference.

\paragraph{Markov Decision Processes}
\jl{MDPs are a class of stochastic models for representing sequential decision-making processes.} In an MDP $\mathcal{M}$, at each step $t$, an agent in state $s_t$ performs some action $a_t$ determined by a policy $\pi$, \jl{ending up in state $s_{t+1} \sim P(\cdot \mid s_t, a_t)$ and receiving reward $R(s_t,a_t)$.}
%
Formally, an MDP is a tuple $\mathcal{M}=(\mathcal{S},\mathcal{A},P,P_I,R)$ where $\mathcal{S}$ is the discrete \emph{state space}, $\mathcal{A}$ is the set of \emph{actions}, $P: (\mathcal{S} \times \mathcal{A} \times \mathcal{S}) \rightarrow [0,1]$ is the \emph{transition probability} function, $P_I:\mathcal{S} \rightarrow [0,1]$ is the \textit{initial state distribution}, and $R: (\mathcal{S} \times \mathcal{A}) \rightarrow \mathbb{R}$ is the \emph{reward function}. 
A (deterministic) \emph{policy} $\pi$ for $\mathcal{M}$ is a function $\pi: \mathcal{S} \rightarrow \mathcal{A}$. \jl{A path $\tau$ of $\mathcal{M}$ under policy $\pi$
%
is a sequence $\tau=(s_0, a_0), \ldots,(s_{T-1},a_{T-1})$ where $T=|\tau|$ is the path length, $P_I(s_0)>0$, $a_t=\pi(s_t)$ for all $t=0,\ldots,T-1$, and $P(s_{t+1}\mid s_t, a_t)>0$ for all $t=0,\ldots,T-2$.}

\paragraph{Counterfactual Inference} Structural causal models (SCMs) \citep{pearl_2009,halpern2005causes} provide a mathematical framework for causal inference. Formally, a SCM is a tuple $\mathcal{C}=(\mathbf{U},\mathbf{V},\mathcal{F},P(\mathbf{U}))$, where $\mathbf{V}$ is the set of endogenous (observed) variables, $\mathbf{U}$ is a set of exogenous (unobserved) variables, $P(\mathbf{U})$ is a joint distribution over the possible values of each $U \in \mathbf{U}$, and $\mathcal{F}$ is a set of structural equations where each $f_V \in \mathcal{F}$ determines the value of endogenous variable $V$, given a fixed realisation of $U_V\in \mathbf{U}$ and set of direct causes (parents) $\mathbf{PA}_V \subseteq \mathbf{V}$\footnote{In general, we denote variables with capital letters and specific values of the variables in lowercase.}.

Using SCMs, we can apply an \textit{intervention} $X \leftarrow x$ on an endogenous variable $X$, and measure the effect on the value of some other endogenous variable $Y$. We can also perform \textit{counterfactual inference} to estimate what causal effect applying an intervention in the past would have had on an observation. This involves inferring the value of the exogenous variables present in the observation $\mathbf{v}$, i.e., $P(\mathbf{U} \mid \mathbf{v})$, then evaluating the structural equations, replacing $P(\mathbf{U})$ with the inferred $P(\mathbf{U} \mid \mathbf{v})$, and applying the intervention. \jl{Counterfactual inference can be challenging because many SCMs may fit the observed and interventional distributions but yield different counterfactual probabilities \citep{pmlr-v162-zhang22ab}. As a result, many approaches require additional assumptions about the causal model for real-world application.}

\paragraph{Counterfactual Inference in MDPs}
The Gumbel-max SCM was developed by \citet{oberst2019counterfactual} to perform counterfactual inference in MDPs and has been widely used in existing work \citep{lorberbom2021learning, benz2022counterfactual, noorbakhsh2022counterfactual, killian2022counterfactually, zhu2020counterfactual, kazemi2022causal, kazemi2024counterfactual} due to its desirable property of counterfactual stability (see Definition \ref{def: counterfactual stability}). The Gumbel-max SCM for an MDP is expressed as:
\begin{equation}\label{eq:gumbel-max-scm}  \begin{aligned}
    S_{t+1} &= f(S_{t},A_{t}, U_{t}=(G_{s,t})_{s\in S}) 
    \\ &= \argmax_{s\in S}\left\{\log\left(P_{\mathcal{M}}(s \mid S_t, A_t)\right) +G_{s,t}\right\}
\end{aligned}  
\end{equation}
where $P_{\mathcal{M}}$ is the MDP's transition probabilities, $S_t$, $A_t$ and $S_{t+1}$ are endogenous variables representing the state, action and next state, and the values of the exogenous variable $U_{t}=(G_{s,t})_{s\in S}$ are sampled from the standard Gumbel distribution\footnote{This formulation relies on the Gumbel-max trick \citep{maddison2014sampling} by which sampling from a categorical distribution with $n$ categories is equivalent to sampling $n$ values $g_0,\dots,g_n$ from the standard Gumbel distribution and evaluating $\argmax_{j}\left\{\log\left(P(Y=j)\right)+ g_j\right\}$ over all the categories, where $Y$ is the output.}. In this model, the Gumbel values $G_{s,t}$ represent the exogenous noise. We can perform (approximate) posterior inference of $P((G_{s,t})_{s\in S} \mid s_t, a_t, s_{t+1})$ through rejection sampling \citep{oberst2019counterfactual} or top-down Gumbel sampling \citep{maddison2014sampling}.
%

\jl{We can define a so-called \textit{counterfactual MDP} $\mathcal{M}_{\tau}$ by solving the SCM \eqref{eq:gumbel-max-scm} for each transition along an observed path $\tau$ in an MDP $\mathcal{M}$. The counterfactual probability for each transition is defined, for $t=0, ..., T-1$, as:}
\begin{equation}
\label{eq:cf_mdp_probs}
\begin{aligned}
&P_{\mathcal{M},t,\tau}(s' \mid s, a) \\ &= P(s' = \argmax_{q \in \mathcal{S}} \left\{\log\left(P_{\mathcal{M}}(q \mid s, a)\right) + 
     G'_{q,t}\right\}) \\
     &\approx \dfrac{1}{N} \sum_{j=0}^{N} \mathbbm{1}\left(s' = \argmax_{q \in \mathcal{S}} \left\{\log\left(P_{\mathcal{M}}(q \mid s, a)\right) + 
     G_{q,t}^{\prime(j)}\right\}\right) \\
\end{aligned}
\end{equation}
\jl{where we sample $N$ values of $G_{q,t}^{\prime(j)}$ from the true posterior distribution $G'_{q,t}$ through rejection sampling or top-down Gumbel sampling.} The indicator function $\mathbbm{1}(\mathbbm{X})$ takes the value $1$ if the condition $\mathbbm{X}$ is satisfied and $0$ otherwise.

\jl{\citet{tsirtsis2021counterfactual} used this counterfactual MDP to derive alternative sequences of actions (i.e., counterfactual paths) that would have produced a higher reward than an observed path. Counterfactual paths can be seen as \textit{counterfactual explanations} for how the observed policy could have been improved. However, since the Gumbel-max SCM is just one of many possible SCMs compatible with any given MDP, this counterfactual MDP may be inaccurate, leading to unreliable counterfactual explanations. To address this limitation, we apply partial counterfactual inference methods to solve this problem of counterfactual unidentifiability.}

\paragraph{Partial Counterfactual Inference}
\jl{Partial counterfactual inference methods bound counterfactual outcomes by considering all SCMs compatible with observational and interventional data. Key works are summarised in Table \ref{table: partial counterfactual inference}. Notably, \cite{pmlr-v162-zhang22ab} developed the first approach capable of handling all causal graph structures. Their \textit{canonical SCM} approach derives a linear or polynomial optimisation problem (depending on the causal graph structure) producing exact bounds. However, this optimisation is very inefficient, as the number of linear constraints grows exponentially with the number and cardinality of the endogenous variables. Research has thus shifted from exact methods to developing approximation methods. One notable exception is recent work by \citet{li2024probabilities}, who identified analytical/closed-form counterfactual probability bounds for systems with categorical variables.} We discuss this work further in Section \ref{sec: equivalence with Li}, as their bounds are equivalent to ours in specific cases.

\begin{table}[]
\centering
\caption{Related works on partial counterfactual inference}
\resizebox{1.0\columnwidth}{!}{
\small
\begin{tabular}{|p{2.0cm}|p{2.0cm}|p{2.5cm}|} %
\hline
\diagbox[innerwidth=2.0cm]{\textbf{Bounds}}{\textbf{Outcomes}}          & \textbf{Binary} & \textbf{Categorical} \\ \hline
\textbf{Exact}       & \citet{balke1994counterfactual, kang2006inequality, cai2008bounds}       & \citet{pmlr-v162-zhang22ab, li2024probabilities}, This paper            \\ \hline
\textbf{Approximate} & \citet{robins1989analysis, manski1990nonparametric} & \citet{zaffalon2020, zaffalon2021, zaffalon2022bounding, zaffalon2023approximating, zaffalon2024, duarte2023, pmlr-v162-zhang22ab}\tablefootnote{\citet{pmlr-v162-zhang22ab} provide both an exact and approximation method in their paper.}            \\ \hline
\end{tabular}
}
\label{table: partial counterfactual inference}
\end{table}
