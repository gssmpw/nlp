\section{Related Work}
\paragraph{Causal MBRL}
MBRL involves training a dynamics model by maximizing the likelihood of collected transitions, known as the world model____.
Due to the exclusion of irrelevant factors from the environment through state abstraction, the application of causal inference in MBRL can effectively improve sample efficiency and generalization____. 
Wang____ proposes a constraint-based causal dynamics learning that explicitly learns causal dependencies by action-sufficient state representations. 
GRADER____ executes variational inference by regarding the causal graph as a latent variable. CDL____ is a causal dynamics learning method based on CIT. CDL employs conditional mutual information to compute the causal relationships between different dimensions of states and actions. For additional related work, please refer to Appendix~\ref{Additional Related Works}.
\vspace{-3mm}
\paragraph{Empowerment in RL} 
Empowerment is an intrinsic motivation to improve the controllability over the environment____. This concept is from the information-theoretic framework, wherein actions and future states are viewed as channels for information transmission. In RL, empowerment is applied to uncover more controllable associations between states and actions or skills____. By quantifying the influence of different behaviors on state transitions, empowerment encourages the agent to explore further to enhance its controllability over the system____. Maximizing empowerment $\max_{\pi} I$ can be used as the learning objective, empowering agents to demonstrate intelligent behavior without requiring predefined external goals. 
% Hence, grounded in the concept of empowerment, we propose a learning framework towards empowerment gain through causal structure learning to improve the controllability for environment. 
\vspace{-3mm}