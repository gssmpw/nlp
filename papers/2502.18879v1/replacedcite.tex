\section{Related Work}
\emph{Safe reinforcement learning} is widely studied____, including several approaches without formal verification. Several policy-search algorithms have been proposed that take into account safety constraints specified separately from the reward signal, but do not offer any guarantee that those constraints will not be violated during training or deployment____. In contrast, this work is part of a family of approaches that are often called \emph{sandboxing} or \emph{shielding-based}____, and where the intended actions of an RL agent are monitored at runtime and overridden by appropriate fallbacks whenever they cannot be proved safe with respect to a model of the environment.

Virtually all existing approaches to shielding aim for full-automation of the shield computation, imposing hard tradeoffs in terms of safety, adaptivity, precision, expressivity and scalability. In circumstances where they run fast enough to be used online, methods based on reachability analysis____ are naturally amenable to a form of adaptivity. However, they only offer bounded-horizon guarantees and precision comes at the expense of runtime efficiency. Methods based on LTL model-checking____ or finite-MDP solving____ can handle infinite time-horizons but require discretization of hybrid dynamics, often at a significant cost in terms of precision and safety. They also tend to scale poorly with increasing dimensionality. Methods based on Hamilton-Jacobi solving____ face similar challenges. In contrast, our proposed framework allows leveraging human ingenuity by extracting shields from nondeterministic, symbolic controllers that are proved safe using interactive theorem proving. It builds on the \emph{Justified Speculative Control} (JSC) framework____, which it generalizes in a crucial way to support adaptivity.

Our framework offers a uniquely expressive language for modelling environment uncertainty via arbitrarily constrained function symbols. For example, reachability analysis typically uses {bounded disturbance terms}____ to model environment uncertainty, which is sufficient to model the example from~Figure~\ref{fig:overview-train-global} but not those from Figures~\ref{fig:overview-train-local}~and~\ref{fig:acas}. To the best of our knowledge, these last two examples cannot be accommodated by any pre-existing approach. Another standard way of representing model uncertainty for the purpose of adaptive shielding is to use \emph{Gaussian processes} to model an unknown, state-dependent term added to the system's dynamics____. This approach offers a different form of modelling flexibility, where assumptions about functional unknowns are implicitly encoded into prior kernels rather than hard logical constraints. However, this also makes the resulting safety guarantees harder to interpret and fundamentally dependent on assumptions that are nearly impossible to validate experimentally.

Another approach has been proposed to extend the JSC framework with a form of adaptivity, in which an agent starts with a finite set of plausible models and then progressively discards those that are found inconsistent with observations____. Our framework handles a more general form of parametric model uncertainty and additionally supports noisy observations and statistical reasoning. Finally, our idea of having experts define nondeterministic inference strategies that are sound by construction and refined via learning -- thereby making shielded agents in charge of their own safety budget -- has, to the best of our knowledge, no equivalent in the literature.