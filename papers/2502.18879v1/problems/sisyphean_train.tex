\begin{figure}[p]
\begin{center}
\begin{sllisting}
\Init \Equiv A > 0 \, \land \, B > 0 \, \land \, T > 0 \, \land \, k > 0 \, \land \, \eta_r > 0 \, \land \, F < B \, \land \, A + F > 0 \, \land \, \\
    \slind[2] \, \lforall{x_1} \lforall{x_2} |f(x_1) - f(x_2)| \le k |x_1 - x_2| \, \land \, \lforall{x} (f(x) \le F \, \land \, f(x) \ge -A)\\
\Ctrl \Equiv y \dlassign \min(y, \TrainFxMax) \seq (?(x + v T + (A + F) \cdot T ^ 2 / 2 + (v + (A + F) \cdot T) ^ 2 / (2 (B - \min(F, y + \\
    \slind[2] \, \, k (v T + (A + F) \cdot T ^ 2 / 2) + k (v + (A + F) \cdot T) ^ 2 / (2 (B - F))))) \le 0) \seq a \dlassign A \cup a \dlassign -B)\\
\Plant \Equiv t \dlassign 0 \seq \dlode{x' = v \commasep v' = a + f(x) \commasep y' = k v \commasep t' = 1 \dldom t \le T \, \land \, v \ge 0}\\
\Inv \Equiv v \ge 0 \, \land \, x + v ^ 2 / (2 (B - \min(F, y + k v ^ 2 / (2 (B - F))))) \le 0 \, \land \, y \ge f(x)\\
\Safe \Equiv x \le 0\\
\pbmonotonicity\\
  \hspace{0.5em}\text{--}\hspace{0.4em}\TrainFxMax_1 \le \TrainFxMax_2 \limply f(x) \le \TrainFxMax_1 \limply f(x) \le \TrainFxMax_2\\
\pbmodel\\
  \hspace{0.5em}\text{--}\hspace{0.4em}\color{red}{\Init \, \land \, f(x) \le \TrainFxMax \, \land \, \Inv \limply \lboxI{\Ctrl \seq \Plant}\Inv}\\
\pbsafe\\
  \hspace{0.5em}\text{--}\hspace{0.4em}\Init \, \land \, \Inv \limply \Safe\\
\pbtotality\\
  \hspace{0.5em}\text{--}\hspace{0.4em}\Init \, \land \, f(x) \le \TrainFxMax \, \land \, \Inv \limply \ldiamond{\Ctrl}{\True}\\
\pbinference\\
  \hspace{0.5em}\text{--}\hspace{0.4em}\Init \, \land \, \Inv \, \land \, \TrainFxMax = F \limply f(x) \le \TrainFxMax\\
  \hspace{0.5em}\text{--}\hspace{0.4em}\Init \, \land \, \Inv \, \land \, \TrainFxMax = \omega_i + k |x - x_i| + \eta_i \, \land \, \omega_i = f(x_i) - \eta_i \limply f(x) \le \TrainFxMax\\
  \hspace{0.5em}\text{--}\hspace{0.4em}\Init \, \land \, \Inv \, \land \, \TrainFxMax = \TrainFxMax_i + k |x - x_i| \, \land \, f(x_i) \le \TrainFxMax_i \limply f(x) \le \TrainFxMax\\
\end{sllisting}
\end{center}
\caption{Proof obligations for the \emph{Sisyphean Train} case study. We write $\bdist{v}{b} \equiv v^2/2b$. The one obligation requiring nontrival guidance (proving the model's invariant) is shown in red. An informal proof of this formula is available in Appendix~\ref{ap:train-invariant-proof}.}
\label{fig:targets-sisyphean-train}
\Description[]{}
\end{figure}
