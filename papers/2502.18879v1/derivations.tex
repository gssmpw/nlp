\section{Derivations of Case Studies}
\label{ap:derivations}
In this section, we want to show that the design choices of shield specifications can be easily derived from first principles. We divide the shield specifications into four parts: assumptions ($\slconstant$, $\slunknown$, $\slassume$, $\slbound$), invariant ($\slinvariant$), controller ($\slcontroller$), inference ($\slnoise$, $\slobserve$, $\slinfer$). For each case study, we will discuss all these parts (we may skip assumptions if assumptions are simple).

\subsection{Sisyphean Train}
The most non-trivial part of this use case is the derivations of the invariant and the controller, which we will explain in detail. After that, we will discuss the inference part.

Here we use the same notation as is shown in Figure~\ref{fig:overview-train-local}. Note that in the case study, we simplify the problem such that there are only two actions: accelerating with an acceleration $A$ or braking with an acceleration $-B$. So the controller becomes
\begin{equation*}
    \begin{aligned}
        & y \dlassign \min(y, \TrainFxMax) \seq \{?(x + vT + \frac{1}{2}(A + F)T^2 + \bdist{v+(A+F)T}{B - \min(F, \, y + k(vT + \frac{1}{2}(A + F)T^2) \ + \\
        & k \cdot \bdist{v + (a+F)T}{B-F})} \le e) \seq a := A \seq \, \cup \, a := -B \seq \}.
    \end{aligned}
\end{equation*}
Other parts of the shield specifications remain the same.

\subsubsection{Invariant}
\begin{figure}
    \includegraphics[width=0.95\textwidth]{img/bootstrap.png}
    \caption{A diagram about how we use bootstrap to improve the braking distance. Assuming that the set of $x$ during braking is $\mathcal{X}$, then the minimal acceleration during braking is $-B+\max_{x\in\mathcal{X}}f(x)$, and braking distance is $\bdist{v}{-B+\max_{x\in\mathcal{X}} f(x)}$.}
    \label{fig:example1-derivations-1}
    \Description[]{}
\end{figure}

The core problem of this use case is how to estimate the upper bound of the braking distance. Here we use the bootstrap method, as is shown in Figure~\ref{fig:example1-derivations-1}. Now we derive it step by step.

For a train with velocity $v$ and acceleration $-b$, the braking distance $\bdist{v}{b} \equiv v^2/2b$. If $b$ is not constant but we can have its lower bound $\underaccent{\bar}{b}$, then an upper bound of the braking distance is $\bdist{v}{\underaccent{\bar}{b}}$.

In our case, $b=B-f(x)$, where $f$ has a global upper bound $F$. So we can let $\underaccent{\bar}{b}=B-F$, leading to a simple upper bound of the braking distance: $\bdist{v}{B - F}$. Note that here $\underaccent{\bar}{b}$ is a global bound that holds for any position $x$, but actually, a local bound that holds during the whole braking is enough. With this idea, we can improve the bound by using the bootstrap method.

We already know that $f$ is $k$-Lipschitz, and an upper bound of the braking distance $\bdist{v}{B - F}$. Assuming that $y$ is an upper bound of $f(x)$ before braking, we can have a local upper bound of $f$ during the whole braking, which is $y + k \cdot \bdist{v}{B - F}$. Combining with the global upper bound $F$, we can let $\underaccent{\bar}{b}=B-\min(F, y + k \cdot \bdist{v}{B - F})$. This gives us a better upper bound of the braking distance, which is $\bdist{v}{B - \min(F, y + k \cdot \bdist{v}{B-F})}$. In the invariant, we require that the braking distance is large enough to stop, i.e., $x + \bdist{v}{B - \min(F, y + k \cdot \bdist{v}{B-F})} \le e$. We can do more bootstraps, but we choose to stop here because it is already good enough for our case study.

The derivations above solve the problem during one control cycle. However, position $x$ changes during the ODE, and there is no guarantee that $y \ge f(x)$ holds in the next control cycle. Note that in the ODE, the change rate of $x$ is $v$, so the change rate of $f(x)$ can be bounded by $kv$ because of the chain rule and the Lipschitz assumption. So we let $y$ increase with a rate of $kv$ in the ODE, which makes it still an upper bound of $f(x)$ in the next control cycle.

Combining all the information, our invariant has three parts: the train moving forward or stop ($v \ge 0$), $y$ being the upper bound of $f(x)$ ($y \ge f(x)$), and the braking distance is large enough to stop ($x + \bdist{v}{B - \min(F, y + k \cdot \bdist{v}{B-F})} \le e$). Also, in Appendix~\ref{ap:acas}, we provide an informal proof about why the invariant is preserved.


\subsubsection{Controller}
For the controller, we firstly update $y$ using the bound parameter ($y \dlassign \min(y, \TrainFxMax)$). We can choose to accelerate when it is safe ($?(x + vT + \frac{1}{2}(A + F)T^2 + \bdist{v+(A+F)T}{B - \min(F, \, y + k(vT + \frac{1}{2}(A + F)T^2) \ + k \cdot \bdist{v + (a+F)T}{B-F})} \le e) \seq a := A$), or we can always brake ($a \dlassign -B$). The core idea for the acceleration case is that with the current acceleration, in the worst case (i.e., the acceleration we have in this control cycle is constantly $A+F$) we still have enough braking distance in the next control cycle.

\subsubsection{Inference}
For the inference part, what we want is the upper bound of $f(x)$: $\TrainFxMax$. We can use the bounded assumption of $f$ ($\TrainFxMax \slassign F$), previous upper bounds with deterioration from the $k$-Lipschitz condition ($\TrainFxMax \slassign \slbest i: \TrainFxMax_i + k|x - x_i|$), and available observations at previous positions with deterioration from the $k$-Lipschitz condition ($\TrainFxMax \slassign \slaggregate i: \omega_i + k|x - x_i| \sland \eta_i$). Our framework will pick the best bound from them.

\subsection{Versatile Train}
The only difference between its shield specifications and the shield specifications of the Sisyphean train use case is the distribution of the noise. For brevity, we skip the derivations for this use case.

\subsection{Crossing the River}
This use case is relatively simple to understand, and we will explain its invariant, controller, and inference parts briefly. Here we use the same notation as is shown in Figure~\ref{fig:river-crossing}.

\subsubsection{Invariant}
The only requirement is that the robot's trajectory has no intersection with the river ($\{(0, y) | y < \rcyb - W \lor y > \rcyb + W\}$). For the unknown value $y_b$, we estimate both its upper bound $\rcybmax$ and its lower bound $\rcybmin$, so the river is a subset of $\{(0, y) | y <\rcybmax -W \lor y > \rcybmin + W\}$ with high probability. Based on these bounds, the invariant is $x = 0 \limply (\rcybmax - W \le y \le \rcybmin + W)$.

\subsubsection{Controller}
For the controller, we choose the velocity ($\rcvx \slassign * \seq \rcvy \slassign *$), check whether the velocity is within its bound ($?(|\rcvx| \le V \land |\rcvy| \le V)$), and check the safety ($?(x \cdot (x + \rcvx T) > 0 \ \lor \  \rcvx = 0 \ \lor \  (\rcvx \ne 0 \land \rcybmax - W \,\le\, y + \rcvy \cdot (-x / \rcvx) \,\le\, \rcybmin + W))$). Note that at each time step, the robot's trajectory is a line segment because it directly chooses its velocity. Based on this information, the safety means whether the line segment $\{(x_t, y_t) | x_t = x + \rcvx t, y_t = y + \rcvy t, t\in [0, 1]\}$ intersects with our estimated region of the river $\{(0, y) | y <\rcybmax -W \lor y > \rcybmin + W\}$.

\subsubsection{Inference}
Regarding the inference part, we only need one aggregation sentence to get the bounds of $\rcyb$. Here the standard deviation of each observation is dependent on the value of $|x|$ (i.e., the closer the robot is to the river, the clearer the robot sees the bridge), which can also be handled by our framework.

\subsection{Revisiting ACAS X}
The most non-trivial part of this use case is the inference part. Here we will start with the assumptions, introduce the invariant and the controller, and finally discuss the inference part. We use the same notation as is shown in Figure~\ref{fig:acas}.

\subsubsection{Assumptions}
For assumptions, we firstly clarify some signs and ranges of values ($\catm \geq 0 ,\, T > 0 ,\, A > 0 ,\, \caAint > 0 ,\, R > 0 ,\, \caSigmaOne > 0 ,\, \caSigmaTwo > 0 ,\, 0 < p < 1$). For the intruder, we require its acceleration to be within $[-\caAint, \caAint]$ ($\lforall{t_1} \lforall{t_2} |\cahint(t_2) - \cahint(t_1) - \cavint(t_1) \cdot (t_2 - t_1)| \leq \caAint \cdot (t_2 - t_1) ^ 2 / 2, \lforall{t_1} \lforall{t_2} |\cavint(t_2) - \cavint(t_1)| \le \caAint |t_2 - t_1|$), maximum velocity and altitude range ($\lforall{t} (|\cavint(t)| \leq V \wedge |\cahint(t)| \leq H)$). We also restrict the sign of its acceleration if it is compliant, depending on its relative altitude in the beginning ($(c = 0 \lor c = 1) \land (c=1 \limply ((\cahint(0) > 0 \rightarrow \lforall{t} (t \leq \catm \rightarrow \cahint(\catm) \geq \cahint(t) + \cavint(t) \cdot (\catm - t))) \ \land \ (\cahint(0) < 0 \rightarrow \lforall{t} (t \leq \catm \rightarrow \cahint(\catm) \leq \cahint(t) + \cavint(t) \cdot (\catm - t)))))$).

\subsubsection{Invariant}
For the invariant, most importantly we want that two planes are far away enough at the meeting time (i.e., $t=\catm$). The altitude of the intruder at time $\catm$ is unknown, but we can estimate its upper bound $\cahmintmax$ and lower bound $\cahmintmin$. If at the meeting time, we can reach an altitude higher than the intruder with a margin $R$ or lower than the intruder with a margin $R$, we are safe. Note that at time $t$, the range of our possible altitude at meeting time is $[h + v \cdot (\catm - t) - A \cdot (\catm - t) ^ 2 / 2, h + v \cdot (\catm - t) + A \cdot (\catm - t) ^ 2 / 2]$, we can get the most important part of the invariant: $((h + v \cdot (\catm - t) + A \cdot (\catm - t) ^ 2 / 2 \,\geq\, \cahmintmax + R) \,\lor\, (h + v \cdot (\catm - t) - A \cdot (\catm - t) ^ 2 / 2 \,\leq\, \cahmintmin - R))$. Another part of the invariant is about the range of time ($t \geq 0 \land t \leq \catm$).

\subsubsection{Controller}
For the controller, we set the acceleration ($a \dlassign *$), check whether it is within its bound ($?(|a| \leq A)$), and check safety by directly checking the invariant at the next time step ($\cahnext \dlassign h + v \cdot \min(T, \catm - t) + a \cdot \min(T, \catm - t) ^ 2 / 2 \seq \cavnext \dlassign v + a \cdot \min(T, \catm - t) \seq \catleftnext \dlassign \max(\catm - t - T, 0) \seq ?((\cahnext + \cavnext \cdot \catleftnext + A \cdot \catleftnext ^ 2 / 2 \,\geq\, \cahmintmax + R)\,\lor\,(\cahnext + \cavnext \cdot \catleftnext - A \cdot \catleftnext ^ 2 / 2 \,\leq\,\cahmintmin - R))$).

\subsubsection{Inference}
The inference part is a little complicated. Although we only need to estimate the bounds of $\cahint(\catm)$, that value is not directly observable. Our observations are the noisy versions of $\cavint(t)$, $\cahint(t)$, and $c$. So we introuce intermediate bound variables $\cavintmax$, $\cavintmin$, $\cahintmax$, $\cahintmin$, $\cahzerointmax$, $\cahzerointmin$ and $\cacmin$ to build a bridge between observations and our goal.

For $\cavintmax$ and $\cavintmin$, we estimate them using the bounded assumption $\lforall{t} |\cavint(t)| \leq V$ and observations ($\cavintmax \slassign V \seq  \cavintmin \slassign -V \seq \cavintmax \slassign \slaggregate i: (\caOmegaOne)_i + \caAint \cdot |t - t_i| \sland (\caEtaOne)_i \seq \cavintmin \slassign \slaggregate i: (\caOmegaOne)_i - \caAint \cdot |t - t_i| \sland (\caEtaOne)_i$). Similarly, we can deal with $\cahintmax$ and $\cahintmin$. Based on the assumption about the range of acceleration, $\cahzerointmax$ and $\cahzerointmin$ can be derived from the bounds we have ($\cahzerointmin \slassign \cahintmax + \cavintmin \cdot (0 - t) + \caAint \cdot (0 - t) ^ 2 / 2 \seq \cahzerointmax \slassign \cahintmin + \cavintmax \cdot (0 - t) - \caAint \cdot (0 - t) ^ 2 / 2$). $\cacmin$ can be derived from observations ($\cacmin \slassign \slaggregate i: 0 \sland 1 - (\caEtaThree)_i  \slwhen (\caOmegaThree)_i = 1$).

With all the information above, we can derive $\cahmintmax$ and $\cahmintmin$. Based on the assumption about the range of acceleration, we can have a simple bound; if the intruder is compliant with an initial altitude lower than us ($\cahzerointmin > 0$, in this case it will not accelerate upward), we have a better bound for $\cahmintmax$; if the intruder is compliant with an initial altitude higher than us ($\cahzerointmin > 0$, in this case it will not accelerate downward), we have a better bound for $\cahmintmin$. These explain the last 4 inference assignments in Figure~\ref{fig:acas}.
