\section{Related Work}

\paragraph{Temporal Reasoning for LLMs.} 
Recent advancements in large language models (LLMs) have brought significant improvements in reasoning capabilities \cite{huang2023towards}, leading to growing interest in temporal reasoning \cite{chu-etal-2024-timebench,qiu2024large}. Existing approaches primarily address this challenge through time-aware language modeling. For example, \citeauthor{kanashiro-pereira-2022-attention, tan2023towards} propose fine-tuning strategies to enhance temporal reasoning, while \citeauthor{zhou2021temporal,yang-etal-2023-upon} introduce auxiliary objectives to incorporate external temporal knowledge. However, studies such as \citeauthor{chu-etal-2024-timebench,qiu2024large} show that state-of-the-art LLMs still exhibit suboptimal performance in temporal reasoning with prompting techniques. Our framework differs from these works by utilizing memory-augmented LLM agents, enhancing memorization through timeline summarization, and integrating neuro-symbolic reasoning as an intermediate step for answering temporal questions.

\paragraph{Multi-session Dialogues.} 
Several studies have focused on multi-session dialogues and developed benchmarks to address this task. \citeauthor{xu-etal-2022-beyond} introduced MSC, the first multi-session dataset incorporating time intervals between sessions. Similarly, \citeauthor{bae-etal-2022-keep} proposed a dynamic memory management method to maintain up-to-date user information and introduced a Korean multi-session dialogue dataset. \citeauthor{jang2023conversation} created the CONVERSATION CHRONICLES dataset, designed for long-term conversations that integrate time intervals and detailed speaker relationships. More recently, \citeauthor{maharana-etal-2024-evaluating} introduced LoCoMo, a dataset featuring long-term dialogues spanning up to 35 sessions. While our work is situated within this context, it specifically targets temporal reasoning, addressing the temporal characteristics of relative time and cross-session dependency, which have not been explicitly explored in prior research.