\section{Related Work}
\label{sec:related}
%Our work pertains to a broad range of research areas, mainly on planning from the multi-agent community and scheduling from the operations research community.

% 1. path finding
% MA-PF/PD
\textbf{Path Finding.}
The study of MAPF aims to develop centralized planning algorithms.
In spite of the computational complexity being NP-hard in general____,
researchers have developed practically fast planners that can even solve instances with hundreds of agents within seconds.
Exemplars can be found via reduction to logic programs____,
prioritized planning____,
conflict-based search____, 
depth-first search____, etc.
Most of them can be extended to the online version of the problem, i.e. MAPD,
where the goals assigned to agents are priorly unknown____.
%Usually, adaptations are then made via communication between robots and broadcasting by the central controller for the purpose of synchronization.
%A notable drawback is that both MAPF and MAPD assume a task assigner should be given.


% 2. task assignment
% TAPF, Anon-MAPF
% Multi-Goal MA-PF/PD
\textbf{Task Assignment.}
The earliest attempt is the formulation of \textit{Anonymous}-MAPF (AMAPF) that does not specify the exact goal that an agent must go to____.
%, but the total number of anonymous goals should be less than or equal to the number of agents
Compared with the labeled version, AMAPF can be solved in polynomial time,
via reduction to max-flow problems____, or target swaps____.
As a generalization,
TAPF explicitly associates each agent with a team____, or with a set of candidate goals____,
and eventually computes a set of collision-free paths as well as the corresponding assignment matrix.
Another analogous formulation is called \textit{Multi-Goal} (MG-)MAPF____ and its lifelong variant MG-MAPD____, which also associates each agent with a set of goals, but the visiting order is pre-specified.
%Imaginably, the further generalized model should consider that tasks arrive as a priorly unknown sequence and are assigned to agents for real-time path finding.
%To the best of our knowledge, this problem is so far open,
%for which we will present a solution system in this paper.

%Hungarian method____
%allow task swapping____

%% 3. comb. opt., scheduling 
%% Job shop scheduling, vehicle routing
%\textbf{Scheduling.}
%One may notice the analogy between TAPF and job-shop scheduling problems (JSSP)____ or vehicle routing problems (VRP)____.
%However, there are two key differences: (1) job durations in JSSP and route lengths in VRP are usually known in advance and (2) the execution of jobs or routes are independent of each other.
%Neither of the two conditions holds in TAPF, especially when the tasks are released online.

%\textcolor{red}{some postponed to Appendix~\ref{apd:more_related_work}}
\textit{We also append some discussion on other less related areas to Appendix~\ref{apd:more_related_work}.}
Despite the rich literature, none of the above directly solves the online problem that a real-world automated warehouse is faced with, which well motivates this work.