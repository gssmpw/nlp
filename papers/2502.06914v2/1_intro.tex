\section{Introduction}

% 1st paragraph: Background of Protein Cleavage Site Prediction such as application.


% Protease-mediated protein cleavage is essential for cellular functions such as protein degradation, signal transduction, and metabolic regulation, influencing processes like insulin secretion, Alzheimer's disease, and tumor invasion\cite{prudova2010multiplex}. Accurately predicting cleavage sites is vital for identifying therapeutic targets and guiding drug design. Computational methods offer efficient, scalable solutions that accelerate substrate discovery and support experimental validations, enhancing our understanding of complex biological systems. \enyan{We need to introduce what is protein hydrolysis. We introduce by summarizing the Sec. 2.1.  also we guide the readers to Fig.~\ref{fig:prelinminary} to quickly get the background knowledge. Also, we need to ensure the consistency of terms like proteolytic enzymes.  Finally, there are no easy-to-follow applications of cleavage site prediction given. Please add them. Rewrite this paragraph.}



% \enyan{In 2nd paragraph, We first introduce it is very important to predict the cleavage sites. And we give some application examples of the cleavage site.}
% % With extensive efforts of biologists, cleavage sites within substrate proteins under the proteolytic enzyme have been annotated using experimental methods. 
% Some studies have advanced protein cleavage site prediction using deep learning and traditional machine learning methods. \enyan{Then, here, we introduce several recent works}. 


In the enzyme-catalyzed protein hydrolysis, protein will split at cleavage sites under the catalysis of proteolytic enzymes. This process is illustrated in Fig.~\ref{fig:bio process}. 
And it is crucial for a variety of physiological processes, including cell proliferation, immune response, and cell death, etc~\cite{dixit2023road}. 
% Due to substrate sequence preferences and structural constraints, different proteolytic enzymes exhibit specificity toward certain amino acid sequences within substrate proteins. 
The positions of enzyme-catalyzed cleavage sites are determined by multiple factors.
% , such as the substrate’s amino acid sequence, its three-dimensional conformation, and the properties of the proteolytic enzyme\cite{fuchs2013substrate, timmer2009structural}. 
Accurate prediction of enzyme-catalyzed cleavage sites in the substrate proteins facilitates the identification of therapeutic targets and guides drug design~\cite{turk2006targeting}. For instance, abnormal protein hydrolysis is closely associated with cancer, viral infections, and neurodegenerative diseases, and predicting the cleavage sites of abnormal proteins under pathological conditions can reveal biomarkers or intervention targets.~\cite{mccauley2016hepatitis, liu2021modular}. 
Additionally, in the design of enzyme inhibitors or prodrugs, identifying key cleavage peptides, such as those cleaved under the catalysis of HIV enzyme, helps enhance drug specificity and minimize off-target effects~\cite{devroe2005hiv, lv2015hiv}.

Cleavage sites of proteins with proteolytic enzymes can be identified through peptide specificity assays or high-throughput mass spectrometry, but these experimental methods are often challenging and expensive~\cite{Zheng2020}. Therefore, developing efficient computational tools to complement experimental work is highly valuable. Recent studies have employed deep learning and traditional machine learning methods to advance the prediction of protein cleavage sites. For example, CAT3~\cite{CAT3} predicts caspase-3 cleavage sites based on position-specific scoring matrices (PSSM). {ProsperousPlus}~\cite{ProsperousPlus} integrates multiple methods to comprehensively evaluate cleavage site predictions.  
% HIV OctaScanner\cite{Sahibzada2025} utilizes machine learning to predict the proteolytic cleavage dynamics of HIV-1 protease substrates. DeepNeuropePred\cite{Wang2023} leverages a deep learning-based protein language model to predict cleavage sites in neuropeptide precursors.

% \enyan{@Chenao, add one or two recent works here.}

However, existing methods generally focus on an individual enzyme system. In contrast, many real-world scenarios require understanding how proteins are cleaved by different proteolytic enzymes. For instance, predicting off-target effects of protein therapeutics demands identifying potential cleavage sites for every relevant proteolytic enzyme in the human body’s complex environment~\cite{Werle2006}. Although one could train separate predictors for each enzyme, this approach neglects crucial information shared among distinct enzyme–substrate interactions.
In addition, the enzyme-specific model lacks the ability to deal with unseen enzymes, limiting its ability to predict cleavage sites for less-studied and de novo enzymes~\cite{https://doi.org/10.1002/cctc.202401542}. Therefore, it is crucial to develop a unified protein cleavage site predictor that can generalize across a diverse range of proteolytic enzymes. 
%\enyan{Complete the cite}

To develop a unified protein cleavage site predictor for diverse proteolytic enzymes, the information of enzyme should be extracted and incorporated for the prediction. However, due to substantial cost of biological experiments, existing cleavage site databases only cover a small number of proteolytic enzymes (Tab.~\ref{tab:data}), which significantly challenge the learning of enzyme information encoder.
Despite the limited coverage of enzymes in existing cleavage site datasets, many proteolytic enzymes are annotated with their active sites, which is the core functional region for catalyzing the protein hydrolysis. Specifically, the unique physicochemical environment of these active sites enables recognition of target substrates and lowers the activation energy required for cleaving specific peptide bonds.
Therefore, we propose to leverage redundant knowledge of enzyme active sites to enhance the modeling of enzymes in enzyme-catalyzed protein cleavage sites.


\begin{figure}[t!]
    \small
    \centering
    \includegraphics[width=\linewidth]{figure/bioprocess_zoomsites.png}
    \vskip -1em
    \caption{Illustration of the enzyme-catalyzed protein hydroysis.}
    \vskip -1.5em
    \label{fig:bio process}
\end{figure}


However, it is non-trivial to achieve a unified cleavage site predictor enhanced with enzyme active-site knowledge. Two major challenges remain to be resolved. \textit{First}, the cleavage process is influenced by various factors of enzymes such as 3D structures and environments of active sites. Check this.
Hence, how to design the the architecture of enzyme encoder to effectively capture useful information for enzyme-catalyzed cleavage site prediction? \textit{Second}, the active site regions of enzymes determine the specificity and efficiency of enzymatic hydrolysis. How can we leverage this rich information of enzyme active sites to improve cleavage site prediction? In an attempt to address the challenges, we propose a novel framework named {\method}. More specifically, a biochemically-informed enzyme encoder is deployed along with the active site-aware pooling to produce high-quality enzyme representations. We further augment the enzyme encoder by pretraining on a supplemented enzyme set for active-site prediction. Furthermore, a joint training of enzyme active site prediction and substrate cleavage site prediction is applied in {\method}. 
% Instead of directly using active site information, we deploy the active-site prediction as an training task. And we further expand this training data of active site with the most similar enzymes from the biology perspective.  In addition, to enhance the identification of active sites, we introduce an energy matrix into the enzyme modeling process based on the energy frustration phenomenon observed in enzyme active sites. To further emphasize the critical role of active sites in cleavage site prediction, the site-aware pooling method assigns higher weights to the active site regions and their surrounding key features. This approach highlights features associated with substrate cleavage during information aggregation.
In summary, our main contributions are:
\begin{itemize} [leftmargin=*]
    \item We investigate a novel and crucial problem of building a unified protein cleavage site predictor that generalizes across diverse proteolytic enzymes;
    \item We propose a novel framework {\method} that effectively integrates the enzyme active-site knowledge to enhance the cleavage site prediction in enzyme-protein interaction;
    \item Extensive experiments demonstrate the effectiveness of our {\method} in predicting cleavage sites of substrate proteins for both seen and unseen proteolytic enzymes.
\end{itemize}

\bigskip

% 3rd Paragraph: Introduce our main idea of protein cleavage site prediction: 1. We want to have uniform model for all enzymes which can generalize to new proteins. 
% 2. We want to use the active site information to enhance the protein cleavage.

% To develop a unified cleavage site predictor for diverse proteolytic enzymes, the information of enzyme should be extracted and incorporated for the prediction. 
% \enyan{Then, we introduce (1) active site of enzyme is very important in the process of protein hydrolysis. (refer to Fig.~\ref{fig:pre2}.); (2) The current number of enzymes in cleavage site prediction is small, which largely challenge the training of enzyme encoder. By contrast, rich annotation of active sties for proteolytic proteins are available. }  Hence, we propose to enhance the protein cleavage site prediction with redundant Enzyme Active-Site Knowledge.









