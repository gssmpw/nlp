\section{Conclusion}\label{sec:conclusion}
% In this paper we provide an overview of the literature on \glspl{stgnn} in short-term load forecasting and present benchmark results for selected algorithms from the literature. Our findings indicate that \acrshort{stgnn} models which use spatial relationships from temporal features offer advantages in forecasting household energy consumption compared to other models that only use temporal features. The results also display how different components in using \glspl{stgnn} can affect the performance of models. Especially, based on the performance, we see that a bipartite graph can model the dynamics of energy consumption data better than deriving direct relationships from signals or learnable embeddings. 

In this paper, we provide an overview of the literature on \glspl{stgnn} in short-term load forecasting and benchmark selected algorithms. Our findings show that integrating spatial relationships with temporal features improves forecasting accuracy for household energy consumption compared to models using only temporal features. Nevertheless, the most effective method for creating a graph to represent the proximity of energy consumption among households remains undetermined, given that even rudimentary graphs, like bipartite or fully connected graphs, can perform better than signal-based graphs. We also discuss how different components of \glspl{stgnn} can affect the performance of models. Notably, bipartite graphs effectively capture energy consumption dynamics, outperforming direct relationships from raw signals or embeddings. However, at the aggregate level, simple models such as SeasonalNaive outperform \gls{stgnn} models.

% This can provide some insight to practitioners and researchers who want to apply \glspl{stgnn} to \acrshort{stlf} in future studies.

However, our research has its limitations; we acknowledge that, for instance, our forecasting scenarios focus on one forecasting horizon (day ahead), which may not provide a comprehensive comparison of the strengths and weaknesses of the models. Furthermore, we did not consider the incorporation of exogenous variables into the STGNN models. Since energy consumption is significantly influenced by exogenous factors, such as weather or time indicators, the inclusion of these variables could improve model performance and provide deeper insight. Finally, given the growing scientific literature around \acrshort{stgnn} for residential \acrshort{stlf}, our selected models for benchmarking may not represent the entire research landscape in this domain. We believe that this incompleteness can motivate future research to provide a broader overview of \acrshort{stgnn} into the residential \acrshort{stlf} problem.

