\begin{abstract}
\acrfull{stlf} plays an important role in traditional and modern power systems. 
Most \acrshort{stlf} models predominantly exploit temporal dependencies from historical data to predict future consumption. Nowadays, with the widespread deployment of smart meters, their data can contain spatial-temporal dependencies. In particular, their consumption data is not only correlated to historical values but also to the values of 'neighboring' smart meters. This new characteristic motivates researchers to explore and experiment with new models that can effectively integrate spatial-temporal interrelations to increase forecasting performance. \glspl{stgnn} can leverage such interrelations by modeling relationships between smart meters as a graph and using these relationships as additional features to predict future energy consumption.
% STGNN meets that requirement because it enhances ...
% ... forecasting energy consumption. (Add forecasting at the beginning...)
While extensively studied in other spatiotemporal forecasting domains such as traffic, environments, or renewable energy generation, their application to load forecasting remains relatively unexplored, particularly in scenarios where the graph structure is not inherently available. This paper overviews the current literature focusing on \glspl{stgnn} with application in \acrshort{stlf}.
%highlights some key features of the applicabilities of \acrshort{stgnn} in \acrshort{stlf} 
Additionally, from a technical perspective, it also benchmarks selected \acrshort{stgnn} models for \acrshort{stlf} at the residential and aggregate levels. The results indicate that incorporating graph features can improve forecasting accuracy at the residential level; however, this effect is not reflected at the aggregate level. 
\end{abstract}


\keywords{Benchmark, \acrlong{stlf}, graph neural network, spatiotemporal forecasting.}
