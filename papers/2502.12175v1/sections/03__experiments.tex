\section{Experiments}\label{sec:exp_result}

\subsection{Dataset and data partition}
 We used an open dataset to train and evaluate the performance of \acrshort{stgnn}. The dataset is \acrfull{lcl} dataset~\cite{strbac_low_2024}, containing historical smart-meter data from 5,567 households over 2013 with a 30-minute resolution. We selected 228 load profiles from one specific sociodemographic group within the dataset (Acorn - D~\cite{acorn}). We selected consumers of the same sociodemographic group so that the pairwise relationship is solely based on the historical data. The selected dataset also has no missing values and zero values, which enables the use of MAPE metrics (Section \ref{subsec:model_benchmark}). As all algorithms are designed to operate solely with past consumption data~\cite{arastehfar_short-term_2022}, we use only historical data as input for all models, as outlined in Section \ref{sec:background}. 
 For data partitioning, to avoid information leakage between training and testing, our proposed train-validation-test split is indicated in Fig. \ref{fig:cross-val}.
\begin{figure}[h!]
    \centering
    \includegraphics[width=0.8\linewidth]{assets/GNN_splits.pdf}
    \caption{Train-validation-test split settings.}
    \label{fig:cross-val}
\end{figure}

Specifically, the training periods for splits 1, 2, and 3 span from January 1, 2013, to the day before July 1, September 1, and November 1, respectively. The validation and testing periods of each split cover the month immediately after the training period. The partition aims to see if the performance comparison is consistent with different time scales.

\subsection{Model training}
  For all models, we do hyperparameter tuning in learning rate, batch size, training window $W$ (Section \ref{sec:background}). The other hyperparameter is fixed for each experiment. 

For training, we use MAE as the loss function.
\begin{equation}
    \text{MAE} = \sum_{n=1}^N\sum_{t=1}^W|x_i - \hat{x}_i|
\end{equation}

with $x_i$ being the measured energy consumption and $\hat{x}_i$ being the predicted consumption of consumers $i$ at time $t$. 

The maximum epoch for training is 300 with early stopping options. To facilitate the implementation of the code, we build the experiment from the tsl package~\cite{Cini_Torch_Spatiotemporal_2022}. The details of the implementation of each model can be found in specified Github repository\footnote{\textbf{\href{https://github.com/Viet1004/Benchmark_STGNN_for_STLF}{https://github.com/Viet1004/Benchmark\_STGNN\_for\_STLF}}}. All the training is carried out in the IRIS cluster of the high-performance computer (HPC) facilities of the University of Luxembourg~\cite{ORBi-aab35225-a6bc-496d-a6e7-621189ebff46}.

\subsection{Model benchmark}\label{subsec:model_benchmark}
To validate the performance of the \acrshort{stgnn} models, we compare them with several benchmark models widely used in time series forecasting.

\begin{itemize}
    \item \textbf{Seasonal Naive}: This simple baseline model uses the value of the same hour on the previous day to predict the corresponding hour on the next day. 
    % It serves as a robust benchmark, particularly for datasets with daily or seasonal patterns.

    \item \textbf{\acrfull{var}}: The \acrfull{var} model is a statistical approach that extends the univariate autoregression (AR) model to multivariate time series. 
    % It captures the linear interdependencies of the historical values of multiple variables over time~\cite{guefano_methodology_2021}. 
    % The \acrshort{var} model is particularly useful for testing whether the nonlinear processing unit of \acrshort{stgnn} provides substantial improvements.

    \item \textbf{\acrshort{gru}}: Recurrent Neural Networks (RNNs) are extensively used in the literature to model sequential data because of their ability to capture temporal dependencies. In this study, we use a variant of RNN, the \acrfull{gru}. This architecture is also used in many \acrshort{stgnn} models in our research.
    % , which is known for its efficiency and effectiveness in handling long sequences without suffering from the vanishing gradient problem~\cite{cho_learning_2014}. 
    % Due to its advantage, it has been studied in \acrshort{stlf} problem~\cite{zheng_short-term_2018}. 

    \item \textbf{Transformer}: Transformer models~\cite{vaswani_attention_2023} represent a paradigm shift in sequence modeling by using self-attention mechanisms to compute pairwise dependencies between elements in a sequence. This architecture has been increasingly applied to \acrshort{stlf}~\cite{zhao_spatial_2023}.
\end{itemize}

Each of these benchmark models offers unique characteristics, allowing us to comprehensively assess the performance of \acrshort{stgnn} against a diverse set of approaches that vary in complexity, interpretability, and scalability. Note that all the models above only take into account the temporal dependency in the sequence. The training configuration for these models is the same as \acrshort{stgnn} models.

The error metrics to evaluate the performance of each model are:
\begin{align}
    \text{MAE} &= \frac{1}{NT}\sum_{n=1}^N\sum_{t=1}^T|x_i - \hat{x}_i| \\
    \text{MAPE} &= \frac{1}{NT}\sum_{n=1}^N\sum_{t=1}^T \left|\frac{x_i - \hat{x}_i}{x_i}\right| \\
    \text{RMSE} &= \sqrt{\frac{1}{NT}\sum_{n=1}^N\sum_{t=1}^T\left(x_i - \hat{x}_i\right)^2}
\end{align}

where N is the number of households and T is the time steps of the testing period.


% \subsection{Results and discussion}

% For the experiments, we test models with different time scales to validate the performance of models.
%  In each experiment, we perform the procedure 5 times using identical parameters to calculate the statistics of the experiment. The results will be displayed as the mean, followed by the standard deviation from the 5 trials.
% \subsubsection{Comparison between graph formation methods}\label{subsubsec:graph_formation}
% Graph formation methods can generate different topologies which allow information to be aggregated accordingly. This section presents the performance of \acrshort{stgnn}s using various predefined graph formation methods. The objective is to examine whether generating graphs based on statistical similarity among time series affects forecasting performance.
% \begin{table}[h!]
% \centering
% \captionsetup{justification=centering}
% \caption{Performance of models regarding graph formation methods in fold 3.}
% \begin{tabular}{lllll}
%                           &     &     & Metrics  &    \\
% \multirow{-2}{*}{Model}   & \multirow{-2}{*}{Graph  formation} & MAE (Wh) & MAPE (\%)    & RMSE (Wh)                                                                                        \\
% \hline
%                           & Euclidean        & \textbf{149.7(0.1)} & 55.9(0.3) & \textbf{295.6(0.6)   }\\
%                           & DTW              & 149.8(0.3) & \textbf{55.4(1.1)} & 295.8(1.0) \\
%                           & Correntropy      & 149.8(0.4) & 55.5(0.6) & 295.7(0.7)   \\
% \multirow{-4}{*}{GRUGCN}                          & Pearson          & 150.7(0.1) & 56.2(0.4) & 297.8(1.0)   \\
%   % & Transfer entropy &  0.151 & 0.556  & 0.299 \\
% \hline
%                           & Euclidean        & \textbf{150.9(0.5)} & \textbf{56.5(0.5)} & 297.9(1.3)   \\
%                           & DTW              & \textbf{150.9(0.4)} & 57.1(1.0) & \textbf{297.3(1.3)}    \\
%                           & Correntropy      & 151.4(0.6) & 56.8(0.6) & 298.8(0.7)  \\
% \multirow{-4}{*}{T-GCN}                           & Pearson          & 151.7(0.3) & 57.2(0.5) & 298.6(0.4) \\
%   % & Transfer entropy & 0.149    & 0.573   & 0.296   \\
% \hline
%                           & Euclidean        & 149.1(0.1) & 55.7(0.4) & 295.7(0.9) \\
%                           & DTW              & 149.2(0.5) & 56.2(0.6) & \textbf{295.5(0.8)}     \\
%                           & Correntropy      & \textbf{149.0(0.4)} & \textbf{55.0(0.4)} & 295.7(0.6) \\
% \multirow{-4}{*}{GCGRU}                          & Pearson          & 149.6(0.2) & 56.4(0.2) & 295.8(0.6)  \\
%  % & Transfer entropy & 0.235  & 1.224  & 0.428 \\
% \hline
% \end{tabular}
% \label{tab:graph_formation}
% \end{table}


% %%%%%%

% %%%% Full result at residential level fold 1 %%%%

% % \begin{tabular}{llll}
% % \toprule
% %  & MAE & MAPE & RMSE \\
% % \midrule
% % GraphWaveNetModel & 91.4(0.2) & 48.3(0.8) & 196.7(0.7) \\
% % RNNModel & 89.5(0.1) & 44.7(0.3) & 194.0(0.5) \\
% % VARModel & 134.3(0.2) & 109.4(0.4) & 218.5(0.3) \\
% % GatedGraphNetworkModel & 88.6(0.1) & 45.7(0.6) & 188.8(0.5) \\
% % AGCRNModel & 90.9(0.2) & 49.0(0.4) & 190.8(0.6) \\
% % TransformerModel & 90.4(0.4) & 45.1(0.9) & 194.5(0.4) \\
% % BiPartiteSTGraphModel & 90.2(0.1) & 49.2(0.9) & 188.3(0.4) \\
% % SameHour & 114.4(0.0) & 66.4(0.0) & 238.6(0.0) \\
% % GRUGCNModel & 89.2(0.8) & 43.5(0.8) & 193.6(0.3) \\
% % TGCNModel_2 & 88.8(0.3) & 43.3(0.8) & 193.4(0.1) \\
% % TGCNModel & 88.2(0.2) & 43.6(0.4) & 191.9(0.2) \\
% % GRUGCNModel & 89.1(0.6) & 43.3(1.0) & 194.1(0.6) \\
% % TGCNModel_2 & 88.9(0.4) & 43.7(0.3) & 193.9(0.7) \\
% % TGCNModel & 88.2(0.2) & 43.5(0.5) & 192.5(0.5) \\
% % GRUGCNModel & 89.0(0.2) & 43.0(1.1) & 193.7(0.6) \\
% % TGCNModel_2 & 88.9(0.5) & 43.8(0.1) & 193.5(1.0) \\
% % TGCNModel & 88.4(0.2) & 43.5(0.9) & 192.5(0.6) \\
% % GRUGCNModel & 89.4(0.2) & 43.8(0.9) & 194.6(0.7) \\
% % TGCNModel_2 & 89.6(0.2) & 43.8(0.5) & 194.6(0.3) \\
% % TGCNModel & 88.5(0.3) & 43.8(1.2) & 193.6(0.5) \\
% % \bottomrule
% % \end{tabular}

% %%%% Full result at aggregate level fold 1 %%%%

% % \begin{tabular}{llll}
% % \toprule
% %  & MAE & MAPE & RMSE \\
% % \midrule
% % GraphWaveNetModel & 8852.3(437.6) & 17.4(1.0) & 10336.5(440.2) \\
% % RNNModel & 8582.6(276.8) & 16.9(0.6) & 10080.4(285.0) \\
% % VARModel & 3439.8(22.7) & 7.1(0.0) & 4562.5(39.6) \\
% % GatedGraphNetworkModel & 8216.2(317.4) & 15.9(0.7) & 9801.8(273.1) \\
% % AGCRNModel & 7909.1(145.7) & 15.2(0.3) & 9427.3(190.6) \\
% % TransformerModel & 8827.3(369.0) & 17.3(0.6) & 10340.2(427.3) \\
% % BiPartiteSTGraphModel & 6614.5(348.2) & 13.1(0.8) & 8068.0(341.1) \\
% % SameHour & 3388.1(0.0) & 6.8(0.0) & 4574.1(0.0) \\
% % GRUGCNModel & 9266.6(154.2) & 18.1(0.4) & 10752.9(172.3) \\
% % TGCNModel_2 & 9268.4(260.1) & 18.1(0.5) & 10684.3(283.6) \\
% % TGCNModel & 8631.9(202.0) & 16.6(0.4) & 10173.6(208.0) \\
% % GRUGCNModel & 9557.0(516.1) & 18.7(1.1) & 11048.1(483.4) \\
% % TGCNModel_2 & 9248.8(291.0) & 18.0(0.5) & 10691.4(334.8) \\
% % TGCNModel & 8826.0(359.2) & 17.1(0.7) & 10249.9(378.8) \\
% % GRUGCNModel & 9383.2(566.2) & 18.2(1.2) & 10926.0(580.0) \\
% % TGCNModel_2 & 9037.7(340.2) & 17.5(0.7) & 10538.1(368.4) \\
% % TGCNModel & 8878.0(540.0) & 17.1(1.1) & 10342.0(508.1) \\
% % GRUGCNModel & 9352.1(602.8) & 18.2(1.3) & 10874.4(588.7) \\
% % TGCNModel_2 & 9206.5(298.9) & 17.9(0.5) & 10751.0(358.4) \\
% % TGCNModel & 8957.3(507.8) & 17.2(1.0) & 10544.0(505.4) \\
% % \bottomrule
% % \end{tabular}

% %%%%  Full result at residential level fold 2 %%%%

% % \begin{tabular}{llll}
% % \toprule
% %  & MAE & MAPE & RMSE \\
% % \midrule
% % GraphWaveNetModel & 128.4(0.3) & 52.9(0.6) & 256.7(1.0) \\
% % RNNModel & 126.5(0.2) & 50.5(0.4) & 254.7(0.4) \\
% % VARModel & 167.6(0.1) & 106.7(0.2) & 283.9(0.4) \\
% % GatedGraphNetworkModel & 121.7(0.4) & 49.5(0.6) & 245.9(1.5) \\
% % AGCRNModel & 123.8(0.3) & 50.6(0.6) & 248.7(0.6) \\
% % TransformerModel & 127.9(0.5) & 52.1(0.4) & 255.4(0.5) \\
% % BiPartiteSTGraphModel & 121.6(0.1) & 49.4(0.5) & 246.0(0.8) \\
% % SameHour & 156.4(0.0) & 75.9(0.0) & 304.1(0.0) \\
% % GRUGCNModel & 125.3(0.2) & 50.0(0.2) & 251.4(0.5) \\
% % TGCNModel_2 & 125.9(0.5) & 49.7(0.6) & 253.7(1.1) \\
% % TGCNModel & 124.6(0.2) & 49.3(0.6) & 251.9(0.4) \\
% % GRUGCNModel & 125.3(0.2) & 51.1(0.5) & 251.0(0.7) \\
% % TGCNModel_2 & 126.1(0.2) & 50.6(0.6) & 253.6(0.6) \\
% % TGCNModel & 124.6(0.2) & 49.6(0.8) & 251.7(0.7) \\
% % GRUGCNModel & 125.1(0.2) & 50.3(0.6) & 250.6(0.6) \\
% % TGCNModel_2 & 125.7(0.3) & 50.0(0.4) & 253.0(0.4) \\
% % TGCNModel & 124.5(0.3) & 49.4(0.6) & 251.2(1.0) \\
% % GRUGCNModel & 126.4(0.2) & 51.3(0.5) & 253.9(0.4) \\
% % TGCNModel_2 & 126.9(0.5) & 50.7(0.6) & 256.1(1.1) \\
% % TGCNModel & 125.3(0.2) & 50.2(0.6) & 252.9(0.3) \\
% % \bottomrule
% % \end{tabular}

% %%%% Full result at aggregate level fold 2 %%%%

% % \begin{tabular}{llll}
% % \toprule
% %  & MAE & MAPE & RMSE \\
% % \midrule
% % GraphWaveNetModel & 13908.5(468.0) & 19.2(0.9) & 17605.2(432.4) \\
% % RNNModel & 13489.2(166.1) & 18.8(0.3) & 16970.4(167.8) \\
% % VARModel & 13730.5(252.8) & 18.0(0.4) & 18669.1(273.7) \\
% % GatedGraphNetworkModel & 12787.0(562.2) & 16.9(0.7) & 16790.6(715.5) \\
% % AGCRNModel & 13264.5(345.9) & 17.9(0.5) & 17339.1(344.5) \\
% % TransformerModel & 13342.9(73.2) & 18.6(0.2) & 16909.1(46.5) \\
% % BiPartiteSTGraphModel & 12774.9(389.0) & 17.1(0.6) & 16559.8(423.0) \\
% % SameHour & 5478.3(0.0) & 8.2(0.0) & 7958.2(0.0) \\
% % GRUGCNModel & 12818.5(221.2) & 17.9(0.3) & 16227.1(242.3) \\
% % TGCNModel_2 & 13930.1(332.2) & 19.4(0.4) & 17580.1(406.7) \\
% % TGCNModel & 13543.2(305.0) & 18.5(0.4) & 17219.4(341.7) \\
% % GRUGCNModel & 12597.9(456.4) & 17.3(0.7) & 16145.7(460.9) \\
% % TGCNModel_2 & 13718.9(318.1) & 19.0(0.4) & 17420.9(366.8) \\
% % TGCNModel & 13316.4(412.7) & 18.2(0.7) & 17057.7(439.2) \\
% % GRUGCNModel & 12706.8(310.7) & 17.7(0.4) & 16171.3(404.7) \\
% % TGCNModel_2 & 13711.6(89.9) & 19.1(0.2) & 17388.7(65.8) \\
% % TGCNModel & 13170.4(543.1) & 18.2(0.9) & 16869.4(528.3) \\
% % GRUGCNModel & 13079.5(256.7) & 18.0(0.6) & 16625.9(194.4) \\
% % TGCNModel_2 & 13878.5(415.5) & 19.2(0.6) & 17674.4(444.5) \\
% % TGCNModel & 13150.4(144.0) & 18.2(0.2) & 16870.8(215.1) \\
% % \bottomrule
% % \end{tabular}

% %%%% Full result at residential level fold 3 %%%%

% % \begin{tabular}{llll}
% % \toprule
% %  & MAE & MAPE & RMSE \\
% % \midrule
% % GraphWaveNetModel & 161.6(12.4) & 64.2(8.4) & 315.2(21.5) \\
% % RNNModel & 153.1(1.7) & 56.9(2.2) & 302.2(0.9) \\
% % VARModel & 198.5(23.1) & 119.1(31.3) & 328.4(13.2) \\
% % GatedGraphNetworkModel & 159.5(25.4) & 71.5(31.8) & 299.2(18.0) \\
% % AGCRNModel & 149.4(1.5) & 58.2(1.0) & 293.6(1.6) \\
% % TransformerModel & 153.3(1.7) & 57.1(0.6) & 301.7(3.3) \\
% % BiPartiteSTGraphModel & 149.2(2.6) & 55.1(1.2) & 295.2(3.9) \\
% % SameHour & 178.7(15.3) & 75.5(11.3) & 345.6(25.2) \\
% % GRUGCNModel & 137.5(24.7) & 53.1(5.0) & 275.0(41.4) \\
% % TGCNModel_2 & 138.7(25.3) & 54.1(5.5) & 277.5(42.7) \\
% % TGCNModel & 136.8(24.3) & 52.7(4.7) & 274.9(41.4) \\
% % GRUGCNModel & 146.0(7.7) & 61.2(11.7) & 286.0(19.7) \\
% % TGCNModel_2 & 138.4(25.0) & 54.3(5.6) & 276.6(41.4) \\
% % TGCNModel & 145.6(7.3) & 61.9(11.4) & 285.8(19.4) \\
% % GRUGCNModel & 144.8(9.8) & 54.9(2.1) & 286.6(18.0) \\
% % TGCNModel_2 & 145.8(10.3) & 55.3(2.5) & 288.6(18.6) \\
% % TGCNModel & 144.1(9.9) & 54.3(2.8) & 286.9(17.7) \\
% % GRUGCNModel & 147.0(7.4) & 62.0(11.5) & 288.0(19.6) \\
% % TGCNModel_2 & 139.4(24.6) & 54.8(4.9) & 277.9(41.4) \\
% % TGCNModel & 146.0(7.3) & 62.2(11.6) & 285.9(19.8) \\
% % \bottomrule
% % \end{tabular}

% %%%% Full result at aggregate level fold 3 %%%%

% % \begin{tabular}{llll}
% % \toprule
% %  & MAE & MAPE & RMSE \\
% % \midrule
% % GraphWaveNetModel & 13206.6(3641.0) & 16.0(4.2) & 16387.4(3990.9) \\
% % RNNModel & 15072.7(658.4) & 17.9(1.0) & 18575.3(444.6) \\
% % VARModel & 10786.8(2450.3) & 12.9(2.8) & 14080.6(2493.0) \\
% % GatedGraphNetworkModel & 12738.7(1743.4) & 15.1(2.0) & 15780.6(1647.5) \\
% % AGCRNModel & 13240.2(297.0) & 15.5(0.3) & 16633.9(331.3) \\
% % TransformerModel & 15338.8(779.1) & 18.0(1.0) & 18905.6(797.1) \\
% % BiPartiteSTGraphModel & 14887.7(303.6) & 17.1(0.6) & 18457.6(249.5) \\
% % SameHour & 7952.3(3969.0) & 9.7(4.2) & 10685.3(4430.2) \\
% % GRUGCNModel & 13008.0(2169.5) & 16.8(0.5) & 15920.5(2871.2) \\
% % TGCNModel_2 & 13363.0(2345.8) & 17.3(0.4) & 16298.3(3070.1) \\
% % TGCNModel & 13233.9(2278.9) & 17.0(0.4) & 16194.4(3001.0) \\
% % GRUGCNModel & 12902.1(3312.5) & 16.2(2.3) & 16046.6(3534.7) \\
% % TGCNModel_2 & 13148.6(2093.8) & 17.1(0.8) & 16039.3(2704.0) \\
% % TGCNModel & 12609.8(2896.0) & 15.9(1.7) & 15795.3(3061.1) \\
% % GRUGCNModel & 13926.1(830.7) & 16.8(0.3) & 17338.3(848.9) \\
% % TGCNModel_2 & 14468.6(870.3) & 17.7(0.8) & 17869.9(815.4) \\
% % TGCNModel & 14194.2(330.0) & 17.3(0.8) & 17629.4(290.5) \\
% % GRUGCNModel & 12458.2(2856.6) & 15.9(1.6) & 15504.1(3087.4) \\
% % TGCNModel_2 & 12943.2(1865.4) & 16.9(0.5) & 15717.3(2405.7) \\
% % TGCNModel & 12038.8(2796.1) & 15.4(1.6) & 15002.0(3033.4) \\
% % \bottomrule
% % \end{tabular}

% %%%%%%


% Based on the results in \autoref{tab:graph_formation}, we conclude that different graph construction methods do not significantly impact the effectiveness of learning, even though the resulting topologies may vary. \acrshort{stgnn}s, as a data-focused method, adapt by learning with different topologies to achieve comparable outcomes. In the following section, for each predefined-graph model, we employ the graph formation technique that results in the minimal MAE. 
% \subsubsection{Performance Comparison with different temporal scale}\label{subsubsec:performance_comparison}

% \begin{table*}[h!]
% \centering
% \captionsetup{justification=centering,margin=2cm}
% \begingroup
% \setlength{\tabcolsep}{4pt} % Default value: 6pt
% \renewcommand{\arraystretch}{1.2} % Default value: 1
% \begin{tabular}{cllllllllll}
% \hline
% \hline
% \multirow{3}{*}{Group} & \multirow{3}{*}{Models} & \multicolumn{9}{c}{Metrics} \\ \cline{3-11}
%                        &                         & \multicolumn{3}{c}{Fold 1}  & \multicolumn{3}{c}{Fold 2} & \multicolumn{3}{c}{Fold 3} \\ \cline{3-11}
%                        &                         & MAE (Wh)    & MAPE (\%)   & RMSE (Wh)   & MAE (Wh)    & MAPE (\%)    & RMSE (Wh) & MAE (Wh)    & MAPE (\%)   & RMSE (Wh) \\ 
% \hline
% \multirow{4}{*}{Benchmark} 
% & SeasonalNaive                              & 114.4(0.0) & 66.4(0.0) & 238.6(0.0) & 156.4(0.0) & 75.9(0.0) & 304.1(0.0) & 186.3(0.0) & 81.1(0.0) & 358.2(0.0) \\
% & VAR                                         & 134.3(0.2) & 109.5(0.5) & 218.5(0.3) & 167.6(0.1) & 106.7(0.2) & 283.9(0.4) & 210.1(0.5) & 134.7(0.5) & 335.0(0.6)      \\
% & Transformer                                  &    90.2(0.3) & 44.9(0.8) & 194.5(0.4) & 127.9(0.5) & 52.1(0.4) & 255.4(0.5) & 154.1(0.0) & 56.7(0.1) & 303.4(0.4)      \\
% \hline
% \multirow{8}{*}{STGNN}  & GRUGCN                                         & \textbf{89.0(0.2)} & \underline{\textbf{43.0(1.1)}} & \textbf{193.7(0.6)}	     & \textbf{125.3(0.2)} & \textbf{50.0(0.2)} & \textbf{251.4(0.5)}       & \textbf{149.7(0.1)} & 55.9(0.3) & \textbf{295.6(0.6)   }    \\
% & GCGRU                                      & \underline{\textbf{88.2(0.2)}} & \textbf{43.6(0.4)} & \textbf{191.9(0.2)}     & \textbf{124.6(0.2)} & \underline{\textbf{49.3(0.6)}} & \textbf{251.9(0.4)}   &    \textbf{149.0(0.4)} & \textbf{55.0(0.4)} & \textbf{295.7(0.6)}  \\
% & T-GCN                                     & \textbf{88.9(0.5)} & \textbf{43.8(0.1)} & \textbf{193.5(1.0)}   &   \textbf{125.0(0.5)}      &   \textbf{49.7(0.6)}      &     \textbf{253.7(1.1)}    & \textbf{150.9(0.5)} & 56.5(0.5) & \textbf{297.9(1.3)}  \\
% & AGCRN                                     & 91.0(0.2) & 49.1(0.4) & \textbf{190.9(0.6)} & \textbf{123.8(0.3)} & 50.6(0.6) & \textbf{248.7(0.6)} & \textbf{150.1(0.3)} & 58.8(0.2) & \textbf{294.2(1.1)}       \\
% & Bipartite                                   &   90.2(0.1) & 48.8(0.5) & \underline{\textbf{188.4(0.3)}} & \underline{\textbf{121.6(0.1)}} & \textbf{49.4(0.5)} & \textbf{246.0(0.8)} & \textbf{148.0(0.1)} & \underline{\textbf{54.7(0.5)}} & \textbf{293.3(0.9) }     \\
% & FC-GNN                            &   \textbf{88.6(0.1)} & 45.7(0.6) & \textbf{189.0(0.3)} & \textbf{121.7(0.4)} & \textbf{49.5(0.6)} & \underline{\textbf{245.9(1.5)}} & \underline{\textbf{146.9(0.1)}} & \textbf{55.6(0.6)} & \underline{\textbf{290.3(1.0)}}     \\
% & GraphWavenet   & 91.4(0.2) & 48.1(0.8) & 196.7(0.8) & 128.4(0.3) & 52.9(0.6) & 256.7(1.0) & 155.9(0.5) & 60.1(0.5) & 304.6(0.8)  \\     
% \hline
% \hline
% \end{tabular}
% \endgroup
% \caption{Performance of different \acrshort{stgnn} models with different experiment settings. The values in \textbf{bold} signify that the performance of \acrshort{stgnn} models is better than all benchmark models. The \underline{underlined values} means the best performance by error metrics.
% }
% \label{tab:result_228}
% \end{table*}

% \begin{figure}[h!]
%     \centering
%     \includegraphics[width=\linewidth]{assets/GNN_splits.pdf}
%     \caption{Train-validation-test split settings}
%     \label{fig:cross-val}
% \end{figure}


% Compared to deep learning models that process only temporal features such as \acrshort{gru} and \acrshort{tfm}, the advantage of \acrshort{stgnn} is evident but varies between time periods. Compared to GRU, which uses matrix multiplication as the unit cell of the recurrent network, GCGRU and T-GCN always achieve better performance by applying GCN as the unit cell. Similarly, GRUGCN also outperforms \acrshort{gru} by applying \acrshort{gcn} on top of it to account for spatial dependency. This showcases the effective use of graph neural networks to model spatial relationships (see Table \ref{tab:result_228}).

% % Comparing GCGRU, T-GCN which use GCN as spatial processing unit to embed node features with RNN which simply use matrix multiplication, we see that there is a small advantage when integrating spatial processing unit. That demonstrates relevance in applying the graph neural network to model spatial relationships.

% For models using a learnable graph, this technique usually gives better results compared to the benchmark models except for GraphWavenet. However, comparing AGCRN, which uses a learnable graph, with GCGRU, which uses a predefined graph from signals, there is a downgrade in performance in some metrics. This is because the learnable graph offers a more flexible way to model the spatial relationship. However, it could make the forecasting task more susceptible to overfitting. When testing on a more distant future (one month after training as outlined in the Fig. \ref{fig:cross-val}), the model actually performs worse than the counterpart that uses a predefined graph. 
% % A special case of this type is that model \acrshort{stegnn} does not converge at the end. The model employs a mechanism to incorporate a learnable graph into their structure. To reduce the computational cost, for each node, this mechanism only keeps top k of the most influential neighbors and makes the rest of the adjacency matrix 0. This mechanism introduces the problem of a sparse gradient in training~\cite{cini_sparse_nodate}, where the gradient becomes zero, making learning difficult. On the other hand, in our experiment settings, the model takes one day before to forecast 1 day ahead. This makes the training data limited in size. By using a small number of training data, the model might not be able to overcome the problem and end up giving a substandard result. 

% In terms of the performance, the results are not always consistent. The bipartite model and FC-GNN model often perform better than their counterparts. These models utilize a weighted sum in the AGGREGATE step, where the weights are derived from the embedded input of each node. This makes the AGGREGATE step more adaptable to the input. An interesting case is the bipartite model. Although it performs slightly worse than FC-GNN, it is more scalable since the interaction between the nodes of FC-GNN is $N^2$, while for the bipartite model, it is only $2KN$ with $K$ being the number of virtual nodes. It also outperforms other models most of the time. One reason might be that, due to the nature of the load profile dataset, the consumption pattern of users is grouped by latent factors such as socio-demographic status~\cite{acorn}. By defining virtual nodes, this model can account for latent factors that are related to original nodes. The bipartite topology allows these virtual nodes to gather information in a cluster-like manner; then, the aggregated information is passed down to each node as additional information for learning.


% % In terms of forecasting result with different periods, we notice that the magnitude of error differs vastly among them, indicating that the nature of data and testing period strongly affect the performance. It suggests that learning from historical data alone is insufficient to capture the dynamics of energy consumption. Incorporating temporal indicators, such as weekends or holidays, could enhance the model's ability to capture these contextual variations more effectively.



% \subsubsection{Forecast at the aggregate level}
% We investigate the performance of load forecasting at the aggregate level simply by aggregating all the forecasts at the residential level (see Table \ref{tab:result_agg_228}).
% % \begin{table}[h!]
% % \centering
% % \captionsetup{justification=centering,margin=1cm}
% % \begingroup
% % \setlength{\tabcolsep}{5pt} % Default value: 6pt
% % \renewcommand{\arraystretch}{1.2} % Default value: 1
% % \begin{tabular}{l|lll}
% % \hline
% % \hline
% % \multicolumn{1}{c|}{\multirow{3}{*}{Models}} & \multicolumn{3}{c}{Metrics} \\ \cline{2-4}
% % \multicolumn{1}{c|}{}  & \multicolumn{3}{c}{Average (over three folds)} \\ \cline{2-4}
% % \multicolumn{1}{c|}{}                        & MAE (kWh)     & MAPE (\%)   & RMSE   \\ 
% % \hline
% % SeasonalNaive                               &   4.495      &  0.075   &  7.00   \\
% % \hline
% % VAR                                         &  9.450      &  0.122      &  11.946 \\
% % \hline
% % GRU                                         &  12.267    &  0.178     &  15.024      \\
% % \hline
% % Tranformer                                  &   12.508     &  0.179    &  15.375          \\
% % \hline
% % GRUGCN                                      & 12.115	      & 0.176   &  14.880	       \\
% % \hline
% % % GCLSTM                                      &    0.139     &   1.22      &  0.239    \\
% % % \hline
% % % T-GCN                                       &  \textbf{0.089 }      &    \textbf{0.444 }    &   0.194    \\
% % % \hline
% % % AGCRNN                                      &   \textbf{0.090 }     &     0.484    &  \textbf{ 0.191 }       \\
% % % \hline
% % % STEGNN                                      &  3.067     &  34.660     &   3.866    \\
% % % \hline
% % TGCN                                        & 11.879       &  0173     &  14.584    \\
% % \hline
% % Bipartite                                   &   11.421     &  0.158    &  14.414   \\
% % \hline
% % % Fully Connected                             &         &         &      \\
% % % \hline
% % % GraphWavenet                                &   0.091      &   0.494      &    0.196    \\     
% % \hline
% % \end{tabular}
% % \endgroup
% % \caption{Performance of different \acrshort{stgnn} models at aggregate level.
% % }
% % \label{tab:result_228_agg}
% % \end{table}


% \begin{table*}[hb!]
% \centering
% \captionsetup{justification=centering,margin=2cm}
% \begingroup
% \setlength{\tabcolsep}{3pt} % Default value: 6pt
% \renewcommand{\arraystretch}{1.2} % Default value: 1
% \begin{tabular}{cllllllllll}
% \hline
% \hline
% \multirow{3}{*}{Group} & \multirow{3}{*}{Models} & \multicolumn{9}{c}{Metrics} \\ \cline{3-11}
%                        &                         & \multicolumn{3}{c}{Fold 1}  & \multicolumn{3}{c}{Fold 2} & \multicolumn{3}{c}{Fold 3} \\ \cline{3-11}
%                        &                         & MAE (kWh)    & MAPE (\%)   & RMSE (kWh)   & MAE (kWh)    & MAPE (\%)    & RMSE (kWh) & MAE (kWh)    & MAPE (\%)   & RMSE (kWh) \\ 
% \hline
% \multirow{4}{*}{Benchmark} 
% & SeasonalNaive                             & \underline{3.39(0.0)} & \underline{6.8(0.0)} & \underline{4.57(0.0)} & \underline{5.48(0.0)} & \underline{8.2(0.0)} & \underline{7.96(0.0)} & \underline{5.97(0.0)} & \underline{7.6(0.0)} & \underline{8.47(0.0)} \\
% & VAR                                         & 3.44(0.23) & 7.1(0.05) & 4.56(0.04) & 13.73(0.25) & 18.0(0.4) & 18.67(0.27) & 9.56(0.19) & 11.5(0.2) & 12.84(0.24) \\
% & GRU  & 8.58(0.28) & 16.9(0.6) & 10.08(0.29) & 13.49(0.17) & 18.8(0.3) & 16.97(0.17) & 15.35(0.39) & 18.2(0.7) & 18.76(0.29) \\
% & Transformer                                 & 8.83(0.37) & 17.3(0.6) & 10.34(0.43) & 13.34(0.73) & 18.6(0.2) & 16.91(0.05) & 15.72(0.17) & 18.5(0.2) & 19.30(0.12) \\
% \hline
% \multirow{8}{*}{STGNN} 
% & GRUGCN                                      & 9.38(0.57) & 18.2(1.2) & 10.93(0.58) & 12.71(0.31) & 17.7(0.4) & 16.17(0.4) & 14.31(0.37) & 16.7(0.3) & 17.71(0.44) \\
% & GCGRU                                       & 8.88(0.54) & 17.1(1.1) & 10.34(0.51) & 13.17(0.54) & 18.2(0.90) & 16.87(0.53) & 14.34(0.17) & 16.9(0.3) & 17.74(0.20) \\
% & T-GCN                                       & 9.27(0.26) & 18.1(0.5) & 10.68(0.28) & 13.93(0.33) & 19.4(0.4) & 17.58(0.41) & 14.52(0.44) & 17.4(0.4) & 17.81(0.55) \\
% & AGCRN                                     & 7.91(0.15) & 15.2(0.3) & 9.43(0.19) & 13.26(0.35) & 17.9(0.5) & 17.34(0.34) & 13.16(0.28) & 15.3(0.2) & 16.66(0.36) \\
% & Bipartite                                   & 6.61(0.35) & 13.1(0.8) & 8.07(0.34) & 12.77(0.39) & 17.1(0.6) & 16.56(0.42) & 14.81(0.29) & 16.9(0.5) & 18.40(0.24) \\
% & FC-GNN                             & 8.22(0.32) & 15.9(0.7) & 9.80(0.27) & 12.79(0.56) & 16.9(0.7) & 16.79(0.72) & 13607.2(165.1) & 16.0(0.4) & 16.60(0.14) \\
% & GraphWavenet   & 8.85(0.44) & 17.4(1.0) & 10.34(0.44) & 13.91(0.47) & 19.2(0.9) & 17.61(0.43) & 15.02(0.44) & 18.1(0.5) & 18.37(0.57) \\     
% \hline
% \hline
% \end{tabular}
% \endgroup
% \caption{Performance of different \acrshort{stgnn} models at the \textbf{aggregate level}. For learnable-graph based model, the average performance is selected. The \underline{underlined values}
% means the best performance by error metrics. 
% }
% \label{tab:result_agg_228}
% \end{table*}

% At the aggregate level, the forecast results obtained through aggregation are inferior to those of the baseline model (SeasonalNaive). This behavior is also observed in other deep learning models such as \acrshort{gru} and \acrshort{tfm}. An explanation is that these models tend to prioritize learning from "easier" periods with consistent patterns while failing to adequately capture atypical events, such as consumption spikes~\cite{zhang_unlocking_2023} (see Fig. \ref{fig:multiple-forecast}). This can lead to underestimation during abnormal periods. At the aggregate level, instead of centering predictions around the actual average consumption, the forecasts may consistently underestimate the consumption spikes, and hence the errors are not balanced out. \acrshort{stgnn}s, by incorporating spatial learning, potentially propagate errors throughout the spatial dimension and therefore do not address this issue.

% \begin{figure}[h!]
%     \centering
%     \includegraphics[width=\linewidth]{assets/MAC000026.pdf}
%     \caption{Load forecast of smart meter MAC000026 in 4 days. We observe that all deep learning models underestimate the spike in their forecasts.}
%     \label{fig:multiple-forecast}
% \end{figure}


% % \begin{table}[h!]
% % \centering
% % \caption{Cross-validation settings for validating \acrshort{stgnn} models.}
% % \begin{tabularx}{\columnwidth}{|X|X|X|X|}
% % \hline
% % \textbf{Phase} & \textbf{Training} & \textbf{Validation} & \textbf{Testing} \\ \hline
% % \textbf{Fold 1} & Jan 1, 2013 -- Jun 30, 2013 & Jul 1, 2013 -- Jul 31, 2013 & Aug 1, 2013 -- Aug 31, 2013 \\ \hline
% % \textbf{Fold 2} & Jan 1, 2013 -- Aug 31, 2013 & Sep 1, 2013 -- Sep 30, 2013 & Oct 1, 2013 -- Oct 31, 2013 \\ \hline
% % \textbf{Fold 3} & Jan 1, 2013 -- Oct 31, 2013 & Nov 1, 2013 -- Nov 30, 2013 & Dec 1, 2013 -- Dec 31, 2013 \\ \hline
% % \end{tabularx}
% % \label{tab:cross_validation}
% % \end{table}


% % \begin{table}[h!]
% % \centering
% % \caption{Cross-validation settings for validating \acrshort{stgnn} models.}
% % \begin{tabular}{|c|c|c|}
% % \hline
% % \textbf{Phase}         & \textbf{Time Period} \\ \hline
% % \textbf{Fold 1} & \begin{tabular}[c]{@{}l@{}}Training: Jan 1, 2013 -- Jun 30, 2013 \\ 
% % Validation: Jul 1, 2013 -- Jul 31, 2013 \\ 
% % Testing: Aug 1, 2013 -- Aug 31, 2013\end{tabular} \\ \hline
% % \textbf{Fold 2} & \begin{tabular}[c]{@{}l@{}}Training: Jan 1, 2013 -- Aug 31, 2013 \\ 
% % Validation: Sep 1, 2013 -- Sep 30, 2013 \\ 
% % Testing: Oct 1, 2013 -- Oct 31, 2013\end{tabular} \\ \hline
% % \textbf{Fold 3} & \begin{tabular}[c]{@{}l@{}}Training: Jan 1, 2013 -- Oct 31, 2013 \\ 
% % Validation: Nov 1, 2013 -- Nov 30, 2013 \\ 
% % Testing: Dec 1, 2013 -- Dec 31, 2013\end{tabular} \\ \hline
% % \end{tabular}
% % \label{tab:cross_validation}
% % \end{table}


