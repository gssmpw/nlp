\section{Introduction}\label{sec:intro}
 \acrfull{stlf} plays a vital role in power systems by supporting grid and market operations and, consequently, helping with the reliability of traditional and modern power systems~\cite{eren2024comprehensive}. Its importance grows as power systems become more complex and decentralized~\cite{big_data_smart_grid}. In particular, modern power systems require better forecasts to deal with the increasing grid (e.g. dispatch, reconfiguration) and market operations (e.g. portfolio balancing) caused by dynamic fluctuations in generation and consumption~\cite{big_data_smart_grid}. Tackling the dynamic fluctuations appropriately beforehand also has a significant economic impact affecting all power system players. For instance, a one-percent increase in forecast accuracy could save up to £10 million annually in the UK~\cite{arastehfar_short-term_2022}.

In general, most \acrshort{stlf} models emerge from statistical or machine learning models~\cite{hong_probabilistic_2016}. Statistical models usually assume the property of time series such as stationarity or invertibility~\cite{arastehfar_short-term_2022}, which might be unsuitable to model volatile energy consumption data down at the residential level~\cite{Moustati2024-oo}. The alternative is to shift to data-driven approaches, such as machine learning and especially deep learning, to effectively model the expanding space of available data, mainly caused by the introduction of smart meters.  
Currently, most \acrshort{stlf} models rely solely on the temporal dependency of historical data for forecasting. However, since consumption data can be collected from multiple smart meters at the household level, the interrelation between different households can also be discovered and used~\cite{WU2023125939}. Incorporating information from 'neighboring' households with strong interconnections could potentially improve forecasting accuracy for a specific household. Consequently, forecasting models should integrate both temporal and spatial dependencies to effectively capture the data's dynamics.

A solution to capture these dependencies is to use \acrfull{stgnn} because, in addition to processing data in temporal order, \acrshort{stgnn} accounts for spatial dependencies through a graph-based approach. 
This deep learning-based method has been the subject of substantial research in fields such as traffic prediction, environmental studies, and energy generation, where spatio-temporal characteristics are evident~\cite{bui_spatial-temporal_2022}. 
Recently, more energy research has applied this architecture to forecast energy consumption at the residential level~\cite{arastehfar_short-term_2022, lin_residential_2021}. Although graphs in other problems such as traffic or energy generation can be derived from geographic locations, spatial proximity does not fully reflect the similarity in energy consumption patterns. This is because consumption behaviors at the residential level are stochastic, and the similarity between usage patterns lies on sociodemographic factors rather than spatial proximity~\cite{Feng2023STGNetSR}. To represent the relationship between households, many \acrshort{stlf} studies have directly extracted the similarity of signals among them~\cite{bloemheuvel_graph_2024}. Another, more flexible way is to model graph structure as learnable parameters and optimize it during training to produce the best forecast~\cite{lin_residential_2021,wei_short-term_2023}. These strategies add the spatial notion in the energy consumption data as a graph and make the application of \glspl{stgnn} on the \acrshort{stlf} problem feasible. However, the absence of an inherent graph structure in energy consumption behaviors necessitates constructing one from historical data. This raises an important question: \textit{Does a temporally informed graph model predict more effectively load that a model based solely on temporal features?}

This question combined with increasing research on \acrshort{stgnn} for \acrshort{stlf} has motivated our research. Our goal is to summarize the current literature on \acrshort{stlf} using \glspl{stgnn} and to identify the key components that influence the performance of the models by:
\begin{itemize}
    \item Evaluating the performance of representative \acrshort{stgnn} models in residential \acrshort{stlf}. 
    \item Identifying relevant factors in the construction of \glspl{stgnn} that affect performance in \acrshort{stlf}. 
\end{itemize}

The remainder of the paper is structured as follows. Section \ref{sec:background} gives a brief overview of the existing \acrshort{stgnn} architectures and models for spatiotemporal forecasting in general and load forecasting in particular. In Section \ref{sec:exp_result}, we designed experiments to validate and compare the performance of \acrshort{stgnn} on \acrshort{stlf} in different time scales. Building upon these results, we provide insight and explanations of the results specific to load forecasting. The study concludes with a summary and future directions in Section \ref{sec:conclusion}.  
