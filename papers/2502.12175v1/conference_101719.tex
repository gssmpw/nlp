% \documentclass[conference]{IEEEtran}
% \IEEEoverridecommandlockouts
% % The preceding line is only needed to identify funding in the first footnote. If that is unneeded, please comment it out.
% \usepackage{cite}
% \usepackage{amsmath,amssymb,amsfonts}
% \usepackage{algorithmic}
% \usepackage{graphicx}
% \usepackage{textcomp}
% \usepackage{xcolor}
% \def\BibTeX{{\rm B\kern-.05em{\sc i\kern-.025em b}\kern-.08em
%     T\kern-.1667em\lower.7ex\hbox{E}\kern-.125emX}}

\documentclass[a4paper]{article}

% \addbibresource{bibliography.bib}
% \IEEEoverridecommandlockouts
% The preceding line is only needed to identify funding in the first footnote. If that is unneeded, please comment it out.
% \usepackage{cite}
\usepackage{amsmath,amssymb,amsfonts}
\usepackage{algorithmic}
\usepackage{graphicx}
\usepackage{textcomp}
\usepackage{multirow}
\usepackage[a4paper, total={6in, 8in}]{geometry}
% \usepackage{subfig}
% \usepackage{caption}
% \usepackage{subcaption}
\usepackage[dvipsnames]{xcolor}
\usepackage{tabularx}
\usepackage{url} %% added to fix the URLs the bibliography
\usepackage[hidelinks]{hyperref}
\usepackage{dblfloatfix}
\def\BibTeX{{\rm B\kern-.05em{\sc i\kern-.025em b}\kern-.08em
    T\kern-.1667em\lower.7ex\hbox{E}\kern-.125emX}}
    
% \usepackage[
% backend=biber,
% style=numeric,
% sorting=none
% ]{biblatex}

\usepackage[acronym, nonumberlist]{glossaries}
\usepackage{adjustbox}
\usepackage[most]{tcolorbox}

% Define an alias for the rounded corner box
% Define an inline box with fixed height and slightly rounded corners
% Define an inline box with a configurable width
\newtcbox{\inlinebox}[1][]{
    colframe=Cyan,    % Light blue border
    colback=white,       % No background color
    arc=0.5mm,             % Slightly rounded corners
    boxrule=0.8pt,       % Thin border
    width=0.3cm,
    height=0.6cm,        % Fixed height
    valign=center,       % Vertical alignment for inline text
    nobeforeafter,       % Avoid spacing issues in inline usage
    #1                   % Allow optional customization
}

\providecommand{\keywords}[1]
{
  \small	
  \textbf{\textit{Keywords---}} #1
}

% \makeglossaries

\newglossaryentry{AI}{
    name=AI: Artificial Intelligence,
    description={The simulation of human intelligence}
}

\newglossaryentry{LLM}{
    name=LLM: Large Language Model,
    description={A type of artificial intelligence designed to understand and generate human-like text}
}

\newglossaryentry{API}{
    name=API: Application Programming Interface,
    description={A software interface for offering a service to other pieces of software}
}

\newglossaryentry{AST}{
    name=AST: Abstract Syntax Tree,
    description={A tree representation of the abstract syntactic structure of source code written in a programming language}
}

\newglossaryentry{ReACC}{
    name=ReACC: Retrieval-Augmented Code Completion,
    description={A framework that enhances code completion by leveraging external context from a large codebase}
}

\newglossaryentry{CMSIS}{
    name=CMSIS: Cortex Microcontroller Software Interface Standard,
    description={A hardware abstraction layer independent of vendor for the Cortex-M processor series}
}

\newglossaryentry{HAL}{
    name=HAL: Hardware Abstraction Layer,
    description={A layer of programming that allows a computer operating system to interact with a hardware device at an abstract level}
}

\newglossaryentry{RAG}{
    name=RAG: Retrieval-Augmented Generation,
    description={A process that enhances large language models by allowing them to respond to prompts using a specified set of documents}
}

\newglossaryentry{STM32F407}{
    name=STM32F407: High-\allowbreak performance Microcontroller,
    description={A microcontroller that offers the performance of the Cortex-M4 core}
}

\newglossaryentry{AURIX TC334}{
    name=AURIX TC334: 32-bit Microcontroller from Infineon,
    description={A microcontroller designed for automotive and industrial applications, featuring a 32-bit TriCore-\allowbreak architecture}
}

\newglossaryentry{LED}{
    name=LED: Light-Emitting Diode,
    description={A semiconductor light source that emits light when current flows through it}
}

\newglossaryentry{CortexM4}{
    name=Cortex-M4: 32-bit processor design from ARM,
    description={A 32-bit processor design optimized for real-time applications with low power consumption}
}

\newglossaryentry{GPIO}{
    name=GPIO: General-Purpose Input/Output Pin,
    description={A versatile pin on a microcontroller that can be configured as either an input or an output. As an \textbf{input}, it can read external signals such as button presses. As an \textbf{output}, it can control devices}
}

\newglossaryentry{Offset}{
    name=Offset: Relative distance of a specific register,
    description={The relative distance or position of a specific register or memory location within a hardware block, measured from a base address}
}

\newglossaryentry{Clock}{
    name=Clock: Synchronization signal for operations,
    description={The signal used to synchronize operations within a microcontroller or hardware system, ensuring consistent timing and execution of tasks}
}

\newglossaryentry{GPT4oMini}{
    name=GPT-4o Mini: Large Language Model from OpenAI,
    description={A compact variant of the GPT-4 language model designed for cost-efficient and versatile tasks}
}

\newglossaryentry{FAISS}{
    name=FAISS: Facebook AI Similarity Search,
    description={An open-source library for efficient similarity search and clustering of high-dimensional vectors}
}




% \addbibresource{bibliography.bib}

% \addbibresource{additional_bibliography.bib}

\begin{document}

\title{Spatiotemporal Graph Neural Networks in short-term load forecasting: Does adding Graph Structure in Consumption Data Improve Predictions?}



\author{
Quoc Viet NGUYEN,
Joaquin DELGADO FERNANDEZ, \\
Sergio POTENCIANO MENCI \\
\textit{Interdisciplinary Centre for Security, Reliability and Trust - SnT,} \\
\textit{University of Luxembourg, Luxembourg }\\
\{quocviet.nguyen, 
joaquin.delgadofernandez,
sergio.potenciano-menci\}@uni.lu
\thanks{This research was funded in part by the Luxembourg National Research Fund (FNR) and PayPal, PEARL grant reference 13342933/Gilbert Fridgen and by FNR grant reference HPC BRIDGES/2022\_Phase2/17886330/DELPHI. For the purpose of open access and in fulfillment of pen access and fulfilling the obligations arising from the grant agreement, the author has applied a Creative Commons Attribution 4.0 International (CC BY 4.0) license to any Author Accepted Manuscript version arising from this submission.
The experiments presented in this paper were carried out using the HPC
facilities of the University of Luxembourg~\cite{ORBi-aab35225-a6bc-496d-a6e7-621189ebff46}– see hpc.uni.lu
}
}


\maketitle

\begin{abstract}
\acrfull{stlf} plays an important role in traditional and modern power systems. 
Most \acrshort{stlf} models predominantly exploit temporal dependencies from historical data to predict future consumption. Nowadays, with the widespread deployment of smart meters, their data can contain spatial-temporal dependencies. In particular, their consumption data is not only correlated to historical values but also to the values of 'neighboring' smart meters. This new characteristic motivates researchers to explore and experiment with new models that can effectively integrate spatial-temporal interrelations to increase forecasting performance. \glspl{stgnn} can leverage such interrelations by modeling relationships between smart meters as a graph and using these relationships as additional features to predict future energy consumption.
% STGNN meets that requirement because it enhances ...
% ... forecasting energy consumption. (Add forecasting at the beginning...)
While extensively studied in other spatiotemporal forecasting domains such as traffic, environments, or renewable energy generation, their application to load forecasting remains relatively unexplored, particularly in scenarios where the graph structure is not inherently available. This paper overviews the current literature focusing on \glspl{stgnn} with application in \acrshort{stlf}.
%highlights some key features of the applicabilities of \acrshort{stgnn} in \acrshort{stlf} 
Additionally, from a technical perspective, it also benchmarks selected \acrshort{stgnn} models for \acrshort{stlf} at the residential and aggregate levels. The results indicate that incorporating graph features can improve forecasting accuracy at the residential level; however, this effect is not reflected at the aggregate level. 
\end{abstract}


\keywords{Benchmark, \acrlong{stlf}, graph neural network, spatiotemporal forecasting.}


\section{Introduction}\label{sec:intro}
 \acrfull{stlf} plays a vital role in power systems by supporting grid and market operations and, consequently, helping with the reliability of traditional and modern power systems~\cite{eren2024comprehensive}. Its importance grows as power systems become more complex and decentralized~\cite{big_data_smart_grid}. In particular, modern power systems require better forecasts to deal with the increasing grid (e.g. dispatch, reconfiguration) and market operations (e.g. portfolio balancing) caused by dynamic fluctuations in generation and consumption~\cite{big_data_smart_grid}. Tackling the dynamic fluctuations appropriately beforehand also has a significant economic impact affecting all power system players. For instance, a one-percent increase in forecast accuracy could save up to £10 million annually in the UK~\cite{arastehfar_short-term_2022}.

In general, most \acrshort{stlf} models emerge from statistical or machine learning models~\cite{hong_probabilistic_2016}. Statistical models usually assume the property of time series such as stationarity or invertibility~\cite{arastehfar_short-term_2022}, which might be unsuitable to model volatile energy consumption data down at the residential level~\cite{Moustati2024-oo}. The alternative is to shift to data-driven approaches, such as machine learning and especially deep learning, to effectively model the expanding space of available data, mainly caused by the introduction of smart meters.  
Currently, most \acrshort{stlf} models rely solely on the temporal dependency of historical data for forecasting. However, since consumption data can be collected from multiple smart meters at the household level, the interrelation between different households can also be discovered and used~\cite{WU2023125939}. Incorporating information from 'neighboring' households with strong interconnections could potentially improve forecasting accuracy for a specific household. Consequently, forecasting models should integrate both temporal and spatial dependencies to effectively capture the data's dynamics.

A solution to capture these dependencies is to use \acrfull{stgnn} because, in addition to processing data in temporal order, \acrshort{stgnn} accounts for spatial dependencies through a graph-based approach. 
This deep learning-based method has been the subject of substantial research in fields such as traffic prediction, environmental studies, and energy generation, where spatio-temporal characteristics are evident~\cite{bui_spatial-temporal_2022}. 
Recently, more energy research has applied this architecture to forecast energy consumption at the residential level~\cite{arastehfar_short-term_2022, lin_residential_2021}. Although graphs in other problems such as traffic or energy generation can be derived from geographic locations, spatial proximity does not fully reflect the similarity in energy consumption patterns. This is because consumption behaviors at the residential level are stochastic, and the similarity between usage patterns lies on sociodemographic factors rather than spatial proximity~\cite{Feng2023STGNetSR}. To represent the relationship between households, many \acrshort{stlf} studies have directly extracted the similarity of signals among them~\cite{bloemheuvel_graph_2024}. Another, more flexible way is to model graph structure as learnable parameters and optimize it during training to produce the best forecast~\cite{lin_residential_2021,wei_short-term_2023}. These strategies add the spatial notion in the energy consumption data as a graph and make the application of \glspl{stgnn} on the \acrshort{stlf} problem feasible. However, the absence of an inherent graph structure in energy consumption behaviors necessitates constructing one from historical data. This raises an important question: \textit{Does a temporally informed graph model predict more effectively load that a model based solely on temporal features?}

This question combined with increasing research on \acrshort{stgnn} for \acrshort{stlf} has motivated our research. Our goal is to summarize the current literature on \acrshort{stlf} using \glspl{stgnn} and to identify the key components that influence the performance of the models by:
\begin{itemize}
    \item Evaluating the performance of representative \acrshort{stgnn} models in residential \acrshort{stlf}. 
    \item Identifying relevant factors in the construction of \glspl{stgnn} that affect performance in \acrshort{stlf}. 
\end{itemize}

The remainder of the paper is structured as follows. Section \ref{sec:background} gives a brief overview of the existing \acrshort{stgnn} architectures and models for spatiotemporal forecasting in general and load forecasting in particular. In Section \ref{sec:exp_result}, we designed experiments to validate and compare the performance of \acrshort{stgnn} on \acrshort{stlf} in different time scales. Building upon these results, we provide insight and explanations of the results specific to load forecasting. The study concludes with a summary and future directions in Section \ref{sec:conclusion}.  


\section{Overview of STGNN for STLF}\label{sec:background}

%%% Define STGNN

\glspl{stgnn} are models designed to handle time series data collected from various locations~\cite{cini_graph_2023}. In the context of \acrshort{stlf}, we consider a dataset collected from $N$ consumers (e.g. households). In the simplest case, we assume 
for each consumer, data only contain historical energy consumption from smart meters. 
Let $x^i_t \in \mathbb{R}$ be the energy consumption of consumer $i$ at time step $t$; each time series $\{x^i_t\}_{t:t+T}$ is the energy consumption of consumer $i$ in period $t \rightarrow t+T$. 
Consequently, by stacking all consumers, the matrix $\mathbf{X}_{t:t+T} \in \mathbb{R}^{N \times T}$ represents the consumption records of N consumers in the period $t \rightarrow t+T$. Given the consumption data $\mathbf{X}_{t-W:t}$ from $W$ previous steps , \acrshort{stgnn} models forecast consumption $\hat{\mathbf{X}}_{t:t+H}$ in $H$ next steps for all consumers. 

In doing so, \acrshort{stgnn} represents consumers and their relationships by a graph structure. A graph $\mathcal{G} = (\mathcal{V}, \mathcal{E})$ consists of a set of nodes $\mathcal{V} = \{v_1,v_2,...,v_N\}$ and a set of edges $\mathcal{E} \subseteq \mathcal{V} \times \mathcal{V}$, where $(v_i,v_j) \in \mathcal{E}$ if node $v_i$ connects to node $v_j$. This connectivity is compactly represented by an adjacency matrix $\mathcal{A}$, where an entry $a_{ij} > 0$ signifies the edge weight between nodes $v_i$ and $v_j$. In the context of residential \acrshort{stlf}, each household is assigned to one node and its features contain time series  of its historical energy consumption. Respectively, the edges can be derived based on the patterns between residential load profiles~\cite{wang_short-term_2023}. 
% The connectivity of the graph decides how the information propagates between nodes. By designing a suitable adjacency matrix, we can enable the learning process to capture meaningful patterns effectively. 

In what follows, we first describe how graph structures are usually constructed, then detail the components of \acrshort{stgnn} and how those components interact with each other. Finally, we present some representative models in the literature.

\subsection{Graph Formation}\label{subsec:graph}
To capture the spatial dependency between different nodes, it is necessary to provide a spatial structure in the form of a graph. The topology of the graph dictates how the features are aggregated between the nodes. Based on how the graph is constructed, the graph formation methods in the literature can be classified as follows~\cite{bloemheuvel_graph_2024}:
% \begin{equation}\label{eq:sim}
%     a_{ij} = 
%     \begin{cases}
%         \text{SIM}(v_1,v_2) & \text{ if }\text{SIM}(v_1,v_2) > \epsilon \\
%         0 & \text{otherwise}
%     \end{cases}
% \end{equation}
\begin{itemize}
    \item \textit{Predefined graph}: The topology of the graph is fixed during training. The edge of the graph may be established using supplementary information and by assessing the similarity between time series. The similarity measure can be based on Pearson coefficient~\cite{fernandez_privacy-preserving_2022}, Euclidean distance, DTW distance~\cite{10202782}, or correntropy~\cite{cini_graph-based_2023}. An edge is considered to exist when the similarity surpasses a defined threshold value~\cite{bloemheuvel_graph_2024}.
    \item \textit{Learnable graph}: Some models integrate graph formation into the learning process. This technique does not require a graph from the dataset before training but self-organizes the graph during training so that it can facilitate the flow of information for graph neural network~\cite{wu_graph_nodate}. We identify some learning algorithms that incorporate graph structure learning for downstream tasks (i.e., forecasting) later in section \ref{subsec:exampleSTGNN}.
\end{itemize}
% \subsubsection{}
% \subsubsection{Learnable graph}

\subsection{Temporal and Spatial Processing Unit as components of STGNN}\label{subsec:processing}

To model spatiotemporal data as in load forecasting, one must process information in temporal and spatial dimensions. 
The most popular deep learning models to process temporal information are through \acrfull{rnn}~\cite{sutskever_sequence_2014}, \acrfull{cnn} models, or \acrfull{mlp}~\cite{rodrigues_short-term_2023}. 
% We denote TEMP as the temporal processing unit. 

Respectively, the most dominant method to propagate information along the graph is through the \acrfull{ms} paradigm~\cite{cini_graph_2023} which involves 2 steps:
\begin{enumerate}
    \item \textit{Message Aggregation}: Each node collects information (or "messages") from its neighboring nodes. This step captures the local structure and features of the graph by pooling or combining the attributes of connected nodes. 
    \item \textit{Feature Update}: After aggregation, each node updates its own representation by combining the aggregated information with its existing attributes. This step refines the node's state, embedding its local and neighboring information into a new feature representation.
\end{enumerate}

There are several models fit into this paradigm, but one of the most popular is \acrfull{gcn}~\cite{mansoor_spatio-temporal_2023}. It is present in numerous examples of \acrshort{stgnn} within our study.

\subsection{STGNN architecture}\label{subsec:architecture}
We refer to the structure of \glspl{stgnn} in the literature. The most prominent architectures of STGNN are \acrfull{tts}, \acrfull{tas}~\cite{gao_equivalence_2022}.

\subsubsection{Time-then-Space (TTS) architecture}\label{subsubsec:tts}
In \acrshort{tts} architecture, the models encode information in the temporal dimension first, as depicted by the solid arrow in Fig. \ref{fig:tts}. The abstract representation of each node is now broadcast (dashed arrow in Fig. \ref{fig:tts}) to their neighbors through the spatial unit to incorporate useful information in the spatial dimension.
% \begin{align}\label{equ:tts}
%     h^i_{T} &= \text{TEMP}(x^i_{[T-W:T]}) \\
%     z^i_{T} &= \text{MS}(h^i_T, \mathcal{A})    
% \end{align}

\subsubsection{Time-and-Space (T\&S) architecture}\label{subsubsec:tas}
Alternatively, the \acrshort{tas} architecture integrates the processing of temporal and spatial features more cohesively. At each time step, the features of individual nodes are propagated and abstracted through a spatial processing unit, updating the node representations in the spatial dimension (dashed arrow in Fig. \ref{fig:tas}). Then, the node representation is processed using a recurrent unit, producing a hidden state at that time step.  
% \begin{align}
%     z^i_{t} &= \text{MS}(x^i_t, \mathcal{A}) \\
%     H_{t} &= \text{RNN}(H_{t-1},Z_t) 
% \end{align}
% Notably, the temporal processing unit TEMP here is an instance of \acrshort{rnn} models. 
Subsequently, this internal hidden state is combined with the upcoming observation to produce the hidden state in the next step (as illustrated by the curved arrow in Fig. \ref{fig:tas}). It is worth noting that a recurrent model is employed as the temporal processing unit in this context~\cite{gao_equivalence_2022}.
% \subsubsection{\acrfull{stt} architecture}

% In addition, although there is also \acrfull{stt} architecture mentioned in ~\cite{cini_graph_2023}, it is not popular and is not the focus of our study. 

% We visualize the operations of two architectures in Fig. \ref{fig:architectures}. 
% The solid arrows indicate processing units in the temporal dimension, and the dashed arrows denote processing units in the spatial dimension. 

\begin{figure}[ht]
    \centering
    % First minipage
    \begin{minipage}[t]{0.46\textwidth} % Adjust width to fit side-by-side
        \centering
        \includegraphics[width=\linewidth]{assets/tts.pdf} % Scale image to fill minipage
        \caption{\acrshort{tts} architecture.}
        \label{fig:tts}
    \end{minipage}
    \hfill % Add flexible spacing between minipages
    % Second minipage
    \begin{minipage}[t]{0.48\textwidth} % Adjust width to fit side-by-side
        \centering
        \includegraphics[width=\linewidth]{assets/tas.pdf} % Scale image to fill minipage
        \caption{\acrshort{tas} architecture.}
        \label{fig:tas}
    \end{minipage}
    % Main caption
    % \caption{Simple visualization of \acrshort{stgnn} architectures.}
    \label{fig:architectures}
\end{figure}


% In the next section, we will review some of the \acrshort{stgnn} models that have been tested in the literature and classify them into the framework in section \ref{subsec:architecture}. 

\subsection{Examples of STGNN models}\label{subsec:exampleSTGNN}
Given the different architectures of \acrshort{stgnn} models, we review  existing models that are representatives of the 
 described architectures. We present and compare the similarity between models based on the components and architectures in Section \ref{subsec:processing} and \ref{subsec:architecture}.

\paragraph{\acrshort{grugcn}~\cite{gao_equivalence_2022}} This model is of type \acrshort{tts}. It uses the \acrfull{gru} (a variant of \acrshort{rnn}) as a temporal processing unit to represent temporal characteristics of each node and then applies \acrshort{gcn} on top of the encoded features to account for spatial dependencies.

\paragraph{\acrshort{gcgru}~\cite{arastehfar_short-term_2022}} This model uses \acrshort{tas} architecture. In particular, it uses \acrshort{gcn} as the spatial processing unit to capture the spatial dependency at time step $t$. This unit acts as a cell in the \acrshort{gru} model, which recursively calculates the hidden state of the entire graph. Note that for a more consistent comparison, we replace the \acrfull{lstm} model in \cite{arastehfar_short-term_2022} by \acrshort{gru} as in other models.

\paragraph{\acrshort{t-gcn}~\cite{huang_gated_2023}} This model updates node features by a 2-layer GCN before processing them by \acrshort{gru} model. It is similar to \acrshort{gcgru} but updates only node features through the \acrshort{gcn}, not the hidden state.

\paragraph{\acrshort{agcrn}~\cite{bai_adaptive_2020}} This model is similar to \acrshort{gcgru} in terms of encoding both spatial and temporal dimensions. However, to model the relationships between nodes more flexibly, this approach introduces a learnable embedding for each node. Consequently, the feature in each node is not only determined by the historical data but also by the embedding. This allows the model to determine the optimal embeddings that maximize the forecast performance.

\paragraph{\acrshort{gw}~\cite{lin_spatial-temporal_2021}} Similar to \acrshort{agcrn}, this model assigns each node a learnable embedding. In terms of forecasting, it employs a temporal convolutional layer to encode historical data of each consumer, followed by a graph convolutional layer to incorporate features in the spatial dimension. By stacking multiple layers of this unit with different parameters, the model can capture various patterns at different temporal and spatial scales. As described, it is of type \acrshort{tts}.  


\paragraph{\acrfull{fc-gnn}~\cite{satorras_multivariate_2022}} This model presumes that every node is interconnected, resulting in a complete graph. However, in the \textit{Message Aggregation} step, the weight of each "message" from neighbors will be adjusted due to the attention mechanism, allowing an adaptable edge weight between nodes even though the topology of the graph does not reflect the relation between time series. In terms of forecasting, it follows the \acrshort{tts} architecture with \acrshort{mlp} as a temporal processing unit and the attention mechanism in the spatial processing unit.

\paragraph{\acrfull{bp-gnn}~\cite{satorras_multivariate_2022}} This model is a variant of the FC-GNN model. However, instead of full connectivity, it defines virtual nodes that connect to all original nodes, forming a bipartite graph. These nodes act as hubs, aggregating, updating, and relaying information between original nodes.

% However, instead of a fully connected topology, it defines virtual nodes that connect to all other nodes in the graph, generating a bi-partite graph. These virtual nodes act as hubs that gather information, update them, and then pass it back to the original nodes. 

The selected models are arranged in Table \ref{tab:overview}, following the structure outlined in Sections \ref{subsec:graph} and \ref{subsec:processing}.

\begin{table}[h]

\caption{Summary of the models in the described framework}
\label{tab:overview}
\begingroup
\setlength{\tabcolsep}{5pt} % Default value: 6pt
\renewcommand{\arraystretch}{1.2} % Default value: 1
\resizebox{0.9\textwidth}{!}{
\begin{tabular}{l|ll|lll}
\hline
\multicolumn{1}{c|}{\multirow{2}{*}{Models}} & \multicolumn{2}{c|}{Graph formation}     & \multicolumn{2}{c}{Architecture} \\ \cline{2-5} 
\multicolumn{1}{c|}{}                        & Predefined graph & Learnable graph & TTS            & T\&S               \\ 
\hline
GRUGCN                                      & \multicolumn{1}{c}{\checkmark}  &                 & \multicolumn{1}{c}{\checkmark}       \\ 
\hline
GCGRU                                      & \multicolumn{1}{c}{\checkmark}  &                 &                & \multicolumn{1}{c}{\checkmark}          \\ 
\hline
T-GCN                                       & \multicolumn{1}{c}{\checkmark}  &                 &  &  \multicolumn{1}{c}{\checkmark}             \\ 
\hline
AGCRN                                      &             & \multicolumn{1}{c|}{\checkmark}      &                &  \multicolumn{1}{c}{\checkmark}            \\ 
\hline
GraphWavenet                                &     &           \multicolumn{1}{c|}{\checkmark}     & \multicolumn{1}{c}{\checkmark}               &            \\ 
\hline
% STEGNN                                      &             & \multicolumn{1}{c|}{\checkmark}      &                &             &  \multicolumn{1}{c}{\checkmark} \\ \hline
FC-GNN                             &   \multicolumn{1}{c}{\checkmark}          &       & \multicolumn{1}{c}{\checkmark}    &            \\ 
\hline
Bipartite                                   &    \multicolumn{1}{c}{\checkmark}         &       & \multicolumn{1}{c}{\checkmark}    &              \\ 
\hline
\end{tabular}
}
\endgroup
\end{table}




\section{Experiments}\label{sec:exp_result}

\subsection{Dataset and data partition}
 We used an open dataset to train and evaluate the performance of \acrshort{stgnn}. The dataset is \acrfull{lcl} dataset~\cite{strbac_low_2024}, containing historical smart-meter data from 5,567 households over 2013 with a 30-minute resolution. We selected 228 load profiles from one specific sociodemographic group within the dataset (Acorn - D~\cite{acorn}). We selected consumers of the same sociodemographic group so that the pairwise relationship is solely based on the historical data. The selected dataset also has no missing values and zero values, which enables the use of MAPE metrics (Section \ref{subsec:model_benchmark}). As all algorithms are designed to operate solely with past consumption data~\cite{arastehfar_short-term_2022}, we use only historical data as input for all models, as outlined in Section \ref{sec:background}. 
 For data partitioning, to avoid information leakage between training and testing, our proposed train-validation-test split is indicated in Fig. \ref{fig:cross-val}.
\begin{figure}[h!]
    \centering
    \includegraphics[width=0.8\linewidth]{assets/GNN_splits.pdf}
    \caption{Train-validation-test split settings.}
    \label{fig:cross-val}
\end{figure}

Specifically, the training periods for splits 1, 2, and 3 span from January 1, 2013, to the day before July 1, September 1, and November 1, respectively. The validation and testing periods of each split cover the month immediately after the training period. The partition aims to see if the performance comparison is consistent with different time scales.

\subsection{Model training}
  For all models, we do hyperparameter tuning in learning rate, batch size, training window $W$ (Section \ref{sec:background}). The other hyperparameter is fixed for each experiment. 

For training, we use MAE as the loss function.
\begin{equation}
    \text{MAE} = \sum_{n=1}^N\sum_{t=1}^W|x_i - \hat{x}_i|
\end{equation}

with $x_i$ being the measured energy consumption and $\hat{x}_i$ being the predicted consumption of consumers $i$ at time $t$. 

The maximum epoch for training is 300 with early stopping options. To facilitate the implementation of the code, we build the experiment from the tsl package~\cite{Cini_Torch_Spatiotemporal_2022}. The details of the implementation of each model can be found in specified Github repository\footnote{\textbf{\href{https://github.com/Viet1004/Benchmark_STGNN_for_STLF}{https://github.com/Viet1004/Benchmark\_STGNN\_for\_STLF}}}. All the training is carried out in the IRIS cluster of the high-performance computer (HPC) facilities of the University of Luxembourg~\cite{ORBi-aab35225-a6bc-496d-a6e7-621189ebff46}.

\subsection{Model benchmark}\label{subsec:model_benchmark}
To validate the performance of the \acrshort{stgnn} models, we compare them with several benchmark models widely used in time series forecasting.

\begin{itemize}
    \item \textbf{Seasonal Naive}: This simple baseline model uses the value of the same hour on the previous day to predict the corresponding hour on the next day. 
    % It serves as a robust benchmark, particularly for datasets with daily or seasonal patterns.

    \item \textbf{\acrfull{var}}: The \acrfull{var} model is a statistical approach that extends the univariate autoregression (AR) model to multivariate time series. 
    % It captures the linear interdependencies of the historical values of multiple variables over time~\cite{guefano_methodology_2021}. 
    % The \acrshort{var} model is particularly useful for testing whether the nonlinear processing unit of \acrshort{stgnn} provides substantial improvements.

    \item \textbf{\acrshort{gru}}: Recurrent Neural Networks (RNNs) are extensively used in the literature to model sequential data because of their ability to capture temporal dependencies. In this study, we use a variant of RNN, the \acrfull{gru}. This architecture is also used in many \acrshort{stgnn} models in our research.
    % , which is known for its efficiency and effectiveness in handling long sequences without suffering from the vanishing gradient problem~\cite{cho_learning_2014}. 
    % Due to its advantage, it has been studied in \acrshort{stlf} problem~\cite{zheng_short-term_2018}. 

    \item \textbf{Transformer}: Transformer models~\cite{vaswani_attention_2023} represent a paradigm shift in sequence modeling by using self-attention mechanisms to compute pairwise dependencies between elements in a sequence. This architecture has been increasingly applied to \acrshort{stlf}~\cite{zhao_spatial_2023}.
\end{itemize}

Each of these benchmark models offers unique characteristics, allowing us to comprehensively assess the performance of \acrshort{stgnn} against a diverse set of approaches that vary in complexity, interpretability, and scalability. Note that all the models above only take into account the temporal dependency in the sequence. The training configuration for these models is the same as \acrshort{stgnn} models.

The error metrics to evaluate the performance of each model are:
\begin{align}
    \text{MAE} &= \frac{1}{NT}\sum_{n=1}^N\sum_{t=1}^T|x_i - \hat{x}_i| \\
    \text{MAPE} &= \frac{1}{NT}\sum_{n=1}^N\sum_{t=1}^T \left|\frac{x_i - \hat{x}_i}{x_i}\right| \\
    \text{RMSE} &= \sqrt{\frac{1}{NT}\sum_{n=1}^N\sum_{t=1}^T\left(x_i - \hat{x}_i\right)^2}
\end{align}

where N is the number of households and T is the time steps of the testing period.


% \subsection{Results and discussion}

% For the experiments, we test models with different time scales to validate the performance of models.
%  In each experiment, we perform the procedure 5 times using identical parameters to calculate the statistics of the experiment. The results will be displayed as the mean, followed by the standard deviation from the 5 trials.
% \subsubsection{Comparison between graph formation methods}\label{subsubsec:graph_formation}
% Graph formation methods can generate different topologies which allow information to be aggregated accordingly. This section presents the performance of \acrshort{stgnn}s using various predefined graph formation methods. The objective is to examine whether generating graphs based on statistical similarity among time series affects forecasting performance.
% \begin{table}[h!]
% \centering
% \captionsetup{justification=centering}
% \caption{Performance of models regarding graph formation methods in fold 3.}
% \begin{tabular}{lllll}
%                           &     &     & Metrics  &    \\
% \multirow{-2}{*}{Model}   & \multirow{-2}{*}{Graph  formation} & MAE (Wh) & MAPE (\%)    & RMSE (Wh)                                                                                        \\
% \hline
%                           & Euclidean        & \textbf{149.7(0.1)} & 55.9(0.3) & \textbf{295.6(0.6)   }\\
%                           & DTW              & 149.8(0.3) & \textbf{55.4(1.1)} & 295.8(1.0) \\
%                           & Correntropy      & 149.8(0.4) & 55.5(0.6) & 295.7(0.7)   \\
% \multirow{-4}{*}{GRUGCN}                          & Pearson          & 150.7(0.1) & 56.2(0.4) & 297.8(1.0)   \\
%   % & Transfer entropy &  0.151 & 0.556  & 0.299 \\
% \hline
%                           & Euclidean        & \textbf{150.9(0.5)} & \textbf{56.5(0.5)} & 297.9(1.3)   \\
%                           & DTW              & \textbf{150.9(0.4)} & 57.1(1.0) & \textbf{297.3(1.3)}    \\
%                           & Correntropy      & 151.4(0.6) & 56.8(0.6) & 298.8(0.7)  \\
% \multirow{-4}{*}{T-GCN}                           & Pearson          & 151.7(0.3) & 57.2(0.5) & 298.6(0.4) \\
%   % & Transfer entropy & 0.149    & 0.573   & 0.296   \\
% \hline
%                           & Euclidean        & 149.1(0.1) & 55.7(0.4) & 295.7(0.9) \\
%                           & DTW              & 149.2(0.5) & 56.2(0.6) & \textbf{295.5(0.8)}     \\
%                           & Correntropy      & \textbf{149.0(0.4)} & \textbf{55.0(0.4)} & 295.7(0.6) \\
% \multirow{-4}{*}{GCGRU}                          & Pearson          & 149.6(0.2) & 56.4(0.2) & 295.8(0.6)  \\
%  % & Transfer entropy & 0.235  & 1.224  & 0.428 \\
% \hline
% \end{tabular}
% \label{tab:graph_formation}
% \end{table}


% %%%%%%

% %%%% Full result at residential level fold 1 %%%%

% % \begin{tabular}{llll}
% % \toprule
% %  & MAE & MAPE & RMSE \\
% % \midrule
% % GraphWaveNetModel & 91.4(0.2) & 48.3(0.8) & 196.7(0.7) \\
% % RNNModel & 89.5(0.1) & 44.7(0.3) & 194.0(0.5) \\
% % VARModel & 134.3(0.2) & 109.4(0.4) & 218.5(0.3) \\
% % GatedGraphNetworkModel & 88.6(0.1) & 45.7(0.6) & 188.8(0.5) \\
% % AGCRNModel & 90.9(0.2) & 49.0(0.4) & 190.8(0.6) \\
% % TransformerModel & 90.4(0.4) & 45.1(0.9) & 194.5(0.4) \\
% % BiPartiteSTGraphModel & 90.2(0.1) & 49.2(0.9) & 188.3(0.4) \\
% % SameHour & 114.4(0.0) & 66.4(0.0) & 238.6(0.0) \\
% % GRUGCNModel & 89.2(0.8) & 43.5(0.8) & 193.6(0.3) \\
% % TGCNModel_2 & 88.8(0.3) & 43.3(0.8) & 193.4(0.1) \\
% % TGCNModel & 88.2(0.2) & 43.6(0.4) & 191.9(0.2) \\
% % GRUGCNModel & 89.1(0.6) & 43.3(1.0) & 194.1(0.6) \\
% % TGCNModel_2 & 88.9(0.4) & 43.7(0.3) & 193.9(0.7) \\
% % TGCNModel & 88.2(0.2) & 43.5(0.5) & 192.5(0.5) \\
% % GRUGCNModel & 89.0(0.2) & 43.0(1.1) & 193.7(0.6) \\
% % TGCNModel_2 & 88.9(0.5) & 43.8(0.1) & 193.5(1.0) \\
% % TGCNModel & 88.4(0.2) & 43.5(0.9) & 192.5(0.6) \\
% % GRUGCNModel & 89.4(0.2) & 43.8(0.9) & 194.6(0.7) \\
% % TGCNModel_2 & 89.6(0.2) & 43.8(0.5) & 194.6(0.3) \\
% % TGCNModel & 88.5(0.3) & 43.8(1.2) & 193.6(0.5) \\
% % \bottomrule
% % \end{tabular}

% %%%% Full result at aggregate level fold 1 %%%%

% % \begin{tabular}{llll}
% % \toprule
% %  & MAE & MAPE & RMSE \\
% % \midrule
% % GraphWaveNetModel & 8852.3(437.6) & 17.4(1.0) & 10336.5(440.2) \\
% % RNNModel & 8582.6(276.8) & 16.9(0.6) & 10080.4(285.0) \\
% % VARModel & 3439.8(22.7) & 7.1(0.0) & 4562.5(39.6) \\
% % GatedGraphNetworkModel & 8216.2(317.4) & 15.9(0.7) & 9801.8(273.1) \\
% % AGCRNModel & 7909.1(145.7) & 15.2(0.3) & 9427.3(190.6) \\
% % TransformerModel & 8827.3(369.0) & 17.3(0.6) & 10340.2(427.3) \\
% % BiPartiteSTGraphModel & 6614.5(348.2) & 13.1(0.8) & 8068.0(341.1) \\
% % SameHour & 3388.1(0.0) & 6.8(0.0) & 4574.1(0.0) \\
% % GRUGCNModel & 9266.6(154.2) & 18.1(0.4) & 10752.9(172.3) \\
% % TGCNModel_2 & 9268.4(260.1) & 18.1(0.5) & 10684.3(283.6) \\
% % TGCNModel & 8631.9(202.0) & 16.6(0.4) & 10173.6(208.0) \\
% % GRUGCNModel & 9557.0(516.1) & 18.7(1.1) & 11048.1(483.4) \\
% % TGCNModel_2 & 9248.8(291.0) & 18.0(0.5) & 10691.4(334.8) \\
% % TGCNModel & 8826.0(359.2) & 17.1(0.7) & 10249.9(378.8) \\
% % GRUGCNModel & 9383.2(566.2) & 18.2(1.2) & 10926.0(580.0) \\
% % TGCNModel_2 & 9037.7(340.2) & 17.5(0.7) & 10538.1(368.4) \\
% % TGCNModel & 8878.0(540.0) & 17.1(1.1) & 10342.0(508.1) \\
% % GRUGCNModel & 9352.1(602.8) & 18.2(1.3) & 10874.4(588.7) \\
% % TGCNModel_2 & 9206.5(298.9) & 17.9(0.5) & 10751.0(358.4) \\
% % TGCNModel & 8957.3(507.8) & 17.2(1.0) & 10544.0(505.4) \\
% % \bottomrule
% % \end{tabular}

% %%%%  Full result at residential level fold 2 %%%%

% % \begin{tabular}{llll}
% % \toprule
% %  & MAE & MAPE & RMSE \\
% % \midrule
% % GraphWaveNetModel & 128.4(0.3) & 52.9(0.6) & 256.7(1.0) \\
% % RNNModel & 126.5(0.2) & 50.5(0.4) & 254.7(0.4) \\
% % VARModel & 167.6(0.1) & 106.7(0.2) & 283.9(0.4) \\
% % GatedGraphNetworkModel & 121.7(0.4) & 49.5(0.6) & 245.9(1.5) \\
% % AGCRNModel & 123.8(0.3) & 50.6(0.6) & 248.7(0.6) \\
% % TransformerModel & 127.9(0.5) & 52.1(0.4) & 255.4(0.5) \\
% % BiPartiteSTGraphModel & 121.6(0.1) & 49.4(0.5) & 246.0(0.8) \\
% % SameHour & 156.4(0.0) & 75.9(0.0) & 304.1(0.0) \\
% % GRUGCNModel & 125.3(0.2) & 50.0(0.2) & 251.4(0.5) \\
% % TGCNModel_2 & 125.9(0.5) & 49.7(0.6) & 253.7(1.1) \\
% % TGCNModel & 124.6(0.2) & 49.3(0.6) & 251.9(0.4) \\
% % GRUGCNModel & 125.3(0.2) & 51.1(0.5) & 251.0(0.7) \\
% % TGCNModel_2 & 126.1(0.2) & 50.6(0.6) & 253.6(0.6) \\
% % TGCNModel & 124.6(0.2) & 49.6(0.8) & 251.7(0.7) \\
% % GRUGCNModel & 125.1(0.2) & 50.3(0.6) & 250.6(0.6) \\
% % TGCNModel_2 & 125.7(0.3) & 50.0(0.4) & 253.0(0.4) \\
% % TGCNModel & 124.5(0.3) & 49.4(0.6) & 251.2(1.0) \\
% % GRUGCNModel & 126.4(0.2) & 51.3(0.5) & 253.9(0.4) \\
% % TGCNModel_2 & 126.9(0.5) & 50.7(0.6) & 256.1(1.1) \\
% % TGCNModel & 125.3(0.2) & 50.2(0.6) & 252.9(0.3) \\
% % \bottomrule
% % \end{tabular}

% %%%% Full result at aggregate level fold 2 %%%%

% % \begin{tabular}{llll}
% % \toprule
% %  & MAE & MAPE & RMSE \\
% % \midrule
% % GraphWaveNetModel & 13908.5(468.0) & 19.2(0.9) & 17605.2(432.4) \\
% % RNNModel & 13489.2(166.1) & 18.8(0.3) & 16970.4(167.8) \\
% % VARModel & 13730.5(252.8) & 18.0(0.4) & 18669.1(273.7) \\
% % GatedGraphNetworkModel & 12787.0(562.2) & 16.9(0.7) & 16790.6(715.5) \\
% % AGCRNModel & 13264.5(345.9) & 17.9(0.5) & 17339.1(344.5) \\
% % TransformerModel & 13342.9(73.2) & 18.6(0.2) & 16909.1(46.5) \\
% % BiPartiteSTGraphModel & 12774.9(389.0) & 17.1(0.6) & 16559.8(423.0) \\
% % SameHour & 5478.3(0.0) & 8.2(0.0) & 7958.2(0.0) \\
% % GRUGCNModel & 12818.5(221.2) & 17.9(0.3) & 16227.1(242.3) \\
% % TGCNModel_2 & 13930.1(332.2) & 19.4(0.4) & 17580.1(406.7) \\
% % TGCNModel & 13543.2(305.0) & 18.5(0.4) & 17219.4(341.7) \\
% % GRUGCNModel & 12597.9(456.4) & 17.3(0.7) & 16145.7(460.9) \\
% % TGCNModel_2 & 13718.9(318.1) & 19.0(0.4) & 17420.9(366.8) \\
% % TGCNModel & 13316.4(412.7) & 18.2(0.7) & 17057.7(439.2) \\
% % GRUGCNModel & 12706.8(310.7) & 17.7(0.4) & 16171.3(404.7) \\
% % TGCNModel_2 & 13711.6(89.9) & 19.1(0.2) & 17388.7(65.8) \\
% % TGCNModel & 13170.4(543.1) & 18.2(0.9) & 16869.4(528.3) \\
% % GRUGCNModel & 13079.5(256.7) & 18.0(0.6) & 16625.9(194.4) \\
% % TGCNModel_2 & 13878.5(415.5) & 19.2(0.6) & 17674.4(444.5) \\
% % TGCNModel & 13150.4(144.0) & 18.2(0.2) & 16870.8(215.1) \\
% % \bottomrule
% % \end{tabular}

% %%%% Full result at residential level fold 3 %%%%

% % \begin{tabular}{llll}
% % \toprule
% %  & MAE & MAPE & RMSE \\
% % \midrule
% % GraphWaveNetModel & 161.6(12.4) & 64.2(8.4) & 315.2(21.5) \\
% % RNNModel & 153.1(1.7) & 56.9(2.2) & 302.2(0.9) \\
% % VARModel & 198.5(23.1) & 119.1(31.3) & 328.4(13.2) \\
% % GatedGraphNetworkModel & 159.5(25.4) & 71.5(31.8) & 299.2(18.0) \\
% % AGCRNModel & 149.4(1.5) & 58.2(1.0) & 293.6(1.6) \\
% % TransformerModel & 153.3(1.7) & 57.1(0.6) & 301.7(3.3) \\
% % BiPartiteSTGraphModel & 149.2(2.6) & 55.1(1.2) & 295.2(3.9) \\
% % SameHour & 178.7(15.3) & 75.5(11.3) & 345.6(25.2) \\
% % GRUGCNModel & 137.5(24.7) & 53.1(5.0) & 275.0(41.4) \\
% % TGCNModel_2 & 138.7(25.3) & 54.1(5.5) & 277.5(42.7) \\
% % TGCNModel & 136.8(24.3) & 52.7(4.7) & 274.9(41.4) \\
% % GRUGCNModel & 146.0(7.7) & 61.2(11.7) & 286.0(19.7) \\
% % TGCNModel_2 & 138.4(25.0) & 54.3(5.6) & 276.6(41.4) \\
% % TGCNModel & 145.6(7.3) & 61.9(11.4) & 285.8(19.4) \\
% % GRUGCNModel & 144.8(9.8) & 54.9(2.1) & 286.6(18.0) \\
% % TGCNModel_2 & 145.8(10.3) & 55.3(2.5) & 288.6(18.6) \\
% % TGCNModel & 144.1(9.9) & 54.3(2.8) & 286.9(17.7) \\
% % GRUGCNModel & 147.0(7.4) & 62.0(11.5) & 288.0(19.6) \\
% % TGCNModel_2 & 139.4(24.6) & 54.8(4.9) & 277.9(41.4) \\
% % TGCNModel & 146.0(7.3) & 62.2(11.6) & 285.9(19.8) \\
% % \bottomrule
% % \end{tabular}

% %%%% Full result at aggregate level fold 3 %%%%

% % \begin{tabular}{llll}
% % \toprule
% %  & MAE & MAPE & RMSE \\
% % \midrule
% % GraphWaveNetModel & 13206.6(3641.0) & 16.0(4.2) & 16387.4(3990.9) \\
% % RNNModel & 15072.7(658.4) & 17.9(1.0) & 18575.3(444.6) \\
% % VARModel & 10786.8(2450.3) & 12.9(2.8) & 14080.6(2493.0) \\
% % GatedGraphNetworkModel & 12738.7(1743.4) & 15.1(2.0) & 15780.6(1647.5) \\
% % AGCRNModel & 13240.2(297.0) & 15.5(0.3) & 16633.9(331.3) \\
% % TransformerModel & 15338.8(779.1) & 18.0(1.0) & 18905.6(797.1) \\
% % BiPartiteSTGraphModel & 14887.7(303.6) & 17.1(0.6) & 18457.6(249.5) \\
% % SameHour & 7952.3(3969.0) & 9.7(4.2) & 10685.3(4430.2) \\
% % GRUGCNModel & 13008.0(2169.5) & 16.8(0.5) & 15920.5(2871.2) \\
% % TGCNModel_2 & 13363.0(2345.8) & 17.3(0.4) & 16298.3(3070.1) \\
% % TGCNModel & 13233.9(2278.9) & 17.0(0.4) & 16194.4(3001.0) \\
% % GRUGCNModel & 12902.1(3312.5) & 16.2(2.3) & 16046.6(3534.7) \\
% % TGCNModel_2 & 13148.6(2093.8) & 17.1(0.8) & 16039.3(2704.0) \\
% % TGCNModel & 12609.8(2896.0) & 15.9(1.7) & 15795.3(3061.1) \\
% % GRUGCNModel & 13926.1(830.7) & 16.8(0.3) & 17338.3(848.9) \\
% % TGCNModel_2 & 14468.6(870.3) & 17.7(0.8) & 17869.9(815.4) \\
% % TGCNModel & 14194.2(330.0) & 17.3(0.8) & 17629.4(290.5) \\
% % GRUGCNModel & 12458.2(2856.6) & 15.9(1.6) & 15504.1(3087.4) \\
% % TGCNModel_2 & 12943.2(1865.4) & 16.9(0.5) & 15717.3(2405.7) \\
% % TGCNModel & 12038.8(2796.1) & 15.4(1.6) & 15002.0(3033.4) \\
% % \bottomrule
% % \end{tabular}

% %%%%%%


% Based on the results in \autoref{tab:graph_formation}, we conclude that different graph construction methods do not significantly impact the effectiveness of learning, even though the resulting topologies may vary. \acrshort{stgnn}s, as a data-focused method, adapt by learning with different topologies to achieve comparable outcomes. In the following section, for each predefined-graph model, we employ the graph formation technique that results in the minimal MAE. 
% \subsubsection{Performance Comparison with different temporal scale}\label{subsubsec:performance_comparison}

% \begin{table*}[h!]
% \centering
% \captionsetup{justification=centering,margin=2cm}
% \begingroup
% \setlength{\tabcolsep}{4pt} % Default value: 6pt
% \renewcommand{\arraystretch}{1.2} % Default value: 1
% \begin{tabular}{cllllllllll}
% \hline
% \hline
% \multirow{3}{*}{Group} & \multirow{3}{*}{Models} & \multicolumn{9}{c}{Metrics} \\ \cline{3-11}
%                        &                         & \multicolumn{3}{c}{Fold 1}  & \multicolumn{3}{c}{Fold 2} & \multicolumn{3}{c}{Fold 3} \\ \cline{3-11}
%                        &                         & MAE (Wh)    & MAPE (\%)   & RMSE (Wh)   & MAE (Wh)    & MAPE (\%)    & RMSE (Wh) & MAE (Wh)    & MAPE (\%)   & RMSE (Wh) \\ 
% \hline
% \multirow{4}{*}{Benchmark} 
% & SeasonalNaive                              & 114.4(0.0) & 66.4(0.0) & 238.6(0.0) & 156.4(0.0) & 75.9(0.0) & 304.1(0.0) & 186.3(0.0) & 81.1(0.0) & 358.2(0.0) \\
% & VAR                                         & 134.3(0.2) & 109.5(0.5) & 218.5(0.3) & 167.6(0.1) & 106.7(0.2) & 283.9(0.4) & 210.1(0.5) & 134.7(0.5) & 335.0(0.6)      \\
% & Transformer                                  &    90.2(0.3) & 44.9(0.8) & 194.5(0.4) & 127.9(0.5) & 52.1(0.4) & 255.4(0.5) & 154.1(0.0) & 56.7(0.1) & 303.4(0.4)      \\
% \hline
% \multirow{8}{*}{STGNN}  & GRUGCN                                         & \textbf{89.0(0.2)} & \underline{\textbf{43.0(1.1)}} & \textbf{193.7(0.6)}	     & \textbf{125.3(0.2)} & \textbf{50.0(0.2)} & \textbf{251.4(0.5)}       & \textbf{149.7(0.1)} & 55.9(0.3) & \textbf{295.6(0.6)   }    \\
% & GCGRU                                      & \underline{\textbf{88.2(0.2)}} & \textbf{43.6(0.4)} & \textbf{191.9(0.2)}     & \textbf{124.6(0.2)} & \underline{\textbf{49.3(0.6)}} & \textbf{251.9(0.4)}   &    \textbf{149.0(0.4)} & \textbf{55.0(0.4)} & \textbf{295.7(0.6)}  \\
% & T-GCN                                     & \textbf{88.9(0.5)} & \textbf{43.8(0.1)} & \textbf{193.5(1.0)}   &   \textbf{125.0(0.5)}      &   \textbf{49.7(0.6)}      &     \textbf{253.7(1.1)}    & \textbf{150.9(0.5)} & 56.5(0.5) & \textbf{297.9(1.3)}  \\
% & AGCRN                                     & 91.0(0.2) & 49.1(0.4) & \textbf{190.9(0.6)} & \textbf{123.8(0.3)} & 50.6(0.6) & \textbf{248.7(0.6)} & \textbf{150.1(0.3)} & 58.8(0.2) & \textbf{294.2(1.1)}       \\
% & Bipartite                                   &   90.2(0.1) & 48.8(0.5) & \underline{\textbf{188.4(0.3)}} & \underline{\textbf{121.6(0.1)}} & \textbf{49.4(0.5)} & \textbf{246.0(0.8)} & \textbf{148.0(0.1)} & \underline{\textbf{54.7(0.5)}} & \textbf{293.3(0.9) }     \\
% & FC-GNN                            &   \textbf{88.6(0.1)} & 45.7(0.6) & \textbf{189.0(0.3)} & \textbf{121.7(0.4)} & \textbf{49.5(0.6)} & \underline{\textbf{245.9(1.5)}} & \underline{\textbf{146.9(0.1)}} & \textbf{55.6(0.6)} & \underline{\textbf{290.3(1.0)}}     \\
% & GraphWavenet   & 91.4(0.2) & 48.1(0.8) & 196.7(0.8) & 128.4(0.3) & 52.9(0.6) & 256.7(1.0) & 155.9(0.5) & 60.1(0.5) & 304.6(0.8)  \\     
% \hline
% \hline
% \end{tabular}
% \endgroup
% \caption{Performance of different \acrshort{stgnn} models with different experiment settings. The values in \textbf{bold} signify that the performance of \acrshort{stgnn} models is better than all benchmark models. The \underline{underlined values} means the best performance by error metrics.
% }
% \label{tab:result_228}
% \end{table*}

% \begin{figure}[h!]
%     \centering
%     \includegraphics[width=\linewidth]{assets/GNN_splits.pdf}
%     \caption{Train-validation-test split settings}
%     \label{fig:cross-val}
% \end{figure}


% Compared to deep learning models that process only temporal features such as \acrshort{gru} and \acrshort{tfm}, the advantage of \acrshort{stgnn} is evident but varies between time periods. Compared to GRU, which uses matrix multiplication as the unit cell of the recurrent network, GCGRU and T-GCN always achieve better performance by applying GCN as the unit cell. Similarly, GRUGCN also outperforms \acrshort{gru} by applying \acrshort{gcn} on top of it to account for spatial dependency. This showcases the effective use of graph neural networks to model spatial relationships (see Table \ref{tab:result_228}).

% % Comparing GCGRU, T-GCN which use GCN as spatial processing unit to embed node features with RNN which simply use matrix multiplication, we see that there is a small advantage when integrating spatial processing unit. That demonstrates relevance in applying the graph neural network to model spatial relationships.

% For models using a learnable graph, this technique usually gives better results compared to the benchmark models except for GraphWavenet. However, comparing AGCRN, which uses a learnable graph, with GCGRU, which uses a predefined graph from signals, there is a downgrade in performance in some metrics. This is because the learnable graph offers a more flexible way to model the spatial relationship. However, it could make the forecasting task more susceptible to overfitting. When testing on a more distant future (one month after training as outlined in the Fig. \ref{fig:cross-val}), the model actually performs worse than the counterpart that uses a predefined graph. 
% % A special case of this type is that model \acrshort{stegnn} does not converge at the end. The model employs a mechanism to incorporate a learnable graph into their structure. To reduce the computational cost, for each node, this mechanism only keeps top k of the most influential neighbors and makes the rest of the adjacency matrix 0. This mechanism introduces the problem of a sparse gradient in training~\cite{cini_sparse_nodate}, where the gradient becomes zero, making learning difficult. On the other hand, in our experiment settings, the model takes one day before to forecast 1 day ahead. This makes the training data limited in size. By using a small number of training data, the model might not be able to overcome the problem and end up giving a substandard result. 

% In terms of the performance, the results are not always consistent. The bipartite model and FC-GNN model often perform better than their counterparts. These models utilize a weighted sum in the AGGREGATE step, where the weights are derived from the embedded input of each node. This makes the AGGREGATE step more adaptable to the input. An interesting case is the bipartite model. Although it performs slightly worse than FC-GNN, it is more scalable since the interaction between the nodes of FC-GNN is $N^2$, while for the bipartite model, it is only $2KN$ with $K$ being the number of virtual nodes. It also outperforms other models most of the time. One reason might be that, due to the nature of the load profile dataset, the consumption pattern of users is grouped by latent factors such as socio-demographic status~\cite{acorn}. By defining virtual nodes, this model can account for latent factors that are related to original nodes. The bipartite topology allows these virtual nodes to gather information in a cluster-like manner; then, the aggregated information is passed down to each node as additional information for learning.


% % In terms of forecasting result with different periods, we notice that the magnitude of error differs vastly among them, indicating that the nature of data and testing period strongly affect the performance. It suggests that learning from historical data alone is insufficient to capture the dynamics of energy consumption. Incorporating temporal indicators, such as weekends or holidays, could enhance the model's ability to capture these contextual variations more effectively.



% \subsubsection{Forecast at the aggregate level}
% We investigate the performance of load forecasting at the aggregate level simply by aggregating all the forecasts at the residential level (see Table \ref{tab:result_agg_228}).
% % \begin{table}[h!]
% % \centering
% % \captionsetup{justification=centering,margin=1cm}
% % \begingroup
% % \setlength{\tabcolsep}{5pt} % Default value: 6pt
% % \renewcommand{\arraystretch}{1.2} % Default value: 1
% % \begin{tabular}{l|lll}
% % \hline
% % \hline
% % \multicolumn{1}{c|}{\multirow{3}{*}{Models}} & \multicolumn{3}{c}{Metrics} \\ \cline{2-4}
% % \multicolumn{1}{c|}{}  & \multicolumn{3}{c}{Average (over three folds)} \\ \cline{2-4}
% % \multicolumn{1}{c|}{}                        & MAE (kWh)     & MAPE (\%)   & RMSE   \\ 
% % \hline
% % SeasonalNaive                               &   4.495      &  0.075   &  7.00   \\
% % \hline
% % VAR                                         &  9.450      &  0.122      &  11.946 \\
% % \hline
% % GRU                                         &  12.267    &  0.178     &  15.024      \\
% % \hline
% % Tranformer                                  &   12.508     &  0.179    &  15.375          \\
% % \hline
% % GRUGCN                                      & 12.115	      & 0.176   &  14.880	       \\
% % \hline
% % % GCLSTM                                      &    0.139     &   1.22      &  0.239    \\
% % % \hline
% % % T-GCN                                       &  \textbf{0.089 }      &    \textbf{0.444 }    &   0.194    \\
% % % \hline
% % % AGCRNN                                      &   \textbf{0.090 }     &     0.484    &  \textbf{ 0.191 }       \\
% % % \hline
% % % STEGNN                                      &  3.067     &  34.660     &   3.866    \\
% % % \hline
% % TGCN                                        & 11.879       &  0173     &  14.584    \\
% % \hline
% % Bipartite                                   &   11.421     &  0.158    &  14.414   \\
% % \hline
% % % Fully Connected                             &         &         &      \\
% % % \hline
% % % GraphWavenet                                &   0.091      &   0.494      &    0.196    \\     
% % \hline
% % \end{tabular}
% % \endgroup
% % \caption{Performance of different \acrshort{stgnn} models at aggregate level.
% % }
% % \label{tab:result_228_agg}
% % \end{table}


% \begin{table*}[hb!]
% \centering
% \captionsetup{justification=centering,margin=2cm}
% \begingroup
% \setlength{\tabcolsep}{3pt} % Default value: 6pt
% \renewcommand{\arraystretch}{1.2} % Default value: 1
% \begin{tabular}{cllllllllll}
% \hline
% \hline
% \multirow{3}{*}{Group} & \multirow{3}{*}{Models} & \multicolumn{9}{c}{Metrics} \\ \cline{3-11}
%                        &                         & \multicolumn{3}{c}{Fold 1}  & \multicolumn{3}{c}{Fold 2} & \multicolumn{3}{c}{Fold 3} \\ \cline{3-11}
%                        &                         & MAE (kWh)    & MAPE (\%)   & RMSE (kWh)   & MAE (kWh)    & MAPE (\%)    & RMSE (kWh) & MAE (kWh)    & MAPE (\%)   & RMSE (kWh) \\ 
% \hline
% \multirow{4}{*}{Benchmark} 
% & SeasonalNaive                             & \underline{3.39(0.0)} & \underline{6.8(0.0)} & \underline{4.57(0.0)} & \underline{5.48(0.0)} & \underline{8.2(0.0)} & \underline{7.96(0.0)} & \underline{5.97(0.0)} & \underline{7.6(0.0)} & \underline{8.47(0.0)} \\
% & VAR                                         & 3.44(0.23) & 7.1(0.05) & 4.56(0.04) & 13.73(0.25) & 18.0(0.4) & 18.67(0.27) & 9.56(0.19) & 11.5(0.2) & 12.84(0.24) \\
% & GRU  & 8.58(0.28) & 16.9(0.6) & 10.08(0.29) & 13.49(0.17) & 18.8(0.3) & 16.97(0.17) & 15.35(0.39) & 18.2(0.7) & 18.76(0.29) \\
% & Transformer                                 & 8.83(0.37) & 17.3(0.6) & 10.34(0.43) & 13.34(0.73) & 18.6(0.2) & 16.91(0.05) & 15.72(0.17) & 18.5(0.2) & 19.30(0.12) \\
% \hline
% \multirow{8}{*}{STGNN} 
% & GRUGCN                                      & 9.38(0.57) & 18.2(1.2) & 10.93(0.58) & 12.71(0.31) & 17.7(0.4) & 16.17(0.4) & 14.31(0.37) & 16.7(0.3) & 17.71(0.44) \\
% & GCGRU                                       & 8.88(0.54) & 17.1(1.1) & 10.34(0.51) & 13.17(0.54) & 18.2(0.90) & 16.87(0.53) & 14.34(0.17) & 16.9(0.3) & 17.74(0.20) \\
% & T-GCN                                       & 9.27(0.26) & 18.1(0.5) & 10.68(0.28) & 13.93(0.33) & 19.4(0.4) & 17.58(0.41) & 14.52(0.44) & 17.4(0.4) & 17.81(0.55) \\
% & AGCRN                                     & 7.91(0.15) & 15.2(0.3) & 9.43(0.19) & 13.26(0.35) & 17.9(0.5) & 17.34(0.34) & 13.16(0.28) & 15.3(0.2) & 16.66(0.36) \\
% & Bipartite                                   & 6.61(0.35) & 13.1(0.8) & 8.07(0.34) & 12.77(0.39) & 17.1(0.6) & 16.56(0.42) & 14.81(0.29) & 16.9(0.5) & 18.40(0.24) \\
% & FC-GNN                             & 8.22(0.32) & 15.9(0.7) & 9.80(0.27) & 12.79(0.56) & 16.9(0.7) & 16.79(0.72) & 13607.2(165.1) & 16.0(0.4) & 16.60(0.14) \\
% & GraphWavenet   & 8.85(0.44) & 17.4(1.0) & 10.34(0.44) & 13.91(0.47) & 19.2(0.9) & 17.61(0.43) & 15.02(0.44) & 18.1(0.5) & 18.37(0.57) \\     
% \hline
% \hline
% \end{tabular}
% \endgroup
% \caption{Performance of different \acrshort{stgnn} models at the \textbf{aggregate level}. For learnable-graph based model, the average performance is selected. The \underline{underlined values}
% means the best performance by error metrics. 
% }
% \label{tab:result_agg_228}
% \end{table*}

% At the aggregate level, the forecast results obtained through aggregation are inferior to those of the baseline model (SeasonalNaive). This behavior is also observed in other deep learning models such as \acrshort{gru} and \acrshort{tfm}. An explanation is that these models tend to prioritize learning from "easier" periods with consistent patterns while failing to adequately capture atypical events, such as consumption spikes~\cite{zhang_unlocking_2023} (see Fig. \ref{fig:multiple-forecast}). This can lead to underestimation during abnormal periods. At the aggregate level, instead of centering predictions around the actual average consumption, the forecasts may consistently underestimate the consumption spikes, and hence the errors are not balanced out. \acrshort{stgnn}s, by incorporating spatial learning, potentially propagate errors throughout the spatial dimension and therefore do not address this issue.

% \begin{figure}[h!]
%     \centering
%     \includegraphics[width=\linewidth]{assets/MAC000026.pdf}
%     \caption{Load forecast of smart meter MAC000026 in 4 days. We observe that all deep learning models underestimate the spike in their forecasts.}
%     \label{fig:multiple-forecast}
% \end{figure}


% % \begin{table}[h!]
% % \centering
% % \caption{Cross-validation settings for validating \acrshort{stgnn} models.}
% % \begin{tabularx}{\columnwidth}{|X|X|X|X|}
% % \hline
% % \textbf{Phase} & \textbf{Training} & \textbf{Validation} & \textbf{Testing} \\ \hline
% % \textbf{Fold 1} & Jan 1, 2013 -- Jun 30, 2013 & Jul 1, 2013 -- Jul 31, 2013 & Aug 1, 2013 -- Aug 31, 2013 \\ \hline
% % \textbf{Fold 2} & Jan 1, 2013 -- Aug 31, 2013 & Sep 1, 2013 -- Sep 30, 2013 & Oct 1, 2013 -- Oct 31, 2013 \\ \hline
% % \textbf{Fold 3} & Jan 1, 2013 -- Oct 31, 2013 & Nov 1, 2013 -- Nov 30, 2013 & Dec 1, 2013 -- Dec 31, 2013 \\ \hline
% % \end{tabularx}
% % \label{tab:cross_validation}
% % \end{table}


% % \begin{table}[h!]
% % \centering
% % \caption{Cross-validation settings for validating \acrshort{stgnn} models.}
% % \begin{tabular}{|c|c|c|}
% % \hline
% % \textbf{Phase}         & \textbf{Time Period} \\ \hline
% % \textbf{Fold 1} & \begin{tabular}[c]{@{}l@{}}Training: Jan 1, 2013 -- Jun 30, 2013 \\ 
% % Validation: Jul 1, 2013 -- Jul 31, 2013 \\ 
% % Testing: Aug 1, 2013 -- Aug 31, 2013\end{tabular} \\ \hline
% % \textbf{Fold 2} & \begin{tabular}[c]{@{}l@{}}Training: Jan 1, 2013 -- Aug 31, 2013 \\ 
% % Validation: Sep 1, 2013 -- Sep 30, 2013 \\ 
% % Testing: Oct 1, 2013 -- Oct 31, 2013\end{tabular} \\ \hline
% % \textbf{Fold 3} & \begin{tabular}[c]{@{}l@{}}Training: Jan 1, 2013 -- Oct 31, 2013 \\ 
% % Validation: Nov 1, 2013 -- Nov 30, 2013 \\ 
% % Testing: Dec 1, 2013 -- Dec 31, 2013\end{tabular} \\ \hline
% % \end{tabular}
% % \label{tab:cross_validation}
% % \end{table}




\section{Results and discussion}

In each experiment, we perform the procedure 5 times using identical parameters to calculate the statistics of the experiment. The results will be displayed as the mean, followed by the standard deviation from the 5 trials. To enhance the presentation of the results, we use \textbf{bold font} to highlight instances where the performance of \acrshort{stgnn} models surpasses all benchmark models, and \underline{underlining} to indicate the best performance according to the error metrics.


\subsection{Comparison between graph formation methods}\label{subsubsec:graph_formation}
Methods for graph formation can produce various topologies that facilitate the aggregation of information. This section aims to explore whether creating graphs based on different types of similarities between time series impacts forecasting performance. The focus is exclusively on the 3 first models in Table \ref{tab:overview} which require signal-based predefined graphs.
\begin{table}[h!]
\centering
\caption{Performance of models which requires predefined graph 
\newline
(Table \ref{tab:overview}) in Split 3.}
\begin{tabular}{@{}lllll@{}}
                          &     &     & Metrics  &    \\
\multirow{-2}{*}{Model}   & \multirow{-2}{*}{Graph  formation} & MAE (Wh) & MAPE (\%)    & RMSE (Wh)                                                                                        \\
\hline
                          & Euclidean        & \underline{149.7(0.1)} & 55.9(0.3) & \underline{295.6(0.6)   }\\
                          & DTW              & 149.8(0.3) & \underline{55.4(1.1)} & 295.8(1.0) \\
                          & Correntropy      & 149.8(0.4) & 55.5(0.6) & 295.7(0.7)   \\
\multirow{-4}{*}{GRUGCN}                          & Pearson          & 150.7(0.1) & 56.2(0.4) & 297.8(1.0)   \\
  % & Transfer entropy &  0.151 & 0.556  & 0.299 \\
\hline
                          & Euclidean        & \underline{150.9(0.5)} & \underline{56.5(0.5)} & 297.9(1.3)   \\
                          & DTW              & \underline{150.9(0.4)} & 57.1(1.0) & \underline{297.3(1.3)}    \\
                          & Correntropy      & 151.4(0.6) & 56.8(0.6) & 298.8(0.7)  \\
\multirow{-4}{*}{T-GCN}                           & Pearson          & 151.7(0.3) & 57.2(0.5) & 298.6(0.4) \\
  % & Transfer entropy & 0.149    & 0.573   & 0.296   \\
\hline
                          & Euclidean        & 149.1(0.1) & 55.7(0.4) & 295.7(0.9) \\
                          & DTW              & 149.2(0.5) & 56.2(0.6) & \underline{295.5(0.8)}     \\
                          & Correntropy      & \underline{149.0(0.4)} & \underline{55.0(0.4)} & 295.7(0.6) \\
\multirow{-4}{*}{GCGRU}                          & Pearson          & 149.6(0.2) & 56.4(0.2) & 295.8(0.6)  \\
 % & Transfer entropy & 0.235  & 1.224  & 0.428 \\
\hline
\end{tabular}
\label{tab:graph_formation}
\end{table}


%%%%%%

%%%% Full result at residential level fold 1 %%%%

% \begin{tabular}{llll}
% \toprule
%  & MAE & MAPE & RMSE \\
% \midrule
% GraphWaveNetModel & 91.4(0.2) & 48.3(0.8) & 196.7(0.7) \\
% RNNModel & 89.5(0.1) & 44.7(0.3) & 194.0(0.5) \\
% VARModel & 134.3(0.2) & 109.4(0.4) & 218.5(0.3) \\
% GatedGraphNetworkModel & 88.6(0.1) & 45.7(0.6) & 188.8(0.5) \\
% AGCRNModel & 90.9(0.2) & 49.0(0.4) & 190.8(0.6) \\
% TransformerModel & 90.4(0.4) & 45.1(0.9) & 194.5(0.4) \\
% BiPartiteSTGraphModel & 90.2(0.1) & 49.2(0.9) & 188.3(0.4) \\
% SameHour & 114.4(0.0) & 66.4(0.0) & 238.6(0.0) \\
% GRUGCNModel & 89.2(0.8) & 43.5(0.8) & 193.6(0.3) \\
% TGCNModel_2 & 88.8(0.3) & 43.3(0.8) & 193.4(0.1) \\
% TGCNModel & 88.2(0.2) & 43.6(0.4) & 191.9(0.2) \\
% GRUGCNModel & 89.1(0.6) & 43.3(1.0) & 194.1(0.6) \\
% TGCNModel_2 & 88.9(0.4) & 43.7(0.3) & 193.9(0.7) \\
% TGCNModel & 88.2(0.2) & 43.5(0.5) & 192.5(0.5) \\
% GRUGCNModel & 89.0(0.2) & 43.0(1.1) & 193.7(0.6) \\
% TGCNModel_2 & 88.9(0.5) & 43.8(0.1) & 193.5(1.0) \\
% TGCNModel & 88.4(0.2) & 43.5(0.9) & 192.5(0.6) \\
% GRUGCNModel & 89.4(0.2) & 43.8(0.9) & 194.6(0.7) \\
% TGCNModel_2 & 89.6(0.2) & 43.8(0.5) & 194.6(0.3) \\
% TGCNModel & 88.5(0.3) & 43.8(1.2) & 193.6(0.5) \\
% \bottomrule
% \end{tabular}

%%%% Full result at aggregate level fold 1 %%%%

% \begin{tabular}{llll}
% \toprule
%  & MAE & MAPE & RMSE \\
% \midrule
% GraphWaveNetModel & 8852.3(437.6) & 17.4(1.0) & 10336.5(440.2) \\
% RNNModel & 8582.6(276.8) & 16.9(0.6) & 10080.4(285.0) \\
% VARModel & 3439.8(22.7) & 7.1(0.0) & 4562.5(39.6) \\
% GatedGraphNetworkModel & 8216.2(317.4) & 15.9(0.7) & 9801.8(273.1) \\
% AGCRNModel & 7909.1(145.7) & 15.2(0.3) & 9427.3(190.6) \\
% TransformerModel & 8827.3(369.0) & 17.3(0.6) & 10340.2(427.3) \\
% BiPartiteSTGraphModel & 6614.5(348.2) & 13.1(0.8) & 8068.0(341.1) \\
% SameHour & 3388.1(0.0) & 6.8(0.0) & 4574.1(0.0) \\
% GRUGCNModel & 9266.6(154.2) & 18.1(0.4) & 10752.9(172.3) \\
% TGCNModel_2 & 9268.4(260.1) & 18.1(0.5) & 10684.3(283.6) \\
% TGCNModel & 8631.9(202.0) & 16.6(0.4) & 10173.6(208.0) \\
% GRUGCNModel & 9557.0(516.1) & 18.7(1.1) & 11048.1(483.4) \\
% TGCNModel_2 & 9248.8(291.0) & 18.0(0.5) & 10691.4(334.8) \\
% TGCNModel & 8826.0(359.2) & 17.1(0.7) & 10249.9(378.8) \\
% GRUGCNModel & 9383.2(566.2) & 18.2(1.2) & 10926.0(580.0) \\
% TGCNModel_2 & 9037.7(340.2) & 17.5(0.7) & 10538.1(368.4) \\
% TGCNModel & 8878.0(540.0) & 17.1(1.1) & 10342.0(508.1) \\
% GRUGCNModel & 9352.1(602.8) & 18.2(1.3) & 10874.4(588.7) \\
% TGCNModel_2 & 9206.5(298.9) & 17.9(0.5) & 10751.0(358.4) \\
% TGCNModel & 8957.3(507.8) & 17.2(1.0) & 10544.0(505.4) \\
% \bottomrule
% \end{tabular}

%%%%  Full result at residential level fold 2 %%%%

% \begin{tabular}{llll}
% \toprule
%  & MAE & MAPE & RMSE \\
% \midrule
% GraphWaveNetModel & 128.4(0.3) & 52.9(0.6) & 256.7(1.0) \\
% RNNModel & 126.5(0.2) & 50.5(0.4) & 254.7(0.4) \\
% VARModel & 167.6(0.1) & 106.7(0.2) & 283.9(0.4) \\
% GatedGraphNetworkModel & 121.7(0.4) & 49.5(0.6) & 245.9(1.5) \\
% AGCRNModel & 123.8(0.3) & 50.6(0.6) & 248.7(0.6) \\
% TransformerModel & 127.9(0.5) & 52.1(0.4) & 255.4(0.5) \\
% BiPartiteSTGraphModel & 121.6(0.1) & 49.4(0.5) & 246.0(0.8) \\
% SameHour & 156.4(0.0) & 75.9(0.0) & 304.1(0.0) \\
% GRUGCNModel & 125.3(0.2) & 50.0(0.2) & 251.4(0.5) \\
% TGCNModel_2 & 125.9(0.5) & 49.7(0.6) & 253.7(1.1) \\
% TGCNModel & 124.6(0.2) & 49.3(0.6) & 251.9(0.4) \\
% GRUGCNModel & 125.3(0.2) & 51.1(0.5) & 251.0(0.7) \\
% TGCNModel_2 & 126.1(0.2) & 50.6(0.6) & 253.6(0.6) \\
% TGCNModel & 124.6(0.2) & 49.6(0.8) & 251.7(0.7) \\
% GRUGCNModel & 125.1(0.2) & 50.3(0.6) & 250.6(0.6) \\
% TGCNModel_2 & 125.7(0.3) & 50.0(0.4) & 253.0(0.4) \\
% TGCNModel & 124.5(0.3) & 49.4(0.6) & 251.2(1.0) \\
% GRUGCNModel & 126.4(0.2) & 51.3(0.5) & 253.9(0.4) \\
% TGCNModel_2 & 126.9(0.5) & 50.7(0.6) & 256.1(1.1) \\
% TGCNModel & 125.3(0.2) & 50.2(0.6) & 252.9(0.3) \\
% \bottomrule
% \end{tabular}

%%%% Full result at aggregate level fold 2 %%%%

% \begin{tabular}{llll}
% \toprule
%  & MAE & MAPE & RMSE \\
% \midrule
% GraphWaveNetModel & 13908.5(468.0) & 19.2(0.9) & 17605.2(432.4) \\
% RNNModel & 13489.2(166.1) & 18.8(0.3) & 16970.4(167.8) \\
% VARModel & 13730.5(252.8) & 18.0(0.4) & 18669.1(273.7) \\
% GatedGraphNetworkModel & 12787.0(562.2) & 16.9(0.7) & 16790.6(715.5) \\
% AGCRNModel & 13264.5(345.9) & 17.9(0.5) & 17339.1(344.5) \\
% TransformerModel & 13342.9(73.2) & 18.6(0.2) & 16909.1(46.5) \\
% BiPartiteSTGraphModel & 12774.9(389.0) & 17.1(0.6) & 16559.8(423.0) \\
% SameHour & 5478.3(0.0) & 8.2(0.0) & 7958.2(0.0) \\
% GRUGCNModel & 12818.5(221.2) & 17.9(0.3) & 16227.1(242.3) \\
% TGCNModel_2 & 13930.1(332.2) & 19.4(0.4) & 17580.1(406.7) \\
% TGCNModel & 13543.2(305.0) & 18.5(0.4) & 17219.4(341.7) \\
% GRUGCNModel & 12597.9(456.4) & 17.3(0.7) & 16145.7(460.9) \\
% TGCNModel_2 & 13718.9(318.1) & 19.0(0.4) & 17420.9(366.8) \\
% TGCNModel & 13316.4(412.7) & 18.2(0.7) & 17057.7(439.2) \\
% GRUGCNModel & 12706.8(310.7) & 17.7(0.4) & 16171.3(404.7) \\
% TGCNModel_2 & 13711.6(89.9) & 19.1(0.2) & 17388.7(65.8) \\
% TGCNModel & 13170.4(543.1) & 18.2(0.9) & 16869.4(528.3) \\
% GRUGCNModel & 13079.5(256.7) & 18.0(0.6) & 16625.9(194.4) \\
% TGCNModel_2 & 13878.5(415.5) & 19.2(0.6) & 17674.4(444.5) \\
% TGCNModel & 13150.4(144.0) & 18.2(0.2) & 16870.8(215.1) \\
% \bottomrule
% \end{tabular}

%%%% Full result at residential level fold 3 %%%%

% \begin{tabular}{llll}
% \toprule
%  & MAE & MAPE & RMSE \\
% \midrule
% GraphWaveNetModel & 161.6(12.4) & 64.2(8.4) & 315.2(21.5) \\
% RNNModel & 153.1(1.7) & 56.9(2.2) & 302.2(0.9) \\
% VARModel & 198.5(23.1) & 119.1(31.3) & 328.4(13.2) \\
% GatedGraphNetworkModel & 159.5(25.4) & 71.5(31.8) & 299.2(18.0) \\
% AGCRNModel & 149.4(1.5) & 58.2(1.0) & 293.6(1.6) \\
% TransformerModel & 153.3(1.7) & 57.1(0.6) & 301.7(3.3) \\
% BiPartiteSTGraphModel & 149.2(2.6) & 55.1(1.2) & 295.2(3.9) \\
% SameHour & 178.7(15.3) & 75.5(11.3) & 345.6(25.2) \\
% GRUGCNModel & 137.5(24.7) & 53.1(5.0) & 275.0(41.4) \\
% TGCNModel_2 & 138.7(25.3) & 54.1(5.5) & 277.5(42.7) \\
% TGCNModel & 136.8(24.3) & 52.7(4.7) & 274.9(41.4) \\
% GRUGCNModel & 146.0(7.7) & 61.2(11.7) & 286.0(19.7) \\
% TGCNModel_2 & 138.4(25.0) & 54.3(5.6) & 276.6(41.4) \\
% TGCNModel & 145.6(7.3) & 61.9(11.4) & 285.8(19.4) \\
% GRUGCNModel & 144.8(9.8) & 54.9(2.1) & 286.6(18.0) \\
% TGCNModel_2 & 145.8(10.3) & 55.3(2.5) & 288.6(18.6) \\
% TGCNModel & 144.1(9.9) & 54.3(2.8) & 286.9(17.7) \\
% GRUGCNModel & 147.0(7.4) & 62.0(11.5) & 288.0(19.6) \\
% TGCNModel_2 & 139.4(24.6) & 54.8(4.9) & 277.9(41.4) \\
% TGCNModel & 146.0(7.3) & 62.2(11.6) & 285.9(19.8) \\
% \bottomrule
% \end{tabular}

%%%% Full result at aggregate level fold 3 %%%%

% \begin{tabular}{llll}
% \toprule
%  & MAE & MAPE & RMSE \\
% \midrule
% GraphWaveNetModel & 13206.6(3641.0) & 16.0(4.2) & 16387.4(3990.9) \\
% RNNModel & 15072.7(658.4) & 17.9(1.0) & 18575.3(444.6) \\
% VARModel & 10786.8(2450.3) & 12.9(2.8) & 14080.6(2493.0) \\
% GatedGraphNetworkModel & 12738.7(1743.4) & 15.1(2.0) & 15780.6(1647.5) \\
% AGCRNModel & 13240.2(297.0) & 15.5(0.3) & 16633.9(331.3) \\
% TransformerModel & 15338.8(779.1) & 18.0(1.0) & 18905.6(797.1) \\
% BiPartiteSTGraphModel & 14887.7(303.6) & 17.1(0.6) & 18457.6(249.5) \\
% SameHour & 7952.3(3969.0) & 9.7(4.2) & 10685.3(4430.2) \\
% GRUGCNModel & 13008.0(2169.5) & 16.8(0.5) & 15920.5(2871.2) \\
% TGCNModel_2 & 13363.0(2345.8) & 17.3(0.4) & 16298.3(3070.1) \\
% TGCNModel & 13233.9(2278.9) & 17.0(0.4) & 16194.4(3001.0) \\
% GRUGCNModel & 12902.1(3312.5) & 16.2(2.3) & 16046.6(3534.7) \\
% TGCNModel_2 & 13148.6(2093.8) & 17.1(0.8) & 16039.3(2704.0) \\
% TGCNModel & 12609.8(2896.0) & 15.9(1.7) & 15795.3(3061.1) \\
% GRUGCNModel & 13926.1(830.7) & 16.8(0.3) & 17338.3(848.9) \\
% TGCNModel_2 & 14468.6(870.3) & 17.7(0.8) & 17869.9(815.4) \\
% TGCNModel & 14194.2(330.0) & 17.3(0.8) & 17629.4(290.5) \\
% GRUGCNModel & 12458.2(2856.6) & 15.9(1.6) & 15504.1(3087.4) \\
% TGCNModel_2 & 12943.2(1865.4) & 16.9(0.5) & 15717.3(2405.7) \\
% TGCNModel & 12038.8(2796.1) & 15.4(1.6) & 15002.0(3033.4) \\
% \bottomrule
% \end{tabular}

%%%%%%


The results in \autoref{tab:graph_formation} suggest that graph construction methods do not significantly impact the effectiveness of learning, even though the resulting topologies may vary. \glspl{stgnn}, as a data-focused method, adapts by learning with different topologies to achieve comparable outcomes. In the following section, for each predefined-graph model, we employ the graph formation technique that results in the minimal MAE. 
\subsection{Model benchmark with different temporal scale at the residential level}\label{subsubsec:performance_comparison}

\begin{table*}[htbp]
\centering
\caption{Performance of different \acrshort{stgnn} models at the \textbf{residential} level. 
}
\begingroup
\setlength{\tabcolsep}{4pt} % Default value: 6pt
\renewcommand{\arraystretch}{1.2} % Default value: 1
\resizebox{\textwidth}{!}{
\begin{tabular}{cllllllllll}
\hline
\hline
\multirow{3}{*}{Group} & \multirow{3}{*}{Models} & \multicolumn{9}{c}{Metrics} \\ \cline{3-11}
                       &                         & \multicolumn{3}{c}{Split 1}  & \multicolumn{3}{c}{Split 2} & \multicolumn{3}{c}{Split 3} \\ \cline{3-11}
                       &                         & MAE (Wh)    & MAPE (\%)   & RMSE (Wh)   & MAE (Wh)    & MAPE (\%)    & RMSE (Wh) & MAE (Wh)    & MAPE (\%)   & RMSE (Wh) \\ 
\hline
\multirow{4}{*}{Benchmark} 
& SeasonalNaive                              & 114.4(0.0) & 66.4(0.0) & 238.6(0.0) & 156.4(0.0) & 75.9(0.0) & 304.1(0.0) & 186.3(0.0) & 81.1(0.0) & 358.2(0.0) \\
& VAR                                         & 134.3(0.2) & 109.5(0.5) & 218.5(0.3) & 167.6(0.1) & 106.7(0.2) & 283.9(0.4) & 210.1(0.5) & 134.7(0.5) & 335.0(0.6)      \\
& GRU &
89.5(0.1) & 44.7(0.3) & 194.0(0.5) & 126.5(0.2) & 50.5(0.4) & 254.7(0.4) & 153.1(1.7) & 56.9(2.2) & 302.2(0.9) \\
& Transformer                                  &    90.2(0.3) & 44.9(0.8) & 194.5(0.4) & 127.9(0.5) & 52.1(0.4) & 255.4(0.5) & 154.1(0.1) & 56.7(0.1) & 303.4(0.4)      \\
\hline
\multirow{8}{*}{STGNN}  & GRUGCN                                         & \textbf{89.0(0.2)} & \underline{\textbf{43.0(1.1)}} & \textbf{193.7(0.6)}	     & \textbf{125.3(0.2)} & \textbf{50.0(0.2)} & \textbf{251.4(0.5)}       & \textbf{149.7(0.1)} & \textbf{55.9(0.3)} & \textbf{295.6(0.6)   }    \\
& GCGRU                                      & \underline{\textbf{88.2(0.2)}} & \textbf{43.6(0.4)} & \textbf{191.9(0.2)}     & \textbf{124.6(0.2)} & \underline{\textbf{49.3(0.6)}} & \textbf{251.9(0.4)}   &    \textbf{149.0(0.4)} & \textbf{55.0(0.4)} & \textbf{295.7(0.6)}  \\
& T-GCN                                     & \textbf{88.9(0.5)} & \textbf{43.8(0.1)} & \textbf{193.5(1.0)}   &   \textbf{125.0(0.5)}      &   \textbf{49.7(0.6)}      &     \textbf{253.7(1.1)}    & \textbf{150.9(0.5)} & \textbf{56.5(0.5)} & \textbf{297.9(1.3)}  \\
& AGCRN                                     & 91.0(0.2) & 49.1(0.4) & \textbf{190.9(0.6)} & \textbf{123.8(0.3)} & 50.6(0.6) & \textbf{248.7(0.6)} & \textbf{150.1(0.3)} & 58.8(0.2) & \textbf{294.2(1.1)}       \\
& GraphWavenet   & 91.4(0.2) & 48.1(0.8) & 196.7(0.8) & 128.4(0.3) & 52.9(0.6) & 256.7(1.0) & 155.9(0.5) & 60.1(0.5) & 304.6(0.8)  \\
& FC-GNN                            &   \textbf{88.6(0.1)} & 45.7(0.6) & \textbf{189.0(0.3)} & \textbf{121.7(0.4)} & \textbf{49.5(0.6)} & \underline{\textbf{245.9(1.5)}} & \underline{\textbf{146.9(0.1)}} & \textbf{55.6(0.6)} & \underline{\textbf{290.3(1.0)}}     \\
& BP-GNN                                   &   90.2(0.1) & 48.8(0.5) & \underline{\textbf{188.4(0.3)}} & \underline{\textbf{121.6(0.1)}} & \textbf{49.4(0.5)} & \textbf{246.0(0.8)} & \textbf{148.0(0.1)} & \underline{\textbf{54.7(0.5)}} & \textbf{293.3(0.9) }     \\
\hline
\hline
\end{tabular}
}
\endgroup
\label{tab:result_228}
\end{table*}

Compared to benchmark models that process only temporal features, the practice of adding relationships between households in \acrshort{stgnn} models increases forecasting performance. Especially, in contrast to \acrshort{gru}, which uses matrix multiplication as the unit cell of the recurrent network, \acrshort{gcgru} and \acrshort{t-gcn} always achieve better performance by applying \acrshort{gcn} as the unit cell. Similarly, \acrshort{grugcn} also outperforms \acrshort{gru} by applying \acrshort{gcn} on top of it to account for spatial dependency. It showcases the effective use of graph neural networks to model spatial relationships (see Table \ref{tab:result_228}).

% Comparing GCGRU, T-GCN which use GCN as spatial processing unit to embed node features with RNN which simply use matrix multiplication, we see that there is a small advantage when integrating spatial processing unit. That demonstrates relevance in applying the graph neural network to model spatial relationships.

Moreover, although models with a learnable graph theoretically offer more flexibility, this approach does not deliver promising outcomes; only \acrshort{agcrn} model performs compatible results with the best benchmark models (\acrshort{gru}). However, when comparing with \acrshort{gcgru}, which uses a predefined graph from signals, there is a downgrade in performance in some metrics. This is because the learnable graph offers a more flexible way to model the spatial relationship. However, it could make the forecasting task more susceptible to overfitting. When testing on a more distant future (one month after training, as outlined in Fig. \ref{fig:cross-val}), the model actually performs worse than the counterpart that uses a predefined graph. 

When testing with 3 splits, the \acrshort{bp-gnn} and \acrshort{fc-gnn} models often perform better than their counterparts. These models presume the topology of the graph without relying on the data (fully connected or bipartite); instead, when aggregating information from neighbor nodes, the models utilize a weighted sum of neighboring information, where the weights are derived from the input. This makes the "message aggregation" step more adaptable to the input. 
However, despite the naive assumption of graph topology, their better performance questions if the graph formation based on signals or learnable parameters is effective in the context of \acrshort{stlf}. 
An interesting case is the \acrshort{bp-gnn} model. Although it performs slightly worse than \acrshort{fc-gnn}, it is more scalable since the interaction between the nodes of \acrshort{fc-gnn} is $N^2$, while for the \acrshort{bp-gnn} model, it is only $2KN$ with $K$ being the number of virtual nodes. It also outperforms other models most of the time. One reason might be that, due to the nature of the load profile dataset, the consumption pattern of users is grouped by latent factors such as socio-demographic status~\cite{acorn}. The virtual nodes defined by the model can account for latent factors that can influence the original nodes (households). The bipartite topology allows these virtual nodes to gather information in a cluster-like manner; then, the aggregated information is passed down to each node as additional information for learning.


% In terms of forecasting result with different periods, we notice that the magnitude of error differs vastly among them, indicating that the nature of data and testing period strongly affect the performance. It suggests that learning from historical data alone is insufficient to capture the dynamics of energy consumption. Incorporating temporal indicators, such as weekends or holidays, could enhance the model's ability to capture these contextual variations more effectively.



\subsection{Model benchmark with different temporal scale at the aggregate level}
We investigate the performance of load forecasting at the aggregate level simply by aggregating all the forecasts at the residential level (see Table \ref{tab:result_agg_228}). At the aggregate level, the forecast results obtained through aggregation are inferior to those of the baseline model (SeasonalNaive). This behavior is also observed in other deep learning models such as \acrshort{gru} and \acrshort{tfm}. 
% An explanation is that these models tend to prioritize learning from easier periods with consistent patterns while failing to adequately capture atypical events, such as consumption spikes~\cite{zhang_unlocking_2023} (see Fig. 4). This can lead to underestimation during abnormal periods. At the aggregate level, instead of centering predictions around the actual average consumption, the forecasts may consistently underestimate the consumption spikes, and hence the errors are not balanced out. STGNNs, by incorporating spatial learning, potentially propagate errors throughout the spatial dimension and therefore do not address this issue.  
To explain this behavior, we visualize in Fig.\ref{fig:multiple-forecast} the histogram of the errors among all the households at peak hour (2013-12-23 19:00). The x-axis represents errors, the y-axis lists top-performing models (Table \ref{tab:result_agg_228}), and the z-axis shows frequency.



% \begin{table}[h!]
% \centering
% \captionsetup{justification=centering,margin=1cm}
% \begingroup
% \setlength{\tabcolsep}{5pt} % Default value: 6pt
% \renewcommand{\arraystretch}{1.2} % Default value: 1
% \begin{tabular}{l|lll}
% \hline
% \hline
% \multicolumn{1}{c|}{\multirow{3}{*}{Models}} & \multicolumn{3}{c}{Metrics} \\ \cline{2-4}
% \multicolumn{1}{c|}{}  & \multicolumn{3}{c}{Average (over three folds)} \\ \cline{2-4}
% \multicolumn{1}{c|}{}                        & MAE (kWh)     & MAPE (\%)   & RMSE   \\ 
% \hline
% SeasonalNaive                               &   4.495      &  0.075   &  7.00   \\
% \hline
% VAR                                         &  9.450      &  0.122      &  11.946 \\
% \hline
% GRU                                         &  12.267    &  0.178     &  15.024      \\
% \hline
% Tranformer                                  &   12.508     &  0.179    &  15.375          \\
% \hline
% GRUGCN                                      & 12.115	      & 0.176   &  14.880	       \\
% \hline
% % GCLSTM                                      &    0.139     &   1.22      &  0.239    \\
% % \hline
% % T-GCN                                       &  \textbf{0.089 }      &    \textbf{0.444 }    &   0.194    \\
% % \hline
% % AGCRNN                                      &   \textbf{0.090 }     &     0.484    &  \textbf{ 0.191 }       \\
% % \hline
% % STEGNN                                      &  3.067     &  34.660     &   3.866    \\
% % \hline
% TGCN                                        & 11.879       &  0173     &  14.584    \\
% \hline
% Bipartite                                   &   11.421     &  0.158    &  14.414   \\
% \hline
% % Fully Connected                             &         &         &      \\
% % \hline
% % GraphWavenet                                &   0.091      &   0.494      &    0.196    \\     
% \hline
% \end{tabular}
% \endgroup
% \caption{Performance of different \acrshort{stgnn} models at aggregate level.
% }
% \label{tab:result_228_agg}
% \end{table}


\begin{table*}[htbp]
\centering
\caption{Performance of different \acrshort{stgnn} models at the \textbf{aggregate} level.
}
\begingroup
\setlength{\tabcolsep}{3pt} % Default value: 6pt
\renewcommand{\arraystretch}{1.2} % Default value: 1
\resizebox{\textwidth}{!}{
\begin{tabular}{cllllllllll}
\hline
\hline
\multirow{3}{*}{Group} & \multirow{3}{*}{Models} & \multicolumn{9}{c}{Metrics} \\ \cline{3-11}
                       &                         & \multicolumn{3}{c}{Split 1}  & \multicolumn{3}{c}{Split 2} & \multicolumn{3}{c}{Split 3} \\ \cline{3-11}
                       &                         & MAE (kWh)    & MAPE (\%)   & RMSE (kWh)   & MAE (kWh)    & MAPE (\%)    & RMSE (kWh) & MAE (kWh)    & MAPE (\%)   & RMSE (kWh) \\ 
\hline
\multirow{4}{*}{Benchmark} 
& SeasonalNaive                             & \underline{3.39(0.0)} & \underline{6.8(0.0)} & \underline{4.57(0.0)} & \underline{5.48(0.0)} & \underline{8.2(0.0)} & \underline{7.96(0.0)} & \underline{5.97(0.0)} & \underline{7.6(0.0)} & \underline{8.47(0.0)} \\
& VAR                                         & 3.44(0.23) & 7.1(0.05) & 4.56(0.04) & 13.73(0.25) & 18.0(0.4) & 18.67(0.27) & 9.56(0.19) & 11.5(0.2) & 12.84(0.24) \\
& GRU  & 8.58(0.28) & 16.9(0.6) & 10.08(0.29) & 13.49(0.17) & 18.8(0.3) & 16.97(0.17) & 15.35(0.39) & 18.2(0.7) & 18.76(0.29) \\
& Transformer                                 & 8.83(0.37) & 17.3(0.6) & 10.34(0.43) & 13.34(0.73) & 18.6(0.2) & 16.91(0.05) & 15.72(0.17) & 18.5(0.2) & 19.30(0.12) \\
\hline
\multirow{8}{*}{STGNN} 
& GRUGCN                                      & 9.38(0.57) & 18.2(1.2) & 10.93(0.58) & 12.71(0.31) & 17.7(0.4) & 16.17(0.4) & 14.31(0.37) & 16.7(0.3) & 17.71(0.44) \\
& GCGRU                                       & 8.88(0.54) & 17.1(1.1) & 10.34(0.51) & 13.17(0.54) & 18.2(0.90) & 16.87(0.53) & 14.34(0.17) & 16.9(0.3) & 17.74(0.20) \\
& T-GCN                                       & 9.27(0.26) & 18.1(0.5) & 10.68(0.28) & 13.93(0.33) & 19.4(0.4) & 17.58(0.41) & 14.52(0.44) & 17.4(0.4) & 17.81(0.55) \\
& AGCRN                                     & 7.91(0.15) & 15.2(0.3) & 9.43(0.19) & 13.26(0.35) & 17.9(0.5) & 17.34(0.34) & 13.16(0.28) & 15.3(0.2) & 16.66(0.36) \\
& GraphWavenet   & 8.85(0.44) & 17.4(1.0) & 10.34(0.44) & 13.91(0.47) & 19.2(0.9) & 17.61(0.43) & 15.02(0.44) & 18.1(0.5) & 18.37(0.57) \\   
& FC-GNN                             & 8.22(0.32) & 15.9(0.7) & 9.80(0.27) & 12.79(0.56) & 16.9(0.7) & 16.79(0.72) & 13.61(0.2) & 16.0(0.4) & 16.60(0.14) \\
& BP-GNN                                   & 6.61(0.35) & 13.1(0.8) & 8.07(0.34) & 12.77(0.39) & 17.1(0.6) & 16.56(0.42) & 14.81(0.29) & 16.9(0.5) & 18.40(0.24) \\
\hline
\hline
\end{tabular}
}
\endgroup
\label{tab:result_agg_228}
\end{table*}



\begin{figure}[h!]
    \centering
    \includegraphics[width=0.6\linewidth]{assets/3D_histogram_plot_.pdf}
    \caption{Distribution of differences between forecasts and ground truth.}
    \label{fig:multiple-forecast}
\end{figure}

% \begin{figure}[h!]
%     \centering
%     \includegraphics[width=\linewidth]{assets/MAC000026.pdf}
%     \caption{Distribution of errors between forecasts and ground truth of 228 households.}
%     \label{fig:multiple-forecast}
% \end{figure}

We observe that the baseline model, although predicting less accurately (errors are more scattered), has its tails more evenly distributed. The error of deep learning-based models (including \glspl{stgnn}) is often more skewed on the left. An explanation is that these models tend to prioritize learning from "easier" periods with consistent patterns while failing to adequately capture atypical events, such as consumption spikes~\cite{zhang_unlocking_2023} (see Fig. \ref{fig:multiple-forecast}). This can lead to underestimation during abnormal periods. At the aggregate level, instead of centering predictions around the actual average consumption, the forecasts may consistently underestimate the consumption spikes, and hence the errors will not be balanced out. In \glspl{stgnn} since it incorporates spatial learning, it can potentially propagate errors throughout the spatial dimension and therefore does not address this issue.




% \begin{table}[h!]
% \centering
% \caption{Cross-validation settings for validating \acrshort{stgnn} models.}
% \begin{tabularx}{\columnwidth}{|X|X|X|X|}
% \hline
% \textbf{Phase} & \textbf{Training} & \textbf{Validation} & \textbf{Testing} \\ \hline
% \textbf{Fold 1} & Jan 1, 2013 -- Jun 30, 2013 & Jul 1, 2013 -- Jul 31, 2013 & Aug 1, 2013 -- Aug 31, 2013 \\ \hline
% \textbf{Fold 2} & Jan 1, 2013 -- Aug 31, 2013 & Sep 1, 2013 -- Sep 30, 2013 & Oct 1, 2013 -- Oct 31, 2013 \\ \hline
% \textbf{Fold 3} & Jan 1, 2013 -- Oct 31, 2013 & Nov 1, 2013 -- Nov 30, 2013 & Dec 1, 2013 -- Dec 31, 2013 \\ \hline
% \end{tabularx}
% \label{tab:cross_validation}
% \end{table}


% \begin{table}[h!]
% \centering
% \caption{Cross-validation settings for validating \acrshort{stgnn} models.}
% \begin{tabular}{|c|c|c|}
% \hline
% \textbf{Phase}         & \textbf{Time Period} \\ \hline
% \textbf{Fold 1} & \begin{tabular}[c]{@{}l@{}}Training: Jan 1, 2013 -- Jun 30, 2013 \\ 
% Validation: Jul 1, 2013 -- Jul 31, 2013 \\ 
% Testing: Aug 1, 2013 -- Aug 31, 2013\end{tabular} \\ \hline
% \textbf{Fold 2} & \begin{tabular}[c]{@{}l@{}}Training: Jan 1, 2013 -- Aug 31, 2013 \\ 
% Validation: Sep 1, 2013 -- Sep 30, 2013 \\ 
% Testing: Oct 1, 2013 -- Oct 31, 2013\end{tabular} \\ \hline
% \textbf{Fold 3} & \begin{tabular}[c]{@{}l@{}}Training: Jan 1, 2013 -- Oct 31, 2013 \\ 
% Validation: Nov 1, 2013 -- Nov 30, 2013 \\ 
% Testing: Dec 1, 2013 -- Dec 31, 2013\end{tabular} \\ \hline
% \end{tabular}
% \label{tab:cross_validation}
% \end{table}




\section{Conclusion}\label{sec:conclusion}
% In this paper we provide an overview of the literature on \glspl{stgnn} in short-term load forecasting and present benchmark results for selected algorithms from the literature. Our findings indicate that \acrshort{stgnn} models which use spatial relationships from temporal features offer advantages in forecasting household energy consumption compared to other models that only use temporal features. The results also display how different components in using \glspl{stgnn} can affect the performance of models. Especially, based on the performance, we see that a bipartite graph can model the dynamics of energy consumption data better than deriving direct relationships from signals or learnable embeddings. 

In this paper, we provide an overview of the literature on \glspl{stgnn} in short-term load forecasting and benchmark selected algorithms. Our findings show that integrating spatial relationships with temporal features improves forecasting accuracy for household energy consumption compared to models using only temporal features. Nevertheless, the most effective method for creating a graph to represent the proximity of energy consumption among households remains undetermined, given that even rudimentary graphs, like bipartite or fully connected graphs, can perform better than signal-based graphs. We also discuss how different components of \glspl{stgnn} can affect the performance of models. Notably, bipartite graphs effectively capture energy consumption dynamics, outperforming direct relationships from raw signals or embeddings. However, at the aggregate level, simple models such as SeasonalNaive outperform \gls{stgnn} models.

% This can provide some insight to practitioners and researchers who want to apply \glspl{stgnn} to \acrshort{stlf} in future studies.

However, our research has its limitations; we acknowledge that, for instance, our forecasting scenarios focus on one forecasting horizon (day ahead), which may not provide a comprehensive comparison of the strengths and weaknesses of the models. Furthermore, we did not consider the incorporation of exogenous variables into the STGNN models. Since energy consumption is significantly influenced by exogenous factors, such as weather or time indicators, the inclusion of these variables could improve model performance and provide deeper insight. Finally, given the growing scientific literature around \acrshort{stgnn} for residential \acrshort{stlf}, our selected models for benchmarking may not represent the entire research landscape in this domain. We believe that this incompleteness can motivate future research to provide a broader overview of \acrshort{stgnn} into the residential \acrshort{stlf} problem.


% \section{Ease of Use}

% \subsection{Maintaining the Integrity of the Specifications}

% The IEEEtran class file is used to format your paper and style the text. All margins, 
% column widths, line spaces, and text fonts are prescribed; please do not 
% alter them. You may note peculiarities. For example, the head margin
% measures proportionately more than is customary. This measurement 
% and others are deliberate, using specifications that anticipate your paper 
% as one part of the entire proceedings, and not as an independent document. 
% Please do not revise any of the current designations.

% \section{Prepare Your Paper Before Styling}
% Before you begin to format your paper, first write and save the content as a 
% separate text file. Complete all content and organizational editing before 
% formatting. Please note sections \ref{AA}--\ref{SCM} below for more information on 
% proofreading, spelling and grammar.

% Keep your text and graphic files separate until after the text has been 
% formatted and styled. Do not number text heads---{\LaTeX} will do that 
% for you.

% \subsection{Abbreviations and Acronyms}\label{AA}
% Define abbreviations and acronyms the first time they are used in the text, 
% even after they have been defined in the abstract. Abbreviations such as 
% IEEE, SI, MKS, CGS, ac, dc, and rms do not have to be defined. Do not use 
% abbreviations in the title or heads unless they are unavoidable.

% \subsection{Units}
% \begin{itemize}
% \item Use either SI (MKS) or CGS as primary units. (SI units are encouraged.) English units may be used as secondary units (in parentheses). An exception would be the use of English units as identifiers in trade, such as ``3.5-inch disk drive''.
% \item Avoid combining SI and CGS units, such as current in amperes and magnetic field in oersteds. This often leads to confusion because equations do not balance dimensionally. If you must use mixed units, clearly state the units for each quantity that you use in an equation.
% \item Do not mix complete spellings and abbreviations of units: ``Wb/m\textsuperscript{2}'' or ``webers per square meter'', not ``webers/m\textsuperscript{2}''. Spell out units when they appear in text: ``. . . a few henries'', not ``. . . a few H''.
% \item Use a zero before decimal points: ``0.25'', not ``.25''. Use ``cm\textsuperscript{3}'', not ``cc''.)
% \end{itemize}

% \subsection{Equations}
% Number equations consecutively. To make your 
% equations more compact, you may use the solidus (~/~), the exp function, or 
% appropriate exponents. Italicize Roman symbols for quantities and variables, 
% but not Greek symbols. Use a long dash rather than a hyphen for a minus 
% sign. Punctuate equations with commas or periods when they are part of a 
% sentence, as in:
% \begin{equation}
% a+b=\gamma\label{eq}
% \end{equation}

% Be sure that the 
% symbols in your equation have been defined before or immediately following 
% the equation. Use ``\eqref{eq}'', not ``Eq.~\eqref{eq}'' or ``equation \eqref{eq}'', except at 
% the beginning of a sentence: ``Equation \eqref{eq} is . . .''

% \subsection{\LaTeX-Specific Advice}

% Please use ``soft'' (e.g., \verb|\eqref{Eq}|) cross references instead
% of ``hard'' references (e.g., \verb|(1)|). That will make it possible
% to combine sections, add equations, or change the order of figures or
% citations without having to go through the file line by line.

% Please don't use the \verb|{eqnarray}| equation environment. Use
% \verb|{align}| or \verb|{IEEEeqnarray}| instead. The \verb|{eqnarray}|
% environment leaves unsightly spaces around relation symbols.

% Please note that the \verb|{subequations}| environment in {\LaTeX}
% will increment the main equation counter even when there are no
% equation numbers displayed. If you forget that, you might write an
% article in which the equation numbers skip from (17) to (20), causing
% the copy editors to wonder if you've discovered a new method of
% counting.

% {\BibTeX} does not work by magic. It doesn't get the bibliographic
% data from thin air but from .bib files. If you use {\BibTeX} to produce a
% bibliography you must send the .bib files. 

% {\LaTeX} can't read your mind. If you assign the same label to a
% subsubsection and a table, you might find that Table I has been cross
% referenced as Table IV-B3. 

% {\LaTeX} does not have precognitive abilities. If you put a
% \verb|\label| command before the command that updates the counter it's
% supposed to be using, the label will pick up the last counter to be
% cross referenced instead. In particular, a \verb|\label| command
% should not go before the caption of a figure or a table.

% Do not use \verb|\nonumber| inside the \verb|{array}| environment. It
% will not stop equation numbers inside \verb|{array}| (there won't be
% any anyway) and it might stop a wanted equation number in the
% surrounding equation.

% \subsection{Some Common Mistakes}\label{SCM}
% \begin{itemize}
% \item The word ``data'' is plural, not singular.
% \item The subscript for the permeability of vacuum $\mu_{0}$, and other common scientific constants, is zero with subscript formatting, not a lowercase letter ``o''.
% \item In American English, commas, semicolons, periods, question and exclamation marks are located within quotation marks only when a complete thought or name is cited, such as a title or full quotation. When quotation marks are used, instead of a bold or italic typeface, to highlight a word or phrase, punctuation should appear outside of the quotation marks. A parenthetical phrase or statement at the end of a sentence is punctuated outside of the closing parenthesis (like this). (A parenthetical sentence is punctuated within the parentheses.)
% \item A graph within a graph is an ``inset'', not an ``insert''. The word alternatively is preferred to the word ``alternately'' (unless you really mean something that alternates).
% \item Do not use the word ``essentially'' to mean ``approximately'' or ``effectively''.
% \item In your paper title, if the words ``that uses'' can accurately replace the word ``using'', capitalize the ``u''; if not, keep using lower-cased.
% \item Be aware of the different meanings of the homophones ``affect'' and ``effect'', ``complement'' and ``compliment'', ``discreet'' and ``discrete'', ``principal'' and ``principle''.
% \item Do not confuse ``imply'' and ``infer''.
% \item The prefix ``non'' is not a word; it should be joined to the word it modifies, usually without a hyphen.
% \item There is no period after the ``et'' in the Latin abbreviation ``et al.''.
% \item The abbreviation ``i.e.'' means ``that is'', and the abbreviation ``e.g.'' means ``for example''.
% \end{itemize}
% An excellent style manual for science writers is \cite{b7}.

% \subsection{Authors and Affiliations}
% \textbf{The class file is designed for, but not limited to, six authors.} A 
% minimum of one author is required for all conference articles. Author names 
% should be listed starting from left to right and then moving down to the 
% next line. This is the author sequence that will be used in future citations 
% and by indexing services. Names should not be listed in columns nor group by 
% affiliation. Please keep your affiliations as succinct as possible (for 
% example, do not differentiate among departments of the same organization).

% \subsection{Identify the Headings}
% Headings, or heads, are organizational devices that guide the reader through 
% your paper. There are two types: component heads and text heads.

% Component heads identify the different components of your paper and are not 
% topically subordinate to each other. Examples include Acknowledgments and 
% References and, for these, the correct style to use is ``Heading 5''. Use 
% ``figure caption'' for your Figure captions, and ``table head'' for your 
% table title. Run-in heads, such as ``Abstract'', will require you to apply a 
% style (in this case, italic) in addition to the style provided by the drop 
% down menu to differentiate the head from the text.

% Text heads organize the topics on a relational, hierarchical basis. For 
% example, the paper title is the primary text head because all subsequent 
% material relates and elaborates on this one topic. If there are two or more 
% sub-topics, the next level head (uppercase Roman numerals) should be used 
% and, conversely, if there are not at least two sub-topics, then no subheads 
% should be introduced.

% \subsection{Figures and Tables}
% \paragraph{Positioning Figures and Tables} Place figures and tables at the top and 
% bottom of columns. Avoid placing them in the middle of columns. Large 
% figures and tables may span across both columns. Figure captions should be 
% below the figures; table heads should appear above the tables. Insert 
% figures and tables after they are cited in the text. Use the abbreviation 
% ``Fig.~\ref{fig}'', even at the beginning of a sentence.

% \begin{table}[htbp]
% \caption{Table Type Styles}
% \begin{center}
% \begin{tabular}{|c|c|c|c|}
% \hline
% \textbf{Table}&\multicolumn{3}{|c|}{\textbf{Table Column Head}} \\
% \cline{2-4} 
% \textbf{Head} & \textbf{\textit{Table column subhead}}& \textbf{\textit{Subhead}}& \textbf{\textit{Subhead}} \\
% \hline
% copy& More table copy$^{\mathrm{a}}$& &  \\
% \hline
% \multicolumn{4}{l}{$^{\mathrm{a}}$Sample of a Table footnote.}
% \end{tabular}
% \label{tab1}
% \end{center}
% \end{table}

% \begin{figure}[htbp]
% \centerline{\includegraphics{fig1.png}}
% \caption{Example of a figure caption.}
% \label{fig}
% \end{figure}

% Figure Labels: Use 8 point Times New Roman for Figure labels. Use words 
% rather than symbols or abbreviations when writing Figure axis labels to 
% avoid confusing the reader. As an example, write the quantity 
% ``Magnetization'', or ``Magnetization, M'', not just ``M''. If including 
% units in the label, present them within parentheses. Do not label axes only 
% with units. In the example, write ``Magnetization (A/m)'' or ``Magnetization 
% \{A[m(1)]\}'', not just ``A/m''. Do not label axes with a ratio of 
% quantities and units. For example, write ``Temperature (K)'', not 
% ``Temperature/K''.

% \section*{Acknowledgment}

% The preferred spelling of the word ``acknowledgment'' in America is without 
% an ``e'' after the ``g''. Avoid the stilted expression ``one of us (R. B. 
% G.) thanks $\ldots$''. Instead, try ``R. B. G. thanks$\ldots$''. Put sponsor 
% acknowledgments in the unnumbered footnote on the first page.


\section*{Declaration of Generative AI and AI-assisted technologies in the writing process}
Statement: During the preparation of this work, the author(s) used ChatGPT~\cite{ChatGPT} to paraphrase and fix grammatical mistakes. After using this tool/service, the author(s) reviewed and edited the content as needed and take(s) full responsibility for the content of the publication.



% \section*{References}

% Please number citations consecutively within brackets \cite{b1}. The 
% sentence punctuation follows the bracket \cite{b2}. Refer simply to the reference 
% number, as in \cite{b3}---do not use ``Ref. \cite{b3}'' or ``reference \cite{b3}'' except at 
% the beginning of a sentence: ``Reference \cite{b3} was the first $\ldots$''

% Number footnotes separately in superscripts. Place the actual footnote at 
% the bottom of the column in which it was cited. Do not put footnotes in the 
% abstract or reference list. Use letters for table footnotes.

% Unless there are six authors or more give all authors' names; do not use 
% ``et al.''. Papers that have not been published, even if they have been 
% submitted for publication, should be cited as ``unpublished'' \cite{b4}. Papers 
% that have been accepted for publication should be cited as ``in press'' \cite{b5}. 
% Capitalize only the first word in a paper title, except for proper nouns and 
% element symbols.

% For papers published in translation journals, please give the English 
% citation first, followed by the original foreign-language citation \cite{b6}.

% \begin{thebibliography}{00}
% \bibitem{b1} G. Eason, B. Noble, and I. N. Sneddon, ``On certain integrals of Lipschitz-Hankel type involving products of Bessel functions,'' Phil. Trans. Roy. Soc. London, vol. A247, pp. 529--551, April 1955.
% \bibitem{b2} J. Clerk Maxwell, A Treatise on Electricity and Magnetism, 3rd ed., vol. 2. Oxford: Clarendon, 1892, pp.68--73.
% \bibitem{b3} I. S. Jacobs and C. P. Bean, ``Fine particles, thin films and exchange anisotropy,'' in Magnetism, vol. III, G. T. Rado and H. Suhl, Eds. New York: Academic, 1963, pp. 271--350.
% \bibitem{b4} K. Elissa, ``Title of paper if known,'' unpublished.
% \bibitem{b5} R. Nicole, ``Title of paper with only first word capitalized,'' J. Name Stand. Abbrev., in press.
% \bibitem{b6} Y. Yorozu, M. Hirano, K. Oka, and Y. Tagawa, ``Electron spectroscopy studies on magneto-optical media and plastic substrate interface,'' IEEE Transl. J. Magn. Japan, vol. 2, pp. 740--741, August 1987 [Digests 9th Annual Conf. Magnetics Japan, p. 301, 1982].
% \bibitem{b7} M. Young, The Technical Writer's Handbook. Mill Valley, CA: University Science, 1989.
% \end{thebibliography}

% \medskip

% \printbibliography


% \vspace{12pt}
% \color{red}
% IEEE conference templates contain guidance text for composing and formatting conference papers. Please ensure that all template text is removed from your conference paper prior to submission to the conference. Failure to remove the template text from your paper may result in your paper not being published.


\bibliographystyle{IEEEtran}
\bibliography{reference}


\end{document}
