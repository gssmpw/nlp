\section{High-Fidelity Prototypes of Design Ideas}
\label{fig:prototypes}

\begin{figure*}[!h]
    \centering
    \includegraphics[width=0.8\linewidth]{inserts/prototypes.jpg}
    \caption{High-fidelity mock-ups of the eight design approaches derived from the co-design study with teen participants. A complete taxonomy of the designs is available in Table \ref{tab:designs}.} 
    % \Description {This figure contains eight prototypes each displaying the UI for the different features of a hypothetical app in support of trust building on social media. (a) System-Generated Prompts: shows a notification on an iPhone home screen that says ``Curious to see today's prompts? Tap to check them out whenever you're up for it.''; (b) Increased Viewer Agency: a system dialog prompting the user if they would like to join a space called ``\#NewSongs''; (c) Interest-Based Segmentation: multiple groups of feed, such as ``FOODIES!!!,'' ``Cute Dog Photo Dump,'' ``LGBTQ+,'' and ``Knitting,'' with an option to create a new feed; (d) Effortful and Specific Reactions: shows the option to react by choosing an emoji from the full list of emojis rather than ``Liking'' with a heart to posts; (e) Low-Profile Sharing: illustrates simple ways to self-disclosure, such as a short text-based note, an emoji-based status, music currently listening to, and a list of interests; (f) Interaction/Profile Contexts: shows a way to add contexts about the user, such as displaying their ``Social Battery Level'' or their specific availability and/or current activity, such as ``On Vacation'' or ``Focus Mode''; (g) Silent Sharing: depicts a feed consisting of a list of user names, with a small purple dot indicator to signal that the user made an update and the number of new posts next to the indicator; (h) Commitment to Trust: a landing page of a hypothetical social media app ``CleanSocial'' with the logo ``Judgment-Free Zone.''}
\end{figure*}
