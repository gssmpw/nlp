\begin{figure}[t]
    \centering
    \includegraphics[width=1.0\linewidth]{inserts/audiences.jpg}
    \caption{Categorizing social media connections based on trust levels. While users have clear trust or distrust for some individuals, a significant portion of ``Friends'' or Followers fall into an ambiguous trust category. This study highlights the importance of fostering meaningful, reciprocal self-disclosure within this group. To achieve this, platform designs should address Communication Fog and Low-Grace Culture, which currently impede what teens perceive to be meaningful self-disclosure, particularly among these potential but not-yet-trusted connections.}
    % \Description {The image features a horizontal axis labeled ``Level of Established Trust,'' ranging from ``HIGH'' on the left to ``LOW'' on the right. On the left, a purple box labeled ``Trusted Friends'' (marked with the number 1) represents individuals with a high level of trust. In the middle, a purple box labeled ``Not-Yet-Trusted `Friends''' (marked with the number 3) indicates individuals with potential trust, though it has not yet been fully established. On the right, a purple box labeled ``Untrusted Individuals'' (marked with the number 2) represents individuals with little to no trust. A yellow-and-black striped barrier separates this group from the ``Not-Yet-Trusted `Friends.''' Above the axis, an arrow labeled ``Repeated, Reciprocated Meaningful Self-Disclosure'' points from ``Not-Yet-Trusted `Friends''' to ``Trusted Friends,'' illustrating the (desired) process that leads to higher levels of trust.}
    \label{fig:audiences}
\end{figure}
