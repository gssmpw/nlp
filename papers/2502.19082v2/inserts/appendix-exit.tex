\section{Exit Interview Protocol}
\label{ref:exit-protocol}

\textbf{Order of Importance (To Sort)}

\begin{itemize}
    \item This bothers me a lot
    \item This bothers me to an extent
    \item This doesn’t bother me
\end{itemize}

\subsubsection*{Larger Themes to Sort}

\begin{itemize}
    \item Prevent getting ignored or receiving no response
    \item Being able to control who sees what you share
    \item Prevent negative reactions (e.g., judgment, misunderstanding)
    \item Defining and communicating privacy rules/norms
    \item Encouraging more effortful communication or validation from friends
    \item Preventing posts from being shared out of context
    \item Increasing trust with friends (followers)
    \item Knowing the right amount of sharing
    \item Making sharing easier, more casual, more spontaneous, and less pressured
    \item Other changes to help share more comfortably
\end{itemize}

\subsubsection*{Subthemes to Sort}

\begin{itemize}
    \item Friend/follower list being a mix of people with different contexts (e.g., family, summer camp)
    \item Not knowing if there is anyone in the follower list whom you don’t approve of
    \item Feeling pressured to add people you’re not close with
    \item Data leaks or hacking
    \item Scams/bots
    \item Doxxing
    \item Impersonation
    \item Getting no response
    \item Not receiving enough validation/support
    \item Trust being broken
    \item Oversharing
    \item Pressure to look good or interesting
    \item Pressure to post things that elicit responses
    \item Pressure to polish content; discomfort with spontaneous sharing
    \item Content being shared with strangers
    \item Content being shared with unintended friends
    \item Screenshots being taken or content saved
    \item Judgment
    \item Gossip
    \item Online harassment
    \item Misunderstanding
    \item Effortless responses/interactions
    \item Lack of trust through positive interactions
    \item Interactions with people you don’t trust
    \item Sharing posts that may be inappropriate in certain contexts (e.g., future relevance)
    \item Time and effort needed to create content
    \item Not knowing where to start
    \item Lack of confidence or feeling anxious
    \item Posts seeming boring or repetitive
    \item Other concerns
\end{itemize}

\subsubsection*{General Questions (For Each Sticky Note, in Order of Importance)}

\begin{itemize}
    \item What does [sticky note] mean to you?
    \item How can we design to make [sticky note] better?
\end{itemize}

\subsubsection*{Post-Sketch Questions (For Each Design Idea)}

\begin{itemize}
    \item Can you explain your design idea? What did you want to change, and how do you think this feature might help?
    \item What might be some benefits of making this change?
    \item What might be some downsides of making this change?
    \item Might the effects of this change differ by context (e.g., content type, platform, or audience)?
    \item When might this be most effective?
    \item When might this be least effective?
\end{itemize}
