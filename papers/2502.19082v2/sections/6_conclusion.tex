\section{Conclusion}
This study highlights the critical role of trust-enabled privacy in shaping adolescent social media experiences. Our findings illustrate how communication fog and low-grace culture hinder self-disclosure, leading teens to default to disengagement rather than risk uncertainty. In response, we propose design interventions that empower teens to regulate their boundaries dynamically, ensuring privacy is not merely about restricting access but about facilitating gradual trust-building in evolving social contexts. Rather than treating privacy as an individual responsibility, platforms should reimagine boundary regulation as a shared, social process. Teens advocate for contextualized audience control, mechanisms for mutual trust reinforcement, and lightweight, low-risk sharing options that align with their relational needs. By integrating these insights, platforms can move beyond rigid information-control models and instead cultivate social media environments that actively support trust development.