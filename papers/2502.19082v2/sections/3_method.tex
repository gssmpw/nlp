\section{Method}
The study consisted of three procedures: 1) a 30-minute entry interview, 2) daily reflection surveys throughout 7 days (with a minimum requirement of 5 or more entries), and 3) a 60-minute exit interview involving co-design activities. The study flow is illustrated in Figure \ref{fig:procedure}.

\begin{figure}[!ht]
    \centering
    \includegraphics[width=1.0\linewidth]{inserts/procedure.jpg}
    \caption{The overall study procedure.}
    % \Description {This diagram represents a multi-phase research study with five stages. The first stage is the screener survey, which has no monetary incentive and involves 47 respondents who take 5 minutes to complete the survey. The second stage is the entry interview, where 20 participants are selected from the survey respondents. They receive \$10 for participating in a 30-minute interview. The third stage is a diary study, involving 19 participants from the entry interview. This stage spans 7 days, where participants are compensated \$25 and are required to spend at least 10 minutes daily logging their diary entries. Following the diary study, the fourth stage is the co-design interview. The same 19 participants complete a 60-minute interview and receive \$20. The final stage is completion, where participants receive an additional \$5 for completing all stages of the study.}
    \label{fig:procedure}
\end{figure}


\subsection{Materials and Procedures}
\vspace{2mm}
\noindent\textbf{Screening survey.}
We first administered a screening survey prior to the interviews. The screening survey included basic demographic questions, three open-ended questions about: concerns or reservations about sharing on social media, existing features that currently facilitate sharing, and suggestions for features that could make sharing easier. The survey also contained 26 questions from the peer attachment scale \cite{Armsden1987-ai}.

\vspace{2mm}
\noindent\textbf{Entry interview.} We introduced the overall purpose and procedure of the study. We also asked about social media platforms where participants felt most at ease sharing posts, what they shared, who they shared with, any concerns they have when sharing, and any specific features or designs that affect the comfort of sharing.

\vspace{2mm}
\noindent\textbf{Diary study.} The diary study captured data in three scenarios: 1) participants shared something on social media, 2) they considered sharing but did not share, and 3) they neither considered nor shared anything. In the first scenario, we asked participants about the content and location of their sharing, the audience, their comfort level, and any features that increased their comfort. We also asked about any risks or benefits they associated with the sharing. For the other two scenarios, we posed similar questions focused on what they had intended or might have shared for the sake of building relationships. Additionally, we asked why they chose not to share and what features could have made them more comfortable doing so. The full questionnaire is available in the supplementary materials. We required at least five diary entries over the seven-day period. Most participants submitted five diary entries, with some submitting between six and 19 entries.

\vspace{2mm}
\noindent\textbf{Co-design interview.} We used a virtual whiteboard, created with the Miro platform \cite{miro}, to anchor a conversation about the barriers participants face to ``sharing comfortably on social media.'' We presented nine sticky notes\footnote{In no particular order, 1) prevent getting ignored; 2) be able to control who gets to see what you share; 3) make sharing easier, more casual, more spontaneous, less pressure; 4) knowing what the right amount of sharing is; 5) defining and communicating privacy rules/norms; 6) encourage more effortful communication/response from friends; 7) prevent negative reactions \eg{judgment, misunderstanding}; 8) increase trust with friends (followers); 9) prevent posts being shared out of context} that encapsulated the major concerns that participants shared in the screening survey and during the entry interviews. We also included additional subcategories related to each of the nine overarching concerns. Using virtual sticky notes, participants listed their concerns in order of personal relevance, and we structured our interviews based on that order. We addressed each sticky note in the sequence they considered most important. For each sticky note, we asked participants to explain the meaning and relevance of each concern and to suggest potential design solutions. A text-document-transformed version of the Miro Board is included in the supplementary materials. 

One member of the research team conducted the semi-structured entry and co-design interviews via Zoom, which lasted 30 to 40 minutes and 60 to 90 minutes, respectively. Participants received \$10 and \$20 Amazon gift cards for the entry and co-design interviews. All participants received a \$25 Amazon gift card for their diary study participation, regardless of the number of entries they submitted. Those who completed the full co-design study received an additional \$5 gift card.

%\begin{table*}[h]
\centering
 \small
\renewcommand{\arraystretch}{1.2} 

\caption{Summative Demographics of the Co-design Interview Participants (N=19)}
\label{ref:demographics-summary}
\begin{tabular}{p{5.5cm} p{9cm}}
\toprule
\textbf{Category} & \textbf{Details} \\
\midrule

{Gender Identity} & Girls (63\%), Boys (26\%), Non-Binary or Third Gender (10\%) \\

{Age} & 13 (10.5\%), 14 (10.5\%), 15 (15.8\%), 16 (21.1\%), 17 (21.1\%), 18 (21.1\%); Mean Age: 15.8 years (SD=1.70) \\

{Race} & White (53\%), Asian (32\%), Black or African American (11\%), White and Asian (5\%) \\

{Hispanic or Latin-American Origin} & No (84\%), Yes (16\%) \\

{Social Media Usage} & Instagram (100\%), BeReal (84\%), Snapchat (74\%), Twitter (68\%), TikTok (63\%), Discord (11\%), Reddit (11\%), Tumblr (5\%) \\

{Platforms Where Participants Reported Sharing Most \textit{Frequently}} & Instagram (42\%), BeReal (32\%), Snapchat (11\%), Twitter (5\%), YouTube (5\%), Discord (5\%) \\

{Platforms Where Participants Reported Sharing Most \textit{Comfortably}} & Instagram (36\%), BeReal (26\%), Snapchat (16\%), Discord (16\%), Pinterest (5\%)\\

\bottomrule
\end{tabular}
\end{table*}


\subsection{Participants and Recruitment}
We invited 80 participants from an established participant pool---composed of individuals who had taken part in one of our previous studies or expressed an interest in future research via other studies---to complete the screener survey for the co-design interview. We selected 22 of the 47 teens, chosen based on the thoroughness of their responses to the open-ended questions and the diversity of their demographic data and peer attachment responses. Of the 22 invitees, 20 enrolled in the study, one withdrew following the entry interview, and 19 completed all three parts of the study.

Twelve participants self-identified as girls, five as boys, and two as third gender. Their age ranged from 13 to 19, with a mean of 15.8 years (SD=1.7). Ten identified as White, six Asian, two as Black or African American, and one as White and Asian. Only three identified as Hispanic or of Latin-American origin. Participants' social media usage was diverse but with a significant predominance of Instagram (19) and BeReal (16). These were also the platforms that participants reported sharing most frequently (8 and 6 participants, respectively) and more comfortably (7 and 5, respectively).
\footnote{See Appendix \ref{ref:individual-demographics} for individual demographic data}


\subsection{Reflexive Thematic Analysis}
The interviews were transcribed and lightly edited for clarity and readability while preserving the original meaning and intent of the participants' responses. We conducted a reflexive thematic analysis~\cite{braun2019reflecting} of the transcripts, ensuring a structured yet flexible approach to analyzing our data in a nuanced and exploratory manner not tied to any one existing framework.

Our analysis process began with three members of the research team independently analyzing the transcripts line by line using Google Docs~\cite{google-docs}. Following this initial coding phase, we convened for weekly discussion sessions to develop an initial set of codes based on our individual analyses. We then proceeded to inductively code the transcripts using Atlas.ti~\cite{atlas}, referring to the established codes while remaining open to identifying new themes. Through collaborative discussions, we reconciled identified codes, resolved disagreements, and progressively refined our themes through several iterations. This collaborative process continued until the second and last authors had reviewed over half of the transcripts (including both entry and co-design interviews), and we reached saturation~\cite{saunders2018saturation} in our themes. The different iterations of the codes are available in the supplementary materials.


Based on the themes that we identified during the thematic analysis, we developed high-fidelity prototypes of each of the eight design ideas that we co-designed with the teen participants. The design ideas and descriptions of the high-fidelity mock-ups of possible implementation of the design ideas are available in Table \ref{tab:designs} and prototypes in Appendix \ref{fig:prototypes}. These are discussed further in Section~\ref{section:4}.

\subsection{Ethical Considerations}
All procedures were approved by our Institutional Review Board (IRB) prior to data collection. Before the interviews, we obtained written consent from participants and parents. At the start of each interview, we reviewed the key points from the consent form and procedures, ensuring participants had the opportunity to ask any questions. Participants were informed of their right to decline questions, turn off cameras, and withdraw from the study without consequence. We emphasized our goal was to understand, not judge, their experiences. We then obtained verbal consent to proceed with the study and to record the interviews.