\subsection{Low-Grace Culture: Barrier to Sustained Trust and Adaptive Boundary Setting}
\label{section:4-3}
In this section, we explore how social media cultivates a ``low-grace culture'' that discourages vulnerability or casual self-disclosure. This environment is characterized by an emphasis on polished self-presentation, judgments based on limited information, the risk of inadvertently burdening others by sharing, and an overall climate of distrust. These pressures inhibit the flexibility needed for adaptive boundary regulation, discouraging trust-building interactions.

\subsubsection{High-Stakes Presentation Expectations and Self-Censorship}
\label{section:4-3-1}
Participants frequently expressed fear about not measuring up to the presentation expectations of platforms such as Instagram. Especially on image-centric platforms, they felt compelled to share only highly polished moments. As \entry{18} stated, \inlinequote{Instagram is only for when the sun sets during the golden hour when I look my best.} This expectation discourages casual posting, as teens perceive that they \inlinequote{need to put significant effort into a post} (\entry{06}). Some worry that deviating from these norms might make them be judged as \inlinequote{underwhelming} or \inlinequote{monotonous} (\diary{11}), leaving them feeling vulnerable (\diary{13}).

To counteract these pressures, many participants expressed a preference for \textbf{low-stakes disclosure} (Figure \ref{fig:prototypes}(a)), which reduces the burden of excessive curation and supports more casual self-expression. They valued features that support ephemeral sharing, such as Instagram's Story or Snapchat posts, which feel \inlinequote{not as big of a deal} (\diary{16}) without \inlinequote{people saving or revisiting them} (\entry{06}). Participants often leveraged these features to share what they felt \inlinequote{doesn't really live up to the standards of what I post on Instagram} (\diary{07}). They also appreciated the support for casual, text-based updates on Bluesky and Twitter, or platforms that made interactions feel more like \inlinequote{having a conversation with someone} (\codesign{11}) rather than putting together a \inlinequote{portfolio} (\codesign{11}). Some participants suggested features like \inlinequote{a little status update} where people can \inlinequote{learn little facts about other people casually} (\codesign{05}).

These features help alleviate concerns about inadvertently exposing too much or misjudging boundaries by supporting and normalizing casual sharing rather than polished, high-effort posts. If casual sharing is the norm, then posting less curated content is less likely to be perceived as a disclosure of something deeply personal or intimate. Instead, it becomes part of a broader culture of mutual disclosure and relationship building, reducing the likelihood of boundary missteps and supporting better trust calibration among peers.



\subsubsection{Sparse Sharing and the Heightened Risk of Misrepresentation}
\label{section:4-3-2}
On platforms like Instagram, where participants perceived infrequent posting as the norm, limited content often becomes the primary basis for one's digital identity. This leaves users vulnerable to misrepresentation from acquaintances who may \inlinequote{make assumptions or misjudge [their] intentions} (\codesign{03}). With such sparse sharing, even a small number of posts can significantly influence how someone is perceived: \inlinequote{Somebody might pull up my Instagram and there are only, like, two posts and I look bad in both of them} (\codesign{12}). They also shared specific examples, such as posting extensively about a single topic, potentially coming across as being \inlinequote{obsessed} (\diary{12}) with that subject; sharing multiple photos from a memorable event might be interpreted as \inlinequote{bragging} (\entry{12}); or inquiring whether other students have completed their summer assignment could be misconstrued as attempting to \inlinequote{push} (\entry{12}) peers to do their work, despite that not being the intention of the post. Such concerns often led participants to lean toward self-censorship.

To mitigate the risk of trust erosion and promote more adaptive boundary regulation, participants suggested features to facilitate \textbf{contextual clarity} (Figure \ref{fig:prototypes} (d))---designs that provide additional context to reduce misunderstandings and support trust-building on platforms like Instagram. These included: indicating personal information such as interests on profiles (\diary{12}), emphasizing the importance of comprehensive information before making judgments (\codesign{03}), supporting explanatory captions for images beyond standard captions (\codesign{07}), and implementing a \inlinequote{social battery} indicator to help manage interaction expectations (\codesign{11}). By fostering mechanisms that encourage giving others the benefit of the doubt, such designs could prevent trust erosion and provide contexts for boundary regulation in online environments where full contextual understanding is not always possible.



\subsubsection{Nonconsensual Exposure and the Fear of Burdening Others}
\label{section:4-3-3}
On broadcast social media platforms, following a user implicitly consents to viewing their content, often leading to viewers feeling overwhelmed and \inlinequote{buried} (\codesign{12}) in posts. As \codesign{12} noted, \inlinequote{If I'm following 100 people and they're all sharing four times a day, then that's 400 things I have to click through.} Conversely, those that are sharing---aware of this dynamic from their own experiences---often fear \inlinequote{spamming} (\entry{03}, \codesign{13}) or \inlinequote{clogging} (\diary{16}) others' feeds. This fear of being perceived as burdensome or irritating was also evident in diary entries, where participants expressed hesitation \inlinequote{that people may think I'm oversharing or like posting too much} (\diary{11}), or regret after realizing, \inlinequote{[I] shared a TON of reels today on my close friends and public for some reason :crying-face:} (\diary{09}).

Participants sought ways to mitigate such unnecessary friction. Many desired \textbf{self-contained disclosure} (Figure \ref{fig:prototypes}(e)) that would allow them to share without imposing their content on others. These unobtrusive communication mechanisms included \inlinequote{a little status update} (\codesign{05}) or sharing minor personal interests on profile pages rather than in followers' feeds. One participant proposed a \inlinequote{red dot} (\codesign{17}) on user profiles to indicate new content subtly, allowing viewers to \inlinequote{not have to go look at it if they don't want to} (\codesign{17}). \codesign{05} saw parallels to Instagram's Story feature where \inlinequote{it doesn't notify people\ldots{} it's not like it goes on their feed\ldots{} it's just casual.} Participants perceived this as a way to foster connection without unnecessary friction and trust erosion by ensuring disclosure remains within a self-regulated and lower-stakes space.



\subsubsection{Absent Norms for Trust-Building and the Cycle of Distrust}
\label{section:4-3-4}
Participants observed social media spaces as often having a pervasive climate of distrust and the lack of measures to regulate such an environment. As \codesign{17} remarked, \inlinequote{On mainstream social media\ldots there's occasional positivity\ldots but mostly it feels draining}. 

They suggested that platforms could shift this culture by explicitly stating their platform expectations toward \textbf{trust-centered norms} (Figure \ref{fig:prototypes} (h)), such as explicitly stating that the platform is \inlinequote{a judgment-free zone} and communicating a clear message to users to \inlinequote{decrease judgment and be more kind} (\codesign{19}). They believed that such an approach would encourage users to \inlinequote{naturally be inclined to conform with the overall vibe and mission of the other users on the app} (\codesign{19}). Once a culture is established, a \inlinequote{selection bias} would occur, attracting like-minded individuals who align with these expectations (\codesign{19}). One participant expanded on this idea: 
\blockquote{I'm tired of how curated social media can be, but I still engage with it\ldots{} When the culture prioritizes authenticity and transparency, you'll get people like me, who are tired of curating, putting in the effort to be authentic and create this culture.}{\codesign{15}}
Similarly, another participant observed:
\blockquote{When a platform declares itself a judgment-free zone, it gives people the power to ensure both their own safety and the safety of others, creating a sense of control.}{\codesign{13}}
Participants emphasized that they believe that setting clear expectations, rules, and guidelines would foster meaningful changes toward safer environments. \codesign{17} noted that \inlinequote{teenagers my age listen when things are specifically told to them,} underscoring the importance of explicit communication in shaping platform norms.

Participants acknowledged that explicit guidelines could help mitigate negative interactions, but several also proposed additional features to actively \inlinequote{regulate toxic behaviors} as a necessary complement. They recognized that \inlinequote{there's always going to be someone that might try to ruin it} (\codesign{18}). To address this, \codesign{07} suggested a \inlinequote{negative comment filter} that users could toggle to block harmful content. Additionally, \codesign{19} proposed a reporting feature that would allow users to flag individuals \inlinequote{who may spread hurtful things or judge people.} While participants were mindful of the risk of implying the platform's distrust in its users, they acknowledged that tools for enforcing platform expectations could be valuable in strengthening their efforts to foster trust-enabling environments.