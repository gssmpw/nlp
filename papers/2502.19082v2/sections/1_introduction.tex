\section{Introduction}
Social media now serves as the foundation of social interactions for teenagers~\cite{Anderson2023-qk}. However, mainstream platforms like Instagram and TikTok have increasingly shifted their focus toward engagement with ``media'' rather than fostering truly ``social'' interactions~\cite{Prinstein2023-pd}. While these platforms excel at maintaining weak ties~\cite{Ellison2011-ld}, they are less effective at fostering the closer relationships crucial for youth well-being~\cite{kim2025socialmediaisntjust}. Research suggests that social media may have contributed to a decline in the quality of social interactions~\cite{Putnam2000-ic, mcpherson2006social}, as it often neglects features that support the development of close interpersonal bonds~\cite{Lam2012-ct}. Teens seek authentic interactions online~\cite{10.1145/3686909, Reddy2024-ic}, yet self-disclosure---one of the most fundamental mechanisms for relational closeness---remains particularly difficult in digital spaces~\cite{Hogan2010-oh, Lenhart2015-cm}.

Privacy concerns play a significant role in shaping adolescent self-disclosure behaviors. Teens must navigate a complex landscape of both actual privacy risks and perceived threats amplified by alarmist narratives~\cite{Malkin-2022-RuntimePermissionsAssistants-p, kim2025privacysocialnormsystematically, de2020contextualizing}. This results in feelings of resignation, fear, and helplessness, discouraging meaningful social interactions. Despite advocacy for resilience-based approaches to privacy~\cite{Wisniewski2018-rc, Agha2023-mu}, mainstream social media platforms and regulatory policies remain heavily prevention-focused~\cite{meta-sue, Other-Other-NYState2023-S7694A-p, Kim-2024-AustraliaBarred-y}, reinforcing a culture of anxiety rather than empowerment~\cite{Weinstein2022-rh}. 

Furthermore, privacy on social media is inherently networked\cite{marwick2014networked, Petronio2002-ce}. While platforms tend to emphasize individual privacy controls, teens' primary concerns often stem from interpersonal privacy risks---such as peer scrutiny, judgment, or parental oversight---rather than corporate data collection\cite{zhao2022understanding}. However, current privacy designs largely ignore the collective nature of boundary regulation, failing to support teens in navigating evolving trust dynamics with their social circles~\cite{Lowens-2025-MisalignmentsDemographicFacebook-d}.

Given the significance of supporting self-disclosure as a mechanism for meaningful social connection and recognizing the role of trust in adolescents' boundary regulation, we explore the following research questions:
\begin{itemize}
    \item \textbf{RQ1:} How do teens navigate self-disclosure within social media environments, and what factors influence their decisions to share?
    \item \textbf{RQ2:} How does social media design support or undermine trust-based self-disclosure among teens?
\end{itemize}
To answer these questions, we conducted a three-part co-design study with 19 adolescents aged 13 to 18. The study comprised an entry interview to understand teens' concerns regarding online self-disclosure, a one-week diary study to capture specific instances where they hesitated to disclose meaningful content, and a co-design exit interview to explore solutions that would better support self-disclosure in digital spaces. 

Our findings reveal that teens desire a social media environment where they feel comfortable sharing everyday moments with their online friends as a way to maintain and strengthen relationships. However, while self-disclosure is more intuitive within trusted circles, interactions with audiences whose trust levels are ambiguous often lead to defaulting to disengagement. Teens recognize that trust can develop over time, yet platform designs often fail to support this progression. Instead, interactions with not-yet-trusted ``Friends'' are shaped by platform norms that encourage shallow engagement and, in many cases, erode rather than build trust. The teen participants suggested several features to counter this by allowing greater alignment with audience expectations, facilitating selective and context-specific sharing, and reducing the (perceived) social risk of disclosure.

Through this work, we present empirical evidence that teens want to engage in self-disclosure but feel constrained by current social media design limitations. We highlight specific interaction patterns that undermine trust and propose design affordances---both existing and new---that can shift these interactions toward trust-building. Additionally, we introduce design guidelines that reframe privacy not as a trade-off between control and disclosure but as a dynamic, \textbf{\textit{trust-enabled}} process that fosters relational development. We encourage social media designers to move beyond traditional information-control-based privacy approaches and consider self-disclosure as a relational practice that requires both initial trust scaffolding and mechanisms for reinforcing trust through repeated interactions. By integrating these insights, platforms can better support teens' social needs, ultimately contributing to a more empowering and trust-enabling online environment.

