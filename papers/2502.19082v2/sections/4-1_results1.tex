\section{Results}
\label{section:4}
Through our interviews and diary study data, we find that teens perceive casual, mundane, and frequent sharing of their daily lives as meaningful self-disclosure---helpful in building relationships but not too intimate or high-stakes. However, many hesitated to share due to concerns about their content being perceived as underwhelming or trivial. Our study also reveals that trust is crucial in how teens regulate boundaries around self-disclosure. When trust in a relationship feels uncertain, teens face a dilemma: self-disclosure is key to strengthening connections, yet within the social media environment, sharing with an ambiguous audience often feels more risky than rewarding. As a result, many teens default to self-censorship, limiting the potential for relationship building.

This dynamic suggests that the way social media privacy is portrayed may discourage trust-building rather than support it. We explore two overarching themes of barriers to meaningful self-disclosure: uncertainties in communication \ie{\textit{communication fog}} and a high-stakes environment that tends to skew these uncertainties towards negative interpretations \ie{\textit{low-grace culture}}. Additionally, we examine teens' design suggestions for addressing these barriers, providing insights into potential platform improvements. We denote quotes with (\entry{XX}) for entry interview data, (\diary{XX}) for diary entries, and (\codesign{XX}) for co-design interview data.

\subsection{Trust as Key for Boundary Regulation}
\label{section:4-1}
Through our interviews and diary studies, we found that trust functions as a key determinant in shaping teens' self-disclosure practices and their ability to regulate boundaries online---trusted friends or interactions that build trust facilitated more open and frequent sharing, while non-trusted individuals or \textit{ambiguous} connections led to hesitations.


\subsubsection{\textbf{Self-Disclosure as a Balancing Act Between Relational Benefits and Privacy Concerns}}
\label{section:4-1-1}
Participants recognized the relational benefits of disclosure. They desired to share mundane details or small updates from their lives on social media but hesitated due to concerns about how their content would be perceived. For example, one participant mentioned wanting to share small updates in their lives, such as \inlinequote{I made some microwave popcorn and at the end there were only 11 pop kernels at the bottom of the bag} (\entry{19}). One participant who initially had concerns about being perceived as oversharing, reflected: 

\blockquote{When\ldots{} I see something like this that maybe shares details that are a little more personal I definitely feel like I connect to it more, and that's when I kind of realize maybe like sharing a couple like personal details what might be considered oversharing to some people wasn't always a bad thing.}{{\codesign{08}}}

However, many teens refrained from posting such content due to an internalized expectation that self-disclosure must be curated. They worried that their posts were \inlinequote{not memorable and significant enough to be shared} (\diary{03}) or \inlinequote{not Instagram worthy} (\diary{08}), leading to a tendency toward self-censorship. Some participants expressed frustration at this pressure, wanting to \inlinequote{just post} (\codesign{09}) without feeling \inlinequote{judged} (\codesign{09}).


\subsubsection{\textbf{Trusted Friends as a Safe Space for Self-Disclosure}}
\label{section:4-1-2}
Teens shared that having a way to carve out their trusted circle of friends supported disclosure. As \diary{16} noted, \inlinequote{The limited nature of my friends on the app helped me feel comfortable, because I trust everyone on there.} Features such as Instagram's ``Close Friends'' reinforced this sense of security: \inlinequote{I shared these on my Close Friends story. These are people I am comfortable talking to at any time without it feeling weird or forced} (\diary{02}). Similarly, \diary{17} noted, \inlinequote{I always post a daily BeReal. I just love the app and how exclusive it is.} Moreover, a trusted circle alleviated concerns about judgment. \diary{16} elaborated, \inlinequote{These are people who either will not judge me or I simply do not care if they do.} \diary{19} echoed this sentiment that with close friends, they do not mind their posts being perceived as \inlinequote{not\ldots{} funny} or even \inlinequote{weird.}

\subsubsection{\textbf{Non-Trusted ``Friends'' as a Barrier to Self-Disclosure}}
\label{section:4-1-3}
The presence of ambiguous or distrusted connections, on the other hand, hindered self-disclosure. Teens frequently described situations where they wanted to share but refrained due to the potential presence of audiences they would not trust. On Instagram, for example, even with a private account, once someone is accepted as a ``follower,'' they gain immediate access to all shared content unless the user curates a ``Close Friends'' list. As \diary{16} explained, \inlinequote{Well I WANTED to share it with friends, but my followers on Instagram encompass more than that. I don't want to be judged.}

Platform norms further complicated boundary management. Many teens felt compelled to accept ``Follow'' requests, even from those who they did not fully trust, to avoid appearing rude. As \codesign{17} described, Instagram often resembled a \inlinequote{popularity contest} where the \inlinequote{follower to following ratio} (\codesign{15}) dictated social standing. This dynamic pressured users to \inlinequote{accept all the requests\ldots{} just to increase [their] follower count} (\codesign{05}) and made it harder to regulate self-disclosure. In contrast, platforms like BeReal, which de-emphasize follower metrics and provide clearer expectations for social boundaries, were seen as more conducive to trust-based boundary regulation: \inlinequote{The hidden follower count helps, so I don't feel pressured to get a lot of followers} (\diary{16}).

\subsubsection{\textbf{Evolving Boundaries and the Role of Trust}}
\label{section:4-1-4}
Our findings show that trust-building on social media mirrors that from prior studies: trust initiation begins with sharing, and repeated reciprocated interactions are essential for trust to develop and solidify. Participants highlighted that building trust with peers online starts with self-disclosure:
\blockquote{For me, building trust looks\ldots{} kind of like getting to post what you like, and maybe having a small two-minute conversation can help build a little more trust.}{\codesign{13}}
Similarly, \codesign{01} shared that trust is fostered through \inlinequote{messaging them and liking or commenting on each other's posts.} 

Participants also confirmed that favorable responses to self-disclosure are critical in trust-building. For example, \codesign{10} noted that trust comes mostly from \inlinequote{the way a person responds to a post that [they] send out.} They elaborated, \inlinequote{if I post a status update and the person maybe relates or comforts me or something, then that can obviously make me trust them.} Reciprocity was seen as key to building trust:
\blockquote{[Trust is a] mutual thing like if you reply more to me and I will talk to you more that trust kind of builds because we get to know each other more.}{\codesign{13}}
The role of effort was also emphasized, as comments that are \inlinequote{sweet, engaging, and creative,} for example, nurture trust and a sense of community (\codesign{06}, \codesign{16}). 

Conversely, a lack of reciprocity often leads to the erosion of trust. For example, \codesign{10} shared, \inlinequote{If they respond to it negatively, then my trust in them definitely decreases.} Non-responsiveness also damages trust, as \codesign{10} explained: if friends or followers \inlinequote{view it and not do anything} when they \inlinequote{explicitly ask for} something, it decreases trust. This lack of engagement, described as giving the \inlinequote{cold shoulder} (\entry{17}), discourages further self-disclosure (\codesign{01}, \entry{17}). Sometimes, trust erosion extends to groups, as \codesign{14} explained: if they \inlinequote{get like 300 views on the story but then only 150 likes}, it often signifies that \inlinequote{people saw it but didn't really do anything,} leading them to \inlinequote{take it the wrong way} and believe that people \inlinequote{don't like [them] anymore.}

\begin{table*}[!ht]
\centering
\small
\caption{Taxonomy of designs derived from the co-design study.}
\label{tab:designs}
\begin{tabular}{p{3.5cm} p{12cm} p{1.5cm}}
\toprule
\textbf{Overall Design Idea} & \textbf{Associated Barrier to Trust-Enabled Privacy} & \textbf{Example Prototype} \\ 
\midrule

\textbf{Guided Disclosure}\newline{}[Section \ref{section:4-2-1}] & Helps clarify \textbf{\textit{Ambiguous Norms}} by providing clear guidelines and expectations around posting, reducing uncertainty and the need for users to interpret implicit social expectations independently & Figure \ref{fig:prototypes}(a) \\

\textbf{Mutual Commitment}\newline{}[Section \ref{section:4-2-2}] & Addresses \textbf{\textit{Ambiguous Loyalty}} by allowing viewers to opt in or out of content while requiring them to agree to the space's rules before joining, ensuring mutual commitment and accountability for both the sharer and the audience. & Figure \ref{fig:prototypes}(g) \\

\textbf{Contextual Disclosure}\newline{}[Section \ref{section:4-2-3}] & Counteracts \textbf{\textit{Ambiguous Relevance}} by enabling segmentation of posts for specific interest groups, ensuring content reaches the right audience & Figure \ref{fig:prototypes}(c) \\

\textbf{Intentional Signaling}\newline{}[Section \ref{section:4-2-4}] & Clarifies \textbf{\textit{Ambiguous Reactions}} by encouraging more meaningful, context-specific reactions beyond simple `Likes' & Figure \ref{fig:prototypes}(f) \\

\textbf{Low-Stakes Disclosure}\newline{}[Section \ref{section:4-3-1}] & Alleviates \textbf{\textit{Presentation Expectations}} by offering casual, subtle, or ephemeral sharing options that lower the pressure to curate perfect content & Figure \ref{fig:prototypes}(b) \\

\textbf{Contextual Clarity}\newline{}[Section \ref{section:4-3-2}] & Reduces the \textbf{\textit{Social Risk of Misrepresentation}} common in CMC by allowing users to add contextual information, reducing likelihood of misunderstandings and misjudgments & Figure \ref{fig:prototypes}(e) \\

\textbf{Self-Contained Disclosure}\newline{}[Section \ref{section:4-3-3}] & Prevents \textbf{\textit{Nonconsensual Exposure}} by ensuring that every user has a dedicated space from the start, enabling them to share without worrying about burdening others by ``clogging'' their feeds & Figure \ref{fig:prototypes}(d) \\

\textbf{Trust-Centered Norms}\newline{}[Section \ref{section:4-3-4}] & Tackles the \textbf{\textit{Cycle of Distrust}} and cultivates positive community norms by setting clear behavioral guidelines, enabling users with aligned intentions to join and mutually enforce these norms & Figure \ref{fig:prototypes}(h) \\

\bottomrule
\end{tabular}
\end{table*}

