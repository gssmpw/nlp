\subsection{Communication Fog: Barrier to Boundary Regulation and Trust Calibration}
\label{section:4-2}
One key aspect that complicated trust-based boundary regulation was the uncertainty about how their posts are received on social media. Participants were unsure of platform norms, leaving them burdened with individual responsibility to navigate about what or how much to share. They also question whether their audience might judge them or not be interested in what they post. Additionally, interpreting reactions---especially low-effort, obligatory responses such as ``Likes''---is confusing, making it difficult to gauge genuine engagement or appreciation. These ambiguities create a ``communication fog'' that complicates trust calibration and boundary regulation, making users hesitant to share in ways that support relationship-building.


\subsubsection{Ambiguous Norms: Unclear Sharing Expectations and Individual Burden}
\label{section:4-2-1}
Participants highlighted the significant influence of peer norms on their self-disclosure behaviors, emphasizing their tendency to align their sharing with the perceived expectations within their social circles. Teens carefully calibrate their disclosures to neither exceed nor fall below the perceived norm, as \entry{02} noted: \inlinequote{Oftentimes, I kind of match what my friends post in terms of how public they are.} This balancing act reflects an ongoing negotiation of social expectations, with \codesign{07} stating: \inlinequote{So you don't want to overshare and you don't want to undershare.} 

However, many platforms lack clearly defined norms for what constitutes an appropriate level of sharing, leaving users to navigate ambiguous expectations independently. Participants interpret implicit norms:
\blockquote{[On Instagram,] Not famous people, in my experience, do not normally post random stuff like this, so it would be weird to make it my first post.}{\diary{16}}
Therefore, Instagram necessitates users to \inlinequote{make an active choice} (\codesign{05}) about when and what to share. Hence, self-initiated posting can sometimes be interpreted as \inlinequote{attention-seeking} (\diary{16}), and teens fear (potential) backlash. For instance, \codesign{12} had been asked \inlinequote{why do you post so much on Instagram?}, and \codesign{11} described having been ``unfollowed'' by a peer who deemed their content uninteresting: \inlinequote{He was like, oh you don't post anything interesting. All you do is post selfies.}

Participants expressed a desire for \textbf{guided disclosure} (Figure \ref{fig:prototypes}(f)), where platforms would offer clear, explicit cues that establish expectations for how much, how often, and when to share. By creating shared reference points for appropriate disclosure, these prompts would reduce the burden of individual decision-making and reinforce the reciprocity of self-disclosure, reassuring users that their vulnerability is not being interpreted as an isolated attempt for attention but rather as a socially expected practice. 

More specifically, participants envisioned being able to make \inlinequote{a quick update on [their] life, even if it's quite boring} (\diary{05}). Platforms like BeReal, which enforce a collective norm of casual sharing at a predefined time, were appreciated for \inlinequote{give[ing] everyone a chance to shout out} (\codesign{05}). Almost all participants advocated for system-generated \textit{prompts} or reminders that would encourage low-stakes sharing and provide implicit permission to disclose, fostering a \inlinequote{sense of community} (\codesign{12}, \codesign{16}) while reducing concerns about \inlinequote{sharing too much} (\codesign{19}). Such interventions would also alleviate uncertainty when users feel like they \inlinequote{can't really figure out what to post} (\codesign{07}) by \inlinequote{giv[ing] ideas about what to share} (\diary{08}), making self-disclosure a more collectively guided and less individually scrutinized process.



\subsubsection{Ambiguous Loyalty: When Followers Feel Like Skeptics}
\label{section:4-2-2}
Participants expressed that social media ``friends'' or ``followers'' often lack the established trust of real-world friendships. While these connections did not warrant removal---sometimes due to fears of social repercussions or backlash---their intentions and judgments remained uncertain. This ambiguity created a challenge in boundary regulation as users struggled to assess how their audience might interpret their posts. As \diary{08} explains: \inlinequote{I don't want some of my friends to think I'm oversharing or being `cringey.' There are always risks. You never can really know who's viewing your content or who's REALLY following you.}

To mitigate these uncertainties and foster a greater sense of trust-based boundary control, participants proposed supporting \textbf{mutual commitment} (Figure \ref{fig:prototypes}(g)) in content consumption. One suggested mechanism involved posters sending invitations, requiring viewers to actively opt-in (\codesign{16}) or out (\codesign{03}): \inlinequote{A pop-up of like `Oh you've been invited to join'\ldots with a yes or a no button,} and offering \inlinequote{the option to either join it or leave if you don't want to follow those rules} (\codesign{11}). This approach redistributes accountability---allowing viewers to regulate their access while relieving sharers of sole responsibility for audience reactions. A participant explained how mutually agreed sharing, such as that on Snapchat private Stories, provides reassurance: 
\blockquote{If they joined that story, [then] they're kind of the ones subjecting themselves to it. If they didn't want to see it, they'd never had to join it.}{\entry{12}}



\subsubsection{Ambiguous Relevance: Struggles to Identify the Right Audience}
\label{section:4-2-3}
Participants expressed a strong desire to curate their audience, even when viewers were not overtly toxic or judgmental. This need stems from their multifaceted identities and diverse interests, which they prefer to compartmentalize for different social groups. Participants shared various interests that they felt would only engage a subset of their audience, from poetry (\entry{11}) to K-Pop (\entry{03}). One participant explained the challenge of this:
\blockquote{sometimes I don't want to bore some of my friends with [a specific interest of theirs]\ldots{} I don't really feel the need to share it with them because as much as I would like them to share my interest in it, I don't want them to find that kind of burdensome.}{\codesign{14}} 

While many platforms employ algorithms to curate content based on inferred interests, participants found these coarse and ineffective. Rather, participants advocated for self-managed and dynamic calibration of social boundaries through \textbf{contextual disclosure} (Figure \ref{fig:prototypes}(b)), allowing them to strategically engage with groups based on the established and/or perceived potential for trust. Some users created elaborate systems of nested groups via Snapchat's private stories feature: 
\blockquote{I have four [private stories], each getting smaller to accommodate things I choose to share with\ldots{} The closer you are to me, the more (and smaller) private stories you're in.}{\diary{15}} 
\entry{16} reflected on the flexibility of Discord as providing similar benefits. 
These user-curated spaces would not just serve as secure places for existing trust but as controlled environments where users could gradually calibrate their social boundaries, engaging in disclosure that fosters emerging trust while minimizing the risks of sharing too broadly.

Participants were particularly keen on interest-based spaces, where shared interests would serve as a bridge to potential trust development. They noted that while common interests alone do not equate to trust, they provide an initial context that reduces the uncertainties of broader public sharing and, ultimately, allows trust to form over time. As \codesign{08} explained, such intentional segmentation \inlinequote{could definitely help people connect more because they'd feel like they had more to talk about.}


\subsubsection{Ambiguous Reactions: When ``Likes'' Fail to Signal Trust}
\label{section:4-2-4}
Participants expressed confusion about interpreting the meaning of reactions they receive to their posts, particularly when those reactions felt \inlinequote{obligatory} (\entry{19}). \entry{19} explained, \inlinequote{you feel more obligated to like a post\ldots [so] it just feels worse to post and not get as many likes.} Similarly, \codesign{08} remarked that when \inlinequote{people you're close with\ldots don't interact with your post,} they begin to question, \inlinequote{Is it really that boring that even people I know don't care?} While ``Likes'' may have been designed to signal validation, their meaning has become obscured in many platforms, making it an ineffective signal for reciprocity or trust.

To reduce ambiguity in social interactions, participants expressed interest in having a broader range of mechanisms for \textbf{intentional signaling} (Figure \ref{fig:prototypes}(d)) beyond effortless reactions such as ``Likes''. They cited examples from existing platforms where engagement from other users served as a clear, indisputable signal of care. For instance, the range and expressive nature of emoji reactions requires more deliberation than \inlinequote{just double tap[ing]}(\codesign{15}). When the chosen emoji aligns with the content, it sends clear signals that the actor actually viewed it. Comments were also valued as another form of clear signals of trust as they were seen as requiring individuals \inlinequote{go out of their way to be nice} (\codesign{08}). Participants also appreciated Instagram's Story Likes, describing them as \inlinequote{not obligatory at all} and as a feature that \inlinequote{shows that\ldots they decided to go out of their way to like it} (\entry{19}). To \codesign{13}, receiving Story Likes brings \inlinequote{a lot of trust[s]} given the private---and therefore more intimate, deliberate, and less likely performative---nature of those reactions. These reflections highlight that more nuanced, intentional reactions could serve as better social signals that reduce the ambiguity in boundary regulation.

