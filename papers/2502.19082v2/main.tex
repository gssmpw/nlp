%%%%%%%%%%%%%%%%%%%%%%%%%%%%%%%%%%%%%%%%%%%%%%%%%%%%%%%%%%%%%%%%%%%%%%%%%%%%%%%%
% Template for USENIX papers.
%
% History:
%
% - TEMPLATE for Usenix papers, specifically to meet requirements of
%   USENIX '05. originally a template for producing IEEE-format
%   articles using LaTeX. written by Matthew Ward, CS Department,
%   Worcester Polytechnic Institute. adapted by David Beazley for his
%   excellent SWIG paper in Proceedings, Tcl 96. turned into a
%   smartass generic template by De Clarke, with thanks to both the
%   above pioneers. Use at your own risk. Complaints to /dev/null.
%   Make it two column with no page numbering, default is 10 point.
%
% - Munged by Fred Douglis <douglis@research.att.com> 10/97 to
%   separate the .sty file from the LaTeX source template, so that
%   people can more easily include the .sty file into an existing
%   document. Also changed to more closely follow the style guidelines
%   as represented by the Word sample file.
%
% - Note that since 2010, USENIX does not require endnotes. If you
%   want foot of page notes, don't include the endnotes package in the
%   usepackage command, below.
% - This version uses the latex2e styles, not the very ancient 2.09
%   stuff.
%
% - Updated July 2018: Text block size changed from 6.5" to 7"
%
% - Updated Dec 2018 for ATC'19:
%
%   * Revised text to pass HotCRP's auto-formatting check, with
%     hotcrp.settings.submission_form.body_font_size=10pt, and
%     hotcrp.settings.submission_form.line_height=12pt
%
%   * Switched from \endnote-s to \footnote-s to match Usenix's policy.
%
%   * \section* => \begin{abstract} ... \end{abstract}
%
%   * Make template self-contained in terms of bibtex entires, to allow
%     this file to be compiled. (And changing refs style to 'plain'.)
%
%   * Make template self-contained in terms of figures, to
%     allow this file to be compiled. 
%
%   * Added packages for hyperref, embedding fonts, and improving
%     appearance.
%   
%   * Removed outdated text.
%
%%%%%%%%%%%%%%%%%%%%%%%%%%%%%%%%%%%%%%%%%%%%%%%%%%%%%%%%%%%%%%%%%%%%%%%%%%%%%%%%

%%% Minor updates for SOUPS 2019 by Michelle Mazurek
%%% Minor updates for SOUPS 2022 by Rick Wash

\documentclass[letterpaper,twocolumn,10pt]{article}
\usepackage{usenix2025_SOUPS}

% to be able to draw some self-contained figs
\usepackage{tikz}
\usepackage{amsmath}
\PassOptionsToPackage{table,xcdraw}{xcolor} 

\usepackage{textgreek}
\usepackage{booktabs}
\usepackage{multirow}
\usepackage{subcaption}
\usepackage{xcolor,colortbl}
\definecolor{gray}{rgb}{0.1,0.1,0.1}
\usepackage[T1]{fontenc}
\usepackage[english]{babel}
\usepackage{graphicx}
\usepackage[flushleft]{threeparttable}
\usepackage{adjustbox}
\usepackage{subcaption}
\usepackage{longtable}
\usepackage{threeparttablex}

% inlined bib file
\usepackage{filecontents}

%-------------------------------------------------------------------------------
\begin{filecontents}{references.bib}
%-------------------------------------------------------------------------------
\end{filecontents}


\usepackage{times}
\usepackage{latexsym}

\usepackage[T1]{fontenc}

\usepackage[utf8]{inputenc}

\usepackage{microtype}

\usepackage{inconsolata}

\usepackage{graphicx}


\usepackage{amsmath}
\usepackage{amssymb}
\usepackage{multirow}
\usepackage{booktabs}
\usepackage{catchfile}

\usepackage[boxed]{algorithm}
\usepackage{varwidth}
\usepackage[noEnd=true,indLines=false]{algpseudocodex}
\usepackage{cleveref}
\makeatletter
\@addtoreset{ALG@line}{algorithm}
\renewcommand{\ALG@beginalgorithmic}{\small}
\algrenewcommand\alglinenumber[1]{\small #1:}
\makeatother

\usepackage[normalem]{ulem}
\usepackage{todonotes}

\usepackage{lipsum}    %
\usepackage{comment}   %
\usepackage{graphicx}  %
\usepackage{pifont}    %

\usepackage[font=small,labelfont=bf]{caption}
\usepackage{float}     %
\usepackage{booktabs}  %
\usepackage{subcaption}  %

\usepackage{listings}

\usepackage{amsthm}  %



%-------------------------------------------------------------------------------
\begin{document}
%-------------------------------------------------------------------------------

%don't want date printed
\date{}

\title{\Large \bf Trust-Enabled Privacy:\\Social Media Designs to Support Adolescent User Boundary Regulation}

% if you leave this blank it will default to a possibly ugly attempt 
% to make the contents of the \author command below into a string
\def\plainauthor{Author name(s) for PDF metadata. Don't forget to anonymize for submission!}

%for single author (just remove % characters)
\author{
{\rm JaeWon Kim}\\
University of Washington
\and
{\rm Robert Wolfe}\\
University of Washington
\and
{\rm Ramya Bhagirathi Subramanian}\\
Smarten Spaces
\and
{\rm Mei-Hsuan Lee}\\
National Renewable Energy Laboratory
\and
{\rm Jessica Colnago}\\
{  }
\and
{\rm Alexis Hiniker}\\
University of Washington
% copy the following lines to add more authors
% \and
% {\rm Name}\\
%Name Institution
} % end author

\maketitle
% \thecopyright

\begin{abstract}
Through a three-part co-design study involving 19 teens aged 13–18, we identify key barriers to effective boundary regulation on social media, including ambiguous audience expectations, social risks associated with oversharing, and the lack of design affordances that facilitate trust-building. Our findings reveal that while adolescents seek casual, frequent sharing to strengthen relationships, existing platform norms and designs often discourage such interactions, leading to withdrawal. To address these challenges, we introduce \textit{trust-enabled privacy} as a design framework that recognizes trust---whether building or eroding---as central to boundary regulation. When trust is supported, boundary regulation becomes more adaptive and empowering; when it erodes, users default to self-censorship or withdrawal. We propose concrete design affordances, including guided disclosure, contextual audience segmentation, intentional engagement signaling, and trust-centered norms, to help platforms foster a more dynamic and nuanced privacy experience for teen social media users. By reframing privacy as a trust-driven process rather than a rigid control-based trade-off, this work provides empirical insights and actionable guidelines for designing social media environments that empower teens to manage their online presence while fostering meaningful social connections.
\end{abstract}

\section{Introduction}

Video generation has garnered significant attention owing to its transformative potential across a wide range of applications, such media content creation~\citep{polyak2024movie}, advertising~\citep{zhang2024virbo,bacher2021advert}, video games~\citep{yang2024playable,valevski2024diffusion, oasis2024}, and world model simulators~\citep{ha2018world, videoworldsimulators2024, agarwal2025cosmos}. Benefiting from advanced generative algorithms~\citep{goodfellow2014generative, ho2020denoising, liu2023flow, lipman2023flow}, scalable model architectures~\citep{vaswani2017attention, peebles2023scalable}, vast amounts of internet-sourced data~\citep{chen2024panda, nan2024openvid, ju2024miradata}, and ongoing expansion of computing capabilities~\citep{nvidia2022h100, nvidia2023dgxgh200, nvidia2024h200nvl}, remarkable advancements have been achieved in the field of video generation~\citep{ho2022video, ho2022imagen, singer2023makeavideo, blattmann2023align, videoworldsimulators2024, kuaishou2024klingai, yang2024cogvideox, jin2024pyramidal, polyak2024movie, kong2024hunyuanvideo, ji2024prompt}.


In this work, we present \textbf{\ours}, a family of rectified flow~\citep{lipman2023flow, liu2023flow} transformer models designed for joint image and video generation, establishing a pathway toward industry-grade performance. This report centers on four key components: data curation, model architecture design, flow formulation, and training infrastructure optimization—each rigorously refined to meet the demands of high-quality, large-scale video generation.


\begin{figure}[ht]
    \centering
    \begin{subfigure}[b]{0.82\linewidth}
        \centering
        \includegraphics[width=\linewidth]{figures/t2i_1024.pdf}
        \caption{Text-to-Image Samples}\label{fig:main-demo-t2i}
    \end{subfigure}
    \vfill
    \begin{subfigure}[b]{0.82\linewidth}
        \centering
        \includegraphics[width=\linewidth]{figures/t2v_samples.pdf}
        \caption{Text-to-Video Samples}\label{fig:main-demo-t2v}
    \end{subfigure}
\caption{\textbf{Generated samples from \ours.} Key components are highlighted in \textcolor{red}{\textbf{RED}}.}\label{fig:main-demo}
\end{figure}


First, we present a comprehensive data processing pipeline designed to construct large-scale, high-quality image and video-text datasets. The pipeline integrates multiple advanced techniques, including video and image filtering based on aesthetic scores, OCR-driven content analysis, and subjective evaluations, to ensure exceptional visual and contextual quality. Furthermore, we employ multimodal large language models~(MLLMs)~\citep{yuan2025tarsier2} to generate dense and contextually aligned captions, which are subsequently refined using an additional large language model~(LLM)~\citep{yang2024qwen2} to enhance their accuracy, fluency, and descriptive richness. As a result, we have curated a robust training dataset comprising approximately 36M video-text pairs and 160M image-text pairs, which are proven sufficient for training industry-level generative models.

Secondly, we take a pioneering step by applying rectified flow formulation~\citep{lipman2023flow} for joint image and video generation, implemented through the \ours model family, which comprises Transformer architectures with 2B and 8B parameters. At its core, the \ours framework employs a 3D joint image-video variational autoencoder (VAE) to compress image and video inputs into a shared latent space, facilitating unified representation. This shared latent space is coupled with a full-attention~\citep{vaswani2017attention} mechanism, enabling seamless joint training of image and video. This architecture delivers high-quality, coherent outputs across both images and videos, establishing a unified framework for visual generation tasks.


Furthermore, to support the training of \ours at scale, we have developed a robust infrastructure tailored for large-scale model training. Our approach incorporates advanced parallelism strategies~\citep{jacobs2023deepspeed, pytorch_fsdp} to manage memory efficiently during long-context training. Additionally, we employ ByteCheckpoint~\citep{wan2024bytecheckpoint} for high-performance checkpointing and integrate fault-tolerant mechanisms from MegaScale~\citep{jiang2024megascale} to ensure stability and scalability across large GPU clusters. These optimizations enable \ours to handle the computational and data challenges of generative modeling with exceptional efficiency and reliability.


We evaluate \ours on both text-to-image and text-to-video benchmarks to highlight its competitive advantages. For text-to-image generation, \ours-T2I demonstrates strong performance across multiple benchmarks, including T2I-CompBench~\citep{huang2023t2i-compbench}, GenEval~\citep{ghosh2024geneval}, and DPG-Bench~\citep{hu2024ella_dbgbench}, excelling in both visual quality and text-image alignment. In text-to-video benchmarks, \ours-T2V achieves state-of-the-art performance on the UCF-101~\citep{ucf101} zero-shot generation task. Additionally, \ours-T2V attains an impressive score of \textbf{84.85} on VBench~\citep{huang2024vbench}, securing the top position on the leaderboard (as of 2025-01-25) and surpassing several leading commercial text-to-video models. Qualitative results, illustrated in \Cref{fig:main-demo}, further demonstrate the superior quality of the generated media samples. These findings underscore \ours's effectiveness in multi-modal generation and its potential as a high-performing solution for both research and commercial applications.
\section{Related Work}
\subsection{Defining Social Media and its Role in Young People's Lives}
Social media does not yet have a widely agreed-upon definition. There is a range of interpretations, from broad (e.g., ``any interactive communication medium that enables two-way interaction and feedback''~\cite{kent2010directions}) to specific (e.g., ``a different kind of Internet-based applications which build the ideological and technological foundations of Web 2.0, and allow users to create content and exchange it with other people through the Internet''~\cite{kaplan2010users}). While some platforms, such as Instagram~\cite{instagram} or Facebook~\cite{facebook-official}, are indisputably considered social media, others, such as YouTube~\cite{youtube} and Discord~\cite{discord}, are less archetypal. There are also different types of social media, and even such classifications have multiple interpretations~\cite{kaplan2010users, sharma2018social, scott2015new}. Some major categories include social networking sites (i.e., ``web-based services that allow individuals to (1) construct a public or semi-public profile within a bounded system, (2) articulate a list of other users with whom they share a connection, and (3) view and traverse their list of connections and those made by others within the system''~\cite{Boyd-2007-SocialNetworkScholarship-i}) and blogs (i.e., ``a way to share that love with the world and encourage an active community of readers who comment on the posts of the author''~\cite{scott2015new}).

% \noindent \textbf{What is social media?} Social media does not yet have a definitive definition. There is a range of interpretations, from broad~\footnote{``any interactive communication medium that enables two-way interaction and feedback''~\cite{kent2010directions})} to specific~\footnote{``a different kind of Internet-based applications which build the ideological and technological foundations of Web 2.0, and allow users to create content and exchange it with other people through the Internet''~\cite{kaplan2010users}}. While some platforms, such as Instagram~\cite{instagram} or Facebook~\cite{facebook-official}, are indisputably considered social media, others, such as YouTube~\cite{youtube} and Discord~\cite{discord}, are less archetypal. There are also different types of social media, and even such classifications have multiple interpretations~\cite{kaplan2010users, sharma2018social, scott2015new}. Some major categories include social networking sites~\footnote{``web-based services that allow individuals to (1) construct a public or semi-public profile within a bounded system, (2) articulate a list of other users with whom they share a connection, and (3) view and traverse their list of connections and those made by others within the system''~\cite{Boyd-2007-SocialNetworkScholarship-i}} and blogs~\footnote{``a way to share that love with the world and encourage an active community of readers who comment on the posts of the author''~\cite{scott2015new}}.

The range of definitions and classifications of social media reflects the wide array of user needs that social media addresses. From a uses and gratifications perspective~\cite{GurevitchKatz-1973-UsesGratificationsResearch-y}, social media supports a variety of utilities, ranging from social interactions to information seeking and entertainment~\cite{whiting2013people}. Even within such broad categories of uses and gratifications, there are diverse ways these needs are pursued. For example, within social interactions, there may be a focus on bridging social capital and weaker ties versus bonding social capital and stronger ties~\cite{phua_uses_2017}. Further, even on a single platform---Instagram, for instance---users engage in a range of activities, from photo sharing for self-promotion and disclosure~\cite{Menon-2022-UsesGratificationsInstagram-x} to more passive consumption of media for entertainment. In other words, Instagram may serve as a personal branding tool for some while functioning as a way to pass the time for others.

% \parHeading{Youth experience with social media} Despite the prevalent narrative that social media is bad, research shows the youth experience with social media is more nuanced. While younger people have bad experiences on social media such as social comparison~\cite{Yau2019-ab}, significant privacy concerns, pressures to curate~\cite{BoydDanah2014ICTS}, time spent feeling meaningless~\cite{HinikerLukoff-2018-WhatMakes-o}, and more mild forms of discomfort~\cite{Landesman-2024-IInstagram-j}, they experience meaningful benefits from those platforms as well. For instance, marginalized populations have been known to benefit particularly from these sites~\cite{AcenaLi-2023-WeReality-s, BellBates-2020-"LetMeDevelopment-u} and these platforms are used as medium for identity development and expression~\cite{Davis2012-bq, Lee-2022-AlgorithmicCrystalTikTok-b}. Given such a nuanced relationship between social media and youth, researchers recommend that it is beneficial to help teens learn to be more resilient with social media experiences rather than focusing on restricting or preventing harm~\cite{Wisniewski2012-nu}.

Despite the prevalent narrative that social media is harmful, the experiences of youth with these platforms are more nuanced. They face challenges such as social comparison~\cite{Yau2019-ab}, significant privacy concerns~\cite{Weinstein2022-rh, kim2024privacysocialnormsystematically, Boyd_2014}, pressures to curate their online presence~\cite{Yau2019-ab}, and feelings of meaningless time spent~\cite{HinikerLukoff-2018-WhatMakes-o}, along with more subtle discomforts~\cite{Landesman-2024-IInstagram-j}. However, these platforms also offer meaningful benefits. For instance, marginalized populations often derive substantial support from these spaces~\cite{AcenaLi-2023-WeReality-s, BellBates-2020-"LetMeDevelopment-u}, and social media serves as an important medium for identity development and self-expression~\cite{Davis2012-bq, Lee-2022-AlgorithmicCrystalTikTok-b}. Given this complex relationship between youth and social media, researchers advocate for strategies that focus on building resilience and equipping teens to navigate these experiences effectively rather than solely aiming to restrict or mitigate harm~\cite{Wisniewski2012-nu, Wisniewski-2018-PrivacyParadox-l, WisniewskiAgha-2023-StrikePrevention-r}.

% paper on Time spent on smartphone feels meaningless vs. meaningful (Kai)

% papers that talk about The bad side
% - correlation between youth social media use and mental health (though not causal)
% - social comparison, mental health, FOMO, filtered and curated posts, only positive ones
% - more mild
% - context collapse -> can't authentically share self

% papers that talk about The good side
% - queer, psychiatric hospitalization
% - identity crystallization
% - political activism
% - emotion regulation

% importance of social media --> shouldn't just ban/restrict --> need to teach resilience



\subsection{Social Technology Design Explorations in HCI}
% HCI research has explored different designs that foster meaningful social interactions online. Ranging from social media platforms such as BeReal~\cite{HinikerKim-2024-SharingDesign-x}, Snapchat~\cite{Bayer2016-fa}, Miitomo~\cite{KaufmanKasunic-2017-BeMeApplication-h}, and TikTok~\cite{Barta2021-yh, Schaadhardt2023-jj} to individual features such as those that support effortful communication~\cite{LiuFannie2021SOUt, LiuZhang-2022-AuggieEncouragingExperiences-n} or genuine connection~\cite{Stepanova2022-vh}. Recently more spatial social technologies such as social virtual reality (VR) platforms~\cite{Zamanifard-2023-SurpriseBirthdayDistance-n, Freeman-2024-MyAudiences-u, Freeman-2020-MyBodyReality-l, Zamanifard-2019-TogethernessCraveRelationships-s}, Gather.town~\cite{duarte2023experience,tu2022meetings}, and VRChat~\cite{chen2024d,deighan2023social,rzeszewski2024social} have been explored.

HCI research has extensively explored designs that foster meaningful social interactions online, examining both platform-level innovations and individual features. Social media platforms such as BeReal~\cite{HinikerKim-2024-SharingDesign-x}, Snapchat~\cite{Bayer2016-fa}, Miitomo~\cite{KaufmanKasunic-2017-BeMeApplication-h}, and TikTok~\cite{Barta2021-yh, Schaadhardt2023-jj} have introduced unique affordances for sharing and connecting, with an emphasis on spontaneity, ephemerality, and creative self-expression. This line of research identified different ways to build and maintain relationships through features that prioritize authenticity and playful interaction. However, users have reported increasingly toxic interactions as they gain popularity, with design choices encouraging users to expand their networks and engage in compulsive behaviors, such as tracking Snap scores~\cite{chambers2022s, van2023snapchat}, which detract from their initial goals of fostering meaningful interactions. 

At the feature level, efforts to support meaningful communication include designs that encourage effortful communication~\cite{LiuFannie2021SOUt, LiuZhang-2022-AuggieEncouragingExperiences-n} and those that foster feelings of genuine connections~\cite{Stepanova2022-vh}. While these features are important, social media experiences are often holistic, involving a complex interplay of multiple design aspects and user behaviors. As a result, such features alone cannot fundamentally alter the overarching narratives or perspectives surrounding social media.

Beyond traditional social media, recent work has investigated spatial social technologies, which emphasize the value of immersive, embodied interactions. Social virtual reality (VR) platforms~\cite{Zamanifard-2023-SurpriseBirthdayDistance-n, Freeman-2024-MyAudiences-u, Freeman-2020-MyBodyReality-l, Zamanifard-2019-TogethernessCraveRelationships-s} provide users with virtual spaces and embodiment for co-presence and shared experiences, facilitating interactions that feel more personal and tangible compared to 2D environments. Similarly, platforms like Gather.town~\cite{duarte2023experience,tu2022meetings} and VRChat~\cite{chen2024d,deighan2023social,rzeszewski2024social} blend elements of gaming and social networking, offering users the ability to navigate virtual environments, customize avatars, and participate in dynamic, context-rich group interactions. However, these platforms often are not generally recognized as social media in the traditional sense.

What youth envision as ideal social media, both at the broader landscape level and the specific feature level remains unclear---especially when considering the possibilities offered by emerging technologies such as spatial computing platforms. Understanding these perspectives is crucial for reimagining the future of social media and ensuring it meets the evolving needs of its users.

% snapscore - \cite{chambers2022s}

% \cite{Stepanova2022-vh}
\subsection{Designing Beyond Doom Narratives}
\parHeading{Social media and design fixation} Design fixation~\cite{jansson1991design}---the tendency to remain constrained by conventional thinking or existing solutions---is a well-documented challenge in design. While generating a wide range of divergent ideas is a critical first step in innovative design, we often impose unnecessary restrictions on our thinking. This self-limiting cognitive process often confines the design process to address localized problems rather than exploring broader, more optimal solutions.

Social media design is no exception. Given the ad-based revenue model, platforms are incentivized to increase time spent on their apps. However, instead of achieving this through meaningful utility and engagement that genuinely enhances users' experiences, they often resort to less healthy, compulsive strategies, such as friend recommendations~\cite{HinikerKim-2024-SharingDesign-x} or engagement metrics (e.g., Snap scores~\cite{rozgonjuk2021comparing}). These tactics drive superficial interactions rather than fostering deeper connections, leaving users dissatisfied and reinforcing the perception that all social media designs and user experiences are fundamentally the same~\cite{Sundaram-Other-SocialMediaSame-g, Pardes-2020-SocialMediaSame-p}.

\parHeading{The Fictional Inquiry (FI) method} FI is an exploratory co-design method that enables participants to transcend real-world constraints through immersion in semi-fictional contexts~\cite{IversenDindler-2007-FictionalInquiry--designSpace-m}. At its core, FI creates an environment where people can imagine beyond present limitations by engaging with carefully crafted scenarios. These scenarios typically incorporate immersive elements such as physical artifacts or role-playing activities, with participants taking on generative roles that help them envision futuristic or fictional settings. The method is particularly effective at helping participants overcome design fixation by encouraging them to draw analogies from fictional scenarios and imagine freely outside real-world constraints. While FI shares the exploratory nature of other ideation techniques, it is distinguished by its emphasis on developing futuristic design materials through collaborative dialogue between designers and users~\cite{IversenDindler-2007-FictionalInquiry--designSpace-m}.

According to Dindler and Iversen~\cite{IversenDindler-2007-FictionalInquiry--designSpace-m}, implementing this method involves three key steps: first, clearly defining the inquiry's purpose, whether it's for staging design situations, exploring future ideas, or driving organizational change; second, developing an appropriate narrative that participants are comfortable with while being distinct enough from current practices to encourage new thinking; and third, creating a compelling plot that introduces tension or conflict within the narrative to motivate specific workshop activities. Unlike traditional role-playing approaches, FI encourages participants to maintain their own identities and expertise while operating within fictional contexts, allowing them to explore new possibilities while drawing from their personal motivations and capabilities, all while circumventing typical societal constraints. 
\section{Study Design}
% robot: aliengo 
% We used the Unitree AlienGo quadruped robot. 
% See Appendix 1 in AlienGo Software Guide PDF
% Weight = 25kg, size (L,W,H) = (0.55, 0.35, 06) m when standing, (0.55, 0.35, 0.31) m when walking
% Handle is 0.4 m or 0.5 m. I'll need to check it to see which type it is.
We gathered input from primary stakeholders of the robot dog guide, divided into three subgroups: BVI individuals who have owned a dog guide, BVI individuals who were not dog guide owners, and sighted individuals with generally low degrees of familiarity with dog guides. While the main focus of this study was on the BVI participants, we elected to include survey responses from sighted participants given the importance of social acceptance of the robot by the general public, which could reflect upon the BVI users themselves and affect their interactions with the general population \cite{kayukawa2022perceive}. 

The need-finding processes consisted of two stages. During Stage 1, we conducted in-depth interviews with BVI participants, querying their experiences in using conventional assistive technologies and dog guides. During Stage 2, a large-scale survey was distributed to both BVI and sighted participants. 

This study was approved by the University’s Institutional Review Board (IRB), and all processes were conducted after obtaining the participants' consent.

\subsection{Stage 1: Interviews}
We recruited nine BVI participants (\textbf{Table}~\ref{tab:bvi-info}) for in-depth interviews, which lasted 45-90 minutes for current or former dog guide owners (DO) and 30-60 minutes for participants without dog guides (NDO). Group DO consisted of five participants, while Group NDO consisted of four participants.
% The interview participants were divided into two groups. Group DO (Dog guide Owner) consisted of five participants who were current or former dog guide owners and Group NDO (Non Dog guide Owner) consisted of three participants who were not dog guide owners. 
All participants were familiar with using white canes as a mobility aid. 

We recruited participants in both groups, DO and NDO, to gather data from those with substantial experience with dog guides, offering potentially more practical insights, and from those without prior experience, providing a perspective that may be less constrained and more open to novel approaches. 

We asked about the participants' overall impressions of a robot dog guide, expectations regarding its potential benefits and challenges compared to a conventional dog guide, their desired methods of giving commands and communicating with the robot dog guide, essential functionalities that the robot dog guide should offer, and their preferences for various aspects of the robot dog guide's form factors. 
For Group DO, we also included questions that asked about the participants' experiences with conventional dog guides. 

% We obtained permission to record the conversations for our records while simultaneously taking notes during the interviews. The interviews lasted 30-60 minutes for NDO participants and 45-90 minutes for DO participants. 

\subsection{Stage 2: Large-Scale Surveys} 
After gathering sufficient initial results from the interviews, we created an online survey for distributing to a larger pool of participants. The survey platform used was Qualtrics. 

\subsubsection{Survey Participants}
The survey had 100 participants divided into two primary groups. Group BVI consisted of 42 blind or visually impaired participants, and Group ST consisted of 58 sighted participants. \textbf{Table}~\ref{tab:survey-demographics} shows the demographic information of the survey participants. 

\subsubsection{Question Differentiation} 
Based on their responses to initial qualifying questions, survey participants were sorted into three subgroups: DO, NDO, and ST. Each participant was assigned one of three different versions of the survey. The surveys for BVI participants mirrored the interview categories (overall impressions, communication methods, functionalities, and form factors), but with a more quantitative approach rather than the open-ended questions used in interviews. The DO version included additional questions pertaining to their prior experience with dog guides. The ST version revolved around the participants' prior interactions with and feelings toward dog guides and dogs in general, their thoughts on a robot dog guide, and broad opinions on the aesthetic component of the robot's design. 

\section{Results}
\label{section:4}
Through our interviews and diary study data, we find that teens perceive casual, mundane, and frequent sharing of their daily lives as meaningful self-disclosure---helpful in building relationships but not too intimate or high-stakes. However, many hesitated to share due to concerns about their content being perceived as underwhelming or trivial. Our study also reveals that trust is crucial in how teens regulate boundaries around self-disclosure. When trust in a relationship feels uncertain, teens face a dilemma: self-disclosure is key to strengthening connections, yet within the social media environment, sharing with an ambiguous audience often feels more risky than rewarding. As a result, many teens default to self-censorship, limiting the potential for relationship building.

This dynamic suggests that the way social media privacy is portrayed may discourage trust-building rather than support it. We explore two overarching themes of barriers to meaningful self-disclosure: uncertainties in communication \ie{\textit{communication fog}} and a high-stakes environment that tends to skew these uncertainties towards negative interpretations \ie{\textit{low-grace culture}}. Additionally, we examine teens' design suggestions for addressing these barriers, providing insights into potential platform improvements. We denote quotes with (\entry{XX}) for entry interview data, (\diary{XX}) for diary entries, and (\codesign{XX}) for co-design interview data.

\subsection{Trust as Key for Boundary Regulation}
\label{section:4-1}
Through our interviews and diary studies, we found that trust functions as a key determinant in shaping teens' self-disclosure practices and their ability to regulate boundaries online---trusted friends or interactions that build trust facilitated more open and frequent sharing, while non-trusted individuals or \textit{ambiguous} connections led to hesitations.


\subsubsection{\textbf{Self-Disclosure as a Balancing Act Between Relational Benefits and Privacy Concerns}}
\label{section:4-1-1}
Participants recognized the relational benefits of disclosure. They desired to share mundane details or small updates from their lives on social media but hesitated due to concerns about how their content would be perceived. For example, one participant mentioned wanting to share small updates in their lives, such as \inlinequote{I made some microwave popcorn and at the end there were only 11 pop kernels at the bottom of the bag} (\entry{19}). One participant who initially had concerns about being perceived as oversharing, reflected: 

\blockquote{When\ldots{} I see something like this that maybe shares details that are a little more personal I definitely feel like I connect to it more, and that's when I kind of realize maybe like sharing a couple like personal details what might be considered oversharing to some people wasn't always a bad thing.}{{\codesign{08}}}

However, many teens refrained from posting such content due to an internalized expectation that self-disclosure must be curated. They worried that their posts were \inlinequote{not memorable and significant enough to be shared} (\diary{03}) or \inlinequote{not Instagram worthy} (\diary{08}), leading to a tendency toward self-censorship. Some participants expressed frustration at this pressure, wanting to \inlinequote{just post} (\codesign{09}) without feeling \inlinequote{judged} (\codesign{09}).


\subsubsection{\textbf{Trusted Friends as a Safe Space for Self-Disclosure}}
\label{section:4-1-2}
Teens shared that having a way to carve out their trusted circle of friends supported disclosure. As \diary{16} noted, \inlinequote{The limited nature of my friends on the app helped me feel comfortable, because I trust everyone on there.} Features such as Instagram's ``Close Friends'' reinforced this sense of security: \inlinequote{I shared these on my Close Friends story. These are people I am comfortable talking to at any time without it feeling weird or forced} (\diary{02}). Similarly, \diary{17} noted, \inlinequote{I always post a daily BeReal. I just love the app and how exclusive it is.} Moreover, a trusted circle alleviated concerns about judgment. \diary{16} elaborated, \inlinequote{These are people who either will not judge me or I simply do not care if they do.} \diary{19} echoed this sentiment that with close friends, they do not mind their posts being perceived as \inlinequote{not\ldots{} funny} or even \inlinequote{weird.}

\subsubsection{\textbf{Non-Trusted ``Friends'' as a Barrier to Self-Disclosure}}
\label{section:4-1-3}
The presence of ambiguous or distrusted connections, on the other hand, hindered self-disclosure. Teens frequently described situations where they wanted to share but refrained due to the potential presence of audiences they would not trust. On Instagram, for example, even with a private account, once someone is accepted as a ``follower,'' they gain immediate access to all shared content unless the user curates a ``Close Friends'' list. As \diary{16} explained, \inlinequote{Well I WANTED to share it with friends, but my followers on Instagram encompass more than that. I don't want to be judged.}

Platform norms further complicated boundary management. Many teens felt compelled to accept ``Follow'' requests, even from those who they did not fully trust, to avoid appearing rude. As \codesign{17} described, Instagram often resembled a \inlinequote{popularity contest} where the \inlinequote{follower to following ratio} (\codesign{15}) dictated social standing. This dynamic pressured users to \inlinequote{accept all the requests\ldots{} just to increase [their] follower count} (\codesign{05}) and made it harder to regulate self-disclosure. In contrast, platforms like BeReal, which de-emphasize follower metrics and provide clearer expectations for social boundaries, were seen as more conducive to trust-based boundary regulation: \inlinequote{The hidden follower count helps, so I don't feel pressured to get a lot of followers} (\diary{16}).

\subsubsection{\textbf{Evolving Boundaries and the Role of Trust}}
\label{section:4-1-4}
Our findings show that trust-building on social media mirrors that from prior studies: trust initiation begins with sharing, and repeated reciprocated interactions are essential for trust to develop and solidify. Participants highlighted that building trust with peers online starts with self-disclosure:
\blockquote{For me, building trust looks\ldots{} kind of like getting to post what you like, and maybe having a small two-minute conversation can help build a little more trust.}{\codesign{13}}
Similarly, \codesign{01} shared that trust is fostered through \inlinequote{messaging them and liking or commenting on each other's posts.} 

Participants also confirmed that favorable responses to self-disclosure are critical in trust-building. For example, \codesign{10} noted that trust comes mostly from \inlinequote{the way a person responds to a post that [they] send out.} They elaborated, \inlinequote{if I post a status update and the person maybe relates or comforts me or something, then that can obviously make me trust them.} Reciprocity was seen as key to building trust:
\blockquote{[Trust is a] mutual thing like if you reply more to me and I will talk to you more that trust kind of builds because we get to know each other more.}{\codesign{13}}
The role of effort was also emphasized, as comments that are \inlinequote{sweet, engaging, and creative,} for example, nurture trust and a sense of community (\codesign{06}, \codesign{16}). 

Conversely, a lack of reciprocity often leads to the erosion of trust. For example, \codesign{10} shared, \inlinequote{If they respond to it negatively, then my trust in them definitely decreases.} Non-responsiveness also damages trust, as \codesign{10} explained: if friends or followers \inlinequote{view it and not do anything} when they \inlinequote{explicitly ask for} something, it decreases trust. This lack of engagement, described as giving the \inlinequote{cold shoulder} (\entry{17}), discourages further self-disclosure (\codesign{01}, \entry{17}). Sometimes, trust erosion extends to groups, as \codesign{14} explained: if they \inlinequote{get like 300 views on the story but then only 150 likes}, it often signifies that \inlinequote{people saw it but didn't really do anything,} leading them to \inlinequote{take it the wrong way} and believe that people \inlinequote{don't like [them] anymore.}

\begin{table*}[!ht]
\centering
\small
\caption{Taxonomy of designs derived from the co-design study.}
\label{tab:designs}
\begin{tabular}{p{3.5cm} p{12cm} p{1.5cm}}
\toprule
\textbf{Overall Design Idea} & \textbf{Associated Barrier to Trust-Enabled Privacy} & \textbf{Example Prototype} \\ 
\midrule

\textbf{Guided Disclosure}\newline{}[Section \ref{section:4-2-1}] & Helps clarify \textbf{\textit{Ambiguous Norms}} by providing clear guidelines and expectations around posting, reducing uncertainty and the need for users to interpret implicit social expectations independently & Figure \ref{fig:prototypes}(a) \\

\textbf{Mutual Commitment}\newline{}[Section \ref{section:4-2-2}] & Addresses \textbf{\textit{Ambiguous Loyalty}} by allowing viewers to opt in or out of content while requiring them to agree to the space's rules before joining, ensuring mutual commitment and accountability for both the sharer and the audience. & Figure \ref{fig:prototypes}(g) \\

\textbf{Contextual Disclosure}\newline{}[Section \ref{section:4-2-3}] & Counteracts \textbf{\textit{Ambiguous Relevance}} by enabling segmentation of posts for specific interest groups, ensuring content reaches the right audience & Figure \ref{fig:prototypes}(c) \\

\textbf{Intentional Signaling}\newline{}[Section \ref{section:4-2-4}] & Clarifies \textbf{\textit{Ambiguous Reactions}} by encouraging more meaningful, context-specific reactions beyond simple `Likes' & Figure \ref{fig:prototypes}(f) \\

\textbf{Low-Stakes Disclosure}\newline{}[Section \ref{section:4-3-1}] & Alleviates \textbf{\textit{Presentation Expectations}} by offering casual, subtle, or ephemeral sharing options that lower the pressure to curate perfect content & Figure \ref{fig:prototypes}(b) \\

\textbf{Contextual Clarity}\newline{}[Section \ref{section:4-3-2}] & Reduces the \textbf{\textit{Social Risk of Misrepresentation}} common in CMC by allowing users to add contextual information, reducing likelihood of misunderstandings and misjudgments & Figure \ref{fig:prototypes}(e) \\

\textbf{Self-Contained Disclosure}\newline{}[Section \ref{section:4-3-3}] & Prevents \textbf{\textit{Nonconsensual Exposure}} by ensuring that every user has a dedicated space from the start, enabling them to share without worrying about burdening others by ``clogging'' their feeds & Figure \ref{fig:prototypes}(d) \\

\textbf{Trust-Centered Norms}\newline{}[Section \ref{section:4-3-4}] & Tackles the \textbf{\textit{Cycle of Distrust}} and cultivates positive community norms by setting clear behavioral guidelines, enabling users with aligned intentions to join and mutually enforce these norms & Figure \ref{fig:prototypes}(h) \\

\bottomrule
\end{tabular}
\end{table*}


\subsection{Communication Fog: Barrier to Boundary Regulation and Trust Calibration}
\label{section:4-2}
One key aspect that complicated trust-based boundary regulation was the uncertainty about how their posts are received on social media. Participants were unsure of platform norms, leaving them burdened with individual responsibility to navigate about what or how much to share. They also question whether their audience might judge them or not be interested in what they post. Additionally, interpreting reactions---especially low-effort, obligatory responses such as ``Likes''---is confusing, making it difficult to gauge genuine engagement or appreciation. These ambiguities create a ``communication fog'' that complicates trust calibration and boundary regulation, making users hesitant to share in ways that support relationship-building.


\subsubsection{Ambiguous Norms: Unclear Sharing Expectations and Individual Burden}
\label{section:4-2-1}
Participants highlighted the significant influence of peer norms on their self-disclosure behaviors, emphasizing their tendency to align their sharing with the perceived expectations within their social circles. Teens carefully calibrate their disclosures to neither exceed nor fall below the perceived norm, as \entry{02} noted: \inlinequote{Oftentimes, I kind of match what my friends post in terms of how public they are.} This balancing act reflects an ongoing negotiation of social expectations, with \codesign{07} stating: \inlinequote{So you don't want to overshare and you don't want to undershare.} 

However, many platforms lack clearly defined norms for what constitutes an appropriate level of sharing, leaving users to navigate ambiguous expectations independently. Participants interpret implicit norms:
\blockquote{[On Instagram,] Not famous people, in my experience, do not normally post random stuff like this, so it would be weird to make it my first post.}{\diary{16}}
Therefore, Instagram necessitates users to \inlinequote{make an active choice} (\codesign{05}) about when and what to share. Hence, self-initiated posting can sometimes be interpreted as \inlinequote{attention-seeking} (\diary{16}), and teens fear (potential) backlash. For instance, \codesign{12} had been asked \inlinequote{why do you post so much on Instagram?}, and \codesign{11} described having been ``unfollowed'' by a peer who deemed their content uninteresting: \inlinequote{He was like, oh you don't post anything interesting. All you do is post selfies.}

Participants expressed a desire for \textbf{guided disclosure} (Figure \ref{fig:prototypes}(f)), where platforms would offer clear, explicit cues that establish expectations for how much, how often, and when to share. By creating shared reference points for appropriate disclosure, these prompts would reduce the burden of individual decision-making and reinforce the reciprocity of self-disclosure, reassuring users that their vulnerability is not being interpreted as an isolated attempt for attention but rather as a socially expected practice. 

More specifically, participants envisioned being able to make \inlinequote{a quick update on [their] life, even if it's quite boring} (\diary{05}). Platforms like BeReal, which enforce a collective norm of casual sharing at a predefined time, were appreciated for \inlinequote{give[ing] everyone a chance to shout out} (\codesign{05}). Almost all participants advocated for system-generated \textit{prompts} or reminders that would encourage low-stakes sharing and provide implicit permission to disclose, fostering a \inlinequote{sense of community} (\codesign{12}, \codesign{16}) while reducing concerns about \inlinequote{sharing too much} (\codesign{19}). Such interventions would also alleviate uncertainty when users feel like they \inlinequote{can't really figure out what to post} (\codesign{07}) by \inlinequote{giv[ing] ideas about what to share} (\diary{08}), making self-disclosure a more collectively guided and less individually scrutinized process.



\subsubsection{Ambiguous Loyalty: When Followers Feel Like Skeptics}
\label{section:4-2-2}
Participants expressed that social media ``friends'' or ``followers'' often lack the established trust of real-world friendships. While these connections did not warrant removal---sometimes due to fears of social repercussions or backlash---their intentions and judgments remained uncertain. This ambiguity created a challenge in boundary regulation as users struggled to assess how their audience might interpret their posts. As \diary{08} explains: \inlinequote{I don't want some of my friends to think I'm oversharing or being `cringey.' There are always risks. You never can really know who's viewing your content or who's REALLY following you.}

To mitigate these uncertainties and foster a greater sense of trust-based boundary control, participants proposed supporting \textbf{mutual commitment} (Figure \ref{fig:prototypes}(g)) in content consumption. One suggested mechanism involved posters sending invitations, requiring viewers to actively opt-in (\codesign{16}) or out (\codesign{03}): \inlinequote{A pop-up of like `Oh you've been invited to join'\ldots with a yes or a no button,} and offering \inlinequote{the option to either join it or leave if you don't want to follow those rules} (\codesign{11}). This approach redistributes accountability---allowing viewers to regulate their access while relieving sharers of sole responsibility for audience reactions. A participant explained how mutually agreed sharing, such as that on Snapchat private Stories, provides reassurance: 
\blockquote{If they joined that story, [then] they're kind of the ones subjecting themselves to it. If they didn't want to see it, they'd never had to join it.}{\entry{12}}



\subsubsection{Ambiguous Relevance: Struggles to Identify the Right Audience}
\label{section:4-2-3}
Participants expressed a strong desire to curate their audience, even when viewers were not overtly toxic or judgmental. This need stems from their multifaceted identities and diverse interests, which they prefer to compartmentalize for different social groups. Participants shared various interests that they felt would only engage a subset of their audience, from poetry (\entry{11}) to K-Pop (\entry{03}). One participant explained the challenge of this:
\blockquote{sometimes I don't want to bore some of my friends with [a specific interest of theirs]\ldots{} I don't really feel the need to share it with them because as much as I would like them to share my interest in it, I don't want them to find that kind of burdensome.}{\codesign{14}} 

While many platforms employ algorithms to curate content based on inferred interests, participants found these coarse and ineffective. Rather, participants advocated for self-managed and dynamic calibration of social boundaries through \textbf{contextual disclosure} (Figure \ref{fig:prototypes}(b)), allowing them to strategically engage with groups based on the established and/or perceived potential for trust. Some users created elaborate systems of nested groups via Snapchat's private stories feature: 
\blockquote{I have four [private stories], each getting smaller to accommodate things I choose to share with\ldots{} The closer you are to me, the more (and smaller) private stories you're in.}{\diary{15}} 
\entry{16} reflected on the flexibility of Discord as providing similar benefits. 
These user-curated spaces would not just serve as secure places for existing trust but as controlled environments where users could gradually calibrate their social boundaries, engaging in disclosure that fosters emerging trust while minimizing the risks of sharing too broadly.

Participants were particularly keen on interest-based spaces, where shared interests would serve as a bridge to potential trust development. They noted that while common interests alone do not equate to trust, they provide an initial context that reduces the uncertainties of broader public sharing and, ultimately, allows trust to form over time. As \codesign{08} explained, such intentional segmentation \inlinequote{could definitely help people connect more because they'd feel like they had more to talk about.}


\subsubsection{Ambiguous Reactions: When ``Likes'' Fail to Signal Trust}
\label{section:4-2-4}
Participants expressed confusion about interpreting the meaning of reactions they receive to their posts, particularly when those reactions felt \inlinequote{obligatory} (\entry{19}). \entry{19} explained, \inlinequote{you feel more obligated to like a post\ldots [so] it just feels worse to post and not get as many likes.} Similarly, \codesign{08} remarked that when \inlinequote{people you're close with\ldots don't interact with your post,} they begin to question, \inlinequote{Is it really that boring that even people I know don't care?} While ``Likes'' may have been designed to signal validation, their meaning has become obscured in many platforms, making it an ineffective signal for reciprocity or trust.

To reduce ambiguity in social interactions, participants expressed interest in having a broader range of mechanisms for \textbf{intentional signaling} (Figure \ref{fig:prototypes}(d)) beyond effortless reactions such as ``Likes''. They cited examples from existing platforms where engagement from other users served as a clear, indisputable signal of care. For instance, the range and expressive nature of emoji reactions requires more deliberation than \inlinequote{just double tap[ing]}(\codesign{15}). When the chosen emoji aligns with the content, it sends clear signals that the actor actually viewed it. Comments were also valued as another form of clear signals of trust as they were seen as requiring individuals \inlinequote{go out of their way to be nice} (\codesign{08}). Participants also appreciated Instagram's Story Likes, describing them as \inlinequote{not obligatory at all} and as a feature that \inlinequote{shows that\ldots they decided to go out of their way to like it} (\entry{19}). To \codesign{13}, receiving Story Likes brings \inlinequote{a lot of trust[s]} given the private---and therefore more intimate, deliberate, and less likely performative---nature of those reactions. These reflections highlight that more nuanced, intentional reactions could serve as better social signals that reduce the ambiguity in boundary regulation.


\subsection{Low-Grace Culture: Barrier to Sustained Trust and Adaptive Boundary Setting}
\label{section:4-3}
In this section, we explore how social media cultivates a ``low-grace culture'' that discourages vulnerability or casual self-disclosure. This environment is characterized by an emphasis on polished self-presentation, judgments based on limited information, the risk of inadvertently burdening others by sharing, and an overall climate of distrust. These pressures inhibit the flexibility needed for adaptive boundary regulation, discouraging trust-building interactions.

\subsubsection{High-Stakes Presentation Expectations and Self-Censorship}
\label{section:4-3-1}
Participants frequently expressed fear about not measuring up to the presentation expectations of platforms such as Instagram. Especially on image-centric platforms, they felt compelled to share only highly polished moments. As \entry{18} stated, \inlinequote{Instagram is only for when the sun sets during the golden hour when I look my best.} This expectation discourages casual posting, as teens perceive that they \inlinequote{need to put significant effort into a post} (\entry{06}). Some worry that deviating from these norms might make them be judged as \inlinequote{underwhelming} or \inlinequote{monotonous} (\diary{11}), leaving them feeling vulnerable (\diary{13}).

To counteract these pressures, many participants expressed a preference for \textbf{low-stakes disclosure} (Figure \ref{fig:prototypes}(a)), which reduces the burden of excessive curation and supports more casual self-expression. They valued features that support ephemeral sharing, such as Instagram's Story or Snapchat posts, which feel \inlinequote{not as big of a deal} (\diary{16}) without \inlinequote{people saving or revisiting them} (\entry{06}). Participants often leveraged these features to share what they felt \inlinequote{doesn't really live up to the standards of what I post on Instagram} (\diary{07}). They also appreciated the support for casual, text-based updates on Bluesky and Twitter, or platforms that made interactions feel more like \inlinequote{having a conversation with someone} (\codesign{11}) rather than putting together a \inlinequote{portfolio} (\codesign{11}). Some participants suggested features like \inlinequote{a little status update} where people can \inlinequote{learn little facts about other people casually} (\codesign{05}).

These features help alleviate concerns about inadvertently exposing too much or misjudging boundaries by supporting and normalizing casual sharing rather than polished, high-effort posts. If casual sharing is the norm, then posting less curated content is less likely to be perceived as a disclosure of something deeply personal or intimate. Instead, it becomes part of a broader culture of mutual disclosure and relationship building, reducing the likelihood of boundary missteps and supporting better trust calibration among peers.



\subsubsection{Sparse Sharing and the Heightened Risk of Misrepresentation}
\label{section:4-3-2}
On platforms like Instagram, where participants perceived infrequent posting as the norm, limited content often becomes the primary basis for one's digital identity. This leaves users vulnerable to misrepresentation from acquaintances who may \inlinequote{make assumptions or misjudge [their] intentions} (\codesign{03}). With such sparse sharing, even a small number of posts can significantly influence how someone is perceived: \inlinequote{Somebody might pull up my Instagram and there are only, like, two posts and I look bad in both of them} (\codesign{12}). They also shared specific examples, such as posting extensively about a single topic, potentially coming across as being \inlinequote{obsessed} (\diary{12}) with that subject; sharing multiple photos from a memorable event might be interpreted as \inlinequote{bragging} (\entry{12}); or inquiring whether other students have completed their summer assignment could be misconstrued as attempting to \inlinequote{push} (\entry{12}) peers to do their work, despite that not being the intention of the post. Such concerns often led participants to lean toward self-censorship.

To mitigate the risk of trust erosion and promote more adaptive boundary regulation, participants suggested features to facilitate \textbf{contextual clarity} (Figure \ref{fig:prototypes} (d))---designs that provide additional context to reduce misunderstandings and support trust-building on platforms like Instagram. These included: indicating personal information such as interests on profiles (\diary{12}), emphasizing the importance of comprehensive information before making judgments (\codesign{03}), supporting explanatory captions for images beyond standard captions (\codesign{07}), and implementing a \inlinequote{social battery} indicator to help manage interaction expectations (\codesign{11}). By fostering mechanisms that encourage giving others the benefit of the doubt, such designs could prevent trust erosion and provide contexts for boundary regulation in online environments where full contextual understanding is not always possible.



\subsubsection{Nonconsensual Exposure and the Fear of Burdening Others}
\label{section:4-3-3}
On broadcast social media platforms, following a user implicitly consents to viewing their content, often leading to viewers feeling overwhelmed and \inlinequote{buried} (\codesign{12}) in posts. As \codesign{12} noted, \inlinequote{If I'm following 100 people and they're all sharing four times a day, then that's 400 things I have to click through.} Conversely, those that are sharing---aware of this dynamic from their own experiences---often fear \inlinequote{spamming} (\entry{03}, \codesign{13}) or \inlinequote{clogging} (\diary{16}) others' feeds. This fear of being perceived as burdensome or irritating was also evident in diary entries, where participants expressed hesitation \inlinequote{that people may think I'm oversharing or like posting too much} (\diary{11}), or regret after realizing, \inlinequote{[I] shared a TON of reels today on my close friends and public for some reason :crying-face:} (\diary{09}).

Participants sought ways to mitigate such unnecessary friction. Many desired \textbf{self-contained disclosure} (Figure \ref{fig:prototypes}(e)) that would allow them to share without imposing their content on others. These unobtrusive communication mechanisms included \inlinequote{a little status update} (\codesign{05}) or sharing minor personal interests on profile pages rather than in followers' feeds. One participant proposed a \inlinequote{red dot} (\codesign{17}) on user profiles to indicate new content subtly, allowing viewers to \inlinequote{not have to go look at it if they don't want to} (\codesign{17}). \codesign{05} saw parallels to Instagram's Story feature where \inlinequote{it doesn't notify people\ldots{} it's not like it goes on their feed\ldots{} it's just casual.} Participants perceived this as a way to foster connection without unnecessary friction and trust erosion by ensuring disclosure remains within a self-regulated and lower-stakes space.



\subsubsection{Absent Norms for Trust-Building and the Cycle of Distrust}
\label{section:4-3-4}
Participants observed social media spaces as often having a pervasive climate of distrust and the lack of measures to regulate such an environment. As \codesign{17} remarked, \inlinequote{On mainstream social media\ldots there's occasional positivity\ldots but mostly it feels draining}. 

They suggested that platforms could shift this culture by explicitly stating their platform expectations toward \textbf{trust-centered norms} (Figure \ref{fig:prototypes} (h)), such as explicitly stating that the platform is \inlinequote{a judgment-free zone} and communicating a clear message to users to \inlinequote{decrease judgment and be more kind} (\codesign{19}). They believed that such an approach would encourage users to \inlinequote{naturally be inclined to conform with the overall vibe and mission of the other users on the app} (\codesign{19}). Once a culture is established, a \inlinequote{selection bias} would occur, attracting like-minded individuals who align with these expectations (\codesign{19}). One participant expanded on this idea: 
\blockquote{I'm tired of how curated social media can be, but I still engage with it\ldots{} When the culture prioritizes authenticity and transparency, you'll get people like me, who are tired of curating, putting in the effort to be authentic and create this culture.}{\codesign{15}}
Similarly, another participant observed:
\blockquote{When a platform declares itself a judgment-free zone, it gives people the power to ensure both their own safety and the safety of others, creating a sense of control.}{\codesign{13}}
Participants emphasized that they believe that setting clear expectations, rules, and guidelines would foster meaningful changes toward safer environments. \codesign{17} noted that \inlinequote{teenagers my age listen when things are specifically told to them,} underscoring the importance of explicit communication in shaping platform norms.

Participants acknowledged that explicit guidelines could help mitigate negative interactions, but several also proposed additional features to actively \inlinequote{regulate toxic behaviors} as a necessary complement. They recognized that \inlinequote{there's always going to be someone that might try to ruin it} (\codesign{18}). To address this, \codesign{07} suggested a \inlinequote{negative comment filter} that users could toggle to block harmful content. Additionally, \codesign{19} proposed a reporting feature that would allow users to flag individuals \inlinequote{who may spread hurtful things or judge people.} While participants were mindful of the risk of implying the platform's distrust in its users, they acknowledged that tools for enforcing platform expectations could be valuable in strengthening their efforts to foster trust-enabling environments.

\section{Discussion and Conclusion}

% \begin{quote}
% \textit{"We believe it is unethical for social workers not to learn... about technology-mediated social work."} (\citeauthor{singer_ai_2023}, 2023)
% \end{quote}

In this study, we uncovered multiple ways in which GenAI can be used in social service practice. While some concerns did arise, practitioners by and large seemed optimistic about the possibilities of such tools, and that these issues could be overcome. We note that while most participants found the tool useful, it was far from perfect in its outputs. This is not surprising, since it was powered by a generic LLM rather than one fine-tuned for social service case management. However, despite these inadequacies, our participants still found many uses for most of the tool's outputs. Many flaws pointed out by our participants related to highly contextualised, local knowledge. To tune an AI system for this would require large amounts of case files as training data; given the privacy concerns associated with using client data, this seems unlikely to happen in the near future. What our study shows, however, is that GenAI systems need not aim to be perfect to be useful to social service practitioners, and can instead serve as a complement to the critical "human touch" in social service.

We draw both inspiration and comparisons with prior work on AI in other settings. Studies on creative writing tools showed how the "uncertainty" \cite{wan2024felt} and "randomness" \cite{clark2018creative} of AI outputs aid creativity. Given the promise that our tool shows in aiding brainstorming and discussion, future social service studies could consider AI tools explicitly geared towards creativity - for instance, providing side-by-side displays of how a given case would fit into different theoretical frameworks, prompting users to compare, contrast, and adopt the best of each framework; or allowing users to play around with combining different intervention modalities to generate eclectic (i.e. multi-modal) interventions.

At the same time, the concept of supervision creates a different interaction paradigm to other uses of AI in brainstorming. Past work (e.g. \cite{shaer2024ai}) has explored the use of GenAI for ideation during brainstorming sessions, wherein all users present discuss the ideas generated by the system. With supervision in social service practice, however, there is a marked information and role asymmetry: supervisors may not have had the time to fully read up on their supervisee's case beforehand, yet have to provide guidance and help to the latter. We suggest that GenAI can serve a dual purpose of bringing supervisors up to speed quickly by summarising their supervisee's case data, while simultaneously generating a list of discussion and talking points that can improve the quality of supervision. Generalising, this interaction paradigm has promise in many other areas: senior doctors reviewing medical procedures with newer ones \cite{snowdon2017does} could use GenAI to generate questions about critical parts of a procedure to ask the latter, confirming they have been correctly understood or executed; game studio directors could quickly summarise key developmental pipeline concerns to raise at meetings and ensure the team is on track; even in academia, advisors involved in rather too many projects to keep track of could quickly summarise each graduate student's projects and identify potential concerns to address at their next meeting.

In closing, we are optimistic about the potential for GenAI to significantly enhance social service practice and the quality of care to clients. Future studies could focus on 1) longitudinal investigations into the long-term impact of GenAI on practitioner skills, client outcomes, and organisational workflows, and 2) optimising workflows to best integrate GenAI into casework and supervision, understanding where best to harness the speed and creativity of such systems in harmony with the experience and skills of practitioners at all levels.

% GenAI here thus serves as a tool that supervisors can use before rather then using the session, taking just a few minutes of their time to generate a list of discussion points with their supervisees.

% Traditional brainstorming comes up with new things that users discuss. In supervision, supervisors can use AI to more efficiently generate talking points with their supervisees. These are generally not novel ideas, since an experienced worker would be able to come up with these on their own. However, the interesting and novel use of AI here is in its use as a preparation tool, efficiently generating talking and discussion points, saving supervisors' time in preparing for a session, while still serving as a brainstorming tool during the session itself.

% The idea of embracing imperfect AI echoes the findings of \citeauthor{bossen2023batman} (2023) in a clinical decision setting, which examined the successful implementation of an "error-prone but useful AI tool". This study frames human-AI collaboration as "Batman and Robin", where AI is a useful but ultimately less skilled sidekick that plays second fiddle to Batman. This is similar to \citeauthor{yang2019unremarkable}'s (2019) idea of "unremarkable AI", systems designed to be unobtrusive and only visible to the user when they add some value. As compared to \citeauthor{bossen2023batman}, however, we see fewer instances of our AI system producing errors, and more examples of it providing learning and collaborative opportunities and other new use cases. We build on the idea of "complementary performance" \cite{bansal2021does}, which discusses how the unique expertise of AI enhances human decision-making performance beyond what humans can achieve alone. Beyond decision-making, GenAI can now enable "complementary work patterns", where the nature of its outputs enables humans to carry out their work in entirely new ways. Our study suggests that rather being a sidekick - Robin - AI is growing into the role of a "second Batman" or "AI-Batman": an entity with distinct abilities and expertise from humans, and that contributes in its own unique way. There is certainly still a time and place for unremarkable AI, but exploring uses beyond that paradigm uncovers entirely new areas of system design.

% % \cite{gero2022sparks} found AI to be useful for science writers to translate ideas already in their head into words, and to provide new perspectives to spark further inspiration. \textit{But how is ours different from theirs?}

% \subsection{New Avenues of Human-AI Collaboration}
% \label{subsubsec:discussionhaicollaboration}

% Past HCI literature in other areas \cite{nah2023generative} has suggested that GenAI represents a "leap" \cite{singh2023hide} in human-AI collaboration, 
% % Even when an AI system sometimes produces irrelevant outputs, it can still provide users 
% % Such systems have been proposed as ways to 
% helping users discover new viewpoints \cite{singh2023hide}, scour existing literature to suggest new hypotheses 
% \cite{cascella2023evaluating} and answer questions \cite{biswas2023role}, stimulate their cognitive processes \cite{memmert2023towards}, and overcome "writer's block" \cite{singh2023hide, cooper2023examining} (particularly relevant to SSPs and the vast amount of writing required of them). Our study finds promise for AI to help SSPs in all of these areas. By nature of being more verbose and capable of generating large amounts of content, GenAI seems to create a new way in which AI can complement human work and expertise. Our system, as LLMs tend to do, produced a lot of "bullshit" (S6) \cite{frankfurt2005bullshit} - superficially true statements that were often only "tangentially related" and "devoid of meaning" \cite{halloran2023ai}. Yet, many participants cited the page-long analyses and detailed multi-step intervention plans generated by the AI system to be a good starting point for further discussion, both to better conceptualize a particular case and to facilitate general worker growth and development. Almost like throwing mud at a wall to see what sticks, GenAI can quickly produce a long list of ideas or information, before the worker glances through it and quickly identifies the more interesting points to discuss. Playing the proposed role as a "scaffold" for further work \cite{cooper2023examining}, GenAI, literally, generates new opportunities for novel and more effective processes and perspectives that previous systems (e.g., PRMs) could not. This represents an entirely new mode of human-AI collaboration.
% % This represents a new mode of collaboration not possible with the largely quantitative AI models (like PRMs) of the past.

% Our work therefore supports and extends prior research that have postulated the the potential of AI's shifting roles from decision-maker to human-supporter \cite{wang_human-human_2020}. \citeauthor{siemon2022elaborating} (2022) suggests the role of AI as a "creator" or "coordinator", rather than merely providing "process guidance" \cite{memmert2023towards} that does not contribute to brainstorming. Similarly, \citeauthor{memmert2023towards} (2023) propose GenAI as a step forward from providing meta-level process guidance (i.e. facilitating user tasks) to actively contributing content and aiding brainstorming. We suggest that beyond content-support, AI can even create new work processes that were not possible without GenAI. In this sense, AI has come full circle, becoming a "meta-facilitator".

% % --- WIP BELOW ---

% % Our work echoes and extends previous research on HAI collaboration in tasks requiring a human touch. \cite{gero2023social} found AI to be a safe space for creative writers to bounce ideas off of and document their inner thoughts. \cite{dhillon2024shaping} reference the idea of appropriate scaffolding in argumentative writing, where the user is providing with guidance appropriate for their competency level, and also warns of decreased satisfaction and ownership from AI use. 

% Separately, we draw parallels with the field of creative writing, where HAI collaboration has been extensively researched. Writers note the "irreducibly human" aspect of creativity in writing \cite{gero2023social}, similar to the "human touch" core to social service practice (D1); both groups therefore expressed few concerns about AI taking over core aspects of their jobs. Another interesting parallel was how writers often appreciated the "uncertainty" \cite{wan2024felt} and "randomness" \cite{clark2018creative} of AI systems, which served as a source of inspiration. This echoes the idea of "imperfect AI" "expanding [the] perspective[s]" (S4) of our participants when they simply skimmed through what the AI produced. \cite{wan2024felt} cited how the "duality of uncertainty in the creativity process advances the exploration of the imperfection of GenAI models". While social service work is not typically regarded as "creative", practitioners nonetheless go through processes of ideation and iteration while formulating a case. Our study showed hints of how AI can help with various forms of ideation, but, drawing inspiration from creative writing tools, future studies could consider designs more explicitly geared towards creativity - for instance, by attempting to fit a given case into a number of different theories or modalities, and displaying them together for the user to consider. While many of these assessments may be imperfect or even downnright nonsensical, they may contain valuable ideas and new angles on viewing the case that the practitioner can integrate into their own assessment.

% % \cite{foong2024designing}, describing the design of caregiver-facing values elicitation tools, cites the "twin scenarios" that caregivers face - private use, where they might use a tool to discover their patient's values, and collaborative use, where they discuss the resulting values with other parties close to the patient. This closely mirrors how SSPs in our study reference both individual and collaborative uses of our tool. Unlike in \cite{foong2024designing}, however, we do not see a resulting need to design a "staged approach" with distinct interface features for both stages.

% % --- END OF WIP ---

% Having mentioned algorithm aversion previously, we also make a quick point here on the other end of the spectrum - automation bias, or blind trust in an automated system \cite{brown2019toward}. LLMs risk being perceived as an "ultimate epistemic authority" \cite{cooper2023examining} due to their detailed, life-like outputs. While automation bias has been studied in many contexts, including in the social sector or adjacent areas, we suggest that the very nature of GenAI systems fundamentally inhibits automation bias. The tendency of GenAI to produce verbose, lengthy explanations prompts users to read and think through the machine's judgement before accepting it, bringing up opportunities to disagree with the machine's opinion. This guards against blind acceptance of the system's recommendations, particularly in the culture of a social work agency where constant dialogue - including discussing AI-produced work - is the norm.


% % : Perception of AI in Social Service Work ??

% \subsection{Redefining the Boundary}
% \label{subsubsec:discussiontheoretical}

% As \citeauthor{meilvang_working_2023} (2023) describes, the social service profession has sought to distance itself from comprising mostly "administrative work" \cite{abbott2016boundaries}, and workers have long tried to tried to reduce their considerable time \cite{socialraadgiverforening2010notat} spent on such tasks in favour of actual casework with clients \cite{toren1972social}. Our study, however, suggests a blurring of the line between "manual" administrative tasks and "mental" casework that draws on practitioner expertise. Many tasks our participants cited involve elements of both: for instance, documenting a case recording requires selecting only the relevant information to include, and planning an intervention can be an iterative process of drafting a plan and discussing it with colleagues and superiors. This all stems from the fact that GenAI can produce virtually any document required by the user, but this document almost always requires revision under a watchful human eye.

% \citeauthor{meilvang_working_2023} (2023) also describes a more recent shift in the perceived accepted boundary of AI interventions in social service work. From "defending [the] jurisdiction [of social service work] against artificial intelligence" in the early days of PRM and other statistical assessment tools, the community has started to embrace AI as a "decision-support ... element in the assessment process". Our study concurs and frames GenAI as a source of information that can be used to support and qualify the assessments of SSPs \cite{meilvang_working_2023}, but suggests that we can take a step further: AI can be viewed as a \textit{facilitator} rather than just a supporter. GenAI can facilitate a wide range of discussions that promote efficiency, encourage worker learning and growth, and ultimately enhance client outcomes. This entails a much larger scope of AI use, where practitioners use the information provided by AI in a range of new scenarios. 

% Taken together, these suggest a new focus for boundary work and, more broadly, HCI research. GAI can play a role not just in menial documentation or decision-support, but can be deeply ingrained into every facet of the social service workflow to open new opportunities for worker growth, workflow optimisation, and ultimately improved client outcomes. Future research can therefore investigate the deeper, organisational-level effects of these new uses of AI, and their resulting impact on the role of profession discretion in effective social service work.

% % MH: oh i feel this paragraph is quite new to me! Could we elaborate this more, and truncate the first two paragraphs a bit to adjust the word propotion?


% % Our study extensively documents this for the first time in social service practice, and in the process reveals new insights about how AI can play such a role.



% \subsection{Design Implications}

% % Add link from ACE diagram?

% % EJ: it would be interesting to discuss how LLMs could help "hands-on experience" in the discussion section

% Addressing the struggle of integrating AI amidst the tension between machine assessment and expert judgement, we reframe AI as an \textit{facilitator} rather than an algorithm or decision-support tool, alleviating many concerns about trust and explainablity. We now present a high-level framework (Figure \ref{fig:hai-collaboration}) on human-AI collaboration, presenting a new perspective on designing effective AI systems that can be applied to both the social service sector and beyond.

% \begin{figure}
%     \centering
%     \includegraphics[scale=0.15]{images/designframework.png}
%     \caption{Framework for Human-AI Collaboration}
%     \label{fig:hai-collaboration}
%     \Description{An image showing our framework for Human-AI Collaboration. It shows that as stakeholder level increases from junior to senior, the directness of use shifts from co-creation to provision.}
% \end{figure}
% % MH: so this paradigm is proposed by us? I wonder if this could a part of results as well..?

% % \subsubsection{From Creation to Provision}

% In Section \ref{sec:stage2findings}, we uncovered the different ways in which SSPs of varying seniorities use, evaluate, and suggest uses of AI. These are intrinsically tied to the perspectives and levels of expertise that each stakeholder possesses. We therefore position the role of AI along the scale of \textit{creation} to \textit{provision}. 

% With junior workers, we recommend \textbf{designing tools for co-creation}: systems that aid the least experienced workers in creating the required deliverables for their work. Rather than \textit{telling} workers what to do - a difficult task in any case given the complexity of social work solutions - AI systems should instead \textit{co-create} deliverables required of these workers. These encompass the multitude of use cases that junior workers found useful: creating reports, suggesting perspectives from which to formulate a case, and providing a starting template for possible intervention plans. Notably, since AI outputs are not perfect, we emphasise the "co" in "co-creation": AI should only be a part of the workflow that also includes active engagement on the part of the SSPs and proactive discussion with supervisors. 

% For more experienced SSPs, we recommend \textbf{designing tools for provision}. Again, this is not the mere provision of recommendations or courses of action with clients, but rather that of resources which complement the needs of workers with greater responsibilities. This notably includes supplying materials to aid with supervision, a novel use case that to our knowledge has not surfaced in previous literature. In addition, senior workers also benefit greatly from manual tasks such as routine report writing and data processing. Since these workers are more experienced and can better spot inaccuracies in AI output, we suggest that AI can "provide" a more finished product that requires less vetting and corrections, and which can be used more directly as part of required deliverables.

% % MH: can we seperate here? above is about the guidance to paradigm, below is the practical roadmap for implementation
% In terms of concrete design features, given the constant focus on discussing AI outputs between colleagues in our FGDs, we recommend that AI tools, particularly those for junior workers, \textbf{include collaborative features} that facilitate feedback and idea sharing between users. We also suggest that designers work closely with domain experts (i.e. social work practitioners and agencies) to identify areas where the given AI model tends to make more mistakes, and to build in features that \textbf{highlight potential mistakes or inadequacies} in the AI's output to facilitate further discussion and avoid workers adopting suboptimal suggestions. 

% We also point out a fundamental difference between GenAI systems and previous systems: that GenAI can now play an important role in aiding users \textit{regardless of its flaws}. The nature of GenAI means that it promotes discussion and opens up new workflows by nature of its verbose and potentially incomplete outputs. Rather than working towards more accurate or explainable outcomes, which may in any case have minimal improvement on worker outcomes \cite{li2024advanced}, designers can also focus on \textbf{understanding how GenAI outputs can augment existing user flows and create new ones}.

% % for more senior workers...

% % how to differentiate levels of workers?

% % \subsubsection{Provider}

% % The most basic and obvious role of modern AI that we identify leverages the main strength of LLMs. They have the ability to produce high-quality writing from short, point-form, or otherwise messy and disjoint case notes that user often have \textit{[cite participant here]}. 

% Finally, given the limited expertise of many workers at using AI, it is important that systems \textbf{explicitly guide users to the features they need}, rather than simply relying on the ability of GAI to understand complex user instructions. For example, in the case of flexibility in use cases (Section \ref{subsubsec:control}), systems should include user flows that help combine multiple intervention and assessment modalities in order to directly meet the needs of workers.

% \subsection{Limitations and Future Work}

% While we attempt to mimic a contextual inquiry and work environment in our study design, there is no substitute for real data from actual system deployment. The use of an AI system in day-to-day work could reveal a different set of insights. Future studies could in particular study how the longitudinal context of how user attitudes, behaviours, preferences, and work outputs change with extended use of AI. 

% While we tried to include practitioners from different agencies, roles, and seniorities, social service practice may differ culturally or procedurally in other agencies or countries. Future studies could investigate different kinds of social service agencies and in different cultures to see if AI is similarly useful there.

% As the study was conducted in a country with relatively high technology literacy, participants naturally had a higher baseline understanding and acceptance of AI and other computer systems. However, we emphasise that our findings are not contingent on this - rather, we suggest that our proposed lens of viewing AI in the social sector is a means for engaging in relevant stakeholders and ensuring the effective design and implementation of AI in the social sector, regardless of how participants feel about AI to begin with. 



% % \subsection{Notes}

% % 1) safety and risks and 2) privacy - what does the emphasis on this say about a) design recommendations and b) approach to designing/PD of such systems?


% % W9 was presented with "Strengths" and "SFBT" output options. They commented, "solution focus is always building on the person's strengths". W9 therefore requested being able to output strengths and SFBT at the same time. But this would suggest that the SFBT output does not currently emphasise strengths strongly enough. However, W9 did not specifically evaluate that, and only made this comment because they saw the "strengths" option available, and in their head, strengths are key to SFBT.
% % What does this say about system design and UI in relation to user mental models?
\paragraph{Summary}
Our findings provide significant insights into the influence of correctness, explanations, and refinement on evaluation accuracy and user trust in AI-based planners. 
In particular, the findings are three-fold: 
(1) The \textbf{correctness} of the generated plans is the most significant factor that impacts the evaluation accuracy and user trust in the planners. As the PDDL solver is more capable of generating correct plans, it achieves the highest evaluation accuracy and trust. 
(2) The \textbf{explanation} component of the LLM planner improves evaluation accuracy, as LLM+Expl achieves higher accuracy than LLM alone. Despite this improvement, LLM+Expl minimally impacts user trust. However, alternative explanation methods may influence user trust differently from the manually generated explanations used in our approach.
% On the other hand, explanations may help refine the trust of the planner to a more appropriate level by indicating planner shortcomings.
(3) The \textbf{refinement} procedure in the LLM planner does not lead to a significant improvement in evaluation accuracy; however, it exhibits a positive influence on user trust that may indicate an overtrust in some situations.
% This finding is aligned with prior works showing that iterative refinements based on user feedback would increase user trust~\cite{kunkel2019let, sebo2019don}.
Finally, the propensity-to-trust analysis identifies correctness as the primary determinant of user trust, whereas explanations provided limited improvement in scenarios where the planner's accuracy is diminished.

% In conclusion, our results indicate that the planner's correctness is the dominant factor for both evaluation accuracy and user trust. Therefore, selecting high-quality training data and optimizing the training procedure of AI-based planners to improve planning correctness is the top priority. Once the AI planner achieves a similar correctness level to traditional graph-search planners, strengthening its capability to explain and refine plans will further improve user trust compared to traditional planners.

\paragraph{Future Research} Future steps in this research include expanding user studies with larger sample sizes to improve generalizability and including additional planning problems per session for a more comprehensive evaluation. Next, we will explore alternative methods for generating plan explanations beyond manual creation to identify approaches that more effectively enhance user trust. 
Additionally, we will examine user trust by employing multiple LLM-based planners with varying levels of planning accuracy to better understand the interplay between planning correctness and user trust. 
Furthermore, we aim to enable real-time user-planner interaction, allowing users to provide feedback and refine plans collaboratively, thereby fostering a more dynamic and user-centric planning process.

\section{Secure Token Pruning Protocols}
\label{app:a}
We detail the encrypted token pruning protocols $\Pi_{prune}$ in Figure \ref{fig:protocol-prune} and $\Pi_{mask}$ in Figure \ref{fig:protocol-mask} in this section.

%Optionally include supplemental material (complete proofs, additional experiments and plots) in appendix.
%All such materials \textbf{SHOULD be included in the main submission.}
\begin{figure}[h]
%vspace{-0.2in}
\begin{protocolbox}
\noindent
\textbf{Parties:} Server $P_0$, Client $P_1$.

\textbf{Input:} $P_0$ and $P_1$ holds $\{ \left \langle Att \right \rangle_{0}^{h}, \left \langle Att \right \rangle_{1}^{h}\}_{h=0}^{H-1} \in \mathbb{Z}_{2^{\ell}}^{n\times n}$ and $\left \langle x \right \rangle_{0}, \left \langle x \right \rangle_{1} \in \mathbb{Z}_{2^{\ell}}^{n\times D}$ respectively, where H is the number of heads, n is the number of input tokens and D is the embedding dimension of tokens. Additionally, $P_1$ holds a threshold $\theta \in \mathbb{Z}_{2^{\ell}}$.

\textbf{Output:} $P_0$ and $P_1$ get $\left \langle y \right \rangle_{0}, \left \langle y \right \rangle_{1} \in \mathbb{Z}_{2^{\ell}}^{n'\times D}$, respectively, where $y=\mathsf{Prune}(x)$ and $n'$ is the number of remaining tokens.

\noindent\rule{13.2cm}{0.1pt} % This creates the horizontal line
\textbf{Protocol:}
\begin{enumerate}[label=\arabic*:, leftmargin=*]
    \item For $h \in [H]$, $P_0$ and $P_1$ compute locally with input $\left \langle Att \right \rangle^{h}$, and learn the importance score in each head $\left \langle s \right \rangle^{h} \in \mathbb{Z}_{2^{\ell}}^{n} $, where $\left \langle s \right \rangle^{h}[j] = \frac{1}{n} \sum_{i=0}^{n-1} \left \langle Att \right \rangle^{h}[i,j]$.
    \item $P_0$ and $P_1$ compute locally with input $\{ \left \langle s \right \rangle^{i} \in \mathbb{Z}_{2^{\ell}}^{n}  \}_{i=0}^{H-1}$, and learn the final importance score $\left \langle S \right \rangle \in \mathbb{Z}_{2^{\ell}}^{n}$ for each token, where  $\left \langle S \right \rangle[i] = \frac{1}{H} \sum_{h=0}^{H-1} \left \langle s \right \rangle^{h}[i]$.
    \item  For $i \in [n]$, $P_0$ and $P_1$ invoke $\Pi_{CMP}$ with inputs  $\left \langle S \right \rangle$ and $ \theta $, and learn  $\left \langle M \right \rangle \in \mathbb{Z}_{2^{\ell}}^{n}$, such that$\left \langle M \right \rangle[i] = \Pi_{CMP}(\left \langle S \right \rangle[i] - \theta) $, where: \\
    $M[i] = \begin{cases}
        1  &\text{if}\ S[i] > \theta, \\
        0  &\text{otherwise}.
            \end{cases} $
    % \item If the pruning location is insensitive, $P_0$ and $P_1$ learn real mask $M$ instead of shares $\left \langle M \right \rangle$. $P_0$ and $P_1$ compute $\left \langle y \right \rangle$ with input $\left \langle x \right \rangle$ and $M$, where  $\left \langle x \right \rangle[i]$ is pruned if $M[i]$ is $0$.
    \item $P_0$ and $P_1$ invoke $\Pi_{mask}$ with inputs  $\left \langle x \right \rangle$ and pruning mask $\left \langle M \right \rangle$, and set outputs as $\left \langle y \right \rangle$.
\end{enumerate}
\end{protocolbox}
\setlength{\abovecaptionskip}{-1pt} % Reduces space above the caption
\caption{Secure Token Pruning Protocol $\Pi_{prune}$.}
\label{fig:protocol-prune}
\end{figure}




\begin{figure}[h]
\begin{protocolbox}
\noindent
\textbf{Parties:} Server $P_0$, Client $P_1$.

\textbf{Input:} $P_0$ and $P_1$ hold $\left \langle x \right \rangle_{0}, \left \langle x \right \rangle_{1} \in \mathbb{Z}_{2^{\ell}}^{n\times D}$ and  $\left \langle M \right \rangle_{0}, \left \langle M \right \rangle_{1} \in \mathbb{Z}_{2^{\ell}}^{n}$, respectively, where n is the number of input tokens and D is the embedding dimension of tokens.

\textbf{Output:} $P_0$ and $P_1$ get $\left \langle y \right \rangle_{0}, \left \langle y \right \rangle_{1} \in \mathbb{Z}_{2^{\ell}}^{n'\times D}$, respectively, where $y=\mathsf{Prune}(x)$ and $n'$ is the number of remaining tokens.

\noindent\rule{13.2cm}{0.1pt} % This creates the horizontal line
\textbf{Protocol:}
\begin{enumerate}[label=\arabic*:, leftmargin=*]
    \item For $i \in [n]$, $P_0$ and $P_1$ set $\left \langle M \right \rangle$ to the MSB of $\left \langle x \right \rangle$ and learn the masked tokens $\left \langle \Bar{x} \right \rangle \in Z_{2^{\ell}}^{n\times D}$, where
    $\left \langle \Bar{x}[i] \right \rangle = \left \langle x[i] \right \rangle + (\left \langle M[i] \right \rangle << f)$ and $f$ is the fixed-point precision.
    \item $P_0$ and $P_1$ compute the sum of $\{\Pi_{B2A}(\left \langle M \right \rangle[i]) \}_{i=0}^{n-1}$, and learn the number of remaining tokens $n'$ and the number of tokens to be pruned $m$, where $m = n-n'$.
    \item For $k\in[m]$, for $i\in[n-k-1]$, $P_0$ and $P_1$ invoke $\Pi_{msb}$ to learn the highest bit of $\left \langle \Bar{x}[i] \right \rangle$, where $b=\mathsf{MSB}(\Bar{x}[i])$. With the highest bit of $\Bar{x}[i]$, $P_0$ and $P_1$ perform a oblivious swap between $\Bar{x}[i]$ and $\Bar{x}[i+1]$:
    $\begin{cases}
        \Tilde{x}[i] = b\cdot \Bar{x}[i] + (1-b)\cdot \Bar{x}[i+1] \\
        \Tilde{x}[i+1] = b\cdot \Bar{x}[i+1] + (1-b)\cdot \Bar{x}[i]
    \end{cases} $ \\
    $P_0$ and $P_1$ learn the swapped token sequence $\left \langle \Tilde{x} \right \rangle$.
    \item $P_0$ and $P_1$ truncate $\left \langle \Tilde{x} \right \rangle$ locally by keeping the first $n'$ tokens, clear current MSB (all remaining token has $1$ on the MSB), and set outputs as $\left \langle y \right \rangle$.
\end{enumerate}
\end{protocolbox}
\setlength{\abovecaptionskip}{-1pt} % Reduces space above the caption
\caption{Secure Mask Protocol $\Pi_{mask}$.}
\label{fig:protocol-mask}
%\vspace{-0.2in}
\end{figure}

% \begin{wrapfigure}{r}{0.35\textwidth}  % 'r' for right, and the width of the figure area
%   \centering
%   \includegraphics[width=0.35\textwidth]{figures/msb.pdf}
%   \caption{Runtime of $\Pi_{prune}$ and $\Pi_{mask}$ in different layers. We compare different secure pruning strategies based on the BERT Base model.}
%   \label{fig:msb}
%   \vspace{-0.1in}
% \end{wrapfigure}

% \begin{figure}[h]  % 'r' for right, and the width of the figure area
%   \centering
%   \includegraphics[width=0.4\textwidth]{figures/msb.pdf}
%   \caption{Runtime of $\Pi_{prune}$ and $\Pi_{mask}$ in different layers. We compare different secure pruning strategies based on the BERT Base model.}
%   \label{fig:msb}
%   % \vspace{-0.1in}
% \end{figure}

\textbf{Complexity of $\Pi_{mask}$.} The complexity of the proposed $\Pi_{mask}$ mainly depends on the number of oblivious swaps. To prune $m$ tokens out of $n$ input tokens, $O(mn)$ swaps are needed. Since token pruning is performed progressively, only a small number of tokens are pruned at each layer, which makes $\Pi_{mask}$ efficient during runtime. Specifically, for a BERT base model with 128 input tokens, the pruning protocol only takes $\sim0.9$s on average in each layer. An alternative approach is to invoke an oblivious sort algorithm~\citep{bogdanov2014swap2,pang2023bolt} on $\left \langle \Bar{x} \right \rangle$. However, this approach is less efficient because it blindly sort the whole token sequence without considering $m$. That is, even if only $1$ token needs to be pruned, $O(nlog^{2}n)\sim O(n^2)$ oblivious swaps are needed, where as the proposed $\Pi_{mask}$ only need $O(n)$ swaps. More generally, for an $\ell$-layer Transformer with a total of $m$ tokens pruned, the overall time complexity using the sort strategy would be $O(\ell n^2)$ while using the swap strategy remains an overall complexity of $O(mn).$ Specifically, using the sort strategy to prune tokens in one BERT Base model layer can take up to $3.8\sim4.5$ s depending on the sorting algorithm used. In contrast, using the swap strategy only needs $0.5$ s. Moreover, alternative to our MSB strategy, one can also swap the encrypted mask along with the encrypted token sequence. However, we find that this doubles the number of swaps needed, and thus is less efficient the our MSB strategy, as is shown in Figure \ref{fig:msb}.

\section{Existing Protocols}
\label{app:protocol}
\noindent\textbf{Existing Protocols Used in Our Private Inference.}  In our private inference framework, we reuse several existing cryptographic protocols for basic computations. $\Pi_{MatMul}$ \citep{pang2023bolt} processes two ASS matrices and outputs their product in SS form. For non-linear computations, protocols $\Pi_{SoftMax}, \Pi_{GELU}$, and $\Pi_{LayerNorm}$\citep{lu2023bumblebee, pang2023bolt} take a secret shared tensor and return the result of non-linear functions in ASS. Basic protocols from~\citep{rathee2020cryptflow2, rathee2021sirnn} are also utilized. $\Pi_{CMP}$\citep{EzPC}, for example, inputs ASS values and outputs a secret shared comparison result, while $\Pi_{B2A}$\citep{EzPC} converts secret shared Boolean values into their corresponding arithmetic values.

\section{Polynomial Reduction for Non-linear Functions}
\label{app:b}
The $\mathsf{SoftMax}$ and $\mathsf{GELU}$ functions can be approximated with polynomials. High-degree polynomials~\citep{lu2023bumblebee, pang2023bolt} can achieve the same accuracy as the LUT-based methods~\cite{hao2022iron-iron}. While these polynomial approximations are more efficient than look-up tables, they can still incur considerable overheads. Reducing the high-degree polynomials to the low-degree ones for the less important tokens can imporve efficiency without compromising accuracy. The $\mathsf{SoftMax}$ function is applied to each row of an attention map. If a token is to be reduced, the corresponding row will be computed using the low-degree polynomial approximations. Otherwise, the corresponding row will be computed accurately via a high-degree one. That is if $M_{\beta}'[i] = 1$, $P_0$ and $P_1$ uses high-degree polynomials to compute the $\mathsf{SoftMax}$ function on token $x[i]$:
\begin{equation}
\mathsf{SoftMax}_{i}(x) = \frac{e^{x_i}}{\sum_{j\in [d]}e^{x_j}}
\end{equation}
where $x$ is a input vector of length $d$ and the exponential function is computed via a polynomial approximation. For the $\mathsf{SoftMax}$ protocol, we adopt a similar strategy as~\citep{kim2021ibert, hao2022iron-iron}, where we evaluate on the normalized inputs $\mathsf{SoftMax}(x-max_{i\in [d]}x_i)$. Different from~\citep{hao2022iron-iron}, we did not used the binary tree to find max value in the given vector. Instead, we traverse through the vector to find the max value. This is because each attention map is computed independently and the binary tree cannot be re-used. If $M_{\beta}[i] = 0$, $P_0$ and $P_1$ will approximate the $\mathsf{SoftMax}$ function with low-degree polynomial approximations. We detail how $\mathsf{SoftMax}$ can be approximated as follows:
\begin{equation}
\label{eq:app softmax}
\mathsf{ApproxSoftMax}_{i}(x) = \frac{\mathsf{ApproxExp}(x_i)}{\sum_{j\in [d]}\mathsf{ApproxExp}(x_j)}
\end{equation}
\begin{equation}
\mathsf{ApproxExp}(x)=\begin{cases}
    0  &\text{if}\ x \leq T \\
    (1+ \frac{x}{2^n})^{2^n} &\text{if}\ x \in [T,0]\\
\end{cases}
\end{equation}
where the $2^n$-degree Taylor series is used to approximate the exponential function and $T$ is the clipping boundary. The value $n$ and $T$ determines the accuracy of above approximation. With $n=6$ and $T=-13$, the approximation can achieve an average error within $2^{-10}$~\citep{lu2023bumblebee}. For low-degree polynomial approximation, $n=3$ is used in the Taylor series.

Similarly, $P_0$ or $P_1$ can decide whether or not to approximate the $\mathsf{GELU}$ function for each token. If $M_{\beta}[i] = 1$, $P_0$ and $P_1$ use high-degree polynomials~\citep{lu2023bumblebee} to compute the $\mathsf{GELU}$ function on token $x[i]$ with high-degree polynomial:
% \begin{equation}
% \mathsf{GELU}(x) = 0.5x(1+\mathsf{Tanh}(\sqrt{2/\pi}(x+0.044715x^3)))
% \end{equation}
% where the $\mathsf{Tanh}$ and square root function are computed via a OT-based lookup-table.

\begin{equation}
\label{eq:app gelu}
\mathsf{ApproxGELU}(x)=\begin{cases}
    0  &\text{if}\ x \leq -5 \\
    P^3(x), &\text{if}\ -5 < x \leq -1.97 \\
    P^6(x), &\text{if}\ -1.97 < x \leq 3  \\
    x, &\text{if}\ x >3 \\
\end{cases}
\end{equation}
where $P^3(x)$ and $P^6(x)$ are degree-3 and degree-6 polynomials respectively. The detailed coefficient for the polynomial is: 
\begin{equation*}
    P^3(x) = -0.50540312 -  0.42226581x - 0.11807613x^2 - 0.01103413x^3
\end{equation*}
, and
\begin{equation*}
    P^6(x) = 0.00852632 + 0.5x + 0.36032927x^2 - 0.03768820x^4 + 0.00180675x^6
\end{equation*}

For BOLT baseline, we use another high-degree polynomial to compute the $\mathsf{GELU}$ function.

\begin{equation}
\label{eq:app gelu}
\mathsf{ApproxGELU}(x)=\begin{cases}
    0  &\text{if}\ x < -2.7 \\
    P^4(x), &\text{if}\   |x| \leq 2.7 \\
    x, &\text{if}\ x >2.7 \\
\end{cases}
\end{equation}
We use the same coefficients for $P^4(x)$ as BOLT~\citep{pang2023bolt}.

\begin{figure}[h]
 % \vspace{-0.1in}
    \centering
    \includegraphics[width=1\linewidth]{figures/bumble.pdf}
    % \captionsetup{skip=2pt}
    % \vspace{-0.1in}
    \caption{Comparison with prior works on the BERT model. The input has 128 tokens.}
    \label{fig:bumble}
\end{figure}

If $M_{\beta}'[i] = 0$, $P_0$ and $P_1$ will use low-degree 
polynomial approximation to compute the $\mathsf{GELU}$ function instead. Encrypted polynomial reduction leverages low-degree polynomials to compute non-linear functions for less important tokens. For the $\mathsf{GELU}$ function, the following degree-$2$ polynomial~\cite{kim2021ibert} is used:
\begin{equation*}
\mathsf{ApproxGELU}(x)=\begin{cases}
    0  &\text{if}\ x <  -1.7626 \\
    0.5x+0.28367x^2, &\text{if}\ x \leq |1.7626| \\
    x, &\text{if}\ x > 1.7626\\
\end{cases}
\end{equation*}


\section{Comparison with More Related Works.}
\label{app:c}
\textbf{Other 2PC frameworks.} The primary focus of CipherPrune is to accelerate the private Transformer inference in the 2PC setting. As shown in Figure \ref{fig:bumble}, CipherPrune can be easily extended to other 2PC private inference frameworks like BumbleBee~\citep{lu2023bumblebee}. We compare CipherPrune with BumbleBee and IRON on BERT models. We test the performance in the same LAN setting as BumbleBee with 1 Gbps bandwidth and 0.5 ms of ping time. CipherPrune achieves more than $\sim 60 \times$ speed up over BOLT and $4.3\times$ speed up over BumbleBee.

\begin{figure}[t]
 % \vspace{-0.1in}
    \centering
    \includegraphics[width=1\linewidth]{figures/pumab.pdf}
    % \captionsetup{skip=2pt}
    % \vspace{-0.1in}
    \caption{Comparison with MPCFormer and PUMA on the BERT models. The input has 128 tokens.}
    \label{fig:pumab}
\end{figure}

\begin{figure}[h]
 % \vspace{-0.1in}
    \centering
    \includegraphics[width=1\linewidth]{figures/pumag.pdf}
    % \captionsetup{skip=2pt}
    % \vspace{-0.1in}
    \caption{Comparison with MPCFormer and PUMA on the GPT2 models. The input has 128 tokens. The polynomial reduction is not used.}
    \label{fig:pumag}
\end{figure}

\textbf{Extension to 3PC frameworks.} Additionally, we highlight that CipherPrune can be also extended to the 3PC frameworks like MPCFormer~\citep{li2022mpcformer} and PUMA~\citep{dong2023puma}. This is because CipherPrune is built upon basic primitives like comparison and Boolean-to-Arithmetic conversion. We compare CipherPrune with MPCFormer and PUMA on both the BERT and GPT2 models. CipherPrune has a $6.6\sim9.4\times$ speed up over MPCFormer and $2.8\sim4.6\times$ speed up over PUMA on the BERT-Large and GPT2-Large models.


\section{Communication Reduction in SoftMax and GELU.}
\label{app:e}

\begin{figure}[h]
    \centering
    \includegraphics[width=0.9\linewidth]{figures/layerwise.pdf}
    \caption{Toy example of two successive Transformer layers. In layer$_i$, the SoftMax and Prune protocol have $n$ input tokens. The number of input tokens is reduced to $n'$ for the Linear layers, LayerNorm and GELU in layer$_i$ and SoftMax in layer$_{i+1}$.}
    \label{fig:layer}
\end{figure}

\begin{table*}[h]
\captionsetup{skip=2pt}
\centering
\scriptsize
\caption{Communication cost (in MB) of the SoftMax and GELU protocol in each Transformer layer.}
\begin{tblr}{
    colspec = {c |c c c c c c c c c c c c},
    row{1} = {font=\bfseries},
    row{2-Z} = {rowsep=1pt},
    % row{4} = {bg=LightBlue},
    colsep = 2.5pt,
    }
\hline
\textbf{Layer Index} & \textbf{0}  & \textbf{1}  & \textbf{2} & \textbf{3} & \textbf{4} & \textbf{5} & \textbf{6} & \textbf{7} & \textbf{8} & \textbf{9} & \textbf{10} & \textbf{11} \\
\hline
Softmax & 642.19 & 642.19 & 642.19 & 642.19 & 642.19 & 642.19 & 642.19 & 642.19 & 642.19 & 642.19 & 642.19 & 642.19 \\
Pruned Softmax & 642.19 & 129.58 & 127.89 & 119.73 & 97.04 & 71.52 & 43.92 & 21.50 & 10.67 & 6.16 & 4.65 & 4.03 \\
\hline
GELU & 698.84 & 698.84 & 698.84 & 698.84 & 698.84 & 698.84 & 698.84 & 698.84 & 698.84 & 698.84 & 698.84 & 698.84\\
Pruned GELU  & 325.10 & 317.18 & 313.43 & 275.94 & 236.95 & 191.96 & 135.02 & 88.34 & 46.68 & 16.50 & 5.58 & 5.58\\
\hline
\end{tblr}
\label{tab:layer}
\end{table*}

{
In Figure \ref{fig:layer}, we illustrate why CipherPrune can reduce the communication overhead of both  SoftMax and GELU. Suppose there are $n$ tokens in $layer_i$. Then, the SoftMax protocol in the attention module has a complexity of $O(n^2)$. CipherPrune's token pruning protocol is invoked to select $n'$ tokens out of all $n$ tokens, where $m=n-n'$ is the number of tokens that are removed. The overhead of the GELU function in $layer_i$, i.e., the current layer, has only $O(n')$ complexity (which should be $O(n)$ without token pruning). The complexity of the SoftMax function in $layer_{i+1}$, i.e., the following layer, is reduced to $O(n'^2)$ (which should be $O(n^2)$ without token pruning). The SoftMax protocol has quadratic complexity with respect to the token number and the GELU protocol has linear complexity. Therefore, CipherPrune can reduce the overhead of both the GELU protocol and the SoftMax protocols by reducing the number of tokens. In Table \ref{tab:layer}, we provide detailed layer-wise communication cost of the GELU and the SoftMax protocol. Compared to the unpruned baseline, CipherPrune can effectively reduce the overhead of the GELU and the SoftMax protocols layer by layer.
}

\section{Analysis on Layer-wise redundancy.}
\label{app:f}

\begin{figure}[h]
    \centering
    \includegraphics[width=0.9\linewidth]{figures/layertime0.pdf}
    \caption{The number of pruned tokens and pruning protocol runtime in different layers in the BERT Base model. The results are averaged across 128 QNLI samples.}
    \label{fig:layertime}
\end{figure}

{
In Figure \ref{fig:layertime}, we present the number of pruned tokens and the runtime of the pruning protocol for each layer in the BERT Base model. The number of pruned tokens per layer was averaged across 128 QNLI samples, while the pruning protocol runtime was measured over 10 independent runs. The mean token count for the QNLI samples is 48.5. During inference with BERT Base, input sequences with fewer tokens are padded to 128 tokens using padding tokens. Consistent with prior token pruning methods in plaintext~\citep{goyal2020power}, a significant number of padding tokens are removed at layer 0.  At layer 0, the number of pruned tokens is primarily influenced by the number of padding tokens rather than token-level redundancy.
%In Figure \ref{fig:layertime}, we demonstrate the number of pruned tokens and the pruning protocol runtime in each layer in the BERT Base model. We averaged the number of pruned tokens in each layer across 128 QNLI samples and then tested the pruning protocol runtime in 10 independent runs. The mean token number of the QNLI samples is 48.5. During inference with BERT Base, input sequences with small token number are padded to 128 tokens with padding tokens. Similar to prior token pruning methods in the plaintext~\citep{goyal2020power}, a large number of padding tokens can be removed at layer 0. We remark that token-level redundancy builds progressively throughout inference~\citep{goyal2020power, kim2022LTP}. The number of pruned tokens in layer 0 mostly depends on the number of padding tokens instead of token-level redundancy.
}

{
%As shown in Figure \ref{fig:layertime}, more tokens are removed in the intermediate layers, e.g., layer $4$ to layer $7$. This suggests there is more redundant information in these intermediate layers. 
In CipherPrune, tokens are removed progressively, and once removed, they are excluded from computations in subsequent layers. Consequently, token pruning in earlier layers affects computations in later layers, whereas token pruning in later layers does not impact earlier layers. As a result, even if layers 4 and 7 remove the same number of tokens, layer 7 processes fewer tokens overall, as illustrated in Figure \ref{fig:layertime}. Specifically, 8 tokens are removed in both layer $4$ and layer $7$, but the runtime of the pruning protocol in layer $4$ is $\sim2.4\times$ longer than that in  layer $7$.
}

\section{Related Works}
\label{app:g}

{
In response to the success of Transformers and the need to safeguard data privacy, various private Transformer Inferences~\citep{chen2022thex,zheng2023primer,hao2022iron-iron,li2022mpcformer, lu2023bumblebee, luo2024secformer, pang2023bolt}  are proposed. To efficiently run private Transformer inferences, multiple cryptographic primitives are used in a popular hybrid HE/MPC method IRON~\citep{hao2022iron-iron}, i.e., in a Transformer, HE and SS are used for linear layers, and SS and OT are adopted for nonlinear layers. IRON and BumbleBee~\citep{lu2023bumblebee} focus on optimizing linear general matrix multiplications; SecFormer~\cite{luo2024secformer} improves the non-linear operations like the exponential function with polynomial approximation; BOLT~\citep{pang2023bolt} introduces the baby-step giant-step (BSGS) algorithm to reduce the number of HE rotations, proposes a word elimination (W.E.) technique, and uses polynomial approximation for non-linear operations, ultimately achieving state-of-the-art (SOTA) performance.
}

{Other than above hybrid HE/MPC methods, there are also works exploring privacy-preserving Transformer inference using only HE~\citep{zimerman2023converting, zhang2024nonin}. The first HE-based private Transformer inference work~\citep{zimerman2023converting} replaces \mysoftmax function with a scaled-ReLU function. Since the scaled-ReLU function can be approximated with low-degree polynomials more easily, it can be computed more efficiently using only HE operations. A range-loss term is needed during training to reduce the polynomial degree while maintaining high accuracy. A training-free HE-based private Transformer inference was proposed~\citep{zhang2024nonin}, where non-linear operations are approximated by high-degree polynomials. The HE-based methods need frequent bootstrapping, especially when using high-degree polynomials, thus often incurring higher overhead than the hybrid HE/MPC methods in practice.
}


\section*{Acknowledgements}
The authors would like to acknowledge the CERES network, the University of Washington Global Innovation Funds (GIF), and the University of Washington Student Technology Funds (STF), which provided support for this work. We additionally thank the anonymous reviewers for their detailed feedback and the participants for sharing their thoughts. We truly appreciate all the help. Alexis Hiniker is a special government employee for the Federal Trade Commission. The content expressed in this manuscript does not reflect the views of the Commission or any of the Commissioners.

%-------------------------------------------------------------------------------
\bibliographystyle{plain}
\bibliography{references}

%%%%%%%%%%%%%%%%%%%%%%%%%%%%%%%%%%%%%%%%%%%%%%%%%%%%%%%%%%%%%%%%%%%%%%%%%%%%%%%%
\end{document}
%%%%%%%%%%%%%%%%%%%%%%%%%%%%%%%%%%%%%%%%%%%%%%%%%%%%%%%%%%%%%%%%%%%%%%%%%%%%%%%%

%%  LocalWords:  endnotes includegraphics fread ptr nobj noindent
%%  LocalWords:  pdflatex acks
