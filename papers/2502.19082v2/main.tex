%%%%%%%%%%%%%%%%%%%%%%%%%%%%%%%%%%%%%%%%%%%%%%%%%%%%%%%%%%%%%%%%%%%%%%%%%%%%%%%%
% Template for USENIX papers.
%
% History:
%
% - TEMPLATE for Usenix papers, specifically to meet requirements of
%   USENIX '05. originally a template for producing IEEE-format
%   articles using LaTeX. written by Matthew Ward, CS Department,
%   Worcester Polytechnic Institute. adapted by David Beazley for his
%   excellent SWIG paper in Proceedings, Tcl 96. turned into a
%   smartass generic template by De Clarke, with thanks to both the
%   above pioneers. Use at your own risk. Complaints to /dev/null.
%   Make it two column with no page numbering, default is 10 point.
%
% - Munged by Fred Douglis <douglis@research.att.com> 10/97 to
%   separate the .sty file from the LaTeX source template, so that
%   people can more easily include the .sty file into an existing
%   document. Also changed to more closely follow the style guidelines
%   as represented by the Word sample file.
%
% - Note that since 2010, USENIX does not require endnotes. If you
%   want foot of page notes, don't include the endnotes package in the
%   usepackage command, below.
% - This version uses the latex2e styles, not the very ancient 2.09
%   stuff.
%
% - Updated July 2018: Text block size changed from 6.5" to 7"
%
% - Updated Dec 2018 for ATC'19:
%
%   * Revised text to pass HotCRP's auto-formatting check, with
%     hotcrp.settings.submission_form.body_font_size=10pt, and
%     hotcrp.settings.submission_form.line_height=12pt
%
%   * Switched from \endnote-s to \footnote-s to match Usenix's policy.
%
%   * \section* => \begin{abstract} ... \end{abstract}
%
%   * Make template self-contained in terms of bibtex entires, to allow
%     this file to be compiled. (And changing refs style to 'plain'.)
%
%   * Make template self-contained in terms of figures, to
%     allow this file to be compiled. 
%
%   * Added packages for hyperref, embedding fonts, and improving
%     appearance.
%   
%   * Removed outdated text.
%
%%%%%%%%%%%%%%%%%%%%%%%%%%%%%%%%%%%%%%%%%%%%%%%%%%%%%%%%%%%%%%%%%%%%%%%%%%%%%%%%

%%% Minor updates for SOUPS 2019 by Michelle Mazurek
%%% Minor updates for SOUPS 2022 by Rick Wash

\documentclass[letterpaper,twocolumn,10pt]{article}
\usepackage{usenix2025_SOUPS}

% to be able to draw some self-contained figs
\usepackage{tikz}
\usepackage{amsmath}
\PassOptionsToPackage{table,xcdraw}{xcolor} 

\usepackage{textgreek}
\usepackage{booktabs}
\usepackage{multirow}
\usepackage{subcaption}
\usepackage{xcolor,colortbl}
\definecolor{gray}{rgb}{0.1,0.1,0.1}
\usepackage[T1]{fontenc}
\usepackage[english]{babel}
\usepackage{graphicx}
\usepackage[flushleft]{threeparttable}
\usepackage{adjustbox}
\usepackage{subcaption}
\usepackage{longtable}
\usepackage{threeparttablex}

% inlined bib file
\usepackage{filecontents}

%-------------------------------------------------------------------------------
\begin{filecontents}{references.bib}
%-------------------------------------------------------------------------------
\end{filecontents}

\usepackage[utf8]{inputenc}

%% full size page
\usepackage{fullpage}
\usepackage[margin = 2.5cm]{geometry}

%% AMS packages
\usepackage{amsmath, amsthm, amssymb}
\usepackage{thmtools}
\usepackage{thm-restate}
\usepackage{mathtools}  
\usepackage{xfrac} 
%\usepackage[normalem]{ulem}

\usepackage{placeins} 
\usepackage{times}

%%bibliography
% \usepackage{amsthm}
\usepackage{url}
\usepackage{array}


%% algorithm
%\usepackage{algorithm}
%\usepackage{algpseudocode}
\usepackage[ruled,vlined]{algorithm2e}
% Create a new environment named "algorithm2e" that behaves like "algorithm"
\newenvironment{algorithm2e}[1][]{%
    \begin{algorithm}[#1]%
}{%
    \end{algorithm}
}

%% graphics and colors
\usepackage{graphicx}
\usepackage{color}
%\usepackage[dvipsnames]{xcolor}

%%bibliography
\usepackage[round]{natbib}
\usepackage[hyperindex,breaklinks]{hyperref}
\usepackage{url}

%%color box figure
\usepackage{tcolorbox}

%%draw figure
\usepackage{tikz}

%%misc
\usepackage{nicefrac}

%\citestyle{acmauthoryear}

% \declaretheorem[name=Theorem]{theorem}

%% define theorems, lemmas, claims
\newtheorem{theorem}{Theorem}[section]
\newtheorem{claim}[theorem]{Claim}
\newtheorem{corollary}[theorem]{Corollary}
\newtheorem{proposition}[theorem]{Proposition}
\newtheorem{lemma}[theorem]{Lemma}
\newtheorem{definition}[theorem]{Definition}
\newtheorem{observation}[theorem]{Observation}
\newtheorem{question}[theorem]{Question}
\newtheorem{assumption}[theorem]{Assumption}
\newtheorem*{remark*}{Remark}


% \newenvironment{numberedtheorem}[1]{%
% \renewcommand{\thetheorem}{#1}%
% \begin{theorem}}{\end{theorem}\addtocounter{theorem}{-1}}

% \newenvironment{numberedlemma}[1]{%
% \renewcommand{\thetheorem}{#1}%
% \begin{lemma}}{\end{lemma}\addtocounter{theorem}{-1}}

% \newenvironment{oneshot}[1]{\@begintheorem{#1}{\unskip}}{\@endtheorem}

% \makeatletter
% \newtheorem*{rep@theorem}{\rep@title}
% \newcommand{\newreptheorem}[2]{%
% \newenvironment{rep#1}[1]{%
%  \def\rep@title{#2 \ref{##1}}%
%  \begin{rep@theorem}}%
%  {\end{rep@theorem}}}
% \makeatother


%cleveref package loaded at the end
\usepackage{cleveref}
\crefname{theorem}{theorem}{theorems}
\Crefname{theorem}{Theorem}{Theorems}

\crefname{lemma}{lemma}{lemmas}
\Crefname{lemma}{Lemma}{Lemmas}

\crefname{claim}{claim}{claims}
\Crefname{claim}{Claim}{Claims}

\crefname{corollary}{corollary}{corollaries}
\Crefname{corollary}{Corollary}{Corollaries}

\crefname{proposition}{proposition}{propositions}
\Crefname{proposition}{Proposition}{Propositions}

\crefname{definition}{definition}{definitions}  % Explicitly set "Definition"
\Crefname{definition}{Definition}{Definitions}

\crefname{observation}{observation}{observations}
\Crefname{observation}{Observation}{Observations}

\crefname{question}{question}{questions}
\Crefname{question}{Question}{Questions}

\crefname{assumption}{assumption}{assumptions}
\Crefname{assumption}{Assumption}{Assumptions}

\crefname{algorithm}{algorithm}{algorithms}
\Crefname{algorithm}{Algorithm}{Algorithms}

\crefname{AlgoLine}{line}{lines}  % If using `algorithm2e` with line numbers
\Crefname{AlgoLine}{Line}{Lines}


%% probability notation
\DeclareMathOperator{\cov}{cov}
\DeclareMathOperator{\sgn}{\mathbf{sgn}}
\DeclareMathOperator{\E}{\mathbf{E}}
\DeclareMathOperator{\Var}{\mathbf{Var}}
\DeclareMathOperator{\one}{\mathbf{1}}
\newcommand{\given}{\mid}
\DeclareMathOperator{\Ball}{Ball}
\DeclareMathOperator{\tr}{tr}

% rounding up and down
\newcommand {\roundup}   [1] {{\lceil {#1} \rceil}}
\newcommand {\rounddown} [1] {{\lfloor {#1} \rfloor}}

%% black board letters
\newcommand{\bbB}{\mathbb{B}}
\newcommand{\bbC}{\mathbb{C}}
\newcommand{\bbR}{\mathbb{R}}
\newcommand{\bbZ}{\mathbb{Z}}

%% calligraphic letters 
\newcommand{\calA}{\mathcal{A}}
\newcommand{\calB}{\mathcal{B}}
\newcommand{\calC}{\mathcal{C}}
\newcommand{\calD}{\mathcal{D}}
\newcommand{\calE}{\mathcal{E}}
\newcommand{\calF}{\mathcal{F}}
\newcommand{\calG}{\mathcal{G}}
\newcommand{\calH}{\mathcal{H}}
\newcommand{\calI}{\mathcal{I}}
\newcommand{\calJ}{\mathcal{J}}
\newcommand{\calK}{\mathcal{K}}
\newcommand{\calL}{\mathcal{L}}
\newcommand{\calM}{\mathcal{M}}
\newcommand{\calN}{\mathcal{N}}
\newcommand{\calO}{\mathcal{O}}
\newcommand{\calP}{\mathcal{P}}
\newcommand{\calQ}{\mathcal{Q}}
\newcommand{\calR}{\mathcal{R}}
\newcommand{\calS}{\mathcal{S}}
\newcommand{\calT}{\mathcal{T}}
\newcommand{\calU}{\mathcal{U}}
\newcommand{\calV}{\mathcal{V}}
\newcommand{\calW}{\mathcal{W}}
\newcommand{\calX}{\mathcal{X}}
\newcommand{\calY}{\mathcal{Y}}
\newcommand{\calZ}{\mathcal{Z}}

\newcommand{\N}{\mathbb{N}}
\newcommand{\R}{\mathbb{R}}
\DeclareMathOperator*{\argmin}{arg\,min}
\DeclareMathOperator*{\argmax}{arg\,max}




\newcommand{\err}{\mathrm{err}}
\newcommand{\errstar}{\mathrm{err}^*}

\DeclareMathOperator{\REC}{REC}
\DeclareMathOperator{\NRD}{NRD}
\DeclareMathOperator{\FN}{FN}
\DeclareMathOperator{\FP}{FP}

\DeclareMathOperator{\Tr}{Tr}

\newcommand{\conv}{\mathrm{conv}}
\newcommand{\cone}{\mathrm{cone}}
\newcommand{\inner}[2]{\langle #1, #2\rangle}


%-------------------------------------------------------------------------------
\begin{document}
%-------------------------------------------------------------------------------

%don't want date printed
\date{}

\title{\Large \bf Trust-Enabled Privacy:\\Social Media Designs to Support Adolescent User Boundary Regulation}

% if you leave this blank it will default to a possibly ugly attempt 
% to make the contents of the \author command below into a string
\def\plainauthor{Author name(s) for PDF metadata. Don't forget to anonymize for submission!}

%for single author (just remove % characters)
\author{
{\rm JaeWon Kim}\\
University of Washington
\and
{\rm Robert Wolfe}\\
University of Washington
\and
{\rm Ramya Bhagirathi Subramanian}\\
Smarten Spaces
\and
{\rm Mei-Hsuan Lee}\\
National Renewable Energy Laboratory
\and
{\rm Jessica Colnago}\\
{  }
\and
{\rm Alexis Hiniker}\\
University of Washington
% copy the following lines to add more authors
% \and
% {\rm Name}\\
%Name Institution
} % end author

\maketitle
% \thecopyright

\begin{abstract}
Through a three-part co-design study involving 19 teens aged 13–18, we identify key barriers to effective boundary regulation on social media, including ambiguous audience expectations, social risks associated with oversharing, and the lack of design affordances that facilitate trust-building. Our findings reveal that while adolescents seek casual, frequent sharing to strengthen relationships, existing platform norms and designs often discourage such interactions, leading to withdrawal. To address these challenges, we introduce \textit{trust-enabled privacy} as a design framework that recognizes trust---whether building or eroding---as central to boundary regulation. When trust is supported, boundary regulation becomes more adaptive and empowering; when it erodes, users default to self-censorship or withdrawal. We propose concrete design affordances, including guided disclosure, contextual audience segmentation, intentional engagement signaling, and trust-centered norms, to help platforms foster a more dynamic and nuanced privacy experience for teen social media users. By reframing privacy as a trust-driven process rather than a rigid control-based trade-off, this work provides empirical insights and actionable guidelines for designing social media environments that empower teens to manage their online presence while fostering meaningful social connections.
\end{abstract}

\section{Introduction}\label{sec:intro}

In computational finance, Monte Carlo simulations are used extensively to estimate the expected value of financial payoffs based on the solution of stochastic differential equations (SDEs) which model the evolution of stock prices, interest rates, exchange rates and other quantities \cite{glasserman04}.  Monte Carlo methods are very general and flexible, but for high accuracy it requires generating a large number of costly SDE path approximations, which has motivated research into a number of variance reduction or, equivalently, cost reduction techniques. One such method is
Multilevel Monte Carlo (MLMC), which was proposed in \cite{GILES2008} and was adapted for various applications that are summarised in \cite{Giles_overview17} and successfully combined with other methods such as quasi-Monte Carlo methods. The main idea of MLMC is to approximate the payoff using different time stepping resolutions when numerically solving the underlying SDE and to generate an optimal number of samples on each level, such that the overall computational cost is minimised subject to the desired bound on the variance. %, such that the total computational cost is minimised. 
The computational savings come from the fact that most samples are computed on the coarser levels and hence are less expensive while only a few samples from the finest levels are required \cite{GILES2008}.


Among the directions in which the computational cost 
of MLMC methods could further be reduced, an important avenue is the use of lower precision calculations, especially for the first Monte Carlo levels where the targeted accuracy is relatively low. 
 An overview of the research on mixed precision for the standard Monte Carlo (MC) framework is provided in \cite{ChowMixedPrecisionStandardMC} but only a few references study the potential of low precision computation in the MLMC framework \cite{Rounding_error_oliver}. To the best of our knowledge, the only MLMC framework with customised precision in the literature is \cite{brugger2014mixed}, but they use a uniform precision for all operations on each Monte Carlo level instead of optimising 
 the precision of each intermediary variable to reduce as much as possible the cost of path generation.
 
An important motivation for an MLMC framework with variable precision would be performing the low precision computations on reconfigurable hardware devices such as Field Programmable Gate Arrays (FPGAs). FPGAs contain customizable logic blocks and connectors that make it easy to adapt the digital circuit architecture for a specific application, leading to a highly parallel and optimised implementation. Therefore they are successfully exploited in applications that require high speed and have high computational workload, such as signal processing \cite{woods2008fpga}, and real time applications like high frequency trading \cite{HFT1,HFT2}. That is why a number of previous works in hardware architecture design implemented the MLMC algorithm to price financial options using FPGAs as accelerators, which resulted in improved speed and power efficiency compared to full CPU architectures \cite{Schryver2013AMM}. The paper \cite{lindsey2016domain} also proposed 
a Domain Specific Language to automate the configuration of FPGAs for this specific application. However, only \cite{brugger2014mixed} proposed a heuristic to reduce the precision in calculations.

In addition, all aforementioned works considered that the random number generation (RNG) is performed in single or double precision. Yet in most cases an important portion of the workload in the overall MLMC simulation comes from the RNG and in \cite{brugger2014mixed} this limited the total computational savings.
To reduce the cost of MLMC simulations in particular those based on the Geometric Brownian Motion (GBM), \cite{approximateICDF_Oliver, NestedOliver} have proposed to use approximate random numbers that are generated by applying an approximation of the inverse CDF to uniform random numbers. In \cite{NestedOliver}, the authors proposed a way to integrate these lower precision random variables into a \textit{nested} MLMC framework and completed a numerical analysis to bound the resulting error at each MC level by a product of the time step and the error in the random number approximation. The same authors show in \cite{approximateICDF_Oliver} that using approximate random variables reduces the cost of path generation by a factor 7.


In this paper we propose a nested MLMC framework that combines the use of approximate random normal variables and lower precision calculations to reduce the computational cost of MLMC even further than \cite{brugger2014mixed,NestedOliver}. We illustrate the efficiency of our framework in Matlab, after making several assumptions on the cost of operations and size of the errors that we carefully justify. We focus on the case of GBM and use the approximate RNG methods presented in \cite{approximateICDF_Oliver} as well as a new slightly modified method that combines CDF inversion and the central limit theorem. To choose the precision of the variables in the low precision path generation, we introduce a novel method to optimise the bit-widths. This optimisation is performed before the main path generation loop is executed and is based on a linear model of the payoff error  
due to rounding when computing in low precision. The error model relies on algorithmic differentiation in a similar manner to \cite{unifying-bwoptim,bitwidth-AD,ADAPT}. The bit-width optimisation procedure can be performed off-line, so this stage can be excluded from the on-line time complexity of our framework. The user specified desired accuracy is then enforced by calculating on-line the number of samples that need to be generated.

In terms of hardware design, we suggest implementing the low precision path generation on FPGAs and the full-precision ones on a CPU or GPU. 
The FPGA offers enough flexibility to define a separate bit-width for every variable in the low precision path generation, and can be reconfigured periodically to update the bit-widths when the market parameters have changed considerably. 


The paper is organized as follows : \Cref{sec:MLMC} introduces MLMC and nested MLMC to make clear the estimator that is implemented in our framework. Then in \Cref{sec:RNG} we detail the methods that could be used to obtain approximate random normally distributed numbers very cheaply for the low precision path generation. In \Cref{sec:error_model} and \Cref{sec:costModel} we propose an error model and a cost model (resp.) that we then use to formulate the optimisation problem that is solved to obtain the optimal bit-widths of fixed point variables in \Cref{sec:optimisation}. Finally we summarise our results and future directions in \Cref{sec:conclusion}.



\section{Related Work}
Dense retrievers~\cite{karpukhin2020dense,xiong2021approximate,izacard2021unsupervised,Yu2021FewShotCD,xiong2021dense,li2021more} have demonstrated superior ranking performance by conducting semantic matching between queries and documents, which helps overcome the problem of vocabulary mismatch~\cite{belkin1982ask}. These models typically leverage Pre-trained Language Models (PLMs) to encode both queries and documents into a shared embedding space, which significantly enhances retrieval effectiveness. To further optimize this embedding space, dense retrievers are often trained contrastively using relevance signals between queries and documents~\cite{karpukhin2020dense,zhan2021optimizing}. Some studies have also developed zero-shot dense retrievers by training on weak supervision data~\cite{xie2023unsupervised} or leveraging Large Language Models (LLMs) for query expansion and reformulation~\cite{gao2022unsupervised}.


Query expansion is a long-standing research direction, originally proposed to rewrite queries and improve the retrieval accuracy of exact matching-based retrievers, such as BM25~\cite{robertson2009probabilistic}. Early query expansion methods primarily aim to bridge the lexical gap between queries and documents~\cite{carpineto2012survey,rocchio1971relevance} by expanding queries with knowledge bases~\cite{bhogal2007review,qiu1993concept,voorhees1994query} or Pseudo-Relevance Feedback (PRF)~\cite{amati2002probabilistic,robertson1990term,rocchio1971relevance}. These PRF-based methods have proven their effectiveness in enhancing reranking techniques~\cite{li2018nprf,ai2018learning,yu2021pgt} and improving dense retrievers by learning better query representations~\cite{yu2021improving}. 


Recent research of query expansion focuses on Generative Relevance Feedback (GRF) methods, which utilize generation models to directly produce query expansion results~\cite{mackie2023generative, claveau2021neural, wang2023generative, jagerman2023query, mackie2023gprf}. These methods often employ LLMs to generate query-related documents~\cite{wang2023query2doc, jagerman2023query, gao2023precise}, leverage Chain-of-Thought (CoT) reasoning results~\cite{wei2022chain, jagerman2023query, trivedi2023interleaving}, or utilize specific keywords~\cite{li2024can, jagerman2023query} to expand queries, thereby enhancing the ranking capabilities of lexical matching based retrieval models~\cite{jagerman2023query, wang2023query2doc}, dense retrieval models~\cite{wang2023query2doc}, and reranking models~\cite{li2024can}.


Although these studies have demonstrated their advantages in improving retrieval models, directly using LLMs for query expansion and reformulation poses potential risks due to LLM hallucinations~\cite{shuster2021retrieval,huang2023survey}. To mitigate this issue, some works use different instructions to reduce inconsistency in query reformulation~\cite{gao2023precise} or leverage rephrased questions as demonstrations to guide LLMs in generating more effective query expansion results~\cite{koo2024optimizing}. RaFe~\cite{mao2024rafe} further enhances the Retrieval-Augmented Generation (RAG) performance by taking reranking scores as training signals to optimize the query rewriting model. In contrast to these approaches, LLM-QE focuses on modeling the ranking preferences of both retrievers and LLMs, rewarding LLMs for generating more effective expansion results, and exploring its potential to build both unsupervised and supervised dense retrieval models.




\section{Methodology}
\paragraph{Preliminaries.}
We primarily focus on the homologous model merging, in which $\boldsymbol{\theta}_i$ all come from the same base model $\boldsymbol{\theta}_{\rm{base}}$. Given $K$ tasks $\{T_1,T_2,\cdots,T_K\}$ and $K$ corresponding fine-tuned models with parameters $\{\boldsymbol{\theta}_1,\boldsymbol{\theta}_2,\cdots,\boldsymbol{\theta}_K\}$, model merging aims to combine $K$ fine-tuned models into one single model simultaneously performing on $\{T_1,T_2,\cdots,T_K\}$ without post-training~\cite{method_p1_1,method_p1_2}.
Task vector~\cite{ilharco2023editing,yang2024adamerging} is a key element in merging method which could enhances the base model‘s ability or enable the model to handle other tasks. Specifically, for task $T_i$, the task vector $\boldsymbol\tau_i\in \mathbb{R}^D$ is defined as the vector obtained by subtracting the SFT weights $\boldsymbol{\theta}_i$ from the base model weight
$\boldsymbol{\theta}_{\rm{base}}$, \emph{i.e.}, $\boldsymbol\tau_i=\boldsymbol{\theta}_i-\boldsymbol{\theta}_{\rm{base}}$. The merged model could be denoted as $\boldsymbol{\theta}_m=\boldsymbol{\theta}_{\rm{base}}+\sum_i \lambda_i\boldsymbol{\tau}_i$, which $\lambda_i$ is the scaling factor measuring the importance of task vector. For clarification, we also denote the neuron set in $\boldsymbol{\theta}_i$ as $\mathcal{N}_i$, the neuron set in $\boldsymbol{\tau}_i$ as $\mathcal{T}_i$.



\begin{algorithm}[!ht]
    \caption{LED-Merging}
    \label{alg1}
    \begin{algorithmic}[1]
        \REQUIRE  base model $\boldsymbol{\theta}_{\rm{base}}$, SFT models $\{\boldsymbol{\theta}_{i}\mid i\in [K]\}$, mask ratios \{$r_{i} \mid i\in [K]\}$, scaling factors $\{\lambda_i\mid i\in[K]\}$, location datasets $\{\mathcal{X}_{i}\mid i\in[K]\}$
        \ENSURE merged parameter $\boldsymbol{\theta}_{m}$
        \STATE $\mathcal{M}\leftarrow\phi$
        \STATE $\boldsymbol{\theta}_{m}\leftarrow \boldsymbol{\theta}_{\rm{base}}$
        \FOR{$i\in [K]$}
        \STATE $I(\boldsymbol{\theta}_i)=\mathbb{E}_{x\sim \mathcal{X}_i}|\boldsymbol{\theta}_{i}\odot \nabla_{\boldsymbol{\theta}_i}\mathcal{L}(x)|$
        \STATE $I(\boldsymbol{\theta}_{\rm{base}})=\mathbb{E}_{x\sim \mathcal{X}_i}|\boldsymbol{\theta}_{\rm{base}}\odot \nabla_{\boldsymbol{\theta}_{\rm{base}}}\mathcal{L}(x)|$
        
        \STATE calculate $\mathcal{T}^{r_i}_{i}$ following Equation \ref{vote}
        \STATE  $\mathcal{M}\leftarrow \mathcal{M}\cup\{\mathcal{T}^{r_i}_i\}$
       
        
   
        
        
        \ENDFOR  
        \FOR{$i\in [K]$}
        
        \STATE calculate $\text{Disjoint}(\mathcal{T}_i^{r_i})$ use Equation~\ref{disjoint_safety}
        \STATE $\boldsymbol{m}_i \leftarrow \boldsymbol{0}$
        \FOR{$d\in \mathcal{T}_i^{r_i}$}
        \STATE $\boldsymbol{m}_{i,d}=1$
        \ENDFOR
        \STATE $\boldsymbol{\theta}_{m}\leftarrow \boldsymbol{\theta}_{m}+\lambda_i \boldsymbol{\tau}_i\odot \boldsymbol{m}_{i}$
        \ENDFOR
    \end{algorithmic}
\end{algorithm}
    %\vspace{-5pt}
\begin{figure*}[h!]
    \centering
    \includegraphics[width=\linewidth]{figs/pipeline_v2.pdf}
    \vspace{-40mm}
    \caption{Overview of our two-stage training pipeline {\ours}.}
    \label{fig:pipeline}
\end{figure*}


\paragraph{LED-Merging: Location, Election, and Disjoint Merging}
To address the neuron misidentification and interference issues in existing model merging methods, we propose LED-Merging (Location, Election, and Disjoint Merging). Specifically, previous studies \cite{modelstock, ilharco2023editing, tiesmerging} fail to accurately identify safety-related neurons in task vectors with a single magnitude score, namely \textit{neuron misidentification}. Meanwhile, there exists an interference between safety-related and utility-related task vector neurons during the merging process, namely \textit{neuron interference}. To address neuron misidentification, we first locate important neurons both in the base and fine-tuned models and then elect neurons from the task vector considering these two scores together. Subsequently, to mitigate the interference, we introduce a disjoint step, isolating these important neurons so that they influence different base neurons. The whole process is illustrated in Figure~\ref{fig:method}. 




In the location and election step, we consider the importance score from base and fine-tuned models simultaneously to locate task-specific neurons. In this way, it is more accurate than relying on the magnitude score alone because task-specific neurons with high importance score in the fine-tuned model may not necessarily score high in the base model, and vice versa.

{\textbf{Location}}.  We first calculate importance scores for each neuron in a base/fine-tuned model. Given a location dataset $\mathcal{X}_i=\{(x,y)_k\}$, where $x$ is the question and $y$ is the answer, we calculate the importance scores for the weight $\boldsymbol{\theta}_i\in\mathbb{R}^D$ in any  layer as follows~\cite{snip,spareseGPT,sun2024a}:
\begin{equation}
    I(\boldsymbol{\theta}_i)=\mathbb{E}_{x\sim \mathcal{X}_i}[\boldsymbol{\theta}_i\odot \nabla _{\boldsymbol{\theta}_i}\mathcal{L}(x)],
    \label{location}
\end{equation}
which $\mathcal{L}(x)=-\log p(y\mid x)$ is the conditional negative log-likelihood loss. We choose the SNIP score~\cite{snip} because it balances computational efficiency and performance~\cite{cq}. Please refer to Sec.~\ref{sec:ablation} for the comparison between different location methods. After computing importance scores, we choose top-$r_i$ neurons as the important neuron subset $\mathcal{N}_{i}^{r_i}$ from $I(\boldsymbol{\theta}_i)$.
 
 % After computing locating scores, we select the neurons scoring both high in base and fine-tuned models as important neurons in task vectors. Then in the disjoint step,  with preventing  polysemantic neurons  from receiving gradient updates towards different directions,
 % we use set difference to isolate the safety   and utility-related neurons  and construct corresponding masks for merging process,

{\textbf{Election}}. A natural question is how to select important neurons in the task vector $\boldsymbol{\tau}_i$ based on $I(\boldsymbol{\theta}_{\rm{base}})$ and $I(\boldsymbol{\theta}_{i})$. The important neurons in the base model may be different from neurons in the fine-tuned model. Therefore, we introduce the following election strategy to select neurons with high scores in both base and fine-tuned models:
\begin{equation}
    \mathcal{T}_i^{r_i}=\mathcal{N}_i^{r_i}\cap \mathcal{N}_{\rm{base}}^{r_i}.
    \label{vote}
\end{equation}
\emph{Remark}. We compare different choosing methods, including scoring low or high in base or fine-tuned model in Section~\ref{sec:ablation} and find that Equation \ref{vote} achieves the best performance.





{\textbf{Disjoint}}. As important neurons from different task vectors may conflict with each other at the same position, we use the set difference to disjoint the neurons from others to prevent interference:
\begin{equation}
    \text{Disjoint}(\mathcal{T}^{r_i}_{i})=\mathcal{T}^{r_i}_{i}-\mathop{\cup}\limits_{{J}\subsetneqq [K],|J|\geq 2}\mathop{\cap}\limits_{j\in {J}}\mathcal{T}^{r_j}_{j}.
    \label{disjoint_safety}
\end{equation}

Next, we construct a mask $\boldsymbol{m}_i\in\mathbb{R}^D$ to implement disjoint in the merging process. Specifically, this mask $\boldsymbol{m}_i$ is used to select neurons from $\mathcal{T}_i$. The mask ratio is $r_i$, where $r\in(0,1]$. The mask $\boldsymbol{m}_i$ can be derived from:
\begin{equation}
    \boldsymbol{m}_{i,d}=\begin{aligned} &\left\{ \begin{array}{ll} 1, & \text{if } d\in \text{Disjoint}(\mathcal{T}_{i}^{r_i}), \\ 0, & \text{otherwise}. \end{array} \right. \end{aligned}
    \label{mask_safety}
\end{equation}


% \subsection{Merging Models with Masks}
{\textbf{Merging}}. The final
merged task vector $\boldsymbol{\tau}_m$ is as follows:
\begin{equation}
    \boldsymbol{\tau}_m= \sum_i \lambda_i\boldsymbol{\tau}_{i}\odot\boldsymbol{m}_i.
    \label{merged_task_vector}
\end{equation}
We summarize the workflow in Algorithm \ref{alg1}.



\section{Results}
\label{section:4}
Through our interviews and diary study data, we find that teens perceive casual, mundane, and frequent sharing of their daily lives as meaningful self-disclosure---helpful in building relationships but not too intimate or high-stakes. However, many hesitated to share due to concerns about their content being perceived as underwhelming or trivial. Our study also reveals that trust is crucial in how teens regulate boundaries around self-disclosure. When trust in a relationship feels uncertain, teens face a dilemma: self-disclosure is key to strengthening connections, yet within the social media environment, sharing with an ambiguous audience often feels more risky than rewarding. As a result, many teens default to self-censorship, limiting the potential for relationship building.

This dynamic suggests that the way social media privacy is portrayed may discourage trust-building rather than support it. We explore two overarching themes of barriers to meaningful self-disclosure: uncertainties in communication \ie{\textit{communication fog}} and a high-stakes environment that tends to skew these uncertainties towards negative interpretations \ie{\textit{low-grace culture}}. Additionally, we examine teens' design suggestions for addressing these barriers, providing insights into potential platform improvements. We denote quotes with (\entry{XX}) for entry interview data, (\diary{XX}) for diary entries, and (\codesign{XX}) for co-design interview data.

\subsection{Trust as Key for Boundary Regulation}
\label{section:4-1}
Through our interviews and diary studies, we found that trust functions as a key determinant in shaping teens' self-disclosure practices and their ability to regulate boundaries online---trusted friends or interactions that build trust facilitated more open and frequent sharing, while non-trusted individuals or \textit{ambiguous} connections led to hesitations.


\subsubsection{\textbf{Self-Disclosure as a Balancing Act Between Relational Benefits and Privacy Concerns}}
\label{section:4-1-1}
Participants recognized the relational benefits of disclosure. They desired to share mundane details or small updates from their lives on social media but hesitated due to concerns about how their content would be perceived. For example, one participant mentioned wanting to share small updates in their lives, such as \inlinequote{I made some microwave popcorn and at the end there were only 11 pop kernels at the bottom of the bag} (\entry{19}). One participant who initially had concerns about being perceived as oversharing, reflected: 

\blockquote{When\ldots{} I see something like this that maybe shares details that are a little more personal I definitely feel like I connect to it more, and that's when I kind of realize maybe like sharing a couple like personal details what might be considered oversharing to some people wasn't always a bad thing.}{{\codesign{08}}}

However, many teens refrained from posting such content due to an internalized expectation that self-disclosure must be curated. They worried that their posts were \inlinequote{not memorable and significant enough to be shared} (\diary{03}) or \inlinequote{not Instagram worthy} (\diary{08}), leading to a tendency toward self-censorship. Some participants expressed frustration at this pressure, wanting to \inlinequote{just post} (\codesign{09}) without feeling \inlinequote{judged} (\codesign{09}).


\subsubsection{\textbf{Trusted Friends as a Safe Space for Self-Disclosure}}
\label{section:4-1-2}
Teens shared that having a way to carve out their trusted circle of friends supported disclosure. As \diary{16} noted, \inlinequote{The limited nature of my friends on the app helped me feel comfortable, because I trust everyone on there.} Features such as Instagram's ``Close Friends'' reinforced this sense of security: \inlinequote{I shared these on my Close Friends story. These are people I am comfortable talking to at any time without it feeling weird or forced} (\diary{02}). Similarly, \diary{17} noted, \inlinequote{I always post a daily BeReal. I just love the app and how exclusive it is.} Moreover, a trusted circle alleviated concerns about judgment. \diary{16} elaborated, \inlinequote{These are people who either will not judge me or I simply do not care if they do.} \diary{19} echoed this sentiment that with close friends, they do not mind their posts being perceived as \inlinequote{not\ldots{} funny} or even \inlinequote{weird.}

\subsubsection{\textbf{Non-Trusted ``Friends'' as a Barrier to Self-Disclosure}}
\label{section:4-1-3}
The presence of ambiguous or distrusted connections, on the other hand, hindered self-disclosure. Teens frequently described situations where they wanted to share but refrained due to the potential presence of audiences they would not trust. On Instagram, for example, even with a private account, once someone is accepted as a ``follower,'' they gain immediate access to all shared content unless the user curates a ``Close Friends'' list. As \diary{16} explained, \inlinequote{Well I WANTED to share it with friends, but my followers on Instagram encompass more than that. I don't want to be judged.}

Platform norms further complicated boundary management. Many teens felt compelled to accept ``Follow'' requests, even from those who they did not fully trust, to avoid appearing rude. As \codesign{17} described, Instagram often resembled a \inlinequote{popularity contest} where the \inlinequote{follower to following ratio} (\codesign{15}) dictated social standing. This dynamic pressured users to \inlinequote{accept all the requests\ldots{} just to increase [their] follower count} (\codesign{05}) and made it harder to regulate self-disclosure. In contrast, platforms like BeReal, which de-emphasize follower metrics and provide clearer expectations for social boundaries, were seen as more conducive to trust-based boundary regulation: \inlinequote{The hidden follower count helps, so I don't feel pressured to get a lot of followers} (\diary{16}).

\subsubsection{\textbf{Evolving Boundaries and the Role of Trust}}
\label{section:4-1-4}
Our findings show that trust-building on social media mirrors that from prior studies: trust initiation begins with sharing, and repeated reciprocated interactions are essential for trust to develop and solidify. Participants highlighted that building trust with peers online starts with self-disclosure:
\blockquote{For me, building trust looks\ldots{} kind of like getting to post what you like, and maybe having a small two-minute conversation can help build a little more trust.}{\codesign{13}}
Similarly, \codesign{01} shared that trust is fostered through \inlinequote{messaging them and liking or commenting on each other's posts.} 

Participants also confirmed that favorable responses to self-disclosure are critical in trust-building. For example, \codesign{10} noted that trust comes mostly from \inlinequote{the way a person responds to a post that [they] send out.} They elaborated, \inlinequote{if I post a status update and the person maybe relates or comforts me or something, then that can obviously make me trust them.} Reciprocity was seen as key to building trust:
\blockquote{[Trust is a] mutual thing like if you reply more to me and I will talk to you more that trust kind of builds because we get to know each other more.}{\codesign{13}}
The role of effort was also emphasized, as comments that are \inlinequote{sweet, engaging, and creative,} for example, nurture trust and a sense of community (\codesign{06}, \codesign{16}). 

Conversely, a lack of reciprocity often leads to the erosion of trust. For example, \codesign{10} shared, \inlinequote{If they respond to it negatively, then my trust in them definitely decreases.} Non-responsiveness also damages trust, as \codesign{10} explained: if friends or followers \inlinequote{view it and not do anything} when they \inlinequote{explicitly ask for} something, it decreases trust. This lack of engagement, described as giving the \inlinequote{cold shoulder} (\entry{17}), discourages further self-disclosure (\codesign{01}, \entry{17}). Sometimes, trust erosion extends to groups, as \codesign{14} explained: if they \inlinequote{get like 300 views on the story but then only 150 likes}, it often signifies that \inlinequote{people saw it but didn't really do anything,} leading them to \inlinequote{take it the wrong way} and believe that people \inlinequote{don't like [them] anymore.}

\begin{table*}[!ht]
\centering
\small
\caption{Taxonomy of designs derived from the co-design study.}
\label{tab:designs}
\begin{tabular}{p{3.5cm} p{12cm} p{1.5cm}}
\toprule
\textbf{Overall Design Idea} & \textbf{Associated Barrier to Trust-Enabled Privacy} & \textbf{Example Prototype} \\ 
\midrule

\textbf{Guided Disclosure}\newline{}[Section \ref{section:4-2-1}] & Helps clarify \textbf{\textit{Ambiguous Norms}} by providing clear guidelines and expectations around posting, reducing uncertainty and the need for users to interpret implicit social expectations independently & Figure \ref{fig:prototypes}(a) \\

\textbf{Mutual Commitment}\newline{}[Section \ref{section:4-2-2}] & Addresses \textbf{\textit{Ambiguous Loyalty}} by allowing viewers to opt in or out of content while requiring them to agree to the space's rules before joining, ensuring mutual commitment and accountability for both the sharer and the audience. & Figure \ref{fig:prototypes}(g) \\

\textbf{Contextual Disclosure}\newline{}[Section \ref{section:4-2-3}] & Counteracts \textbf{\textit{Ambiguous Relevance}} by enabling segmentation of posts for specific interest groups, ensuring content reaches the right audience & Figure \ref{fig:prototypes}(c) \\

\textbf{Intentional Signaling}\newline{}[Section \ref{section:4-2-4}] & Clarifies \textbf{\textit{Ambiguous Reactions}} by encouraging more meaningful, context-specific reactions beyond simple `Likes' & Figure \ref{fig:prototypes}(f) \\

\textbf{Low-Stakes Disclosure}\newline{}[Section \ref{section:4-3-1}] & Alleviates \textbf{\textit{Presentation Expectations}} by offering casual, subtle, or ephemeral sharing options that lower the pressure to curate perfect content & Figure \ref{fig:prototypes}(b) \\

\textbf{Contextual Clarity}\newline{}[Section \ref{section:4-3-2}] & Reduces the \textbf{\textit{Social Risk of Misrepresentation}} common in CMC by allowing users to add contextual information, reducing likelihood of misunderstandings and misjudgments & Figure \ref{fig:prototypes}(e) \\

\textbf{Self-Contained Disclosure}\newline{}[Section \ref{section:4-3-3}] & Prevents \textbf{\textit{Nonconsensual Exposure}} by ensuring that every user has a dedicated space from the start, enabling them to share without worrying about burdening others by ``clogging'' their feeds & Figure \ref{fig:prototypes}(d) \\

\textbf{Trust-Centered Norms}\newline{}[Section \ref{section:4-3-4}] & Tackles the \textbf{\textit{Cycle of Distrust}} and cultivates positive community norms by setting clear behavioral guidelines, enabling users with aligned intentions to join and mutually enforce these norms & Figure \ref{fig:prototypes}(h) \\

\bottomrule
\end{tabular}
\end{table*}


\subsection{Communication Fog: Barrier to Boundary Regulation and Trust Calibration}
\label{section:4-2}
One key aspect that complicated trust-based boundary regulation was the uncertainty about how their posts are received on social media. Participants were unsure of platform norms, leaving them burdened with individual responsibility to navigate about what or how much to share. They also question whether their audience might judge them or not be interested in what they post. Additionally, interpreting reactions---especially low-effort, obligatory responses such as ``Likes''---is confusing, making it difficult to gauge genuine engagement or appreciation. These ambiguities create a ``communication fog'' that complicates trust calibration and boundary regulation, making users hesitant to share in ways that support relationship-building.


\subsubsection{Ambiguous Norms: Unclear Sharing Expectations and Individual Burden}
\label{section:4-2-1}
Participants highlighted the significant influence of peer norms on their self-disclosure behaviors, emphasizing their tendency to align their sharing with the perceived expectations within their social circles. Teens carefully calibrate their disclosures to neither exceed nor fall below the perceived norm, as \entry{02} noted: \inlinequote{Oftentimes, I kind of match what my friends post in terms of how public they are.} This balancing act reflects an ongoing negotiation of social expectations, with \codesign{07} stating: \inlinequote{So you don't want to overshare and you don't want to undershare.} 

However, many platforms lack clearly defined norms for what constitutes an appropriate level of sharing, leaving users to navigate ambiguous expectations independently. Participants interpret implicit norms:
\blockquote{[On Instagram,] Not famous people, in my experience, do not normally post random stuff like this, so it would be weird to make it my first post.}{\diary{16}}
Therefore, Instagram necessitates users to \inlinequote{make an active choice} (\codesign{05}) about when and what to share. Hence, self-initiated posting can sometimes be interpreted as \inlinequote{attention-seeking} (\diary{16}), and teens fear (potential) backlash. For instance, \codesign{12} had been asked \inlinequote{why do you post so much on Instagram?}, and \codesign{11} described having been ``unfollowed'' by a peer who deemed their content uninteresting: \inlinequote{He was like, oh you don't post anything interesting. All you do is post selfies.}

Participants expressed a desire for \textbf{guided disclosure} (Figure \ref{fig:prototypes}(f)), where platforms would offer clear, explicit cues that establish expectations for how much, how often, and when to share. By creating shared reference points for appropriate disclosure, these prompts would reduce the burden of individual decision-making and reinforce the reciprocity of self-disclosure, reassuring users that their vulnerability is not being interpreted as an isolated attempt for attention but rather as a socially expected practice. 

More specifically, participants envisioned being able to make \inlinequote{a quick update on [their] life, even if it's quite boring} (\diary{05}). Platforms like BeReal, which enforce a collective norm of casual sharing at a predefined time, were appreciated for \inlinequote{give[ing] everyone a chance to shout out} (\codesign{05}). Almost all participants advocated for system-generated \textit{prompts} or reminders that would encourage low-stakes sharing and provide implicit permission to disclose, fostering a \inlinequote{sense of community} (\codesign{12}, \codesign{16}) while reducing concerns about \inlinequote{sharing too much} (\codesign{19}). Such interventions would also alleviate uncertainty when users feel like they \inlinequote{can't really figure out what to post} (\codesign{07}) by \inlinequote{giv[ing] ideas about what to share} (\diary{08}), making self-disclosure a more collectively guided and less individually scrutinized process.



\subsubsection{Ambiguous Loyalty: When Followers Feel Like Skeptics}
\label{section:4-2-2}
Participants expressed that social media ``friends'' or ``followers'' often lack the established trust of real-world friendships. While these connections did not warrant removal---sometimes due to fears of social repercussions or backlash---their intentions and judgments remained uncertain. This ambiguity created a challenge in boundary regulation as users struggled to assess how their audience might interpret their posts. As \diary{08} explains: \inlinequote{I don't want some of my friends to think I'm oversharing or being `cringey.' There are always risks. You never can really know who's viewing your content or who's REALLY following you.}

To mitigate these uncertainties and foster a greater sense of trust-based boundary control, participants proposed supporting \textbf{mutual commitment} (Figure \ref{fig:prototypes}(g)) in content consumption. One suggested mechanism involved posters sending invitations, requiring viewers to actively opt-in (\codesign{16}) or out (\codesign{03}): \inlinequote{A pop-up of like `Oh you've been invited to join'\ldots with a yes or a no button,} and offering \inlinequote{the option to either join it or leave if you don't want to follow those rules} (\codesign{11}). This approach redistributes accountability---allowing viewers to regulate their access while relieving sharers of sole responsibility for audience reactions. A participant explained how mutually agreed sharing, such as that on Snapchat private Stories, provides reassurance: 
\blockquote{If they joined that story, [then] they're kind of the ones subjecting themselves to it. If they didn't want to see it, they'd never had to join it.}{\entry{12}}



\subsubsection{Ambiguous Relevance: Struggles to Identify the Right Audience}
\label{section:4-2-3}
Participants expressed a strong desire to curate their audience, even when viewers were not overtly toxic or judgmental. This need stems from their multifaceted identities and diverse interests, which they prefer to compartmentalize for different social groups. Participants shared various interests that they felt would only engage a subset of their audience, from poetry (\entry{11}) to K-Pop (\entry{03}). One participant explained the challenge of this:
\blockquote{sometimes I don't want to bore some of my friends with [a specific interest of theirs]\ldots{} I don't really feel the need to share it with them because as much as I would like them to share my interest in it, I don't want them to find that kind of burdensome.}{\codesign{14}} 

While many platforms employ algorithms to curate content based on inferred interests, participants found these coarse and ineffective. Rather, participants advocated for self-managed and dynamic calibration of social boundaries through \textbf{contextual disclosure} (Figure \ref{fig:prototypes}(b)), allowing them to strategically engage with groups based on the established and/or perceived potential for trust. Some users created elaborate systems of nested groups via Snapchat's private stories feature: 
\blockquote{I have four [private stories], each getting smaller to accommodate things I choose to share with\ldots{} The closer you are to me, the more (and smaller) private stories you're in.}{\diary{15}} 
\entry{16} reflected on the flexibility of Discord as providing similar benefits. 
These user-curated spaces would not just serve as secure places for existing trust but as controlled environments where users could gradually calibrate their social boundaries, engaging in disclosure that fosters emerging trust while minimizing the risks of sharing too broadly.

Participants were particularly keen on interest-based spaces, where shared interests would serve as a bridge to potential trust development. They noted that while common interests alone do not equate to trust, they provide an initial context that reduces the uncertainties of broader public sharing and, ultimately, allows trust to form over time. As \codesign{08} explained, such intentional segmentation \inlinequote{could definitely help people connect more because they'd feel like they had more to talk about.}


\subsubsection{Ambiguous Reactions: When ``Likes'' Fail to Signal Trust}
\label{section:4-2-4}
Participants expressed confusion about interpreting the meaning of reactions they receive to their posts, particularly when those reactions felt \inlinequote{obligatory} (\entry{19}). \entry{19} explained, \inlinequote{you feel more obligated to like a post\ldots [so] it just feels worse to post and not get as many likes.} Similarly, \codesign{08} remarked that when \inlinequote{people you're close with\ldots don't interact with your post,} they begin to question, \inlinequote{Is it really that boring that even people I know don't care?} While ``Likes'' may have been designed to signal validation, their meaning has become obscured in many platforms, making it an ineffective signal for reciprocity or trust.

To reduce ambiguity in social interactions, participants expressed interest in having a broader range of mechanisms for \textbf{intentional signaling} (Figure \ref{fig:prototypes}(d)) beyond effortless reactions such as ``Likes''. They cited examples from existing platforms where engagement from other users served as a clear, indisputable signal of care. For instance, the range and expressive nature of emoji reactions requires more deliberation than \inlinequote{just double tap[ing]}(\codesign{15}). When the chosen emoji aligns with the content, it sends clear signals that the actor actually viewed it. Comments were also valued as another form of clear signals of trust as they were seen as requiring individuals \inlinequote{go out of their way to be nice} (\codesign{08}). Participants also appreciated Instagram's Story Likes, describing them as \inlinequote{not obligatory at all} and as a feature that \inlinequote{shows that\ldots they decided to go out of their way to like it} (\entry{19}). To \codesign{13}, receiving Story Likes brings \inlinequote{a lot of trust[s]} given the private---and therefore more intimate, deliberate, and less likely performative---nature of those reactions. These reflections highlight that more nuanced, intentional reactions could serve as better social signals that reduce the ambiguity in boundary regulation.


\subsection{Low-Grace Culture: Barrier to Sustained Trust and Adaptive Boundary Setting}
\label{section:4-3}
In this section, we explore how social media cultivates a ``low-grace culture'' that discourages vulnerability or casual self-disclosure. This environment is characterized by an emphasis on polished self-presentation, judgments based on limited information, the risk of inadvertently burdening others by sharing, and an overall climate of distrust. These pressures inhibit the flexibility needed for adaptive boundary regulation, discouraging trust-building interactions.

\subsubsection{High-Stakes Presentation Expectations and Self-Censorship}
\label{section:4-3-1}
Participants frequently expressed fear about not measuring up to the presentation expectations of platforms such as Instagram. Especially on image-centric platforms, they felt compelled to share only highly polished moments. As \entry{18} stated, \inlinequote{Instagram is only for when the sun sets during the golden hour when I look my best.} This expectation discourages casual posting, as teens perceive that they \inlinequote{need to put significant effort into a post} (\entry{06}). Some worry that deviating from these norms might make them be judged as \inlinequote{underwhelming} or \inlinequote{monotonous} (\diary{11}), leaving them feeling vulnerable (\diary{13}).

To counteract these pressures, many participants expressed a preference for \textbf{low-stakes disclosure} (Figure \ref{fig:prototypes}(a)), which reduces the burden of excessive curation and supports more casual self-expression. They valued features that support ephemeral sharing, such as Instagram's Story or Snapchat posts, which feel \inlinequote{not as big of a deal} (\diary{16}) without \inlinequote{people saving or revisiting them} (\entry{06}). Participants often leveraged these features to share what they felt \inlinequote{doesn't really live up to the standards of what I post on Instagram} (\diary{07}). They also appreciated the support for casual, text-based updates on Bluesky and Twitter, or platforms that made interactions feel more like \inlinequote{having a conversation with someone} (\codesign{11}) rather than putting together a \inlinequote{portfolio} (\codesign{11}). Some participants suggested features like \inlinequote{a little status update} where people can \inlinequote{learn little facts about other people casually} (\codesign{05}).

These features help alleviate concerns about inadvertently exposing too much or misjudging boundaries by supporting and normalizing casual sharing rather than polished, high-effort posts. If casual sharing is the norm, then posting less curated content is less likely to be perceived as a disclosure of something deeply personal or intimate. Instead, it becomes part of a broader culture of mutual disclosure and relationship building, reducing the likelihood of boundary missteps and supporting better trust calibration among peers.



\subsubsection{Sparse Sharing and the Heightened Risk of Misrepresentation}
\label{section:4-3-2}
On platforms like Instagram, where participants perceived infrequent posting as the norm, limited content often becomes the primary basis for one's digital identity. This leaves users vulnerable to misrepresentation from acquaintances who may \inlinequote{make assumptions or misjudge [their] intentions} (\codesign{03}). With such sparse sharing, even a small number of posts can significantly influence how someone is perceived: \inlinequote{Somebody might pull up my Instagram and there are only, like, two posts and I look bad in both of them} (\codesign{12}). They also shared specific examples, such as posting extensively about a single topic, potentially coming across as being \inlinequote{obsessed} (\diary{12}) with that subject; sharing multiple photos from a memorable event might be interpreted as \inlinequote{bragging} (\entry{12}); or inquiring whether other students have completed their summer assignment could be misconstrued as attempting to \inlinequote{push} (\entry{12}) peers to do their work, despite that not being the intention of the post. Such concerns often led participants to lean toward self-censorship.

To mitigate the risk of trust erosion and promote more adaptive boundary regulation, participants suggested features to facilitate \textbf{contextual clarity} (Figure \ref{fig:prototypes} (d))---designs that provide additional context to reduce misunderstandings and support trust-building on platforms like Instagram. These included: indicating personal information such as interests on profiles (\diary{12}), emphasizing the importance of comprehensive information before making judgments (\codesign{03}), supporting explanatory captions for images beyond standard captions (\codesign{07}), and implementing a \inlinequote{social battery} indicator to help manage interaction expectations (\codesign{11}). By fostering mechanisms that encourage giving others the benefit of the doubt, such designs could prevent trust erosion and provide contexts for boundary regulation in online environments where full contextual understanding is not always possible.



\subsubsection{Nonconsensual Exposure and the Fear of Burdening Others}
\label{section:4-3-3}
On broadcast social media platforms, following a user implicitly consents to viewing their content, often leading to viewers feeling overwhelmed and \inlinequote{buried} (\codesign{12}) in posts. As \codesign{12} noted, \inlinequote{If I'm following 100 people and they're all sharing four times a day, then that's 400 things I have to click through.} Conversely, those that are sharing---aware of this dynamic from their own experiences---often fear \inlinequote{spamming} (\entry{03}, \codesign{13}) or \inlinequote{clogging} (\diary{16}) others' feeds. This fear of being perceived as burdensome or irritating was also evident in diary entries, where participants expressed hesitation \inlinequote{that people may think I'm oversharing or like posting too much} (\diary{11}), or regret after realizing, \inlinequote{[I] shared a TON of reels today on my close friends and public for some reason :crying-face:} (\diary{09}).

Participants sought ways to mitigate such unnecessary friction. Many desired \textbf{self-contained disclosure} (Figure \ref{fig:prototypes}(e)) that would allow them to share without imposing their content on others. These unobtrusive communication mechanisms included \inlinequote{a little status update} (\codesign{05}) or sharing minor personal interests on profile pages rather than in followers' feeds. One participant proposed a \inlinequote{red dot} (\codesign{17}) on user profiles to indicate new content subtly, allowing viewers to \inlinequote{not have to go look at it if they don't want to} (\codesign{17}). \codesign{05} saw parallels to Instagram's Story feature where \inlinequote{it doesn't notify people\ldots{} it's not like it goes on their feed\ldots{} it's just casual.} Participants perceived this as a way to foster connection without unnecessary friction and trust erosion by ensuring disclosure remains within a self-regulated and lower-stakes space.



\subsubsection{Absent Norms for Trust-Building and the Cycle of Distrust}
\label{section:4-3-4}
Participants observed social media spaces as often having a pervasive climate of distrust and the lack of measures to regulate such an environment. As \codesign{17} remarked, \inlinequote{On mainstream social media\ldots there's occasional positivity\ldots but mostly it feels draining}. 

They suggested that platforms could shift this culture by explicitly stating their platform expectations toward \textbf{trust-centered norms} (Figure \ref{fig:prototypes} (h)), such as explicitly stating that the platform is \inlinequote{a judgment-free zone} and communicating a clear message to users to \inlinequote{decrease judgment and be more kind} (\codesign{19}). They believed that such an approach would encourage users to \inlinequote{naturally be inclined to conform with the overall vibe and mission of the other users on the app} (\codesign{19}). Once a culture is established, a \inlinequote{selection bias} would occur, attracting like-minded individuals who align with these expectations (\codesign{19}). One participant expanded on this idea: 
\blockquote{I'm tired of how curated social media can be, but I still engage with it\ldots{} When the culture prioritizes authenticity and transparency, you'll get people like me, who are tired of curating, putting in the effort to be authentic and create this culture.}{\codesign{15}}
Similarly, another participant observed:
\blockquote{When a platform declares itself a judgment-free zone, it gives people the power to ensure both their own safety and the safety of others, creating a sense of control.}{\codesign{13}}
Participants emphasized that they believe that setting clear expectations, rules, and guidelines would foster meaningful changes toward safer environments. \codesign{17} noted that \inlinequote{teenagers my age listen when things are specifically told to them,} underscoring the importance of explicit communication in shaping platform norms.

Participants acknowledged that explicit guidelines could help mitigate negative interactions, but several also proposed additional features to actively \inlinequote{regulate toxic behaviors} as a necessary complement. They recognized that \inlinequote{there's always going to be someone that might try to ruin it} (\codesign{18}). To address this, \codesign{07} suggested a \inlinequote{negative comment filter} that users could toggle to block harmful content. Additionally, \codesign{19} proposed a reporting feature that would allow users to flag individuals \inlinequote{who may spread hurtful things or judge people.} While participants were mindful of the risk of implying the platform's distrust in its users, they acknowledged that tools for enforcing platform expectations could be valuable in strengthening their efforts to foster trust-enabling environments.
\section{Discussion}
\label{section:discussion}


\subsection{Practical Implications for Feedforward Prompting}

Of course, prompting an LLM continuously before the user submits their prompt is significantly most costly over submitting the prompt just once, once the user is ready.

% But user might not be ready, and the cognitive costs is pretty heavy.


\subsection{}


% Does this work well with Chain of Thought actually?
% Maybe this approach will actually incentivize self-prompt-chaining???
% What are the implications of this?


% A benefit of this is certainly more transparency in the LLM
% LLM is so flexible that adding this kind of structure is still okay for the LLM



% What's more costly, entering a prompt, then responding and saying, no i want this, or typing a prompt, and tuning the prompt/expected output to reduce message exchanges?

% Learning to become a better prompter. One is by trial and error experience. Perhaps another is through this feedforward that tells you what you might be able to anticipate.
\section{Concluding Remarks}
In this paper, we proposed a novel approach utilizing multimodal LLMs to generate gesture-aware speech recognition transcripts for patients with language disorders. Our framework integrates verbal speech and iconic gestures, enabling the generation of enriched transcripts that capture the latent meaning conveyed through both modalities. Through extensive experimentation, we demonstrated that the proposed method effectively contextualizes incomplete or disfluent speech by incorporating gesture information, leading to more accurate and meaningful representations of the speaker's intent. These findings highlight the potential of our approach to significantly contribute to the field of speech and language therapy, offering innovative tools that can enhance the quality of life for individuals with language disorders by facilitating better communication and assessment methods.

\subsection{Ethical Statement} 
Our dataset was obtained from AphasiaBank with the approval of the Institutional Review Board (IRB) and adheres to the data sharing guidelines set by TalkBank\footnote{https://talkbank.org/share/ethics.html}. This includes complying with the Ground Rules for all TalkBank databases, which are based on the American Psychological Association Code of Ethics~\cite{american2002ethical}.

\subsection{Limitation \& Future Work} 
%This study represents a preliminary investigation into using multimodal LLMs to generate gesture-aware speech recognition transcripts. 
While the results are promising, we recognize several limitations and outline our plans to extend this work further.

One primary limitation is the absence of a definitive ground truth for quantitative evaluation. Since our model generates transcripts by synthesizing speech and gesture data from scratch, traditional benchmarks, such as comparisons with standard speech recognition outputs, are insufficient. Moreover, existing original transcripts lack gesture annotations, making direct comparisons challenging. In future work, we aim to address this gap by collaborating with certified pathologists to conduct qualitative assessments, such as A-B preference tests, to evaluate the effectiveness of gesture-enriched transcripts in accurately conveying the speaker's intentions.

To support quantitative evaluations, we plan to develop novel metrics that assess transcript quality, including grammar accuracy, semantic consistency, and the integration of multimodal information. Such metrics will provide a more objective basis for assessing our model's performance and facilitate comparisons with other multimodal and unimodal approaches.

Another limitation of this study is its focus on structured gestures from a specific task, the Peanut Butter Sandwich Task. While this task offers a controlled context for testing our approach, it does not encompass the diversity of gestures and communication patterns seen in everyday scenarios. As part of our future work, we plan to expand the scope of our model to include tasks such as the Cinderella Story Recall Task~\cite{bird1996cinderella}, which involves unstructured and complex narrative gestures. This expansion will allow us to evaluate the adaptability and robustness of our model in handling varied linguistic and gestural contexts.

In summary, while this study establishes a strong foundation for gesture-aware speech recognition, we aim to refine and extend our methods through collaborative qualitative evaluations, the development of robust quantitative metrics, and broader task applications. These efforts will ensure that our approach continues to evolve, ultimately contributing to more effective communication tools and interventions for individuals with language disorders.





\section{Metric}
\label{sec:metric}

\textbf{Mean Squared Error (MSE)} Mean Squared Error (MSE) is a common statistical metric used to assess the difference between predicted and actual values. The formula is:
\begin{equation}
    MSE = \frac{1}{n} \sum_{i=1}^{n} (y_i - \hat{y}_i)^2
\end{equation}
where $ n $ is the number of samples, $ y_i $ is the actual value, and $ \hat{y}_i $ is the predicted value.

\textbf{Relative L2 Error} Relative L2 error measures the relative difference between predicted and actual values, commonly used in time series prediction. The formula is:
\begin{equation}
    \text{Relative L2 Error} = \frac{\| Y_{\text{pred}} - Y_{\text{true}} \|_2}{\| Y_{\text{true}} \|_2}
\end{equation}
where $ Y_{\text{pred}} $ is the predicted value and $ Y_{\text{true}} $ is the actual value.

\textbf{Structural Similarity Index Measure (SSIM)} The Structural Similarity Index (SSIM) measures the similarity between two images in terms of luminance, contrast, and structure. The formula is:
\begin{equation}
    SSIM(x, y) = \frac{(2\mu_x \mu_y + C_1)(2\sigma_{xy} + C_2)}{(\mu_x^2 + \mu_y^2 + C_1)(\sigma_x^2 + \sigma_y^2 + C_2)}
\end{equation}
where $ \mu_x $ and $ \mu_y $ are the mean values, $ \sigma_x $ and $ \sigma_y $ are the standard deviations, $ \sigma_{xy} $ is the covariance.

\section{Related Work}
\subsection{Deep Learning based Weather Forecasting}
\textbf{Global Weather Forecasting.} Global weather forecasting has seen significant progress with deep learning models. FourCastNet, based on Fourier neural operators, provides global forecasts comparable to traditional numerical methods like IFS, but at much higher speeds~\cite{pathak2022fourcastnet}. Pangu, utilizing the Swin Transformer, exceeds NWP methods, incorporating earth-specific location embeddings for better performance~\cite{bi2023accurate}. The Spherical Fourier Neural Operator (SFNO) extends Fourier methods using spherical harmonics, offering more stable long-term predictions~\cite{bonev2023spherical}. FuXi focuses on long-term forecasting, achieving a 15-day forecasts comparable to ECMWF~\cite{chen2023fuxi}. GraphCast leverages message-passing networks to improve efficiency and forecasting accuracy~\cite{lam2023learning}, and GenCast builds on this to enhance ensemble forecasting~\cite{price2023gencast}. Further, diffusion models like those in~\cite{li2024generative} generate probabilistic ensembles by sampling, while NeuralGCM~\cite{kochkov2024neural} focuses on atmospheric circulation with a dynamic core, offering climate simulation capabilities but at higher training and inference costs. 

\textbf{Regional Weather Forecasting.} The goal of regional weather forecasting is to enhance local prediction accuracy with high-resolution models. CorrDiff~\cite{mardani2023generative} combines U-Net and diffusion models to improve local forecasts. MetaWeather~\cite{kim2024metaweather} adapts global forecasts to regional contexts using meta-learning. GNNs are also widely applied in regional forecasting, with Graphcast~\cite{lam2023learning} enhancing accuracy by modeling complex spatial dependencies. MetNet-3~\cite{espeholt2022deep} offers high-accuracy forecasts for weather variables, such as precipitation, temperature, and wind speed, at 2-minute intervals and 1–4 km resolution, outperforming traditional models like HRRR. NowcastNet~\cite{zhang2023skilful} and DGMR~\cite{ravuri2021skilful} excel in short-term extreme precipitation forecasts using deep generative models and radar data. In spatiotemporal prediction, NMO~\cite{wu2024neural} models the evolution of physical dynamics, providing new insights for local weather forecasting. Similarly, SimVP~\cite{gao2022simvp} and PastNet~\cite{wu2024pastnet} achieve good results in forecasting local precipitation evolution using spatiotemporal convolution methods.
    
% Despite these advances, none of these methods effectively address the challenge of balancing global and regional high-resolution forecasts or capturing the fine-grained, dynamic interactions important for extreme event prediction.
    
\subsection{Numerical analysis methods}
Multigrid methods~\cite{mccormick1987multigrid,wesseling1995introduction,hackbusch2013multi,bramble2019multigrid,hiptmair1998multigrid,brandt1983multigrid,borzi2009multigrid} and nested grid strategies~\cite{miyakoda1977one,zhang2012nested,sullivan1996grid} are widely used to solve PDEs and handle multi-scale problems~\cite{debreu2008two,xue2000advanced}. Multigrid methods use grids of different resolutions to transfer information and accelerate iterations. They efficiently solve large-scale problems and improve computational accuracy. By eliminating low-frequency errors on coarse grids and high-frequency errors on fine grids, multigrid methods effectively handle error convergence at different scales~\cite{he2019mgnet,he2023mgno,shao2022fast}. Nested grid strategies embed higher-resolution fine grids into regions of interest based on a global coarse grid to capture local complex physical phenomena in detail. In weather forecasting, this method provides large-scale background fields on a global scale while refining the grid for target regions to accurately simulate the evolution of local weather systems and the occurrence of extreme events~\cite{bacon2000dynamically}. 

% Our proposed neural nested grid method helps address challenges like boundary information loss in regional forecasting and multi-scale feature capture.

\section{Additional Results}
%
We present more additional results in Figure \ref{fig_0.25-day}, \ref{fig_0.5-day}, \ref{fig_1.0-day} \ref{fig_1.5-day}, \ref{fig_2.0-day}, \ref{fig_2.5-day}, \ref{fig_3.0-day}, \ref{fig_3.5-day}, \ref{fig_4.0-day}, \ref{fig_4.5-day}, \ref{fig_5.0-day}, \ref{fig_5.5-day}, \ref{fig_6.0-day}, \ref{fig_6.5-day}, \ref{fig_7.0-day}, \ref{fig_7.5-day},
\ref{fig_8.0-day}, \ref{fig_8.5-day}, \ref{fig_9.0-day}, \ref{fig_9.5-day},
\ref{fig_10.0-day}, including 18 variables that are importmant to weather forecasting, each with results ranging from 6 hours to 10 days. These additional results further demonstrate the effectiveness of OneForecast. Same as the Figure \ref{fig:visual_results}
, the initial conditions is 00:00 UTC, 1 January 2020.


\begin{figure*}[h]
\centering
\includegraphics[width=1\linewidth]{figures/fig_0.25-day.jpg}
\vspace{-20pt}
\caption{6-hour forecast results of different models.}
\label{fig_0.25-day}
\end{figure*}

\begin{figure*}[h]
\centering
\includegraphics[width=1\linewidth]{figures/fig_0.5-day.jpg}
\vspace{-20pt}
\caption{0.5-day forecast results of different models.}
\label{fig_0.5-day}
\end{figure*}

\begin{figure*}[h]
\centering
\includegraphics[width=1\linewidth]{figures/fig_1.0-day.jpg}
\vspace{-20pt}
\caption{1-day forecast results of different models.}
\label{fig_1.0-day}
\end{figure*}

\begin{figure*}[h]
\centering
\includegraphics[width=1\linewidth]{figures/fig_1.5-day.jpg}
\vspace{-20pt}
\caption{1.5-day forecast results of different models.}
\label{fig_1.5-day}
\end{figure*}

\begin{figure*}[h]
\centering
\includegraphics[width=1\linewidth]{figures/fig_2.0-day.jpg}
\vspace{-20pt}
\caption{2-day forecast results of different models.}
\label{fig_2.0-day}
\end{figure*}


\begin{figure*}[h]
\centering
\includegraphics[width=1\linewidth]{figures/fig_2.5-day.jpg}
\vspace{-20pt}
\caption{2.5-day forecast results of different models.}
\label{fig_2.5-day}
\end{figure*}

\begin{figure*}[h]
\centering
\includegraphics[width=1\linewidth]{figures/fig_3.0-day.jpg}
\vspace{-20pt}
\caption{3-day forecast results of different models.}
\label{fig_3.0-day}
\end{figure*}

\begin{figure*}[h]
\centering
\includegraphics[width=1\linewidth]{figures/fig_3.5-day.jpg}
\vspace{-20pt}
\caption{3.5-day forecast results of different models.}
\label{fig_3.5-day}
\end{figure*}

\begin{figure*}[h]
\centering
\includegraphics[width=1\linewidth]{figures/fig_4.0-day.jpg}
\vspace{-20pt}
\caption{4-day forecast results of different models.}
\label{fig_4.0-day}
\end{figure*}

\begin{figure*}[h]
\centering
\includegraphics[width=1\linewidth]{figures/fig_4.5-day.jpg}
\vspace{-20pt}
\caption{4.5-day forecast results of different models.}
\label{fig_4.5-day}
\end{figure*}


\begin{figure*}[h]
\centering
\includegraphics[width=1\linewidth]{figures/fig_5.0-day.jpg}
\vspace{-20pt}
\caption{5.0-day forecast results of different models.}
\label{fig_5.0-day}
\end{figure*}

\begin{figure*}[h]
\centering
\includegraphics[width=1\linewidth]{figures/fig_5.5-day.jpg}
\vspace{-20pt}
\caption{5.5-day forecast results of different models.}
\label{fig_5.5-day}
\end{figure*}

\begin{figure*}[h]
\centering
\includegraphics[width=1\linewidth]{figures/fig_6.0-day.jpg}
\vspace{-20pt}
\caption{6.0-day forecast results of different models.}
\label{fig_6.0-day}
\end{figure*}

\begin{figure*}[h]
\centering
\includegraphics[width=1\linewidth]{figures/fig_6.5-day.jpg}
\vspace{-20pt}
\caption{6.5-day forecast results of different models.}
\label{fig_6.5-day}
\end{figure*}

\begin{figure*}[h]
\centering
\includegraphics[width=1\linewidth]{figures/fig_7.0-day.jpg}
\vspace{-20pt}
\caption{7.0-day forecast results of different models.}
\label{fig_7.0-day}
\end{figure*}

\begin{figure*}[h]
\centering
\includegraphics[width=1\linewidth]{figures/fig_7.5-day.jpg}
\vspace{-20pt}
\caption{7.5-day forecast results of different models.}
\label{fig_7.5-day}
\end{figure*}

\begin{figure*}[h]
\centering
\includegraphics[width=1\linewidth]{figures/fig_8.0-day.jpg}
\vspace{-20pt}
\caption{8.0-day forecast results of different models.}
\label{fig_8.0-day}
\end{figure*}

\begin{figure*}[h]
\centering
\includegraphics[width=1\linewidth]{figures/fig_8.5-day.jpg}
\vspace{-20pt}
\caption{8.5-day forecast results of different models.}
\label{fig_8.5-day}
\end{figure*}

\begin{figure*}[h]
\centering
\includegraphics[width=1\linewidth]{figures/fig_9.0-day.jpg}
\vspace{-20pt}
\caption{9.0-day forecast results of different models.}
\label{fig_9.0-day}
\end{figure*}

\begin{figure*}[h]
\centering
\includegraphics[width=1\linewidth]{figures/fig_9.5-day.jpg}
\vspace{-20pt}
\caption{9.5-day forecast results of different models.}
\label{fig_9.5-day}
\end{figure*}

\begin{figure*}[h]
\centering
\includegraphics[width=1\linewidth]{figures/fig_10.0-day.jpg}
\vspace{-20pt}
\caption{10.0-day forecast results of different models.}
\label{fig_10.0-day}
\end{figure*}


\section{Detailed Mathematical Proof}
\label{sec:proof}
\textbf{Proof of Theorem 1}

Now we have N augmented data and we need to select the best from them. We consider both the quality and the diversity of these data and get the sampling strategy from an optimization problem.

We model the sampling strategy as a multinomial distribution supported on all the augmented data $S = \{\mathbf{X}_j\}_{j=1}^N$, which means that the sampling strategy $\pi=(\pi_1,...,\pi_N)^\top$ is the corresponding probabilities of selecting $\mathbf{X}_1,...,\mathbf{X}_N$, then we can model the expectation of the similarity as:
$$\begin{aligned}
 & \mathbb{E}_{Y_x,Y_{x^{\prime}}\in\mathcal{C}}\{g(x,x^{\prime})\mid S\} \\
 & =\quad\int g(\mathbf{x},\mathbf{x}^{\prime})\boldsymbol{\pi}(\mathbf{x})\mathrm{Pr}_{S}(Y_{x}\in\mathcal{C}\mid\boldsymbol{x}=\mathbf{x})\boldsymbol{\pi}(\mathbf{x}^{\prime})\mathrm{Pr}_{S}(Y_{x}\in\mathcal{C}\mid\boldsymbol{x}=\mathbf{x}^{\prime})d\mathbf{x}d\mathbf{x}^{\prime} \\
 & =\quad\sum_{i,j=1}^Ng(\mathbf{X}_i,\mathbf{X}_j)\pi_i\pi_j\mathrm{Pr}_{S}(Y_x\in\mathcal{C}\mid\boldsymbol{x}=\mathbf{X}_i)\mathrm{Pr}_{S}(Y_x\in\mathcal{C}\mid\boldsymbol{x}=\mathbf{X}_j),
\end{aligned}$$
where the set $\mathcal{C}$ denotes the criterion of selection we are using, the function $g$ can be chosen as any similarity metric function and $x$ means a random variable.

The core to solving the above optimization problem is to use predictive inference to approximate the conditional probability of $\{Y_x\in\mathcal{C}\}$ given $x = \mathbf{X}$
Let $\mu ( \mathbf{x} ) : = \mathbb{E} ( Y\mid \mathbf{X} = \mathbf{x} )$ be the oracle associated with $( \mathbf{X} , Y) .$ Denote $\theta_j=\mathbb{I}\{Y_j\in\mathcal{C}\}$. As the augmented data
$\mathbf{X}_1,...,\mathbf{X}_N$ are independently identically distributed, $\theta_1,...,\theta_N$ can be regarded as independent Bernoulli($q)$ variables with $q=\Pr(Y_j\in\mathcal{C}).$ The probability distribution of the predicted result $W_j$ for $j=1,...,N$ is
$$\Pr(W_j\mid\theta_j)=(1-\theta_j)f_0+\theta_jf_1,\quad$$
where $f_0$ and $f_1$ are the conditional distributions of $W_j$ on $Y_j \in \mathcal{C}$ or not.

Denote $T(w) = \frac{(1-q)f_0(W_j)}{f(W_j)}$, we can rewrite the expectation of the similarity as
$$\mathbb{E}_{Y_x,Y_{x^{\prime}}\in\mathcal{C}}\{g(x,x^{\prime})|S\}=\sum_{i,j=1}^Ng(\mathbf{X}_i,\mathbf{X}_j)\pi_i\pi_j(1-T_i)(1-T_j)=\boldsymbol{\pi}^\top A_\mathbb{T}\boldsymbol{\pi},$$

Next, we use the expectation to control the quality of the data.
$$\mathbb{E}\{\mathbb{I}(Y_x\not\in\mathcal{C})\mid S\}=\sum_{i=1}^N\Pr(Y_i\not\in\mathcal{C}\mid\mathbf{X}_i)\pi_i=\sum_{i=1}^N\pi_iT_i\leq\alpha,$$

In all, the optimization problem can be modeled as 
\begin{align}
    & \arg\min_{\boldsymbol{\pi}}\quad h(\boldsymbol{\pi},\mathbb{T}):=\boldsymbol{\pi}^\top A_\mathbb{T}\boldsymbol{\pi}, \\
    & \text{subject to} \quad
        \begin{cases}
            \sum_{i = 1}^N\pi_iT_i\leq\alpha, \\
            \sum_{i = 1}^N\pi_i = 1, \\
            0\leq\pi_i\leq m^{-1}, \quad 1\leq i\leq N.
        \end{cases}
\end{align}

where $m$ is used to control the maximum selection.

The best selection of K is determined by the strategy $\pi$ which serves as the solution to the above optimization problem.

\section{Additional Experiments}
\label{sec:more_experiments}
\subsection{Long-term forecasting experiment expansion}

In the long-term forecasting experiments, we compare the performance of different backbone models on the SWE benchmark, evaluating the relative L2 error for three variables (U, V, and H). Our setup inputs 5 frames and predicts 50 frames. For the SimVP-v2 model, using \method{} reduces the relative L2 error for SWE (u) from 0.0187 to 0.0154, SWE (v) from 0.0387 to 0.0342, and SWE (h) from 0.0443 to 0.0397. We visualize SWE (h) in 3D as shown in Figure~\ref{fig:case} [\textcolor{red}{I}]. For the ConvLSTM model, applying \method{} reduces the relative L2 error for SWE (u) from 0.0487 to 0.0321, SWE (v) from 0.0673 to 0.0351, and SWE (h) from 0.0762 to 0.0432. For the FNO model, using \method{} reduces the relative L2 error for SWE (u) from 0.0571 to 0.0502, SWE (v) from 0.0832 to 0.0653, and SWE (h) from 0.0981 to 0.0911. Overall, \method{} significantly improves the long-term forecasting accuracy of different backbone models.

\begin{figure*}[h]
    \centering
    \includegraphics[width=\textwidth]{image/casestudy.pdf}
    \caption{
    \textcolor{red}{I.} 3D visualization of the SWE(h), showing Ground-truth, SimVP-V2+BeamVQ predictions, and Error at T=1, 10, 20, 30, 40, 50. The first row shows Ground-truth, the second SimVP-V2+BeamVQ predictions, and the third Error. \textcolor{red}{II.} A case study. Building fire simulation with ventilation settings added to Wu's Prometheus~\cite{wu2024prometheus}. (a) Layout and HRR growth. (b) Comparison of physical metrics for different methods. (c) Ground-truth, ResNet+BeamVQ, and ResNet predictions.
    }
    \label{fig:case} 
\end{figure*}


\subsection{Experiment Statistical Significance}
\label{sec:significance}
To measure the statistical significance of our main experiment results, we choose three backbones to train on two datasets to run 5 times. 
Table~\ref{tab:significance} records the average and standard deviation of the test MSE loss.
The results prove that our method is statistically significant to outperform the baselines
because our confidence interval is always upper than the confidence interval of the baselines. 
Due to limited computation resources, we do not cover all ten backbones and five datasets, 
but we believe these results have shown that our method has consistent advantages.


\begin{table}[h]
\label{tab:significance}
\centering
\begin{scriptsize}
    \begin{sc}
    \caption{ The average and standard deviation of MSE in 5 runs}
    \label{tab:significance}
    \centering
        \renewcommand{\multirowsetup}{\centering}
        \setlength{\tabcolsep}{10pt}
        \begin{tabular}{l|cc|cc}
            \toprule
            
            \multirow{4}{*}{Model} & \multicolumn{4}{c}{Benchmarks}  \\
            \cmidrule(lr){2-5}
            & \multicolumn{2}{c}{NSE} &   \multicolumn{2}{c}{SEVIR}   \\
            \cmidrule(lr){2-5}
           & Ori & + BeamVQ & Ori & + BeamVQ  \\
            \midrule
            ConvLSTM &0.4092$\pm$0.0002 &\textbf{0.1277$\pm$0.0001}  & 0.1762 0.0007  & \textbf{0.1279$\pm$0.0009}  \\
            FNO &  0.2227$\pm$0.0003 &\textbf{0.1007 $\pm$0.0002}& 0.0787$\pm$0.0012 & \textbf{ 0.0437$\pm$0.0013} \\
            CNO & 0.2192 $\pm$0.0008 &\textbf{ 0.1492$\pm$0.0011}& 0.0057$\pm$0.0005 & \textbf{ 0.0053$\pm$0.0006} \\
            \bottomrule
        \end{tabular}
    \end{sc}

\end{scriptsize}
\end{table}


\section*{Acknowledgements}
The authors would like to acknowledge the CERES network, the University of Washington Global Innovation Funds (GIF), and the University of Washington Student Technology Funds (STF), which provided support for this work. We additionally thank the anonymous reviewers for their detailed feedback and the participants for sharing their thoughts. We truly appreciate all the help. Alexis Hiniker is a special government employee for the Federal Trade Commission. The content expressed in this manuscript does not reflect the views of the Commission or any of the Commissioners.

%-------------------------------------------------------------------------------
\bibliographystyle{plain}
\bibliography{references}

%%%%%%%%%%%%%%%%%%%%%%%%%%%%%%%%%%%%%%%%%%%%%%%%%%%%%%%%%%%%%%%%%%%%%%%%%%%%%%%%
\end{document}
%%%%%%%%%%%%%%%%%%%%%%%%%%%%%%%%%%%%%%%%%%%%%%%%%%%%%%%%%%%%%%%%%%%%%%%%%%%%%%%%

%%  LocalWords:  endnotes includegraphics fread ptr nobj noindent
%%  LocalWords:  pdflatex acks
