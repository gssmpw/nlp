\section{Introduction: Unsettling the Hegemony of Intention}
\label{intro}

% \assign[Andre]{write draft}

% \andre{to write together at monday 1/13 meeting}

\textit{What does ``a Founding Father'' look like? What does ``Jesus'' look like? What about ``a migrant,'' ``a doctor,'' ``World War II''?}
The production and consumption of visual meanings is a \textit{sociopolitical} endeavor.
When people consider their mental image of ``a Founding Father,'' their preconceptions are in conversation with others' interpretations about \textit{who} a Founding Father is and --- just as importantly --- is not.
Today, our collective ``answers'' to these questions increasingly draw from images that are generated by biased AI models and then proliferate in our digital media ecosystem~\cite{jingnan2024ai, diresta2024spammers, herrieetal2024democratization}. The political context of image generation was further underscored when in March 2024 Google's Gemini model came under criticism for generating images with inaccurate racial and gender diversity --- for example, depicting Nazi soldiers as people of color. 
Such examples show how the values embedded in image generation tools shape how we view and understand our social and political world.
% For this reason, image generation might be understood as a \textit{political act} embedded in a constant negotiation of 
% collective meanings.
% within a public realm of ``visual citizens,'' demanding political structures for 
% in which the meaning of images is constantly being negotiated between ``visual citizens.''
% rewrote to put an emphasis on political structures.
Despite the increasing politicization of image generation tools, the dominant paradigm behind most current AI image generation tools is the \textbf{actualization of intention}: they should enable anyone to \textit{see exactly what they were intending}.
This ``intention is all you need'' paradigm~\cite{sarkar2024intention} neglects the political dimension of image generation. 
By inhibiting engagement with competing interpretations of user prompts, intention-centric paradigms run the risk of entrenching problematic user intentions \cite{cai2024antagonisticai}.

Certainly, the Gemini controversy was as a meaningful attempt to grapple with the political dimensions of image generation by implementing corrections to address a lack of diversity in image generation models. 
Yet, while Gemini's diversity intervention may have challenged the ```hegemony of intention'', it did so in a way that caused significant public backlash, inciting ``\textit{an explosion of negative commentary from figures such as Elon Musk who saw it as another front in the culture wars}''~\cite{milmoandkern2024gemini}. 
The swift pushback indicates that top-down interventions are ill-suited to the dynamic political landscape that image generation tools find themselves situated in.
In short, the problem is that shallow efforts that do not involve users in discourse but simply alter their intent can come across to users as patronizing. 

% These are all questions that we have some kind of visual answer to, and that we may be invested in as members of ethnicities, communities, nations, and other social groups with shared histories, values, and knowledge.
% As members of various social groups with shared histories, values, and knowledge, the images produced 
% As people increasingly populate the digital media ecosystem with AI-generated images, they contribute towards collective ``answers'' to these questions. In this way, the values embedded in the design of image generation tools play a role in how we view and understand our social and political world.
% This makes it important to interrogate \textit{what values are at play in their design}.
% As millions of AI image generation users produce billions of images with AI image generation tools populate the digital media ecosystem, AI-generated images will play an important role in political processes (e.g., elections) and shape how people understand the world.
% As these tools become more powerful, it is important to interrogate \textit{what values are at play in their design}.
% A dominant value in the literature on image generation is the \textbf{actualization of intention}: AI models should enable anyone to ``see what they were thinking'' (\S\ref{current-work}).
% However, this ``intention is all you need'' paradigm \cite{sarkar2024intention} entrenches users within their isolated spheres rather than exposing them to other competing interpretations, as the production and consumption of image meanings is a social endeavor.
% risks dogmatically entrenching existing user intentions.
% at the cost of obscuring the social 
% restricts image production to an isolated sphere of what users already think, when
% Vast scholarship from the social sciences argue that (\S\ref{discursive-approaches}) the production and consumption of image meanings is a social endeavor with important roles in the social world.
% This means that image generation might instead be understood as a \textit{political act}, occurring within a political sphere in which the meaning of images --- including their ethical, social, political, etc. dimensions --- must be negotiated between people.
% When users generate an image of ``a Founding Father,'' ``Jesus,'' ``a migrant,'' etc., their preexisting intentions are always in conversation with others' interpretations in the broader discourses about what images mean and how they are used.

In this work, we aim to build and evaluate an image generation interface that acknowledges the sensitive political contexts of image generation, including questions about diversity and representation, but unlike the Gemini case, encourages users to engage and reflect on their own intentions in relation to others on their own terms.
Our design draws from \textbf{agonistic pluralism} (from the Greek \textit{agon} meaning ``struggle''), a political philosophy emphasizing the importance of conflict in the political realm and the productive ways it can be channeled instead of avoided.
Given a user prompt, our image generation interface, \agonistic, collects and presents a range of interpretations of the prompt, prioritizing more controversial or challenging interpretations (Figure~\ref{fig:interface-screenshots}).
Interpretations come with a justification grounded in concepts extracted from Wikipedia.
Only after examining different interpretations and selecting one does a user then receive image generations.

A critical part of agonistic pluralism is \textit{reflection} --- the process of considering other viewpoints and challenging dominant assumptions. 
For this reason, we conduct a comparative lab evaluation of our interface for its ability to foster productive critical reflection.
We build three other paradigmatic interfaces for comparison --- 
\baseline: produces images generated from the user's prompt with no further interaction;
\diverse: rewrites the user's prompt to be more diverse before image generation;
and \reformulative: provides prompt reformulations to produce more aesthetic and preferable images, inspired by existing work in prompt reformulation.
Twenty-nine participants used these interfaces to iteratively build and refine a collage of 10 images representing a human subject of their choice while thinking aloud. 
% This task required participants to make decisions on visual representation.
% We collect self-reported quantitative measurements, participant-produced artifacts (collages, written statements), and qualitative responses to understand how different aspects of these interfaces contribute to or detract from reflection.

Across several measures of reflection, we find that \agonistic~induces the highest reflection out of any interface (\S\ref{how-reflection}), followed by \reformulative, then \diverse.
For instance, when participants went to add images to a collage, they described the image as an expansion of their original intention 21\% of the time while using \agonistic, but only 8\%, 7\%, and 3\% of the time for \reformulative, \diverse, and \baseline, respectively.
% For example, only $62\%$ of all images added with \agonistic~were directly intended by the user and $21\%$ were added after users' intents were expanded, whereas $97\%$ of images added under \baseline~were directly intended by the user.
% and $5\%$ after users' intents were significantly challenged. In contrast, for the \baseline, $97\%$ of images were directly intended by the user and only $3\%$ added due to expanded intent (\S\ref{adding-images}).
In addition, we find that participants express diversity as a justification for adding images to their collage
 $48\%$ of the time when using \agonistic, but only $28\%$ under \diverse.
Our findings confirm that participants are likely to reject Gemini-style interventions due to a lack of factuality and authenticity, while \agonistic~more authentically expresses diversity by situating user prompts and the generated images in relevant sociopolitical discourse.
We find that getting users to acknowledge and prioritize diversity requires situating diversity in authentic political controversies, a different finding from current interventions that either seek to impose diversity for its own sake or constrain diversity with factuality.
\agonistic~aids critical reflection by giving users a sense of control and provide contextualizing elements to help them interpret images.
% $48\%$ of images added under \agonistic~
% participants invoked diversity as a value for adding $48\%$ of images with \agonistic, compared to only $28\%$ with \diverse.
These results point to richer ways that researchers and developers of AI tools can address ethical questions in generative AI. 
Instead of making top-down interventions, our work shows that AI tools can instead create space for open-ended contestation over values like user intention, factuality, and diversity. 
Ultimately, our findings highlight the importance of bringing HCI work to bear on AI ethics and provide more general design elements for building reflective AI systems.
% \andre{make the ending more punchy}


% \andre{add RQs in? copied from 4:}
% To evaluate different interfaces with respect to user reflection, we investigate
% \textbf{(RQ1.)} \textit{How} each interface induces reflection using a variety of measures for user reflection (\S\ref{how-reflection}); and
% \textbf{(RQ2.)} \textit{Why} each interface induces reflection by examining how other (non-reflection) variables could explain the reflection observations in RQ1 (\S\ref{why-reflection}).

% In contrast, we find that our proposed interface --- motivated by agonistic pluralism --- may do a better job at expanding user assumptions about diversity, with 
% We find that \agonistic~generally encourages the most reflection out of all four interfaces, followed by \reformulative, and lastly \baseline~and \diverse~(\S\ref{how-reflection}) --- as measured by participants' self-reported change in mental image after using the interface (\S\ref{change-mental-image}), change in intent when adding images (\S\ref{adding-images}), and change in written collage summaries (\S\ref{design-statement}).
% Notably, only $62\%$ of all images added with \agonistic~were directly intended by the user, whereas $21\%$ were added after users' intents were expanded and $5\%$ after users' intents were significantly challenged. In contrast, for the \baseline, $97\%$ of images were directly intended by the user and only $3\%$ added due to expanded intent (\S\ref{adding-images}).
% We further investigate why different interfaces achieve different levels of reflection.
% We find that users' perception of control and appropriateness of interface content is a prerequisite for reflection (\S\ref{interface-properties}).
% Qualitatively, we observe that interface-produced contextualizing text helps users interpret and decide on images. We also find that reflection can be stimulated by a value we call ``authentic diversity,'' or a form of diversity that is not only grounded in factuality but also is communicated in ways that increase perceived authenticity (\S\ref{qualitative-themes}).

% \andre{can but by 1/2; group stats together and make main points, reference appropriate results section in other places}
% \andre{goal: cut intro to end at page 2 so the interfaces figure can be on page 3 and then background starts right after}

% \andre{Can have a more forceful opening, e.g.:}
% What does "Jesus" look like? What does "a Democrat" look like? What does a "mass shooter" look like? These are all questions that we are investigating as political citizens, as historically invested people, as ... but millions of people are using AI to contribute towards the digital ecosystem of visual representations of these issues.

% \begin{enumerate}
%     \item Image generation interfaces are used by many people to produce lots of AI images, they're becoming part of the digital media ecosystem
%     \item Increasingly more powerful models are being produced. There is a question of \textit{what values are at play when we design these systems}
%     \item Techniques to improve the user experience for these models are developed, like prompt reformulation. Interfaces are also being developed in HCI work for image generation. Common to all of these approaches are that they are motivated by this value that we must help users actualize their intention -- what they want, they shall see. AI image generation lets anyone be a creator, an artist, a designer, an influencer...
%     \item But we quickly realize that there are additional things that we care about in image generation, such as diversity. We want image generation models to feature people, for example, of diverse demographics.
%     \item However, perhaps this can be ``taken too far''. The Gemini scandal, hereafter referred to as the ``Gemini case'', appears to be a conflict between the value of diversity and the value of actualizing intentions.
%     \item The response to this incident seems to have been to put intention back on the throne, and find a way to insert diversity ``when appropriate'', to subordinate diversity still under intention. One way to appeal to this is via the value of factuality, that is by suggesting that diversity should only be instituted  ``when appropriate''. Therefore, we end up at a piecewise picture of the values of these interfaces is more or less ``let's try to portray things as how they actually are, generally according to what is reasonable or what accords to the user's intention"
%     \item What is complicating this narrative is that images aren't blankly factual or not, and that when users create images they are entering into a broader community of discourse and relations about visual representation of a topic. For example, consider the historical depiction of Jesus. Even though most individuals in the West implicitly intend an image of Jesus with White features, he was ``actually'' Jewish with more middle eastern figures. These are part of the visual representation of Jesus. It's not just about showing Jesus in different skin tones ``just for the sake of diversity'', it's because there are contested meanings over the visual representation of this thing. It hasn't been settled yet, and by engaging in the act of image generation, users should participate in this discourse.
%     \item In this paper, we explore how to design image generation interfaces that can be a site of this kind of reflection, which challenge and thwart intention, which can help users realize when their intentions may be incomplete or recontextualize them in new lights, to contextualize their image in the broader discourse they are entering by doing image generation.
%     \item We focus specifically on image generation of people. We build four interfaces (\S\ref{paradigms-interfaces}) representing four different paradigms for approaching the design of image generation interfaces: vanilla (baseline), diverse, reformulative, and agonistic. We conducted hour-long interviews with 29 participants in which participants use al four interfaces to create a collage representing a human subject of their choice (give examples). They face decisions about what images to include and to include, thinking through these subjects.
%     \item \note{Go over how results}. We find that ... (\S\ref{how-reflection})
%     \item \note{Go over why results}. To explain, we found that ... (\S\ref{why-reflection})
%     \item We find that building reflective image generation interfaces requires...
%     \begin{itemize}
%         \item finding 1
%         \item finding 2
%         \item etc.
%     \end{itemize}
%     \item In summary, our main contributions are: a) raising the general issue of image generation as a site for reflection, which has not been discussed much in thee xisting literature; b) the contribution of an interface and interactions under the \textit{agonistic paradigm} to stimulate more reflection in image generation interactions; c) comparisons of this interface with other dominant paradigms / values in image generation interface design via interviews with 29 participants; and d) design principles and lessons for building reflective image generation interfaces.
% \end{enumerate}




% \begin{figure}[htbp]
%     \centering
%     % First subfigure
%     \begin{subfigure}{0.48\textwidth}
%         \centering
%         \includegraphics[width=\textwidth]{assets/collage-spread/Sample Collage Demos.pdf}
%         \caption{Describe the subject}
%         \label{fig:subfig1}
%     \end{subfigure}
%     \hfill
%     % Second subfigure
%     \begin{subfigure}{0.48\textwidth}
%         \centering
%         \includegraphics[width=\textwidth]{assets/collage-spread/Sample Collage Demos.pdf}
%         \caption{\andre{Will be a different collage in final version}}
%         \label{fig:subfig2}
%     \end{subfigure}
%     % Main caption
%     \caption{These figures demonstrate the progression of collages in two user studies, with the replacements shown fully and the kept images transparent. \andre{Will replace A, B, C, D with \baseline, \diverse, etc. later.} \andre{Maybe we can add little annotations explaining why images were added? it may be hard to see on your own}}
%     \label{fig:mainfig}
% \end{figure}
