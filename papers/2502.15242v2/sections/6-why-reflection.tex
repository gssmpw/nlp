\section{\textit{Why} do different interfaces induce reflection?}
\label{why-reflection}

To explain the observed differences in reflection under different interfaces as discussed in \S\ref{how-reflection}, we look at the relationship between reflection and other variables: perceived interface-level properties like appropriateness and control (\S\ref{interface-properties}), participants' values when adding images (\S\ref{why-add}) and why they reject images (\S\ref{why-reject}), and qualitative themes (\S\ref{qualitative-themes}).
% (explaining interface -> variables -> intent)}

% \assign[Andre]{write section header}

\subsection{Perceived Interface-Level Properties}
 % \assign[Andre]{...}}
\label{interface-properties}

% \begin{wrapfigure}{l}{0.55\linewidth} % 'l' aligns left, and width is set to 50% of the text width
%     \centering
%     \includegraphics[width=\linewidth]{assets/results-visuals/properties.png}
%     \caption{Mean participant responses on a 5-point scale for three dimensions (Appropriateness, Control, Interest). The ``Rethink'' response is re-included for comparison.}
%     \label{fig:enter-label}
% \end{wrapfigure}

\textbf{\reformulative~and \agonistic~improve upon \baseline~and \diverse~in appropriateness and control ($p \le 0.05$ for all).}
Indeed, in general, there is a weak but significant correlation between rethinking and both appropriateness ($r = 0.22$, $p = 0.02$) and control ($r = 0.27$, $p = 0.003$) in general.
\agonistic~is ranked above \diverse~by $81\%$ of participants for appropriateness and $86\%$ for control; \reformulative~by $85\%$ for appropriateness and $92\%$ for control.
% Given that \reformulative~and \agonistic~dominate \diverse~in rethinking, 
% This suggests that interfaces which produce results that end up seeming appropriate and engage users in ways that make them feel in control may encourage more rethinking.
% Indeed, there is qualitative evidence to support this, which we discuss in \S\ref{qualitative-themes}. 
% \andre{We can also discuss the qualitative evidence here.}
Although \diverse~cannot be distinguished significantly from \baseline~by appropriateness, it can by control --- \diverse~has lower control than \baseline~($p \ll 0.01$), meaning that rewriting user prompts to encourage diversity results in a noticeable feeling of decreased control among participants.
% \andre{worth discussion this?}
That being said, \agonistic~and \reformulative~are not themselves distinguishable by control or appropriateness ratings ($p \gg 0.05$).
% Indeed, only $59\%$ of people prefer \agonistic~to \reformulative~for appropriateness and exactly $50\%$ for control.
% Given that there is a noticeable difference between \agonistic~and \reformulative~for rethinking (as reported in \S\ref{how-reflection}), what explains the difference between these two, if not appropriateness and control?
Additionally, participants find \agonistic's possible interpretations more interesting than \reformulative's suggested reformulations ($p \ll 0.05$).

% although both are rated highly  in general ($\ge 4$).
% To continue explaining the difference in observed reflection between the two, we conduct other forms of analysis both at lower (\S\ref{why-add}, \S\ref{why-reject}) and higher (\S\ref{qualitative-themes}) levels.

% \amy{more detail than the reader cares to know?}
% \amy{you can do it a little bit but it seems like there's a good bit of interpretation happening in these findings sections, can you more directly say the findings and do most of the interpretation in the Discussion?}


% We can further analyze correlation with satisfaction...
% Very interestingly, while there is a significant weak correlation between satisfaction and both appropriateness ($r = 0.31$, $p \ll 0.01$) and control ($r = 0.23$, $p = 0.01$), there is no correlation with rethinking at all ($r = 0.02$, $p \gg 0.05$).
% [Maybe also measure with satisfaction deltas]
% \andre{disentangle satisfaction from appropriateness -- important concept here}
% Almost near zero correlation between satisfaction and reflection, suggesting that people do not necessarily feel more satisfied even when there is reflection.
% \andre{need to do more analysis on this part -- not just abslute/relative, alos exclude when the satisfaction has reached 5 and there is nowhere to go, or otherwise scale by satisfaction...}


% (interface -> properties -> intent)}

% After interacting with each interface, users report on the interface's appropriateness, control, and interestingness. \andre{introduction to the data collection source, maybe even specify what each of the ends are}

% The 

% \andre{add table with features (appropriateness, control, interest) by interface}

% The appropriateness and control attributes help distinguish C and D from A and B, which corresponds with an observed gap between C and D and B.

% Even though we do observe this broad interface-level observation, the subject-level variation is actually only quite weak: 0.20 for rethinking and appropriateness, 0.30 for rethinking and control.

% \andre{standardize by each subject -- mean zero and unit variance, or something like that}

% We have qualitative reasons to believe that both appropriateness and control are important necessary conditions for rethinking, as observed by their presence generally in \reformulative~and \agonistic~and absence in \baseline~and \diverse.

% \andre{provide qualitative descriptions}

% \andre{segueue into next section}
% However, D produces notably more reflection than C by several measurements (reference previous section).
% This is not only explained by appropriateness and control, which appear to be necessary but not sufficient conditions for high reflection.
% The interestingness may get at it but there is not that much difference either.
% Therefore, the mystery now is to consider how D varies by C. To do so, we look at a more granular level at the \textit{image-level} values before taking a broader view at \textit{qualitative themes} that distinguish the interfaces


\subsection{Image-Level Values: Why users add images}
\label{why-add}

% (interface -> values -> intent)}



\begin{figure}[!t]
    \begin{minipage}[t]{0.53\textwidth}
        \centering
        \includegraphics[height=4.4cm]{assets/results-visuals/values.png}
        \caption{Distribution of values cited for images added with each interface (discussed in \S\ref{why-add}; Table~\ref{tab:values-interfaces}).}
        \label{fig:values-interfaces}
    \end{minipage}
    \hfill
    \begin{minipage}[t]{0.45\textwidth}
        \centering
        \includegraphics[height=4.4cm]{assets/results-visuals/intents-values-distribution.png}
        \caption{Breakdown of intents by value cited when adding image (discussed in \S\ref{why-add}); Table~\ref{tab:values-intents-rel}.}
        \label{fig:intents-values}
    \end{minipage}
\end{figure}



% \andre{Can probably put these side by side}


We also explore whether different values may serve as a relevant explanatory variable for types of reflection (intents). Values code for \textit{why} participants add images, or in other words the reasons that participants invoke when selecting images. The values in our coding ontology are as follows: 1) \realism: the image represents how the user thinks the world actually is; 2) \familiarity: the image fits the user's assumptions, regardless of whether the user thinks they are representative of how the world actually is; 3) \diversity: the image portrays an underrepresented aspect of the prompt that the user believes is normatively important to include; 4) \aesthetics: the user otherwise likes how the image looks. The distribution of values across interfaces and intents is shown in Fig. \ref{fig:values-interfaces} and \ref{fig:intents-values}.

\textbf{\agonistic~is the only interface for which \diversity~is invoked more frequently than \familiarity.} In all interfaces \textit{except} \agonistic, \familiarity~is invoked most frequently, followed by \realism, \diversity, and \aesthetics. In contrast, for \agonistic, participants invoke \diversity~most frequently, followed by \realism, \familiarity, and \aesthetics. Interestingly, compared to \baseline, participants invoke \diversity~\textit{less} frequently for \diverse~($-2\%$), but more frequently for \reformulative~($+12\%$) and \agonistic~($+19\%$).


% \textbf{\reformulative~evokes the lowest amount of \realism~but the highest amount of \familiarity.} While participants invoke \realism~at similar rates across all other interfaces ($39\%$-$42\%$), they do so at significantly lower rates for \reformulative~($24\%$). However, they also invoke \familiarity~at much higher rates for \reformulative~than other interfaces ($53\%$ over $42\%$-$47\%$).
% \amy{happens elsewhere too but I feel like there's got to be a way to say your major findings more directly and succinctly. feels like lots of little things being reported, most of which are not that useful to know?}
% \andrew{TODO: cut down to most important findings}

\textbf{\diversity~is associated with the highest rate of reflection.}
The majority of images across values are added with \direct~intent. Among values, \aesthetics~ is associated with the highest highest proportion of \direct~intent ($96\%$), followed by \familiarity($95\%$), \realism~($89\%$), and \diversity~($75\%$). \diversity~is associated with the highest degrees of reflection, with $7\%$ of images added with \reminder, $16\%$ of images added with \expansion, and $2\%$ of images added with \challenge. Since \agonistic~also has the highest proportion of \diversity~values, these findings suggest that \diversity~serves as a strong explanatory variable for reflection.

% \andre{Note --- we may find that users accept/reject on the basis of un/familiarity much more than other types of values like diversity and factuality/realism, if this is true then there are big design implications, people aren't talking very much about familiarity and the associated concepts right now, could be good to highlight that while we care about factuality, diversity, etc. the values that people are invoking most even when thinking about these political/identity/historical issues is familiarity, so we should keep that in mind when designing. might also be a good discussion section --- discussed in discussion section}

\subsection{Image-Level Values: Why users reject images}
\label{why-reject}

% \assign[Andrew]{coordinate section}
% \assign[Andre]{add compelling pieces}


% \andre{Andrew: compute reasons for rejections as proportions by interface -- what percent type for each interface; we can qualify that it may not be super quantitatively rigorous but it's one number which could be interesting --- going to try just qualitative first}

We carry out a qualitative analysis of image rejection in lieu of a quantitative analysis because the unit of rejection is not as well-defined as adding images.

\textbf{Participants reject images due to factual inaccuracies, especially when interacting with \diverse.}
% When \diverse~generated an image of Black and White military members working together for the prompt ``World War II,'' \pref{17} responded: ``\textit{the army was only desegregated after World War II, so this photo [...] is just like a fantasy.}'' Similarly, 
\pref{7} said that racially diverse pictures of `The Chinese Communist Revolution' ``\textit{looked like modern people were slapped into the wrong time period,}'' and \pref{28} experienced ``\textit{pushback}'' against racially diverse pictures of ``The Signing of the Declaration of the Independence.'' 
In other interfaces, participants similarly cited lack of factuality as a reason for rejecting images.
% though the inaccuracies were less often related to inaccurate demographic diversity. 
For instance, \pref{2} reacted ``\textit{Middle class people don't dine out very often, and especially not at fancy places},'' when \baseline~generated an image of people at an upscale restaurant for the prompt ``A middle class person.'' 
% \amy{to save space, can just do 1-2 examples per point}

% On the Constitutional Convention, \pref{24} said ``\textit{it's unfortunate, but it was kind of a reality}''

\textbf{Participants also reject images because of an excessive adherence to factuality.} 
\pref{26}, generating images of ``A Syrian refugee'' noted that \agonistic~``\textit{presented more of what I know to be true, [...] the harsh reality of it. [...] The other interfaces [...] were more instrumental in helping me expand beyond that.}'' 
\pref{7} shared that they preferred \reformulative~over \agonistic~because \reformulative~did a better job ``\textit{highlighting all different aspects of [their prompt], instead of just the realism images}.'' Such comments suggest that users do value an some level of diversity in generated images that may extend beyond what they consider to be strictly factual.

\textbf{Participants reject images due to lack of familiarity, even where they acknowledge that an image may be factual.} 
When using \agonistic, participants may believe interpretations to be factual (being extracted from Wikipedia) but still reject them because they are unfamiliar. 
% We observe that lack of familiarity is often invoked for \agonistic, which justifies generations with information from Wikipedia. 
For example, with the prompt ``An Arab person'', \pref{25} reacted to an image of a Sudanese Arab person from \agonistic~by remarking, ``\textit{Oh, Sudanese Arab [...] I think that's an identity that I haven't represented here that I would like to.}'' 
They later decided to reject the image by providing the following reasoning: ``\textit{It just does not resonate with me. I guess I'm also not familiar with Sudanese Arabs as much.}''
% \andrew{@Andre @Andrew find a potentially better example?}
% Nonetheless, we emphasize that reflection can still occur when participants reject images, such as how the suggestion of Sudanese Arabs reminded the participant of an aspect of Arab identity that they had previously not considered.

% \subsection{Affective Responses to Interfaces}

% \subsection{W}
% (interface -> anything -> intent)}

% \assign[Andre]{...}

% \assign[Andrew]{...}

% Collection of themes

% We want to investigate how people's emotions / effects correlate to reflection.
% \andre{maybe give definition of affect / emotion}


% We also have other measures of affect, such as qualitatively, e.g.: frustration, offense, boredom, etc.?

% Reflection is often encountered in the form of... confusion, ... ?



% What are people seeing from A, B, C< D, etc/ What uniquely are they getting?

% Affective responses to interfaces (interface -> emotions)
% qualitative



% features:
% - person
% - 10 slots for images -- drop down to code intention and value
%     - repeat 4x for every image -- only change the one that was replaced
% - also have list for removing images
% - capture our entire ontology for adding/not adding images/suggestions
% - have an open ended feature for discussing affective responses
% - have an open ended feature about how the participant reflected
% - nice qualitative quotes / ideas / "what this participant brings to the table / our study"


\subsection{Qualitative Themes}
\label{qualitative-themes}

We present several themes from qualitative analysis further explaining the differences in reflection observed in \S\ref{how-reflection}.

% \textbf{Affective responses to interfaces.}
% [Describe different affects.]
% Surprise seems important...

\textbf{Text helps users contextualize, interpret, and decide on images.} 
% \andre{mine}
% Roland Barthes remarked that ``\textit{all images are polysemous; they imply, underlying their signifiers, a ``floating chain'' of signifieds, the reader able to choose some and ignore others}''~\cite{barthes_rhetoric_image} --- images are `pure visual fields' that can be contextualized and given meanings in many different ways.
% \amy{move to Discussion?}
Throughout the user studies, we find that participants prefer interfaces that help them contextualize and interpret images (\reformulative~and \agonistic~over \baseline~and \diverse), even though ultimately they are only adding the images (and not the accompanying text) to the collage.
Firstly, contextualizing content is cognitively stimulating: 
``\textit{[the] suggestion feature... by nature makes your brain think about alternatives}'' (\pref{24});
it ``\textit{I wouldn’t have thought of [these perspectives] without the text}'' (\pref{1}).
% rompted me to think about the subject in different ways. 
% Indeed, ``\textit{even if [the suggestions are] inaccurate [...] it's still somewhat stimulating [...] your brain to think about, `oh, it's not accurate because of this'}'' (\pref{23}). 
% \amy{quote is confusing, does it need more punctuation? cut bits and replace with ellipses?}
Participants feel that it is meaningful and important to interpret the image, which is mediated by thinking about the relevant text.
\pref{1} said that the generated images ``\textit{adhere to the prompt but you have to figure out the image yourself}''.
\pref{15} said of \reformulative: ``\textit{I like with this interface. It gives me a lot more of an explanation. It kind of tells a story that I might not know just by looking at it... I think that it could help me kind of change what I'm looking for.}''
% Sometimes, accompanying text can be explanatory or informative: when interacting with \agonistic, \pref{27} said ``\textit{I didn’t know this was a historically attired Christian. So I think that explains a lot about previous imagery that generated}'', in reference to seeing similarly dressed individuals in \baseline~and \diverse~(but without explanation).

% Moreover, contextual content helps participants better correlate model inputs and outputs, developing a cognitive model of the interface and therefore becoming a more effective user.
% \andre{fill in}




% Images aren't produced in a vacuum; people find text to be really helpful towards explaining and understanding the images they're producing, they care about this. So image interfaces should be designed with supplemental text to encourage reflective generation.
% People want to have text to help them interpret images! Further supports the previously discussed work in photography and film that images are never ``just images'', but have different ethical, political, cultural, etc. dimensions that people want to be part of and engage with.
% Situating the input.
% \andre{@Andrew -- dump any examples that could be relevant}
% \andre{incorporate this into previous} \textbf{People find it important to correlate / mental model tracking inputs and outputs (even interpretability).}
% It not only helps interpret, but helps correlate inputs and outputs -- what is the process of going from my input text to an output? Providing text helps with this.
% Hard to follow along what's going on with B.
% Appropriateness
% James Duff: `It also just has a suggestion feature that, you know, by nature makes your brain think about alternatives.''
% Jenny Suk: ``The detailed text below or next to each picture in suggestions prompted me to think about the subject in different ways. I wouldn’t have thought of [these perspectives] without the text.''


\textbf{`Authenticity' situates diversity in an engaged political context.}
Although both \diverse~and \agonistic~end up displaying demographically diverse images to users, we find overwhelmingly that participants prefer the type of diversity offered by \agonistic~to that offered by \diverse~due to a greater perception of \textit{authenticity}. 
% \andrew{move definition of authenticity here?} 
% \pref{17}, for instance, reacted that they didn't ``\textit{get a clear idea from anything}'' generated by \diverse.
\diverse is not only perceived as factually inaccurate (\S\ref{why-reject}), but \textit{nontransparent, alienating, and ungenuine}.
\pref{17} said they couldn't ``\textit{get a clear idea from anything}'' from the interface.
% For their prompt ``A gun owner'', \pref{10} remarked that the racially diverse images by \diverse~``\textit{don't really connect with that idea of someone who would own a gun.}'' 
\pref{5}
% who described themselves as ``\textit{a person who believes highly in representation and diversity,}'' 
reacted when racially and gender diverse World War II soldiers generated by \diverse: ``\textit{I find them not just [...] historically inaccurate, I think it gives narratives that some people will not dig deeper into, and [...] can lead to a lot of misconceptions down the line of colorblind racism.}'' 
% Though participants generally describe understanding the value of diversity, they clearly do not think that \diverse~embodies that value authentically. Rather, the comments above suggest that diversity appears more authentic to participants when it is historically informed and connected to salient features of the prompt.
In contrast to \diverse, participants report that the perceived authenticity of \agonistic~engages them in the political context of image generation (\S\ref{agonistic-democracy}).
\pref{10} liked that \agonistic~was ``\textit{more grounded in reality}'' --- ``\textit{you're like, `yeah, real people exist.'}''
% not really as fun because 
% I like the [images] from [\agonistic] a lot, because it 
\pref{28} described \agonistic: ``\textit{[it] changed my perception of what the prompt itself was [...] but [\diverse] feels like it's [...] trying to change our perception of the event itself that I was already thinking of [...\agonistic] tries to bring in diverse perspectives and increase representation of other groups [...] in a way that felt more authentic and can change people's assumptions in a healthier way.}'' 
% In these and other interviews, we observe that participants are more likely when interacting with \agonistic~than \diverse~to acknowledge the political context of image generation, considering images as alternative interpretations of their prompt that ``\textit{real people}'' might hold. Participants thereby feel \agonistic~to be more ``\textit{authentic}'' than \diverse, and report being more open to changing their assumptions when interacting with \agonistic. \\

% Similar to \pref{28}'s comment (also mentioned above in \S\ref{why-reject}) that \diverse~was ``\textit{less grounded in [...] history,}'' \pref{10} remarked that they found \agonistic~preferable to \diverse~because it was ``\textit{more grounded in reality}.''
% To this effect, \pref{9} also commented that \agonistic~differed from other interfaces in that it felt more like ``\textit{the AI and the person working together to get to the final image generation.}'' In short, confrontational design can be aided by authenticity cues to more effectively situate image generation in broader political discourse and help users navigate politically sensitive topics.
% \pref{10} said that ``\textit{I find it interesting... but it doesn't \textit{honor} the situation} [of what happened in WWII]''.



% \andre{complete this one -- talk about it may have the highest appropriateness but it really treads the line.}
% Maybe can put the comments about feeling like it's one step forward, one step back (e.g. Calvin) here.
% Language of ``not what I'm looking for'', dynamic between having something not be what you are looking for (using what you want as a compass/measurement to accept/reject vs. changing your compass)

% \andre{put one more theme here @Andrew}


% % \andre{maybe we want to move this to the results section previously?}
% \andre{somewhere here: put this idea of A (specificity) vs. B (quality of images) trade off as interesting to examine with and without prompt reformulation}

