\section{Background and Context}
\label{background}

% \andre{@Amy @Ranjay currently there are a lot of ideas that are important to motivating our work which are described in this section, but it's kind of complicated to organize the ideas in a clear narrative}

% This section aims to provide theoretical motivation for our contribution (\S\ref{discursive-approaches}, \ref{agonistic-democracy}), as well as to better describe the current values in image generation interfaces (S\ref{current-work}) as well as to situate our work in the broader context of ongoing research on reflective human-AI interaction~(\S\ref{hai-reflection}).
% We begin by investigating what the meaning of an image is, drawing upon perspectives from philosophy, photography, and history to demonstrate that images always enter into a community of discourses and dialogues~(\S\ref{discursive-approaches}.
% Then, we draw upon political philosophy, particular on agonistic democracy, to consider image generation as a political act of entering this set of discourses and considering how agonism can help us design interfaces that navigate the challenges of image generation~(\S\ref{agonistic-democracy}.
% Afterwards, we provide an overview of current work on user-facing aspects of image generation models and interfaces, demonstrating that most work is justified from the point of view of actualizing user intent~(\S\ref{current-work}.
% Lastly, we aim to situate our work in existing  human-AI reflection~(\S\ref{hai-reflection}).


\subsection{Images as socio-political entities}
\label{discursive-approaches}

% \assign[Andre]{}

% \amy{I'd drop this para to 1 sentence, merge with 2nd para}
% Understanding what images are and how they function may help us design image generation interfaces that better reflect the values we want from image generation.
% It is useful to understand how philosophers have investigated the meaning of \textit{signs in general} (images, words, sounds -- anything that stands for or represents something else).
% Although some theories take the meaning of signs to be the things that they refer to~\cite{kripke1980naming} or logically apply to~\cite{russell1905ondenoting} (e.g., ``cat'' refers to that cat or the category of cats), other thinkers emphasize the social and active aspects of signs.
% The primary nature of signs may be that they are constructed and used in cooperative communication~\cite{grice1989logic}, governed by the rules of a ``language game''~\cite{wittgenstein1953philosophical}.
% The invocation of a sign can \textit{do something in} or \textit{change the state of} the social world (e.g. a teacher saying ``your grade is $80\%$'' makes it so)~\cite{austin1956performative, austin1962how}.
% Signs can be ``performed'' --- iterated repeatedly --- to construct and demonstrate membership within social categories and groups~\cite{butler1990gender, hall1990cultural, hall1997representation}.
% The same applies for signs (images) produced by generative AI --- they  play active roles in social life.

Philosophers of meaning have argued that signs (things like words or images representing something other than themselves) can be ``performed'' to demonstrate membership within social categories \cite{butler1990gender, hall1990cultural, hall1997representation} and invoked to effect changes in the world \cite{austin1956performative, austin1962how} --- that is, \textit{signs do not just represent, but also play active roles in social life}. 
Likewise, thinkers in history have challenged concepts of ``objective'' truth by calling attention to complex networks of competing narratives, power relations, and institutional practices~\cite{certeau1988writing, white1973metahistory} --- what Michel Foucault calls ``discourse''~\cite{foucault1972archaeology} --- that shape historical study.
% \andrew{@Andre is there a specific name that we might call this view?} 
Thus, historical questions of ``how things were really like'' are far from straightforward because appealing to historical accuracy also requires considering ways that discourse shapes the historical record itself.
% \andre{rephrase, don't like phrasing}
% Therefore, demanding AI image generation models to simply produce ``historically accurate'' images may be an oversimplification, as it may be necessary to consider and present relevant discursive factors.
Thinkers in media studies have further elaborated these insights in the specific context of visual media.
Images are themselves political not only in the sense that interpretation of them is conditioned by cultural context \cite{barthes1981camera}, but also in the sense that the act of creating or capturing an images always involves ``\textit{imposing standards on [one's] subject}''~\cite{sontag1977on}. 
For example, Laura Mulvey's concept of the ``male gaze'' describes how the movement and position of the camera can frame women as passive objects of visual pleasure \cite{mulvey1975visual}. 
% \andrew{@Andre find a more straightforward example than Sivan?}
These works suggest that all image production and consumption --- including AI image generation --- is implicated in social, political, and ethical issues: ``\textit{individuals implicitly engage in ongoing struggles over visual dominance and its articulation with social formations through their involvement in image production, distribution, exposure, and consumption}'' \cite{herrieetal2024democratization}. 
Our aim in this paper is to build an image generation interface which engages users in this complexity of image generation.
% \andre{revise to be more eloquent}

% Indeed, the Gemini case is a prime example of the way in which image generation can easily become part of larger political discourses, with ``\textit{an explosion of negative commentary from figures such as Elon Musk who saw it as another front in the culture wars.}'' \cite{milmoandkern2024gemini}.

% Thinkers in photography and film studies have further elaborated these insights in the specific context of visual media.
% Roland Barthes writes that cultural context forms the general field of meaning that more individual, subjective experience of a photograph stands relative to~\cite{barthes1981camera}.
% Susan Sontag famously argues that photographs are not merely records but active interpretations of the world: ``\textit{in deciding how a picture should
% look, in preferring one exposure to another, photographers are always imposing standards on their subject}''~\cite{sontag1977on}.
% For example, Laura Mulvey's concept of the ``male gaze'' describes how the movement and position of the camera can frame women as passive objects of visual pleasure~\cite{mulvey1975visual}.
% Likewise, filmmaker Eyal Sivan comments on how even films which operate on the pretense of objectivity (like documentaries) can enact political and ethical functions: documentaries, especially covering conflict, often ``\textit{have a victim that is suffering for you [the spectator] and through his suffering he redeems the spectator and more: He says, you are human because you feel my suffering. So it comforts the spectator, [affirming his] position of being the ``good one''} \cite{silver2012against}.
% These works suggest that \textit{all image production and consumption} --- including AI image generation, and especially of people --- \textit{is implicated in social, political, and ethical issues}.
% Our aim in this paper is to build an image generation interface which engages users in such issues.

% Many of the issues at play in the Gemini case featured historical inaccuracies.
% Although truth is a central and perhaps irrevocable value in historical scholarship, truth in history is more complex.
% Often it is about examining competing narratives, advancing different interpretations.
% Concept of discourse in history.


% and complex social interaction.
% de Saussure proposed the ``arbitrariness of the sign'' --- that there is no necessary relationship between the form of a sign and the thing it represents --- and instead suggested that the meaning of signs depends on a system of differences between signs (example?) which are a matter of social convention (cite).
% Derrida further destabilizes any sign's claim to a fixed or ultimate meaning, emphasizing that a sign's meaning's relation to other signs ``infinitely defers'' its meaning.
% Others suggest that the meaning of signs is held in communication with other humans (Grice), what we end up doing with them (Austin), and how we ``perform'' them (Butler) -- demonstrating them without ever achieving them (useful in the analysis of hate speech (cite) and gender, provocations of thought (Deleuze).
% These ideas become important in the analysis of cultural ideas --- that may be constructed and experienced as real even if not real in the sense of referring to a reality (Stuart Hall).
% In all these cases, turning away from thinking of a sign as just reflecting a reality towards a more complex social notion of a sign which both captures the dynamiss of the social world and also affects how we view the world changes things.
% But there are two concrete applications of interest here when we consider how meaning is considered in photography and history, given that many of the contested issues in the Gemini issue were about history.

% The meaning of things is not fixed, but is continually being renegotiated through the processes of representation and discourse. -- Stuart Hall (maybe not legit?)

% In order to be, we might say, we must become recognizable, but to challenge the norms by which recognition is conferred is, in some ways, to risk one’s very being, to become questionable in one’s ontology, to risk one’s very recognizability as a subject.

% \textit{Photography and Film.}
% While photography might appear to be a passive act of capturing reality, understanding images as active signs reveals their role in shaping discourse and meaning. 
% Susan Sontag (On Photography, 1977) argues that photographs are not merely records but interpretations of the world, noting that ``in deciding how a picture should look, in preferring one exposure to another, photographers are always imposing standards on their subjects.''
% Roland Barthes (Camera Lucida, 1980) further develops this by introducing the concept of the ``studium'' (cultural-political reading) and ``punctum'' (personal, emotional response), demonstrating how photographs create meaning beyond mere representation.
% The concept of the ``male gaze,'' introduced by Laura Mulvey (Visual Pleasure and Narrative Cinema, 1975), illustrates how even seemingly objective camera angles can embed particular viewpoints and power dynamics. 
% Documentary filmmaker Eyal Sivan comments on how even films which operate on the pretense of objectivity (like documentaries) can perform political and ethical functions: ``\textit{[in documentaries], you have a victim that is suffering for you [the spectator] and through his suffering he redeems the spectator and more: He says, you are human because you feel my suffering. So it comforts the spectator, [affirming his] position of being the ``good one''.}''
% We can express the role of the creator in participating in the political and ethical natures of image production to only be heightened with AI image generation.

% \textit{History.}
% While we might assume that images simply reflect historical reality, Michel Foucault's analysis in "The Archaeology of Knowledge" (1969) reveals a more complex process. Rather than accepting historical images as neutral documents, Foucault shows how they emerge from and participate in networks of power relations, institutional practices, and competing narratives. What appears as historical "truth" in images is actually produced through these ongoing contestations - different groups advancing competing interpretations, institutions legitimizing certain representations over others, and power relations determining which visual narratives become dominant.
% This discursive approach to understanding images manifests across numerous domains of visual representation: the construction of racial categories through colonial photography (Said, "Orientalism," 1978), gendered depictions in medical and professional imagery (Kitch, 2009), religious iconography varying across cultures (Morgan, "The Sacred Gaze," 2005), orientalist portrayals of non-Western peoples (Said, 1978), representations of occupations and social roles (Gherardi, 1995), and evolving depictions of historical figures like Cleopatra that reflect changing cultural narratives about race and power (Bianchi, 2021). These visual representations don't simply document reality but actively shape how we understand social categories and their relationships.

% \andre{look at pinned springer article}


% Describe things that are useful to explaining design decisions and the premise, and then also do more philosophy in the discussion.. could be more interesting to talk about the philosophy and paradigm there. 

% how does prior work inform your prior views.

% Andre

% Susan Sontag, Barthes, Derrida, Butler. What is a photograph? What is meaning? Why is it always elusive?

% Foucault and friends -- discursive methods in historcial scholarship. e.g. what is ``sexuality''? etc.

% I am not a pipe -- 

% Look at pinned Springer article


\subsection{Agonistic Pluralism: A political framework for image generation}
% \subsection{Agonistic Democracy and Image Generation}
\label{agonistic-democracy}

Given the political nature of image generation, what kind of political structures should be used to deal with this conflict?
Some commentators have found the problem with the Gemini case to be that ``\textit{tuning to ensure that Gemini showed a range of people failed to account for cases that should clearly not show a range}'' \cite{milmoandkern2024gemini}. 
If image generation is political, however, then there may not always be a clear division between cases that should or should not show diversity, but rather a pluralism of conflicting, sincerely held beliefs. 
For instance, though some reacted negatively to generated images of non-White Founding Fathers, public acclaim of the Broadway play \textit{Hamilton}, which used people of color to portray Founding Fathers, implies that depicting historical figures in racially inaccurate ways is not necessarily always problematic.
Consequently, we draw from democracy theory in this paper to inform our design because it offers a long tradition of scholarship aimed at answering questions of pluralism. 
Herrie et. al. suggest that within democracy theory, agonistic pluralism is apt for analyzing the political dimensions of image generation \cite{herrieetal2024democratization}. 
First developed by Chantal Mouffe, agonistic pluralism is a philosophy that sees struggles over meaning, like those in the Gemini case, as inherent and ineradicable parts of democratic society. 
% In agonistic democracy, ``\textit{all material practices are considered discursive, and social formations are in constant struggle against one another}'' \cite{herrieetal2024democratization}. 
The task for democracy then is not to eradicate such struggles, but to channel them into productive ends. 
Doing so requires at least two major conditions which we highlight here as relevant to our work. 
First, democracy should always allow for the possibility of political struggle by providing avenues for hegemonic views to be challenged. 
Second, Mouffe qualifies that agonistic pluralism also involves a basic level of respect for others: in contrast to purely antagonistic struggle in which opponents view each other as enemies to be destroyed, agonistic struggle requires viewing opponents legitimate adversaries ``\textit{whose ideas we combat but whose right to defend those ideas we do not put into question}'' \cite{mouffe2000democraticparadox}. 

% \andre{need to revisit this section and make it punchier, more eloquent}


\subsection{Current values in image generation interfaces}
\label{current-work}

% \assign[Andre]{...}

Many current works aim to help users produce ``better'' images without necessitating additional user interaction, invoking the value of \textbf{intention actualization}.
Prompt reformulation methods automatically rewrite prompts to produce more aesthetic or generally preferred~\cite{zhan2024capability, xu2024imagereward, datta-etal-2024-prompt, chen2024tailored} images, motivated by the goal of ``\textit{convey[ing] the user’s \underline{intended ideas}}''\cite{mo2024dynamicpromptoptimizingtexttoimage} or ``\textit{preserving the \underline{original user intentions}}''~\cite{hao2024optimizing}
Other work builds interfaces to help users explore and iterate over prompts, invoking similar values:
```\textit{align the user’s creative \underline{intentions} with the model’s generation'' }~\cite{wang2024promptcharm};
``\textit{align with [users'] \underline{intended creative output}}''~\cite{brade2023promptify}.
This emphasis on actualizing user intentions is described by \citet{sarkar2024intention} as the ``intention is all you need'' paradigm: ``\textit{Contrary to the assumption that GenAI merely executes human intentions, it also shapes them [...] induc[ing] `mechanised convergence,' homogenising creative output, and reducing diversity in thought}.''
\citet{cai2024antagonisticai} characterizes interfaces that overly emphasize cognitive ease of use as ``\textit{easy to learn at the expense of the power and complexity necessary to do hard but valuable work or learn uncomfortable truths}''.
In response to these critiques, our work explores how users might \textit{question} and \textit{revise} their intentions, for example by reflecting upon limiting assumptions. 
% \amy{feel like this para could half as long, don't need multiple quotes for a point}
% \andre{I prefer keeping quotes in to qualitatively show how common the intention paradigm is, but I significantly shortened them}

Another notable value in image generation is \textbf{diversity}.
Image generation models have well-documented issues with diversity, often failing to appropriately represent demographic minorities and resorting to stereotypes~\citep[\textit{inter alia}]{ali2024demographic, wang-etal-2023-t2iat, wang2024newjob, chauhan2024identifying, naik2023t2i}.
Approaches to address diversity issues include finetuning the model with some diversity objective~\cite{miao2024training, esposito2023mitigatingstereotypicalbiasestext}, intervening on internal model states~\cite{friedrich2023fairdiffusioninstructingtexttoimage, zhang2023iti, zameshina2023diversediffusionenhancingimage, chuang2023debiasingvisionlanguagemodelsbiased}, or rewriting the prompt to emphasize diversity~\cite{microsoft_prompt_transformation}.
Our work takes another angle on diversity, aiming to engage users with competing visual interpretations of a subject rather than directly generating ``diverse'' images. We further discuss this issue in \S\ref{inauthentic-diversity}.

% \cite{cai}
% ``easy to learn at the expense of the power and complexity necessary to do hard but valuable work or learn uncomfortable truths'' (Cai et. al. 2024) --- detrimental to reflection

% ``guide users towards `ethically preferable' choices that serve as much to protect users as to shield the company from legal and sociopolitical critique.'' (Cai et. al. 2024) --- very apt for Gemini

% Treats ethico-political issues as already settled in the realm of human-AI interaction, either by imposing some ethical standard or by relegating control to the user. For this reason, we refer to this current dominant paradigm as the ``deliberative'' model of HCI below.

\subsection{Agonistic design and reflection in HCI}
\label{hai-reflection}

% \assign[Andrew]{...}

In contrast to this intention-centric paradigm, Kate Crawford poses the question: ``\textit{What might happen if we brought a model of agonism to understanding algorithms?}'' \cite{crawford2014agonistic}. 
She answers that agonistic design requires acknowledging that ``\textit{complex, shifting negotiations are occurring between people, algorithms, and institutions, always acting in relation to each other}'' \cite{crawford2014agonistic}.
These suggestions are in line with more recent literature in support of design principles that similarly seek to disrupt rather than reproduce user intention \cite{cai2024antagonisticai, sarkar2024intention, herrieetal2024democratization}. 
As applied to image generation, agonistic design entails that developers might avoid imposing a single standard on image generation outputs in favor of bringing controversy to users' attention and providing ``\textit{resources for more people to participate in that dissensus}'' \cite{crawford2014agonistic}.

Agonistic design also finds affinities in existing HCI work on designing for reflection, understood here as ``\textit{intellectual and affective activities in which individuals engage to explore their experiences in order to lead to new understandings and appreciations}'' \cite{boudetal1985reflection}.
Based on a review of HCI research about reflection, Fleck and Fitzpatrick set forth a five-level framework of reflection, in which each level is distinguished from the last by increasingly deep consideration of different views. The last stage of reflection, called critical reflection, involves challenging personal assumptions by considering a wide range of social and ethical issues \cite{fleckandfitzpatrick2010reflectingonreflection}.
% \begin{enumerate}[topsep=0pt]
%     \item Description: description without further explanation, \textbf{not reflective}
%     \item Reflective Description: description including explanation or justification
%     \item Dialogic Reflection: considering different explanations, hypotheses, and points of view
%     \item Transformative Reflection: challenging personal assumptions leading to a change in understanding
%     \item Critical Reflection: considering wider social and ethical issues
% \end{enumerate}
Halbert and Nathan further elaborate this last stage of critical reflection as the result of a two-stage process in which users first encounter a discomforting experience, and then become critically aware of their own biases by working through discomforting feelings, rather than avoiding them \cite{halbertandnathan2015criticalreflection}. 
% Importantly, they caution that the exposure to discomfort required for critical reflection may not be productive for all populations, especially for marginalized people who already regularly face discomforting experiences. 
% Nonetheless, we believe that our work on critical reflection in this study remains a valuable way to surface insights that can be used to disrupt the more harmful biases that image generation interfaces are prone to reproduce. \andre{maybe put these two sentences into an ethics statement}
Though comparatively little HCI research on reflection has been done for art and culture domains~\cite{bentvelzenetal2022revisitingreflection}, we aim to fill that gap by designing image generation interfaces for user reflection in this paper. 
% More recently, a structured literature review by Bentvelzen et. al. show that that while HCI research on reflection has largely tended to focus on applications for health and well-being, daily life, and learning and education, comparatively less work has been done on reflection in domains like art and culture \cite{bentvelzenetal2022revisitingreflection}. Consequently, the potential for more recent generative visual technologies like image generation models to play a role in user reflection also remains underexplored. 
% In this paper, we aim to fill this gap by designing and testing an image generation interface that supports user reflection.

% We draw on comparison and discovery techniques used in other reflection-centered interfaces, which encourage reflection by letting users compare their status with either an absolute or social reference (comparison), or  evoking the user to see a topic in a new light (discovery) \cite{bentvelzenetal2022revisitingreflection}. \andrew{Should I just move this mention of comparison and discovery techniques to \S\ref{paradigms-interfaces} where we describe how we design the interfaces?}  We further elaborate specific reflection-centered interfaces and their influence on our interface design in \S\ref{paradigms-interfaces}.


% \andrew{To move to \S\ref{paradigms-interfaces} to describe how \agonistic~draws from these studies}
% Tools designed to support user reflection include StarryThoughts from Kim et. al. (2021), who design an interface that organizes political opinions along various identity groups to encourage users to explore different perspectives and combat political polarization \cite{kimetal2021starrythoughts}. Paramita et. al. (2024) take a participatory design approach to design user interventions that raise awareness of search engine biases, including displaying various tags and pieces of information that indicate the bias of a search engine result \cite{paramitaetal2024searchenginebiases}. Cox et. al. (2024) leverage distances between language embeddings to enhance user creativity by directing prompts away from existing user ideas \cite{coxetal2021directeddiversity}.

% More generally,  \amy{bit of a laundry list. don't have to summarize every paper, just cite them in a group. only bring up points that relate to us} \andrew{TODO: explain these resources further} 

% Our work here also differs from existing literature on reflection in human-AI interaction. Sarkar observes, ``current explorations of improving critical thinking with GenAI [...] are strictly additive: let’s augment AI interaction and output with prompts, text, visualisations, etc. that get the user thinking.'' (Sarkar 2024). The problem is that since information increases the cognitive burden on users, the amount of critical thinking is inherently limited by human cognitive capacity. Instead, Sarkar suggests that we begin from a different question of \textit{resistance}: ``How can we build GenAI tools with inherent, productive resistances that are part of working with the tool, not an additional thing that users need to `pay' attention to?'' (Sarkar 2024)



 

% \note{Andre may want to add some philosophical work here briefly, mainly Heidegger on tool breaking}
