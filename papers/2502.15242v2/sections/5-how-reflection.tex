\section{\textit{How} do interfaces induce reflection?} 
\label{how-reflection}


\begin{figure}[!b]
    \centering
    % First subfigure
    \begin{subfigure}[t]{0.55\textwidth}
        \centering
        \includegraphics[height=4.3cm]{assets/results-visuals/basicimg-a.png}
        \caption{Overall, by interface.}
        \label{fig:subfig1}
    \end{subfigure}
    \hfill
    % Second subfigure
    \begin{subfigure}[t]{0.44\textwidth}
        \centering
        \includegraphics[height=4.3cm]{assets/results-visuals/bascimg-b.png}
        \caption{Breakdown within interface by prompt categories: ``H'' for History, ``P'' for Politics, ``I'' for identity.}
        \label{fig:subfig2}
    \end{subfigure}

    % Whole figure caption
    \caption{Participant ratings on a 5-point scale for how much interacting with the interface made them rethink their mental picture.}
    \label{fig:mental-image}
\end{figure}


% (establishing interface -> intent)}

% \assign[Andre]{write section header}

In this section, we explore \textit{how} different interfaces induce reflection by investigating variables that directly measure some aspect of reflection collected in the study.
The aim is to paint a holistic picture of how reflection occurred under different interfaces. 
% we attempt to explain \textit{why} by investigating other variables in \S\ref{why-reflection}.
In Table~\ref{tab:qualitative-examples} we provide two examples of reflection for each of the interfaces from our interviews.
To characterize reflection across different interfaces, we investigate users' self-reported change in mental image (\S\ref{change-mental-image}) and reflection on a per-image level (\S\ref{adding-images}) as described in \S\ref{measures}.

\begin{table}[!t]
\centering
\begin{tabularx}{\textwidth}{|l|c|X|}
\hline
 & \textbf{PID} & \textbf{Examples of Participant Reflection} \\ \hline
\multirow{3}{*}{\rotatebox{90}{\baselinebox}}
    & \pref{24} & \footnotesize The participant noticed that all the images for ``the constitutional convention'' featured the delegates sitting. In subsequent generations, they specified that the delegates should be standing. They reflected: ``\textit{when I kept generating those prompts and they were sitting [...] I don't think I would have specified whether they were sitting or standing before. Since I thought about it, I had to... play around with the prompt to get them to stand}''. They further remarked: ``\textit{Interacting with this interface [...] made me pay more detailed attention to what I perceive would have actually happened at that historical event.}'' \\ \cline{2-3}
    & \pref{4} & \footnotesize After generating a few images for the subject ``a queer community'' which seemed to ``\textit{rely on heavy stereotypes [...] like pride parades}'', the participant wanted to generate ``\textit{everyday examples of queer people}''. After producing a few images, they remarked ``\textit{now you lose the queer part}.'' After a few moments, they reflected: ``\textit{Well now you really got me questioning myself [...] what is a queer community? Do they all have to be wearing pride flags? How do you tell? Now I'm getting in my own head about it.}'' \\ \hline
\multirow{3}{*}{\rotatebox{90}{\diversebox}}
    & \pref{19} & \footnotesize The participant was attempting to generate an image of ``A Syrian refugee working in Germany'' when \diverse~generated an image of a woman in a healthcare setting. The participant paused and remarked, ``\textit{can you train to be a doctor [in] 12 years?}'' They later explained that they were unsure whether a Syrian refugee would have been able to study long enough to become a licensed medical professional since leaving Syria, but reasoned that ``\textit{the situation has gone on long enough [that] they have studied and integrated in the societies they're in, where [they're] able to actually finish something in the medical sciences.}'' \\ \cline{2-3}
    % & PID 5 &  \\ \cline{2-3}
    & \pref{29} & \footnotesize The participant had previously generated all images of White people for the prompt ``a Jewish person'' using \baseline. After \diverse~produced images of Black people, the participant felt that the images were not appropriate but became confused upon further reflection: ``\textit{Okay, I'm focused on the Black [...] person here because, this is because my lack of knowledge [...] I don't know what makes a person Jewish other than the religious element [but...] to be Jewish, I think you have to be born in a family [...] I could be wrong.}'' \\ \hline
\multirow{3}{*}{\rotatebox{90}{\reformulativebox}} 
    & \pref{10} & \footnotesize With the prompt ``A gun owner,'' \reformulative~generated the suggestion, ``An elderly man polishing his grandfather's antique shotgun.'' Upon reading the suggestion, the participant reacted, ``\textit{I didn't even think about that kind [of] collector
gun owner,}'' before using the suggestion to generate and add an image to their collage. They later reflected that the interface ``\textit{reminded [them] of things [they] forgot about}'' and ``\textit{sparked a new mental image}'' for them. \\ \cline{2-3}
    % & PID 8 &  \\ \cline{2-3}
    & \pref{19} & \footnotesize Given the prompt ``A gun control advocate,'' \reformulative~generated a suggestion of ``A gun control advocate engaging in a community discussion.'' The participant reacted, ``\textit{I like that it's suggested positioning the advocate in a community space, where all of the other images [from previous interfaces] were very portrait-style.}'' They used the suggestion to generate and add an image to their collage, later sharing that they liked how \reformulative~helped them ``\textit{think of things that [they] wouldn't think to ask.}'' \\ \hline
\multirow{3}{*}{\rotatebox{90}{\agonisticbox}}
    % & \pref{6} & \footnotesize After previously choosing not to include a darker-skinned picture of the subject ``Jesus'' produced by \diverse, the participant was presented with two interpretations of Jesus --- Jesus as an olive-skinned Jew and as a red-haired man (from Islamic records) --- both of which they ended up accepting after much thought and reading the respective Wikipedia pages. They reflected: ``\textit{It's really interesting knowing there are so many interpretations of who Jesus is, I think I only thought about how I got taught in the Bible at my Church, but I never really thought about other people think about who Jesus is.}'' \\ \cline{2-3}
    & \pref{5} & \footnotesize After interacting with \agonistic~on the subject ``World War II'' which showed many less well-known and more local actualizations of WWII, the participant (whose family was strongly impacted by WWII) revised their design statement from ``\textit{WWII involved mass conflicts that...}'' to ``\textit{WWII involved conflicts that...}''. They reflected: ``\textit{I think it just involves conflicts in general now. It's making me rethink that word [mass]: history tends to focus on real large-scale things, but I find that all of them are important to me personally. I feel like it's actually made me reconnect to the way I think about World War II. And instead of trying to project what I would like, think that other people need to see about World War II.}'' \\ \cline{2-3}
    & \pref{28} & \footnotesize The participant initially described their mental image of the prompt ``the signing of the Declaration of Independence'' as ``\textit{A group of [...] white men like Thomas Jefferson standing around a table signing a document.}'' When \agonistic~generated images of non-White people, drawing from sources like ``Haitian Declaration of Independence,'' the participant observed, ''\textit{when [...] I put  Declaration of Independence, I assume a mental image of the US Declaration of Independence, but so many nations have Declarations of Independence.}'' They continued, ``\textit{It makes me realize [...] this assumption of bias... Oh my gosh, I was so narrow-minded in my approach.}'' \\ \hline
\end{tabularx}
\caption{Two examples of reflection observed during our interviews when using each interface.}
% \caption{\andre{We will probably just do two examples of reflection from each interface. I think we do want to have concrete examples of what reflection looks like in the main paper and not all in the appendix though} \amy{somewhat hard to tell what this table is about, I suggest renaming "Example".}}
\label{tab:qualitative-examples}
\end{table}



% Some examples of reflection being induced by various interfaces

% Some examples of reflection \textit{not} being induced

% Need to do 

% \begin{wraptable}{r}{0.35\textwidth} % r for right, 0.5\textwidth for table width
% \vspace{-\baselineskip}
% \centering
% \small % Makes the table compact
% \setlength{\tabcolsep}{4pt} % Adjusts column spacing
% \begin{tabular}{@{}lcc@{}}
% \toprule
% \textbf{Interface} & \textbf{Rethinking} & \textbf{Min-Max Scaled} \\ \midrule
% \agonistic~        & $2.97 \pm 1.19$     & $0.78 \pm 0.39$         \\
% \reformulative~     & $1.93 \pm 1.05$     & $0.37 \pm 0.41$         \\
% \diverse~          & $1.48 \pm 0.93$     & $0.17 \pm 0.35$         \\
% \baseline~         & $2.03 \pm 1.00$     & $0.43 \pm 0.44$         \\ \bottomrule
% \end{tabular}
% \caption{Rethinking scores (5-point Likert-style scale) and min-max scaled scores (scaling done per-participant).}
% \label{tab:rethinking_scores}
% \vspace{-\baselineskip}
% \end{wraptable}

% \begin{table}[!t]
% \vspace{-\baselineskip}
% \centering
% \small % Compact font size
% \setlength{\tabcolsep}{6pt} % Adjust column spacing
% \begin{tabular}{@{}ccccccc@{}}
% \toprule
% & \multirow{2}{*}{\textbf{Interface}} & \multicolumn{4}{c}{\textbf{5-Point Rating}} & \multirow{2}{*}{\textbf{Min-Max Scaled}} \\ \cmidrule(lr){3-6}
% &                                      & \textbf{Overall}   & \textbf{Identity} & \textbf{Politics} & \textbf{History} & \\ \midrule
% & \baselinebox                  & $2.03 \pm 1.00$    & $1.90 \pm 1.14$    & $2.20 \pm 0.87$    & $2.00 \pm 0.94$   & $0.43 \pm 0.44$ \\
% & \diversebox                   & $1.48 \pm 0.93$    & $1.50 \pm 1.02$    & $1.60 \pm 1.02$    & $1.33 \pm 0.67$  & $0.17 \pm 0.35$ \\
% & \reformulativebox             & $1.93 \pm 1.05$    & $1.80 \pm 0.98$    & $2.60 \pm 1.11$    & $1.33 \pm 0.47$  & $0.37 \pm 0.41$ \\
% & \agonisticbox                 & $2.97 \pm 1.19$    & $2.90 \pm 0.94$    & $3.00 \pm 1.18$    & $3.00 \pm 1.41$   & $0.78 \pm 0.39$ \\ \bottomrule
% \end{tabular}
% \caption{Comparison of overall, identity, politics, history prompt ratings, and min-max scaled scores across interfaces. Min-max scaling is applied to each participant, with their lowest rating set to 0 and their highest set to 1. \andre{Maybe remove groupings by category?}}
% \label{tab:grouped_ratings}
% \vspace{-\baselineskip}
% \end{table}



\subsection{Change in mental image}
% \assign[Andre]{...} }
\label{change-mental-image}

The change in mental image is reported by participants after they use each interface and is measured by agreement with the statement ``Interacting with the interface made me rethink the mental picture of the subject I had right before using this interface'' on a 5-point Likert-style scale.
% where 1 is ``Not at all; my mental picture stayed the exact same'', 3 is ``Somewhat; my mental picture changed in some ways'', and 5 is ``Entirely; my mental picture is very different''.
We take higher reported changes in mental image to be indicative of more reflection.
Averages are reported in Figure~\ref{fig:mental-image} and Table~\ref{tab:grouped_ratings}. 
% \amy{figure instead? grouped bar chart of avgs with SD lines for a set of the questions? *s for significance}

% \amy{consider a bolded topic sentence in front of each para(s) with the summarized finding.}
\textbf{\agonistic~induces the most reflection as measured by self-reported rethinking.} Participants' mental images change more under \agonistic~than \reformulative~($p \ll 0.01$), and more under \reformulative~than \diverse~($p = 0.05$). Neither \reformulative~nor \diverse~surpass the \baseline (likely due to the fact that participants interact with \baseline~first), but \agonistic~does ($p \ll 0.01$).
% Given that participants interact with \baseline~first, this suggests that only \agonistic~provides a larger \textit{unique} change in mental picture which \textit{exceeds} than the initial \baseline.
% \diverse~induces lower mental change than \baseline~with $p = 0.03$. \amy{confusingly worded}
The `hierarchy' of interfaces by reflection is thus $\left[ \agonistic> \{ \baseline, \reformulative\} > \diverse\right]$.
% These differences are even more pronounced when min-max scaling each participant's responses such that the lowest value across the four interfaces is mapped to 0 and the highest to 1.
A similar `hierarchy' emerges when considering how participants rank interfaces by how much their mental image changed retroactively at the end of the study (see \S\ref{rankings-mental-image-change}), as well as measuring how much participants' ``design statements'' changed after using each interface (see \S\ref{design-statement}). \agonistic~further induces greater unique rethinking based on an analysis of ordering effects (see \S\ref{reflection-ordering}).
% \agonistic~induces more reflection than \reformulative~by a variety of measurements, and likewise \reformulative~over \diverse.

\textbf{Rethinking under \agonistic~is robust to prompt category.} 
Though  \diverse~and \reformulative~see significant drops in rethinking for historical prompts compared to political prompts ($-0.27$ and $-1.27$, respectively) the same drop is not observed for \agonistic~($-0.00$).
% Interestingly, for historical topics, low rethinking happens under \textit{both} \diverse~(1.33) and \reformulative~(1.33), and much higher with \agonistic~(3.00).
% The same rethinking happens under \agonistic~(3.00) for political topics, but also substantially under \reformulative~(2.60).
While all domains have roughly similar rethinking under \agonistic, political subjects lend themselves to more rethinking overall under \diverse~and \reformulative~than historical subjects.
% \amy{maybe also a figure broken down by category? these things are much easier and shorter to eyeball than try to describe, the text is kinda convoluted.}


\subsection{Reflection when adding images}
% \assign[Andrew]{...}}
\label{adding-images}

% \begin{table}[!h]
% \vspace{-\baselineskip}
% \centering
% \small % Compact font size
% \setlength{\tabcolsep}{6pt} % Adjust column spacing
% \begin{tabular}{@{}ccccccc@{}}
% \toprule
% & \multirow{2}{*}{\textbf{Interface}} & \multicolumn{4}{c}{\textbf{Intents}} \\ \cmidrule(lr){3-6}
% &                                      & \textbf{Direct}   & \textbf{Reminder} & \textbf{Expansion} & \textbf{Challenge} & \\ \midrule
% & \baselinebox                  & \textbf{0.97} & $0.00$ & $0.03$ & $0.00$   \\
% & \diversebox                   & \textbf{0.91} & $0.02$ & $0.07$ & $0.00$   \\
% & \reformulativebox             & \textbf{0.87} & $0.04$ & $0.08$ & $0.00$   \\
% & \agonisticbox                 & \textbf{0.62} & $0.12$ & $0.21$ & $0.05$   \\
% \bottomrule
% \end{tabular}
% \caption{Proportion of added images per interface with each type of intent. \amy{explain this more here} \amy{this could be a stacked bar chart instead maybe}}
% \label{tab:grouped_ratings}
% \vspace{-\baselineskip}
% \end{table}

Next, we aim to capture a richer picture of reflection by analyzing the distribution of intent codes across interfaces. 
As explained in \S\ref{measures}, intents code for how user intents changed while adding an image to their collage.
Intent codes are defined as follows: 1) \direct: the user already intended to generate the image; 2) \reminder: the user is reminded of a detail about the prompt they already accepted but forgot; 3) \expansion: the user accepts an image as a different valid interpretation of the prompt than their original intent; 4) \challenge: the user realizes an error or otherwise experiences a significant change in their original intent. \direct~intent represents little to no reflection, whereas \reminder, \expansion, and \challenge~represent increasing degrees of reflection. The distribution of intents across interfaces is shown in Fig. \ref{fig:intents_distributed}
% \amy{forgot now. wonder if you want to move the definitions here instead.}

\textbf{\agonistic~induces the most reflection as measured by quantitative interview analysis.}
Participants replace the most images on average with \agonistic (3.1), followed by \reformulative~(2.9) and \diverse~(1.7).
We observe \direct~intent most frequently with \baseline~($98\%$), followed by \diverse~($95\%$), \reformulative~($90\%$), and a sharp drop in \agonistic~($67\%$). \agonistic~also has the highest proportion of \reminder~and \expansion~of intent across all interfaces ($11\%$ and $19\%$ respectively, compared to the next-highest $4\%$ and $6\%$ for \reformulative), and is the only interface to \challenge~ intent ($4\%$). 
An example of \challenge~intent by \pref{28} is given in Table~\ref{tab:qualitative-examples}. A minor difference with \S\ref{change-mental-image} is the finding that \diverse~induces more frequent reflection than \baseline~(see \S\ref{intent-coding-appendix}).

% The quantitative analysis of intents largely supports and extends the findings from \S\ref{change-mental-image}.
% % \amy{maybe an example here...}

% \begin{wrapfigure}{l}{0.55\linewidth} % 'l' aligns left, and width is set to 50% of the text width
%     \centering
%     \includegraphics[width=\linewidth]{assets/results-visuals/intent-distribution.png}
%     \caption{Breakdown of reported intents when adding images for each interface.}
%     \label{fig:enter-label}
% \end{wrapfigure}






% Changes to the design statement reflect changes in how the user views.
% design statements can also change even when users don't add any images / few images because their view on what they've created changes. so it adds this different dimension to things

% Compare automatic

% Also investigate changes at all vs. no changes

\begin{figure}[!b]
    \begin{minipage}[t]{0.45\textwidth}
        \centering
        \includegraphics[height=4.4cm]{assets/results-visuals/intent-distribution.png}
        \caption{Breakdown of how user intents changed when adding images for each interface (discussed in`\S\ref{adding-images}; Table~\ref{tab:intent-distribution-table}).}
        \label{fig:intents_distributed}
    \end{minipage}
    \hfill
    \begin{minipage}[t]{0.53\textwidth}
        \centering
        \includegraphics[height=4.4cm]{assets/results-visuals/properties.png}
        \caption{Mean responses on a 5-point scale for Rethink, Appropriateness, Control, and Interest (discussed in \S\ref{interface-properties}; Table~\ref{tab:explicit_values_with_std_properties}).}
        \label{fig:properties-visual}
    \end{minipage}
\end{figure}