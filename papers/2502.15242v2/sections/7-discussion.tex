\section{Discussion}
\label{discussion}

\subsection{The inefficacy of inauthentic diversity}
\label{inauthentic-diversity}

% \andre{mine}

% The \diverse~interface was loosely modeled after the reported behavior of the Gemini case, and provides an interesting point of analysis on the question of diversity in image generation interfaces.
Though it is common to conceive of diversity and factuality as conflicting values, our findings suggest that context is more important for authentic diversity.
Some participants preferred \diverse~to \baseline~because they felt that the demographic diversity of \diverse~was appropriate and either more true to representing the subject or more familiar to their experience. 
For example, in response to images of a diverse crowd worshiping Jesus, \pref{6} said,``\textit{It's nice having [...] diverse people, [be]cause I think that's just how it is.}'' Nonetheless, as discussed in \S\ref{why-reject}, many participants echoed concerns about conflicts between diversity and factuality --- diversity's ``factuality tax''~\cite{wang-etal-2023-t2iat} --- that also surfaced in the online reaction to the Gemini case~\cite{milmoandkern2024gemini, robertson2024google}. Yet notions of factuality are seldom so straightforward (\S\ref{discursive-approaches}).
For example, when creating a collage for ``the women's suffrage movement'', \pref{18} was hesitant to include images of non-White women because many American suffragette leaders espoused racist views, but changed their mind after engaging with an interpretation from \agonistic~highlighting the role that Black women played in leading their own suffragette organizations. 
Here, adding the image did not obscure the facts of historical racism but highlighted other relevant aspects in the complex history of the suffragette movement.
Thus although the ``factuality tax'' is perceived as a necessary cost of diversity, it is not clear that image generation must always be thought of along a strict dichotomy between factuality and diversity.
Given that images are situated in a social context, developers should turn attention from a narrow focus on generating only the right kinds of  \textit{images} to also generating the right kinds of \textit{conte(x/n)t} --- something that users in our study found valuable (\S\ref{qualitative-themes}) --- which can give rise to new ways of seeing and understanding the prompt.

% Normatively, this means that ``image'' generation model interfaces should produce more than just ``images'' for reflection or even UX, given that images involve much more than just the visual.
% It also means that not all questions of image generation (e.g., diversity) need to or even should be resolved in terms of producing the right kinds of images, but also the right kind of (possibly non-image) accompanying content, explicitly opening up another dimension of image generation interface interactions.

% \amy{strong point, feels like this could be highlighted better}
Several participants bring up aspects of a less discussed angle with \diverse's approach to diversity: its \textit{authenticity}.
Taking a definition that ``\textit{to say that something is authentic is to say that it is what it professes to be, or what it is reputed to be, in origin or authorship}''~\cite{sep-authenticity}, \textbf{\diverse~can be critiqued not only as unfactual but also insufficiently aligned with the underlying goals of diversity}---such as respecting and representing all groups of people.
In \S\ref{qualitative-themes}, we documented how situating diversity in a political context encourages user perception of authenticity, as further exemplified by \pref{18}'s reflection about the suffragette movement mentioned above.
These findings support the idea that the ``factuality tax'' is an overly restrictive framework to conceptualize \textit{authentic} diversity, which should be understood neither in terms of diversity for the sake of diversity nor diversity rigidly constrained by factuality.
Moreover, our findings that participants invoke \diversity~at higher rates when adding images with \agonistic~ ($45\%$ of images) than \diverse~($24\%$ of images) suggest that agonistic pluralism offers a superior framework for actualizing authentic diversity by engaging the user with relevant political context.  

% \andre{put in unique change in mental image result}
% In combination with our finding that, this suggests that 
% \textit{Agonism helps actualize authentic diversity} in generative AI interfaces because it engages the user with the relevant debates about diversity. In an agonistically designed interface, diversity is not presented on a silver platter, but shown with the messiness and complexity that goes into negotiating how people and things are represented.
% \andre{@Andrew I think it's too verbose, please leave revisions/feedback}



% \andre{@Andrew please help with the argumentation here.}.
% \andre{cite the dicussion of inauthenticity in the diversity qual theme.}

% We argue that authentic diversity isn't only about striving towards some singular notion of ``factual diversity'' (given previously discussed complications about what is ``factual'' in \S\ref{discursive-approaches}), but about \textit{contextualized diversity} --- what is more important than ``getting the facts right'' (for a narrow sense of what ``facts'' are) is \textit{engaging users in the stories told and debates had about the facts and what they are}.
% \diverse~cuts off users from contextualizing debates, whereas \agonistic~engages them with it; therefore, we see that $48\%$ of images added with \agonistic~cite the diversity value whereas only $28\%$ with \diverse~do.
% \andre{Maybe do more interpretative work with the numbers here.}
% In sum, \textit{an agonistic approach may more authentically embody the value of diversity.}


% Note that we code diversity as rebelling against stereotypes and \textit{lack of diversity} as giving into them, which means sometimes people reject from B \textit{because} of "lack of diversity" -- of genuine diversity, where the diversity is giving into another kind of illness of representation.

% We counterpose \diverse~against \agonistic~in this way.

% \andre{@andrew you can bring in some of the agonistic philosophy stuff here}

% ``"I understand what it's trying to do, putting women at the face of this battalion, but to be honest, it was a very patriarchal society back then, and that was not the reality bck then, and the boys who joined the 442nd had to join to prove loyalty for their families, and so I don't that this tells that story... A lot of them are smiling... I do think there was joy in that. But then there's also the harsh reality that the 442nd was almost a necessity for survival for many people."''


% \andre{@andrew agonistic phil here too?}

\subsection{Engaging with users experientially}

Our study suggests that something is missing in a picture in which user engagement with interfaces is driven by the interface's factuality (realism, consistency) and diversity: the role of \textit{familiarity}.
As discussed in \S\ref{why-add} and \S\ref{why-reject}, many images are accepted and rejected because they are un/familiar rather than because they are un/factual or un/diverse.
If we want to steer users towards desired ideals like critical reflection, interfaces should treat them not solely as judges for factuality and diversity, but as people who engage with images on the basis of familiarity and experience.
The design of \reformulative leans into giving users aestheticized and familiar experiences, possibly at the expense of reflection:
``\textit{
% I just think the way that it was provided different variations... 
if there was a change that I wanted to implement in these pictures, but I just didn’t have the words for it, I think [\reformulative]... gave the options. \underline{I didn’t have to... think about what I was looking for}. I could just see examples of what I was looking for}'' (\pref{19}).
Meanwhile, \agonistic~focused not only on producing factual and diverse outputs via Wikipedia-grounded interpretation sources, but also in explicitly engaging with and challenging users' experience and sense of familiarity.
Our findings therefore further illuminate how core principles of agonistic pluralism, which emphasizes the ``\textit{crucial role played by passions and affects in securing allegiance to democratic values.}''~\cite{mouffe2000democraticparadox}.
% can be brought to bear on user design for democratic engagement with AI technologies.


% To date, both computer vision and HCI researchers seem to have focused on factuality (realism, consistency) and diversity as the primary attributes that users want out of image generation systems.
% have tended to view the main ethical tradeoff of image generation as one between diversity and factuality \cite{wan-etal-2024-factuality}.
% Our work highlights that \textit{familiarity plays a much more important role in user interactions} with image generation systems than current research suggests.
% \S\ref{why-add} shows that $42\%$ of images produced with \baseline~are added because they are \textit{familiar} to the user --- the users think is it is similar to something they have seen before, the image matches their personal experiences and assumptions, etc. --- and not necessarily because it is realistic or diverse.
% The gap between the proportion of images added due to familiarity is \textit{highest} under \reformulative~($54\%$) and \textit{lowest} under \agonistic~($39\%$).
% Correspondingly, the proportion of images added under \reformulative~due to realism ($21\%$) and diversity ($39\%$) and much higher under \agonistic~($38\%$ and $48\%$, respectively).







% \S\ref{why-add} suggests roughly that across interfaces, the more images added under familiarity, the fewer added under realism and diversity.
% Moreover, \textbf{different image generation interface designs can change the distribution of values/reasons salient to users} (\S\ref{why-reject}); e.g., a comparatively large number of images added under \reformulative are justified by \familiarity ($53\%$) and few by realism ($24\%$).
% % Similarly, our qualitative analysis of image rejection in \S\ref{why-reject} revealed that participants also often invoked familiarity when \textit{rejecting} images, including in cases where they explicitly acknowledged the factuality of generated images.
% Practically, this means that \textbf{focusing on how users engage with interfaces \textit{experientially}} (on the basis of what is/n't familiar to them, their emotional/affective responses to outputs, etc.) rather than only as judges of realism and factuality 
% Adding images on the basis of familiarity, although certainly not in all cases, often comes at the expense of reflection.
% Therefore, to encourage reflection, the \agonistic~interface incorporates several different design decisions that seek to challenge users with unfamiliar experiences and encourage them to engage with this unfamiliarity.
% As we discuss in \S\ref{qualitative-themes}, design techniques like incorporating text into image generation help to reduce user resistance to unfamiliarity by enhancing the interpretability and control of image generation interfaces --- making the ``unfamiliar familiar'', the ``inappropriate appropriate'' (as discussed in \S\ref{interface-properties}), the ``darkness conscious'' (referencing the epigraph).
% On the other hand, interfaces whose design encourages familiarity --- for users to find what they were already thinking and intending --- tend to induce less reflection.
% \pref{19} remarked while using \reformulative: ``\textit{I just think the way that it was provided different variations... if there was a change that I wanted to implement in these pictures, but I just didn’t have the words for it, I think [\reformulative]... gave the options. I didn’t have to... think about what I was looking for. I could just see examples of what I was looking for.}''
% Our findings therefore further illuminate how core principles of agonistic pluralism, which emphasizes the ``\textit{crucial role played by passions and affects in securing allegiance to democratic values,}''~\cite{mouffe2000democraticparadox} can be brought to bear on user design for democratic engagement with AI technologies.

% These findings provide richer avenues for developers and researchers of AI tools to engage with users on questions of diversity. Beyond making backend interventions to align foundation models with a static baseline of diversity and factuality, developers grappling with questions of diversity might also use HCI work to inform other interventions that work on users' emotional or affective reactions to AI outputs. Our \agonistic~interface, for instance, incorporates a number of different design decisions that seek to challenge users with unfamiliar experiences and encourage them to engage with that unfamiliarity in ethical ways. As we discuss in \S\ref{qualitative-themes}, design techniques like incorporating text into image generation help to reduce user resistance to unfamiliarity by enhancing the interpretability and control of image generation interfaces. Our findings therefore further illuminate how core principles of agonistic pluralism, which emphasizes the ``\textit{crucial role played by passions and affects in securing allegiance to democratic values,}''~\cite{mouffe2000democraticparadox} can be brought to bear on user design for democratic engagement with AI technologies.



% However, our work points to the importance that familiarity plays in mediating user interactions with AI systems beyond diversity and factuality. As discussed in \S\ref{why-add}, participants in our study invoked `Familiarity' for a greater proportion of added images than `Realism' across \textit{all interfaces}. 

% These differences were most pronounced in \reformulative, with participants invoking `Familiarity' for $54\%$ of images compared to invoking `Realism' for $21\%$ of images. Further, our qualitative analysis of image rejection in \S\ref{why-reject} revealed that participants also often invoked familiarity when rejecting images, including in cases where they explicitly acknowledged the factuality of generated images. In short, participants tended to evaluate images based on how well those images comported with their personal assumptions and experiences, regardless of whether they thought those assumptions were strongly indexed to facts or not.

% These findings demonstrate that developers of computational and AI tools for reflection should aim to design in ways that disrupt user familiarity rather than facilitate it. Whereas \reformulative~was designed to allow users to select the most familiar images that aligned with their intentions, \pref{19} reacted by remarking, ``\textit{I didn't have to think about what I was looking for, just see examples.}'' \andrew{This feels like a really obvious point??? So I decided not to include it @Andre}

% \andrew{TODO: make clearer the connection between familiarity and affect, define affect or use another term}

% This means that invoking some basic amount of reflection in people may not be so difficult, since as soon as they are made conscious about their tendency towards familiarity, they tend to self-correct and just do research
% Analyze \reformulative~as representative of the intention actualization paradigm, ``vibes-based'' ``stylistic'' --- tie this to the observation that familiarity is highest for \reformulative~and actually realism is uniquely lowest for \reformulative, also aesthetics highest for \reformulative
% % Asma: "I don't have to think about it" (on reformulative)
% Sam with D: "I have to think through it myself"
% may be helpful: (on \reformulative) \pref{23} ``the somewhat historically accurate elements that we're seeing with D I think. like some of that is lost. You don't really see that as much with this?''


% Can cite Asma on reformulative: ``"If there was a change that I wanted to implement in these, but I didn't have the words for it, interface C gave you the options; I didn't have to think about what I was looking for, just see examples"''

% \subsection{``I don't have to think about it'': Aestheticizied Unreflection }
% \andre{mine}

% \andre{maybe merge with familiarity section below -- there is a peak in familiarity with reformulative, aestheticized}
% C overflowing with inspiration and ideas and satisfaction but not as much rethinking.
% Asma: "I don't have to think about it"
% Sam with D: "I have to think through it myself"
% Indeed, sometimes C induces less reflection (measure) than A because it's so good at matching mental images well whereas D has some other difficulties
% This is the issue with frictionless tools, or tools that are too nice to use, too immediate, give too much control.
% It does end up being important that users think that image generation tools are appropriate after the fact, even though they may not think it is appropriate beforehand.
% But there does need to be a str


% Several participants reported

% Emphasizes we can't have one-size fits all ways and reflection will always be leaking in excess of the plans we as designers have for how to corrale it.
% People find unexpectedness and results in many ways. Interacting with something which is different from them, which encounters some other kind of selfhood or initiative \andre{cite myself lmao}, allows for some maount of reflection, as we see in Table \andre{cite}.
% You can find reflection in many different places. We found meaningful examples of reflection in every interface (see Table).
% This means that...
% \andre{can add some derrida or whatever here, you always encounter something of the other in meaning and representation.}

% uggle, a designing of struggle tools.


\subsection{Limitations and Future Work}
% \andrew{mine}
% We are limited by...

% \andrew{will convert from bullet list into paragraph form later}

% \amy{maybe split up future work and limitations}

\textbf{Interface-level.}
Wikipedia entries are overtly sites of discourse and negotiation, and prone to political, cultural, and other forms of bias~\cite{hube2017bias, umarova2019partisanship}.
Even though our discovery process attempts to find varying views on subjects, this bias may filter into the process in a way that hinders agonistic user engagement.
% \textbf{Interface-level.}
% Although the \agonistic~interface was designed with agonistic principles in mind, there are several limitations.
% Just like images and the narratives told about them, Wikipedia entries are overtly sites of discourse and negotiation, and prone to political, cultural, and other forms of bias~\cite{hube2017bias, umarova2019partisanship} that filtered into \agonistic's research process
% Even though our discovery process attempts to find varying views on subjects, this bias may filter into the process in a way that hinders agonistic user engagement.
% Indeed, these biases were picked up on by some users in our study: for example, \pref{14} said, ``\textit{I think the Wikipedia one [referencing \agonistic] is a little bit dangerous, because you can get a very strong impression from an article that the AI will then pursue, that maybe only one part of that article, and so isn't super nuanced.}''
% \pref{25} said, ``\textit{I feel like on Wikipedia the things that showing me is notable Arabs, right? [...] The collage I'm trying to create is of a person who I can talk to in a daily day to day.}''
% \pref{7} said about \agonistic, ``\textit{depending on the prompt that Wikipedia would pick it sometimes just completely misses the mark because it can just pick a bad snippet essentially, and then get some from freaking World War 2, which is, I think it did at some points.}'' (the prompt was Chinese Communist Revolution)
Future work might better understand the way that knowledge communities (e.g. Wikipedia, Britannica~\cite{Greenstein2014DoEO}) connect with user interfaces that seek to expose users to the social and discursive processes (e.g., Wikipedia page edit history) by which information is produced.
% Search techniques could be incorporated to improve the precision and efficiency of the research process.
Future work on agonistic principles in generative AI interfaces might consider more \textit{explicit and iterative representations of discourse} (e.g., graph-based) rather than \agonistic's summarizing and linear presentation of discourse/interpretations.
% and understanding how more aggressive agonistic design elements might be used while ensuring productive agonistic experiences and keeping in mind the lived experiences that users --- especially from already marginalized communities --- already come to an interface with, possibly via personalization.
% \andre{@andrew any lit you wanted to add here?}
% Moreover, interfaces could be designed to reap the ``best of all worlds'' for the value that users find with each interface, without compromising on the integrity of agonism.

% \andrew{Add limitation/future work of improving on pipeline to be more efficient (for both time and environmental reasons), could do better controversy score, non-AI based relevance filtering, non-AI based subject detection (these are things that the Wikipedia search function does well, for example, that their API doesn't actually give us access to)---maybe incorporate some search techniques here}


\textbf{Study-level.}
Although we tried to mitigate these factors, our study may be prone to novelty and participant response bias~\cite{dell2012yoursbetter}.
Our study favored lower scale (29 participants) abut more high-resolution (1-hour interviews, collecting self-reported measures, task artifacts, qualitative responses, etc.) examinations of user experiences.
Additionally, our sample population was recruited from university students in the United States, all from a single university. 
% This population may be more likely to hold political, social, and other views/values which impact their predisposition to reflection.
Future work might systematically examine how larger-scale populations engage with agonism in generative (image) AI interfaces.

% \amy{maybe say something about novelty bias, participant response bias https://dl.acm.org/doi/abs/10.1145/2207676.2208589}
% \begin{itemize}
%     \item Sample size -- only 29 participants, we favored lower scale and more rich study (1 hour interviews collecting lots of quantiative self-report + behavioral as well as qualitative data) over higher scale and less rich per-sample study
%     \item Sample population -- we chose US college students in a specific area of the country which may be more predisposed to certain political, historical, social etc. views that may have impacted disposition to reflection 
%     \item Coding for only for added images --- some types of reflection aren't captured
%     \item Wikipedia --- limited to notable info, left-leaning bias \andre{cite source}
%     \item use of DALL-E alleged leaked prompt though it seems reasonable
%     \item \andre{@Amy @Ranjay any other limitations worth mentioning?}
% \end{itemize}


% \subsection{Future Directions for Work}
% 
% \andrew{mine}

% \andre{will convert from bullet list into paragraph form later}

% \begin{itemize}
%     \item More research on frontend design decisions -- our main contribution seems to be the backend algorithm, we have some limited frontend elements for agonism.
%     \item More research on backend controversy detection and relevance
%     \item Personalization
%     \item Efficiency (energy and time) --- speed up agonistic workflow
%     \item different data source than Wikipedia
%     \item Combining different approaches for ``best of both worlds''? Several users report X Y Z
% \end{itemize}

% \textbf{Form matters!}
% Many people didn't even bother looking at the Wikipeida sources and still had good reflection.
% For some, Wikipedia was an instrumental part of the reflection. But for others, reading the way we had framed the text as ``You may assume ...'' was helpful.
% \andre{cite such examples in interviews}
% [Who did we ask about this and the format helped them?]
% \textit{Even formal interventions -- that is, with the structure of reflection induction -- can be helpful}
% \andre{future workify this}


% deeper reflection is hard!
% Brief note about how reflection is difficult to do and reiteration that it may not be pleasant to do always, there is a time and place for it, and there may be trade-offs with other values in interface design.
% Reflection on how even D the average is that it changed mental picture in \textit{some ways}; the challenge tag for intent was fairly rare, etc.
% How to do this?

% Also, at broader scale, and trying out other demographics, not just college students, who may already be predisposed to reflection more than average.