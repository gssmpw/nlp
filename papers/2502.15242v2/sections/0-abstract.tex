%% article.
\begin{abstract}
% \andre{remember to modify abstract}
% - Image generation is political
% - Intention -- does not recognize political
% - Response 1: Gemini
% - Response 2: Agonism -- a political structure to understand and deal with political negotiation over image meanings; critical reflection is an important part of agonism.
% - Study details
% - Discussion topics on diversity

Current image generation paradigms prioritize \textit{actualizing user intention} --- ``see what you intend'' --- but often neglect the sociopolitical dimensions of this process. 
As these tools become cornerstones of the media landscape, it is increasingly evident that image generation is \textit{political}, contributing to broader social struggles over visual meaning.
This sociopolitical aspect was highlighted by the March 2024 Gemini controversy, where Gemini faced criticism for inappropriately injecting demographic diversity into user prompts. 
Although the developers sought to redress image generation's sociopolitical dimension by introducing diversity ``corrections,'' their opaque imposition of a standard for ``diversity'' ultimately proved counterproductive.
In this paper, we present an alternative approach: an image generation interface designed to embrace open negotiation along the sociopolitical dimensions of image creation. 
Grounded in the principles of agonistic pluralism (from the Greek \textit{agon}, meaning struggle), our interface actively engages users with competing visual interpretations of their prompts.
Through a lab study with 29 participants, we evaluate our agonistic interface on its ability to facilitate reflection --- engagement with other perspectives and challenging dominant assumptions --- a core principle that underpins agonistic contestation. 
We compare it to three existing paradigms: a standard interface emulating current image generation tools, 
a Gemini-style interface that produces ``diverse'' images, 
and an intention-centric interface that suggests aestheticized prompt refinements before generation.
Our findings demonstrate that the agonistic interface enhances reflection across multiple measures, but also that reflection depends on users perceiving the interface as both appropriate and empowering; introducing diversity without grounding it in relevant political contexts was perceived as inauthentic.
Our results suggest that diversity and user intention should not be treated as opposing values to be balanced. Instead, interfaces can productively surface and navigate tensions between competing perspectives, enabling users to engage with and evolve their intentions in meaningful ways.



% In a lab study with 29 participants, we evaluate our agonistic interface by how it facilitates \textit{reflection} --- engagement with a wide range of alternative perspectives and challenging of personal assumptions that is critical for agonistic contestation.
% We compare it against interfaces representing three common current paradigms: 
% an interface mimicking standard image generation tools, 
% an Gemini-style interface that returns ``diverse'' generations, 
% and an intention-centric interface that suggests aestheticized prompt reformulations to the user before generation.
% We find that the agonistic interface induces significantly more reflection than other
% approaches across multiple measures, including self-reported changes in mental image. 
% However, this reflection appears contingent
% on maintaining user perceptions of appropriateness and control; introducing diverse content without grounding them in the relevant historical and social context felt inauthentic to users. 
% Our work suggests that rather than treating diversity and user intention as competing
% values to be traded off, interfaces can productively surface tensions between different perspectives while engaging and evolving user intentions.

% Recent advances in AI image generation have sparked discussions about representation, bias, and user agency. 
% While current approaches prioritize user intention --- helping users actualize their desired images --- image generation interfaces could also facilitate critical reflection about visual representation. 
% Inspired by the political theory of agonistic pluralism, in this work, we reimagine user interactions during image generation to deliberately surface tensions and competing perspectives in visual representation.
% In a lab study with 27 participants, we compare an ``agonistic'' interface against three common current approaches: an interface mimicking standard image generation tools, an identical interface that returns more diverse generations, and an interface that suggests more detailed prompts to the user before generation.
% We find that simply injecting diversity often feel inauthentic to users but grounding diverse representations in historical and social context results in deeper engagement and reflection. 
% We find that our agonistic interface induced significantly more reflection than other approaches across multiple measures including self-reported changes in mental models.
% However, this reflection appears contingent on maintaining user perceptions of appropriateness  and control --- simply introducing diverse content without grounding them in historical and social context felt inauthentic. 
% Based on these findings, we develop design principles for building image generation interfaces that balance user agency with critical reflection, contributing to ongoing discussions about responsible AI development. 
% Our work suggests that rather than treating diversity and user intention as competing values to be traded off, interfaces can productively surface tensions between different perspectives while still satisfying users' queries.
\end{abstract}