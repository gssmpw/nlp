\newpage
\appendix

\section{Full Study Materials and Information}

\subsection{Participant Backgrounds}
\label{participant-background}

Participants came from the following departments:
Computer Science,
Physics,
Math,
Economics,
Informatics,
Comparative History of Ideas,
English,
History,
Business,
American Ethnic Studies,
Law,
Journalism,
Education,
Library and Information Sciences,
Gender Women and Sexuality Studies,
Communications,
Cinema and Media Studies,
and
Political Science.
Out of the 29 participants in the study, 21 were undergraduate students and 8 were graduate students.

\begin{table}[h!]
\centering
\begin{tabular}{|c|l|l|l|}
\hline
\textbf{PID} & \textbf{Selected Prompt}                             & \textbf{Category} & \textbf{Interface Order}              \\ \hline
\pref{1}  & A Korean person                    & Identity   & \diverse, \reformulative, \agonistic~\\ \hline
\pref{2}  & A middle-class person              & Identity   & \agonistic, \diverse, \reformulative~\\ \hline
\pref{3}  & A designer                         & Identity   & \reformulative, \diverse, \agonistic~\\ \hline
\pref{4}  & A community of queer people        & Identity   & \agonistic, \reformulative, \diverse~\\ \hline
\pref{5}  & World War II                       & Politics   & \reformulative, \agonistic, \diverse~\\ \hline
\pref{6}  & Jesus                              & History    & \diverse, \agonistic, \reformulative~\\ \hline
\pref{7}  & The Chinese Communist Revolution   & History    & \diverse, \agonistic, \reformulative~\\ \hline
\pref{8}  & An Asian person                    & Identity   & \agonistic, \diverse, \reformulative~\\ \hline
\pref{9}  & An immigrant                       & Identity   & \agonistic, \reformulative, \diverse~\\ \hline
\pref{10} & A gun owner                        & Politics   & \diverse, \reformulative, \agonistic~\\ \hline
\pref{11} & The French Revolution              & History    & \agonistic, \reformulative, \diverse~\\ \hline
\pref{12} & A fascist                          & Politics   & \reformulative, \diverse, \agonistic~\\ \hline
\pref{13} & A socialist                        & Politics   & \diverse, \agonistic, \reformulative~\\ \hline
\pref{14} & The Israel-Palestine Conflict      & Politics   & \reformulative, \agonistic, \diverse~\\ \hline
\pref{15} & A mass shooter                     & Politics   & \diverse, \reformulative, \agonistic~\\ \hline
\pref{16} & A gun control advocate             & Politics   & \reformulative, \diverse, \agonistic~\\ \hline
\pref{17} & World War II                       & History    & \agonistic, \diverse, \reformulative~\\ \hline
\pref{18} & The women's suffrage movement      & History    & \agonistic, \diverse, \reformulative~\\ \hline
\pref{19} & A Kamala supporter                 & Politics   & \reformulative, \diverse, \agonistic~\\ \hline
\pref{20} & Lord Krishna                       & History    & \diverse, \reformulative, \agonistic~\\ \hline
\pref{21} & An activist                        & Identity   & \agonistic, \reformulative, \diverse~\\ \hline
\pref{22} & A Democrat                         & Politics   & \reformulative, \agonistic, \diverse~\\ \hline
\pref{23} & A Roman gladiator                  & History    & \agonistic, \diverse, \reformulative~\\ \hline
\pref{24} & The Constitutional Convention      & History    & \reformulative, \agonistic, \diverse~\\ \hline
\pref{25} & An Arab person                     & Identity   & \reformulative, \diverse, \agonistic~\\ \hline
\pref{26} & A Syrian refugee                   & Politics   & \diverse, \reformulative, \agonistic~\\ \hline
\pref{27} & A Christian person                 & Identity   & \diverse, \agonistic, \reformulative~\\ \hline
\pref{28} & Declaration of Independence Signing& History    & \agonistic, \reformulative, \diverse~\\ \hline
\pref{29} & A Jewish person                    & Identity   & \diverse, \agonistic, \reformulative~\\ \hline
\end{tabular}
\caption{Participant IDs with their selected prompt, category, and the order in which they encountered each of the non-\baseline~interfaces.}
\label{tab:prompts}
\end{table}

\newpage


\subsection{Full Topic List}
\label{topic-list}

Participants were provided the following list of example topics/subjects within each of the three categories.

\begin{itemize}[label=$\bullet$, leftmargin=1.5em]

    \item \textbf{Identity \& Demographics}
    \begin{itemize}[label=$\circ$]
        \item \textit{Race:} ``a Black person'', ``a White person'', ``an Asian person''
        \item \textit{Ethnicity:} ``a Hispanic person'', ``a Jewish person'', ``an Arab person''
        \item \textit{Nationality:} ``an American person'', ``a Chinese person'', ``a Brazilian person'', ``a Swedish person'', ``a South African person''
        \item \textit{Class:} ``a working class person'', ``a middle class person'', ``a member of the elite''
        \item \textit{Family history:} ``an immigrant'', ``a refugee'', ``an indigenous person''
        \item \textit{Occupation:} ``a nurse'', ``a student'', ``a businessperson'', ``a CEO'', ``a/the Pope''
        \item \textit{Religion:} ``a Christian person'', ``a Muslim person'', ``a Buddhist person''
        \item \textit{LGBTQ+:} ``a transgender person'', ``two men getting married''
    \end{itemize}

    \item \textbf{History}
    \begin{itemize}[label=$\circ$]
        \item \textit{Historical people:} ``a Founding Father'', ``a Nazi soldier'', ``Jesus'', ``the prophet Muhammad'', ``a Roman gladiator''
        \item \textit{Historical events:} ``World War I'', ``the French Revolution'', ``the Holocaust'', ``the Great Depression'', ``the American Civil War'', ``the Chinese Communist Revolution'', ``the fall of the Berlin Wall'', ``the Trail of Tears'', ``D-Day'', ``The Declaration of Independence Signing'', ``the Nakba'', ``the Napoleonic Wars''
    \end{itemize}

    \item \textbf{Politics}
    \begin{itemize}[label=$\circ$]
        \item \textit{Ideological orientations:} ``a Republican'', ``a Democrat'', ``a rightist'', ``a leftist'', ``a centrist'', ``a Trump supporter'', ``a Harris supporter'', ``a libertarian'', ``a progressive'', ``a nationalist'', ``an anarchist'', ``a socialist'', ``a capitalist'', ``a communist'', ``a fascist'', ``an authoritarian''
        \item \textit{Current conflicts and issues:}
        \begin{itemize}[label=$-$]
            \item \textit{Geopolitical:} ``the Israel-Palestine conflict'', ``the Ukraine-Russia war'', ``a protester in Hong Kong'', ``a Kurdish fighter'', ``a Syrian refugee''
            \item \textit{Gun Rights and Policing:} ``a mass shooter'', ``a gun owner'', ``a gun control advocate'', ``a mass shooting survivor'', ``a police officer in a high-crime area'', ``a police brutality victim''
            \item \textit{Immigration and Borders:} ``a migrant crossing the US border'', ``a border patrol agent''
        \end{itemize}
    \end{itemize}

\end{itemize}


\newpage


\subsection{Full Survey Questions}
\label{survey-questions}

We ask users to provide responses on a 5-point Likert-style scale (1 $=$ ``not at all'', 3 $=$ ``somewhat'', 5 = ``entirely'') after interacting with each interface for each of the following statements:

\begin{itemize}
    \item \textit{Satisfaction}: ``I am satisfied with the collage I created''
    \item \textit{Rethinking}: ``Interacting with the interface made me rethink the mental picture of the subject I had right before using this interface''
    \item \textit{Appropriateness}: ``I thought the content generated from the interface (e.g. images, suggestions) was appropriate to the prompt''
    \item \textit{Control}: ``I feel like the interface gives me detailed control over the image generation process''
    \item  \textit{Interest}: ``I found the content of the suggestions interesting''
\end{itemize}

At the end of the study, participants were also asked to rank each of the interfaces by agreement with the rethinking, appropriateness, and control statements.
This provided another data signal as well as a moment for participants to reflect upon the entire interview and make open-ended comments.
Participants were allowed to give two interfaces the same rank if there were strong ties.


% \section{Additional Examples from User Studies}

% \subsection{Design Statements}
% \label{app-design-statements}

% \andre{Add qualitative investigation of change from initial mental image to interface A (in section on change in designer's statement as final addendum) as a brief note on how just interacting with this thing for the first time can change mental picture}


% \section{Further Analysis}

% \subsection{Orderwise Comparisons}
% \label{orderwise-comparisons}

% \subsection{Satisfaction}


% \section{"Confounder Variables"}

\newpage

\section{Detailed View of Interfaces}
\label{detailed-view-interfaces}

\subsection{Design Iteration for \agonistic}
\label{design-iteration}
In the first iteration of \agonistic, we generated different set of suggestions for each subject in the user's prompt (an example with only one subject is shown in \ref{fig:agonistic-v0}). Finding that generating suggestions for multiple subjects was too costly and slow for interview purposes, we stopped organizing suggestions by subject and instead only generated suggestions for the main subject of the prompt, as shown in \ref{fig:interface-screenshots}.

We also asked users to input their ``mental image'' of their subject before generating images in the initial iteration of \agonistic, for the purpose of generating suggestions that challenged the user's mental image. Since we found this feature to be prohibitively difficult to use when regenerating suggestions or changing prompts, however, we removed it in favor of using LLM-generated mental image descriptions for the same purpose, as detailed in \S\ref{paradigms-interfaces}.

\begin{figure}[ht]
    \centering
    \includegraphics[width=0.75\textwidth]{assets/appendix/agonistic-v0.png}
    \caption{First iteration of \agonistic~interface}
    \label{fig:agonistic-v0}
\end{figure}

\newpage

\subsection{\diverse}

We use GPT-3.5 with the full alleged leaked DALL-E system prompt~\cite{milmoandkern2024gemini} listed below to rewrite the user's prompt four times before passing each rewritten prompt into the image generation model.

\begin{quote}
You are given the prompt "{prompt}."
Diversify depictions of people in this prompt to always include always DESCENT and GENDER for EACH person using direct terms.
Adjust only human descriptions.
EXPLICITLY specify these attributes, not abstractly reference them.
The attributes should be specified in a minimal way and should directly describe their physical form.
Your choices should be grounded in reality. For example, all of a given OCCUPATION should not be the same gender or race.
Additionally, focus on creating diverse, inclusive, and exploratory scenes via the properties you choose during rewrites.
Make choices that may be insightful or unique sometimes.
Use "various" or "diverse" ONLY IF the description refers to groups of more than 3 people.
Do not change the number of people requested in the original description.
Don't alter memes, fictional character origins, or unseen people.
Maintain the original prompt's intent and prioritize quality.
Do not create any imagery that would be offensive.
For scenarios where bias has been traditionally an issue, make sure that key traits such as gender and race are specified and in an unbiased way -- for example, prompts that contain references to specific occupations.
\end{quote}

\newpage
\subsection{\reformulative}

Users begin by entering their prompt in the prompt box at the top of the interface.

\begin{figure}[!h]
    \centering
    \includegraphics[width=0.6\linewidth]{assets/detailed-view/reformulative/a.png}
    % \caption{Caption}
    % \label{fig:enter-label}
\end{figure}

\noindent
When they press enter or click on the $\gg$ symbol on the prompt box, reformulations are generated and displayed to the user in the ``Suggestions'' box to the left of the interface.
Each reformulation includes a thumbnail to visualize how the reformulated prompt might be visualized.

\begin{figure}[!h]
    \centering
    \includegraphics[width=0.6\linewidth]{assets/detailed-view/reformulative/b.png}
\end{figure}

\newpage

When users click on a suggestion, a new ``Prompt Workspace'' pane opens.
In this pane, users can edit the reformulation in a text box.
Clicking the ``Generate images'' button generates images using the prompt in the text box.
The images are persistent and stay on the panel even when the user clicks the $\gg$ button in the Prompt Workpace Pane and returns to the Suggestions list.

\begin{figure}[!h]
    \centering
    \includegraphics[width=0.6\linewidth]{assets/detailed-view/reformulative/c.png}
    % \caption{Caption}
    % \label{fig:enter-label}
\end{figure}

% The full prompt is available here: ...


\newpage
\subsection{\agonistic}


Users begin by entering their prompt in the prompt box at the top of the interface.

\begin{figure}[!h]
    \centering
    \includegraphics[width=0.6\linewidth]{assets/detailed-view/agonistic/a.png}
\end{figure}

\noindent
When they press enter or click on the $\gg$ symbol on the prompt box, interpretations are generated and displayed to the user in the ``Possible Interpretations'' box to the left of the interface.
Each card includes a decription of the interpretation, a source including the Wikipedia page title and section, and a thumbnail.

\begin{figure}[!h]
    \centering
    \includegraphics[width=0.6\linewidth]{assets/detailed-view/agonistic/b.png}
\end{figure}


\newpage 


When users click on a card, it expands and a justification of the form ``\textit{You may assume [X], but [Y]}'' is displayed, along with a clickable link to the Wikipedia page and section the interpretation references.
After three seconds, an ``Accept'' button appears.

\begin{figure}[!h]
    \centering
    \includegraphics[width=0.6\linewidth]{assets/detailed-view/agonistic/c.png}
\end{figure}

Clicking the ``Accept'' button opens the Prompt Workspace pane.
In this pane, users can edit the reformulation in a text box.
Clicking the ``Generate images'' button generates images using the prompt in the text box.
The images are persistent and stay on the panel even when the user clicks the $\gg$ button in the Prompt Workpace Pane and returns to the Possible Interpretations list.

\begin{figure}[!h]
    \centering
    \includegraphics[width=0.6\linewidth]{assets/detailed-view/agonistic/d.png}
\end{figure}


% We first instruct GPT-4o to generate 5 mental images an ``average person'' might have of the main subject and provide this in-context to the interpretation generation call to reference in the \textit{justification} field.
% For each page, GPT-4o then selects 4 sections based on their titles to read and produces an \textit{interpretation} with 4 fields: 1) a \textit{section summary} explaining the main points of the cited page section; 2) a \textit{description} of what the user's main subject (e.g., ``Jesus'') looks like (e.g., ``\textit{a typical 1st-century Jewish man, with olive-brown skin, brown eyes, and short dark hair}''), taking into account the user's full prompt; 3) a \textit{source} with the referenced page and section (e.g., ``\textit{Jesus -- Language, ethnicity, and appearance}''); 4) a \textit{justification} explaining how the section content justifies the description (e.g., ``\textit{you may assume that Jesus is solely a spiritual figure in church communities, but he also was a real Jewish man with specific physical characteristics}'') (The examples are taken from \pref{6}.)
% The section summary is not shown to the user but is generated first to reduce model hallucinations.

\newpage


% \section{Research Methods}

% \subsection{Part One}

% Lorem ipsum dolor sit amet, consectetur adipiscing elit. Morbi
% malesuada, quam in pulvinar varius, metus nunc fermentum urna, id
% sollicitudin purus odio sit amet enim. Aliquam ullamcorper eu ipsum
% vel mollis. Curabitur quis dictum nisl. Phasellus vel semper risus, et
% lacinia dolor. Integer ultricies commodo sem nec semper.

% \subsection{Part Two}

% Etiam commodo feugiat nisl pulvinar pellentesque. Etiam auctor sodales
% ligula, non varius nibh pulvinar semper. Suspendisse nec lectus non
% ipsum convallis congue hendrerit vitae sapien. Donec at laoreet
% eros. Vivamus non purus placerat, scelerisque diam eu, cursus
% ante. Etiam aliquam tortor auctor efficitur mattis.

% \section{Online Resources}

% Nam id fermentum dui. Suspendisse sagittis tortor a nulla mollis, in
% pulvinar ex pretium. Sed interdum orci quis metus euismod, et sagittis
% enim maximus. Vestibulum gravida massa ut felis suscipit
% congue. Quisque mattis elit a risus ultrices commodo venenatis eget
% dui. Etiam sagittis eleifend elementum.

% Nam interdum magna at lectus dignissim, ac dignissim lorem
% rhoncus. Maecenas eu arcu ac neque placerat aliquam. Nunc pulvinar
% massa et mattis lacinia.

\section{Inter-Rater Reliability Methodology}
\label{irr-methodology}
We calculated inter-rater reliability (IRR) on value codes. the two first co-authors first independently coded a subset of 3 participant interviews using the coding ontology described in \S\ref{measures}. Since some codes were not mutually exclusive with each other, we calculated a separate Cohen's kappa score for each code, with the following results: 0.64 (\direct), 0.00 (\reminder), 0.65 (\expansion), 1.00 (\challenge), 0.66 (\realism), 0.83 (\familiarity), 0.60 (\diversity), 0.38 (\aesthetics) We then took an average of Cohen's kappa scores across value codes, weighted by the frequency of each value code, to calculate our final IRR of 0.67.

We noticed that the Cohen's kappa score for several codes might be unreliable to the low frequency of certain codes in the subset of interviews used to calculate IRR (for instance, there were only 3 instances of \reminder~ and 1 instance of \challenge~). This issue was especially severe for intent codes, due to the extremely high proportion of \direct~codes compared to other intents. For this reason, two first co-authors then reviewed all non-\direct intent codes together to establish greater agreement. We note that while this approach reduced the chance of false positives for non-\direct intent codes, it did not reduce the chance of false negatives. Therefore, the amount of reflection captured by our coding in this study is a conservative estimate of the amount of reflection that may have occurred.

\section{Extended Results}

\subsection{Rankings for change in mental image}
\label{rankings-mental-image-change}

At the end of the study, participants are asked to rank the interfaces by how much each made them rethink their mental picture (compared to right before using it), allowing equal ranks between strongly tied interfaces.
This ranking allows participants to rank the interfaces with the benefit of retrospection (e.g., participants may not realize how interacting with an interface changed their mental picture until later, the scale may have been disrupted) and provides another view into reflection.
\agonistic~was ranked higher than \reformulative~by $79\%$ of participants, \diverse~by $86\%$, and \baseline~by $89\%$.
\reformulative~was ranked higher than \diverse~by $70\%$ and \baseline~by $68\%$; \diverse~higher than \baseline~by $52\%$.
This suggests that participants often believe \reformulative~made them rethink their mental picture, although they may not have realized this at the moment.
\diverse~and \baseline~are ranked above each other by $50\%$ of participants and therefore are overall similar in retroactive attribution towards reflection.
% \andre{Not sure how important this paragraph on ranking is.}

\subsection{Interface Ordering Analysis}
\label{reflection-ordering}
Considering how much the mental image changed given the ordering of the interfaces also allows us to more closely examine the \textit{unique} changes in mental image that each interface offers.
That is, is the kind of reflection induced by one interface (roughly) a subset of the reflection induced by another?
Participants interacting with \diverse~\textbf{before} \agonistic~report 1.67 for \diverse, but this falls by 0.38 or 23$\%$ to 1.29 \textbf{after} \agonistic.
Meanwhile, participants interacting with \agonistic~\textbf{before} \diverse~report 3.07 for \agonistic, falling by 0.22 or 7$\%$ to 2.87 \textbf{after} \diverse.
This indicates that much of the rethinking provided by \diverse~is accounted for (that is, encompassed by) by \agonistic, but not vice versa does not hold (\agonistic~offers more unique rethinking).
% The full comparisons are available in \S\ref{orderwise-comparisons}.
% \amy{feels like a minor point, could maybe be dropped?}

\subsection{Change in design statement}
% \assign[Andre]{...}}
\label{design-statement}

% \andre{@Amy @Ranjay may axe this section and move to supplementary?}

Participants write an ``design statement'' --- one or two sentences describing the most important aspects of the subject represented in the collage -- after producing an initial collage with \baseline.
The design statements provided both a more standardized and a richer way to record the progression of participant assumptions over different interfaces beyond the data collected by the self-reported rankings.
After iterating through each following interface, participants copied over their previous design statement and made modifications to reflect the changes to their collage. They could opt to make no changes even if images were replaced, should they feel that the previous design statement satisfactorily described the new collage.
% Participants write an initial design statement reflecting the most important choices they made in putting together their collage with \baseline, then update the design statement after interacting with each other interface.
The design statement reflects a different dimension of reflection than either the change in mental image or how images are added:
for example,
some participants did not report a substantive changes in mental picture of the \textit{subject} but still modified the design statement to reflect a change in how they viewed their collage, like by emphasizing some aspects of the collage;
some participants also replaced images in the collage but did not change the design statement because they felt that it applied just as well (the changes fell within the scope of the previous design statement).
The design statement uniquely captures the self-perception of the participant as a collage designer synthesizing their choices.

\begin{table}[!h]
% \vspace{-\baselineskip}
\centering
\small % Compact font size
\setlength{\tabcolsep}{6pt} % Adjust column spacing
\begin{tabular}{@{}cccccc@{}}
\toprule
& \multirow{2}{*}{\textbf{Interface}} & \multicolumn{2}{c}{\textbf{Levenshtein}} & \multicolumn{2}{c}{\textbf{Embeddings}} \\ \cmidrule(lr){3-4} \cmidrule(lr){5-6}
&                 & \textbf{Raw}   & \textbf{Scaled} & \textbf{Raw}   & \textbf{Scaled} \\ \midrule
& \diversebox        & 6.72  & 0.27 & 0.02 & 0.28  \\
& \reformulativebox  & 9.24  & 0.38 & 0.05 & 0.32 \\
& \agonisticbox      & 12.31 & 0.57 & 0.08 & 0.56 \\\bottomrule
\end{tabular}
\caption{Mean change in participant design statement after using each interface, as measured by Levenshtein and embedding-based distance. ``Scaled'' indicates 0-1 min-max scaling, where the smallest change is mapped to 0 and the highest is mapped to 1.}
\label{tab:design_statement_changes}
% \vspace{-\baselineskip}
% \vspace{-\baselineskip}
\end{table}

We can measure reflection under each interface via change in design statement by computing the mean text distance after using an interface.
One purely syntactic metric is Levenshtein distance across words, which measures the number of insertions, deletions, and additions needed to transform one text to another.
Another metric that captures more semantic information is to compare the cosine similarity between the BERT representations of the text samples.
In each instance, \agonistic~has a higher level of change than \diverse~($p \ll 0.05$ except for Levenshtein raw, where $p = 0.06$).
The relationship between \reformulative~and either \diverse~or \agonistic~is not statistically significant, except for scaled embeddings, in which \agonistic~is larger than \reformulative~with $p = 0.04$.
% \andre{Not sure how interesting this stuff is to report, also not sure how convincing or needed the embeddings results are, maybe Levenshtein is enough.}
% This suggests that \reformulative~tends to induce a substantive amount of thought or cognition about the design objective, even if it doesn't actually result in a substantive change in mental image or in replacing actions.
% We also include some representative qualitative examples of design statements in the appendix \S\ref{app-design-statements}. \amy{could drop or shorten significantly}
% \andre{revisit interpretative work here}

\newpage

\subsection{Example Collage Progression}

Figure~\ref{fig:example-collage-progression} shows an example of a collage progression from \pref{6} for the subject ``Jesus''.
Observe that under \diverse, the participant adds images featuring demographically diverse individuals worshipping Jesus.
After previously choosing not to include a darker-skinned picture of the subject ``Jesus'' produced by \diverse, the participant was presented with two interpretations of Jesus by \agonistic --- Jesus as an olive-skinned Jew and as a red-haired man (from Islamic records) --- both of which they ended up accepting after much thought and reading the respective Wikipedia pages.
They reflected: ``\textit{It's really interesting knowing there are so many interpretations of who Jesus is, I think I only thought about how I got taught in the Bible at my Church, but I never really thought about other people think about who Jesus is.}''
Under \reformulative, the participant added several images that were closer to what they originally imagined Jesus to look like, notably with two of the images in an illustration style.

\begin{figure}[ht]
    \centering
    % All images in one row
    \begin{subfigure}[b]{0.24\textwidth}
        \centering
        \includegraphics[width=\textwidth]{assets/collage-spread/ss-1.png}
        \caption{\baselinebox}
        \label{fig:image1}
    \end{subfigure}
    \hfill
    \begin{subfigure}[b]{0.24\textwidth}
        \centering
        \includegraphics[width=\textwidth]{assets/collage-spread/ss-2.png}
        \caption{ \diversebox}
        \label{fig:image2}
    \end{subfigure}
    \hfill
    \begin{subfigure}[b]{0.24\textwidth}
        \centering
        \includegraphics[width=\textwidth]{assets/collage-spread/ss-3.png}
        \caption{ \agonisticbox}
        \label{fig:image3}
    \end{subfigure}
    \hfill
    \begin{subfigure}[b]{0.24\textwidth}
        \centering
        \includegraphics[width=\textwidth]{assets/collage-spread/ss-4.png}
        \caption{\reformulativebox}
        \label{fig:image4}
    \end{subfigure}

    \caption{Example collage progression (left-to-right). Images added/replaced in each new round are fully shown; kept images are faded.}
    % \andre{can probably make just a bit smaller} \amy{is this the best example? also wonder if the first figure should introduce the agonistic interface more instead/intro all the interfaces}}
    \label{fig:example-collage-progression}
\end{figure}


\newpage

\subsection{Table Form for Results}



\begin{table}[!h]
% \vspace{-\baselineskip}
\centering
\small % Compact font size
\setlength{\tabcolsep}{6pt} % Adjust column spacing
\begin{tabular}{@{}ccccccc@{}}
\toprule
& \multirow{2}{*}{\textbf{Interface}} & \multicolumn{4}{c}{\textbf{5-Point Rating}} & \multirow{2}{*}{\textbf{Min-Max Scaled}} \\ \cmidrule(lr){3-6}
&                                      & \textbf{Overall}   & \textbf{Identity} & \textbf{Politics} & \textbf{History} & \\ \midrule
& \baselinebox                  & $2.03 \pm 1.00$    & $1.90 \pm 1.14$    & $2.20 \pm 0.87$    & $2.00 \pm 0.94$   & $0.43 \pm 0.44$ \\
& \diversebox                   & $1.48 \pm 0.93$    & $1.50 \pm 1.02$    & $1.60 \pm 1.02$    & $1.33 \pm 0.67$  & $0.17 \pm 0.35$ \\
& \reformulativebox             & $1.93 \pm 1.05$    & $1.80 \pm 0.98$    & $2.60 \pm 1.11$    & $1.33 \pm 0.47$  & $0.37 \pm 0.41$ \\
& \agonisticbox                 & $2.97 \pm 1.19$    & $2.90 \pm 0.94$    & $3.00 \pm 1.18$    & $3.00 \pm 1.41$   & $0.78 \pm 0.39$ \\ \bottomrule
\end{tabular}
\caption{Comparison of overall, identity, politics, history prompt ratings, and min-max scaled scores across interfaces (mean $\pm$ standard deviation). Min-max scaling is applied to each participant, with their lowest rating set to 0 and their highest set to 1. Visualized in Figure~\ref{fig:mental-image}.}
\label{tab:grouped_ratings}
\vspace{-\baselineskip}
\end{table}

\begin{table}[!h]
% \vspace{-\baselineskip}
\centering
\small % Compact font size
\setlength{\tabcolsep}{6pt} % Adjust column spacing
\begin{tabular}{@{}cc|c|ccc@{}}
\toprule
& \textbf{Interface} & \textbf{Rethink} & \textbf{Appropriateness} & \textbf{Control} & \textbf{Interestingness} \\ \midrule
& \baselinebox        & $2.03 \pm 1.00$ & $3.31 \pm 0.91$ & $3.03 \pm 1.22$ & --- \\
& \diversebox         & $1.48 \pm 0.93$ & $3.07 \pm 1.20$ & $2.59 \pm 1.16$ & --- \\
& \reformulativebox   & $1.93 \pm 1.05$ & $3.93 \pm 0.91$ & $3.79 \pm 1.21$ & $4.00 \pm 1.11$ \\
& \agonisticbox       & $2.97 \pm 1.19$ & $3.83 \pm 1.15$ & $3.90 \pm 0.99$ & $4.28 \pm 0.91$ \\ \bottomrule
\end{tabular}
\caption{Comparison of rethink, appropriateness, control, and interestingness (mean \(\pm\) standard deviation) across interfaces. Visualized in Figure~\ref{fig:properties-visual}.}
\label{tab:explicit_values_with_std_properties}
% \vspace{-\baselineskip}
\end{table}

\begin{table}[!h]
% \vspace{-\baselineskip}
\centering
\small % Compact font size
\setlength{\tabcolsep}{6pt} % Adjust column spacing
\begin{tabular}{@{}ccccccc@{}}
\toprule
& \multirow{2}{*}{\textbf{Interface}} & \multicolumn{4}{c}{\textbf{Values}} \\ \cmidrule(lr){3-6}
&                                      & \textbf{Realism}   & \textbf{Familiarity} & \textbf{Diversity} & \textbf{Aesthetics} \\ \midrule
& \baselinebox                  & $0.40$ & $0.44$ & $0.26$ & $0.08$   \\
& \diversebox                   & $0.42$ & $0.47$ & $0.24$ & $0.05$   \\
& \reformulativebox             & $0.24$ & $0.53$ & $0.38$ & $0.10$   \\
& \agonisticbox                 & $0.39$ & $0.42$ & $0.45$ & $0.08$   \\
\bottomrule
\end{tabular}
\caption{Comparison of the proportion of images added under each value, across interfaces. Rows do not sum to 1 because values are non-exclusively coded. Visualized in Figure~\ref{fig:values-interfaces}.}
\label{tab:values-interfaces}
% \vspace{-\baselineskip}
\end{table}



\begin{table}[!h]
% \vspace{-\baselineskip}
\centering
\small % Compact font size
\setlength{\tabcolsep}{6pt} % Adjust column spacing
\begin{tabular}{@{}ccccccc@{}}
\toprule
& \multirow{2}{*}{\textbf{Interface}} & \multicolumn{4}{c}{\textbf{Intents}} \\ \cmidrule(lr){3-6}
&                                      & \textbf{Direct}   & \textbf{Reminder} & \textbf{Expansion} & \textbf{Challenge} & \\ \midrule
& \baselinebox                  & \textbf{0.98} & $0.00$ & $0.02$ & $0.00$   \\
& \diversebox                   & \textbf{0.95} & $0.00$ & $0.05$ & $0.00$   \\
& \reformulativebox             & \textbf{0.90} & $0.04$ & $0.06$ & $0.00$   \\
& \agonisticbox                 & \textbf{0.67} & $0.11$ & $0.19$ & $0.04$   \\
\bottomrule
\end{tabular}
\caption{Comparison of the proportion of images added with each type of intent, across interfaces. Rows sum to 1. Visualized in Figure~\ref{fig:intents_distributed}.}
\label{tab:intent-distribution-table}
% \vspace{-\baselineskip}
\end{table}


\begin{table}[!h]
% \vspace{-\baselineskip}
\centering
\small % Compact font size
\setlength{\tabcolsep}{6pt} % Adjust column spacing
\begin{tabular}{@{}cccccc@{}}
\toprule
& \multirow{2}{*}{\textbf{Values}} & \multicolumn{4}{c}{\textbf{Intents}} \\ \cmidrule(lr){3-6}
&                                      & \direct   & \reminder & \expansion & \challenge \\ \midrule
& \realism                    & \textbf{0.879}    & $0.047$           & $0.065$            & $0.009$            \\
& \familiarity                & \textbf{0.954}    & $0.031$           & $0.015$            & $0.000$            \\
& \diversity                  & \textbf{0.753}    & $0.067$           & $0.157$            & $0.022$            \\
& \aesthetics                 & \textbf{0.960}    & $0.000$           & $0.040$            & $0.000$            \\
\bottomrule
\end{tabular}
\caption{Comparison of the proportion of ratings across values and intents. Rows sum to 1. Visualized in Figure~\ref{fig:intents-values}.}
\label{tab:values-intents-rel}
% \vspace{-\baselineskip}
\end{table}


\subsection{Quantitative Coding Analysis}
\label{intent-coding-appendix}
A notable difference between the quantitative analysis of intent code occurrences and \ref{change-mental-image} is the finding that \diverse~induces greater degrees of `Reminder' and `Expansion' of intent than \baseline~($+0.02$ and $+0.04$, respectively), in contrast to self-reported findings that \baseline~induces greater reflection than \diverse. These differences may be explained by the fact that coding for added images excludes some forms of reflection, such as reflection that does not explicitly correlate to a particular image or reflection caused by images that participants ultimately reject, as further discussed in \S\ref{why-reflection}. For this reason, the methodology of this section may more accurately capture the unique reflection due to each interface because it gives less weight to reflection from external causes like performing the task itself. The `hierarchy' of interfaces suggested by our findings in this section is thus $\left[ \agonistic~\gg \reformulative~> \{\baseline, \diverse\}\right]$.




\subsection{Additional Qualitative Themes}
\label{qual-themes}
\textbf{\agonistic~can feel like a double-edged sword, treading the line between appropriate and inappropriate.}
The results in \S\ref{interface-properties} suggest that appropriateness is correlated with higher rethinking.
However, by design, \agonistic~often presents users with interpretations that are perceived \textit{initially} by users as unrelated, inappropriate, or even offensive.
For many participants who underwent reflection, these interpretations eventually seemed appropriate, explaining the high appropriateness rating of \agonistic.
However, for other participants, these interpretations remained inappropriate (even after reflection).
Several participants commented on the dually appropriate and inappropriate character of \agonistic.
\pref{4} suggested that ``\textit{it [\agonistic] was giving me the most appropriate, but it was also giving me the most inappropriate [outputs] at the same time... [it] would sometimes hit a home run, sometimes wildly off track.'}'
\pref{16} said that \agonistic~was ``\textit{good good and bad bad [...] I like the specificity [of \agonistic]. I just think the specificity was kind of whacked out in a way where it just wasn't super on the bar of what I was thinking about}''.
Together, these suggest that the experience of using \agonistic~can vary widely, and that perhaps designing for reflection via agonism means that interface outputs will not seem appropriate to many users by its own nature.

\section{Extended Discussion}
\subsection{The Many Heads of Reflection}
We find it important to note that people find reflection across all the interfaces, as demonstrated by Table~\ref{tab:qualitative-examples}; although the overall occurrence of reflection does vary substantively (as discussed in \S\ref{how-reflection}), reflection is still a very subjective, personal experience.
Indeed, while \baseline~was uninspiring for many --- ``\textit{I typed what I thought and I got what I thought I would get}'' \pref{21} -- it was a highly reflective experience for others ---  ``\textit{I have a lot of other potential images in my mind of it now, and I don't know what is [...] actually true}'' \pref{28}.
We find that reflection generally seems to emerge from an initially unexpected but eventually rationalizable encounter with the interface (either the images themselves or contextualizing content).
This reflects a host of work on reflection and related cognitive activities emphasizing encounter with alterity or the Other~\cite[\textit{inter alia}]{ye2024languagemodelscriticalthinking, halbertandnathan2015criticalreflection, levinas_totality_1969, lacan_ecrits}.
Although reflection often occurred in planned ways (i.e., as a result of design elements we had introduced with the intention of inducing reflection), it also often emerged in spontaneous but nonetheless personally significant ways.
This emphasizes that image generation interfaces may benefit from more adaptive design, but more importantly that users will always reflect in ways that interface designers cannot completely control.