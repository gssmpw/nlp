\section{Coda: Bricks and Windows}
\label{conclusion}

% Roland Barthes eloquently remarked that ``\textit{all images are polysemous; they imply, underlying their signifiers, a ``floating chain'' of signifieds, the reader able to choose some and ignore others}''~\cite{barthes_rhetoric_image};
% from this polysemy of images emerges a \textit{political world} in which individuals negotiate over the meanings of images.
We introduced \textit{agonistic pluralism} as a challenge to the `hegemony of intention' in image generation interface design, demonstrating that an interface built on agonistic principles engaged individuals with competing interpretations in the salient discourse, making them more critically reflective.
Gilles Deleuze and Felix Guattari delightfully noted that ``\textit{A concept is a brick. It can be used to build a courthouse of reason. Or it can be thrown through the window}''~\cite{deleuze_guattari_1987}.
Instead of obsessing over ``structurally sound'' interfaces that perfectly embody capital-I Intention, capital-F Factuality, and/or capital-D Diversity, maybe our interfaces should give users some bricks to throw at the windows --- they were always brittle anyways, bordering on stale, ready to be broken.
Perhaps the view outside will be clearer.
% \andre{admittedly a bit polemical and whimsical but I like it a lot lmao}



% We present \textit{agonism} as a challenge to the dominant paradigm of intention-based design in image generation and AI development. 
% Image generation is not just a technical process of translating individual desires into pixels, but an inherently political act that participates in ongoing social negotiations over visual representation of people and the baggage they carry.
% Our empirical findings reveal that an agonistic approach to interface design can successfully create space for democratic struggle between competing values of intention and diversity via reflection and authenticity.
% The \agonistic~interface significantly outperformed other paradigmatic approaches across multiple measures of reflection, challenging the assumption that AI systems must choose between respecting user agency, promoting diversity, and being factual.
% Instead, by explicitly surfacing tensions and contestations in visual representation, our interface enabled users to engage more deeply with questions of diversity while maintaining their sense of control and authenticity.
% Several key design principles emerge from our work: 
% the importance of contextualizing elements that help users interpret and reason about AI outputs; 
% the value of challenging familiar representations in favor of historically-informed alternatives; 
% and the productive role of guiding users to find initially inappropriate outputs appropriate through reflection rather than prescription.
% % These findings have implications beyond image generation, suggesting new ways to approach ethical questions in AI development more broadly.
% Our work ultimately demonstrates the limitations of current approaches that attempt to resolve tensions between user intention and social values through technical means alone. 
% Rather than trying to determine the ``right'' balance between diversity and intention programmatically, we show that interfaces can productively engage users in the political work of negotiating these tensions themselves. 
% This reframing opens new possibilities for generative AI interfaces and systems that don't just reflect or enforce particular values, but rather create spaces for democratic deliberation over what those values should be.
% In doing so, we challenge the field to move beyond simplistic framings of AI ethics as a matter of aligning systems with predetermined values. Instead, we suggest that AI interfaces can and should be designed as sites of productive political struggle – not to resolve tensions between competing values, but to help users engage with them more thoughtfully and critically.


% We present \textit{agonism} as an alternative to dominant intention-based design paradigms in image generation and AI development.
% Because image generation is an inherently \textit{active} and \textit{social} process of negotiating meanings, image generation interfaces must (implicitly or explicitly) play a \textit{political role} in arranging how these meanings are negotiated. Agonism is a useful way to understand such a role for image generation interfaces.
% % We suggest that while research about diversity in image generation in particular has largely sought to align image generation models with user intentions, agonistic design requires viewing image generation as a political practice with a pluralism of conflicting intentions.
% We apply agonistic design principles to develop an image generation interface that supports critical reflection and test our interface against other paradigms of image generation. 
% We find that \agonistic~is more effective at encouraging reflection on politically sensitive topics compared to other paradigmatic interfaces on a variety of measures for reflection.
% % Users of our interface self-report an average $+0.94$ increase in reflection on a 5-point scale over the baseline, and 
% % The \agonistic~interfaces achieves the highest rate of multiple types of reflection based on a quantitative analysis of interview coding. 
% % We find that in general, UI interventions are crucial in scaffolding productive forms of critical reflection. 
% % Adding text to describe image generation outputs assists users in contextualizing and interpreting AI outputs, which can be helpful in sparking forms of reflection. 
% % Reflection can further be facilitated by authentic diversity, in which users are confronted with alternative perspectives that are realistically and historically informed. 
% Our findings may be useful to others seeking to build tools for reflection: among other insights, we find that reflection is encouraged by the presence of contextualizing elements (e.g. descriptive text next to images) that help users interpret AI outputs; emphases on challenging familiarity in favor of other values (like realism and diversity); and the process of going from inappropriate to appropriate.
% % Importantly, our work highlights that factuality is \textit{not} a primary value for users of image generation interfaces, who tend to invoke familiarity more often in evaluating image generation outputs. 
% % Our work therefore also highlights the importance of attending to the more affective dimension of human-AI interaction. 
% In our view, ongoing research tends to overly restrict the space of user-model interactions in ways that bracket out potentially productive struggles over political questions of diversity. 
% Instead, by situating user-model interactions in a broader political context, we show that agonistic design is effective at meaningfully broadening the diversity of images in ways that respect user agency and control.

