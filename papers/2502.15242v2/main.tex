%%
%% This is file `sample-manuscript.tex',
%% generated with the docstrip utility.
%%
%% The original source files were:
%%
%% samples.dtx  (with options: `all,proceedings,bibtex,manuscript')
%% 
%% IMPORTANT NOTICE:
%% 
%% For the copyright see the source file.
%% 
%% Any modified versions of this file must be renamed
%% with new filenames distinct from sample-manuscript.tex.
%% 
%% For distribution of the original source see the terms
%% for copying and modification in the file samples.dtx.
%% 
%% This generated file may be distributed as long as the
%% original source files, as listed above, are part of the
%% same distribution. (The sources need not necessarily be
%% in the same archive or directory.)
%%
%%
%% Commands for TeXCount
%TC:macro \cite [option:text,text]
%TC:macro \citep [option:text,text]
%TC:macro \citet [option:text,text]
%TC:envir table 0 1
%TC:envir table* 0 1
%TC:envir tabular [ignore] word
%TC:envir displaymath 0 word
%TC:envir math 0 word
%TC:envir comment 0 0
%%
%% The first command in your LaTeX source must be the \documentclass
%% command.
%%
%% For submission and review of your manuscript please change the
%% command to \documentclass[manuscript, screen, review]{acmart}.
%%
%% When submitting camera ready or to TAPS, please change the command
%% to \documentclass[sigconf]{acmart} or whichever template is required
%% for your publication.
%%
%%
\documentclass[manuscript,screen]{acmart}
% \documentclass[manuscript,screen,review,anonymous]{acmart}
%%
%% \BibTeX command to typeset BibTeX logo in the docs
\AtBeginDocument{%
  \providecommand\BibTeX{{%
    Bib\TeX}}}

%% Rights management information.  This information is sent to you
%% when you complete the rights form.  These commands have SAMPLE
%% values in them; it is your responsibility as an author to replace
%% the commands and values with those provided to you when you
%% complete the rights form.
\setcopyright{acmlicensed}
\copyrightyear{2018}
\acmYear{2018}
\acmDOI{XXXXXXX.XXXXXXX}
%% These commands are for a PROCEEDINGS abstract or paper.
\acmConference[Conference acronym 'XX]{Conference title.}{June 03--05,
  2018}{Woodstock, NY}
%%
%%  Uncomment \acmBooktitle if the title of the proceedings is different
%%  from ``Proceedings of ...''!
%%
%%\acmBooktitle{Woodstock '18: ACM Symposium on Neural Gaze Detection,
%%  June 03--05, 2018, Woodstock, NY}
\acmISBN{978-1-4503-XXXX-X/18/06}


%%
%% Submission ID.
%% Use this when submitting an article to a sponsored event. You'll
%% receive a unique submission ID from the organizers
%% of the event, and this ID should be used as the parameter to this command.
%%\acmSubmissionID{123-A56-BU3}

%%
%% For managing citations, it is recommended to use bibliography
%% files in BibTeX format.
%%
%% You can then either use BibTeX with the ACM-Reference-Format style,
%% or BibLaTeX with the acmnumeric or acmauthoryear sytles, that include
%% support for advanced citation of software artefact from the
%% biblatex-software package, also separately available on CTAN.
%%
%% Look at the sample-*-biblatex.tex files for templates showcasing
%% the biblatex styles.
%%

%%
%% The majority of ACM publications use numbered citations and
%% references.  The command \citestyle{authoryear} switches to the
%% "author year" style.
%%
%% If you are preparing content for an event
%% sponsored by ACM SIGGRAPH, you must use the "author year" style of
%% citations and references.
%% Uncommenting
%% the next command will enable that style.
%%\citestyle{acmauthoryear}


\usepackage{xcolor}
\usepackage{epigraph}
\usepackage{tcolorbox}
\usepackage{tabularx}
\usepackage{graphicx} % For including the table
\usepackage{booktabs} % For better table styling
\usepackage{multicol} % For multi-column layout
\usepackage{adjustbox} % For adjusting table placement
\usepackage{xfrac}
\usepackage{wrapfig} % For wrapping tables or figures in text
\usepackage{lipsum}
\usepackage{multirow}
\usepackage{enumitem}

\newcommand{\assign}[2][unknown]{%
  \colorbox{yellow!60}{%
    \hspace{3pt}\textcolor{gray}{\textbf{assigned to #1: #2}}\hspace{3pt}%
  }%
}
\newcommand{\note}[1]{\textcolor{red}{[!!!: #1]}}
\newcommand{\andrew}[1]{\textcolor{blue}{[andrew: #1]}}
\newcommand{\andre}[1]{\textcolor{purple}{}}

\newcommand{\amy}[1]{\textcolor{orange}{[amy: #1]}}

\newcommand{\fixedbox}[3]{%
  \hspace{1pt}\raisebox{0.1ex}{% Adjust vertical alignment slightly
    \colorbox{#1}{%
      \hspace{-1pt}% Add minimal horizontal padding
      \rule[-0.1ex]{0pt}{1.5ex}% Adjust height and depth to make it tighter
      \textcolor{#2}{\textsf{\small #3}}%
      \hspace{-1pt}% Add minimal horizontal padding
    }%
  }%
}

% Define specific colored boxes
\newcommand{\agonisticbox}{\fixedbox{yellow!20}{yellow!80!black}{\textsf{\textsc{Agonistic}}}~}
\newcommand{\baselinebox}{\fixedbox{gray!10}{gray!90}{\textsf{\textsc{Baseline}}}~}
\newcommand{\reformulativebox}{\fixedbox{red!10}{red!80!black}{\textsf{\textsc{Reformulative}}}~}
\newcommand{\diversebox}{\fixedbox{blue!10}{blue!80!black}{\textsf{\textsc{Diverse}}~}}

\newcommand{\agonistic}{\textsf{\textsc{Agonistic}}}
\newcommand{\baseline}{\textsf{\textsc{Baseline}}}
\newcommand{\reformulative}{\textsf{\textsc{Reformulative}}}
\newcommand{\diverse}{\textsf{\textsc{Diverse}}}

% Define codes
\newcommand{\direct}{\textsf{Direct}}
\newcommand{\reminder}{\textsf{Reminder}}
\newcommand{\expansion}{\textsf{Expansion}}
\newcommand{\challenge}{\textsf{Challenge}}

\newcommand{\realism}{\textsf{Realism}}
\newcommand{\familiarity}{\textsf{Familiarity}}
\newcommand{\diversity}{\textsf{Diversity}}
\newcommand{\aesthetics}{\textsf{Aesthetics}}

\usepackage{tikz} % Required for drawing shapes
\usepackage{tabularx}
\usepackage{subcaption}

% % Define the circular tag function
% \newcommand{\pref}[1]{%
%   \tikz[baseline=(char.base)]{
%     \node[%draw=purple!80!black, % Circle border color
%           fill=purple!20, % Circle fill color
%           text=black, % Text color
%           circle, % Makes it a circle
%           minimum size=1.5em, % Diameter of the circle
%           inner sep=0pt, % No extra padding
%           line width=0pt,
%           font=\sffamily] (char) {{\small P}#1}; % Content: P followed by number
%   }%
% }

\newcommand{\pref}[1]{\textsf{P#1}}




%%
%% end of the preamble, start of the body of the document source.
\begin{document}

%%
%% The "title" command has an optional parameter,
%% allowing the author to define a "short title" to be used in page headers.
\title{Unsettling the Hegemony of Intention: Agonistic Image Generation}

%%
%% The "author" command and its associated commands are used to define
%% the authors and their affiliations.
%% Of note is the shared affiliation of the first two authors, and the
%% "authornote" and "authornotemark" commands
%% used to denote shared contribution to the research.
\author{Andrew Shaw}
\authornote{Both authors contributed equally to this research.}
\email{shawan@uw.edu}
\orcid{0009-0007-4579-718X}
\author{Andre Ye}
\authornotemark[1]
\email{andreye@uw.edu}
\orcid{0000-0003-4936-7914}
\affiliation{%
  \institution{University of Washington}
  \city{Seattle}
  \state{Washington}
  \country{USA}
}

\author{Ranjay Krishna}
\affiliation{%
  \institution{University of Washington, Allen Institute for Artificial Intelligence}
  \city{Seattle}
  \state{Washington}
  \country{USA}}
\email{ranjay@cs.washington.edu}

\author{Amy X. Zhang}
\affiliation{%
  \institution{University of Washington, Allen Institute for Artificial Intelligence}
  \city{Seattle}
  \state{Washington}
  \country{USA}
}
\email{axz@cs.uw.edu}

% \author{Aparna Patel}
% \affiliation{%
%  \institution{Rajiv Gandhi University}
%  \city{Doimukh}
%  \state{Arunachal Pradesh}
%  \country{India}}

% \author{Huifen Chan}
% \affiliation{%
%   \institution{Tsinghua University}
%   \city{Haidian Qu}
%   \state{Beijing Shi}
%   \country{China}}

% \author{Charles Palmer}
% \affiliation{%
%   \institution{Palmer Research Laboratories}
%   \city{San Antonio}
%   \state{Texas}
%   \country{USA}}
% \email{cpalmer@prl.com}

% \author{John Smith}
% \affiliation{%
%   \institution{The Th{\o}rv{\"a}ld Group}
%   \city{Hekla}
%   \country{Iceland}}
% \email{jsmith@affiliation.org}

% \author{Julius P. Kumquat}
% \affiliation{%
%   \institution{The Kumquat Consortium}
%   \city{New York}
%   \country{USA}}
% \email{jpkumquat@consortium.net}

%%
%% By default, the full list of authors will be used in the page
%% headers. Often, this list is too long, and will overlap
%% other information printed in the page headers. This command allows
%% the author to define a more concise list
%% of authors' names for this purpose.
\renewcommand{\shortauthors}{Shaw $\&$ Ye et al.}

%%
%% The abstract is a short summary of the work to be presented in the
Large language model (LLM)-based agents have shown promise in tackling complex tasks by interacting dynamically with the environment. 
Existing work primarily focuses on behavior cloning from expert demonstrations and preference learning through exploratory trajectory sampling. However, these methods often struggle in long-horizon tasks, where suboptimal actions accumulate step by step, causing agents to deviate from correct task trajectories.
To address this, we highlight the importance of \textit{timely calibration} and the need to automatically construct calibration trajectories for training agents. We propose \textbf{S}tep-Level \textbf{T}raj\textbf{e}ctory \textbf{Ca}libration (\textbf{\model}), a novel framework for LLM agent learning. 
Specifically, \model identifies suboptimal actions through a step-level reward comparison during exploration. It constructs calibrated trajectories using LLM-driven reflection, enabling agents to learn from improved decision-making processes. These calibrated trajectories, together with successful trajectory data, are utilized for reinforced training.
Extensive experiments demonstrate that \model significantly outperforms existing methods. Further analysis highlights that step-level calibration enables agents to complete tasks with greater robustness. 
Our code and data are available at \url{https://github.com/WangHanLinHenry/STeCa}.

%%
%% The code below is generated by the tool at http://dl.acm.org/ccs.cfm.
%% Please copy and paste the code instead of the example below.
%%
\begin{CCSXML}
<ccs2012>
   <concept>
       <concept_id>10003120.10003121.10003124</concept_id>
       <concept_desc>Human-centered computing~Interaction paradigms</concept_desc>
       <concept_significance>500</concept_significance>
       </concept>
   <concept>
       <concept_id>10003120.10003123.10011758</concept_id>
       <concept_desc>Human-centered computing~Interaction design theory, concepts and paradigms</concept_desc>
       <concept_significance>500</concept_significance>
       </concept>
 </ccs2012>
\end{CCSXML}

\ccsdesc[500]{Human-centered computing~Interaction paradigms}
\ccsdesc[500]{Human-centered computing~Interaction design theory, concepts and paradigms}

%%
%% Keywords. The author(s) should pick words that accurately describe
%% the work being presented. Separate the keywords with commas.
\keywords{agonism, reflection, image generation interfaces}

\received{20 February 2007}
\received[revised]{12 March 2009}
\received[accepted]{5 June 2009}


%%
%% This command processes the author and affiliation and title
%% information and builds the first part of the formatted document.
\maketitle

% \setlength\epigraphwidth{0.86\textwidth}
% \epigraph{``\textit{One does not become enlightened by imagining figures of light, but by making the darkness conscious.}'' -- Carl Jung}{}

% \vspace{-5mm}

% \agonistic
% \diverse
% \reformulative
% \baseline

% Example usage: \pref{5}, \pref{10}, \ref{42}.

% One does not become enlightened by imagining figures of light, but by making the darkness conscious.
% Carl Jung

% There are painters who transform the sun to a yellow spot, but there are others who with the help of their art and their intelligence, transform a yellow spot into the sun.
% Pablo Picasso



% Overall research question: How do different image generation interface design principles influence user reflection on cultural, political, social, etc. debates over visual representation?

% Observed behavior
% Emotional reaction

% 1. What causes people to reflect?
% - Values

% 2. How do interface features influence these causes?

% 1. How do interfaces induce reflection?
% - Survey results -- report statistics, do all of our normalizations and explanations (does order matter? compute variance): not including appropriateness, control, etc. -- only doing analysis about mental image
% - Coding results -- just report what kinds of intention/reflection is happening with each image, do analysis of image adding dynamics. Maybe also talk about why people reject
% - Qualitatively -- giving examples of reflection that people had with interface X; examples of bad/no reflection with Y

% 2. Why do different interfaces induce reflection?
% - Survey results -- consider correlations between appropriateness, control and reflection; do interfaces / engagemetns with lower appropriateness correlate with higher reflection, etc.
% - Bring in values and other analysis which tries to more richly characterize / explain the results above
% - More qualitative stuff -- explanations



%!TEX root = gcn.tex
\section{Introduction}
Graphs, representing structural data and topology, are widely used across various domains, such as social networks and merchandising transactions.
Graph convolutional networks (GCN)~\cite{iclr/KipfW17} have significantly enhanced model training on these interconnected nodes.
However, these graphs often contain sensitive information that should not be leaked to untrusted parties.
For example, companies may analyze sensitive demographic and behavioral data about users for applications ranging from targeted advertising to personalized medicine.
Given the data-centric nature and analytical power of GCN training, addressing these privacy concerns is imperative.

Secure multi-party computation (MPC)~\cite{crypto/ChaumDG87,crypto/ChenC06,eurocrypt/CiampiRSW22} is a critical tool for privacy-preserving machine learning, enabling mutually distrustful parties to collaboratively train models with privacy protection over inputs and (intermediate) computations.
While research advances (\eg,~\cite{ccs/RatheeRKCGRS20,uss/NgC21,sp21/TanKTW,uss/WatsonWP22,icml/Keller022,ccs/ABY318,folkerts2023redsec}) support secure training on convolutional neural networks (CNNs) efficiently, private GCN training with MPC over graphs remains challenging.

Graph convolutional layers in GCNs involve multiplications with a (normalized) adjacency matrix containing $\numedge$ non-zero values in a $\numnode \times \numnode$ matrix for a graph with $\numnode$ nodes and $\numedge$ edges.
The graphs are typically sparse but large.
One could use the standard Beaver-triple-based protocol to securely perform these sparse matrix multiplications by treating graph convolution as ordinary dense matrix multiplication.
However, this approach incurs $O(\numnode^2)$ communication and memory costs due to computations on irrelevant nodes.
%
Integrating existing cryptographic advances, the initial effort of SecGNN~\cite{tsc/WangZJ23,nips/RanXLWQW23} requires heavy communication or computational overhead.
Recently, CoGNN~\cite{ccs/ZouLSLXX24} optimizes the overhead in terms of  horizontal data partitioning, proposing a semi-honest secure framework.
Research for secure GCN over vertical data  remains nascent.

Current MPC studies, for GCN or not, have primarily targeted settings where participants own different data samples, \ie, horizontally partitioned data~\cite{ccs/ZouLSLXX24}.
MPC specialized for scenarios where parties hold different types of features~\cite{tkde/LiuKZPHYOZY24,icml/CastigliaZ0KBP23,nips/Wang0ZLWL23} is rare.
This paper studies $2$-party secure GCN training for these vertical partition cases, where one party holds private graph topology (\eg, edges) while the other owns private node features.
For instance, LinkedIn holds private social relationships between users, while banks own users' private bank statements.
Such real-world graph structures underpin the relevance of our focus.
To our knowledge, no prior work tackles secure GCN training in this context, which is crucial for cross-silo collaboration.


To realize secure GCN over vertically split data, we tailor MPC protocols for sparse graph convolution, which fundamentally involves sparse (adjacency) matrix multiplication.
Recent studies have begun exploring MPC protocols for sparse matrix multiplication (SMM).
ROOM~\cite{ccs/SchoppmannG0P19}, a seminal work on SMM, requires foreknowledge of sparsity types: whether the input matrices are row-sparse or column-sparse.
Unfortunately, GCN typically trains on graphs with arbitrary sparsity, where nodes have varying degrees and no specific sparsity constraints.
Moreover, the adjacency matrix in GCN often contains a self-loop operation represented by adding the identity matrix, which is neither row- nor column-sparse.
Araki~\etal~\cite{ccs/Araki0OPRT21} avoid this limitation in their scalable, secure graph analysis work, yet it does not cover vertical partition.

% and related primitives
To bridge this gap, we propose a secure sparse matrix multiplication protocol, \osmm, achieving \emph{accurate, efficient, and secure GCN training over vertical data} for the first time.

\subsection{New Techniques for Sparse Matrices}
The cost of evaluating a GCN layer is dominated by SMM in the form of $\adjmat\feamat$, where $\adjmat$ is a sparse adjacency matrix of a (directed) graph $\graph$ and $\feamat$ is a dense matrix of node features.
For unrelated nodes, which often constitute a substantial portion, the element-wise products $0\cdot x$ are always zero.
Our efficient MPC design 
avoids unnecessary secure computation over unrelated nodes by focusing on computing non-zero results while concealing the sparse topology.
We achieve this~by:
1) decomposing the sparse matrix $\adjmat$ into a product of matrices (\S\ref{sec::sgc}), including permutation and binary diagonal matrices, that can \emph{faithfully} represent the original graph topology;
2) devising specialized protocols (\S\ref{sec::smm_protocol}) for efficiently multiplying the structured matrices while hiding sparsity topology.


 
\subsubsection{Sparse Matrix Decomposition}
We decompose adjacency matrix $\adjmat$ of $\graph$ into two bipartite graphs: one represented by sparse matrix $\adjout$, linking the out-degree nodes to edges, the other 
by sparse matrix $\adjin$,
linking edges to in-degree nodes.

%\ie, we decompose $\adjmat$ into $\adjout \adjin$, where $\adjout$ and $\adjin$ are sparse matrices representing these connections.
%linking out-degree nodes to edges and edges to in-degree nodes of $\graph$, respectively.

We then permute the columns of $\adjout$ and the rows of $\adjin$ so that the permuted matrices $\adjout'$ and $\adjin'$ have non-zero positions with \emph{monotonically non-decreasing} row and column indices.
A permutation $\sigma$ is used to preserve the edge topology, leading to an initial decomposition of $\adjmat = \adjout'\sigma \adjin'$.
This is further refined into a sequence of \emph{linear transformations}, 
which can be efficiently computed by our MPC protocols for 
\emph{oblivious permutation}
%($\Pi_{\ssp}$) 
and \emph{oblivious selection-multiplication}.
% ($\Pi_\SM$)
\iffalse
Our approach leverages bipartite graph representation and the monotonicity of non-zero positions to decompose a general sparse matrix into linear transformations, enhancing the efficiency of our MPC protocols.
\fi
Our decomposition approach is not limited to GCNs but also general~SMM 
by 
%simply 
treating them 
as adjacency matrices.
%of a graph.
%Since any sparse matrix can be viewed 

%allowing the same technique to be applied.

 
\subsubsection{New Protocols for Linear Transformations}
\emph{Oblivious permutation} (OP) is a two-party protocol taking a private permutation $\sigma$ and a private vector $\xvec$ from the two parties, respectively, and generating a secret share $\l\sigma \xvec\r$ between them.
Our OP protocol employs correlated randomnesses generated in an input-independent offline phase to mask $\sigma$ and $\xvec$ for secure computations on intermediate results, requiring only $1$ round in the online phase (\cf, $\ge 2$ in previous works~\cite{ccs/AsharovHIKNPTT22, ccs/Araki0OPRT21}).

Another crucial two-party protocol in our work is \emph{oblivious selection-multiplication} (OSM).
It takes a private bit~$s$ from a party and secret share $\l x\r$ of an arithmetic number~$x$ owned by the two parties as input and generates secret share $\l sx\r$.
%between them.
%Like our OP protocol, o
Our $1$-round OSM protocol also uses pre-computed randomnesses to mask $s$ and $x$.
%for secure computations.
Compared to the Beaver-triple-based~\cite{crypto/Beaver91a} and oblivious-transfer (OT)-based approaches~\cite{pkc/Tzeng02}, our protocol saves ${\sim}50\%$ of online communication while having the same offline communication and round complexities.

By decomposing the sparse matrix into linear transformations and applying our specialized protocols, our \osmm protocol
%($\prosmm$) 
reduces the complexity of evaluating $\numnode \times \numnode$ sparse matrices with $\numedge$ non-zero values from $O(\numnode^2)$ to $O(\numedge)$.

%(\S\ref{sec::secgcn})
\subsection{\cgnn: Secure GCN made Efficient}
Supported by our new sparsity techniques, we build \cgnn, 
a two-party computation (2PC) framework for GCN inference and training over vertical
%ly split
data.
Our contributions include:

1) We are the first to explore sparsity over vertically split, secret-shared data in MPC, enabling decompositions of sparse matrices with arbitrary sparsity and isolating computations that can be performed in plaintext without sacrificing privacy.

2) We propose two efficient $2$PC primitives for OP and OSM, both optimally single-round.
Combined with our sparse matrix decomposition approach, our \osmm protocol ($\prosmm$) achieves constant-round communication costs of $O(\numedge)$, reducing memory requirements and avoiding out-of-memory errors for large matrices.
In practice, it saves $99\%+$ communication
%(Table~\ref{table:comm_smm}) 
and reduces ${\sim}72\%$ memory usage over large $(5000\times5000)$ matrices compared with using Beaver triples.
%(Table~\ref{table:mem_smm_sparse}) ${\sim}16\%$-

3) We build an end-to-end secure GCN framework for inference and training over vertically split data, maintaining accuracy on par with plaintext computations.
We will open-source our evaluation code for research and deployment.

To evaluate the performance of $\cgnn$, we conducted extensive experiments over three standard graph datasets (Cora~\cite{aim/SenNBGGE08}, Citeseer~\cite{dl/GilesBL98}, and Pubmed~\cite{ijcnlp/DernoncourtL17}),
reporting communication, memory usage, accuracy, and running time under varying network conditions, along with an ablation study with or without \osmm.
Below, we highlight our key achievements.

\textit{Communication (\S\ref{sec::comm_compare_gcn}).}
$\cgnn$ saves communication by $50$-$80\%$.
(\cf,~CoGNN~\cite{ccs/KotiKPG24}, OblivGNN~\cite{uss/XuL0AYY24}).

\textit{Memory usage (\S\ref{sec::smmmemory}).}
\cgnn alleviates out-of-memory problems of using %the standard 
Beaver-triples~\cite{crypto/Beaver91a} for large datasets.

\textit{Accuracy (\S\ref{sec::acc_compare_gcn}).}
$\cgnn$ achieves inference and training accuracy comparable to plaintext counterparts.
%training accuracy $\{76\%$, $65.1\%$, $75.2\%\}$ comparable to $\{75.7\%$, $65.4\%$, $74.5\%\}$ in plaintext.

{\textit{Computational efficiency (\S\ref{sec::time_net}).}} 
%If the network is worse in bandwidth and better in latency, $\cgnn$ shows more benefits.
$\cgnn$ is faster by $6$-$45\%$ in inference and $28$-$95\%$ in training across various networks and excels in narrow-bandwidth and low-latency~ones.

{\textit{Impact of \osmm (\S\ref{sec:ablation}).}}
Our \osmm protocol shows a $10$-$42\times$ speed-up for $5000\times 5000$ matrices and saves $10$-2$1\%$ memory for ``small'' datasets and up to $90\%$+ for larger ones.

\section{Background}
\label{sec:background}


\subsection{Code Review Automation}
Code review is a widely adopted practice among software developers where a reviewer examines changes submitted in a pull request \cite{hong2022commentfinder, ben2024improving, siow2020core}. If the pull request is not approved, the reviewer must describe the issues or improvements required, providing constructive feedback and identifying potential issues. This step involves review commment generation, which play a key role in the review process by generating review comments for a given code difference. These comments can be descriptive, offering detailed explanations of the issues, or actionable, suggesting specific solutions to address the problems identified \cite{ben2024improving}.


Various approaches have been explored to automate the code review comments process  \cite{tufano2023automating, tufano2024code, yang2024survey}. 
Early efforts centered on knowledge-based systems, which are designed to detect common issues in code. Although these traditional tools provide some support to programmers, they often fall short in addressing complex scenarios encountered during code reviews \cite{dehaerne2022code}. More recently, with advancements in deep learning, researchers have shifted their focus toward using large-language models to enhance the effectiveness of code issue detection and code review comment generation.

\subsection{Knowledge-based Code Review Comments Automation}

Knowledge-based systems (KBS) are software applications designed to emulate human expertise in specific domains by using a collection of rules, logic, and expert knowledge. KBS often consist of facts, rules, an explanation facility, and knowledge acquisition. In the context of software development, these systems are used to analyze the source code, identifying issues such as coding standard violations, bugs, and inefficiencies~\cite{singh2017evaluating, delaitre2015evaluating, ayewah2008using, habchi2018adopting}. By applying a vast set of predefined rules and best practices, they provide automated feedback and recommendations to developers. Tools such as FindBugs \cite{findBugs}, PMD \cite{pmd}, Checkstyle \cite{checkstyle}, and SonarQube \cite{sonarqube} are prominent examples of knowledge-based systems in code analysis, often referred to as static analyzers. These tools have been utilized since the early 1960s, initially to optimize compiler operations, and have since expanded to include debugging tools and software development frameworks \cite{stefanovic2020static, beller2016analyzing}.



\subsection{LLMs-based Code Review Comments Automation}
As the field of machine learning in software engineering evolves, researchers are increasingly leveraging machine learning (ML) and deep learning (DL) techniques to automate the generation of review comments \cite{li2022automating, tufano2022using, balachandran2013reducing, siow2020core, li2022auger, hong2022commentfinder}. Large language models (LLMs) are large-scale Transformer models, which are distinguished by their large number of parameters and extensive pre-training on diverse datasets.  Recently, LLMs have made substantial progress and have been applied across a broad spectrum of domains. Within the software engineering field, LLMs can be categorized into two main types: unified language models and code-specific models, each serving distinct purposes \cite{lu2023llama}.

Code-specific LLMs, such as CodeGen \cite{nijkamp2022codegen}, StarCoder \cite{li2023starcoder} and CodeLlama \cite{roziere2023code} are optimized to excel in code comprehension, code generation, and other programming-related tasks. These specialized models are increasingly utilized in code review activities to detect potential issues, suggest improvements, and automate review comments \cite{yang2024survey, lu2023llama}. 




\subsection{Retrieval-Augmented Generation}
Retrieval-Augmented Generation (RAG) is a general paradigm that enhances LLMs outputs by including relevant information retrieved from external databases into the input prompt \cite{gao2023retrieval}. Traditional LLMs generate responses based solely on the extensive data used in pre-training, which can result in limitations, especially when it comes to domain-specific, time-sensitive, or highly specialized information. RAG addresses these limitations by dynamically retrieving pertinent external knowledge, expanding the model's informational scope and allowing it to generate responses that are more accurate, up-to-date, and contextually relevant \cite{arslan2024business}. 

To build an effective end-to-end RAG pipeline, the system must first establish a comprehensive knowledge base. It requires a retrieval model that captures the semantic meaning of presented data, ensuring relevant information is retrieved. Finally, a capable LLM integrates this retrieved knowledge to generate accurate and coherent results \cite{ibtasham2024towards}.




\subsection{LLM as a Judge Mechanism}

LLM evaluators, often referred to as LLM-as-a-Judge, have gained significant attention due to their ability to align closely with human evaluators' judgments \cite{zhu2023judgelm, shi2024judging}. Their adaptability and scalability make them highly suitable for handling an increasing volume of evaluative tasks. 

Recent studies have shown that certain LLMs, such as Llama-3 70B and GPT-4 Turbo, exhibit strong alignment with human evaluators, making them promising candidates for automated judgment tasks \cite{thakur2024judging}

To enable such evaluations, a proper benchmarking system should be set up with specific components: \emph{prompt design}, which clearly instructs the LLM to evaluate based on a given metric, such as accuracy, relevance, or coherence; \emph{response presentation}, guiding the LLM to present its verdicts in a structured format; and \emph{scoring}, enabling the LLM to assign a score according to a predefined scale \cite{ibtasham2024towards}. Additionally, this evaluation system can be enriched with the ability to explain reasoning behind verdicts, which is a significant advantage of LLM-based evaluation \cite{zheng2023judging}. The LLM can outline the criteria it used to reach its judgment, offering deeper insights into its decision-making process.





\section{Four Paradigmatic Interfaces for Image Generation Interactions}
\label{paradigms-interfaces}


\begin{figure}[]
    \centering
    % Top left
    \begin{subfigure}[b]{0.35\textwidth}
        \centering
        \includegraphics[height=6cm]{assets/interface-demos/demo-baseline.pdf}
        \phantomsubcaption
        \caption*{(a) \baselinebox}
        \label{fig:image1}
    \end{subfigure}
    \hfill
    % Top right
    \begin{subfigure}[b]{0.64\textwidth}
        \centering
        \includegraphics[height=6cm]{assets/interface-demos/demo-reformulative.pdf}
        \phantomsubcaption
        \caption*{(c) \reformulativebox}
        \label{fig:image3}
    \end{subfigure}
    \vspace{4mm}
    % Bottom left
    \begin{subfigure}[b]{0.35\textwidth}
        \centering
        \includegraphics[height=6cm]{assets/interface-demos/demo-diverse.pdf}
        \phantomsubcaption
        \caption*{(b) \diversebox}
        \label{fig:image2}
    \end{subfigure}
    \hfill
    % Bottom right
    \begin{subfigure}[b]{0.64\textwidth}
        \centering
        \includegraphics[height=6cm]{assets/interface-demos/demo-agonistic.pdf}
        \phantomsubcaption
        \caption*{(d) \agonisticbox}
        \label{fig:image4}
    \end{subfigure}
    \caption{Annotated screenshots from each of the paradigmatic image generation interfaces evaluated in our study. See \S\ref{paradigms-interfaces} for details.}
    \label{fig:interface-screenshots}
\end{figure}


To holistically evaluate how agonistic design might encourages reflection (\S\ref{hai-reflection}), we create \textbf{four image generation interfaces}: \baseline, which accepts the user's prompt and outputs generated images with no further interaction, and three \textit{paradigmatic interfaces} each embodying a dominant approach or value --- \diverse~for diversity, \reformulative~for intention actualization, and \agonistic~for agonism.
In this section, we describe the important frontend and backend elements of each interface, with more details in Appendix \S\ref{detailed-view-interfaces}. 
Annotated screenshots are displayed in Figure~\ref{fig:interface-screenshots}.


% \andre{@andrew: will need to design a single figure which illustrates how each of these interfaces works. can also be a $2 \times 2$ multifigure. good option is to design in google slides and then export as .pdf}

% \begin{figure}
%     \centering
%     \includegraphics[width=0.8\linewidth]{assets/Four Interfaces Mockup.pdf}
%     \caption{Screenshots of each of the interfaces, taken from the interview with participant 18. For a more detailed visual walkthrough of each interface, please see the supplementary \S\ref{detailed-view-interfaces}. \andre{Final version will be prettier}}
%     \label{fig:interface-screenshots}
% \end{figure}




% \S\ref{paradigms-interfaces}

% \assign[Andrew]{...}

% \subsection{\baseline}
\textbf{\baselinebox}~$\cdot$~
We design a baseline interface as a control for the study. In this interface, users enter a prompt and are displayed four images generated from the prompt.
We use a lightweight open-source image generation model (Black Forest Labs, \verb|flux-schnell|) throughout all interfaces in the study, to avoid relying on interfaces that perform under-the-hood prompt rewriting (like the DALL-E API).

% \subsection{\diverse}
\textbf{\diversebox}~$\cdot$~
\diverse~represents the paradigm of image generation that explicitly corrects for diversity issues by rewriting prompts in more diverse ways, as was reported in the Gemini case. 
To create this interface, we use a leaked alleged DALL-E prompt as reported by \cite{milmoandkern2024gemini}, which instructs the model to ``\textit{diversify depictions of ALL images with people to include DESCENT and GENDER for EACH person using direct term.}''
We instruct GPT-3.5 to rewrite the user prompt with the above instructions and use the rewritten prompt to generate four images that are then displayed to the user.
Although we recognize that the prompt may not match the exact approach used in the Gemini case, it anecdotally yields similar results to reported images for a variety of historical prompts.
Moreover, this approach makes \diverse~a realistic interface to compare against, since it uses the (alleged) prompt for the DALL-E API, which is widely in use.



% \andre{Add comment that even though B is supposed to be similar to the Gemini paradigm, we're taking it from DALL-E, and actually it's a pretty reasonable prompt, or at least it seems so, and is widely in use -- so it's a practical think to be comparing}

% \subsection{\reformulative}
\textbf{\reformulativebox}~$\cdot$~
% \reformulative~represents an approach that attempts to resolve controversies over image generation diversity by delegating control to individual users.
The \reformulative~interface embodies the value of intention actualization and is inspired by \S\ref{current-work}; it supports users by producing prompt reformulations that add detail to the prompt (e.g., adding useful semantics or stylistic indicators) in ways that are likely to result in more aesthetic or preferred images.
This interface is similar to interfaces like Promptify, which allows users to iteratively refine their prompts based on a variety of AI-generated reformulations \cite{brade2023promptify}.
Skilton and Cardinal propose using AI to generate a list of ``suggested descriptors'' that the user can then choose from and modify to create images that depart from stereotypical representations \cite{skiltonandcardinal2024inclusivepromptengineering}. 
We instruct GPT-4o to generate a diverse set of detail-adding reformulations of the user prompt with in-context examples from or inspired by Brade et al. 
% \amy{we should include all the prompts in our study in the appendix}
To aid visual navigation of reformulations, we generate a sample image (``thumbnail'') for each reformulation and display both in a list of suggested reformulations to the user.
The user can choose a suggestion and modify the suggestion before using it to generate images.
This interface is designed to have a comparable interaction level and similar features where applicable to \agonistic~for fair comparison.

% \subsection{\agonistic}
\textbf{\agonisticbox}~$\cdot$~
The \agonistic~interface attempts to forefront ambiguities and controversies over visual interpretation to the user, drawing from \S\ref{agonistic-democracy} and \S\ref{hai-reflection}.
We implement \agonistic~by creating a multi-step retrieval-augmented workflow that researches controversies about the user prompt from Wikipedia and presents them to the user.
Our work builds on insights from Cox et. al., who find that presenting users with maximally distant phrases (measured by language embeddings) is effective at enhancing diversity in creative ideation \cite{coxetal2021directeddiversity}.
Because Wikipedia allows public contributions, we believe it is more likely to capture relevant prompt controversies.% 
~$\cdot$~
We identify the main subject of the prompt with GPT-3.5 and perform a Wikipedia search for 50 pages related to the main subject. 
We then use GPT-4o to filter for 40 pages relevant to the user prompt based on their titles (excluding, for example, soccer player Gabriel Jesus from a search for ``Jesus'').
Next, we compute a controversy score for each page and rank pages by controversy, selecting the top 20 most controversial pages. 
The controversy score is calculated by dividing the total number of edits by the number of unique editors, following findings from Kittur et. al. that controversy is positively correlated with total number of edits and negatively correlated with the number of unique editors \cite{kitturetal2007conflictwikipedia}.
We chose this particular controversy metric for its efficiency and strong qualitative performance.
% to keep generation time relatively low for study interviews.
~$\cdot$~
% From this set of 40 pages, we take a random subset of 20 pages to generate interpretations of the user prompt.
We then instruct GPT-4o to generate 5 mental images an ``average person'' might have of the main subject and provide this in-context to the interpretation generation call.
For each page, GPT-4o then selects 4 sections based on their titles to read and produces an \textit{interpretation} with 4 fields: 1) a \textit{section summary} explaining the main points of the cited page section; 2) a \textit{description} of what the user's main subject looks like, taking into account the user's full prompt; 3) a \textit{source} with the referenced page and section; 4) a \textit{justification} explaining how the section content justifies the description.
The section summary is not shown to the user but is generated first to reduce model hallucinations.
~$\cdot$~
We instruct GPT-4o to generate descriptions that challenge the provided mental images, and phrase justifications in the form ``you may assume..., but... .'' These decisions were made to increase the likelihood that users would encounter discomforting suggestions, creating opportunities for critical reflection. 
Finally, like in \reformulative, we generate a ``thumbnail'' for each interpretation.
Users are shown a list of interpretations with descriptions, thumbnails, and sources; when users click on an interpretation, the card expands and the justification becomes visible.
After a 3-second wait period to encourage users to read the justification, the user can ``accept'' the interpretation and generate images, as well as edit the description text and re-generate.
The design iteration of this interface is documented in \S\ref{design-iteration}.
\section{Experimental Design}
\label{experiment}

To evaluate all four interfaces with respect to  reflection, we conduct a within-subjects comparative lab study and
ask 
\textbf{(RQ1):} \textit{How} each interface induces reflection using a variety of measures for user reflection; and
\textbf{(RQ2):} \textit{Why} each interface induces reflection by examining how other  variables could explain the reflection observations in RQ1.
% We describe the user task and interview structure (\S\ref{task-structure}), the recruiting process (\S\ref{recruiting-procedure}), and our measurements (\S\ref{measures}).
% \amy{wonder if you want to start out with some RQs/summarize them here, as it was unclear at this point what the experiment is testing}
% We begin by describing the user task and the interview structure as well as the data we collect~(\S\ref{task-structure}) for each interviewee.
% Then, we describe the recruiting process and give more information about the interviewees~(\S\ref{recruiting-procedure}).
% Lastly, we generally describe the measures we take during the study to answer RQ1 and RQ2~(\S\ref{measures}).

\subsection{User Task and Interview Structure}
% \assign[Andre]{...}}
\label{task-structure}

Each lab study session was conducted 1-on-1, lasted around one hour, and consisted of the \textbf{setup} and \textbf{user task}.

\textbf{Setup ($\sim$15 mins).}
% Each interface was referenced by a letter (A, B, C, D) in participant-facing communication to avoid prejudicing participants based on interface names.
% , where \baseline~$\to$ A, \diverse~$\to$ B, \reformulative~$\to$ C, and \agonistic~$\to$ D. \amy{prob too much detail? not necessary to know letters?}
Each participant was randomly assigned to an ordering of interfaces, with \baseline~fixed as the first interface (e.g., [\baseline~\agonistic, \diverse, \reformulative] --- 6 unique orderings), to account for learning effects between non-\baseline~interfaces.
The interviewer received consent to record the interview and use anonymized data.
% collect artifacts produced during the interview, and use anonymized quotes from interview.
% Participants were asked to share their screen.
Then, to familiarize the participants with the interfaces, participants were guided through using each interface in their assigned order with the prompt ``a person.''
Afterwards, participants selected a subject (prompt) for their task; they were told to choose one of interest to them and encouraged to choose one in a randomly pre-assigned category.
Subjects all featured people and were divided into three categories --- identity  / demographics, history, and political issues. 
This choice of categories allowed for potential reflection on issues of visual representation.
% Participants were encouraged to choose a subject in a randomly pre-assigned category.
% (such that $\approx \sfrac{1}{3}$ of participants fall in each category) 
% but allowed to choose a subject in a different category if strongly desired.
Participants were provided with examples in each category for inspiration but also allowed come up with their own subjects. 
See \S\ref{participant-background} for participants' chosen prompts and \S\ref{topic-list} for example prompts.
% \amy{starts out past tense and then weirdly shifts to present tense}

\textbf{User Task ($\sim$45 mins).}
After setup, participants were informed of their task: to create a collage of ten images which represented all relevant aspects of the subject.
This task forced participants to make nuanced choices about what to include and exclude in visual representation of their chosen subject.
% for example by including two images which represent different aspects of the subject.
Participants constructed an initial collage of ten images with \baseline.
Then, they \textit{improved} their collage by interacting with the other interfaces (\diverse, \reformulative, \agonistic) in their assigned order. 
If participants produced an image they wanted to add to their collage, they picked one of their existing images to replace.
% They were also allowed not to replace any images at all.
% They were allowed to forgo replacing images if they did not produce better ones with the interface at hand.
The task design of accumulating collage improvement gave permanence to users' decisions about visual representation, provided a holistic picture of how users interacted with interfaces across different stages of exploration, and was a more natural and engaging task for participants compared with other experiment designs tested.
% (e.g., using \diverse, \reformulative, and \agonistic~to independently improve upon \baseline's collage).
% Participants interacted with each of the four interfaces for approximately ten minutes.
Participants were instructed to ``think out loud,'' commenting on their choices to include or exclude images in their collage.
Figure~\ref{fig:example-collage-progression} displays an example collage progression from the study.

% maybe note somewhere that participants were instructed to disregard accuracy of text in image generations


\subsection{Recruiting Procedure and Study Chronology}
% \assign[Andre]{...}}
\label{recruiting-procedure}

After receiving IRB approval from our university, we sent recruiting materials to undergraduate and graduate students in university departments and student groups via online communication channels.
We aimed to represent a variety of backgrounds, experiences, and interests in our participant pool.
We piloted an initial version of our study with 11 individuals and revised the task to its current form (as described in \S\ref{task-structure}) in response to time constraints and perceived confusion about the task.
Then, the two co-first authors individually conducted interviews with 29 individuals over Zoom.
Participants were compensated $\$20$ per hour.
More information about participants is available in \S\ref{participant-background}.
% \andre{This section is kind of short. Do we need it as standalone or maybe we can integrate into previous section?}
% \amy{some other details can be dropped but we should keep at least some participant info in the main body.}



\subsection{Measures}
\label{measures}

Given the work discussed in \S\ref{hai-reflection},
we measured reflection by looking quantitative and qualitative signals of how much individuals questioned or changed prior assumptions after using our interfaces. 
While it is more common to use purely qualitative metrics when measuring reflection \cite{bentvelzenetal2022revisitingreflection}, we believed a mixed-methods approach offered unique benefits for our study because quantitative metrics would allow us to more precisely compare reflection across interfaces.  
We drew from common metrics in mixed methods studies of reflection such as interviews, questionnaires, and self-reports \cite{bentvelzenetal2022revisitingreflection}. 

\textbf{Survey.}
After participants interacted with each interface, they provided responses on a 5-point Likert-style scale
% (1 $=$ ``not at all'', 3 $=$ ``somewhat'', 5 = ``entirely'')
to measure perceived rethinking (how much their mental image changed), satisfaction, appropriateness, and control (see \S\ref{survey-questions}). 
For the \reformulative~and \agonistic~interfaces, we also collected responses about how interesting the suggestions/interpretations were.
Quantitatively measuring self-reported rethinking helped capture reflection not have been recorded by other artifacts like collages (for instance, when an image caused reflection but was not added). 
We measured non-reflection variables (e.g., control) to explain why interfaces might have different reflection.

\textbf{Interview Coding.}
After conducing interviews, the two co-first authors coded interview transcripts to systematically extract further information about user experience. 
They independently coded a small subset of interviews before coming together to build a coding ontology. 
We divided codes into \textit{intents} (how user intents changed while adding an image, roughly correlated to different degrees of reflection) and \textit{values} (reasons users give for adding an image). 
Values are not mutually exclusive, because a user can invoke multiple values when adding an image.
% We decided to code for each image individually to be able to perform more rigorous quantitative coding analyses.
% We therefore note that the reflection captured by our coding ontology is a subset of all reflection during interviews, because reflection at times does not correlate to any particular image or is caused by images that were ultimately rejected by the participant.
The two first co-authors then independently coded a subset of three interviews to compute inter-rater reliability (IRR), revising the coding ontology and independently re-coding interviews as needed to resolve disagreements. 
The final IRR was 0.67 based on a weighted average of Cohen's Kappa scores across value codes. 
After coding all interviews, the two co-first authors reviewed all images marked with non-\direct~intent together to establish more agreement on intent codes.
See \S\ref{irr-methodology} for our IRR methodology and \S\ref{adding-images}, \S\ref{why-add} for a delination of intents and values.

% \andrew{TODO: move to findings}
% \amy{hmm, often codebooks are in an appendix, do you think you need it here for ppl to understand the findings?}
% \amy{after reading the rest of the paper, I'd move these to the relevant part of the findings, maybe as a table along with results counts/quotes. can do a longer codebook in appendix}




% \amy{I feel like something that's missing is something like "Measures" - basically what outcome measures are we trying to capture and how are we capturing it? The big thing I want to be able to do here is to put forth a definition of "reflection" and how we measure it grounded in prior lit. need to write defensively with R2 in mind as this is tricky to measure.}
\section{\textit{How} do interfaces induce reflection?} 
\label{how-reflection}


\begin{figure}[!b]
    \centering
    % First subfigure
    \begin{subfigure}[t]{0.55\textwidth}
        \centering
        \includegraphics[height=4.3cm]{assets/results-visuals/basicimg-a.png}
        \caption{Overall, by interface.}
        \label{fig:subfig1}
    \end{subfigure}
    \hfill
    % Second subfigure
    \begin{subfigure}[t]{0.44\textwidth}
        \centering
        \includegraphics[height=4.3cm]{assets/results-visuals/bascimg-b.png}
        \caption{Breakdown within interface by prompt categories: ``H'' for History, ``P'' for Politics, ``I'' for identity.}
        \label{fig:subfig2}
    \end{subfigure}

    % Whole figure caption
    \caption{Participant ratings on a 5-point scale for how much interacting with the interface made them rethink their mental picture.}
    \label{fig:mental-image}
\end{figure}


% (establishing interface -> intent)}

% \assign[Andre]{write section header}

In this section, we explore \textit{how} different interfaces induce reflection by investigating variables that directly measure some aspect of reflection collected in the study.
The aim is to paint a holistic picture of how reflection occurred under different interfaces. 
% we attempt to explain \textit{why} by investigating other variables in \S\ref{why-reflection}.
In Table~\ref{tab:qualitative-examples} we provide two examples of reflection for each of the interfaces from our interviews.
To characterize reflection across different interfaces, we investigate users' self-reported change in mental image (\S\ref{change-mental-image}) and reflection on a per-image level (\S\ref{adding-images}) as described in \S\ref{measures}.

\begin{table}[!t]
\centering
\begin{tabularx}{\textwidth}{|l|c|X|}
\hline
 & \textbf{PID} & \textbf{Examples of Participant Reflection} \\ \hline
\multirow{3}{*}{\rotatebox{90}{\baselinebox}}
    & \pref{24} & \footnotesize The participant noticed that all the images for ``the constitutional convention'' featured the delegates sitting. In subsequent generations, they specified that the delegates should be standing. They reflected: ``\textit{when I kept generating those prompts and they were sitting [...] I don't think I would have specified whether they were sitting or standing before. Since I thought about it, I had to... play around with the prompt to get them to stand}''. They further remarked: ``\textit{Interacting with this interface [...] made me pay more detailed attention to what I perceive would have actually happened at that historical event.}'' \\ \cline{2-3}
    & \pref{4} & \footnotesize After generating a few images for the subject ``a queer community'' which seemed to ``\textit{rely on heavy stereotypes [...] like pride parades}'', the participant wanted to generate ``\textit{everyday examples of queer people}''. After producing a few images, they remarked ``\textit{now you lose the queer part}.'' After a few moments, they reflected: ``\textit{Well now you really got me questioning myself [...] what is a queer community? Do they all have to be wearing pride flags? How do you tell? Now I'm getting in my own head about it.}'' \\ \hline
\multirow{3}{*}{\rotatebox{90}{\diversebox}}
    & \pref{19} & \footnotesize The participant was attempting to generate an image of ``A Syrian refugee working in Germany'' when \diverse~generated an image of a woman in a healthcare setting. The participant paused and remarked, ``\textit{can you train to be a doctor [in] 12 years?}'' They later explained that they were unsure whether a Syrian refugee would have been able to study long enough to become a licensed medical professional since leaving Syria, but reasoned that ``\textit{the situation has gone on long enough [that] they have studied and integrated in the societies they're in, where [they're] able to actually finish something in the medical sciences.}'' \\ \cline{2-3}
    % & PID 5 &  \\ \cline{2-3}
    & \pref{29} & \footnotesize The participant had previously generated all images of White people for the prompt ``a Jewish person'' using \baseline. After \diverse~produced images of Black people, the participant felt that the images were not appropriate but became confused upon further reflection: ``\textit{Okay, I'm focused on the Black [...] person here because, this is because my lack of knowledge [...] I don't know what makes a person Jewish other than the religious element [but...] to be Jewish, I think you have to be born in a family [...] I could be wrong.}'' \\ \hline
\multirow{3}{*}{\rotatebox{90}{\reformulativebox}} 
    & \pref{10} & \footnotesize With the prompt ``A gun owner,'' \reformulative~generated the suggestion, ``An elderly man polishing his grandfather's antique shotgun.'' Upon reading the suggestion, the participant reacted, ``\textit{I didn't even think about that kind [of] collector
gun owner,}'' before using the suggestion to generate and add an image to their collage. They later reflected that the interface ``\textit{reminded [them] of things [they] forgot about}'' and ``\textit{sparked a new mental image}'' for them. \\ \cline{2-3}
    % & PID 8 &  \\ \cline{2-3}
    & \pref{19} & \footnotesize Given the prompt ``A gun control advocate,'' \reformulative~generated a suggestion of ``A gun control advocate engaging in a community discussion.'' The participant reacted, ``\textit{I like that it's suggested positioning the advocate in a community space, where all of the other images [from previous interfaces] were very portrait-style.}'' They used the suggestion to generate and add an image to their collage, later sharing that they liked how \reformulative~helped them ``\textit{think of things that [they] wouldn't think to ask.}'' \\ \hline
\multirow{3}{*}{\rotatebox{90}{\agonisticbox}}
    % & \pref{6} & \footnotesize After previously choosing not to include a darker-skinned picture of the subject ``Jesus'' produced by \diverse, the participant was presented with two interpretations of Jesus --- Jesus as an olive-skinned Jew and as a red-haired man (from Islamic records) --- both of which they ended up accepting after much thought and reading the respective Wikipedia pages. They reflected: ``\textit{It's really interesting knowing there are so many interpretations of who Jesus is, I think I only thought about how I got taught in the Bible at my Church, but I never really thought about other people think about who Jesus is.}'' \\ \cline{2-3}
    & \pref{5} & \footnotesize After interacting with \agonistic~on the subject ``World War II'' which showed many less well-known and more local actualizations of WWII, the participant (whose family was strongly impacted by WWII) revised their design statement from ``\textit{WWII involved mass conflicts that...}'' to ``\textit{WWII involved conflicts that...}''. They reflected: ``\textit{I think it just involves conflicts in general now. It's making me rethink that word [mass]: history tends to focus on real large-scale things, but I find that all of them are important to me personally. I feel like it's actually made me reconnect to the way I think about World War II. And instead of trying to project what I would like, think that other people need to see about World War II.}'' \\ \cline{2-3}
    & \pref{28} & \footnotesize The participant initially described their mental image of the prompt ``the signing of the Declaration of Independence'' as ``\textit{A group of [...] white men like Thomas Jefferson standing around a table signing a document.}'' When \agonistic~generated images of non-White people, drawing from sources like ``Haitian Declaration of Independence,'' the participant observed, ''\textit{when [...] I put  Declaration of Independence, I assume a mental image of the US Declaration of Independence, but so many nations have Declarations of Independence.}'' They continued, ``\textit{It makes me realize [...] this assumption of bias... Oh my gosh, I was so narrow-minded in my approach.}'' \\ \hline
\end{tabularx}
\caption{Two examples of reflection observed during our interviews when using each interface.}
% \caption{\andre{We will probably just do two examples of reflection from each interface. I think we do want to have concrete examples of what reflection looks like in the main paper and not all in the appendix though} \amy{somewhat hard to tell what this table is about, I suggest renaming "Example".}}
\label{tab:qualitative-examples}
\end{table}



% Some examples of reflection being induced by various interfaces

% Some examples of reflection \textit{not} being induced

% Need to do 

% \begin{wraptable}{r}{0.35\textwidth} % r for right, 0.5\textwidth for table width
% \vspace{-\baselineskip}
% \centering
% \small % Makes the table compact
% \setlength{\tabcolsep}{4pt} % Adjusts column spacing
% \begin{tabular}{@{}lcc@{}}
% \toprule
% \textbf{Interface} & \textbf{Rethinking} & \textbf{Min-Max Scaled} \\ \midrule
% \agonistic~        & $2.97 \pm 1.19$     & $0.78 \pm 0.39$         \\
% \reformulative~     & $1.93 \pm 1.05$     & $0.37 \pm 0.41$         \\
% \diverse~          & $1.48 \pm 0.93$     & $0.17 \pm 0.35$         \\
% \baseline~         & $2.03 \pm 1.00$     & $0.43 \pm 0.44$         \\ \bottomrule
% \end{tabular}
% \caption{Rethinking scores (5-point Likert-style scale) and min-max scaled scores (scaling done per-participant).}
% \label{tab:rethinking_scores}
% \vspace{-\baselineskip}
% \end{wraptable}

% \begin{table}[!t]
% \vspace{-\baselineskip}
% \centering
% \small % Compact font size
% \setlength{\tabcolsep}{6pt} % Adjust column spacing
% \begin{tabular}{@{}ccccccc@{}}
% \toprule
% & \multirow{2}{*}{\textbf{Interface}} & \multicolumn{4}{c}{\textbf{5-Point Rating}} & \multirow{2}{*}{\textbf{Min-Max Scaled}} \\ \cmidrule(lr){3-6}
% &                                      & \textbf{Overall}   & \textbf{Identity} & \textbf{Politics} & \textbf{History} & \\ \midrule
% & \baselinebox                  & $2.03 \pm 1.00$    & $1.90 \pm 1.14$    & $2.20 \pm 0.87$    & $2.00 \pm 0.94$   & $0.43 \pm 0.44$ \\
% & \diversebox                   & $1.48 \pm 0.93$    & $1.50 \pm 1.02$    & $1.60 \pm 1.02$    & $1.33 \pm 0.67$  & $0.17 \pm 0.35$ \\
% & \reformulativebox             & $1.93 \pm 1.05$    & $1.80 \pm 0.98$    & $2.60 \pm 1.11$    & $1.33 \pm 0.47$  & $0.37 \pm 0.41$ \\
% & \agonisticbox                 & $2.97 \pm 1.19$    & $2.90 \pm 0.94$    & $3.00 \pm 1.18$    & $3.00 \pm 1.41$   & $0.78 \pm 0.39$ \\ \bottomrule
% \end{tabular}
% \caption{Comparison of overall, identity, politics, history prompt ratings, and min-max scaled scores across interfaces. Min-max scaling is applied to each participant, with their lowest rating set to 0 and their highest set to 1. \andre{Maybe remove groupings by category?}}
% \label{tab:grouped_ratings}
% \vspace{-\baselineskip}
% \end{table}



\subsection{Change in mental image}
% \assign[Andre]{...} }
\label{change-mental-image}

The change in mental image is reported by participants after they use each interface and is measured by agreement with the statement ``Interacting with the interface made me rethink the mental picture of the subject I had right before using this interface'' on a 5-point Likert-style scale.
% where 1 is ``Not at all; my mental picture stayed the exact same'', 3 is ``Somewhat; my mental picture changed in some ways'', and 5 is ``Entirely; my mental picture is very different''.
We take higher reported changes in mental image to be indicative of more reflection.
Averages are reported in Figure~\ref{fig:mental-image} and Table~\ref{tab:grouped_ratings}. 
% \amy{figure instead? grouped bar chart of avgs with SD lines for a set of the questions? *s for significance}

% \amy{consider a bolded topic sentence in front of each para(s) with the summarized finding.}
\textbf{\agonistic~induces the most reflection as measured by self-reported rethinking.} Participants' mental images change more under \agonistic~than \reformulative~($p \ll 0.01$), and more under \reformulative~than \diverse~($p = 0.05$). Neither \reformulative~nor \diverse~surpass the \baseline (likely due to the fact that participants interact with \baseline~first), but \agonistic~does ($p \ll 0.01$).
% Given that participants interact with \baseline~first, this suggests that only \agonistic~provides a larger \textit{unique} change in mental picture which \textit{exceeds} than the initial \baseline.
% \diverse~induces lower mental change than \baseline~with $p = 0.03$. \amy{confusingly worded}
The `hierarchy' of interfaces by reflection is thus $\left[ \agonistic> \{ \baseline, \reformulative\} > \diverse\right]$.
% These differences are even more pronounced when min-max scaling each participant's responses such that the lowest value across the four interfaces is mapped to 0 and the highest to 1.
A similar `hierarchy' emerges when considering how participants rank interfaces by how much their mental image changed retroactively at the end of the study (see \S\ref{rankings-mental-image-change}), as well as measuring how much participants' ``design statements'' changed after using each interface (see \S\ref{design-statement}). \agonistic~further induces greater unique rethinking based on an analysis of ordering effects (see \S\ref{reflection-ordering}).
% \agonistic~induces more reflection than \reformulative~by a variety of measurements, and likewise \reformulative~over \diverse.

\textbf{Rethinking under \agonistic~is robust to prompt category.} 
Though  \diverse~and \reformulative~see significant drops in rethinking for historical prompts compared to political prompts ($-0.27$ and $-1.27$, respectively) the same drop is not observed for \agonistic~($-0.00$).
% Interestingly, for historical topics, low rethinking happens under \textit{both} \diverse~(1.33) and \reformulative~(1.33), and much higher with \agonistic~(3.00).
% The same rethinking happens under \agonistic~(3.00) for political topics, but also substantially under \reformulative~(2.60).
While all domains have roughly similar rethinking under \agonistic, political subjects lend themselves to more rethinking overall under \diverse~and \reformulative~than historical subjects.
% \amy{maybe also a figure broken down by category? these things are much easier and shorter to eyeball than try to describe, the text is kinda convoluted.}


\subsection{Reflection when adding images}
% \assign[Andrew]{...}}
\label{adding-images}

% \begin{table}[!h]
% \vspace{-\baselineskip}
% \centering
% \small % Compact font size
% \setlength{\tabcolsep}{6pt} % Adjust column spacing
% \begin{tabular}{@{}ccccccc@{}}
% \toprule
% & \multirow{2}{*}{\textbf{Interface}} & \multicolumn{4}{c}{\textbf{Intents}} \\ \cmidrule(lr){3-6}
% &                                      & \textbf{Direct}   & \textbf{Reminder} & \textbf{Expansion} & \textbf{Challenge} & \\ \midrule
% & \baselinebox                  & \textbf{0.97} & $0.00$ & $0.03$ & $0.00$   \\
% & \diversebox                   & \textbf{0.91} & $0.02$ & $0.07$ & $0.00$   \\
% & \reformulativebox             & \textbf{0.87} & $0.04$ & $0.08$ & $0.00$   \\
% & \agonisticbox                 & \textbf{0.62} & $0.12$ & $0.21$ & $0.05$   \\
% \bottomrule
% \end{tabular}
% \caption{Proportion of added images per interface with each type of intent. \amy{explain this more here} \amy{this could be a stacked bar chart instead maybe}}
% \label{tab:grouped_ratings}
% \vspace{-\baselineskip}
% \end{table}

Next, we aim to capture a richer picture of reflection by analyzing the distribution of intent codes across interfaces. 
As explained in \S\ref{measures}, intents code for how user intents changed while adding an image to their collage.
Intent codes are defined as follows: 1) \direct: the user already intended to generate the image; 2) \reminder: the user is reminded of a detail about the prompt they already accepted but forgot; 3) \expansion: the user accepts an image as a different valid interpretation of the prompt than their original intent; 4) \challenge: the user realizes an error or otherwise experiences a significant change in their original intent. \direct~intent represents little to no reflection, whereas \reminder, \expansion, and \challenge~represent increasing degrees of reflection. The distribution of intents across interfaces is shown in Fig. \ref{fig:intents_distributed}
% \amy{forgot now. wonder if you want to move the definitions here instead.}

\textbf{\agonistic~induces the most reflection as measured by quantitative interview analysis.}
Participants replace the most images on average with \agonistic (3.1), followed by \reformulative~(2.9) and \diverse~(1.7).
We observe \direct~intent most frequently with \baseline~($98\%$), followed by \diverse~($95\%$), \reformulative~($90\%$), and a sharp drop in \agonistic~($67\%$). \agonistic~also has the highest proportion of \reminder~and \expansion~of intent across all interfaces ($11\%$ and $19\%$ respectively, compared to the next-highest $4\%$ and $6\%$ for \reformulative), and is the only interface to \challenge~ intent ($4\%$). 
An example of \challenge~intent by \pref{28} is given in Table~\ref{tab:qualitative-examples}. A minor difference with \S\ref{change-mental-image} is the finding that \diverse~induces more frequent reflection than \baseline~(see \S\ref{intent-coding-appendix}).

% The quantitative analysis of intents largely supports and extends the findings from \S\ref{change-mental-image}.
% % \amy{maybe an example here...}

% \begin{wrapfigure}{l}{0.55\linewidth} % 'l' aligns left, and width is set to 50% of the text width
%     \centering
%     \includegraphics[width=\linewidth]{assets/results-visuals/intent-distribution.png}
%     \caption{Breakdown of reported intents when adding images for each interface.}
%     \label{fig:enter-label}
% \end{wrapfigure}






% Changes to the design statement reflect changes in how the user views.
% design statements can also change even when users don't add any images / few images because their view on what they've created changes. so it adds this different dimension to things

% Compare automatic

% Also investigate changes at all vs. no changes

\begin{figure}[!b]
    \begin{minipage}[t]{0.45\textwidth}
        \centering
        \includegraphics[height=4.4cm]{assets/results-visuals/intent-distribution.png}
        \caption{Breakdown of how user intents changed when adding images for each interface (discussed in`\S\ref{adding-images}; Table~\ref{tab:intent-distribution-table}).}
        \label{fig:intents_distributed}
    \end{minipage}
    \hfill
    \begin{minipage}[t]{0.53\textwidth}
        \centering
        \includegraphics[height=4.4cm]{assets/results-visuals/properties.png}
        \caption{Mean responses on a 5-point scale for Rethink, Appropriateness, Control, and Interest (discussed in \S\ref{interface-properties}; Table~\ref{tab:explicit_values_with_std_properties}).}
        \label{fig:properties-visual}
    \end{minipage}
\end{figure}
\section{\textit{Why} do different interfaces induce reflection?}
\label{why-reflection}

To explain the observed differences in reflection under different interfaces as discussed in \S\ref{how-reflection}, we look at the relationship between reflection and other variables: perceived interface-level properties like appropriateness and control (\S\ref{interface-properties}), participants' values when adding images (\S\ref{why-add}) and why they reject images (\S\ref{why-reject}), and qualitative themes (\S\ref{qualitative-themes}).
% (explaining interface -> variables -> intent)}

% \assign[Andre]{write section header}

\subsection{Perceived Interface-Level Properties}
 % \assign[Andre]{...}}
\label{interface-properties}

% \begin{wrapfigure}{l}{0.55\linewidth} % 'l' aligns left, and width is set to 50% of the text width
%     \centering
%     \includegraphics[width=\linewidth]{assets/results-visuals/properties.png}
%     \caption{Mean participant responses on a 5-point scale for three dimensions (Appropriateness, Control, Interest). The ``Rethink'' response is re-included for comparison.}
%     \label{fig:enter-label}
% \end{wrapfigure}

\textbf{\reformulative~and \agonistic~improve upon \baseline~and \diverse~in appropriateness and control ($p \le 0.05$ for all).}
Indeed, in general, there is a weak but significant correlation between rethinking and both appropriateness ($r = 0.22$, $p = 0.02$) and control ($r = 0.27$, $p = 0.003$) in general.
\agonistic~is ranked above \diverse~by $81\%$ of participants for appropriateness and $86\%$ for control; \reformulative~by $85\%$ for appropriateness and $92\%$ for control.
% Given that \reformulative~and \agonistic~dominate \diverse~in rethinking, 
% This suggests that interfaces which produce results that end up seeming appropriate and engage users in ways that make them feel in control may encourage more rethinking.
% Indeed, there is qualitative evidence to support this, which we discuss in \S\ref{qualitative-themes}. 
% \andre{We can also discuss the qualitative evidence here.}
Although \diverse~cannot be distinguished significantly from \baseline~by appropriateness, it can by control --- \diverse~has lower control than \baseline~($p \ll 0.01$), meaning that rewriting user prompts to encourage diversity results in a noticeable feeling of decreased control among participants.
% \andre{worth discussion this?}
That being said, \agonistic~and \reformulative~are not themselves distinguishable by control or appropriateness ratings ($p \gg 0.05$).
% Indeed, only $59\%$ of people prefer \agonistic~to \reformulative~for appropriateness and exactly $50\%$ for control.
% Given that there is a noticeable difference between \agonistic~and \reformulative~for rethinking (as reported in \S\ref{how-reflection}), what explains the difference between these two, if not appropriateness and control?
Additionally, participants find \agonistic's possible interpretations more interesting than \reformulative's suggested reformulations ($p \ll 0.05$).

% although both are rated highly  in general ($\ge 4$).
% To continue explaining the difference in observed reflection between the two, we conduct other forms of analysis both at lower (\S\ref{why-add}, \S\ref{why-reject}) and higher (\S\ref{qualitative-themes}) levels.

% \amy{more detail than the reader cares to know?}
% \amy{you can do it a little bit but it seems like there's a good bit of interpretation happening in these findings sections, can you more directly say the findings and do most of the interpretation in the Discussion?}


% We can further analyze correlation with satisfaction...
% Very interestingly, while there is a significant weak correlation between satisfaction and both appropriateness ($r = 0.31$, $p \ll 0.01$) and control ($r = 0.23$, $p = 0.01$), there is no correlation with rethinking at all ($r = 0.02$, $p \gg 0.05$).
% [Maybe also measure with satisfaction deltas]
% \andre{disentangle satisfaction from appropriateness -- important concept here}
% Almost near zero correlation between satisfaction and reflection, suggesting that people do not necessarily feel more satisfied even when there is reflection.
% \andre{need to do more analysis on this part -- not just abslute/relative, alos exclude when the satisfaction has reached 5 and there is nowhere to go, or otherwise scale by satisfaction...}


% (interface -> properties -> intent)}

% After interacting with each interface, users report on the interface's appropriateness, control, and interestingness. \andre{introduction to the data collection source, maybe even specify what each of the ends are}

% The 

% \andre{add table with features (appropriateness, control, interest) by interface}

% The appropriateness and control attributes help distinguish C and D from A and B, which corresponds with an observed gap between C and D and B.

% Even though we do observe this broad interface-level observation, the subject-level variation is actually only quite weak: 0.20 for rethinking and appropriateness, 0.30 for rethinking and control.

% \andre{standardize by each subject -- mean zero and unit variance, or something like that}

% We have qualitative reasons to believe that both appropriateness and control are important necessary conditions for rethinking, as observed by their presence generally in \reformulative~and \agonistic~and absence in \baseline~and \diverse.

% \andre{provide qualitative descriptions}

% \andre{segueue into next section}
% However, D produces notably more reflection than C by several measurements (reference previous section).
% This is not only explained by appropriateness and control, which appear to be necessary but not sufficient conditions for high reflection.
% The interestingness may get at it but there is not that much difference either.
% Therefore, the mystery now is to consider how D varies by C. To do so, we look at a more granular level at the \textit{image-level} values before taking a broader view at \textit{qualitative themes} that distinguish the interfaces


\subsection{Image-Level Values: Why users add images}
\label{why-add}

% (interface -> values -> intent)}



\begin{figure}[!t]
    \begin{minipage}[t]{0.53\textwidth}
        \centering
        \includegraphics[height=4.4cm]{assets/results-visuals/values.png}
        \caption{Distribution of values cited for images added with each interface (discussed in \S\ref{why-add}; Table~\ref{tab:values-interfaces}).}
        \label{fig:values-interfaces}
    \end{minipage}
    \hfill
    \begin{minipage}[t]{0.45\textwidth}
        \centering
        \includegraphics[height=4.4cm]{assets/results-visuals/intents-values-distribution.png}
        \caption{Breakdown of intents by value cited when adding image (discussed in \S\ref{why-add}); Table~\ref{tab:values-intents-rel}.}
        \label{fig:intents-values}
    \end{minipage}
\end{figure}



% \andre{Can probably put these side by side}


We also explore whether different values may serve as a relevant explanatory variable for types of reflection (intents). Values code for \textit{why} participants add images, or in other words the reasons that participants invoke when selecting images. The values in our coding ontology are as follows: 1) \realism: the image represents how the user thinks the world actually is; 2) \familiarity: the image fits the user's assumptions, regardless of whether the user thinks they are representative of how the world actually is; 3) \diversity: the image portrays an underrepresented aspect of the prompt that the user believes is normatively important to include; 4) \aesthetics: the user otherwise likes how the image looks. The distribution of values across interfaces and intents is shown in Fig. \ref{fig:values-interfaces} and \ref{fig:intents-values}.

\textbf{\agonistic~is the only interface for which \diversity~is invoked more frequently than \familiarity.} In all interfaces \textit{except} \agonistic, \familiarity~is invoked most frequently, followed by \realism, \diversity, and \aesthetics. In contrast, for \agonistic, participants invoke \diversity~most frequently, followed by \realism, \familiarity, and \aesthetics. Interestingly, compared to \baseline, participants invoke \diversity~\textit{less} frequently for \diverse~($-2\%$), but more frequently for \reformulative~($+12\%$) and \agonistic~($+19\%$).


% \textbf{\reformulative~evokes the lowest amount of \realism~but the highest amount of \familiarity.} While participants invoke \realism~at similar rates across all other interfaces ($39\%$-$42\%$), they do so at significantly lower rates for \reformulative~($24\%$). However, they also invoke \familiarity~at much higher rates for \reformulative~than other interfaces ($53\%$ over $42\%$-$47\%$).
% \amy{happens elsewhere too but I feel like there's got to be a way to say your major findings more directly and succinctly. feels like lots of little things being reported, most of which are not that useful to know?}
% \andrew{TODO: cut down to most important findings}

\textbf{\diversity~is associated with the highest rate of reflection.}
The majority of images across values are added with \direct~intent. Among values, \aesthetics~ is associated with the highest highest proportion of \direct~intent ($96\%$), followed by \familiarity($95\%$), \realism~($89\%$), and \diversity~($75\%$). \diversity~is associated with the highest degrees of reflection, with $7\%$ of images added with \reminder, $16\%$ of images added with \expansion, and $2\%$ of images added with \challenge. Since \agonistic~also has the highest proportion of \diversity~values, these findings suggest that \diversity~serves as a strong explanatory variable for reflection.

% \andre{Note --- we may find that users accept/reject on the basis of un/familiarity much more than other types of values like diversity and factuality/realism, if this is true then there are big design implications, people aren't talking very much about familiarity and the associated concepts right now, could be good to highlight that while we care about factuality, diversity, etc. the values that people are invoking most even when thinking about these political/identity/historical issues is familiarity, so we should keep that in mind when designing. might also be a good discussion section --- discussed in discussion section}

\subsection{Image-Level Values: Why users reject images}
\label{why-reject}

% \assign[Andrew]{coordinate section}
% \assign[Andre]{add compelling pieces}


% \andre{Andrew: compute reasons for rejections as proportions by interface -- what percent type for each interface; we can qualify that it may not be super quantitatively rigorous but it's one number which could be interesting --- going to try just qualitative first}

We carry out a qualitative analysis of image rejection in lieu of a quantitative analysis because the unit of rejection is not as well-defined as adding images.

\textbf{Participants reject images due to factual inaccuracies, especially when interacting with \diverse.}
% When \diverse~generated an image of Black and White military members working together for the prompt ``World War II,'' \pref{17} responded: ``\textit{the army was only desegregated after World War II, so this photo [...] is just like a fantasy.}'' Similarly, 
\pref{7} said that racially diverse pictures of `The Chinese Communist Revolution' ``\textit{looked like modern people were slapped into the wrong time period,}'' and \pref{28} experienced ``\textit{pushback}'' against racially diverse pictures of ``The Signing of the Declaration of the Independence.'' 
In other interfaces, participants similarly cited lack of factuality as a reason for rejecting images.
% though the inaccuracies were less often related to inaccurate demographic diversity. 
For instance, \pref{2} reacted ``\textit{Middle class people don't dine out very often, and especially not at fancy places},'' when \baseline~generated an image of people at an upscale restaurant for the prompt ``A middle class person.'' 
% \amy{to save space, can just do 1-2 examples per point}

% On the Constitutional Convention, \pref{24} said ``\textit{it's unfortunate, but it was kind of a reality}''

\textbf{Participants also reject images because of an excessive adherence to factuality.} 
\pref{26}, generating images of ``A Syrian refugee'' noted that \agonistic~``\textit{presented more of what I know to be true, [...] the harsh reality of it. [...] The other interfaces [...] were more instrumental in helping me expand beyond that.}'' 
\pref{7} shared that they preferred \reformulative~over \agonistic~because \reformulative~did a better job ``\textit{highlighting all different aspects of [their prompt], instead of just the realism images}.'' Such comments suggest that users do value an some level of diversity in generated images that may extend beyond what they consider to be strictly factual.

\textbf{Participants reject images due to lack of familiarity, even where they acknowledge that an image may be factual.} 
When using \agonistic, participants may believe interpretations to be factual (being extracted from Wikipedia) but still reject them because they are unfamiliar. 
% We observe that lack of familiarity is often invoked for \agonistic, which justifies generations with information from Wikipedia. 
For example, with the prompt ``An Arab person'', \pref{25} reacted to an image of a Sudanese Arab person from \agonistic~by remarking, ``\textit{Oh, Sudanese Arab [...] I think that's an identity that I haven't represented here that I would like to.}'' 
They later decided to reject the image by providing the following reasoning: ``\textit{It just does not resonate with me. I guess I'm also not familiar with Sudanese Arabs as much.}''
% \andrew{@Andre @Andrew find a potentially better example?}
% Nonetheless, we emphasize that reflection can still occur when participants reject images, such as how the suggestion of Sudanese Arabs reminded the participant of an aspect of Arab identity that they had previously not considered.

% \subsection{Affective Responses to Interfaces}

% \subsection{W}
% (interface -> anything -> intent)}

% \assign[Andre]{...}

% \assign[Andrew]{...}

% Collection of themes

% We want to investigate how people's emotions / effects correlate to reflection.
% \andre{maybe give definition of affect / emotion}


% We also have other measures of affect, such as qualitatively, e.g.: frustration, offense, boredom, etc.?

% Reflection is often encountered in the form of... confusion, ... ?



% What are people seeing from A, B, C< D, etc/ What uniquely are they getting?

% Affective responses to interfaces (interface -> emotions)
% qualitative



% features:
% - person
% - 10 slots for images -- drop down to code intention and value
%     - repeat 4x for every image -- only change the one that was replaced
% - also have list for removing images
% - capture our entire ontology for adding/not adding images/suggestions
% - have an open ended feature for discussing affective responses
% - have an open ended feature about how the participant reflected
% - nice qualitative quotes / ideas / "what this participant brings to the table / our study"


\subsection{Qualitative Themes}
\label{qualitative-themes}

We present several themes from qualitative analysis further explaining the differences in reflection observed in \S\ref{how-reflection}.

% \textbf{Affective responses to interfaces.}
% [Describe different affects.]
% Surprise seems important...

\textbf{Text helps users contextualize, interpret, and decide on images.} 
% \andre{mine}
% Roland Barthes remarked that ``\textit{all images are polysemous; they imply, underlying their signifiers, a ``floating chain'' of signifieds, the reader able to choose some and ignore others}''~\cite{barthes_rhetoric_image} --- images are `pure visual fields' that can be contextualized and given meanings in many different ways.
% \amy{move to Discussion?}
Throughout the user studies, we find that participants prefer interfaces that help them contextualize and interpret images (\reformulative~and \agonistic~over \baseline~and \diverse), even though ultimately they are only adding the images (and not the accompanying text) to the collage.
Firstly, contextualizing content is cognitively stimulating: 
``\textit{[the] suggestion feature... by nature makes your brain think about alternatives}'' (\pref{24});
it ``\textit{I wouldn’t have thought of [these perspectives] without the text}'' (\pref{1}).
% rompted me to think about the subject in different ways. 
% Indeed, ``\textit{even if [the suggestions are] inaccurate [...] it's still somewhat stimulating [...] your brain to think about, `oh, it's not accurate because of this'}'' (\pref{23}). 
% \amy{quote is confusing, does it need more punctuation? cut bits and replace with ellipses?}
Participants feel that it is meaningful and important to interpret the image, which is mediated by thinking about the relevant text.
\pref{1} said that the generated images ``\textit{adhere to the prompt but you have to figure out the image yourself}''.
\pref{15} said of \reformulative: ``\textit{I like with this interface. It gives me a lot more of an explanation. It kind of tells a story that I might not know just by looking at it... I think that it could help me kind of change what I'm looking for.}''
% Sometimes, accompanying text can be explanatory or informative: when interacting with \agonistic, \pref{27} said ``\textit{I didn’t know this was a historically attired Christian. So I think that explains a lot about previous imagery that generated}'', in reference to seeing similarly dressed individuals in \baseline~and \diverse~(but without explanation).

% Moreover, contextual content helps participants better correlate model inputs and outputs, developing a cognitive model of the interface and therefore becoming a more effective user.
% \andre{fill in}




% Images aren't produced in a vacuum; people find text to be really helpful towards explaining and understanding the images they're producing, they care about this. So image interfaces should be designed with supplemental text to encourage reflective generation.
% People want to have text to help them interpret images! Further supports the previously discussed work in photography and film that images are never ``just images'', but have different ethical, political, cultural, etc. dimensions that people want to be part of and engage with.
% Situating the input.
% \andre{@Andrew -- dump any examples that could be relevant}
% \andre{incorporate this into previous} \textbf{People find it important to correlate / mental model tracking inputs and outputs (even interpretability).}
% It not only helps interpret, but helps correlate inputs and outputs -- what is the process of going from my input text to an output? Providing text helps with this.
% Hard to follow along what's going on with B.
% Appropriateness
% James Duff: `It also just has a suggestion feature that, you know, by nature makes your brain think about alternatives.''
% Jenny Suk: ``The detailed text below or next to each picture in suggestions prompted me to think about the subject in different ways. I wouldn’t have thought of [these perspectives] without the text.''


\textbf{`Authenticity' situates diversity in an engaged political context.}
Although both \diverse~and \agonistic~end up displaying demographically diverse images to users, we find overwhelmingly that participants prefer the type of diversity offered by \agonistic~to that offered by \diverse~due to a greater perception of \textit{authenticity}. 
% \andrew{move definition of authenticity here?} 
% \pref{17}, for instance, reacted that they didn't ``\textit{get a clear idea from anything}'' generated by \diverse.
\diverse is not only perceived as factually inaccurate (\S\ref{why-reject}), but \textit{nontransparent, alienating, and ungenuine}.
\pref{17} said they couldn't ``\textit{get a clear idea from anything}'' from the interface.
% For their prompt ``A gun owner'', \pref{10} remarked that the racially diverse images by \diverse~``\textit{don't really connect with that idea of someone who would own a gun.}'' 
\pref{5}
% who described themselves as ``\textit{a person who believes highly in representation and diversity,}'' 
reacted when racially and gender diverse World War II soldiers generated by \diverse: ``\textit{I find them not just [...] historically inaccurate, I think it gives narratives that some people will not dig deeper into, and [...] can lead to a lot of misconceptions down the line of colorblind racism.}'' 
% Though participants generally describe understanding the value of diversity, they clearly do not think that \diverse~embodies that value authentically. Rather, the comments above suggest that diversity appears more authentic to participants when it is historically informed and connected to salient features of the prompt.
In contrast to \diverse, participants report that the perceived authenticity of \agonistic~engages them in the political context of image generation (\S\ref{agonistic-democracy}).
\pref{10} liked that \agonistic~was ``\textit{more grounded in reality}'' --- ``\textit{you're like, `yeah, real people exist.'}''
% not really as fun because 
% I like the [images] from [\agonistic] a lot, because it 
\pref{28} described \agonistic: ``\textit{[it] changed my perception of what the prompt itself was [...] but [\diverse] feels like it's [...] trying to change our perception of the event itself that I was already thinking of [...\agonistic] tries to bring in diverse perspectives and increase representation of other groups [...] in a way that felt more authentic and can change people's assumptions in a healthier way.}'' 
% In these and other interviews, we observe that participants are more likely when interacting with \agonistic~than \diverse~to acknowledge the political context of image generation, considering images as alternative interpretations of their prompt that ``\textit{real people}'' might hold. Participants thereby feel \agonistic~to be more ``\textit{authentic}'' than \diverse, and report being more open to changing their assumptions when interacting with \agonistic. \\

% Similar to \pref{28}'s comment (also mentioned above in \S\ref{why-reject}) that \diverse~was ``\textit{less grounded in [...] history,}'' \pref{10} remarked that they found \agonistic~preferable to \diverse~because it was ``\textit{more grounded in reality}.''
% To this effect, \pref{9} also commented that \agonistic~differed from other interfaces in that it felt more like ``\textit{the AI and the person working together to get to the final image generation.}'' In short, confrontational design can be aided by authenticity cues to more effectively situate image generation in broader political discourse and help users navigate politically sensitive topics.
% \pref{10} said that ``\textit{I find it interesting... but it doesn't \textit{honor} the situation} [of what happened in WWII]''.



% \andre{complete this one -- talk about it may have the highest appropriateness but it really treads the line.}
% Maybe can put the comments about feeling like it's one step forward, one step back (e.g. Calvin) here.
% Language of ``not what I'm looking for'', dynamic between having something not be what you are looking for (using what you want as a compass/measurement to accept/reject vs. changing your compass)

% \andre{put one more theme here @Andrew}


% % \andre{maybe we want to move this to the results section previously?}
% \andre{somewhere here: put this idea of A (specificity) vs. B (quality of images) trade off as interesting to examine with and without prompt reformulation}


\section{Discussion}
The development of foundation models has increasingly relied on accessible data support to address complex tasks~\cite{zhang2024data}. Yet major challenges remain in collecting scalable clinical data in the healthcare system, such as data silos and privacy concerns. To overcome these challenges, MedForge integrates multi-center clinical knowledge sources into a cohesive medical foundation model via a collaborative scheme. MedForge offers a collaborative path to asynchronously integrate multi-center knowledge while maintaining strong flexibility for individual contributors.
This key design allows a cost-effective collaboration among clinical centers to build comprehensive medical models, enhancing private resource utilization across healthcare systems.

Inspired by collaborative open-source software development~\cite{raffel2023building, github}, our study allows individual clinical institutions to independently develop branch modules with their data locally. These branch modules are asynchronously integrated into a comprehensive model without the need to share original data, avoiding potential patient raw data leakage. Conceptually similar to the open-source collaborative system, iterative module merging development ensures the aggregation of model knowledge over time while incorporating diverse data insights from distributed institutions. In particular, this asynchronous scheme alleviates the demand for all users to synchronize module updates as required by conventional methods (e.g., LoRAHub~\cite{huang2023lorahub}).


MedForge's framework addresses multiple data challenges in the cycle of medical foundation model development, including data storage, transmission, and leakage. As the data collection process requires a large amount of distributed data, we show that dataset distillation contributes greatly to reducing data storage capacity. In MedForge, individual contributors can simply upload a lightweight version of the dataset to the central model developer. As a result, the remarkable reduction in data volume (e.g., 175 times less in LC25000) alleviates the burden of data transfer among multiple medical centers. For example, we distilled a 10,500 image training set into 60 representative distilled data while maintaining a strong model performance. We choose to use a lightweight dataset as a transformed representation of raw data to avoid the leakage of sensitive raw information.
Second, the asynchronous collaboration mode in MedForge allows flexible model merging, particularly for users from various local medical centers to participate in model knowledge integration. 
Third, MedForge reformulates the conventional top-down workflow of building foundational models by adopting a bottom-up approach. Instead of solely relying on upstream builders to predefine model functionalities, MedForge allows medical centers to actively contribute to model knowledge integration by providing plugin modules (i.e., LoRA) and distilled datasets. This approach supports flexible knowledge integration and allows models to be applicable to wide-ranging clinical tasks, addressing the key limitation of fixed functionalities in traditional workflows.

We demonstrate the strong capacity of MedForge via the asynchronous merging of three image classification tasks. MedForge offered an incremental merging strategy that is highly flexible compared to plain parameter average~\cite{wortsman2022model} and LoRAHub~\cite{huang2023lorahub}. Specifically, plain parameter averaging merges module parameters directly and ignores the contribution differences of each module. Although LoRAHub allows for flexible distribution of coefficients among modules, it lacks the ability to continuously update, limiting its capacity to incorporate new knowledge during the merging process. In contrast, MedForge shows its strong flexibility of continuous updates while considering the contribution differences among center modules. The robustness of MedForge has been demonstrated by shuffling merging order (Tab~\ref{tab:order}), which shows that merging new-coming modules will not hurt the model ability of previous tasks in various orders, mitigating the model catastrophic forgetting. 
MedForge also reveals a strong generality on various choices of component modules. Our experiments on dataset distillation settings (such as DC and without DSA technique) and PEFT techniques (such as DoRA) emphasize the extensible ability of MedForge's module settings. 

To fully exploit multi-scale clinical data, it will be necessary to include broader data modalities (e.g., electronic health records and radiological images). Managing these diverse data formats and standards among numerous contributors can be challenging due to the potential conflict between collaborators. 
Moreover, since MedForge integrates multiple clinical tasks that involve varying numbers of classification categories, conventional classifier heads with fixed class sizes are not applicable. However, the projection head of the CLIP model, designed to calculate similarities between image and text, is well-suited for this scenario. It allows MedForge to flexibly handle medical datasets with different category numbers, thus overcoming the challenge of multi-task classification. That said, this design choice also limits the variety of model architectures that can be utilized, as it depends specifically on the CLIP framework. Future investigations will explore extensive solutions to make the overall architecture more flexible. Additionally, incorporating more sophisticated data anonymization, such as synthetic data generation~\cite{ding2023large}, and encryption methods can also be considerable. To improve data privacy protection, test-time adaptation technique~\cite{wang2020tent, liang2024comprehensive} without substantial training data can be considered to alleviate the burden of data sharing in the healthcare system.



             

\section{Conclusion}
We reveal a tradeoff in robust watermarks: Improved redundancy of watermark information enhances robustness, but increased redundancy raises the risk of watermark leakage. We propose DAPAO attack, a framework that requires only one image for watermark extraction, effectively achieving both watermark removal and spoofing attacks against cutting-edge robust watermarking methods. Our attack reaches an average success rate of 87\% in detection evasion (about 60\% higher than existing evasion attacks) and an average success rate of 85\% in forgery (approximately 51\% higher than current forgery studies). 




%%
%% The next two lines define the bibliography style to be used, and
%% the bibliography file.
\bibliographystyle{ACM-Reference-Format}
\bibliography{0main}

% \newpage 

\newpage
\appendix
\section{Appendix}

\subsection{Conversational agent prompts for generating stable diffusion prompts in art-making phase}

\textbf{Role:} You will be able to capture the essence of the sessions and drawings in the recordings based on the art therapy session recordings I have given you and summarize them into a short sentence that will be used to guide the PROMPT for the Stable Diffusion model.

\vspace{0.5em} % 添加一些垂直间距

\textbf{Example input:}

\begin{itemize}[leftmargin=*]
    \item \textbf{USER:} [user-drawn] I drew the ocean. [canvas content] There is nothing on the canvas right now.
    \item \textbf{ASSISTANT:} What kind of ocean is this?
    \item \textbf{USER:} [user-drawn] I drew grass. [canvas content] Now there is an ocean on the canvas.
    \item \textbf{ASSISTANT:} What kind of grass is this?
    \item \textbf{USER:} [user-drawn] I drew the sky. [canvas content] Now there is grass and ocean on the canvas.
    \item \textbf{ASSISTANT:} What kind of sky is this?
    \item \textbf{USER:} [user-drawn] I drew mountains. [canvas content] Now there is sky, grass, and ocean on the canvas.
    \item \textbf{ASSISTANT:} What kind of mountain is this?
    \item \textbf{USER:} [user-drawn] I drew clouds. [canvas content] Now there is sky, mountain, grass, and ocean on the canvas.
    \item \textbf{ASSISTANT:} What kind of cloud is this?
    \item \textbf{USER:} [user dialogue] Colorful clouds, emerald green mountains and grass, choppy ocean
\end{itemize}

\vspace{0.5em} % 添加一些垂直间距

\textbf{Task:}

\begin{enumerate}[label=\textbf{Step \arabic*:}]
    \item \textbf{[Step 0]} Read the given transcript of the art therapy session, focusing on the content of \texttt{user: [user drawing]} and \texttt{user: [user dialog]}: Go to \textbf{[Step 1]}.
    \item \textbf{[Step 1]} Based on the input, find the last entry of user's input with \texttt{[canvas content]}, find the keywords of the screen elements that the canvas now contains (in the example input above, it is: sky, grass, sea), separate the keywords of each element with a comma, and add them to the generated result. Examples: [keyword1], [keyword2], [keyword3], \dots, [keyword n].
    \item \textbf{[Step 2]} Find whether there are more specific descriptions of the keywords of the painting elements in \texttt{[Step 1]} in \texttt{[User Dialog]} according to the input. If not, this step ends into \textbf{[Step 3]}; if there are, combine these descriptions and the keywords corresponding to the descriptions into a new descriptive phrase, and replace the previous keywords with the new phrases. Examples: [description of keyword 1] [keyword 1], [keyword 2 description of keyword 2], [description of keyword 3], \dots. Based on the above example input, the output is: rough sea, lush grass, blue sky.
    \item \textbf{[Step 3]} Based on the input, find out if there is a description of the painting style in the \texttt{[User Dialog]} in the dialog record, and if there is, add the style of the picture as a separate phrase after the corresponding phrase generated in \texttt{[Step 2]}, separated by commas. For example: [description of keyword 1] [keyword 1], [description of keyword 2] [keyword 2], \dots, [screen style phrase 1], [screen style phrase 2], [screen style phrase 3], \dots, [Picture Style Phrase n].
\end{enumerate}

\vspace{0.5em} % 添加一些垂直间距

\textbf{Output:} 

Only need to output the generated result of \textbf{[Step 3]}.

\vspace{0.5em} % 添加一些垂直间距

\textbf{Example output:} 

\emph{Rough sea, lush grass}

\subsection{Conversational agent prompts for discussion phase}

\textbf{Role:} <therapist\_name>, Professional Art Therapist

\textbf{Characteristics:} Flexible, empathetic, honest, respectful, trustworthy, non-judgmental.

\vspace{0.5em} % 添加垂直间距

\textbf{Task:} Based on the user's dialogic input, start sequentially from step [A], then step [B], to step [C], step [D], step [E] \dots Step [N] will be asked in a dialogical order, and after step [N], you can go to \textbf{Concluding Remarks}. You can select only one question to be asked at a time from the sample output display of step [N]! You have the flexibility to ask up to one round of extended dialog questions at step [N] based on the user's answers. Lead the user to deeper self-exploration and emotional expression, rather than simply asking questions.

\vspace{0.5em} % 添加垂直间距

\textbf{Operational Guidelines:}

\begin{enumerate}
    \item You must start with the first question and proceed sequentially through the steps in the conversational process (step [A], step [B], step [C], step [D], step [E], \dots, step [N]).
    \item Do not include references like step '[A]', step '[B]' directly in your reply text.
    \item You may include one round of extended dialog questions at any step [N] depending on the user's responses and situation. After that, move on to the next step.
    \item Always ensure empathy and respect are present in your responses, e.g., re-telling or summarizing the user's previous answer to show empathy and attention.
\end{enumerate}

\vspace{0.5em} % 添加垂直间距

\textbf{Therapist’s Configuration:}

\textbf{Principle 1:}  
\textit{Sample question:} How are you feeling about what you are creating in this moment?

\vspace{0.5em}

\textbf{Principle 2:}  
\textit{Sample question:} Can you share with me what this artwork represents to you personally? 

\vspace{0.5em}

\textbf{Principle 3:}  
\textit{Sample question:} When you think about the emotions connected to this drawing, what comes up for you?

\vspace{0.5em}

\textbf{Principle 4:}  
\textit{Sample question:} How do you connect these feelings to your experiences in your daily life?

\vspace{0.5em} % 添加垂直间距

\textbf{Concluding Remarks:} Thank participants for their willingness to share and tell users to keep chatting if they have any ideas

\vspace{1em} % 添加额外的间距

\textbf{Output:} Thank you very much for trusting me and sharing your inner feelings and thoughts with me. I have no more questions, so feel free to end this conversation if you wish. Or, if you wish, we can continue to talk.

\subsection{AI summary prompts}

\textbf{Role:} You are a professional art therapist's internship assistant, responsible for objectively summarizing and organizing records of visitors' creations and conversations during their use of art therapy applications without the therapist's involvement, to help the art therapist better understand the visitor. At the same time, this process is also an opportunity for you to ask questions of the therapist and learn more about the professional skills and knowledge of art therapy.

\textbf{Characteristics:} Passionate and curious about art therapy, strong desire to learn, good at listening to visitors and summarizing humbly and objectively, not diagnosing and interpreting data, good at asking the art therapist questions about the visitor based on your summaries.

\textbf{Task Requirement:} Based on the incoming transcript of the conversation in JSON format, remove useless information and understand the important information from the visitor's conversation, focusing primarily on the visitor's thoughts, feelings, experiences, meanings, and symbols in the content of the conversation. Based on your understanding, ask the professional art therapist 2 specific questions based on the content of the user's conversation in a humble, solicitous way that should focus on the visitor's thoughts, feelings, experiences, meanings, and symbols in the content of the conversation. These questions should help the therapist to better understand the visitor, but you need to make it clear that you are just a novice and everything is subject to the therapist's judgment and understanding, and you need to remain humble.

\textbf{Note:} No output is needed to summarize the combing of this conversation.





\end{document}


\endinput
%%
%% End of file `sample-manuscript.tex'.
