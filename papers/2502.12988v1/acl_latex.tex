% This must be in the first 5 lines to tell arXiv to use pdfLaTeX, which is strongly recommended.
\pdfoutput=1
% In particular, the hyperref package requires pdfLaTeX in order to break URLs across lines.
\PassOptionsToPackage{dvipsnames}{xcolor}
\documentclass[11pt]{article}

% Remove the "review" option to generate the final version.
\usepackage{acl}

% Standard package includes
\usepackage{times}
\usepackage{latexsym}



% For proper rendering and hyphenation of words containing Latin characters (including in bib files)
\usepackage[T1]{fontenc}
% For Vietnamese characters
% \usepackage[T5]{fontenc}
% See https://www.latex-project.org/help/documentation/encguide.pdf for other character sets

% This assumes your files are encoded as UTF8
\usepackage[utf8]{inputenc}
\usepackage[most]{tcolorbox}
% This is not strictly necessary, and may be commented out,
% but it will improve the layout of the manuscript,
% and will typically save some space.
\usepackage{microtype}
\usepackage{amsfonts} 
\usepackage{pifont}
\usepackage{cancel}
\usepackage{soul}
% \usepackage{xcolor}
\usepackage{algorithm}
\usepackage{algorithmic}
\usepackage{booktabs}
\usepackage{xspace}
\usepackage{fontawesome}
\usepackage{mathrsfs}
\usepackage{multirow}
\usepackage{amsmath}
\usepackage{colortbl}
\usepackage{subfig}
\usepackage{graphicx}
\usepackage{tabularx}
\usepackage{arydshln}
\usepackage{makecell}
\usepackage{CJKutf8}


% \usepackage[table]{xcolor} % 为了让表格支持颜色
\usepackage{caption}       % 优化表格/图的 caption

\usepackage[dvipsnames]{xcolor}

\newcolumntype{M}{>{\columncolor{BrickRed!9}}c}      % Multiple-Choice
\newcolumntype{I}{>{\columncolor{YellowGreen!12}}c}    % Generative Question Answering
\newcolumntype{S}{>{\columncolor{Aquamarine!12}}c}   % Style Transfer

\newcommand{\ie}{\textit{i.e.,}\xspace}
\newcommand{\eg}{\textit{e.g.,}\xspace}
\definecolor{bleudefrance}{rgb}{0.19, 0.55, 0.91}
\definecolor{yes}{RGB}{239,211,69}
\definecolor{carminered}{rgb}{1.0, 0.0, 0.22}
\definecolor{crimsonglory}{rgb}{0.75, 0.0, 0.2}
\newcommand{\hlc}[2][yellow]{{%
    \colorlet{foo}{#1}%
    \sethlcolor{foo}\hl{#2}}%
}

\definecolor{err}{RGB}{255,0,128}
\definecolor{corr}{RGB}{65,105,225}
\newcommand{\erritalic}[1]{\textcolor{err}{\textit{#1}}}
\newcommand{\corritalic}[1]{\textcolor{corr}{\textit{#1}}}



\title{Beyond Profile: From Surface-Level Facts to\\ Deep Persona Simulation in LLMs}


\author{
Zixiao Wang\textsuperscript{1} ~~~Duzhen Zhang\textsuperscript{1} ~~~Ishita Agrawal\textsuperscript{1}\\
\textbf{\large Shen Gao\textsuperscript{2}} ~~~
\textbf{\large Le Song\textsuperscript{1}} ~~~
\textbf{\large Xiuying Chen\textsuperscript{1}\thanks{~~Corresponding Author.}~}
\\
\textsuperscript{1}Mohamed bin Zayed University of Artificial Intelligence, \\
\textsuperscript{2}Shandong University \\
\texttt{\{zixiao.wang,duzhen.zhang,xiuying.chen\}@mbzuai.ac.ae}\\
}



\begin{document}
\maketitle
\begin{abstract}
Previous approaches to persona simulation large language models
(LLMs) have typically relied on learning basic biographical information, or using limited role-play dialogue datasets to capture a character’s responses.
However, a holistic representation of an individual goes beyond surface-level facts or conversations to deeper thoughts and thinking.
In this work, we introduce CharacterBot, a model designed to replicate both the linguistic patterns and distinctive thought processes of a character. 
Using Lu Xun, a renowned Chinese writer, as a case study, we propose four training tasks derived from his 17 essay collections.
These include a pre-training task focused on mastering external linguistic structures and knowledge, as well as three fine-tuning tasks: multiple-choice question answering, generative question answering, and style transfer, each aligning the LLM with Lu Xun’s internal ideation and writing style. 
To optimize learning across these tasks, we introduce a CharLoRA parameter updating mechanism, where a general linguistic style expert collaborates with other task-specific experts to better study both the language style and the understanding of deeper thoughts.
We evaluate CharacterBot on three tasks for linguistic accuracy and opinion comprehension, demonstrating that it significantly outperforms the baselines on our adapted metrics.
We hope that this work inspires future research on deep character persona simulation LLM: \faGithub  \href{https://anonymous.4open.science/r/characterbot-9445}{CharacterBot}.
\end{abstract}

\section{Introduction}

\begin{figure}[t]
\centering
\includegraphics[scale=0.75]{figs/intro.pdf}
\caption{
Comparison of Characterbot and other models in responding to Lu Xun-related questions.
}
\label{fig:compare}
\end{figure}

Developing language models that simulate character personas by embodying specific individuals' personalities and worldviews has been a longstanding objective in NLP. 
Researchers in this area have explored various approaches. 
Some models are fine-tuned to memorize essential \textit{profile information}, such as birthdate, occupation, and other background traits~\cite{shao2023character, wang2023rolellm, bai2024baijia}. 
Others leverage \textit{conversational data} from novels, screenplays, and multimedia content to adapt language models through dialogue-based fine-tuning~\cite{li2023chatharuhi, zhang2024thinking, tu2024charactereval}. 
Additionally, prompt engineering techniques dynamically inject \textit{profile-specific information} during generation, enabling models to emulate character traits without extensive fine-tuning~\cite{han2022meet, shanahan2023role, tu2023characterchat}.

As shown in the related works above, current methods often oversimplify persona representation, limiting it to superficial dialogues or basic profile descriptors.
We argue that \textit{robust persona simulation must move beyond basic attributes and narrow conversational patterns} to incorporate deeper aspects of human identity, such as an individual's worldview, ethical frameworks, context-dependent viewpoints, and foundational beliefs.
Echoing Virginia Woolf's idea that ``books are the mirrors of the soul'', we propose a new paradigm: deep persona simulation based on a writer's works, which inherently reflect their beliefs, insights, and reactions to diverse themes and topics, thus capturing their essence more authentically.

In this work, we explore the problem of character persona simulation using Lu Xun as an exemplar—a renowned Chinese writer known for his critical essays on sociocultural issues. We develop CharacterBot to model his persona, as shown in Figure~\ref{fig:compare}.
We curated a corpus comprising 17 essay collections, including 638 individual works containing titles, full texts, and segmented passages. 
To enable a comprehensive acquisition of both stylistic patterns and ideological depth, we implemented four distinct tasks. 
First, next-token prediction pre-training combined with authorial perspective reframing facilitates mastery of Lu Xun's linguistic style and cognitive frameworks. 
Subsequently, in the fine-tuning stage, the model learns to: 
1) resolve multiple-choice questions testing comprehension of authorial viewpoints,
2) extract core ideological propositions from textual segments, and
3) perform style transfer by transforming styleless text inputs into outputs that emulate the author's distinctive stylistic patterns.
For model design, we develop a CharLoRA framework that extends LoRA~\cite{hulora} by optimizing knowledge integration specifically for persona simulation. 
Unlike standard LoRA, which primarily focuses on adapting models to specific tasks via low-rank updates, CharLoRA introduces a structured decomposition: a general matrix pair \(\mathbf{A}_{\text{pt}}\) and \(\mathbf{B}_i\) that jointly capture both linguistic patterns and cognitive foundations. 
While task-specific \(\mathbf{B}_i\) matrices specialize in handling task-relevant persona patterns, the shared \(\mathbf{A}_{\text{pt}}\) matrix enables cross-task knowledge synthesis.
This novel design not only preserves persona consistency across tasks but also enhances deep persona representation through multi-objective learning, distinguishing CharLoRA from conventional LoRA approaches.
Training data was generated using GPT-4o with carefully designed prompts and rigorously validated by humans to ensure quality.

Since no established metrics exist for evaluating personalized LLMs with both linguistic depth and philosophical insight, we propose novel evaluation criteria and compare our model against latest role-play models. 
CharacterBot shows significant improvements in both style and ideological depth. Additionally, we conduct a human evaluation where participants rank outputs from our model and baselines against reference texts. 
Results confirm that our model consistently outperforms baselines in generating contextually accurate and stylistically aligned outputs.

Our main contributions are as follows:  
(1) We introduce a novel framework for character persona simulation in LLMs that captures both linguistic style and deep ideological perspectives beyond superficial profile memorization.
(2) We design a multitask architecture with CharLoRA that combines general linguistic knowledge with task-specific persona modeling through tasks like question answering and style transfer.
(3) Experiments show that CharacterBot outperforms baselines in linguistic accuracy, style preservation, and opinion comprehension, simulating complex character personas.


\section{Related Work}

\paragraph{Character Persona Simulation.}
Character persona simulation assigns virtual personality traits to LLMs and has received substantial attention \cite{cheng2023towards, tseng2024two, samuel2024personagym}. Two primary approaches are used: prompt-based and fine-tuning-based methods.
Prompt-based methods integrate character settings directly into prompts. For instance, \citet{shanahan2023role} introduce a role-playing mechanism through prompts, while \citet{tu2023characterchat} construct a dialogue system using the MBTI personality framework. \citet{agatsuma2024building} simulate patient personas to train nursing students in health guidance. However, prompt engineering often fails to capture intricate traits and dynamic behaviors.
Fine-tuning-based methods address these limitations by training pre-trained LLMs. \citet{shao2023character} fine-tune models using character profiles and an experience upload strategy. \citet{lu2024large} extract character attributes, background details, and dialogue features from knowledge bases to fine-tune models. 
\citet{park2024enhancing} further refine role-specific behaviors through trait extraction from novel summaries. Despite these advancements, fine-tuning often produces models that mimic surface-level style but miss deeper traits.

\paragraph{Style Transfer.}  
Style transfer for text involves altering the stylistic attributes of a source text while preserving its core meaning \cite{li2023stylized}.
\citet{reif2022recipe} introduce an Augmented Zero-Shot Learning method, which leverages LLMs to achieve versatile text-style transformations without requiring task-specific training.
\citet{pu2023chatgpt} evaluate the performance of ChatGPT in sentence style transfer by comparing ChatGPT generated texts with those created by humans.
\citet{zhang2024distilling} explore a novel approach to style transfer by integrating the Chain-of-Thought reasoning capabilities of LLMs with knowledge distillation techniques.
However, these methods predominantly focus on linguistic style without engaging with deeper ideological dimensions.

\paragraph{Parameter-Efficient Fine-Tuning.}
Parameter-Efficient Fine-Tuning (PEFT) techniques are crucial for adapting large pre-trained models efficiently, especially in the context of LLMs \cite{xu2023parameter, ding2023parameter, chen2024flexible}. Prefix Tuning optimizes learnable prefixes added to input embeddings or intermediate representations \cite{li2021prefix}, while Prompt Tuning adjusts learnable prompt embeddings to guide downstream task adaptation \cite{lester2021power}. Among PEFT methods, LoRA is widely used, introducing low-rank matrices to fine-tune models efficiently \cite{hulora}. Its variant, QLoRA, further improves efficiency through quantization \cite{dettmers2024qlora}.
Recent innovations have improved LoRA’s domain-specific adaptability. \citet{zhao2024loraretriever} propose a dynamic mechanism for retrieving and combining LoRA modules in mixed task scenarios. \citet{tan2024personalized} enable efficient personalized LLMs by sharing partial PEFT parameters across users, and \citet{zhao2024merging} introduce a flexible LoRA merging strategy suited for multi-task learning without additional training. In our work, we adapt LoRA for deep persona simulation.

\begin{figure*}[t]
    \centering
    \includegraphics[width=0.8\linewidth]{figs/method1.pdf}
    \caption{In pre-training, reframed essays from Authorial Perspective Reframing (APR) train the base model. In fine-tuning, multiple-choice question answering (MCQ), generative question answering (GQA), and style transfer (ST) refine their modules within CharLoRA to align with the target persona.}
    \label{fig:method}
\end{figure*}



\section{Problem Formulation}  
We formally define the task of deep persona simulation by introducing key notations and concepts.
A single model \( M \) is designed to perform multiple tasks, which together assess its ability to understand and generate character-specific responses.

The first task is Multiple-Choice Question Answering, where the model \( M \) is given a question \( Q \) and a set of candidate answers \( \{A_1, A_2, \dots, A_n\} \). The model predicts the correct answer \( A^* \) based on its understanding of the character’s perspective:  
\[
A^* = \arg \max_{A_i} M(Q, \{A_1, ..., A_n\}; \theta),
\]  
where \( \theta \) denotes the model’s parameters.

The second task is Generative Question Answering, where the model \( M \) generates an appropriate answer \( A \) given a question \( Q \), reflecting the character’s knowledge, opinions, or personality:  
\[
A = M(Q; \theta).
\]  
The final task is Style Transfer, where the model \( M \) rewrites an input text \( T_{\text{input}} \) into an output text \( T_{\text{output}} \) that aligns with a specified persona style \( S \):  
\[
T_{\text{output}} = M(T_{\text{input}}, S; \theta).
\]  

\section{Method}

As shown in Figure~\ref{fig:method}, our CharacterBot is first pre-trained and then fine-tuned on three downstream tasks.
In the following sections, we introduce the task-specific objectives and adapted training datasets for each stage, followed by a detailed explanation of our customized CharLoRA.

\subsection{Pre-training}

Although conventional LLMs are typically trained on books that may include the works of Lu Xun, they are fine-tuned on broad-coverage corpora that, while fostering general linguistic competence, lack the domain-specific precision necessary to replicate nuanced authorial traits.
To address this gap, we specifically pre-train the LLM on Lu Xun’s corpus to capture his distinctive narrative style and linguistic patterns.


\paragraph{Authorial Perspective Reframing (APR).}
To enhance the model’s understanding of authorial perspectives, we introduce APR, a pre-training technique that aligns textual viewpoints with their original intellectual contexts by transforming first-person narratives into third-person perspectives.
This transformation clarifies attribution by explicitly linking content to its creator, thereby reducing ambiguity. 
It also improves the model’s ability to distinguish between described concepts and the author’s stance, while reinforcing the connection between viewpoints and their originators.
In the following, we show an example:
\begin{tcolorbox}[colback=gray!10, left=1mm, right=1mm, top=1mm, bottom=1mm] \small [Original Essay] Comparison is the best thing. Before understanding phonetic scripts, people would find it difficult to realize the challenges of pictographic characters... \end{tcolorbox}
\begin{tcolorbox}[colback=gray!10, left=1mm, right=1mm, top=1mm, bottom=1mm] \small  [APR Processed] \textbf{Lu Xun discusses} the importance of comparison in his essay. \textbf{He points out} that before understanding phonetic scripts, it is hard for people to realize the challenges of pictographic characters... \end{tcolorbox}


\subsection{Fine-tuning}

Following pre-training, we fine-tune the LLM on three tasks: Multiple Choice Questions, Generative Question Answering and Style Transfer to align it with the cognitive patterns, style, and ideology of the target character. For examples of three tasks, refer to Appendix~\ref{sec:data-examples}.

\paragraph{Task 1: Multiple Choice Questions.}
The motivation for this task comes from the common use of multiple choice questions in reading comprehension exams in various languages, making it an effective way to assess whether the bot truly understands the author’s ideas within the essay.
Our first fine-tuning task involves multiple-choice answering, where each question, derived from an essay, provides four answer options with one correct answer.
To ensure alignment with the essay content, the dataset generation process follows two key constraints: The questions should be directly grounded in the essay, avoiding any extrapolation, and they must use second-person framing by addressing the character as ``you''.

\paragraph{Task 2: Generative Question Answering.}  
Similarly, another common type of reading comprehension question involves asking respondents to answer based on their understanding of an article.  
In this task, the generative question answering creates pairs of questions and answers that investigate the underlying arguments of the character.  
The questions are framed as ideological probes in the second person, while the answers reflect the rhetorical and ideological patterns of the character.  
For each essay, three pairs are created, following two principles: semantic fidelity, ensuring that the answers derive solely from the essay, and stylistic consistency, preserving the unique lexical and syntactic characteristics of the character.

\paragraph{Task 3: Style Transfer.}  
The final training task focuses on style transfer, commonly used in personalization tasks where the model rewrites sentences to follow a specific style. For each essay, three sentences that are representative of the character’s style are extracted using an advanced language model. 
These sentences are then rewritten into a neutral, styleless form that preserves their original meaning, argumentative intent, and emotional tone while removing distinct stylistic features.  
This parallel structure enables the model to learn how to rewrite text in the author’s style while maintaining the intended meaning and ideological consistency. 





\subsection{CharLoRA}  

To optimize computational and storage efficiency while ensuring high persona fidelity, we introduce CharLoRA, an adaptation of the LoRA method. 

\paragraph{LoRA Fundamentals.}  
For a frozen pre-trained weight matrix \( W_0 \in \mathbb{R}^{d \times k} \) in any linear layer, LoRA injects trainable low-rank matrices \( \mathbf{B} \in \mathbb{R}^{d \times r} \) and \( \mathbf{A} \in \mathbb{R}^{r \times k} \), with \( r \ll \min(d,k) \), to approximate parameter updates:  
\[
W_0 + \Delta W = W_0 + \mathbf{B}\mathbf{A},  
\]   
Given input activations \( \mathbf{x} \in \mathbb{R}^k \), the modified forward pass becomes:  
\[
\mathbf{h}' = W_0\mathbf{x} + \mathbf{B}\mathbf{A}\mathbf{x}.  
\]  
During optimization, \( W_0 \) remains fixed, while \( \mathbf{A} \) and \( \mathbf{B} \) accumulate gradients - reducing trainable parameters versus full fine-tuning.  

\paragraph{CharLoRA in Pre-Training Stage.}
In the pre-training stage, the primary goal of CharLoRA is to inject persona-specific knowledge into a base language model without disrupting its general pre-trained knowledge. To achieve this, we leverage low-rank matrix adaptations through LoRA while maintaining a separation between the original model parameters and persona-specific updates.  

The process begins with the reframed versions of the original texts using APR. 
CharLoRA applies a low-rank adaptation to the weight matrix of the frozen base model \( W_0 \), introducing an update matrix \( \Delta W_{\text{pt}} \) defined as:  
\[
\mathbf{h}'_{\text{pt}} = W_0 \mathbf{x} + \Delta W_{\text{pt}} \mathbf{x} = W_0 \mathbf{x} + \mathbf{B}_{\text{pt}} \mathbf{A}_{\text{pt}} \mathbf{x},
\]  
where \( \mathbf{B}_{\text{pt}} \in \mathbb{R}^{d \times r} \) and \( \mathbf{A}_{\text{pt}} \in \mathbb{R}^{r \times k} \) are low-rank matrices with \( r \ll \min(d, k) \).  

During training, the original weight matrix \( W_0 \) remains frozen, while only the low-rank matrices \( \mathbf{B}_{\text{pt}} \) and \( \mathbf{A}_{\text{pt}} \) are optimized. The update \( \mathbf{B}_{\text{pt}} \mathbf{A}_{\text{pt}} \) encodes fine-grained persona-specific traits, such as writing style, tone, and ideological perspectives. 
For instance, when pre-training on Lu Xun’s original essays and their reframed versions, this update effectively captures his distinctive voice and literary style. 
By decoupling the base model from low-rank adaptations, CharLoRA ensures that new persona-specific knowledge is efficiently integrated while preserving the general knowledge of the original language model.

\paragraph{CharLoRA in the Fine-Tuning Stage.}  
Building on the persona-specific knowledge encoded during pre-training, CharLoRA is designed to handle diverse downstream tasks while preserving the learned persona consistency across tasks. 
However, each task has unique demands, such as varying output formats and contextual focus, which necessitate task-specific adaptations. 
To address this, CharLoRA adopts a hybrid parameter-sharing strategy by decoupling shared persona knowledge from task-specific updates.

\begin{table*}[t]
    \centering
    \small
    \begin{tabularx}{\linewidth}{>{\centering\arraybackslash}X M I I S S S}
    \toprule
    \multirow{2}{*}{\textbf{Model}}
    & \multicolumn{1}{c}{\textbf{M-C Questions}} 
    & \multicolumn{2}{c}{\textbf{Generative Question Answering}} 
    & \multicolumn{3}{c}{\textbf{Style Transfer}}               
    \\ 
    \cmidrule(lr){2-2}
    \cmidrule(lr){3-4}
    \cmidrule(lr){5-7}
    & \textbf{Accuracy} 
    & \textbf{Content Score} 
    & \textbf{Style Score} 
    & \textbf{BLEU} 
    & \textbf{ROUGE-1} 
    & \textbf{Style Matching} 
    \\ 
    \midrule
    Llama 3.1-8B         & 0.614    & 2.370            & 1.354            & 0.113          & 0.264          & 0.267          \\
    Qwen2.5-7B           & 0.787    & 2.828            & 2.818            & 0.115          & 0.233          & 0.456          \\
    GPT-4o               & 0.734    & \textbf{3.214}   & 2.542            & 0.088          & 0.196          & 0.471          \\
    CharacterGLM-6B      & 0.073    & 1.984            & 1.729            & 0.017          & 0.084          & 0.351          \\
    Baichuan-NPC-Turbo   & 0.568    & 2.620            & 2.052            & 0.124          & 0.185          & 0.518          \\
    Tongyi Xingchen      & \underline{0.788}    & 3.172            & \underline{2.823} & 0.101          & 0.187          & \underline{0.534}          \\
    LuXun-GPT            & -        & -                & -                & \underline{0.127}          & \underline{0.283}          & 0.387          \\
    Ours                 & \textbf{0.880} 
                         & \textbf{3.214} 
                         & \textbf{2.885}
                         & \textbf{0.293} 
                         & \textbf{0.410} 
                         & \textbf{0.937} 
    \\ 
    \bottomrule
    \end{tabularx}
    \caption{Results on Multiple-Choice Questions, Generative Question Answering, and Style Transfer. 
    M-C Questions denotes Multiple-Choice Questions.
    Note that LuXun-GPT is solely designed for the style transfer task, so only style transfer results are available. 
    Higher values indicate better performance.
    The best results are shown in bold, and the second best scores are underlined.}
    \label{tab:experiment}
\end{table*}
The shared low-rank matrix \( \mathbf{A}_{\text{pt}} \), initialized during pre-training, encodes core persona attributes like writing style and themes. For fine-tuning, CharLoRA replicates the pre-trained matrix \( \mathbf{B}_{\text{pt}} \) into task-specific matrices \( \mathbf{B}_i \), allowing customization for each downstream task \( i \) (e.g. multiple choice questions, generative QA, or style transfer) while retaining global persona knowledge. The forward pass for task \( i \) is defined as:
\[
\mathbf{h}_i = W_0 \mathbf{x} + \Delta W_i \mathbf{x} = W_0 \mathbf{x} + \mathbf{B}_i \mathbf{A}_{\text{pt}} \mathbf{x}.
\]  
% where \( \Delta W_i = \mathbf{B}_i \mathbf{A}_{\text{pt}} \) represents the task-specific low-rank update.

During fine-tuning, only \( \mathbf{B}_i \) and the shared \( \mathbf{A}_{\text{pt}} \) are updated for the active task \( i \), while the matrices corresponding to other tasks \( \mathbf{B}_j \) (\( j \neq i \)) remain frozen to prevent interference. 
% This selective update mechanism ensures that the model can efficiently adapt to diverse tasks without overwriting the persona knowledge or causing negative transfer between tasks.
By maintaining \( \mathbf{A}_{\text{pt}} \) as a shared cross-task component and allowing each task-specific \( \mathbf{B}_i \) to capture task-specific details, CharLoRA achieves an optimal balance between efficiency and adaptability. For example, \( \mathbf{A}_{\text{pt}} \) ensures that Lu Xun’s characteristic voice is preserved throughout the tasks, while each \( \mathbf{B}_i \) allows nuanced adaptations, such as adjusting the tone for answering questions or generating stylistically consistent content in different contexts. This design allows CharLoRA to deliver robust performance in multi-task learning scenarios while maintaining persona consistency.



\section{Experiments}
\label{sec:experiment}
\subsection{Datasets}

In this work, we examined the writings of the renowned Chinese author Lu Xun as a case study, utilizing 17 of his essay collections comprising 638 essays. 
The pre-training stage used both the original essay texts and their transformed versions processed through Authorial Perspective Reframing (APR), which were employed in a next-token prediction task. 
The pre-training dataset, sourced from Wikisource, includes major essay collections such as \textit{False Freedom}, \textit{Grave}, and \textit{Hot Wind}. 
These collections, which vividly convey Lu Xun’s ideological perspectives, provide essential material for capturing his unique narrative style. 
A complete list of the collections can be found in the Appendix~\ref{sec:essay-lists}.
For the fine-tuning stage, we constructed three datasets derived from Lu Xun’s essays using OpenAI’s GPT-4o API. 
Each dataset was randomly split into training (85\%), validation (5\%), and testing (10\%) sets to ensure unbiased evaluation.
While brief descriptions of these datasets were provided during the task explanations, Table~\ref{tab:dataset} summarizes key statistics for further reference.

\begin{table}[htbp]
\centering
\small
\resizebox{0.482\textwidth}{!}{
\begin{tabular}{@{}ccc@{}}
\toprule
\textbf{\begin{tabular}[c]{@{}c@{}}Essay\\ Collections\end{tabular}} & \textbf{\begin{tabular}[c]{@{}c@{}}Total\\ Essays\end{tabular}} & \textbf{Task Type (Count)} \\ \midrule
\multirow{3}{*}{17} & \multirow{3}{*}{638} & Multiple-Choice Questions (1914) \\
                    &                      & Generative Question Answering (1914) \\
                    &                      & Style Transfer (1907) \\ \bottomrule
\end{tabular}
}
\caption{Summary of Lu Xun’s essay collections and the three fine-tuning tasks, showing the task types and their respective instance counts.}
\label{tab:dataset}
\end{table}


\subsection{Baselines}



To evaluate the performance of our proposed model, we conducted benchmark evaluations of advanced open-source and proprietary conversational LLMs to assess their performance:
(1) Llama 3.1-8B \cite{dubey2024llama}: A multilingual LLM by Meta for diverse language tasks.
(2) Qwen2.5-7B \cite{yang2024qwen2}: An LLM by Alibaba, optimized for advanced Chinese processing.
(3) GPT-4o: A multimodal LLM by OpenAI with state-of-the-art cross-modal capabilities.
Additionally, we included models explicitly designed for role-playing applications:
(4) CharacterGLM-6B \cite{zhou2023characterglm}: A role-based dialogue model built upon the ChatGLM series.
(5) \href{https://npc.baichuan-ai.com}{Baichuan-NPC-Turbo}: An advanced LLM developed by Baichuan Intelligence, focusing on dynamic role-playing scenarios.
(6) \href{https://tongyi.aliyun.com/xingchen}{Tongyi Xingchen}: A personalized role-dialogue platform launched by Alibaba Cloud, built on the Qwen LLM framework.
Beyond general-purpose and role-playing models, we also evaluated a specialized model for text style transfer:
(7) \href{https://github.com/Suffoquer-fang/LuXun-GPT}{LuXun-GPT}: An open-source project designed to transform input into the linguistic style of Lu Xun.

\begin{figure*}[htb]
    \centering
    \includegraphics[width=1\linewidth]{figs/ablation.pdf}
    \caption{Ablation study results: (a) multiple-choice questions, (b) generative question answering, and (c) style transfer. The full architecture outperforms ablated versions, demonstrating the contributions of CharLoRA and APR to task performance.}
    \label{fig:ablation}
\end{figure*}
\subsection{Implementation Details}
The experiments were conducted using PyTorch on an NVIDIA A100 GPU. Since Lu Xun’s writings are in Chinese, we selected the Qwen2.5-7B-Instruct model, an LLM optimized for the Chinese language, as the base model. 
During the pre-training stage, CharLoRA was applied with a LoRA rank of 64 to introduce low-rank adaptations. 
The training configuration included a token cutoff length of 2048, a learning rate of \(5.0 \times 10^{-5}\), three training epochs, a batch size of 4. 
For the fine-tuning stage, the LoRA rank was kept at 64, and the same learning rate, batch size, and validation set size were used. 
The token cutoff length was reduced to 1024, and fine-tuning was performed over three epochs. 


\subsection{Evaluation Metrics}
We design classic and advanced aspect-based metrics to evaluate performance on different tasks.

For the multiple-choice questions task, performance is measured using accuracy, calculated as the number of correct answers divided by the total number of test cases.
For the generative question answering task, two metrics are used to evaluate performance: the content score and the style score. The content score assesses how well the generated answers align with the core ideological ideas on a scale from 1 to 5, while the style score evaluates the adherence to the author’s linguistic patterns, also on a scale from 1 to 5. 
For the style transfer task, we compared the model-generated sentences for each test example with sentences from Lu Xun's original work. The evaluation used BLEU \cite{papineni2002bleu}, ROUGE-1 \cite{lin2004rouge}, and the style matching score, which measures the degree to which the generated texts reflect the author’s distinctive style.

The evaluation process is conducted using DeepSeek-V3 \cite{liu2024deepseek}, 671B the strong capabilities of Chinese LLMs, and two human annotators: a native speaker with a PhD and a PhD candidate, both possessing extensive literary expertise. Detailed scoring guidelines and prompts are provided in the Appendix~\ref{sec:prompts}.




\begin{table}[htb]
\centering
\scriptsize
\begin{tabular}{@{}cccc@{}}
\toprule
\multirow{2}{*}{\textbf{Model}} & \multicolumn{2}{c}{\textbf{Generative Question Answering}} & \textbf{Style Transfer} \\ \cmidrule(l){2-4} 
                                & \textbf{Content Score}        & \textbf{Style Score}       & \textbf{Style Matching} \\ \midrule
GPT-4o                          & 3.492                         & 3.150                      & 0.467                   \\
Tongyi Xingchen                 & 3.008                         & 2.667                      & 0.616                   \\
Ours                            & \textbf{3.758}                         & \textbf{3.550 }                     & \textbf{0.917}                   \\ \bottomrule
\end{tabular}
\caption{Human evaluation results.}
\label{tab:human}
\end{table}



\begin{table*}[t]
\centering
\scriptsize
\begin{tabularx}{\textwidth}{X}
\toprule
\multicolumn{1}{M}{\textit{Multiple-Choice Questions}} \\
\midrule
Resolved Error type: \textbf{Distractor Bias}\\[0.5em]
Question: Please choose the most appropriate answer from the following four options. Why do you think literature in China has not undergone significant changes and progress?\\
A: Because literary figures lack sufficient passion\\
B: Because revolutionaries dislike literature\\
C: Because there is no revolution and progress as a backdrop\\
D: Because the influence of old-school figures is too significant\\[0.5em]
\textit{\textbf{Baseline (GPT-4o)}}: D: Because \erritalic{the influence of old-school figures} is too significant\\[0.5em]
\textit{\textbf{Characterbot}}: C: Because there is \corritalic{no revolution and progress as a backdrop}\\[0.5em]
\textit{\textbf{Essay Excerpt:}}: ``China claims to have undergone revolution - politically perhaps true, but literarily unchanged. \corritalic{This implies literature's stagnation directly reflects the absence of true revolution and progress} - though revolutionaries might dislike this interpretation.'' — \textit{Ancient Texts and Vernacular Chinese}\\
\midrule
\multicolumn{1}{I}{\textit{Generative Question Answering}} \\
\midrule
Resolved Error type: \textbf{Faulty Attribution}\\[0.5em]
Question: Why do you think that the most effective critic of classical texts is precisely someone who has read them?\\[0.5em]
\textit{\textbf{Baseline (Llama3.1-8B)}}: This is because they \erritalic{once possessed passion and reverence for the classical texts, but as time passed, that passion and reverence gradually faded}, leaving behind a void and insincere knowledge.\\[0.5em]
\textit{\textbf{Characterbot}}: Because those who have read the classical texts are often \corritalic{able to spot their flaws and mistakes}, they are more likely to criticize and dismiss the texts.\\[0.5em]
\textit{\textbf{Essay Excerpt:}}: ``The most effective disparagers of ancient texts are precisely those who have read them thoroughly. \corritalic{Only they can expose the inherent contradictions and 'turn their own weapons against them' through deep familiarity}…'' — \textit{Overview of Modern New Literature}\\
\midrule
\multicolumn{1}{S}{\textit{Style Transfer}} \\
\midrule
Resolved Error type: \textbf{Misaligned Metaphor}\\[0.5em]
Question: Rephrase the following passage in your own tone: ``My art takes delicacy as its life, and wields a sharpness akin to an anatomical scalpel as its strength.''\\[0.5em]
\textit{\textbf{Baseline (Llama3.1-8B)}}: My art is practically an extension of my life. \erritalic{Delicacy is the blood of my life}, requiring a force as sharp as an anatomical scalpel.\\[0.5em]
\textit{\textbf{Characterbot}}: My art is delicate—\corritalic{its very essence is defined by delicacy}, yet it is supported by a force as incisive as an anatomical scalpel.\\
\bottomrule
\end{tabularx}
\caption{Case studies comparing the responses of our model \textit{\textbf{Characterbot}} with those of the baseline models (including \textit{\textbf{GPT-4o}} and \textit{\textbf{Llama3.1-8B)}}.
\erritalic{Red} text indicates errors, while \corritalic{blue} text shows correct responses. 
The complete version is available in Table~\ref{tab:case_full} in Appendix~\ref{sec:case-study}.}
\label{tab:case}
\end{table*}

\subsection{Main Results}
We show the performance of our model in Table~\ref{tab:experiment}. \textit{Our model achieves the highest accuracy in multiple-choice questions, excelling at capturing complex ideological nuances.} It achieves an accuracy of 0.880, outperforming Tongyi Xingchen (0.788) and Qwen2.5-7B (0.787). This highlights its effectiveness in understanding personality-driven narratives. Tongyi Xingchen and Qwen2.5-7B also outperform models like Llama 3.1-8B (0.614) due to their extensive use of Chinese-language data, emphasizing the importance of linguistic alignment. Further pre-training and fine-tuning enhance Qwen2.5-7B’s grasp of Lu Xun’s ideological nuances.

For the generative question answering task, \textit{our model excels in preserving both content and stylistic fidelity, outperforming key baselines in style.} It achieves a content score of 3.214 and a style score of 2.885, surpassing GPT-4o’s 2.542 and Tongyi Xingchen’s 2.823. This demonstrates its superior ability to reflect the author’s style. Additionally, the correlation between content scores and multiple-choice accuracy shows that improved character understanding enhances stylistic alignment.

In the style transfer task, \textit{our model excels in capturing and reproducing the author’s style while maintaining content integrity}. It outperforms baselines with scores of 0.293 (BLEU), 0.410 (ROUGE-1), and 0.937 (style matching), demonstrating strong stylistic fidelity and textual coherence. These results underscore the importance of integrating ideological and stylistic frameworks for authentic reproduction, surpassing LuXun-GPT in overall performance.

\section{Analysis and Discussion}



\subsection{Ablation Study}
We assess the impact of individual components by evaluating ablated versions across three tasks. As shown in Figure~\ref{fig:ablation}, the full architecture achieves the best performance, confirming that both CharLoRA and APR are essential for improving downstream tasks.
Removing CharLoRA (w/o CharLoRA) degrades the model to a standard LoRA setup, leading to asymmetric performance drops. While it slightly improves BLEU scores due to fewer stylistic constraints (Figure~\ref{fig:ablation}(c)), it significantly reduces style scores (Figure~\ref{fig:ablation}(b)) and content scores (Figure~\ref{fig:ablation}(a)), underscoring CharLoRA’s role in robust style representation. Similarly, excluding APR (w/o APR) weakens the model’s ability to process stylistic information, causing a sharp decline in style score to 2.536 (Figure~\ref{fig:ablation}(b)) and style evaluation to 0.895 (Figure~\ref{fig:ablation}(c)). This highlights APR’s crucial role in decoupling narrative perspective from content through third-person reframing.
These results demonstrate that CharLoRA and APR work together to enhance style-sensitive tasks while preserving content fidelity, reinforcing the effectiveness of our design choices.

\subsection{Human Evaluation}
\label{sec:human-evaluation}

We perform human evaluations using the two strongest baseline models, GPT-4o and Tongyi Xingchen, with scores provided by two annotators, as shown in Table~\ref{tab:human}. 
Both evaluators possess extensive literary expertise, including a native speaker with a PhD and a PhD candidate. 
The inter-annotator agreement, measured using Cohen’s kappa, yields a value of 0.72, indicating substantial consistency. 
The evaluation results show that our model surpasses the baselines in content and stylistic fidelity, highlighting its ability to capture Lu Xun’s nuanced ideas and writing style.

We classify errors into distractor bias, faulty attribution, context neglect, misaligned metaphors, and concept drift (full cases in Appendix~\ref{sec:case-study}) and analyze selected examples (Table~\ref{tab:case}).
In \textit{multiple-choice questions}, baseline models are often misled by superficially plausible but contextually inaccurate options.
For example, GPT-4o wrongly attributes literary stagnation to `old-guard influence', while the passage states it reflects `a lack of genuine revolution and progress'.
CharacterBot correctly identifies the core argument.
In \textit{generative question answering}, baselines misinterpret causality. Llama3.1-8B overemphasizes emotion instead of explaining how deep reading exposes textual flaws. CharacterBot, in contrast, follows the passage’s logic: only deep engagement reveals contradictions for effective critique.
In \textit{style transfer}, baseline models introduce metaphors that distort meaning. Llama3.1-8B replaces a phrase about artistic refinement with `blood'', failing to capture the intended nuance of precision as the essence of art''. CharacterBot preserves both meaning and stylistic sharpness, ensuring semantic and tonal alignment with the original text.
These cases show CharacterBot outperforms baselines by avoiding superficial cues, maintaining fidelity, and ensuring logical consistency.


\section{Conclusion}


In this paper, we present CharacterBot, a model that simulates both the linguistic patterns and deeper thought processes of a character. Using Lu Xun as a case study, we adopt a multi-task approach, combining pre-training on linguistic structures with fine-tuning through tasks like question answering and style transfer. 
We introduce the CharLoRA updating mechanism, enabling collaboration between general and task-specific experts.
Experimental results show that CharacterBot outperforms baselines in linguistic accuracy and opinion comprehension. 
In future work, we aim to extend it to diverse personas and multi-layered character simulations.




\section*{Limitations}

Despite the effectiveness of CharacterBot in simulating deep character personas, several limitations remain.
First, our approach primarily relies on collecting personal essays and reflective writings to capture an individual's thought processes. While these sources provide direct insight into a person's beliefs and reasoning, novels and fictional works can also reflect the author's ideology, though in a more indirect way. However, using novels for persona simulation requires a deeper understanding of thematic elements, narrative structures, and symbolic representations, which remains an open challenge.
Second, deep persona simulation extends beyond the three tasks we proposed. While multiple-choice question answering, generative question answering, and style transfer help align the model with an individual's linguistic patterns and ideological stance, other aspects of persona simulation, such as cognitive reasoning, emotional modeling, and long-term consistency in beliefs, remain unexplored. Future research should consider a broader range of tasks to enhance the depth and fidelity of simulated personas.

\section*{Ethical Consideration}


The development of CharacterBot involves several ethical considerations, particularly regarding intellectual property, authenticity, and responsible use. While the goal is to simulate a character’s linguistic patterns and thought processes rather than replicate exact works, it is important to ensure that such simulations are respectful and do not misrepresent the original author’s ideas. Additionally, generating text that mimics a historical figure’s writing style raises concerns about potential misinformation or misuse. To address these issues, we advocate for the responsible use of CharacterBot as an assistive tool for literary study and creative exploration, rather than as a means to replace original authorship. Possible safeguards include clearly distinguishing AI-generated content from authentic writings, ensuring transparency in how the model is trained, and limiting its use to educational and creative applications that respect the legacy of the original writer.


\bibliography{custom}
% This must be in the first 5 lines to tell arXiv to use pdfLaTeX, which is strongly recommended.
\pdfoutput=1
% In particular, the hyperref package requires pdfLaTeX in order to break URLs across lines.

\documentclass[11pt]{article}

% Change "review" to "final" to generate the final (sometimes called camera-ready) version.
% Change to "preprint" to generate a non-anonymous version with page numbers.
\usepackage[final]{acl}

% Standard package includes
\usepackage{times}
\usepackage{latexsym}

% For proper rendering and hyphenation of words containing Latin characters (including in bib files)
\usepackage[T1]{fontenc}
% For Vietnamese characters
% \usepackage[T5]{fontenc}
% See https://www.latex-project.org/help/documentation/encguide.pdf for other character sets

% This assumes your files are encoded as UTF8
\usepackage[utf8]{inputenc}

% This is not strictly necessary, and may be commented out,
% but it will improve the layout of the manuscript,
% and will typically save some space.
\usepackage{microtype}

% This is also not strictly necessary, and may be commented out.
% However, it will improve the aesthetics of text in
% the typewriter font.
\usepackage{inconsolata}

%Including images in your LaTeX document requires adding
%additional package(s)
\usepackage{graphicx}
\usepackage{xcolor}

% If the title and author information does not fit in the area allocated, uncomment the following
%
%\setlength\titlebox{<dim>}
%
% and set <dim> to something 5cm or larger.

\usepackage{booktabs}
\usepackage{hyperref}
\usepackage{multirow}
\usepackage{multicol}
\usepackage[most]{tcolorbox}
\usepackage{adjustbox}
\usepackage{graphicx}
\usepackage{fullpage}
\usepackage{times}
\usepackage{fancyhdr,graphicx,amsmath,amssymb}
%\usepackage[ruled,vlined]{algorithm2e}
\usepackage{algorithm}
\usepackage{algpseudocode}
\usepackage{booktabs}
\usepackage{adjustbox}
\usepackage{url}
\usepackage{hyperref}
\usepackage{amssymb}
\usepackage{marvosym}
\usepackage{multirow}
\usepackage{subcaption}
\DeclareMathOperator*{\argmax}{arg\,max}
\DeclareMathOperator*{\argmin}{arg\,min}


\newtcolorbox{promptbox}[2][]{
  colback=gray!10,
  colframe=gray!50,
  arc=3mm,
  boxrule=1pt,
  left=10pt,
  right=10pt,
  top=8pt,
  bottom=8pt,
  before skip=12pt,
  after skip=12pt,
  fonttitle=\bfseries,
  title=#2,
  #1
}

\title{Quality-Aware Decoding: Unifying Quality Estimation and Decoding}

% Author information can be set in various styles:
% For several authors from the same institution:
% \author{Author 1 \and ... \and Author n \\
%         Address line \\ ... \\ Address line}
% if the names do not fit well on one line use
%         Author 1 \\ {\bf Author 2} \\ ... \\ {\bf Author n} \\
% For authors from different institutions:
% \author{Author 1 \\ Address line \\  ... \\ Address line
%         \And  ... \And
%         Author n \\ Address line \\ ... \\ Address line}
% To start a separate ``row'' of authors use \AND, as in
% \author{Author 1 \\ Address line \\  ... \\ Address line
%         \AND
%         Author 2 \\ Address line \\ ... \\ Address line \And
%         Author 3 \\ Address line \\ ... \\ Address line}

\author{Sai Koneru$^{1}$,
  Matthias Huck$^{2}$,
  Miriam Exel$^{2}$, \textnormal{and}
  Jan Niehues$^{1}$ \\
  $^{1}$ Karlsruhe Institute of Technology \\
  $^{2}$ SAP SE, Dietmar-Hopp-Allee 16, 69190 Walldorf, Germany \\
  \texttt{\{sai.koneru, jan.niehues\}@kit.edu} \\
  \texttt{\{matthias.huck, miriam.exel\}@sap.com}}

%\author{
%  \textbf{First Author\textsuperscript{1}},
%  \textbf{Second Author\textsuperscript{1,2}},
%  \textbf{Third T. Author\textsuperscript{1}},
%  \textbf{Fourth Author\textsuperscript{1}},
%\\
%  \textbf{Fifth Author\textsuperscript{1,2}},
%  \textbf{Sixth Author\textsuperscript{1}},
%  \textbf{Seventh Author\textsuperscript{1}},
%  \textbf{Eighth Author \textsuperscript{1,2,3,4}},
%\\
%  \textbf{Ninth Author\textsuperscript{1}},
%  \textbf{Tenth Author\textsuperscript{1}},
%  \textbf{Eleventh E. Author\textsuperscript{1,2,3,4,5}},
%  \textbf{Twelfth Author\textsuperscript{1}},
%\\
%  \textbf{Thirteenth Author\textsuperscript{3}},
%  \textbf{Fourteenth F. Author\textsuperscript{2,4}},
%  \textbf{Fifteenth Author\textsuperscript{1}},
%  \textbf{Sixteenth Author\textsuperscript{1}},
%\\
%  \textbf{Seventeenth S. Author\textsuperscript{4,5}},
%  \textbf{Eighteenth Author\textsuperscript{3,4}},
%  \textbf{Nineteenth N. Author\textsuperscript{2,5}},
%  \textbf{Twentieth Author\textsuperscript{1}}
%\\
%\\
%  \textsuperscript{1}Affiliation 1,
%  \textsuperscript{2}Affiliation 2,
%  \textsuperscript{3}Affiliation 3,
%  \textsuperscript{4}Affiliation 4,
%  \textsuperscript{5}Affiliation 5
%\\
%  \small{
%    \textbf{Correspondence:} \href{mailto:email@domain}{email@domain}
%  }
%}

\begin{document}
\maketitle
\begin{abstract}
% Neural Machine Translation (NMT) has achieved high-quality translations in many scenarios, pushing the boundaries of tasks such as instruction-following and multimodal translation. 
Quality Estimation (QE) models for Neural Machine Translation (NMT) predict the quality of the hypothesis without having access to the reference.
An emerging research direction in NMT involves the use of QE models, which have demonstrated high correlations with human judgment and can enhance translations through Quality-Aware Decoding. Although several approaches have been proposed based on sampling multiple candidate translations and picking the best candidate, none have integrated these models directly into the decoding process. In this paper, we address this by proposing a novel token-level QE model capable of reliably scoring partial translations. We build a uni-directional QE model for this, as decoder models are inherently trained and efficient on partial sequences. We then present a decoding strategy that integrates the QE model for Quality-Aware decoding and demonstrate that the translation quality improves when compared to the N-best list re-ranking with state-of-the-art QE models (up to $1.39$ XCOMET-XXL $\uparrow$). Finally, we show that our approach provides significant benefits in document translation tasks, where the quality of N-best lists is typically suboptimal\footnote{Code can be found at \url{https://github.com/SAP-samples/quality-aware-decoding-translation}}
\end{abstract}
\section{Introduction}

Large language models (LLMs) have significantly impacted various Natural Language Processing (NLP) tasks \citep{brown2020language, jiang2023mistral, dubey2024llama}, including Neural Machine Translation (NMT). The field of NMT is transitioning from using dedicated encoder-decoder transformers \citep{vaswani2017attention, nllb2024scaling} to leveraging decoder-only LLM-based translation models \citep{kocmi2024findings}. This shift is driven by LLMs' ability to retain knowledge, handle large contexts, and follow instructions, learned during extensive pre-training \citep{xu2024contrastive, alves2024tower}. As a result, LLM-based MT models have achieved state-of-the-art translation quality \citep{kocmi2024findings}.

In parallel, Quality Estimation (QE) has become a well-researched subfield within NMT. QE models are trained to predict the quality of a translation without requiring access to the reference \citep{rei2021references,rei2022cometkiwi}. Interestingly, QE models can achieve performance in assessing translation quality that is comparable to MT evaluation models, which do have access to the reference \citep{zerva2024findings}.

This led to the question: "\textit{Can we integrate QE into the current translation process to improve quality?}" Incorporating QE into NMT offers several benefits. First, having a expert QE model guiding the decoding can further improve the quality. Second, by adapting the QE model with feedback from human annotators, we can generate future translations guided with the newly obtained feedback.

\begin{figure*}[!ht]
\includegraphics[width=\textwidth]{Figures/nbestlist_problem.png}
 \caption{Example from WMT'23 English → German \#ID: 10: The paragraph begins with 'Department of Homeland Security,' which should be translated as 'Ministerium für \textbf{I}nnere Sicherheit.' However, the top 25 beams do not contain the correct translation and begin with an error, making N-best list re-ranking insufficient. Although the top-5 tokens at the decoding contain the correct forms 'Inn' or 'Inner,' the probabilities split among them giving highest mass to the incorrect token 'inn.' Quality-Aware decoding can prevent errors with earlier integration.}
\label{fig:nbestlist}
\end{figure*}


Several approaches have been explored to integrate QE into the translation process. These include re-ranking the N-best list \citep{fernandes2022quality}, applying minimum Bayes risk (MBR) decoding on a quality-filtered N-best list \citep{tomani2024quality}, and training additional models for post-editing based on QE-predicted errors \citep{treviso2024xtower}. However, all these methods operate on fully generated sequences before the QE model can exert influence. Integrating QE earlier in the decoding process, referred in this paper as \textit{Quality-Aware Decoding}, could enhance translation quality and reduce reliance on the N-best list. This is especially relevant when dealing with long inputs as good translations during decoding are likely to be pruned and may need sampling larger number of finished hypothesis. We illustrate this in Figure \ref{fig:nbestlist}.

To achieve this, a QE model capable of predicting the quality of partial translations is required. However, current leading QE models face challenges in this area, as they are typically not trained to predict scores for incomplete hypotheses. \textit{Therefore, developing QE models that can handle partial translations is essential for implementing Quality-Aware Decoding during the translation process}.

In this work, we propose adapting LLM-based MT models to perform QE on partial translations and incorporating this model into the decoding. We create a token-level synthetic QE dataset using WMT Multidimensional Quality Metrics (MQM) data \citep{burchardt2013multidimensional, freitag2024llms}. We then adapt a uni-directional LLM-based MT model to predict whether a token is \textit{Good} or \textit{Bad}. Training QE models on these token-level tasks alleviates the data challenge and allows us to exploit the MQM data while simultaneously making the task easier for the model compared to predicting a score directly.

\begin{figure*}[!ht]
\includegraphics[width=\textwidth]{Figures/annotation_scheme.png}
 \caption{Token-level label annotation scheme using the MQM error tags. \textit{MASK} indicates that this token label will not be used in training to prevent incorrect learning signal.}
\label{fig:annotation}
\end{figure*}

Furthermore, integrating the QE model into NMT during decoding is not trivial, as we need to combine the QE estimates during decoding. Therefore, we use the decoding strategy from \citet{koneru2024plug}, and modify it to incorporate token-level predictions efficiently with the adapted QE model to provide real-time feedback during the decoding process. We summarize our main findings and contributions below.

\begin{itemize}
    \item We present a novel uni-directional QE model which estimates quality on incomplete hypotheses by averaging the probabilities of each token being classified as \textit{Good}. 
    
    %We demonstrate that it achieves improved correlation with human annotations on WMT 23 English $\rightarrow$ German, compared to the log probabilities of the same LLM-based NMT model.

    \item We propose a decoding strategy that combines the token-level QE model on partial hypothesis and the NMT model to perform Quality-Aware Decoding. 
    
    \item We show through experiments that early integration is essential and the translation quality is improved even when compared to re-ranking the N-best list with state-of-the-art QE models.

    \item We highlight the significance of our approach in document translation scenarios, where post-generation QE techniques fall short due to their reliance on the quality of the N-best list, a challenge that becomes more difficult as the input length increases.
\end{itemize}



\section{Quality-Aware Decoding}

The primary objective of this paper is to achieve Quality-Aware Decoding in MT. To accomplish this, it is essential to predict the quality of partial translations and integrate this information during the decoding process. Our approach proposes using one NMT model for generating translations and another adapted NMT model to predict the quality of the candidate translations produced by the first model.

First, we explain why relying solely on the NMT model to predict the quality of a hypothesis is insufficient and why an additional model is necessary. Next, we outline the adaptation of the NMT model for QE on partial translations, detailing the creation of a token-level QE dataset, the modifications made to the NMT model for this task, and the process of estimating the sentence-level quality score. Finally, we describe the algorithm used to incorporate the QE score into the decoding process.

\subsection{Decomposing Decoding: Translation + QE}
NMT models generate a token-by-token sequence and provide the probability of each token at the decoding step. The average of the log-probabilities is often used as a proxy to score the candidate during search. 

While NMT models are capable of generating high-quality translations, using the average log-probabilities of hypotheses as a scoring metric tends to yield poor correlation with actual translation quality \citep{eikema2020map, freitag2020bleu}. In many cases, a translation can continue in several different ways, all of which may be acceptable. If the starting tokens for these continuations differ, the probability mass may be spread across multiple options which is used during the search. However, from a quality perspective, all these continuations could still achieve a high score, as the QE scores are independent and need not sum to $1$.

Therefore, we propose a expert model that focuses on quality to estimate the scores better during decoding and  improve the search space leading to a better hypothesis.


% Therefore, relying solely on the average log-probabilities during decoding is not ideal, as it computes the score independently for each token and does not fully correlate with the overall quality of the current hypothesis.

\subsection{Quality Estimation on Partial Sequences}

% NMT models decode sequences token-by-token. 
To provide a quality score during decoding, the QE model must be capable of handling incomplete sequences. It should not penalize a sequence if there is a potential extension that could lead to a perfect translation.

Estimating the score in this way is not feasible with current QE models, such as COMET \citep{rei2021references}, as they were not trained for this specific task and cannot provide reliable scores in the context of partial translations. Hence, we need to develop a partial QE system.

When building a partial QE system, several factors need to be considered. First, should the model use a uni-directional or bi-directional architecture? A \textbf{uni-directional} model is more efficient, as it allows for caching the hidden states, which can then be used for subsequent steps without re-encoding, unlike a bi-directional model.

Next, we need to decide whether to predict the QE score at the sequence level or at the token level. For \textbf{token-level QE}, we can directly use data from MQM annotations, as we already know which tokens are \textit{Good} or \textit{Bad}. However, for segment-level scoring, we need to consider how to synthetically create the training data. 

% Additionally, COMET models are encoder-only architectures pre-trained on full sentences, rather than partial sentences as required in this case. Moreover, predicting the score of partial translations naturally favors decoder-only models due to their efficiency. New tokens only need to process the preceding sequence, avoiding the need to re-encode the entire sequence. Additionally, this approach simplifies training, as we do not require synthetically shorter samples. 

%  Furthermore, there is no readily available dataset containing partial translations along with their quality scores. Hence, we need to design the adaptation process with a QE model that is uni-directional and exploit already available human annotated data.

% \subsubsection{Token-level Quality Estimation}

Therefore, we decide adapt the uni-directional model into a token-level QE system that predicts whether each token is \textit{Good} or \textit{Bad} (a binary decision) by adding an additional classifier head. This adaptation enables us to estimate the score for a sequence by calculating the average probability that each token is classified as \textit{Good}. We hypothesize that adapting the model in this way, rather than directly predicting the score, provides greater stability, as the last hidden states inherently contain token-level information and do not require mapping the entire sequence to a single score.

For training this model, we leverage the WMT MQM data containing error annotations in MT outputs. We can treat tokens before an error as \textit{Good} and those containing inside an error as \textit{Bad}. Then, we can train in uni-directional manner where each token's label is predicted using only the preceding context in the hypothesis. This is crucial as we only have the preceding context to estimate the quality for partial hypothesis.

\subsubsection{Learning the Right Signal}

\begin{algorithm*}[!t]
\caption{Computing merged score of partial hypothesis with translation and token-level QE models.}
\begin{algorithmic}[1]
\setlength{\baselineskip}{1.2em}
\Procedure{MergeScore}{}
    \State \textbf{Input:}   Hypothesis tokens $h_1, h_2, h_3, \dots, h_{n}$, Translation Model $\mathcal{M}_{NMT}$, QE model $\mathcal{M}_{QE}$, Source sentence $\mathcal{S}$, Re-ranking weight $\alpha$,
    \State \textbf{Output:} $merged\_score$
    \State $Score_{NMT} \gets \frac{1}{n}\sum \log \mathcal{P}(h_1, h_2,\dots, h_{n}|\mathcal{S};\mathcal{M}_{NMT})$ 
    \State $Score_{QE} \gets \frac{1}{n}\sum \log \mathcal{P}(0_{1}, 0_{2},\dots,0_{n} | h_1, h_2,\dots, h_{n},\mathcal{S};\mathcal{M}_{QE})$ 
    \State $merged\_score \gets (\alpha) \times Score_{NMT} + (1 - \alpha) \times Score_{QE}$
\EndProcedure
\end{algorithmic}
\label{alg:joint}
\end{algorithm*}

The straightforward approach to creating labels is to assign $1$ to all tokens within the error span and $0$ otherwise. However, MQM annotations can mark errors from words to phrases, and the starting tokens of an error span may not always be wrong. This is illustrated in Figure \ref{fig:annotation}.

For example, consider the German sentence \textit{"Ich spiele Tennis"} translated by three different NMT systems, each annotated with MQM error labels. In this work, we focus on learning a binary decision: whether an error is present, ignoring error severity.

\textbf{System 1: No error}: The translation \textit{"I play Tennis"} is perfect, and all tokens are labeled as "\textit{Good}."

\textbf{System 2: Partial error}: The translation \textit{"I played Tennis"} has an error in the verb form ("played" instead of "play"). The error is in the token span \textit{"played"}, but not all tokens in this span are incorrect (e.g., "pla" is correct). Assigning a "\textit{Bad}" label to the entire span would lead to incorrect learning. A more refined approach is needed to mark errors accurately at the token level.

\textbf{System 3: Full error}: The translation \textit{"I enjoy Tennis"} contains an error in \textit{"enjoy"}, so all tokens in this span should be labeled as "\textit{Bad}."

It is not trivial to decide when the prefix of an error span is correct/incorrect. To achieve accurate labeling, we propose the following scheme:

\begin{itemize} \item Apply a \texttt{<MASK>} operation to all tokens within the error span. \item Only the last token in the span is assigned the label "\textit{Bad}", as the error is considered complete at the end of the span. \end{itemize}

If the error token is in the middle, we still train the model to predict "\textit{Bad}" in the end and let the model determine which tokens should be part of the error span during inference. This approach ensures that errors are identified without explicitly defining the error span. 

\subsubsection{Sequence-Level Quality Estimation}


After fine-tuning a token-level classification model to predict the quality of the tokens, we still need to map these predictions into a sequence-level score that can be integrated during the decoding process. There are several potential ways to achieve this.

One approach is to simply count how many tokens are classified as \textit{Bad} in the current hypothesis. However, this method has limitations. The number of errors should be normalized based on the length of the hypothesis to account for varying sizes. Additionally, converting the probabilities into a fixed number of error tokens would need to account for different error types according to the MQM format, as each error counts differently.

To avoid such strict scoring schemes, we take a simpler approach. We average the log probabilities of all tokens that are classified as \textit{Good}. This method inherently accounts for the length of the hypothesis, and it provides a score on the scale of log probabilities, which aligns with the decoding process. Therefore, we use this averaged log probability as a proxy metric for the QE score, where a higher score indicates better quality
(\textbf{Line 5} in Algorithm \ref{alg:joint}).

\subsubsection{Fusing Translation and Quality}

We can use a token-level QE system to evaluate the quality of a source and partial hypothesis during decoding. However, integrating these probabilities into all candidates is computationally expensive, as each beam considers extensions equal to the vocabulary size.

To address this, we adopt a simplified decoding strategy from \citet{koneru2024plug}, which ensembles models with different vocabularies. By adapting the same MT model for token-level QE, we simplify the merging process, as the vocabularies match. This restriction is reasonable, as it is also beneficial to leverage the knowledge learned by the specialized MT for token-level QE.

The core idea is to re-rank the top candidates at each decoding step using the QE model. After re-ranking, the translation and QE scores are merged, and the process repeats until the end-of-sentence token is generated, for each beam. This strategy allows us to efficiently incorporate the QE model’s estimate, improving translation quality.

During decoding, at each step, we have scores for $n$ beams and $V$ possible extensions from the vocabulary. In typical beam search, we select the top $n$ extensions and expand the hypothesis. To make the decoding process Quality-aware, we estimate the quality of these extensions. Since estimating all extensions is computationally expensive, we limit the candidates by selecting a specified number of top candidates.

To achieve this, we use a hyper-parameter $topk$, which selects the best $topk$ extensions for each beam. For each of these top $topk$ extensions, we compute a combined score, detailed in Algorithm \ref{alg:joint}. This combined score incorporates both the translation model score and the quality estimation score, ensuring the quality is considered during decoding.

For a top extension at decoding step $n$, let the current tokens be $h_1, h_2, h_3, \dots, h_n$. The NMT model score is computed as the average log probabilities of each token (Line 4). For the token-level QE model, we compute the average probability of each token being classified as '\textit{Good}' (Line 5). The merged score is equal to weighted linear combination of these probabilities, with weight $\alpha$ (Line 6).

Thus, to make the decoding process Quality-Aware, we first train a token-level QE system by adapting the same NMT model to ensure vocabulary matching. We then combine the scores from both models to improve the sequence estimates explored during search.


\begin{table*}[!ht]
\resizebox{2\columnwidth}{!}{
\begin{tabular}{@{}ccccc@{}}
\toprule
\multicolumn{1}{c|}{Model}            & \multicolumn{1}{c|}{Beams}                & \multicolumn{1}{c|}{Re-ranking}              & MetricX ($\downarrow$)     & XCOMET-XXL ($\uparrow$)    \\ \midrule
\multicolumn{5}{c}{\textit{English $\rightarrow$ German}}                                                                                                          \\ \midrule
\multicolumn{1}{c|}{Tower}            & \multicolumn{1}{c|}{5}                    & \multicolumn{1}{c|}{\_}                      & 2.52          & 86.93          \\
\multicolumn{1}{c|}{Tower}            & \multicolumn{1}{c|}{25}                   & \multicolumn{1}{c|}{XCOMET-XL QE}            & 2.37          & 87.79          \\
\multicolumn{1}{c|}{Tower}            & \multicolumn{1}{c|}{25}                   & \multicolumn{1}{c|}{Tower QE} & 2.38          & 87.40          \\
\multicolumn{1}{c|}{Tower + Tower QE} & \multicolumn{1}{c|}{5 (25* for Tower QE)} & \multicolumn{1}{c|}{\_}                      & 2.12          & 88.95          \\
\multicolumn{1}{c|}{Tower + Tower QE} & \multicolumn{1}{c|}{5 (25* for Tower QE)} & \multicolumn{1}{c|}{XCOMET-XL QE}            & \textbf{2.09} & \textbf{89.08} \\ \midrule
\multicolumn{5}{c}{\textit{Chinese $\rightarrow$ English}}                                                                                                         \\ \midrule
\multicolumn{1}{c|}{Tower}            & \multicolumn{1}{c|}{5}                    & \multicolumn{1}{c|}{\_}                      & 2.42          & 88.91          \\
\multicolumn{1}{c|}{Tower}            & \multicolumn{1}{c|}{25}                   & \multicolumn{1}{c|}{XCOMET-XL QE}            & 2.30          & 89.49          \\
\multicolumn{1}{c|}{Tower}            & \multicolumn{1}{c|}{25}                   & \multicolumn{1}{c|}{Tower QE} & 2.32          & 89.51          \\
\multicolumn{1}{c|}{Tower + Tower QE} & \multicolumn{1}{c|}{5 (25* for Tower QE)} & \multicolumn{1}{c|}{\_}                      & 2.26          & 89.82          \\
\multicolumn{1}{c|}{Tower + Tower QE} & \multicolumn{1}{c|}{5 (25* for Tower QE)} & \multicolumn{1}{c|}{XCOMET-XL QE}            & \textbf{2.24} & \textbf{90.00} \\ \bottomrule
\end{tabular}
}
\caption{Translation Quality on WMT23 English $\rightarrow$ German Test set. Both XCOMET and MetricX columns use reference for reporting translation quality where as XCOMET-XL QE does not use for re-ranking.}
\label{tab:qadecoding}
\end{table*}

\begin{table}[!ht]
\resizebox{\columnwidth}{!}{
\centering
\begin{tabular}{@{}c|ccc@{}}
\toprule
                                                                                      & Pearson        & Spearmann      & Kendall        \\ \midrule
COMETQE                                                                               & \textbf{44.41} & 41.29          & 31.19          \\ \midrule
COMETQE-XL                                                                            & 41.23          & \textbf{42.17} & \textbf{31.84} \\ \midrule
Tower Avg. Log Prob                                                                        & 32.32          & 16.74          & 12.77          \\ \midrule
\begin{tabular}[c]{@{}c@{}}Tower QE\end{tabular} & 40.56          & 33.96          & 25.87          \\ \bottomrule
\end{tabular}
}
\caption{Correlation on WMT 23 for English $\rightarrow$ German Test set. The scores are calculated after removing the few sentences labeled for hallucination detection. Best scores according to each coefficient are highlighted in \textbf{bold}.}
\label{tab:correlation}
\end{table}

\section{Experimental Setup}
\paragraph{Datasets:} We focus on two language directions given their availability of MQM data: English $\rightarrow$ German and Chinese $\rightarrow$ English. To train our token-level QE systems, we use the MQM datasets\footnote{https://github.com/google/wmt-mqm-human-evaluation} from WMT \citep{freitag2021experts}. Specifically, we use the datasets until 2022 for training, 2024 for validation, and 2023 for testing \citep{kocmi2024findings}. This setup is consistent with all the other QE metrics, and we do not use any additional data beyond these datasets.
\vspace{-0.1cm}
\paragraph{Models:} 
Our proposed approach achieves Quality-Aware decoding by combining an NMT model with a token-level QE model, where we adapt the same NMT for QE by adding a classification head. We use the state-of-the-art NMT model, Tower 7B\footnote{Unbabel/TowerInstruct-7B-v0.2} \citep{alves2024tower}, which provides high-quality translations and has already been exposed to MQM data during instruction-tuning. This ensures that the gains observed in our approach stem from integrating Quality-Aware decoding into the NMT process, rather than introducing new data. Additional details on training and hyper-parameters are provided in Appendix \ref{sec:training_detail}.
\vspace{-0.1cm}
\paragraph{Metrics:}
For reporting the translation quality, we consistently use XCOMET-XXL\footnote{Unbabel/XCOMET-XXL} \citep{guerreiro2024xcomet} and MetricX\footnote{google/metricx-24-hybrid-xl-v2p6} \citep{juraska2024metricx} \textbf{with the reference}. To compare with N-best list re-ranking, we use the XCOMET-XL QE\footnote{Unbabel/XCOMET-XL} \textbf{without the reference}. This approach allows us to avoid biasing toward a single metric during the re-ranking process and enables us to measure the gains achieved by differently trained metrics. 

\section{Results}



\begin{table*}[!ht]
\centering
\resizebox{2\columnwidth}{!}{
\begin{tabular}{@{}ccccc@{}}
\toprule
\multicolumn{1}{c|}{Model}            & \multicolumn{1}{c|}{Beams}                        & \multicolumn{1}{c|}{Re-ranking}               & MetricX ($\downarrow$) & XCOMET-XXL ($\uparrow$) \\ \midrule
\multicolumn{5}{c}{\textit{English $\rightarrow$ German}}                                                                                                          \\ \midrule
\multicolumn{1}{c|}{Tower}            & \multicolumn{1}{c|}{25}                           & \multicolumn{1}{c|}{XCOMET-XL QE}             & 2.37     & 87.79      \\
\multicolumn{1}{c|}{Tower}            & \multicolumn{1}{c|}{25}                           & \multicolumn{1}{c|}{Tower QE}         & 2.38     & 87.40      \\
\multicolumn{1}{c|}{Tower}            & \multicolumn{1}{c|}{25}                           & \multicolumn{1}{c|}{Tower Distill QE} & 2.38     & 87.39      \\
\multicolumn{1}{c|}{Tower + Tower QE} & \multicolumn{1}{c|}{5 (25* for Tower QE)}         & \multicolumn{1}{c|}{\_}                       & 2.12     & \textbf{88.95}      \\
\multicolumn{1}{c|}{Tower + Tower QE} & \multicolumn{1}{c|}{5 (25* for Tower Distill QE)} & \multicolumn{1}{c|}{\_}                       & \textbf{2.11}     & 88.76      \\ \bottomrule
\end{tabular}
}
\caption{Performance of Unidirectional QE trained with/without distillation on WMT23 English $\rightarrow$ German Test set. Best scores according to each metric are highlighted in \textbf{bold}.}
\label{tab:towerdistill}
\end{table*}


\begin{table*}[!ht]
\centering
\resizebox{2\columnwidth}{!}{
\begin{tabular}{@{}cccccc@{}}
\toprule
\multicolumn{1}{c|}{Model}            & \multicolumn{1}{c|}{Beams}                & \multicolumn{1}{c|}{Re-ranking}       & XCOMET-XL ($\uparrow$)     & \multicolumn{1}{c|}{XCOMET-XXL ($\uparrow$)}     & Impact                                                                                       \\ \midrule
\multicolumn{6}{c}{\textit{Paragraph-Level}}                                                                                                                                                                                                                                    \\ \midrule
\multicolumn{1}{c|}{Tower}            & \multicolumn{1}{c|}{25}                   & \multicolumn{1}{c|}{XCOMET-XL QE}     & \textbf{86.56} & \multicolumn{1}{c|}{87.79}          & \multirow{3}{*}{\begin{tabular}[c]{@{}c@{}}$\delta$ = + 1.16\\ (88.95 - 87.79)\end{tabular}} \\
\multicolumn{1}{c|}{Tower}            & \multicolumn{1}{c|}{25}                   & \multicolumn{1}{c|}{Tower QE} & 85.40          & \multicolumn{1}{c|}{87.40}          &                                                                                              \\
\multicolumn{1}{c|}{Tower + Tower QE} & \multicolumn{1}{c|}{5 (25* for Tower QE)} & \multicolumn{1}{c|}{\_}               & 86.36          & \multicolumn{1}{c|}{\textbf{88.95}} &                                                                                              \\ \midrule
\multicolumn{6}{c}{\textit{Sentence-Level}}                                                                                                                                                                                                                                     \\ \midrule
\multicolumn{1}{c|}{Tower}            & \multicolumn{1}{c|}{25}                   & \multicolumn{1}{c|}{XCOMET-XL QE}     & \textbf{86.42}          & \multicolumn{1}{c|}{87.68}          & \multirow{3}{*}{\begin{tabular}[c]{@{}c@{}}$\delta$ = + 0.38\\ (88.06 - 87.68)\end{tabular}} \\
\multicolumn{1}{c|}{Tower}            & \multicolumn{1}{c|}{25}                   & \multicolumn{1}{c|}{Tower QE} & 85.23          & \multicolumn{1}{c|}{87.41}          &                                                                                              \\
\multicolumn{1}{c|}{Tower + Tower QE} & \multicolumn{1}{c|}{5 (25* for Tower QE)} & \multicolumn{1}{c|}{\_}               & 85.96          & \multicolumn{1}{c|}{\textbf{88.06}}          &                                                                                              \\ \bottomrule
\end{tabular}
}
\caption{Impact of integrating Unidirectional QE during decoding with paragraphs vs sentences on WMT23 English $\rightarrow$ German Test set. $\delta$ denotes the improvement in translation quality from re-ranking N-best list with XCOMET-XL QE to integrating unidirectional Tower QE during the decoding. Best scores according to each metric are highlighted in \textbf{bold}.}
\label{tab:sentvspara}
\end{table*}



We conduct a series of experiments to validate the effectiveness of Quality-Aware decoding and identify the scenarios where it provides the most benefit. First, we evaluate whether our token-level QE model can better estimate sequence quality compared to the log probabilities of the NMT model. Next, we assess the impact of Quality-Aware decoding by comparing it with other approaches to determine if it improves translation quality. We also perform an ablation study to examine whether training the QE model on errors from the same NMT model enhances its performance. Finally, we explore the impact of source sentence length to highlight the limitations of N-best list re-ranking.

\subsection{Quality Estimation Performance}

First, we evaluate the agreement between the Tower-based token-level QE model (\textbf{Tower QE}) and human scores for a given hypothesis. It is only beneficial if we achieve higher correlation than the average of the NMT model log probabilities to show the need to integrate it during decoding. Therefore, we report the correlation with human scores of different models on WMT 23 English $\rightarrow$ German in Table \ref{tab:correlation}. 

We observe that the best-performing systems are the Comet QE models, which predict a single score using the full hypothesis. This is expected, as these models assess quality after the hypothesis is fully generated. In contrast, both log probabilities and Tower QE scores are based on the predicted token of each decoding step, using only the preceding context. Log probabilities perform poorly in this setup, while our proposed model, Tower QE, achieves twice the correlation with human judgments compared to log probabilities, despite scoring token by token with preceding context. This result highlights the potential of integrating our approach into the decoding process.

\subsection{Unified Decoding for NMT}


To validate the effectiveness of our unified decoding approach, we compare it with several baselines in Table \ref{tab:qadecoding}. First, we evaluate whether our approach outperforms generating translations with the NMT model alone. Next, we check if the quality of translations improves compared to N-best list re-ranking. To make the setups comparable, we set $topk$ and $num\_beams$ to $5$ and compare with re-ranking the top $25$ beams using XCOMET-XL. Finally, to demonstrate that re-ranking the N-best list remains a viable and complementary approach, we re-rank the top $5$ beams obtained from Quality-Aware decoding using the same QE model. 

We find that re-ranking with XCOMET-XL and Tower QE yields similar results, indicating that our partial QE model does not over-fit to any specific metric. Furthermore, we observe that the unified decoding approach outperforms N-best list re-ranking across both metrics in both language pairs. For example, the MetricX score improves from $2.37$ to $2.12$ for English $\rightarrow$ German. Note that Tower has already seen this data during instruction-tuning and the improvement is not from new data but from Quality-Aware decoding. Moreover, re-ranking the top $5$ beams obtained from unified decoding with XCOMET-XL leads to a slight further improvement in quality. This highlights the robustness and generalizability of our approach across different evaluation metrics.
%\footnotetext{\href{https://github.com/WMT-QE-Task/wmt-qe-2023-data}{WMT 23 English $\rightarrow$ German QE Data}}

\subsection{Adapting for Tower Errors}

We use the MQM annotations from WMT to train our Tower QE model, which contains error annotations from other systems. However, a viable alternative would be to adapt Tower QE specifically to the errors it typically makes. To maintain a similar data setup, we first generate translations using Tower on these source sentences. Then, we annotate the generated hypotheses with XCOMET-XL using the reference and fine-tune Tower QE on this synthetic dataset, which we refer to as \textbf{Tower Distill QE}. We evaluate the performance of the new distill QE model and report the results in Table \ref{tab:towerdistill}.

We observe that the distilled QE model performs very similarly to the QE model trained on errors from other systems. This indicates that there was no significant benefit in adapting the QE model to the specific errors typically made by Tower. However, further analysis on larger datasets and different domains is needed to fully validate the effectiveness of the distillation approach as the current synthetic data generated is small.

\subsection{Sentence vs Document-level Translation}

From Table \ref{tab:qadecoding}, we observe that the gains for English $\rightarrow$ German (paragraph-level) are much higher than for Chinese $\rightarrow$ English (sentence-level). We hypothesize that this discrepancy arises from the length of the sentences, as the N-best list re-ranking is likely sufficient for shorter sentences. To confirm this, we take the English paragraphs and split them into sentences using a tokenizer while tracking the paragraph IDs. We then perform the entire decoding process similarly, and later join the sentences back using the paragraph IDs before evaluation. We report the results in Table \ref{tab:sentvspara}.

We define the impact as the improvement in translation quality from re-ranking the N-best list with XCOMET-XL QE to integrating Tower QE. Comparing the results at the paragraph level to those at the sentence level, we observe that the impact decreases, which confirms our hypothesis. Additionally, we obtain better scores at the document level, further highlighting the potential benefits of Quality-Aware Decoding.

\section{Related Work}

\textbf{Integrating QE in NMT:} Several advancements have been made in improving QE for NMT over the years \citep{rei2021references, rei2022cometkiwi, blain2023findings, zerva2024findings, guerreiro2024xcomet}. These developments have led to the integration of QE in various ways.
One common approach involves applying QE after generating multiple sequences through techniques such as QE re-ranking \citep{fernandes2022quality, faria2024quest} or Minimum Bayes Risk (MBR) decoding \citep{tomani2024quality}. Another direction focuses on removing noisy data using QE models, followed by fine-tuning on high-quality data \citep{xu2024contrastive, finkelstein2024introducing}. \citet{vernikos2024don} proposes to generate diverse translations as a first step and then combine them. We perform this explicitly by integrating the QE directly into decoding.
Recently, \citet{zhang2024learning} exploited the MQM data by training models to penalize tokens within an error span, improving translation quality. In contrast, our approach adopts a modular framework, where we propose an expert QE model that is trained independently for targeted training. This modular approach aims to improve performance by decomposing the task into separate translation and QE components.

\textbf{Reward Modeling in NLG:}  Quality-Aware decoding shares several similarities with controllable text generation methods, particularly in the use of an additional "Quality/Reward" model that guides the decoding. A well-explored approach for controlling text is altering the decoding with a reward model (Weighted Decoding) \citep{yang2021fudge}. This method modifies the decoding by adjusting token probabilities based on the reward model, allowing for more controlled generation.
Similarly, \citet{deng-raffel-2023-reward} also used a uni-directional reward model, with the aim of maintaining efficiency during generation. This approach minimizes computational complexity while still benefiting from the guiding influence of the reward model. Moreover, recent work by \citet{li-etal-2024-reinforcement} introduced a token-level reinforcement learning-based reward model, providing more fine-grained feedback that enhances control over text generation at a granular level. While similar, the key contribution in our work lies in the development of the first uni-directional QE model for translation. 


\section{Conclusion}
We have shown the importance of Quality-Aware decoding to improve translation quality, rather than relying on post-generation techniques. In this work, we demonstrated how MQM data can be used to build a uni-directional token-level QE model, which is then integrated into the decoding process. Through a series of experiments, we showed that our Quality-Aware decoding approach results in measurable improvements in translation quality. Notably, we did not introduce new training data to the NMT model, and show that the gains stem from Quality-Aware decoding.


\section{Limitations}
While our Quality-Aware decoding improves translation quality, it adds considerable computational complexity to the inference process. Theoretically, this approach would double the time needed to generate a translation and require additional memory to utilize the token-level QE model. One potential solution to mitigate this issue could be to use token-level QE as a reward model for training via Reinforcement Learning.

Additionally, we trained our model on a limited set of human-annotated MQM data. However, current QE models, such as XCOMET, are capable of predicting error tags using the reference with reasonable quality. This suggests that further improvements could be achieved if these models were trained on larger-scale datasets, providing more nuanced feedback and refining translation quality even further.

Lastly, our proposed token-level QE model does not account for error severity. Ideally, it should be able to predict the category of errors, allowing for more nuanced feedback and enabling the model to generate translations with only minor errors when necessary.


% Bibliography entries for the entire Anthology, followed by custom entries
%\bibliography{anthology,custom}
% Custom bibliography entries only
\bibliography{custom}

\appendix

\section{Appendix}
\label{sec:appendix}

% \begin{table*}[!ht]
% \centering
% \begin{tabular}{@{}ccccc@{}}
% \toprule
% \multicolumn{1}{c|}{Model}            & \multicolumn{1}{c|}{Beams}                & \multicolumn{1}{c|}{Re-ranking}              & XCOMET-XL      & XCOMET-XXL     \\ \midrule
% \multicolumn{5}{c}{\textit{English $\rightarrow$ German}}                                                                                                          \\ \midrule
% \multicolumn{1}{c|}{Tower}            & \multicolumn{1}{c|}{5}           & \multicolumn{1}{c|}{\_}                      & 84.93          & 86.93          \\
% \multicolumn{1}{c|}{Tower}            & \multicolumn{1}{c|}{25}                   & \multicolumn{1}{c|}{\textbf{\_}}             & 84.87 & 86.45          \\
% \multicolumn{1}{c|}{Tower MBR}        & \multicolumn{1}{c|}{25}                   & \multicolumn{1}{c|}{\_}                      & 85.23          & 87.09          \\
% \multicolumn{1}{c|}{Tower}            & \multicolumn{1}{c|}{25}                   & \multicolumn{1}{c|}{XCOMET-XL QE}            & 86.56          & 87.79          \\
% \multicolumn{1}{c|}{Tower}            & \multicolumn{1}{c|}{5}                    & \multicolumn{1}{c|}{Tower QE} & 85.34          & 87.33          \\
% \multicolumn{1}{c|}{Tower}            & \multicolumn{1}{c|}{25}                   & \multicolumn{1}{c|}{Tower QE} & 85.40          & 87.40          \\
% \multicolumn{1}{c|}{Tower + Tower QE} & \multicolumn{1}{c|}{5 (25* for Tower QE)} & \multicolumn{1}{c|}{\_}                      & 86.36          & 88.95          \\
% \multicolumn{1}{c|}{Tower + Tower QE} & \multicolumn{1}{c|}{5 (25* for Tower QE)} & \multicolumn{1}{c|}{XCOMET-XL QE}            & \textbf{86.88} & \textbf{89.08} \\ \midrule
% \multicolumn{5}{c}{\textit{Chinese $\rightarrow$ English}}                                                                                                         \\ \midrule
% \multicolumn{1}{c|}{Tower}            & \multicolumn{1}{c|}{5}                    & \multicolumn{1}{c|}{\_}                      & 85.38          & 88.91          \\
% \multicolumn{1}{c|}{Tower}            & \multicolumn{1}{c|}{25}                   & \multicolumn{1}{c|}{\_}                      & 85.29          & 88.71          \\
% \multicolumn{1}{c|}{Tower MBR}        & \multicolumn{1}{c|}{25}                   & \multicolumn{1}{c|}{\_}                      & 86.00          & 89.23          \\
% \multicolumn{1}{c|}{Tower}            & \multicolumn{1}{c|}{25}                   & \multicolumn{1}{c|}{XCOMET-XL QE}            & 87.04          & 89.49          \\
% \multicolumn{1}{c|}{Tower}            & \multicolumn{1}{c|}{5}                    & \multicolumn{1}{c|}{Tower QE} & 85.64          & 89.10          \\
% \multicolumn{1}{c|}{Tower}            & \multicolumn{1}{c|}{25}                   & \multicolumn{1}{c|}{Tower QE} & 85.93          & 89.51          \\
% \multicolumn{1}{c|}{Tower + Tower QE} & \multicolumn{1}{c|}{5 (25* for Tower QE)} & \multicolumn{1}{c|}{\_}                      & 86.01          & 89.82          \\
% \multicolumn{1}{c|}{Tower + Tower QE} & \multicolumn{1}{c|}{5 (25* for Tower QE)} & \multicolumn{1}{c|}{XCOMET-XL QE}            & \textbf{86.67} & \textbf{90.00} \\ \bottomrule
% \end{tabular}
% \caption{COMET scores on WMT23 English $\rightarrow$ German Test set. Both XCOMET metric columns use reference for reporting translation quality and do not when used for re-ranking }
% \end{table*}


\subsection{Training details}
\label{sec:training_detail}

We use the transformers library \citep{wolf-etal-2020-transformers} for training and inference with Tower-Instruct V2.  For adapting Tower to token-level QE, we use LoRA \citep{hulora} based fine-tuning with an additional classifier head. Therefore, we only train the adapters and the weights for classification head.

We add the adapters to the modules \textit{q\_proj,k\_proj,v\_proj,gate\_proj,up\_proj} and \textit{down\_proj}. We set a batch size for each device to 12 initially and enable \textit{auto\_find\_batch\_size} to \textit{True} on 4 NVIDIA RTX A6000 GPU's. For having a  larger batch size during training, we set \textit{gradient\_accumulation\_steps} to 6. We use a \textit{learning\_rate} of $1e^{-5}$. We set the \textit{eval\_steps} to $50$ and \textit{num\_train\_epochs} to $10$. The other parameters are set to default.

Using the cross-entropy loss for token-level QE directly is insufficient due to the fact that the majority of tokens are classified as '\textit{Good}'. Hence, we find that the weighted cross-entropy loss is essential when fine-tuning the model. For the training on human MQM data, we set the weights to $0.05,0.95$ to '\textit{Good}' and '\textit{Bad}' labels respectively. In the case of distilling from XCOMET, we observed more errors. Therefore, we find that setting them $0.2,0.8$ to '\textit{Good}' and '\textit{Bad}' labels respectively provided stable training.

We train on data until WMT'22 for training and use WMT'24 for validation. We calculate the macro '\textit{F1}' on token-level predictions as the validation metric and stop training if it does not improve for 10 consecutive \textit{eval\_steps}.

\subsection{Partial vs Full Sequence Quality Estimation}

We also compare the difference in performance between our proposed token-level QE for partial sequences with Tower trained for full sequence QE. We achieve this by adding a regression head to predict the score at the end-of-sentence token. Hence, the model uses the source and hypothesis to predict the score using regression head at the end.

We fine-tune the model using only direct assesment data \citep{zerva2024findings} (\textbf{Tower Full DA}). Furthermore, we use this as initialisation and continue fine-tuning on the MQM data (\textbf{Tower Full DA + MQM}). We also use LoRA similarly to the previous model with a regression head to adapt the model. We report the scores in Table \ref{tab:correlation_ablation}.

We see that the both Tower QE models based on full sentences outperforms the partial model. However, this is expected as it has seen the entire context and was also trained on larger amounts of data. Nonetheless, the partial model still achieves much higher correlaiton that the log probabilities showcasing its potential for Quality-Aware decoding.

\subsection{Robustness to re-ranking weight}

In our method, we introduce a hyperparameter, $\alpha$, to merge the probabilities from the token-level QE model and the translation model. This section analyzes the impact of $\alpha$ on the final translation quality.

To efficiently evaluate its effect, we re-rank the N-best list using different values of $\alpha$. This approach allows us to estimate the ideal value of $\alpha$ without the need for joint decoding multiple times. If the re-ranking model (in this case, Tower QE) is beneficial, we expect that any $\alpha$ less than 1 will improve translation quality, as it demonstrates that incorporating the probabilities from the QE model is helpful.

We visualize this impact in Figure \ref{fig:mainfigure}. The results show that using an $\alpha$ less than 1 leads to improved translation quality in both scenarios. This indicates that relying entirely on the NMT model does not yield the best results and highlights the importance of the Tower QE model.

Thus, we emphasize that re-ranking the N-best list provides an effective way to tune the value of $\alpha$, and it remains robust to different values.

\begin{figure*}[!htpb]
\begin{promptbox}[title={Tower Translation Prompt}]
    \small
    <|im\_start|>user\\
    Translate the sentence from English into German.\\
    English: \{src\_sent\}\\
    German:\\
    <|im\_end|>\\
    <|im\_start|>assistant
\end{promptbox}

\begin{promptbox}[title={Tower Token-Level QE Prompt}]
    \small
    English:\{src\_sent\}\\
    German: \{tgt\_sent\}
\end{promptbox}
\caption{Prompts used in our experiments for translation and QE model. \{src\_sent\} and \{tgt\_sent\} represent the source and target sentence. We replace the language with Chinese and English when experimenting with that language pair.}
\end{figure*}

\begin{figure*}[!htpb]
    \centering
    % First subfigure
    \begin{subfigure}[b]{0.5\textwidth}
        \centering
        \includegraphics[width=\textwidth]{Figures/alphas_ende_25.png} % Replace with your image path
        \caption{English $\rightarrow$ German}
        \label{fig:subfigure1}
    \end{subfigure}
    
    \vspace{0.5cm} % Adjust space between the two subfigures

    % Second subfigure
    \begin{subfigure}[b]{0.5\textwidth}
        \centering
        \includegraphics[width=\textwidth]{Figures/alphas_zhen_25.png} % Replace with your image path
        \caption{Chinese $\rightarrow$ English}
        \label{fig:subfigure2}
    \end{subfigure}
    
    \caption{Impact of $\alpha$ when re-ranking with token-level Tower QE on WMT'23 Test sets.}
    \label{fig:mainfigure}
\end{figure*}


\begin{table*}[!ht]
\centering
\begin{tabular}{@{}c|ccc@{}}
\toprule
                                                                                      & Pearson        & Spearmann      & Kendall        \\ \midrule
COMETQE                                                                               & \textbf{44.41} & 41.29          & 31.19          \\ \midrule
COMETQE-XL                                                                            & 41.23          & \textbf{42.17} & \textbf{31.84} \\ \midrule
\begin{tabular}[c]{@{}c@{}}COMETQE Scratch\\      Fine-tuned (ours)\end{tabular}      & 36.32          & 33.66          & 25.24          \\ \midrule
Tower Log Prob                                                                        & 32.32          & 16.74          & 12.77          \\ \midrule
\begin{tabular}[c]{@{}c@{}}Tower Partial QE\end{tabular} & 40.56          & 33.96          & 25.87          \\ \midrule
Tower Full DA                                                                        & 33.67          & 36.46          & 27.38          \\ \midrule
Tower Full DA + MQM                                                                 & 32.03          & 40.85          & 30.38          \\ \bottomrule
\end{tabular}
\caption{Full Correlation results on WMT 23 for English $\rightarrow$ German Test set. Partial indicates that the QE model predict scores via token-level where as full indicates predicting the score at the end-of-sentence token. The scores are calculated after removing the few sentences labelled for hallucination detection. Best scores according to each coefficient are highlighted in \textbf{bold}.}
\label{tab:correlation_ablation}
\end{table*}




\end{document}



\newpage
\appendix
\section*{Appendix}

\section{Essay Lists}
\label{sec:essay-lists}

\begin{table}[htbp]
\centering
\small
\renewcommand{\arraystretch}{1.2} 
\begin{tabular}{@{}c c@{}}
\toprule
\textbf{Collection} & \textbf{Articles} \\ \midrule
\textit{Essays from Qiejie Pavilion} & 38 \\
\textit{Essays from Qiejie Pavilion II} & 48 \\
\textit{Final Essays from Qiejie Pavilion} & 14 \\
\textit{False Freedom} & 35 \\
\textit{Grave} & 25 \\
\textit{Hot Wind} & 42 \\
\textit{Marginalia} & 61 \\
\textit{Quasi-Romance} & 66 \\
\textit{Unlit Collection} & 33 \\
\textit{Sequel to Unlit Collection} & 28 \\
\textit{Sequel to Sequel of Unlit Collection} & 7 \\
\textit{Southern Accent Northern Tune} & 50 \\
\textit{That Is All} & 31 \\
\textit{Two Hearts} & 39 \\
\textit{Three Leisures} & 34 \\
\textit{Uncollected Works} & 33 \\
\textit{Supplement to Uncollected Works} & 54 \\ \midrule
\textit{All Essays} & 638 \\ \bottomrule
\end{tabular}
\caption{List of Lu Xun's essay collections and article counts.}
\label{tab:essay-collections}
\end{table}

The experimental dataset comprises 17 essay collections by Lu Xun, totaling 638 articles sourced from Wikisource (see Table \ref{tab:essay-collections}). 
These collections span the entirety of Lu Xun's mature essayistic output, reflecting diverse themes of his intellectual career. 
All texts fall within the public domain under the copyright law, ensuring unrestricted scholarly usage. 

This comprehensive compilation enables systematic analysis of linguistic patterns, thematic evolution, and rhetorical strategies across Lu Xun's essays.





\section{Data Examples}
\label{sec:data-examples}


\begin{table}[htbp]
    \small
    \centering
    \begin{tabular}{@{}p{\linewidth}@{}}
        \toprule
        \textbf{Multiple-Choice Question Example} \\
        \midrule
        \textbf{Instruction:} Do you think critics and creators should take on the same role? \\
        A: They should not, because criticism and creation require different skill sets.  \\
        B: They can, because the two can mutually enhance each other.  \\
        C: They should not, because creating work can affect the fairness of criticism.  \\
        D: They can, as long as the critic has sufficient creative talent. \\
        
        ~
        
        \textbf{Response:} A \\
        
        \bottomrule
    \end{tabular}
    \caption{Example of the Multiple-Choice Question task.}
    \label{tab:mcq_example}
\end{table}

\begin{table}[htbp]
    \small
    \centering
    \begin{tabular}{@{}p{\linewidth}@{}}
        \toprule
        \textbf{Generative Question Answering Example} \\
        \midrule
        \textbf{Instruction:} What aspects of responsibility do you think contribute to poor translation quality? \\
        
        ~
        
        \textbf{Response:} \\
        The responsibility for poor translation quality lies not only with the translators but also with the literary world and publishing industry, especially critics, who should also share the blame. To address this situation, there must be proper criticism to point out the bad and reward the good. \\
        
        \bottomrule
    \end{tabular}
    \caption{Example of the Generative Question Answering task.}
    \label{tab:idea_extraction}
\end{table}

\begin{table}[htbp]
    \small
    \centering
    \begin{tabular}{@{}p{\linewidth}@{}}
        \toprule
        \textbf{Style Transfer Example} \\
        \midrule
        \textbf{Instruction:} Describe the following paragraph in your tone: \\
        When I was sick, reading these books, though heavy on the heart, made me realize that some intelligent readers can still find a touch of leisure amidst their suffering. \\
        
        ~ \\

        \textbf{Response:} Reading these books while ill, in the end, still leaves one feeling stifled. Yet, I have come to understand that some wise readers can still manage to find a sense of leisure even amidst pools of blood. \\
        \bottomrule
    \end{tabular}
    \caption{Example of the Style Transfer task.}
    \label{tab:style_transfer_example}
\end{table}

We illustrate, through concrete examples, the diverse task types and data forms encompassed in the research dataset. 
Table~\ref{tab:mcq_example} presents a typical multiple-choice question task that employs an instruction-option structure, requiring the model to select accurately based on the semantic content of the question, thereby demonstrating its classification and judgment capabilities. 
Table~\ref{tab:idea_extraction} showcases a generative question-answering task in which the model is asked to articulate its viewpoints in response to open-ended questions, testing its logical reasoning and language expression abilities. 
Table~\ref{tab:style_transfer_example}, on the other hand, demonstrates a style transfer task that requires the model to reconstruct the language style while preserving the original meaning, thereby assessing its ability to maintain semantics and adapt its language. 
Although all three examples utilize a standardized instruction-answer format, the distinct design of each task validates the model's capabilities across different dimensions—namely, closed choice, open generation, and style migration. 
Collectively, these data instances form a multidimensional test benchmark for evaluating language understanding and generation capabilities.






\section{Case Study}
\label{sec:case-study}

We identify five key error types in baseline models:
(1) Distractor Bias: Over-reliance on superficially plausible but contextually irrelevant options in multiple-choice tasks.
(2) Faulty Attribution: Misrepresenting causal relationships or textual evidence in generative answers.
(3) Context Neglect: Ignoring textual or cultural context when interpreting statements.
(4) Misaligned Metaphor: Introducing semantically inconsistent analogies during style transfer.
(5) Concept Drift: Altering core concepts or terminology during paraphrasing, distorting original intent.


In Section \ref{sec:human-evaluation}, we have analyzed the error types for Distractor Bias, Faulty Attribution, and Misaligned Metaphor. 
To further illustrate the advantages of CharacterBot, we present additional case studies across various tasks, comparing its performance against baselines such as GPT-4o, Llama 3.1-8B, and Tongyi Xingchen, as shown in Table~\ref{tab:case_full}. The corresponding Chinese version can be found in Table~\ref{tab:case_full_chinese}.

In generative question answering, baseline models often overlook crucial details. 
For example, Tongyi Xingchen’s response overlooks the context and misidentifies Lu Sao, even drawing an erroneous connection to Rousseau, thereby failing to recognize that the punishment was driven by the influence of his ideas.
In contrast, Characterbot correctly indicates that Lu Sao was penalized for the impact of his ideas rather than for any wrongdoing.

In style transfer, baseline models frequently reinterpret key phrases in ways that diverge from the original intent.
For instance, GPT-4o rephrases ``eliminating disasters'' as ``changing our fate,'' which shifts the rhetorical emphasis and the original statement.
CharacterBot avoids this issue by preserving both semantic integrity and rhetorical intent, maintaining the accuracy and expressiveness of the original text.


\begin{table*}[t]
\centering
\small
\begin{tabularx}{\textwidth}{X}
\toprule
\multicolumn{1}{c}{\textbf{Case Study}} \\
\midrule
\multicolumn{1}{M}{\textit{Multiple-Choice Questions}} \\
\midrule
Resolved Error type: \textbf{Distractor Bias}\\[0.5em]
Question: Please choose the most appropriate answer from the following four options. Why do you think literature in China has not undergone significant changes and progress?\\
A: Because literary figures lack sufficient passion\\
B: Because revolutionaries dislike literature\\
C: Because there is no revolution and progress as a backdrop\\
D: Because the influence of old-school figures is too significant\\[0.5em]
\textit{\textbf{Baseline (GPT-4o)}}: D: Because \erritalic{the influence of old-school figures} is too significant\\[0.5em]
\textit{\textbf{Characterbot}}: C: Because there is \corritalic{no revolution and progress as a backdrop}\\[0.5em]
\textit{\textbf{Essay Excerpt:}}: ``China claims to have undergone revolution - politically perhaps true, but literarily unchanged. \corritalic{This implies literature's stagnation directly reflects the absence of true revolution and progress} - though revolutionaries might dislike this interpretation''. — \textit{An Overview of Contemporary New Literature}\\
\midrule
\multicolumn{1}{I}{\textit{Generative Question Answering}} \\
\midrule
Resolved Error type: \textbf{Faulty Attribution}\\[0.5em]
Question: Why do you think that the most effective critic of classical texts is precisely someone who has read them?\\[0.5em]
\textit{\textbf{Baseline (Llama3.1-8B)}}: This is because they \erritalic{once possessed passion and reverence for the classical texts, but as time passed, that passion and reverence gradually faded}, leaving behind a void and insincere knowledge.\\[0.5em]
\textit{\textbf{Characterbot}}: Because those who have read the classical texts are often \corritalic{able to spot their flaws and mistakes}, they are more likely to criticize and dismiss the texts.\\[0.5em]
\textit{\textbf{Essay Excerpt:}}: ``The most effective disparagers of ancient texts are precisely those who have read them thoroughly. \corritalic{Only they can expose the inherent contradictions and 'turn their own weapons against them' through deep familiarity}…'' — \textit{Ancient Texts and Vernacular Chinese}\\
\midrule
Resolved Error type: \textbf{Context Neglect}\\[0.5em]
Question: Why do you mention that the punishment Lu Sao received was ``a crime of influence rather than an actual crime''?\\[0.5em]
\textit{\textbf{Baseline (Tongyi Xingchen)}}: \erritalic{I have never mentioned "Lu Sao," nor do I know why he was punished}. Are you referring to Rousseau? As for the so-called ``crime of influence,'' I do not understand your meaning.\\[0.5em] 
\textit{\textbf{Characterbot}}: Because Lu Sao was not inherently an evil person. The punishment he received was \corritalic{due to the influence of his thoughts and words on others}, rather than for any actual crime of his own.\\[0.5em] 
\textit{\textbf{Essay Excerpt:}}: As for his (Liang Shiqiu's) attack on Lu Sao, the second reason is that ``\corritalic{Lu Sao... has already become a representative model of the general behavior of romantic writers}. An attack on Lu Sao's morality can be said to be an attack on the behavior of romantic people in general...'' \corritalic{The punishment he is now receiving is for the crime of influence, not for an actual crime—how lamentable!} — \textit{Head}\\
\midrule
\multicolumn{1}{S}{\textit{Style Transfer}} \\
\midrule
Resolved Error type: \textbf{Misaligned Metaphor}\\[0.5em]
Question: Rephrase the following passage in your own tone: ``My art takes delicacy as its life, and wields a sharpness akin to an anatomical scalpel as its strength.''\\[0.5em]
\textit{\textbf{Baseline (Llama3.1-8B)}}: My art is practically an extension of my life. \erritalic{Delicacy is the blood of my life}, requiring a force as sharp as an anatomical scalpel.\\[0.5em]
\textit{\textbf{Characterbot}}: My art is delicate, \corritalic{its very essence is defined by delicacy}, yet it is supported by a force as incisive as an anatomical scalpel.\\
\midrule
Resolved Error type: \textbf{Concept Drift}\\[0.5em]
Question: Rephrase the following passage in your own tone: ``Aren’t we unworthy of eliminating disasters ourselves? In the future, should we rely on Panchen Lama, or Mei Lanfang, or perhaps Xu Lai and Hu Die?''\\[0.5em]
\textit{\textbf{Baseline (GPT-4o)}}: Have we already lost the power to \erritalic{change our destiny on our own}, thereby placing our hope in religious symbols like Panchen Lama, or in theatrical celebrities such as Mei Lanfang, Xu Lai, and Hu Die?\\[0.5em]
\textit{\textbf{Characterbot}}: Could it be that we are no longer capable of \corritalic{eradicating disasters ourselves}, so that henceforth, everything must depend on Panchen Lama, Mei Lanfang, Xu Lai, and Hu Die?\\
\bottomrule
\end{tabularx}
\caption{Complete case studies comparing the responses of our model \textit{\textbf{Characterbot}} with those of the baseline models (including \textit{\textbf{GPT-4o}}, \textit{\textbf{Llama3.1-8B}} and \textit{\textbf{Tongyi Xingchen}}). \erritalic{Red} text indicates errors, while \corritalic{blue} text indicates correct responses.}
\label{tab:case_full}
\end{table*}


\begin{table*}[t]
\centering
\small
\begin{tabularx}{\textwidth}{X}
\toprule
\multicolumn{1}{c}{\textbf{Case Study}} \\
\midrule
\multicolumn{1}{M}{\textit{Multiple-Choice Questions}} \\
\midrule
Resolved Error type: \textbf{Distractor Bias}\\[0.5em]
Question: \begin{CJK}{UTF8}{gbsn}请在以下四个选项中选择一个最合适的答案。你认为为什么文学在中国没有发生太大变化和进步?\end{CJK}\\
A: \begin{CJK}{UTF8}{gbsn}因为文学家们不够热情\end{CJK}\\
B: \begin{CJK}{UTF8}{gbsn}因为革命家不喜欢文学\end{CJK}\\
C: \begin{CJK}{UTF8}{gbsn}因为没有革命和进步作为背景\end{CJK}\\
D: \begin{CJK}{UTF8}{gbsn}因为旧式人物的影响太大\end{CJK}\\[0.5em]
\textit{\textbf{Baseline (GPT-4o)}}: D: \begin{CJK}{UTF8}{gbsn}\erritalic{旧式人物的影响}太大\end{CJK}\\[0.5em]
\textit{\textbf{Characterbot}}: C: \begin{CJK}{UTF8}{gbsn}\corritalic{没有革命和进步作为背景}\end{CJK}\\[0.5em]
\textit{\textbf{Essay Excerpt:}}: \begin{CJK}{UTF8}{gbsn}``中国,据说,自然是已经革了命,—政治上也许如此罢,但在文艺上,却并没有改变。\corritalic{这照我上面所讲的推论起来,就是文学并不变化和兴旺,所反映的便是并无革命和进步},—虽然革命家听了也许不大喜欢。''\end{CJK} — \begin{CJK}{UTF8}{gbsn}\textit{现今的新文学的概观}\end{CJK}\\
\midrule
\multicolumn{1}{I}{\textit{Generative Question Answering}} \\
\midrule
Resolved Error type: \textbf{Faulty Attribution}\\[0.5em]
Question: \begin{CJK}{UTF8}{gbsn}为何你认为菲薄古书的最有力者恰恰是读过古书的人?\end{CJK}\\[0.5em]
\textit{\textbf{Baseline (Llama3.1-8B)}}: \begin{CJK}{UTF8}{gbsn}这是因为他们\erritalic{曾经拥有过对古书的热情和崇敬,然而随着时间的流逝,他们的热情和崇敬逐渐消散},留下的是一种空虚和虚伪的知识。\end{CJK}\\[0.5em]
\textit{\textbf{Characterbot}}: \begin{CJK}{UTF8}{gbsn}因为读过古书的人,往往\corritalic{能够看出其中的破绽与错误},所以他们更有可能对古书进行批评和否定。\end{CJK}\\[0.5em]
\textit{\textbf{Essay Excerpt:}}: \begin{CJK}{UTF8}{gbsn}``菲薄古书者,惟读过古书者最有力,这是的确的。\corritalic{因为他洞知弊病,能‘以子之矛攻子之盾’}…''\end{CJK} — \begin{CJK}{UTF8}{gbsn}\textit{古书与白话}\end{CJK}\\
\midrule
Resolved Error type: \textbf{Context Neglect}\\[0.5em]
Question: \begin{CJK}{UTF8}{gbsn}为什么你提到卢骚受到的处罚是``影响罪而非本罪''?\end{CJK}\\[0.5em]
\textit{\textbf{Baseline (Tongyi Xingchen)}}: \begin{CJK}{UTF8}{gbsn}\erritalic{我从未提及``卢骚'',也不知道他为何受罚}。你指的是卢梭?至于所谓``影响罪'',我不明白你的意思。\end{CJK}\\[0.5em]
\textit{\textbf{Characterbot}}: \begin{CJK}{UTF8}{gbsn}因为卢骚本非作恶之人,他所受的处罚是\corritalic{因为他的思想和言论影响了他人,而不是他自己的本罪}。\end{CJK}\\[0.5em]
\textit{\textbf{Essay Excerpt:}}: \begin{CJK}{UTF8}{gbsn}至于他(梁实秋)之攻击卢骚,理由之二,则在``\corritalic{卢骚……已然成为一般浪漫文人行为之标类的代表},对于卢骚的道德的攻击,可以说即是给一般浪漫的人的行为的攻击……''……\corritalic{他现在所受的罚,是因为影响罪,不是本罪了,可叹也夫!}''\end{CJK} — \begin{CJK}{UTF8}{gbsn}\textit{头}\end{CJK}\\
\midrule
\multicolumn{1}{S}{\textit{Style Transfer}} \\
\midrule
Resolved Error type: \textbf{Misaligned Metaphor}\\[0.5em]
Question: \begin{CJK}{UTF8}{gbsn}用你的口吻叙述下列段落:我的艺术以细腻为生命,同时以像解剖刀一样的锋利作为力量。\end{CJK}\\[0.5em]
\textit{\textbf{Baseline (Llama3.1-8B)}}: \begin{CJK}{UTF8}{gbsn}我的艺术,简直是我的生命的延续。\erritalic{细腻是我的生命的血液},需要一种像解剖刀一样的锋利的力量。\end{CJK}\\[0.5em]
\textit{\textbf{Characterbot}}: \begin{CJK}{UTF8}{gbsn}我的艺术是纤细的,\corritalic{以纤细为生命的},但又以解剖刀一般尖锐的力量为后盾。\end{CJK}\\
\midrule
Resolved Error type: \textbf{Concept Drift}\\[0.5em]
Question: \begin{CJK}{UTF8}{gbsn}用你的口吻叙述下列段落:我们自己是不是不配消除灾难了,以后该依靠班禅大师,还是梅兰芳,或者是徐来和胡蝶呢?\end{CJK}\\[0.5em]
\textit{\textbf{Baseline (GPT-4o)}}: \begin{CJK}{UTF8}{gbsn}我们是否已经丧失了\erritalic{自力改变命运的力量},而将希望寄托于班禅大师那般的宗教符号,或是梅兰芳、徐来、胡蝶这样的戏剧名伶呢?\end{CJK}\\[0.5em]
\textit{\textbf{Characterbot}}: \begin{CJK}{UTF8}{gbsn}莫非我们\corritalic{自己已经不能铲除灾祸},于是此后的事,都得靠班禅大师,梅兰芳,徐来和胡蝶了么?\end{CJK}\\
\bottomrule
\end{tabularx}
\caption{Complete case studies comparing the responses of our model \textit{\textbf{Characterbot}} with those of the baseline models (including \textit{\textbf{GPT-4o}}, \textit{\textbf{Llama3.1-8B}} and \textit{\textbf{Tongyi Xingchen}}) in Chinese. \erritalic{Red} text indicates errors, while \corritalic{blue} text indicates correct responses.}
\label{tab:case_full_chinese}
\end{table*}


\section{Prompts}
\label{sec:prompts}

We present various prompt templates in this section. Tables~\ref{tab:prompt1_generate_mcq} and \ref{tab:prompt2_generate_qa} outline the prompts to generate question-answer pairs and multiple-choice questions, respectively. 
Table~\ref{tab:prompt3_sentence_conversion} provides a template for selecting representative sentences from Lu Xun’s work and transforming them into modern vernacular Chinese, resulting in pairs of sentences, one in a style-neutral version and the other in Lu Xun’s distinctive style. 
Table~\ref{tab:prompt4_paraphrase} is the prompt using Authorial Perspective Reframing to convert Lu Xun's essay texts into reframed data for pre-training. 
Finally, Tables~\ref{tab:prompt5_evaluation} and \ref{tab:prompt_language_style_consistency} describe the evaluation criteria used to assess the responses in generative question answering and style transfer tasks.

\begin{table*}[htbp]
    \small
    \centering
    \begin{tabular}{@{}p{\linewidth}@{}}
        \toprule
        \textbf{Prompt for Generating Multiple-Choice Questions from Lu Xun’s Article} \\
        \midrule
        This article titled \{\textit{title}\} was written by Lu Xun. The content is as follows: \{\textit{input\_article}\}. Please generate three multiple-choice questions with answers based on the article. \\
        \\
        Each question should directly inquire about Lu Xun's viewpoints, phrased in second-person (``you'') addressing him directly, avoiding terms like ``the author,'' ``Lu Xun,'' ``this article,'' or ``the text.'' Each question must have four options with one correct answer. If involving specific concepts, explain them within the question. All content must strictly derive from the article without external references. \\
        \\
        Output must be in a strict JSON array format without extra characters, with each object containing only ``question,'' ``option,'' and ``answer'' keys (with the answer value being ``A''/ ``B''/ ``C''/ ``D''). \\
        \bottomrule
    \end{tabular}
    \caption{Prompt for generating Multiple-Choice Questions from Lu Xun’s article.}
    \label{tab:prompt1_generate_mcq}
\end{table*}

\begin{table*}[htbp]
    \small
    \centering
    \begin{tabular}{@{}p{\linewidth}@{}}
        \toprule
        \textbf{Prompt for Generating Question-Answer Pairs from Lu Xun’s Article} \\
        \midrule
        This article titled \{\textit{title}\} was written by Lu Xun. The content is as follows: \{\textit{input\_article}\}. Please generate three question-answer pairs based on this article. \\
        \\
        Each question should directly inquire about Lu Xun's viewpoints based on the article's content. Questions must be phrased in second-person (``you'') addressing Lu Xun directly, without mentioning terms like ``the author,'' ``Lu Xun,'' ``this article,'' or ``the text.'' If involving specific concepts from the article, explain them within the question. All questions must strictly adhere to the article's content without introducing external elements. Answers should reflect Lu Xun's original language style and perspective, maintain detailed accuracy aligned with the text, and avoid terms like ``Lu Xun,'' ``author,'' or ``article.'' \\
        \\
        Output must be in a strict JSON array format without extra characters, with each object containing only ``question'' and ``answer'' keys. \\
        \bottomrule
    \end{tabular}
    \caption{Prompt for generating Question-Answer Pairs from Lu Xun’s article.}
    \label{tab:prompt2_generate_qa}
\end{table*}

\begin{table*}[htbp]
    \small
    \centering
    \begin{tabular}{@{}p{\linewidth}@{}}
        \toprule
        \textbf{Prompt for Transforming Sentence Style from Lu Xun’s Article} \\
        \midrule
        You are familiar with Lu Xun's writing style and modern vernacular Chinese. Given a passage from Lu Xun's work, select three representative sentences and convert them to modern vernacular. 
        
        ~
        
        [Requirements]
        
        1. Selected sentences should exemplify Lu Xun's stylistic features and reflect the article's main themes.\\
        2. The length of each selected sentence should be around 100 characters (not less than 80 characters and not more than 120 characters) to ensure the sentence is complete and informative.\\
        3. When converting, use smooth modern vernacular Chinese to convey the original meaning of the sentence, avoiding overly complex or obscure expressions.\\
        4. The output must be in strict JSON array format, for example: [{``original'': ``Original Sentence 1,'' ``plain'': ``Vernacular Version 1''}, {``original'': ``Original Sentence 2,'' ``plain'': ``Vernacular Version 2''}]. 
        Do not include any extra characters or explanations.\\
        
        ~
        
        [Input Article]  
        
        \{\textit{input\_article}\} \\
        \bottomrule
    \end{tabular}
    \caption{Prompt for transforming Sentence Style from Lu Xun’s article.}
    \label{tab:prompt3_sentence_conversion}
\end{table*}



\begin{table*}[htbp]
    \small
    \centering
    \begin{tabular}{@{}p{\linewidth}@{}}
        \toprule
        \textbf{Prompt for Paraphrasing Lu Xun's Article Content using Authorial Perspective Reframing} \\
        \midrule
        Please use modern Chinese (vernacular) to paraphrase the following Lu Xun article from a third-person perspective and accurately convey the original information.
        
        ~
        
        [Requirements]
        
        1. Clear Attribution: Use phrases such as ``Lu Xun pointed out,'' ``Lu Xun believed,'' ``Lu Xun said,'' ``Lu Xun criticized,'' ``Lu Xun mocked,'' or other appropriate expressions to indicate that the viewpoints belong to Lu Xun. 
        Each paraphrased paragraph must include at least one explicit attribution to Lu Xun; when necessary, the attribution must clearly state that the viewpoint belongs to Lu Xun.\\
        2. Sentence-by-Sentence Paraphrasing: Each sentence of the original text must be paraphrased individually in smooth modern Chinese from a third-person perspective. 
        The paraphrasing should maintain clear logic, ensure no information is omitted, and avoid adding any personal interpretation.\\
        3. Use fluent modern Chinese expressions (no classical Chinese).\\
        4. Include Lu Xun's name as frequently as possible throughout the paraphrased article.\\
        5. Output only the paraphrased text.\\

        ~
        
        [Input Article]  
        
        \{\textit{input\_article}\} \\
        \bottomrule
    \end{tabular}
    \caption{Prompt for paraphrasing Lu Xun's article content using Authorial Perspective Reframing.}
    \label{tab:prompt4_paraphrase}
\end{table*}



\begin{table*}[htbp]
    \small
    \centering
    \begin{tabular}{@{}p{\linewidth}@{}}
        \toprule
        \textbf{Prompt for Evaluating Responses in the Generative Question Answering Task} \\
        \midrule
        You are a reviewer and scoring expert who is very familiar with Lu Xun's literature. 
        Some AI chatbots are simulating Lu Xun's responses. 
        You are now required to evaluate the following AI chatbots' responses simulating Lu Xun, based solely on the provided original Lu Xun text.

        ~
        
        1. Evaluate whether the response conforms to the language style of the provided original Lu Xun text:
        
           - Consider only language style factors; do not take into account the core content meaning or any other factors.
           
           [Direct deduction conditions]
           
           - Using modern vernacular Chinese,
           
           - Using non-Chinese language,
           
           - Using commonly used vocabulary in modern vernacular Chinese that does not match the style of the provided original Lu Xun text,
           
           - Using a preachy tone.

        ~
        
        2. Evaluate whether the response conforms to the core content meaning of the provided original Lu Xun text (including whether it aligns with the facts of the provided original Lu Xun text, and whether it is consistent with the ideas, emotions, or stances expressed in the provided original Lu Xun text):
        
           - Consider only the core content meaning factors; do not take into account the language style or any other factors.
           
           [Direct deduction conditions]
           
           - Being vague and off-topic, unrelated to the original Lu Xun text.

        ~
        
        3. Score these two aspects separately (on a scale of 1-5), where 1 indicates extremely non-compliant and 5 indicates highly compliant.

        ~
        
        Please evaluate each AI chatbots's response according to the following format. 
        For each AI chatbots's response, output only 4 lines, with the 2nd and 4th lines containing only the numeric scores:
        
        Line 1: A brief evaluation of the language style.
        
        Line 2: Language style score (1-5).
        
        Line 3: A brief evaluation of the core content meaning.
        
        Line 4: Core content meaning score (1-5).

        ~
        
        Evaluation begins:

        ~
        
        [Original Lu Xun Text]  
        
        \{\textit{original\_text}\}

        ~
        
        [User's Question]  
        
        \{\textit{user\_question}\}

        ~
        
        [AI Chatbots's Response]  
        
        \{\textit{ai\_response}\} \\
        \bottomrule
    \end{tabular}
    \caption{Prompt for evaluating responses in the Generative Question Answering task.}
    \label{tab:prompt5_evaluation}
\end{table*}




\begin{table*}[htbp]
    \small
    \centering
    \begin{tabular}{@{}p{\linewidth}@{}}
        \toprule
        \textbf{Prompt for Evaluating Language Style Matching Consistency} \\
        \midrule
        You are an expert in language style analysis. Please determine whether the language style of the generated text and the answer is consistent, with 1 indicating consistency and 0 indicating inconsistency. Please output only the number 1 or 0, without any additional content. \\
        \\
        Question: \\
        \{\textit{question}\} \\
        \\
        Generated Text: \\
        \{\textit{generated\_text}\} \\
        \\
        Answer: \\
        \{\textit{answer}\} \\
        \bottomrule
    \end{tabular}
    \caption{Prompt for evaluating Language Style Matching Consistency.}
    \label{tab:prompt_language_style_consistency}
\end{table*}




\end{document}

