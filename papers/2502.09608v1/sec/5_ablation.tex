\section{Limitations and Future Work}
While our method successfully segments scene sketches across various styles and challenging cases, it has some limitations.
First, our bounding box filtering technique may still include undesired boxes, potentially introducing artifacts when combining masks into the final segmentation (\Cref{fig:limitations}a).
Second, our mask generation relies on SAM, which generally produces good masks but can occasionally introduce artifacts, particularly for objects occupying large regions in the sketch (\Cref{fig:limitations}b). Even after applying our refinement stage, some artifacts may persist in the final segmentation. Future work could address this issue by fine-tuning SAM specifically for sketches or incorporating a learned refinement stage.
% Lastly, oveerlapping masks still pose a challenge, while out depth-based approach aim to choose the, depth cues may not always provide a reliable signal for resolving ambiguous pixels in overlapping masks, leading to artifacts, especially along shared contours between overlapping objects.



\begin{figure}
    \includegraphics[width=1\linewidth]{figs/limitations_seg.pdf}
    \caption{Examples showcasing the limitations of our approach. (a) The bounding box filtering process can still retain undesired boxes, leading to artifacts in the final segmentation. (b) SAM masks oftentimes are decent, but sometimes do contain noticeable artifacts for larger objects. }
    \label{fig:limitations}
\end{figure}

 

\section{Conclusions}
We introduced SketchSeg, a method for instance segmentation of raster scene sketches. Our approach adapts Grounding DINO, an object detection model trained on natural images, to the sketch domain through class-agnostic fine-tuning. We utilized Segment Anything (SAM) for segmentation along with a refinement stage that incorporates depth cues to resolve ambiguous pixels.
Our method significantly improves upon state-of-the-art approaches in this domain, demonstrating the utility of natural image priors for sketch understanding tasks. We additionally provide a synthetic scene-level annotated sketch dataset encompassing a wide range of object categories and significant variations in drawing styles. Our experiments demonstrate that SketchSeg is robust to these variations, achieving consistent performance across diverse datasets.
% The dataset will be made publicly available to support further research and advancements in sketch understanding.