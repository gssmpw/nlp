
% \documentclass[10pt,twocolumn,letterpaper]{article}

% %%%%%%%%% PAPER TYPE  - PLEASE UPDATE FOR FINAL VERSION
% % \usepackage{cvpr}              % To produce the CAMERA-READY version
% % \usepackage[review]{cvpr}      % To produce the REVIEW version
% \usepackage[pagenumbers]{cvpr} % To force page numbers, e.g. for an arXiv version
% \usepackage{minitoc}
% \renewcommand \thepart{}
% \renewcommand \partname{}

% % Import additional packages in the preamble file, before hyperref
% \usepackage[dvipsnames]{xcolor}
% \usepackage[export]{adjustbox}
% \usepackage{wrapfig}
% \usepackage[subrefformat=parens]{subcaption}

% % \setlength{\intextsep}{0.25em plus 2pt minus 2pt}
% % \newcommand{\CG}{\mathcal{G}\xspace}
\newcommand{\CV}{\mathcal{V}\xspace}
\newcommand{\CE}{\mathcal{E}\xspace}
\newcommand{\CA}{\mathcal{A}\xspace}
\newcommand{\CF}{\mathcal{F}\xspace}
\newcommand{\CR}{\mathcal{R}\xspace}
\newcommand{\CB}{\mathcal{B}\xspace}
\newcommand{\CX}{\mathcal{X}\xspace}
\newcommand{\CK}{\mathcal{K}\xspace}
\newcommand{\CM}{\mathcal{M}\xspace}
\newcommand{\CC}{\mathcal{C}\xspace}
\newcommand{\CL}{\mathcal{L}\xspace}
\newcommand{\CI}{\mathcal{I}\xspace}
\newcommand{\CQ}{\mathcal{Q}\xspace}
\newcommand{\CO}{\mathcal{O}\xspace}
\newcommand{\CP}{\mathcal{P}\xspace}
\newcommand{\CS}{\mathcal{S}\xspace}
\newcommand{\CT}{\mathcal{T}\xspace}
\newcommand{\CJ}{\mathcal{J}\xspace}
\usepackage[para]{footmisc}
\usepackage{subfig}
% \usepackage{subcaption}
% \usepackage{array}
% \usepackage{colortbl}



% % It is strongly recommended to use hyperref, especially for the review version.
% % hyperref with option pagebackref eases the reviewers' job.
% % Please disable hyperref *only* if you encounter grave issues, 
% % e.g. with the file validation for the camera-ready version.
% %
% % If you comment hyperref and then uncomment it, you should delete *.aux before re-running LaTeX.
% % (Or just hit 'q' on the first LaTeX run, let it finish, and you should be clear).
% \definecolor{cvprblue}{rgb}{0.21,0.49,0.74}
% \usepackage[pagebackref,breaklinks,colorlinks,citecolor=cvprblue]{hyperref}
% \usepackage{tikz,lipsum}
% \usepackage[most]{tcolorbox}
% \usepackage[capitalize]{cleveref}
% \usepackage{multirow}
% \usepackage{listings}
% \usepackage{xcolor}
% \usepackage{minted}
% \usepackage{tabularx}


% \usepackage{booktabs} % For formal tables
% \usepackage[capitalize]{cleveref}
% \usepackage[dvipsnames]{xcolor}
% \usepackage{booktabs}
% % TOG prefers author-name bib system with square brackets
% \citestyle{acmauthoryear}

% \usepackage{diagbox}
% \usepackage{multirow}
% \usepackage{makecell}
% \usepackage{xhfill}

% % \lstset{backgroundcolor=\color{lightgray}}

% %%%%%%%%% PAPER ID  - PLEASE UPDATE
% \def\confName{CVPR}
% \def\confYear{2025}


% %%%%%%%%% AUTHORS - PLEASE UPDATE
% \author{\hspace{-1cm} Mia Tang$^{1}$ \hspace{-0.8cm} 
% \and Yael Vinker$^{2}$ \hspace{-0.8cm}
% \and Chuan Yan$^{1}$ \hspace{-0.8cm} 
% \and Lvmin Zhang$^{1}$ \hspace{-0.8cm}
% \and Maneesh Agrawala$^{1}$ \hspace{-1cm} 
% % \hspace{0.9\linewidth}
% \and \hspace{0.9\linewidth} \and
% $^{1}$Stanford University \\ {\tt\small $\{$miatang, chuanyan, lvmin, maneesh$\}$@stanford.edu }  \and $^{2}$MIT  \\ {\tt\small yaelvink@mit.edu} \vspace{-0.5cm}\\ \and {\tt\small \href{https://sketchseg.github.io/sketch-seg/}{https://sketchseg.github.io/sketch-seg/} } \vspace{-0.5cm}
% }

% \begin{document}
% \doparttoc % Tell to minitoc to generate a toc for the parts
% \faketableofcontents % Run a fake tableofcontents command for the partocs
% % Title portion
% \title{Instance Segmentation of Scene Sketches Using Natural Image Priors -- Supplementary Material}
% \maketitle



\section{Sketch Editing Interface}
\begin{figure}
    \centering
    \includegraphics[width=1\linewidth]{figs/interface.png}
    \caption{Interactive interface for sketch editing, powered by our instance segmentation and layer completion algorithm.}
    \label{fig:layers-inpaint}
\end{figure}

Our sketch segmentation and layering technique facilitates sketch editing, allowing users to drag or manipulate segmented objects without the need to manually sketch the affected regions. We demonstrate this through an interactive sketch editing interface (\Cref{fig:layers-inpaint}) that enables users to upload a sketch, which is then segmented and transformed into completed, ordered layers as detailed in our paper. This facilitates more efficient sketch editing by allowing artists to easily move, copy, or delete pixels associated with specific object instances, as the sketch is represented as an ordered list of layers. \textbf{Please see demo video for more examples.}


% We offer an interactive sketch editing interface that enables users to upload a sketch, which is then segmented and transformed into completed, ordered layers as detailed in our paper. This facilitates more efficient sketch editing by allowing artists to easily move, copy, or delete pixels associated with specific object instances, as the sketch is represented as an ordered list of layers.

\section{Synthetic Dataset}
In this section, we provide additional details on our synthetic data creation process. 

\subsection{Generating Vector Scene Sketches}
We employ CLIPasso \shortcite{vinker2022clipasso} and SketchAgent \shortcite{vinker2024sketchagent} to generate diverse vector sketches of single objects. For each generation method, we create 10 distinct object instances for all 45 classes in the SketchyScene dataset \shortcite{Zou18SketchyScene}, ensuring sufficient variability in our synthetic scenes.
For CLIPasso's image-to-vector conversion, we first generate photorealistic synthetic images using SDXL \shortcite{rombach2021highresolution}. The generation process uses a consistent prompt template: \textit{``A realistic image of a \{class\_name\} with a blank background"}. Figure~\ref{fig:clipasso_component} demonstrates representative pairs of synthetic input images and their corresponding generated vector sketches. 
For SketchAgent, we generate sketches directly from class labels as text prompts, producing 10 samples per class. Figure~\ref{fig:sketchagent_component} illustrates representative examples of the generated object sketches.
\begin{figure}
    \centering
    \setlength{\tabcolsep}{2pt}
    {\small
    \resizebox{0.9\linewidth}{!}{ 
    \begin{tabular}{c c  c c }
        Input Synthetic Image & Object Vector Sketch & Input Synthetic Image & Object Vector Sketch \\
        \frame{\includegraphics[width=0.32\linewidth]{figs_supp/dataset/cat_input.png}} &
        \frame{\includegraphics[width=0.32\linewidth]{figs_supp/dataset/cat_2_16strokes_seed0_best.png}} &
        \frame{\includegraphics[width=0.32\linewidth]{figs_supp/dataset/apple_input.png}} &
        \frame{\includegraphics[width=0.32\linewidth]{figs_supp/dataset/apple_0_16strokes_seed2000_best.png}} \\

        \frame{\includegraphics[width=0.32\linewidth]{figs_supp/dataset/duck_input.png}} &
        \frame{\includegraphics[width=0.32\linewidth]{figs_supp/dataset/duck_4_16strokes_seed0_best.png}} &
        \frame{\includegraphics[width=0.32\linewidth]{figs_supp/dataset/cloud_input.png}} &
        \frame{\includegraphics[width=0.32\linewidth]{figs_supp/dataset/cloud_3_16strokes_seed1000_best.png}} \\

        \frame{\includegraphics[width=0.32\linewidth]{figs_supp/dataset/bench_input.png}} &
        \frame{\includegraphics[width=0.32\linewidth]{figs_supp/dataset/bench_1_16strokes_seed2000_best.png}} &
        \frame{\includegraphics[width=0.32\linewidth]{figs_supp/dataset/bicycle_input.png}} &
        \frame{\includegraphics[width=0.32\linewidth]{figs_supp/dataset/bicycle_2_16strokes_seed2000_best.png}} \\
    \end{tabular}
    }}
    \vspace{-0.3cm}  \caption{Examples pairs of input synthetic image and output generated object vector sketch. }
    \label{fig:clipasso_component}
\end{figure}


\begin{figure}
    \centering
    \setlength{\tabcolsep}{2pt}
    {\small
    \resizebox{0.9\linewidth}{!}{ 
    \begin{tabular}{c c  c c }
        ``bus" & ``bird" & ``cat" & ``butterfly"\\ 
        \frame{\includegraphics[width=0.32\linewidth]{figs_supp/dataset/agent_bus.png}} &
        \frame{\includegraphics[width=0.32\linewidth]{figs_supp/dataset/agent_bird.png}} &
        \frame{\includegraphics[width=0.32\linewidth]{figs_supp/dataset/agent_cat.png}} &
        \frame{\includegraphics[width=0.32\linewidth]{figs_supp/dataset/agent_butterfly.png}} \\

        ``chair" & ``flower" & ``people" & ``house" \\
        \frame{\includegraphics[width=0.32\linewidth]{figs_supp/dataset/agent_chair.png}} &
        \frame{\includegraphics[width=0.32\linewidth]{figs_supp/dataset/agent_flower.png}} &
        \frame{\includegraphics[width=0.32\linewidth]{figs_supp/dataset/agent_person.png}} &
        \frame{\includegraphics[width=0.32\linewidth]{figs_supp/dataset/agent_house.png}} \\

        ``basket" & ``bench" & ``cup" & ``car" \\
        \frame{\includegraphics[width=0.32\linewidth]{figs_supp/dataset/agent_basket.png}} &
        \frame{\includegraphics[width=0.32\linewidth]{figs_supp/dataset/agent_bench.png}} &
        \frame{\includegraphics[width=0.32\linewidth]{figs_supp/dataset/agent_cup.png}} &
        \frame{\includegraphics[width=0.32\linewidth]{figs_supp/dataset/agent_car.png}} \\
    \end{tabular}
    }}
    \vspace{-0.3cm}  \caption{Examples of object vector sketches generated by SketchAgent.}
    \label{fig:sketchagent_component}
\end{figure}

\subsection{Generating Sketches from Natural Images}
To expand beyond SketchyScene's object categories, we employ InstantStyle \shortcite{Wang2024InstantStyleFL} to transform a subset of Visual Genome \shortcite{VisualGenome2017} images into sketches, using a single CLIPasso object sketch as the style reference. Figure~\ref{fig:instant_gallery} showcases a gallery of examples from our InstantStyle-generated scene sketch dataset.
 
\section{Additional Qualitative Comparisons}
We present additional qualitative comparisons between our approach and baseline methods across  benchmark scene sketch datasets in Fig.~\ref{fig:comparison_instance_sketchyscene} for SketchyScene, Fig.~\ref{fig:comparison_instance_sketchagent} for SketchAgent, Fig.~\ref{fig:comparison_instance_clipasso} for CLIPasso, Fig.~\ref{fig:comparison_instance_instantstyle} for InstantStyle, to accompany our numerical evaluations included in the paper. 
To compare with semantic segmentation method, namely OpenVocab, by Bourouis \etal \cite{bourouis2024open}, we created a filtered version of all seven datasets, where each scene contains at most one instance per object class. These filtered scenes remain challenging for existing methods despite their reduced complexity. We show qualitative results in Fig. ~\ref{fig:comparison_openvocab_sketchyscene} for filtered SketchyScene dataset, Fig. ~\ref{fig:comparison_openvocab_sketchagent} for filtered SketchAgent dataset, and Fig ~\ref{fig:comparison_openvocab_clipasso} for filtered CLIPasso dataset, to accompany our qualitative results shown in the paper. 

%%%% FIGS
%%%% dataset: InstantStyle


\begin{figure*}
    \centering
    \setlength{\tabcolsep}{2pt}
    {\small
    \resizebox{0.88\textwidth}{!}{ 
    \begin{tabular}{c c c}
        \frame{\includegraphics[trim=0 1.3cm 0 0.1cm,clip,width=0.32\linewidth]{figs_supp/dataset_scene/InstantStyle/17590.png}} &
        \frame{\includegraphics[trim=0 1.3cm 0 0,clip,width=0.32\linewidth]{figs_supp/dataset_scene/InstantStyle/31188.png}} &
        \frame{\includegraphics[trim=0 1.3cm 0 1.2cm,clip,width=0.32\linewidth]{figs_supp/dataset_scene/InstantStyle/33444.png}} \\

        \frame{\includegraphics[trim=0 3.2cm 0 0cm,clip,width=0.32\linewidth]{figs_supp/dataset_scene/InstantStyle/221223.png}} &
        \frame{\includegraphics[trim=0 1.3cm 0 0,clip,width=0.32\linewidth]{figs_supp/dataset_scene/InstantStyle/520657.png}} &
        \frame{\includegraphics[trim=0 1.3cm 0 1.2cm,clip,width=0.32\linewidth]{figs_supp/dataset_scene/InstantStyle/283809.png}} \\

        \frame{\includegraphics[trim=0 1.3cm 0 1.2cm,clip,width=0.32\linewidth]{figs_supp/dataset_scene/InstantStyle/291981.png}} &
        \frame{\includegraphics[trim=0 1.3cm 0 0,clip,width=0.32\linewidth]{figs_supp/dataset_scene/InstantStyle/400571.png}} &
        \frame{\includegraphics[trim=0 1.3cm 0 2cm,clip,width=0.32\linewidth]{figs_supp/dataset_scene/InstantStyle/528966.png}} \\

        \frame{\includegraphics[trim=0 0 0 2.4cm,clip,width=0.32\linewidth]{figs_supp/dataset_scene/InstantStyle/229598.png}} &
        \frame{\includegraphics[trim=0 0 0 0,clip,width=0.32\linewidth]{figs_supp/dataset_scene/InstantStyle/302657.png}} &
        \frame{\includegraphics[trim=0 1.3cm 0 1cm,clip,width=0.32\linewidth]{figs_supp/dataset_scene/InstantStyle/332270.png}} \\

        \frame{\includegraphics[trim=0 0 0 2.4cm,clip,width=0.32\linewidth]{figs_supp/dataset_scene/InstantStyle/407139.png}} &
        \frame{\includegraphics[trim=0 0 0 0.5cm,clip,width=0.32\linewidth]{figs_supp/dataset_scene/InstantStyle/409092.png}} &
        \frame{\includegraphics[trim=0 0.5cm 0 0,clip,width=0.32\linewidth]{figs_supp/dataset_scene/InstantStyle/410898.png}} \\

        \frame{\includegraphics[trim=0 0 0 0,clip,width=0.32\linewidth]{figs_supp/dataset_scene/InstantStyle/503500.png}} &
        \frame{\includegraphics[trim=0 0 0 0cm,clip,width=0.32\linewidth]{figs_supp/dataset_scene/InstantStyle/509864.png}} &
        \frame{\includegraphics[trim=0 0 0 0,clip,width=0.32\linewidth]{figs_supp/dataset_scene/InstantStyle/539694.png}} \\
    \end{tabular}
    }
    \vspace{-0.1cm}
    \caption{\textbf{Example sketches from our InstantStyle dataset}. These sketches are derived from Visual Genome ~\shortcite{VisualGenome2017} containing 5 to 10 annotated objects. For visual clarity, we mask unsegmented regions in the generated sketches. This dataset contains 54 new categories beyond SketchyScene's original 45 classes, and contain 1068 sketches in total. }
    \label{fig:instant_gallery}
    }
\end{figure*}


\begin{figure*}
    \centering
    \setlength{\tabcolsep}{2pt}
    {\small
    \resizebox{0.88\textwidth}{!}{ 
    \begin{tabular}{c @{\hskip 10pt}  @{\hskip 10pt} c c c}
        Input & SketchyScene & Grounded SAM & \textbf{Ours} \\
        
        \frame{\includegraphics[width=0.23\linewidth]{figs_supp/SketchyScene/L0_sample78.png}} &
        \frame{\includegraphics[width=0.23\linewidth]{figs_supp/SketchyScene/L0_sample78_SketchyScene.png}} &
        \frame{\includegraphics[width=0.23\linewidth]{figs_supp/SketchyScene/L0_sample78_grounded_SAM.png}} &
        \frame{\includegraphics[width=0.23\linewidth]{figs_supp/SketchyScene/L0_sample78_ours.png}} \\

        \frame{\includegraphics[width=0.23\linewidth]{figs_supp/SketchyScene/L0_sample14.png}} &
        \frame{\includegraphics[width=0.23\linewidth]{figs_supp/SketchyScene/L0_sample14_SketchyScene.png}} &
        \frame{\includegraphics[width=0.23\linewidth]{figs_supp/SketchyScene/L0_sample14_grounded_SAM.png}} &
        \frame{\includegraphics[width=0.23\linewidth]{figs_supp/SketchyScene/L0_sample14_ours.png}} \\

        \frame{\includegraphics[width=0.23\linewidth]{figs_supp/SketchyScene/L0_sample17.png}} &
        \frame{\includegraphics[width=0.23\linewidth]{figs_supp/SketchyScene/L0_sample17_SketchyScene.png}} &
        \frame{\includegraphics[width=0.23\linewidth]{figs_supp/SketchyScene/L0_sample17_grounded_SAM.png}} &
        \frame{\includegraphics[width=0.23\linewidth]{figs_supp/SketchyScene/L0_sample17_ours.png}} \\

        \frame{\includegraphics[width=0.23\linewidth]{figs_supp/SketchyScene/L0_sample52.png}} &
        \frame{\includegraphics[width=0.23\linewidth]{figs_supp/SketchyScene/L0_sample52_SketchyScene.png}} &
        \frame{\includegraphics[width=0.23\linewidth]{figs_supp/SketchyScene/L0_sample52_grounded_SAM.png}} &
        \frame{\includegraphics[width=0.23\linewidth]{figs_supp/SketchyScene/L0_sample52_ours.png}} \\

        \frame{\includegraphics[width=0.23\linewidth]{figs_supp/SketchyScene/L0_sample62.png}} &
        \frame{\includegraphics[width=0.23\linewidth]{figs_supp/SketchyScene/L0_sample62_SketchyScene.png}} &
        \frame{\includegraphics[width=0.23\linewidth]{figs_supp/SketchyScene/L0_sample62_grounded_SAM.png}} &
        \frame{\includegraphics[width=0.23\linewidth]{figs_supp/SketchyScene/L0_sample62_ours.png}} \\
    \end{tabular}
    }
    \vspace{-0.1cm}
    \caption{\textbf{Qualitative comparison of instance segmentation methods on the SketchyScene dataset.} Our method surpasses SketchyScene and Grounded SAM by delivering fine-grained, semantically consistent segmentations with precise boundaries. In the urban scene (row 1), our approach accurately segments all object instances, cleanly separating the sun, tree, building, and character from each other. For the park scenes (rows 2 and 3), it captures intricate details, such as the dog's face and picnic items, which are either missed or over-segmented by other methods. In the residential and cottage scenes (rows 4 and 5), our method effectively delineates repetitive patterns like fences and handles dense objects like trees, preserving structural integrity where other approaches struggle. These results highlight the robustness of our method in managing complex and detailed sketches.}
    \label{fig:comparison_instance_sketchyscene}
    }
\end{figure*}
% 
\begin{figure*}
    \centering
    \setlength{\tabcolsep}{2pt}
    {\small
    \resizebox{0.88\textwidth}{!}{ 
    \begin{tabular}{c @{\hskip 10pt} | @{\hskip 10pt} c c c}
        Input & SketchyScene & Grounded SAM & Ours \\
        
        \frame{\includegraphics[width=0.23\linewidth]{figs_supp/SketchyScene/L0_sample78.png}} &
        \frame{\includegraphics[width=0.23\linewidth]{figs_supp/SketchyScene/L0_sample78_SketchyScene.png}} &
        \frame{\includegraphics[width=0.23\linewidth]{figs_supp/SketchyScene/L0_sample78_grounded_SAM.png}} &
        \frame{\includegraphics[width=0.23\linewidth]{figs_supp/SketchyScene/L0_sample78_ours.png}} \\
\\
    \end{tabular}
    }
    \vspace{-0.1cm}
    \caption{\textbf{Qualitative comparison of instance segmentation methods on the SketchAgent dataset.} }
    \label{fig:comparison_instance_sketchagent}
    }
\end{figure*}

\begin{figure*}
    \centering
    \setlength{\tabcolsep}{2pt}
    {\small
    \resizebox{0.88\textwidth}{!}{ 
    \begin{tabular}{c @{\hskip 10pt}  @{\hskip 10pt} c c c}
        Input & SketchyScene & Grounded SAM & \textbf{Ours} \\
        
        \frame{\includegraphics[width=0.23\linewidth]{figs_supp/SketchAgent/1031.png}} &
        \frame{\includegraphics[width=0.23\linewidth]{figs_supp/SketchAgent/1031_SketchyScene.png}} &
        \frame{\includegraphics[width=0.23\linewidth]{figs_supp/SketchAgent/1031_grounded_SAM.png}} &
        \frame{\includegraphics[width=0.23\linewidth]{figs_supp/SketchAgent/1031_ours.png}} \\

        \frame{\includegraphics[width=0.23\linewidth]{figs_supp/SketchAgent/1099.png}} &
        \frame{\includegraphics[width=0.23\linewidth]{figs_supp/SketchAgent/1099_SketchyScene.png}} &
        \frame{\includegraphics[width=0.23\linewidth]{figs_supp/SketchAgent/1099_grounded_SAM.png}} &
        \frame{\includegraphics[width=0.23\linewidth]{figs_supp/SketchAgent/1099_ours.png}} \\

        \frame{\includegraphics[width=0.23\linewidth]{figs_supp/SketchAgent/1110.png}} &
        \frame{\includegraphics[width=0.23\linewidth]{figs_supp/SketchAgent/1110_SketchyScene.png}} &
        \frame{\includegraphics[width=0.23\linewidth]{figs_supp/SketchAgent/1110_grounded_SAM.png}} &
        \frame{\includegraphics[width=0.23\linewidth]{figs_supp/SketchAgent/1110_ours.png}} \\

        \frame{\includegraphics[width=0.23\linewidth]{figs_supp/SketchAgent/0618.png}} &
        \frame{\includegraphics[width=0.23\linewidth]{figs_supp/SketchAgent/0618_SketchyScene.png}} &
        \frame{\includegraphics[width=0.23\linewidth]{figs_supp/SketchAgent/0618_grounded_SAM.png}} &
        \frame{\includegraphics[width=0.23\linewidth]{figs_supp/SketchAgent/0618_ours.png}} \\

        \frame{\includegraphics[width=0.23\linewidth]{figs_supp/SketchAgent/0002.png}} &
        \frame{\includegraphics[width=0.23\linewidth]{figs_supp/SketchAgent/0002_SketchyScene.png}} &
        \frame{\includegraphics[width=0.23\linewidth]{figs_supp/SketchAgent/0002_grounded_SAM.png}} &
        \frame{\includegraphics[width=0.23\linewidth]{figs_supp/SketchAgent/0002_ours.png}} \\
    \end{tabular}
    }
    \vspace{-0.1cm}
    \caption{\textbf{Qualitative comparison of instance segmentation methods on the SketchAgent dataset.} Both SketchyScene and Grounded SAM struggle to segment these abstract sketches, which differ significantly from their training data of clipart-like and real objects. Our method successfully segments individual instances while maintaining object boundaries, even in challenging cases like the last row where objects overlap with a grid-patterned picnic blanket.}
    \label{fig:comparison_instance_sketchagent}
    }
\end{figure*}



\begin{figure*}
    \centering
    \setlength{\tabcolsep}{2pt}
    {\small
    \resizebox{0.88\textwidth}{!}{ 
    \begin{tabular}{c @{\hskip 10pt} @{\hskip 10pt} c c c}
        Input & SketchyScene & Grounded SAM & \textbf{Ours}  \\
        \frame{\includegraphics[width=0.23\linewidth]{figs_supp/CLIPasso_base/0051.png}} &
        \frame{\includegraphics[width=0.23\linewidth]{figs_supp/CLIPasso_base/0051_grounded_SAM.png}} &
        \frame{\includegraphics[width=0.23\linewidth]{figs_supp/CLIPasso_base/0051_SketchyScene.png}} &
        \frame{\includegraphics[width=0.23\linewidth]{figs_supp/CLIPasso_base/0051_ours.png}} \\

        \frame{\includegraphics[width=0.23\linewidth]{figs_supp/CLIPasso_base/0078.png}} &
        \frame{\includegraphics[width=0.23\linewidth]{figs_supp/CLIPasso_base/0078_grounded_SAM.png}} &
        \frame{\includegraphics[width=0.23\linewidth]{figs_supp/CLIPasso_base/0078_SketchyScene.png}} &
        \frame{\includegraphics[width=0.23\linewidth]{figs_supp/CLIPasso_base/0078_ours.png}} \\
         
        \frame{\includegraphics[width=0.23\linewidth]{figs_supp/CLIPasso_base/0047.png}} &
        \frame{\includegraphics[width=0.23\linewidth]{figs_supp/CLIPasso_base/0047_grounded_SAM.png}} &
        \frame{\includegraphics[width=0.23\linewidth]{figs_supp/CLIPasso_base/0047_SketchyScene.png}} &
        \frame{\includegraphics[width=0.23\linewidth]{figs_supp/CLIPasso_base/0047_ours.png}} \\

        \frame{\includegraphics[width=0.23\linewidth]{figs_supp/CLIPasso_base/0158.png}} &
        \frame{\includegraphics[width=0.23\linewidth]{figs_supp/CLIPasso_base/0158_grounded_SAM.png}} &
        \frame{\includegraphics[width=0.23\linewidth]{figs_supp/CLIPasso_base/0158_SketchyScene.png}} &
        \frame{\includegraphics[width=0.23\linewidth]{figs_supp/CLIPasso_base/0158_ours.png}} \\

        \frame{\includegraphics[width=0.23\linewidth]{figs_supp/CLIPasso_base/1104.png}} &
        \frame{\includegraphics[width=0.23\linewidth]{figs_supp/CLIPasso_base/1104_grounded_SAM.png}} &
        \frame{\includegraphics[width=0.23\linewidth]{figs_supp/CLIPasso_base/1104_SketchyScene.png}} &
        \frame{\includegraphics[width=0.23\linewidth]{figs_supp/CLIPasso_base/1104_ours.png}} \\
    \end{tabular}
    }
    \vspace{-0.1cm}
    \caption{\textbf{Qualitative comparison of instance segmentation methods on the CLIPasso base dataset.} Our method successfully detects both large and small objects with ambiguous openings in their silhouettes, whereas baseline methods either merge multiple instances into one or fail to detect them entirely.}
    \label{fig:comparison_instance_clipasso}
    }
\end{figure*}


\begin{figure*}
    \centering
    \setlength{\tabcolsep}{2pt}
    {\small
    \resizebox{0.88\textwidth}{!}{ 
    \begin{tabular}{c @{\hskip 10pt}  @{\hskip 10pt} c c c}
        Input & SketchyScene & Grounded SAM & \textbf{Ours}  \\
        \frame{\includegraphics[width=0.23\linewidth]{figs_supp/InstantStyle/36012.png}} &
        \frame{\includegraphics[width=0.23\linewidth]{figs_supp/InstantStyle/36012_SketchyScene.png}} &
        \frame{\includegraphics[width=0.23\linewidth]{figs_supp/InstantStyle/36012_grounded_SAM.png}} &
        \frame{\includegraphics[width=0.23\linewidth]{figs_supp/InstantStyle/36012_ours.png}} \\

       

        \frame{\includegraphics[width=0.23\linewidth]{figs_supp/InstantStyle/86750.png}} &
        \frame{\includegraphics[width=0.23\linewidth]{figs_supp/InstantStyle/86750_SketchyScene.png}} &
        \frame{\includegraphics[width=0.23\linewidth]{figs_supp/InstantStyle/86750_grounded_SAM.png}} &
        \frame{\includegraphics[width=0.23\linewidth]{figs_supp/InstantStyle/86750_ours.png}} \\

        \frame{\includegraphics[width=0.23\linewidth]{figs_supp/InstantStyle/162285.png}} &
        \frame{\includegraphics[width=0.23\linewidth]{figs_supp/InstantStyle/162285_SketchyScene.png}} &
        \frame{\includegraphics[width=0.23\linewidth]{figs_supp/InstantStyle/162285_grounded_SAM.png}} &
        \frame{\includegraphics[width=0.23\linewidth]{figs_supp/InstantStyle/162285_ours.png}} \\

        
        \frame{\includegraphics[width=0.23\linewidth]{figs_supp/InstantStyle/193547.png}} &
        \frame{\includegraphics[width=0.23\linewidth]{figs_supp/InstantStyle/193547_SketchyScene.png}} &
        \frame{\includegraphics[width=0.23\linewidth]{figs_supp/InstantStyle/193547_grounded_SAM.png}} &
        \frame{\includegraphics[width=0.23\linewidth]{figs_supp/InstantStyle/193547_ours.png}} \\

        \frame{\includegraphics[width=0.23\linewidth]{figs_supp/InstantStyle/574350.png}} &
        \frame{\includegraphics[width=0.23\linewidth]{figs_supp/InstantStyle/574350_SketchyScene.png}} &
        \frame{\includegraphics[width=0.23\linewidth]{figs_supp/InstantStyle/574350_grounded_SAM.png}} &
        \frame{\includegraphics[width=0.23\linewidth]{figs_supp/InstantStyle/574350_ours.png}} \\

        
        \frame{\includegraphics[width=0.23\linewidth]{figs_supp/InstantStyle/576973.png}} &
        \frame{\includegraphics[width=0.23\linewidth]{figs_supp/InstantStyle/576973_SketchyScene.png}} &
        \frame{\includegraphics[width=0.23\linewidth]{figs_supp/InstantStyle/576973_grounded_SAM.png}} &
        \frame{\includegraphics[width=0.23\linewidth]{figs_supp/InstantStyle/576973_ours.png}} \\

        \frame{\includegraphics[width=0.23\linewidth]{figs_supp/InstantStyle/176038.png}} &
        \frame{\includegraphics[width=0.23\linewidth]{figs_supp/InstantStyle/176038_SketchyScene.png}} &
        \frame{\includegraphics[width=0.23\linewidth]{figs_supp/InstantStyle/176038_grounded_SAM.png}} &
        \frame{\includegraphics[width=0.23\linewidth]{figs_supp/InstantStyle/176038_ours.png}} \\
    \end{tabular}
    }
    \vspace{-0.1cm}
    \caption{\textbf{Qualitative comparison of instance segmentation methods on the InstantStyle dataset.} This dataset poses a significantly greater challenge compared to the others due to its increased complexity, including diverse perspectives, intricate textures, and frequent occlusions. Despite these difficulties, our method effectively locates object instances, such as the glass of water in row 2, the fork in row 3, and the food in dishes and bottles in row 4. Even with ambiguous shapes, as shown in row 5, our method outperforms GroundedSAM by successfully segmenting the umbrella on the right separately from the person holding it. In the final row, our approach demonstrates its capability by accurately segmenting both the pile of clothes on the couch and the couch itself.}
    \label{fig:comparison_instance_instantstyle}
    }
\end{figure*}


\begin{figure*}
    \centering
    \setlength{\tabcolsep}{2pt}
    {\small
    \resizebox{0.7\textwidth}{!}{ 
    \begin{tabular}{c @{\hskip 10pt}  @{\hskip 10pt} c c}
        Input & OpenVocab & \textbf{Ours}  \\
        \frame{\includegraphics[width=0.23\linewidth]{figs_supp/OpenVocab/SketchyScene/L0_sample3.png}} &
        \frame{\includegraphics[width=0.23\linewidth]{figs_supp/OpenVocab/SketchyScene/L0_sample3_OpenVocab.png}} &
        \frame{\includegraphics[width=0.23\linewidth]{figs_supp/OpenVocab/SketchyScene/L0_sample3_ours.png}} \\

        \frame{\includegraphics[width=0.23\linewidth]{figs_supp/OpenVocab/SketchyScene/L0_sample8.png}} &
        \frame{\includegraphics[width=0.23\linewidth]{figs_supp/OpenVocab/SketchyScene/L0_sample8_OpenVocab.png}} &
        \frame{\includegraphics[width=0.23\linewidth]{figs_supp/OpenVocab/SketchyScene/L0_sample8_ours.png}} \\

        \frame{\includegraphics[width=0.23\linewidth]{figs_supp/OpenVocab/SketchyScene/L0_sample14.png}} &
        \frame{\includegraphics[width=0.23\linewidth]{figs_supp/OpenVocab/SketchyScene/L0_sample14_OpenVocab.png}} &
        \frame{\includegraphics[width=0.23\linewidth]{figs_supp/OpenVocab/SketchyScene/L0_sample14_ours.png}} \\

        \frame{\includegraphics[width=0.23\linewidth]{figs_supp/OpenVocab/SketchyScene/L0_sample491.png}} &
        \frame{\includegraphics[width=0.23\linewidth]{figs_supp/OpenVocab/SketchyScene/L0_sample491_OpenVocab.png}} &
        \frame{\includegraphics[width=0.23\linewidth]{figs_supp/OpenVocab/SketchyScene/L0_sample491_ours.png}} \\

        \frame{\includegraphics[width=0.23\linewidth]{figs_supp/OpenVocab/SketchyScene/L0_sample737.png}} &
        \frame{\includegraphics[width=0.23\linewidth]{figs_supp/OpenVocab/SketchyScene/L0_sample737_OpenVocab.png}} &
        \frame{\includegraphics[width=0.23\linewidth]{figs_supp/OpenVocab/SketchyScene/L0_sample737_ours.png}} \\
    \end{tabular}
    }
    \vspace{-0.1cm}
    \caption{\textbf{Qualitative comparison of segmentation on filtered SketchyScene dataset. } We prompt the semantic segmentation model with ground truth class labels; however, it struggles to locate the objects due to the significant deviation of this sketch style from the data it was trained on. Our performance on both complete and filtered scenes are equally as robust, detecting and segmenting object instances precisely. }
    \label{fig:comparison_openvocab_sketchyscene}
    }
\end{figure*}


\begin{figure*}
    \centering
    \setlength{\tabcolsep}{2pt}
    {\small
    \resizebox{0.7\textwidth}{!}{ 
    \begin{tabular}{c @{\hskip 10pt}  @{\hskip 10pt} c c}
        Input & OpenVocab & \textbf{Ours}  \\
        \frame{\includegraphics[width=0.23\linewidth]{figs_supp/OpenVocab/SketchAgent/0749.png}} &
        \frame{\includegraphics[width=0.23\linewidth]{figs_supp/OpenVocab/SketchAgent/0749_OpenVocab.png}} &
        \frame{\includegraphics[width=0.23\linewidth]{figs_supp/OpenVocab/SketchAgent/0749_ours.png}}\\

        \frame{\includegraphics[width=0.23\linewidth]{figs_supp/OpenVocab/SketchAgent/0032.png}} &
        \frame{\includegraphics[width=0.23\linewidth]{figs_supp/OpenVocab/SketchAgent/0032_OpenVocab.png}} &
        \frame{\includegraphics[width=0.23\linewidth]{figs_supp/OpenVocab/SketchAgent/0032_ours.png}}\\

        \frame{\includegraphics[width=0.23\linewidth]{figs_supp/OpenVocab/SketchAgent/0035.png}} &
        \frame{\includegraphics[width=0.23\linewidth]{figs_supp/OpenVocab/SketchAgent/0035_OpenVocab.png}} &
        \frame{\includegraphics[width=0.23\linewidth]{figs_supp/OpenVocab/SketchAgent/0035_ours.png}}\\

        \frame{\includegraphics[width=0.23\linewidth]{figs_supp/OpenVocab/SketchAgent/0565.png}} &
        \frame{\includegraphics[width=0.23\linewidth]{figs_supp/OpenVocab/SketchAgent/0565_OpenVocab.png}} &
        \frame{\includegraphics[width=0.23\linewidth]{figs_supp/OpenVocab/SketchAgent/0565_ours.png}}\\

        
        \frame{\includegraphics[width=0.23\linewidth]{figs_supp/OpenVocab/SketchAgent/0017.png}} &
        \frame{\includegraphics[width=0.23\linewidth]{figs_supp/OpenVocab/SketchAgent/0017_OpenVocab.png}} &
        \frame{\includegraphics[width=0.23\linewidth]{figs_supp/OpenVocab/SketchAgent/0017_ours.png}}\\
    \end{tabular}
    }
    \vspace{-0.1cm}
    \caption{\textbf{Qualitative comparison of segmentation on filtered SketchAgent dataset. } We prompt the semantic segmentation model with ground truth class labels; however, OpenVocab struggles to accurately identify instances of the correct class, often assigning multiple class labels to the same object, as indicated by the color gradients. Sometimes it is unable to label any sketch pixels in the scene (row 5). In contrast, our method effectively segments object instances, ensuring clear separation and consistent labeling.}
    \label{fig:comparison_openvocab_sketchagent}
    }
\end{figure*}



\begin{figure*}
    \centering
    \setlength{\tabcolsep}{2pt}
    {\small
    \resizebox{0.7\textwidth}{!}{ 
    \begin{tabular}{c @{\hskip 10pt}  @{\hskip 10pt} c c}
        Input & OpenVocab & \textbf{Ours}  \\
        \frame{\includegraphics[width=0.23\linewidth]{figs_supp/OpenVocab/CLIPasso_base/0303.png}} &
        \frame{\includegraphics[width=0.23\linewidth]{figs_supp/OpenVocab/CLIPasso_base/0303_OpenVocab.png}} &
        \frame{\includegraphics[width=0.23\linewidth]{figs_supp/OpenVocab/CLIPasso_base/0303_ours.png}}\\

        \frame{\includegraphics[width=0.23\linewidth]{figs_supp/OpenVocab/CLIPasso_base/0500.png}} &
        \frame{\includegraphics[width=0.23\linewidth]{figs_supp/OpenVocab/CLIPasso_base/0500_OpenVocab.png}} &
        \frame{\includegraphics[width=0.23\linewidth]{figs_supp/OpenVocab/CLIPasso_base/0500_ours.png}}\\

        \frame{\includegraphics[width=0.23\linewidth]{figs_supp/OpenVocab/CLIPasso_base/0505.png}} &
        \frame{\includegraphics[width=0.23\linewidth]{figs_supp/OpenVocab/CLIPasso_base/0505_OpenVocab.png}} &
        \frame{\includegraphics[width=0.23\linewidth]{figs_supp/OpenVocab/CLIPasso_base/0505_ours.png}}\\

        \frame{\includegraphics[width=0.23\linewidth]{figs_supp/OpenVocab/CLIPasso_base/0649.png}}&
        \frame{\includegraphics[width=0.23\linewidth]{figs_supp/OpenVocab/CLIPasso_base/0649_OpenVocab.png}}&
        \frame{\includegraphics[width=0.23\linewidth]{figs_supp/OpenVocab/CLIPasso_base/0649_ours.png}}\\
        
        \frame{\includegraphics[width=0.23\linewidth]{figs_supp/OpenVocab/CLIPasso_base/0688.png}}&
        \frame{\includegraphics[width=0.23\linewidth]{figs_supp/OpenVocab/CLIPasso_base/0688_OpenVocab.png}} &
        \frame{\includegraphics[width=0.23\linewidth]{figs_supp/OpenVocab/CLIPasso_base/0688_ours.png}}\\
    \end{tabular}
    }
    \vspace{-0.1cm}
    \caption{\textbf{Qualitative comparison of segmentation on filtered CLIPasso dataset. } We prompt the semantic segmentation model with ground truth class labels; however, OpenVocab struggles to accurately identify instances of the correct class, often assigning multiple class labels to the same object, as indicated by the color gradients. Sometimes it is unable to label any sketch pixels in the scene (row 4). In contrast, our method effectively segments object instances, ensuring clear separation and consistent labeling.}
    \label{fig:comparison_openvocab_clipasso}
    }
\end{figure*}

% \clearpage
% \bibliographystyle{ACM-Reference-Format}
% \bibliography{main}
% \end{document}


\begin{figure*}
    \centering
    \setlength{\tabcolsep}{3pt}
    \renewcommand{\arraystretch}{2}
    % \addtolength{\belowcaptionskip}{-7.5pt}
    {\small
    \begin{tabular}{c c c c c c c}
        \frame{\includegraphics[width=0.13\linewidth]{figs/ours/SketchyScene/L0_sample1066.png}} &
        \frame{\includegraphics[width=0.13\linewidth]{figs/ours/SketchyScene/L0_sample1084.png}} &
        \frame{\includegraphics[width=0.13\linewidth]{figs/ours/SketchyScene/L0_sample1086.png}} &
        \frame{\includegraphics[width=0.13\linewidth]{figs/ours/SketchyScene/L0_sample1055.png}} &
        \frame{\includegraphics[width=0.13\linewidth]{figs/ours/SketchyScene/L0_sample1069.png}} &
        \frame{\includegraphics[width=0.13\linewidth]{figs/ours/SketchyScene/L0_sample1056.png}} &
        \frame{\includegraphics[width=0.13\linewidth]{figs/ours/SketchyScene/L0_sample1013.png}} \\
        
        \frame{\includegraphics[width=0.13\linewidth]{figs/ours/CLIPasso_3/0012.png}} &
        \frame{\includegraphics[width=0.13\linewidth]{figs/ours/CLIPasso_3/1017.png}} &
        \frame{\includegraphics[width=0.13\linewidth]{figs/ours/CLIPasso_3/0679.png}} &
        \frame{\includegraphics[width=0.13\linewidth]{figs/ours/CLIPasso_3/0121.png}} &
        \frame{\includegraphics[width=0.13\linewidth]{figs/ours/CLIPasso_3/0428.png}} &
        \frame{\includegraphics[width=0.13\linewidth]{figs/ours/CLIPasso_3/0018.png}} &
        \frame{\includegraphics[width=0.13\linewidth]{figs/ours/CLIPasso_3/0102.png}} \\

        \frame{\includegraphics[width=0.13\linewidth]{figs/ours/CLIPasso_1/0044.png}} &
        \frame{\includegraphics[width=0.13\linewidth]{figs/ours/CLIPasso_1/0863.png}} &
        \frame{\includegraphics[width=0.13\linewidth]{figs/ours/CLIPasso_1/0755.png}} &
        \frame{\includegraphics[width=0.13\linewidth]{figs/ours/CLIPasso_1/0726.png}} &
        \frame{\includegraphics[width=0.13\linewidth]{figs/ours/CLIPasso_1/0137.png}} &
        \frame{\includegraphics[width=0.13\linewidth]{figs/ours/CLIPasso_1/0087.png}} &
        \frame{\includegraphics[width=0.13\linewidth]{figs/ours/CLIPasso_1/0174.png}} \\

        
        \frame{\includegraphics[width=0.13\linewidth]{figs/ours/SketchAgent/1017.png}} &
        \frame{\includegraphics[width=0.13\linewidth]{figs/ours/SketchAgent/0501.png}} &
        \frame{\includegraphics[width=0.13\linewidth]{figs/ours/SketchAgent/0812.png}} &
        \frame{\includegraphics[width=0.13\linewidth]{figs/ours/SketchAgent/0595.png}} &
        \frame{\includegraphics[width=0.13\linewidth]{figs/ours/SketchAgent/1066.png}} &
        \frame{\includegraphics[width=0.13\linewidth]{figs/ours/SketchAgent/1062.png}} &
        \frame{\includegraphics[width=0.13\linewidth]{figs/ours/SketchAgent/0876.png}} \\
        
    \\     
    \end{tabular}
    }
    \caption{ours on SketchyScene inputs (first row), clipasso two styles (second row), sketchagent(last row)}
    \label{fig:qualitative3}
\end{figure*}



% \clearpage
% \bibliographystyle{ACM-Reference-Format}
% \bibliography{main}

% \end{document}
