\begin{abstract}
%RAG
Retrieval-Augmented Generation (RAG) enhances the performance of LLMs across various tasks by retrieving relevant information from external sources, particularly on text-based data. 
For structured data, such as knowledge graphs, GraphRAG has been widely used to retrieve relevant information. However, recent studies have revealed that structuring implicit knowledge from text into graphs can benefit certain tasks, extending the application of GraphRAG from graph data to general text-based data. 
Despite their successful extensions, most applications of GraphRAG for text data have been designed for specific tasks and datasets, lacking a systematic evaluation and comparison between RAG and GraphRAG on widely used text-based benchmarks.
In this paper, we systematically evaluate RAG and GraphRAG on well-established benchmark tasks, such as Question Answering and Query-based Summarization. 
Our results highlight the distinct strengths of RAG and GraphRAG across different tasks and evaluation perspectives. 
Inspired by these observations, we investigate strategies to 
integrate their strengths to improve downstream tasks. Additionally, we provide an in-depth discussion of the shortcomings of current GraphRAG approaches and outline directions for future research.


\end{abstract}