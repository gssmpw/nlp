

\typeout{IJCAI--25 Instructions for Authors}


\documentclass{article}
\pdfpagewidth=8.5in
\pdfpageheight=11in

\usepackage{ijcai25}
\usepackage{amssymb}
\usepackage{multirow}
\usepackage{color}
\usepackage[table,xcdraw]{xcolor}
\usepackage{graphicx}
\usepackage{booktabs}  %
\usepackage{xcolor} 
\usepackage{array} 

\usepackage{multirow}
\usepackage{times}
\usepackage{soul}
\usepackage{url}
\usepackage[hidelinks]{hyperref}
\usepackage[utf8]{inputenc}
\usepackage[small]{caption}
\usepackage{graphicx}
\usepackage{amsmath}
\usepackage{amsthm}
\usepackage{booktabs}
\usepackage{algorithm}
\usepackage{algorithmic}
\usepackage[switch]{lineno}


\urlstyle{same}


\newtheorem{example}{Example}
\newtheorem{theorem}{Theorem}






\pdfinfo{
/TemplateVersion (IJCAI.2025.0)
}

\title{PedDet: Adaptive Spectral Optimization for Multimodal Pedestrian Detection}


\author{
Rui Zhao$^{1}$\thanks{Equal contribution. $^\dag$Project lead. $^\ddag$Corresponding author (y.zhao2@latrobe.edu.au).},~ 
Zeyu Zhang$^{23*\dag}$,~ %
Yi Xu$^4$,~ %
Yi Yao$^5$,~ %
Yan Huang$^6$,\\
Wenxin Zhang$^7$,~
Zirui Song$^{68}$,~
Xiuying Chen$^8$,~
Yang Zhao$^{3\ddag}$\\
\affiliations
$^1$JD.com~
$^2$The Australian National University~
$^3$La Trobe University\\
$^4$Central South University~
$^5$NavInfo Co., Ltd.~
$^6$University of Technology Sydney~\\
$^7$University of Chinese Academy of Science~
$^8$Mohamed bin Zayed University of Artificial Intelligence
}




\begin{document}

\maketitle

\begin{abstract}
Pedestrian detection in intelligent transportation systems has made significant progress but faces two critical challenges: (1) insufficient fusion of complementary information between visible and infrared spectra, particularly in complex scenarios, and (2) sensitivity to illumination changes, such as low-light or overexposed conditions, leading to degraded performance. To address these issues, we propose \textbf{PedDet}, an adaptive spectral optimization complementarity framework which specifically enhanced and optimized for multispectral pedestrian detection. PedDet introduces the Multi-scale Spectral Feature Perception Module \textbf{(MSFPM)} to adaptively fuse visible and infrared features, enhancing robustness and flexibility in feature extraction. Additionally, the Illumination Robustness Feature Decoupling Module \textbf{(IRFDM)} improves detection stability under varying lighting by decoupling pedestrian and background features. We further design a contrastive alignment to enhance intermodal feature discrimination. Experiments on \textbf{LLVIP} and \textbf{MSDS }datasets demonstrate that PedDet achieves state-of-the-art performance, improving the mAP by 6.6 \% with superior detection accuracy even in low-light conditions, marking a significant step forward for road safety. 
Code will be available at \url{https://github.com/AIGeeksGroup/PedDet}.
\end{abstract}

\section{Introduction}
Pedestrian detection, a critical component in ensuring road safety, has garnered significant attention from both academia and industry. With the advancement of intelligent transportation systems, pedestrian detection plays an essential role in scenarios such as autonomous driving and public transportation monitoring. However, the complexity of road environments presents significant challenges to pedestrian detection, including issues like occlusion, diverse target sizes, and varying illumination conditions. Addressing these challenges necessitates more precise and robust detection algorithms to enhance traffic safety.


\begin{figure}[t]
	\centering
	\includegraphics[width=0.99\linewidth]{fig1.pdf}
	\caption{Modules A and B illustrate the distinctions between the existing pedestrian detection frameworks and our proposed approach. Methods in Module A identify and locate targets only by fusing features from multiple levels. Module B shows that our method takes advantage of the Multi-scale Spectral Feature Perception Module (MSFPM) and the Illumination Robustness Feature Decoupling Module (IRFDM) to improve detection performance}
	\label{fig:first}
\end{figure}

\begin{figure*}[t]
	\centering
	\includegraphics[width=\textwidth]{fig2.pdf}
	\caption{The overall pipeline of PedDet.}
	\label{fig:main}
\end{figure*}

Despite progress in pedestrian detection, current methods face two major challenges. First, existing algorithms struggle to fully exploit the complementary information between RGB and infrared spectra, resulting in suboptimal multimodal feature fusion. In complex scenarios, single-modal pedestrian detection often exhibits instability under unfavorable illumination conditions. Second, traditional methods \cite{1,2,3} demonstrate a high sensitivity to environmental changes, particularly under low-light or overexposed conditions. This sensitivity hampers the models' ability to extract reliable pedestrian features, leading to a significant decline in detection performance. These challenges hinder the widespread real-world deployment of existing pedestrian detection models.

To address these challenges, we propose PedDet, an optimization framework tailored for all object detection models. 
PedDet is specifically designed to optimize pedestrian detection by leveraging the complementary information between RGB and infrared, thereby improving both robustness and accuracy in diverse lighting scenarios, as shown in Figure \ref{fig:first}. 

First, we introduce the Multi-scale Spectral Feature Perception Module (MSFPM), which processes RGB and infrared spectral data in parallel. MSFPM adaptively extracts and fuses features from multiple spectral scales, dynamically adjusting the feature weights of each modality based on illumination conditions. This enables the model to maximize spectral complementarity and maintain robustness across varying environmental complexities. In contrast to traditional single-modal approaches, MSFPM effectively integrates information from both spectra, significantly enhancing detection stability and accuracy.

Additionally, we design the Illumination Robustness Feature Decoupling Module (IRFDM) to address the adverse effects of extreme lighting conditions on detection performance. IRFDM focuses on decoupling pedestrian-specific features from background noise, thereby mitigating interference from complex environments. Traditional pedestrian detection methods often struggle to differentiate pedestrians from the background under extreme conditions, such as strong light, low light, or uneven illumination, where the accuracy significantly deteriorates due to occlusion, shadows, or light reflections. By incorporating IRFDM, our model effectively separates illumination-induced disturbances during feature extraction, ensuring high detection accuracy even under extreme lighting conditions. IRFDM learns robust feature representations across varying illumination scenarios, reducing the negative impacts of environmental lighting variations.

To further improve feature discrimination, we adopt a contrastive learning paradigm. This strategy compares the differences between pedestrian and background features across visible and infrared spectra, strengthening the discrimination between pedestrian and background features and improving the model’s performance in complex scenarios. Comprehensive experiments demonstrate that PedDet achieves state-of-the-art performance, significantly outperforming existing methods in accuracy and robustness under diverse and challenging conditions.

The primary contributions of this work are summarized as follows:
\begin{itemize}
	\item We propose \textbf{PedDet}, which uses a Multi-scale Spectral Feature Perception Module (\textbf{MSFPM}) to fuse visible and infrared features, adjusting weights based on lighting conditions for improved pedestrian detection and accuracy compared to single-modal methods.
	\item To address illumination challenges, we design the Illumination Robustness Feature Decoupling Module (\textbf{IRFDM}), which isolates pedestrian features from background noise, enhancing robustness across different lighting conditions.
	\item A contrastive learning strategy is integrated to improve feature discrimination, helping the model better distinguish pedestrians from background in both visible and infrared modalities, further enhancing robustness in complex environments.
\end{itemize}


\section{Related Works}
Pedestrian detection in intelligent transportation systems has made some strides, but existing studies continue to face two primary challenges: \textit{multimodal feature fusion} and \textit{illumination robustness}. While various methods have been proposed to address these issues, they still exist limitations in real-world scenarios. For related works, see \textbf{Supplementary Section 1: Related Works}.



\section{Methodology}
\subsection{Overview}
In this section, we present the proposed PedDet in detail. As illustrated in Figure \ref{fig:main}, PedDet consists of four key components: an MSFPM, an IRFDM, a contrastive learning paradigm, and a detection head. The following subsections provide an in-depth explanation of each component, concluding with a description of the model’s optimization objectives. 


\subsection{Feature Extractor}
We adopt the improved YOLOv10 \cite{4} as the backbone network, as illustrated in Figure \ref{fig:extactor}. YOLOv10 introduces systematic optimizations at multiple levels, significantly enhancing computational efficiency and feature extraction capability. By refining the backbone architecture, YOLOv10 reduces computational complexity and memory consumption while maintaining high model accuracy, making it particularly suitable for real-time applications with stringent latency requirements.




YOLOv10 serves as the backbone for feature extraction, processing RGB and infrared images through parallel branches, each generating modality-specific feature matrices. The RGB branch captures rich color details in daylight, while the infrared branch enhances object contours in low-light conditions. Feature extraction operates across multiple scales, producing feature maps at:  

\begin{equation}
F_1 \in \mathbb{R}^{80 \times 80 \times 256}, \quad  
F_2 \in \mathbb{R}^{40 \times 40 \times 512}, \quad  
F_3 \in \mathbb{R}^{20 \times 20 \times 512}
\end{equation}

where \( F_i \) represents feature maps at different scales. This multiscale approach ensures robust pedestrian detection across varying sizes and perspectives. The final feature representation \( F_{\text{final}} \) is obtained via fusion:  

\begin{equation}
F_{\text{final}} = \mathbf{w}^T \cdot \mathbf{F}
\end{equation}

where \( w_i \) are learnable weights, balancing contributions from different scales to enhance adaptability across detection scenarios.  

\begin{figure}
	\centering
	\includegraphics[width=\linewidth]{fig3.pdf}
	\caption{The Architecture of Feature Extractor.}
	\label{fig:extactor}
\end{figure}

\subsection{Multi-Scale Spectral Feature Perception Module (MSFPM)}
The MSFPM plays a critical role in addressing the challenges of pedestrian detection under complex illumination conditions. Specifically designed to leverage the complementary information between RGB and infrared spectra, MSFPM effectively overcomes the limitations of existing methods in utilizing multimodal advantages in challenging scenarios.
The core of multimodal feature fusion is to effectively integrate RGB and infrared features while maximizing their complementary strengths across diverse environments. MSFPM achieves this by stacking features along the depth dimension and dynamically adjusting their contribution weights:

\begin{equation}
F_{\text{fused}} = w_{\text{RGB}} F_{\text{RGB}} + w_{\text{IR}} F_{\text{IR}}
\end{equation}

where \( w_{\text{RGB}} \) and \( w_{\text{IR}} \) are adaptive weights that prioritize RGB features in well-lit conditions and infrared features in low-light scenarios, ensuring robustness.

After fusion, MSFPM applies dimensionality reduction to enhance computational efficiency while preserving semantic richness:

\begin{equation}
F_{\text{reduced}} = W_{\text{reduce}} F_{\text{fused}}
\end{equation}

where \( W_{\text{reduce}} \) is a learnable projection matrix that compresses feature dimensions while maintaining cross-modal synergy. The SiLU activation function is then applied to improve non-linear representation and robustness:

\begin{equation}
F_{\text{activated}} = F_{\text{reduced}} \cdot \sigma(F_{\text{reduced}})
\end{equation}

where \( \sigma(x) = x \cdot \text{sigmoid}(x) \). Finally, the fused features undergo multi-scale convolutional processing for pedestrian detection, leveraging modality complementarity to ensure accurate results under varying illumination conditions.




\subsection{Illumination Robustness Feature Decoupling Module}
Feature decoupling \cite{34} aims to decompose latent representations into independent factors, reducing redundancy and enhancing feature clarity. To improve detection accuracy, we propose the Illumination Robustness Feature Decoupling Module (IRFDM), which separates visible light image features into human-related and background-related components. This minimizes background interference, refining pedestrian detection.  

As shown in the middle of Figure \ref{fig:main}, IRFDM employs two parallel processing paths to handle visible (\( f_v \)) and infrared (\( f_i \)) features separately, ensuring robust feature extraction under varying illumination conditions.



\paragraph{Process of Visible Light Images ($f_v$).}
The IRFDM process enhances feature separation by sequentially applying pooling, convolution, and activation functions. Max Pooling highlights key human-related features, while Average Pooling captures background details. Then, \( 1 \times 1 \) convolutions refine feature representations:

\begin{equation}
F' = W_{1\times1} * F
\end{equation}

where \( W_{1\times1} \) is the learnable weight matrix. Convolution and pooling layers further reduce feature dimensionality while improving stability.

Finally, a three-layer Multi-Layer Perceptron (MLP) refines the separation, producing decoupled human-related (\( f_v^{H} \)) and background-related (\( f_v^{B} \)) features:

\begin{equation}
f_v = f_v^{H} + f_v^{B}
\end{equation}

This structured decoupling minimizes background interference, enhancing pedestrian detection accuracy.



\paragraph{Process of Visible Light Images ($f_i$).}

The process begins with a Multi-Layer Perceptron (MLP), which captures complex nonlinear relationships for fine-grained feature differentiation. Max Pooling then reduces spatial dimensions while preserving the most relevant information. Next, the Sigmoid activation function normalizes feature values to \([0,1]\):
Finally, two \( 1 \times 1 \) convolutional layers separate features into human-related (\( f_v^{H} \)) and background-related (\( f_v^{B} \)) components:

\begin{equation}
f_v^{H}, f_v^{B} = W_H * F_{\text{norm}}, \quad W_B * F_{\text{norm}}
\end{equation}

where \( W_H \) and \( W_B \) are learnable transformation matrices. This structured feature decoupling enhances pedestrian detection by reducing background interference.

\paragraph{Orthogonal Regularization Loss Function.}
To ensure the independence of decoupled human and background features, an orthogonal regularization method be applied. This method involves adding a regularization term to the loss function to penalize the similarity between human features and background features. Specifically, this can be achieved by minimizing the dot product between these two sets of feature vectors, as orthogonal or uncorrelated vectors have a dot product of zero. The loss function can be formulated as:

\begin{equation}
\mathbf{L}_{\mathbf{con}} = \sum_{\mathbf{i} = 1}^{\mathbf{N}} \left| \mathbf{f}_{\mathbf{v}, \mathbf{i}}^{\mathbf{H}} \cdot \mathbf{f}_{\mathbf{v}, \mathbf{i}}^{\mathbf{B}} \right|^{2} + \mathbf{\lambda} \sum_{\mathbf{i} = 1}^{\mathbf{N}} \left| \mathbf{f}_{\mathbf{i}, \mathbf{i}}^{\mathbf{H}} \cdot \mathbf{f}_{\mathbf{i}, \mathbf{i}}^{\mathbf{B}} \right|^{2}
\end{equation}

where

\begin{itemize}
	\item $\mathbf{f}_{\mathbf{v,i}}^{\mathbf{H}}$ and $\mathbf{f}_{\mathbf{v,i}}^{\mathbf{B}}$ are the decoupled human and background features from visible light images, respectively.
	\item $\mathbf{f}_{\mathbf{i,i}}^{\mathbf{H}}$ and $\mathbf{f}_{\mathbf{i,i}}^{\mathbf{B}}$ are the decoupled human and background features from infrared images, respectively.
	\item $\mathbf{\lambda}$ is a hyperparameter that controls the weight of the infrared images term.
	\item $\mathbf{N}$ is the number of samples. Each sample $\mathbf{i}$ contributes to the overall loss, ensuring that the regularization is applied across all the data.
	\item The regularization terms $\mathbf{f}_{\mathbf{v,i}}^{\mathbf{H}} \cdot \mathbf{f}_{\mathbf{v,i}}^{\mathbf{B}}$ and $\mathbf{f}_{\mathbf{i,i}}^{\mathbf{H}} \cdot \mathbf{f}_{\mathbf{i,i}}^{\mathbf{B}}$ aim to minimize the correlation between human features and background features within the same image modality.
\end{itemize}

This approach not only effectively separates the features but also ensures their independence, which is crucial for improving the accuracy and robustness of object detection systems.

\subsection{Contrastive Learning Paradigm}
After decoupling human and background features, contrastive learning is essential to further enhance feature discrimination. This approach strengthens the model’s ability to distinguish humans from the background, improving recognition and classification across diverse lighting and environmental conditions.

\paragraph{Selection of Visible Light and Infrared Image Pairs.}
We select visible and infrared image pairs from the MSDS and LLVIP Datasets (Figure \ref{fig:rgb_vs_ir}) under similar scene conditions, enabling the model to recognize pedestrians across lighting variations. Careful pair selection ensures high-quality positive and negative samples, crucial for effective contrastive learning.

\begin{figure}
    \centering
    \includegraphics[width=\linewidth]{fig4.png}
    \caption{Example of RGB and Infrared Image Pairs. Left: RGB Image. Right: Infrared Image.}
    \label{fig:rgb_vs_ir}
\end{figure}
For accurate multimodal pedestrian detection, visible and infrared image pairs must be strictly aligned in targets, background, and dimensions, sharing unified labels for seamless feature fusion. This alignment minimizes modality discrepancies and enhances robustness.  

We implement a contrastive learning framework (Figure \ref{fig:cl}) using visible and infrared features to distinguish humans from the background. Triplet Loss optimizes feature learning by maximizing inter-class distances and minimizing intra-class distances, with a margin term defining decision boundaries. The model’s core components—Anchor (A), Positive (P), and Negative (N)—form the basis of the Triplet Loss function, ensuring effective feature discrimination.



\paragraph{Triplet Loss Function.}
The triplet loss is used in the contrastive learning framework to train the model by increasing the distance between dissimilar features (negative samples) and decreasing the distance between similar features (positive samples). The triplet loss function is defined as:
\[
\mathbf{L} = \sum_{\mathbf{i} = \mathbf{1}}^{\mathbf{N}} \max\left( \mathbf{d}\left( \mathbf{f}_{\mathbf{a,i}}, \mathbf{f}_{\mathbf{p,i}} \right) - \mathbf{d}\left( \mathbf{f}_{\mathbf{a,i}}, \mathbf{f}_{\mathbf{n,i}} \right) + \text{margin}, 0 \right)
\]

Where:

\begin{itemize}
	\item $\mathbf{f}_{\mathbf{a,i}}$ is the feature vector of the anchor sample for the i-th triplet.
	\item $\mathbf{f}_{\mathbf{p,i}}$ is the feature vector of the positive sample which is similar to the anchor.
	\item $\mathbf{f}_{\mathbf{n,i}}$ is the feature vector of the negative sample which is dissimilar to the anchor.
	\item $\mathbf{d}\left( \mathbf{x}, \mathbf{y} \right)$ represents the distance function, typically Euclidean distance, measuring the dissimilarity between two feature vectors.
	\item $\text{margin}$ is a predefined threshold that dictates the minimum difference required between the anchor-positive distance and the anchor-negative distance to encourage effective learning.
\end{itemize}

\begin{figure}
	\centering
	\includegraphics[width=\linewidth]{fig5.pdf}
	\caption{Contrastive learning paradigm.}
	\label{fig:cl}
\end{figure}

\paragraph{For Background Features.}
For background features, the positive sample is the background-related feature, while the negative sample is the human feature (background-unrelated). Through contrastive learning, we minimize the distance between all positive samples (visible background and infrared background) and maximize the distance between positive and negative samples. This approach enhances the model to learn background features more effectively, thereby improving the distinction between human and background features.

To minimize the distance between positive samples ($\mathbf{f}_{\mathbf{v}}$ background features to $\mathbf{f}_{\mathbf{i}}$ background features) and maximize the distance between negative samples and the anchor (Background to human), thereby enhancing the model's ability to discern background features difference apart from human features. The triplet loss for background features can be mathematically represented as:
\[
L_{\text{bg}} = \sum_{i = 1}^{N} \max\left( d\left( f_{a,i}^{B}, f_{p,i}^{B} \right) - d\left( f_{a,i}^{B}, f_{n,i}^{H} \right) + \text{margin}, 0 \right)
\]

\paragraph{For Human Features.}
Similarly, for human features, the positive sample is the human-related feature, and the negative sample is the background feature(human-unrelated). Through contrastive learning, we minimize the distance between all positive samples (human in visible images and human in infrared images), enabling the model to learn human features more effectively. By maximizing the distance between positive and negative samples, the model improves its ability to distinguish between human and background features.

To reduce the feature distance among human instances (positive samples) while expanding the gap between humans and their backgrounds (negative samples), thus facilitating a clearer distinction between human and environmental features. The triplet loss for human features can be mathematically represented as:
\[
L_{\text{human}} = \sum_{i = 1}^{N} \max\left( d\left( f_{a,i}^{H}, f_{p,i}^{H} \right) - d\left( f_{a,i}^{H}, f_{n,i}^{B} \right) + \text{margin}, 0 \right)
\]
\subsection{Detection Head}
The category prediction is responsible for classifying detected objects and outputting their category probabilities. As this study focuses on pedestrian detection, the module specifically predicts the probability of an object belonging to the pedestrian category. To achieve this, a fully connected layer is employed, followed by a softmax function to convert the raw predictions into a probability distribution. The formulation for category prediction is as follows:
\[
C_{\text{pred}} = \text{Softmax}\left( W_{c} \cdot F + b_{c} \right)
\]
Where $W_{c}$ is the weight matrix for category prediction, $F$ is the fused multimodal feature matrix, $b_{c}$ is the bias term, and $\text{Softmax}$ is the activation function used to calculate the probability of the target belonging to each category.

By integrating features from both RGB and infrared spectra, the category prediction module effectively addresses classification tasks under complex illumination conditions. This integration significantly enhances the accuracy of pedestrian detection by reducing false positives in low-light or cluttered background scenarios.

\paragraph{Bounding Box Prediction.}
The bounding box prediction is responsible for localizing detected objects by predicting their center coordinates $(x, y)$, width $w$, and height $h$. An anchor-based mechanism and regression to refine the bounding box parameters. The bounding box prediction is formulated as follows:

\[
B_{\text{pred}} = \left( \sigma\left( t_{x} \right) + x_{\text{cell}}, \sigma\left( t_{y} \right) + y_{\text{cell}}, p_{w}e^{t_{w}}, p_{h}e^{t_{h}} \right)
\]

Where $(t_{x}, t_{y}, t_{w}, t_{h})$ are the predicted adjustment parameters for the bounding box, $(x_{\text{cell}}, y_{\text{cell}})$ are the top-left coordinates of the grid cell, $(p_{w}, p_{h})$ are the width and height of the anchor box, and $\sigma(\cdot)$ is the Sigmoid activation function, which constrains the coordinates to the range $[0, 1]$.

\paragraph{Confidence Prediction.}
The confidence prediction evaluates whether a detected bounding box contains a pedestrian and assesses the precision of the box. The confidence score is determined by the probability of the object's presence, $P_{\text{obj}}$, and the Intersection over Union (IoU) between the predicted bounding box and the ground truth box. The formula is expressed as:

\[
S_{\text{conf}} = P_{\text{obj}} \times \text{IoU}\left( B_{\text{pred}}, B_{\text{true}} \right)
\]

Where $S_{\text{conf}}$ is the probability that the predicted bounding box contains a target object, and $\text{IoU}\left( B_{\text{pred}}, B_{\text{true}} \right)$ measures the overlap between the predicted bounding box $B_{\text{pred}}$ and the ground truth box $B_{\text{true}}$.

\begin{algorithm}[t] \caption{PedDet Algorithm Overview} \label{alg:overview} \begin{algorithmic}[1] \REQUIRE RGB image $I_{rgb}$, Infrared image $I_{ir}$ \ENSURE Detection results $B_{pred}$, $C_{pred}$, $S_{conf}$ \STATE \textbf{Feature Extraction:} \STATE $F_{rgb} \leftarrow \text{YOLOv10}(I_{rgb})$ \COMMENT{Generate multi-scale RGB features} \STATE $F_{ir} \leftarrow \text{YOLOv10}(I_{ir})$ \COMMENT{Generate multi-scale IR features} \STATE \textbf{MSFPM Processing:} \STATE $F_{fused} \leftarrow w_{rgb}F_{rgb} + w_{ir}F_{ir}$ \COMMENT{Dynamic fusion} \STATE $F_{reduced} \leftarrow W_{reduce}F_{fused}$ \COMMENT{Dimensionality reduction} \STATE $F_{activated} \leftarrow \text{SiLU}(F_{reduced})$ \STATE \textbf{IRFDM Processing:} \STATE $f_v^H, f_v^B \leftarrow \text{DecoupleFeatures}(F_{activated})$ \COMMENT{Visible branch} \STATE $f_i^H, f_i^B \leftarrow \text{DecoupleFeatures}(F_{activated})$ \COMMENT{Infrared branch} \STATE \textbf{Contrastive Learning:} \STATE $\mathcal{L}_{bg} \leftarrow \text{ComputeTripletLoss}(f_v^B, f_i^B, f_v^H)$ \STATE $\mathcal{L}_{human} \leftarrow \text{ComputeTripletLoss}(f_v^H, f_i^H, f_v^B)$ \STATE \textbf{Detection Head:} \STATE $C_{pred} \leftarrow \text{Softmax}(W_cF_{activated} + b_c)$ \STATE $B_{pred} \leftarrow \text{DecodeBox}(t_x, t_y, t_w, t_h)$ \STATE $S_{conf} \leftarrow P_{obj} \times \text{IoU}(B_{pred}, B_{true})$ \STATE \textbf{Joint Optimization:} \STATE $\mathcal{L}_{total} = \mathcal{L}_{cls} + \mathcal{L}_{box} + \mathcal{L}_{conf} + \lambda_1\mathcal{L}_{con} + \lambda_2(\mathcal{L}_{bg} + \mathcal{L}_{human})$ \end{algorithmic} \end{algorithm}

By leveraging the enhanced features from the IRFDM, the confidence prediction effectively differentiates targets from complex backgrounds. This enhanced capability ensures accurate confidence scores even under challenging conditions, such as low-light or high-brightness environments, maintaining robustness and reliability in pedestrian detection tasks.

\paragraph{Joint Loss Function.}
The overall prediction loss function integrates multiple components, including regularization constraints, the contrastive learning paradigm, classification, bounding box regression, and confidence prediction losses. To achieve comprehensive optimization of the model, we define a joint loss function, $L_{\text{total}}$, as follows:

\[
L_{\text{total}} = L_{\text{cls}} + L_{\text{box}} + L_{\text{conf}} + \lambda_{1}L_{\text{con}} + \lambda_{2}\left( L_{\text{bg}} + L_{\text{human}} \right)
\]

where $L_{\text{cls}}$ is the classification loss, $L_{\text{box}}$ is the localization loss, $L_{\text{conf}}$ is the confidence loss, $L_{\text{con}}$ is the decoupling constraint loss, and $L_{\text{bg}} + L_{\text{human}}$ is the contrastive loss. The terms $\lambda_{1}$ and $\lambda_{2}$ are hyperparameters that control the relative importance of the decoupling constraint loss and the contrastive loss, respectively.

The weights of these loss components can be adjusted to balance model performance. The choice of hyperparameters $\lambda_{1}$ and $\lambda_{2}$ depends on the specific characteristics of the dataset and the task requirements. By optimizing all these components, the overall prediction loss $L_{\text{total}}$ ensures that the model performs robustly across all detection tasks. This design enables accurate and reliable pedestrian detection, even in complex illumination conditions and diverse scenarios.

\section{Experiments}
\subsection{Experimental Setup and Datasets}
 \paragraph{Setup.}
 During the training phase, we adopted transfer learning by initializing the model with pre-trained weights from the MS COCO dataset \cite{lin2014microsoft}. By leveraging these pre-trained weights, the network could utilize general features from a large-scale dataset, significantly reducing training time and improving initial convergence speed. During validation, all input images were resized to a consistent resolution of 256$\times$256 pixels. This resizing ensures uniformity in the input data, minimizing computational overhead while maintaining detection accuracy. The validation process strictly adhered to the hyperparameter configurations used during training, ensuring the reliability and consistency of evaluation results. For training and validation configuration and hyperparameters, see \textbf{Supplementary Section 2: Implement Details}.

\paragraph{Datasets.}
We primarily evaluated our model on two benchmark datasets: the MSRS \cite{34} and LLVIP \cite{35} datasets. These datasets have become key benchmarks in the field as they reflect real-world scenarios. See \textbf{Supplementary Section 3: Datasets}.




\subsection{Comparative Study}

\begin{figure}[t]
	\centering
	\includegraphics[width=\linewidth]{fig6.png}
	\caption{Comparison of traditional RGB-modal (a, b) and ours multi-modal (c) pedestrian detection result. (a) RGB image, (b) Infrared image, and (c) Detection result of the proposed multi-modal algorithm.}
	\label{fig:ours}
\end{figure}

To comprehensively evaluate the performance of PedDet, we compared its detection accuracy against several state-of-the-art multimodal pedestrian detection algorithms, including: SSD \cite{36}, RetinaNet \cite{1}, YOLOv10 \cite{4}, Faster R-CNN \cite{10}, DDQ-DETR \cite{37}, Halfway fusion \cite{9}, ProbEn \cite{18}, ARCNN-Extension \cite{38}, PoolFuser \cite{39}, LENFusion \cite{40}. The evaluation was conducted on two benchmark datasets, MSRS and LLVIP, with standard metrics including mean Average Precision (mAP) and its at 50\% Intersection over Union (mAP@50). The experimental results are shown in Table \ref{tab:comparison} below.

\begin{table}[H]
  \centering
\resizebox{\linewidth}{!}{
  \begin{tabular}{l c c c c c}
    \toprule
    Method & Backbone & Dataset & mAP@50$\uparrow$ & mAP$\uparrow$ & Modality \\
    \midrule

    \multicolumn{6}{c}{\textbf{Infrared}} \\
    \midrule
    SSD & VGG16 & LLVIP & 90.2 & 53.5 & Infrared \\
    RetinaNet & ResNet50 & LLVIP & 94.8 & 55.1 & Infrared \\
    YOLOv10 & CSPDarknet & LLVIP & 93.1 & 53.7 & Infrared \\
    Faster R-CNN & ResNet50 & LLVIP & 94.6 & 54.5 & Infrared \\
    DDQ-DETR & ResNet50 & LLVIP & 93.9 & 58.6 & Infrared \\
    \midrule

    \multicolumn{6}{c}{\textbf{RGB}} \\
    \midrule
    SSD & VGG16 & LLVIP & 82.6 & 39.8 & RGB \\
    RetinaNet & ResNet50 & LLVIP & 88.0 & 42.8 & RGB \\
    YOLOv10 & CSPDarknet & LLVIP & 87.8 & 50.2 & RGB \\
    Faster R-CNN & ResNet50 & LLVIP & 87.0 & 47.5 & RGB \\
    DDQ-DETR & ResNet50 & LLVIP & 86.1 & 46.7 & RGB \\
    \midrule

    \multicolumn{6}{c}{\textbf{Infrared + RGB}} \\
    \midrule
    Halfway Fusion & VGG16 & LLVIP & 91.4 & 55.1 & Infrared + RGB \\
    ProbEn & ResNet50 & LLVIP & 93.4 & 51.5 & Infrared + RGB \\
    ARCNN-Extension & VGG16 & LLVIP & 89.2 & 56.2 & Infrared + RGB \\
    PoolFuser & ResNet-34s & LLVIP & 80.3 & 38.4 & Infrared + RGB \\
    LENFusion & CSPDarknet & LLVIP & 81.6 & 53.0 & Infrared + RGB \\
    \rowcolor{yellow} \textbf{PedDet (Ours)} & CSPDarknet & LLVIP & \textbf{95.8} & \textbf{56.8} & Infrared + RGB \\
    \bottomrule
  \end{tabular}
}
\caption{\textbf{Comparison of detection performance of SOTA methods.} The best values are in \textbf{bold}.}
\label{tab:comparison}
\end{table}

The detection results as shown in Figure \ref{fig:ours}(a), visible-light images exhibit challenges in distinguishing humans from the background under poor illumination conditions. This limitation is evident in the low contrast between objects and their surroundings, which makes pedestrian detection difficult. In contrast, infrared images provide distinct object contours, enabling easier differentiation of pedestrians from the background. The proposed PedDet effectively integrates the complementary features of visible and infrared spectra, highlighting pedestrians in the fused feature representation. This integration results in robust detection performance, particularly in low-light scenarios as shown in  Figure \ref{fig:vis1}.

\begin{figure}[h]
    \centering
    \begin{minipage}{0.48\linewidth}
    \centering
    \includegraphics[width=\linewidth]{EVR.png}
    \end{minipage}
    \begin{minipage}{0.51\linewidth}
        \centering
    \includegraphics[width=\linewidth]{EVR2.png}
    \end{minipage}
    \caption{Comparison of detection visualization effects with and without infrared assistance.}
    \label{fig:vis1}
\end{figure}



\subsection{Ablation Study}
\paragraph{Effect of Contrastive Learning.}
In this section, we evaluate the impact of contrastive learning (CL) on the performance of different backbones. The results are presented in Table~\ref{tab:contrastive_learning}. As shown, the introduction of contrastive learning consistently improves the performance across all backbones. For instance, YOLOv10-S achieves a performance of 54.8  \% with CL, which is an improvement of 1.6  \% compared to its performance without CL. Similar improvements are observed for YOLOv10-M and YOLOv10-L, with performance gains of 1.6  \% and 1.7  \%, respectively. These results highlight the effectiveness of contrastive learning in enhancing the model's capability.

\begin{table}[h!]
\centering
\caption{Impact of different backbones on the task with and without contrastive learning.}
\label{tab:contrastive_learning}
\resizebox{0.9\linewidth}{!}{ %
\begin{tabular}{>{\centering\arraybackslash}p{2.5cm} |>{\centering\arraybackslash}p{2cm} >{\centering\arraybackslash}p{2.8cm}} 
\toprule
\textbf{Backbone} & \textbf{w/o CL} & \textbf{w/ CL} \\
\midrule
YOLOv10-S & 53.2  \% & 54.8  \%  \textcolor{red}{ ($\uparrow$ 1.6  \%)} \\
YOLOv10-M & 55.1  \% & 56.7  \%  \textcolor{red}{   ($\uparrow$ 1.6  \%)} \\
YOLOv10-L & 56.4  \% & 58.1  \%  \textcolor{red}{   ($\uparrow$ 1.7  \%)} \\
\bottomrule
\end{tabular}
}
\end{table}


\paragraph{Effect of Additional Modality.}
We further investigated the impact of different modalities (AM) on task performance, with the results summarized in Table~\ref{label3}. The inclusion of additional modalities significantly improved performance. After introducing MSFPM and IRFDM, PedDet achieved a performance of 95.8  \%, representing an improvement of 1.1  \% compared to when CL was not used. This demonstrates the importance of leveraging additional modalities to enhance the model's effectiveness.



\begin{table}[h!]
\centering
\caption{Performance comparison of different models.}
\label{label3}
\resizebox{0.9\linewidth}{!}{ %
\begin{tabular}{c|c c}
\toprule
\textbf{Models} & \textbf{mAP@50} & \textbf{mAP} \\
\midrule
\rowcolor{yellow} PedDet    & 95.8   & 56.8    \\
PedDet (w/o MSFPM)   & 93.4      & 54.3      \\
PedDet (w/o IRFDM)   & 94.1       & 55.0     \\
PedDet (w/o CL)   & 94.7      & 55.5      \\
PedDet (w/o MSFPM and IRFDM)   & 91.2      & 51.8      \\
PedDet (w/o MSFPM and CL)   & 92.8      & 53.1      \\
PedDet (w/o IRFDM and CL)   & 93.0      & 53.5       \\
\bottomrule
\end{tabular}
}
\end{table}

\paragraph{T-SNE Visualization of Feature Decoupling Results.}
To further demonstrate the effectiveness of our feature decoupling module, we performed t-SNE visualization. As shown in the Figure \ref{fig:decouple} below, the t-SNE visualization shows the model's ability to distinguish between \textit{human relevant (green color) and human irrelevant (red color)} features over different training epochs (100, 200, 500, 1000).  In the initial stages of training, the human-relevant feature and human-irrelevant feature points are highly mixed, indicating the model's initial weak capability in feature distinction.

However, as training progresses to 200 epochs, the data points begin to show some clustering trends, although there is still a lot of overlap. By 500 epochs, the separation between human-relevant features and human-irrelevant features becomes more clearer, indicating a significant improvement in the model's ability to discriminate features. At 1000 epochs, the clustering of human-relevant and human irrelevant features is very obvious, with minimal overlap, which shows that the model has effectively learned to differentiate between the two types of features, accurately separating and grouping them. 

This gradual improvement of the learning process indicates that the training strategy of the model is successful, enabling the model to accurately perform tasks such as classification, recognition in complex datasets.
\begin{figure}[h]
	\centering
	\includegraphics[width=\linewidth]{fig9.png}
	\caption{t-SNE visualization of feature decoupling results.}
	\label{fig:decouple}
\end{figure}

\section{Conclusions}

In this paper, we propose PedDet, an adaptive framework that enhances pedestrian detection under complex illumination by leveraging spectral complementarity. It integrates the MSFPM and IRFDM to optimize detection across visible and infrared spectra. MSFPM adjusts feature weights, while IRFDM mitigates illumination noise by decoupling pedestrian and background features. A contrastive learning approach further improves feature discrimination. Experimental results show PedDet outperforms existing methods, establishing it as a reliable and effective solution for multimodal pedestrian detection and a benchmark in intelligent transportation systems.








\clearpage
%%%% ijcai25.tex

\typeout{IJCAI--25 Instructions for Authors}

% These are the instructions for authors for IJCAI-25.

\documentclass{article}
\pdfpagewidth=8.5in
\pdfpageheight=11in

% The file ijcai25.sty is a copy from ijcai22.sty
% The file ijcai22.sty is NOT the same as previous years'
\usepackage{ijcai25}

% Use the postscript times font!
\usepackage{times}
\usepackage{soul}
\usepackage{url}
\usepackage[hidelinks]{hyperref}
\usepackage[utf8]{inputenc}
\usepackage[small]{caption}
\usepackage{graphicx}
\usepackage{amsmath}
\usepackage{amsthm}
\usepackage{amssymb}
\usepackage{booktabs}
\usepackage[table,xcdraw]{xcolor}
\usepackage{algorithm}
\usepackage{algorithmic}
\usepackage[switch]{lineno}

\usepackage{xcolor}
\usepackage{subcaption} % For subfigures


% Comment out this line in the camera-ready submission
%\linenumbers

\urlstyle{same}

% the following package is optional:
%\usepackage{latexsym}

% See https://www.overleaf.com/learn/latex/theorems_and_proofs
% for a nice explanation of how to define new theorems, but keep
% in mind that the amsthm package is already included in this
% template and that you must *not* alter the styling.
\newtheorem{example}{Example}
\newtheorem{theorem}{Theorem}
\newtheorem{definition}{Definition}
\newtheorem{proposition}{Proposition}
\newtheorem{remark}{Remark}

% Following comment is from ijcai97-submit.tex:
% The preparation of these files was supported by Schlumberger Palo Alto
% Research, AT\&T Bell Laboratories, and Morgan Kaufmann Publishers.
% Shirley Jowell, of Morgan Kaufmann Publishers, and Peter F.
% Patel-Schneider, of AT\&T Bell Laboratories collaborated on their
% preparation.

% These instructions can be modified and used in other conferences as long
% as credit to the authors and supporting agencies is retained, this notice
% is not changed, and further modification or reuse is not restricted.
% Neither Shirley Jowell nor Peter F. Patel-Schneider can be listed as
% contacts for providing assistance without their prior permission.

% To use for other conferences, change references to files and the
% conference appropriate and use other authors, contacts, publishers, and
% organizations.
% Also change the deadline and address for returning papers and the length and
% page charge instructions.
% Put where the files are available in the appropriate places.


% PDF Info Is REQUIRED.

% Please leave this \pdfinfo block untouched both for the submission and
% Camera Ready Copy. Do not include Title and Author information in the pdfinfo section
\pdfinfo{
/TemplateVersion (IJCAI.2025.0)
}

\title{The Combined Problem of Online Task Assignment and Lifelong Path Finding\\ in Logistics Warehouses: A Case Study}


% Single author syntax
%\author{
%    Author Name
%    \affiliations
%    Affiliation
%    \emails
%    email@example.com
%}

%\author{Anonymous}

% Multiple author syntax (remove the single-author syntax above and the \iffalse ... \fi here)
%\iffalse
\author{
Fengming Zhu$^1$
\and
Fangzhen Lin$^1$
\and
Weijia Xu$^2$
\And
Yifei Guo$^2$\\
\affiliations
$^1$CSE Department, HKUST\\
$^2$Meituan Academy of Robotics Shenzhen\\
\emails
fzhuae@connect.ust.hk,
flin@cse.ust.hk,
\{xuweijia, guoyifei02\}@meituan.com
}
%\fi

\begin{document}

\maketitle

%\vspace{-3mm}

\begin{abstract}
%We study the online problem of task assignment and path finding in modern automated warehouses, which has significant applications in the logistics industry.
%%%% what assumption
%%The existing literature mostly considers ideal abstractions of such problems by imposing potentially unrealistic assumptions.
%%To this end, we propose a system that aims at mitigating those gaps between simulation and real-world deployment.
%%%%
%%The combined problem of task assignment and path finding for robot swarms is one of the key challenges in modern automated warehouses, especially in the logistics .
%The existing literature 
%(1)~mostly considers idealized formulations of such problems by assuming
%robot models that neglect rotational costs and focusing on well-formed layouts,
%and (2)~has not fully explored the benefit of deliberate task assignment.
%To address the above issues, our proposed system redesigns these two modules, namely
%(i)~a lifelong path planner that is directly tailored for a more practical robot model in possibly \textit{non}-well-formed layouts, and
%%(ii)~online task assignment that reacts on real-time states.
%(ii)~an task assigner that can adapt to the underlying path planner.
%Simulation experiments conducted in warehouse scenarios at \textit{Meituan}, one of the largest shopping platforms in China, demonstrate that
%(a)~\textit{in terms of time efficiency},
%our system takes only 83.77\% of the execution time needed for the currently deployed system at Meituan, outperforming other SOTA algorithms by 8.09\%;
%(b)~\textit{in terms of economic efficiency},
%ours can achieve the same throughput with only 60\% of the agents of the current scale.
%%%%
%% Simulation experiments conducted in warehouse scenarios at \textit{Meituan}, one of the biggest shopping platforms in China, demonstrate that our system, in terms of throughput, 
%% outperforms other SOTA algorithms by 10.2\%, and outperforms the currently deployed system at Meituan by 19.4\%.
%% From the economical side, our system can achieve the same throughput with only 60\% agents of the current scale.
%%%%
%Our observations also imply that rule-based methods, though may not be general, are sometimes effective enough in practice.




We study the combined problem of online task assignment and lifelong path finding, which is crucial for the logistics industries.
However, most literature either (1) focuses on lifelong path finding assuming a given task assigner, or (2) studies the offline version of this problem where tasks are known in advance.
We argue that, to maximize the system throughput, the online version that integrates these two components should be tackled directly.
To this end, we introduce a formal framework of the combined problem and its solution concept.
Then, we design a rule-based lifelong planner under a practical robot model that works well even in environments with severe local congestion.
Upon that, we automate the search for the task assigner with respect to the underlying path planner.
Simulation experiments conducted in warehouse scenarios at \textit{Meituan}, one of the largest shopping platforms in China, demonstrate that
(a)~\textit{in terms of time efficiency},
our system requires only 83.77\% of the execution time needed for the currently deployed system at Meituan, outperforming other SOTA algorithms by 8.09\%;
(b)~\textit{in terms of economic efficiency},
ours can achieve the same throughput with only 60\% of the agents currently in use.
\end{abstract}



\section{Introduction}

%% background problem
%We consider the problem of real-world warehouse automation, where a fleet of robots are programmed to pick up and deliver packages without any collision.
%The investigation of such a problem will significantly benefit logistics companies. 
%However, the problem is inherently hard, primarily for two reasons:
%(1)~the computational complexity of multi-agent path finding is notorious especially when the size of the robot fleet is considerably large, and
%(2)~the stream of tasks arrive in an online fashion of which the accomplishment also depends on the subsequent execution by the path finding module. 

% Lin's version
We consider the problem present in highly automated real-world warehouses where a fleet of robots is programmed to pick up and deliver packages without any collision.
This is a significant problem for logistics companies as it has a major impact on their operational efficiency.  
It is a difficult problem for at least the following  two reasons:
(1)~the computational complexity of multi-agent path finding is notoriously high, especially when the number of robots is large, and
(2)~the dynamic and real-time assignment of tasks to the robots both depends on and affects the subsequent path finding.


% Existing mainstream solutions and main issues
There is a vast literature that studies idealized abstractions of such real-world problems.
The most commonly seen formulation is to assume a given (or naive) task assigner, and therefore, the focus is merely on the path-finding part, which is usually termed as one-shot \textit{Multi-Agent Path Finding} (MAPF)~\cite{yu2013structure,erdem2013general,sharon2015conflict,li2021eecbs,okumura2022priority,okumura2023lacam}
or its lifelong version \textit{Multi-Agent Pickup and Delivery} (MAPD)~\cite{ma2017lifelong,vsvancara2019online,li2021lifelong,okumura2022priority}.
%Note that a task assigner tightly depends on the underlying path finding. In other words, to maximize the throughput of the whole pipeline, one should adapt a task assigner to a path planner.
However, to maximize the throughput of the whole production pipeline, the task assigner should also be deliberately designed with respect to the particular underlying path planner.
To this end, some recent work has further investigated the combined problem of \textit{Task Assignment and Path Finding} (TAPF)~\cite{yu2013multi,ma2016optimal,honig2018conflict,liu2019task,chen2021integrated,tang2023solving}.
%Nevertheless, this line of work is mostly restricted to  offline scenarios, i.e., tasks (and/or their release times) are priorly known.
Nevertheless, this line of work is mostly restricted to  offline scenarios, i.e., tasks (and/or their release times) are assumed to be known.
%However, so far only offline scenarios, i.e. tasks (and/or their release times) are assumed upfront, are considered.
In practice, for example in a sorting center, orders may come dynamically in real-time.

% Other implementation-related problems
Besides, we draw attention to two seemingly minor but indeed fundamental aspects.
\textbf{For one}, the robots are usually abstracted to agents doing unit-cost unit-distance cardinal actions,
i.e., \{\texttt{stop}, \texttt{$\uparrow$}, \texttt{$\downarrow$}, \texttt{$\leftarrow$}, \texttt{$\rightarrow$}\}, what we term as the \texttt{Type}$\oplus$ robot model.
The planned paths are later post-processed to executable motions regarding kinematic constraints~\cite{honig2016multi} and action dependencies~\cite{honig2019persistent}, as a real-world robot has to rotate before going in a different direction.
Imaginably, when the rotational cost is not negligible compared to the translational cost, the quality of the plans computed for the \texttt{Type}$\oplus$ robot model will be largely compromised when instantiated to motions.
A candidate solution is to revisit and reimplement the existing algorithms over an alternative set of atomic actions \{\texttt{stop}, \texttt{forward}, \texttt{$\circlearrowright$90}, \texttt{$\circlearrowleft$90}\}, which we advocate in this paper as the \texttt{Type}$\odot$ robot model.
\textbf{For another}, most of the literature assumes the problem instance to be \textit{well-formed}~\cite{ma2017lifelong,liu2019task,xu2022multi} to guarantee completeness of their methods, which is actually a strong condition requiring that every agent can find a collision-free path to her current goal even if the others are stationary.
%In fact, \textit{non-well-formed} instances are also commonly seen in modern warehouse, illustrated as an example in Figure~\ref{fig:eg_non_wf}.
%The planner that we will present later makes no use of this potentially unrealistic assumption.
However, this assumption is often not met in modern warehouses. In particular, the instance (Figure~\ref{fig:eg_non_wf}) that we consider in this work does not satisfy this condition.

%wellformed instance too strong!

%\textit{
%Given a potentially non-well-formed layout with the number of agents varying,
%is there a good way for online TAPF that can take the advantage of the layout.
%}
%\textbf{That is why this is a case study}


%Might be minor for academia, but critical for industrial deployment

%MAPF-POST and Temporal Plan Graph (TPG) ~\cite{honig2016multi}

%allow task swapping~\cite{okumura2023solving}

%robust warehouse execution~\cite{honig2019persistent}

%capacitated mapd~\cite{chen2021integrated}

%probably no need for a general algorithm, but a practical one for certain structured layouts


\begin{figure*}[tb]
\centering
\vspace{-3mm}
\includegraphics[width=180mm]{fig/warehouse3-compressed}
\caption{A \textit{non-well-formed} instance 
%(see~\protect\cite{xu2022multi}) 
currently deployed in Meituan warehouses. The white cells near \textsc{Green} dots are delivery ports, while the ones near \textsc{Red} dots are pickup ports. Colored circles heading to different directions with numbers are agents. The colored box (blue) is a pickup port currently assigned to the agent in the same color (ID 45 in the lower right area). Congestion happens a lot near the pickup ports.}
\vspace{-1mm}
\label{fig:eg_non_wf}
%\vspace{-2mm}
\end{figure*}

% Our Contribution
Considering the aforementioned issues,
%we here propose a system that organically integrates the task assignment and the path planning in an online manner.
%so that the mismatch between algorithmic simulation and real-world deployment shall be alleviated. 
we introduce a formal framework to study the combined problem that organically integrates task assignment and path finding in an online manner.
%\textcolor{red}{Rigorously formalize the online problem}
%More specifically,
To solve the formalized problem,
we \textbf{first} develop lifelong path finding algorithms directly for the \texttt{Type}$\odot$ robot model (assuming an arbitrary task assigner), including those adapted from the existing literature and our new rule-based planner which performs both efficiently and effectively, even for non-well-formed instances.
%Note that an additional strong assumption of the aforementioned work is that the layout should better be \textit{well-formed}~\cite{ma2017lifelong},
%which is dropped in our system, since non-well-formed layouts are commonly seen in modern warehouses.
%We illustrate a non-well-formed example in Figure~\ref{fig:eg_non_wf}.
\textbf{Secondly}, we propose a novel formulation that addresses the online problem of task assignment as optimally solving a Markov Decision Process~(MDP), with path planners as hyper-parameters.
Due to the complex state space and transition of the formulated MDP, we resort to approximated solutions by reinforcement learning (RL), as well as other non-trivial rule-based ones with insightful observations.
%To our best knowledge, this is the first compound system that directly tackles both task assigner and path planner simultaneously in an online fashion.
\textit{Simulation experiments on warehouse scenarios at Meituan, one of the largest shopping platforms in China, have shown that our system
(1)~takes only 83.77\% of the execution time needed for the currently deployed system at Meituan, outperforming other SOTA algorithms by 8.09\%; and
(2)~can achieve the same throughput with only 60\% of the agents of the current scale.}
We also draw an important lesson from this study
that both path finding and task assignment should fully exploit the warehouse layout, as it is normally fixed in a relatively long period of time after deployment, though the number of agents may still vary.
To this end, it might be more worthwhile to investigate layout-dependent-agent-independent solutions instead of entirely general-purpose solutions.




This paper is organized as follows.
We first list a few related areas in Section~\ref{sec:related}.
Then the problem formulation is provided in Section~\ref{sec:problem}.
We later present our system in two parts: the path planners in Section~\ref{sec:pf}, and the task assigners in Section~\ref{sec:ta}.
Experimental results are shown in Section~\ref{sec:exp}, mainly conducted for Meituan warehouse scenarios with various scales of agents.
We conclude the paper with a few insightful discussions in Section~\ref{sec:conclusion}.

% Our system


% Organization


\section{Related Work}
\label{sec:related}
%Our work pertains to a broad range of research areas, mainly on planning from the multi-agent community and scheduling from the operations research community.

% 1. path finding
% MA-PF/PD
\textbf{Path Finding.}
The study of MAPF aims to develop centralized planning algorithms.
In spite of the computational complexity being NP-hard in general~\cite{yu2013structure},
researchers have developed practically fast planners that can even solve instances with hundreds of agents within seconds.
Exemplars can be found via reduction to logic programs~\cite{erdem2013general},
prioritized planning~\cite{silver2005cooperative,ma2019searching,okumura2022priority},
conflict-based search~\cite{sharon2015conflict,li2021eecbs}, 
depth-first search~\cite{okumura2023lacam}, etc.
Most of them can be extended to the online version of the problem, i.e. MAPD,
where the goals assigned to agents are priorly unknown~\cite{ma2017lifelong,vsvancara2019online,li2021lifelong}.
%Usually, adaptations are then made via communication between robots and broadcasting by the central controller for the purpose of synchronization.
%A notable drawback is that both MAPF and MAPD assume a task assigner should be given.


% 2. task assignment
% TAPF, Anon-MAPF
% Multi-Goal MA-PF/PD
\textbf{Task Assignment.}
The earliest attempt is the formulation of \textit{Anonymous}-MAPF (AMAPF) that does not specify the exact goal that an agent must go to~\cite{stern2019multi}.
%, but the total number of anonymous goals should be less than or equal to the number of agents
Compared with the labeled version, AMAPF can be solved in polynomial time,
via reduction to max-flow problems~\cite{yu2013multi}, or target swaps~\cite{okumura2023solving}.
As a generalization,
TAPF explicitly associates each agent with a team~\cite{ma2016optimal}, or with a set of candidate goals~\cite{honig2018conflict,tang2023solving},
and eventually computes a set of collision-free paths as well as the corresponding assignment matrix.
Another analogous formulation is called \textit{Multi-Goal} (MG-)MAPF~\cite{surynek2021multi,ijcai2024p0028} and its lifelong variant MG-MAPD~\cite{xu2022multi}, which also associates each agent with a set of goals, but the visiting order is pre-specified.
%Imaginably, the further generalized model should consider that tasks arrive as a priorly unknown sequence and are assigned to agents for real-time path finding.
%To the best of our knowledge, this problem is so far open,
%for which we will present a solution system in this paper.

%Hungarian method~\cite{kuhn1955hungarian}
%allow task swapping~\cite{okumura2023solving}

%% 3. comb. opt., scheduling 
%% Job shop scheduling, vehicle routing
%\textbf{Scheduling.}
%One may notice the analogy between TAPF and job-shop scheduling problems (JSSP)~\cite{manne1960job,jain1999deterministic} or vehicle routing problems (VRP)~\cite{toth2014vehicle,braekers2016vehicle}.
%However, there are two key differences: (1) job durations in JSSP and route lengths in VRP are usually known in advance and (2) the execution of jobs or routes are independent of each other.
%Neither of the two conditions holds in TAPF, especially when the tasks are released online.

%\textcolor{red}{some postponed to Appendix~\ref{apd:more_related_work}}
\textit{We also append some discussion on other less related areas to Appendix~\ref{apd:more_related_work}.}
Despite the rich literature, none of the above directly solves the online problem that a real-world automated warehouse is faced with, which well motivates this work.



\section{Problem Definition}
\label{sec:problem}

%\textcolor{red}{N as a number and a set}

We consider a set of agents $N$, moving along a 4-neighbor grid map given as an undirected graph $G = (V, E)$,
where $V$ is the set of vertices and $E$ is the set of unit-cost edges.
Let $P, D\subseteq V$ denote the set of pickup ports and the set of delivery ports, respectively.
Usually, these two sets are disjoint and are specified alongside the map graph.


Let $k$ starting from 0 denote any arbitrary timestep (to be distinguished from the later notations of tasks).
At each timestep $k$, the local \textit{agent-state} of agent $i$, denoted as $\phi_i^k$, is a 3-tuple consisting of
her current location $l_i^k\in V$,
direction $d_i^k\in \{n,s,w,e\}$,
and goal $g_i^k \in P\cup D$.
$\Phi_i$ is the set of all possible local states of agent $i$, and consequently $\Phi = \prod_{i\in N} \Phi_i$ is the set of all possible \textit{joint agent-states}.
Each agent is associated with a set of four unit-cost actions $A=\{\texttt{stop}, \texttt{forward}, \texttt{$\circlearrowright$90}, \texttt{$\circlearrowleft$90}\}$ (called the \texttt{Type}$\odot$ robot model), with their usual meaning specified using the deterministic function $move$.
For example,
\[
\begin{split}
	move(((3,28), w),\ \ \ \ \ \ \ \ \ \texttt{$\circlearrowright$90}) &\to ((3,28), n) \\
	move(((3,28), e), \texttt{forward}) &\to ((3,29), e)
\end{split}
\]
While planning paths for agents, we need to avoid the following types of collisions.
\begin{definition}[Collision types~\cite{stern2019multi}]
Let $i$ and $j$ be any arbitrary pair of agents,
	\begin{itemize}
		\item Vertex-collision: $l_i^k = l_j^k$,
		\item Edge-collision: $l_i^k = l_j^{k+1} \land l_j^{k} = l_i^{k+1}$,
%		\item Following-conflict: $l_i(k+1) = l_j(k)$.
	\end{itemize}
\end{definition}

%In fact, there is an additional type of conflicts, namely \textit{cycle-conflict}, which is a special case when the group of agents that run into a following-conflict form a cycle.
%Mathematically, a cycle-conflict happens when there exists a subset of agents
%$\{i, i+1, \cdots, j-1, j\}$,
%such that $l_i(k+1) = l_{i+1}(k)\land l_{i+1}(k+1) = l_{i+2}(k)\land \cdots\land l_{j}(k+1) = l_{i}(k)$.
%Note that we do not explicitly include cycle-conflicts in our discussion because forbidding following-conflicts implies forbidding cycle-conflicts.

%\begin{definition}[Lock types]
%ss
%	\begin{itemize}
%		\item Dead-lock.
%		\item Live-lock.
%	\end{itemize}
%\end{definition}

%The so-called tasks are composed of a sequence $I$ of typed items. 
%The type of an item $\iota \in I$ is also known as a \textit{stock keeping unit}\footnote{https://en.wikipedia.org/wiki/Stock\_keeping\_unit} (SKU).
%We let $T$ denote the set of all possible types (SKUs).
The so-called tasks are composed of a sequence $I$ of typed items. 
Each item $\iota \in I$ is associated with a type $t\in T$.
A back-end demand database specifies for each type a subset of delivery ports that items of the type need to be delivered to. 
Here we model such a database as a lookup table $L: T\mapsto 2^D$.
As $L$ will be changed in real-time, we also let $L^k$ denote the demand database at timestep~$k$.
When an agent has finished her last delivery job and returned to a pickup port at timestep $k$,
an item, say of type $t$, will be loaded onto this idle agent.
The system will then check the lookup table $L^k$, choose one target delivery port $g\in L^k(t)$ to assign to this agent, and delete this demand, i.e. $L^{k+1}(t) = L^k(t)\backslash \{g\}$.

Also, we have an assignment table $\eta$ that keeps track of which one of the loaded items is assigned to which agent, i.e., $\eta(\iota) = i$ means the item $\iota$ is currently carried by agent $i$.
An item $\iota$ is \textit{delivered} if there exists a timestep $k$ such that $l_{\eta(\iota)}^k = g_{\eta(\iota)}^k$, i.e., the agent carrying this item has reached her goal.
Upon successful delivery, the item $\iota$ will be deleted from the entry of $\eta$.
As $\eta$ is being changed  over time, we also use the time-indexed version $\eta^t$.
 
%Once an agent is assigned a goal, she cannot swap her goal with that of anyone else, as in real-world warehouses those robots are not equipped with arms and cannot swap packages.
We assume (1) $L$ has recorded the demands of a long enough period of time, and therefore, will not be enlarged;
and (2) an item will be appended to the system
only when it is loaded onto an agent.
%We also assume that no two agents simultaneously arrives at two pickup ports or two delivery ports due to the continuous-time nature of the real-world problem, but we allow the case when one is at a pickup port and the other is at a delivery port.
%Note that this does not imply any two agents will not appear simultaneously at the same location and hence no impossibility of any collisions.
%The above assumption is merely about no two agents will be switched to such a status waiting for assignment.

%(3)~no two agents arrive at their pickup ports simultaneously.
%Therefore, the path planner has no knowledge of the forthcoming items and their types.

With the above notations, we formally define the dynamics of the whole system as a deterministic \textit{system-transition} function over \textit{system-states}.
\begin{enumerate}
	\item
	A \textit{system-state} $\psi$ is a tuple consisting of the joint agent-state $\phi = \{\phi_i\}_{i \in N}$, the lookup table $L$, and the assignment table $\eta$ at that moment. % illegal state?
	Let $\Psi$ denote all possible system states.
	Among them, there is an initial system-state $\psi^0 = (\phi_1^0, \cdots, \phi_N^0, L^0, \eta^0)$.
%	A system state is \textit{legal} if no two agents are at pickup locations simultaneously.
	
	\item
	The space of \textit{system-actions} is $\Omega = (A\cup P\cup D)^N$. That is, a \textit{system-action} $\omega\in \Omega$ is an ordered tuple of the atomic actions of agents where any of them can be substituted by an assignment decision.
	A system-action $\omega$ is \textit{executable} under a legal system-state $\psi$ iff
	\[
	\begin{split}
	 \forall i \in N.\ 
	 & [\l_i \in V\backslash (P\cup D) \land \omega_i \in A] \lor \\
	 & [\l_i \in P \land \omega_i \in D] \lor
	[\l_i \in D \land \omega_i \in P]
	\end{split}
	\]
	
	\item $\Gamma: \Psi \times \Omega \mapsto \Psi$ is the \textit{system-transition} function, which means (1) if no agents are at the pickup/delivery ports, then the system proceeds by deterministically moving agents by their reported actions, which will not change the goal component $g_i$ in each agent-state $\phi_i$; or (2) if any agent arrives at any pickup (resp. delivery) port, then the  system  needs to re-assign the agent the next delivery (resp. pickup) port, which will change the goal of that particular agent to the corresponding new location and temporarily force her to stay in-place, and change the demand table $L$ as well as the assignment table $\eta$ accordingly.
\end{enumerate}
%\textcolor{red}{one item at a time}
An additional minor assumption is, even if two agents arrive at two different pickup (resp. delivery) ports simultaneously,
%they will not be ``switched'' to the status of waiting for new-delivery (resp. pickup) assignments asimultaneously, as in real-world both time and motion execution are continuous.
they will eventually get assigned certain new ports within the next one single timestep one by one in a random order. We do not care about the case where one is waiting for a new-delivery assignment while another is waiting for a new-pickup assignment.

In summary, a \textit{problem instance} is a following tuple $$<N, G, P, D, A, I, L>,$$
and consequently a \textit{principle solution} is threefold:
\begin{enumerate}
	\item $\pi_N: \Phi\mapsto A^N$ is the routing policy that outputs the next action for each agent given their current states. It is unnecessary to compute the entire $\pi_N$ completely upfront, instead, execution can be interleaved with replanning.
	\item $\pi_{D}: \Psi \times 2^D \mapsto D$ is the delivery selection policy which assigns an agent a delivery port among the candidates according to $L(t)$ when she is at one of the pickup ports and given an item of type $t$.
%	Note that $t$ is not one of the input parameters as the loaded item is not within control. 
	\item $\pi_{P}: \Psi \mapsto P$ is the pickup selection policy which assigns an agent a pickup port to return to when she has finished the delivery.
\end{enumerate}
Note that (1) both $\pi_D$ and $\pi_P$ assign one new goal at a time, as we assumed before;
(2) both $\pi_D$ and $\pi_P$ will change the goal of that particular agent to the corresponding port,
while $\pi_N$ will not;
(3) if an agent is at a pickup or delivery port, then her agent-action, even if specified by $\pi_N$, will be overwritten to $\texttt{stop}$ by the decision of $\pi_D$ or $\pi_P$.

%condition for feasible policy%
\begin{definition}[Feasibility]
Given any system-state $\psi$ and the system-transition $\Gamma$, the application of the above solution policy $(\pi_N, \pi_D, \pi_P)$ deterministically outputs a successor system-state $\psi'$. If there is no aforementioned collision between $\psi$ and $\psi'$, then $(\pi_N, \pi_D, \pi_P)$ is a feasible solution.
\end{definition}

%A goal $g\in P\cup D$ is reached under $\pi_N$ if there exists an agent $i$ and a timestep $k$, such that $l_i(k) = g$.
%An item of type $t$ is delivered if the target delivery port assigned by $\pi_{D}$ is reached. 

%TODO: two items with identical goals

%The system ends immediately once it has delivered every item in the online sequence $I$, in a given period of time.
We finally define \textit{makespan} as our evaluation metric.
\begin{definition}[Makespan]
	Given a problem instance with its initial system-state $\psi^0$, and a feasible solution $(\pi_N, \pi_D, \pi_P)$, an execution trajectory will be generated by the sequential applications of the solution policy $\{\psi^0, \psi^1, \cdots\}$. The makespan is the minimum timestep $k$ such that every item in I is delivered at $\psi^k$. 
\end{definition}

However, in real-world warehouses, the pickup ports are usually concentrated in a restricted local area for operational convenience, e.g., in the top right corner of Figure~\ref{fig:eg_non_wf}. Therefore, $\pi_{P}$ is normally implemented for the purpose of balancing the loads over all pickup ports.
\textbf{In this work, we merely aim at a \textit{practical solution} consisting of only $(\pi_N, \pi_D)$, assuming $\pi_{P}$ is given and is not part of the desired solution.}







%\textbf{Evaluation Protocol.}
%Since usually the underlying order database $\mathcal{T}$ is quite large, we will run the system through relatively small sets of tasks to test the elapsed \textit{makespans}, i.e., the number of timesteps to accomplish the given set of tasks.

%\begin{example}[System pipeline]
%As in Figure~\ref{fig:eg_non_wf},
%$\textsc{Robot}_1 \sim \textsc{Robot}_{50}$ initially rest in random locations upon the completion of the system execution the last time. Now the system is launched and every robot starts moving towards the pickup ports, namely $\textsc{Red}_1$ and $\textsc{Red}_2$. Suppose $\textsc{Robot}_{31}$ first reaches $\textsc{Red}_2$, the human operator loads one item, say a dozen of eggs, onto $\textsc{Robot}_{31}$, and the system checks the demand database and finds out that there are three orders requesting a dozen of eggs, with corresponding delivery ports $\textsc{Green}_{69}$, $\textsc{Green}_{142}$, and $\textsc{Green}_{83}$. After deliberate consideration, the system decides to send $\textsc{Robot}_{31}$ to $\textsc{Green}_{83}$ this time (and later ones to the remaining two delivery ports), and plans a path for $\textsc{Robot}_{31}$ to go from $\textsc{Red}_2$ to $\textsc{Green}_{83}$ while avoiding potential collisions with others, and so forth.
%\end{example}

\begin{example}[System pipeline]
As shown in Figure~\ref{fig:eg_non_wf}, $\textsc{Robot}_1 \sim \textsc{Robot}_{50}$ initially rest in random locations after the last system execution. Once the system is launched, each robot moves towards the pickup ports, $\textsc{Red}_1$ and $\textsc{Red}_2$. When $\textsc{Robot}_{31}$ reaches $\textsc{Red}_2$, the human operator loads a dozen eggs onto it. The system checks the demand database and finds three orders for a dozen eggs, with delivery ports $\textsc{Green}_{69}$, $\textsc{Green}_{142}$, and $\textsc{Green}_{83}$. After consideration, the system decides to send $\textsc{Robot}_{31}$ to $\textsc{Green}_{83}$ this time, planning a path while avoiding potential collisions, with subsequent deliveries to the other two ports.
\end{example}

One may see potential connections between our problem and the standard formulation MAPD in the existing literature, \textit{we postpone some remarks elaborating on the differences to Appendix~\ref{apd:relate_to_mapd}, due to the limited space}.
%\begin{remark}[Relation to MAPD]
%In MAPD, an online task $t_i$ is characterized by a pickup port $s_i$ and a delivery port $g_i$ with a priorly unknown release time. Once an agent becomes idle, she will select one task $t^* = (s^*, g^*)$ of her best interest from the released ones, and then plan a path from her current location to $g^*$ through $s^*$.
%Mapping to our settings, an agent becomes idle only when she arrives at a pickup port, and shall be assigned one delivery port from the candidate ones, say $\{g_1, \cdots, g_k\}$.
%Suppose the system will simply pair each delivery port with a pickup port immediately, for which the particular agent will return to after the delivery. Then it is equivalent to, in the language of MAPD, releasing $k$ tasks $\{(g_1, \pi_P(g_1)), \cdots, (g_k, \pi_P(g_k)\}$.
%However, after choosing one from the $k$ tasks and assigning it to an agent, the rest $(k-1)$ tasks will be temporarily removed, or say ``deactivated'', from the pool of released tasks until the next item of the same type arrives.
%\end{remark}


We clarify that in the rest of the paper, by ``path finding'' we mean to compute $\pi_N$, and by ``task assignment'' we mean to compute $\pi_D$.


\section{Path Finding}

\label{sec:pf}

In this section, we first review several representative algorithms that can plan collision-free paths in a lifelong fashion.
However, they are not always effective for resolving collisions under the \texttt{Type}$\odot$ robot model for non-well-formed instances like Figure~\ref{fig:eg_non_wf}.
To this end, we propose a simple-yet-powerful rule-based planner
%that takes advantage of the map layout,
 which is capable of efficiently and robustly moving robots without collisions or deadlocks.

%\begin{definition}[Well-formed instance~\cite{xu2022multi}]
%	The vertices in $P\cup D$ are defined as task endpoints, while the initial locations of agents $\{l_i^0\}_{i \in N}$ are defined as non-task endpoints. A problem instance is well-formed iff (1)~the set of non-task endpoints is disjoint with the set of task endpoints, and (2)~there exists a path between any two endpoints that traverses no other endpoints.
%\label{def:wellform}
%\end{definition}

%\textcolor{red}{why this layout is hard}
%
%\textcolor{red}{not effective examples in Appendix~\ref{apd:pf_notgood_eg}}


\subsection{Existing Lifelong Path Finding Algorithms}

\textbf{Prioritized Planning.}
A straightforward way is to prioritize path finding for each agent by assigning them  distinct priorities, known as \textit{Cooperative A$^*$} (CA$^*$)~\cite{silver2005cooperative}, which can also be extended to lifelong situations.
In descending order of priority, the agents will plan their paths one by one.
Once an agent with a higher priority has found her path, those $(state, time)$ pairs along the path will be \textit{reserved} for this particular agent.
All subsequent agents with lower priorities will view those reservations as states that are unreachable at the corresponding timesteps, i.e. as spatio-temporal obstacles.
Therefore, each agent will need to conduct optimal search over the joint space of state and time.
Understandably,
there is a chance that an agent with a lower priority cannot compute any feasible path given the preceding path computed by a higher-priority, which makes the algorithm itself incomplete.
This situation is even worse under the \texttt{Type}$\odot$ robot model, as an agent often needs to rotate in-place before going to an adjacent vertex, which adds extra difficulty of avoiding collisions. \textit{An illustrative example is provided in Appendix~\ref{apd:pf_notgood_eg}}.

\textbf{Rolling Horizon Collision Resolution.}
A more systematic approach for lifelong path finding is to \textit{window} the search process~\cite{silver2005cooperative}.
This idea is further developed by~\cite{li2021lifelong} as 
the \textit{Rolling Horizon Collision Resolution} (RHCR) framework.
The framework takes two use-specified parameters: (1)~the replanning frequency $h$ and (2)~the length of the collision resolution window $w\geq h$, ensuring that no collisions occur within the next $w$ timesteps.
The framework is general enough to encompass most MAPF algorithms.
An example is to
extend conflict-based search (CBS)~\cite{sharon2015conflict,li2021eecbs} to the lifelong version using this RHCR framework, where the high-level constraint tree is expanded only if there are still collisions within the first $w$ timesteps, resulting in a much smaller constraint tree.
However, under the \texttt{Type}$\odot$ robot model, neighboring agents often require more timesteps to resolve collisions, especially in crowded local areas.
\textit{An example is provided in
Appendix~\ref{apd:pf_notgood_eg}.}

%
%\subsection{Prioritized Planning}
%
%A straightforward way of extending single-agent path finding to multi-agent path finding is to assign each agent a priority ordering, known as \textit{Cooperative A$^*$} (CA$^*$)~\cite{silver2005cooperative}.
%According to descending order of the priorities, the agents will plan their path one by one.
%Once an agent of a higher priority has found her path, those $(state, time)$ pairs along the path will be \textit{reserved} for this particular agent.
%All subsequent agents with lower priorities will see those reservations as states that are unreachable at corresponding timesteps, i.e. as spatio-temporal obstacles.
%In principle, each agent will need to conduct optimal search over the joint space of state-and-time, adding a bit overhead compared with optimal search over only states.
%
%% greedy
%
%Despite of being easy to implement and thus highly extendable, this paradigm is understandably incomplete, as it may not find any collision-free solution given a certain priority ordering whereas the problem instance is indeed solvable.
%A recent study~\cite{ma2019searching} also shows that there exists certain problem instances that are unsolvable for any order of static priorities.
%The authors therefore propose a method called \textit{Priority-Based Search} (PBS), which resembles the two-level method \textit{Conflict-Based Search} (CBS) but searches for a feasible priority ordering in its high level.
%
%\textcolor{red}{incomplete, if higher pri plans first while the lower one got no way to go}
%
%Note that all of the above mentioned algorithms assume that (1) each agent will serve as an obstacle when she reaches the goal, and (2) the system should terminate only when all agents arrives at their goals.
%Adaptations need to be made in situations of lifelong path finding, where agents are repeatedly assigned new goal locations.
%For decoupled algorithms like CA$^*$, one can easily reuse the reservation table for the previously planned paths, and plan new paths for the currently idle agents that are assigned new goals.
%For centralized algorithms like PBS, the central controller has to replan for \textit{all the agents} even if only one agent is assigned a new goal.
%% must be a different one from anyone else
%
%In this paper, we only implement the lifelong CA$^*$ for the \texttt{Type}$\odot$ robot model as one of the baselines of prioritized planning due to its flexibility, later denoted as \textbf{PP}.
%
%
%
%\subsection{Rolling Horizon Collision Resolution}
%A more systematic approach to extending one-shot MAPF algorithms to lifelong ones is to \textit{window} the search process~\cite{silver2005cooperative}.
%This idea is further developed by~\cite{li2021lifelong} as 
%the \textit{Rolling Horizon Collision Resolution} (RHCR) framework.
%The framework takes two use-specified parameters: (1)~the replanning frequency $h$ and (2)~the length of the collision resolution window $w$, ensuring $h \leq w$.
%The framework is general enough to encompass most of the MAPF algorithms, for example
%\begin{enumerate}
%	\item For CA$^*$, the state-time reservations made for the previously planned agents with higher priorities will only be effective within the first $w$ timesteps, and therefore, the size of the reservation table will reduce from $|states| \times T_{max}$ to $|states| \times w$. 
%	\item For CBS, the high-level contraint tree will be extended only if there are still collisions happening within the first $w$ timesteps, and therefore, the depth the constraint tree will be much smaller, similar for PBS.
%\end{enumerate}
%In this paper, we only implement RHCR upon CBS for our \texttt{Type}$\odot$ robot model due to 
%its completeness\footnote{CBS is complete for one-shot MAPF instances. Note that even with certain deadlock avoidance mechanisms mentioned in~\cite{li2021lifelong} the RHCR framework is still incomplete.} in theory, later denoted as \textbf{RHCR-CBS}.
%
%may need to reduce to $w\gets \infty$ when congestion around pickup ports
%

\subsection{Our Solution: Touring With Early Exit}



% traffic jam near pickup ports leads to failures or time-outs for the existing algorithms.




Instead of doing deliberate planning, we here present a simple-yet-effective rule-based planner, named \textit{Touring With Early Exit} (later denoted as \textbf{Touring} for short).
As summarized in Algorithm~\ref{alg:touring},
this planner consists of three main rules \textsc{Rule1-touring}, \textsc{Rule2-early-exit}, and \textsc{Rule3-safe}.
We will explain them one by one.

%\vspace{-3mm}
\begin{algorithm}[tb]
\small
    \caption{Touring with early exit}
    \label{alg:touring}
    \textbf{Input}: $states = (\{l_i\}_{i\in N}, \{d_i\}_{i\in N}, \{g_i\}_{i\in N})$\\
    \textbf{Parameter}: for any arbitrary timestep $k$, omitted below\\
    \textbf{Output}: next joint-action $actions$
    \begin{algorithmic}[1] %[1] enables line numbers
        \STATE $actions \gets \textsc{Rule1-touring}(states)$
        %\textcolor{red}{double-check}
        \IF{$\textsc{Exists-deadlock}(states)$}
        	\STATE $actions \gets \textsc{Rule3-safe}(states, actions)$
        	\STATE \textbf{return} actions
        \ENDIF
        \STATE $actions \gets \textsc{Rule2-early-exit}(states, actions)$
        \STATE $actions \gets \textsc{Rule3-safe}(states, actions)$
        \STATE \textbf{return} actions
    \end{algorithmic}
\end{algorithm}

\textbf{Firstly}, a tour $\tau$ is defined as a simple cycle in the map graph $G$.
Let $V_\tau\subset V$ denote the vertices in $\tau$.
For \textsc{Rule1-touring}, we partition the graph into disjoint tours $\{\tau_p\}_{p\in P}$, ensuring that each tour covers distinct pickup ports, i.e., 
\[
\begin{split}
& \big( \forall p_1, p_2 \in P.\
p_1 \in \tau_1 \land p_2\in \tau_2 \land \tau_1 \neq \tau_2 \iff p_1 \neq p_2 \big) \\
%\land
%& \big( \forall p_1, p_2 \in P.\
%p_1 \in \tau \land p_2\in \tau \implies p_1 = p_2 \big) \\
& \land
\bigcup_{p\in P} V_{\tau_p} = V \land \bigcap_{p\in P} V_{\tau_p} = \emptyset\\
\end{split}
\]
\textsc{Rule1-touring} further specifies the fixed direction along which agents will traverse the tour regardless of their goal locations, i.e. blind touring.
Figure~\ref{fig:touring}(a) shows an example with two tours (in dashed orange), one covering \textsc{F2} clockwise and the other covering \textsc{G2} counter-clockwise.
An agent may need more than one action at certain cells for touring, e.g., need a \texttt{$\circlearrowleft$90} followed by a \texttt{forward} at the corner \textsc{A4}.

\textbf{Secondly}, for each tour $\tau$, $\textsc{Rule2-early-exit}$ marks certain vertices as turnings, where an agent currently in $\tau$ can exit the tour. The set of turnings is denoted as $V_\tau^{turn} \subseteq V_\tau$. 
An agent $i$ can \textit{exit} her tour $\tau$ if she is at the turning
%\textcolor{red}{along the touring direction}
that is the closest to her goal by choosing the next action of her shortest plan towards the goal,
%i.e., $l_i \in \arg\min_{l\in V_\tau^{turn}}\textsc{ShortestDistance}(l, g_i)$, 
or continue touring otherwise. Note that it may not be the case that $g_i \in V_\tau$, which may require agents to go across tours.
An exiting action is prioritized over a touring action.
Two agents who are exiting their own tours simultaneously but moving towards each other will be prioritized by their IDs:
the one with the larger ID will exit, while the other continues touring until reaching the next second-best turning.
The blue cells in Figure~\ref{fig:touring} represent the turnings of the respective tours, with \ref{fig:touring}(b) and \ref{fig:touring}(c) illustrating the two aforementioned prioritized cases.
These turnings can be either user-specified, or searched in terms of minimizing the makespan.
\textit{We provide an illustration of how the frequency of the turnings affects the eventual makespan
in Appendix~\ref{apd:param_search}}.

\textbf{Finally}, \textsc{Rule3-safe}
is applied to revise those actions to collision-free ones.
%Imaginably, if a preceding agent is rotating, then the following agent should not go \texttt{forward}, otherwise collisions will happen.
For example, if a preceding agent is rotating, the following agent should not move \texttt{forward}; otherwise, collisions may occur.
Thus,
we design \textsc{Rule3-safe} conservatively: 
for each agent $i$ (1)~she observes her adjacent agents but assumes their actions specified by the prior rules may or may not be executed successfully, (2)~for either outcome, she checks whether her next action, if it is \texttt{forward}, will lead to a collision, (3) if any potential collision is detected, she revises her action to \texttt{stop}. 
Intuitively, this ensures that every agent maintains a safe distance from one another.
%Each agent is associated with a fixed priority, say her identity number, and each action is also tagged by the number of the last rule that revise it, i.e. 1 or 2.
%The eventual priority is an ordered tuple of (agent-priority, action-priority) and the comparison reduces to the comparison between ordered tuples. Suppose two agents are going against each other, possible cases are (1) if they are both about to cross tours, then only the one with higher agent-priority will make it and the other one continues touring until the next second-best turning; (2) if one is about to turn and the other one is touring, then the latter one with a lower action-priority will wait until the former one finishes her turn.
%\textcolor{red}{actually implies no following-collision~\cite{stern2019multi}}
\textit{In fact, this conservative rule also prevents potential following-collisions} (which will not be discussed in this paper; see~\cite{stern2019multi}).

\textbf{Last but not least},
one may notice that
if there is a subset of agents forming a cycle where each one is about to go to the next location that is currently occupied by another agent in the cycle, \textsc{Rule3-safe} will overwrite the actions of all involved agents to \texttt{stop}, resulting in a deadlock.
Since the planner consisting of only the three main rules is merely a one-step reactive planner, the identical planning step will repeat indefinitely once a deadlock is formed.
Therefore, additional inspections need to be made (Line 2 in Algorithm~\ref{alg:touring}),
within which a depth-first search is conducted to see if any cycles (and thus the deadlock) exist.
 Once a deadlock is found, all the \textit{exiting} agents will be interrupted and resume \textit{touring}.

%
%\textcolor{red}{As we mentioned, in real-world warehouses, agents may encounter a variety of contingencies like failing to execute planed actions or (un)load packages.}
%Hence, \textsc{Rule3-safe} is to simply revise those action profiles that may cause following-conflict to safe ones. Formally, given an action profile $\{a_i\}_{i\in N}$ at timestep $k$,
%if 
%$\exists j.\ l_j^k = move(i, a_i)$, then $a_i$ will be revised to \texttt{stop}.
%Note that this rule will only affect \texttt{forward} actions, as in-place spinning does not move agents to different locations.
%
%Given the above three main rules, the planner is still incomplete as it may stuck in local areas.
%We further polish the rules in two aspects.
%\begin{enumerate}
%	\item \textbf{Priorities.} Each agent is associated with a fixed priority, say her identity number, and each action is also tagged by the last rule that revise it. The eventual priority is an ordered tuple of (agent-priority, action-priority) and the comparison reduces to the comparison between ordered tuples. That is, suppose there are two agents go against each other, (1) if they are both about to cross tours, then only the one with higher agent-priority will make it and the other one continues touring until the next second-best turning; (2) if one is about to turn and the other one is touring, then the latter one with a lower action-priority will wait until the former one finishes her turn.
%
%	\item \textbf{Deadlocks.} \textcolor{red}{like how?} Since the planner is a purely \textcolor{red}{one-step planner} reactive one, understandably it may cause deadlocks. Once a deadlock is formed, it will never be escaped automatically as the identical planning step will repeat forever. Therefore, additional inspections are made (Line 2 in Algorithm~\ref{alg:touring}). Once a deadlock is detected \textcolor{red}{by DFS}, all the turnings will be interrupted and everyone continues touring (along a fixed direction and thus no deadlock).
%\end{enumerate}
%


\begin{figure}[tb]
\vspace{-3mm}
\centering
\includegraphics[width=80mm]{fig/touring}
\caption{Illustration of \textsc{Rule1-Touring} (a) and two prioritized cases (b, c).
%\textcolor{red}{more elaboration}
Colored boxes are the goals.}
\label{fig:touring}
\vspace{-3mm}
\end{figure}

Our \textbf{Touring} planner  eliminates potential collisions by implementing safety rules and avoids deadlocks in real-time. The worst case is to continue touring until the goal is reached.
Hence, our \textbf{Touring} planner is both \textit{sound} and \textit{complete} as
(1)~it will not cause any vertex-collision or edge-collision, %(or even following-collision),
and (2)~every item will be delivered in finite number of steps.

%
%PRIMAL~\cite{sartoretti2019primal,damani2021primal}
%
%\textcolor{red}{Although our experiments only focus on the specific layout, that is because there is only reference value at Meituan. But the design principle is obvious and to some extent general, though simple}


\subsection{Comparison for Path Finding Algorithms}
Before adding task assignment to the context, here we first conduct a brief comparison among the above path finding algorithms, assuming a sequence of goals arrives online.
We implement the lifelong CA$^*$ as a baseline for prioritized planning (denoted as \textbf{PP}),
and CBS under the RHCR framework with $h=1, w=5$ as a baseline for windowed search (denoted as \textbf{RHCR-CBS}).
We also implement two heuristics for the underlying single-agent search, namely $h_{slow}$, which merely computes the Manhattan distance between the current location and the goal, and $h_{fast}$, which additionally counts the minimum number of \texttt{$\circlearrowleft$90}/\texttt{$\circlearrowright$90} needed. Hence, here we have $2\times2=4$ combined baselines. \textit{We report the computing time, even for various scales, in Appendix~\ref{apd:comp_time}.} It turns out \textbf{RHCR-CBS-$h_{slow}$} is too costly for a multi-run evaluation.

As we mentioned, our testing environment (Figure~\ref{fig:eg_non_wf}) may not be well-formed.
\textbf{PP} may fail due to improper priority orderings.
\textbf{RHCR-CBS} may also take a long time for collision resolution, especially when there is a traffic jam near the pickup ports.
We treat a replanning of \textbf{RHCR-CBS} as failure if the number of leaf nodes in the high-level constraint tree exceeds 50, indicating severe congestion.
Once these two methods fail, they will be temporarily switched to \textbf{Touring}, and later be switched back.

In Figure~\ref{fig:pf_vp}, we present the entire distributions of the tested makespans over multiple runs.
As one can clearly see that our \textbf{Touring} planner largely outperforms the other three, and the computing time is entirely negligible compared to the others, as reported in Appendix~\ref{apd:comp_time}.
Besides, \textbf{RHCR-CBS} outperforms \textbf{PP} in most cases, though the average performances are close, as it poorly handles some extreme cases.



%\begin{table}[tb]
%\centering
%\begin{tabular}{@{}lr@{}}
%\toprule
%(50 agents)                      & \textbf{Planning Time (s)} \\ \midrule
%\textbf{Touring (ours)}             &             0.017                    \\
%\textbf{PP $h_{fast}$}       &             0.112               \\
%\textbf{PP $h_{slow}$}       &             0.170               \\
%\textbf{RHCR-CBS $h_{fast}$} &             1.581               \\
%\textbf{RHCR-CBS $h_{slow}$} &             5.995                \\ \bottomrule
%\end{tabular}
%\caption{Average (re)planning time \textit{per step} in 50-agent Meituan warehouse simulation throughout multiple runs.}
%\label{tab:plan_time}
%\end{table}

\begin{figure}[tb]
\vspace{-4mm}
\centering
\includegraphics[width=80mm]{fig/pf_vp}
\caption{The tested makespans of lifelong path finding algorithms in 50-agent Meituan warehouse scenarios. Dotted lines represent the 25-/75-quantiles, and white dots are the means. The means correspond to the leftmost column of the 50-agent scenario in Table~\ref{tab:eval_full}. \underline{416.09} is the reference makespan by Meituan's current system.}
\label{fig:pf_vp}
\vspace{-3mm}
\end{figure}




\section{Task Assignment}
\label{sec:ta}

In the offline setting where tasks are known \textit{a priori}
the assignment problem is well-studied~\cite{ma2016optimal,honig2018conflict,liu2019task,tang2023solving}, 
where the combinatorial search of the best task assignment is coupled with path finding.
However, when it comes to the online setting, it seems that the best approach so far is to greedily assign tasks at each decision point~\cite{ma2017lifelong,okumura2022priority}, i.e., to pick up the task so as to minimize the path costs from the current location to starting location of the task.
Projecting onto our settings, a greedy assignment is to select among those candidates the delivery port that is closest to the current location.
However, no evidence has witnessed that greedy assignments are rational and effective, given the fact that forthcoming tasks are totally unknown.
To this end, we extend this greedy strategy into a broader class of strategies, divided into three categories
(1)~stateless assignment, (2)~adaptive assignment, and (3)~predictive assignment.

%By a classic lesson taken from CPU process scheduling in operating systems,
%an obvious drawback of such greedy assignment is that there might always be certain tasks that keep being preempted. 
%One can also simply argue that committing to a greedy strategy does not make any sense if the future is even unknown.
%To this end, we extend this naive greedy strategy to a broader class of strategies, in three categories (1) stateless assignment, (2) adaptive assignment, and (3) predictive assignment.

\subsection{Stateless Assignment}

%A stateless assignment makes no use of any system-state information.
As shown in Algorithm~\ref{alg:statless}, \textsc{MeasureFunc} is a user-specified function that encodes a measure between the location of the current agent waiting for assignment and the candidate delivery ports. Straightforward options are
\begin{enumerate}
	\item \textbf{Shortest distances}. This reduces to the greedy strategies that select the closest delivery port.
	\item \textbf{Negative shortest distances}. This is equivalent to selecting the farthest delivery port. It is usually counter-intuitive, but makes some sense since it may alleviate congestion around the pickup ports, especially when the scale of the agents is large.
	\item \textbf{Random numbers}. It reduces to random assignments.
\end{enumerate}

%\vspace{-5mm}
\begin{algorithm}[tb]
%\vspace{-5mm}
\small
    \caption{Stateless assignment}
    \label{alg:statless}
    \textbf{Input}: agent $i$ waiting for assignment, item $\iota$ of type $t$, candidate delivery ports $L(t)$\\
    \textbf{Parameter}: any arbitrary timestep $k$ (omitted below)\\
    \textbf{Output}: A selected goal $\in L(t)$ % $g^* \in L(t)$
    \begin{algorithmic}[1] %[1] enables line numbers
%        \STATE $g^* \gets 
        \STATE \textbf{return} $\arg\min_{g \in L(t)} \textsc{MeasureFunc}(g, l_i)$
%        \STATE \textbf{return} $g^*$
    \end{algorithmic}
\end{algorithm}
%\vspace{-3mm}





\subsection{Adaptive Assignment}

Stateless assignments do not make use of system-state information, e.g., the current locations of all agents.
As revealed in Figure~\ref{fig:occu_ratio}, the occupation ratio, defined as the fraction of the number of agents over the number of passable cells,
of the left half differs significantly from that of the right half.
The \textit{closest-first} strategy will inevitably cause high-level congestion around the pickup ports, while \textit{farthest-first} strategy unnecessarily sends agents to farther locations, even though it alleviates traffic jams.
The random strategy somehow
%unconsciously
balances between the former two.

Inspired by this insight, Algorithm~\ref{alg:adaptive} further develops an adaptive version, which takes in a congestion threshold $\alpha$ and makes dynamic assignment decisions based on the current state. If there is a heavy traffic in the right half of the map, the system will send agents to farther goals, and similarly otherwise. One can clearly see in Figure~\ref{fig:occu_ratio}(d) that the occupation ratio fluctuates more responsively.

The threshold parameter $\alpha$ can be specified by users or searched in terms of minimizing the makespan.
\textit{We report comprehensive search results in
Figure~\ref{fig:alpha_bp} in Appendix~\ref{apd:param_search}.}


\begin{algorithm}[t]
\small
    \caption{Adaptive assignment}
    \label{alg:adaptive}
    \textbf{Input}: agent $i$ waiting for assignment, item $\iota$ of type $t$, candidate delivery ports $L(t)$, all agents' locations $\{l_i\}_{i\in N}$\\
    \textbf{Parameter}: A threshold $\alpha$, any timestep $k$ (omitted below)\\
    \textbf{Output}:  A selected goal $\in L(t)$ % $g^* \in L(t)$
    \begin{algorithmic}[1] %[1] enables line numbers
    	\STATE $occu_{l}, occu_{r} \gets \textsc{OccupationRatio}(\{l_i\}_{i\in N})$
    	\IF{$occu_{r} \leq \alpha$}
%    		\STATE $g^* \gets
    		\STATE \textbf{return} $\arg\min_{g \in L(t)} \textsc{ShortestDistance}(g, l_i)$
%    	\ELSIF{s}
%    		\STATE s
		\ELSE
%			\STATE $g^* \gets
			\STATE \textbf{return} $\arg\max_{g \in L(t)} \textsc{ShortestDistance}(g, l_i)$
    	\ENDIF
%        \STATE \textbf{return} $g^*$
    \end{algorithmic}
\end{algorithm}

\begin{figure}[t]
\vspace{-2mm}
\centering
\includegraphics[width=85mm]{fig/occu_ratio}
\caption{The dynamics of the occupation ratios for different assignment strategies in 50-agent cases. Dashed lines represent the means.}
\vspace{-1mm}
\label{fig:occu_ratio}
\vspace{-1mm}
\end{figure}


%%%%%%%%%%%%%%%%%%%%%%%%%%%%%%%%%%%%%
%\vspace{5mm}
\begin{table*}[tb]
\vspace{-6mm}
\small
\begin{tabular}{@{}lrrrrrrrrr@{}}
\toprule
(30 agents)                  & \textbf{Random}      & \textbf{Closest}      & \textbf{Farthest}     & \textbf{Adapt$^{1st}$} & \textbf{Adapt$^{2nd}$} & \textbf{Adapt$^{3rd}$} & \textbf{RL$^{avg}$}  & \textbf{RL$^{best}$}  & \textbf{RL$^{worst}$}  \\ \midrule
\textbf{Touring (ours)}      & 467.00               & \textcolor{orange}{$\textbf{412.90}$}               & 530.85               & $443.45^{0.158}$       & $454.15^{0.152}$       & $454.15^{0.155}$       & 425.00                  & $425.00^{412}$          & $425.00^{439}$          \\
\textbf{PP $h_{slow}$}       & 659.45               & $\textbf{550.75}$               & 678.50               & $587.70^{0.158}$        & $590.10^{0.149}$        & $605.30^{0.146}$        & \cellcolor[HTML]{EFEFEF}569.45               & \cellcolor[HTML]{EFEFEF}$569.45^{482}$       & \cellcolor[HTML]{EFEFEF}$569.45^{762}$       \\
\textbf{PP $h_{fast}$}       & 641.30               & $\textbf{561.50}$               & 681.50               & $589.10^{0.152}$        & $589.50^{0.155}$        & $610.20^{0.149}$        & \cellcolor[HTML]{EFEFEF}588.40                & \cellcolor[HTML]{EFEFEF}$588.40^{492}$        & \cellcolor[HTML]{EFEFEF}$588.40^{800}$        \\
\textbf{RHCR-CBS $h_{fast}$} & 645.00               & $\textbf{539.75}$               & 726.05               & $645.20^{0.152}$        & $646.20^{0.155}$        & $655.10^{0.146}$        & \cellcolor[HTML]{EFEFEF}641.95               & \cellcolor[HTML]{EFEFEF}$641.95^{495}$       & \cellcolor[HTML]{EFEFEF}$641.95^{800}$     \vspace{1.5mm}  \\
%                             &                      &                      &                      &                        &                        &                        &                      &                      &                      \\
(40 agents)                  &                      &                      &                      &                        &                        &                        &                      &                      &                      \\ \midrule
\textbf{Touring (ours)}      & 392.10               & 376.30               & 422.70               & $382.40^{0.219}$        & $387.30^{0.211}$        & $387.30^{0.215}$        & \textcolor{orange}{$\textbf{372.05}$}               & $372.05^{348}$       & $383.65^{399}$       \\
\textbf{PP $h_{slow}$}       & 474.50               & $\textbf{427.10}$               & 518.35               & $443.50^{0.215}$        & $447.20^{0.211}$        & $449.05^{0.207}$       & \cellcolor[HTML]{EFEFEF}452.80                & \cellcolor[HTML]{EFEFEF}$452.80^{425}$        & \cellcolor[HTML]{EFEFEF}$473.90^{565}$        \\
\textbf{PP $h_{fast}$}       & 467.00               & $\textbf{426.70}$               & 516.40               & $443.70^{0.219}$        & $449.15^{0.215}$       & $451.25^{0.211}$       & \cellcolor[HTML]{EFEFEF}445.20                & \cellcolor[HTML]{EFEFEF}$445.20^{417}$        & \cellcolor[HTML]{EFEFEF}$475.45^{535}$       \\
\textbf{RHCR-CBS $h_{fast}$} & 463.00               & 444.95               & 523.75               & $438.85^{0.211}$       & $438.85^{0.215}$       & $443.10^{0.219}$        & \cellcolor[HTML]{EFEFEF}$\textbf{431.55}$               & \cellcolor[HTML]{EFEFEF}$431.55^{394}$       & \cellcolor[HTML]{EFEFEF}$447.05^{481}$    \vspace{1.5mm}   \\
%                             &                      &                      &                      &                        &                        &                        &                      &                      &                      \\
\textcolor{teal}{\textbf{(50 agents)}}         &                      &                      &                      &                        &                        &                        &                      &                      &                      \\ \midrule
\textcolor{teal}{\textbf{Touring (ours)}}      & 362.55               & 358.35               & 375.15               & \textcolor{orange}{$\textbf{348.55}^{0.235}$}       & $349.35^{0.265}$       & $349.80^{0.240}$         & 350.15               & $352.70^{316}$        & $350.15^{363}$       \\
\textcolor{teal}{\textbf{PP $h_{slow}$}}       & 424.65               & 392.85               & 435.45               & $\textbf{388.35}^{0.280}$        & $392.65^{0.275}$       & $400.55^{0.255}$       & \cellcolor[HTML]{EFEFEF}409.95               & \cellcolor[HTML]{EFEFEF}$397.10^{359}$        & \cellcolor[HTML]{EFEFEF}$409.95^{509}$       \\
\textcolor{teal}{\textbf{PP $h_{fast}$}}       & 410.70               & $\textbf{390.15}$               & 434.20               & $396.70^{0.280}$         & $399.40^{0.265}$        & $400.15^{0.260}$        & \cellcolor[HTML]{EFEFEF}398.25               & \cellcolor[HTML]{EFEFEF}$402.70^{361}$        & \cellcolor[HTML]{EFEFEF}$398.25^{444}$       \\
\textcolor{teal}{\textbf{RHCR-CBS $h_{fast}$}} & 409.60               & 401.00               & 415.90               & $384.25^{0.280}$        & $385.20^{0.275}$        & $386.10^{0.265}$        & \cellcolor[HTML]{EFEFEF}384.90                & \cellcolor[HTML]{EFEFEF}$\textbf{382.20}^{363}$        & \cellcolor[HTML]{EFEFEF}$384.90^{468}$    \vspace{1.5mm}    \\
%                             &                      &                      &                      &                        &                        &                        &                      &                      &                      \\
(60 agents)                  & \multicolumn{1}{l}{} & \multicolumn{1}{l}{} & \multicolumn{1}{l}{} & \multicolumn{1}{l}{}   & \multicolumn{1}{l}{}   & \multicolumn{1}{l}{}   & \multicolumn{1}{l}{} & \multicolumn{1}{l}{} & \multicolumn{1}{l}{} \\ \midrule
\textbf{Touring (ours)}      & 350.60                & 352.50                & 352.40                & \textcolor{orange}{$\textbf{335.50}^{0.281}$}        & $337.10^{0.293}$        & $337.10^{0.299}$        & 342.70               & $342.70^{308}$        & $342.70^{359}$        \\
\textbf{PP $h_{slow}$}       & 390.90                & 380.45               & 411.00                & $\textbf{369.35}^{0.287}$       & $370.2^{0.293}$        & $373.10^{0.329}$        & \cellcolor[HTML]{EFEFEF}375.10                & \cellcolor[HTML]{EFEFEF}$375.10^{345}$        & \cellcolor[HTML]{EFEFEF}$375.10^{403}$        \\
\textbf{PP $h_{fast}$}       & 394.15               & 382.80                & 397.15               & $\textbf{371.75}^{0.299}$       & $372.65^{0.329}$       & $378.45^{0.311}$       & \cellcolor[HTML]{EFEFEF}391.05               & \cellcolor[HTML]{EFEFEF}$391.05^{356}$       & \cellcolor[HTML]{EFEFEF}$391.05^{499}$       \\
\textbf{RHCR-CBS $h_{fast}$} & 372.50                & 370.00                & 375.90                & $\textbf{357.35}^{0.305}$       & $360.55^{0.287}$       & $360.85^{0.323}$       & \cellcolor[HTML]{EFEFEF}372.85               & \cellcolor[HTML]{EFEFEF}$372.85^{354}$       & \cellcolor[HTML]{EFEFEF}$372.85^{469}$   \vspace{1.5mm}    \\
%                             &                      &                      &                      &                        &                        &                        &                      &                      &                      \\
(70 agents)                  &                      &                      &                      &                        &                        &                        &                      &                      &                      \\ \midrule
\textbf{Touring (ours)}      & 346.45               & 354.65               & 344.50                & \textcolor{orange}{$\textbf{333.40}^{0.353}$}        & $333.60^{0.339}$        & $334.10^{0.360}$         & 338.80                & $338.80^{308}$        & $338.80^{354}$        \\
\textbf{PP $h_{slow}$}       & 375.95               & 381.15               & 393.85               & $374.95^{0.325}$       & $375.40^{0.388}$        & $375.95^{0.304}$       & \cellcolor[HTML]{EFEFEF}$\textbf{373.50}$                & \cellcolor[HTML]{EFEFEF}$373.50^{347}$        & \cellcolor[HTML]{EFEFEF}$373.50^{394}$        \\
\textbf{PP $h_{fast}$}       & 371.25               & 372.10                & 372.10                & $\textbf{364.95}^{0.332}$       & $364.95^{0.360}$        & $365.35^{0.304}$       & \cellcolor[HTML]{EFEFEF}390.90                & \cellcolor[HTML]{EFEFEF}$390.90^{340}$        & \cellcolor[HTML]{EFEFEF}$390.90^{513}$        \\
\textbf{RHCR-CBS $h_{fast}$} & 362.50                & 377.20                & 365.55               & $\textbf{351.90}^{0.367}$        & $353.30^{0.353}$        & $354.10^{0.381}$        & \cellcolor[HTML]{EFEFEF}362.20                & \cellcolor[HTML]{EFEFEF}$362.20^{337}$        & \cellcolor[HTML]{EFEFEF}$362.20^{435}$        \\ \bottomrule
\end{tabular}
\vspace{-1mm}
\caption{Evaluation results. The numbers are the average makespans ($\downarrow$).
The reference makespan is \underline{416.09} by the currently deployed system at Meituan. 
For each path planner, the result performed by the best task assigner is marked in \textbf{bold}. 
The scenario in \textcolor{teal}{teal} is the current scale at Meituan. 
Those in \textcolor{orange}{orange} represent the best combinations at each scale.
Some cells are marked in \textcolor{gray}{gray} as the RL policies are not explicitly trained for those scenarios.
For the adaptive assignment strategies, we report the top-3 ones along with the corresponding thresholds in superscripts.
For the RL strategies, besides the best ones of average performances, we also present the best ones of best-case (resp. worst-case) performances, along with the corresponding best-case (resp. worst-case) makespans in superscripts.}
\label{tab:eval_full}
\vspace{-1mm}
\end{table*}
%%%%%%%%%%%%%%%%%%%%%%%%%%%%%%%%%%%%%

\subsection{Predictive Assignment}
\label{sec:ta_rl} 
One can further argue that purely reactive assignments like the above ones do not capture the dynamics of the system.
%A natural debate is, \textit{even if the system has monitored that a certain area is currently of low-level congestion, (1) is it a good idea to send agents there, and (2) will it be still vacuum when the agent gets there?}
To this end, one needs to make good use of the underlying path finding module which may hint about the dynamic flow of the agent swarm.

A systematic way is to formulate the assignment problem as a Markov Decision Process (MDP), taking the path finding module, i.e. the routing policy $\pi_N$, as a hyperparameter.
The MDP is defined as a 5-tuple
$<\mathcal{S}, \mathcal{A}, T, R, \gamma>$:
\begin{enumerate}
	\item \textbf{States} $\mathcal{S}$: each $S \in \mathcal{S} \subset \Psi$ is a collection of all system-states where there exists an agent at a pickup port waiting for a new-delivery assignment. We call these \textit{assignment states} to avoid ambiguity.
%	\item \textbf{States} $\mathcal{S}$: each $S \in \mathcal{S}$ is a collection of all individual agents' instantaneous local states, including their locations, directions, etc. Not all such joint-states are in $\mathcal{S}$, only those where there exists an agent waiting for task assignment are valid ones. We call them \textit{assignment states} to avoid ambiguity.
	\item \textbf{Actions} $\mathcal{A} = D$: all possible delivery ports. Given a loaded item of type $t$, the available actions are the delivery ports in $L(t)$.
%	\item \textbf{Transition function} $T: \mathcal{S}\times \mathcal{A} \mapsto \Delta(\mathcal{S})$. Once a valid task is assigned to an idle agent, the system will proceed according to the underlying path finding module until the next assignment state. The transition is implicitly stochastic since there might be intermediate contingencies between two assignment states, e.g., failing to execute plans or unload packages by certain robots.
	\item \textbf{Transition function} $T: \mathcal{S}\times \mathcal{A} \mapsto \Delta(\mathcal{S})$. Once a new delivery port is assigned to an agent, the system will proceed according to $\pi_N$ (and the given $\pi_P$) until the next assignment state. 
	\item \textbf{Reward function} $R: \mathcal{S}\times \mathcal{A} \times \mathcal{S} \mapsto \mathbb{R}$. The reward is the negative time cost spent between two assignment states. Note that the reward signals are quite ``delayed'', in the sense that for two adjacent assignment states, the immediate reward received at the latter one might not reflect the delivery cost for the task assigned in the former state (it is usually not delivered yet). Nevertheless, the total accumulated rewards of an episode is indeed the negative makespan to complete the attended item sequence.
	\item \textbf{Discount factor} $\gamma \in (0, 1)$.
\end{enumerate}
Note that to solve the above MDP is to search for the policy $\pi_D$ while fixing $\pi_N$.
%The solution concept here is an assignment (possibly randomized) policy $\pi: \mathcal{S}\mapsto \mathcal{A}$ that maximizes the expected accumulated rewards. 
By definition, once the optimal policy $\pi_D^*$ is found, it will instruct the system to assign tasks at any possible assignment state, and therefore, the initial resting locations of agents
%(where the system is launched from)
do not matter.
We adopt PPO~\cite{schulman2017proximal} as the RL algorithm to solve the above MDP, \textit{and will postpone further training details to Appendix~\ref{apd:rl_train}}. 

% online?





\section{Experimental Results}
\label{sec:exp}

%\textcolor{red}{why ours is way better: coincide with universal plans}

In this section, we report the main experimental results in Table~\ref{tab:eval_full}, conducted on Meituan simulated warehouse scenarios.
The online sequences of items are retrieved from roughly 5-minute segments of the system's log containing around $140$ items % 135
of approximately 50 types, % 46
while the demand database $L$ is made from mobile orders made by the customers in a longer period of around 6 hours. % 23:40-17:52
The two dimensions of this table represent the choices of path planners and task assigners, respectively.
%\textcolor{red}{tested by Meituan log data}
The numbers are the average \textit{makespans} over multiple runs launched with agents initialized in random locations.
In each run, the system is required to deliver a sequence of items of a fixed length that arrives online.
In other words, we evaluate the average cost it takes to accomplish the same amount of throughput, for each pair of path planners and task assigners.
For the adaptive assignment strategies, we report the top-3 ones along with the corresponding thresholds in superscripts.
For the RL strategies, besides the best ones for average performances, we also present the best ones for best-case (resp. worst-case) performances, along with the corresponding best-case (resp. worst-case) makespans in superscripts.
We only train RL policies over the \textbf{Touring} planner, since the others are not fast enough and will take a tremendous amount of time for RL training.
However, we can slightly abuse a trained assignment policy by testing it with the other three path planners as the state spaces are same.

\underline{As an overview},
our \textbf{Touring} planner outperforms the other three regardless of the task assigner.
We highly conjecture that this planner, to some extent, coincides with a near-optimal universal plan~\cite{zhu2023computing}; \textit{see more discussion in Appendix~\ref{apd:more_related_work}}.
As for the task assigner,
(1)~when the number of agents is $\geq 50$,
adaptive strategies are surprisingly effective, even slightly better than RL ones, and the \textit{closest-first} strategy is not necessarily better than the \textit{farthest-first} strategy.
(2)~when the number of agents is $< 50$,
it might be redundant to use adaptive strategies as the density of agents is quite low; instead, stateless ones or RL ones are better choices.
Although RL strategies can achieve comparable performances in practice (even optimal performance in theory if trained well),
it depends on the user whether the training cost is a worthwhile effort.

%\textcolor{red}{farthest v.s. closest not much difference}

\underline{Looking closer}, we point out two insights:
\begin{enumerate}
\item \textit{Time efficiency}.
Regarding the current scale of Meituan (50 agents),
our system only needs 83.77\% of the makespan to deliver the same amount of throughput, compared to their current system $^{(348.55 / 416.09)}$, outperforming the best$^{(382.20 / 416.09)}$ among the rest by 8.09\%.

\item \textit{Economic efficiency}.
Note that there is a continuing improvement$^{(348.55 \rightarrow 335.50)}$ while increasing the number of agents to 60. However, the marginal gain of further increasing to 70 agents is negligible$^{(335.50 \rightarrow 333.40)}$.
In fact, only 30 agents can fulfill the current throughput with even slightly shorter time$^{(412.90)}$, resulting in a 40\% reduction in fixed costs for purchasing robots. 

\end{enumerate}

%	\item \textbf{RQ1} (time efficiency): Given the currently deployed layout with 50 agents, what is the best combined option of those path finding and task assignment modules?
%	\item \textbf{RQ2} (economic efficiency): Is it profitable to deploy more agents to further increase the throughput, or the other way around, to deploy less agents while still maintaining the same throughput?
%
%Regarding \textbf{RQ1},
%
%Regarding \textbf{RQ2},

%\textcolor{red}{why ours is way better: coincide with universal plans}



\section{Conclusion and Discussion}
\label{sec:conclusion}



In this paper, we conduct a case study on the real-world problem of warehouse automation by combining lifelong multi-robot path finding and dynamic task assignment in an online fashion.
As a result, we manage to speed up package delivery given the current scale at Meituan,
and also identify potential profitable upgrades of the system.

An important lesson from this study is that given the layout of the warehouse, once deployed, is normally fixed in a relatively long period of time, it is worthwhile to have both the routing module and the assignment module that take advantage of the layout.
However, both modules should be general enough to account for the varying number of robots available.

A limitation of this work is that we search an assignment policy with respect to a fixed routing policy, which is an open-loop control. A natural next step is to couple the search of these two, though it will be computationally challenging.

%PRIMAL
%
%Another lesson is 
%
%1. The design of warehouse layouts should not be oblivious to the design of robot control strategies in later phases.
%
%2. policy search for a specific warehouse
%
%draw two lessons 
%
%robust RL
%
%Contextual RL

%\section*{Ethical Statement}
%
%There are no ethical issues.

%\section*{Acknowledgments}
%
%The preparation of these instructions and the \LaTeX{} and Bib\TeX{}
%files that implement them was supported by Schlumberger Palo Alto
%Research, AT\&T Bell Laboratories, and Morgan Kaufmann Publishers.
%Preparation of the Microsoft Word file was supported by IJCAI.  An
%early version of this document was created by Shirley Jowell and Peter
%F. Patel-Schneider.  It was subsequently modified by Jennifer
%Ballentine, Thomas Dean, Bernhard Nebel, Daniel Pagenstecher,
%Kurt Steinkraus, Toby Walsh, Carles Sierra, Marc Pujol-Gonzalez,
%Francisco Cruz-Mencia and Edith Elkind.

%\clearpage
%% The file named.bst is a bibliography style file for BibTeX 0.99c
\bibliographystyle{named}
\bibliography{ijcai25}


\newpage
\onecolumn
\appendix

\section{Relation to MAPD}
\label{apd:relate_to_mapd}


In MAPD, an online task $t_i$ is characterized by a pickup port $s_i$ and a delivery port $g_i$ with a priorly unknown release time. Once an agent becomes idle, she will select one task $t^* = (s^*, g^*)$ of her best interest from the released ones, and then plan a path from her current location to $g^*$ through $s^*$.
Mapping to our settings, an agent becomes idle only when she arrives at a pickup port, and shall then be assigned one delivery port from the candidates, say $\{g_1, \cdots, g_k\}$.
Suppose the system will simply pair each delivery port with a pickup port immediately, for which the particular agent will return to after the delivery. Then it is equivalent to, in the language of MAPD, releasing $k$ tasks $\{(g_1, \pi_P(g_1)), \cdots, (g_k, \pi_P(g_k)\}$.
However, after choosing one from the $k$ tasks and assigning it to an agent, the rest $(k-1)$ tasks will be temporarily removed, or ``deactivated'', from the pool of released tasks until the next item of the same type arrives.


\section{More Discussion on Related Work}
\label{apd:more_related_work}

The problem presented in this paper pertains to some other important areas in planning and operations research.

\textbf{Universal Planning.}
Unlike the idealized one-shot MAPF, fully automating real-world warehouses requires lifelong path finding.
However, most of the existing work~\cite{ma2017lifelong,vsvancara2019online,li2021lifelong,xu2022multi} still focuses on the solution concept as a set of collision-free paths, which is a sequence of joint actions.
Such a solution concept is vulnerable if there is any uncertainty, e.g. unknown future goals, or even system contingencies.
We argue that one can turn to the solution concept of universal plans~\cite{schoppers1987universal,ginsberg1989universal}.
Although universal plans are even harder to compute, there are some exemplars using multi-agent reinforcement learning~\cite{sartoretti2019primal,damani2021primal}, or via reduction to logic programs~\cite{zhu2023computing}.

% 3. comb. opt., scheduling 
% Job shop scheduling, vehicle routing
\textbf{Scheduling.}
One may also notice the analogy between TAPF and job-shop scheduling problems (JSSP)~\cite{manne1960job,jain1999deterministic} or vehicle routing problems (VRP)~\cite{toth2014vehicle,braekers2016vehicle}.
However, there are at least two key differences: (1) job durations in JSSP and route lengths in VRP are usually known in advance and (2) the execution of jobs or routes is independent of each other.
Neither of these two conditions holds in TAPF, especially when the tasks are released online.


\section{Side Effects of the \texttt{Type}$\odot$ Robot Model}
\label{apd:pf_notgood_eg}

When the state of an agent is lifted from pure locations to (location, direction) pairs, there will be extra difficulty resolving collisions. We here show three examples for (a)~\textit{cooperative A$^*$} (CA$^*$)~\cite{silver2005cooperative}, (b)~\textit{conflict-based search} (CBS)~\cite{sharon2015conflict}, and (c)~\textit{priority inheritance with backtracking} (PIBT)~\cite{okumura2022priority}, respectively.
\begin{enumerate}
	\item Figure~\ref{fig:pf_notgood_eg}(a) shows a case where CA$^*$ fails for the \texttt{Type}$\odot$ robot model. Suppose agent 2 is prioritized over agent 1, then agent 1 will move away immediately under the \texttt{Type}$\oplus$ robot model. However, under the \texttt{Type}$\odot$ robot model, agent 1 has to rotate first and thus cannot manage to avoid collision at the very next timestep.
	\item Figure~\ref{fig:pf_notgood_eg}(b) shows a case where it takes CBS a longer time  to resolve collisions under the \texttt{Type}$\odot$ robot model. The main reason is still due to the rotational cost. Similarly for the execution of \textit{priority-based search} (PBS)~\cite{ma2019searching}. 
% move to appendix?
%Another successful exemplar in designing dynamic priorities is to toggle priority update and inheritance at each planning step, bringing the solver named \textit{Priority Inheritance With Backtracking} (PIBT)~\cite{okumura2022priority}.
	\item Figure~\ref{fig:pf_notgood_eg}(c) shows a failed case due to deeper theoretical reasons.
	Instead of performing any best-first search, PIBT repeats one-timestep planning until the terminal state, and therefore, it needs a crucial lemma to make sure the total number of execution steps is always bounded (see \textbf{Lemma 1} in~\cite{okumura2022priority}), i.e., \textit{at each timestep the agent with the highest priority will manage to move one step closer to her goal}.
Nevertheless, when the states of each agent are lifted from only locations to (location, direction) pairs, this lemma no longer holds as a counter-example is provided in Figure~\ref{fig:pf_notgood_eg}(c).
\end{enumerate}



%\begin{figure}[!ht]
%\includegraphics[width=110mm]{fig/pp}
%
%\includegraphics[width=176mm]{fig/cbs}
%
%\includegraphics[width=132mm]{fig/pibt}
%\caption{A counter-example that shows the reachability lemma holds for the \texttt{Type}$\oplus$ robot model (upper), but fails for the \texttt{Type}$\odot$ robot model (lower).}
%\label{fig:pibt_fail}
%\end{figure}

\begin{figure*}[!ht]
%    \centering
    % First subfigure
    \begin{subfigure}[b]{\textwidth}
        \includegraphics[width=110mm]{fig/pp} 
        \caption{Cooperative A$^*$ may fail for the \texttt{Type}$\odot$ Robot Model}
    \end{subfigure}
    \vspace{0.5mm}
    
    % Second subfigure
    \begin{subfigure}[b]{\textwidth}
        \includegraphics[width=176mm]{fig/cbs}
        \caption{Conflict-based search works but takes more timesteps (same execution results by priority-based search in this particular case).}
    \end{subfigure}
    \vspace{0.5mm}
    
    % Third subfigure
    \begin{subfigure}[b]{\textwidth}
        \includegraphics[width=132mm]{fig/pibt}
        \caption{PIBT fails.}
    \end{subfigure}


    \caption{Examples showing the added difficulty of resolving collisions with the \texttt{Type}$\odot$ robot model.}
    \label{fig:pf_notgood_eg}
\end{figure*}

%\newpage
\section{Computing Time}
\label{apd:comp_time}

We report the planning time per step in Table~\ref{tab:plan_time}.
Experiments are conducted on a
MacBook Air with Apple M2 CPU and 16 GB memory.
The planners are all implemented in Python, therefore, those numbers are merely for relative comparisons within this work.

\begin{table}[!ht]
\centering
\begin{tabular}{@{}lrrrrr@{}}
\toprule
                             & \textbf{30} & \textbf{40} & \textbf{50} & \textbf{60} & \textbf{70} \\ \midrule
\textbf{Touring (ours)}      & $\textbf{0.008}$       & $\textbf{0.012}$       & $\textbf{0.018}$       & $\textbf{0.025}$       & $\textbf{0.032}$       \\
\textbf{PP $h_{fast}$}       & 0.066       & 0.115       & 0.225       & 0.500       & 0.713       \\
\textbf{PP $h_{slow}$}       & 0.169       & 0.287       & 0.448       & 0.892       & 1.028       \\
\textbf{RHCR-CBS $h_{fast}$} & 0.765       & 0.718       & 1.842       & 2.642       & 2.448       \\
\textbf{RHCR-CBS $h_{slow}$} & 3.023       & 3.070       & 7.081       & 7.821       & 8.952       \\ \bottomrule
\end{tabular}
\caption{Planning time per step in seconds, implemented in Python.}
\label{tab:plan_time}
\end{table}


%\newpage
\section{Parameter Search}
\label{apd:param_search}
In the design of both the \textbf{Touring} and adaptive task assignment, there are certain hyper-parameters.
We here show how the best option is searched in terms of minimizing the eventual makespan.
\begin{enumerate}
	\item Turning frequency (Figure~\ref{fig:turning_freq}). Figure~\ref{fig:eg_non_wf} has presented the extreme where every possible cell that can be a turning is set as a turning, i.e., of frequency 1. One can gradually ``sparsify'' the turnings to see if the overall makespan gets worse. It turns out, the more turnings you have, the better the makespan on average will be. 
	\item Adaptive Threshold (Figure~\ref{fig:alpha_bp}). As the occupation ratio is defined as the number of agents over the number of passable cells in that part of area, the spectrum of tested thresholds in $N$-agent scenarios will be considerably less than those in $N'$-agent scenarios if $N < N'$.
	One can clearly observe that our \textbf{Touring} planner significantly outperforms the other three, and the threshold that makes the lowest box plot is the most desired one. Another observation from Figure~\ref{fig:alpha_bp} is that ours is also more stable than the other three, as the variations (the length of those boxes) are relatively small in most cases.
\end{enumerate}


%\subsection{Turning Frequencies}
\begin{figure*}[!ht]
\centering
\includegraphics[width=170mm]{fig/turning_freq}
\caption{Makespans over different turning frequency in various scales of agents in Meituan warehouse simulation. The X-axis means ``there will be a turning every $x$ cells''.}
\label{fig:turning_freq}
\end{figure*}



%\newpage
%\subsection{Adaptive Thresholds}

\begin{figure*}[!ht]
\centering
\includegraphics[width=167mm]{fig/alpha_bp_30}
\includegraphics[width=167mm]{fig/alpha_bp_40}
\includegraphics[width=167mm]{fig/alpha_bp_50}
\includegraphics[width=167mm]{fig/alpha_bp_60}
\includegraphics[width=167mm]{fig/alpha_bp_70}
\caption{The box-plot of makespans over different adaptive thresholds with various scales of agents in Meituan warehouse simulation.}
\label{fig:alpha_bp}
\end{figure*}


\newpage
\section{RL Training Details}
\label{apd:rl_train}

Here we reveal the details of RL training skipped in Section~\ref{sec:ta_rl}.

\textbf{Actions.} We directly mask out unavailable actions (those delivery ports that do not need the item) at each assignment state, instead of signaling large negative rewards. In principle, these two are equivalent in terms of the value of the eventual optimal policy, but the former one will guide the policy optimization to converge faster~\cite{huang2022closer}.

\textbf{State features.} As defined in Section~\ref{sec:ta_rl}, assignment states contain necessary information from system-states. Here we make each state of size $num\_of\_agents\times (2 + 1)$,
	which means to mark each agent's location and direction (converted to [0, 90, 80, 270]). The location feature is further normalized by the layout shape, and the direction feature is normalized by 360.
	
\textbf{Episodes.} We train the RL agents over one set of item sequences while evaluate it over another set of item sequences.


\textbf{Hyper-parameters.} Both the value network and the policy network are MLPs of size $H \times H \times H \times H$ followed by respective value/policy heads.
We attach some training samples in Figure~\ref{fig:rl_log}, for $H$ chosen from [128, 256, 1024]. The number of total training steps can also be seen in this figure. It turns out networks with $H=1024$ tend to overfit in most cases.

%\subsection{Training Protocol}
%
%\subsection{Different Network Sizes}
\begin{figure*}[!ht]
\centering
\includegraphics[width=170mm]{fig/train_log_30a}
\includegraphics[width=170mm]{fig/train_log_40a}
\includegraphics[width=170mm]{fig/train_log_50a}
\includegraphics[width=170mm]{fig/train_log_60a}
\includegraphics[width=170mm]{fig/train_log_70a}
\caption{Training processes for different network sizes in various scales of agents. }
\label{fig:rl_log}
\end{figure*}

\newpage
\section{Running Example Recordings}
We attach three video clips for the \textbf{Touring} planner coupled with the closest-first assigner, the adaptive assigner with $\alpha~=~0.235$, and the RL assignment with the best best-case performance, respectively.
\begin{enumerate}
	\item \texttt{warehouse\_50\_touring\_closest.mp4} with makespan 355.
	\item \texttt{warehouse\_50\_touring\_alpha0235.mp4} with makespan 325.
	\item \texttt{warehouse\_50\_touring\_rl.mp4} with makespan 316.
\end{enumerate}
In all the above instances, the initial states are the same, i.e. the corresponding agents are in the same locations and towards the same directions at the beginning, and the online item sequences are also the same.

\end{document}







\clearpage

%\title{Generating 3D \hl{Small} Binding Molecules Using Shape-Conditioned Diffusion Models with Guidance}
%\date{\vspace{-5ex}}

%\author{
%	Ziqi Chen\textsuperscript{\rm 1}, 
%	Bo Peng\textsuperscript{\rm 1}, 
%	Tianhua Zhai\textsuperscript{\rm 2},
%	Xia Ning\textsuperscript{\rm 1,3,4 \Letter}
%}
%\newcommand{\Address}{
%	\textsuperscript{\rm 1}Computer Science and Engineering, The Ohio Sate University, Columbus, OH 43210.
%	\textsuperscript{\rm 2}Perelman School of Medicine, University of Pennsylvania, Philadelphia, PA 19104.
%	\textsuperscript{\rm 3}Translational Data Analytics Institute, The Ohio Sate University, Columbus, OH 43210.
%	\textsuperscript{\rm 4}Biomedical Informatics, The Ohio Sate University, Columbus, OH 43210.
%	\textsuperscript{\Letter}ning.104@osu.edu
%}

%\newcommand\affiliation[1]{%
%	\begingroup
%	\renewcommand\thefootnote{}\footnote{#1}%
%	\addtocounter{footnote}{-1}%
%	\endgroup
%}



\setcounter{secnumdepth}{2} %May be changed to 1 or 2 if section numbers are desired.

\setcounter{section}{0}
\renewcommand{\thesection}{S\arabic{section}}

\setcounter{table}{0}
\renewcommand{\thetable}{S\arabic{table}}

\setcounter{figure}{0}
\renewcommand{\thefigure}{S\arabic{figure}}

\setcounter{algorithm}{0}
\renewcommand{\thealgorithm}{S\arabic{algorithm}}

\setcounter{equation}{0}
\renewcommand{\theequation}{S\arabic{equation}}


\begin{center}
	\begin{minipage}{0.95\linewidth}
		\centering
		\LARGE 
	Generating 3D Binding Molecules Using Shape-Conditioned Diffusion Models with Guidance (Supplementary Information)
	\end{minipage}
\end{center}
\vspace{10pt}

%%%%%%%%%%%%%%%%%%%%%%%%%%%%%%%%%%%%%%%%%%%%%
\section{Parameters for Reproducibility}
\label{supp:experiments:parameters}
%%%%%%%%%%%%%%%%%%%%%%%%%%%%%%%%%%%%%%%%%%%%%

We implemented both \SE and \methoddiff using Python-3.7.16, PyTorch-1.11.0, PyTorch-scatter-2.0.9, Numpy-1.21.5, Scikit-learn-1.0.2.
%
We trained the models using a Tesla V100 GPU with 32GB memory and a CPU with 80GB memory on Red Hat Enterprise 7.7.
%
%We released the code, data, and the trained model at Google Drive~\footnote{\url{https://drive.google.com/drive/folders/146cpjuwenKGTd6Zh4sYBy-Wv6BMfGwe4?usp=sharing}} (will release to the public on github once the manuscript is accepted).

%===================================================================
\subsection{Parameters of \SE}
%===================================================================


In \SE, we tuned the dimension of all the hidden layers including VN-DGCNN layers
(Eq.~\ref{eqn:shape_embed}), MLP layers (Eq.~\ref{eqn:se:decoder}) and
VN-In layer (Eq.~\ref{eqn:se:decoder}), and the dimension $d_p$ of generated shape latent embeddings $\shapehiddenmat$ with the grid-search algorithm in the 
parameter space presented in Table~\ref{tbl:hyper_se}.
%
We determined the optimal hyper-parameters according to the mean squared errors of the predictions of signed distances for 1,000 validation molecules that are selected as described in Section ``Data'' 
in the main manuscript.
%
The optimal dimension of all the hidden layers is 256, and the optimal dimension $d_p$ of shape latent embedding \shapehiddenmat is 128.
%
The optimal number of points $|\pc|$ in the point cloud \pc is 512.
%
We sampled 1,024 query points in $\mathcal{Z}$ for each molecule shape.
%
We constructed graphs from point clouds, which are employed to learn $\shapehiddenmat$ with VN-DGCNN layer (Eq.~\ref{eqn:shape_embed}), using the $k$-nearest neighbors based on Euclidean distance with $k=20$.
%
We set the number of VN-DGCNN layers as 4.
%
We set the number of MLP layers in the decoder (Eq.~\ref{eqn:se:decoder}) as 2.
%
We set the number of VN-In layers as 1.

%
We optimized the \SE model via Adam~\cite{adam} with its parameters (0.950, 0.999), %betas (0.95, 0.999), 
learning rate 0.001, and batch size 16.
%
We evaluated the validation loss every 2,000 training steps.
%
We scheduled to decay the learning rate with a factor of 0.6 and a minimum learning rate of 1e-6 if 
the validation loss does not decrease in 5 consecutive evaluations.
%
The optimal \SE model has 28.3K learnable parameters. 
%
We trained the \SE model %for at most 80 hours 
with $\sim$156,000 training steps.
%
The training took 80 hours with our GPUs.
%
The trained \SE model achieved the minimum validation loss at 152,000 steps.


\begin{table*}[!h]
  \centering
      \caption{{Hyper-Parameter Space for \SE Optimization}}
  \label{tbl:hyper_se}
  \begin{threeparttable}
 \begin{scriptsize}
      \begin{tabular}{
%	@{\hspace{2pt}}l@{\hspace{2pt}}
	@{\hspace{2pt}}l@{\hspace{5pt}} 
	@{\hspace{2pt}}r@{\hspace{2pt}}         
	}
        \toprule
        %Notation &
          Hyper-parameters &  Space\\
        \midrule
        %$t_a$    & 
         %hidden layer dimension         & \{16, 32, 64, 128\} \\
         %atom/node embedding dimension &  \{16, 32, 64, 128\} \\
         %$\latent^{\add}$/$\latent^{\delete}$ dimension        & \{8, 16, 32, 64\} \\
         hidden layer dimension            & \{128, 256\}\\
         dimension $d_p$ of \shapehiddenmat        &  \{64, 128\} \\
         \#points in \pc        & \{512, 1,024\} \\
         \#query points in $\mathcal{Z}$                & 1,024 \\%1024 \\%\bo{\{1024\}}\\
         \#nearest neighbors              & 20          \\
         \#VN-DGCNN layers (Eq~\ref{eqn:shape_embed})               & 4            \\
         \#MLP layers in Eq~\ref{eqn:se:decoder} & 4           \\
        \bottomrule
      \end{tabular}
%  	\begin{tablenotes}[normal,flushleft]
%  		\begin{footnotesize}
%  	
%  	\item In this table, hidden dimension represents the dimension of hidden layers and 
%  	atom/node embeddings; latent dimension represents the dimension of latent embedding \latent.
%  	\par
%  \end{footnotesize}
%  
%\end{tablenotes}
%      \begin{tablenotes}
%      \item 
%      \par
%      \end{tablenotes}
\end{scriptsize}
  \end{threeparttable}
\end{table*}

%
\begin{table*}[!h]
  \centering
      \caption{{Hyper-Parameter Space for \methoddiff Optimization}}
  \label{tbl:hyper_diff}
  \begin{threeparttable}
 \begin{scriptsize}
      \begin{tabular}{
%	@{\hspace{2pt}}l@{\hspace{2pt}}
	@{\hspace{2pt}}l@{\hspace{5pt}} 
	@{\hspace{2pt}}r@{\hspace{2pt}}         
	}
        \toprule
        %Notation &
          Hyper-parameters &  Space\\
        \midrule
        %$t_a$    & 
         %hidden layer dimension         & \{16, 32, 64, 128\} \\
         %atom/node embedding dimension &  \{16, 32, 64, 128\} \\
         %$\latent^{\add}$/$\latent^{\delete}$ dimension        & \{8, 16, 32, 64\} \\
         scalar hidden layer dimension         & 128 \\
         vector hidden layer dimension         & 32 \\
         weight of atom type loss $\xi$ (Eq.~\ref{eqn:loss})  & 100           \\
         threshold of step weight $\delta$ (Eq.~\ref{eqn:diff:obj:pos}) & 10 \\
         \#atom features $K$                   & 15 \\
         \#layers $L$ in \molpred             & 8 \\
         %\# \eqgnn/\invgnn layers     &  8 \\
         %\# heads {$n_h$} in $\text{MHA}^{\mathtt{x}}/\text{MHA}^{\mathtt{v}}$                               & 16 \\
         \#nearest neighbors {$N$}  (Eq.~\ref{eqn:geometric_embedding} and \ref{eqn:attention})            & 8          \\
         {\#diffusion steps $T$}                  & 1,000 \\
        \bottomrule
      \end{tabular}
%  	\begin{tablenotes}[normal,flushleft]
%  		\begin{footnotesize}
%  	
%  	\item In this table, hidden dimension represents the dimension of hidden layers and 
%  	atom/node embeddings; latent dimension represents the dimension of latent embedding \latent.
%  	\par
%  \end{footnotesize}
%  
%\end{tablenotes}
%      \begin{tablenotes}
%      \item 
%      \par
%      \end{tablenotes}
\end{scriptsize}
  \end{threeparttable}

\end{table*}


%===================================================================
\subsection{Parameters of \methoddiff}
%===================================================================

Table~\ref{tbl:hyper_diff} presents the parameters used to train \methoddiff.
%
In \methoddiff, we set the hidden dimensions of all the MLP layers and the scalar hidden layers in GVPs (Eq.~\ref{eqn:pred:gvp} and Eq.~\ref{eqn:mess:gvp}) as 128. %, including all the MLP layers in \methoddiff and the scalar dimension of GVP layers in Eq.~\ref{eqn:pred:gvp} and Eq.~\ref{eqn:mess:gvp}. %, and MLP layer (Eq.~\ref{eqn:diff:graph:atompred}) as 128.
%
We set the dimensions of all the vector hidden layers in GVPs as 32.
%
We set the number of layers $L$ in \molpred as 8.
%and the number of layers in graph neural networks as 8.
%
Both two GVP modules in Eq.~\ref{eqn:pred:gvp} and Eq.~\ref{eqn:mess:gvp} consist of three GVP layers. %, which consisa GVP modset the number of layer of GVP modules %is a multi-head attention layer ($\text{MHA}^{\mathtt{x}}$ or $\text{MHA}^{\mathtt{h}}$) with 16 heads.
% 
We set the number of VN-MLP layers in Eq.~\ref{eqn:shaper} as 1 and the number of MLP layers as 2 for all the involved MLP functions.
%

We constructed graphs from atoms in molecules, which are employed to learn the scalar embeddings and vector embeddings for atoms %predict atom coordinates and features  
(Eq.~\ref{eqn:geometric_embedding} and \ref{eqn:attention}), using the $N$-nearest neighbors based on Euclidean distance with $N=8$. 
%
We used $K=15$ atom features in total, indicating the atom types and its aromaticity.
%
These atom features include 10 non-aromatic atoms (i.e., ``H'', ``C'', ``N'', ``O'', ``F'', ``P'', ``S'', ``Cl'', ``Br'', ``I''), 
and 5 aromatic atoms (i.e., ``C'', ``N'', ``O'', ``P'', ``S'').
%
We set the number of diffusion steps $T$ as 1,000.
%
We set the weight $\xi$ of atom type loss (Eq.~\ref{eqn:loss}) as $100$ to balance the values of atom type loss and atom coordinate loss.
%
We set the threshold $\delta$ (Eq.~\ref{eqn:diff:obj:pos}) as 10.
%
The parameters $\beta_t^{\mathtt{x}}$ and $\beta_t^{\mathtt{v}}$ of variance scheduling in the forward diffusion process of \methoddiff are discussed in 
Supplementary Section~\ref{supp:forward:variance}.
%
%Please note that as in \squid, we did not perform extensive hyperparameter optimization for \methoddiff.
%
Following \squid, we did not perform extensive hyperparameter tunning for \methoddiff given that the used 
hyperparameters have enabled good performance.

%
We optimized the \methoddiff model via Adam~\cite{adam} with its parameters (0.950, 0.999), learning rate 0.001, and batch size 32.
%
We evaluated the validation loss every 2,000 training steps.
%
We scheduled to decay the learning rate with a factor of 0.6 and a minimum learning rate of 1e-5 if 
the validation loss does not decrease in 10 consecutive evaluations.
%
The \methoddiff model has 7.8M learnable parameters. 
%
We trained the \methoddiff model %for at most 60 hours 
with $\sim$770,000 training steps.
%
The training took 70 hours with our GPUs.
%
The trained \methoddiff achieved the minimum validation loss at 758,000 steps.

During inference, %the sampling, 
following Adams and Coley~\cite{adams2023equivariant}, we set the variance $\phi$ 
of atom-centered Gaussians as 0.049, which is used to build a set of points for shape guidance in Section ``\method with Shape Guidance'' 
in the main manuscript.
%
We determined the number of atoms in the generated molecule using the atom number distribution of training molecules that have surface shape sizes similar to the condition molecule.
%
The optimal distance threshold $\gamma$ is 0.2, and the optimal stop step $S$ for shape guidance is 300.
%
With shape guidance, each time we updated the atom position (Eq.~\ref{eqn:shape_guidance}), we randomly sampled the weight $\sigma$ from $[0.2, 0.8]$. %\bo{(XXX)}.
%
Moreover, when using pocket guidance as mentioned in Section ``\method with Pocket Guidance'' in the main manuscript, each time we updated the atom position (Eq.~\ref{eqn:pocket_guidance}), we randomly sampled the weight $\epsilon$ from $[0, 0.5]$. 
%
For each condition molecule, it took around 40 seconds on average to generate 50 molecule candidates with our GPUs.



%%%%%%%%%%%%%%%%%%%%%%%%%%%%%%%%%%%%%%%%%%%%%%
\section{Performance of \decompdiff with Protein Pocket Prior}
\label{supp:app:decompdiff}
%%%%%%%%%%%%%%%%%%%%%%%%%%%%%%%%%%%%%%%%%%%%%%

In this section, we demonstrate that \decompdiff with protein pocket prior, referred to as \decompdiffbeta, shows very limited performance in generating drug-like and synthesizable molecules compared to all the other methods, including \methodwithpguide and \methodwithsandpguide.
%
We evaluate the performance of \decompdiffbeta in terms of binding affinities, drug-likeness, and diversity.
%
We compare \decompdiffbeta with \methodwithpguide and \methodwithsandpguide and report the results in Table~\ref{tbl:comparison_results_decompdiff}.
%
Note that the results of \methodwithpguide and \methodwithsandpguide here are consistent with those in Table~\ref{tbl:overall_results_docking2} in the main manuscript.
%
As shown in Table~\ref{tbl:comparison_results_decompdiff}, while \decompdiffbeta achieves high binding affinities in Vina M and Vina D, it substantially underperforms \methodwithpguide and \methodwithsandpguide in QED and SA.
%
Particularly, \decompdiffbeta shows a QED score of 0.36, while \methodwithpguide substantially outperforms \decompdiffbeta in QED (0.77) with 113.9\% improvement.
%
\decompdiffbeta also substantially underperforms \methodwithpguide in terms of SA scores (0.55 vs 0.76).
%
These results demonstrate the limited capacity of \decompdiffbeta in generating drug-like and synthesizable molecules.
%
As a result, the generated molecules from \decompdiffbeta can have considerably lower utility compared to other methods.
%
Considering these limitations of \decompdiffbeta, we exclude it from the baselines for comparison.

\begin{table*}[!h]
	\centering
		\caption{Comparison on PMG among \methodwithpguide, \methodwithsandpguide and \decompdiffbeta}
	\label{tbl:comparison_results_decompdiff}
\begin{threeparttable}
	\begin{scriptsize}
	\begin{tabular}{
		@{\hspace{2pt}}l@{\hspace{2pt}}
		%
		%@{\hspace{2pt}}l@{\hspace{2pt}}
		%
		@{\hspace{2pt}}r@{\hspace{2pt}}
		@{\hspace{2pt}}r@{\hspace{2pt}}
		%
		@{\hspace{6pt}}r@{\hspace{6pt}}
		%
		@{\hspace{2pt}}r@{\hspace{2pt}}
		@{\hspace{2pt}}r@{\hspace{2pt}}
		%
		@{\hspace{5pt}}r@{\hspace{5pt}}
		%
		@{\hspace{2pt}}r@{\hspace{2pt}}
		@{\hspace{2pt}}r@{\hspace{2pt}}
		%
		@{\hspace{5pt}}r@{\hspace{5pt}}
		%
		@{\hspace{2pt}}r@{\hspace{2pt}}
	         @{\hspace{2pt}}r@{\hspace{2pt}}
		%
		@{\hspace{5pt}}r@{\hspace{5pt}}
		%
		@{\hspace{2pt}}r@{\hspace{2pt}}
		@{\hspace{2pt}}r@{\hspace{2pt}}
		%
		@{\hspace{5pt}}r@{\hspace{5pt}}
		%
		@{\hspace{2pt}}r@{\hspace{2pt}}
		@{\hspace{2pt}}r@{\hspace{2pt}}
		%
		@{\hspace{5pt}}r@{\hspace{5pt}}
		%
		@{\hspace{2pt}}r@{\hspace{2pt}}
		@{\hspace{2pt}}r@{\hspace{2pt}}
		%
		@{\hspace{5pt}}r@{\hspace{5pt}}
		%
		@{\hspace{2pt}}r@{\hspace{2pt}}
		%@{\hspace{2pt}}r@{\hspace{2pt}}
		%@{\hspace{2pt}}r@{\hspace{2pt}}
		}
		\toprule
		\multirow{2}{*}{method} & \multicolumn{2}{c}{Vina S$\downarrow$} & & \multicolumn{2}{c}{Vina M$\downarrow$} & & \multicolumn{2}{c}{Vina D$\downarrow$} & & \multicolumn{2}{c}{{HA\%$\uparrow$}}  & & \multicolumn{2}{c}{QED$\uparrow$} & & \multicolumn{2}{c}{SA$\uparrow$} & & \multicolumn{2}{c}{Div$\uparrow$} & %& \multirow{2}{*}{SR\%$\uparrow$} & 
		& \multirow{2}{*}{time$\downarrow$} \\
	    \cmidrule{2-3}\cmidrule{5-6} \cmidrule{8-9} \cmidrule{11-12} \cmidrule{14-15} \cmidrule{17-18} \cmidrule{20-21}
		& Avg. & Med. &  & Avg. & Med. &  & Avg. & Med. & & Avg. & Med.  & & Avg. & Med.  & & Avg. & Med.  & & Avg. & Med.  & & \\ %& & \\
		%\multirow{2}{*}{method} & \multirow{2}{*}{\#c\%} &  \multirow{2}{*}{\#u\%} &  \multirow{2}{*}{QED} & \multicolumn{3}{c}{$\nmax=50$} & & \multicolumn{2}{c}{$\nmax=1$}\\
		%\cmidrule(r){5-7} \cmidrule(r){8-10} 
		%& & & & \avgshapesim(std) & \avggraphsim(std  &  \diversity(std  & & \avgshapesim(std) & \avggraphsim(std \\
		\midrule
		%Reference                          & -5.32 & -5.66 & & -5.78 & -5.76 & & -6.63 & -6.67 & & - & - & & 0.53 & 0.49 & & 0.77 & 0.77 & & - & - & %& 23.1 & & - \\
		%\midrule
		%\multirow{4}{*}{PM} 
		%& \AR & -5.06 & -4.99 & &  -5.59 & -5.29 & &  -6.16 & -6.05 & &  37.69 & 31.00 & &  0.50 & 0.49 & &  0.66 & 0.65 & & - & - & %& 7.0 & 
		%& 7,789 \\
		%& \pockettwomol   & -4.50 & -4.21 & &  -5.70 & -5.27 & &  -6.43 & -6.25 & &  48.00 & 51.00 & &  0.58 & 0.58 & &  \textbf{0.77} & \textbf{0.78} & &  0.69 & 0.71 &  %& 24.9 & 
		%& 2,544 \\
		%& \targetdiff     & -4.88 & \underline{-5.82} & &  -6.20 & \underline{-6.36} & &  \textbf{-7.37} & \underline{-7.51} & &  57.57 & 58.27 & &  0.50 & 0.51 & &  0.60 & 0.59 & &  0.72 & 0.71 & % & 10.4 & 
		%& 1,252 \\
		 \decompdiffbeta             & -4.72 & -4.86 & & \textbf{-6.84} & \textbf{-6.91} & & \textbf{-8.85} & \textbf{-8.90} & &  {72.16} & {72.16} & &  0.36 & 0.36 & &  0.55 & 0.55 & & 0.59 & 0.59 & & 3,549 \\ 
		%-4.76 & -6.18 & &  \textbf{-6.86} & \textbf{-7.52} & &  \textbf{-8.85} & \textbf{-8.96} & &  \textbf{72.7} & \textbf{89.8} & &  0.36 & 0.34 & &  0.55 & 0.57 & & 0.59 & 0.59 & & 15.4 \\
		%& \decompdiffref  & -4.58 & -4.77 & &  -5.47 & -5.51 & &  -6.43 & -6.56 & &  47.76 & 48.66 & &  0.56 & 0.56 & &  0.70 & 0.69  & &  0.72 & 0.72 &  %& 15.2 & 
		%& 1,859 \\
		%\midrule
		%\multirow{2}{*}{PC}
		\methodwithpguide       &  \underline{-5.53} & \underline{-5.64} & & {-6.37} & -6.33 & &  \underline{-7.19} & \underline{-7.52} & &  \underline{78.75} & \textbf{94.00} & &  \textbf{0.77} & \textbf{0.80} & &  \textbf{0.76} & \textbf{0.76} & & 0.63 & 0.66 & & 462 \\
		\methodwithsandpguide   & \textbf{-5.81} & \textbf{-5.96} & &  \underline{-6.50} & \underline{-6.58} & & -7.16 & {-7.51} & &  \textbf{79.92} & \underline{93.00} & &  \underline{0.76} & \underline{0.79} & &  \underline{0.75} & \underline{0.74} & & 0.64 & 0.66 & & 561\\
		\bottomrule
	\end{tabular}%
	\begin{tablenotes}[normal,flushleft]
		\begin{footnotesize}
	\item 
\!\!Columns represent: {``Vina S'': the binding affinities between the initially generated poses of molecules and the protein pockets; 
		``Vina M'': the binding affinities between the poses after local structure minimization and the protein pockets;
		``Vina D'': the binding affinities between the poses determined by AutoDock Vina~\cite{Eberhardt2021} and the protein targets;
		``QED'': the drug-likeness score;
		``SA'': the synthesizability score;
		``Div'': the diversity among generated molecules;
		``time'': the time cost to generate molecules.}
		
		\par
		\par
		\end{footnotesize}
	\end{tablenotes}
	\end{scriptsize}
\end{threeparttable}
  \vspace{-10pt}    
\end{table*}



%===================================================================
\section{{Additional Experimental Results on SMG}}
\label{supp:app:results}
%===================================================================

%-------------------------------------------------------------------------------------------------------------------------------------
\subsection{Comparison on Shape and Graph Similarity}
\label{supp:app:results:overall_shape}
%-------------------------------------------------------------------------------------------------------------------------------------

%\ziqi{Outline for this section:
%	\begin{itemize}
%		\item \method can consistently generate molecules with novel structures (low graph similarity) and similar shapes (high shape similarity), such that these molecules have comparable binding capacity with the condition molecules, and potentially better properties as will be shown in Table~\ref{tbl:overall_results_quality_10}.
%	\end{itemize}
%}

\begin{table*}[!h]
	\centering
		\caption{Similarity Comparison on SMG}
	\label{tbl:overall_sim}
\begin{threeparttable}
	\begin{scriptsize}
	\begin{tabular}{
		@{\hspace{0pt}}l@{\hspace{8pt}}
		%
		@{\hspace{8pt}}l@{\hspace{8pt}}
		%
		@{\hspace{8pt}}c@{\hspace{8pt}}
		@{\hspace{8pt}}c@{\hspace{8pt}}
		%
	    	@{\hspace{0pt}}c@{\hspace{0pt}}
		%
		@{\hspace{8pt}}c@{\hspace{8pt}}
		@{\hspace{8pt}}c@{\hspace{8pt}}
		%
		%@{\hspace{8pt}}r@{\hspace{8pt}}
		}
		\toprule
		$\delta_g$  & method          & \avgshapesim$\uparrow$(std) & \avggraphsim$\downarrow$(std) & & \maxshapesim$\uparrow$(std) & \maxgraphsim$\downarrow$(std)       \\ %& \#n\%$\uparrow$  \\ 
		\midrule
		%\multirow{5}{0.079\linewidth}%{\hspace{0pt}0.1} & \dataset   & 0.0             & 0.628(0.139)          & 0.567(0.068)          & 0.078(0.010)          &  & 0.588(0.086)          & 0.081(0.013)          & 4.7              \\
		%&  \squid($\lambda$=0.3) & 0.0             & 0.320(0.000)          & 0.420(0.163)          & \textbf{0.056}(0.032) &  & 0.461(0.170)          & \textbf{0.065}(0.033) & 1.4              \\
		%& \squid($\lambda$=1.0) & 0.0             & 0.414(0.177)          & 0.483(0.184)          & \underline{0.064}(0.030)  &  & 0.531(0.182)          & \underline{0.073}(0.029)  & 2.4              \\
		%& \method               & \underline{1.6}     & \textbf{0.857}(0.034) & \underline{0.773}(0.045)  & 0.086(0.011)          &  & \underline{0.791}(0.053)  & 0.087(0.012)          & \underline{5.1}      \\
		%& \methodwithsguide      & \textbf{3.7}    & \underline{0.833}(0.062)  & \textbf{0.812}(0.037) & 0.088(0.009)          &  & \textbf{0.835}(0.047) & 0.089(0.010)          & \textbf{6.2}     \\ 
		%\cmidrule{2-10}
		%& improv\% & - & 36.5 & 43.2 & -53.6 &  & 42.0 & -33.8 & 31.9  \\
		%\midrule
		\multirow{6}{0.059\linewidth}{\hspace{0pt}0.3} & \dataset             & 0.745(0.037)          & \textbf{0.211}(0.026) &  & 0.815(0.039)          & \textbf{0.215}(0.047)      \\ %    & \textbf{100.0}   \\
			& \squid($\lambda$=0.3) & 0.709(0.076)          & 0.237(0.033)          &  & 0.841(0.070)          & 0.253(0.038)        \\ %  & 45.5             \\
		    & \squid($\lambda$=1.0) & 0.695(0.064)          & \underline{0.216}(0.034)  &  & 0.841(0.056)          & 0.231(0.047)        \\ %  & 84.3             \\
			& \method               & \underline{0.770}(0.039)  & 0.217(0.031)          &  & \underline{0.858}(0.038)  & \underline{0.220}(0.046)  \\ %& \underline{87.1}     \\
			& \methodwithsguide     & \textbf{0.823}(0.029) & 0.217(0.032)          &  & \textbf{0.900}(0.028) & 0.223(0.048)  \\ % & 86.0             \\ 
		%\cmidrule{2-7}
		%& improv\% & 10.5 & -2.8 &  & 7.0 & -2.3  \\ % & %-12.9  \\
		\midrule
		\multirow{6}{0.059\linewidth}{\hspace{0pt}0.5} & \dataset & 0.750(0.037)          & \textbf{0.225}(0.037) &  & 0.819(0.039)          & \textbf{0.236}(0.070)          \\ %& \textbf{100.0}   \\
			& \squid($\lambda$=0.3)  & 0.728(0.072)          & 0.301(0.054)          &  & \underline{0.888}(0.061)  & 0.355(0.088)          \\ %& 85.9             \\
			& \squid($\lambda$=1.0)  & 0.699(0.063)          & 0.233(0.043)          &  & 0.850(0.057)          & 0.263(0.080)          \\ %& \underline{99.5}     \\
			& \method               & \underline{0.771}(0.039)  & \underline{0.229}(0.043)  &  & 0.862(0.036)          & \textbf{0.236}(0.065) \\ %& 99.2             \\
			& \methodwithsguide    & \textbf{0.824}(0.029) & \underline{0.229}(0.044)  &  & \textbf{0.903}(0.027) & \underline{0.242}(0.069)  \\ %& 99.0             \\ 
		%\cmidrule{2-7}
		%& improv\% & 9.9 & -1.8 &  & 1.7 & 0.0 \\ %& -0.8  \\
		\midrule
		\multirow{6}{0.059\linewidth}{\hspace{0pt}0.7} 
		& \dataset &  0.750(0.037) & \textbf{0.226}(0.038) & & 0.819(0.039) & \underline{0.240}(0.081) \\ %& \textbf{100.0} \\
		%& \dataset & 12.3            & 0.736(0.076)          & 0.768(0.037)          & \textbf{0.228}(0.042) &  & 0.819(0.039)          & \underline{0.242}(0.085)  & \textbf{100.0}   \\
			& \squid($\lambda$=0.3) &  0.735(0.074)          & 0.328(0.070)          &  & \underline{0.900}(0.062)  & 0.435(0.143)          \\ %& 95.4             \\
			& \squid($\lambda$=1.0) &  0.699(0.064)          & 0.234(0.045)          &  & 0.851(0.057)          & 0.268(0.090)          \\ %& \underline{99.9}     \\
			& \method               &  \underline{0.771}(0.039)  & \underline{0.229}(0.043)  &  & 0.862(0.036)          & \textbf{0.237}(0.066) \\ %& 99.3             \\
			& \methodwithsguide     &  \textbf{0.824}(0.029) & 0.230(0.045)          &  & \textbf{0.903}(0.027) & 0.244(0.074)          \\ %& 99.2             \\ 
		%\cmidrule{2-7}
		%& improv\% & 9.9 & -1.3 &  & 0.3 & 1.3 \\%& -0.7  \\
		\midrule
		\multirow{6}{0.059\linewidth}{\hspace{0pt}1.0} 
		& \dataset & 0.750(0.037)          & \textbf{0.226}(0.038) &  & 0.819(0.039)          & \underline{0.242}(0.085)  \\%& \textbf{100.0}  \\
		& \squid($\lambda$=0.3) & 0.740(0.076)          & 0.349(0.088)          &  & \textbf{0.909}(0.065) & 0.547(0.245)       \\ %   & \textbf{100.0}  \\
		& \squid($\lambda$=1.0) & 0.699(0.064)          & 0.235(0.045)          &  & 0.851(0.057)          & 0.271(0.097)          \\ %& \textbf{100.0}   \\
		& \method               & \underline{0.771}(0.039)  & \underline{0.229}(0.043)  &  & 0.862(0.036)          & \textbf{0.237}(0.066) \\ %& \underline{99.3}  \\
		& \methodwithsguide      & \textbf{0.824}(0.029) & 0.230(0.045)          &  & \underline{0.903}(0.027)  & 0.244(0.076)          \\ %& 99.2            \\
		%\cmidrule{2-7}
		%& improv\% &  9.9               & -1.3              &  & -0.7              & -2.1           \\ %       & -0.7 \\
		\bottomrule
	\end{tabular}%
	\begin{tablenotes}[normal,flushleft]
		\begin{footnotesize}
	\item 
\!\!Columns represent: ``$\delta_g$'': the graph similarity constraint; 
%``\#d\%'': the percentage of molecules that satisfy the graph similarity constraint and are with high \shapesim ($\shapesim>=0.8$);
%``\diversity'': the diversity among the generated molecules;
``\avgshapesim/\avggraphsim'': the average of shape or graph similarities between the condition molecules and generated molecules with $\graphsim<=\delta_g$;
``\maxshapesim'': the maximum of shape similarities between the condition molecules and generated molecules with $\graphsim<=\delta_g$;
``\maxgraphsim'': the graph similarities between the condition molecules and the molecules with the maximum shape similarities and $\graphsim<=\delta_g$;
%``\#n\%'': the percentage of molecules that satisfy the graph similarity constraint ($\graphsim<=\delta_g$).
%
``$\uparrow$'' represents higher values are better, and ``$\downarrow$'' represents lower values are better.
%
 Best values are in \textbf{bold}, and second-best values are \underline{underlined}. 
\par
		\par
		\end{footnotesize}
	\end{tablenotes}
\end{scriptsize}
\end{threeparttable}
  \vspace{-10pt}    
\end{table*}
%\label{tbl:overall_sim}


{We evaluate the shape similarity \shapesim and graph similarity \graphsim of molecules generated from}
%Table~\ref{tbl:overall_sim} presents the comparison of shape-conditioned molecule generation among 
\dataset, \squid, \method and \methodwithsguide under different graph similarity constraints  ($\delta_g$=1.0, 0.7, 0.5, 0.3). 
%
%During the evaluation, for each molecule in the test set, all the methods are employed to generate or identify 50 molecules with similar shapes.
%
We calculate evaluation metrics using all the generated molecules satisfying the graph similarity constraints.
%
Particularly, when $\delta_g$=1.0, we do not filter out any molecules based on the constraints and directly calculate metrics on all the generated molecules.
%
When $\delta_g$=0.7, 0.5 or 0.3, we consider only generated molecules with similarities lower than $\delta_g$.
%
Based on \shapesim and \graphsim as described in Section ``Evaluation Metrics'' in the main manuscript,
we calculate the following metrics using the subset of molecules with \graphsim lower than $\delta_g$, from a set of 50 generated molecules for each test molecule and report the average of  these metrics across all test molecules:
%
(1) \avgshapesim\ measures the average \shapesim across each subset of generated molecules with $\graphsim$ lower than $\delta_g$; %per test molecule, with the overall average calculated across all test molecules; }%the 50 generated molecules for each test molecule, averaged across all test molecules;
(2) \avggraphsim\ calculates the average \graphsim for each set; %, with these means averaged across all test molecules}; %} 50 molecules, %\bo{@Ziqi rephrase}, with results averaged on the test set;\ziqi{with the average computed over the test set; }
(3) \maxshapesim\ determines the maximum \shapesim within each set; %, with these maxima averaged across all test molecules; }%\hl{among 50 molecules}, averaged across all test molecules;
(4) \maxgraphsim\ measures the \graphsim of the molecule with maximum \shapesim in each set. %, averaged across all test molecules; }%\hl{among 50 molecules}, averaged across all test molecules;

%
As shown in Table~\ref{tbl:overall_sim}, \method and \methodwithsguide demonstrate outstanding performance in terms of the average shape similarities (\avgshapesim) and the average graph similarities (\avggraphsim) among generated molecules.
%
%\ziqi{
%Table~\ref{tbl:overall} also shows that \method and \methodwithsguide consistently outperform all the baseline methods in average shape similarities (\avgshapesim) and only slightly underperform 
%the best baseline \dataset in average graph similarities (\avggraphsim).
%}
%
Specifically, when $\delta_g$=0.3, \methodwithsguide achieves a substantial 10.5\% improvement in \avgshapesim\ over the best baseline \dataset. 
%
In terms of \avggraphsim, \methodwithsguide also achieves highly comparable performance with \dataset (0.217 vs 0.211, in \avggraphsim, lower values indicate better performance).
%
%This trend remains consistent across various $\delta_g$ values.
This trend remains consistent when applying various similarity constraints (i.e., $\delta_g$) as shown in Table~\ref{tbl:overall_sim}.


Similarly, \method and \methodwithsguide demonstrate superior performance in terms of the average maximum shape similarity across generated molecules for all test molecules (\maxshapesim), as well as the average graph similarity of the molecules with the maximum shape similarities (\maxgraphsim). %maximum shape similarities of generated molecules (\maxshapesim) and the average graph similarities of molecules with the maximum shape similarities (\maxgraphsim). %\bo{\maxgraphsim is misleading... how about $\text{avgMSim}_\text{g}$}
%
%\bo{
%in terms of the maximum shape similarities (\maxshapesim) and the maximum graph similarities (\maxgraphsim) among all the generated molecules.
%@Ziqi are the metrics maximum values or the average of maximum values?
%}
%
Specifically, at \maxshapesim, Table~\ref{tbl:overall_sim} shows that \methodwithsguide outperforms the best baseline \squid ($\lambda$=0.3) when $\delta_g$=0.3, 0.5, and 0.7, and only underperforms
it by 0.7\% when $\delta$=1.0.
%
We also note that the molecules generated by {\methodwithsguide} with the maximum shape similarities have substantially lower graph similarities ({\maxgraphsim}) compared to those generated by {\squid} ({$\lambda$}=0.3).
%\hl{We also note that the molecules with the maximum shape similarities generated by {\methodwithsguide} are with significantly lower graph similarities ({\maxgraphsim}) than those generated by {\squid} ({$\lambda$}=0.3).}
%
%\bo{@Ziqi please rephrase the language}
%
%\bo{
%@Ziqi the conclusion is not obvious. You may want to remind the meaning of \maxshapesim and \maxgraphsim here, and based on what performance you say this.
%}
%
%\bo{\st{This also underscores the ability of {\methodwithsguide} in generating molecules with similar shapes to condition molecules and novel graph structures.}}
%
As evidenced by these results, \methodwithsguide features strong capacities of generating molecules with similar shapes yet novel graph structures compared to the condition molecule, facilitating the discovery of promising drug candidates.
%

\begin{comment}
\ziqi{replace \#n\% with the percentage of novel molecules that do not exist in the dataset and update the discussion accordingly}
%\ziqi{
Table~\ref{tbl:overall_sim} also presents \bo{\#n\%}, the percentage of molecules generated by each method %\st{(\#n\%)} 
with graph similarities lower than the constraint $\delta_g$. 
%
%\bo{
%Table~\ref{tbl:overall_sim} also presents \#n\%, the percentage of generated molecules with graph similarities lower than the constraint $\delta_g$, of different methods. 
%}
%
As shown in Table~\ref{tbl:overall_sim},  when a restricted constraint (i.e., $\delta_g$=0.3) is applied, \method and \methodwithsguide could still generate a sufficient number of molecules satisfying the constraint.
%
Particularly, when $\delta_g$=0.3, \method outperforms \squid with $\lambda$=0.3 by XXX and \squid with $\lambda$=1.0 by XXX.
% achieve the second and the third in \#n\% and only underperform the best baseline \dataset.
%
This demonstrates the ability of \method in generating molecules with novel structures. 
%
When $\delta_g$=0.5, 0.7 and 1.0, both methods generate over 99.0\% of molecules satisfying the similarity constraint $\delta_g$.
%
%Note that \dataset is guaranteed to identify at least 50 molecules satisfying the $\delta_g$ by searching within a training dataset of diverse molecules.
%
Note that \dataset is a search algorithm that always first identifies the molecules satisfying $\delta_g$ and then selects the top-50 molecules of the highest shape similarities among them. 
%
Due to the diverse molecules in %\hl{the subset} \bo{@Ziqi why do you want to stress subset?} of 
the training set, \dataset can always identify at least 50 molecules under different $\delta_g$ and thus achieve 100\% in \#n\%.
%
%\bo{
%Note that \dataset is a search algorithm that always generate molecules XXX
%@Ziqi
%We need to discuss here. For \dataset, \#n\% in this table does not look aligned with that in Fig 1 if the highlighted defination is correct...
%}
%
%Thus, \dataset achieves 100.0\% in \#n\% under different $\delta_g$.
%
It is also worth noting that when $\delta_g$=1.0, \#n\% reflects the validity among all the generated molecules. 
%
As shown in Table~\ref{tbl:overall_sim}, \method and \methodwithsguide are able to generate 99.3\% and 99.2\% valid molecules.
%
This demonstrates their ability to effectively capture the underlying chemical rules in a purely data-driven manner without relying on any prior knowledge (e.g., fragments) as \squid does.
%
%\bo{
%@Ziqi I feel this metric is redundant with the avg graph similarity when constraint is 1.0. Generally, if the avg similarity is small. You have more mols satisfying the requirement right?
%}
\end{comment}

Table~\ref{tbl:overall_sim} also shows that by incorporating shape guidance, \methodwithsguide
%\bo{
%@Ziqi where does this come from...
%}
substantially outperforms \method in both \avgshapesim and \maxshapesim, while maintaining comparable graph similarities (i.e., \avggraphsim\ and \maxgraphsim).
%
Particularly, when $\delta_g$=0.3, \methodwithsguide 
establishes a considerable improvement of 6.9\% and 4.9\%
%\bo{\st{achieves 6.9\% and 4.9\% improvements}} 
over \method in \avgshapesim and \maxshapesim, respectively. 
%
%\hl{In the meanwhile}, 
%\bo{@Ziqi it is not the right word...}
Meanwhile, \methodwithsguide achieves the same \avggraphsim with \method and only slightly underperforms \method in \maxgraphsim (0.223 vs 0.220).
%\bo{
%XXX also achieves XXX
%}
%it maintains the same \avggraphsim\ with \method and only slightly underperforms \method in \maxgraphsim (0.223 vs 0.220).
%
%Compared with \method, \methodwithsguide consistently generates molecules with higher shape similarities while maintaining comparable graph similarities.
%
%\bo{
%@Ziqi you may want to highlight the utility of "generating molecules with higher shape similarities while maintaining comparable graph similarities" in real drug discovery applications.
%
%
%\bo{
%@Ziqi You did not present the details of method yet...
%}
%
%\methodwithsguide leverages additional shape guidance to push the predicted atoms to the shape of condition molecules \bo{and XXX (@Ziqi boosts the shape similarities XXX)} , as will be discussed in Section ``\method with Shape Guidance'' later.
%
The superior performance of \methodwithsguide suggests that the incorporation of shape guidance effectively boosts the shape similarities of generated molecules without compromising graph similarities.
%
%This capability could be crucial in drug discovery, 
%\bo{@Ziqi it is a strong statement. Need citations here}, 
%as it enables the discovery of drug candidates that are both more potentially effective due to the improved shape similarities and novel induced by low graph similarities.
%as it could enable the identification of candidates with similar binding patterns %with the condition molecule (i.e., high shape similarities) 
%(i.e., high shape similarities) and graph structures distinct from the condition molecules (i.e., low graph similarities).
%\bo{\st{and enjoys novel structures (i.e., low graph similarities) with potentially better properties. } \ziqi{change enjoys}}
%\bo{
%and enjoys potentially better properties (i.e., low graph similarities). \ziqi{this looks weird to me... need to discuss}
%}
%\st{potentially better properties (i.e., low graph similarities).}}

%-------------------------------------------------------------------------------------------------------------------------------------
\subsection{Comparison on Validity and Novelty}
\label{supp:app:results:valid_novel}
%-------------------------------------------------------------------------------------------------------------------------------------

We evaluate the ability of \method and \squid to generate molecules with valid and novel 2D molecular graphs.
%
We calculate the percentages of the valid and novel molecules among all the generated molecules.
%
As shown in Table~\ref{tbl:validity_novelty}, both \method and \methodwithsguide outperform \squid with $\lambda$=0.3 and $\lambda$=1.0 in generating novel molecules.
%
Particularly, almost all valid molecules generated by \method and \methodwithsguide are novel (99.8\% and 99.9\% at \#n\%), while the best baseline \squid with $\lambda$=0.3 achieves 98.4\% in novelty.
%
In terms of the percentage of valid and novel molecules among all the generated ones (\#v\&n\%), \method and \methodwithsguide again outperform \squid with $\lambda$=0.3 and $\lambda$=1.0.
%
We also note that at \#v\%,  \method (99.1\%) and \methodwithsguide (99.2\%) slightly underperform \squid with $\lambda$=0.3 and $\lambda$=1.0 (100.0\%) in generating valid molecules.
%
\squid guarantees the validity of generated molecules by incorporating valence rules into the generation process and ensuring it to avoid fragments that violate these rules.
%
Conversely, \method and \methodwithsguide use a purely data-driven approach to learn the generation of valid molecules.
%
These results suggest that, even without integrating valence rules, \method and \methodwithsguide can still achieve a remarkably high percentage of valid and novel generated molecules.

\begin{table*}
	\centering
		\caption{Comparison on Validity and Novelty between \method and \squid}
	\label{tbl:validity_novelty}
	\begin{scriptsize}
\begin{threeparttable}
%	\setlength\tabcolsep{0pt}
	\begin{tabular}{
		@{\hspace{3pt}}l@{\hspace{10pt}}
		%
		@{\hspace{10pt}}r@{\hspace{10pt}}
		%
		@{\hspace{10pt}}r@{\hspace{10pt}}
		%
		@{\hspace{10pt}}r@{\hspace{3pt}}
		}
		\toprule
		method & \#v\% & \#n\% & \#v\&n\% \\
		\midrule
		\squid ($\lambda$=0.3) & \textbf{100.0} & 96.7 & 96.7 \\
		\squid ($\lambda$=1.0) & \textbf{100.0} & 98.4 & 98.4 \\
		\method & 99.1 & 99.8 & 98.9 \\
		\methodwithsguide & 99.2 & \textbf{99.9} & \textbf{99.1} \\
		\bottomrule
	\end{tabular}%
	%
	\begin{tablenotes}[normal,flushleft]
		\begin{footnotesize}
	\item 
\!\!Columns represent: ``\#v\%'': the percentage of generated molecules that are valid;
		``\#n\%'': the percentage of valid molecules that are novel;
		``\#v\&n\%'': the percentage of generated molecules that are valid and novel.
		Best values are in \textbf{bold}. 
		\par
		\end{footnotesize}
	\end{tablenotes}
\end{threeparttable}
\end{scriptsize}
\end{table*}


%-------------------------------------------------------------------------------------------------------------------------------------
\subsection{Additional Quality Comparison between Desirable Molecules Generated by \method and \squid}
\label{supp:app:results:quality_desirable}
%-------------------------------------------------------------------------------------------------------------------------------------

\begin{table*}[!h]
	\centering
		\caption{Comparison on Quality of Generated Desirable Molecules between \method and \squid ($\delta_g$=0.5)}
	\label{tbl:overall_results_quality_05}
	\begin{scriptsize}
\begin{threeparttable}
	\begin{tabular}{
		@{\hspace{0pt}}l@{\hspace{16pt}}
		@{\hspace{0pt}}l@{\hspace{2pt}}
		%
		@{\hspace{6pt}}c@{\hspace{6pt}}
		%
		%@{\hspace{3pt}}c@{\hspace{3pt}}
		@{\hspace{3pt}}c@{\hspace{3pt}}
		@{\hspace{3pt}}c@{\hspace{3pt}}
		@{\hspace{3pt}}c@{\hspace{3pt}}
		@{\hspace{3pt}}c@{\hspace{3pt}}
		%
		%
		}
		\toprule
		group & metric & 
        %& \dataset 
        & \squid ($\lambda$=0.3) & \squid ($\lambda$=1.0)  &  \method & \methodwithsguide  \\
		%\multirow{2}{*}{method} & \multirow{2}{*}{\#c\%} &  \multirow{2}{*}{\#u\%} &  \multirow{2}{*}{QED} & \multicolumn{3}{c}{$\nmax=50$} & & \multicolumn{2}{c}{$\nmax=1$}\\
		%\cmidrule(r){5-7} \cmidrule(r){8-10} 
		%& & & & \avgshapesim(std) & \avggraphsim(std  &  \diversity(std  & & \avgshapesim(std) & \avggraphsim(std \\
		\midrule
		\multirow{2}{*}{stability}
		& atom stability ($\uparrow$) & 
        %& 0.990 
        & \textbf{0.996} & 0.995 & 0.992 & 0.989     \\
		& mol stability ($\uparrow$) & 
        %& 0.875 
        & \textbf{0.948} & 0.947 & 0.886 & 0.839    \\
		%\midrule
		%\multirow{3}{*}{Drug-likeness} 
		%& QED ($\uparrow$) & 
        %& \textbf{0.805} 
        %& 0.766 & 0.760 & 0.755 & 0.751    \\
	%	& SA ($\uparrow$) & 
        %& \textbf{0.874} 
        %& 0.814 & 0.813 & 0.699 & 0.692    \\
	%	& Lipinski ($\uparrow$) & 
        %& \textbf{4.999} 
        %& 4.979 & 4.980 & 4.967 & 4.975    \\
		\midrule
		\multirow{4}{*}{3D structures} 
		& RMSD ($\downarrow$) & 
        %& \textbf{0.419} 
        & 0.907 & 0.906 & 0.897 & \textbf{0.881}    \\
		& JS. bond lengths ($\downarrow$) & 
        %& \textbf{0.286} 
        & 0.457 & 0.477 & 0.436 & \textbf{0.428}    \\
		& JS. bond angles ($\downarrow$) & 
        %& \textbf{0.078} 
        & 0.269 & 0.289 & \textbf{0.186} & 0.200    \\
		& JS. dihedral angles ($\downarrow$) & 
        %& \textbf{0.151} 
        & 0.199 & 0.209 & \textbf{0.168} & 0.170    \\
		\midrule
		\multirow{5}{*}{2D structures} 
		& JS. \#bonds per atoms ($\downarrow$) & 
        %& 0.325 
        & 0.291 & 0.331 & \textbf{0.176} & 0.181    \\
		& JS. basic bond types ($\downarrow$) & 
        %& \textbf{0.055} 
        & \textbf{0.071} & 0.083 & 0.181 & 0.191    \\
		%& JS. freq. bond types ($\downarrow$) & 
        %& \textbf{0.089} 
        %& 0.123 & 0.130 & 0.245 & 0.254    \\
		%& JS. freq. bond pairs ($\downarrow$) & 
        %& \textbf{0.078} 
        %& 0.085 & 0.089 & 0.209 & 0.221    \\
		%& JS. freq. bond triplets ($\downarrow$) & 
        %& \textbf{0.089} 
        %& 0.097 & 0.114 & 0.211 & 0.223    \\
		%\midrule
		%\multirow{3}{*}{Rings} 
		& JS. \#rings ($\downarrow$) & 
        %& 0.142 
        & 0.280 & 0.330 & \textbf{0.043} & 0.049    \\
		& JS. \#n-sized rings ($\downarrow$) & 
        %& \textbf{0.055} 
        & \textbf{0.077} & 0.091 & 0.099 & 0.112    \\
		& \#Intersecting rings ($\uparrow$) & 
        %& \textbf{6} 
        & \textbf{6} & 5 & 4 & 5    \\
		%\method (+bt)            & 100.0 & 98.0 & 100.0 & 0.742 & 0.772 (0.040) & 0.211 (0.033) & & 0.862 (0.036) & 0.211 (0.033) & 0.743 (0.043) \\
		%\methodwithguide (+bt)    & 99.8 & 98.0 & 100.0 & 0.736 & 0.814 (0.031) & 0.193 (0.042) & & 0.895 (0.029) & 0.193 (0.042) & 0.745 (0.045) \\
		%
		\bottomrule
	\end{tabular}%
	\begin{tablenotes}[normal,flushleft]
		\begin{footnotesize}
	\item 
\!\!Rows represent:  {``atom stability'': the proportion of stable atoms that have the correct valency; 
		``molecule stability'': the proportion of generated molecules with all atoms stable;
		%``QED'': the drug-likeness score;
		%``SA'': the synthesizability score;
		%``Lipinski'': the Lipinski 
		``RMSD'': the root mean square deviation (RMSD) between the generated 3D structures of molecules and their optimal conformations; % identified via energy minimization;
		``JS. bond lengths/bond angles/dihedral angles'': the Jensen-Shannon (JS) divergences of bond lengths, bond angles and dihedral angles;
		``JS. \#bonds per atom/basic bond types/\#rings/\#n-sized rings'': the JS divergences of bond counts per atom, basic bond types, counts of all rings, and counts of n-sized rings;
		%``JS. \#rings/\#n-sized rings'': the JS divergences of the total counts of rings and the counts of n-sized rings;
		``\#Intersecting rings'': the number of rings observed in the top-10 frequent rings of both generated and real molecules. } \par
		\par
		\end{footnotesize}
	\end{tablenotes}
\end{threeparttable}
\end{scriptsize}
\end{table*}

%\label{tbl:overall_quality05}

\begin{table*}[!h]
	\centering
		\caption{Comparison on Quality of Generated Desirable Molecules between \method and \squid ($\delta_g$=0.7)}
	\label{tbl:overall_results_quality_07}
	\begin{scriptsize}
\begin{threeparttable}
	\begin{tabular}{
		@{\hspace{0pt}}l@{\hspace{14pt}}
		@{\hspace{0pt}}l@{\hspace{2pt}}
		%
		@{\hspace{4pt}}c@{\hspace{4pt}}
		%
		%@{\hspace{3pt}}c@{\hspace{3pt}}
		@{\hspace{3pt}}c@{\hspace{3pt}}
		@{\hspace{3pt}}c@{\hspace{3pt}}
		@{\hspace{3pt}}c@{\hspace{3pt}}
		@{\hspace{3pt}}c@{\hspace{3pt}}
		%
		%
		}
		\toprule
		group & metric & 
        %& \dataset 
        & \squid ($\lambda$=0.3) & \squid ($\lambda$=1.0)  &  \method & \methodwithsguide  \\
		%\multirow{2}{*}{method} & \multirow{2}{*}{\#c\%} &  \multirow{2}{*}{\#u\%} &  \multirow{2}{*}{QED} & \multicolumn{3}{c}{$\nmax=50$} & & \multicolumn{2}{c}{$\nmax=1$}\\
		%\cmidrule(r){5-7} \cmidrule(r){8-10} 
		%& & & & \avgshapesim(std) & \avggraphsim(std  &  \diversity(std  & & \avgshapesim(std) & \avggraphsim(std \\
		\midrule
		\multirow{2}{*}{stability} 
		& atom stability ($\uparrow$) & 
        %&  0.990 
        & \textbf{0.995} & 0.995 & 0.992 & 0.988 \\
		& molecule stability ($\uparrow$) & 
        %& 0.876 
        & 0.944 & \textbf{0.947} & 0.885 & 0.839 \\
		\midrule
		%\multirow{3}{*}{Drug-likeness} 
		%& QED ($\uparrow$) & 
        %& \textbf{0.805} 
        %& 0.766 & 0.760 & 0.755 & 0.751    \\
	%	& SA ($\uparrow$) & 
        %& \textbf{0.874} 
        %& 0.814 & 0.813 & 0.699 & 0.692    \\
	%	& Lipinski ($\uparrow$) & 
        %& \textbf{4.999} 
        %& 4.979 & 4.980 & 4.967 & 4.975    \\
	%	\midrule
		\multirow{4}{*}{3D structures} 
		& RMSD ($\downarrow$) & 
        %& \textbf{0.420} 
        & 0.897 & 0.906 & 0.897 & \textbf{0.881}    \\
		& JS. bond lengths ($\downarrow$) & 
        %& \textbf{0.286} 
        & 0.457 & 0.477 & 0.436 & \textbf{0.428}    \\
		& JS. bond angles ($\downarrow$) & 
        %& \textbf{0.078} 
        & 0.269 & 0.289 & \textbf{0.186} & 0.200    \\
		& JS. dihedral angles ($\downarrow$) & 
        %& \textbf{0.151} 
        & 0.199 & 0.209 & \textbf{0.168} & 0.170    \\
		\midrule
		\multirow{5}{*}{2D structures} 
		& JS. \#bonds per atoms ($\downarrow$) & 
        %& 0.325 
        & 0.285 & 0.329 & \textbf{0.176} & 0.181    \\
		& JS. basic bond types ($\downarrow$) & 
        %& \textbf{0.055} 
        & \textbf{0.067} & 0.083 & 0.181 & 0.191    \\
	%	& JS. freq. bond types ($\downarrow$) & 
        %& \textbf{0.089} 
        %& 0.123 & 0.130 & 0.245 & 0.254    \\
	%	& JS. freq. bond pairs ($\downarrow$) & 
        %& \textbf{0.078} 
        %& 0.085 & 0.089 & 0.209 & 0.221    \\
	%	& JS. freq. bond triplets ($\downarrow$) & 
        %& \textbf{0.089} 
        %& 0.097 & 0.114 & 0.211 & 0.223    \\
	%	\midrule
	%	\multirow{3}{*}{Rings} 
		& JS. \#rings ($\downarrow$) & 
        %& 0.143 
        & 0.273 & 0.328 & \textbf{0.043} & 0.049    \\
		& JS. \#n-sized rings ($\downarrow$) & 
        %& \textbf{0.055} 
        & \textbf{0.076} & 0.091 & 0.099 & 0.112    \\
		& \#Intersecting rings ($\uparrow$) & 
        %& \textbf{6} 
        & \textbf{6} & 5 & 4 & 5    \\
		%\method (+bt)            & 100.0 & 98.0 & 100.0 & 0.742 & 0.772 (0.040) & 0.211 (0.033) & & 0.862 (0.036) & 0.211 (0.033) & 0.743 (0.043) \\
		%\methodwithguide (+bt)    & 99.8 & 98.0 & 100.0 & 0.736 & 0.814 (0.031) & 0.193 (0.042) & & 0.895 (0.029) & 0.193 (0.042) & 0.745 (0.045) \\
		%
		\bottomrule
	\end{tabular}%
	\begin{tablenotes}[normal,flushleft]
		\begin{footnotesize}
	\item 
\!\!Rows represent:  {``atom stability'': the proportion of stable atoms that have the correct valency; 
		``molecule stability'': the proportion of generated molecules with all atoms stable;
		%``QED'': the drug-likeness score;
		%``SA'': the synthesizability score;
		%``Lipinski'': the Lipinski 
		``RMSD'': the root mean square deviation (RMSD) between the generated 3D structures of molecules and their optimal conformations; % identified via energy minimization;
		``JS. bond lengths/bond angles/dihedral angles'': the Jensen-Shannon (JS) divergences of bond lengths, bond angles and dihedral angles;
		``JS. \#bonds per atom/basic bond types/\#rings/\#n-sized rings'': the JS divergences of bond counts per atom, basic bond types, counts of all rings, and counts of n-sized rings;
		%``JS. \#rings/\#n-sized rings'': the JS divergences of the total counts of rings and the counts of n-sized rings;
		``\#Intersecting rings'': the number of rings observed in the top-10 frequent rings of both generated and real molecules. } \par
		\par
		\end{footnotesize}
	\end{tablenotes}
\end{threeparttable}
\end{scriptsize}
\end{table*}

%\label{tbl:overall_quality07}

\begin{table*}[!h]
	\centering
		\caption{Comparison on Quality of Generated Desirable Molecules between \method and \squid ($\delta_g$=1.0)}
	\label{tbl:overall_results_quality_10}
	\begin{scriptsize}
\begin{threeparttable}
	\begin{tabular}{
		@{\hspace{0pt}}l@{\hspace{14pt}}
		@{\hspace{0pt}}l@{\hspace{2pt}}
		%
		@{\hspace{4pt}}c@{\hspace{4pt}}
		%
		%@{\hspace{3pt}}c@{\hspace{3pt}}
		@{\hspace{3pt}}c@{\hspace{3pt}}
		@{\hspace{3pt}}c@{\hspace{3pt}}
		@{\hspace{3pt}}c@{\hspace{3pt}}
		@{\hspace{3pt}}c@{\hspace{3pt}}
		%
		%
		}
		\toprule
		group & metric & 
        %& \dataset 
        & \squid ($\lambda$=0.3) & \squid ($\lambda$=1.0)  &  \method & \methodwithsguide \\
		%\multirow{2}{*}{method} & \multirow{2}{*}{\#c\%} &  \multirow{2}{*}{\#u\%} &  \multirow{2}{*}{QED} & \multicolumn{3}{c}{$\nmax=50$} & & \multicolumn{2}{c}{$\nmax=1$}\\
		%\cmidrule(r){5-7} \cmidrule(r){8-10} 
		%& & & & \avgshapesim(std) & \avggraphsim(std  &  \diversity(std  & & \avgshapesim(std) & \avggraphsim(std \\
		\midrule
		\multirow{2}{*}{stability}
		& atom stability ($\uparrow$) & 
        %& 0.990 
        & \textbf{0.995} & \textbf{0.995} & 0.992 & 0.988     \\
		& mol stability ($\uparrow$) & 
        %& 0.876 
        & 0.942 & \textbf{0.947} & 0.885 & 0.839    \\
		\midrule
	%	\multirow{3}{*}{Drug-likeness} 
	%	& QED ($\uparrow$) & 
        %& \textbf{0.805} 
        %& \textbf{0.766} & 0.760 & 0.755 & 0.751    \\
	%	& SA ($\uparrow$) & 
        %& \textbf{0.874} 
        %& \textbf{0.813} & \textbf{0.813} & 0.699 & 0.692    \\
	%	& Lipinski ($\uparrow$) & 
        %& \textbf{4.999} 
        %& 4.979 & \textbf{4.980} & 4.967 & 4.975    \\
	%	\midrule
		\multirow{4}{*}{3D structures} 
		& RMSD ($\downarrow$) & 
        %& \textbf{0.420} 
        & 0.898 & 0.906 & 0.897 & \textbf{0.881}    \\
		& JS. bond lengths ($\downarrow$) & 
        %& \textbf{0.286} 
        & 0.457 & 0.477 & 0.436 & \textbf{0.428}    \\
		& JS. bond angles ($\downarrow$) & 
        %& \textbf{0.078} 
        & 0.269 & 0.289 & \textbf{0.186} & 0.200   \\
		& JS. dihedral angles ($\downarrow$) & 
        %& \textbf{0.151} 
        & 0.199 & 0.209 & \textbf{0.168} & 0.170    \\
		\midrule
		\multirow{5}{*}{2D structures} 
		& JS. \#bonds per atoms ($\downarrow$) & 
        %& 0.325 
        & 0.280 & 0.330 & \textbf{0.176} & 0.181    \\
		& JS. basic bond types ($\downarrow$) & 
        %& \textbf{0.055} 
        & \textbf{0.066} & 0.083 & 0.181 & 0.191   \\
	%	& JS. freq. bond types ($\downarrow$) & 
        %& \textbf{0.089} 
        %& \textbf{0.123} & 0.130 & 0.245 & 0.254    \\
	%	& JS. freq. bond pairs ($\downarrow$) & 
        %& \textbf{0.078} 
        %& \textbf{0.085} & 0.089 & 0.209 & 0.221    \\
	%	& JS. freq. bond triplets ($\downarrow$) & 
        %& \textbf{0.089} 
        %& \textbf{0.097} & 0.114 & 0.211 & 0.223    \\
		%\midrule
		%\multirow{3}{*}{Rings} 
		& JS. \#rings ($\downarrow$) & 
        %& 0.143 
        & 0.269 & 0.328 & \textbf{0.043} & 0.049    \\
		& JS. \#n-sized rings ($\downarrow$) & 
        %& \textbf{0.055} 
        & \textbf{0.075} & 0.091 & 0.099 & 0.112    \\
		& \#Intersecting rings ($\uparrow$) & 
        %& \textbf{6} 
        & \textbf{6} & 5 & 4 & 5    \\
		%\method (+bt)            & 100.0 & 98.0 & 100.0 & 0.742 & 0.772 (0.040) & 0.211 (0.033) & & 0.862 (0.036) & 0.211 (0.033) & 0.743 (0.043) \\
		%\methodwithguide (+bt)    & 99.8 & 98.0 & 100.0 & 0.736 & 0.814 (0.031) & 0.193 (0.042) & & 0.895 (0.029) & 0.193 (0.042) & 0.745 (0.045) \\
		%
		\bottomrule
	\end{tabular}%
	\begin{tablenotes}[normal,flushleft]
		\begin{footnotesize}
	\item 
\!\!Rows represent:  {``atom stability'': the proportion of stable atoms that have the correct valency; 
		``molecule stability'': the proportion of generated molecules with all atoms stable;
		%``QED'': the drug-likeness score;
		%``SA'': the synthesizability score;
		%``Lipinski'': the Lipinski 
		``RMSD'': the root mean square deviation (RMSD) between the generated 3D structures of molecules and their optimal conformations; % identified via energy minimization;
		``JS. bond lengths/bond angles/dihedral angles'': the Jensen-Shannon (JS) divergences of bond lengths, bond angles and dihedral angles;
		``JS. \#bonds per atom/basic bond types/\#rings/\#n-sized rings'': the JS divergences of bond counts per atom, basic bond types, counts of all rings, and counts of n-sized rings;
		%``JS. \#rings/\#n-sized rings'': the JS divergences of the total counts of rings and the counts of n-sized rings;
		``\#Intersecting rings'': the number of rings observed in the top-10 frequent rings of both generated and real molecules. } \par
		\par
		\end{footnotesize}
	\end{tablenotes}
\end{threeparttable}
\end{scriptsize}
\end{table*}

%\label{tbl:overall_quality10}

Similar to Table~\ref{tbl:overall_results_quality_desired} in the main manuscript, we present the performance comparison on the quality of desirable molecules generated by different methods under different graph similarity constraints $\delta_g$=0.5, 0.7 and 1.0, as detailed in Table~\ref{tbl:overall_results_quality_05}, Table~\ref{tbl:overall_results_quality_07}, and Table~\ref{tbl:overall_results_quality_10}, respectively.
%
Overall, these tables show that under varying graph similarity constraints, \method and \methodwithsguide can always generate desirable molecules with comparable quality to baselines in terms of stability, 3D structures, and 2D structures.
%
These results demonstrate the strong effectiveness of \method and \methodwithsguide in generating high-quality desirable molecules with stable and realistic structures in both 2D and 3D.
%
This enables the high utility of \method and \methodwithsguide in discovering promising drug candidates.


\begin{comment}
The results across these tables demonstrate similar observations with those under $\delta_g$=0.3 in Table~\ref{tbl:overall_results_quality_desired}.
%
For stability, when $\delta_g$=0.5, 0.7 or 1.0, \method and \methodwithsguide achieve comparable performance or fall slightly behind \squid ($\lambda$=0.3) and \squid ($\lambda$=1.0) in atom stability and molecule stability.
%
For example, when $\delta_g$=0.5, as shown in Table~\ref{tbl:overall_results_quality_05}, \method achieves similar performance with the best baseline \squid ($\lambda$=0.3) in atom stability (0.992 for \method vs 0.996 for \squid with $\lambda$=0.3).
%
\method underperforms \squid ($\lambda$=0.3) in terms of molecule stability.
%
For 3D structures, \method and \methodwithsguide also consistently generate molecules with more realistic 3D structures compared to \squid.
%
Particularly, \methodwithsguide achieves the best performance in RMSD and JS of bond lengths across $\delta_g$=0.5, 0.7 and 1.0.
%
In JS of dihedral angles, \method achieves the best performance among all the methods.
%
\method and \methodwithsguide underperform \squid in JS of bond angles, primarily because \squid constrains the bond angles in the generated molecules.
%
For 2D structures, \method and \methodwithsguide again achieve the best performance 
\end{comment}

%===================================================================
\section{Additional Experimental Results on PMG}
\label{supp:app:results_PMG}
%===================================================================

%\label{tbl:comparison_results_decompdiff}


%-------------------------------------------------------------------------------------------------------------------------------------
%\subsection{{Additional Comparison for PMG}}
%\label{supp:app:results:docking}
%-------------------------------------------------------------------------------------------------------------------------------------

In this section, we present the results of \methodwithpguide and \methodwithsandpguide when generating 100 molecules. 
%
Please note that both \methodwithpguide and \methodwithsandpguide show remarkable efficiency over the PMG baselines.
%
\methodwithpguide and \methodwithsandpguide generate 100 molecules in 48 and 58 seconds on average, respectively, while the most efficient baseline \targetdiff requires 1,252 seconds.
%
We report the performance of \methodwithpguide and \methodwithsandpguide against state-of-the-art PMG baselines in Table~\ref{tbl:overall_results_docking_100}.


%
According to Table~\ref{tbl:overall_results_docking_100}, \methodwithpguide and \methodwithsandpguide achieve comparable performance with the PMG baselines in generating molecules with high binding affinities.
%
Particularly, in terms of Vina S, \methodwithsandpguide achieves very comparable performance (-4.56 kcal/mol) to the third-best baseline \decompdiff (-4.58 kcal/mol) in average Vina S; it also achieves the third-best performance (-4.82 kcal/mol) among all the methods and slightly underperforms the second-best baseline \AR (-4.99 kcal/mol) in median Vina S
%
\methodwithsandpguide also achieves very close average Vina M (-5.53 kcal/mol) with the third-best baseline \AR (-5.59 kcal/mol) and the third-best performance (-5.47 kcal/mol) in median Vina M.
%
Notably, for Vina D, \methodwithpguide and \methodwithsandpguide achieve the second and third performance among all the methods.
%
In terms of the average percentage of generated molecules with Vina D higher than those of known ligands (i.e., HA), \methodwithpguide (58.52\%) and \methodwithsandpguide (58.28\%) outperform the best baseline \targetdiff (57.57\%).
%
These results signify the high utility of \methodwithpguide and \methodwithsandpguide in generating molecules that effectively bind with protein targets and have better binding affinities than known ligands.

In addition to binding affinities, \methodwithpguide and \methodwithsandpguide also demonstrate similar performance compared to the baselines in metrics related to drug-likeness and diversity.
%
For drug-likeness, both \methodwithpguide and \methodwithsandpguide achieve the best (0.67) and the second-best (0.66) QED scores.
%
They also achieve the third and fourth performance in SA scores.
%
In terms of the diversity among generated molecules,  \methodwithpguide and \methodwithsandpguide slightly underperform the baselines, possibly due to the design that generates molecules with similar shapes to the ligands.
%
These results highlight the strong ability of \methodwithpguide and \methodwithsandpguide in efficiently generating effective binding molecules with favorable drug-likeness and diversity.
%
This ability enables them to potentially serve as promising tools to facilitate effective and efficient drug development.

\begin{table*}[!h]
	\centering
		\caption{Additional Comparison on PMG When All Methods Generate 100 Molecules}
	\label{tbl:overall_results_docking_100}
\begin{threeparttable}
	\begin{scriptsize}
	\begin{tabular}{
		@{\hspace{2pt}}l@{\hspace{2pt}}
		%
		@{\hspace{2pt}}r@{\hspace{2pt}}
		%
		@{\hspace{2pt}}r@{\hspace{2pt}}
		@{\hspace{2pt}}r@{\hspace{2pt}}
		%
		@{\hspace{6pt}}r@{\hspace{6pt}}
		%
		@{\hspace{2pt}}r@{\hspace{2pt}}
		@{\hspace{2pt}}r@{\hspace{2pt}}
		%
		@{\hspace{5pt}}r@{\hspace{5pt}}
		%
		@{\hspace{2pt}}r@{\hspace{2pt}}
		@{\hspace{2pt}}r@{\hspace{2pt}}
		%
		@{\hspace{5pt}}r@{\hspace{5pt}}
		%
		@{\hspace{2pt}}r@{\hspace{2pt}}
	         @{\hspace{2pt}}r@{\hspace{2pt}}
		%
		@{\hspace{5pt}}r@{\hspace{5pt}}
		%
		@{\hspace{2pt}}r@{\hspace{2pt}}
		@{\hspace{2pt}}r@{\hspace{2pt}}
		%
		@{\hspace{5pt}}r@{\hspace{5pt}}
		%
		@{\hspace{2pt}}r@{\hspace{2pt}}
		@{\hspace{2pt}}r@{\hspace{2pt}}
		%
		@{\hspace{5pt}}r@{\hspace{5pt}}
		%
		@{\hspace{2pt}}r@{\hspace{2pt}}
		@{\hspace{2pt}}r@{\hspace{2pt}}
		%
		@{\hspace{5pt}}r@{\hspace{5pt}}
		%
		@{\hspace{2pt}}r@{\hspace{2pt}}
		%@{\hspace{2pt}}r@{\hspace{2pt}}
		%@{\hspace{2pt}}r@{\hspace{2pt}}
		}
		\toprule
		\multirow{2}{*}{method} & \multicolumn{2}{c}{Vina S$\downarrow$} & & \multicolumn{2}{c}{Vina M$\downarrow$} & & \multicolumn{2}{c}{Vina D$\downarrow$} & & \multicolumn{2}{c}{{HA\%$\uparrow$}}  & & \multicolumn{2}{c}{QED$\uparrow$} & & \multicolumn{2}{c}{SA$\uparrow$} & & \multicolumn{2}{c}{Div$\uparrow$} & %& \multirow{2}{*}{SR\%$\uparrow$} & 
		& \multirow{2}{*}{time$\downarrow$} \\
	    \cmidrule{2-3}\cmidrule{5-6} \cmidrule{8-9} \cmidrule{11-12} \cmidrule{14-15} \cmidrule{17-18} \cmidrule{20-21}
		 & Avg. & Med. &  & Avg. & Med. &  & Avg. & Med. & & Avg. & Med.  & & Avg. & Med.  & & Avg. & Med.  & & Avg. & Med.  & & \\ %& & \\
		%\multirow{2}{*}{method} & \multirow{2}{*}{\#c\%} &  \multirow{2}{*}{\#u\%} &  \multirow{2}{*}{QED} & \multicolumn{3}{c}{$\nmax=50$} & & \multicolumn{2}{c}{$\nmax=1$}\\
		%\cmidrule(r){5-7} \cmidrule(r){8-10} 
		%& & & & \avgshapesim(std) & \avggraphsim(std  &  \diversity(std  & & \avgshapesim(std) & \avggraphsim(std \\
		\midrule
		Reference                          & -5.32 & -5.66 & & -5.78 & -5.76 & & -6.63 & -6.67 & & - & - & & 0.53 & 0.49 & & 0.77 & 0.77 & & - & - & %& 23.1 & 
		& - \\
		\midrule
		\AR & \textbf{-5.06} & -4.99 & &  -5.59 & -5.29 & &  -6.16 & -6.05 & &  37.69 & 31.00 & &  0.50 & 0.49 & &  0.66 & 0.65 & & 0.70 & 0.70 & %& 7.0 & 
		& 7,789 \\
		\pockettwomol   & -4.50 & -4.21 & &  -5.70 & -5.27 & &  -6.43 & -6.25 & &  48.00 & 51.00 & &  0.58 & 0.58 & &  \textbf{0.77} & \textbf{0.78} & &  0.69 & 0.71 &  %& 24.9 & 
		& 2,150 \\
		\targetdiff     & -4.88 & \textbf{-5.82} & &  \textbf{-6.20} & \textbf{-6.36} & &  \textbf{-7.37} & \textbf{-7.51} & &  57.57 & 58.27 & &  0.50 & 0.51 & &  0.60 & 0.59 & &  \textbf{0.72} & 0.71 & % & 10.4 & 
		& 1,252 \\
		%& \decompdiffbeta                    & 63.03 & %-4.72 & -4.86 & & \textbf{-6.84} & \textbf{-6.91} & & \textbf{-8.85} & \textbf{-8.90} & &  \textbf{72.16} & \textbf{72.16} & &  0.36 & 0.36 & &  0.55 & 0.55 & & 0.59 & 0.59 & & 14.9 \\ 
		%-4.76 & -6.18 & &  \textbf{-6.86} & \textbf{-7.52} & &  \textbf{-8.85} & \textbf{-8.96} & &  \textbf{72.7} & \textbf{89.8} & &  0.36 & 0.34 & &  0.55 & 0.57 & & 0.59 & 0.59 & & 15.4 \\
		\decompdiffref  & -4.58 & -4.77 & &  -5.47 & -5.51 & &  -6.43 & -6.56 & &  47.76 & 48.66 & &  0.56 & 0.56 & &  0.70 & 0.69  & &  \textbf{0.72} & \textbf{0.72} &  %& 15.2 & 
		& 1,859 \\
		\midrule
		%\method & 14.04 & 9.74 & &  -2.80 & -3.87 & &  -6.32 & -6.41 & &  42.37 & 40.40 & &  0.70 & 0.71 & &  0.73 & 0.72 & & 0.71 & 0.74 & & 42 \\
		%\methodwithsguide & 1.04 & -0.33 & &  -4.23 & -4.39 & &  -6.31 & -6.46 & &  46.18 & 44.00 & &  0.69 & 0.71 & &  0.72 & 0.71 & & 0.70 & 0.73 & 53 \\
		\methodwithpguide      & -4.15 & -4.59 & &  -5.41 & -5.34 & &  -6.49 & -6.74 & &  \textbf{58.52} & 59.00 & &  \textbf{0.67} & \textbf{0.69} & &  0.68 & 0.68 & & 0.67 & 0.70 & %& 28.0 & 
		& 48 \\
		\methodwithsandpguide  & -4.56 & -4.82 & &  -5.53 & -5.47 & &  -6.60 & -6.78 & &  58.28 & \textbf{60.00} & &  0.66 & 0.68 & &  0.67 & 0.66 & & 0.68 & 0.71 &
		& 58 \\
		\bottomrule
	\end{tabular}%
	\begin{tablenotes}[normal,flushleft]
		\begin{footnotesize}
	\item 
\!\!Columns represent: {``Vina S'': the binding affinities between the initially generated poses of molecules and the protein pockets; 
		``Vina M'': the binding affinities between the poses after local structure minimization and the protein pockets;
		``Vina D'': the binding affinities between the poses determined by AutoDock Vina~\cite{Eberhardt2021} and the protein pockets;
		``HA'': the percentage of generated molecules with Vina D higher than those of condition molecules;
		``QED'': the drug-likeness score;
		``SA'': the synthesizability score;
		``Div'': the diversity among generated molecules;
		``time'': the time cost to generate molecules.}
		\par
		\par
		\end{footnotesize}
	\end{tablenotes}
	\end{scriptsize}
\end{threeparttable}
\end{table*}


%\label{tbl:overall_results_docking_100}

%-------------------------------------------------------------------------------------------------------------------------------------
%\subsection{{Comparison of Pocket Guidance}}
%\label{supp:app:results:docking}
%-------------------------------------------------------------------------------------------------------------------------------------


\begin{comment}
%-------------------------------------------------------------------------------------------------------------------------------------
\subsection{\ziqi{Simiarity Comparison for Pocket-based Molecule Generation}}
%-------------------------------------------------------------------------------------------------------------------------------------


\begin{table*}[t!]
	\centering
	\caption{{Overall Comparison on Similarity for Pocket-based Molecule Generation}}
	\label{tbl:docking_results_similarity}
	\begin{small}
		\begin{threeparttable}
			\begin{tabular}{
					@{\hspace{0pt}}l@{\hspace{5pt}}
					%
					@{\hspace{3pt}}l@{\hspace{3pt}}
					%
					@{\hspace{3pt}}r@{\hspace{8pt}}
					@{\hspace{3pt}}c@{\hspace{3pt}}
					%
					@{\hspace{3pt}}c@{\hspace{3pt}}
					@{\hspace{3pt}}c@{\hspace{3pt}}
					%
					@{\hspace{0pt}}c@{\hspace{0pt}}
					%
					@{\hspace{3pt}}c@{\hspace{3pt}}
					@{\hspace{3pt}}c@{\hspace{3pt}}
					%
					@{\hspace{3pt}}r@{\hspace{3pt}}
				}
				\toprule
				$\delta_g$  & method          & \#d\%$\uparrow$ & $\diversity_d$$\uparrow$(std) & \avgshapesim$\uparrow$(std) & \avggraphsim$\downarrow$(std) & & \maxshapesim$\uparrow$(std) & \maxgraphsim$\downarrow$(std)       & \#n\%$\uparrow$  \\ 
				\midrule
				%\multirow{6}{0.059\linewidth}{\hspace{0pt}0.1} 
				%& \AR   & 4.4 & 0.781(0.076) & 0.511(0.197) & \textbf{0.056}(0.020) & & 0.619(0.222) & 0.074(0.024) & 21.4  \\
				%& \pockettwomol & 6.6 & 0.795(0.099) & 0.519(0.216) & 0.063(0.020) & & 0.608(0.236) & 0.076(0.022) & \textbf{24.1}  \\
				%& \targetdiff & 2.0 & 0.872(0.041) & 0.619(0.110) & 0.068(0.018) & & 0.721(0.146) & 0.075(0.023) & 17.7  \\
				%& \decompdiffbeta & 0.0 & - & 0.374(0.138) & 0.059(0.031) & & 0.414(0.141) & \textbf{0.058}(0.032) & 9.8  \\
				%& \decompdiffref & 8.5 & 0.805(0.096) & 0.810(0.070) & 0.076(0.018) & & 0.861(0.085) & 0.076(0.020) & 11.3  \\
				%& \methodwithpguide   &  9.9 & \textbf{0.876}(0.041) & 0.795(0.058) & 0.073(0.015) & & 0.869(0.073) & 0.076(0.020) & 17.7  \\
				%& \methodwithsandpguide & \textbf{11.9} & 0.872(0.036) & \textbf{0.813}(0.052) & 0.075(0.014) & & \textbf{0.874}(0.069) & 0.080(0.014) & 17.0  \\
				%\cmidrule{2-10}
				%& improv\% & 40.4$^*$ & 8.8$^*$ & 0.4 & -30.4$^*$ &  & 1.6 & -30.0$^*$ & -26.3$^*$  \\
				%\midrule
				\multirow{7}{0.059\linewidth}{\hspace{0pt}1.0} 
				& \AR & 14.6 & 0.681(0.163) & 0.644(0.119) & 0.236(0.123) & & 0.780(0.110) & 0.284(0.177) & 95.8  \\
				& \pockettwomol & 18.6 & 0.711(0.152) & 0.654(0.131) &   \textbf{0.217}(0.129) & & 0.778(0.121) &   \textbf{0.243}(0.137) &  \textbf{98.3}  \\
				& \targetdiff & 7.1 &  \textbf{0.785}(0.085) & 0.622(0.083) & 0.238(0.122) & & 0.790(0.102) & 0.274(0.158) & 90.4  \\
				%& \decompdiffbeta & 0.1 & 0.589(0.030) & 0.494(0.124) & 0.263(0.143) & & 0.567(0.143) & 0.275(0.162) & 67.7  \\
				& \decompdiffref & 37.3 & 0.721(0.108) & 0.770(0.087) & 0.282(0.130) & & \textbf{0.878}(0.059) & 0.343(0.174) & 83.7  \\
				& \methodwithpguide   &  27.4 & 0.757(0.134) & 0.747(0.078) & 0.265(0.165) & & 0.841(0.081) & 0.272(0.168) & 98.1  \\
				& \methodwithsandpguide &\textbf{45.2} & 0.724(0.142) &   \textbf{0.789}(0.063) & 0.265(0.162) & & 0.876(0.062) & 0.264(0.159) & 97.8  \\
				\cmidrule{2-10}
				& Improv\%  & 21.2$^*$ & -3.6 & 2.5$^*$ & -21.7$^*$ &  & -0.1 & -8.4$^*$ & -0.2  \\
				\midrule
				\multirow{7}{0.059\linewidth}{\hspace{0pt}0.7} 
				& \AR   & 14.5 & 0.692(0.151) & 0.644(0.119) & 0.233(0.116) & & 0.779(0.110) & 0.266(0.140) & 94.9  \\
				& \pockettwomol & 18.6 & 0.711(0.152) & 0.654(0.131) & \textbf{0.217}(0.129) & & 0.778(0.121) & \textbf{0.243}(0.137) & \textbf{98.2}  \\
				& \targetdiff & 7.1 & \textbf{0.786}(0.084) & 0.622(0.083) & 0.238(0.121) & & 0.790(0.101) & 0.270(0.151) & 90.3  \\
				%& \decompdiffbeta & 0.1 & 0.589(0.030) & 0.494(0.124) & 0.263(0.142) & &0.567(0.143) & 0.273(0.156) & 67.6  \\
				& \decompdiffref & 36.2 & 0.721(0.113) & 0.770(0.086) & 0.273(0.123) & & \textbf{0.876}(0.059) & 0.325(0.139) & 82.3  \\
				& \methodwithpguide   &  27.4 & 0.757(0.134) & 0.746(0.078) & 0.263(0.160) & & 0.841(0.081) & 0.271(0.164) & 96.8  \\
				& \methodwithsandpguide      & \textbf{45.0} & 0.732(0.129) & \textbf{0.789}(0.063) & 0.262(0.157) & & \textbf{0.876}(0.063) & 0.262(0.153) & 96.2  \\
				\cmidrule{2-10}
				& Improv\%  & 24.3$^*$ & -3.6 & 2.5$^*$ & -20.8$^*$ &  & 0.0 & -7.6$^*$ & -1.5  \\
				\midrule
				\multirow{7}{0.059\linewidth}{\hspace{0pt}0.5} 
				& \AR   & 14.1 & 0.687(0.160) & 0.639(0.124) & 0.218(0.097) & & 0.778(0.110) & 0.260(0.130) & 89.8  \\
				& \pockettwomol & 18.5 & 0.711(0.152) & 0.649(0.134) & \textbf{0.209}(0.114) & & 0.777(0.121) & \textbf{0.240}(0.131) & \textbf{93.2}  \\
				& \targetdiff & 7.1 & \textbf{0.786}(0.084) & 0.621(0.083) & 0.230(0.111) & & 0.788(0.105) & 0.254(0.127) & 86.5  \\
				%&\decompdiffbeta & 0.1 & 0.595(0.025) & 0.494(0.124) & 0.254(0.129) & & 0.565(0.142) & 0.259(0.138) & 63.9  \\
				& \decompdiffref & 34.7 & 0.730(0.105) & 0.769(0.086) & 0.261(0.109) & & 0.874(0.080) & 0.301(0.117) & 77.3   \\
				& \methodwithpguide  &  27.2 & 0.765(0.123) & 0.749(0.075) & 0.245(0.135) & & 0.840(0.082) & 0.252(0.137) & 88.6  \\
				& \methodwithsandpguide & \textbf{44.3} & 0.738(0.122) & \textbf{0.791}(0.059) & 0.247(0.132) &  & \textbf{0.875}(0.065) & 0.249(0.130) & 88.8  \\
				\cmidrule{2-10}
				& Improv\%   & 27.8$^*$ & -2.7 & 2.9$^*$ & -17.6$^*$ &  & 0.2 & -3.4 & -4.7$^*$  \\
				\midrule
				\multirow{7}{0.059\linewidth}{\hspace{0pt}0.3} 
				& \AR   & 12.2 & 0.704(0.146) & 0.614(0.146) & 0.164(0.059) & & 0.751(0.138) & 0.206(0.059) & 66.4  \\
				& \pockettwomol & 17.1 & 0.731(0.129) & 0.617(0.163) & \textbf{0.155}(0.056) & & 0.740(0.159) & \textbf{0.190}(0.076) & \textbf{71.0}  \\
				& \targetdiff & 6.2 & \textbf{0.809}(0.061) & 0.619(0.087) & 0.181(0.068) & & 0.768(0.119) & 0.196(0.076) & 61.7  \\				
                %& \decompdiffbeta & 0.0 & - & 0.489(0.124) & 0.195(0.080) & & 0.547(0.139) & 0.203(0.087) & 42.0  \\
				& \decompdiffref & 27.7 & 0.775(0.081) & 0.767(0.086) & 0.202(0.062) & & 0.854(0.093) & 0.216(0.068) & 52.6  \\
				& \methodwithpguide   &  24.4 & 0.805(0.084) & 0.763(0.066) & 0.180(0.074) & & 0.847(0.080) & \textbf{0.190}(0.059) & 61.4  \\
				& \methodwithsandpguide & \textbf{36.3} & 0.789(0.081) & \textbf{0.800}(0.056) & 0.181(0.071) & &\textbf{0.878}(0.067) & \textbf{0.190}(0.078) & 61.8  \\
				\cmidrule{2-10}
				& improv\% & 31.1$^*$ & 3.9$^*$ & 4.3$^*$ & -16.5$^*$ &  & 2.8$^*$ & 0.0 & -12.9$^*$  \\
				\bottomrule
			\end{tabular}%
			\begin{tablenotes}[normal,flushleft]
				\begin{footnotesize}
					\item 
					\!\!Columns represent: \ziqi{``$\delta_g$'': the graph similarity constraint; ``\#n\%'': the percentage of molecules that satisfy the graph similarity constraint ($\graphsim<=\delta_g$);
						``\#d\%'': the percentage of molecules that satisfy the graph similarity constraint and are with high \shapesim ($\shapesim>=0.8$);
						``\avgshapesim/\avggraphsim'': the average of shape or graph similarities between the condition molecules and generated molecules with $\graphsim<=\delta_g$;
						``\maxshapesim'': the maximum of shape similarities between the condition molecules and generated molecules with $\graphsim<=\delta_g$;
						``\maxgraphsim'': the graph similarities between the condition molecules and the molecules with the maximum shape similarities and $\graphsim<=\delta_g$;
						``\diversity'': the diversity among the generated molecules.
						%
						``$\uparrow$'' represents higher values are better, and ``$\downarrow$'' represents lower values are better.
						%
						Best values are in \textbf{bold}, and second-best values are \underline{underlined}. 
					} 
					%\todo{double-check the significance value}
					\par
					\par
				\end{footnotesize}
			\end{tablenotes}
		\end{threeparttable}
	\end{small}
	\vspace{-10pt}    
\end{table*}
%\label{tbl:docking_results_similarity}

\bo{@Ziqi you may want to check my edits for the discussion in Table 1 first.
%
If the pocket if known, do you still care about the shape similarity in real applications?
}

\ziqi{Table~\ref{tbl:docking_results_similarity} presents the overall comparison on similarity-based metrics between \methodwithpguide, \methodwithsandpguide and other baselines under different graph similarity constraints  ($\delta_g$=1.0, 0.7, 0.5, 0.3), similar to Table~\ref{tbl:overall}. 
%
As shown in Table~\ref{tbl:docking_results_similarity}, regarding desirable molecules,  \methodwithsandpguide consistently outperforms all the baseline methods in the likelihood of generating desirable molecules (i.e., $\#d\%$).
%
For example, when $\delta_g$=1.0, at $\#d\%$, \methodwithsandpguide (45.2\%) demonstrates significant improvement of $21.2\%$ compared to the best baseline \decompdiff (37.3\%).
%
In terms of $\diversity_d$, \methodwithpguide and \methodwithsandpguide also achieve the second and the third best performance. 
%
Note that the best baseline \targetdiff in $\diversity_d$ achieves the least percentage of desirable molecules (7.1\%), substantially lower than \methodwithpguide and \methodwithsandpguide.
%
This makes its diversity among desirable molecules incomparable with other methods. 
%
When $\delta_g$=0.7, 0.5, and 0.3, \methodwithsandpguide also establishes a significant improvement of 24.3\%, 27.8\%, and 31.1\% compared to the best baseline method \decompdiff.
%
It is also worth noting that the state-of-the-art baseline \decompdiff underperforms \methodwithpguide and \methodwithsandpguide in binding affinities as shown in Table~\ref{tbl:overall_results_docking}, even though it outperforms \methodwithpguide in \#d\%.
%
\methodwithpguide and \methodwithsandpguide also achieve the second and the third best performance in $\diversity_d$ at $\delta_g$=0.7, 0.5, and 0.3. 
%
The superior performance of \methodwithpguide and \methodwithsandpguide in $\#d\%$ at small $\delta_g$ indicates their strong capacity in generating desirable molecules of novel graph structures, thereby facilitating the discovery of novel drug candidates.
%
}

\ziqi{Apart from the desirable molecules, \methodwithpguide and \methodwithsandpguide also demonstrate outstanding performance in terms of the average shape similarities (\avgshapesim) and the average graph similarities (\avggraphsim).
%
Specifically, when $\delta_g$=1.0, \methodwithsandpguide achieves a significant 2.5\% improvement in \avgshapesim\ over the best baseline \decompdiff. 
%
In terms of \avggraphsim, \methodwithsandpguide also achieves higher performance than the baseline \decompdiff of the highest \avgshapesim (0.265 vs 0.282).
%
Please note that all the baseline methods except \decompdiff achieve substantially lower performance in \avgshapesim than \methodwithpguide and \methodwithsandpguide, even though these methods achieve higher \avggraphsim values.
%
This trend remains consistent when applying various similarity constraints (i.e., $\delta_g$) as shown in Table~\ref{tbl:overall_results_docking}.
}

\ziqi{Similarly, \methodwithpguide and \methodwithsandpguide also achieve superior performance in \maxshapesim and \maxgraphsim.
%
Specifically, when $\delta_g$=1.0, for \maxshapesim, \methodwithsandpguide achieves highly comparable performance in \maxshapesim\ compared to the best baseline \decompdiff (0.876 vs 0.878).
%
We also note that \methodwithsandpguide achieves lower \maxgraphsim\ than the \decompdiff with 23.0\% difference. 
%
When $\delta_g$ gets smaller from 0.7 to 0.3, \methodwithsandpguide maintains a high \maxshapesim value around 0.876, while the best baseline \decompdiff has \maxshapesim decreased from 0.878 to 0.854.
%
This demonstrates the superior ability of \methodwithsandpguide in generating molecules with similar shapes and novel structures.
%
}

\ziqi{
In terms of \#n\%, when $\delta_g$=1.0, the percentage of molecules with \graphsim below $\delta_g$ can be interpreted as the percentage of valid molecules among all the generated molecules. 
%
As shown in Table~\ref{tbl:docking_results_similarity}, \methodwithpguide and \methodwithsandpguide are able to generate 98.1\% and 97.8\% of valid molecules, slightly below the best baseline \pockettwomol (98.3\%). 
%
When $\delta_g$=0.7, 0.5, or 0.3, all the methods, including \methodwithpguide and \methodwithsandpguide, can consistently find a sufficient number of novel molecules that meet the graph similarity constraints.
%
The only exception is \decompdiff, which substantially underperforms all the other methods in \#n\%.
}
\end{comment}

%%%%%%%%%%%%%%%%%%%%%%%%%%%%%%%%%%%%%%%%%%%%%
\section{Properties of Molecules in Case Studies for Targets}
\label{supp:app:results:properties}
%%%%%%%%%%%%%%%%%%%%%%%%%%%%%%%%%%%%%%%%%%%%%

%-------------------------------------------------------------------------------------------------------------------------------------
\subsection{Drug Properties of Generated Molecules}
\label{supp:app:results:properties:drug}
%-------------------------------------------------------------------------------------------------------------------------------------

Table~\ref{tbl:drug_property} presents the drug properties of three generated molecules: NL-001, NL-002, and NL-003.
%
As shown in Table~\ref{tbl:drug_property}, each of these molecules has a favorable profile, making them promising drug candidates. 
%
{As discussed in Section ``Case Studies for Targets'' in the main manuscript, all three molecules have high binding affinities in terms of Vina S, Vina M and Vina D, and favorable QED and SA values.
%
In addition, all of them meet the Lipinski's rule of five criteria~\cite{Lipinski1997}.}
%
In terms of physicochemical properties, all these properties of NL-001, NL-002 and NL-003, including number of rotatable bonds, molecule weight, LogP value, number of hydrogen bond doners and acceptors, and molecule charges, fall within the desired range of drug molecules. 
%
This indicates that these molecules could potentially have good solubility and membrane permeability, essential qualities for effective drug absorption.

These generated molecules also demonstrate promising safety profiles based on the predictions from ICM~\cite{Neves2012}.
%
In terms of drug-induced liver injury prediction scores, all three molecules have low scores (0.188 to 0.376), indicating a minimal risk of hepatotoxicity. 
%
NL-001 and NL-002 fall under `Ambiguous/Less concern' for liver injury, while NL-003 is categorized under 'No concern' for liver injury. 
%
Moreover, all these molecules have low toxicity scores (0.000 to 0.236). 
%
NL-002 and NL-003 do not have any known toxicity-inducing functional groups. 
%
NL-001 and NL-003 are also predicted not to include any known bad groups that lead to inappropriate features.
%
These attributes highlight the potential of NL-001, NL-002, and NL-003 as promising treatments for cancers and Alzheimer’s disease.

%\begin{table*}
	\centering
		\caption{Drug Properties of Generated Molecules}
	\label{tbl:binding_drug_mols}
	\begin{scriptsize}
\begin{threeparttable}
	\begin{tabular}{
		@{\hspace{6pt}}r@{\hspace{6pt}}
		@{\hspace{6pt}}r@{\hspace{6pt}}
		@{\hspace{6pt}}r@{\hspace{6pt}}
		@{\hspace{6pt}}r@{\hspace{6pt}}
		@{\hspace{6pt}}r@{\hspace{6pt}}
		@{\hspace{6pt}}r@{\hspace{6pt}}
		@{\hspace{6pt}}r@{\hspace{6pt}}
		@{\hspace{6pt}}r@{\hspace{6pt}}
		@{\hspace{6pt}}r@{\hspace{6pt}}
		%
		}
		\toprule
Target & Molecule & Vina S & Vina M & Vina D & QED   & SA   & Logp  & Lipinski \\
\midrule
\multirow{3}{*}{CDK6} & NL-001 & -6.817      & -7.251    & -8.319     & 0.834 & 0.72 & 1.313 & 5        \\
& NL-002 & -6.970       & -7.605    & -8.986     & 0.851 & 0.74 & 3.196 & 5        \\
\cmidrule{2-9}
& 4AU & 0.736       & -5.939    & -7.592     & 0.773 & 0.79 & 2.104 & 5        \\
\midrule
\multirow{2}{*}{NEP} & NL-003 & -11.953     & -12.165   & -12.308    & 0.772 & 0.57 & 2.944 & 5        \\
\cmidrule{2-9}
& BIR & -9.399      & -9.505    & -9.561     & 0.463 & 0.73 & 2.677 & 5        \\
		\bottomrule
	\end{tabular}%
	\begin{tablenotes}[normal,flushleft]
		\begin{footnotesize}
	\item Columns represent: {``Target'': the names of protein targets;
		``Molecule'': the names of generated molecules and known ligands;
		``Vina S'': the binding affinities between the initially generated poses of molecules and the protein pockets; 
		``Vina M'': the binding affinities between the poses after local structure minimization and the protein pockets;
		``Vina D'': the binding affinities between the poses determined by AutoDock Vina~\cite{Eberhardt2021} and the protein pockets;
		``HA'': the percentage of generated molecules with Vina D higher than those of condition molecules;
		``QED'': the drug-likeness score;
		``SA'': the synthesizability score;
		``Div'': the diversity among generated molecules;
		``time'': the time cost to generate molecules.}
\!\! \par
		\par
		\end{footnotesize}
	\end{tablenotes}
\end{threeparttable}
\end{scriptsize}
  \vspace{-10pt}    
\end{table*}

%\label{tbl:binding_drug_mols}

\begin{table*}
	\centering
		\caption{Drug Properties of Generated Molecules}
	\label{tbl:drug_property}
	\begin{scriptsize}
\begin{threeparttable}
	\begin{tabular}{
		@{\hspace{0pt}}p{0.23\linewidth}@{\hspace{5pt}}
		%
		@{\hspace{1pt}}r@{\hspace{2pt}}
		@{\hspace{2pt}}r@{\hspace{6pt}}
		@{\hspace{6pt}}r@{\hspace{6pt}}
		%
		}
		\toprule
		Property Name & NL-001 & NL-002 & NL-003 \\
		\midrule
Vina S & -6.817 &  -6.970 & -11.953 \\
Vina M & -7.251 & -7.605 & -12.165 \\
Vina D & -8.319 & -8.986 & -12.308 \\
QED    & 0.834  & 0.851  & 0.772 \\
SA       & 0.72    & 0.74    & 0.57    \\
Lipinski & 5 & 5 & 5 \\
%bbbScore          & 3.386                                                                                        & 4.240                                                                                        & 3.892      \\
%drugLikeness      & -0.081                                                                                       & -0.442                                                                                       & -0.325     \\
%molLogP1          & 1.698                                                                                        & 2.685                                                                                        & 2.382      \\
\#rotatable bonds          & 3                                                                                        & 2                                                                                        & 2      \\
molecule weight         & 267.112                                                                                      & 270.117                                                                                      & 390.206    \\
molecule LogP           & 1.698                                                                                        & 2.685                                                                                        & 2.382     \\
\#hydrogen bond doners           & 1                                                                                        & 1                                                                                        & 2      \\
\#hydrogen bond acceptors           & 5                                                                                       & 3                                                                                        & 5      \\
\#molecule charges   & 1                                                                                        & 0                                                                                        & 0      \\
drug-induced liver injury predScore    & 0.227                                                                                        & 0.376                                                                                        & 0.188      \\
drug-induced liver injury predConcern  & Ambiguous/Less concern                                                                       & Ambiguous/Less concern                                                                       & No concern \\
drug-induced liver injury predLabel    & Warnings/Precautions/Adverse reactions & Warnings/Precautions/Adverse reactions & No match   \\
drug-induced liver injury predSeverity & 2                                                                                        & 3                                                                                        & 2      \\
%molSynth1         & 0.254                                                                                        & 0.220                                                                                        & 0.201      \\
%toxicity class         & 0.480                                                                                        & 0.480                                                                                        & 0.450      \\
toxicity names         & hydrazone                                                                                    &   -                                                                                           &   -         \\
toxicity score         & 0.236                                                                                        & 0.000                                                                                        & 0.000      \\
bad groups         & -                                                                                             & Tetrahydroisoquinoline:   allergies                                                          &   -         \\
%MolCovalent       &                                                                                              &                                                                                              &            \\
%MolProdrug        &                                                                                              &                                                                                              &            \\
		\bottomrule
	\end{tabular}%
	\begin{tablenotes}[normal,flushleft]
		\begin{footnotesize}
	\item ``-'': no results found by algorithms
\!\! \par
		\par
		\end{footnotesize}
	\end{tablenotes}
\end{threeparttable}
\end{scriptsize}
  \vspace{-10pt}    
\end{table*}

%\label{tbl:drug_property}

%-------------------------------------------------------------------------------------------------------------------------------------
\subsection{Comparison on ADMET Profiles between Generated Molecules and Approved Drugs}
\label{supp:app:results:properties:admet}
%-------------------------------------------------------------------------------------------------------------------------------------

\paragraph{Generated Molecules for CDK6}
%
Table~\ref{tbl:admet_cdk6} presents the comparison on ADMET profiles between two generated molecules for CDK6 and the approved CDK6 inhibitors, including Abemaciclib~\cite{Patnaik2016}, Palbociclib~\cite{Lu2015}, and Ribociclib~\cite{Tripathy2017}.
%
As shown in Table~\ref{tbl:admet_cdk6}, the generated molecules, NL-001 and NL-002, exhibit comparable ADMET profiles with those of the approved CDK6 inhibitors. 
%
Importantly, both molecules demonstrate good potential in most crucial properties, including Ames mutagenesis, favorable oral toxicity, carcinogenicity, estrogen receptor binding, high intestinal absorption and favorable oral bioavailability.
%
Although the generated molecules are predicted as positive in hepatotoxicity and mitochondrial toxicity, all the approved drugs are also predicted as positive in these two toxicity.
%
This result suggests that these issues might stem from the limited prediction accuracy rather than being specific to our generated molecules.
%
Notably, NL-001 displays a potentially better plasma protein binding score compared to other molecules, which may improve its distribution within the body. 
%
Overall, these results indicate that NL-001 and NL-002 could be promising candidates for further drug development.


\begin{table*}
	\centering
		\caption{Comparison on ADMET Profiles among Generated Molecules and Approved Drugs Targeting CDK6}
	\label{tbl:admet_cdk6}
	\begin{scriptsize}
\begin{threeparttable}
	\begin{tabular}{
		%@{\hspace{0pt}}p{0.23\linewidth}@{\hspace{5pt}}
		%
		@{\hspace{6pt}}l@{\hspace{5pt}}
		@{\hspace{6pt}}r@{\hspace{6pt}}
		@{\hspace{6pt}}r@{\hspace{6pt}}
		@{\hspace{6pt}}r@{\hspace{6pt}}
		@{\hspace{6pt}}r@{\hspace{6pt}}
		@{\hspace{6pt}}r@{\hspace{6pt}}
		%
		%
		@{\hspace{6pt}}r@{\hspace{6pt}}
		%@{\hspace{6pt}}r@{\hspace{6pt}}
		%
		}
		\toprule
		\multirow{2}{*}{Property name} & \multicolumn{2}{c}{Generated molecules} & & \multicolumn{3}{c}{FDA-approved drugs} \\
		\cmidrule{2-3}\cmidrule{5-7}
		 & NL--001 & NL--002 & & Abemaciclib & Palbociclib & Ribociclib \\
		\midrule
\rowcolor[HTML]{D2EAD9}Ames   mutagenesis                             & --   &  --  & & + &  --  & --  \\
\rowcolor[HTML]{D2EAD9}Acute oral toxicity (c)                           & III & III & &  III          & III          & III         \\
Androgen receptor binding                         & +                          & +            &              & +            & +            & +             \\
Aromatase binding                                 & +                          & +            &              & +            & +            & +            \\
Avian toxicity                                    & --                          & --          &                & --            & --            & --            \\
Blood brain barrier                               & +                          & +            &              & +            & +            & +            \\
BRCP inhibitior                                   & --                          & --          &                & --            & --            & --            \\
Biodegradation                                    & --                          & --          &                & --            & --            & --           \\
BSEP inhibitior            & +                          & +            &              & +            & +            & +        \\
Caco-2                                            & +                          & +            &              & --            & --            & --            \\
\rowcolor[HTML]{D2EAD9}Carcinogenicity (binary)                          & --                          & --             &             & --            & --            & --          \\
\rowcolor[HTML]{D2EAD9}Carcinogenicity (trinary)                         & Non-required               & Non-required   &            & Non-required & Non-required & Non-required  \\
Crustacea aquatic toxicity & --                          & --            &              & --            & --            & --            \\
 CYP1A2 inhibition                                 & +                          & +            &              & --            & --            & +             \\
CYP2C19 inhibition                                & --                          & +            &              & +            & --            & +            \\
CYP2C8 inhibition                                 & --                          & --           &               & +            & +            & +            \\
CYP2C9 inhibition                                 & --                          & --           &               & --            & --            & +             \\
CYP2C9 substrate                                  & --                          & --           &               & --            & --            & --            \\
CYP2D6 inhibition                                 & --                          & --           &               & --            & --            & --            \\
CYP2D6 substrate                                  & --                          & --           &               & --            & --            & --            \\
CYP3A4 inhibition                                 & --                          & +            &              & --            & --            & --            \\
CYP3A4 substrate                                  & +                          & --            &              & +            & +            & +            \\
\rowcolor[HTML]{D2EAD9}CYP inhibitory promiscuity                        & +                          & +                    &      & +            & --            & +            \\
Eye corrosion                                     & --                          & --           &               & --            & --            & --            \\
Eye irritation                                    & --                          & --           &               & --            & --            & --             \\
\rowcolor[HTML]{D8E7FF}Estrogen receptor binding                         & +                          & +                    &      & +            & +            & +            \\
Fish aquatic toxicity                             & --                          & +            &              & +            & --            & --            \\
Glucocorticoid receptor   binding                 & +                          & +             &             & +            & +            & +            \\
Honey bee toxicity                                & --                          & --           &               & --            & --            & --            \\
\rowcolor[HTML]{D2EAD9}Hepatotoxicity                                    & +                          & +            &              & +            & +            & +             \\
Human ether-a-go-go-related gene inhibition     & +                          & +               &           & +            & --            & --           \\
\rowcolor[HTML]{D8E7FF}Human intestinal absorption                       & +                          & +             &             & +            & +            & +    \\
\rowcolor[HTML]{D8E7FF}Human oral bioavailability                        & +                          & +              &            & +            & +            & +     \\
\rowcolor[HTML]{D2EAD9}MATE1 inhibitior                                  & --                          & --              &            & --            & --            & --    \\
\rowcolor[HTML]{D2EAD9}Mitochondrial toxicity                            & +                          & +                &          & +            & +            & +    \\
Micronuclear                                      & +                          & +                          & +            & +            & +           \\
\rowcolor[HTML]{D2EAD9}Nephrotoxicity                                    & --                          & --             &             & --            & --            & --             \\
Acute oral toxicity                               & 2.325                      & 1.874    &     & 1.870        & 3.072        & 3.138        \\
\rowcolor[HTML]{D8E7FF}OATP1B1 inhibitior                                & +                          & +              &            & +            & +            & +             \\
\rowcolor[HTML]{D8E7FF}OATP1B3 inhibitior                                & +                          & +              &            & +            & +            & +             \\
\rowcolor[HTML]{D2EAD9}OATP2B1 inhibitior                                & --                          & --             &             & --            & --            & --             \\
OCT1 inhibitior                                   & --                          & --        &                  & +            & --            & +             \\
OCT2 inhibitior                                   & --                          & --        &                  & --            & --            & +             \\
P-glycoprotein inhibitior                         & --                          & --        &                  & +            & +            & +     \\
P-glycoprotein substrate                          & --                          & --        &                  & +            & +            & +     \\
PPAR gamma                                        & +                          & +          &                & +            & +            & +      \\
\rowcolor[HTML]{D8E7FF}Plasma protein binding                            & 0.359        & 0.745     &    & 0.865        & 0.872        & 0.636       \\
Reproductive toxicity                             & +                          & +          &                & +            & +            & +           \\
Respiratory toxicity                              & +                          & +          &                & +            & +            & +         \\
Skin corrosion                                    & --                          & --        &                  & --            & --            & --           \\
Skin irritation                                   & --                          & --        &                  & --            & --            & --         \\
Skin sensitisation                                & --                          & --        &                  & --            & --            & --          \\
Subcellular localzation                           & Mitochondria               & Mitochondria  &             & Lysosomes    & Mitochondria & Mitochondria \\
Tetrahymena pyriformis                            & 0.398                      & 0.903         &             & 1.033        & 1.958        & 1.606         \\
Thyroid receptor binding                          & +                          & +             &             & +            & +            & +           \\
UGT catelyzed                                     & --                          & --           &               & --            & --            & --           \\
\rowcolor[HTML]{D8E7FF}Water solubility                                  & -3.050                     & -3.078              &       & -3.942       & -3.288       & -2.673     \\
		\bottomrule
	\end{tabular}%
	\begin{tablenotes}[normal,flushleft]
		\begin{footnotesize}
	\item Blue cells highlight crucial properties where a negative outcome (``--'') is desired; for acute oral toxicity (c), a higher category (e.g., ``III'') is desired; and for carcinogenicity (trinary), ``Non-required'' is desired.
	%
	Green cells highlight crucial properties where a positive result (``+'') is desired; for plasma protein binding, a lower value is desired; and for water solubility, values higher than -4 are desired~\cite{logs}.
\!\! \par
		\par
		\end{footnotesize}
	\end{tablenotes}
\end{threeparttable}
\end{scriptsize}
  \vspace{--10pt}    
\end{table*}

%\label{tbl:admet_cdk6}

\paragraph{Generated Molecule for NEP}
%
Table~\ref{tbl:admet_nep} presents the comparison on ADMET profiles between a generated molecule for NEP targeting Alzheimer's disease and three approved drugs, Donepezil, Galantamine, and Rivastigmine, for Alzheimer's disease~\cite{Hansen2008}.
%
Overall, NL-003 exhibits a comparable ADMET profile with the three approved drugs.  
%
Notably, same as other approved drugs, NL-003 is predicted to be able to penetrate the blood brain barrier, a crucial property for Alzheimer's disease.
%  
In addition, it demonstrates a promising safety profile in terms of Ames mutagenesis, favorable oral toxicity, carcinogenicity, estrogen receptor binding, high intestinal absorption, nephrotoxicity and so on.
%
These results suggest that NL-003 could be promising candidates for the drug development of Alzheimer's disease.

\begin{table*}
	\centering
		\caption{Comparison on ADMET Profiles among Generated Molecule Targeting NEP and Approved Drugs for Alzhimer's Disease}
	\label{tbl:admet_nep}
	\begin{scriptsize}
\begin{threeparttable}
	\begin{tabular}{
		@{\hspace{6pt}}l@{\hspace{5pt}}
		%
		@{\hspace{6pt}}r@{\hspace{6pt}}
		@{\hspace{6pt}}r@{\hspace{6pt}}
		@{\hspace{6pt}}r@{\hspace{6pt}}
		@{\hspace{6pt}}r@{\hspace{6pt}}
		@{\hspace{6pt}}r@{\hspace{6pt}}
		%
		%
		%@{\hspace{6pt}}r@{\hspace{6pt}}
		%
		}
		\toprule
		\multirow{2}{*}{Property name} & Generated molecule & & \multicolumn{3}{c}{FDA-approved drugs} \\
\cmidrule{2-2}\cmidrule{4-6}
			& NL--003 & & Donepezil	& Galantamine & Rivastigmine \\
		\midrule
\rowcolor[HTML]{D2EAD9} 
Ames   mutagenesis                            & --                      &              & --                                    & --                                 & --                     \\
\rowcolor[HTML]{D2EAD9}Acute oral toxicity (c)                       & III           &                       & III                                  & III                               & II                      \\
Androgen receptor binding                     & +      &      & +            & --         & --         \\
Aromatase binding                             & --     &       & +            & --         & --        \\
Avian toxicity                                & --     &                               & --                                    & --                                 & --                        \\
\rowcolor[HTML]{D8E7FF} 
Blood brain barrier                           & +      &                              & +                                    & +                                 & +                        \\
BRCP inhibitior                               & --     &       & --            & --         & --         \\
Biodegradation                                & --     &                               & --                                    & --                                 & --                        \\
BSEP inhibitior                               & +      &      & +            & --         & --         \\
Caco-2                                        & +      &      & +            & +         & +         \\
\rowcolor[HTML]{D2EAD9} 
Carcinogenicity (binary)                      & --     &                               & --                                    & --                                 & --                        \\
\rowcolor[HTML]{D2EAD9} 
Carcinogenicity (trinary)                     & Non-required    &                     & Non-required                         & Non-required                      & Non-required             \\
Crustacea aquatic toxicity                    & +               &                     & +                                    & +                                 & --                        \\
CYP1A2 inhibition                             & +               &                     & +                                    & --                                 & --                        \\
CYP2C19 inhibition                            & +               &                     & --                                    & --                                 & --                        \\
CYP2C8 inhibition                             & +               &                     & --                                    & --                                 & --                        \\
CYP2C9 inhibition                             & --              &                      & --                                    & --                                 & --                        \\
CYP2C9 substrate                              & --              &                      & --                                    & --                                 & --                        \\
CYP2D6 inhibition                             & --              &                      & +                                    & --                                 & --                        \\
CYP2D6 substrate                              & --              &                      & +                                    & +                                 & +                        \\
CYP3A4 inhibition                             & --              &                      & --                                    & --                                 & --                        \\
CYP3A4 substrate                              & +               &                     & +                                    & +                                 & --                        \\
\rowcolor[HTML]{D2EAD9} 
CYP inhibitory promiscuity                    & +               &                     & +                                    & --                                 & --                        \\
Eye corrosion                                 & --     &       & --            & --         & --         \\
Eye irritation                                & --     &       & --            & --         & --         \\
Estrogen receptor binding                     & +      &      & +            & --         & --         \\
Fish aquatic toxicity                         & --     &                               & +                                    & +                                 & +                        \\
Glucocorticoid receptor binding             & --      &      & +            & --         & --         \\
Honey bee toxicity                            & --    &                                & --                                    & --                                 & --                        \\
\rowcolor[HTML]{D2EAD9} 
Hepatotoxicity                                & +     &                               & +                                    & --                                 & --                        \\
Human ether-a-go-go-related gene inhibition & +       &     & +            & --         & --         \\
\rowcolor[HTML]{D8E7FF} 
Human intestinal absorption                   & +     &                               & +                                    & +                                 & +                        \\
\rowcolor[HTML]{D8E7FF} 
Human oral bioavailability                    & --    &                                & +                                    & +                                 & +                        \\
\rowcolor[HTML]{D2EAD9} 
MATE1 inhibitior                              & --    &                                & --                                    & --                                 & --                        \\
\rowcolor[HTML]{D2EAD9} 
Mitochondrial toxicity                        & +     &                               & +                                    & +                                 & +                        \\
Micronuclear                                  & +     &       & --            & --         & +         \\
\rowcolor[HTML]{D2EAD9} 
Nephrotoxicity                                & --    &                                & --                                    & --                                 & --                        \\
Acute oral toxicity                           & 2.704  &      & 2.098        & 2.767     & 2.726     \\
\rowcolor[HTML]{D8E7FF} 
OATP1B1 inhibitior                            & +      &                              & +                                    & +                                 & +                        \\
\rowcolor[HTML]{D8E7FF} 
OATP1B3 inhibitior                            & +      &                              & +                                    & +                                 & +                        \\
\rowcolor[HTML]{D2EAD9} 
OATP2B1 inhibitior                            & --     &                               & --                                    & --                                 & --                        \\
OCT1 inhibitior                               & +      &      & +            & --         & --         \\
OCT2 inhibitior                               & --     &       & +            & --         & --         \\
P-glycoprotein inhibitior                     & +      &      & +            & --         & --         \\
\rowcolor[HTML]{D8E7FF} 
P-glycoprotein substrate                      & +      &                              & +                                    & +                                 & --                        \\
PPAR gamma                                    & +      &      & --            & --         & --         \\
\rowcolor[HTML]{D8E7FF} 
Plasma protein binding                        & 0.227   &                             & 0.883                                & 0.230                             & 0.606                    \\
Reproductive toxicity                         & +       &     & +            & +         & +         \\
Respiratory toxicity                          & +       &     & +            & +         & +         \\
Skin corrosion                                & --      &      & --            & --         & --         \\
Skin irritation                               & --      &      & --            & --         & --         \\
Skin sensitisation                            & --      &      & --            & --         & --         \\
Subcellular localzation                       & Mitochondria & &Mitochondria & Lysosomes & Mitochondria  \\
Tetrahymena pyriformis                        & 0.053           &                     & 0.979                                & 0.563                             & 0.702                        \\
Thyroid receptor binding                      & +       &     & +            & +         & --             \\
UGT catelyzed                                 & --      &      & --            & +         & --             \\
\rowcolor[HTML]{D8E7FF} 
Water solubility                              & -3.586   &                            & -2.425                               & -2.530                            & -3.062                       \\
		\bottomrule
	\end{tabular}%
	\begin{tablenotes}[normal,flushleft]
		\begin{footnotesize}
	\item Blue cells highlight crucial properties where a negative outcome (``--'') is desired; for acute oral toxicity (c), a higher category (e.g., ``III'') is desired; and for carcinogenicity (trinary), ``Non-required'' is desired.
	%
	Green cells highlight crucial properties where a positive result (``+'') is desired; for plasma protein binding, a lower value is desired; and for water solubility, values higher than -4 are desired~\cite{logs}.
\!\! \par
		\par
		\end{footnotesize}
	\end{tablenotes}
\end{threeparttable}
\end{scriptsize}
  \vspace{--10pt}    
\end{table*}

%\label{tbl:admet_nep}

\clearpage
%%%%%%%%%%%%%%%%%%%%%%%%%%%%%%%%%%%%%%%%%%%%%
\section{Algorithms}
\label{supp:algorithms}
%%%%%%%%%%%%%%%%%%%%%%%%%%%%%%%%%%%%%%%%%%%%%

Algorithm~\ref{alg:shapemol} describes the molecule generation process of \method.
%
Given a known ligand \molx, \method generates a novel molecule \moly that has a similar shape to \molx and thus potentially similar binding activity.
%
\method can also take the protein pocket \pocket as input and adjust the atoms of generated molecules for optimal fit and improved binding affinities.
%
Specifically, \method first calculates the shape embedding \shapehiddenmat for \molx using the shape encoder \SEE described in Algorithm~\ref{alg:see_shaperep}.
%
Based on \shapehiddenmat, \method then generates a novel molecule with a similar shape to \molx using the diffusion-based generative model \methoddiff as in Algorithm~\ref{alg:diffgen}.
%
During generation, \method can use shape guidance to directly modify the shape of \moly to closely resemble the shape of \molx.
%
When the protein pocket \pocket is available, \method can also use pocket guidance to ensure that \moly is specifically tailored to closely fit within \pocket.

\begin{algorithm}[!h]
    \caption{\method}
    \label{alg:shapemol}
         %\hspace*{\algorithmicindent} 
	\textbf{Required Input: $\molx$} \\
 	%\hspace*{\algorithmicindent} 
	\textbf{Optional Input: $\pocket$} 
    \begin{algorithmic}[1]
        \FullLineComment{calculate a shape embedding with Algorithm~\ref{alg:see_shaperep}}
        \State $\shapehiddenmat$, $\pc$ = $\SEE(\molx)$
        \FullLineComment{generate a molecule conditioned on the shape embedding with Algorithm~\ref{alg:diffgen}}
         \If{\pocket is not available}
        \State $\moly = \diffgenerative(\shapehiddenmat, \molx)$
        \Else
        \State $\moly = \diffgenerative(\shapehiddenmat, \molx, \pocket)$
        \EndIf
        \State \Return \moly
    \end{algorithmic}
\end{algorithm}
%\label{alg:shapemol}

\begin{algorithm}[!h]
    \caption{\SEE for shape embedding calculation}
    \label{alg:see_shaperep}
    \textbf{Required Input: $\molx$}
    \begin{algorithmic}[1]
        %\Require $\molx$
        \FullLineComment{sample a point cloud over the molecule surface shape}
        \State $\pc$ = $\text{samplePointCloud}(\molx)$
        \FullLineComment{encode the point cloud into a latent embedding (Equation~\ref{eqn:shape_embed})}
        \State $\shapehiddenmat = \SEE(\pc)$
        \FullLineComment{move the center of \pc to zero}
        \State $\pc = \pc - \text{center}(\pc)$
        \State \Return \shapehiddenmat, \pc
    \end{algorithmic}
\end{algorithm}
%\label{alg:see_shaperep}

\begin{algorithm}[!h]
    \caption{\diffgenerative for molecule generation}
    \label{alg:diffgen}
    	\textbf{Required Input: $\molx$, \shapehiddenmat} \\
 	%\hspace*{\algorithmicindent} 
	\textbf{Optional Input: $\pocket$} 
    \begin{algorithmic}[1]
        \FullLineComment{sample the number of atoms in the generated molecule}
        \State $n = \text{sampleAtomNum}(\molx)$
        \FullLineComment{sample initial positions and types of $n$ atoms}
        \State $\{\pos_T\}^n = \mathcal{N}(0, I)$
        \State $\{\atomfeat_T\}^n = C(K, \frac{1}{K})$
        \FullLineComment{generate a molecule by denoising $\{(\pos_T, \atomfeat_T)\}^n$ to $\{(\pos_0, \atomfeat_0)\}^n$}
        \For{$t = T$ to $1$}
            \IndentLineComment{predict the molecule without noise using the shape-conditioned molecule prediction module \molpred}{1.5}
            \State $(\tilde{\pos}_{0,t}, \tilde{\atomfeat}_{0,t}) = \molpred(\pos_t, \atomfeat_t, \shapehiddenmat)$
            \If{use shape guidance and $t > s$}
                \State $\tilde{\pos}_{0,t} = \shapeguide(\tilde{\pos}_{0,t}, \molx)$
                %\State $\tilde{\pos}_{0,t} = \pos^*_{0,t}$
            \EndIf
            \IndentLineComment{sample $(\pos_{t-1}, \atomfeat_{t-1})$ from $(\pos_t, \atomfeat_t)$ and $(\tilde{\pos}_{0,t}, \tilde{\atomfeat}_{0,t})$}{1.5}
            \State $\pos_{t-1} = P(\pos_{t-1}|\pos_t, \tilde{\pos}_{o,t})$
            \State $\atomfeat_{t-1} = P(\atomfeat_{t-1}|\atomfeat_t, \tilde{\atomfeat}_{o,t})$
            \If{use pocket guidance and $\pocket$ is available}
                \State $\pos_{t-1} = \pocketguide(\pos_{t-1}, \pocket)$
                %\State $\pos_{t-1} = \pos_{t-1}^*$
            \EndIf  
        \EndFor
        \State \Return $\moly = (\pos_0, \atomfeat_0)$
    \end{algorithmic}
\end{algorithm}
%\label{alg:diffgen}

%\input{algorithms/train_SE}
%\label{alg:train_se}

%\begin{algorithm}[!h]
    \caption{Training Procedure of \methoddiff}
    \label{alg:diffgen}
    \begin{algorithmic}[1]
        \Require $\shapehiddenmat, \molx, \pocket$
        \FullLineComment{sample the number of atoms in the generated molecule}
    \end{algorithmic}
\end{algorithm}
%\label{alg:train_diff}

%---------------------------------------------------------------------------------------------------------------------
\section{{Equivariance and Invariance}}
\label{supp:ei}
%---------------------------------------------------------------------------------------------------------------------

%.................................................................................................
\subsection{Equivariance}
\label{supp:ei:equivariance}
%.................................................................................................

{Equivariance refers to the property of a function $f(\pos)$ %\bo{is it the property of the function or embedding (x)?} 
that any translation and rotation transformation from the special Euclidean group SE(3)~\cite{Atz2021} applied to a geometric object
$\pos\in\mathbb{R}^3$ is mirrored in the output of $f(\pos)$, accordingly.
%
This property ensures $f(\pos)$ to learn a consistent representation of an object's geometric information, regardless of its orientation or location in 3D space.
%
%As a result, it provides $f(\pos)$ better generalization capabilities~\cite{Jonas20a}.
%
Formally, given any translation transformation $\mathbf{t}\in\mathbb{R}^3$ and rotation transformation $\mathbf{R}\in\mathbb{R}^{3\times3}$ ($\mathbf{R}^{\mathsf{T}}\mathbf{R}=\mathbb{I}$), %\xia{change the font types for $^{\mathsf{T}}$ and $\mathbb{I}$ in the entire manuscript}), 
$f(\pos)$ is equivariant with respect to these transformations %$g$ (\bo{where is $g$...})
if it satisfies
\begin{equation}
f(\mathbf{R}\pos+\mathbf{t}) = \mathbf{R}f(\pos) + \mathbf{t}. %\ \text{where}\ \hiddenpos = f(\pos).
\end{equation}
%
%where $\hiddenpos=f(\pos)$ is the output of $\pos$. 
%
In \method, both \SE and \methoddiff are developed to guarantee equivariance in capturing the geometric features of objects regardless of any translation or rotation transformations, as will be detailed in the following sections.
}

%.................................................................................................
\subsection{Invariance}
\label{supp:ei:invariance}
%.................................................................................................

%In contrast to equivariance, 
Invariance refers to the property of a function that its output {$f(\pos)$} remains constant under any translation and rotation transformations of the input $\pos$. %a geometric object's feature $\pos$.
%
This property enables $f(\pos)$ to accurately capture %a geometric object's 
the inherent features (e.g., atom features for 3D molecules) that are invariant of its orientation or position in 3D space.
%
Formally, $f(\pos)$ is invariant under any translation $\mathbf{t}$ and  rotation $\mathbf{R}$ if it satisfies
%
\begin{equation}
f(\mathbf{R}\pos+\mathbf{t}) = f(\pos).
\end{equation}
%
In \method, both \SE and \methoddiff capture the inherent features of objects in an invariant way, regardless of any translation or rotation transformations, as will be detailed in the following sections.

%%%%%%%%%%%%%%%%%%%%%%%%%%%%%%%%%%%%%%%%%%%%%
\section{Point Cloud Construction}
\label{supp:point_clouds}
%%%%%%%%%%%%%%%%%%%%%%%%%%%%%%%%%%%%%%%%%%%%%

In \method, we represented molecular surface shapes using point clouds (\pc).
%
$\pc$
serves as input to \SE, from which we derive shape latent embeddings.
%
To generate $\pc$, %\bo{\st{create this}}, \bo{generate $\pc$}
we initially generated a molecular surface mesh using the algorithm from the Open Drug Discovery Toolkit~\cite{Wjcikowski2015oddt}.
%
Following this, we uniformly sampled points on the mesh surface with probability proportional to the face area, %\xia{how to uniformly?}, ensuring the sampling is done proportionally to the face area, with
using the algorithm from PyTorch3D~\cite{ravi2020pytorch3d}.
%
This point cloud $\pc$ is then centralized by setting the center of its points to zero.
%
%

%%%%%%%%%%%%%%%%%%%%%%%%%%%%%%%%%%%%%%%%%%%%%
\section{Query Point Sampling}
\label{supp:training:shapeemb}
%%%%%%%%%%%%%%%%%%%%%%%%%%%%%%%%%%%%%%%%%%%%%

As described in Section ``Shape Decoder (\SED)'', the signed distances of query points $z_q$ to molecule surface shape $\pc$ are used to optimize \SE.
%
In this section, we present how to sample these points $z_q$ in 3D space.
%
Particularly, we first determined the bounding box around the molecular surface shape, using the maximum and minimum \mbox{($x$, $y$, $z$)-axis} coordinates for points in our point cloud \pc,
denoted as $(x_\text{min}, y_\text{min}, z_\text{min})$ and $(x_\text{max}, y_\text{max}, z_\text{max})$.
%
We extended this box slightly by defining its corners as \mbox{$(x_\text{min}-1, y_\text{min}-1, z_\text{min}-1)$} and \mbox{$(x_\text{max}+1, y_\text{max}+1, z_\text{max}+1)$}.
%
For sampling $|\mathcal{Z}|$ query points, we wanted an even distribution of points inside and outside the molecule surface shape.
%
%\ziqi{Typically, within this bounding box, molecules occupy only a small portion of volume, which makes it more likely to sample
%points outside the molecule surface shape.}
%
When a bounding box is defined around the molecule surface shape, there could be a lot of empty spaces within the box that the molecule does not occupy due to 
its complex and irregular shape.
%
This could lead to that fewer points within the molecule surface shape could be sampled within the box.
%
Therefore, we started by randomly sampling $3k$ points within our bounding box to ensure that there are sufficient points within the surface.
%
We then determined whether each point lies within the molecular surface, using an algorithm from Trimesh~\footnote{https://trimsh.org/} based on the molecule surface mesh.
%
If there are $n_w$ points found within the surface, we selected $n=\min(n_w, k/2)$ points from these points, 
and randomly choose the remaining 
%\bo{what do you mean by remaining? If all the 3k sampled points are inside the surface, you get no points left.} 
$k-n$ points 
from those outside the surface.
%
For each query point, we determined its signed distance to the molecule surface by its closest distance to points in \pc with a sign indicating whether it is inside the surface.

%%%%%%%%%%%%%%%%%%%%%%%%%%%%%%%%%%%%%%%%%%%%%
\section{Forward Diffusion (\diffnoise)}
\label{supp:forward}
%%%%%%%%%%%%%%%%%%%%%%%%%%%%%%%%%%%%%%%%%%%%%

%===================================================================
\subsection{{Forward Process}}
\label{supp:forward:forward}
%===================================================================

Formally, for atom positions, the probability of $\pos_t$ sampled given $\pos_{t-1}$, denoted as $q(\pos_t|\pos_{t-1})$, is defined as follows,
%\xia{revise the representation, should be $\beta^x_t$ -- note the space} as follows,
%
\begin{equation}
q(\pos_t|\pos_{t-1}) = \mathcal{N}(\pos_t|\sqrt{1-\beta^{\mathtt{x}}_t}\pos_{t-1}, \beta^{\mathtt{x}}_t\mathbb{I}), 
\label{eqn:noiseposinter}
\end{equation}
%
%\xia{should be a comma after the equation. you also missed it. }
%\st{in which} 
where %\hl{$\pos_0$ denotes the original atom position;} \xia{no $\pos_0$ in the equation...}
%$\mathbf{I}$ denotes the identity matrix;
$\mathcal{N}(\cdot)$ is a Gaussian distribution of $\pos_t$ with mean $\sqrt{1-\beta_t^{\mathtt{x}}}\pos_{t-1}$ and covariance $\beta_t^{\mathtt{x}}\mathbf{I}$.
%\xia{what is $\mathcal{N}$? what is $q$? you abused $q$. need to be crystal clear... }
%\bo{Should be $\sim$ not $=$ in the equation}
%
Following Hoogeboom \etal~\cite{hoogeboom2021catdiff}, 
%the forward process for the discrete atom feature $\atomfeat_t\in\mathbb{R}^K$ adds 
%categorical noise into $\atomfeat_{t-1}$ according to a variance schedule $\beta_t^v\in (0, 1)$. %as follows, %\hl{$\betav_t\in (0, 1)$} as follows,
%\xia{presentation...check across the entire manuscript... } as follows,
%
%\ziqi{Formally, 
for atom features, the probability of $\atomfeat_t$ across $K$ classes given $\atomfeat_{t-1}$ is defined as follows,
%
\begin{equation}
q(\atomfeat_t|\atomfeat_{t-1}) = \mathcal{C}(\atomfeat_t|(1-\beta^{\mathtt{v}}_t) \atomfeat_{t-1}+\beta^{\mathtt{v}}_t\mathbf{1}/K),
\label{eqn:noisetypeinter}
\end{equation}
%
where %\hl{$\atomfeat_0$ denotes the original atom positions}; 
$\mathcal{C}$ is a categorical distribution of $\atomfeat_t$ derived from the %by 
noising $\atomfeat_{t-1}$ with a uniform noise $\beta^{\mathtt{v}}_t\mathbf{1}/K$ across $K$ classes.
%adding an uniform noise $\beta^v_t$ to $\atomfeat_{t-1}$ across K classes.
%\xia{there is always a comma or period after the equations. Equations are part of a sentence. you always missed it. }
%\xia{what is $\mathcal{C}$? what does $q$ mean? it is abused. }

Since the above distributions form Markov chains, %} \xia{grammar!}, 
the probability of any $\pos_t$ or $\atomfeat_t$ can be derived from $\pos_0$ or $\atomfeat_0$:
%samples $\mol_0$ as follows,
%
\begin{eqnarray}
%\begin{aligned}
& q(\pos_t|\pos_{0}) & = \mathcal{N}(\pos_t|\sqrt{\cumalpha^{\mathtt{x}}_t}\pos_0, (1-\cumalpha^{\mathtt{x}}_t)\mathbb{I}), \label{eqn:noisepos}\\
& q(\atomfeat_t|\atomfeat_0)  & = \mathcal{C}(\atomfeat_t|\cumalpha^{\mathtt{v}}_t\atomfeat_0 + (1-\cumalpha^{\mathtt{v}}_t)\mathbf{1}/K), \label{eqn:noisetype}\\
& \text{where }\cumalpha^{\mathtt{u}}_t & = \prod\nolimits_{\tau=1}^{t}\alpha^{\mathtt{u}}_\tau, \ \alpha^{\mathtt{u}}_\tau=1 - \beta^{\mathtt{u}}_\tau, \ {\mathtt{u}}={\mathtt{x}} \text{ or } {\mathtt{v}}.\;\;\;\label{eqn:noiseschedule}
%\end{aligned}
\label{eqn:pos_prior}
\end{eqnarray}
%\xia{always punctuations after equations!!! also use ``eqnarray" instead of ``equation" + ``aligned" for multiple equations, each
%with a separate reference numbering...}
%\st{in which}, 
%where \ziqi{$\cumalpha^u_t = \prod_{\tau=1}^{t}\alpha^u_\tau$ and $\alpha^u_\tau=1 - \beta^u_\tau$ ($u$=$x$ or $v$)}.
%\xia{no such notations in the above equations; also subscript $s$ is abused with shape};
%$K$ is the number of categories for atom features.
%
%The details about noise schedules $\beta^x_t$ and $\beta^v_t$ are available in Supplementary Section \ref{XXX}. \ziqi{add trend}
%
Note that $\bar{\alpha}^{\mathtt{u}}_t$ ($\mathtt{u}={\mathtt{x}}\text{ or }{\mathtt{v}}$)
%($u$=$x$ or $v$) 
is monotonically decreasing from 1 to 0 over $t=[1,T]$. %\xia{=???}. 
%
As $t\rightarrow 1$, $\cumalpha^{\mathtt{x}}_t$ and $\cumalpha^{\mathtt{v}}_t$ are close to 1, leading to that $\pos_t$ or $\atomfeat_t$ approximates 
%the original data 
$\pos_0$ or $\atomfeat_0$.
%
Conversely, as  $t\rightarrow T$, $\cumalpha^{\mathtt{x}}_t$ and $\cumalpha^{\mathtt{v}}_t$ are close to 0,
leading to that $q(\pos_T|\pos_{0})$ %\st{$\rightarrow \mathcal{N}(\mathbf{0}, \mathbf{I})$} 
resembles  {$\mathcal{N}(\mathbf{0}, \mathbb{I})$} 
and $q(\atomfeat_T|\atomfeat_0)$ %\st{$\rightarrow \mathcal{C}(\mathbf{I}/K)$} 
resembles {$\mathcal{C}(\mathbf{1}/K)$}.

Using Bayes theorem, the ground-truth Normal posterior of atom positions $p(\pos_{t-1}|\pos_t, \pos_0)$ can be calculated in a
closed form~\cite{ho2020ddpm} as below,
%
\begin{eqnarray}
& p(\pos_{t-1}|\pos_t, \pos_0) = \mathcal{N}(\pos_{t-1}|\mu(\pos_t, \pos_0), \tilde{\beta}^\mathtt{x}_t\mathbb{I}), \label{eqn:gt_pos_posterior_1}\\
&\!\!\!\!\!\!\!\!\!\!\!\mu(\pos_t, \pos_0)\!=\!\frac{\sqrt{\bar{\alpha}^{\mathtt{x}}_{t-1}}\beta^{\mathtt{x}}_t}{1-\bar{\alpha}^{\mathtt{x}}_t}\pos_0\!+\!\frac{\sqrt{\alpha^{\mathtt{x}}_t}(1-\bar{\alpha}^{\mathtt{x}}_{t-1})}{1-\bar{\alpha}^{\mathtt{x}}_t}\pos_t, 
\tilde{\beta}^\mathtt{x}_t\!=\!\frac{1-\bar{\alpha}^{\mathtt{x}}_{t-1}}{1-\bar{\alpha}^{\mathtt{x}}_{t}}\beta^{\mathtt{x}}_t.\;\;\;
\end{eqnarray}
%
%\xia{Ziqi, please double check the above two equations!}
Similarly, the ground-truth categorical posterior of atom features $p(\atomfeat_{t-1}|\atomfeat_{t}, \atomfeat_0)$ can be calculated~\cite{hoogeboom2021catdiff} as below,
%
\begin{eqnarray}
& p(\atomfeat_{t-1}|\atomfeat_{t}, \atomfeat_0) = \mathcal{C}(\atomfeat_{t-1}|\mathbf{c}(\atomfeat_t, \atomfeat_0)), \label{eqn:gt_atomfeat_posterior_1}\\
& \mathbf{c}(\atomfeat_t, \atomfeat_0) = \tilde{\mathbf{c}}/{\sum_{k=1}^K \tilde{c}_k}, \label{eqn:gt_atomfeat_posterior_2} \\
& \tilde{\mathbf{c}} = [\alpha^{\mathtt{v}}_t\atomfeat_t + \frac{1 - \alpha^{\mathtt{v}}_t}{K}]\odot[\bar{\alpha}^{\mathtt{v}}_{t-1}\atomfeat_{0}+\frac{1-\bar{\alpha}^{\mathtt{v}}_{t-1}}{K}], 
\label{eqn:gt_atomfeat_posterior_3}
%\label{eqn:atomfeat_posterior}
\end{eqnarray}
%
%\xia{Ziqi: please double check the above equations!}
%
where $\tilde{c}_k$ denotes the likelihood of $k$-th class across $K$ classes in $\tilde{\mathbf{c}}$; 
$\odot$ denotes the element-wise product operation;
$\tilde{\mathbf{c}}$ is calculated using $\atomfeat_t$ and $\atomfeat_{0}$ and normalized into $\mathbf{c}(\atomfeat_t, \atomfeat_0)$ so as to represent
probabilities. %\xia{is this correct? is $\tilde{c}_k$ always greater than 0?}
%\xia{how is it calculated?}.
%\ziqi{the likelihood distribution $\tilde{c}$ is calculated by $p(\atomfeat_t|\atomfeat_{t-1})p(\atomfeat_{t-1}|\atomfeat_0)$, according to 
%Equation~\ref{eqn:noisetypeinter} and \ref{eqn:noisetype}.
%\xia{need to write the key idea of the above calculation...}
%
The proof of the above equations is available in Supplementary Section~\ref{supp:forward:proof}.

%===================================================================
\subsection{Variance Scheduling in \diffnoise}
\label{supp:forward:variance}
%===================================================================

Following Guan \etal~\cite{guan2023targetdiff}, we used a sigmoid $\beta$ schedule for the variance schedule $\beta_t^{\mathtt{x}}$ of atom coordinates as below,

\begin{equation}
\beta_t^{\mathtt{x}} = \text{sigmoid}(w_1(2 t / T - 1)) (w_2 - w_3) + w_3
\end{equation}
in which $w_i$($i$=1,2, or 3) are hyperparameters; $T$ is the maximum step.
%
We set $w_1=6$, $w_2=1.e-7$ and $w_3=0.01$.
%
For atom types, we used a cosine $\beta$ schedule~\cite{nichol2021} for $\beta_t^{\mathtt{v}}$ as below,

\begin{equation}
\begin{aligned}
& \bar{\alpha}_t^{\mathtt{v}} = \frac{f(t)}{f(0)}, f(t) = \cos(\frac{t/T+s}{1+s} \cdot \frac{\pi}{2})^2\\
& \beta_t^{\mathtt{v}} = 1 - \alpha_t^{\mathtt{v}} = 1 - \frac{\bar{\alpha}_t^{\mathtt{v}} }{\bar{\alpha}_{t-1}^{\mathtt{v}} }
\end{aligned}
\end{equation}
in which $s$ is a hyperparameter and set as 0.01.

As shown in Section ``Forward Diffusion Process'', the values of $\beta_t^{\mathtt{x}}$ and $\beta_t^{\mathtt{v}}$ should be 
sufficiently small to ensure the smoothness of forward diffusion process. In the meanwhile, their corresponding $\bar{\alpha}_t$
values should decrease from 1 to 0 over $t=[1,T]$.
%
Figure~\ref{fig:schedule} shows the values of $\beta_t$ and $\bar{\alpha}_t$ for atom coordinates and atom types with our hyperparameters.
%
Please note that the value of $\beta_{t}^{\mathtt{x}}$ is less than 0.1 for 990 out of 1,000 steps. %\bo{\st{, though it increases when $t$ is close to 1,000}}.
%
This guarantees the smoothness of the forward diffusion process.
%\bo{add $\beta_t^{\mathtt{x}}$ and $\beta_t^{\mathtt{v}}$ in the legend of the figure...}
%\bo{$\beta_t^{\mathtt{v}}$ does not look small when $t$ is close to 1000...}

\begin{figure}
	\begin{subfigure}[t]{.45\linewidth}
		\centering
		\includegraphics[width=.7\linewidth]{figures/var_schedule_beta.pdf}
	\end{subfigure}
	%
	\hfill
	\begin{subfigure}[t]{.45\linewidth}
		\centering
		\includegraphics[width=.7\linewidth]{figures/var_schedule_alpha.pdf}
	\end{subfigure}
	\caption{Schedule}
	\label{fig:schedule}
\end{figure}

%===================================================================
\subsection{Derivation of Forward Diffusion Process}
\label{supp:forward:proof}
%===================================================================

In \method, a Gaussian noise and a categorical noise are added to continuous atom position and discrete atom features, respectively.
%
Here, we briefly describe the derivation of posterior equations (i.e., Eq.~\ref{eqn:gt_pos_posterior_1}, and   \ref{eqn:gt_atomfeat_posterior_1}) for atom positions and atom types in our work.
%
We refer readers to Ho \etal~\cite{ho2020ddpm} and Kong \etal~\cite{kong2021diffwave} %\bo{add XXX~\etal here...} \cite{ho2020ddpm,kong2021diffwave} 	
for a detailed description of diffusion process for continuous variables and Hoogeboom \etal~\cite{hoogeboom2021catdiff} for
%\bo{add XXX~\etal here...} \cite{hoogeboom2021catdiff} for
the description of diffusion process for discrete variables.

For continuous atom positions, as shown in Kong \etal~\cite{kong2021diffwave}, according to Bayes theorem, given $q(\pos_t|\pos_{t-1})$ defined in Eq.~\ref{eqn:noiseposinter} and 
$q(\pos_t|\pos_0)$ defined in Eq.~\ref{eqn:noisepos}, the posterior $q(\pos_{t-1}|\pos_{t}, \pos_0)$ is derived as below (superscript $\mathtt{x}$ is omitted for brevity),

\begin{equation}
\begin{aligned}
& q(\pos_{t-1}|\pos_{t}, \pos_0)  = \frac{q(\pos_t|\pos_{t-1}, \pos_0)q(\pos_{t-1}|\pos_0)}{q(\pos_t|\pos_0)} \\
& =  \frac{\mathcal{N}(\pos_t|\sqrt{1-\beta_t}\pos_{t-1}, \beta_{t}\mathbf{I}) \mathcal{N}(\pos_{t-1}|\sqrt{\bar{\alpha}_{t-1}}\pos_{0}, (1-\bar{\alpha}_{t-1})\mathbf{I}) }{ \mathcal{N}(\pos_{t}|\sqrt{\bar{\alpha}_t}\pos_{0}, (1-\bar{\alpha}_t)\mathbf{I})}\\
& =  (2\pi{\beta_t})^{-\frac{3}{2}} (2\pi{(1-\bar{\alpha}_{t-1})})^{-\frac{3}{2}} (2\pi(1-\bar{\alpha}_t))^{\frac{3}{2}} \times \exp( \\
& -\frac{\|\pos_t - \sqrt{\alpha}_t\pos_{t-1}\|^2}{2\beta_t} -\frac{\|\pos_{t-1} - \sqrt{\bar{\alpha}}_{t-1}\pos_{0} \|^2}{2(1-\bar{\alpha}_{t-1})} \\
& + \frac{\|\pos_t - \sqrt{\bar{\alpha}_t}\pos_0\|^2}{2(1-\bar{\alpha}_t)}) \\
& = (2\pi\tilde{\beta}_t)^{-\frac{3}{2}} \exp(-\frac{1}{2\tilde{\beta}_t}\|\pos_{t-1}-\frac{\sqrt{\bar{\alpha}_{t-1}}\beta_t}{1-\bar{\alpha}_t}\pos_0 \\
& - \frac{\sqrt{\alpha_t}(1-\bar{\alpha}_{t-1})}{1-\bar{\alpha}_t}\pos_{t}\|^2) \\
& \text{where}\ \tilde{\beta}_t = \frac{1-\bar{\alpha}_{t-1}}{1-\bar{\alpha}_t}\beta_t.
\end{aligned}
\end{equation}
%\bo{marked part does not look right to me.}
%\bo{How to you derive from the second equation to the third one?}

Therefore, the posterior of atom positions is derived as below,

\begin{equation}
q(\pos_{t-1}|\pos_{t}, \pos_0)\!\!=\!\!\mathcal{N}(\pos_{t-1}|\frac{\sqrt{\bar{\alpha}_{t-1}}\beta_t}{1-\bar{\alpha}_t}\pos_0 + \frac{\sqrt{\alpha_t}(1-\bar{\alpha}_{t-1})}{1-\bar{\alpha}_t}\pos_{t}, \tilde{\beta}_t\mathbf{I}).
\end{equation}

For discrete atom features, as shown in Hoogeboom \etal~\cite{hoogeboom2021catdiff} and Guan \etal~\cite{guan2023targetdiff},
according to Bayes theorem, the posterior $q(\atomfeat_{t-1}|\atomfeat_{t}, \atomfeat_0)$ is derived as below (supperscript $\mathtt{v}$ is omitted for brevity),

\begin{equation}
\begin{aligned}
& q(\atomfeat_{t-1}|\atomfeat_{t}, \atomfeat_0) =  \frac{q(\atomfeat_t|\atomfeat_{t-1}, \atomfeat_0)q(\atomfeat_{t-1}|\atomfeat_0)}{\sum_{\scriptsize{\atomfeat}_{t-1}}q(\atomfeat_t|\atomfeat_{t-1}, \atomfeat_0)q(\atomfeat_{t-1}|\atomfeat_0)} \\
%& = \frac{\mathcal{C}(\atomfeat_t|(1-\beta_t)\atomfeat_{t-1} + \beta_t\frac{\mathbf{1}}{K}) \mathcal{C}(\atomfeat_{t-1}|\bar{\alpha}_{t-1}\atomfeat_0+(1-\bar{\alpha}_{t-1})\frac{\mathbf{1}}{K})} \\
\end{aligned}
\end{equation}

For $q(\atomfeat_t|\atomfeat_{t-1}, \atomfeat_0)$, we have % $\atomfeat_t=\atomfeat_{t-1}$ with probability $1-\beta_t+\beta_t / K$, and $\atomfeat_t \neq \atomfeat_{t-1}$
%with probability $\beta_t / K$.
%
%Therefore, this function can be symmetric, that is, 
%
\begin{equation}
\begin{aligned}
q(\atomfeat_t|\atomfeat_{t-1}, \atomfeat_0) & = \mathcal{C}(\atomfeat_t|(1-\beta_t)\atomfeat_{t-1} + \beta_t/{K})\\
& = \begin{cases}
1-\beta_t+\beta_t/K,\!&\text{when}\ \atomfeat_{t} = \atomfeat_{t-1},\\
\beta_t / K,\! &\text{when}\ \atomfeat_{t} \neq \atomfeat_{t-1},
\end{cases}\\
& = \mathcal{C}(\atomfeat_{t-1}|(1-\beta_t)\atomfeat_{t} + \beta_t/{K}).
\end{aligned}
%\mathcal{C}(\atomfeat_{t-1}|(1-\beta_{t})\atomfeat_{t} + \beta_t\frac{\mathbf{1}}{K}).
\end{equation}
%
Therefore, we have
%\bo{why it can be symmetric}
%
\begin{equation}
\begin{aligned}
& q(\atomfeat_t|\atomfeat_{t-1}, \atomfeat_0)q(\atomfeat_{t-1}|\atomfeat_0) \\
& = \mathcal{C}(\atomfeat_{t-1}|(1-\beta_t)\atomfeat_{t} + \beta_t\frac{\mathbf{1}}{K}) \mathcal{C}(\atomfeat_{t-1}|\bar{\alpha}_{t-1}\atomfeat_0+(1-\bar{\alpha}_{t-1})\frac{\mathbf{1}}{K}) \\
& = [\alpha_t\atomfeat_t + \frac{1 - \alpha_t}{K}]\odot[\bar{\alpha}_{t-1}\atomfeat_{0}+\frac{1-\bar{\alpha}_{t-1}}{K}].
\end{aligned}
\end{equation}
%
%\bo{what is $\tilde{\mathbf{c}}$...}
Therefore, with $q(\atomfeat_t|\atomfeat_{t-1}, \atomfeat_0)q(\atomfeat_{t-1}|\atomfeat_0) = \tilde{\mathbf{c}}$, the posterior is as below,

\begin{equation}
q(\atomfeat_{t-1}|\atomfeat_{t}, \atomfeat_0) = \mathcal{C}(\atomfeat_{t-1}|\mathbf{c}(\atomfeat_t, \atomfeat_0)) = \frac{\tilde{\mathbf{c}}}{\sum_{k}^K\tilde{c}_k}.
\end{equation}

%%%%%%%%%%%%%%%%%%%%%%%%%%%%%%%%%%%%%%%%%%%%%
\section{{Backward Generative Process} (\diffgenerative)}
\label{supp:backward}
%%%%%%%%%%%%%%%%%%%%%%%%%%%%%%%%%%%%%%%%%%%%%

Following Ho \etal~\cite{ho2020ddpm}, with $\tilde{\pos}_{0,t}$, the probability of $\pos_{t-1}$ denoised from $\pos_t$, denoted as $p(\pos_{t-1}|\pos_t)$,
can be estimated %\hl{parameterized} \xia{???} 
by the approximated posterior $p_{\boldsymbol{\Theta}}(\pos_{t-1}|\pos_t, \tilde{\pos}_{0,t})$ as below,
%
\begin{equation}
\begin{aligned}
p(\pos_{t-1}|\pos_t) & \approx p_{\boldsymbol{\Theta}}(\pos_{t-1}|\pos_t, \tilde{\pos}_{0,t}) \\
& = \mathcal{N}(\pos_{t-1}|\mu_{\boldsymbol{\Theta}}(\pos_t, \tilde{\pos}_{0,t}),\tilde{\beta}_t^{\mathtt{x}}\mathbb{I}),
\end{aligned}
\label{eqn:aprox_pos_posterior}
\end{equation}
%
where ${\boldsymbol{\Theta}}$ is the learnable parameter; $\mu_{\boldsymbol{\Theta}}(\pos_t, \tilde{\pos}_{0,t})$ is an estimate %estimation
of $\mu(\pos_t, \pos_{0})$ by replacing $\pos_0$ with its estimate $\tilde{\pos}_{0,t}$ 
in Equation~{\ref{eqn:gt_pos_posterior_1}}.
%
Similarly, with $\tilde{\atomfeat}_{0,t}$, the probability of $\atomfeat_{t-1}$ denoised from $\atomfeat_t$, denoted as $p(\atomfeat_{t-1}|\atomfeat_t)$, 
can be estimated %\hl{parameterized} 
by the approximated posterior $p_{\boldsymbol{\Theta}}(\atomfeat_{t-1}|\atomfeat_t, \tilde{\atomfeat}_{0,t})$ as below,
%
\begin{equation}
\begin{aligned}
p(\atomfeat_{t-1}|\atomfeat_t)\approx p_{\boldsymbol{\Theta}}(\atomfeat_{t-1}|\atomfeat_{t}, \tilde{\atomfeat}_{0,t}) 
=\mathcal{C}(\atomfeat_{t-1}|\mathbf{c}_{\boldsymbol{\Theta}}(\atomfeat_t, \tilde{\atomfeat}_{0,t})),\!\!\!\!
\end{aligned}
\label{eqn:aprox_atomfeat_posterior}
\end{equation}
%
where $\mathbf{c}_{\boldsymbol{\Theta}}(\atomfeat_t, \tilde{\atomfeat}_{0,t})$ is an estimate of $\mathbf{c}(\atomfeat_t, \atomfeat_0)$
by replacing $\atomfeat_0$  
with its estimate $\tilde{\atomfeat}_{0,t}$ in Equation~\ref{eqn:gt_atomfeat_posterior_1}.



%===================================================================
\section{\method Loss Function Derivation}
\label{supp:training:loss}
%===================================================================

In this section, we demonstrate that a step weight $w_t^{\mathtt{x}}$ based on the signal-to-noise ratio $\lambda_t$ should be 
included into the atom position loss (Eq.~\ref{eqn:diff:obj:pos}).
%
In the diffusion process for continuous variables, the optimization problem is defined 
as below~\cite{ho2020ddpm},
%
\begin{equation*}
\begin{aligned}
& \arg\min_{\boldsymbol{\Theta}}KL(q(\pos_{t-1}|\pos_t, \pos_0)|p_{\boldsymbol{\Theta}}(\pos_{t-1}|\pos_t, \tilde{\pos}_{0,t})) \\
& = \arg\min_{\boldsymbol{\Theta}} \frac{\bar{\alpha}_{t-1}(1-\alpha_t)}{2(1-\bar{\alpha}_{t-1})(1-\bar{\alpha}_{t})}\|\tilde{\pos}_{0, t}-\pos_0\|^2 \\
& = \arg\min_{\boldsymbol{\Theta}} \frac{1-\alpha_t}{2(1-\bar{\alpha}_{t-1})\alpha_{t}} \|\tilde{\boldsymbol{\epsilon}}_{0,t}-\boldsymbol{\epsilon}_0\|^2,
\end{aligned}
\end{equation*}
where $\boldsymbol{\epsilon}_0 = \frac{\pos_t - \sqrt{\bar{\alpha}_t}\pos_0}{\sqrt{1-\bar{\alpha}_t}}$ is the ground-truth noise variable sampled from $\mathcal{N}(\mathbf{0}, \mathbf{1})$ and is used to sample $\pos_t$ from $\mathcal{N}(\pos_t|\sqrt{\cumalpha_t}\pos_0, (1-\cumalpha_t)\mathbf{I})$ in Eq.~\ref{eqn:noisetype};
$\tilde{\boldsymbol{\epsilon}}_0 = \frac{\pos_t - \sqrt{\bar{\alpha}_t}\tilde{\pos}_{0, t}}{\sqrt{1-\bar{\alpha}_t}}$ is the predicted noise variable. 

%A simplified training objective is proposed by Ho \etal~\cite{ho2020ddpm} as below,
Ho \etal~\cite{ho2020ddpm} further simplified the above objective as below and
demonstrated that the simplified one can achieve better performance:
%
\begin{equation}
\begin{aligned}
& \arg\min_{\boldsymbol{\Theta}} \|\tilde{\boldsymbol{\epsilon}}_{0,t}-\boldsymbol{\epsilon}_0\|^2 \\
& = \arg\min_{\boldsymbol{\Theta}} \frac{\bar{\alpha}_t}{1-\bar{\alpha}_t}\|\tilde{\pos}_{0,t}-\pos_0\|^2,
\end{aligned}
\label{eqn:supp:losspos}
\end{equation}
%
where $\lambda_t=\frac{\bar{\alpha}_t}{1-\bar{\alpha}_t}$ is the signal-to-noise ratio.
%
While previous work~\cite{guan2023targetdiff} applies uniform step weights across
different steps, we demonstrate that a step weight should be included into the atom position loss according to Eq.~\ref{eqn:supp:losspos}.
%
However, the value of $\lambda_t$ could be very large when $\bar{\alpha}_t$ is close to 1 as $t$ approaches 1.
%
Therefore, we clip the value of $\lambda_t$ with threshold $\delta$ in Eq.~\ref{eqn:diff:obj:pos}.




\end{document}
