\section{Related Works}
\label{sec:firstpage}

Searches were carried out in the BDTD (Base of National Theses and Dissertations), as well as in Google Scholar, in the events WIT (Women in Technology), the CBIE (Brazilian Congress of Informatics in Education) and the ANPED (National Association of Graduate and Research in Education), with the aim of obtaining studies that analyze the difficulties faced by students in computing courses in the national scenario. The quality criteria used in the selection of works were rigor, credibility and relevance.

\cite{Santos:21} and \cite{maia_2016} analyzed the female scenario of higher IT courses in Brazil, concluding the evident inequality of freshmen and graduates from the years 2014 to 2019.

For \cite{Paganini:20}, in her study on the low participation of women in Hackathon competitions, analyzing points such as motivational aspects and gender problems are key issues to find ways to understand this female scarcity in events, as well as the lack of of empirical evidence studies on the reasons for this.

However, unlike works presented previously, this article presents an analysis of female participation in Programming Marathons, which are high-performance competitions, showing the importance of this research, never before carried out in Brazil, seeking to think of ways to expand this representativeness and female participation.