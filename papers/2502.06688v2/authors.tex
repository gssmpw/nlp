% Author: Patrik Goldschmidt
% Date: 2025-02-10

\begin{wrapfigure}{l}{0.33\linewidth}
    \vspace{-12pt}
    \includegraphics[width=\linewidth]{authors/goldschmidt.jpg}
    \vspace{-15pt}
\end{wrapfigure}
\noindent
\textbf{Patrik Goldschmidt} is a Ph.D. candidate at the Faculty of Information Technology (FIT) at Brno University of Technology (BUT), Czechia and a research assistant at Kempelen Institute of Intelligent Technologies (KInIT), Slovakia. With an interest in cybersecurity and computer networking since high school, he graduated with Bachelor's and Master's degrees from FIT BUT, focusing on DDoS detection and mitigation. In his Ph.D., Patrik studies the application of artificial intelligence and machine learning for cybersecurity, notably Network Intrusion Detection Systems (NIDSs), to enhance their performance, robustness, and trustworthiness while aiming to bridge the gap between academic research and practice.

\vspace{1em}

\begin{wrapfigure}{l}{0.33\linewidth}
    \includegraphics[width=\linewidth]{authors/chuda.jpg}
    \vspace{-15pt}
\end{wrapfigure}
\noindent
\textbf{Daniela Chud\'a} is an associate professor currently working at Faculty of Electrical Engineering and Information Technology (FEI), Slovak University of Technology (STU) and as an expert researcher and a PhD supervisor at the Kempelen Institute of Intelligent Technologies (KInIT). She has authored or co-authored more than 60 publications in scientific journals and conferences, and regularly reviews submissions for international conferences and scientific journals. Her research is focused on  information security and privacy, in particular behavioral biometrics in the context of user authentication, malware analysis and detection, network anomaly detection, phishing and malicious behavior.