\section{Related Work}
\label{sec:related_work}

This section compares the presented survey with other works discussing network intrusion detection datasets. While data-specific NID surveys (this paper) are rare, general NID surveys often include datasets as well. Naturally, they typically lack detailed discussion on data-specific topics yet still provide a relevant reference point for many readers. We thus compare our work to those focused explicitly on NID data and five general surveys with data-related sections, summarized in Table~\ref{tab:related_work_comparison}. Based on the related work analysis, we conclude that our paper:

\begin{table*}[t]
\tabcolsep=0.09cm
    \centering
    \small
    \caption{Comparison of related work with this study. As depicted, our study covers most NID datasets (with the bonus property that all of them are publicly available), while it is the only study to perform a Systematic Literature Review (SLR) while specifically focusing on the data.}
    \vspace*{1em}
    \begin{tabular}{>{\raggedright\arraybackslash}m{4cm}  >{\centering\arraybackslash}m{1.33cm}  >{\centering\arraybackslash}m{1.35cm}  >{\centering\arraybackslash}m{0.9cm}  >{\centering\arraybackslash}m{1.2cm}  >{\centering\arraybackslash}m{1.45cm}  >{\centering\arraybackslash}m{0.95cm}  >{\centering\arraybackslash}m{1.7cm}  >{\centering\arraybackslash}m{1cm}  >{\centering\arraybackslash}m{1.9cm}}
    
    \textbf{Paper} & \textbf{Covered until} & \textbf{\# NID datasets} & \textbf{SLR-based} & \textbf{Data-focused} & \textbf{Compar-ative} & \textbf{Data limits} & \textbf{Popularity analysis} & \textbf{Data trends} & \textbf{Recommen-dations} \\ \toprule
    
    Ring et al., 2019~\cite{ring2019_nids_datasets_survey} & 2018 & 34 & \ding{55} & \ding{51} & \ding{51} & \ding{55} & \ding{55} & \ding{55} & \ding{51} \\ \midrule
    
    Molina-Coronado et al., 2019~\cite{molinacoronado2020_survey_nids_kdd} & 2018 & 11 & \ding{55} & \ding{55} & \ding{51} & \ding{51} & \ding{51} & \ding{55} & $\circledbullet$ \\ \midrule

    Ferrag et al., 2020~\cite{ferrah2020_deep_learning_ids} & 2018 & 23 & \ding{55} & \ding{55} & \ding{55} & \ding{55} & \ding{51} & \ding{55} & \ding{55} \\ \midrule

    Hindy et al., 2020~\cite{hindy2020_network_threats_taxonomy} & 2018 & 21 & \ding{55} & \ding{51} & \ding{51} & \ding{51} & \ding{51} & $\circledbullet$ & \ding{51} \\ \midrule

    Kenyon et al., 2020~\cite{kenyon2020_public_ids_datasets_fit} & 2019 & 34 & \ding{55} & \ding{51} & \ding{51} & \ding{51} & \ding{55} & $\circledbullet$ & \ding{51} \\ \midrule

    G\"{u}m\"{u}\c{s}ba\c{s} et al., 2021~\cite{gumusbas2021_survey_db_dl_cybersec_ids} & 2020 & 25 & \ding{55} & $\circledbullet$ & $\circledbullet$ & $\circledbullet$ & \ding{55} & \ding{55} & \ding{55} \\ \midrule

    Thakkar et al., 2022~\cite{thakkar2022_ids_survey} & 2020 & 13 & \ding{51} & \ding{55} & \ding{55} & \ding{55} & \ding{55} & \ding{55} & \ding{55} \\ \midrule

    Yang et al., 2022~\cite{yang2022_slr_anids} & 2021 & 52 & \ding{51} & \ding{55} & \ding{51} & \ding{55} & \ding{51} & \ding{55} & \ding{55} \\ \midrule

    This paper & 2023 & 89 & \ding{51} & \ding{51} & \ding{51} & \ding{51} & \ding{51} & \ding{51} & \ding{51} \\
    \bottomrule
    \end{tabular}
    \label{tab:related_work_comparison}
    \vspace*{0.75em}
    {\raggedleft \small Symbols description: \ding{51}: fulfilled, $\circledbullet$: partially fulfilled, \ding{55}: not fulfilled \par}
\end{table*}

\begin{enumerate}[topsep=0.1em,itemsep=0.4em,parsep=0cm]
    \item Is the only Systematic Literature Review (SLR)-based study specifically focused on NID data,
    
    \item Lists the highest number of public datasets, all of which are publicly available with provided download links,
    
    \item Collects the most detailed properties (e.g., type of normal/attack traffic) for the listed datasets,
    
    \item Performs data popularity analysis in a fully transparent way based exclusively on Tier 1 security conferences.
\end{enumerate}

The closest study to ours, and its inspiration in many aspects, is the work by Ring et al.~\cite{ring2019_nids_datasets_survey}. Considered one of the first surveys on NID data, it formalized data properties, compared existing datasets, and provided data-related recommendations. However, since it was published in 2019, it does not include newer datasets. Our study builds on this work by refining some established properties (e.g., removing the data balance property, as NID data are naturally imbalanced) and incorporating newly released datasets. We also introduce sections on data limitations and trends to address recent findings and emphasize data quality, aiming to provide a holistic view of NID data.

The work of Kenyon et al.~\cite{kenyon2020_public_ids_datasets_fit} also overlaps with our paper, as it looks at NID data from different perspectives. The paper discusses various data properties and limitations, compares 34 datasets, and closes with recommendations for NID datasets. In contrast, our study covers more datasets with detailed properties in an SLR-based approach, as well as discusses dataset popularity to better aid in selecting suitable data.

Although Hindy et al.~\cite{hindy2020_network_threats_taxonomy} cover aspects similar to this paper, they primarily focus on the network threat landscape and its relation to 21 NID datasets (until 2018). The study further discusses NIDS algorithms' and datasets' popularity, threat taxonomy, IDS limitations, and recommendations for future research. In contrast, our study includes more datasets, compares them by their properties rather than by attacks, and provides a more detailed discussion of NID data-related aspects.

Yang et al.~\cite{yang2022_slr_anids} conducted an SLR on the NID domain as a whole, highlighting popular ML-based methods and datasets for benchmarking. Although not explicitly data-focused, the study presents the largest published dataset list at the time (52), compares them, and analyzes their popularity, with KDD\textquotesingle99 and NSL-KDD coming as the most popular. In contrast, our SLR specifically focuses on the data, listing more datasets and elaborating on data-specific topics absent in Yang et al.'s work.

As mentioned in Section~\ref{ssec:meth_scope}, our survey focuses on network intrusion datasets regardless of the environment, with the key condition being that the data must be captured from the network. Therefore, we exclude IDS datasets where most of the data originates from non-network sources like end-host devices or sensors, as typical for Industrial Control Systems (ICS) or Internet of Things (IoT) datasets. Although relevant for environment-specific intrusion detection, we consider such datasets out-of-scope. For this reason, we recommend exploring other, more specialized surveys for data in specific areas. For instance, Conti et al.~\cite{conti2021_survey_ics_testbeds_datasets} and Koay et al.~\cite{koay2023_ml_ics_landscape} review ICS testbeds and datasets, whereas IoT-specific datasets are covered by Keersmaeker et al.~\cite{keersmaeker2023_survey_public_iot_datasets} and Kaur et al.~\cite{kaur2023_iot_security_dataset_evolution}.

%%%%%%%%%%%%%%%%%%%%%%%%%%%%%%%%%%%%%%%%%%%%%%%%%%%%%%%%%%%%%%%%%%%%%%%%
%%%%%%%%%%%%%%%%%%%%%%%%%%%%%%%%%%%%%%%%%%%%%%%%%%%%%%%%%%%%%%%%%%%%%%%%