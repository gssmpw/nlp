\section{Related Work}
Extensive research has explored the effects of conversational agents, particularly their impacts on user experience. Our work underscores a significant evolution in how statistical analyses are performed and how developers interact with software tools. While statistical software has been crucial in data analysis, conversational agents are widely used in customer service, healthcare, and education \cite{Folstad2017}. They aim to enhance user productivity through simplified and natural language interaction. In this section we provide a comparative approach to identify and quantify the specific contributions of conversational agents to user satisfaction and operational efficiency.
\subsection{Conversational Agent Applications}
A wide range of current studies investigates how conversational agents are applied successfully in a variety of contexts.\cite{catania2023conversational} In \textit{customer service}, They have proven scalable for improving content curation and aligning with user needs \citep{candello2022}, including machine-teaching strategies that lower the barriers to conversational agent adoption in areas like banking and telecommunications. 

In \textit{healthcare}, researchers have developed tools such as conversational agents to deliver social and emotional support to patients \citep{wang2021, he2023conversational}, while wearable systems such as CommSense enhance patient-clinician interactions by integrating conversational data analytics \cite{wang2024}. Recent advances in large language models (LLMs) further demonstrate their potential to evaluate and improve communication quality in palliative care and HIV mHealth interactions, offering actionable feedback to enhance rapport and empathy \citep{Wang2025, Wang2024chi}. Meanwhile, \textit{public health interventions} leverage AI-driven messaging to boost the persuasiveness and effectiveness of campaigns such as pro-vaccination efforts \cite{karinshak2023}. 

In the field of \textit{education}, conversational agents help refine course evaluations (EVA) and facilitate informal learning—Design. For instance, Quizzer \cite{schmitt2022} structures community feedback to improve visual design skills\cite{peng2024}, and DebateBot \cite{kim2021} fosters structured, collaborative discussions in classrooms \cite{wambsganss2022}. 

For \textit{group collaboration}, multi-agent platforms such as CommunityBots \cite{Jiang2023} and moderator-focused conversational agents facilitate balanced participation, fairness, and improved decision-making \cite{Do2022} and moderator-focused conversational agents\cite{ bagmar2022}, leading to higher user engagement\cite{kim2021}, better response quality\cite{Do2023}, and fewer conversational disruptions compared to single-agent approaches \citep{Jiang2023}. 

In \textit{design and creativity}, cycles—ProtoChat \cite{choi2021} supports iterative feedback and integrates crowd responses to enhance chatbot scripts, and DesignQuizzer \cite{peng2024} guides novice users in applying visual design principles. Finally, in \textit{advisory services}, multi-party conversational agents augment workflows by contributing social presence and adaptive feedback to bolster user trust and perceived competence, thereby improving both client satisfaction and the advisor's professional standing \citep{bucher2024, schmid2022}.

\subsection{Conversational Agent Impact on User Satisfaction}
Recent studies underscore the multifaceted benefits of conversational agents across several dimensions. With respect to \textit{user satisfaction}, conversational agents that follow user-centered design principles have exhibited significant improvements in user enjoyment and trust \citep{schmitt2022}, higher response quality \citep{wambsganss2022}, and enhanced perceived intelligence through explanation strategies \citep{Do2023}. Furthermore, incorporating conversational repair approaches---strategies for agent error resolution that enable user-initiated corrections, clarifications, or challenges—--can alleviate the impact of false-positive errors and thereby improve the discussion experience \citep{Do2022}, while certain response styles can foster deeper engagement \citep{cho2020}, and even paradoxically, a metaphor signaling lower competence can heighten user satisfaction \citep{khadpe2020}. 

In terms of \textit{efficiency and effectiveness}, conversational agents that pose automatic follow-up questions reduce dropout rates and elicit more informative responses \citep{Hu2024}. They have also demonstrated their potential to gather higher-quality input in course evaluations \cite{wambsganss2022} and group moderation settings \cite{bagmar2022}. Also, multi-agent platforms enhance user engagement and input quality \cite{Jiang2023}, along with efficiency, by reducing context switching and cognitive load \cite{Luger2016}. 

To bolster \textit{user trust and acceptance}, researchers emphasize user-centered designs \citep{schmitt2022, Amershi2019}. For example, learning by teaching paradigms for crowdworkers \citep{Chhibber2022}, and strategies such as algorithmic explanation and effective error-repair \citep{Do2023, Do2022}, while a strong social presence can further elevate perceived competence \citep{schmid2022}. 

Lastly, from a \textit{design and implementation} standpoint, measuring productivity in software projects involves assessing code quality, development speed, and developer satisfaction \citep{McConnell2006, Sadowski2015}, prompting researchers to explore alternative metrics. Within these workflows, refining question-asking techniques \citep{Hu2024} and integrating crowd feedback into conversation design \citep{choi2021} have proven fruitful, as has employing AI-generated text to improve message persuasiveness \citep{karinshak2023}. Studies have shown that conversational tools can also enhance the developer experience by providing immediate, context-aware support \citep{Wang2018}. 

Collectively, these findings reveal that well-designed, context-aware conversational agent can substantively enhance user satisfaction, efficiency, trust, collaboration, and design outcomes. Despite these promising efforts, we didn't find a work that assists in statistical analysis and data engineering with a conversational agent, which is the focus of this research.