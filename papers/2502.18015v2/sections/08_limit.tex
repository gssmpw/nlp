% 1. data collection time expensive and require exhaustive computational time (although it automatically proceed), if we have more skills, then it might take more time.
% 2.  single vs multiple object, single environment  (domain-specific) (object, domain generalization)
% 3. fixed object & shape, fixed environment
% 3-a. pre-deinfed relative grasp pose.
% 3-b. predefined environment region with SE(2)
% 4. non-region based skills cannot be usable (e.g., unrealted to object pose)
% 5. 

There are several limitations to our work. First, our work generalizes across different initial and goal object poses, but not across object shapes. Second, it assumes a fixed environment, and cannot generalize across different environment shapes. Third, it takes a significant amount of data collection time, as we need to solve many planning problems. Further, as the number of skills increases, it would take longer as we would need to simulate more skills. Fourth, in our algorithm \lazyskillrrt, we create a connecting node $\connectingnode$ assuming that the object configuration remains the same as in the nearest node $v$. This makes it impossible to use with skills that end with high object velocity such as throwing or batting.

%while our framework handles a fixed object, it cannot be directly applied to different objects within the same environment, limiting its generalizability. Additionally, the skills in our approach are defined for a single object, making the method insufficient for environments containing multiple objects, where interactions between objects must be considered. Another limitation lies in the time-intensive nature of data collection, which requires significant computational resources despite being automated. As the number of skills increases, the data collection process scales proportionally, potentially rendering it impractical for large-scale applications. Addressing these challenges in future work will help extend the applicability of our approach to more complex and diverse scenarios.
