
% % \begin{table}[H]
% \centering
% \begin{adjustbox}{width=0.5\textwidth} % Scales the table to half the page width
% \begin{tabular}{|l|c|p{4cm}|} % Adjust the last column width to fit half-page format
% \hline
% \textbf{Name} & \textbf{Dimension} & \textbf{Explanation} \\
% \hline
% \texttt{joint position} & 9 & Current positions of the robot's joints. \\
% \hline
% \texttt{previous joint position} & 9 & Previous timestep's positions of the robot's joints. \\
% \hline
% \texttt{object keypoints} & 24 & 3D object keypoints. \\
% \hline
% \texttt{EE keypoints} & 24 & 3D robot end-effector keypoints. \\
% \hline
% \texttt{tip positions} & 6 & 3D positions of the left and right finger tips. \\
% \hline
% \texttt{relative EE keypoints} & 24 & 3D robot end-effector keypoints with respect to the object. \\
% \hline
% \texttt{relative tip position} & 6 & 3D positions of the finger tips with respect to the object. \\
% \hline
% \texttt{is gripper executable} & 1 & Binary value indicating whether the gripper is currently executable or not. \\
% \hline
% \makecell[l]{\texttt{previous gripper} \\ \texttt{target position}} & 1 & Previous timestep's robot action. \\
% \hline
% \texttt{goal object keypoints} & 24 & 3D goal object pose Keypoints. \\
% \hline
% \end{tabular}
% \end{adjustbox}
% \vspace{-0.2cm}
% \begin{flushleft}
% \footnotesize
% \caption{State space of distillation policy.}\label{table:IL_state}
% \end{flushleft}
% \vspace{-0.6cm}
% \end{table}

% \begin{table}[H]
% \centering
% \begin{adjustbox}{width=0.5\textwidth}
% \begin{tabular}{|c|c|c|}
% \hline
% \textbf{Extract from} & \textbf{Symbol} & \textbf{Name} \\
% \hline
% \multirow{4}{*}{robot}  & $s_t.q_r \in \mathbb{R}^9$ &robot joint position  \\
% \cline{1-2}
%  & $s_{t-1}.q_r \in \mathbb{R}^9$ & previous robot joint position  \\
% \cline{1-2}
% & $\big((s_t.q_r)^\text{ee}\big)^\text{keypoint} \in \mathbb{R}^{24}$ & robot end-effector keypoint positions. \\
% \cline{1-2}
%  & $(s_t.q_r)^\text{tip} \in \mathbb{R}^6$ & Robot gripper tip positions \\
% \hline
% \multirow{2}{*}{Robot w.r.t. $s_t.q_\text{obj}$}& $\big((s_t.q_r)^\text{rel, ee}\big)^\text{keypoint} \in \mathbb{R}^{24}$ & \makecell{relative robot end-effector \\ keypoint positions w.r.t $s_t.q_\text{obj}$}. \\
% \cline{1-2}
%  & $(s_t.q_r)^\text{rel, tip} \in \mathbb{R}^6$ & \makecell{relative robot gripper \\ tip positions  w.r.t $s_t.q_\text{obj}$} \\
% \hline
% object & $(s_t.q_\text{obj})^\text{keypoint} \in \mathbb{R}^{24}$ &object keypoint positions  \\
% \hline
% simulator & $ (t \mod 12) == 0$ & gripper executable. \\
% \hline
% action & $\in \mathbb{R}$ & Previous timestep's robot gripper width action. \\
% \hline
% target object pose & $(q'_\text{obj})^\text{keypoint}$ &target object keypoint positions \\
% \hline
% \end{tabular}
% \end{adjustbox}
% \begin{flushleft}
% \footnotesize
% \caption{State space of distillation policy.}\label{table:IL_state}
% \end{flushleft}
% \end{table}

\begin{table}[h!]
\centering
\begin{adjustbox}{width=\columnwidth}
\setlength\tabcolsep{20 pt}
\begin{tabular}{|c|c|}
\hline
\textbf{Symbol} & \textbf{Description} \\
\hline
$q^\text{(t)}_r \in \mathbb{R}^9$ &Robot joint position  \\
\cline{1-2}
 $q^\text{(t-1)}_r \in \mathbb{R}^9$ & Previous robot joint position  \\
\cline{1-2}
 $p^\text{(t)}_\text{ee} \in \mathbb{R}^{24}$ & \makecell{Robot end-effector keypoint positions \\ (computed from $q^\text{(t)}_r$)} \\
%  $\big((s_t.q_r)^\text{ee}\big)^\text{keypoint} \in \mathbb{R}^{24}$ & robot end-effector keypoint positions. \\
\cline{1-2}
 $p^\text{(t)}_\text{tip} \in \mathbb{R}^6$ & \makecell{Robot gripper tip positions \\ (computed from $q^\text{(t)}_r$)} \\
\hline
$p_\text{ee-rel}^\text{(t)} \in \mathbb{R}^{24}$ & \makecell{Relative robot end-effector \\ keypoint positions w.r.t $q^\text{(t)}_\text{obj}$} \\
\cline{1-2}
 $p^\text{(t)}_\text{tip-rel}\in \mathbb{R}^6$ & \makecell{Relative robot gripper \\ tip positions w.r.t $q^\text{(t)}_\text{obj}$} \\
\hline
  $p^\text{(t)}_\text{obj} \in \mathbb{R}^{24}$ &\makecell{Object keypoint positions \\ (computed from $q^\text{(t)}_\text{obj}$)}  \\
\hline
%  \mathbbm{1}(t-t_{last_grip_exec}) > 1.2 sec
% $ \big((t \mod 12) == 0\big) \in \{0, 1\}$ & gripper executable. \\
 $ \mathbbm{1}^\text{(t)}_\text{gripper-execute} $ & \makecell{Gripper action executability \\ (executable every 1.2 seconds)} \\
% $ \mathbb{I}\big( (t-t_\text{last\_grip\_exec})> 1.2 sec\big) $ & gripper executable. \\
\hline
 $ a^\text{(t-1)}_\text{width}\in \mathbb{R}$ & \makecell{Previous timestep's \\ robot gripper width action} \\
\hline
$p^\text{g}_\text{obj} \in \mathbb{R}^{24}$ & Goal object keypoint positions \\
\hline
\end{tabular}
\end{adjustbox}
\vspace{-0.2cm}
\begin{flushleft}
\footnotesize
\caption{The components of state space $S$ of distillation policy.}\label{table:IL_state}
\end{flushleft}
\vspace{-1.0cm}
\end{table}

% \begin{table}[H]
% \centering
% \begin{adjustbox}{width=0.5\textwidth} % Scales the table to half the page width
% \begin{tabular}{|l|c|p{4cm}|} % Adjust the last column width to fit half-page format
% \hline
% \textbf{Name} & \textbf{Dimension} & \textbf{Explanation} \\
% \hline
% \texttt{residual joint position} & 7 & The change in the positions of the robot's joints (excluding the two gripper joints). \\ \hline
% \texttt{target gripper position} & 1 & The absolute joint position of the robot's gripper. \\ \hline
% \texttt{joint gain} & 7 & The gain values for the robot's joints (excluding the two gripper joints). \\ \hline
% \texttt{joint damping} & 7 & The damping values for the robot's joints (excluding the two gripper joints). \\ \hline
% \end{tabular}
% \end{adjustbox}
% \vspace{-0.2cm}
% \begin{flushleft}
% \footnotesize
% \caption{Action space of distillation policy}\label{table:IL_action}
% % \textbf{Table \ref{table:IL_action}} — This table describes the action information used by the imitation learning policy.
% \end{flushleft}
% \vspace{-0.6cm}
% \end{table}

\begin{table}[h!]
\centering
% \begin{adjustbox}{width=0.5\textwidth}
\setlength\tabcolsep{20 pt}
\begin{tabular}{|c|c|}
\hline
 \textbf{Symbol} & \textbf{Description} \\
\hline
$\Delta q_\text{joint} \in \mathbb{R}^7$ & \makecell{Changes in the robot joint positions \\ (except the two gripper joints)}  \\
\hline
 $q_\text{width} \in \mathbb{R}$ & Target gripper width  \\
\hline
 $k_p \in \mathbb{R}^{7}$ & \makecell{Proportional gain of robot joints \\ (except the two gripper joints).} \\
 \hline
 $\rho \in \mathbb{R}^{7}$ & \makecell{Damping of robot joints \\ (except the two gripper joints).} \\
\hline
\end{tabular}
% \end{adjustbox}
\vspace{-0.2cm}
\begin{flushleft}
\footnotesize
\caption{The components of action space $A$ of distillation policy.}\label{table:IL_action}
\end{flushleft}
\vspace{-0.6cm}
\end{table}


We train the diffusion policy with a U-Net backbone from the diffusion policy \cite{chi2023diffusion} codebase. To achieve faster inference times, we remove action chunking and state history.

For the state components, as shown in Table \ref{table:IL_state}, we use the previous robot joint position \( q^\text{(t-1)}_r \) instead of the robot joint velocity $\dot{q}^\text{(t)}_r$ due to the large sim-to-real gap in joint velocity. The relative positions of the robot's end-effector keypoints $p^\text{(t)}_\text{ee-rel}$ and gripper tip $p^\text{(t)}_\text{tip-rel}$, with respect to the object pose $q_\text{obj}$, are used to explicitly represent the relationship between the object and the robot. We also use \( a^\text{(t-1)}_\text{width} \) to identify the previous width action in order to provide previous gripper commands for efficient execution. Other state components (\( q^\text{(t)}_r, p^\text{(t)}_\text{ee}, p^\text{(t)}_\text{tip}, p^\text{(t)}_\text{obj}, \) and \( p^\text{g}_\text{obj} \)) correspond to the same components in the state components of the prehensile skill post-contact policy. The state component \( \mathbbm{1}^\text{(t)}_\text{gripper-execute} \) corresponds to the connector policy's state component.

% Additionally, we include a binary indicator to specify whether the gripper is currently controllable, as the gripper of the Franka Research 3 cannot accept new commands until the previous gripper command is completed.

For the action components, as described in Table \ref{table:IL_action}, it consists of four components to perform torque control for non-prehensile manipulation. Their gains and damping values correspond to the same components in the action components of the non-prehensile skill post-contact policy. The distillation policy uses changes in the robot joint positions $\Delta q_\text{joint}$ instead of changes in the end-effector $\Delta q_\text{ee}$, eliminating the need for inverse kinematics (IK) computation and enabling faster inference. Hyperparameters related to training diffusion policies are summarized in Table \ref{table:IL_hyper}. These parameters remain consistent across the card flip, bookshelf, and kitchen domains.