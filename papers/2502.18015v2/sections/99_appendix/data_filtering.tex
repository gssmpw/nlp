We generate skill plans for multiple PNP problems. To increase the success rate, we filter out skill plans based on the replay success rate. The replay success rate is defined as the ratio of successful executions to the total number of trials when re-executing skill plans. To measure the replay success rate of a skill plan, we first create $N$ parallel environments in simulation. For a given skill plan $\skillplan = \{\pi^{(t)},q^{(t)}\}_{t=1}^{T}$, we sequentially execute $\pi^{(t)},q^{(t)}$, where the next skill is executed only when the previous skill succeeds. If all skills are successfully executed and the goal object pose $q^g{\text{obj}}$ is reached, the skill plan is considered successful. We count the total number of successful replays, $N_{success}$, and compute the replay success rate as $N_{success}/N$. If the replay success rate exceeds a threshold $m$, i.e., $N_{success}/N > m$, the skill plan is used for training the distillation policy. 