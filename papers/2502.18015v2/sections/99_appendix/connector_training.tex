To train the connector policy \( \pi_C\), we introduce (1) the initial problem setup and (2) the state space $S$, the action space $A$, and the reward function $R$ used for training. The problem setup for connector policy is provided in Section~\ref{method:train_connector}.
% \Romannum{3}.C.
%~\ref{method:train_connector}.

For training connector policy $\pi_C$, the state space, the action space, and the reward function are defined as follows. 

\begin{itemize}
    \item \textbf{State (\( S \)):} % The state space of connector policy $\pi_C$ consist of variables listed in Table~\ref{table:Connector_state}.
    % \begin{table}[H]
% \centering
% \caption{State Information for Connector Skill Policy}\label{table:Connector_state}
% \begin{adjustbox}{width=0.5\textwidth} % Adjusts the table width to half the page
% \begin{tabular}{|l|c|p{5cm}|}
% \hline
% \textbf{Name} & \textbf{Dimension} & \textbf{Explanation} \\
% \hline
% \texttt{joint\_position} & 9 & Current positions of the robot's joint. \\
% \hline
% \texttt{joint\_velocity} & 9 & Current velocities of the robot's joint. \\
% \hline
% \texttt{object\_keypoints} & 24 & 3D object keypoints.  \\
% \hline
% \texttt{tool\_pose} & 7 & Position and orientation of the tool in 3D space. \\
% \hline
% \texttt{EE\_keypoints} & 24 & Current keypoints representing predefined positions on the end-effector. \\
% \hline
% \texttt{goal\_EE\_keypoints} & 24 & Goal keypoints representing predefined positions on the end-effector. \\
% \hline
% \texttt{tip\_position} & 6 & Current 3D positions of the finger tips. \\
% \hline
% \texttt{goal\_tip\_position} & 6 & Goal 3D positions of the finger tips. \\
% \hline
% \texttt{is\_gripper\_executable} & 1 & Binary value indicating whether the gripper action is currently executable or not. \\
% \hline
% \texttt{previous\_action} & 21 & Previous timestep's robot action. \\
% \hline
% \end{tabular}
% \end{adjustbox}
% \end{table}

\begin{table}[H]
\centering
% \begin{adjustbox}{width=\textwidth} % Adjusts the table width to half the page
\begin{tabular}{|c|c|c|}
\hline
\textbf{Extract from} & \textbf{Symbol} & \textbf{Description} \\
\hline
& $q^\text{(t)}_r\in\mathbb{R}^9$ &robot joint position   \\
\cline{2-3}
\multirow{5}{*}{Robot configuration}
& $\dot{q}^\text{(t)}_r\in\mathbb{R}^9$ &Robot joint velocity  \\
\cline{2-3}
 & $T^\text{(t)}_\text{tool}\in\mathbb{R}^7$& Robot tool pose  \\
\cline{2-3}
 & $p^\text{(t)}_\text{ee}\in\mathbb{R}^{24}$& Robot end-effector keypoint positions \\
 \cline{2-3}
 & $p^\text{(t)}_\text{tip}\in\mathbb{R}^6$& Robot gripper tip positions  \\
 \cline{1-3}
 Object configuration& $p^\text{(t)}_\text{obj}\in\mathbb{R}^{24}$ &Object keypoint positions   \\
\hline
Action & $a^\text{(t-1)}\in\mathbb{R}^{21}$ & Previous action  \\
\hline
Simulator & $\mathbbm{1}^\text{(t)}_\text{gripper-execute} $ & Gripper action executability  \\
\hline
\multirow{2}{*}{Target robot configuration} & $p^\text{g}_\text{ee}\in\mathbb{R}^{24}$ &Target end-effector keypoint positions \\
\cline{2-3}
 & $p^\text{g}_\text{tip}\in\mathbb{R}^{24}$ &Target gripper tip positions  \\
\hline
\end{tabular}
% \end{adjustbox}
\caption{The components of state space $S$ of connector policy $\pi_C$}\label{table:Connector_state}
\end{table}
    The state for the skill's connector policy, \( K.\pi_C \), consists of ten components outlined in Table~\ref{table:Connector_state}. 
    % The state for the Connector Skill Policy \( \pi_C \) consists of nine components: robot joint position \( q^\text{(t)}_r \), robot joint velocity \( \dot{q}^\text{(t)}_r \), robot tool pose \( T^\text{(t)}_\text{tool} \), robot end-effector keypoint positions \( p^\text{(t)}_\text{ee} \), robot gripper tip positions \( p^\text{(t)}_\text{tip} \), object keypoint positions \( p^\text{(t)}_\text{obj} \), previous action \( a^\text{(t-1)} \), target end-effector keypoint positions \( p^\text{g}_\text{ee} \), and target gripper tip positions \( p^\text{g}_\text{tip} \).
    % The robot information includes the joint position \( q^\text{(t)}_r \), joint velocity \( \dot{q}^\text{(t)}_r \), tool pose \( T^\text{(t)}_\text{tool} \), end-effector keypoint positions \( p^\text{(t)}_\text{ee} \), and gripper tip positions \( p^\text{(t)}_\text{tip} \), which together provide a representation of the robot's state.
    % The object information is provided by the object keypoint positions \( p^\text{(t)}_\text{obj} \). The previous timestep action \( a^\text{(t-1)} \) is included to provide sufficient statistics.
    \(\mathbbm{1}^\text{(t)}_\text{gripper-execute}\), provided by the simulator, represents a binary indicator of the executability of the gripper action, where a value of 1 indicates that the action is executable. The gripper action is executable every 1.2 seconds because the Franka Research 3 gripper cannot accept new commands until the previous gripper command is completed. The goal is represented by the target end-effector keypoint positions \( p^\text{g}_\text{ee} \) and the target gripper tip positions \( p^\text{g}_\text{tip} \), which define the desired configurations for the robot to achieve. The remaining state components are identical to the state components of the prehensile skill post-contact policy.

    \medskip

    \item \textbf{Action (\( A \)):} The action \( a \in A \) represents the control inputs applied to the robot. The action consists of four components as show in Table \ref{table:Connector_action}. The delta end-effector pose, the proportional gain for robot joints, and joint damping correspond to the action components of the NP skill's post-contact policy. The fourth component, $q_\text{width}$, is the target gripper tip width. For the robot joints (excluding the gripper tip), the control process follows the same procedure as the post-contact policy control process used in the non-prehensile skill. For the gripper tip, the gripper width is adjusted to \( q_\text{width} \).
    \begin{table}[H]
\centering
% \begin{adjustbox}{width=0.5\textwidth} % Scales the table to half the page width
\begin{tabular}{|c|c|} % Adjust the last column width to fit half-page format
\hline
\textbf{Symbol} & \textbf{Description}  \\
\hline
$\Delta q_\text{ee}\in\mathbb{R}^6$& \text{Delta end-effector pose}  \\ \hline
$k_p\in\mathbb{R}^7$ & \text{Proportional gain}   \\ \hline
$\rho\in\mathbb{R}^7$ & \text{Joint damping}   \\ \hline
$q_\text{width}\in\mathbb{R}$& \text{Target gripper width}  \\ \hline

\end{tabular}
% \end{adjustbox}
\caption{The components of action space $A$ of connector policy $\pi_C$}\label{table:Connector_action}
\end{table}


    \medskip
    \item \textbf{Reward (\( R(s_t, a_t, s_{t+1}) \)):} 
    \begin{enumerate}
        \item \textbf{End-Effector Distance Reward:} Encourages the end-effector to move closer to its target position. Here, $p^\text{(t)}_\text{ee}$ represents the end-effector keypoint positions at timestep $t$, and $p^\text{g}_\text{ee}$ denotes the target end-effector keypoint positions:
        \[
        \begin{aligned}
        r^\text{(t)}_{\text{ee}} = \epsilon^{\text{ee}}_0 \big( &\exp(-\epsilon^{\text{ee}}_1 \| p^\text{(t)}_\text{ee} - p^\text{g}_\text{ee} \|_2) - \exp(-\epsilon^{\text{ee}}_1 \| p^\text{(t-1)}_\text{ee} - p^\text{g}_\text{ee} \|_2) \big)
        \end{aligned}
        \]
    
        \item \textbf{Gripper Tip Position Reward:} Aligns the gripper tips with their target positions. Here, $p^\text{(t)}_\text{ee}$ represents the gripper tip positions at timestep $t$, and $p^\text{g}_\text{tip}$ denotes the target tip positions:
        \[
        \begin{aligned}
        r^\text{(t)}_{\text{tip}} = \epsilon^{\text{tip}}_0 \big( &\exp(-\epsilon^{\text{tip}}_1 \| p^\text{(t)}_\text{ee} - p^\text{g}_\text{tip} \|_2)
        - \exp(-\epsilon^{\text{tip}}_1 \| p^\text{(t-1)}_\text{ee} - p^\text{g}_\text{tip} \|_2) \big)
        \end{aligned}
        \]
    
        \item \textbf{Object Movement Penalty:} Penalizes unnecessary object movement to ensure the preconditions of subsequent skills remain valid. Here, $p^\text{(t)}_\text{obj}$ represents the object keypoint positions at timesteps $t$:
        \[
        r^\text{(t)}_{\text{obj-move}} = -w^{\text{move}} \| p^\text{(t)}_\text{obj} - p^\text{(t-1)}_\text{obj} \|_2
        \]
    
        \item \textbf{Success Reward:} A success reward, \( r_{\text{succ}} \), is given when both the end-effector and gripper tip reach their respective target positions, \( p^\text{g}_\text{ee} \) and \( p^\text{g}_\text{tip} \), within the allowable error thresholds \( \delta_\text{ee} \) and \( \delta_\text{tip} \).
    \[
    r^\text{(t)}_{\text{success}} =
    \begin{cases} 
    r_{\text{succ}} & \text{if } ||p^\text{(t)}_\text{ee} - p^\text{g}_\text{ee}||_2 < \delta_\text{ee}, \text{ and }  ||p^\text{(t)}_\text{tip} - p^\text{g}_\text{tip}||_2 < \delta_\text{tip}, \\
    0 & \text{otherwise}
    \end{cases}
    \]
    \end{enumerate}

    The overall reward is defined as:
    \[
    r^\text{(t)}_{\text{connector}} = r^\text{(t)}_{\text{ee}} + r^\text{(t)}_{\text{tip}} + r^\text{(t)}_{\text{obj-move}} + r^\text{(t)}_{\text{success}}
    \]
    The hyperparameters of the reward function, $\epsilon_0$, $\epsilon_1$, $\omega$, $\epsilon_\text{vel}$, $\delta_\text{ee}$, and $\delta_\text{tip}$, vary depending on the specific connector being trained. The values of these reward hyperparameters for each connector are provided in Table~\ref{table:connector_reward}.
    
\end{itemize}

\begin{table}[H]
\centering
% \begin{adjustbox}{width=\textwidth} % Adjusts the table width to half the page
\begin{tabular}{|l|c|c|c|c|c|c|c|c|c|}
\hline
\textbf{Domain} & \multicolumn{2}{c|}{\textbf{Card Flip}} & \multicolumn{3}{c|}{\textbf{Bookshelf}} & \multicolumn{4}{c|}{\textbf{Kitchen}} \\ \hline
\textbf{Skills} & slide & prehensile & topple & prehensile & push & sink & prehensile & l-cupboard & r-cupboard \\ \hline
$\epsilon^\text{ee}_0$       &  \multicolumn{5}{c|}{40} & 40  & 40 & 40 & 40 \\ \hline
$\epsilon^\text{ee}_1$       &  \multicolumn{5}{c|}{0.9}  & 1.0 & 1.0 & 1.0 & 1.0 \\ \hline
$\epsilon^\text{tip}_0$       &  \multicolumn{5}{c|}{40} & 40 & 40 & 30 & 30\\ \hline
$\epsilon^\text{tip}_1$       & \multicolumn{5}{c|}{1.0}  & 1.0 & 1.0 & 1.0 & 1.0\\ \hline
$\omega^{\text{move}}$ &  \multicolumn{5}{c|}{-0.3}  & -0.3 & -3.0 & -10 & -10\\ \hline
$r_\text{succ}$ &   \multicolumn{5}{c|}{1000}  & 150 & 150 & 100 & 100\\ \hline
$\delta_{\text{ee}}$ & \multicolumn{9}{c|}{0.01}  \\ \hline
$\delta_{\text{tip}}$ & \multicolumn{9}{c|}{0.003}  \\ \hline
\end{tabular}
% \end{adjustbox}
\caption{Reward hyperparameter values for training connector policy}\label{table:connector_reward}
\end{table}


Connector policies utilize a multilayer perceptron (MLP) architecture to generate low-level robot actions based on the state and target robot configuration. Each connector policy \( \pi_C \) employs a five-layer MLP with an input dimension of 131 and an output dimension of 21. Other components of the network architecture are identical to those of the NP skill post-contact policy's network architecture.
% The MLP has hidden dimensions of $[512, 256, 256, 128]$ and an output dimension of 21. ELU is used as the activation function for the hidden layers, while Identity is applied as the final activation function.
\begin{table}[H]
    \fontsize{8}{8}\selectfont
    \centering
    
    % \resizebox{0.8\textwidth}{!}{
\begin{tabular}{c|c|c|c|c|c}
    \toprule
    & \makecell{input\\dimensions}& \makecell{hidden\\dimensions} & \makecell{output\\dimensions} &\makecell{hidden\\activations} &\makecell{output\\activation} \\
    \cmidrule(lr){1-1}\cmidrule(lr){2-2}\cmidrule(lr){3-3}\cmidrule(lr){4-4}\cmidrule(lr){5-5}\cmidrule(lr){6-6}
    $\pi_C$
    & 131
    & {$[512, 256, 256, 128]$}
    & 21
    & {ELU}
    & Identity
    \\
    \bottomrule
    \end{tabular}
    \label{table:connector_MLP_architecture}
    \caption{The network architecture of the connector policy}

\end{table}