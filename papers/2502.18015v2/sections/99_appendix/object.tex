In each experimental environment, we fixed the object to be manipulated as follows:
\begin{itemize}
    \item \textbf{Card:} A rectangular cuboid with dimensions [0.05, 0.07, 0.005] meters.
    \item \textbf{Book:} A rectangular cuboid with dimensions [0.14, 0.20, 0.03] meters.
    \item \textbf{Cup:} A model created based on the real-world cup.
\end{itemize}

% % \begin{table}[ht]
% \centering
% \caption{Objects Images in Simulation and Real World}
% \begin{adjustbox}{max width=\textwidth}
% \begin{tabular}{|>{\centering\arraybackslash}m{0.1\textwidth}|>{\centering\arraybackslash}m{0.3\textwidth}|>{\centering\arraybackslash}m{0.3\textwidth}|}
% \hline
% \textbf{Object} & \textbf{Simulation} & \textbf{Real World} \\
% \hline
% \textbf{Card} & \includegraphics[width=0.25\textwidth]{figures/object/sim_card.png} & \includegraphics[width=0.25\textwidth]{figures/object/real_card.jpg} \\
% \hline
% \textbf{Book} & \includegraphics[width=0.25\textwidth]{figures/object/sim_book.png} & \includegraphics[width=0.25\textwidth]{figures/object/real_book.jpg} \\
% \hline
% \textbf{Cup} & \includegraphics[width=0.25\textwidth]{figures/object/sim_cup.png} & \includegraphics[width=0.25\textwidth]{figures/object/real_cup.jpg} \\
% \hline
% \end{tabular}
% \end{adjustbox}
% \label{tab:objects_sim_real}
% \end{table}

\begin{table}[ht]
\centering
\small
\caption{Objects Images in Simulation and Real World}
\begin{adjustbox}{max width=0.9\textwidth}
\begin{tabular}{|>{\centering\arraybackslash}m{0.1\textwidth}|>{\centering\arraybackslash}m{0.3\textwidth}|>{\centering\arraybackslash}m{0.3\textwidth}|>{\centering\arraybackslash}m{0.3\textwidth}|}
\hline
\textbf{Object} & \textbf{Card} & \textbf{Book} & \textbf{Cup} \\
\hline
\textbf{Simulation} & \includegraphics[width=0.25\textwidth]{figures/object/sim_card_white_background.png} & \includegraphics[width=0.25\textwidth]{figures/object/sim_book.png} & \includegraphics[width=0.25\textwidth]{figures/object/sim_cup.png} \\
\hline
\textbf{Real} & \includegraphics[width=0.25\textwidth]{figures/object/real_card_white_background.png} & \includegraphics[width=0.25\textwidth]{figures/object/real_book_white_background.png} & \includegraphics[width=0.25\textwidth]{figures/object/real_cup_white_backgound.png} \\
\hline
\end{tabular}
\end{adjustbox}
\label{tab:objects_sim_real}
\end{table}

% \begin{figure}[h!] % Adjust to fit within a single column
% \centering
% \includegraphics[width=1.0\columnwidth]{figures/object/object_images_v2.png}
% \vspace{-10mm}
% \caption{Each object image in real world} % Adjust caption text if needed
% \label{fig:object_images} % Place the label after the caption
% \end{figure}

There are two predefined components for each object: (1) keypoints, and (2) relative grasp pose and grasp width.

First, we pre-define eight keypoints (\(x, y, z\) positions for each keypoint) on the object's surface. We can use their pose to represent the object's position and orientation. However, due to the nature of quaternions, which cannot directly compute the distance between two poses in Euclidean space, these keypoints are strategically placed to capture the geometry and spatial configuration of the object. These keypoints correspond to the eight vertices of cuboid-shaped objects (such as the card and book). For the cup, we define eight points along its surface. These keypoint positions are used to calculate distances between two different object poses and state component of the skill policy.

\textcolor{red}{explain in detail how you have pre-defined grasp poses for each objec}
Second, we pre-define the relative grasp poses to an object frame and corresponding gripper width for each object. The relative grasp pose is used to represent the position and orientation of the gripper relative to the object. 

% The grasp width defines the extent to which the gripper opens, which is crucial for determining how the object is manipulated. This can be extended in future work by using additional grasp pose samplers, such as GraspNet \cite{mousavian20196, sundermeyer2021contact}.