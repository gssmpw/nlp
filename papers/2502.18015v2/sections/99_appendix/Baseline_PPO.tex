%Flat RL
% This section describes the baselines: PPO and \texttt{MAPLE}.

Baseline PPO is a flat reinforcement learning method whose training pipeline follows the same process as skill training. To train the policy \( \pi_\text{post} \), we detail: (1) the initial problem setup, (2) the state space \( S \), the action space \( A \), and the reward function \( R \), and (3) a policy architecture used for training.

The problem for training the post-contact policy, \(\pi_\text{post} \), consists of: (1) the initial object pose \( q^\text{init}_\text{obj} \), (2) the initial robot configuration \( q^\text{init}_r \), and (3) the target object pose \( q^\text{g}_\text{obj} \). We randomly sample the initial and desired object poses, the same as in Skill-RRT in Appendix~\ref{Appendix:Problem}.
%We randomly sample the initial and target object poses from \( \Qobj \). 
Subsequently, we compute \( q^\text{init}_r \) by \( \pi_\text{pre}(q^\text{init}_\text{obj},q^\text{g}_\text{obj}) \). If the computed \( q^\text{init}_r \) does not cause collisions, either between the robot and the environment or between the robot and the object, the problem is generated. Otherwise, it is excluded from the dataset.


\begin{itemize}
    \medskip
    \item \textbf{State (\( S \)):} The components of the state space for the PPO policy are identical to those of the distillation policy. However, the PPO policy uses the robot’s joint velocity $\dot{q}^\text{(t)}_r$ instead of the previous joint position $q^\text{(t-1)}_r$ and excludes the gripper action executability $\mathbbm{1}_\text{gripper-execute}$. Additionally, it incorporates not only the previous gripper width action $ a^\text{(t-1)}_\text{width}$ but also all previous actions $a^\text{(t-1)}$ as components of the state space.
    % The state for the PPO policy consists of nine components outlined in Table~\ref{table:PPO_state}.
    % It includes:
    %\begin{table}[h]
\centering
% \begin{adjustbox}{width=0.5\textwidth}
\caption{State configuration of the baseline PPO policy.}\label{table:PPO_state}
\begin{tabular}{|c|c|c|}
\hline
\textbf{Extract From} &  \textbf{Symbol} & \textbf{Description} \\
\hline
\multirow{4}{*}{Robot configuration}& $q^\text{(t)}_r \in \mathbb{R}^9$ &Robot joint position  \\
\cline{2-3}
& $\dot{q}^\text{(t)}_r \in \mathbb{R}^9$ &Robot joint velocity  \\
\cline{2-3}
&  $p^\text{(t)}_\text{ee} \in \mathbb{R}^{24}$ & Robot end-effector keypoint positions. \\
\cline{2-3}
&  $p^\text{(t)}_\text{tip} \in \mathbb{R}^6$ & Robot gripper tip positions \\
\hline
\multirow{2}{*}{Robot and object configuration}& $p^\text{(t)}_\text{ee-rel} \in \mathbb{R}^{24}$ & \makecell{Relative robot end-effector \\ keypoint positions w.r.t $q^\text{(t)}_\text{obj}$}. \\
\cline{2-3}
&  $p^\text{(t)}_\text{tip-rel} \in \mathbb{R}^6$ & \makecell{Relative robot gripper \\ tip positions  w.r.t $q^\text{(t)}_\text{obj}$} \\
\hline
Object configuration&  $p^\text{(t)}_\text{obj} \in \mathbb{R}^{24}$ &Object keypoint positions  \\
\hline
Target object pose & $p^\text{g}_\text{obj} \in \mathbb{R}^{24}$ &Target object keypoint positions \\
\hline
Action & $a^\text{(t-1)} \in \mathbb{R}^{21}$ &Previous action \\
\hline
\end{tabular}
% \end{adjustbox}
\vspace{-0.2cm}
\begin{flushleft}
\footnotesize
\end{flushleft}
\end{table}


    \medskip
    \item \textbf{Action (\( A \)):}
    The action components are identical to action components of connector policy. % The action \( a \in A \) represents the control inputs applied to the robot.
    % \begin{table}[H]
\centering
\caption{Action Information for the Baseline PPO Policy}\label{table:baseline_ppo_action}
% \begin{adjustbox}{width=0.5\textwidth} % Scales the table to half the page width
\begin{tabular}{|c|c|} % Adjust the last column width to fit half-page format
\hline
\textbf{Symbol} & \textbf{Description}  \\
\hline
$\Delta q_\text{ee} \in \mathbb{R}^6$& \text{Delta end-effector pose} \\ \hline
$k_p\in\mathbb{R}^7$ & \text{proportional gain}  \\ \hline
$\rho\in\mathbb{R}^7$ & \text{joint damping}  \\ \hline
$q_\text{width}\in\mathbb{R}$& \text{absolute gripper width} \\ \hline

\end{tabular}
% \end{adjustbox}
\end{table}


    \medskip
    \item \textbf{Reward (\( R \)):} The reward function \( R(s_t, a_t, s_{t+1}) \) is designed to encourage the robot to manipulate object to the goal object pose \( q_{\text{obj}}^\text{g} \). The reward consists of three main components:
    \begin{enumerate}

        \item \textbf{Object Keypoint Distance Reward:} The reward $r^\text{(t)}_\text{obj}$ computation is identical to the object keypoint distance reward used in non-prehensile skill post-contact policy training. % Encourages moving the object closer to its target object pose. Here, $p^\text{(t)}_\text{obj}$ represents the object's keypoint position at timestep $t$, and $p^\text{g}_\text{obj}$ denotes the keypoint position of the target object pose:
        % \[
        % \begin{aligned}
        % r^\text{(t)}_{\text{obj}} = {\epsilon_0^{\text{obj}}\over{\|p^\text{(t)}_\text{obj} - p^\text{g}_\text{obj}\| + \epsilon^{\text{obj}}_1}}
        % - {\epsilon_0^{\text{obj}}\over{\|p^\text{(t-1)}_\text{obj} - p^\text{g}_\text{obj}\| + \epsilon^{\text{obj}}_1}}
        % \end{aligned}
        % \]

        \item \textbf{Tip Contact Reward:} The reward $r^\text{(t)}_\text{tip-contact}$ computation is identical to the tip contact reward used in non-prehensile post-contact policy training. % Encourages maintaining contact between the gripper tips and the object. Here, $(u_\text{tip})_t$ represents the robot gripper tip positions at timestep $t$, $q^\text{(t)}_\text{obj}.\text{pos}$ represents the object position at timestep $t$, and $\epsilon^{\text{tip-obj}}_0$, $\epsilon^{\text{tip-obj}}_1$ are reward hyperparameters:
        % \[
        % \begin{aligned}
        % r^\text{(t)}_{\text{tip-contact}} = {\epsilon_0^{\text{tip-obj}}\over{\|p^\text{(t)}_\text{tip} - q^\text{(t)}_\text{obj}.\text{pos}\| + \epsilon^{\text{tip-obj}}_1}}
        % - {\epsilon_0^{\text{tip-obj}}\over{\|p^\text{(t-1)}_\text{obj} -     q^\text{(t-1)}_\text{obj}.\text{pos}\| + \epsilon^{\text{tip-obj}}_1}}
        % \end{aligned}
        % \]

        \item \textbf{Domain-Oriented Reward:} Encourages the successful completion of domain-specific objectives. The exact reward varies depending on the domain as shown in Table \ref{table:PPO_reward_condition}:

        \[
        \begin{aligned}
        r^\text{(t)}_{\text{domain}} = {\epsilon_0^{\text{domain}}} \cdot \mathbb{I}[\text{domain-specific conditions}]
        \end{aligned}
        \]

        \begin{table}[H]
\centering
% \begin{adjustbox}{width=0.5\textwidth} % Adjusts the table width to half the page
\begin{tabular}{|c|c|}
\hline
\textbf{Domain} & \multicolumn{1}{c|}{domain-specific conditions} \\ \hline
\textbf{Card Flip} & \multicolumn{1}{c|}{The card is flipped} \\ \hline
\textbf{Bookshelf} & \multicolumn{1}{c|}{The book is placed on the box} \\ \hline
\textbf{Kitchen} & \multicolumn{1}{c|}{The cup is placed on the shelf} \\ \hline
\end{tabular}
\caption{domain-specific conditions for training the Baseline PPO policy}\label{table:PPO_reward_condition}
\end{table}

        % \begin{itemize}
        %     \item 
        %     For the Card Flip domain, $r^{\text{domain}} = {\epsilon_0^{\text{domain}}} \cdot \mathbb{I}[\text{object is flipped}]$
        %     \item
        %     For the Bookshelf domain, $r^{\text{domain}} = {\epsilon_0^{\text{domain}}} \cdot \mathbb{I}[\text{object is placed on the box}]$
        %     \item
        %     For the Kitchen domain, $r^{\text{domain}} = {\epsilon_0^{\text{domain}}} \cdot \mathbb{I}[\text{object is placed on the shelf}]$

        % \end{itemize}

    
        \item \textbf{Success Reward:} The reward $r^\text{(t)}_\text{success}$ computation is identical to the success reward used in non-prehensile post-contact policy training.% Provides a success reward, $r^{\text{succ}}$, when the object is successfully manipulated to the target goal pose $q^{\text{obj}}_\text{g}$.
    \end{enumerate}

    The overall reward is defined as:
    \[
    r^\text{(t)}_{\text{PPO}} = r^\text{(t)}_{\text{obj}} + r^\text{(t)}_{\text{tip-obj}} + r^\text{(t)}_{\text{domain}} + r^\text{(t)}_{\text{success}}
    \]
    The hyperparameters of the reward function, $\epsilon_0^{\text{obj}}$, $\epsilon_1^{\text{obj}}$, $\epsilon_0^{\text{tip-obj}}$, $\epsilon_1^{\text{tip-obj}}$, $\epsilon_0^{\text{domain}}$ and $r_{\text{succ}}$, are provided in Table~\ref{table:baseline_ppo_reward}.
    \begin{table}[H]
\centering
% \begin{adjustbox}{width=0.5\textwidth} % Adjusts the table width to half the page
\begin{tabular}{|c|c|c|c|}
\hline
\textbf{Domain} & \multicolumn{1}{c|}{\textbf{Card Flip}} & \multicolumn{1}{c|}{\textbf{Bookshelf}} & \multicolumn{1}{c|}{\textbf{Kitchen}} \\ \hline
$\epsilon_0^{\text{obj}}$       & \multicolumn{3}{c|}{0.02} \\ \hline
$\epsilon_1^{\text{obj}}$       & \multicolumn{3}{c|}{0.02} \\ \hline
$\epsilon_0^{\text{tip-obj}}$       & \multicolumn{3}{c|}{0.03} \\ \hline
$\epsilon_1^{\text{tip-obj}}$       & \multicolumn{3}{c|}{0.03} \\ \hline
$\epsilon_0^{\text{domain}}$       & \multicolumn{3}{c|}{0.5} \\ \hline
$r_{\text{succ}}$ & \multicolumn{3}{c|}{1000} \\ \hline
$\delta_\text{obj}$ & \multicolumn{3}{c|}{0.005} \\ \hline
\end{tabular}
\caption{Reward hyperparameter values for training the Baseline PPO policy}\label{table:baseline_ppo_reward}
\end{table}
\end{itemize}

% \subsubsection{Baseline PPO Policy architecture}
The baseline PPO policies utilize a multilayer perceptron (MLP) architecture to generate low-level robot actions based on the state and goal object pose information. Each policy, $\pi_{\text{pre}}$ and $\pi_{\text{post}}$, employs a five-layer MLP. $\pi_{\text{pre}}$ and $\pi_{\text{post}}$ have input dimensions of 14 and 147, respectively. $\pi_{\text{pre}}$ and $\pi_{\text{post}}$ have output dimensions of 9 and 21, respectively. The other components of the architecture are identical to those of the non-prehensile skill post-contact policy's network architecture.
\begin{table}[H]
    \fontsize{8}{8}\selectfont
    \centering
    % \resizebox{0.8\textwidth}{!}{
\begin{tabular}{c|c|c|c|c|c}
    \toprule
    & \makecell{input\\dimensions}& \makecell{hidden\\dimensions} & \makecell{output\\dimensions} &\makecell{hidden\\activations} &\makecell{output\\activation} \\
    \cmidrule(lr){1-1}\cmidrule(lr){2-2}\cmidrule(lr){3-3}\cmidrule(lr){4-4}\cmidrule(lr){5-5}\cmidrule(lr){6-6}
    $\pi_\text{pre}$
    & 14
    & \multirow{2}{*}{$[512, 256, 256, 128]$}
    & 9 
    &
    \multirow{2}{*}{ELU} & 
    \multirow{2}{*}{Identity}
    \\
    \cmidrule(lr){1-1}\cmidrule(lr){2-2}\cmidrule(lr){4-4}
    $\pi_{\text{post}}$
    & 147
    & 
    & 21
    &
    &
    \\
    \bottomrule
    \end{tabular}
    \caption{The network architecture of the Baseline PPO policies}\label{table:baseline_ppo_network_architecture}
\end{table}

