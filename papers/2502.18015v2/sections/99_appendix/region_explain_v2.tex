For each domain, we define a set of regions of interest, $\Qobj$, which is a subset of the $SE(3)$ space where object poses can be stable. In the card flip domain, the space of stable and relevant object configurations, $\Qobj \subset \text{SE(3)}$, is defined as the set of object poses lying on a table, regardless of whether the card is flipped or not. In the bookshelf domain, there are two types of $\Qobj$. The first is $\Qobj^\text{upper-shelf}$, where the book is fitted between books on either side in the upper shelf. The second is $\Qobj^\text{lower-shelf}$, where the book is lying on the lower shelf. In the kitchen domain, there are three types of $\Qobj$. The first is $\Qobj^\text{sink}$, where the cup is in the sink. The second is $\Qobj^\text{l-cupboard}$, where the cup is lying on the left side of the cupboard. The last is $\Qobj^\text{r-cupboard}$, where the cup is lying on the right side of the cupboard. The visualization of $\Qobj$ for each domain is depicted in Figure \ref{fig:skill}.

\begin{figure*}[h]
\centering
\resizebox{\textwidth}{!}{
    \includegraphics{figures/region.png}
}
\caption{Regions for each domain.}\label{fig:region}
% \vspace{-1mm}
\end{figure*}
