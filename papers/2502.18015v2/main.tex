\documentclass[conference]{IEEEtran}
\usepackage{times}

% numbers option provides compact numerical references in the text. 
\usepackage[numbers]{natbib}
\usepackage{multicol}
\usepackage[bookmarks=true]{hyperref}
% basic
%\usepackage{color,xcolor}
\usepackage{color}
\usepackage{epsfig}
\usepackage{graphicx}
\usepackage{algorithm,algorithmic}
% \usepackage{algpseudocode}
%\usepackage{ulem}

% figure and table
\usepackage{adjustbox}
\usepackage{array}
\usepackage{booktabs}
\usepackage{colortbl}
\usepackage{float,wrapfig}
\usepackage{framed}
\usepackage{hhline}
\usepackage{multirow}
% \usepackage{subcaption} % issues a warning with CVPR/ICCV format
% \usepackage[font=small]{caption}
\usepackage[percent]{overpic}
%\usepackage{tikz} % conflict with ECCV format

% font and character
\usepackage{amsmath,amsfonts,amssymb}
% \let\proof\relax      % for ECCV llncs class
% \let\endproof\relax   % for ECCV llncs class
\usepackage{amsthm} 
\usepackage{bm}
\usepackage{nicefrac}
\usepackage{microtype}
\usepackage{contour}
\usepackage{courier}
%\usepackage{palatino}
%\usepackage{times}

% layout
\usepackage{changepage}
\usepackage{extramarks}
\usepackage{fancyhdr}
\usepackage{lastpage}
\usepackage{setspace}
\usepackage{soul}
\usepackage{xspace}
\usepackage{cuted}
\usepackage{fancybox}
\usepackage{afterpage}
%\usepackage{enumitem} % conflict with IEEE format
%\usepackage{titlesec} % conflict with ECCV format

% ref
% commenting these two out for this submission so it looks the same as RSS example
% \usepackage[breaklinks=true,colorlinks,backref=True]{hyperref}
% \hypersetup{colorlinks,linkcolor={black},citecolor={MSBlue},urlcolor={magenta}}
\usepackage{url}
\usepackage{quoting}
\usepackage{epigraph}

% misc
\usepackage{enumerate}
\usepackage{paralist,tabularx}
\usepackage{comment}
\usepackage{pdfpages}
% \usepackage[draft]{todonotes} % conflict with CVPR/ICCV/ECCV format



% \usepackage{todonotes}
% \usepackage{caption}
% \usepackage{subcaption}

\usepackage{pifont}% http://ctan.org/pkg/pifont

% extra symbols
\usepackage{MnSymbol}



\usepackage{xr}
\externaldocument{main_appendix}

\pdfinfo{
   /Author (Homer Simpson)
   /Title  (Robots: Our new overlords)
   /CreationDate (D:20101201120000)
   /Subject (Robots)
   /Keywords (Robots;Overlords)
}

\begin{document}

% paper title
\title{From planning to policy: distilling \texttt{Skill-RRT} for long-horizon prehensile and non-prehensile manipulation}

% You will get a Paper-ID when submitting a pdf file to the conference system
% \author{Haewon Jung*, Donguk Lee*, Haecheol Park, Kim Jun Hyeop, Beomjoon Kim}
\author{
    Haewon Jung*, Donguk Lee*, Haecheol Park, JunHyeop Kim, Beomjoon Kim \\ 
    \small Korea Advanced Institute of Science and Technology (KAIST) % Affiliation in a smaller font
}

\fancypagestyle{firstpage}{
    \fancyhf{} % Clear all headers and footers
    \renewcommand{\headrulewidth}{0pt} % Remove the unwanted top line
    \fancyfoot[L]{%
        \noindent\hspace{0pt}\rule{0.1\linewidth}{0.4pt}\\ % Half-width line, left-aligned
        \parbox[t]{0.5\linewidth}{\footnotesize * equal contribution. Correspondence to: Haewon Jung (zora07@kaist.ac.kr), and Donguk Lee (du.lee@kaist.ac.kr)}
    }
}
%\authorblockA{School of Electrical and\\Computer Engineering\\
%Georgia Institute of Technology\\
%Atlanta, Georgia 30332--0250\\
%Email: mshell@ece.gatech.edu}
%\and
%\authorblockN{Homer Simpson}
%\authorblockA{Twentieth Century Fox\\
%Springfield, USA\\
%Email: homer@thesimpsons.com}
%\and
%\authorblockN{James Kirk\\ and Montgomery Scott}
%\authorblockA{Starfleet Academy\\
%San Francisco, California 96678-2391\\
%Telephone: (800) 555--1212\\
%Fax: (888) 555--1212}}


% avoiding spaces at the end of the author lines is not a problem with
% conference papers because we don't use \thanks or \IEEEmembership


% for over three affiliations, or if they all won't fit within the width
% of the page, use this alternative format:
% 
% \author{\authorblockN{Michael Shell\authorrefmark{1},
% Homer Simpson\authorrefmark{2},
% James Kirk\authorrefmark{3}, 
% Montgomery Scott\authorrefmark{3} and
% Eldon Tyrell\authorrefmark{4}}
% \authorblockA{\authorrefmark{1}School of Electrical and Computer Engineering\\
% Georgia Institute of Technology,
% Atlanta, Georgia 30332--0250\\ Email: mshell@ece.gatech.edu}
% \authorblockA{\authorrefmark{2}Twentieth Century Fox, Springfield, USA\\
% Email: homer@thesimpsons.com}
% \authorblockA{\authorrefmark{3}Starfleet Academy, San Francisco, California 96678-2391\\
% Telephone: (800) 555--1212, Fax: (888) 555--1212}
% \authorblockA{\authorrefmark{4}Tyrell Inc., 123 Replicant Street, Los Angeles, California 90210--4321}}


% \maketitle

\twocolumn[{
	\renewcommand\twocolumn[1][]{#1}%
	\maketitle
        \vspace{-7mm}
	\begin{center}
		\includegraphics[width=\textwidth]{figures/figure_task_v2.png}
                \vspace{-5mm}

		\captionof{figure}{\label{fig:CPNP_tasks} Overview of our tasks. The first column shows objects. Yellow and green lines mark the initial and goal object poses in the second column. (First row) the robot must flip a thin card initially in an ungraspable pose by first moving it to the end of the table, flipping it, and putting it in the target pose. (Second row) the robot must topple a book out of the upper shelf and put it in the lower shelf where the robot gripper cannot fit by first toppling the book, picking and placing it at the end of the lower shelf and then pushing it inside. (Third row) the robot must upright the cup in a sink, grab it, place it on the cupboard, and ensure the handle is inside the desired region by re-orienting it. The red arrow indicates the movement of the object, while the orange arrow represents the movement of the robot. We highly recommend our \href{https://www.youtube.com/watch?v=17u5FhO8eso}{video}.}
	\end{center}
}]
\begin{abstract}
\begin{abstract}
  In this work, we present a novel technique for GPU-accelerated Boolean satisfiability (SAT) sampling. Unlike conventional sampling algorithms that directly operate on conjunctive normal form (CNF), our method transforms the logical constraints of SAT problems by factoring their CNF representations into simplified multi-level, multi-output Boolean functions. It then leverages gradient-based optimization to guide the search for a diverse set of valid solutions. Our method operates directly on the circuit structure of refactored SAT instances, reinterpreting the SAT problem as a supervised multi-output regression task. This differentiable technique enables independent bit-wise operations on each tensor element, allowing parallel execution of learning processes. As a result, we achieve GPU-accelerated sampling with significant runtime improvements ranging from $33.6\times$ to $523.6\times$ over state-of-the-art heuristic samplers. We demonstrate the superior performance of our sampling method through an extensive evaluation on $60$ instances from a public domain benchmark suite utilized in previous studies. 


  
  % Generating a wide range of diverse solutions to logical constraints is crucial in software and hardware testing, verification, and synthesis. These solutions can serve as inputs to test specific functionalities of a software program or as random stimuli in hardware modules. In software verification, techniques like fuzz testing and symbolic execution use this approach to identify bugs and vulnerabilities. In hardware verification, stimulus generation is particularly vital, forming the basis of constrained-random verification. While generating multiple solutions improves coverage and increases the chances of finding bugs, high-throughput sampling remains challenging, especially with complex constraints and refined coverage criteria. In this work, we present a novel technique that enables GPU-accelerated sampling, resulting in high-throughput generation of satisfying solutions to Boolean satisfiability (SAT) problems. Unlike conventional sampling algorithms that directly operate on conjunctive normal form (CNF), our method refines the logical constraints of SAT problems by transforming their CNF into simplified multi-level Boolean expressions. It then leverages gradient-based optimization to guide the search for a diverse set of valid solutions.
  % Our method specifically takes advantage of the circuit structure of refined SAT instances by using GD to learn valid solutions, reinterpreting the SAT problem as a supervised multi-output regression task. This differentiable technique enables independent bit-wise operations on each tensor element, allowing parallel execution of learning processes. As a result, we achieve GPU-accelerated sampling with significant runtime improvements ranging from $10\times$ to $1000\times$ over state-of-the-art heuristic samplers. Specifically, we demonstrate the superior performance of our sampling method through an extensive evaluation on $60$ instances from a public domain benchmark suite utilized in previous studies.

\end{abstract}

\begin{IEEEkeywords}
Boolean Satisfiability, Gradient Descent, Multi-level Circuits, Verification, and Testing.
\end{IEEEkeywords}
\end{abstract}

\IEEEpeerreviewmaketitle
\thispagestyle{firstpage} % Apply the footer only to the first page


% \footnotetext{* Equal contribution.}

\section{Introduction}\label{sec:Introduction}
% \section{Introduction}\label{sec:Intro} 


Novel view synthesis offers a fundamental approach to visualizing complex scenes by generating new perspectives from existing imagery. 
This has many potential applications, including virtual reality, movie production and architectural visualization \cite{Tewari2022NeuRendSTAR}. 
An emerging alternative to the common RGB sensors are event cameras, which are  
 bio-inspired visual sensors recording events, i.e.~asynchronous per-pixel signals of changes in brightness or color intensity. 

Event streams have very high temporal resolution and are inherently sparse, as they only happen when changes in the scene are observed. 
Due to their working principle, event cameras bring several advantages, especially in challenging cases: they excel at handling high-speed motions 
and have a substantially higher dynamic range of the supported signal measurements than conventional RGB cameras. 
Moreover, they have lower power consumption and require varied storage volumes for captured data that are often smaller than those required for synchronous RGB cameras \cite{Millerdurai_3DV2024, Gallego2022}. 

The ability to handle high-speed motions is crucial in static scenes as well,  particularly with handheld moving cameras, as it helps avoid the common problem of motion blur. It is, therefore, not surprising that event-based novel view synthesis has gained attention, although color values are not directly observed.
Notably, because of the substantial difference between the formats, RGB- and event-based approaches require fundamentally different design choices. %

The first solutions to event-based novel view synthesis introduced in the literature demonstrate promising results \cite{eventnerf, enerf} and outperform non-event-based alternatives for novel view synthesis in many challenging scenarios. 
Among them, EventNeRF \cite{eventnerf} enables novel-view synthesis in the RGB space by assuming events associated with three color channels as inputs. 
Due to its NeRF-based architecture \cite{nerf}, it can handle single objects with complete observations from roughly equal distances to the camera. 
It furthermore has limitations in training and rendering speed: 
the MLP used to represent the scene requires long training time and can only handle very limited scene extents or otherwise rendering quality will deteriorate. 
Hence, the quality of synthesized novel views will degrade for larger scenes. %

We present Event-3DGS (E-3DGS), i.e.,~a new method for novel-view synthesis from event streams using 3D Gaussians~\cite{3dgs} 
demonstrating fast reconstruction and rendering as well as handling of unbounded scenes. 
The technical contributions of this paper are as follows: 
\begin{itemize}
\item With E-3DGS, we introduce the first approach for novel view synthesis from a color event camera that combines 3D Gaussians with event-based supervision. 
\item We present frustum-based initialization, adaptive event windows, isotropic 3D Gaussian regularization and 3D camera pose refinement, and demonstrate that high-quality results can be obtained. %

\item Finally, we introduce new synthetic and real event datasets for large scenes to the community to study novel view synthesis in this new problem setting. 
\end{itemize}
Our experiments demonstrate systematically superior results compared to EventNeRF \cite{eventnerf} and other baselines. 
The source code and dataset of E-3DGS are released\footnote{\url{https://4dqv.mpi-inf.mpg.de/E3DGS/}}. 




  
% Humans have a remarkable capability to use a long sequence of prehensile and non-prehensile skills to manipulate objects. To flip a thin card on a table and reposition it, we drag the card to the table’s edge to enable grasping, pick it up, flip it, and drag it again to the desired spot. Similarly, taking a book from a crowded bookshelf and placing it in a narrow space involve toppling the book for grasping, moving it to the edge of the target space, and then pushing it into place. These are extremely intricate and complex tasks: they require considering not only the feasibility of the skills, but also their long-term consequences in a way that they \textit{chain}, such as toppling a book to enable a grasp. 

Humans have a remarkable capability to use a long sequence of prehensile and non-prehensile skills to manipulate objects. To take a book from a crowded shelf and place it in a tight space where our entire hand would not fit, we topple the book to enable grasping, move it to the edge of the target space, and then push it into place using just our fingers. Such problems are extremely intricate and complex: they require considering not only the feasibility of the skills, but also their long-term consequences in a way that they chain, such as toppling a book to enable a grasp.

Our goal is to enable robots to solve such \textit{Prehensile-Non-Prehensile (PNP)} problems — tasks that require long sequences of prehensile and non-prehensile skills to move an object to a target pose, like those shown in Figure \ref{fig:CPNP_tasks}. We assume access to a set of independently trained skills, where their training processes do not assume the knowledge of other skills. Each skill consists of a \emph{goal-conditioned policy} that takes as an input a desired object pose and outputs a sequence of joint torques to achieve that pose, and an \emph{applicability checker} which, given a state, checks whether the skill can be applied in that state. We also assume access to a physics engine that can simulate these skills.
%The objective is to find a sequence of joint torques to achieve the ultimate goal object pose.

%One approach to solve PNP problems is to use Task and Motion Planning (TAMP) algorithms that have been shown to be effective for long-horizon problems~\cite{garrett2021integrated}. TAMP combines discrete task planning with continuous motion planning, analogous to selecting a sequence of skills from a discrete set and the associated intermediate object poses in PNP problems. However, the issue with applying TAMP to PNP problems is that TAMP requires preconditions and effects of a skill. These are descriptions of state sets that ensure a skill, when executed in a state satisfying its preconditions, will deterministically lead to a state satisfying its effects. However, in PNP problems, such information is not readily available. While an applicability checker can determine whether a skill can be applied in a given state, it does not ensure the skill's success in moving the object to the desired pose. Further, even when the skill succeeds, the object is not positioned precisely at the desired pose but rather within a specified distance threshold. These uncertainties render the effect states undefined. Another problem is that TAMP algorithms are generally computationally expensive and prone to frequent re-planning when unexpected transitions occur, particularly in contact-rich tasks where modeling contact dynamics is inherently challenging.

One approach to PNP problems is to use Task and Motion Planning (TAMP) algorithms that are effective for long-horizon problems~\cite{garrett2021integrated}. However, the issue with applying TAMP to PNP problems is that TAMP requires 
preconditions and effects of a skill, which are state sets that explicitly define the deterministic transition model of the skill. However, in PNP problems, such information is not readily available. While an applicability checker can determine whether a skill \emph{can} be applied in a given state, it does not ensure the skill's success in moving the object to the desired pose. Further, even when the skill succeeds, we have a stochastic transition where the object is not positioned precisely at the desired pose but rather at a random pose within a specified distance threshold.  Another problem is that TAMP algorithms, as with any planning algorithms, are generally computationally expensive, as they involve exploring different sequences of actions and expensive feasibility checks, such as the existence of a motion plan.


%TAMP combines discrete task planning with continuous motion planning, analogous to selecting a sequence of skills from a discrete set and the associated intermediate object poses in PNP problems. However, the issue with applying TAMP to PNP problems is that TAMP requires preconditions and effects of a skill. These are descriptions of state sets that ensure a skill, when executed in a state satisfying its preconditions, will deterministically lead to a state satisfying its effects. However, in PNP problems, such information is not readily available. While an applicability checker can determine whether a skill can be applied in a given state, it does not ensure the skill's success in moving the object to the desired pose. Further, even when the skill succeeds, the object is not positioned precisely at the desired pose but rather within a specified distance threshold. These uncertainties render the effect states undefined. Another problem is that TAMP algorithms are generally computationally expensive and prone to frequent re-planning when unexpected transitions occur, particularly in contact-rich tasks where modeling contact dynamics is inherently challenging.

Alternatively, learning-based methods, such as reinforcement learning (RL) and imitation learning (IL), offer faster decision-making by leveraging neural network inference to compute actions.  However, RL often faces challenges in long-horizon tasks due to sparse rewards and the credit assignment problem. These difficulties are particularly pronounced in our tasks where we often have to temporarily move away from the goal to ultimately achieve it, such as positioning a card near the edge of a table, away from a goal pose, to facilitate grasping (e.g. Figure~\ref{fig:CPNP_tasks}, top row). IL has shown impressive performance in robotics \cite{chi2023diffusion, fu2024mobile, Reuss-RSS-23, wang2023mimicplay,Zhao-RSS-23}, but is typically limited to acquiring short-horizon skills as it is expensive to collect large-scale long-horizon demonstrations.


%To apply RL, we can formulate the PNP problem as  Parameterized Action Markov Decision Processes (PAMDPs)~\cite{hausknecht2015deep,jain2020generalization, jiang2024hacmanpp, masson2016reinforcement,  nasiriany2022augmenting, wei2018hierarchical, xiong2018parametrized, li2021hyar} where each action is a skill represented by a pair of skill index and continuous parameters, and use RL to train a high-level policy for selecting skills and their parameters. 


Instead, we propose a two-stage framework: (1) generate high-quality, large-scale demonstrations for PNP problems using a novel planner, and (2) use IL to reduce the online computation time. To efficiently generate demonstrations, we propose \texttt{Skill-RRT}, which extends RRT~\cite{lavalle1998rapidly}, a time-tested method for computing a collision-free motion to a goal, to object poses and skills. RRT efficiently explores the search space by randomly sampling a robot configuration, extending from the nearest node in the tree if the configuration is collision-free, and rejecting it otherwise. \texttt{Skill-RRT} has the same structure, but introduces two key differences. First, instead of sampling a robot configuration, it samples object pose and skill from the space of object poses and the discrete set of skills. Second, it extends from the nearest node in the tree only if the sampled skill is both applicable and succeeds when simulated with the sampled object pose as its goal. If these conditions are not met, the sampled pose and skill are rejected.

\newcommand{\skillrrt}{\texttt{Skill-RRT}}
\newcommand{\lazyskillrrt}{\texttt{Lazy Skill-RRT}}

%\skillrrt{} enables efficient planning in the space of skills and intermediate object poses by adopting the exploration strategy of RRT. However, one problem with \skillrrt{} is that the resulting state of one skill might not align with the applicable states of the next skill as skills are not necessarily made to chain \cite{konidaris2009skill}. For instance, in the card domain shown in Figure~\ref{fig:CPNP_tasks}, a sliding skill ends when the robot achieves the given object pose, with its gripper at the top of the card, closed. To pick the object, the robot must transition to a pre-grasp configuration at the side of the card with the gripper opened, as that is the applicable state of the prehensile skill, but this is a motion that belongs neither to the sliding skill nor the prehensile skill. A naive solution is to use a collision-free motion planner to move the gripper, but we found this to be unworkable: transitioning to the next skill’s applicable state often involves breaking and making contacts, and the stiff motions from motion planners tend to disturb the object, causing undesirable events such as card falling off the table.

\skillrrt{} facilitates efficient planning in the space of skills and intermediate object poses by leveraging the exploration strategy of RRT. However, a key challenge with \skillrrt{} is that the resulting state of one skill may not align with the applicable states required by the next skill, as skills are not inherently designed to chain~\cite{konidaris2009skill}. For example, in the card domain, a sliding skill ends when the robot positions the object at the desired pose, with the gripper closed and positioned on top of the card (Figure~\ref{fig:CPNP_tasks}, third column). To transition to a prehensile skill, the robot must move to a pre-grasp configuration at the side of the card with the gripper opened, as this is the applicable state of the prehensile skill (Figure~\ref{fig:CPNP_tasks}, fourth column). However, this motion belongs to neither the sliding nor the prehensile skill. One naive solution is to use a collision-free motion planner to move the gripper, but we found the stiff motions generated by motion planners tend to disturb the object, causing undesirable outcomes such as the card falling off the table.

To address this, we introduce \textit{connectors} -- goal-conditioned policies that, given the current state and goal robot configuration, breaks the contact at the current configuration and remake the contact at a new configuration with the minimum disturbance to the object’s pose. For each skill, we have an associated connector that moves the robot to a state where the skill is applicable, similar to Mason's idea of ``funnels''~\cite{mason1985mechanics}. 

To train connectors, we use RL. However, the challenge is defining the right distribution of problems: if we, say, uniform-randomly generate initial state and goal configurations for connectors, we might end up training connectors on states that are irrelevant to the given PNP problem. If on the other hand, we manually design these problems, then the connector might end up facing out-of-distribution problems.

Instead, we propose \texttt{Lazy Skill-RRT}, which tentatively assumes connectors exist, and teleport the robot to the next skill's applicable state. We use \texttt{Lazy Skill-RRT} to solve PNP problems, while logging the states where we teleported the robot, hence requires a connector. We then train the connector on these logged problems, which allows us to focus our connectors only on states that are likely to be encountered on a solution path.

With the connectors, \texttt{Skill-RRT} can generate complete skill plans for PNP problems. We use \texttt{Skill-RRT} to create solution dataset for diverse initial and goal object poses and distill it to a policy via IL to eliminate expensive online planning. Because the choice of intermediate object poses results in multi-modal trajectories, we use Diffusion Policy \cite{chi2023diffusion}. However, as already observed in the previous paper \cite{dalal2023imitating}, not all data from a planner is useful, as low-quality trajectories may degrade policy performance by leading the robot to a high-risk state. To mitigate this, we filter data by replaying the \texttt{Skill-RRT} plans with noise in simulation, and discard those whose success rate is lower than a threshold. 
%For example, unstable card poses positioned very close to the edge of the table might solve the task but risk dropping the card. 

We generate data and train all policies entirely in simulation, and zero-shot transfer to the real world. In three contact-rich, real-world long-horizon PNP problems (Figure \ref{fig:CPNP_tasks}), our policy achieves more than 80\% success rates across all domains. We also show that the distilled policy outperforms pure planner, \texttt{Skill-RRT}, and \texttt{MAPLE}~\cite{nasiriany2022augmenting}, the state-of-the-art skill-based RL algorithm, in terms of computation time, and success rate.

  

\section{Related Works}\label{sec:Realted_works}
\subsection{Learning prehensile and non-prehensile skills}
% Prehensile skill
% Grasp prediction + motion planning: DexNet, contact-graspnet, Anygrasp.
% End-to-end
% Qt-opt, Joshi et al., 
There are two common approaches to learning prehensile skills. The first approach involves incorporating grasp pose prediction and motion planning \cite{fang2023anygrasp, mahler2017dex, sundermeyer2021contact}. Grasp pose prediction modules output a grasp pose given an RGB-D or point cloud observation. These grasp pose predictions are typically trained on a synthesized dataset in a supervised manner, and low-level robot joint actions are generated by a motion planner. The second approach involves training an end-to-end policy using RL \cite{joshi2020robotic, kalashnikov2018scalable, wang2022goal}. RL policy learns to grasp objects of diverse shapes by optimizing a policy through trial and error, using rewards to enhance stable grasps while penalizing failures. For a more comprehensive review of prehensile skills, we encourage readers to refer to the survey by \citet{xie2023learning}. In this work, we train prehensile skills with predefined grasps using RL which can leverage GPU-parallelization to offer faster computation than a planner.

% Non-prehensile skill
% Hacman, CRM, CORN
% Recently, several works have proposed to leverage learning to directly map sensory inputs to actions in nonprehensile manipulation tasks, circumventing the computational cost of planning, inaccuracy of hybrid dynamics models, and difficulty in system identification from high dimensional sensory inputs. In Kimet al. (2023), they propose an approach that outputs diverse motions and effectively performs time-varying hybrid force and position control (Bogdanovic et al., 2020), by using the end-effector target pose and controller gains as action space. However, since they represent object geometry via its bounding box, their approach has a limited generalization capability across shapes. Similarly, other approaches only handle simple shapes such as cuboids (Yuan et al., 2018; 2019; Ferrandis et al., 2023; Zhou & Held, 2023), pucks (Peng et al., 2018), or cylinders (Lowrey et al., 2018).

% explain how previous works using RL or IL to train the Non-prehensile policies

% yuan2018rearrangement -> cube, planar push, tabletop
% yuan2019end -> cube, planar push, tabletop
% ferrandis2023nonprehensile -> cube, planar push, tabletop
% peng2018sim -> pucks, planar push, tabletop
% lowrey2018reinforcement -> pucks, planar push, tabletop
% zhou2023learning -> variety objects. move to grasp the object, 
%  \citet{kim2023pre} -> diverse cuboids, not only tabletop, not only push (flip the object using environment)

% To construct a set of skills for our problem, our non-prehensile skills should be trained to manipulate a fixed object, regardless of its shape. Also, the skills manipulate the object into the desired pose within fixed environments, which can differ from just the tabletop. % add not only the planar push
% do not reuquire generalization on diverse objects and envirnomnet
% Along with HACMan, other methods use simple primitives (Yuan et al., 2019; Zhou et al., 2023). But for problems we are contemplating, use of such open-loop primitives precludes dense feedback motions. This is problematic for highly unstable (e.g. rolling or pivoting) objects that move erratically, which require the robot to continuously adjust the contact
% jiang2024hacmanpp

Recent studies use learning-based approaches for non-prehensile manipulation, directly mapping sensory inputs to actions while avoiding the computational burden of planning, inaccuracies in hybrid dynamics models, and challenges in system identification from high-dimensional inputs. Several works \cite{ferrandis2023nonprehensile, yuan2018rearrangement, yuan2019end} focus on planar in pushing using simple cuboid objects on a tabletop, while \citet{peng2018sim} train policies for pushing pucks. \citet{zhou2023learning} train a policy for various object shapes, where the goal is to reposition the object to enable grasping, a sub-problem in PNP problems. HACMan \cite{zhou2023hacman} proposes object-centric action representations, which consist of a target point on the object point cloud and a poking direction, enabling generalization to diverse object shapes. Most of these approaches are limited to short-horizon problems, and have a restricted motion repertoire, relying primarily on planar pushing or hand-crafted primitives.

Recently, \citet{kim2023pre} have proposed an RL-based approach for non-prehensile manipulation based on a general action space that automatically discovers various skills such as toppling, sliding, pushing, pulling, etc. Moreover, this work has several features that make it suitable for PNP problems. More concretely, it decomposes the policy into a pre-contact policy, which outputs the robot configuration for contacting the object for manipulation, and a post-contact policy, which performs the manipulation. This decomposition naturally aligns with our approach to defining connectors: we train connectors to transition the robot to the pre-contact configuration, allowing skills to chain. We adopt this approach for training our skills.



%To construct a set of skills for our problem, we need an algorithm that can train non-prehensile skills with various object shapes, extending beyond planar pushing on a flat tabletop to include tasks such as manipulating a cup in a sink. We apply the approach from \citet{kim2023pre} to construct a robust skill set for NP manipulation.

\subsection{Reinforcement learning with given skills}
% One way to address PNP problems is to formulate the problem as Parameterized Action Markov Decision Processes (PAMDPs)~\cite{masson2016reinforcement} where each action is a skill represented by a pair of skill index and continuous parameters and use RL. For instance, PADDPG~\cite{hausknecht2015deep} treats the hybrid action space as continuous by selecting skills via probabilities, with a single policy generating both probabilities and parameters simultaneously, but neglecting their dependency. PDQN~\cite{xiong2018parametrized} combines DQN~\cite{mnih2013playing} for discrete actions and DDPG for parameters but introduces redundancy by modeling all parameters jointly. MAHHQN~\cite{fu2019deep} and PATRPO~\cite{wei2018hierarchical} adopt hierarchical policies, conditioning low-level parameter policies on high-level skill outputs, but require fixed parameter dimensions. \citet{li2021hyar} address this limitation by leveraging a latent action space to adapt parameters for each skill, enabling dimension flexibility. While these methods focus on simple domains like robot soccer, we target complex, high-dimensional robotic tasks.

% RAPS \cite{dalal2021accelerating} and MAPLE \cite{nasiriany2022augmenting} propose similar approaches, where a high-level policy learns to predict the logits of skill indices and the mean and variance of parameters using RL, and introduces low-level primitives as the provided primitives are often insufficient to solve tasks. MAPLE specifically suggests an affordance reward function that helps evaluate the utility of skills in specific states, facilitating exploration by providing additional rewards that guide the system towards states satisfying the skill's preconditions. HACMan++~\cite{jiang2024hacmanpp} computes skill parameters for each skill and object point in the point cloud. The critic network generates Q-values for every skill-parameter-point combination. The policy then selects the skill index and its parameter by sampling from this Q-value map using a softmax distribution. While these methods are applicable to our problems, RL often struggles in long-horizon problems with sparse rewards. We instead use a planning algorithm to solve problems, and use IL to learn a policy.

One way to address PNP problems is to formulate them as parameterized action Markov decision processes (PAMDPs), where each action is a skill represented by a skill index and continuous parameters, and solve them using RL. General RL algorithms, designed for purely discrete or continuous action spaces, require modifications to handle parameterized action spaces. PADDPG~\cite{hausknecht2015deep} converts the parameterized action space into a continuous space for DDPG, using a single policy to generate both skill probabilities and continuous parameters simultaneously, regardless of the selected skill. Similarly, PDQN~\cite{xiong2018parametrized} combines DQN for discrete skill selection and DDPG for continuous parameter generation. However, both PADDPG and PDQN are unstable because the action-value \( Q \)-function is influenced by parameters generated for unselected skills~\cite{li2021hyar}. This introduces noise and disrupts stability in learning. To address this, MAHHQN~\cite{fu2019deep} and PATRPO~\cite{wei2018hierarchical} adopt hierarchical policies, where low-level parameter policies are conditioned on high-level skill outputs.

Our problem also can be addressed using PAMDPs, where a skill index is selected, and its continuous desired object pose is generated. However, we also need low-level actions to connect the skills. To address this, RAPS~\cite{dalal2021accelerating} and MAPLE~\cite{nasiriany2022augmenting} introduce a low-level primitive that directly controls single-timestep end-effector motions, in addition to the set of skills which are multi-timestep hard-coded closed-loop controllers. They then learn a high-level policy that uses these to solve long-horizon problems. However, exploration remains challenging, particularly in determining skill parameters such as where to grasp or place. MAPLE addresses this by introducing an affordance reward function, which evaluates the utility of skills in specific states and guides exploration by rewarding skill's applicability. % However, generating an affordance reward function for each skill individually makes it challenging to define the utility of a skill, making the approach difficult to scale.
HACMan++~\cite{jiang2024hacmanpp} also solves PNP problems using a PAMDP formulation. The critic network generates Q-values for each combination, and the policy selects the skill index and its parameter by sampling from the Q-value map using a softmax distribution. However, they assume that skills can chain.

In contrast to these approaches, we train connectors to guide the robot to a state where skills can be applied. Moreover, RL often struggles in long-horizon tasks with sparse reward settings. Instead, we use a planning algorithm to solve these problems and apply imitation learning (IL) to train a policy.

%However, the issue with RL is that they struggle to address long-horizon problems due to sparse rewards and the credit assignment problem, particularly in scenarios that require temporarily moving away from the goal to ultimately achieve it. In contrast, we utilize a planning algorithm, \texttt{Skill-RRT}, which efficiently explores the search space.

% Like our setup, there are RL-based approaches for using pre-defined skills. In particular, reinforcement Learning for Parameterized Action Markov Decision Processes (PAMDPs) \cite{hausknecht2015deep, masson2016reinforcement} focus on sequencing skills and their continuous parameters. \citet{masson2016reinforcement} use DDPG~\cite{lillicrap2015continuous} to train a single policy that outputs both skill probabilities and parameters, but it does not consider dependencies between discrete actions and continuous parameters. To address this, \citet{xiong2018parametrized} propose a hybrid approach, using DQN~\cite{mnih2013playing} to select discrete actions and DDPG to determine continuous parameters, though it remains flat-level training with dependent continuous parameters. \citet{fu2019deep} introduce a hierarchical framework where a high-level policy selects discrete actions, and a low-level policy generates continuous parameters conditioned on the discrete action. However, this method struggles with varying skill parameter dimensions across discrete actions. \citet{li2021hyar} address this limitation by using a latent action space to generate continuous parameters specific to each discrete action, enabling better handling of dimension variations.


% To apply RL, we can formulate the PNP problem as  Parameterized Action Markov Decision Processes (PAMDPs)~\cite{hausknecht2015deep,jain2020generalization, jiang2024hacmanpp, masson2016reinforcement,  nasiriany2022augmenting, wei2018hierarchical, xiong2018parametrized, li2021hyar, dalal2021accelerating} where each action is a skill represented by a pair of skill index and continuous parameters, and use RL to train a high-level policy for selecting skills and their parameters. RAPS \cite{dalal2021accelerating} and MAPLE \cite{nasiriany2022augmenting} propose similar approaches, where a high-level policy learns to predict the logits of skill indices and the mean and variance of parameters using RL, and introduces low-level primitives as the provided primitives are often insufficient to solve tasks. MAPLE specifically suggests an affordance reward function that helps evaluate the utility of skills in specific states, facilitating exploration by providing additional rewards that guide the system towards states satisfying the skill's preconditions. 

% The issue with RL is that they struggle to address long-horizon problems due to sparse rewards and the credit assignment problem, particularly in scenarios that require temporarily moving away from the goal to ultimately achieve it. In contrast, we utilize a planning algorithm, \texttt{Skill-RRT}, which efficiently explores the search space.



\subsection{Planning with given skills}

% Sampling based method (dogar, barry, mao)
% Optimization based planning method (STAP, LSP)
% Train skill dynamic model + planning (Liang, 
    % DLPA: DLPA~\cite{zhang2024model} trains a parameterized-action-conditioned dynamics model, which includes a transition predictor, a reward predictor, and a continue predictor to model task termination in PAMDPs. It generates actions by planning with a modified Cross-Entropy Method, maximizing cumulative returns using trajectory rollouts predicted by the dynamics model over a planning horizon.

Given a set of skills and their preconditions and effects, Task and Motion Planning (TAMP) performs integrated discrete task planning and continuous motion planning for long-horizon problems~\cite{garrett2021integrated}. There are two common methods for solving TAMP: sampling-based and optimization-based approaches. Sampling-based methods~\cite{garrett2018ffrob, garrett2020pddlstream,kaelbling2011hierarchical, ren2024extended} typically work by finding a task plan using a symbolic planner and using sampling to find a feasible set of continuous parameters. Optimization-based approaches, such as Logic-Geometric Programming (LGP) \cite{migimatsu2020object, toussaint2015logic, toussaint2018differentiable} first solve symbolic task planning, and leverages optimization to find continuous parameters with the constraints imposed by the symbolic plan. While these methods have shown impressive performance in long-horizon problems \cite{du2023video, kim2022representation, mendez2023embodied, vu2024coast, zhu2021hierarchical}, using TAMP requires defining the preconditions and effects of skills, which are not available in PNP problems.
% to use TAMP, you need to define preconditions and effects of skills, which we do not have in PNP problems.

%. These are descriptions of state sets that ensure a skill, when executed in a state satisfying its preconditions, will deterministically lead to a state satisfying its effects. However, in PNP problems, such information is not readily available. While an applicability checker can determine whether a skill can be applied in a given state, it does not ensure the skill's success in moving the object to the desired pose. Further, even when the skill succeeds, the object is not positioned precisely at the desired pose but rather within a specified distance threshold. These uncertainties render the effect states undefined. 

Similar to our \texttt{Skill-RRT}, \citet{barry2013hierarchical} propose an RRT-based hierarchical skill planning algorithm that first generates a collision-free object path and realizes it by identifying a sequence of skills that could achieve that object path. It is similar to our algorithm in that it extends a tree toward randomly sampled configurations through skills using RRT. The key differences are that we utilize RL-based skills, which we assume to be independently trained, and use connectors to bridge gaps between the skills. In general, the core problem with pure planners, including \texttt{Skill-RRT}, is that they are computationally expensive and prone to frequent re-planning when unexpected transitions occur, particularly in contact-rich tasks where modeling contact dynamics is inherently challenging. We solve this by distilling \texttt{Skill-RRT} using IL.

%formulates TAMP as a joint optimization problem over discrete symbolic actions with task constraints, often expressed as a logic program, and continuous motion parameters with a geometric model of the robot and environment. 

%generating symbolic action plans with corresponding continuous parameters, sampling and refining trajectories to satisfy both the plan's discrete constraints and motion feasibility. 

%Logic-Geometric Programming (LGP) \cite{toussaint2015logic, toussaint2018differentiable, migimatsu2020object} formulates TAMP as a joint optimization problem over discrete symbolic actions with task constraints, often expressed as a logic program, and continuous motion parameters with a geometric model of the robot and environment. It leverages optimization techniques to find feasible solutions that satisfy both logical and geometric constraints. 


% Leaning skill dynamic models and planning
\iffalse
%BK: this could potentially be useful
To acquire the effects of skills, several works \cite{liang2022search, shi2023skill, zhang2024model, lidexdeform} train skill dynamics models and utilize them in planning. \citet{liang2022search} train GNN-based skill effect models, which take as input the current state and a skill parameter and predict the terminal state reached as well as the heuristic-based execution cost of the skills. They then integrate these skill effect models with a graph search by constructing a graph based on their skill effect predictions. \citet{shi2023skill} and \citet{lidexdeform} jointly train skill dynamics models and skills during both the pre-training and downstream adaptation phases and employ the skill dynamics models in model-based RL, which simulates the outcomes of candidate skills and selects the best skill. DLPA~\cite{zhang2024model} trains a parameterized-action-conditioned dynamics model, comprising a transition predictor, a reward predictor, and a continuation predictor, to model task termination in PAMDPs. It generates actions by planning with a modified Cross-Entropy Method, maximizing cumulative returns through trajectory rollouts predicted by the dynamics model over a planning horizon. However, small prediction errors from learned skill dynamics models are critical for our tasks, as the feasible state of the next skill is narrow. For instance, when predicting the terminal state of pushing a card to the edge of a table, even a few millimeters of prediction error could incorrectly determine the feasibility of the prehensile skill. As a result, this could render the plan infeasible and lead to inefficiencies in finding a feasible plan. In contrast, we obtain the terminal states of skills by simulating the skills in parallel.


% Optimizing value function.
% Problem undefined
% The key to generalization is planning actions that maximize the probability of long-horizon task success, which we model using the product of learned Q-values.

Another line of work focuses on skill planning to find an optimal sequence of skills and their parameters that maximize the probability of long-horizon task success. To do so, they optimize a skill sequence and its parameters based on the total product or summation of the value functions of each skill with its corresponding parameters in the sequence \cite{agia2023stap, xue2024logic}. STAP \cite{agia2023stap} performs sampling-based optimization on skill parameters to maximize the product of the Q-values of a plan skeleton which can be integrated with a task planner, such as PDDLStream \cite{garrett2020pddlstream}. LSP \cite{xue2024logic} generates a skill plan using Monte Carlo Tree Search (MCTS) \cite{browne2012survey} and optimizes skill parameters with the Cross-Entropy Method (CEM) \cite{de2005tutorial} to maximize the sum of the value functions. For better value function approximation, skills are trained based on Generalized Policy Iteration using Tensor Train (TTPI). While these methods require careful reward function design to prevent the dominance of one skill, we can utilize skills regardless of how the reward function is designed.
\fi


\subsection{Imitation learning for robotics}
There are three challenges in applying IL to PNP problems: (1) addressing action multi-modality that stems from different choices of object intermediate poses, (2) how to generate a large dataset, and (3) how to generate a high-quality dataset. 

To cope with action multi-modality, \citet{lee2024behavior} and \citet{shafiullah2022behavior} introduce methods combining discrete action mode predictors with continuous offset correctors to identify and refine distinct modes in the action distribution. ACT~\cite{Zhao-RSS-23} uses conditional variational autoencoders (CVAE) to handle multimodality within a transformer architecture to generate a fixed-length action sequence, called action chunks. OPTIMUS~\cite{dalal2023imitating} uses a Gaussian Mixture Model (GMM) with five modes to model diverse paths, grasps, and placements, outperforming unimodal approaches based on Mean Squared Error (MSE) loss. More recently, Diffusion Policy~\cite{chi2023diffusion} addresses action multi-modality using diffusion models, capturing diverse strategies like left or right end-effector movements to push a block. Their policy demonstrates strong performance in simulations and real-world settings. Our method also leverages the diffusion policy to capture multi-modal behaviors.

To gather a large amount of data, many studies rely on teleoperation systems to collect robotic demonstration data \cite{handa2020dexpilot, heo2023furniturebench, mandlekar2018roboturk, mandlekar2019scaling, mandlekar2020human, stepputtis2022system, Zhao-RSS-23}. However, teleoperation-based data collection is labor-intensive and time-consuming, creating challenges in scaling datasets for diverse tasks, such as varying initial and goal object poses \cite{brohan2022rt, dalal2023imitating, jang2022bc, xiang2019task, yang2021trail}. 

Alternatively, recent approaches automate data collection for robotics tasks using simulators and planning algorithms, such as TAMP, to generate large-scale datasets and train a policy with IL. \citet{driess2021learning} use Logic Geometric Programming (LGP)~\cite{toussaint2018differentiable} to generate high-level action sequences, with Model Predictive Control (MPC) producing robot joint controls.  \citet{mcdonald2022guided} train the high-level policy to imitate action types and their parameters from TAMP solutions. Their parallelized Modular TAMP approach uses FastForward~\cite{hoffmann2001ff} to generate initial action sequences, refined into executable trajectories by motion planners. OPTIMUS ~\cite{dalal2023imitating} uses a single-level IL policy to directly imitate end-effector control from generated data. Their method generates TAMP solutions using the PDDLStream planning framework~\cite{garrett2020pddlstream} with an adaptive, sampling-based algorithm, incorporating samplers for grasp generation, placement, inverse kinematics, and motion planning. We take much inspiration from these works, but \texttt{Skill-RRT} generates data from a planner tailored for PNP problems.

Training with high-quality demonstrations is essential for IL \cite{mandlekar2022matters}, as low-quality data can lead to high-risk states, causing failures or out-of-distribution (OOD) states. To mitigate this, OPTIMUS~\cite{dalal2023imitating} imposes a containment constraint within a bounding box, pruning trajectories where the end-effector exits the workspace. This ensures the learned policy avoids OOD states but is limited to a setup where end-effector location defines problem characteristic. In our work, we evaluate the quality of plans generated by \texttt{Skill-RRT} replaying plans under simulator stochasticity, and keeping only the ones that are robust under disturbance.

% \section{Backgrounds}\label{sec:Backgrounds}
% % \vspace{-1em}
\section{Background}
\label{sec:background}

In this section, we discuss the structure and operational principles of FeFETs, and review existing CAM design works.

\begin{figure}%[H]
    \centering
    \includegraphics[width=1\linewidth]{Figures/1FRIV1.png}
   % \vspace{-0.4cm}
    \caption{\textbf{(a)} FeFET polarization directions and channel conditions after memory write operations;  \textbf{(b)} The FeFET $\textit{I}_\textit{D}$-$\textit{V}_\textit{G}$ characteristics after positive/negative gate write; % Source is grounded; 
    \textbf{(c)} 1FeFET-1R structure and equivalent circuit; \textbf{(d)} The 1FeFET-1R $\textit{I}_\textit{D}$-$\textit{V}_\textit{G}$ characteristics after positive/negative gate write.
    %Source is grounded.
    }
   
 
    \label{fig:fefet}
   %  \vspace{-0.4cm}
\end{figure}

%\vspace{-2ex}
%\vspace{-0cm}
\subsection{FeFET Basics}
\label{sec:device}
\setlength{\abovecaptionskip}{2pt}
\setlength{\belowcaptionskip}{2pt}


Recent advancements in ferroelectric material, particularly hafnium oxide ($\text{HfO}_\text{2}$), have spurred research interest in ferroelectric transistors  and the development of non-volatile circuit designs compatible with CMOS technology \cite{yin2020fecam}. 
FeFETs 
%belong to the subclass of metal-oxide-semiconductor field-effect transistors (MOSFETs) and 
incorporate a ferroelectric 
(FE) layer  within the gate stack. These devices exhibit unique electrical hysteresis characteristics, exhibiting reversible polarization states upon an applied voltage-driven electric field. 
%Integration of a ferroelectric capacitor with the MOSFET gate capacitor confers FeFETs with adjustable hysteresis characteristics. 
The FE layer induces a shift in the threshold voltage of the FeFET depending on the orientation of FE polarization \cite{FeFET-capacitor}, enabling non-volatile (NV) storage capabilities. 
By applying gate voltage pulses, such as -4V/+4V, to a FeFET device, as depicted in \autoref{fig:fefet}(a), it can be programmed to store low and high $\textit{V}_\textit{TH}$ states corresponding to logic ‘0’ and ‘1’, respectively. 
The associated hysteresis  $\textit{I}_\textit{D}$-$\textit{V}_\textit{G}$ transfer characteristics are shown in \autoref{fig:fefet}(b) \cite{transfer-characteristics}. FeFETs, being voltage-driven for read and write operations, exhibit superior energy efficiency compared to two-terminal current-driven NVMs.

%In recent years, with the continuous advancement of ferroelectric material hafnium oxide ($\rm{HfO_2}$),  there has been a growing focus among researchers on ferroelectric transistors and the exploration of non-volatile circuit structures compatible with CMOS technology \cite{yin2020fecam}. Ferroelectric gate field-effect transistors (FeFETs) represent a subclass of metal-oxide-semiconductor field-effect transistors (MOSFETs) that incorporate a ferroelectric layer (FE) within the gate stack, These devices possess distinctive electrical properties capable of reversible polarization under an applied electric field. The integration of a ferroelectric capacitor with the MOSFET gate capacitor grants FeFETs adjustable hysteresis characteristics. The ferroelectric layer introduces a shift in the threshold voltage contingent upon the orientation of ferroelectric polarization \cite{FeFET-capacitor}, resulting in non-volatile (NV) storage capabilities. 
%By applying gate voltage pulses, such as -4V/+4V, to a FeFET device, as illustrated in \autoref{fig:fefet}(a), the device can be programmed to exhibit low $V_{TH}$ and high $V_{TH}$ states corresponding to logic `0' and `1', respectively.
%FeFET have high energy efficiency and low energy consumption, making them a new type of device with great potential for application.
%We define that only two states of information exist in FeFET: logic '0' (high $ V_{TH}$) and logic '1' (low $ V_{TH}$), and logic '0' and logic '1' can be written by applying -4V/+4V pulses as shown in \autoref{fig:fefet}(a). 
 %The corresponding $I_D$-$V_{G}$ transfer characteristics  are depicted in \autoref{fig:fefet}(b) \cite{transfer-characteristics}.
%Given that FeFET read and write operations are voltage-driven, FeFETs demonstrate superior energy efficiency compared to two-terminal current-driven NVMs.

When the FeFET operates as a current source, its ON current gradually increases with the rise in gate voltage, as depicted in \autoref{fig:fefet}(b). Consequently, there's a certain variability in the conduction current regarding the gate read voltage. 
To ensure stable ON current during operation and enhance the design robustness, a current limiter is connected to the source of the FeFET, as shown in the equivalent circuit of \autoref{fig:fefet}(c).
Prior studies \cite{1FeFET1R-transfer, yin2023ultracompact} have shown that a series resistor on the drain/source of a FeFET can regulate the ON current, with 1FeFET-1R integration experimentally demonstrated \cite{area}. Such integration suppresses the ON current variability, making it independent of the $\textit{V}_\textit{TH}$ state and gate voltage when the series resistor is sufficiently large. 
The transfer characteristic curve of the 1FeFET-1R structure is depicted in \autoref{fig:fefet}(d).  
We adopt the 1FeFET-1R structure using a series resistor as a current limiter in this work. 
This approach mitigates the impact of ON current variability on \textit{ML} discharging in a CAM array 
%and reduces high energy consumption, 
achieving low power consumption and robust tunable approximate matching functionality.


%\autoref{fig:fefet}(b) demonstrates that even after entering the saturation region, the ferroelectric transistor behaves as a current source, with the ON current continuing to gradually increase with the rise in gate voltage. Consequently, there exists a certain variability in the conduction current concerning the gate read voltage. Given the requirement for approximate search functionality in the designed circuit, heightened demands are placed on the precision and stability of the current. 
%increases continuously with the increase of gate voltage. 
%After entering the saturation region, there will still be a slight increase in current. 
%Therefore, in order to ensure the stable ON current of the bitcell during operation and enhance the overall robustness of the circuit, a current limiter is connected to the source of the FeFET to constrain the variability of the ON current. In the equivalent circuit, this scenario presents a voltage division between the resistance of the ferroelectric transistor and the current limiter, as depicted in \autoref{fig:fefet}(c). 
%Previous studies \cite{1FeFET1R-transfer, yin2023ultracompact} have demonstrated that a series resistor on the drain/source of FeFETs can govern the ON current of FeFET devices, and such 1FeFET-1R integration has been experimentally verified \cite{area}.
%As a result, the variability in ON current is notably suppressed, and the ON current becomes independent of the $V_{TH}$ state, determined solely by the series resistor when the resistor is sufficiently large. The transfer characteristic curve of the 1FeFET-1R circuit is depicted in \autoref{fig:fefet}(d). 
%, which adds a limiter to control the change of the opening current, that is, a resistor with a larger resistance value is connected in series, and the resistance value of the resistor should be greater than the effective conductance resistance of the FeFET, so that the obtained opening current is not affected by $V_G$ and $V_{TH}$, but only controlled by $V_D$ and $R_S$ in \autoref{fig:fefet}(d).
%In this work, we adopt the 1FeFET-1R structure utilizing a series resistor as a current limiter. This approach not only mitigates the impact of ON current variability on the matchline discharging of a CAM array, but also reduces the high energy consumption caused by excessive ON current, thereby achieving low power consumption and tunable approximate matching functionality.

%This can improve the stability of the current \cite{stability}, and the obtained $ I_D$-$ V_{GS}$ curve is calibrated based on Preisach's FeFET model \cite{transfer-characteristics}.


%It's the voltage and the pulse width applied to gate that determine the memory window in FeFET devices\cite{Ni_2019},(Fig.\ref{FeFET}).
% \begin{figure}[H]
%     \centering
%     \begin{minipage}{0.47\linewidth}
%     \includegraphics[width=\linewidth]{Figures/FeFET.png}
%     \caption*{(a)}
%     \end{minipage}
%     \hspace{.1ex}
%     \begin{minipage}{0.47\linewidth}
%     \includegraphics[width=\linewidth]{Figures/i_v64.eps}
%     \caption*{(b)}
%     \end{minipage}
%     \caption{(a) FeFET polarization changes after applying a voltage pulse at the gate terminal; (b) FeFET I-V  curve we used in SPICE simulation.}
%     \label{FeFET}
    
% \end{figure}


   
%\vspace{-1ex}
\subsection{Existing CAM Designs}
\label{sec:existing_work}

\begin{figure}
    \centering
    \includegraphics[width=\linewidth]{Figures/bg4.png}
  %  \vspace{-0.4cm}
    \caption{Schematics of \textbf{(a)} 16T CMOS TCAM cell; \textbf{(b)} 2T-2ReRAM TCAM cell; \textbf{(c)} 20T-6MTJ TCAM cell; \textbf{(d)} 2FeFET TCAM cell.}
  %  \vspace{-0.4cm}
\label{fig:CAM}
\end{figure}


%\begin{figure}
%    \centering
%    \includegraphics[width=\linewidth]{Figures/conventional_CAM.pdf}
%    \caption{(a)Architecture of an M × N TCAM array.(b)NOR-type CAM bitcell.}
%\label{fig:array}
%\end{figure}


Various CAM designs have been proposed based on CMOS technology and NVM devices. A conventional 16T CMOS TCAM cell is shown in \autoref{fig:CAM}(a). CAMs leveraging NVM typically demonstrate enhanced performance over CMOS-based counterparts. For example, a 2T-2R TCAM design based on ReRAM was proposed in \cite{jing2t2r} for its compact structure, as shown in \autoref{fig:CAM}(b). While it consumes less area compared with conventional CMOS-based CAM designs, %issues arise primarily due to 
the low HRS/LRS ratio, low variable resistance and current-driven write-in mechanism associated with large access transistors  make the write and search energy significant concerns. 
\cite{20T6MTJ} proposed a 20T-6MTJ TCAM design as illustrated in \autoref{fig:CAM}(c), greatly enhancing the search speed and search performance. However, the reduced sense margin caused by the limited TMR ratio of STT-MRAM necessitates numerous transistors to address this issue, thus severely impacting area and power consumption.

Among NVM based CAM designs, utilizing FeFET stands out due to its high ON/OFF current ratio, efficient voltage-driven write mechanisms, low energy consumption, and cost-effectiveness, enabling significant performance improvements compared to conventional CMOS designs and other NVM-based designs. Building upon advanced FeFET models, researchers have proposed various FeFET CAM designs, particularly designs of TCAM. 
The 2FeFET TCAM design as depicted in \autoref{fig:CAM}(d) offers a compact alternative than CMOS counterparts \cite{2FeFET}. 2FeFET TCAM features a smaller cell area, reduced write and search energy consumption, and search delay. 
However, it faces limitations such as the lack of support for approximate matching functionality. 
%Our design will focus on addressing these issues.



%\autoref{fig:CAM}(a) illustrates a 4T-2FeFET cell, which incurs considerable energy and area overheads due to the presence of 4T cells. Furthermore, 
%Here we introduce a 2FeFET-2R CAM cell, depicted in \autoref{fig:CAM}(c), which incorporates the use of a drain resistor, similar to the 1FeFET-1R cell used for compute-in-memory applications \cite{power2}. The 2FeFET-2R cell extends beyond exact match functionality by employing a self-referenced sense amplifier to measure the Hamming distance (HD) between the input query and stored entries. Non-trivial sensing circuitry is required to effectively sample the limited HD.

%Various CAM designs have been proposed based on both CMOS and NVM devices so far. Utilizing NV technology such as FeFET distinguishes itself among various memory technologies due to its high ON/OFF current ratio, efficient voltage-driven write mechanisms, low energy consumption, and cost-effectiveness, thus enabling significant performance improvements compared to conventional CMOS designs. Building upon advanced FeFET models, researchers have proposed various FeFET CAM designs, particularly prevalent in the field of TCAM.  \autoref{fig:CAM}(a) shows a 4T-2FeFET cell, which incurs considerable energy and area overheads due to the presence of 4T cells. Furthermore, 2FeFET cells have been introduced as depicted in \autoref{fig:CAM}(b), which offer a more compact alternative than CMOS counterparts \cite{2FeFET}.
%Here we introduce a 2FeFET-2R CAM cell as shown in \autoref{fig:CAM}(c), which incorporates the use of a drain resistor previously studied in several works, similar to the 1FeFET-1R cell used for compute-in-memory applications \cite{power2}. 
%The 2FeFET-2R cell extends beyond the exact match functionality by employing a self-referenced sense amplifier to measure the Hamming distance (HD) between the input query and stored entries. Non-trivial sensing circuitry is required to effectively sample the limited HD.

\subsection{Threshold Matching Concepts and Related Works}
\label{sec:existing_work}

\begin{figure}
    \centering
    \includegraphics[width=\linewidth]{Figures/threshold_match.png}
  %  \vspace{-0.4cm}
    \caption{\textbf{(a) Exact match:} The stored entry that matches exactly with the query; \textbf{(b) Best match:} The stored entry that has the smallest distance to the query; \textbf{(c) Threshold match:} The stored entry whose distance to the query is below specified thresholds.}
 %   \vspace{-0.4cm}
\label{fig:Matchstyle}
\end{figure}


Most CMOS and NVM based CAM designs discussed earlier prioritize exact matching, as depicted in \autoref{fig:Matchstyle}(a), 
limiting their adaptability for data-intensive applications. 
%For data-centric applications,
In contrast, approximate matching gains favor due to its potential to enhance hardware utilization while maintaining acceptable accuracy. 
As a means to achieve approximate matching, best match CAMs, as illustrated in \autoref{fig:Matchstyle}(b), aim to output the stored entry with the highest similarity to the search query. 
For example, A-HAM \cite{AHAM} evaluates similarities across stored entries and identifies the closest Hamming distance to the input query. 
4T-2MTJ utilizing STT-MRAM \cite{STTMRAM} measures similarity between input query and stored entries in terms of \textit{ML} current and outputs the entry with the highest similarity. \cite{bestmatch} introduced a CAM design for minimum Hamming distance search using digital circuits for bit comparison. A Winner-Take-All (WTA) circuit at the output selects the entry with the highest degree of matching to the search query. However, CAMs designed for best matching may fail in applications requiring the output of multiple entries with specific similarities. Therefore, threshold matching CAMs were devised. 

Threshold matching CAMs, as illustrated in \autoref{fig:Matchstyle}(c), aim to provide multiple stored entries with similarity within a predefined Hamming distance (HD) threshold. 
For instance, the HD-CAM proposed in \cite{conventionalCAM} utilizes a 10T CMOS-based design incorporating \textit{ML} charge redistribution, enabling threshold matching with large HD tolerance, notably used in virus DNA classification. However, the SRAM based HD-CAM cell incurs substantial area and energy overheads. Furthermore, its effectiveness is limited in discerning patterns with substantial HDs due to the intricate tuning of \textit{ML} discharge current, making bit-by-bit tuning of HD thresholds impractical.
\cite{liu2023reconfigurable} introduced MHCAM, a multi-state CAM design encoding multiple CAM cells into distinct multi-states per dimension to perform both dimension-wise exact matching and reconfigurable threshold matching. However, additional transistors introduce fixed bit precisions (1-bit/2-bit/4-bit/8-bit per dimension), restricting fine-grained tunability in threshold matching and adaptability to applications demanding multi-state HD. The ReRAM-based CAM proposed in \cite{MASC} implements threshold matching by leveraging voltage scaling and controlling the precharge period. However, the current-driven mechanisms of ReRAMs result in high power consumption during operation and limited HD thresholds can be achieved due to the large \textit{ML} discharge current and non-trivial threshold-associated period sampling. \cite{2FeFETa} implements approximate matching functionality based on 2FeFET TCAM. It calculates the HD between search and stored vectors in a parallel manner by sensing the discharge rate of \textit{ML}. While achieving high energy efficiency and density in TCAM, it lacks precise control over the degree of approximate searching.

These threshold search CAMs all face a common issue, that they cannot precisely control the degree of approximate matching. Therefore, our design will focus on implementing bit-by-bit tuning of threshold to control the degree of approximate matching.



% To address these challenges, we propose a 2FeFET-2R CAM design, leveraging the advantages of 1FeFET-1R structure for compact, energy-efficient, and flexible bit-by-bit tunable HD threshold matching. We elaborate on our design in subsequent sections.




%Threshold matching CAMs illustrated in \autoref{fig:Matchstyle}(b) aim to provide multiple stored entries with similarity within a predefined Hamming distance (HD) threshold. 
%For instance, the HD-CAM proposed in \cite{conventionalCAM} employs a 10T CMOS-based design incorporating ML charge redistribution, enabling threshold matching with large HD tolerance, notably used in virus DNA classification.
%However, the SRAM based HD tolerant CAM cell incurs substantial area and energy overheads. Furthermore, its effectiveness diminishes in discerning patterns with substantial HDs due to the intricate tuning of ML discharge current, making bit-by-bit tuning of HD thresholds impractical.

%Most of aforementioned CMOS and NVM based CAM designs primarily focus on performing exact match as depicted in \autoref{fig:Matchstyle}(a), which restricts their adaptability to a wide range of emerging applications in the era of big data.
%However, in the context of data-centric applications,  approximate matching is increasingly favored. This approach offers the potential to significantly enhance hardware utilization while maintaining an acceptable level of  accuracy.
%In particular, threshold matching CAMs, illustrated in \autoref{fig:Matchstyle}(b), are designed to provide multiple stored entries that have a similarity within a specified Hamming distance (HD) threshold with the search query.



%An example is the HD tolerant CAM proposed in
%\cite{conventionalCAM}, which employs a 10T CMOS-based design incorporating a ML charge redistribution technique.
%This design implements threshold match with large HD tolerance, and is notably used in virus DNA classification. 
%Nevertheless, it's worth noting that the HD tolerant CAM, while functional, consumes substantial area and energy overheads. Moreover, its effectiveness is limited in discerning patterns  with substantial HDs  due to the intricate tuning of ML discharge current. This renders the bit-by-bit tuning of HD thresholds impractical. 

%The ReRAM-based CAM proposed in \cite{MASC} implements threshold matching by leveraging voltage scaling and controlling the precharge period.
%However, the current-driven mechanisms of ReRAMs result in high power consumption during operation, limiting achievable HD thresholds due to the large ML discharge current and non-trivial threshold-associated period sampling.

%\cite{liu2023reconfigurable} introduced MHCAM, a multi-state CAM design encoding multiple CAM cells into distinct multi-states per dimension to perform both dimension-wise exact matching and reconfigurable threshold matching. However, additional transistors introduce fixed bit precisions (1-bit/2-bit/4-bit/8-bit per dimension), restricting fine-grained tunability in threshold matching, and adaptability to applications demanding multi-state HD.

%\cite{2FeFETa} implements approximate matching functionality based on 2FeFET TCAM. It calculates the Hamming distance between search vectors and storage vectors in a massively parallel manner by sensing the discharge rate of ML. While achieving high energy efficiency and density in TCAM, it lacks precise control over the degree of approximate searching, which is the focus of our design breakthrough.

%The ReRAM-based CAM proposed in \cite{MASC} implements threshold matching by leveraging voltage scaling and controlling the precharge period. However, the current-driven mechanisms of ReRAMs result in high power consumption during operation, and limited HD thresholds can be achieved due to the large ML discharge current and non-trivial threshold associated period sampling.

%\cite{liu2023reconfigurable} proposed MHCAM, a multi-state CAM design that encodes multiple CAM cells into distinct multi-states per dimension to perform both dimension-wise exact matching and reconfigurable threshold matching. However, this approach introduces additional transistors to implement fixed bit precisions (i.e., 1-bit/2-bit/4-bit/8-bit per dimension), limiting its adaptability to specific applications demanding multi-state HD.




%The ReRAM-based CAM proposed in \cite{MASC} implements the threshold matching by leveraging voltage scaling and controlling the precharge period.
%However, the current-driven mechanisms of ReRAMs result in high power consumption during the operation and limited HD thresholds can be achieved due to the large ML discharge current and non-trivial threshold associated period sampling.
%to tolerant a few HD, thus  controls the search mode as either exact or approximate by selectively controlling the precharge period. 
%\cite{liu2023reconfigurable} proposed MHCAM, a multi-state CAM design that encodes multiple CAM cells into distinct multi-states per dimension  to perform both dimension-wise exact matching and reconfigurable threshold matching. However, additional transistors are introduced to implement fixed bit precisions (i.e., 1-bit/2-bit/4-bit/8-bit per dimension). Yet, this approach unavoidably restricts the extent of tunability in threshold matching, limiting its adaptability to specific applications demanding multi-state HD.
 

%The conventional n$\times$m CAM model is shown in the 
%CAM  searches  the input query across the stored entries in parallel, conducting a single input multiple output operation.
%\autoref{fig:array}(a) shows the conceptual schematic of a CAM array, which  performs bit-wise comparison between input query  and stored entries in parallel, conducting single input multiple output operations. 
%Each ML corresponds to the storage content in each row of the CAM, and all MLs are connected to a sense amplifier (SA). 
%Searchlines (SL/$\overline{SL}$) are placed vertically to write stored data and apply input data to the cells within the same column, while wordlines (WLs) are shared horizontally by the cells within a row.
%The storage in each column is connected to a pair of complementary search lines (SL/$\overline{SL}$), which control the writing and searching of rows. 
%Precharge transistors are used to precharge the MLs.
%The sense amplifier (SA) of each word measures the voltage of matchline (ML), which connects all the cells within the word, and is discharged depending on the accumulated bit-wise comparison results.
%and the voltage change of the ML is amplified and characterized by the SA \cite{conventionalCAM}.

%A single CAM unit stores content using a pair of cross-coupled inverters, and the MOS transistor on the row is activated by the writeline (WL) to drive SL and $\overline{SL}$ with complementary voltage values for writing. After the writing is completed, the SLs are precharged for reading and searching. The process and principle of searching are described as follows: first, the ML needs to be precharged to the high level of VDD, and during this process, the SLs need to be controlled at a low level to prevent ML from discharging. Then, turn off the precharge transistor and perform searching on the SL. If the search content on the SL matches the content stored in the unit (SL=D), the gate voltage of $M_{c3}$ is low, and ML cannot discharge, remaining at a high level. If it does not match (SL=$\overline{D}$), $M_{c1}$ and $M_{c2}$ control the gate voltage of $M_{c3}$ to be high, $M_{c3}$ opens, and ML discharges, lowering the voltage level. Therefore, any mismatched bit will be manifested as a decrease in ML voltage after amplification by the SA, resulting in an overall mismatch.

%The most basic content-addressable memory is a 16T CMOS CAM, which operates based on NOR and NAND cells. During the search process of the NAND cell, the transistor responsible for pre-charging needs to be charged to the power supply voltage. When searching for a storage match, all nMOS transistors are turned on, creating a path between the ML and ground, allowing the ML to discharge. If there is a bit miss, the ML will not discharge and remains at a high level. The disadvantage of the NAND matchline is that it depends on the quadratic delay of the number of cells, resulting in large parasitic capacitance and series resistance to ground, and low noise margin, so it is not widely used \cite{16TCMOS}.
%In comparison to the NAND cell, the NOR cell is more commonly used. The search cycle of the NOR cell is divided into three stages: searchline pre-charging, matchline pre-charging, and matchline evaluation. First, disconnect the matchline from ground, pre-charge the searchline to a low level, and charge the matchline to a high level. Then, drive the searchline with the content to be searched to evaluate the ML. The content to be searched is compared with the stored content based on the level change of the ML. The NOR cell has a fast evaluation speed. In the slowest 1-bit miss case, the critical matchline is composed of two series-connected transistors, but the number of transistors is high, resulting in high cost and power consumption.


%\vspace{-1ex}




%Correcting-Match Scheme to achieve soft-error tolerance, but typically only allow for a limited Hamming Distance of 1-4 bits \cite{softerror1}, \cite{softerror2}. 
%In \cite{conventionalCAM}, a dynamic and configurable search method was proposed, where the user can decide the threshold of the allowed mismatch, resulting in the design of a Hamming Distance Tolerant CAM (HD-CAM) whose parameters can be adjusted by the user. The schematic diagram of a bitcell in the HD-CAM is similar to \autoref{fig:accuracy}(a). It adds an evaluation transistor to the NOR-type CAM to control the evaluation voltage (i.e., the threshold voltage for mismatches). When Veval is set to the maximum voltage $V_{DD}$, it can perform a precise search. Alternatively, $V_{eval}$ can be set to other values less than $V_{DD}$ to perform an approximate search.

%However, the HD-CAM design using the 65nm CMOS technology has a large bitcell area and high cost, and consumes a large amount of power during search. It cannot distinguish between cases where the number of mismatched bits is close, which are all areas for future improvements of HD-CAM.


%\vspace{-1ex}
%\subsection{CAM Designs Based on FeFET}

%\label{sec:existing_work}
%To address the cost and power consumption issues, researchers have used new FeFET technology to design 4T-2FeFET TCAM \cite{4T2FeFET} and 2FeFET TCAM structure \cite{2FeFET}. Based on the material's non-volatility, FeFETs have the characteristic of storing and computing in one unit, thereby reducing the number of transistors used for writing and searching in the CMOS process, reducing the design area and cost. 

%First, we consider a 4T-2FeFET design in the context of a multi-domain Preisach model in Sec. \ref{sec:existing_work}(a). The design sets the write and read voltages to be 4V and 1V, respectively, and the write operation is divided into two steps. For example, when writing logic '1' to a TCAM cell, a high-level write voltage is used to drive the writeline (WL), which opens the $T_{3}$ and $T_{4}$ transistors. In the first step, $V_{write}$/0 is used to drive the gates of the $M_{1}$/$M_{2}$ FeFETs, polarizing $M_{1}$ and writing logic '1'. In the second step, 0/-$V_{write}$ is used to drive the gates of the $M_{1}$/$M_{2}$ FeFETs, writing logic '0' to $M_{2}$. During the search process, $V_{search}$ is used to drive the WL and two bitlines, thereby opening the FeFETs, while 0/$V_{search}$ is used to drive the searchline \cite{2FeFET}.

%For a fair comparison, we take 2FeFET design into consideration. The 2FeFET TCAM cell unit is shown in the Sec. \ref{sec:existing_work}(b), where a pair of parallel FeFETs connect the drain to the matchline (ML), the gate is connected to the bitline/searchline (BL/$\overline{SL}$ and $\overline{BL}$/SL), and the source is connected to the sourceline (ScL), driven by the write voltage through ScL.

%Taking writing logic '1' to the 2FeFET TCAM as an example, the specific writing process is described as follows:

%first, writing logic '1' to M1 requires adding $V_{write}$ to the BL/$\overline{SL}$ end, grounding the $\overline{BL}$/SL and ScL ends, ensuring that M1 $V_{GS}$ is high, and the ferroelectric is polarized, successfully writing logic '1' to $M_{1}$.Then, writing logic '0' to $M_{2}$ is similar to the first step, except that ScL is connected to $V_{write}$ to ensure that $M_{2}$ $V_{GS}$ is low, equivalent to writing logic '0' to $M_{2}$.
%The writing process for other content is shown in the table. 
%To perform a search, the ML voltage needs to be precharged and then apply the corresponding search voltage on SL/$\overline{SL}$. Logic '1' corresponds to a high voltage, and logic '0' corresponds to a low voltage. The search content is judged by the voltage change on the ML.

%Compared with conventional CMOS technology and 4T-2FeFET design, the 2FeFET TCAM structure has the advantages of low power consumption, low delay, and small area. However, it also has some limitations: the 2FeFET structure cannot select whether to match at a fixed voltage point, and there is no fixed sensetime to distinguish between cases where the number of mismatched bits is close. 






%In addition to threshold matching CAMs, many CMOS and NVM-based CAM designs incorporate approximate matching functionality through other means. 




%In this paper, we concentrate on CAMs with threshold matching functionality.
%To overcome the aforementioned challenges faced by these threshold matching CAMs, we propose a 2FeFET-2R CAM design, which implements compact, energy-efficient, and flexible bit-by-bit tunable HD threshold matching by exploiting the advantages of 1FeFET-1R structure, i.e., suppressed ON current with negligible variability, single transistor $AND$ logic, voltage-driven write and read mechanisms, etc. We elaborate on our design in the following sections.
%Therefore, considering the approximate search function and the high-energy efficient new devices based on FeFET comprehensively, we propose a 2FeFET-2R structure in Sec. \ref{sec:existing_work}(c), which inherits the advantages of FeFET's low power consumption, low delay, and low cost, while also being able to distinguish between cases where the number of mismatched bits is quite close. Compared with existing TCAM structures, it has richer functions and superior features.

%In addition to the previously discussed threshold matching CAMs, many CMOS and NVM based  CAM designs have also been developed to incorporate  approximate matching capabilities. 
%For instance, the A-HAM proposed in \cite{AHAM} evaluates the similarities across all stored entries and identifies the entry with the closest HD to the input query. 
%The PPAC in \cite{PPAC} calculates HD similarity by counting the number of `1's in  XNOR outputs of all CAM cells within a word. 
%The Hamming distance search CAM proposed in \cite{delayCAM} generates a delayed scoring signal when a bit mismatch occurs, and the Hamming distance is proportional to the time delay. 
%The STT-MRAM proposed in \cite{STTMRAM} measures the similarity between the input query and the stored entries in terms of $ML$ current. \cite{bestmatch} introduces a CAM design for minimum Hamming distance search, which utilizes digital circuits for bit comparison. Additionally, a Winner-Take-All (WTA) circuit is integrated at the end of the output voltage. It can be used to output the entry with the highest degree of matching to the search query, as depicted in \autoref{fig:Matchstyle}(c).
%However, these approximate matching CAMs are facing challenges such as high power consumption or constrained parallelism. 
%In this paper, we concentrate on the CAMs with threshold matching functionality.
%In addition, for the ReRAM-based MASC, the tolerance level for approximate Hamming distance search is low. Other structures use a digital search method, which results in low parallelism and so on.


\section{Methodology}\label{sec:Methodology}
% how constructed methodology
% \begin{table}[h!]
\large
\centering
\begin{adjustbox}{width=\columnwidth} % Automatically fit within column width
\begin{tabular}{|c|p{0.75\columnwidth}|} % Adjust the second column width proportionally
\hline
\textbf{Symbol} & \textbf{Explanation} \\ \hline
$q^{\text{obj}} \in SE(2)$ & Object pose \\ \hline
$q^{\text{robot}} \in \mathbb{R}^9$ & Robot configuration (9 DoF robot joint position) \\ \hline
$q^{\text{obj}}_\text{sg} \in SE(2)$ & Subgoal object pose \\ \hline
$q^{\text{robot}}_\text{sg} \in SE(3) \times \mathbb{R}^3$ & Subgoal robot configuration (End-effector pose in $SE(3)$ and gripper tip positions in 3D space) \\ \hline
$q^{\text{obj}}_{\text{init}} \in SE(2)$ & Initial object pose of the task \\ \hline
$q^{\text{robot}}_{\text{init}} \in \mathbb{R}^9$ & Initial robot configuration of the task (9 DoF robot joint position) \\ \hline
$q^{\text{obj}}_{\text{goal}} \in SE(2)$ & Goal object pose of the task \\ \hline
$s$ & State \\ \hline
% $sg$ & Subgoal information (e.g., subgoal object pose, robot configuration) \\ \hline
\end{tabular}
\end{adjustbox}
\caption{Notation table for task parameters}
\label{tab:state_notation}
\end{table}

% \begin{table}[h!]
\large
\centering
\begin{adjustbox}{width=\columnwidth} % Automatically fit within column width
\begin{tabular}{|c|p{0.75\columnwidth}|} % Adjust the second column width proportionally
\hline
\textbf{Symbol} & \textbf{Explanation} \\ \hline

$\mathcal{O}$ & Skill library containing all skills: $\mathcal{K} = \{o_{\text{NP}_1}, \dots, o_{\text{NP}_n}, o_{\text{Place}}\}$ \\ \hline
$i \in \{1, 2, \dots, n\}$ & Index for Non-Prehensile skills \\ \hline
$\alpha \in \{\text{NP}_1, \text{NP}_2, \dots, \text{NP}_n, \text{Place}\}$ & Index for skills in the skill library \\ \hline
$o_{\text{NP}_i}$ & $i$-th Non-Prehensile skill for manipulating objects in region $\mu_i$ \\ \hline
$o_{\text{Place}}$ & Skill for placing the object between different regions \\ \hline
% $\mu^i \subset SE(2)$ & Set of object poses manipulated by the $i$-th Non-Prehensile skill $o_{\text{NP}_i}$ \\ \hline
$\mu^i_\mathcal{I} \subset SE(2)$ & Set of initial object poses manipulated by the $i$-th Non-Prehensile skill $o_{\text{NP}_i}$ \\ \hline
$\mu^i_\beta \subset SE(2)$ & Set of subgoal object poses manipulated by the $i$-th Non-Prehensile skill $o_{\text{NP}_i}$ \\ \hline
$\mathcal{M}$ & Set of regions: $\mathcal{M} = \{(\mu^1_{\mathcal{I}}, \mu^1_\beta), (\mu^2_{\mathcal{I}}, \mu^2_\beta), \dots, (\mu^n_{\mathcal{I}}, \mu^n_\beta) \}$ \\ \hline
$\pi_\alpha(a \mid s, q^{\text{obj}}_\text{sg})$ & Policy function for generating actions based on the current and subgoal states of skill $o_\alpha$\\ \hline
$f_\alpha^{\text{pre}}(q^{\text{obj}}, q^{\text{obj}}_\text{sg})$ & Pre-contact Robot Configuration Sampler of skill $o_\alpha$ \\ \hline
$q^{\text{robot}}_{\text{pre}}$ & Pre-contact robot configuration \\ \hline
$\mathrm{pre}^{\text{level}}_\alpha(q^{\text{obj}}, q^{\text{robot}}, q^{\text{obj}}_{\text{sg}})$ & Precondition function for the skill $o_\alpha$ that takes $q^{\text{obj}}$, $q^{\text{robot}}$, and $q^{\text{obj}}_{\text{sg}}$ with abstract level \\ \hline
$c_\alpha$ & Connector for skill $o_\alpha$ \\ \hline
$\pi^c_\alpha(a \mid s, q^{\text{robot}}_{\text{sg}})$ & Policy function that generates actions for the connector $c_\alpha$, given $s$, and $q^{\text{robot}}_\text{sg}$ \\ \hline

\end{tabular}
\end{adjustbox}
\caption{Notation table for skill}
\label{tab:skill_notation}
\end{table}

% \begin{table}[h!]
\large
\centering
\begin{adjustbox}{width=\columnwidth} % Automatically fit within column width
\begin{tabular}{|c|p{0.75\columnwidth}|} % Adjust the second column width proportionally
\hline
\textbf{Symbol} & \textbf{Explanation} \\ \hline

$d$ & Metric function which calculates the distance between two object poses $q^{\text{obj}}$ and $q^{\text{obj}}_{\text{sg}}$ \\ \hline
$p_g$ & Goal object pose sampling probability during skill-RRT \\ \hline
$N_{max}$ & Maximum number of extend nodes to the tree during skill-RRT \\ \hline
$T$ & Tree in skill-RRT \\ \hline
$v$ & Node in the tree \\ \hline
$e$ & Edge in the tree \\ \hline
$T.V$ & Set of nodes in the tree $T$ \\ \hline
$v_0$ & Initial node \\ \hline
$v_{\text{near}}$ & Nearest node to the new node \\ \hline
$v_{\text{new}}$ & Newly added node \\ \hline
$v_{\text{goal}}$ & Node where the goal object pose serves as its subgoal object pose \\ \hline
$e_{\text{new}}$ & Newly added edge \\ \hline
$v.q^{\text{obj}}$ & Object pose at the node \\ \hline
$v.q^{\text{robot}}$ & Robot configuration at the node \\ \hline
$v.q^{\text{obj}}_{\text{sg}}$ & Subgoal object pose at the node \\ \hline
$\texttt{simulate}$ & Simulation function (IsaacGym) \\ \hline
$\tau_{\text{skill}}$ & Skill plan, consisting of a sequence of skills, connectors and their subgoals \\ \hline
$o_k$ & $k$-th skill in the skill plan $\tau_{\text{skill}}$ \\ \hline
$v_k$ & $k$-th node in the sequence of nodes from $v_0$ to $v_\text{goal}$ \\ \hline

\end{tabular}
\end{adjustbox}
\caption{Notation table for SKILL-RRT}
\label{tab:skillrrt_notation}
\end{table}
% \begin{table}[h!]
\large
\centering
\begin{adjustbox}{width=\columnwidth} % Automatically fit within column width
\begin{tabular}{|c|p{0.75\columnwidth}|} % Adjust the second column width proportionally
\hline
\textbf{Symbol} & \textbf{Explanation} \\ \hline

$c_\alpha$ & connector skill for skill $o_\alpha$ \\ \hline
$C$ & connector library contatining all connectors $C = \{c_{\text{NP}_1}, c_{\text{NP}_2}, \dots, c_{\text{Place}}\}$ \\ \hline
$\tau_{\text{skill}}^\text{abstract}$ & An abstract skill plan that consists solely of a sequence of skills and their associated subgoals, excluding any connector information. \\ \hline
$r^{\text{ee}}$ & End-effector distance reward: encourages the end-effector to move closer to its target position \\ \hline
$r^{\text{tip}}$ & Gripper tip position reward: aligns the gripper tips with their target positions \\ \hline
$r^{\text{success}}$ & Success reward: awarded when both the end-effector pose and gripper tip reach their target positions successfully \\ \hline
$r^{\text{object\_vel}}$ & Object velocity penalty: penalizes unnecessary object movement to maintain the validity of subsequent skill preconditions \\ \hline
$\epsilon^{\text{ee}}_0, \epsilon^{\text{ee}}_1$ & Parameters for the end-effector distance reward \\ \hline
$\epsilon^{\text{tip}}_0, \epsilon^{\text{tip}}_1$ & Parameters for the gripper tip position reward \\ \hline
$w^{\text{vel}}$ & Weight parameter for the object velocity penalty \\ \hline
$x_t^{\text{ee}}$ & End-effector keypoints at timestep $t$ \\ \hline
$\mathbf{x}_g^{\text{ee}}$ & Subgoal end-effector keypoints \\ \hline
$\mathbf{x}_t^{\text{tip}}$ & Gripper tip position at timestep $t$ \\ \hline
$\mathbf{x}_g^{\text{tip}}$ & Subgoal gripper tip position \\ \hline
$\mathbf{x}_t^{\text{obj}}$ & Object keypoints at timestep $t$ \\ \hline

\end{tabular}
\end{adjustbox}
\caption{Notation table for connector}
\label{tab:connector_notation}
\end{table}

% \begin{table}[h!]
\large
\centering
\begin{adjustbox}{width=\columnwidth} % Automatically fit within column width
\begin{tabular}{|c|p{0.75\columnwidth}|} % Adjust the second column width proportionally
\hline
\textbf{Symbol} & \textbf{Explanation} \\ \hline

$\theta$ & Replay success rate threshold \\ \hline
$\pi_{\text{IL}}$ & Imitation learned policy \\ \hline
$\mathcal{D}_{\text{imitate}}$ & Dataset consisting of state-action pairs from the replay of skill plans $\tau_{\text{skill}}$ \\ \hline

\end{tabular}
\end{adjustbox}
\caption{Notation table for imitation learning}
\label{tab:IL_notation}
\end{table}
\begin{figure*}[ht] % Force the figure at the top of the page
\centering
\vspace{-10mm}
\resizebox{\textwidth}{!}{
    \includegraphics{figures/cpnp_overview_v7.png}
}
% \caption{Overview of our method: (a) A skill library $\mathcal{O}$ that consists of $n$ number of non-prehensile (NP) and $m$ number of prehensile (P) skills are given. (b) We first train connector skills. To do this, we use \texttt{Abstract Skill-RRT} to collect pairs of states that the connector needs to connect, based on the preconditions and effects of the skills in $\mathcal{O}$, and the connectors are trained using reinforcement learning (RL). (c) We then generate data for imitation learning by running \texttt{Skill-RRT} with the skills in $\mathcal{O}$ and connector skill. Given an initial node $q_{\text{init}}$ (red circle) and a goal object pose $q^{\text{obj}}_{\text{goal}}$ (green circle), we generate a skill plan $\tau_{\text{skill}}$, which includes a sequence of skills (blue lines), connectors (yellow lines), and their associated subgoals (black circle). (d) We filter out low-quality data by replaying each skill plan N times, and filtering skill plans with low replay success rate based on a pre-defined threshold $\theta$. The filtered trajectories are used to train diffusion policy. The final diffusion policy is deployed in the real world in a zero-shot manner. }\label{fig:CPNP_overview} % Place the label after the caption
\caption{Overview of our method: (a) Examples of given skills in the bookshelf domain~(Figure~\ref{fig:CPNP_overview}, row 2). $K_\text{P}$ is a prehensile skill, $K_{\text{NP}_1}$ is a toppling skill that topples a book, and $K_{\text{NP}_n}$ is a pushing skill that pushes an object at a bottom shelf. (b) We first use \texttt{Lazy Skill-RRT} to collect a set of $(v.s, v_{\text{connect}}.s, K)$ triplets which defines the connections the connectors need to make, and $v$ denotes a node of RRT. The middle of (b) shows an example where the connector has to move the gripper from the end of the prehensile skill's state, $v.s$, to the beginning of the pushing skill, $v_{\text{connect}}.s$. We use RL to train connectors. (c) \texttt{Skill-RRT} is run with the trained connectors $\mathcal{C}$ and the set of skills $\mathcal{K}$ to generate a skill plan $\tau_{\text{skill}}$. Starting from the initial node $s_{0}$ (red circle) and the goal object pose $q_{\text{obj}}^{g}$ (green circle), the skill plan consists of a sequence of skills, connectors, and their associated desired object poses, or desired robot configuration (black circles). (d) We use IL to train skills using skill plans. To filter data, we replay each skill plan $N$ times, and those with a replay success rate below a predefined threshold $m$ are filtered out. The remaining high-quality trajectories are used to train a diffusion policy, which is deployed in the real world in a zero-shot manner.}\label{fig:CPNP_overview}
\vspace{-2mm}
\end{figure*}

\subsection{Problem formulation for prehensile-and-non-prehensile (PNP) manipulation problems}\label{method:PF}
\newcommand{\skillset}{\mathcal{K}}
\newcommand{\skill}{K}
\newcommand{\npskill}{K_{\text{NP}}}
\newcommand{\pskill}{K_{\text{P}}}
\newcommand{\applicabilitychecker}{\phi}
\newcommand{\policy}{\pi}
\newcommand{\qobj}{q_\text{obj}}
\newcommand{\qobjg}{q_\text{obj}^g}
\newcommand{\vobj}{\dot{q}_\text{obj}}
\newcommand{\Qobj}{Q_\text{obj}}
\newcommand{\qrobot}{q_\text{r}}
\newcommand{\vrobot}{\dot{q}_\text{r}}
\newcommand{\fsim}{f_{\text{sim}}}
\newcommand{\skillplan}{\tau_{\text{skill}}}

\newcommand{\node}{v}
\newcommand{\pipre}{\pi_{\text{pre}}}
\newcommand{\pipost}{\pi_{\text{post}}}


We denote the state space as $S$, action space as $A$, and space of stable and relevant object configurations as $\Qobj \subset SE(3)$, which only includes regions where the object can occupy, determined by environmental constraints like the presence of shelves. A state $s \in S$ includes object pose and velocity, robot joint positions, and velocities, denoted $\qobj, \vobj, \qrobot$ and $\vrobot$ respectively. An action $a \in  A$ consists of the target end-effector pose, the gain and damping values for a low-level differential inverse kinematics (IK) controller, and the target gripper width.

We assume we are given a set of manipulation skills, $\skillset=\{\pskill,K_{\text{NP}_1},\cdots,K_{\text{NP}_n}\}$, one of which is prehensile and the rest are non-prehensile (NP) skills. For instance, in the bookshelf domain, we have a \textit{topple} and \textit{push} non-prehensile skills, and one prehensile skill for placing the book in a desired region. We assume that an NP skill has an associated region defined on $\Qobj$; for instance, the topple skill is associated with object poses where the object is nearly upright with respect to the shelf. 

Each skill $K$ is a tuple of two functions, $K =\{ \applicabilitychecker, \pi\}$, where $\pi: S \times \Qobj \rightarrow A$ is a goal-conditioned policy that maps a state $s$ and a desired object pose $\qobj$ to an action\footnote{More accurately, instead of using $s$ directly, we use quantities derived from $s$. See Appendix~\ref{Appendix:NP_Skill},~\ref{Appendix:P_Skill} and~\ref{Appendix:Connector}}. We train our skills using RL, using the same policy computation structure from~\cite{kim2023pre}, where our policy $\pi$ consists of $\pipre$ and $\pipost$\footnote{$\pipre$ uses a different action space to ensure contact. See Appendix~\ref{Appendix:NP_Skill} and~\ref{Appendix:P_Skill}.}. $\pipre$ is a \emph{pre-contact} policy that predicts the robot configuration prior to executing the \emph{post-contact} policy that actually manipulates the object. For instance, for our prehensile skill, $\pipre$ puts the robot at a pre-grasp configuration, and $\pipost$ places the object. 

% The function $\applicabilitychecker:S \times \Qobj \rightarrow \{0,1\}$ is an applicability checker that tests whether it is possible to execute the skill policy in $s$ with $\qobj \in \Qobj$ as its goal. This could be an IK solver that checks the existence of a feasible grasp for both the current and target object poses for a prehensile skill, or a simple check for whether the desired object pose and the current object pose belong to the same region for an NP skill. Note that $\phi$ only checks whether skill \textit{can} be applied in the current state, and not whether the skill would succeed. For a more detailed skill examples, see experiment section, Table ~\ref{tab:skill_defn}.


The function $\applicabilitychecker:S \times \Qobj \rightarrow \{0,1\}$ is an applicability checker that tests whether it is possible to execute the skill policy in $s$ with $\qobj \in \Qobj$ as its goal. This could be an IK solver that checks the existence of a feasible grasp for both the current and target object poses for a prehensile skill, or a simple check for whether the desired object pose and the current object pose belong to the same region for an NP skill. Note that $\phi$ only checks whether skill \textit{can} be applied in the current state, and not whether the skill would succeed. For a more detailed skill examples, see experiment section, Table ~\ref{tab:skill_defn}.

Lastly, we assume we have a simulator $\fsim$ that takes in state $s$ and $\pi(\cdot;\qobj)$, a policy conditioned on desired object pose $\qobj$, and simulates the policy for $N_{sim}$ number of time steps, and returns the next state, $\fsim: S \times \Pi \rightarrow S$ where $\Pi$ is the set of policies of skills defined in $\skillset$. 

Given a pair of an initial state $s_0 \in S$ and the ultimate goal object pose $\qobj^g \in \Qobj$, the objective of a PNP problem is to find a sequence of actions for achieving $\qobj^g$. Because our policies output actions, this problem reduces to finding the sequence of skills and associated intermediate object poses, that we call a \textit{skill plan}, $\skillplan = \{\pi^{(t)},q^{(t)}\}_{t=1}^{T}$, such that when we simulate the sequence of $T$ policies from $s_0$, the resulting state would have $\qobj^g$ as its object pose.

At a high-level, our method uses \texttt{Skill-RRT} to collect data, and then distill it to a policy using IL. But to find the trajectories using \skillrrt, we first need connectors, and to do that, we need \lazyskillrrt~to collect relevant skills and states for which to train connectors. See Figure~\ref{fig:CPNP_overview} for the overview. We first introduce \texttt{Skill-RRT}, from which you can define \lazyskillrrt.

% i think we need to introduce the notion of sim state, and that we store everything inside the node
% to simulate the state. I am tempted to write Skill RRT to take as input the initial simulation state, and formulate the entire problem with this initial sim state as given.

%TODO: I need to mention the GPU acceleration

\subsection{Skill-RRT}
\newcommand{\extend}{\texttt{Extend}}
\newcommand{\connectorskillset}{\mathcal{C}}

\begin{algorithm}[h]
\caption{\skillrrt($s_0, \qobj^g, \skillset, \connectorskillset,\Qobj$)}
\label{algo:skill-rrt}
\begin{algorithmic}[1]
\State $T=\emptyset$   
\State $v_{0} \gets \{(\emptyset, \emptyset), s_0\}$ 
\State $T$.\texttt{AddNode}($parent=\emptyset, child=v_{0}$)

\For{$i = 1$ \textbf{to} $N_{\text{max}}$}
    \State ${K}, \qobj \gets $\hyperref[algo:UnifSmplSkillAndSubgoal]{\texttt{UnifSmplSkillAndSubgoal}}($\skillset,\Qobj$)

    \State ${v}_{\text{near}} \gets$ \hyperref[algo:GetFeasibleNearestNode]{\texttt{GetApplicableNearestNode}}($T, K, \qobj$)
    \If{$v_\text{near}$ is $\emptyset$} 
        \State \textbf{continue}
    \EndIf
    %TODO 
    %   A bit of complication here; what do we do with connector policies? 
    %   Ideally, we would pass in the connector set, and if that is empty, we would declare this as a lazy RRT. How about the following fix?
    \State \hyperref[algo:extend]{\texttt{Extend}}$\big(T, v_{\text{near}}, K, \qobj, \connectorskillset\big)$
    \If{\texttt{Near}($\qobj^g,T$)}
        \State \textbf{Return} \texttt{Retrace}($\qobj^g, T$)
    \EndIf
\EndFor
\State \textbf{Return} None

% \Statex
% \hrulefill
\end{algorithmic}
\label{algo:skillrrtbackbone}
\end{algorithm}
Algorithm~\ref{algo:skillrrtbackbone} provides the pseudocode for \skillrrt. The algorithm takes as input an initial state $s_0$, goal object configuration $\qobjg$, set of skills $\skillset$, set of connectors $\connectorskillset$, and object regions $\Qobj$. As we will soon explain, setting $\connectorskillset$ to an empty set makes it \lazyskillrrt.

The algorithm begins by initiating the tree, $T$, with an empty set, defining the root node, and adding it to the tree (L2-3). A node $v$ consists of a state $s$ and a pair of a skill policy $\pi$ and its desired object pose $\qobj$ that have been applied to the parent state to achieve the current state. For the root node, the policy and pose are set to an empty set (L2). We then begin the main for-loop. We first uniform-randomly sample a skill $K$ and the desired object pose for the skill, $\qobj$, using the function \texttt{UnifSmplSkillAndSubgoal} (L5), and then compute the nearest node from the tree among the nodes where $K$ can be applied (L6). Specifically, \texttt{GetApplicableNearestNode} function returns the nearest node $v$ where $K.\phi(v.s, \qobj)$ is true, and an empty set if no such node exists. If the function returns an empty set, we discard the sample, and move to the next iteration (L7-8). Otherwise, we use \texttt{Extend} function with $v_{near}$ (L9). Lastly, at every iteration, we check if $\qobjg$ is close enough to any of the nodes in the tree, and if it is, return the path by calling the \texttt{Retrace} function that computes a sequence of $v$ from root node to that node (L10-11). If no such node can be found after $N_{max}$ number of iterations, we return None. 

\iffalse
\begin{algorithm}[H]
\caption{\texttt{GetFeasibleNearestNode}($T, K, \qobj,\texttt{Dist})$}\label{algo:NearestNode}
\begin{algorithmic}[1]
    \State $V_{\text{sorted}} \gets \texttt{SortBy}(T.V, \qobj, \texttt{Dist})$
    \For{$v$ in $V_\text{near}$}
        \If{$K.\phi(v.s, \qobj)$} \Comment{Checks applicability}
            \State \Return $v$
        \EndIf
    \EndFor
    \State \Return $\emptyset$
\end{algorithmic}
\end{algorithm}
\fi

\newcommand{\connectingnode}{v_{\text{connect}}}
Algorithm~\ref{algo:extend} shows the $\texttt{Extend}$ function. The function takes in the tree $T$, skill to be simulated $K$, node $v$ that we will use $K$ from, desired object pose we will extend to, $\qobj'$, and set of connectors $\connectorskillset$. The algorithm begins by computing a pre-contact robot configuration for $K$, $\qrobot'$, using $K$'s pre-contact policy at state $v.s$ with $\qobj'$ as a goal (L1). It then creates a connecting node, using $\texttt{ComputeConnectingNode}$ shown in Algorithm~\ref{algo:computeconnectingnode}. This algorithm first checks if $\connectorskillset$ is empty, in which case this will instantiate \lazyskillrrt. In this case, we create a connecting node $\connectingnode$ with empty skill and object pose, with the same state as $v$, except the robot joint configuration is teleported to $\qrobot'$ (L2-4). Otherwise, we retrieve the connector $\pi_C$ for the given skill $K$ (L6), simulate it from $v.s$ with pre-contact configuration $\qrobot'$ as a goal (L7), and create $\connectingnode$ using $\pi_C$, $\qrobot'$, and the resulting state $s'$ (L8). We then return the connecting node $\connectingnode$, to the \texttt{Extend} function. From the connecting node's state, $\connectingnode.s$, we simulate the post-contact policy $\pipost$  with $\qobj'$ as its goal. If the skill fails to achieve $\qobj'$, then we return without modifying the tree (Algorithm~\ref{algo:extend}, L3-5).
Otherwise, we create a new node $\node'$ with the resulting state $s'$, simulated policy $\skill.\pipost$, and the desired object pose $\qobj'$ (L7). We add $\connectingnode$ with its parent as $\node$, and $\node'$ with $\connectingnode$ as its parent.

%For \lazyskillrrt, we pass in empty set for $\connectorskillset$ to \skillrrt, 
%We use this backbone to define \lazyskillrrt~that teleports the robot to the pre-contact configuration in simulation. Algorithm~\ref{algo:lazyextend} describes the algorithm. The algorithm takes as input the tree $T$, skill that will be simulated, $K$, the node to extend from, $v$, and the target object pose for the skill, $\qobj'$. The algorithm begins by computing the pre-contact robot configuration for the skill, $\qrobot'$ (L1). It then creates the node that \textit{would have} resulted \emph{if} we had executed a connector policy from $v$, $\connectingnode$, which practically have the same state as $\node$ except the robot configuration at $\qrobot'$ and $K.\pipre$ as its policy (L2-3). We then simulate the post-contact policy from the state of $\connectingnode$ with $\qobj'$ as its goal to get the resulting state $s'$, and check if the skill has achieved $\qobj'$. If not, we return without modifying the tree (L4-6). Otherwise, we create a new node $\node'$ with the resulting state $s'$, simulated policy $\skill.\pipost$, and the target object pose (L7). We add $\connectingnode$ with its parent as $\node$, and also $\node'$ with $\connectingnode$ as its parent. \lazyskillrrt~is defined as an instantiation of \skillrrtbackbone~with \texttt{LazyExtend} as the extend function.

\begin{algorithm}[h]
\caption{\texttt{Extend}($T, \skill, \node, \qobj', \connectorskillset$)}\label{algo:extend}
\begin{algorithmic}[1]
\State $\qrobot' \gets \skill.\pipre(\node.s.\qobj;\qobj')$
\State $\connectingnode \gets \hyperref[algo:computeconnectingnode]{\texttt{ComputeConnectingNode}}(\qrobot', \node, \skill, \connectorskillset)$
\State $s' \gets \fsim(\connectingnode.s, \skill.\pipost(\connectingnode.s;\qobj'))$
\If{$\hyperref[algo:Failed]{\texttt{Failed}}(s', \qobj')$}
\State \Return
\EndIf
\State $v' \gets (\skill.\pipost, \qobj',s')$
\State $T.\texttt{Add}(parent=\node, child=\connectingnode)$
\State $T.\texttt{Add}(parent=\connectingnode, child=\node')$
\end{algorithmic}
\end{algorithm}

\begin{algorithm}[h]
\caption{\texttt{ComputeConnectingNode}($\qrobot',\node,\skill,\connectorskillset$)}\label{algo:computeconnectingnode}
\begin{algorithmic}[1]
\State IsLazy $\gets \connectorskillset == \emptyset$
\If{IsLazy}
\State $\connectingnode \gets (\emptyset, \emptyset, \node.s)$
\State $\connectingnode.s.\qrobot \gets \qrobot'$ \Comment{Teleport the robot's config}
\Else
\State $\pi_C \gets \texttt{GetConnectorForSkill}(\connectorskillset,\skill) $
\State $s' \gets \fsim(\node.s, \pi_C(v.s;\qrobot'))$
\State $\connectingnode \gets (\pi_C,\qrobot', s')$
\EndIf
\Return $\connectingnode$
\end{algorithmic}
\end{algorithm}

Our description of \skillrrt~gives only a high-level description of some of the subroutines such as \texttt{UnifSmplSkillAndSubgoal}, \texttt{GetFeasibleNearestNode} or \texttt{Failed}~for brevity and clarity. Their detailed pseudocodes are in Appendix~\ref{Appendix:Skill-RRT Details}. Also, we use GPU-parallelized version of \texttt{Skill-RRT} in our experiments, which is described in Appendix~\ref{Appendix:Skill-RRT Details}.

\subsection{Training the connectors}\label{method:train_connector}
To train a set of connectors $\connectorskillset$, we collect problems using \lazyskillrrt{} by first creating a PNP problem that consists of a pair $(s_0,\qobjg)$, and then running Algorithm~\ref{algo:skill-rrt} with $\connectorskillset=\emptyset$. This will return a solution node sequence, some of which will be $\connectingnode$. We then collect a set of $(\node.s,\connectingnode.s,K)$ triplet from the node sequence, where $\node$ is the parent of $\connectingnode$, and $K$ is the skill that was used from $\connectingnode$ to get its child node.

% There is something odd about the definition of K above.
% Why is this a skill used from v_connect? Because we need to know what skill came after the connecting node, because we need the skill associated with the connector that got us to v_connect

We then train a connector $\pi_C$ for each $K$, whose goal is to go from $\node.s$ to the connecting node's robot configuration $\connectingnode.s.\qrobot$, using PPO~\cite{schulman2017proximal}, with minimum disturbance to the object pose in $\node.s$. This is a goal conditioned policy that maps a state to an action, $\pi_C(\cdot;\qrobot): S \rightarrow A$. The reward consists of four main components,
\[
    r_{\text{connector}} = r_{\text{ee}} + r_{\text{tip}} + r_{\text{obj-move}} + r_{\text{success}},
\]
where $r_{\text{ee}}$ and $r_{\text{tip}}$ are dense rewards based on the distances between the current and target end-effector poses and gripper tip positions, respectively. $r_{\text{obj-move}}$ penalizes an action if it changes the object pose from previous object pose, and $r_{\text{success}}$ gives a large success reward for achieving the target end-effector pose and gripper width. More details about the implementation of these reward terms can be found in Appendix~\ref{Appendix:Connector}.




%%% Continue from here

\subsection{Distilling \skillrrt~to a policy using IL}\label{method:IL}
%Directly applying a skill plan \(\tau_{\text{skill}}\), generated via \skillrrt, in the real world is impractical because planning requires significant online computation time, and frequent re-planning is necessary because of the modeling error. So, we distill \skillrrt~into a policy using IL. In particular, we use diffusion policy~\cite{chi2023diffusion}, which is able to handle multi-modal data without having to specify the number of modes, as in Gaussian Mixture Models~\cite{robomimic2021}, are easier to train than Generative Adversarial Nets (GANs)~\cite{goodfellow2014generative}, and is able to generate much sharper outputs than Variational Autoencoders (VAEs)~\cite{kingma2013auto} .


We distill \skillrrt~into a policy using Diffusion Policy~\cite{chi2023diffusion}, which have shown promising results in robotics, by imitating the solutions generated by \skillrrt. Despite their high performance, diffusion models have slow inference speeds due to the iterative denoising process. To achieve faster inference, we use a U-Net \cite{ronneberger2015u} architecture as the backbone instead of Transformers \cite{vaswani2017attention}, as Transformers incur high computational costs due to attention mechanisms. Additionally, we further reduce inference time by removing action chunking and state history from the diffusion policy. As a result, the policy is able to output actions at 75 Hz on an Nvidia 4090 GPU. Details of imitation learning such as hyperparameters are described in Appendix~\ref{Appendix:imitation_learning}.
%Inference speed is especially critical in our tasks because the robot must react quickly based on perceptual feedback. 

An action space of our diffusion policy uses target robot joint positions instead of the target end-effector pose, as used in the skill's policy. This eliminates the need for IK computation and enables faster inference. The policy uses a slightly different state space from the skill's, as shown in Table~\ref{table:IL_state}, to accommodate the simulation-to-reality gap. For instance, we found that the joint velocity of our robot (Franka Research 3) has a significant sim-to-real gap, so we use joint position from the previous time step, $q^{(t-1)}_r$, instead of using velocity directly. We also found that Franka's gripper cannot accept another command until the previous command is completed, which takes approximately 1.2 seconds. So, we include a binary input indicating whether that time has elapsed and is ready for another gripper command, $\mathbbm{1}_\text{gripper-executable}$.

Further, we use key points instead of poses for end-effector, object, and gripper tip, denoted $p_\text{ee}, p_\text{obj}$ and $p_\text{tip}$ respectively. The key points are the locations of eight corners of a bounding box for the gripper and object, and the locations of the tip of two fingers on the gripper for the gripper tip. We use key points rather than poses because poses involve orientations, which are known to be discontinuous and difficult for a neural network to learn~\cite{zhou2019continuity}.
%On the Continuity of Rotation Representations in Neural Networks
% \begin{table}[H]
% \centering
% \begin{adjustbox}{width=0.5\textwidth} % Scales the table to half the page width
% \begin{tabular}{|l|c|p{4cm}|} % Adjust the last column width to fit half-page format
% \hline
% \textbf{Name} & \textbf{Dimension} & \textbf{Explanation} \\
% \hline
% \texttt{joint position} & 9 & Current positions of the robot's joints. \\
% \hline
% \texttt{previous joint position} & 9 & Previous timestep's positions of the robot's joints. \\
% \hline
% \texttt{object keypoints} & 24 & 3D object keypoints. \\
% \hline
% \texttt{EE keypoints} & 24 & 3D robot end-effector keypoints. \\
% \hline
% \texttt{tip positions} & 6 & 3D positions of the left and right finger tips. \\
% \hline
% \texttt{relative EE keypoints} & 24 & 3D robot end-effector keypoints with respect to the object. \\
% \hline
% \texttt{relative tip position} & 6 & 3D positions of the finger tips with respect to the object. \\
% \hline
% \texttt{is gripper executable} & 1 & Binary value indicating whether the gripper is currently executable or not. \\
% \hline
% \makecell[l]{\texttt{previous gripper} \\ \texttt{target position}} & 1 & Previous timestep's robot action. \\
% \hline
% \texttt{goal object keypoints} & 24 & 3D goal object pose Keypoints. \\
% \hline
% \end{tabular}
% \end{adjustbox}
% \vspace{-0.2cm}
% \begin{flushleft}
% \footnotesize
% \caption{State space of distillation policy.}\label{table:IL_state}
% \end{flushleft}
% \vspace{-0.6cm}
% \end{table}

% \begin{table}[H]
% \centering
% \begin{adjustbox}{width=0.5\textwidth}
% \begin{tabular}{|c|c|c|}
% \hline
% \textbf{Extract from} & \textbf{Symbol} & \textbf{Name} \\
% \hline
% \multirow{4}{*}{robot}  & $s_t.q_r \in \mathbb{R}^9$ &robot joint position  \\
% \cline{1-2}
%  & $s_{t-1}.q_r \in \mathbb{R}^9$ & previous robot joint position  \\
% \cline{1-2}
% & $\big((s_t.q_r)^\text{ee}\big)^\text{keypoint} \in \mathbb{R}^{24}$ & robot end-effector keypoint positions. \\
% \cline{1-2}
%  & $(s_t.q_r)^\text{tip} \in \mathbb{R}^6$ & Robot gripper tip positions \\
% \hline
% \multirow{2}{*}{Robot w.r.t. $s_t.q_\text{obj}$}& $\big((s_t.q_r)^\text{rel, ee}\big)^\text{keypoint} \in \mathbb{R}^{24}$ & \makecell{relative robot end-effector \\ keypoint positions w.r.t $s_t.q_\text{obj}$}. \\
% \cline{1-2}
%  & $(s_t.q_r)^\text{rel, tip} \in \mathbb{R}^6$ & \makecell{relative robot gripper \\ tip positions  w.r.t $s_t.q_\text{obj}$} \\
% \hline
% object & $(s_t.q_\text{obj})^\text{keypoint} \in \mathbb{R}^{24}$ &object keypoint positions  \\
% \hline
% simulator & $ (t \mod 12) == 0$ & gripper executable. \\
% \hline
% action & $\in \mathbb{R}$ & Previous timestep's robot gripper width action. \\
% \hline
% target object pose & $(q'_\text{obj})^\text{keypoint}$ &target object keypoint positions \\
% \hline
% \end{tabular}
% \end{adjustbox}
% \begin{flushleft}
% \footnotesize
% \caption{State space of distillation policy.}\label{table:IL_state}
% \end{flushleft}
% \end{table}

\begin{table}[h!]
\centering
\begin{adjustbox}{width=\columnwidth}
\setlength\tabcolsep{20 pt}
\begin{tabular}{|c|c|}
\hline
\textbf{Symbol} & \textbf{Description} \\
\hline
$q^\text{(t)}_r \in \mathbb{R}^9$ &Robot joint position  \\
\cline{1-2}
 $q^\text{(t-1)}_r \in \mathbb{R}^9$ & Previous robot joint position  \\
\cline{1-2}
 $p^\text{(t)}_\text{ee} \in \mathbb{R}^{24}$ & \makecell{Robot end-effector keypoint positions \\ (computed from $q^\text{(t)}_r$)} \\
%  $\big((s_t.q_r)^\text{ee}\big)^\text{keypoint} \in \mathbb{R}^{24}$ & robot end-effector keypoint positions. \\
\cline{1-2}
 $p^\text{(t)}_\text{tip} \in \mathbb{R}^6$ & \makecell{Robot gripper tip positions \\ (computed from $q^\text{(t)}_r$)} \\
\hline
$p_\text{ee-rel}^\text{(t)} \in \mathbb{R}^{24}$ & \makecell{Relative robot end-effector \\ keypoint positions w.r.t $q^\text{(t)}_\text{obj}$} \\
\cline{1-2}
 $p^\text{(t)}_\text{tip-rel}\in \mathbb{R}^6$ & \makecell{Relative robot gripper \\ tip positions w.r.t $q^\text{(t)}_\text{obj}$} \\
\hline
  $p^\text{(t)}_\text{obj} \in \mathbb{R}^{24}$ &\makecell{Object keypoint positions \\ (computed from $q^\text{(t)}_\text{obj}$)}  \\
\hline
%  \mathbbm{1}(t-t_{last_grip_exec}) > 1.2 sec
% $ \big((t \mod 12) == 0\big) \in \{0, 1\}$ & gripper executable. \\
 $ \mathbbm{1}^\text{(t)}_\text{gripper-execute} $ & \makecell{Gripper action executability \\ (executable every 1.2 seconds)} \\
% $ \mathbb{I}\big( (t-t_\text{last\_grip\_exec})> 1.2 sec\big) $ & gripper executable. \\
\hline
 $ a^\text{(t-1)}_\text{width}\in \mathbb{R}$ & \makecell{Previous timestep's \\ robot gripper width action} \\
\hline
$p^\text{g}_\text{obj} \in \mathbb{R}^{24}$ & Goal object keypoint positions \\
\hline
\end{tabular}
\end{adjustbox}
\vspace{-0.2cm}
\begin{flushleft}
\footnotesize
\caption{The components of state space $S$ of distillation policy.}\label{table:IL_state}
\end{flushleft}
\vspace{-1.0cm}
\end{table}



To generate training data, we create multiple PNP problems and solve them using \skillrrt. To improve the success rate, we filter the generated data based on quality, ensuring that the dataset remains robust to perception and modeling errors. Specifically, we prioritize solutions that are more likely to reach the goal despite these uncertainties. For example, in the card flip domain, placing the card at the far end of the table might technically solve the problem, but even a slight perception error could cause the object to fall, making such solutions less reliable.

To create such a robust dataset we evaluate each skill plan, $\skillplan$, by executing it $N$ times while introducing noise into both the state and the torque output of the low-level controller. For object pose, we apply different scales of noise in different domains. We also add random noise to the commanded torque to reflect real-world noise effects in robot controller. The noises for each domain are summarized in Table \ref{table:DR_params}. We only accept skill plans whose success rate exceeds a threshold, $m$, i.e., if $N_{success}/N > m$. A high success rate during these replays indicates that the plan is reliable even under disturbances. 



%For example, when a book is in between two other books in the bookshelf domain, the noise in the object pose is larger because the book is only partially visible. 



 %Additionally, we randomize the friction of the environment since real-world friction is unknown and

% The noise injection process is described in detail in Appendix~\ref{Appendix:DR}.


\begin{table}[H]
\begin{adjustbox}{width=\columnwidth} % Automatically fit within column width

\centering
\begin{tabular}{|c|c|c|c|}
\hline
\textbf{}                   & \multicolumn{3}{c|}{\textbf{Range}}           \\ \hline
\textbf{Domain}                   & \textbf{Card Flip}     & \textbf{Bookshelf}     & \textbf{Kitchen}           \\ \hline
Object Pose (Position)              & $+ N[0.0, 0.003]$      & $+ N[0.0, 0.005]$ & $+ N[0.0, 0.003]$ \\ \hline
Object Pose (Orientation)           & $+ N[0.0, 0.03]$       & $+ N[0.0, 0.05]$ & $+ N[0.0, 0.03]$ \\ \hline
Robot joint position                & \multicolumn{3}{c|}{$+ N[0.0, 0.005]$}   \\ \hline
EE Pose (Position)                  & \multicolumn{3}{c|}{$+ N[0.0, 0.001]$}   \\ \hline
EE Pose (Orientation)               & \multicolumn{3}{c|}{$+ N[0.0, 0.01]$}    \\ \hline
Environment friction                & \multicolumn{3}{c|}{$\times U[0.8, 1.2]$}\\ \hline
Robot EE surface friction           & \multicolumn{3}{c|}{$\times U[0.9, 1.1]$}\\ \hline
Object mass                         & \multicolumn{3}{c|}{$\times U[0.8, 1.2]$}\\ \hline
Torque noise                        & \multicolumn{3}{c|}{$+ N[0.0, 0.03]$}    \\ \hline
% 3D key point noise                  & $+ N[0.0, 0.03]$      & &    \\ \hline
% Sensor noise                        & $+ N[0.0, 0.01]$      & &    \\ \hline
\end{tabular}
\end{adjustbox}
% \begin{flushleft}
% \footnotesize
% \textbf{Table \ref{table:DR_params}} — Range for domain randomization. \( U[\text{min}, \text{max}] \) denotes uniform distribution, and \( N[\mu, \sigma] \) denotes normal distribution. The symbol "$+$" represents the summation operation, and "$\times$" represents the scaling operation.
% \end{flushleft}
\caption{Ranges of noise for data filtering. \( U[\text{min}, \text{max}] \) denotes uniform distribution, and \( N[\mu, \sigma] \) denotes normal distribution. The symbol "$+$" represents the summation operation, and "$\times$" represents the product operation.}\label{table:DR_params}
\end{table}


\iffalse
\begin{algorithm}[H]
\caption{Data filtering}\label{algo:Collect_data}
\begin{algorithmic}[1]
\State \textbf{Input:} Skill library ${\mathcal{O}} = \{o_{\text{NP}_1}, o_{\text{NP}_2}, \ldots, o_{\text{NP}_n}, o_{\text{Place}}\}$
\State \textbf{Initialize:} Counter $n \leftarrow 0$, dataset $\mathcal{D} \leftarrow \emptyset$
\While{$n < 500$}
    \State Randomly sample initial $q^\text{obj}_{\text{init}}$ and goal object pose $q^\text{obj}_{\text{goal}}$
    \State Generate skill plan $\tau_{\text{skill}} = \text{Skill-RRT}(q^\text{obj}_{\text{init}}, q^\text{obj}_{\text{goal}}, {\mathcal{O}})$
    \State Replay the skill plan $\tau_{\text{skill}}$ 400 times to obtain full trajectories $\tau_{\text{full}}^{(i)}$ for $i = 1, \ldots, 400$
    \State Compute replay success rate $r = \frac{1}{400} \sum \text{success}(\tau_{\text{full}}^{(i)})$
    \If{$r \geq 0.9$}
        \State $\mathcal{D} \leftarrow \mathcal{D} \cup \{\text{selected 30 successful trajectories}\}$
        \State $n \leftarrow n + 1$
    \EndIf
\EndWhile
\end{algorithmic}
\end{algorithm}
\fi
%Using entire trajectories generated by \texttt{Skill-RRT} can degrade the performance of the distillation policy. Even if a skill plan is successfully generated, it may not consistently achieve the task due to uncertainties in state transitions caused by contact-rich dynamics. For example, if a subgoal is positioned very close to the edge of the table, it might solve the task but risk dropping the card due to dynamic uncertainties in the real world. Therefore, we need a metric to measure whether the generated skill plan is robust enough to reliably solve the given task and to filter out those with poor performance.

% TODO this needs to be moved to the related work
%Previous works \cite{agia2023stap, xue2024logic} propose a metric for measuring the optimality of skill plans as the sum of value functions. However, these approaches require either additional modules, such as skill dynamics models and uncertainty quantification of skills, or specific forms of RL training to accurately approximate the value function. Instead, we propose a simpler metric for measuring the robustness of skill plans by replaying them $N$ times in simulation. Specifically, for a given skill plan, we sequentially replay the skills and their subgoals $N$ times. We then calculate the replay success rate, defined as the ratio of successful replays. A skill plan replay is considered successful if all the skills are executed successfully and reach the given goal, while it is considered a failure if any skill fails to achieve its subgoal (e.g., due to timeout or dropping an object). The replay success rate of all skill plans is measured, and plans with a replay success rate lower than a predefined threshold $\theta$ (e.g. 90\%) re filtered out. Trajectories consisting of state-action data pairs generated by replaying skill plans with high replay success rates are then used to train the distillation policy.

% we evaluate each skill plan by replaying it $N$ times in simulation. When a skill plan $\tau_\text{skill}=\{ q^{\text{obj}}_{\text{init}}, q^{\text{obj}}_{\text{goal}}, \{(c,q^{\text{robot}}_{\text{pre}})_k, (o,q^{\text{obj}}_{\text{sg}})_k\}_{k=0}^K) \}$ is found, its skills and connectors are re-executed N times. The replay success rate of the skill plan is then measured based on whether all skills and connectors succeed during replay. Skill plans with a replay success rate lower than a predefined threshold $\theta$ (e.g. 90\%) are filtered out. A high replay success rate indicates that the skill plan consistently succeeds in reaching the goal despite uncertainties in state transitions and domain randomizations. Training the IL policy with such high-quality trajectories improves its overall performance.

%To address this, we retain only skill plans with a high success rate, ensuring the policy performs reliably across trials and generalizes better during deployment. Specifically, only skill plans with a success rate exceeding a predefined threshold \( \theta \) are included in the dataset \( \mathcal{D}_{\text{imitate}} \). The dataset is curated by replaying skill plans on randomly selected tasks and retaining only successful trajectories, resulting in high-quality training data.

% Once the dataset \(\mathcal{D}_{\text{imitate}}\) is collected, we employ a U-Net-based diffusion model \cite{ho2020denoising, chi2023diffusion} to train the imitation learned policy \(\pi_{\text{IL}}\). Diffusion models are particularly suitable for our task, as they excel in handling multimodal and high-dimensional data. In our dataset, the same state may lead to different actions depending on the skill or subgoal, making multimodality a key challenge.

% The policy is trained to predict a single action for each state at a single time step. Since our task is contact-rich, the policy performs better with frequent observation feedback. Also we use only the single time-step observation during training for reducing computational complexity. The input and output configuration of the policy $\pi_\text{IL}$ is outlined in Table \ref{table:IL_state} and \ref{table:IL_action}.

%The trained distillation policy is directly deployed in real-world manipulation tasks in a zero-shot manner. By leveraging the diversity and quality of trajectories, along with the generalization capabilities of the diffusion model, the policy is able to handle complex manipulation tasks without requiring additional fine-tuning.



\section{Experiments}\label{sec:Experiments}
\subsection{Experiment Setup}
We conduct experiments across three distinct domains: card flip, bookshelf, and kitchen shown in Figure~\ref{fig:CPNP_tasks}. We compare against baselines in simulations of these environments and then show real-world results. Table~\ref{tab:skill_defn} shows the definitions of non-prehensile skills used in each domain. We include the definition of the prehensile skill in Appendix~\ref{Appendix:P_Skill}. 
\begin{table*}[ht]
\centering
\begin{tabular}{|c|c|cc|cc|}
\hline
\textbf{\begin{tabular}[c]{@{}c@{}}Domain\\ names\end{tabular}} & \textbf{Card flip}                                                                                    & \multicolumn{2}{c|}{\textbf{Bookshelf}}                                                                                                                                                                                                                                  & \multicolumn{2}{c|}{\textbf{Kitchen}}                                                                                                                                                                                                                                                                           \\ \hline
\textbf{NP skills}                                                 & $K_{\text{slide}}$                                                                                    & \multicolumn{1}{c|}{$K_{\text{topple}}$}                                                                                                      & $K_{\text{push}}$                                                                                                        & \multicolumn{1}{c|}{$K_{\text{sink}}$}                                                                                                     & $K_{\text{cupboard}}$                                                                                                                                              \\ \hline
$\phi(q,q')$                                                    & \begin{tabular}[c]{@{}c@{}}$q,q' \in \Qobj$, \\ $R_x(q) = R_x(q')$,\\ $R_y(q) = R_y(q')$\end{tabular} & \multicolumn{1}{c|}{\begin{tabular}[c]{@{}c@{}}$q,q' \in \Qobj^{\text{(upper-shelf)}}$, \\ $R_x(q) = R_x(q')$,\\ $R_z(q) = R_z(q')$\end{tabular}} & \begin{tabular}[c]{@{}c@{}}$q,q' \in \Qobj^{\text{(lower-shelf)}}$, \\ $R_x(q) = R_x(q')$,\\ $R_y(q) = R_y(q')$\end{tabular} & \multicolumn{1}{c|}{\begin{tabular}[c]{@{}c@{}}$q,q' \in \Qobj^{\text{(sink)}}$, \end{tabular}} & \begin{tabular}[c]{@{}c@{}}$q,q' \in \Qobj^{\text{(l-cupboard)}} \text{ or } \Qobj^{\text{(r-cupboard)}}$,\\ $R_x(q) = R_x(q')$,\\ $R_y(q) = R_y(q')$\end{tabular} \\ \hline
$\pi$                                                           & $\pi_{\text{slide}}$                                                                                  & \multicolumn{1}{c|}{$\pi_{\text{topple}}$}                                                                                                    & $\pi_{\text{push}}$                                                                                                      & \multicolumn{1}{c|}{$\pi_{\text{sink}}$}                                                                                                   & $\pi_{\text{cupboard}}$                                                                                                                                            \\ \hline
\end{tabular}
\caption{Non-prehensile manipulation skills trained using RL for each domain. The row $\phi(q,q')$ denotes the applicability checker for each skill. We write $q = \qobj$ and $\phi(q,q')$ instead of $\phi(s,q')$ with abuse of notation for brevity and clarity. Here, $R_x$, $R_y$ and $R_z$ denote the rotation matrices of pose $q$ with respect to $x,y$, and $z$ axes. For the card flip domain, slide skill is applicable if the desired pose $q'$ involves only translation and rotation wrt the z-axis from $q$. For the bookshelf domain, we have two sub-regions $\Qobj^{\text{(upper-shelf)}}$ and $\Qobj^{\text{(lower-shelf)}}$,  as shown in Figure~\ref{fig:region} (middle). To apply the topple or push skill, $q$ and $q'$ must belong to the same sub-region. Toppling skill is applicable only for orientation wrt the y-axis. The pushing skill is applicable for orientation wrt the z-axis and any translation within the lower shelf. For the kitchen domain, we have three sub-regions: $\Qobj^{\text{(sink)}}$, $\Qobj^{\text{(l-cupboard)}}$, and $\Qobj^{\text{(r-cupboard)}}$. $K_{\text{sink}}$ is applicable for any orientation or translation, as long as the object stays within the sink. $K_{\text{cupboard}}$ is applicable for an orientation wrt z-axis and any translation, as long as the object moves within a left or right cupboard. The last row, $\pi$, indicates different policies trained for different skills. The details of training are in Appendix~\ref{Appendix:NP_Skill}.}\label{tab:skill_defn}
\end{table*} 
\begin{figure*}[h]
\centering
\resizebox{\textwidth}{!}{
    \includegraphics{figures/region.png}
}
\caption{Regions for each domain. The world frame is shown in the bottom left of each figure, where the red line represents the $x$-axis, the green line represents the $y$-axis, and the blue line represents the $z$-axis.}\label{fig:region}
% \vspace{-1mm}
\end{figure*}

%In the card flip domain, the robot has to slide the card, flip it using pick-and-place, and slide it again to reach the goal. In the bookshelf domain, the robot must move a book from an upper shelf to the goal pose on a lower shelf. The book is initially surrounded by other books, so the robot needs to topple them to enable a grasp. Additionally, the lower shelf has a tight space, requiring the robot to push the book into place rather than directly placing it. In the kitchen domain, the robot's task is to move a cup, initially in an ungraspable pose inside the sink, to either the left or right cupboard.

%The object pose space $\Qobj$ for each domain is outlined in Table~\ref{table:exp_setup}. In the card flip domain, there are two regions: $R^\text{up}$ and $R^\text{down}$, which are stable card poses on a table with the card facing upward and downward respectively. In the bookshelf domain, there are two regions: \(R^\text{uppershelf}\), are poses of the book at the upper shelf where it is in between other books, and \(R^\text{lowershelf}\), which are stable poses on the lower shelf. In the kitchen domain, there are three regions: \( R^\text{sink} \), \( R^\text{r-cupboard} \), and \( R^\text{l-cupboard} \), which represent stable poses in the sink, right cupboard, and left cupboard, respectively. The detailed region descriptions are outlined in the Appendix \ref{Appendix:region}.



% The set of skills $\mathcal{K}$ for each domain is shown in Table~\ref{table:exp_setup}.

\iffalse
, as outlined in Definition~\ref{Def:Skill}. The skill consists of two components: (1) a goal-conditioned policy \( \pi \) and (2) an applicability checker \( \phi \). Every goal-conditioned policy \( \pi \) consists of two different policies: (1) a \textit{pre-contact} policy \( \pi_\text{pre} \), and (2) a \textit{post-contact} policy \( \pi_\text{post} \). Additionally, we define an applicability checker function of each skill in each domain in Definition~\ref{Def:phi_applicability}. The details of each policy, the applicability checker functions, and the space of stable object poses $\Qobj$ are described in Appendix~\ref{Appendix:region} and~\ref{Appendix:Skill_each_domain}.

% \noindent\hrulefill

% \noindent \textbf{Non-Prehensile (NP) Skill Definition} $K_{\text{NP}}$

% \begin{itemize}\label{Def:NP}
%     \item \textbf{Goal-conditioned Policy} \(\pi\)

%     \begin{itemize}

%         \item \(\pi = 
%             \begin{cases} 
%             \text{pre-contact policy, } \pi_\text{pre}(s.q_\text{obj}, q'_\text{obj}): Q_\text{obj} \times Q_\text{obj} \rightarrow Q_r\\
%             \text{post-contact policy, } \pi_\text{post}(s, q'_\text{obj}): S \times Q_\text{obj} \rightarrow A \\
%             \text{connector policy, } \pi_C(s, q'_r): S \times Q_r \rightarrow A

%             \end{cases}\)
%     \end{itemize}

%     \item \textbf{Region $R \subsetneq Q_\text{obj}$}
    
%     \item \textbf{Applicability checker} $\phi: Q_\text{obj} \times Q_\text{obj} \rightarrow \{0, 1\}$

%     \begin{itemize}
%         \item[] \underline{\textbf{Card Domain}}
%         \begin{itemize}
%             \item \(\phi_\text{slide}(q_\text{obj}, q'_\text{obj}) = 
%             \begin{cases} 
%             1, & \text{if } q'_\text{obj} \in R, \, q_\text{obj} \in R, \text{ and } \pi_\text{pre}(s.q_\text{obj}, q'_\text{obj}) \text{ is collision-free}, \\ 
%             0, & \text{otherwise.} 
%             \end{cases}\)
%         \end{itemize}
    

%         \item[] \underline{\textbf{Bookshelf Domain}}
%             \begin{itemize}
%                 \item \(\phi_\text{topple}(q_\text{obj}, q'_\text{obj}) = 
%                 \begin{cases} 
%                 1, & \text{if } q'_\text{obj} \in R, \, q_\text{obj} \in R, \text{ and } \pi_\text{pre}(s.q_\text{obj}, q'_\text{obj}) \text{ is collision-free}, \\ 
%                 0, & \text{otherwise.} 
%                 \end{cases}\)
%                 \item \(\phi_\text{push}(q_\text{obj}, q'_\text{obj}) = 
%                 \begin{cases} 
%                 1, & \text{if } q'_\text{obj} \in R, \, q_\text{obj} \in R, \text{ and } \pi_\text{pre}(s.q_\text{obj}, q'_\text{obj}) \text{ is collision-free}, \\ 
%                 0, & \text{otherwise.} 
%                 \end{cases}\)
%             \end{itemize}

%         \item[] \underline{\textbf{Kitchen Domain}}
%             \begin{itemize}
%                 \item \(\phi_\text{sink}(q_\text{obj}, q'_\text{obj}) = 
%                 \begin{cases} 
%                 1, & \text{if } q'_\text{obj} \in R, \, q_\text{obj} \in R, \text{ and } \pi_\text{pre}(s.q_\text{obj}, q'_\text{obj}) \text{ is collision-free}, \\ 
%                 0, & \text{otherwise.} 
%                 \end{cases}\)
%                 \item \(\phi_\text{l-cupboard}(q_\text{obj}, q'_\text{obj}) = 
%                 \begin{cases} 
%                 1, & \text{if } q'_\text{obj} \in R, \, q_\text{obj} \in R, \text{ and } \pi_\text{pre}(s.q_\text{obj}, q'_\text{obj}) \text{ is collision-free}, \\ 
%                 0, & \text{otherwise.} 
%                 \end{cases}\)
%                 \item \(\phi_\text{r-cupboard}(q_\text{obj}, q'_\text{obj}) = 
%                 \begin{cases} 
%                 1, & \text{if } q'_\text{obj} \in R, \, q_\text{obj} \in R, \text{ and } \pi_\text{pre}(s.q_\text{obj}, q'_\text{obj}) \text{ is collision-free}, \\ 
%                 0, & \text{otherwise.} 
%                 \end{cases}\)
%             \end{itemize}

%     \end{itemize}

% \end{itemize}

% \noindent\hrulefill


% \noindent\hrulefill

% \noindent \textbf{Prehensile (P) Skill Definition} $K_{\text{P}}$

% \begin{itemize}\label{Def:P}
%     \item \textbf{Goal-conditioned Policy $\pi$}
%     \begin{itemize}

%         \item \(\pi = 
%             \begin{cases} 
%             \text{pre-contact policy, } \pi_\text{pre}(s.q_\text{obj}, q'_\text{obj}): Q_\text{obj} \times Q_\text{obj} \rightarrow Q_r\\
%             \text{post-contact policy, } \pi_\text{post}(s, q'_\text{obj}): S \times Q_\text{obj} \rightarrow A \\
%             \text{connector policy, } \pi_C(s, q'_r): S \times Q_r \rightarrow A

%             \end{cases}\)
%     \end{itemize}

%     \item \textbf{Region $R = Q_\text{obj}$}

%     \item \textbf{Applicability checker} $\phi: S \times Q_\text{obj} \rightarrow \{0, 1\}$
%     \begin{itemize}
%         \item \(\phi(s, q'_\text{obj}) = 
%         \begin{cases} 
%         1, & \text{if } \pi_\text{pre}(s.q_\text{obj}, q'_\text{obj}) \text{ is collision-free}, \\ 
%         0, & \text{otherwise.} 
%         \end{cases}\)
%     \end{itemize}

% \end{itemize}
% \noindent\hrulefill

\begin{definition}[\textbf{Skill Definition}] \label{Def:Skill}
\noindent\hrulefill
\normalfont
% \noindent \textbf{Skill Definition} $K$

\begin{itemize}\label{Def:NP}
    \item \textbf{$K$'s goal-conditioned policy} \(K.\pi\)

    \begin{itemize}
        \item \(\pi  
            \begin{cases} 
            \text{pre-contact policy,} \\ 
            \quad \pi_\text{pre}(q_\text{obj}, q'_\text{obj}): Q_\text{obj} \times Q_\text{obj} \rightarrow Q_r, \\[5pt]
            
            \text{post-contact policy,} \\ 
            \quad \pi_\text{post}(s, q'_\text{obj}): S \times Q_\text{obj} \rightarrow A, \\[5pt]
    
            % \text{connector policy,} \\ 
            % \quad \pi_C(s, q'_r): S \times Q_r \rightarrow A.
            \end{cases}\)
    \end{itemize}

    % \item \textbf{Region $R \subsetneq Q_\text{obj}$}
    
    \item \textbf{$K$'s applicability checker} $K.\phi: Q_\text{obj} \times Q_\text{obj} \rightarrow \{0, 1\}$
\end{itemize}
\end{definition}

\noindent\hrulefill



% \noindent\hrulefill
\begin{definition}[\textbf{Applicability checker of each skill}] \label{Def:phi_applicability}
\noindent\hrulefill
\normalfont
% \noindent \textbf{Skill Definition} $K$

    % \item \textbf{Region $R \subsetneq Q_\text{obj}$}
    
    % \item \textbf{Applicability checker} $\phi: Q_\text{obj} \times Q_\text{obj} \rightarrow \{0, 1\}$
    
    % \noindent\hrulefill

    % \noindent\hrulefill
    % The applicability checker function for each skill in each domain.
    \begin{itemize}
        \item[]
        \item \underline{\textbf{Card Flip Domain}} $\mathcal{K}=\{K_{\text{NP}_\text{slide}}, K_\text{P}\}$
            \begin{itemize}
                \item \(K_{\text{NP}_\text{slide}}.\phi(q_\text{obj}, q'_\text{obj})\) \\
                \(\begin{cases} 
                1, & \text{if } q_\text{obj}, q'_\text{obj} \in \Qobj, R_x(\qobj) =R_x(q'_\text{obj}), \\ 
                   & \quad \text{and } R_y(\qobj) = R_y(q'_\text{obj}), \\ 
                0, & \text{otherwise.} 
                \end{cases}\)
                \item \(K_\text{P}.\phi(\qobj, q'_\text{obj}) \\
                \begin{cases} 
                1, & \text{if } q_\text{obj}, q'_\text{obj} \in \Qobj, \\
                &\quad \pi_\text{pre}(q_\text{obj}, q'_\text{obj}) \text{ is collision-free} \\
                & \quad \text{and not null} \\ 
                0, & \text{otherwise.} 
                \end{cases}\)
            \end{itemize}
        where \( R_x \) and \( R_y \) are notations for the \( x \) and \( y \) axes in the global frame.

        \item \underline{\textbf{Bookshelf Domain}} $\mathcal{K}=\{K_{\text{NP}_\text{topple}},K_{\text{NP}_\text{push}},K_{\text{P}}\}$
            \begin{itemize}
                \item \(K_{\text{NP}_\text{topple}}.\phi(q_\text{obj}, q'_\text{obj}) \\ 
                \begin{cases} 
                1, & \text{if } q_\text{obj}, q'_\text{obj} \in \Qobj^\text{upper-shelf}, R_x(\qobj) = R_x(q'_\text{obj}), \\ 
                & \quad \text{and } R_z(\qobj) = R_z(q'_\text{obj}), \\ 
                0, & \text{otherwise.} 
                \end{cases}\)
                \item \(K_{\text{NP}_\text{push}}.\phi(q_\text{obj}, q'_\text{obj}) \\
                \begin{cases} 
                1, & \text{if } q_\text{obj}, q'_\text{obj} \in \Qobj^\text{lower-shelf}, R_x(\qobj) = R_x(q'_\text{obj}), \\ 
                & \quad \text{and } R_y(\qobj) = R_y(q'_\text{obj}), \\ 
                0, & \text{otherwise.} 
                \end{cases}\)

                \item \(K_{\text{P}}.\phi(\qobj, q'_\text{obj}) \\
                \begin{cases} 
                1, & \text{if } q_\text{obj}, q'_\text{obj} \in \Qobj, \\
                &\quad \pi_\text{pre}(q_\text{obj}, q'_\text{obj}) \text{ is collision-free} \\
                & \quad \text{and not null} \\ 
                0, & \text{otherwise.} 
                \end{cases}\)
            \end{itemize}
        where \( R_z \) is a notation for the \( z \) axis in the global frame. $Q_{\text{obj}}^{\text{upper-shelf}} \subset \Qobj$ and $Q_{\text{obj}}^{\text{lower-shelf}} \subset \Qobj$ are subspaces of $\Qobj$. $Q_{\text{obj}}^{\text{upper-shelf}}$ is a space at the upper shelf, between other books where the target book is upright. $\Qobj^\text{lower-shelf}$ is a space at the lower shelf, where the book is on the lower shelf.

        \item \underline{\textbf{Kitchen Domain}} \\ $\mathcal{K}=\{K_{\text{NP}_\text{sink}},K_{\text{NP}_\text{l-cupboard}},K_{\text{NP}_\text{r-cupboard}},K_{\text{P}}\}$
            \begin{itemize}
                \item \(K_{\text{NP}_\text{sink}}.\phi(q_\text{obj}, q'_\text{obj})  \\
                \begin{cases} 
                1, & \text{if } q_\text{obj}, q'_\text{obj} \in \Qobj^\text{sink} \\ 
                0, & \text{otherwise.} 
                \end{cases}\)
                \item \(K_{\text{NP}_\text{l-cupboard}}.\phi(q_\text{obj}, q'_\text{obj}) \\
                \begin{cases} 
                1, & \text{if } q_\text{obj}, q'_\text{obj} \in \Qobj^\text{l-cupboard}, R_x(\qobj) = R_x(q'_\text{obj}), \\ 
                & \quad \text{and } R_y(\qobj) = R_y(q'_\text{obj}), \\ 
                0, & \text{otherwise.} 
                \end{cases}\)
                \item \(K_{\text{NP}_\text{r-cupboard}}.\phi(q_\text{obj}, q'_\text{obj}) \\
                \begin{cases} 
                1, & \text{if } q_\text{obj}, q'_\text{obj} \in \Qobj^\text{r-cupboard}, R_x(\qobj) = R_x(q'_\text{obj}), \\ 
                & \quad \text{and } R_y(\qobj) = R_y(q'_\text{obj}), \\ 
                0, & \text{otherwise.} 
                \end{cases}\)
                \item \(K_{\text{P}}.\phi(\qobj, q'_\text{obj}) \\
                \begin{cases} 
                1, & \text{if } q_\text{obj}, q'_\text{obj} \in \Qobj, \\
                &\quad \pi_\text{pre}(q_\text{obj}, q'_\text{obj}) \text{ is collision-free} \\
                & \quad \text{and not null} \\ 
                0, & \text{otherwise.} 
                \end{cases}\)
            \end{itemize}
        where $\Qobj^\text{sink} \subset \Qobj, \Qobj^\text{l-cupboard} \subset \Qobj,$ and $\Qobj^\text{r-cupboard} \subset \Qobj$ are the subspaces of $\Qobj$. $\Qobj^\text{sink}$ is the space that belongs to the sink, $\Qobj^\text{l-cupboard}$ is space that belongs to the left side of the cupboard, and $\Qobj^\text{r-cupboard}$ is the space that belongs the right side of the cupboard.
    \end{itemize}

\noindent\hrulefill
\end{definition}
\fi

%the skill library includes \(K_\text{P}\), which moves the card across regions (\(R^\text{up}\) and \(R^\text{down}\)), as well as two non-prehensile (NP) skills: \(K_{\text{NP}_\text{up}}\) and \(K_{\text{NP}_\text{down}}\), which manipulate the card within \(R^\text{up}\) and \(R^\text{down}\), respectively. In the bookshelf domain, the library consists of \(K_\text{P}\), which moves a book between the upper and lower shelves, \(K_{\text{NP}_\text{topple}}\), which topples a book on the upper shelf, and \(K_{\text{NP}_\text{push}}\), which pushes a book inward on the lower shelf. In the kitchen domain, the skill library includes \(K_\text{P}\), which moves a cup from the sink to the shelf, as well as \(K_{\text{NP}_\text{sink}}\), \(K_{\text{NP}_\text{leftshelf}}\), and \(K_{\text{NP}_\text{rightshelf}}\), which manipulate the cup within the sink, left shelf, and right shelf regions, respectively. The detailed skill descriptions and the training methods for the skills are outlined in Appendix.% ~\ref{Appendix:P_Skill} and~\ref{Appendix:NP_Skill}.


\subsection{Baselines}
We compare our method with the following baselines:

\begin{itemize}
    \item PPO \cite{schulman2017proximal}: An end-to-end RL policy that outputs actions directly, without utilizing a set of skills $\mathcal{K}$. The MDP definition and training hyperparameters are described in the Appendix~\ref{Appendix:baseline_PPO}.

    \item \texttt{MAPLE} \cite{nasiriany2022augmenting}: A state-of-the-art RL method for PAMDPs. The skills are used as parameterized actions. We use sparse rewards where we reward the robot when we achieve a goal, or when the high-level policy selects a feasible skill. To adapt the method to the PNP problem, we introduce several modifications. The details of these modifications, including the reward functions and hyperparameter configurations, are provided in Appendix~\ref{Appendix:baseline_MAPLE}.
    % Modifications to adapt to the PNP problem, the reward function and hyperparameters are provided in Appendix \ref{Appendix:baseline_MAPLE}.
        
    \item \texttt{Skill-RRT} (low-level action): Pure planner that executes a sequence of actions given by \texttt{Skill-RRT} in an open-loop manner.
    
    \item \texttt{Skill-RRT} (skill plan): Pure planner that executes the sequence of skill policies and their associate object poses as given by \texttt{Skill-RRT}. It executes the next skill policy in the plan if the current skill succeeds. It is semi-open-loop in that skills execute closed-loop policies but we do not change the high-level skill plan based on states.

    \item Ours: Diffusion policy trained with data from \skillrrt. Filtered data with replay success rate threshold $m=0.9$. The hyperparameters for diffusion policy is described in Appendix \ref{Appendix:imitation_learning}.
\end{itemize}
The summary of these baselines are shown in the first two columns of Table~\ref{table:main_exp}. 
% To adapt MAPLE to our PNP problem, we apply several modifications. First, we omit the affordance reward, which in MAPLE guides the high-level policy toward the desired manipulation region. Instead, we integrate the connector and skill, using a applicability checker $\applicabilitychecker$. The connector activates only feasible skills, thereby eliminating the need for an affordance reward. Additionally, we remove the explicit target location parameter, $x_\text{reach}$ used in MAPLE, computing it instead through the pre-contact policy $\pi_\text{pre}$. We also remove atomic primitives, low-level actions used to fill in gaps that cannot be fulfilled by skills, since our connectors are already trained to handle these gaps. Detailed description for training MAPLE can be found in Appendix~\ref{Appendix:baseline}.
% To adapt \texttt{MAPLE} to the PNP problem, we introduce several modifications. First, we remove the affordance reward, which in \texttt{MAPLE} guides the high-level policy toward the desired manipulation region. Instead, we integrate the connectors and skills, incorporating an applicability checker \(\applicabilitychecker\). The connector, which moves the robot to a state where the corresponding skill is applicable, is executed only when the predicted desired object pose is applicable (i.e., when the corresponding \(\applicabilitychecker\) holds true); otherwise, it is not executed. This replaces the need for an affordance reward.
% Furthermore, we replace the explicit initial end-effector position parameter, \(x_{\text{reach}}\), in \texttt{MAPLE} with the output of the pre-contact policy \(\pi_\text{pre}\). \(x_{\text{reach}}\) serves as an input to skills, specifying the initial end-effector position for skill execution. Instead, this position is now computed by the pre-contact policy, and the connector moves to the computed position before skill execution. 
% Additionally, we eliminate atomic primitives, low-level actions used to fill in gaps that cannot be fulfilled by skills, since our connectors are already trained to handle these gaps. A detailed description for training \texttt{MAPLE} can be found in Appendix~\ref{Appendix:baseline_MAPLE}.
For the PPO baseline, we use a total of 2.5B state-action pairs across all domains. For training the high-level policy in \texttt{MAPLE}, we use 0.27B, 0.33B, and 0.27B state-action pairs for the card flip, bookshelf, and kitchen domains, respectively. For our method, we collect 500 skill plans with a replay success rate threshold \( m = 0.9 \) in each domain, resulting in 0.0029B, 0.0028B, and 0.0032B state-action pairs for training our IL policy in card flip, bookshelf, and kitchen domain respectively.

% maple, ppo # of state-action pair
% ppo totally 2.5B state-action pair.
% MAPLE card flip 0.27B (low-level) 0.29M (high-level)
% MAPLE bookshelf 0.33B (low-level) 0.72M (high-level)
% MAPLE kitchen 0.27B (low-level) 0.29M (high-level)

%For \texttt{Skill-RRT} (low-level action), given an initial state and goal, we first store the low-level actions from \texttt{Skill-RRT}. Then, we re-execute the stored low-level actions starting from the same initial state and evaluate whether the object reaches the goal.  For \skillrrt (skill action), each skill in the skill plan is executed sequentially, moving to the next skill if the current skill satisfies its success condition.

% For diffusion policy distillation, we collect 500 skill plans with a replay success rate threshold \( m = 0.9 \), filtered from 13,900, 2900, and 4400 skill plan samples for the card flip, bookshelf, and kitchen domains, respectively. From the 500 skill plans over replay success rate 0.9, 2.9M, 2.8M, and 3.2M state-action pairs are used to train our IL policy for the card flip, bookshelf, and kitchen domains.

%For diffusion policy distillation, we collect 500 skill plans with a replay success rate threshold \( m = 0.9 \). This is after filtering out 13,400, 2,400, and 3,900 skill plans for the Card Flip, Bookshelf, and Kitchen domains, respectively, which failed to satisfy the replay success rate threshold $m=0.9$. 



% We evaluate the baselines using two metrics: (1) success rate, which is the number of successfully executed trajectories divided by the total number of problems solved, (2) Runtime, which measures the total time the algorithm takes to solve the problem. For planning-based algorithms, runtime includes both the planning and execution phases, while for other methods, it primarily corresponds to the execution phase.



\subsection{Result and Analysis}

In this paper, we make the following claims.

\begin{itemize}
    \item Claim 1: For long-horizon PNP problems with sparse rewards, leveraging a planner as opposed to pure RL achieves better success rates.
    % compare open-loop versus closed-loop
    \item Claim 2: Distilling a planner to a policy achieves better computation time and success rate than pure planning.
    %\item Claim 3: Combining planning (\texttt{Skill-RRT}) with imitation learning enhances execution efficiency and robustness by refining plans based on the current state, enabling faster and more successful skill execution.
    % split
\end{itemize}
To validate these claims, we evaluate the baselines using two metrics: (1) success rate, which is the number of successfully executed trajectories divided by the number of problem-solving attempts, and (2) computation time, which measures the total elapsed time for computing the entire sequence of actions. For \texttt{Skill-RRT}, the computation time includes both planning and inference times, as it involves using skill policies in addition to tree search. The planning and inference times of \texttt{Skill-RRT} are described in Appendix I.

% \begin{table*}[ht]
% \begin{adjustbox}{width=2\columnwidth} % Automatically fit within column width

% \centering

% \begin{tabular}{cccc|ccccccc}
%     \toprule
%     & \multicolumn{3}{c}{Components} & \multicolumn{6}{c}{Problem Domain} \\
%     \cmidrule(lr){2-4} \cmidrule(lr){5-10}
%     & \multirow{3}{*}{Method}
%     % \multirow{2}{*}{\makecell{RL \\ IL \\ Planning.}}
%     & \multirow{3}{*}{\makecell{Control \\ Strategies}} & \multirow{3}{*}{\makecell{Action \\ Type}} &
%     \multicolumn{2}{c}{Card Flip} & \multicolumn{2}{c}{Bookshelf} & \multicolumn{2}{c}{Kitchen}  \\
%     & & & &
%     \makecell{Success \\ rate (\%)} & \makecell{Run \\ time (s)} &\makecell{Success \\ rate (\%)} & \makecell{Run \\ time (s)} &\makecell{Success \\ rate (\%)} & \makecell{Run \\ time (s)}  \\
%     \cmidrule(lr){1-1} \cmidrule(lr){2-4} \cmidrule(lr){5-6}\cmidrule(lr){7-8}\cmidrule(lr){9-10}
%     PPO \cite{schulman2017proximal} & 
%     \textcolor{red}{RL} & 
%     \textcolor{blue}{Closed} & 
%     \textcolor{blue}{Low-level action} & 
%     {\makecell{$0.0\pm0.0$}} & 
% &
%     {\makecell{$0.0\pm0.0$}} &
%     &
%     {\makecell{$0.0\pm0.0$}} &
%     \\
%     \midrule
%     MAPLE (sparse) \cite{nasiriany2022augmenting} & 
%     \textcolor{red}{RL} & 
%     $\textcolor{blue}{Closed}$ & 
%     \textcolor{red}{Skill Parameter} & 
%     {\makecell{}} & 
% &
%     {\makecell{$94.9\pm0.7$}} &
%     &
%     {\makecell{}} &
%     \\
%     \midrule
% %     MAPLE (dense) \cite{nasiriany2022augmenting} & 
% %     \textcolor{red}{RL} & 
% %     $\textcolor{blue}{Closed}$ & 
% %     \textcolor{red}{Skill Parameter} & 
% %     {\makecell{}} & 
% % &
% %     {\makecell{$95.0\pm0.6$}} &
% %     &
% %     {\makecell{}} &
% %     \\
% %     \midrule
%     \makecell{\texttt{Skill-RRT} \\ (low-level action)} & 
%     \textcolor{blue}{Planning} & 
%     $\textcolor{red}{Open}$ & 
%     \textcolor{blue}{Low-level action} & 
%     0 & 
%     &
%     0 &
%     &
%     1 &
%     \\
%     \midrule
%     \makecell{\texttt{Skill-RRT} \\ (skill action)} & 
%     \textcolor{blue}{Planning} & 
%     $\textcolor{red}{Open}$ & 
%     \textcolor{red}{Skill Parameter} & 
%     55 & 
%     &
%     75 &
%     &
%     72 &
%     \\
%     \midrule
%     % \makecell{Skill-RRT \\ \& Guided TAMP \cite{mcdonald2022guided}} & 
%     % $\textcolor{blue}{Planning}$ & 
%     % $\textcolor{blue}{\cmark}$ & 
%     % \textcolor{red}{Skill Parameter} & 
%     % {\makecell{}} & 
%     % \makecell{157.55 \\ $\pm$ 46.13} & 
%     % {\makecell{}} &
%     % &
%     % {\makecell{}} &
%     % \\
%     % \midrule $K_{\text{P}}$
%     Ours (\texttt{Skill-RRT}+IL) & 
%     \textcolor{blue}{Planning + IL} & 
%     $\textcolor{blue}{Closed}$ & 
%     \textcolor{blue}{Low-level action} & 
%     {\makecell{95$\,\pm\,$0.5}} & 
%      & 
%     {\makecell{93$\,\pm\,$1.4}} &
%     &
%     {\makecell{98$\,\pm\,$0.5}} &
%     \\
%     \bottomrule
% \end{tabular}
% \end{adjustbox}

% \caption{Main Results: Comparison of various baselines based on their components (method, control strategies, and action type) and performance metrics (success rate and execution time) with their average and standard deviation across three different seeds for each problem domain: Card Flip, Bookshelf, and Kitchen.}
% \label{table:main_exp}
% \end{table*}

\begin{table*}[ht]
\vspace{-10mm}

\begin{adjustbox}{width=2\columnwidth} % Automatically fit within column width

\centering
\begin{tabular}{ccc|ccccccc}
    \toprule
    & \multicolumn{2}{c}{Components} & \multicolumn{6}{c}{Problem Domain} \\
    \cmidrule(lr){2-3} \cmidrule(lr){4-9}
    & \multirow{2}{*}{Method}
    & \multirow{3}{*}{\makecell{Action \\ Type}} &
    \multicolumn{2}{c}{Card Flip} & \multicolumn{2}{c}{Bookshelf} & \multicolumn{2}{c}{Kitchen}  \\
    & & &
    \makecell{Success \\ rate (\%)} & \makecell{Computation \\ time (s)} &\makecell{Success \\ rate (\%)} & \makecell{Computation \\ time (s)} &\makecell{Success \\ rate (\%)} & \makecell{Computation \\ time (s)}  \\
    \cmidrule(lr){1-1} \cmidrule(lr){2-3} \cmidrule(lr){4-5}\cmidrule(lr){6-7}\cmidrule(lr){8-9}
    PPO \cite{schulman2017proximal} & 
    \textcolor{red}{RL} & 
    \textcolor{blue}{Low-level action} & 
    {\makecell{$0.0\pm0.0$}} & N/A
    &
    {\makecell{$0.0\pm0.0$}} & N/A
    &
    {\makecell{$0.0\pm0.0$}} & N/A
    \\
    \midrule
    \texttt{MAPLE} \cite{nasiriany2022augmenting} & 
    \textcolor{red}{RL} & 
    \textcolor{red}{Skill $\&$ Parameter} & 
    {\makecell{$0.0\pm0.0$}} &
    N/A &
    {\makecell{$83.3\pm2.4$}} &
    $5.3\pm2.7$&
    {\makecell{$0.0\pm0.0$}} & 
    N/A
    \\
    \midrule
    % MAPLE (dense) \cite{nasiriany2022augmenting} & 
    % \textcolor{red}{RL} & 
    % \textcolor{red}{Skill Parameter} & 
    % {\makecell{}} & 
    % &
    % {\makecell{$95.0\pm0.6$}} &
    % &
    % {\makecell{}} &
    % \\
    % \midrule
    \makecell{\texttt{Skill-RRT} \\ (low-level action)} & 
    \textcolor{darkgreen}{Planning} & 
    \textcolor{blue}{Low-level action} & 
    $0.0 \pm 0.0$ & 
    N/A &
    $0.0 \pm 0.0$ &
    N/A &
    $0.0 \pm 0.0$ & N/A
    \\
    \midrule
    \makecell{\texttt{Skill-RRT} \\ (skill action)} & 
    \textcolor{darkgreen}{Planning} & 
    \textcolor{red}{Skill $\&$ Parameter} & 
    $39.0 \pm 0.0 $ & $85.3 \pm 48.7$
    & $66.0 \pm 0.0 $
    & $79.2 \pm 67.1$
    & $64.0 \pm 0.0 $
    & $121 \pm 39.5$
    \\
    \midrule
    % \makecell{Skill-RRT \\ \& Guided TAMP \cite{mcdonald2022guided}} & 
    % \textcolor{blue}{Planning} & 
    % \textcolor{red}{Skill Parameter} & 
    % {\makecell{}} & 
    % &
    % {\makecell{}} &
    % &
    % {\makecell{}} &
    % \\
    % \midrule $K_{\text{P}}$
    Ours (\texttt{Skill-RRT}+IL) & 
    \textcolor{blue}{Planner distilled via IL} & 
    \textcolor{blue}{Low-level action} & 
    {\makecell{$95.0 \pm 0.5$}} & 
    \makecell{$2.68 \pm 0.61$}
    &
    {\makecell{$93.0 \pm 1.4$}} &
    \makecell{$2.93 \pm 2.05$}
    &
    {\makecell{$98.0 \pm 0.5$}} &
    \makecell{$3.02 \pm 0.65$}
    \\
    \bottomrule
\end{tabular}
\end{adjustbox}

\caption{Comparison of baselines based on their components (method and action type) and performance metrics (success rate and Computation time) with their average and standard deviation across three different seeds for each problem domain. The success rate is measured on a set of 100 test problems. Computation time refers to the average elapsed time required to solve the given 100 problems when the method successfully completes them.}
\label{table:main_exp}
\end{table*}


% To support Claim 1, we compare our framework with MAPLE \cite{nasiriany2022augmenting}. As shown in Table \ref{table:main_exp}, MAPLE achieves a zero success rate in the card and kitchen domains. This outcome is caused by the difficulty of sampling a target object pose that is applicable for the next skill under sparse rewards. For example, in the card domain, the probability of MAPLE outputting the desired card pose at the end of the table through random actions is low, reducing the applicability of the P skill. In contrast, in the bookshelf domain, the P skill becomes applicable once the book is toppled, making skill chaining more easy.
% To support Claim 1, we compare our framework with MAPLE~\cite{nasiriany2022augmenting} and PPO. As shown in Table \ref{table:main_exp}, MAPLE achieves a success rate of zero in both the card flip and kitchen domains. This poor performance is mainly due to the challenge of predicting an object target pose that enables the successful execution of subsequent skills through random exploration. For instance, in the card domain, task success depends on releasing the gripper and moving it to the card without falling the card after completing the prehensile skill. However, the region where the subsequent non-prehensile skill can successfully execute is narrow, making it challenging for RL-based methods, which rely on random exploration, to find a viable solution without dense reward signals to guide the search.

\newcommand{\maple}{\texttt{MAPLE}}

Table~\ref{table:main_exp} shows the results.
To support Claim 1, we first compare our method with PPO~\cite{schulman2017proximal} and \texttt{MAPLE}~\cite{nasiriany2022augmenting}. PPO exhibits poor performance across all domains. Because this is a flat RL policy, it struggles with long-horizon problems that have sparse rewards, a perennial problem in RL. \maple{} also shows zero success rate in card flip and kitchen domains. While \maple{} is a PAMDP-based hierarchical RL method, it struggles in these domains due to the narrow passage problem~\cite{hsu1998finding}. For instance, in the card flip domain, the card can only be flipped near the edges of the table, a very small region in $\Qobj$. Additionally, even at the edge, to prevent the card from dropping during the transition from sliding to the prehensile skill, it must be positioned in a precise manner: it should be graspable yet remain sufficiently close to the table to avoid falling during the connection (See Appendix Figure~\ref{fig:card_y_position_histograms} for visual illustration).  Similarly, in the kitchen domain, the robot must position the cup such that the prehensile skill can be applied following the non-prehensile skill in the sink. But the cup poses for which you can apply prehensile skill is very narrow: it must be collision-free, and an inverse kinematics solution should exist for picking. For these reasons, most object subgoal pose exploration leads to failure for \maple{}, which ends up learning only locally optimal behaviors, such as bringing the card to the edge of the table but not executing the prehensile policy as it would often lead to dropping the card. For the bookshelf domain, where the robot simply has to topple the book to enable grasping, \maple{} shows 83\% success rate.


Pure planner, \skillrrt~(low-level action) achieves zero success rates across all domains, as its open-loop nature cannot respond to state transitions that are different from the plan. \skillrrt~(skill action) does better, because it reacts to unexpected transitions within skills. However, it tends to use risky desired object poses, such as sliding the card to the very edge, which, even with a slight difference in actions or state transitions, risks dropping the card.


In contrast to these approaches, our method achieves success rates above 93\% across all domains, thanks to the sophisticated exploration strategy in RRT combined with a data filtering mechanism that filters plans that involve risky object subgoals. Furthermore, because ours is a policy, we can achieve full closed-loop control, which can react to unexpected state transitions.




%In both domains, the range in which a successful transition between skills can occur is extremely narrow. In RL, if a skill with a specific desired object pose exhibits a high failure rate, the agent often learns to suppress its selection due to the reward-driven nature of policy optimization. This tendency can result in convergence to suboptimal policies and a fall into local minima, even though the skill with the desired object pose is essential for task completion, such as the sliding skill with the object pose near the edge. 


%. Specifically, to ensure task completion, the reward function would need to be designed in a way that sacrifices distance. As a result, PPO struggles to achieve high success rates in these domains.

%s presented in Table~\ref{table:main_exp}, 


% In contrast, our framework uses \skillrrt~to systematically explores the leading to higher success rates even in challenging scenarios where RL struggles due to issues with random exploration and credit assignment. Additionally, the bookshelf domain exhibits a broader region in which the subsequent skill could succeed. Specifically, the prehensile skill becomes feasible to execute if the topple skill is successful, enabling a higher success rate with the PAMDP method in this domain.

% In MAPLE, if a skill frequently fails, the agent often learns to avoid selecting that skill, leading to frequent cases of getting stuck in local minima, even though the skill is necessary for task success. In contrast, planning (\texttt{Skill-RRT}) continuously executes the failed skill along with a new desired object pose, even after failure. When successful, the new state is stored as a new node, and the search continues toward the goal. As a result, planning avoids local minima and maintains continuous exploration.
% bookshelf -> we can generate the high success rate plan 
% It is hard to sample target object pose that is feasible for the next skill with random exploration under sparse reward setting. (except bookshelf) 
% need shield -> bookshelf domain's characteristic
% 

% In RL, if a skill with a specific desired object pose exhibits a high failure rate, the agent often learns to suppress its selection due to the reward-driven nature of policy optimization. This tendency can result in convergence to suboptimal policies and a fall into local minima, even though the skill with the desired object pose is essential for task completion, such as the sliding skill with the object pose near the edge. 

%Exceptionally, the bookshelf domain allows for a wider range of successful skill transitions. Specifically, if the topple skill is performed successfully, the prehensile skill becomes applicable, leading to a higher success rate with the PAMDP method in this domain.

%To support claim 2, we compare our distillation policy with \texttt{Skill-RRT} (low-level actions). As shown in Table \ref{table:main_exp}, this approach achieves zero success rates across all domains, as its open-loop nature cannot respond to state transitions that are different from the plan. \skillrrt (skill action) does better, because it reacts to unexpected transitions within skills. However, it tends to use risky desired object poses, such as sliding the card to the very edge, which, even with a slight difference in actions or state transitions, risk dropping the card. In contrast, our method (\texttt{Skill-RRT} + IL) filters out skill plans with low replay success rates by replaying them multiple times and trains the distillation policy on trajectories generated from such robust skill plans. As a result, our method demonstrates higher success rates compared to simply executing skill plans.


% The poor performance is due to the contact-rich nature of our PNP tasks and simulation stochasticity of GPU-based simulation, which leads to increased dynamic uncertainty. Executing actions in an open-loop manner cannot actively respond to stochastic state transitions because it lacks state feedback, whereas our distillation policy can adapt its action to the current state immediately.

%To support Claim 3, we compare our method with \skillrrt (skill action). However, this method still fails to achieve high success rates. This is because there are risky target object poses where the skills do not succeed 100\%, even though these target poses succeed during the planning stage. For example, pushing a card to the very edge of a table might succeed during planning but is likely to result in the card falling off in execution. In contrast, our method (\texttt{Skill-RRT} + IL) filters out skill plans with low replay success rates by replaying them multiple times and trains the distillation policy on trajectories generated from such robust skill plans. As a result, our method demonstrates higher success rates compared to simply executing skill plans.

%Another benefit of distillation is that the distillation policy can solve tasks much faster than planning because it requires significantly less online computation time compared to planning. 

To validate our argument about computation time, we compare the total elapsed time for computing the sequence of actions between our method and \skillrrt{}. Note that only successful episodes are measured, as failure episodes, such as those involving object falling or timeout, result in extremely short or extremely long computation time. The results shown in Table \ref{table:main_exp} demonstrate that our distillation policy significantly reduces the computation time compared to a pure planner, \texttt{Skill-RRT}, as it eliminates expensive search procedure.

% Action Type: By executing the skill plan, if it fails, we cannot revise the skill with the current setting (as there is no function $\bets$ for termination the skill). Therefore, state-action imitation provides a better approach to address the problem.

% \textcolor{red}{result analysis, not yet}

\subsection{Real World Experiments}
We evaluate our distilled diffusion policy in real-world domains shown in Figure~\ref{fig:CPNP_tasks} by zero-shot transferring the policy. For each experiment, we use a set of 20 test problems consisting of different initial object poses, initial robot configurations, and goal object poses. The shapes of the real environments and objects are identical to those in the simulation. The policy achieves a success rate of over 80\% for all domains, as shown in Table \ref{table:real_world_result}. The main failure cases across domains include torque limit violations of the robot hardware caused by impacts between the robot and the object, and object dropping due to unstable object placement poses. Detailed setups for the real-world experiments are described in Appendix~\ref{Appendix:real_exp}.

\begin{table}[ht]
\centering
\setlength\tabcolsep{10 pt}
\begin{adjustbox}{max width=\columnwidth}
\begin{tabular}{cccc}
\toprule
           & \textbf{Card Flip} & \textbf{Bookshelf} & \textbf{Kitchen} \\
\midrule
\makecell{Success rate \\ (Success/Trials)}  &  85\% (17/20)   & 90\% (18/20)   &  80\% (16/20) \\
\bottomrule
\end{tabular}
\end{adjustbox}
\caption{Real-world experimental results on three domains.}
\label{table:real_world_result}
\vspace{-3mm}
\end{table}

\subsection{Ablations of distillation policy architecture with its data filtering method} 

As described in Section~\ref{method:IL}, we train the distillation policy using the Diffusion Policy~\cite{chi2023diffusion} through imitation learning. For data collection, we use state-action pairs gathered by replaying skill plans whose success rate exceeds a threshold of \( m = 0.9 \). To understand the contribution of our choice of distillation policy architecture, IL algorithm, and and the data filtering method, we conduct ablation studies. The ablation study of the distillation policy architecture can be found in Appendix~\ref{Appendix:IL_ablation}, while the ablation study of the filtering method is presented in Appendix~\ref{Appendix:data_qual}.



% \section{Ablation}\label{sec:ablation}
% % \textbf{Evaluate Connector in Bridging the Gap Between Effect and Preconditions}
% % purpose of this section
% We examine the effectiveness of the connector in bridging the gap between the effect of one skill and the precondition of the next. To evaluate its performance, we compared our trained connector policies against a motion planner \cite{sucan2012the-open-motion-planning-library}, which generates a sequence of long-horizon robot joint actions to reposition the robot and satisfy the precondition of the subsequent skill.

% % How did we implement connector and motion planner
% We replace the connector policy with the Open Motion Planning Library (OMPL) \cite{sucan2012the-open-motion-planning-library} and collect test problems similarly for connector training. Each problem consists of the effect of the skill, the skill's terminated state \( s_{k-1} \), the subsequent skill \( o_k \), and its subgoal \( (q^\text{obj}_\text{sg})_k \). The target robot configuration, \( q^\text{robot}_\text{pre} \), is determined using the subsequent skill's pre-contact robot configuration sampler \( f^\text{pre}_{k} \), the terminated state's object pose \( s_{k-1}.q^\text{obj} \), and \( (q^\text{obj}_\text{sg})_k \).

% The connector computes low-level actions based on the current state \( s \) and the target robot configuration \( q^\text{robot}_\text{pre} \). For the motion planner, a trajectory of robot joint positions is computed from the current robot joint positions \( s_{k-1}.q^\text{robot} \) to \( q^\text{robot}_\text{pre} \). If the motion planning computation is successful, the robot executes the trajectory by following the sequence of joint positions generated by the motion planner.

% % The measurement, result and analysis
% We evaluate the effectiveness of each method using three metrics: 1) execution time to assess practicability with skill-RRT, 2) the rate of satisfying the subsequent skill's precondition, and 3) the success rate of the subsequent skill in manipulating the object to \( q^\text{obj}_\text{sg} \), as shown in Figure.

% \begin{figure}[ht!] % Adjust to fit within a single column
% \centering
% % \vspace{-10mm}
% \includegraphics[width=\columnwidth]{figures/results/Ablation_subsequent_success_rate.png}
% \caption{Ablation Result:} % Adjust caption text if needed
% \label{fig:ablation_subsequent_success} % Place the label after the caption
% \end{figure}


% % \renewcommand{\arraystretch}{1.5}
\begin{table*}[ht]
\begin{adjustbox}{width=2\columnwidth} % Automatically fit within column width

\centering

\begin{tabular}{cccccccccc}
    \toprule
    \multirow{2}{*}{Method} & \multirow{2}{*}{Metric} & \multicolumn{2}{c}{Card Flip} & \multicolumn{3}{c}{Bookshelf} & \multicolumn{3}{c}{Kitchen} \\
    \cmidrule(lr){3-4} \cmidrule(lr){5-7} \cmidrule(lr){8-10}
    & & {Non-Prehensile} & Place & Topple & Push & Place & NP in sink & Push in Shelf & Place \\
    \cmidrule(lr){1-1}\cmidrule(lr){2-2}\cmidrule(lr){3-4} \cmidrule(lr){5-7} \cmidrule(lr){8-10}
    
     & \makecell{Execution Time (s)}
     & 
     & 
     & 
     & 
     &
    &
     &

    \\
    \cdashline{2-10}
    {Motion Planner \cite{sucan2012the-open-motion-planning-library}} & \makecell{Precondition rate}
     & 
     & 
     & 
     & 
     &
    &
     &
    \\
    \cdashline{2-10}
     & \makecell{Skill success}
     & 
     & 
     & 
     & 
     &
    &
     &
    \\
    \midrule

     & \makecell{Execution Time (s)}
     & 
     & 
     & 
     & 
     &
    &
     &
    \\
    \cdashline{2-10}
    {\makecell{Connector}} & \makecell{Precondition rate}
& 
     & 
     & 
     & 
     &
    &
     &
    \\
    \cdashline{2-10}
     & \makecell{Skill success}
     & 
     & 
     & 
     & 
     &
    &
     &
    \\
    \bottomrule
\end{tabular}
\end{adjustbox}

\caption{ablation connector versus motion planner.}\label{table:ablation_connector}
\begin{flushleft}
\footnotesize
\textbf{Table \ref{table:ablation_connector}} — This table evaluates each method using three metrics: 1) execution time for practicability with skill-RRT, 2) precondition rate for satisfying the subsequent skill's precondition, and 3) skill success for effectively connecting skills to achieve \( q^\text{obj}_\text{sg} \).
\end{flushleft}
\end{table*}

% \textcolor{red}{result and analysis needed}
% % However, motion planners primarily focus on the robot's movement, often neglecting the object's motion during the connector phase.

% textbf{Effect of Replay Success Rate Threshold ($m$)}
\subsection{Impact of our data filtering mechanism}

% what we want to ablate
%High-quality demonstrations are essential for imitation learning (IL) \cite{mandlekar2022matters}, as low-quality data can lead to high-risk states. We collect state-action trajectories for training a distillation policy by replaying skill plans $\tau_{\text{skill}}$ that exceed a replay success rate threshold $m$ during the replay process. We investigate the performance of the distillation policy by training with data collected from state-action trajectories with different replay success rates threshold $m$.

% How to set other baselines
We evaluate the impact of $m$ and the use of the data filtering mechanism. Figure \ref{fig:ablation_filtering_success} shows the result. Baseline Without Replay uses 15,000 skill plans to train a diffusion policy, without filtering, with a trajectory per skill plan. On the other hand, Replay $(m=0.1)$ and Replay $(m=0.9)$ filter out skill plans that do not meet their respective replay success rate threshold $m$. A skill plan's replay success rate is measured by
\[
\frac{\text{Number of successful replays of }\skillplan: N_{success}}{\text{Total number of replays of }\skillplan: N}
\]
as mentioned in Section~\ref{method:IL}. To measure the success rate, we replay each skill plan \( N = 400 \) times and collect the successful trajectories for each skill plan. If a skill plan's \( N_{success} \) exceeds 40, it is included in the Replay with \( m = 0.1 \). If a skill plan's \( N_{success} \) exceeds 360, it is included in the Replay with \( m = 0.9 \). Both methods collect 500 skill plans. From these, we randomly select 30 trajectories from the successful executions for each skill plan. While all successful trajectories can be used, each method collects a total of 15,000 trajectories to ensure a fair comparison when training the distillation policy. Across all domains, we see that $m=0.9$ with the filtering mechanism achieves the highest success rate, by preferring desired object poses that prevents risky states, where object might drop. Detailed characteristics of skill plans under each filtering method, including the distribution of desired object poses, are provided in Appendix~\ref{Appendix:data_qual}.

% Replay $(m=0.1)$ uses data filtering with $(m=0.1)$ on 500 skill plans, and collect 30 successful trajectories per each skill plan. Replay $(m=0.9)$ uses the same setup, but uses a high success rate threshold $(m=0.9)$. All methods collect 15,000 trajectories in total to train the distillation policy, ensuring a fair comparison. Across all domains, we see that $m=0.9$ with the filtering mechanism achieves the highest success rate, by preferring subgoal object poses that prevents risky states, where object might drop. Detailed characteristics of skill plans under each filtering method, including the distribution of desired object poses, are provided in Appendix~\ref{Appendix:data_qual}.

%We evaluate the effectiveness of each method using the distillation policy's success rate in simulation as shown in . %\ref{}.

\begin{figure}[ht!] % Adjust to fit within a single column
\centering
% \vspace{-10mm}
\includegraphics[width=\columnwidth]{figures/results/success_rates_across_environments.png}
\caption{Success rate for each data filtering method.}

%We provide a more comprehensive presented in Appendix~\ref{Appendix:IL_ablation}}%Table \ref{table:ablation_dataQ}.} % Adjust caption text if needed
\label{fig:ablation_filtering_success} % Place the label after the caption
\end{figure}

%Our filtering rule ensures data quality by restricting skill plans to those with a high replay success rate (Replay \(m=0.9\)). This approach consistently outperforms replaying low-success-rate plans (Replay \(m=0.1\)) and collecting data without replay (Without Replay), which includes plans with lower success rates and leads to decreased performance. While collecting 15,000 plans without replay improves generalization (initial and goal object poses generalization), the lack of filtering low-success-rate plans results in inferior performance. High replay success rate plans are characterized by intermediate desired object poses within skills that minimize risky states. For instance, in the card flip domain, NP skill target poses positioned slightly inward from the table's edge facilitate grasping and maintain stability, whereas lower success rate plans place the card too close to the edge, increasing the risk of falling. Detailed characteristics of skill plans under each filtering method, including the distribution of desired object poses, are provided in Appendix~\ref{Appendix:data_qual}.

\subsection{Design choices for IL}

In this section, we study how the design choices of imitation learning (IL) affect the performance of the distillation policy in simulation. We compare the diffusion policy \cite{chi2023diffusion} with the following architectures: 1) ResNet \cite{he2016deep}, a simple IL model with large parameters; 2) LSTM+GMM, which has been shown to be effective for multimodal data in RoboMimic \cite{robomimic2021}; 3) cVAE \cite{kingma2013auto}, another conditional generative model; and 4) Transformer \cite{vaswani2017attention}, a widely used architecture for multimodal data such as language. Each architecture is trained on the same dataset, with the similar model parameter size ($\approx$70M) and the same training duration of 100 epochs, across three different seeds. Detailed training settings such as hyperparameters are described in Appendix~\ref{Appendix:IL_ablation}. 
The success rate is measured at every training epoch on 100 fixed test problems and we report the maximum success rates achieved during the training epochs. The success rates for each architecture are summarized in Table \ref{table:IL_arch_ablation}. 

In the card flip domain, the diffusion policy significantly outperforms the other architectures. We hypothesize that this is because of the high diversity of intermediate target poses in this domain—for instance, an object pose can be at either the left or right edge of the table. In the bookshelf and kitchen domains, ResNet, LSTM+GMM, and Transformer achieve success rates similar to the diffusion policy, which might because the level of intermediate object poses diversity is lower than in the card flip domain. cVAE shows poor performance across all domains, indicating that the smoothing effect of VAE negatively impacts imitation learning.

\begin{table}[h]
    \centering
    % \setlength\extrarowheight{-1pt}
    \setlength\tabcolsep{12 pt}
    \begin{tabular}{r|ccc}
    \toprule
                        & Card Flip & Bookshelf & Kitchen \\
    \midrule
    ResNet \cite{he2016deep}                    & \small 54$\, \pm \,$5.7  & \small 85$\, \pm \,$2.6  & \small 98$\, \pm \,$0.5 \\ 
    LSTM+GMM \cite{robomimic2021}               & \small 55$\, \pm \,$5.7  & \small 95$\, \pm \,$1.7  & \small \textbf{99$\, \pm \,$0.5} \\ 
    cVAE \cite{kingma2013auto}                  & \small 22$\, \pm \,$2.1  & \small 31$\, \pm \,$3.3  & \small 41$\, \pm \,$4.5 \\ 
    Tranformer    \cite{vaswani2017attention}   & \small 63$\, \pm \,$1.4  & \small \textbf{96$\, \pm \,$1.4}  & \small 98$\, \pm \,$0.8 \\ 
    Diffusion Policy  \cite{chi2023diffusion}   & \small \textbf{95$\, \pm \,$0.5}   & \small 93$\, \pm \,$1.4  & \small 98$\, \pm \,$0.5 \\ 
    \bottomrule
    \end{tabular}
    \caption{Success rates averaged over 3 seeds in simulation for each imitation learning design.}
    \label{table:IL_arch_ablation}
\end{table}


\section{Conclusion}\label{sec:conclusion}
\section{Conclusion}
We present live monitoring and mid-run interventions for multi-agent systems. We demonstrate that monitors based on simple statistical measures can effectively predict future agent failures, and these failures can be prevented by restarting the communication channel. Experiments across multiple environments and models show consistent gains of up to 17.4\%-20\% in system performance, with an addition in inference-time compute.
Our work also introduces \ourenv{}, a new environment for studying multi-agent cooperation.


\section{Limitations}
% 1. data collection time expensive and require exhaustive computational time (although it automatically proceed), if we have more skills, then it might take more time.
% 2.  single vs multiple object, single environment  (domain-specific) (object, domain generalization)
% 3. fixed object & shape, fixed environment
% 3-a. pre-deinfed relative grasp pose.
% 3-b. predefined environment region with SE(2)
% 4. non-region based skills cannot be usable (e.g., unrealted to object pose)
% 5. 

There are several limitations to our work. First, our work generalizes across different initial and goal object poses, but not across object shapes. Second, it assumes a fixed environment, and cannot generalize across different environment shapes. Third, it takes a significant amount of data collection time, as we need to solve many planning problems. Further, as the number of skills increases, it would take longer as we would need to simulate more skills. Fourth, in our algorithm \lazyskillrrt, we create a connecting node $\connectingnode$ assuming that the object configuration remains the same as in the nearest node $v$. This makes it impossible to use with skills that end with high object velocity such as throwing or batting.

%while our framework handles a fixed object, it cannot be directly applied to different objects within the same environment, limiting its generalizability. Additionally, the skills in our approach are defined for a single object, making the method insufficient for environments containing multiple objects, where interactions between objects must be considered. Another limitation lies in the time-intensive nature of data collection, which requires significant computational resources despite being automated. As the number of skills increases, the data collection process scales proportionally, potentially rendering it impractical for large-scale applications. Addressing these challenges in future work will help extend the applicability of our approach to more complex and diverse scenarios.


% \section*{Acknowledgments}

%% Use plainnat to work nicely with natbib. 

% \newpage
\bibliographystyle{plainnat}
\bibliography{references}

\onecolumn
\newpage
\appendix
% \section{Proposed model}
% \label{section:app:model}
% Table \ref{table:define} lists the symbols and their definitions used in this paper. \par
% % \vspace{-1.5em}
% % \TSK{
% % \begin{table}[t]
\vspace{-0.5em}
\centering
\small
% \footnotesize
\caption{Symbols and definitions.}
\label{table:define}
\vspace{-1.2em}
\begin{tabular}{l|l}
\toprule
Symbol & Definition \\
\midrule
$d$ & Number of dimensions \\
$t_c$ & Current time point \\
% $N$ & Current window length \\
$\mX$ & Co-evolving multivariate data stream (semi-infinite) \\
$\mX^c$ & Current window, i.e., $\mX^c = \mX[t_m:t_c]\in\R^{d\times N}$ \\
\midrule
$h$ & Embedding dimension \\
% $\mH$ & Hankel matrix, i.e., $\begin{bmatrix}
%             \embed{\vx_1} & \embed{\vx_2} & \cdots & \embed{\vx_{n-h+1}}
%         \end{bmatrix}$ \\
$\embed{\cdot}$ & Observable for time-delay embedding, i.e., $g\colon\R \rightarrow \R^{h}$ \\
% $\mH$ & Hankel matrix of $\mX$, i.e., $\mH = [\embed{\vx_1}~\embed{\vx_2} ~ ... ~ \embed{\vx_{n-h+1}}]$ \\
$\mH$ & Hankel matrix \\
$\nmodes$ & Number of modes \\
% $\imode$ & Modes of the system for $i$-th dimension of $\mX$, i.e., $\imode \in \R^{h \times r_i}$ \\
$\modes$ & Modes of the system, i.e., $\modes \in \R^{h\times\nmodes}$ \\
% $\ieig$ & Eigenvalues of the system for $i$-th dimension of $\mX$, $i.e., \ieig \in \R^{r_i \times r_i}$ \\
$\eigs$ & Eigenvalues of the system, i.e., $\eigs \in \R^{\nmodes\times\nmodes}$ \\
$\demixing$ & Demixing matrix, i.e., $\mW = [\rowvect{w}_1, ..., \rowvect{w}_d]^\top \in \R^{d \times d}$ \\
$\mB$ & Causal adjacency matrix, i.e., $\mB \in \R^{d \times d}$ \\
\midrule
% $\vs(t)$ & Latent variables at time point $t$, i.e., $\vs(t) = \{ \vs_1(t), ..., \vs_d(t) \} $ \\
$\ind(t)$ & Inherent signal at time point $t$, i.e., $\ind(t) = \{ \ith{e}(t) \}_{i=1}^d$ \\
$\mat{S}(t)$ & Latent vectors at time point $t$, i.e., $\mat{S}(t) = \{ 
\ith{\vs}(t) \}_{i=1}^d$ \\
$\vvec(t)$ & Estimated vector at time point $t$, i.e., $\vvec(t) = \{ \ith{v}(t) \}_{i=1}^d$ \\
\midrule
$\mathcal{D}$ & Self-dynamics factor set, i.e., $\mathcal{D} = \{\modes, \eigs\}$\\
$\regime$ & Regime parameter set, i.e., $\regime = \{ \mW, \mathcal{D}_{(1)}, ..., \mathcal{D}_{(d)} \}$\\
\midrule
$R$ & Number of regimes \\
$\regimeset$ & Regime set, i.e., $\regimeset 
 = \{ \regime^1, ..., \regime^R \}$\\
 $\mathcal{B}$ & \Relation, i.e., $\mathcal{B} = \{\mB^1, ..., \mB^R\}$\\
$\updateset$ & Update parameter set,  i.e., $\updateset 
 = \{ \update^1, ..., \update^R \}$ \\
\midrule
 $\modelparam$ & Full parameter set,  i.e., $\modelparam 
 = \{ \regimeset, \updateset \}$ \\

\bottomrule
% \midrule
\end{tabular}
\normalsize
% \vspace{-2.0em}
\vspace{1.0em}
\end{table}

% % }
% \vspace{0.6em}

\section{Optimization Algorithm}
% Algorithm \ref{alg:model} is the overall procedure of \method. Algorithm \ref{alg:estimator}, namely, \modelestimator continuously updates the full parameter set $\modelparam$ and the model candidate $\candparam$, which describes the current window $\mX^c$.
\TSK{
\begin{figure}[!h]
\vspace{-5.0ex}
\begin{algorithm}[H]
    \normalsize
    \caption{\method($\vx(t_c), \modelparam, \candparam$)}
    \label{alg:model}
    \begin{algorithmic}[1]
        \STATE {\bf Input:}
        \hspace{0mm}    (a) New value $\vx(t_c)$ at time point $t_c$ \\
        \hspace{9.5mm} (b) Full parameter set $\modelparam = \{\regimeset, \updateset\}$ \\
        \hspace{9.68mm} (c) Model candidate $\candparam = \{\regime^c, \update^c, \bm{s}^c_{en}\}$
        \STATE {\bf Output:}
        \hspace{0mm}    (a) Updated full parameter set $\modelparam'$ \\
        \hspace{11.8mm} (b) Updated model candidate $\candparam'$ \\
        \hspace{11.9mm} (c) $l_s$-steps-ahead future value $\vect{v}(t_c+l_s)$ \\
        \hspace{11.8mm} (d) Causal adjacency matrix $\mB$
        \STATE /* Update current window $\mX^c$ */
        \STATE $\mX^c \leftarrow \mX[t_m : t_c]$
        \STATE /* Estimate optimal regime $\regime$ */
        \STATE $\{\modelparam', \candparam'\} \leftarrow$ \modelestimator($\mX^c, \modelparam$, $\candparam$)
        \STATE /* Forecast future value and discover causal relationship */
        \STATE $\{\vect{v}(t_c+l_s),~\mB\} \leftarrow$ \modelgenerator($\candparam'$)
        \STATE /* Update regime $\regime$ */
        \IF{NOT create new regime}
            \STATE $\candparam' \leftarrow \regimeupdate(\mX^c, \candparam')$
        \ENDIF
    \RETURN $\{\modelparam', \candparam', \vect{v}(t_c+l_s), \mB\}$
    \end{algorithmic}
\end{algorithm}
\vspace{-4.5em}
\end{figure}
}\par
\TSK{
\begin{figure}[!h]
\vspace{-0.0ex}
\begin{algorithm}[H]
    \normalsize
    \caption{\modelestimator($\mX^c, \modelparam, \candparam$)}
    \label{alg:estimator}
    \begin{algorithmic}[1]
        \STATE {\bf Input:}
        \hspace{0.0mm}  (a) Current window $\mX^c$ \\
        \hspace{9.5mm} (b) Full parameter set $\modelparam$ \\
        \hspace{9.68mm} (c) Model candidate $\candparam$
        \STATE {\bf Output:}
        \hspace{0.0mm}  (a) Updated full parameter set $\modelparam'$ \\
        \hspace{11.8mm} (b) Updated model candidate $\candparam'$
        \STATE /* Calculate optimal initial conditions */
        \STATE $\mat{S}_{0}^c \leftarrow \argmin_{\mat{S}_{0}^c} f(\mX^c; \mat{S}_{0}^c, \regime^c)$
        \IF{$f(\mX^c; \mat{S}_{0}^c, \regime^c) > \tau$}
            \STATE /* Find better regime in $\bm{\Theta}$ */
            \STATE $\{ \mat{S}_{0}^c, \regime^c \} \leftarrow \argmin_{\mat{S}_{0}^c, \regime \in \regimeset} \,f(\mX^c; \mat{S}_{0}^c, \regime^c)$
            \IF{$f(\mX^c; \mat{S}_{0}^c, \regime^c) > \tau$}
                \STATE /* Create new regime */
                \STATE $\{ \regime^c, \update^c \} \leftarrow \textsc{RegimeCreation}(\mX^c)$
                \STATE $\regimeset \leftarrow \regimeset \cup \regime^c$; $\updateset \leftarrow \updateset \cup \update^c$
            \ENDIF
        \ENDIF
        \STATE $\modelparam' \leftarrow \{\regimeset, \updateset\}$; $\candparam' \leftarrow \{\regime^c, \update^c, \mat{S}_{en}^c\}$
        \RETURN $\modelparam', \candparam'$
    \end{algorithmic}
\end{algorithm}
\vspace{-3.3em}
\end{figure}
}
% Algorithm \ref{alg:model} shows the overall procedure for \method,
% including \modelestimator (Algorithm \ref{alg:estimator}).
% \modelestimator continuously updates the full parameter set $\modelparam$ and
% the model candidate $\candparam$, which describes the current window $\mX^c$.
% \par
\label{section:app:algorithm}
% \myparaitemize{Details in Eq. \eqref{eq:update_trans}}
\subsection{Details of Eq. (\ref{eq:update_trans})}
% \myparaitemize{Details of Eq. (\ref{eq:update_trans})}
Here, we introduce the recurrence relation of transition matrix $\ith{\trans}$.
As mentioned earlier, we use the following cost function (below, index $i$ denoting $i$-th dimension is omitted for the sake of simplicity, e.g., we write $\ith{\trans}$ as $\trans$):
\begin{align*}
    \mathcal{E} &= \sum_{t'=t_m+h}^{t_c}\forgetting^{t_c-t'}||\embed{e(t')} - \trans\embed{e(t'-1)}||_2^2 \\
    &= \sum_{l=1}^h (\mat{L}(l, :) - \trans(l, :)\mat{R})\Forgetting(\mat{L}(l, :) - \trans(l, :)\mat{R})^\top
\end{align*}
where,
% $\Forgetting = diag(\forgetting^{N-2}, ..., \forgetting^0) \in \R^{(N-1) \times (N-1)}$ and
$\Forgetting, \mat{L}$ and $\mat{R}$ are synonymous with the definition in Section \ref{section:alg:creation}.
Because we want to obtain $\trans$ that minimizes this cost function $\mathcal{E}$, we differentiate it with respect to $\trans$.
\begin{align*}
    \dfrac{\partial}{\partial\trans(l, :)}\mathcal{E}
    &= -2(\mat{L}(l, :) - \trans(l, :)\mat{R}) \Forgetting \mat{R}^\top
\end{align*}
Solving the equation $\partial\mathcal{E}/\partial\trans(l, :) = 0$ for each $l$, $1 \leq l \leq h$,
the optimal solution for $\trans$ is given by
% the optimal solution is $ \trans = (\mat{L}\Forgetting\mat{R}^\top)(\mat{R}\Forgetting\mat{R}^\top)^{-1} $
$$ \trans = (\mat{L}\Forgetting\mat{R}^\top)(\mat{R}\Forgetting\mat{R}^\top)^{-1} $$
where we define
\begin{align*}
    \mat{Q} = \mat{L}\Forgetting\mat{R}^\top,\quad
    \mat{P} = (\mat{R}\Forgetting\mat{R}^\top)^{-1}
\end{align*}
The recurrence relations of $\mat{Q}$ can be written as
\begin{align*}
    \mat{Q} &= \sum_{t'=t_m+h}^{t_c}\forgetting^{t_c-t'}\embed{e(t')}\embed{e(t'-1)}^\top \\
    &= \forgetting\sum_{t'=t_m+h}^{t_c-1}\forgetting^{t_c-t'-1}\embed{e(t')}\embed{e(t'-1)}^\top + \embed{e(t_c)}\embed{e(t_c-1)}^\top
\end{align*}
\begin{align}
    \label{eq:Q}
    \therefore \mat{Q}^{new} = \forgetting\mat{Q}^{prev} + \embed{e(t_c-1)}\embed{e(t_c)}^\top
\end{align}
and similarly
\begin{align}
    \label{eq:bP}
    \mat{P}^{new} &= (\forgetting{(\mat{P}^{prev})}^{-1} + \embed{e(t_c)}\embed{e(t_c)}^\top)^{-1}
\end{align}Here, we apply the Sherman-Morrison formula~\cite{sherman1950adjustment} to the RHS of Eq. \eqref{eq:bP}.
Note that $\embed{e(t_c)}^\top\mat{P}^{prev}\embed{e(t_c)} > 0$
because $\mat{P}^{-1} = \mat{R}\Forgetting\mat{R}^\top$ is positive definite by definition.
\begin{align}
    \label{eq:P}
    \therefore \mat{P}^{new} = \frac{1}{\forgetting}(\mat{P}^{prev} - \frac{\mat{P}^{prev}\embed{e(t_c-1)}\embed{e(t_c-1)}^\top\mat{P}^{prev}}{\forgetting + \embed{e(t_c-1)}^\top\mat{P}^{prev}\embed{e(t_c-1)}})
\end{align}
Finally, combining Eq. \eqref{eq:Q} and Eq. \eqref{eq:P} gives the recurrence relations of $\trans$ for Eq. \eqref{eq:update_trans}.
\begin{align*}
    \begin{split}
        \trans^{new} &= \trans^{prev} + (\embed{e(t_c)} - \trans^{prev}\embed{e(t_c-1)}\boldsymbol\gamma \\
        \boldsymbol\gamma &= \frac{\embed{e(t_c-1)}^\top\mat{P}^{prev}}{\forgetting + \embed{e(t_c-1)}^\top\mat{P}^{prev}\embed{e(t_c-1)}}
    \end{split}
\end{align*}
    
% \end{enumerate}
\par
% \TSK{
% \begin{figure*}[t]
    \begin{tabular}{cccc}
      \hspace{-1.5em}
      \begin{minipage}[c]{0.24\linewidth}
        \centering
        % \vspace{1em}
        \includegraphics[width=\linewidth]{results/web/original1_ver1.0.pdf}
        \vspace{-2em} \\
        \hspace{1.5em}
        % (a-i) Snapshot
        % (a-i) $l_s$-steps-ahead future value forecasting
        (a-i) Original data $\mX^c$
        \label{fig:web:forecast}
      \end{minipage} &
      \hspace{-1.5em}
      \begin{minipage}[c]{0.24\linewidth}
        \centering
        \includegraphics[width=\linewidth]{results/web/latent1_ver1.2.pdf}
        \vspace{-2em} \\
        \hspace{1.5em}
        (a-ii) Inherent signals $\mE$
      \end{minipage} &
      \hspace{-1.5em}
      \begin{minipage}[c]{0.24\linewidth}
        \centering
        \includegraphics[width=0.8\linewidth]{results/web/causal1_ver1.0.pdf}
        % \vspace{-2em}
        \\
        % \hspace{2.0em}
        (a-iii) Causal relationship $\mB$
      \end{minipage} &
      \hspace{-1.5em}
      \begin{minipage}[c]{0.24\linewidth}
        \centering
        \includegraphics[width=0.95\linewidth]{results/web/mode1_ver1.0.pdf}
        % \vspace{-2em}
        \\
        \hspace{-0.7em}
        (a-iv) Latent dynamics $\eigs$
      \end{minipage} \vspace{0.5em} \\
      % \caption{(a) Snapshot at the current time point $t_c = 208$}
      \multicolumn{4}{c}{\textbf{(a) Snapshots at current time point $t_c=208$.}}
      \vspace{0.5em} \\
      \hspace{-1.5em}
      \begin{minipage}[c]{0.24\linewidth}
        \centering
        % \vspace{1em}
        \includegraphics[width=\linewidth]{results/web/original2_ver1.0.pdf}
        \vspace{-2em} \\
        \hspace{1.5em}
        % (a-i) Snapshot
        % (b-i) $l_s$-steps-ahead future value forecasting
        (b-i) Original data $\mX^c$
      \end{minipage} &
      \hspace{-1.5em}
      \begin{minipage}[c]{0.24\linewidth}
        \centering
        \includegraphics[width=\linewidth]{results/web/latent2_ver1.2.pdf}
        \vspace{-2em} \\
        \hspace{1.5em}
        (b-ii) Inherent signals $\mE$
      \end{minipage} &
      \hspace{-1.5em}
      \begin{minipage}[c]{0.24\linewidth}
        \centering
        \includegraphics[width=0.8\linewidth]{results/web/causal2_ver1.0.pdf}
        % \vspace{-2em}
        \\
        % \hspace{2.0em}
        (b-iii) Causal relationship $\mB$
      \end{minipage} &
      \hspace{-1.5em}
      \begin{minipage}[c]{0.24\linewidth}
        \centering
        \includegraphics[width=0.95\linewidth]{results/web/mode2_ver1.0.pdf}
        % \vspace{-2em} 
        \\
        \hspace{-0.7em}
        (b-iv) Latent dynamics $\eigs$
      \end{minipage} \vspace{0.5em} \\
      \multicolumn{4}{c}{\textbf{(b) Snapshots at current time point $t_c=443$.}}
    \end{tabular}
    \vspace{-1.0em}
    \caption{\method modeling for a web-click activity stream related to beer query sets (i.e., \googletrend).
      Two sets of
      % invaluable knowledge
      snapshots taken
      at two different time points
      % (i.e., $t_c = 208, 443$, respectively)
      % on December 27, 2007 (top) and July 14, 2011 (bottom)
      show:
      (a/b-i) the current window of the original data stream,
      % where, the blue right vertical and red axes represent the current and $l_s$-steps-ahead time points, (i.e., $t_c, t_c+l_s$), respectively;
      % where, the blue vertical line to the right represents the current time point $t_c$;
      % where, the blue right vertical axis represents the current time point $t_c$;
      (a/b-ii) independent signals $\mE$ specific to each observation;
      % (c) time-evolving relationships with each other based on variables generating processes (i.e., \relations) and
      (a/b-iii) causal relationships $\mB\in\mathcal{B}$ and
      (a/b-iv) interpretable latent dynamics $\eigs$,
      where the argument and the absolute value of each point correspond to
      the temporal frequency and decay rate of modes, respectively.
      }
    \label{fig:web}
    \vspace{-1.2em}
\end{figure*}
% }
\setcounter{lemma}{1}
\subsection{Proof of Lemma \ref{lemma:create_time}}
\begin{proof}
The dominant steps in \textsc{RegimeCreation} are I, IV, and VI.
The decomposition $\mX$ into $\demixing^{-1}$ and $\mE$ using ICA requires $O(d^2N)$.
For each observation,
the SVD of $\ith{\mat{R}}\mat{M}$ requires $O(h^2N)$, and the eigendecomposition of $\ith{\tilde{\trans}}$ takes $O(k_i^3)$.
The straightforward way to
process IV and VI
is to perform the calculation $d$ times sequentially, i.e., they require $O(dh^2N+\sum_ik_i^3)$ in total.
However, since these operations do not interfere with each other,
they are simultaneously computed by parallel processing.
Therefore, the time complexity of \textsc{RegimeCreation} is $O(N(d^2+h^2)+k^3)$, where $k=\max_i(k_i)$.
\end{proof}
\subsection{Proof of Lemma \ref{lemma:causal}}
\begin{proof}
First, we need to formulate the causal structure.
Here, we utilize the structural equation model~\cite{pearl2009causality}, denoted by $\mX_{\text{sem}} = \mB_{\text{sem}}\mX_{\text{sem}} + \mE_{\text{sem}}$.
Because this model is known as the general formulation of causality, if $\mB_{\text{sem}}$ in this model is identified, then it can be said that we discover causality.
In other words, we need to prove that our proposed algorithm can find the causal adjacency matrix $\mB$ aligning with this model.
Solving the structural equation model for $\mX_{\text{sem}}$, we obtain 
$\mX_{\text{sem}} = \demixing^{-1}_{\text{sem}}\mE_{\text{sem}}$
where $\demixing_{\text{sem}} = \mat{I} - \mB_{\text{sem}}$.
It is shown that we can identify $\demixing_{\text{sem}}$ in the above equation by ICA,
except for the order and scaling of the independent components, if the observed data is a linear, invertible mixture of non-Gaussian independent components~\cite{comon1994independent}.
Thus, demonstrating that \modelgenerator precisely resolves the two indeterminacies of a mixing matrix $\mW^{-1}$ (i.e., the inverse of $\demixing \in \regime^c$) suffices to complete the proof because $\demixing$ is computed by ICA in \textsc{RegimeCreation}. \par
First, we reveal that our algorithm can resolve the order indeterminacy.
We can permutate the causal adjacency matrix $\mB$ to strict lower triangularity thanks to the acyclicity assumption~\cite{bollen1989structural}.
%, which is without loss of generality.
Thus, correctly permuted and scaled $\mW$
is a lower triangular matrix with all ones on the diagonal.
It is also said that there would only be one way to permutate $\mW$, which meets the above condition~\cite{shimizu2006linear}.
Thus, \modelgenerator can identify the order of a mixing matrix by the process in step I (i.e., finding the permutation of rows of a mixing matrix that yields a matrix without any zeros on the main diagonal).
Next, with regard to the scale of indeterminacy,
it is apparent that we only need to focus on the diagonal element,
remembering that the permuted and scaled $\mW$ has all ones on the diagonal.
Therefore, we prove that \modelgenerator can resolve the order and scaling of the indeterminacies of a mixing matrix $\demixing^{-1}$.
\end{proof}
% \myparaitemize{Proof of Lemma \ref{lemma:time}} \par
\subsection{Proof of Lemma \ref{lemma:stream_time}}
\begin{proof}
For each time point, \method first runs \modelestimator,
which estimates the optimal full parameter set $\modelparam$ and the model candidate $\candparam$ for the current window $\mX^c$.
If the current regime $\regime^c$ fits well,
it takes $O(N\sum_i k_i)$ time.
Otherwise, it takes $O(RN\sum_i k_i)$ time to find a better regime in $\regimeset\in\modelparam$.
Furthermore, if \method encounters an unknown pattern,
it runs \textsc{RegimeCreation}, which takes $O(N(d^2+h^2)+k^3)$ time.
Subsequently, it runs \modelgenerator to identify the causal adjacency matrix and forecast an $l_s$-steps-ahead future value,
which takes $O(d^2)$ and $O(l_s)$ time, respectively.
Note that $l_s$ is negligible because of the small constant value.
Finally, when \method does not create a new regime,
it executes \regimeupdate, which needs $O(dh^2)$ time.
Thus, the total time complexity is at least $O(N\sum_ik_i+dh^2)$ time and at most $O(RN\sum_i k_i+N(d^2+h^2)+k^3)$ time per process.
\end{proof}

% \input{components/table_acc_app_forecast}
% \begin{table*}[t]
    % \small
    \centering
    \caption{Ablation study results with forecasting steps $l_s\in\{5, 10, 15\}$ for both synthetic and real-world datasets.}
    \vspace{-1.0em}
    \begin{tabular}{c|c|cc|cc|cc|cc|cc}
    \toprule
    % \:Datasets\:
    \multicolumn{2}{c|}{Datasets}
    % & \#0 & \#1 & \#2 & \#3 & \#4 \\
    & \multicolumn{2}{c|}{\synthetic} & \multicolumn{2}{c|}{\covid} & \multicolumn{2}{c|}{\googletrend} & \multicolumn{2}{c|}{\chickendance} & \multicolumn{2}{c}{\exercise} \\
    \midrule
    \multicolumn{2}{c|}{Metrics}
    % \:Metrics\:
    & \:RMSE & MAE\:\,
    & \:RMSE & MAE\:\,
    & \:RMSE & MAE\:\,
    & \:RMSE & MAE\:
    & \:RMSE & MAE\:\: \\
    \midrule
    \multirow[t]{3}{*}{\:\:\method (full)\:\:}
    % \:\:\method (full)\:\:
    & 5 & \:0.722 & 0.528\:\, & \:0.588 & 0.268\:\, & \:0.573 & 0.442\:\, & \:0.353 & 0.221\: & \:0.309 & 0.177\:\, \\
    & 10 & \:0.829 & 0.607\:\, & \:0.740 & 0.361\:\, & \:0.620 & 0.481\:\, & \:0.511 & 0.325\: & \:0.501 & 0.309\:\, \\
    & 15 & \:0.923 & 0.686\:\, & \:0.932 & 0.461\:\, & \:0.646 & 0.505\:\, & \:0.653 & 0.419\: & \:0.687 & 0.433\:\, \\
    \midrule
    \multirow[t]{3}{*}{\:\:w/o causality\:\:}
    & 5 & \:0.759 & 0.563\:\, & \:0.758 & 0.374\:\, & \:0.575 & 0.437\:\, & \:0.391 & 0.262\: & \:0.375 & 0.218\:\, \\
    & 10 & \:0.925 & 0.696\:\, & \:0.848 & 0.466\:\, & \:0.666 & 0.511\:\, & \:0.590 & 0.398\: & \:0.707 & 0.433\:\, \\
    & 15 & \:1.001 & 0.760\:\, & \:1.144 & 0.583\:\, & \:0.708 & 0.545\:\, & \:0.821 & 0.537\: & \:0.856 & 0.533\:\, \\
    \bottomrule
    \end{tabular}
    \label{table:ablation}
    \vspace{-0.75em}
\end{table*}

\section{Experimental Setup}
% \label{section:app:experiments}
\label{section:app:experiments:setting}
In this section, we describe the experimental setting in detail.
% \subsection{Experimental Setting}
% \myparaitemize{Experimental Setting}
We conducted all our experiments on
% \unclear{<server spec>}.
an Intel Xeon Platinum 8268 2.9GHz quad core CPU
with 512GB of memory and running Linux.
We normalized the values of each dataset based on their mean and variance (z-normalization).
The length of the current window $N$ was $50$ steps in all experiments.
\par
\myparaitemize{Generating the Datasets}
We randomly generated synthetic multivariate data streams containing multiple clusters, each of which exhibited a certain causal relationship.
For each cluster, the causal adjacency matrix $\mB$ was generated from a well-known random graph model, namely Erdös-Rényi (ER)~\cite{erdos1960evolution} with edge density $0.5$ and the number of observed variables $d$ was set at 5.
The data generation process was modeled as a structural equation model~\cite{pearl2009causality},
where each value of the causal adjacency matrix $\mB$ was sampled from a uniform distribution $\mathcal{U}(-2, -0.5)\cup(0.5, 2)$.
In addition, to demonstrate the time-changing nature of exogenous variables, 
we allowed the inherent signals variance $\sigma^2_{i, t}$ (i.e., $\ith{e}(t)\sim\text{Laplace}(0, \sigma_{i, t}^2)$)
to change over time.
Specifically, we introduced $h_{i, t}=\text{log}(\sigma^2_{i, t})$, which evolves according to an autoregressive model, where the coefficient and noise variance of the autoregressive model were sampled from $\mathcal{U}(0.8, 0.998)$ and $\mathcal{U}(0.01, 0.1)$, respectively.
% however, 

The overall data stream was then generated by constructing a temporal sequence of cluster segments and each segment had $500$ observations (e.g., ``$1,2,1$'' consists of three segments containing two types of causal relationships and its total sample size is $1,500$). We ran our experiments on five different temporal sequences: ``$1,2,1$'', ``$1,2,3$'', ``$1,2,2,1$'', ``$1,2,3,4$'', and ``$1,2,3,2,1$'' to encompass various types of real-world scenarios.
\par
\myparaitemize{Baselines}
The details of the baselines we used throughout our extensive experiments are summarized as follows:
\par\noindent
(1) Causal discovering methods
{\setlength{\leftmargini}{11pt}
\vspace{-0.3ex}
\begin{itemize}
    \item CASPER~\cite{liu2023discovering}: is a state-of-the-art method for causal discovery, integrating the graph structure into the score function and reflecting the causal distance between estimated and ground truth causal structure. We tuned the parameters by following the original paper setting.
    \item DARING~\cite{he2021daring}: introduces an adversarial learning strategy to impose an explicit residual independence constraint for causal discovery. We searched for three types of regularization penalties $\{\alpha, \beta, \gamma\}\in\{0.001, 0.01, 0.1, 1.0, 10\}$.
    % aiming to improve the learning of acyclic graphs.
    \item NoCurl~\cite{yu2021dag}: uses a two-step procedure: initialize a cyclic solution first and then employ the Hodge decomposition of graphs. We set the optimal parameter presented in the original paper.
    % and learn a DAG structure by projecting the cyclic graph to the gradient of a potential function.
    \item NOTEARS-MLP~\cite{zheng2020learning}: is an extension of NOTEARS~\cite{zheng2018dags} (mentioned below) for nonlinear settings, which aims to approximate the generative structural equation model by MLP.
    We used the default parameters provided in authors' codes\footnote[2]{\url{https://github.com/xunzheng/notears} \label{fot:notears}}.
    \item NOTEARS~\cite{zheng2018dags}:
    % is specifically designed for linear settings and
    is a differentiable optimization method with an acyclic regularization term to estimate a causal adjacency matrix.
    We used the default parameters provided in authors' codes\footref{fot:notears}.
    % estimates the true causal graph by minimizing the fixed reconstruction loss with the continuous acyclicity constraint.
    \item LiNGAM~\cite{shimizu2006linear}:
    exploits the non-Gaussianity of data to determine the direction of causal relationships. It has no parameters to set.
    % and we used the authors source codes\footnote{https://github.com/cdt15/lingam}.
    \item GES~\cite{chickering2002optimal}: is a traditional score-based bayesian algorithm that discovers causal relationships in a greedy manner.
    It has no parameters to set.
    We employed BIC as the score function and utilized the open-source in~\cite{kalainathan2020causal}.
\end{itemize}
\vspace{-0.5ex}}
\par\noindent
(2) Time series forecasting methods
{\setlength{\leftmargini}{11pt}
\vspace{-0.3ex}
\begin{itemize}
    \item TimesNet/PatchTST~\cite{wu2023timesnet, Yuqietal-2023-PatchTST}: are state-of-the-art TCN-based and Transformer-based methods, respectively.
    The past sequence length was set as 16 (to match the current window length).
    % Other parameters follow the parameter settings suggested in the original paper.
    Other parameters followed the original paper setting.
    % \item PatchTST~\cite{Yuqietal-2023-PatchTST}: is a state-of-the-art Transformer-based method for time series forecasting. The past sequence length is set as 16 for the same reason as above.
    \item DeepAR~\cite{salinas2020deepar}: models probabilistic distribution in the future, based on RNN. We built the model with 2-layer 64-unit RNNs. We used Adam optimization~\cite{adam} with a learning rate of 0.01 and let it learn for 20 epochs with early stopping.
    % to choose the best model.
    \item OrbitMap~\cite{matsubara2019dynamic}:
    % is a stream forecasting algorithm that finds important time-evolving patterns with multiple discrete non-linear dynamical systems.
    finds important time-evolving patterns for stream forecasting.
    We determined the optimal transition strength $\rho$ to minimize the forecasting error in training.
    \item ARIMA~\cite{box1976arima}: is one of the traditional time series forecasting approaches based on linear
    equations. We determined the optimal parameter set using AIC.
\end{itemize}}


\end{document}


