\renewcommand{\arraystretch}{1.5}
\begin{table*}[ht]
\begin{adjustbox}{width=2\columnwidth} % Automatically fit within column width

\centering

\begin{tabular}{cccccccccc}
    \toprule
    \multirow{2}{*}{Method} & \multirow{2}{*}{Metric} & \multicolumn{2}{c}{Card Flip} & \multicolumn{3}{c}{Bookshelf} & \multicolumn{3}{c}{Kitchen} \\
    \cmidrule(lr){3-4} \cmidrule(lr){5-7} \cmidrule(lr){8-10}
    & & {Non-Prehensile} & Place & Topple & Push & Place & NP in sink & Push in Shelf & Place \\
    \cmidrule(lr){1-1}\cmidrule(lr){2-2}\cmidrule(lr){3-4} \cmidrule(lr){5-7} \cmidrule(lr){8-10}
    
     & \makecell{Execution Time (s)}
     & 
     & 
     & 
     & 
     &
    &
     &

    \\
    \cdashline{2-10}
    {Motion Planner \cite{sucan2012the-open-motion-planning-library}} & \makecell{Precondition rate}
     & 
     & 
     & 
     & 
     &
    &
     &
    \\
    \cdashline{2-10}
     & \makecell{Skill success}
     & 
     & 
     & 
     & 
     &
    &
     &
    \\
    \midrule

     & \makecell{Execution Time (s)}
     & 
     & 
     & 
     & 
     &
    &
     &
    \\
    \cdashline{2-10}
    {\makecell{Connector}} & \makecell{Precondition rate}
& 
     & 
     & 
     & 
     &
    &
     &
    \\
    \cdashline{2-10}
     & \makecell{Skill success}
     & 
     & 
     & 
     & 
     &
    &
     &
    \\
    \bottomrule
\end{tabular}
\end{adjustbox}

\caption{ablation connector versus motion planner.}\label{table:ablation_connector}
\begin{flushleft}
\footnotesize
\textbf{Table \ref{table:ablation_connector}} — This table evaluates each method using three metrics: 1) execution time for practicability with skill-RRT, 2) precondition rate for satisfying the subsequent skill's precondition, and 3) skill success for effectively connecting skills to achieve \( q^\text{obj}_\text{sg} \).
\end{flushleft}
\end{table*}
