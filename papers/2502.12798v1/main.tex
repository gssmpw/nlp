\documentclass[11pt,final]{article}

%Added after EC
\usepackage{xcolor}


%Make sure the order of begin document, title, author, abstract, make title is correct
%%%%%%%%%%%%%%%%


\usepackage{fullpage}
\usepackage{amsmath,amssymb,amsthm}
\usepackage{lmodern}



\usepackage[numbers,sort&compress]{natbib} 
\setcitestyle{numbers}

\usepackage[colorlinks=true]{hyperref}
\hypersetup{
     colorlinks  = true,
     urlcolor    = teal,
	 citecolor   = teal,
	 linkcolor   = red
}




\usepackage[ruled,noend,linesnumbered]{algorithm2e} % For algorithms
\renewcommand{\algorithmcfname}{ALGORITHM}
\SetAlFnt{\small}
\SetAlCapFnt{\small}
\SetAlCapNameFnt{\small}
\SetAlCapHSkip{0pt}
\IncMargin{-\parindent}



\usepackage[utf8]{inputenc} % allow utf-8 input
\usepackage[T1]{fontenc}    % use 8-bit T1 fonts
\usepackage{lmodern}
\usepackage{url}            % simple URL typesetting
\usepackage{amsfonts}       % blackboard math symbols
\usepackage{nicefrac}       % compact symbols for 1/2, etc.
\usepackage{microtype}      % microtypography
\usepackage{booktabs}       % professional-quality tables
%\usepackage[linesnumbered,ruled,noend]{algorithm2e}

%Make sure this appears AFTER ams packages, etc.
\newtheorem{theorem}{Theorem}
\newtheorem{lemma}{Lemma}
\newtheorem{proposition}{Proposition}
\newtheorem{corollary}{Corollary}
\newtheorem{conjecture}{Conjecture}
\newtheorem{definition}{Definition}
%\newtheorem{example}{Example}
\newtheorem{remark}{Remark}
\newtheorem{property}{Property}
\newtheorem{claim}{Claim}
\newtheorem{observation}{Observation}
\newtheorem{assumption}{Assumption}

\theoremstyle{definition} % Ensures normal (non-italic) text
\newtheorem{example}{Example}


\newcount\Includeappendix %include appendices or not
\Includeappendix=1
\newcommand{\appnx}[1]{{\ifnum\Includeappendix=1{#1}\else{the appendix}\fi}}


% Choose a citation style by commenting/uncommenting the appropriate line:
%\setcitestyle{acmnumeric}
\newcommand{\stap}{\bS_{\rm TAP}}
\newcommand{\slamp}{\bS_{\rm LAMP}}
\newcommand{\gout}{\bg_{\rm out}}

\newcommand{\Py}{\mathsf{Z}}
\newcommand{\I}{\mathbb{I}}
\newcommand{\Zout}{\Py}
\newcommand{\dgout}{\bG}

\newcommand{\bSigma}{\boldsymbol{\Sigma}}

% Probability
\renewcommand{\P}{\mathbb{P}}
\newcommand{\E}{\mathbb{E}}
\newcommand{\Var}{\text{Var}}
\newcommand{\Cov}{\mathrm{Cov}}
\newcommand{\cN}{\mathcal{N}}

% Sets
\newcommand{\Z}{\mathbb{Z}}
\newcommand{\R}{\mathbb{R}}
\newcommand{\C}{\mathbb{C}}
\newcommand{\N}{\mathbb{N}}
\renewcommand{\S}{\mathbb{S}}
\def\ball{{\mathsf B}}

% Variables
\newcommand{\eps}{\varepsilon} 
\newcommand{\vphi}{\varphi}
\def\id{{\mathbf I}}


% Math
\renewcommand{\d}{\textup{d}}
\renewcommand{\l}{\vert}
\newcommand{\dl}{\Vert}
\newcommand{\<}{\langle}
\renewcommand{\>}{\rangle}
\newcommand{\sign}{\text{sign}}
\newcommand{\diag}{\text{diag}}
%\newcommand{\tr}{\text{tr}}
%\newcommand{\op}{{\rm op}}
\newcommand{\ones}{\bm{1}}
\newcommand{\what}{\widehat}
%\newcommand{\grad}{\boldsymbol{\nabla}}
\def\sT{{\mathsf T}}
\def\bzero{{\boldsymbol 0}}
\newcommand{\bomega}{\boldsymbol{\omega}}
\newcommand{\bOmega}{\boldsymbol{\Omega}}
\newcommand{\flatten}{\operatorname{flat}}
\newcommand{\bcT}{\boldsymbol{\mathcal{T}}}


\DeclareMathOperator*{\argmin}{arg\,min}
\DeclareMathOperator*{\argmax}{arg\,max}
\DeclareMathOperator*{\argsup}{arg\,sup}
\DeclareMathOperator*{\arginf}{arg\,inf}
\newcommand{\eqnd}{\, {\buildrel d \over =} \,} 
\newcommand{\eqndef}{\mathrel{\mathop:}=}
\def\doteq{{\stackrel{\cdot}{=}}}
\newcommand{\goto}{\longrightarrow}
\newcommand{\gotod}{\buildrel d \over \longrightarrow} 
\newcommand{\gotoas}{\buildrel a.s. \over \longrightarrow} 
\def\simiid{{\stackrel{i.i.d.}{\sim}}}


% Notations 
\newcommand{\notate}[1]{\textcolor{red}{\textbf{[#1]}}}
\newcommand{\cc}[1]{\textcolor{blue}{\textbf{[CC:#1]}}}
\newcommand{\yw}[1]{\textcolor{pink}{\textbf{[YW:#1]}}}
\newcommand{\mc}[1]{\mathcal{#1}}
\newcommand{\mb}[1]{\mathbf{#1}}


% Theorem
\newtheorem{question}{Question}
\newtheorem{property}{Property}
\newtheorem{objective}{Objective}
\newtheorem{claim}{Claim}
\newtheorem{example}{Example}



%\usepackage[inline]{showlabels}

\DeclareSymbolFont{rsfs}{U}{rsfs}{m}{n}
\DeclareSymbolFontAlphabet{\mathscrsfs}{rsfs}



% Bold symbols
\def\bA{{\boldsymbol A}}
\def\bB{{\boldsymbol B}}
\def\bC{{\boldsymbol C}}
\def\bD{{\boldsymbol D}}
\def\bE{{\boldsymbol E}}
\def\bF{{\boldsymbol F}}
\def\bG{{\boldsymbol G}}

\def\bH{{\boldsymbol H}}
\def\bI{{\boldsymbol I}}
\def\bJ{{\boldsymbol J}}
\def\bK{{\boldsymbol K}}
\def\bL{{\boldsymbol L}}
\def\bM{{\boldsymbol M}}
\def\bN{{\boldsymbol N}}
\def\bO{{\boldsymbol O}}
\def\bP{{\boldsymbol P}}
\def\bQ{{\boldsymbol Q}}
\def\bR{{\boldsymbol R}}
\def\bS{{\boldsymbol S}}
\def\bT{{\boldsymbol T}}
\def\bU{{\boldsymbol U}}
\def\bV{{\boldsymbol V}}
\def\bW{{\boldsymbol W}}
\def\bX{{\boldsymbol X}}
\def\bY{{\boldsymbol Y}}
\def\bZ{{\boldsymbol Z}}

\def\ba{{\boldsymbol a}}
\def\bb{{\boldsymbol b}}
\def\be{{\boldsymbol e}}
\def\boldf{{\boldsymbol f}}
\def\bg{{\boldsymbol g}}
\def\bh{{\boldsymbol h}}
\def\bi{{\boldsymbol i}}
\def\bj{{\boldsymbol j}}
\def\bk{{\boldsymbol k}}
\def\bt{{\boldsymbol t}}
\def\bu{{\boldsymbol u}}
\def\bv{{\boldsymbol v}}
\def\bw{{\boldsymbol w}}
\def\bx{{\boldsymbol x}}
\def\by{{\boldsymbol y}}
\def\bz{{\boldsymbol z}}

\def\bmu{{\boldsymbol \mu}}
\def\bbeta{{\boldsymbol \beta}}
\def\bdelta{{\boldsymbol\delta}}
\def\beps{{\boldsymbol \eps}}
\def\blambda{{\boldsymbol \lambda}}
\def\bpsi{{\boldsymbol \psi}}
\def\bphi{{\boldsymbol \phi}}
\def\btheta{{\boldsymbol \theta}}
\def\bvphi{{\boldsymbol \vphi}}
\def\bxi{{\boldsymbol \xi}}

\def\bDelta{{\boldsymbol \Delta}}
\def\bLambda{{\boldsymbol \Lambda}}
\def\bPsi{{\boldsymbol \Psi}}
\def\bPhi{{\boldsymbol \Phi}}
\def\bSigma{{\boldsymbol \Sigma}}
\def\bTheta{{\boldsymbol \Theta}}

\def\bfzero{{\boldsymbol 0}}
\def\bfone{{\boldsymbol 1}}
\def\bPi{{\boldsymbol \Pi}}


% Symbols with hat
\def\hba{{\hat {\boldsymbol a}}}
\def\hf{{\hat f}}
\def\ha{{\hat a}}
\def\tcT{\widetilde{\mathcal T}}
\def\tK{\widetilde{K}}


\def\cR{\mathcal{R}}
\def\test{{\rm test}}
\def\train{{\rm train}}
\def\CV{\text{CV}}
\def\GCV{\text{GCV}}
\def\sfs{{\sf s}}

% rm symbols
\def\spn{{\rm span}}
\def\supp{{\rm supp}}
\def\Easy{{\rm E}}
\def\Hard{{\rm H}}
\def\post{{\rm post}}
\def\pre{{\rm pre}}
\def\Rot{{\rm Rot}}
\def\Sft{{\rm Sft}}
\def\endd{{\rm end}}
\def\KR{{\rm KR}}
\def\bbHe{{\rm He}}
\def\sk{{\rm sk}}
\def\de{{\rm d}}
\def\Tr{{\rm Tr}}
\def\lin{{\rm lin}}
\def\res{{\rm res}}
\def\degzero{{\rm deg0}}
\def\degone{{\rm deg1}}
\def\Poly{{\rm Poly}}
\def\Poly{{\rm Poly}}
\def\Coeff{{\rm Coeff}}
\def\de{{\rm d}}
\def\Unif{{\rm Unif}}
\def\lin{{\rm lin}}
\def\res{{\rm res}}
\def\RF{{\rm RF}}
\def\NT{{\rm NT}}
\def\Cyc{{\rm Cyc}}
\def\RC{{\rm RC}}
\def\kernel{\rm Ker}
\def\image{{\rm Im}}
\def\Easy{{\rm E}}
\def\Hard{{\rm H}}
\def\post{{\rm post}}
\def\pre{{\rm pre}}
\def\Rot{{\rm Rot}}
\def\Sft{{\rm Sft}}
\def\ddiag{{\rm ddiag}}
\def\KR{{\rm KR}}
\def\RR{{\rm RR}}
\def\bbHe{{\rm He}}
\def\eff{{\rm eff}}

\def\spn{{\rm span}}


%mathcal symbols
\def\cV{{\mathcal V}}
\def\cG{{\mathcal G}}
\def\cO{{\mathcal O}}
\def\cP{{\mathcal P}}
\def\cW{{\mathcal W}}
\def\cT{{\mathcal T}}
\def\cC{{\mathcal C}}
\def\cQ{{\mathcal Q}}
\def\cL{{\mathcal L}}
\def\cF{{\mathcal F}}
\def\cE{{\mathcal E}}
\def\cS{{\mathcal S}}
\def\cI{{\mathcal I}}
\def\cV{{\mathcal V}}
\def\cG{{\mathcal G}}
\def\cO{{\mathcal O}}
\def\cP{{\mathcal P}}
\def\cW{{\mathcal W}}
\def\cT{{\mathcal T}}
\def\cH{{\mathcal H}}
\def\cA{{\mathcal A}}


\def\tbA{\Tilde \bA}

%mathbb mathsf sf symbols
\def\K{{\mathbb K}}
\def\H{{\mathbb H}}
\def\T{{\mathbb T}}
\def\bbV{{\mathbb V}}
\def\W{{\mathbb W}}
\def\sM{{\mathsf M}}
\def\sW{{\mathsf W}}
\def\Unif{{\sf Unif}}
\def\normal{{\sf N}}
\def\proj{{\mathsf P}}
\def\ik{{\mathsf k}}
\def\il{{\mathsf l}}
\def\sM{{\sf M}}
\def\RKHS{{\sf RKHS}}
\def\RF{{\sf RF}}
\def\NT{{\sf NT}}
\def\NN{{\sf NN}}
\def\reals{{\mathbb R}}
\def\integers{{\mathbb Z}}
\def\naturals{{\mathbb N}}
\def\Top{{\mathbb T}}
\def\Kop{{\mathbb K}}
\def\Aop{{\mathbb A}}
\def\normal{{\sf N}}
\def\proj{{\mathsf P}}
\def\bbV{{\mathbb V}}
\def\sW{{\mathsf W}}
\def\sM{{\mathsf M}}
\def\T{{\mathbb T}}
\def\K{{\mathbb K}}
\def\H{{\mathbb H}}
\def\Unif{{\sf Unif}}
\def\normal{{\sf N}}
\def\Uop{{\mathbb U}}
\def\Hop{{\mathbb H}}
\def\Sop{{\mathbb S}}
\def\proj{{\mathsf P}}
\def\ik{{\mathsf k}}
\def\il{{\mathsf l}}
\def\sM{{\sf M}}
\def\RKHS{{\sf RKHS}}
\def\RF{{\sf RF}}
\def\NT{{\sf NT}}
\def\NN{{\sf NN}}
\def\reals{{\mathbb R}}
\def\integers{{\mathbb Z}}
\def\naturals{{\mathbb N}}
\def\proj{{\mathsf P}}
\def\Hop{{\mathbb H}}
\def\Uop{{\mathbb U}}
\def\App{{\rm App}}
\def\sU{{\sf U}}
\def\sV{{\sf V}}
\def\sfp{{\sf p}}
\def\tcE{\widetilde{\cE}}
\def\tmu{\widetilde  \mu}
\def\tbD{\widetilde{\bD}}




\def\stest{\mbox{\tiny\rm test}}

\def\seff{\mbox{\tiny\rm eff}}

\def\Ker{K}
\def\tKer{\tilde{K}}
\def\oKop{\overline{{\mathbb K}}}
\def\oKer{\overline{K}}
\def\ocV{\overline{{\mathcal V}}}

\def\th{\tilde{h}}
\def\tQ{\tilde{Q}}
\def\tsigma{\Tilde{\sigma}}


\def\hba{{\hat {\boldsymbol a}}}
\def\hf{{\hat f}}
\def\hy{{\hat y}}
\def\hU{\widehat{U}}
\def\hUop{\widehat{\mathbb U}}
\def\tbDelta{\widetilde{\bDelta}}


\def\tcT{\widetilde{\mathcal T}}

\def\Cyc{{\rm Cyc}}
\def\inv{{\rm inv}}


\def\cE{{\mathcal E}}
\def\cD{{\mathcal D}}
\def\cX{{\mathcal X}}
\def\cF{{\mathcal F}}
\def\cS{{\mathcal S}}
\def\cI{{\mathcal I}}



\def\He{{\rm He}}
\def\lin{{\rm lin}}
\def\res{{\rm res}}
\def\degzero{{\rm deg0}}
\def\degone{{\rm deg1}}
\def\Poly{{\rm Poly}}
\def\Coeff{{\rm Coeff}}
\def\de{{\rm d}}
\def\Unif{{\rm Unif}}
\def\RF{{\rm RF}}
\def\NT{{\rm NT}}
\def\Cyc{{\rm Cyc}}
\def\RC{{\rm RC}}

\def\tK{\widetilde{K}}
\def\stest{\mbox{\tiny\rm test}}


\def\ttau{\tilde{\tau}}


\def\cE{{\mathcal E}}
\def\bt{{\boldsymbol t}}
\def\normal{{\sf N}}

\def\bDelta{{\boldsymbol \Delta}}










\def\cX{{\mathcal X}}
\def\CKR{{\rm CKR}}
\def\bproj{{\overline \proj}}
\def\quadratic{{\rm quad}}
\def\cube{{\rm cube}}
\def\Cube{{\mathscrsfs Q}}

\def\Poly{{\rm Poly}}
\def\Coeff{{\rm Coeff}}
\def\RF{{\rm RF}}
\def\NT{{\rm NT}}
\def\bA{{\boldsymbol A}}
\def\btheta{{\boldsymbol \theta}}
\def\bTheta{{\boldsymbol \Theta}}
\def\bLambda{{\boldsymbol \Lambda}}
\def\blambda{{\boldsymbol \lambda}}

\def\cM{{\mathcal M}}

\def\cT{{\mathcal T}}
\def\cV{{\mathcal V}}
\def\bP{{\boldsymbol P}}
\def\diag{{\rm diag}}
\def\bS{{\boldsymbol S}}
\def\bO{{\boldsymbol O}}
\def\bD{{\boldsymbol D}}
\def\bPsi{{\boldsymbol \Psi}}
\def\bsh{{\boldsymbol h}}
\def\bL{{\boldsymbol L}}



\def\osigma{\overline{\sigma}}
\def\tbu{\Tilde \bu}
\def\tbZ{\Tilde \bZ}
\def\tbphi{\Tilde \bphi}
\def\tbpsi{\Tilde \bpsi}

\def\tbf{\Tilde \boldf}
\def\hbU{\hat{{\boldsymbol U}}_\lambda }
\def\hbUi{\hat{{\boldsymbol U}}_\lambda^{-1} }
\def\bb{{\boldsymbol b}}
\def\bsigma{{\boldsymbol \sigma}}

\def\hf{\hat f}
\def\hbf{\hat \boldf}
\def\bR{{\boldsymbol R}}
\def\bpsi{{\boldsymbol \psi}}
\def\cuH{\mathscrsfs{H}}

\def\noisestd{\sigma_{\varepsilon}}

\def\evn{{\mathsf m}}
\def\evN{{\mathsf M}}

\def\lvn{{\mathsf s}}
\def\lvN{{\mathsf S}}

\def\bc{{\boldsymbol c}}
\def\bC{{\boldsymbol C}}
\def\oba{\overline{{\boldsymbol a}}}
\def\uba{\underline{{\boldsymbol a}}}

\def\barsigma{\bar{\sigma}}

\def\tbN{\Tilde \bN}
\def\dv{{D}}

\def\tbV{\Tilde \bV}
\def\hiota{{\hat \iota}}
\def\biota{{\boldsymbol \iota}}
\def\hbiota{{\hat {\boldsymbol \iota}}}

\def\bzeta{{\boldsymbol \zeta}}
\def\hbzeta{{\hat {\boldsymbol \zeta}}}
\def\oproj{{\overline \proj}}
\def\barHop{\bar{\Hop}}
\def\barUop{\bar{\Uop}}
\def\barU{\bar{U}}
\def\barH{\bar{H}}
\def\ind{\mathbbm{1}}

\def\tC{\Tilde C}
\def\tQ{\Tilde Q}
\def\balpha{\boldsymbol{\alpha}}
\def\bgamma{\boldsymbol{\gamma}}
\def\cU{\mathcal{U}}
\def\tbC{\Tilde \bC}
\def\tba{\Tilde \ba}
\def\tbeta{\Tilde \beta}
\def\tbbeta{\Tilde \bbeta}
\def\boldf{\boldsymbol{f}}
\def\bXi{\boldsymbol{\Xi}}
\def\cB{\mathcal{B}}
\def\MP{{\rm MP}}
\def\complex{\mathbbm{C}}
\def\Im{{\rm Im}}
\def\tbM{\Tilde \bM}

\def\sR{\mathsf R}
\def\sV{\mathsf V}
\def\sB{\mathsf B}

\def\obR{\overline{\bR}}
\def\obM{\overline{\bM}}
\def\wbM{\widetilde{\bM}}
\def\tbR{\widetilde{\bR}}
\def\tbM{\widetilde{\bM}}

\def\ulambda{\overline{\lambda}}
\def\hbtheta{\hat \btheta}
\def\rr{{\rm r}}

\def\rC{\textcolor{red}{C}}

\def\rSQ{{\rm SQ}}

\def\rdc{{\rm dc}}
\def\rmc{{\rm mc}}
\def\cY{\mathcal{Y}}
\def\cZ{\mathcal{Z}}
\def\rdeg{{\rm deg}}


\def\dom{{\rm dom}}
\def\prox{{\rm prox}}
\def\hE{\widehat{\E}}
\def\okappa{\overline{\kappa}}
\def\otau{\overline{\tau}}
\def\br{{\boldsymbol r}}
\def\bGamma{{\boldsymbol \Gamma}}
\def\cJ{\mathcal{J}}
\def\oxi{\overline{\xi}}
\def\hbalpha{\hat{\balpha}}
\def\sfG{\textsf{G}}
\def\sfMG{\textsf{MG}}
\def\obz{\overline{\bz}}
\def\obZ{\overline{\bZ}}
\def\obg{\overline{\bg}}
\def\obG{\overline{\bG}}
\def\tbU{\Tilde{\bU}}
\def\obx{\overline{\bx}}
\def\ox{\overline{x}}



\def\tC{\Tilde C}
\def\tQ{\Tilde Q}
\def\balpha{\boldsymbol{\alpha}}
\def\bgamma{\boldsymbol{\gamma}}
\def\cU{\mathcal{U}}
\def\tbC{\Tilde \bC}
\def\tba{\Tilde \ba}
\def\tbeta{\Tilde \beta}
\def\tbbeta{\Tilde \bbeta}
\def\boldf{\boldsymbol{f}}
\def\bXi{\boldsymbol{\Xi}}
\def\cB{\mathcal{B}}
\def\MP{{\rm MP}}
\def\complex{\mathbbm{C}}
\def\Im{{\rm Im}}
\def\tbM{\Tilde \bM}

\def\sR{\mathsf R}
\def\sV{\mathsf V}
\def\sB{\mathsf B}

\def\ulambda{\overline{\lambda}}
\def\hbtheta{\hat \btheta}
\def\oPhi{\overline{\Phi}}
\def\sfPhi{\mathsf \Phi}

\def\hbSigma{\hat{\bSigma}}
\def\sfC{{\sf C}}
\def\sfc{{\sf c}}
\def\sfD{{\sf D}}
\def\sfM{{\sf M}}
\def\rmI{{\rm I}}
\def\rmII{{\rm II}}
\def\obQ{\overline{\bQ}}
\def\tS{\widetilde{S}} 
\def\tbS{\widetilde{\bS}}  
\def\obtheta{\overline{\btheta}}
\def\onu{\overline{\nu}}
\def\oT{\overline{T}}
\def\sL{\mathsf{L}}
\def\bq{\boldsymbol{q}}
\def\og{\overline{g}}
\def\oq{\overline{q}}
\def\ske{{\sf ske}}
\def\bs{{\boldsymbol s}}
\def\obD{\overline{\bD}}
\def\osfD{{\overline{{\sf D}}}}
\def\sflf{{\sf leaf}}
\def\sfT{{\sf T}}
\def\sfG{{\sf G}}
\def\bsfT{{\boldsymbol \sfT}}
\def\bsfG{{\boldsymbol \sfG}}
\def\obi{\overline{\bi}}
\def\obsfT{\overline{\bsfT}}
\def\obsfG{\overline{\bsfG}}
\def\oi{\overline{i}}
\def\osfT{\overline{\sfT}}
\def\osfG{\overline{\sfG}}
\def\sfH{{\sf H}}
\def\tbD{\widetilde{\bD}}
\def\polylog{\text{polylog}}
\def\tcL{{\widetilde{\cL}}}
\def\tsL{{\widetilde{\sL}}}

\def\seff{{\sf eff}}
\def\sG{\mathsf{G}}
\def\sKL{\mathsf{KL}}
\def\oevn{\overline{\evn}}
\def\obeta{\overline{\beta}}
\def\oC{\overline{C}}

\def\tnu{\Tilde{\nu}}
\def\hbSigma{\widehat{\bSigma}}
\def\tmu{\Tilde{\mu}}
\def\sK{{\sf K}}
\def\sA{{\sf A}}
\def\tPhi{\widetilde{\Phi}}
\def\obF{\overline{\bF}}
\def\oboldf{\overline{\boldf}}
\def\tr{\widehat{r}}
\def\hxi{\hat{\xi}}
\def\hr{\widehat{r}}
\def\hrho{\widehat{\rho}}
\def\trho{\widetilde{\rho}}
\def\tcA{\widetilde{\cA}}
\def\obv{\overline{\bv}}
\def\tsB{\widetilde{\sB}}
\def\tbG{\widetilde{\bG}}


\newcommand{\G}{\mathbf{G}}
\newcommand{\GT}{\mathbf{G}^\top}
\newcommand{\bet}{\boldsymbol{\beta}}
\newcommand{\U}{\mathbf{U}}
\newcommand{\V}{\mathbf{V}}
\newcommand{\D}{\mathbf{D}}
%\newcommand{\R}{\mathbb{R}}
%\newcommand{\E}{\mathbb{E}}
\newcommand{\Sph}{\mathbb{S}}
%\newcommand{\I}{\mathbb{I}}
%\newcommand{\Pr}{\mathbb{P}}
%\newcommand{\bx}{\boldsymbol{x}}
%\newcommand{\bw}{\boldsymbol{w}}
%\newcommand{\bz}{\boldsymbol{z}}
\newcommand{\bblV}{{\color{blue}\bV}}
% Title. Note the optional short title for running heads. In the interest of anonymization, please do not include any acknowledgements.
\begin{document}

\title{Envious Explore and Exploit}

% Anonymized submission.
\author{
Omer Ben{-}Porat%
\thanks{%
    {Technion---Israel Institute of Technology (\url{omerbp@technion.ac.il})}, corresponding author}
\and Yotam Ganfi%
\thanks{%
    {Weizmann Institute (\url{yotam.gafni@gmail.com})}}
\and Or Markovetzki%
\thanks{%
    {Technion---Israel Institute of Technology (\url{ormar@campus.technion.ac.il})}}
}
\maketitle
% Abstract. Note that this must come before \maketitle.
\begin{abstract}
Explore-and-exploit tradeoffs play a key role in recommendation systems (RSs), aiming at serving users better by learning from previous interactions. Despite their commercial success, the societal effects of explore-and-exploit mechanisms are not well understood, especially regarding the utility discrepancy they generate between different users.
In this work, we measure such discrepancy using the economic notion of envy. We present a multi-armed bandit-like model in which every round consists of several sessions, and rewards are realized once per round. We call the latter property \emph{reward consistency}, and show that the RS can leverage this property for better societal outcomes. On the downside, doing so also generates envy, as late-to-arrive users enjoy the information gathered by early-to-arrive users. We examine the generated envy under several arrival order mechanisms and virtually any \emph{anonymous} algorithm, i.e., any algorithm that treats all similar users similarly without leveraging their identities. We provide tight envy bounds on uniform arrival and upper bound the envy for nudged arrival, in which the RS can affect the order of arrival by nudging its users. Furthermore, we study the efficiency-fairness trade-off by devising an algorithm that allows constant envy and approximates the optimal welfare in restricted settings. Finally, we validate our theoretical results empirically using simulations.
\end{abstract}


% Paper body
\section{Introduction}
\label{sec:intro}

\begin{figure*}[tb]
    \centering
    \includegraphics[width=0.848\linewidth]{figs/circuitnn.pdf} 
    \caption{Illustration of differentiable CircuitNN. CircuitNN is designed based on differentiable NAND gates. After DAS is guided by PI and PO pairs of the truth table, CircuitNN can get the precise circuit architecture logic equivalent to the truth table.}
    \label{fig:circuitnn}
\end{figure*}

% 1. Describe the importance of logic synthesis
% 2. Existing Problems
% (a) Neural Architecture Search: Unstable, Predefined Setting, etc.
% (b) Circuit Generation: Probabilistic Model, Logic Equivalence

With the rapid advancement of technology, the scale of integrated circuits (ICs) has expanded exponentially. 
This expansion has introduced significant challenges in chip manufacturing, particularly concerning power and area metrics.
A primary objective in IC design is achieving the same circuit function with fewer transistors, thereby reducing power usage and area occupancy.

Logic synthesis~\cite{hachtel2005logicsynth}, a critical step in electronic design automation (EDA), transforms behavioral-level circuit designs into optimized gate-level circuits, ultimately yielding the final IC layout. 
The primary goal of logic synthesis is to identify the physical implementation with the fewest gates for a given circuit function. 
This task constitutes a challenging NP-hard combinatorial optimization problem. 
Current logic synthesis tools~\cite{brayton2010abc, wolf2013yosys} rely on human-designed heuristics, often leading to sub-optimal outcomes.

Differentiable architecture search (DAS) techniques~\cite{liu2018darts, chu2020darts} offer novel perspectives on addressing challenges in this problem.
Circuit functions can be represented through truth tables, which map binary inputs to their corresponding outputs. 
Truth tables provide a precise representation of input-output relationships, ensuring the design of functionally equivalent circuits.
Inspired by this, researchers~\cite{deepmind2024ai4sys, wang2024tnet} have begun exploring the application of DAS to synthesize circuits directly from truth tables.
Specifically, \citet{deepmind2024ai4sys} proposed CircuitNN, a framework that learns differentiable connection structures with logic gates, enabling the automatic generation of logic circuits from truth tables.
This approach significantly reduces the complexity of traditional circuit generation. 
Building on this, \citet{wang2024tnet} introduced T-Net, a triangle-shaped variant of CircuitNN, incorporating regularization techniques to enhance the efficiency of DAS.

Despite these advancements, several challenges remain. 
The computational complexity of DAS grows quadratically with the number of gates, posing scalability issues.
Although triangle-shaped architecture~\cite{wang2024tnet} partially mitigates this problem, redundancy persists. 
%Additionally, DAS is susceptible to converging to local optima, limiting the ability to search architectures that satisfy the given truth tables~\cite{liu2018darts}. 
%Furthermore, hyperparameters (network depth and layer width) require extensive searches, introducing complexity and prolonging the synthesis process. 
Additionally, DAS is susceptible to converging to local optima~\cite{liu2018darts} and hyperparameters (network depth and layer width) require extensive searches. 
The challenges arise from the vast search space in DAS. 
% Even with predefined settings for CircuitNN, finding a configuration that meets the truth table requires extensive trial and error during the DAS process. 
Intuitively, limiting the search space through predefined parameters (network depth, gates per layer, and connection probabilities) can significantly reduce the complexity.

Recent advances~\cite{openai2023gpt4, abramson2024alphafold3, esser2024sd3, li2024mar} in conditional generative models have demonstrated remarkable performance across language, vision, and graph generation tasks. 
Motivated by these developments, we propose a novel approach to circuit generation that generates preliminary circuit structures to guide DAS in generating refined circuits matching specified truth tables. 
Firstly, we introduce CircuitVQ, a tokenizer with a discrete codebook for circuit tokenization. 
Built upon our Circuit AutoEncoder framework~\cite{hou2022graphmae,li2023maskgae,wu2025mgvga}, CircuitVQ is trained through a circuit reconstruction task. 
Specifically, the CircuitVQ encoder encodes input circuits into discrete tokens using a learnable codebook, while the decoder reconstructs the circuit adjacency matrix based on these tokens.
Subsequently, the CircuitVQ encoder serves as a circuit tokenizer for CircuitAR pretraining, which employs a masked autoregressive modeling paradigm~\cite{chang2022maskgit, li2023mage}. 
In this process, the discrete codes function as supervision signals. 
After training, CircuitAR can generate discrete tokens progressively, which can be decoded into initial circuit structures by the decoder of the CircuitVQ. 
These prior insights can guide DAS in producing refined circuits that match the target truth tables precisely.

Our key contributions can be summarized as follows:
\begin{itemize}
\item We introduce CircuitVQ, a circuit tokenizer that facilitates graph autoregressive modeling for circuit generation, based on our Circuit AutoEncoder framework;
\item Develop CircuitAR, a model trained using masked autoregressive modeling, which generates initial circuit structures conditioned on given truth tables;
\item Propose a refinement framework that integrates differentiable architecture search to produce functionally equivalent circuits guided by target truth tables;
\item Comprehensive experiments demonstrating the scalability and capability emergence of our CircuitAR and the superior performance of the proposed circuit generation approach.
\end{itemize}

% Motivation
% (a) Diffusion (Vision, Graph), Autoregressive (Language, Vision)
% (b) Circuit Generation for Predefined Setting
% (c) Neural Architecture Search for Strict Logic Equivalence

% Contribution
% (a) Circuit Tokenizer (new transformer arch, training strategy)
% (b) CircuitAR (train and gen strategies, post-ar strategy)
% (c) Extensive Evaluation including BitD (Bit Distance) for Scalability
 %WIP
To illustrate equilibria and dynamics of performative prediction games, we focus on a scenario in which a \emph{duopoly} of mortgage companies, i.e. banks, compete to sell loans to customers.

\paragraph{Customer Model:} In our game, each bank is trying to attract customers from a given population $\mathcal{P}$. We model this population as comprised of individuals with a single-dimensional type: we denote individual $j$'s type as $y_j \in [0,1]$. For simplicity, we assume that \(y\) represents the customer’s probability of repaying the loan\footnote{In practice, a customer's (normalized) credit score can be interpreted as a noisy observation of $y_j$. This also corresponds to credit scores being \emph{calibrated}.}, i.e., $y_j := \P[Y_j = 1]$, where $Y_j$ is a random variable such that $Y_j = 0$ means that $j$ defaults on their loan, and $Y_j = 1$ means they repay their loan. Customer types in the population are drawn from a known distribution $D_y$ supported on $[0,1]$. 

\paragraph{Game between Banks:} Each Bank \(i \in \{1, 2\}\) selects two parameters \( (\tau_i, \gamma_i) := \theta_i\), where:
\begin{itemize}
    \item \(\tau_i \in \{\tau_l,\tau_h\}\) is the credit score threshold for approving a customer\footnote{We restrict the bank to only pick between two thresholds, $\tau_l$ and $\tau_h$. However, we highlight how our results are affected when we expand the strategy space to $n > 2$ actions in our experiments of Appendix \ref{app:3gamma}.}. Specifically, a customer $j$ with credit score \(y_j\) is approved by Bank $i$ if and only if \(y_j \geq \tau_i\);
    \item \(\gamma_i \in \{\gamma_l, \gamma_h\}\) is the interest rate offered to approved customers.
\end{itemize}
We denote as shorthand the space of allowable thresholds by $\Gamma := [0,1]$ and allowable interests rates by $\Lambda := [0,1]$. %The latter is set without loss of generality---we simply normalize the rates to be at most $1$. 
% {\color{red} Vidya: just thinking about this but is it natural to restrict interest rate to $1$? I don't think it would affect the equilibrium structure of the game but theoretically I think the interest rate could be anything in $[0,\infty)$.} {\color{green} Guanghui: Could we say something like this is without loss of generality} \gua{changed.}\juba{I think we repeated this twice, the next sentence already had this}
The loan amount is normalized to $1$ in the entire paper, without loss of generality; in this case, if a customer chooses Bank $i$, and the customer is approved by the bank at an interest rate of $\gamma_i$, the expected utility for the bank is equal to
\[
(1+\gamma_i)\cdot \P[Y_i = 1]-\P[Y_i = 0] = (1+\gamma_i)y_i-(1-y_i).
\]


%In practice, the credit score \(y\) serves as a noisy observation of the true likelihood of the customer's repayment. 

\paragraph{Banks' Utilities:} For given parameter choices \(\theta_1 = (\tau_1, \gamma_1)\) by Bank 1 and \(\theta_2 = (\tau_2, \gamma_2)\) by Bank 2, a \emph{rational} customer with credit score $y$ acts as follows:

\begin{enumerate}
    \item \textbf{Qualified for a single bank}: 
        \begin{itemize}
        \item If \(\tau_1 \leq y < \tau_2\), the customer goes to Bank 1, as the score qualifies for Bank 1 but not Bank 2. Conversely, if \(\tau_2 \leq y < \tau_1\), the customer chooses Bank 2.
    \end{itemize}
    \item \textbf{Qualified for both banks}:
     \begin{itemize}
        \item If \(\tau_1, \tau_2 \leq y\) and \(\gamma_1 < \gamma_2\), the customer selects Bank 1 for its lower interest rate. Conversely, if \(\gamma_1 > \gamma_2\), the customer chooses Bank 2.
        \item If \(\gamma_1 = \gamma_2\), the customer picks each bank with probability $1/2$. 
    \end{itemize}
    \item \textbf{Unqualified for both banks}:
    \begin{itemize}
        \item If \(y < \tau_1\) and \(y < \tau_2\), the customer is rejected by both banks.
    \end{itemize}
\end{enumerate}

The expected reward for Bank 1, denoted as \(u_1(\theta_1, \theta_2)\), can then be expressed as:
\begin{align}\label{eq:utility}
    u_1(\theta_1, \theta_2) 
    &=  \mathbb{E}_{y \sim D_y} \left[ \mathbb{I}\{\underbrace{\tau_1 \leq y < \tau_2 \ \cup \ (\tau_1, \tau_2 \leq y \ \cap \ \gamma_1 < \gamma_2)}_{\text{accepted by Bank 1}}\} \cdot \big((1+\gamma_1)y - (1-y)\big) \right] \nonumber\\
    & + \frac{1}{2} \mathbb{E}_{y \sim D_y} \left[ \mathbb{I}\{\underbrace{\tau_1, \tau_2 \leq y \ \cap \ \gamma_1 = \gamma_2}_{\text{accepted by both Banks}}\} \cdot \big((1+\gamma_1)y - (1-y)\big) \right].
\end{align}
Note that the problem is \emph{symmetric}, i.e., the utility function for Bank 2 can be derived by swapping the roles of \(\theta_1\) and \(\theta_2\). I.e., $u_2(\theta_1, \theta_2) = u_1(\theta_2, \theta_1)$. 

% If a bank only attracts customers between thresholds $\tau_a$ and $\tau_b$, for $\tau_a<\tau_b$, we call $[\tau_a,\tau_b]$ the \emph{threshold} range for that bank. For example, if Bank $1$ sets a threshold of $\tau_1$, Bank $2$ a threshold of $\tau_2 > \tau_1$, and $\gamma_1 > \gamma_2$, then Bank 1 has a threshold range of $[\tau_1,\tau_2]$, while bank $2$ has a threshold range of $[\tau_2,1]$.
% Note that the parameters set by \emph{both} banks, i.e. $(\theta_1,\theta_2)$ both influence the threshold range for each of Bank 1 and 2.  If $\tau_1>\tau_2$, $\gamma_1>\gamma_2$, then $\tau_a>\tau_b$, and the bank does not attract any customers. 
% {\color{red} is it possible for $\tau_a > \tau_b$, leading to the bank never attracting customers?} \gua{if $\gamma_1>\gamma_2$, $\tau_1>\tau_2$, then it gets no customer. I think it also makes sense.}\juba{I think we said we wanted to delete the discussion of the threshold range, no?}

% \noindent \textbf{Discrete Model}   
% We now present the discrete version of our model, where the interest rates and thresholds are selected from finite sets \(\Gamma\) and \(\Lambda\), respectively, with $\tau\in[0,1], \gamma\in[0,1]$,  for all $\tau\in\Lambda$ and $\gamma\in\Gamma$, \(|\Gamma| = n\) and \(|\Lambda| = m\). Let \(p_1, p_2 \in \Delta(\Gamma \times \Lambda)\) represent the mixed strategies of the two banks, where \(\Delta(\Gamma \times \Lambda)\) denotes the set of probability distributions over the discrete decision space \(\Gamma \times \Lambda\).


% \begin{Remark}
%    Note that our proposed problem can be reformulated as a standard multi-player performative prediction problem \citep{narang2023multiplayer}. However, in our problem, the data distribution faced by each learner breaks the Lipschitzness assumption of previous work~\citep{hardt2023performative,narang2023multiplayer}. A small modification in one of the learner's thresholds can completely change how demand is allocated across both learners, as is often the case in Bertrand-style games. 
% \end{Remark} 

% \gua{I made some changes to Remark 1, please have a look}
\begin{Remark}
   Previous works in multi-learner performative prediction~\citep{narang2023multiplayer} resort to an insensitivity assumption, i.e., the data distribution faced by each player can only changes slightly when the parameters also change slightly; formally, the data distribution faced by each player is Lipschitz in their decisions. This is immediately not true in our setting: the bank slightly changing its parameters can completely changes the demand distribution of customers it faces. Intuitively, this is because of Bertrand-competition-style effects, where if two banks have similar rates, one bank that lowers their rate by a small amount suddenly captures the entire customer demand that is eligible for that rate.%\juba{made further light edits adding intuition}
   
   In Appendix \ref{Appendix:refumulation}, we discuss this problem more carefully by reformulating our problem in the standard multi-learner performative prediction form given by~\citep{narang2023multiplayer}. We show the distribution is not Lipschitz with respect to the parameters, and thus does not satisfy the insensitivity assumption. 
%Prior work~\citep{hardt2023performative,narang2023multiplayer} showed that, for a general multi-agent performative prediction framework to work, insensitivity assumptions are needed: in the \textbf{worst case}, they can construct settings where the insensitivity assumption does not hold and simple dynamics do not converge anymore. We add nuance to this picture. We will show that our dynamics often converge, even absent insensitivity assumptions, highlighting that while the impossibility results of previous work hold in the worst case, they may not hold in the ``average case'' and especially not in problems motivated by applications. In particular, we will show convergence to a variety of equilibria of our game, and often to symmetric Nash equilibria where insensitivity is immediately violated.
     
\end{Remark}



% \paragraph{Relationship to Performative Prediction} A central point of our work is to highlight that \textcolor{red}{needs writing from intro}. We highlight how our work specifically ties to ``Performative Prediction'' below:


%\textcolor{red}{needs a definition environment}



%Here, \(\E_{\theta_1, \theta_2}\) represents the expected utility of the banks over their respective strategies \((\theta_1, \theta_2)\). These inequalities ensure that neither bank can unilaterally improve its expected utility by deviating from its mixed strategy in the equilibrium.



%and  for all $\tau\in\Gamma$, we have $\tau\in\Lambda$, $(\tau,\gamma)\in[0,1]^2$. Let $\Gamma\times\Lambda$
%In this paper, we focus on the most fundamental case, where there are two choices for each parameter: $0\leq\tau_{\ell}<\tau_{h}\leq 1$, and $0\leq \gamma_{\ell}< \gamma_{h}\leq 1$. In this case, the utility for each pair of decisions forms a $4\times4$ matrix (given in Table \ref{tab:my-table}). We consider the canonical case where $\tau_{\ell}=\frac{1}{2+\gamma_{h}}$, and $\tau_{h}=\frac{1}{2+\gamma_{\ell}}.$ Note that these are natural choices for the thresholds, in the sense that, if there is only one bank and the interest rate is set to be $\gamma$, then $\frac{1}{2+\gamma}$ is the optimal threshold corresponding to the fixed $\gamma$.


%and the thresholds are chosen in $\Lambda=\{\tau^{(1)},\dots,\tau^{(m)}\}$. Here, we only assume that, for each $\gamma\in\Gamma$, there at least exist one $\tau\in\Lambda$ such that $f(\gamma,\tau,1)>0$. Note that this is a very minor assumption, in the sense that, if for a $\gamma$ such that $f(\gamma,\tau,1)<0$ for all $\tau\in\Lambda$, then adopting this decision will lead to negative utility regardless of the opponent's decision, and thus is not an interesting case. 

%\textcolor{red}{The model section is missing the dynamic version of the game. We should clearly define the one-shot and the dynamic game}
% we only considered one-shot case in our paper


 
\section{Uniform Arrival}
\label{sec:uniform}
In this section, we assume agents arrive uniformly; that is, the uniform arrival function, $\uniord$, picks in every round $t$ a permutation $\ordv_t$ uniformly at random from the set of all distributions $\textnormal{Perm}([N])$. We start with an insight into reward discrepancies for uniform arrival. Then, in Subsections~\ref{subsec:uni upper} and~\ref{subsec:uni lower}, we provide upper and lower bounds on the expected envy, respectively. %and high probability envy.


Recall that $\dif^t_{i,j}$ is the signed reward discrepancy between agents~$i$ and~$j$ at round~$t$. This quantity depends on both the algorithm, the reward distribution, and the arrival function. 
For example, if an algorithm always pulls $a_1$, then $\dif^t_{i,j} = 0$ almost surely for every $i,j,t$, since all agents receive the same reward. By contrast, for more general algorithms, $\dif^t_{i,j}$ can vary almost arbitrarily, reflecting different approaches to exploration and exploitation over time. Under uniform arrival, each round's agent order is chosen uniformly at random from all permutations of $[N]$, ensuring identical treatment of all agents in expectation.  Consequently, the reward discrepancies exhibit a symmetry:
\begin{remark}\label{remark: symmetric dif}
Under $\uniord$ and any algorithm, the random variables $\dif^t_{i,j}$ in round $t$ are identically distributed for all pairs $(i,j)\in [N]^2$.
\end{remark}

Due to Remark~\ref{remark: symmetric dif}, we simplify the notation in this section and use $\dif^t$ to denote the reward discrepancy distribution between any two agents. Clearly, $\E{\dif^t}=0$. Note that the random variables $\adift{i_1}{j_1}$ and $\adift{i_2}{j_2}$ are still correlated for different pairs of agents $(i_1,j_1)$ and $(i_2,j_2)$. For example, $\adift{i}{j}= -\adift{j}{i}$ for all $i,j$. 
Moreover, $\dif^t$ might depend on the rewards obtained in previous rounds, reflecting temporal correlations that are typical in multi-armed bandit settings (e.g., stationary rewards).  As a result, a high reward discrepancy in earlier rounds might suggest a more thorough exploration, leading to a low reward discrepancy in later rounds. Consequently, we cannot assume that $(\dif^t)_{t=1}^T$ are independent.

%Moreover, $\dif^t$ might depend on the rewards obtained in previous rewards, which is often the case in multi-armed bandit settings: The algorithm explores in earlier rounds to exploit in later rounds. As a result, a high reward discrepancy in earlier rounds might suggest a more thorough exploration, leading to a low reward discrepancy in later rounds. Consequently, we cannot assume that $(\dif^t)_{t=1}^T$ are independent.

%\omer{I might want to put the below definition in a dedicated subsection, in the meantime it parks here in iffalse}
\iffalse
explore-first. We say that an algorithm satisfies the explore-first property if, for every round $t$, there exists a session $q(t)$ such that the algorithm behaves as follows:
\begin{itemize}
    \item Before session $q(t)$, the algorithm pulls only unobserved arms.
    \item From session $q(t)$ onward, the algorithm exclusively pulls a single already-observed arm.
\end{itemize}
Formally, 
\begin{definition}[Explore-First Algorithm]
\label{def: threshold}
An algorithm satisfies the \emph{explore-first} property if, for every round $t$, if the algorithm pulls arm $a_i$ for some $i \in [K]$ in sessions $q$ and $q'$ with $q < q'$, then the algorithm pulls $a_i$ in every session $w$ such that $w > q'$.
\end{definition}
Notice that every explore-first algorithm can only commit to one arm in each round, but that arm could vary from round to round and depend on the realized rewards. We make 
\fi


    % in every round $t$ it explores only in the first $K$ sessions and then commit to one arm.
    % I.e., if the algorithm pulls arm $a_i$ in session $q_1$ of round $t$, $K+1 \leq q_1 \leq N$, then
    % \[
    % \abs{\left\{ w \mid 1\leq w\leq K, \at{(w)} = a_i \right\} }\neq 0 ,
    % \]
    % and for every other session $q_2$ of round $t$ s.t. $K+1 \leq q_2 \leq N$, $\at{(q_2)} = a_i$.
% \omer{In the below, you don't really use the fact that the algorithm has a threshold structure. For instance, in Prop 3.2.2, you only assume that the algorithm is rational:
% \begin{definition}[Threshold Algorithm]
% \label{def: rational}
%     An algorithm is called \emph{rational} if, in every session, it either picks a previously unexplored arm or pulls an already explored arm with the best-observed reward.
% \end{definition}
% }

\subsection{Upper Bound}\label{subsec:uni upper}
Our first result is an upper bound on the expected envy.
\begin{theorem}\label{thm: uni upper-bound}
When executing any algorithm, it holds that
\[\E{\max_{1\leq t \leq T} \env^t (\uniord)} \leq 2\sqrt{\ln{(N)} \sum^{T}_{t=1}{\var{\dift}} }.\]
\end{theorem}
\begin{proof}[Proof of Theorem~\ref{thm: uni upper-bound}]
The proof leverages properties of martingales and subgaussian random variables. Since the rewards are bounded, the resulting envy is also bounded and therefore subgaussian with some parameter. However, because envy is a sum of \emph{possibly dependent} random variables, additional arguments are needed to obtain a sharper subgaussian parameter. The following proposition is the primary technical ingredient in achieving this refinement.
\begin{proposition}\label{prop:envy is good SG}
For any two arbitrary agents $i,j\in [N]$, $\max_{1\leq t \leq T} \env^t_{i,j}$ is $\sqrt{\sum^T_{t=1}{\var{\dift}}}$-subgaussian.
\end{proposition}
We prove this proposition in \ifapp{Section~\ref{appendix:uni}}{the appendix}. Equipped with Proposition~\ref{prop:envy is good SG}, we can use the following well-known property of the maximum of subgaussian random variables.
\begin{claim}\label{claim: sg max}
    Let $Y_1, \ldots, Y_n$ be (possibly dependent) $\sigma$-subgaussian random variables. It holds that
    \[
    \E{\max_{i \in [n]}{\{Y_i\}} } \leq \sqrt{2\sigma^2 \ln{(n)} }.
    \]
\end{claim}
Claim~\ref{claim: sg max} is folklore, but we include its proof in \ifapp{Section~\ref{appendix:uni}}{the appendix} for completeness. Using this claim along with Proposition~\ref{prop:envy is good SG}, we obtain 
\[
\E{\max_{1\leq t \leq T} \env^t} =
\E{\max_{i,j \in [N]}{\left\{ {\max_{1\leq t \leq T} \env^t_{i,j}} \right\}}} \leq
\sqrt{2 \ln{(N^2)} \sum^{T}_{t=1}{\var{\dift}} } =
2\sqrt{\ln{(N)} \sum^{T}_{t=1}{\var{\dift}} }.
\]
This concludes the proof of Theorem~\ref{thm: uni upper-bound}.
\end{proof}
\subsubsection{Refining the Variance of Reward Discrepancy}
Although Theorem~\ref{thm: uni upper-bound} contains a factor of $\ln N$, it does not explicitly capture the influence of $N$ or $K$, as these parameters are embedded within $\var{\dift}$. To clarify this further, we refine our analysis by focusing on the subclass of algorithms known as \emph{explore-first} algorithms. 

We say that an algorithm satisfies the explore-first property if, for every round $t$, once it repeats an arm selection (i.e., chooses $a_i$ in two sessions), it commits to $a_i$ for all subsequent sessions in that round. More formally, 
\begin{definition}[Explore-first]\label{def:explore-first}
An algorithm is \emph{explore-first} if for every $t$ there exists a session $q(t)$ such that:
\begin{itemize}
    \item Before session $q(t)$, the algorithm pulls only unobserved arms.
    \item From session $q(t)$ onward, the algorithm exclusively pulls a single already-observed arm.
\end{itemize} 
\end{definition}
Observe that an explore-first algorithm commits to a single arm in each round, but this arm may vary across rounds and depend on the observed rewards. The explore-first property is natural and applies, e.g., to algorithms that use a threshold to start exploitation (like Algorithm~\ref{alguni}). Relying on Definition~\ref{def:explore-first}, we can express $\var{\dift}$ in terms of $N$ and $K$.
\begin{proposition}\label{thm: threshold var}
    When executing any explore-first algorithm with $K$ arms and $N$ agents, it holds that $\var{\dift} \leq \min\left\{1, \frac{(2N-K)(K-1)}{N(N-1)} \right\}$.  
\end{proposition}
The intuition behind this proposition is as follows. Since at most $K$ agents are exploring, the remaining $N-K$ agents are exploiting and receive identical rewards, so no discrepancy occurs between any two exploiting agents. As $N$ grows relative to $K$, most agent pairs consist of two exploiting agents, leaving only a small fraction of pairs---those involving at least one exploring agent---that can contribute to the variance. This decrease in the number of discrepancy-generating pairs results in a tighter overall bound on $\var{\dift}$.

Combining Proposition~\ref{thm: threshold var} with Theorem~\ref{thm: uni upper-bound}, we get the following corollary.
\begin{corollary} \label{thm: sqrt TK N}
    When executing any explore-first algorithm with $K$ arms and $N$ agents, the expected envy is $\E{\max_{1\leq t \leq T} \env^t  (\uniord)}=O\left( \sqrt{\frac{TK\ln(N)}{N}} \right)$.
\end{corollary}

\subsection{Lower Bound}\label{subsec:uni lower}
\label{sec: lb}

Next, we move to develop a lower bound on the expected envy. As is apparent, not all algorithms generate envy. To demonstrate, consider the following complementary cases:
\begin{enumerate}
    \item An instance in which all arms have the same \emph{deterministic} reward. In this case, any algorithm produces zero envy among all agents as all receive the same reward.
    \item An algorithm that pulls the same arm in all sessions. Even if rewards are stochastic, the algorithm does not generate any envy due to reward consistency.
\end{enumerate}
These examples show that envy will not accumulate if the instance or the algorithm are degenerate, motivating to focus on \emph{executions}: A pair of algorithm and reward distributions. In what follows, we characterize a class of non-degenerate executions for which envy always accumulates. we call such compositions \emph{sufficiently random executions}.
Formally,
\begin{definition}[Sufficiently Random]\label{def: sufficiently random}
    An execution is called \emph{sufficiently random} if it holds that
\begin{equation}\label{eq:def suff}
\sum_{t=1}^T{\var{\dift}} \geq \sqrt{T}.   
\end{equation}
%\omer{I dont need $c_0$, I belive. The old version is still here.}
    % An execution is called \emph{sufficiently random} if it there exists $c_0 > 0$ and $T_0 \in \mathbb N$ such that for all $T > T_0$, it holds that
    % \[
    % \sum_{t=1}^T{\var{\dift}} \geq c_0\sqrt{T}.
    % \]    
\end{definition}
In other words, an execution is sufficiently random if the average variance of the reward discrepancy between two agents in a round is greater or equal to $\frac{1}{\sqrt{T}}$. To further illustrate, we provide the following example. 
\begin{example}\label{example: uni suff}
Recall the execution in Example~\ref{example 1}, with $K=2$ arms with  $\uni{0,1}$ rewards, $N=2$ agents, and Algorithm~\ref{alguni}. As we formally show in \ifnum\Includeappendix=0{the appendix}\else{Claim~\ref{claim: uni suff}}\fi, it holds that $\var{\dift} = \frac{1}{12}$; thus, as long as $T \geq 144$, we have $\sum_{t=1}^T{\var{\dift}} = \frac{T}{12}  {\geq } \sqrt{T}
$; hence, Inequality~\eqref{eq:def suff} holds and this execution is sufficiently random. 
\end{example}
Intuitively speaking, any ``reasonable'' algorithm and reward distributions form a sufficiently random execution. As long as the reward distributions are constant w.r.t. the horizon $T$ and the algorithm conducts enough exploration, the execution will be sufficiently random. To demonstrate cases that are insufficiently random, consider the following example.
\begin{example}\label{example: ber suff}
Assume $K$ i.i.d. arms with rewards distributed $Bernoulli(p)$ for some $p \in (0,1)$, and $N$ agents for $N\geq 2$. We focus on the socially optimal algorithm: It picks fresh arms until a reward of $1$ is obtained and exploits it afterward for all the remaining agents. We show in \ifnum\Includeappendix=0{the appendix}\else{Claim~\ref{claim:example ber suff}}\fi~ that $\var{\Delta^t} \geq \frac{2p(1 - p)}{N}$ and $\var{\Delta^t} \leq  2pK$. 


Consequently, as long as  $\frac{2p(1 - p)}{N} \geq \frac{1}{\sqrt T}$ holds, i.e., for $T \geq \frac{N^2}{4p^2(1-p)^2}$, the execution is sufficiently random. However, if $2pK <\frac{1}{\sqrt T}$, which is the case if $p<\frac{1}{2K\sqrt T}$, the execution is insufficiently random.
\end{example}
Next, we derive a lower bound on the envy for sufficiently random executions.
%%%%%%%%%%%%%%%%%%%%%%%%%%%%%%%%%%%%%%%%%%%%%%%%%%%%%%
\begin{theorem}\label{thm: uni lower-bound}
Any algorithm as part of sufficiently random execution generates an envy of
\[\E{\max_{1\leq t \leq T} \env^t  (\uniord)} \geq c\sqrt{ \sum^{T}_{t=1}{\var{\dif^t}}},\]
where $c>0$ is a global constant that does not depend on the instance. 
\end{theorem}

\begin{proof}[Proof of Theorem~\ref{thm: uni lower-bound}]
    Noticeably, we can bound $\env^t$ using the envy between two specific agents.
    That is,
    \begin{align}\label{thm: mp uni lower-bound 1}
        \E{\max_{1\leq t \leq T} \env^t} =
        \E{\max_{i,j \in [N]}{\left\{ \max_{1\leq t \leq T} \env^t_{i,j} \right\}}} \geq
        \E{\max_{1\leq t \leq T} \abs{\env^t_{1,N}}} =
        %\E{ \max_{1\leq t \leq T}\abs{\sum^T_{t=1}{\adift{1}{N}}} } =
        \E{\max_{1\leq t \leq T}\abs{\sum^T_{t=1}{\dift}} },
    \end{align}
    where the last inequality is due to  Remark~\ref{remark: symmetric dif}. To express $\E{\max_{1\leq t \leq T}\abs{\sum^T_{t=1}{\dift}} }$ as a function of $\sum_t\var{\dift}$, we can lower bound its quadratic variation as we have done in the proof of the upper bound. Next, 
    we use the Burkholder-Davis-Gundy inequality~\cite{davis1970intergrability}. We present a simplified version of the theorem, as we are only interested in the special case of a discrete-time martingale with $L_1$ norm. %\footnote{The reader may wonder why we do not use the more intuitive and popular Marcinkiewicz–Zygmund (MZ) inequality~\cite{chow2003probability, ibragimov1999analogues}. The MZ inequality assumes independent random variables, while the sequence $({\dift})_t$ exhibits dependence, as learning algorithms can base actions on past rounds.}
    \begin{theorem}\label{thm:BDG}[Burkholder-Davis-Gundy inequality]
    Let $\{M^t\}_{t \geq 0}$ be a discrete-time martingale with $M_0 = 0$. There exist positive constants $A_1$ and $B_1$ such that 
    \[ 
    A_1 \, \E{\sqrt{\sum_{t=1}^T (M^t - M^{t-1})^2}}\leq \E{\max_{0 \leq t \leq T} \abs{M^t}}\leq B_1 \, \E{\sqrt{\sum_{t=1}^T (M^t - M^{t-1})^2}}.
    \]    
    \end{theorem}
    As we formally show in \ifnum\Includeappendix=0{the appendix}\else{Observation~\ref{envy is martingale}}\fi, the sequence $(\env^t_{1,N})_t$ forms a martingale; hence, an immediate corollary from Theorem~\ref{thm:BDG} and Inequality~\eqref{thm: mp uni lower-bound 1} is that\footnote{The reader might be tempted to use this theorem for the upper bound as well, obtaining a straightforward bound without additional intricate arguments. However, the maximal envy $\env^T$ is not a martingale, so we can only apply it to the envy between two agents. Replacing the $\max$ operator with a summation results in an upper bound of $O\left(N^2 \sqrt{\sum_{l=1}^{T} \var{\dift}}\right)$, which includes an  additional multiplicative factor of $\nicefrac{N^2}{\ln^2 N}$ to the bound of Theorem~\ref{thm: uni upper-bound}.}
    \begin{align}\label{eq: gdgggsds}
    \E{\max_{1\leq t \leq T} \env^t} \geq A_1 \E{ \sqrt{\sum^T_{t=1}{ (\dift)^2}} }.
    \end{align}
    
    %%%%%%%%%%%%%%%%%%%%%%%%%%%%%%%%%%%%%%%%%%%%%%%%%%%%%%%%%%%%%%%%%%%%%
    % The proof uses heavily the following theorem.
    % \begin{theorem}[Marcinkiewicz–Zygmund Inequality \cite{chow2003probability, ibragimov1999analogues}]\label{thm: MZ}
    % Let $Y_1, \dots, Y_n$ be independent random variables with $\E{Y_i} = 0$ and $\E{|Y_i|^p} <\infty$ for $1\leq p < \infty$. Then there exist global constants $A_p, B_p$ such that
    % \[
    %     {\displaystyle A_{p}\mathbb E\left[\left(\sum _{i=1}^{n}\left\vert Y_{i}\right\vert ^{2}\right)_{}^{p/2}\right]\leq \mathbb E\left[\left\vert \sum _{i=1}^{n}Y_{i}\right\vert ^{p}\right]\leq B_{p}\mathbb E\left[\left(\sum _{i=1}^{n}\left\vert Y_{i}\right\vert ^{2}\right)_{}^{p/2}\right]}.
    % \]
    % \end{theorem}
    % Due to the algorithm being symmetric, the series $\dift$ is a series of centered random variables (\autoref{thm: sum dif sg}) with finite moments (as rewards are bounded).
    % Additionally, the memory-freeness property implies the random variables are independent.
    % Thus, we can apply \autoref{thm: MZ} on the series $\dift$. Picking $p=1$ and plugging the theorem into Inequity~\eqref{thm: mp uni lower-bound 1}, we get
    % \begin{align*}
    %     \E{\env^T} \geq
    %     \E{ \abs{\sum^T_{t=1}{\dift}} } \geq
    %     A_1 \E{ \sqrt{\sum^T_{t=1}{ \dift^2}} }.
    % \end{align*}
    Next, we use the following auxiliary proposition, which provides a reverse Jensen's-like inequality of the square root function.
    \begin{proposition}\label{thm: board}
        Let $Y$ be a non-negative random variable with a finite second moment. It holds
        \begin{align*}
            \E{\sqrt{Y}} \geq \sqrt{\E{Y}}\left(1- \frac{\var{Y}}{2\E{Y}^2}\right).
            \end{align*}
    \end{proposition}
    We prove Proposition~\ref{thm: board} in \ifnum\Includeappendix=0{the appendix}\else{Appendix~\ref{appendix:uni}}\fi. Applying Proposition~\ref{thm: board} with $Y = \sum^T_{t=1}{ \dift^2}$ to Inequality~\eqref{eq: gdgggsds},
    \begin{align}\label{thm: mp uni lower-bound 2}
        \E{\max_{1\leq t \leq T} \env^t} \geq
        A_1 \E{ \sqrt{\sum^T_{t=1}{ \dift^2}} }\geq
        A_1 \sqrt{\E{ \sum^T_{t=1}{ \dift^2}} }
        \left(
        1 - \frac{\var{ \sum^T_{t=1}{ \dift^2} } }{2 \E{ \sum^T_{t=1}{ \dift^2} }^2 }
        \right) .
    \end{align}
    Next, due to our assumption that the execution is sufficiently random, we get  
    \begin{align}\label{thm: mp uni lower-bound 3}
        2 \left(\E{ \sum^T_{t=1}{ \dift^2} }\right)^2 =2 \left( \sum^T_{t=1}{ \E{ \dift^2} - \E{ \dift}^2}\right)^2 = 2 \left( \sum^T_{t=1}{\var{  \dif^t } }\right)^2 \geq 2T.
    \end{align}
    Furthermore, notice that $\var{ \sum^T_{t=1}{ \dift^2} } \leq T$ as the discrepancies $(\dift)_t$ are supported in the $[-1, 1]$ segment. Combining this fact with Inequality~\eqref{thm: mp uni lower-bound 3}, 
    \begin{align}\label{eq:m,bnhjikw}
    \frac{\var{ \sum^T_{t=1}{ \dift^2} } }{2 \E{ \sum^T_{t=1}{ \dift^2} }^2 }\leq \frac{T}{2T}=\frac{1}{2}.
    \end{align}
    Plugging Inequality~\eqref{eq:m,bnhjikw} to Inequality~\eqref{thm: mp uni lower-bound 2}, we conclude that 
    % \begin{align*}
    % \E{\env^T}
    % \geq
    % A_1 \sqrt{\E{ \sum^T_{t=1}{ \dift^2}} }
    % \left(1 - \frac{ T }{2 \sqrt{T}^2 }\right),
    % \end{align*}
    % where the last inequity holds due to the fact that 
    
    % and due to the execution being sufficiently random, i.e., there exists some $T_0$ s.t. if $T > T_0$ then $\sum^T_{t=1}{\var{\dift}}\geq1\cdot\sqrt{T}$ (\autoref{def: sufficiently random}).
    % hence,
    \[
    \E{\max_{1\leq t \leq T} \env^t} \geq
    A_1 \sqrt{\E{ \sum^T_{t=1}{ \dift^2}} }
    \left(1 -\frac{1}{2}\right)
    =
    \frac{A_1}{2}\sqrt{\sum^T_{t=1}{ \var{ \dift }}},
    \]
    where the last inequality is again due to $\E{\dift}=0$. This concludes the proof of Theorem~\ref{thm: uni lower-bound}.
\end{proof}
Theorems~\ref{thm: uni lower-bound} and~\ref{thm: uni upper-bound} imply the following corollary.
\begin{corollary}\label{cor: uni-envy}
Any algorithm as part of sufficiently random execution generates an envy of $\E{\max_{1\leq t \leq T} \env^t  (\uniord)} = \Theta\left(\sqrt{ \sum^{T}_{t=1}{\var{\dif^t}} }\right)$.  
\end{corollary}
Before we complete the section, we remark that high envy may still arise under insufficiently random executions. Indeed, although Definition~\ref{def: sufficiently random} gives a sufficient condition, it is not a necessary one. From a technical perspective, Theorem~\ref{thm: uni lower-bound} applies whenever the left-hand side of Inequality~\eqref{eq:m,bnhjikw} is constant. In certain cases, this allows us to relax the requirement for sufficiently random executions. The following proposition shows that this relaxation holds in the instance described in Example~\ref{example: ber suff}. Recall that the execution in this example is insufficiently random for $p < \frac{1}{2K\sqrt T}$. However, the next proposition shows that the tight bound in Corollary~\ref{cor: uni-envy} can still hold if $p$ is significantly smaller.
\begin{proposition}\label{prop:insufficient}
    For in the execution in Example~\ref{example: ber suff}, as long as    
    $p\in \left[\frac{N}{cT}, 1-\frac{N}{cT}\right]$ for a constant $c \geq 2$ and $T \geq N$, the envy satisfies $\E{\max_{1\leq t \leq T} \env^t  (\uniord)} = \Theta\left(\sqrt{ \sum^{T}_{t=1}{\var{\dif^t}} }\right)$.
\end{proposition}

 
\section{Nudged and Adversarial Arrival}
\label{sec:nudge}
In this section, we address nudged arrival: We assume an exogenous arrival mechanism can influence the order in which agents arrive. Practically, this captures scenarios where the system sends push notifications or otherwise encourages some agents to arrive earlier or later. Our goal is to analyze the envy of arbitrary algorithms without changing the way they select arms. We stress that our analysis still assumes anonymous algorithms: The decision-making process is not affected by agent identities.

% \omer{USE THIS SOMEHOW
% In this section, we examine the envy dynamics of algorithms that \emph{ignore} agent identities \emph{during} the sessions. That is, \omer{we try to minimize envy by making high rewraded agents arrive ealier} the algorithm can affect the arrival order at the beginning of the round, but does not condition the decision on which arm to choose in each session on the arriving agent. \omer{this is good but not perfect}}
Subsection~\ref{subsec:nudged prot} introduces the nudged arrival protocol, $\sugord$, and the accompanying assumptions. Later,  Subsection~\ref{Envy Analysis} presents our main result of the section: An upper bound of $\env^T({\sugord})$ for a broad class of algorithms. Interestingly, we show that the bound depends on the instance parameters but not on the horizon $T$. Finally, we complete this section by adopting a complementary approach, where an adversary can pick the worst arrival order in terms of envy, and show that an $\Omega(T)$ regret is inevitable. For clarity, we remind the reader that for $q,i\in [N]$, $\ordv_t (q)=i$ implies that agent $i$ has arrive in the $q$'th session in round $t$ (similarly for $\ordv_t^{-1}(i)=q$). 
\subsection{Nudged Arrival Protocol}\label{subsec:nudged prot}
\begin{algorithm}[t]
\caption{Nudged Arrival}
\label{alg: sugg arr}
\SetAlgoLined
\LinesNumbered
\KwIn{horizon $T$, nudge parameter $\delta$} \label{line:input}
\For{round $t = 1$ to $T$}{ \label{line:for_loop} 
    let $\sigv_t$ be an $[N]\rightarrow [N]$ mapping such that 
    \[
    R^{t-1}_{\sigma_t(N)} \leq R^{t-1}_{\sigma_t(N-1)} \leq \dots \leq R^{t-1}_{\sigma_t(2)} \leq R^{t-1}_{\sigma_t(1)}.    
    \]
    \label{line:mapping}\\
    sample an arrival order $\ordv_t \sim \sugord(\sigma_t,\delta)$ \label{line:request}
}
\end{algorithm}
We describe the arrival protocol in Algorithm~\ref{alg: sugg arr}. It receives the horizon $T$ and a nudge parameter~$\delta$ as input, and interacts with any recommender algorithm through the horizon. In each round in the for loop of Line~\ref{line:for_loop}, we pick an \textit{ideal permutation} $\sigma_t$ that orders the agents according to their cumulative rewards.  The mapping $\sigma_t$ prioritizes agents according to their cumulative rewards in previous rounds, from the most rewarded one to the least rewarded one. In our notation, $\sigma_t : \{1,\dots,n\} \to [N]$, so that $\sigma_t(i)$ is the name of the agent in position $i$. Hence, $\sigma_t(1)$ is the agent with the highest cumulative reward so far, and $\sigma_t(N)$ is the agent with the lowest. While we assume the agents can be nudged toward this ordering $\sigma_t$, we do not claim that it is implemented or enforced in its exact form. Instead, in Line~\ref{line:request}, we sample an arrival order $\ordv$ from   $\sugord(\sigma_t,\delta)$, which is a distribution over permutations of $N$. Particularly, we assume that $\sugord$ satisfies the \emph{nudged arrival} property.

\begin{property}[Nudged Arrival]\label{prop:nudge}
Given a scalar \( \delta \in (0,1) \) and a mapping \( \sigv : [N] \rightarrow [N] \),  for every two agents \( i,j \) with \( \sigma^{-1}(i) < \sigma^{-1}(j) \) the distribution $\sugord(\sigv,\delta)$ satisfies
\begin{equation}\label{def: nudged} 
\Pr_{\ordv \sim \sugord(\sigma,\delta)}\left(\ordv^{-1}(i) < \ordv^{-1}(j)\right) \geq \frac{1+\delta}{2}. 
\end{equation}  
\end{property}
In other words, Algorithm~\ref{alg: sugg arr} introduces a structured form of randomness in the sequence of agent arrivals. Each round allows for the selection of a permutation $\sigma_t$, representing a preferred (but not guaranteed) ordering of agents. For any two agents $i$ and $j$, if the permutation prioritizes agent $i$'s arrival over agent $j$'s arrival, i.e., $\sigma_t^{-1}(i) < \sigma_t^{-1}(j)$, then agent $i$ is more likely to arrive before agent $j$. Particularly, the probability of agent $i$ preceding agent $j$ in the realized arrival order $\ordv_t$ is at least $\frac{1}{2} + \frac{\delta}{2}$, ensuring a consistent bias $\delta$ toward the preferred ordering. Property~\ref{prop:nudge} is inspired by several well-known models of stochastic ranking, like Mallows model~\cite{mallows1957non}, Plackett-Luce~\cite{marden1996analyzing} and also noisy comparison models~\cite{braverman2007noisy}. In \ifnum\Includeappendix=0{the appendix}\else{Section~\ref{appendix:nudge-models}}\fi, we describe how to derive $\delta$ from each of these models. By specifying Property~\ref{prop:nudge} rather than the full underlying ordering model, we preserve flexibility in how global orderings are derived.  
Next, we illustrate nudged arrival and the role it plays in envy dynamics.
\begin{example}[Nudged Arrival and Envy Dynamics]  \label{example: envy with sugg}
We reconsider the setting of Example~\ref{example 1}, with $K=2$ arms with  $\uni{0,1}$ rewards, $N=2$ agents, Algorithm~\ref{alguni}, but with the nudged arrival ${\sugord}$. To ease readability, we keep using $\env^T$ to denote $\env^T({\sugord})$, omitting the dependence on  ${\sugord}$. Suppose that after $t-1$ rounds, for some arbitrary $t \in [T]$, agent rewards satisfy $\env^{t-1} =\env^{t-1}_{1,2} = R^{t-1}_1 - R^{t-1}_2 > 1$, indicating that agent 2 envies agent 1.


In this scenario, the expected envy after round $t$ is given by:  
{  
\thinmuskip=2mu  
\medmuskip=3mu plus 2mu minus 3mu  
\thickmuskip=4mu plus 5mu minus 2mu  
\begin{align}\label{eq:env_update}  
\mathbb{E}[\env^t \mid \env^{t-1}_{1,2} > 1]  
&= \mathbb{E}\left[\left| \env^{t-1}_{1,2} + \Delta^t_{1,2} \right| \mid \env^{t-1}_{1,2} > 1 \right] = \env^{t-1}_{1,2} + \mathbb{E}\left[\Delta^t_{1,2} \mid\env^{t-1}_{1,2} > 1 \right]. 
\end{align}  
}  



The term $\mathbb{E}[\Delta^t_{1,2} \mid \env^{t-1}_{1,2} > 1 ]$ can be expressed as:  
{  
\thinmuskip=2mu  
\medmuskip=3mu plus 2mu minus 3mu  
\thickmuskip=4mu plus 5mu minus 2mu  
\begin{align}\label{eq:delta_expression}  
\E{\Delta^t_{1,2} \mid \env^{t-1}_{1,2} > 1} &= \E{r_{(2)}^t - r_{(1)}^t \mid \env^{t-1}_{1,2} > 1, \ordv_t = (2,1)}\prb{\ordv_t = (2,1) \mid \env^{t-1}_{1,2} > 1} \nonumber \\  
& \qquad + \E{r_{(1)}^t - r_{(2)}^t \mid \env^{t-1}_{1,2} > 1, \ordv_t  = (1,2)}\prb{\ordv_t = (1,2) \mid \env^{t-1}_{1,2} > 1} \nonumber\\
&= \frac{1}{8} \left[\prb{\ordv_t = (2,1) \mid \env^{t-1}_{1,2} > 1} - \prb{\ordv_t = (1,2) \mid \env^{t-1}_{1,2} > 1} \right],  
\end{align}  
}%
where we have used the fact that $\E{\rt{(2)} - \rt{(1)}} = \frac{1}{8}$ as Subsection~\ref{subsec:information} suggests and the fact that the rewards $r_{(i)}^t$ for $i\in \{1,2 \}$ are independent of the arriving agent's identity.

Under nudged arrival, the ideal permutation is $\sigma_t = (1,2)$, prioritizing agent $1$'s arrival over agent $2$'s arrival;therefore, the permutation $(1,2)$ is more likely than $(2,1)$, resulting in
\[
\prb{\ordv_t = (2,1) \mid \env^{t-1}_{1,2} > 1} - \prb{\ordv_t = (1,2) \mid \env^{t-1}_{1,2} > 1} \leq -\delta.
\]  
Thus, rewriting Equation~\eqref{eq:delta_expression}, we have $\E{\Delta^t_{1,2} \mid \env^{t-1}_{1,2} > 1}  \leq -\frac{\delta}{8}$. To conclude this example, we plug this result into Equation~\eqref{eq:env_update} and obtain
\[
\mathbb{E}[\env^t \mid \env^{t-1}_{1,2} > 1] \leq \env^{t-1}_{1,2}-\frac{\delta}{8},
\]
suggesting that the cumulative envy $\env^t$ is likely to decrease in round $t$ by a non-negligible value.
\end{example}
To be able to analyze envy dynamics and show that it cannot grow too much, we need to have some regularity assumptions on the way algorithms we analyze operate. Indeed, the envy reduction in Example~\ref{example: envy with sugg} relies heavily on the fact that $\E{\rt{(2)} - \rt{(1)}} = \frac{1}{8} > 0$. This inequality ensures that, in expectation, arriving second leads to a higher reward. While we aim to analyze any arbitrary algorithm, we need to ensure that the ordering $\sigma_t$ in Line~\ref{line:mapping} reduces envy in expectation.
The following natural assumption generalizes the behavior of Algorithm~\ref{alguni} in Example~\ref{example: envy with sugg}.\begin{assumption}\label{assumption: nudge alg ref}
In every round $t \in [T]$, the algorithm picks arms so that
\[
    \E{r_{(1)}^t} \leq \E{r_{(2)}^t}\leq \cdots \leq \E{r_{(N)}^t}.
\]
\end{assumption}
To satisfy this assumption,\footnote{In fact, all of our results holds for the much broader case where there exists a permutation $\sigma:N \rightarrow N$ and the algorithm picks arms so that 
$\E{r_{(\sigma(1))}^t} \leq \E{r_{(\sigma(2))}^t}\leq \cdots \leq \E{r_{(\sigma(N))}^t}$. In such a case, we would pick $\sigma_t$ in Line~\ref{line:mapping} of Algorithm~\ref{alg: sugg arr} so that $\left(R^{t-1}_{\sigma_t(\sigma(i))}\right)_i$ is an increasing series.} the algorithm at hand must have some information about the expected rewards; Bayesian information is a sufficient condition, although not necessary.


Furthermore, without loss of generality, we shall assume that $\prb{\Delta^t_{(N),(1)} \neq 0}>0$ in every round $t$. This is indeed without loss of generality, as Assumption~\ref{assumption: nudge alg ref} already guarantees that $\E{\Delta^t_{(N),(1)}}=\E{r^t_{(N)}-r^t_{(1)}} \geq 0 $, and any round in which $\prb{\Delta^t_{(N),(1)} \neq 0}=0$ does not affect envy and can be disregarded. Additionally, to simplify our analysis, we introduce the new notation $\tilde{\dif}$, denoting
\begin{equation}\label{eq def tdif}
\tdif = \min_{1\leq i <j \leq N, t\in [T]:  \prb{\Delta^t_{(j),(i)} \neq 0}>0 }\left\{ \E{\Delta^t_{(j),(i)} \mid \Delta^t_{(j),(i)} \neq 0} \right\}    
\end{equation}
The quantity $\tdif >0$ %, which is well defined as long as since we assume that $\prb{\Delta^t_{(N),(1)} \neq 0}>0$ for every $t$, 
represents a lower bound on our ability to decrease envy from round to round. Recall that Assumption~\ref{assumption: nudge alg ref} implies that $\E{\Delta^t_{(j),(i)}}=\E{r^t_{(j)}-r^t_{(i)}} \geq 0 $ for $j>i$. However, $\Delta^t_{(j),(i)}$ can take 0 sometimes,\footnote{For instance, if $N > K+1$, any explore-first algorithm would have $\Delta^t_{(N-1),(N)}=0$ as the algorithm will pick the same arm for both sessions.} which means there is no scope for nudged arrival to further reduce envy. As long as Assumption~\ref{assumption: nudge alg ref} holds and $ \prb{\Delta^t_{(j),(i)} \neq 0}>0$, we know that
\[
\E{\Delta^t_{(j),(i)} \mid \Delta^t_{(j),(i)} \neq 0}=\frac{ \E{\Delta^t_{(j),(i)}}}{\prb{\Delta^t_{(j),(i)} \neq 0}};
\] thus, we expect $\tdif$ to be significant. To illustrate, recall that in Example~\ref{example: envy with sugg} it holds that $\E{\Delta^t_{(2),(1)}}=\frac{1}{8}$, whereas $\E{\Delta^t_{(2),(1)} \mid \Delta^t_{(2),(1)} \neq 0}=\frac{1}{4}$.
\subsection{Envy Analysis}\label{Envy Analysis}
We are ready to present the main result of the section: Upper bounding the envy under nudged arrival.
\begin{theorem}\label{thm: sugg-envy}
    When executing any algorithm that satisfies Assumption~\ref{assumption: nudge alg ref} with nudged arrival, the expected envy is
    \[\E{\env^T({\sugord})}\leq (N-1)\left(2+\frac{128}{15\delta \tdif}\right) .\]
\end{theorem}
Notice that this upper bound does not depend on the horizon $T$. Intuitively, under nudged arrival, envy behaves like a random walk with a drift toward zero. Although each round may introduce a discrepancy (akin to a random fluctuation), the nudging mechanism consistently pushes the cumulative difference back toward zero. Furthermore, the bound is inversely proportional to $\delta$ and $\tdif$: As $\delta$ decreases, the nudging effect weakens and nudged arrival increasingly resembles uniform arrival. We suspect the terms in the bound are not tight; we discuss it in Section~\ref{sec:discussion}.
\begin{proof}[Proof of Theorem~\ref{thm: sugg-envy}]
Fix any arbitrary algorithm satisfying  Assumption~\ref{assumption: nudge alg ref} and any arbitrary~$t \in [T]$. The proof is outlined as follows:
\begin{enumerate}
    \item Step 1 introduces envy gaps as stochastic processes and the concept of envy excursions.
    \item Demonstrating that envy gaps are nontrivial to analyze, Step 2 presents a more friendly stochastic process that we prove to upper bound the envy gap almost surely.
    \item Step 3 leverages Property~\ref{prop:nudge} and concentration inequalities to upper bound large deviations of the friendly stochastic process.
    \item Lastly, Step 4 uses the tail formula and the concentration from Step 3 to bound to cumulative envy.
\end{enumerate}
\textbf{Step 1: Envy Gap and Excursion}
For every $t$, let $\sigma_t: [N] \rightarrow [N]$ be the ideal permutation from Line~\ref{line:mapping}, i.e.,  
\[
R^{t-1}_{\sigma_t(N)} \leq  R^{t-1}_{\sigma_t(N-1)} \leq \cdots \leq R^{t-1}_{\sigma_t(2)} \leq R^{t-1}_{\sigma_t(1)} .
\]
For every $i, 1\leq i \leq N-1$, we define the \emph{envy gap} $G_i^t = R^t_{\sigma_t(i)} - R^t_{\sigma_t(i+1)}$, representing the envy between the agent with the $i$-th highest reward and the agent with the $(i+1)$-th highest reward.
Consequently, we can define $\env^t$ using the envy gap sequence $(G^t_i)_i$ by
\begin{equation}\label{eq:jknbvfbjj}
\env^t=R^t_{\sigma_t(1)}-R^t_{\sigma_t(N)}=\sum_{i=1}^{N-1} R^t_{\sigma_t(i)}-R^t_{\sigma_t(i+1)} = \sum_{i=1}^{N-1} G_i^t.   
\end{equation}

Next, fix any arbitrary $i$ in the range. We continue by analyzing \emph{excursions} from low envy to high envy and showing they are relatively short, meaning that the expected envy $\E{G_i^t}$ is low. We define an excursion as a sequence of consecutive rounds during which the gap $G_i^\cdot$ exceeds $1$. 
Let $\underline{t} = \max{\left\{ \tau \mid 1 \leq \tau \leq t, G_i^{\tau} \leq 1 \right\}}$ denote last round $\tau$ before $t$ that $G_i^{\tau}$ was less than 1. Similarly,  let $\bar{t} = \min{\left\{ \tau \mid t\leq \tau \leq T, G_i^{\tau} \leq 1 \right\}}$ be the first round $\tau$ after $t$ where $G_i^{\tau}$ is less than 1. For the extreme case where $\bar{t}$ is undefined, we set $\bar{t} = T + 1$. Furthermore, let $D(t)$ denote $t$'s excursion, namely, the set of all the consecutive rounds $\tau$ that includes $t$ during which $G_i^\tau \geq 1$. Formally, $D(t) = \{\tau | \underline{t} < \tau < \bar{t}\}$. Notice that $D(t)$ is an empty set if and only if $G_i^t \leq 1$.

\textbf{Step 2: Auxiliary Stochastic Process}
The sequence $\{G_i^\tau\}_{\tau \in D(t)}$ is challenging to work with because the agents occupying the $i$-th and $(i+1)$-th highest reward position may change from round to round.  We refer to these changes as \emph{rank swaps}, which cause increments like  $G_i^{\tau+1} - G_i^\tau$ to lack a consistent structure. To address this complexity, we introduce the stochastic process $(M^\tau)_\tau$, which is easier to analyze.
%$G_i^\tau$ for $\tau \in D(t)$ is challenging to work with. To demonstrate, notice that the identities of the agent with the $i$-th highest reward and the agent with the $(i+1)$-th highest reward can change from round to round; thus, the increment $G_i^{\tau+1}-G_i^{\tau}$ has no clear structure. 
%To that end, we introduce the stochastic process $(M^\tau)_\tau$, which is easier to analyze. We define it as follows:
\begin{align*}
    M^\tau =
    \begin{cases}
        2 & \text{if } \tau = \underline{t}+1 \\
        M^{\tau-1}+r^\tau_{(i)}-r^\tau_{(i+1)} & \text{else}
        \end{cases}.
\end{align*}
Unlike $G_i^{\tau+1}-G_i^{\tau}$, the increments $M^{\tau+1}-M^{\tau} = r^\tau_{(i)}-r^\tau_{(i+1)}$ are more straightforward and negative in expectation due to Assumption~\ref{assumption: nudge alg ref}. The next proposition demonstrates that $M^\tau_i$ can assist when analyzing $G^\tau_i$.
\begin{proposition}\label{prop G less than M}
For every $\tau \in D(t)$, it holds that $G^\tau_i \leq M^\tau_i$ almost surely.
\end{proposition}
The proof of Proposition~\ref{prop G less than M} appears in \ifapp{Section~\ref{appendix:nudge}}{the appendix}. The main argument enabling this statement is that rank swaps can only decrease the increments $G_i^{\tau+1}-G_i^{\tau}$, but do not affect the increments $M^{\tau+1}-M^{\tau}$.

\textbf{Step 3: Concentration}
The recursive definition of $M^\tau$ implies that for every $\tau \in D(t)$, $M^\tau = 2+ \sum_{n= \underline{t}+2 }^{\tau} r^n_{(i)}-r^n_{(i+1)}$. The next proposition bounds large deviations of $M^\tau$.
\begin{proposition}\label{prop: sugg-m concentration}
    For any $n \in \mathbb{N}$ and $\tau \in D(t)$, it holds that
    \begin{align*}
    \prb{M^\tau > n}\leq \exp\left\{-\frac{(n-2)(\delta \tdif)}{8}\right\}.
    \end{align*}
\end{proposition}
%Intuitively, the bound in Proposition~\ref{prop: sugg-m concentration}  
The proof of Proposition~\ref{prop: sugg-m concentration} appears in \ifapp{Section~\ref{appendix:nudge}}{the appendix}. It leverages the Azuma-Hoeffding inequality and several algebraic tricks to obtain the bound. As we expect, as $\delta$ and $\tdif$ increase, the right-hand side becomes more significant. Alternatively, if the term $\delta \tdif$ approach zero, this bound become irrelevant as $\exp\{0\}=1$.

\textbf{Step 4: Tail Sum}
To finalize the proof, we use the tail-sum formula. Since $G_i^t$ is non-negative,
\begin{align*}
\E{G_i^t} &=
    \int_{x=0}^{\infty} \prb{G_i^t > x}dx \leq
    \sum_{n=0}^{\infty}{\int_{x=n}^{n+1} \prb{G_i^t > n} \,dx}=
    \sum_{n=0}^{\infty}{\prb{G_i^t > n}}.
\end{align*}    
Next, by applying Propositions~\ref{prop G less than M} and~\ref{prop: sugg-m concentration}, we conclude that
\begin{align}\label{eq:tail formula}
    \sum_{n=0}^{\infty}{\prb{G_i^t > n}}& \stackrel{\textnormal{Prop. \ref{prop G less than M}}}{\leq} \sum_{n=0}^{\infty}{\prb{M_i^t > n}} \leq 2+ \sum_{n=2}^{\infty}{\prb{M_i^t > n}}   \stackrel{\textnormal{Prop. \ref{prop: sugg-m concentration}}}{\leq} 2+ \sum_{n=2}^{\infty}\exp{\left\{-\frac{(n-2)\delta \tdif}{8} \right\}}\nonumber \\
    &= 2+ \sum_{n=0}^{\infty}\left(e^{-\frac{\delta \tdif}{8}} \right)^n = 2+\frac{1}{1-\exp{\left(\frac{-\delta \tdif}{8} \right)}} \stackrel{e^{-x}\leq 1-x+\frac{x^2}{2}}{\leq}
    2+\frac{1}{\frac{\delta \tdif}{8}-\frac{(\delta \tdif)^2}{128}}
 \nonumber \\
    & \stackrel{\delta \tdif \leq 1}{\leq} 2+\frac{1}{\frac{\delta \tdif}{8}-\frac{\delta \tdif}{128}} = 2+\frac{128}{15\delta \tdif}.
\end{align}
Ultimately, recall that Inequality~\eqref{eq:tail formula} applies to $\E{G_i^t}$ for every $i$; therefore, Equation~\eqref{eq:jknbvfbjj} ensures that $\E{\env^t}=\E{\sum_{i=1}^{N-1} G_i^t} \leq (N-1)\left(2+\frac{128}{15\delta \tdif}\right) $. This concludes the proof of Theorem~\ref{thm: sugg-envy}.
\end{proof}

{\color{green}




}
\subsection{Adversarial Arrival}
\label{sec: advord}
We end this section by focusing on the adversarial arrival order $\advord$. Intuitively, an adversary seeking to maximize envy would reverse the ideal permutation, placing agents in descending order of their current cumulative rewards. That is, sets the order $\ordv_t$ in round $t$ such that
\[
R^{t-1}_{\ordv_t(N)} \geq R^{t-1}_{\ordv_t(N-1)} \geq \dots \geq R^{t-1}_{\ordv_t(2)} \geq R^{t-1}_{\ordv_t(1)}.
\]
Indeed, it is easy to see that:
\begin{proposition}\label{thm: adv-envy}
When executing any algorithm that satisfies Assumption~\ref{assumption: nudge alg ref} with $\advord$, the expected envy is
    \[\E{\env^T ({\advord})}\geq \tilde{\dif}T.\]
\end{proposition}
\begin{proof}[Proof of Proposition~\ref{thm: adv-envy}]
Assume that the adversary picks agent 1 to be the first and agent $N$ to be the last, i.e., $\ordv_t(1)=1$ and $\ordv_t(N)=N$ for all $t\in [T]$. In such a case,
\begin{align*}
\E{\env^T({\advord})} &=
\E{\max_{i,j\in [N]}{\left\{ \sum^T_{t=1}{\adift{i}{j}} \right\} }} \geq \E{\sum^T_{t=1}{\dif_{N,1}^t}} = \E{\sum^T_{t=1}{\dif^t_{(N),(1)}}}  \geq T\E{\min_{1 \leq t \leq T}{\left\{\sdift{N}{1} \right\} }} \\
&\geq \tdif T.    
\end{align*}
This concludes the proof of Proposition~\ref{thm: adv-envy}.
\end{proof}


\section{Extension: Trading Envy and Welfare}
\label{sec:extensions}








In the previous section, we have shown that coordinating agents' arrival order alone can significantly reduce envy, without affecting the algorithm's core decision-making process. In this section, we take an initial step toward understanding the efficiency-fairness tradeoff, a well-established concept in the literature on fair allocation~\cite{varian1973equity,bertsimas2012efficiency}  and fair classification~\cite{menon2018cost,zafar2017fairness}. 
Specifically, we extend the definition of algorithms from Section~\ref{sec:model} to allow agent‐specific treatment. In other words, algorithms can now observe agent identities and maintain agents accumulated rewards in their memory. Formally, the relevant histories contain triples of the form (agent index, pulled arm, realized reward). We hope to leverage this additional capability to balance social welfare and envy. 


%Misusing agent-specific information can trivially drive envy to $\Theta(T)$. For example, an algorithm that discriminates against agent~$N$ by always assigning it the arm with the lowest expected reward accrues a constant reward gap each round, leading to $\Theta(T)$ envy. Hence, we must proceed cautiously, 

We focus on the special case of our running example (Example~\ref{example 1}): $N=2$ agents, $K=2$ arms with rewards drawn from the uniform distribution, $X_1, X_2 \sim \uni{0,1}$, and the uniform arrival $\uniord$. Furthermore, we assume Bayesian information, i.e., the prior distributions are known. We stress that our results are preliminary, albeit non-trivial. In Subsection~\ref{subsec:ext welfare}, we analyze the socially optimal algorithm from a welfare and envy perspective. Later, in Subsection~\ref{subsec:ext efc}, we develop $\efc$, our welfare-envy balancing algorithm. 
\subsection{Optimal Welfare and Optimal Envy}\label{subsec:ext welfare}
We first analyze the maximal social welfare for this setting. As it turns out, Algorithm~\ref{alguni} is a special case of the optimal two-agent algorithm, as we prove in \ifapp{Section~\ref{appendix:sociallyopt}}{the appendix}. Along with our results from Section~\ref{sec:uniform}, we conclude that:
\begin{observation}\label{obs:opt for tradeoff}
When executing Algorithm~\ref{alguni} on the instance of Example~\ref{example 1} and $\uniord$, it achieves an expected social welfare of (1+$\frac{1}{8}) T$ and induces an expected envy of $\env^T(\uniord)=\tilde \Theta(\sqrt T)$. Furthermore, this is the optimal welfare.
\end{observation}
%The observation suggests that Algorithm~\ref{alguni} is positioned on an extreme point of the Pareto frontier of welfare-envy XXXX, with maximal welfare and high envy. 
This observation indicates that Algorithm~\ref{alguni} occupies an extreme point on the Pareto frontier of the welfare-envy tradeoff: It achieves maximum welfare but also incurs high envy. Another point on the frontier is the algorithm we call $NE$ (\textbf{N}o \textbf{E}nvy), guaranteeing $\env^t=0$ in every round $t$ almost surely. $NE$ draws the same arm in both sessions of every round, as this is essential to maintain zero envy (since the rewards are uniformly distributed and stochastic by nature). Of course, $NE$ has an expected social welfare of $\sw= T$. 


A compelling way to address the social welfare of any algorithm is by examining its ability to exploit the information obtained in the earlier sessions. For example, comparing the performance of $NE$ and Algorithm~\ref{alguni} highlights this difference: $NE$ does not utilize information from the first session, whereas Algorithm~\ref{alguni} leverages it to secure a better reward in the second round. This strategic use of information by Algorithm~\ref{alguni} results in a welfare increase of $\frac{1}{8}$ each round, but also generates envy. This is a key element in the algorithm we propose next.
\subsection{Envy-freeness up to $C$}\label{subsec:ext efc}
%The algorithmic2e version!
% \begin{algorithm}[t]
% \caption{Envy-freeness up to 1 ($\ef$)}
% \label{alg: ef1}
% \begin{algorithmic}[1]
% \Require {horizon $T$}
% \For{round $t = 1 \ldots T$}\label{alg: ef1 1}
%     \State{pull $a_{1}$}\label{alg: ef1 2}
%     \If{$x^t_1 > \frac{1}{2}$}\label{alg: ef1 3}
%         \State{pull $a_{1}$}\label{alg: ef1 4}
%     \Else\label{alg: ef1 5}
%         \If{$\abs{R^{t-1}_{(1)} + \rt{(1)} - R^{t-1}_{(2)}} \leq 1$ and $\abs{R^{t-1}_{(1)} + \rt{(1)} - R^{t-1}_{(2)} - 1} \leq 1$}\label{alg: ef1 6}
%             \State{pull $a_{2}$}\label{alg: ef1 7}
%         \Else\label{alg: ef1 8}
%             \State{pull $a_{1}$}\label{alg: ef1 9}
%         \EndIf
%     \EndIf
% \EndFor
% \end{algorithmic}
% \end{algorithm}
\begin{algorithm}[t]
\caption{Envy-freeness up to $C$ ($\efc$)}
\label{alg:efc}
\SetAlgoLined
\DontPrintSemicolon
\LinesNumbered
\KwIn{horizon $T$, envy bound $C$}
\For{round $t = 1$ to $T$\label{efcline:for}}{
    pull $a_{1}$\label{efclin:pull_a1}\\
    \lIf{$x^t_1 > \frac{1}{2}$}{pull $a_{1}$\label{efclin:pull_a1_again}}
    \Else{\label{efclin:else}
        \lIf{\textnormal{there exists $r\in [0,1]$ such that $\abs{R^{t-1}_{(1)} + r^t_{(1)} - R^{t-1}_{(2)}-r} > C$}}{pull $a_{1}$\label{efclin:pull_a1_cond}}
        %\lIf{$\abs{R^{t-1}_{(1)} + \rt{(1)} - R^{t-1}_{(2)}} \leq C$ and $\abs{R^{t-1}_{(1)} + \rt{(1)} - R^{t-1}_{(2)} - 1} \leq C$}{Pull $a_{2}$\label{efclin:pull_a2}}\textbf{}
        \lElse{pull $a_{2}$\label{efclin:pull_a2_cond_else}}
    }
}
\end{algorithm}

In what follows, we introduce $\efc$, which is an abbreviation of \textbf{E}nvy-\textbf{F}reeness up to $\textbf{C}$, and is
implemented in Algorithm~\ref{alg:efc}. $\efc$ operates by selectively limiting the exploitation of information when the gap between agents' rewards could potentially exceed a predefined envy threshold $C$. This mechanism enforces envy-freeness up to $C$, allowing for better welfare compared to $NE$ while maintaining low envy.

We now describe how $\efc$ works. It interacts with agents for $T$ rounds (Line~\ref{efcline:for}). In every round~$t$, $\efc$ pulls arm $a_1$ for the agent arriving in the first session (Line~\ref{efclin:pull_a1}). The decision to pull $a_1$ first is arbitrary since both arms are identically distributed. If $a_1$ realizes a high reward, i.e., $r^t_{(1)}=x^t_1 > \frac{1}{2}$, $\efc$ pulls it again for the agent in the second session (Line~\ref{efclin:pull_a1_again}). Otherwise, we enter the ``else'' clause in Line~\ref{efclin:else}.

If $a_1$ yields a low reward, the welfare-wise correct action is to pull $a_2$; however, recall that $\efc$ aims to keep the envy lower than $C$. As a result, it ensures that the envy $\abs{R^t_{(1)}-R^t_{(2)}}$ at the end of round $t$ is lower or equal to $C$. Specifically, the ``if'' clause in Line~\ref{efclin:pull_a1_cond} asks whether there exists a realization $r^t_{(2)}=r$ for which the envy would exceed $C$ by the end of the round. If such a realization exists, it pulls $a_1$. Otherwise, it pulls $a_2$ in Line~\ref{efclin:pull_a2_cond_else}. We term $\ef$ the special case of $\efc$ for $C=1$. 
\begin{theorem}\label{thm:ef1evny+sw}
When executing $\efc$ on the instance of Example~\ref{example 1} and $\uniord$, the following hold:
\begin{enumerate}
    \item For all $t$, $\env^t \leq C$ almost surely.
    \item For $C=1$, the social welfare is $\sw \geq \left(1 + \frac{1}{16}\right)T$.
\end{enumerate}
\end{theorem}
Interestingly, Theorem~\ref{thm:ef1evny+sw} implies that $\ef$ recovers half of the social welfare increase due to exploiting information, $(1+\nicefrac{1}{16})T$ versus $(1+\nicefrac{1}{8})T$ for Algorithm~\ref{alguni} and $T$ for $NE$, while limiting the maximal envy to 1 almost surely. We provide a proof sketch below and defer the full proof to \ifapp{Section~\ref{appendix:sociallyopt}}{the appendix}.
\begin{proof}[Proof sketch of Theorem~\ref{thm:ef1evny+sw}]
The first part of the theorem follows directly, as Line~\ref{efclin:pull_a1_cond} allows the envy at $t$ to change only if
\[
\prb{\env^t> C\middle| R^{t-1}_{(1)}, R^{t-1}_{(2)},r^t_{(1)}}=
\prb{\abs{R^{t-1}_{(1)} + r^t_{(1)} - R^{t-1}_{(2)}-r^t_{(2)}} > C\middle| R^{t-1}_{(1)}, R^{t-1}_{(2)},r^t_{(1)}}=0.
\]
The second part of the theorem requires a more detailed argument. We need to understand how often $\efc$ can exploit the information of the first session and enter the ``if'' clause in Line~\ref{efclin:pull_a1_cond}. Since the probability of entering the clause depends on the current level of envy, we must first understand how envy behaves over time. The main technical ingredient we use is the following proposition.
\begin{proposition}\label{prop:ef1 uni dominance}%{thm: ef1 stochastic dominance}
In every round $t$, the distribution of $\env^t$ is stochastically dominated by the uniform distribution over $[0,1]$. I.e., for any $x \in [0,1]$, it holds that $\prb{\env^t \leq x} \geq x$.
\end{proposition}
Despite its intuitive nature, proving this claim demands careful and thorough case analysis. Equipped with Proposition~\ref{prop:ef1 uni dominance}, we turn to analyze how often $\ef$ pulls $a_2$ after observing a low reward $r^t_{(1)} \leq \frac{1}{2}$ in the first session.
\begin{proposition}\label{prop:ef1 open arm}
%\label{thm: ef1 open arm}
        In every round $t$ with $\rt{(1)}\ \leq \frac{1}{2}$, it holds that $\prb{a^t_{(2)} \neq a^t_{(1)} \mid \rt{(1)} \leq \frac{1}{2}} \geq \frac{1}{2}$. 
\end{proposition}
We complete the proof by computing $\sw(\ef)$ via the law of total expectation, using Proposition~\ref{prop:ef1 open arm} to show that the welfare in every round is $1+\frac{1}{16}$, matching the statement of the theorem.
\end{proof}

\subsection{Beyond $C=1$}
The envy analysis in Theorem~\ref{thm:ef1evny+sw} concerns $\ef$, which is a special case of $\efc$ with $C=1$. Unfortunately, our techniques rely heavily on this fact, and extending it would require a different approach. Our preliminary investigation has led us to the following conjecture.
\begin{conjecture}\label{thm: efc sw}
When executing $\efc$ on the instance of Example~\ref{example 1} with any $C \geq 1$ and $\uniord$, the expected social welfare of at least $\sw \geq (1+\frac{1}{8}-\frac{1}{16C})T$.
\end{conjecture}
Simulations we conducted and appear in \ifapp{Section~\ref{sec:simulations}}{the appendix} suggest that Conjecture~\ref{thm: efc sw} holds, and we hope future work could formally prove it.
%While we could not prove this conjecture, we validate it empirically in Chapter~\ref{chap: simulations}.


This work identifies signal collapse as a critical bottleneck in one-shot neural network pruning. Performance loss in pruned networks is due to \textbf{signal collapse} in addition to the removal of critical parameters. We propose \textbf{REFLOW} (\textbf{Re}storing \textbf{F}low of \textbf{Low}-variance signals), a simple yet effective method that mitigates signal collapse without computationally expensive weight updates. By focusing on signal preservation, REFLOW highlights the importance of mitigating signal collapse in sparse networks and enables magnitude pruning to match or surpass state-of-the-art one-shot pruning methods such as CHITA, CBS, and WF.

REFLOW consistently achieves state-of-the-art accuracy across diverse architectures, restoring ResNeXt-101 from under 4.1\% to 78.9\% top-1 accuracy at 80\% sparsity on ImageNet. Its lightweight design makes it a practical solution for both research and deployment, delivering high-quality sparse models without the overhead of traditional approaches. These findings challenge the traditional emphasis on weight selection strategies and underscore the critical role of signal propagation for achieving high-quality sparse networks in the context of one-shot pruning.


 


\section*{Acknowledgments}\label{sec:Acknowledgments}
The work of O. Ben-Porat was supported by the Israel Science Foundation (ISF; Grant No. 3079/24). 

\bibliographystyle{plainnat}
\documentclass{MITstyle}

%\usepackage[table]{xcolor}
\usepackage{chngcntr}
\usepackage{hyperref}
\usepackage{microtype}

\title{A Lightweight and Extensible Cell Segmentation and Classification Model for Whole Slide Images}

\author{Nikita Shvetsov~$^{1, }$\footnote{Correspondence e-mail: nikita.shvetsov@uit.no}, Thomas K. Kilvaer~$^{2, 3}$, Masoud Tafavvoghi~$^{4}$, Anders Sildnes~$^{1}$, \\ Kajsa Møllersen~$^{4}$, Lill-Tove Rasmussen Busund~$^{5, 6}$, Lars Ailo Bongo~$^{1}$ \\
%
\vspace{1em} % Space between authors and afilliations
%
\normalfont{\small $^{1}$Department of Computer Science, UiT The Arctic University of Norway}\\
\normalfont{\small $^{2}$Department of Oncology, University Hospital of North Norway}\\
\normalfont{\small $^{3}$Department of Clinical Medicine, UiT The Arctic University of Norway}\\
\normalfont{\small $^{4}$Department of Community Medicine, UiT The Arctic University of Norway}\\
\normalfont{\small $^{5}$Department of Medical Biology, UiT The Arctic University of Norway} \\
\normalfont{\small $^{6}$Department of Clinical Pathology, University Hospital of North Norway} %\vspace{2em}
}

\begin{document}
\maketitle

\section*{Abstract}

% \begin{abstract}
% Developing clinically useful cell-level analysis tools in digital pathology remains challenging due to limitations in dataset granularity, inconsistent annotations, computational demands of advanced models, and difficulties in integrating new technologies into clinical workflows. To address these challenges, we propose a multi-faceted solution that enhances data quality, model performance, and usability to create a lightweight and extensible cell segmentation and classification model.

% First, we update data labels by employing a cross-relabeling process that refines the labels of two existing datasets, PanNuke and MoNuSAC, to create a new unified dataset with enhanced granularity, encompassing seven distinct cell types. Second, we leverage the H-Optimus foundation model as a fixed encoder to improve feature representation for simultaneous cell segmentation and classification tasks. Third, to address the computational demands of foundation models, we employ knowledge distillation to reduce model size and complexity while maintaining comparable performance. Finally, to facilitate integration into clinical workflows, we integrate the distilled model into the QuPath software, a widely used open-source platform in digital pathology.

% Our results demonstrate improvements in cell segmentation and classification performance using the H‑Optimus-based model compared to a CNN-based model. Specifically, the average $R^2$ improved from 0.575 to 0.871, and the average $PQ$ score improved from 0.450 to 0.492, indicating better alignment with actual cell counts and enhanced segmentation and classification quality. Furthermore, the distilled student model maintains performance comparable to the larger foundation model while reducing the parameter count by a factor of 48.
% Overall, by reducing computational complexity and integrating it into existing workflows, the proposed approach may significantly impact diagnostic processes, reduce the workload of pathologists, and contribute to improved patient outcomes. Though our approach shows potential enhancements in efficiency and usability of cell segmentation and classification models in digital pathology, extensive validation is needed to deploy these models in clinical practice.
% \end{abstract}

%%% shortened abstract
\begin{abstract}
Developing clinically useful cell-level analysis tools in digital pathology remains challenging due to limitations in dataset granularity, inconsistent annotations, high computational demands, and difficulties integrating new technologies into workflows. To address these issues, we propose a solution that enhances data quality, model performance, and usability by creating a lightweight, extensible cell segmentation and classification model. 

First, we update data labels through cross-relabeling to refine annotations of PanNuke and MoNuSAC, producing a unified dataset with seven distinct cell types. Second, we leverage the H-Optimus foundation model as a fixed encoder to improve feature representation for simultaneous segmentation and classification tasks. Third, to address foundation models' computational demands, we distill knowledge to reduce model size and complexity while maintaining comparable performance. Finally, we integrate the distilled model into QuPath, a widely used open-source digital pathology platform. 

Results demonstrate improved segmentation and classification performance using the H-Optimus-based model compared to a CNN-based model. Specifically, average $R^2$ improved from 0.575 to 0.871, and average $PQ$ score improved from 0.450 to 0.492, indicating better alignment with actual cell counts and enhanced segmentation quality. The distilled model maintains comparable performance while reducing parameter count by a factor of 48. By reducing computational complexity and integrating into workflows, this approach may significantly impact diagnostics, reduce pathologist workload, and improve outcomes. Although the method shows promise, extensive validation is necessary prior to clinical deployment.
\end{abstract}
\clearpage

\section{Introduction}
In digital pathology, accurate segmentation and classification of cells are crucial for many diagnostic, prognostic, and predictive analyses \cite{Jaber_Beziaeva_etal._2019,Lin_Pan_etal._2022,Park_Ock_etal._2022,Shen_Choi_etal._2024}. Nowadays, developments in computational pathology offer multiple solutions \cite{H._Qu_P._Wu_etal._2020,Javed_Mahmood_etal._2020} to utilize cell-level datasets to train machine learning models that solve these problems. The quality and specificity of training datasets are critical for robust and accurate models. Adhering to the principle of "garbage in, garbage out", it is essential to ensure that these datasets are extensively and accurately labeled with distinct classes that reflect the diverse biological characteristics of different cell types. Unfortunately, the number of open-source datasets comprising such high-quality annotations is limited. Existing cell segmentation datasets \cite{Gamper_Koohbanani_etal._2019,Graham_Vu_etal._2019,Verma_Kumar_etal._2021} may offer extensive annotations for certain cell types while providing more general labels for others. For example, in PanNuke, which is one of the largest open-source datasets comprising labeled cells, various types of morphologically and functionally different inflammatory cells like macrophages and lymphocytes are clustered in a broad "inflammatory" class. Consequently, these classes are frequently omitted from analyses or aggregated into broader meta-classes \cite{Gamper_Koohbanani_etal._2020} and likely interfere with other cell classes included in the dataset. This and similar inconsistencies in annotation granularity limit the ability of machine learning models to learn the comprehensive and nuanced features necessary for accurate cell segmentation and classification. To address these challenges, methods for refining and standardizing dataset annotations are essential to enhance the quality of training data.

A complementary approach to mitigate the absence of high-quality training data is the use of foundation models. Foundation models as encoders are defined as large-scale, versatile networks pre-trained on vast, diverse datasets using self-supervised learning, contrasting with convolutional neural network (CNN) pre-trained encoders that rely on supervised learning with labeled data. In practice, foundation models leverage enormous amounts of weakly or unlabeled data from millions of whole slide images (WSIs) and employ self-attention mechanisms to capture long-range dependencies and global context \cite{Chen_Ding_etal._2024,Saillard_Jenatton_etal._2024,Vorontsov_Bozkurt_etal._2024,Xu_Usuyama_etal._2024}. As a consequence, foundation models are able to produce transferable feature representations across different cell types and tissue environments. The feature representations can be leveraged by decoder networks to produce segmentation masks and pixel-level classifications. Because foundation models have comprehensive feature representations, they can be effectively fine-tuned using much smaller amounts of cell-level data compared to the large datasets needed to train models from scratch. Furthermore, foundation models incorporate adversarial training elements or contrastive learning \cite{Chen_Ding_etal._2024,Xu_Usuyama_etal._2024}, enhancing their resilience and adaptability by exposing them to challenging and varied scenarios during training. This may result in more generalizable models, often making them well-suited for diverse and complex tasks in digital pathology.

Despite the inherent advantages of foundation models, their deployment for practical use faces its own obstacles. In particular, they require substantial computational power, financial investments and rigorous testing to ensure reliability and efficacy for a given task \cite{Akkus_Dangott_etal._2022,Dragomir_Cocuz_etal._2022,Go_2022,Jafri_Farooqui_etal._2024}. Moreover, while foundation models enhance feature representation and performance, they depend on the quality of available annotations for decoder fine-tuning and, like any other model, cannot resolve existing inconsistencies or ambiguities in data labels. Therefore, there remains a critical need for solutions that address both data quality and practical deployment considerations.
Further, integrating new technologies into existing clinical workflows often encounters resistance, as it necessitates adjustments to established diagnostic processes. So, there is a need to develop solutions that could be integrated into current practices, minimizing the burden on medical professionals to adopt new tools \cite{King_Williams_etal._2023}.

Existing solutions \cite{Goldsborough_Philps_etal._2024,Hörst_Rempe_etal._2024}, while addressing some aspects of these challenges, fall short in providing a comprehensive approach. To address the data quality and clinical deployment issues, we propose a multi-faceted solution that encompasses data refinement, model optimization, and integration with existing pathology tools (\hyperref[fig:fig1]{Figure 1}). The outcome is a lightweight cell segmentation and classification model that can be integrated into digital pathology workflows for practical clinical use.

\begin{figure}[h!]
    \centering
    \includegraphics[width=\textwidth, height=0.82\textheight, keepaspectratio]{images/Figure_1.pdf}
    \caption{Overview of the proposed solution, including 1) Data refinement using cross-relabeling, 2) Teacher model development and fine tuning, 3) Student model optimization with knowledge distillation and 4) Student model and QuPath integration}
    \label{fig:fig1}
\end{figure}
\clearpage

Our approach begins with preparing the data for the fine-tuning and training of the machine learning models. We create a refined dataset, acquired via cross-relabeling two cell-level datasets, enhancing annotation specificity and consistency of the labeled data. Subsequently, we create a cell segmentation and classification model based on the foundation model. We leverage the foundation model as a fixed encoder and fine-tune a decoder using the refined dataset to improve generalization across diverse tissue- and cell types.
To ensure that the model remains lightweight and deployable in a possibly resource-constrained environment, we employ knowledge distillation to approximate the functionality of the foundation model. Finally, to facilitate the practical application of our model in digital pathology workflows, we integrate it with the QuPath \cite{Bankhead_Loughrey_etal._2017} application. Each methodological component contributes to the overarching goal of enhancing model performance, generalizability, and usability in clinical settings.

The primary contributions of this paper are:
\begin{enumerate}
    \item \textit{Data labels refinement through cross-relabeling:}
    
    We propose a new method for refining labels of cell-level datasets through cross-relabeling. This method employs classification models to re-label broad and ambiguous instances, resulting in a more diverse dataset. Our evaluation demonstrates that these classification models achieve high accuracy on test subsets, indicating the reliability of the method for label refinement.

    \item \textit{Enhanced model performance via foundation models:}
    
    We employ a foundation model as a feature extractor for the cell segmentation and classification task. In comparison with training a CNN model from scratch, the foundation model backbone only needs fine-tuning, which significantly reduces training time, computational resources and data requirements. We show that using a foundation model encoder leads to better performance in cell segmentation and classification networks than using a CNN-based encoder. This improvement may enable the model to generalize more effectively across various tissue types and imaging methods.
    
    \item \textit{Model optimization through knowledge distillation:}
    
    We show that a smaller student model trained using knowledge distillation on the refined dataset obtained via our cross-relabeling approach from a foundation model achieves comparable performance in cell segmentation and quantification tasks. As a result, this model is more suitable for deployment in environments without high-performance computing resources.
    
    \item \textit{Integration with QuPath:}
    
    We integrate the distilled cell segmentation and classification model into QuPath, a widely used open-source digital pathology platform, to accelerate clinical adaptation by enabling pathologists to more easily incorporate advanced computational tools into their existing workflows.
\end{enumerate}

Through these methodological steps, we aim to bridge the gap between advanced machine learning techniques and practical clinical applications, making accurate and efficient digital pathology accessible in a broader range of healthcare settings.

\section{Refining Existing Datasets Using Cross-Relabeling}
To address the limitations of sparse and ambiguous labeling of cell-level datasets, we propose a generalizable cross-relabeling strategy that can be applied to any dataset containing broadly categorized or imprecisely labeled cell types. This approach involves training and subsequently leveraging classification models to refine broad categories into more specific or biologically relevant classes.
When applied to cell-level data, the methodology includes extracting individual cell images from the dataset patches, preprocessing these images to standardize the size and accommodate partial cells, and then training deep learning classifiers capable of distinguishing between the finer cell subtypes within the coarser categories. 
To illustrate our approach, we focus on the PanNuke \cite{Gamper_Koohbanani_etal._2020, Gamper_Koohbanani_etal._2019} and MoNuSAC \cite{Verma_Kumar_etal._2021} datasets that we have used to train models for cell quantification in our previous works \cite{Shvetsov_Grønnesby_etal._2022,Shvetsov_Sildnes_etal._2024}. We find that for better cell differentiation we have to introduce more granular labels. PanNuke includes a broad classification of "inflammatory" cells, encompassing lymphocytes, macrophages, and neutrophils. Each cell type differs significantly in structure, function, and clinical relevance. Conversely, MoNuSAC uses the label "epithelial" for a class that comprises both benign epithelial cells and malignant neoplastic cells. This practice makes it challenging to differentiate between benign and malignant epithelial cells in the dataset, which is a critical distinction when identifying tumor areas within tissue samples. To address these issues, we implement a cross-relabeling strategy as shown in \hyperref[fig:fig2]{Figure 2}. The key components are two classification models: one is trained on singular cell images from PanNuke data to classify the epithelial meta-class into epithelial and neoplastic classes. The other is trained on MoNuSAC to refine the inflammatory class into lymphocytes, neutrophils, and macrophages.

\begin{figure}[h!]
    \centering
    \includegraphics[width=\textwidth]{images/Figure_2.pdf}
    \caption{Refined dataset generation via cross relabeling}
    \label{fig:fig2}
\end{figure}

The refining approach consists of three consecutive steps. The first is the preprocessing step, in which we extract individual cells from both datasets (\hyperref[fig:fig3]{Figure 3}). The specifics of PanNuke and MoNuSAC patch preparation before cell preprocessing are provided in \hyperref[chap:S1]{Appendix S1}.

\begin{figure}[h!]
    \centering
    \includegraphics[width=\textwidth]{images/Figure_3.pdf}
    \caption{Cell instances preprocessing including (1) cell map extraction, (2) bounding box delineation, (3) adjusting cell boxes and (4) cropping and resizing of cell images}
    \label{fig:fig3}
\end{figure}

During preprocessing, we extract cell type maps from the ground truth label mask and calculate bounding boxes around each cell instance. To accommodate partial cells at patch borders, a common issue in cropped patch images, we employ mirror padding and extend the field of view of the cell label by 15 pixels to capture adjacent cells. We then crop and resize the identified regions to $64 \times 64$ pixels using bicubic interpolation.

The preprocessed PanNuke dataset comprises 68,031 neoplastic and 23,207 epithelial cell images, while MoNuSAC comprises  33,104 lymphocytes, 1,252 neutrophils, and 1,695 macrophages, which we subsequently use in training cell classification models and classifying the cell image data \hyperref[fig:S2]{Appendix Figure S2 (1)}. 

The next step is to train two distinct ResNet50-based classifiers tailored to address the specific labeling challenges inherent in each dataset. We use ResNet50 for classification models due to its proven effectiveness for image classification tasks in histopathology \cite{pan2022reviewmachinelearningapproaches}, and its compatibility with small images. For the PanNuke dataset, we design the classifier, trained on MoNuSAC data, to disaggregate the heterogeneous "inflammatory" cell category into distinct subtypes: lymphocytes, macrophages, and neutrophils. Similarly, for the MoNuSAC dataset, the classifier is trained on PanNuke data and distinguishes between benign and malignant epithelial cells within the overarching "epithelial" label. By applying these targeted classifiers to their respective datasets, we assign more specific labels to individual cell instances, thus enabling us to create a unified dataset.
To ensure a balanced representation of classes, we train both models on datasets that had been equalized to match the size of the least represented class. Thus, we obtain datasets comprising 23,207 samples per class for PanNuke and 1,252 samples per class for MoNuSAC data. Next, we partition both of them into training (70\%), validation (20\%), and testing (10\%) subsets. To mitigate the risk of overfitting, we use a single dropout layer with a rate of p=0.5 in both models and data augmentation using randomized color perturbations, rotation, and horizontal and vertical flipping. We employ AdamW optimizer and the cross-entropy loss function for the training criterion.

To evaluate the two trained models, we measure the classification accuracy on the respective test subsets. The accuracies on the test subset for both classifiers are presented in \hyperref[tab:1]{Table 1}. The PanNuke model achieves an average accuracy of 93.57\%, with higher accuracy for neoplastic cells (96.06\%) compared to epithelial cells (86.26\%). The confusion matrix in Figure A3.1 shows that the model predominantly distinguishes accurately between epithelial and neoplastic tissues, with a substantial number of correct classifications and relatively few misclassifications. The MoNuSAC model demonstrates an average accuracy of 98.92\%, excelling in classifying lymphocytes (99.67\%) and macrophages (94.12\%), with lower performance for neutrophils (85.71\%). The confusion matrix in Figure A3.2 shows that the model identifies lymphocytes and performs reasonably well with macrophages and neutrophils.

\begin{table}[h!]
\renewcommand{\arraystretch}{1.5}
  \centering
  \caption{Cell classification results for PanNuke and MoNuSAC trained models (CI 95\%).}
  \label{tab:1}
  \begin{tabular}{|l|c|c|}
   \hline
   %\rowcolor{gray!30}
    Accuracy               & PanNuke model              & MoNuSAC model              \\
    \hline
    Average      & 0.936 (0.931--0.941)         & 0.989 (0.986--0.993)        \\
    \hline
    Neoplastic   & 0.961 (0.956--0.965)         & -                          \\
    \hline
    Epithelial   & 0.863 (0.849--0.877)         & -                          \\
    \hline
    Lymphocytes  & -                          & 0.997 (0.995--0.999)        \\
    \hline
    Neutrophils  & -                          & 0.857 (0.796--0.918)        \\
    \hline
    Macrophages  & -                          & 0.941 (0.906--0.976)        \\
    \hline
  \end{tabular}
\end{table}

Finally, during the last step, we use the model trained on PanNuke data for epithelial cells in MoNuSAC and the model trained on MoNuSAC for the inflammatory cells class in PanNuke. Specifically, we use classifier models to relabel epithelial cells in MoNuSAC and inflammatory cells in PanNuke data. Then we combine cells with refined labels and the rest of the cells in both datasets to create a refined dataset (\hyperref[fig:S2]{Appendix Figure S2 (2)}). The process of relabeling cells and visualizing them on a patch is shown in \hyperref[fig:fig4]{Figure 4}. The cell counts in the refined dataset are provided in \hyperref[tab:S4]{Appendix Table S4}.

\begin{figure}[h!]
    \centering
    \includegraphics[width=\textwidth, height=0.42\textheight, keepaspectratio]{images/Figure_4.pdf}
    \caption{Cell relabeling procedure for epithelial and inflammatory cell classes}
    \label{fig:fig4}
\end{figure}

%\hfill

Relabeling and combining datasets have been explored in a prior study \cite{Parulekar_Kanwat_etal._2023}, where consecutive fine-tuning on multiple datasets was employed to account for hierarchical class label structures. While the method presented in \cite{Parulekar_Kanwat_etal._2023} is intuitive, it often lacks consistency and requires multiple fine-tuning runs, which can be cumbersome and time-consuming. 
In contrast, cross-relabeling simplifies this process by using specialized classification models tailored to each dataset's specific labeling challenges. This approach provides better transparency and produces a unified dataset encompassing seven distinct cell types across multiple tissue samples, enhancing data diversity for further model training or fine-tuning.

Despite these improvements, cross-relabeling does not entirely resolve issues related to poor labeling quality or the amount of labeled data. Specifically, our results show lower accuracies persist for underrepresented classes, such as macrophages, which may stem from a limited sample availability and intrinsic challenges in distinguishing these cells based solely on H\&E staining. Furthermore, while our method enhances label specificity, it relies on the initial quality of the broad labels; thus, any fundamental inaccuracies in the original annotations can propagate through the relabeling process. Addressing the overall problem of limited data labels may require integrating additional data sources or utilizing complementary immunohistochemical staining methods.
Although the reported performance metrics are obtained from evaluations on the native test sets of each dataset, it is important to note that the primary application of these classifiers is to perform cross-relabeling, where a model trained on one dataset (e.g., PanNuke) is applied to another (e.g., MoNuSAC) and vice versa. We acknowledge that a more systematic evaluation of cross-dataset generalization is needed and could be performed in future work.

Overall, the refined dataset produced by our approach can enhance the supervised training or fine-tuning of cell segmentation and classification models, especially those that utilize pre-trained foundation models to improve feature extraction robustness. In addition, these models can detect nuanced classes that enable researchers to conduct more detailed analyses of biological processes in computational pathology.

\section{Foundation models for robust cell segmentation and classification}

Accurate cell segmentation and classification in digital pathology are hindered by limited labeled data and the fact that conventional CNNs are unable to capture global contextual information due to their local receptive field constraints \cite{Gheflati_Rivaz_2022,Yang_Marcus_etal.}. Traditional approaches in cell quantification have predominantly relied on CNN encoders, such as ResNet50, given their proven effectiveness in semantic segmentation tasks \cite{Deshmane_2023,Graham_Vu_etal._2019,Mukasheva_Koishiyeva_etal._2024,Stringer_Wang_etal._2021}. However, approaches that include fine-tuning of pretrained CNNs, data augmentation, and stain normalization to partially increase data variability and address staining differences often fail to achieve the necessary generalization and robustness across diverse tissue types and staining conditions \cite{G._Wang_W._Li_etal._2018,Gao_Bagci_etal._2018,Karim_El_Khoury_Martin_Fockedey_etal._2021}.

To overcome these challenges, we leverage an encoder-decoder network that uses a foundation model as the encoder and a CNN upsampling decoder (\hyperref[fig:fig5]{Figure 5}) for simultaneous cell segmentation and classification in 2D patches extracted from WSIs. Foundation models with transformer-based architectures are viable alternatives to CNN-based encoders \cite{Shamshad_Khan_etal._2023,Sourget_2023}. They enable the creation of more advanced architectures that can decode or transform learned features more effectively \cite{Chen_Duan_etal._2023,Cheng_Misra_etal._2022,Xie_Wang_etal._2021}.

\begin{figure}[h!]
    \centering
    \includegraphics[width=\textwidth]{images/Figure_5.pdf}
    \caption{UNETR-like model with foundational model as backbone}
    \label{fig:fig5}
\end{figure}

By utilizing a transformer-based encoder, we incorporate global contextual information into the feature extraction process, which is a key advantage of such architectures \cite{Chen_Lu_etal._2021}. This foundation model integration facilitates accurate pixel-wise segmentation and classification without the need for extensive encoder training, thereby potentially improving generalization across varied cellular structures and tissue types.
In our implementation, we employ a modified UNETR \cite{Hatamizadeh_Tang_etal._2021} architecture that combines a vision transformer (ViT) \cite{Dosovitskiy_Beyer_etal._2021} encoder with a CNN-based decoder. The encoder utilizes the pretrained H-Optimus foundation model, which contains 1.1 billion parameters and is trained on over 500,000 H\&E stained WSIs \cite{Saillard_Jenatton_etal._2024}. We extract outputs from four evenly spaced transformer blocks $Z_i$, where $i \in [1, 14, 26, 38]$, to serve as residual connections for the CNN decoder. We select these blocks based on our observation that features from non-adjacent levels of the encoder lead to better overall performance on the test subset.

The CNN decoder upsamples the feature representations, acquired from the transformer blocks, to generate an intermediate vector that is handled by two task-specific layers that generate cell segmentation and classification masks. The first task-specific layer is the ‘Cellpose head’,  which is used to delineate cell instances. The layer generates horizontal and vertical gradient maps to form vector fields that are refined through gradient tracking in a post-processing step using the Cellpose algorithm \cite{Stringer_Wang_etal._2021}, known for its efficacy in cell segmentation tasks and generalizability across multiple domains \cite{Pachitariu_Stringer_2022,Stringer_Pachitariu_2024}. The second task-specific layer is the "Cell type head", which assigns labels to individual pixels. In the post-processing step, we determine the output classification label of each segmented cell instance by majority voting over the labeled pixels that comprise the cell in the segmentation map.

To evaluate model performance and measure the impact of adding a foundation model as backbone, we compare it to a ResNet50-based model. ResNet50 is a widely used solution for encoders in segmentation architectures in the medical domain \cite{Deshmane_2023,Graham_Vu_etal._2019,Mukasheva_Koishiyeva_etal._2024,Stringer_Wang_etal._2021}. For the H-Optimus-based model, we utilize frozen weights for the encoder and only fine-tune the decoder to take advantage of the extensive pre-training of the foundation model. For the ResNet50-based model we start with ImageNet \cite{Deng_Dong_etal.} weights and train both encoder and decoder parts. Hyperparameters for the training step are set to be identical, where possible, for comparable evaluation. 
For this evaluation, we deliberately use the PanNuke dataset to provide a standardized and controlled comparison between the H‑Optimus and ResNet50-based models (\hyperref[fig:S2]{Appendix Figure S2 (3)}). Specifically, we use two of the default PanNuke dataset splits (66\%) for training and validation, and reserve the third split (33\%) for testing.

To address the challenge of cell class imbalance in the PanNuke dataset, which is a common characteristic in most cell-level H\&E patch datasets, both models’ training processes employ a weighted loss function comprising cross-entropy and focal loss \cite{Lin_Goyal_etal._2018}. The focal loss component is adjusted with coefficients derived from each cell class' instance frequency, emphasizing learning from underrepresented classes and enhancing the model's sensitivity to rare but significant cellular patterns. The cross-entropy loss is augmented with spectral decoupling regularization \cite{Pezeshki_Kaba_etal._2021,Pohjonen_Stürenberg_etal._2022} and spatially varying label smoothing \cite{Islam_Glocker_2021}, which potentially stabilizes training and improves generalization in case of complex tissue morphologies. For optimization, we employ the \textit{AdamW} \cite{Loshchilov_Hutter_2019} to counter unbalanced class scenarios, with cosine annealing learning rate scheduler.

We utilize the scikit-learn library \cite{Van_der_Walt_Schönberger_etal._2014} and HoVer-Net \cite{Graham_Vu_etal._2019} implementations of $R^2$ (the coefficient of determination) and $PQ$ (panoptic quality) to evaluate our experiments. Complete mathematical formulations and detailed explanations of these metrics are provided in \hyperref[chap:S5]{Appendix S5}. To compute confidence intervals, we use nonparametric bootstrapping, where after calculating the metric on the full sample, we generated 1000 bootstrap replicates by resampling with replacement and then determined the 95\% confidence intervals as the 2.5th and 97.5th percentiles of the resulting empirical distribution.

%\hfill

The model comparisons are summarized in \hyperref[tab:2]{Table 2}. The H‑Optimus-based model achieves higher $R^2$ across all cell classes compared to the ResNet50-based model, which means that its predictions are more closely aligned with the PanNuke cell counts, indicating a stronger correlation with the observed data. Notably, the improvement of $R^2_{dead}$ may be an indicator of better global contextual representations provided by the foundation model backbone. In terms of segmentation and classification quality combined, measured by the PQ score, the H‑Optimus-based model demonstrates notable improvements across most cell classes. Overall, the average $R^2$ improved from 0.575 to 0.871, while the average $PQ$ score improved from 0.450 to 0.492, demonstrating better performance of the H-Optimus-based model.

\begin{table}[h!]
\renewcommand{\arraystretch}{1.5}
  \centering
  \caption{Cell quantification metrics for baseline and proposed models (CI 95\%).}
  \label{tab:2}
  \begin{tabular}{|l|c|c|}
    \hline
    %\rowcolor{gray!30}
    Metric             & Resnet50-based            & H-optimus-based              \\
    \hline
    $R^2_{neoplastic}$    & 0.681 (0.576--0.769)       & \textbf{0.941 (0.917--0.960)} \\
    \hline
    $R^2_{inflammatory}$  & 0.863 (0.778--0.903)       & \textbf{0.949 (0.918--0.966)} \\
    \hline
    $R^2_{connective}$    & 0.600 (0.488--0.698)       & 0.609 (0.436--0.772)          \\
    \hline
    $R^2_{dead}$          & 0.097 (-11.389--0.669)     & 0.925 (0.404--0.982)          \\
    \hline
    $R^2_{epithelial}$    & 0.635 (0.490--0.747)       & \textbf{0.930 (0.886--0.964)} \\
    \hline
    $PQ_{neoplastic}$       & 0.517 (0.499--0.535)       & \textbf{0.589 (0.575--0.604)} \\
    \hline
    $PQ_{inflammatory}$     & 0.455 (0.429--0.482)       & \textbf{0.528 (0.507--0.549)} \\
    \hline
    $PQ_{connective}$       & 0.416 (0.400--0.431)       & \textbf{0.451 (0.436--0.465)} \\
    \hline
    $PQ_{dead}$             & 0.374 (0.342--0.408)       & 0.292 (0.209--0.365)          \\
    \hline
    $PQ_{epithelial}$       & 0.488 (0.460--0.519)       & \textbf{0.599 (0.579--0.618)} \\
    \hline
  \end{tabular}
\end{table}

Our results  show that integrating the H‑Optimus foundation model within the UNETR architecture enhances the model's ability to segment and classify cells across diverse tissues from PanNuke data. The pretrained transformer encoder provides robust feature representations, resulting in higher average $R^2$ and $PQ$ scores compared to the CNN-based model. This leads to more reliable cell quantification and more accurate downstream analysis. Additionally, the streamlined fine-tuning process reduces computational overhead and training time, making the model more adaptable for new data.

Despite these advancements, the foundation model-based approach does not fully resolve all challenges related to cell segmentation and classification. We observe lower metric scores for underrepresented classes in the training data. Furthermore, foundation models typically encompass billions of parameters, resulting in substantial computational and memory requirements. It therefore poses challenges for deployment in resource-constrained environments, limiting their practical applicability in certain clinical settings.

\section{Model optimization via Knowledge Distillation}

To address the limitations posed by the extensive size of foundation models, we implement knowledge distillation — a model compression technique that leverages the teacher-student paradigm \cite{Hinton_Vinyals_etal._2015}. By training a smaller, more efficient student model to replicate the output of a larger, pre-trained teacher model, we retain performance while significantly reducing the model's complexity and resource requirements (\hyperref[fig:fig6]{Figure 6}).

\begin{figure}[h!]
    \centering
    \includegraphics[width=\textwidth, height=0.45\textheight, keepaspectratio]{images/Figure_6.pdf}
    \caption{Knowledge distillation framework for training a student model using a pre-trained teacher}
    \label{fig:fig6}
\end{figure}

We employ knowledge distillation to compress the H‑Optimus-based teacher model into a more efficient student model. The teacher model is the modified UNETR architecture with the H‑Optimus foundation model described in the previous chapter. The student model is based on a UNet architecture augmented with residual connections and incorporates a smaller ViT encoder with 9 million parameters \cite{Steiner_Kolesnikov_etal._2022,Wightman_2019}. 

First, we fine-tune the teacher model using the refined dataset from the cross-relabeling procedure (Section 2). Initially we train the decoder of the teacher model while keeping the encoder weights frozen. We split the refined dataset into train (70\%), validation (20\%) and test (10\%) subsets (\hyperref[fig:S2]{Appendix Figure S2 (4)}). During fine-tuning, we use the train and validation subsets, while leaving the test subset for model evaluation. We set the training procedure and model hyperparameters to be identical to those that were used to demonstrate the utility of foundation models for the simultaneous cell segmentation and classification task.

Next, we perform knowledge distillation from teacher to student using the refined dataset used to fine-tune the teacher model. The student model is trained to replicate the teacher model's outputs. We utilize a specialized loss function that aligns the student's predicted probability distribution with the teacher's, incorporating the teacher's class probability distribution derived from the output. Following the methodology of Hinton et al. \cite{Hinton_Vinyals_etal._2015}, we experiment with various hyperparameter settings for the temperature ($T$) and the balancing coefficients ($\alpha$ and $\beta$) in the loss function. We vary $T$ from 1 to 20 and adjust $\alpha$ and $\beta$ to balance the distillation and student losses. Through iterative tuning and evaluation, we identify that setting $T=14$, $\alpha=0.3$, and $\beta=0.7$ yields a configuration that converges and closely approximates the teacher model's performance during training.

Finally, we assess the performance of both models using the $R^2$ and $PQ$ (defined in \hyperref[chap:S5]{Appendix S5}) on the test set of the refined dataset (\hyperref[tab:3]{Table 3}). We observe that the 95\% confidence intervals overlap for most cell types, so we cannot claim statistically significant performance differences between the teacher and student models. One exception appears in the neoplastic class. The teacher model produces an $R^2$ of 0.919, while the student model shows an $R^2$ of 0.852. In addition, the student model achieves higher $PQ$ values for the neoplastic and connective classes, though the confidence intervals show overlap.

\begin{table}[h!]
\renewcommand{\arraystretch}{1.5}
  \centering
  \caption{Cell quantification metrics for teacher and distilled student models (CI 95\%).}
  \label{tab:3}
  \begin{tabular}{|l|c|c|}
    \hline
    %\rowcolor{gray!30}
    Metric & Teacher & Student \\
    \hline
    $R^2_{neoplastic}$    & \textbf{0.919} (0.898--0.939) & 0.852 (0.800--0.891) \\
    \hline
    $R^2_{lymphocyte}$    & 0.969 (0.956--0.977)         & 0.969 (0.956--0.978) \\
    \hline
    $R^2_{connective}$    & 0.694 (0.548--0.809)         & 0.618 (0.469--0.741) \\
    \hline
    $R^2_{dead}$          & 0.755 (0.400--0.908)         & 0.424 (0.100--0.731) \\
    \hline
    $R^2_{epithelial}$    & 0.922 (0.870--0.958)         & 0.843 (0.738--0.917) \\
    \hline
    $R^2_{macrophage}$    & 0.384 (-0.369--0.724)        & 0.704 (0.352--0.859) \\
    \hline
    $R^2_{neutrofil}$     & 0.854 (0.578--0.929)         & 0.833 (0.502--0.925) \\
    \hline
    $PQ_{neoplastic}$       & 0.581 (0.569--0.593)         & 0.601 (0.588--0.613) \\
    \hline
    $PQ_{lymphocyte}$       & 0.536 (0.520--0.553)         & 0.563 (0.544--0.579) \\
    \hline
    $PQ_{connective}$       & 0.436 (0.421--0.451)         & 0.457 (0.441--0.474) \\
    \hline
    $PQ_{dead}$             & 0.272 (0.235--0.315)         & 0.279 (0.201--0.369) \\
    \hline
    $PQ_{epithelial}$       & 0.522 (0.500--0.545)         & 0.530 (0.506--0.555) \\
    \hline
    $PQ_{macrophage}$       & 0.524 (0.459--0.588)         & 0.474 (0.405--0.543) \\
    \hline
    $PQ_{neutrofil}$        & 0.541 (0.490--0.592)         & 0.565 (0.522--0.607) \\
    \hline
  \end{tabular}
\end{table}


We further decompose the $PQ$ metric into its $SQ$ and $DQ$ components (\hyperref[tab:S6]{Appendix Table S6}). Both models produce nearly identical $SQ$ values, which indicates that they predict instance boundaries with similar precision. Although the student model shows some improvement in $DQ$ scores for certain classes, the confidence intervals overlap and do not confirm a statistically significant difference.

We observe that the student and teacher models yield comparable detection performance despite the student model using a much smaller and simpler architecture. A model with fewer parameters reduces the risk of overfitting when training data are scarce relative to the model’s complexity \cite{Farias_Ludermir_etal._2022}. The knowledge distillation process also encourages the student model to focus on the most generalizable detection features learned from the teacher. These factors enable the student model to achieve similar detection performance across different cell types.

Additionally, considering the model sizes reported in \hyperref[tab:4]{Table 4}, the distilled model achieves a significant reduction compared to the teacher model, with a 48-fold decrease in parameter count and a 5.5-fold reduction in on-disk size. In inference mode, the teacher model requires 16 GB of VRAM for a batch size of 32, while the distilled model only needs 3 GB of VRAM for the same batch size. These reductions make the distilled model significantly more practical for fine-tuning and deployment in resource-constrained environments.

\begin{table}[h!]
\renewcommand{\arraystretch}{1.5}
  \centering
  \caption{Parameter counts and size of teacher and distilled model}
  \label{tab:4}
  \adjustbox{max width=\textwidth}{%
  \begin{tabular}{|l|c|c|c|}
    \hline
    %\rowcolor{gray!30}
    Metric & H-optimus-based (Teacher) & mobileViT-based (Student) & Magnitude of difference \\
    \hline
    Parameters count       & 1,158,917,906   & \textbf{24,093,393}   & \textbf{48x}  \\
    \hline
    Estimated Total Size (MB) & 87,912       & \textbf{15,935}    & \textbf{5.5x} \\
    \hline
  \end{tabular}%
}
\end{table}

%\hfill

With recent advancements in complex network architectures and the use of pretrained encoders to achieve state-of-the-art performance \cite{Baumann_Dislich_etal._2024,Hörst_Rempe_etal._2024} in cell segmentation and classification tasks, model size, computational complexity, and processing times have increased. This limits the scalability and accessibility of these models. As we demonstrate, this may be mitigated using knowledge distillation. Studies in the field of natural language processing have demonstrated the efficacy of knowledge distillation in retaining the capabilities of the teacher model while achieving significant reductions in size and complexity \cite{Huangpu_Gao_2024,Sun_Yu_etal.}. 

We demonstrate the feasibility of knowledge distillation in digital pathology, specifically for cell segmentation and classification tasks. Moreover, we achieve this performance while also significantly reducing the parameter count. In addressing the challenge of knowledge transfer, we found that distillation from a transformer-based model to a smaller transformer is more straightforward than attempting to map transformer features to CNN blocks. In our experiments, using a CNN-based network as a student results in worse cell quantification performance due to the structural constraints of CNN feature space dimensions. 

Although our primary approach relies on a transformer-based student model that performs well, it can be further optimized to incorporate advantages from CNN architectures. For example, employing alternative techniques such as using ViT adapters \cite{Chen_Duan_etal._2023} or $1 \times 1$ convolutions to adjust feature map sizes may be beneficial for harnessing CNN advantages like enhanced local feature extraction. Moreover, if additional performance improvements are desired, the process can be further enhanced by applying supplementary knowledge distillation techniques, such as self-distillation \cite{Zhang_Song_etal._2019} or online distillation \cite{Houyon_Cioppa_etal._2023}.

Despite these promising results, further validation on independent datasets is necessary to fully understand the model's limitations. Underrepresented classes may pose challenges when addressing complex cases. Pathologists need to validate these models to adopt them in clinical settings. While the distilled models are smaller and more deployable, a technological gap persists because pathologists traditionally rely on established methods for inspecting WSIs and diagnosing diseases. Addressing the complexities involved in deploying models for inference and supporting pathologists in adopting new tools is essential for integrating these models into clinical workflows.

\section{Model integration with QuPath}
Digital pathology tools with graphical user interfaces are essential for visualizing and analyzing WSIs. To make our student model useful in clinical pathology workflows, it needs to be integrated into a tool that enables inspecting regions, creating annotations, and providing quantitative analyses of biomarkers. Therefore, we integrate the trained student model from the previous chapter into the QuPath open‑source platform \cite{Bankhead_Loughrey_etal._2017}. QuPath provides the required annotation, visualization, and analysis tools to interpret complex histological data, including workflows for cell segmentation, classification, and quantification (\hyperref[fig:fig7]{Figure 7}). 

\begin{figure}[h!]
    \centering
    \includegraphics[width=\textwidth]{images/Figure_7.pdf}
    \caption{Visualization of model-generated cell quantification annotations (left) and the corresponding unannotated slide (right) in QuPath}
    \label{fig:fig7}
\end{figure}

To identify the regions in a WSI critical for prognosticating tumor development, such as specific tumor areas or border regions without overlapping healthy tissue, the pathologist uses QuPath to outline these regions. Then, the pathologist initiates a cell segmentation and classification script through the QuPath interface for the selected regions. The resulting annotations and quantified cell information are then directly overlaid onto the WSI in the QuPath interface. Additional design and implementation details are in \hyperref[chap:S7]{Appendix S7}. 

Two common approaches for integrating deep learning models into QuPath are Java‑based native QuPath extensions \cite{Goldsborough_Philps_etal._2024} and the execution of RESTful API requests to a model server coupled with handling the response via an extension, as demonstrated in the application of cell segmentation models applied to immunofluorescence images \cite{Sugawara_2023}. While the community is actively working on these integration strategies, there is currently no universal solution that fully addresses all integration and performance requirements.

Extensions may offer better integration with QuPath, allowing slightly improved performance and more widespread usage of the built-in QuPath models, but they lack the flexibility to customize models and modify their behavior. For example, the newest version of QuPath includes models such as StarDist \cite{Weigert_Schmidt} and InstanSeg \cite{Goldsborough_Philps_etal._2024} that can perform cell segmentation. Both models pose limitations when applied to simultaneous cell segmentation and classification. StarDist performs well only on convex, round shapes by design, whereas some neoplastic, inflammatory, and connective cells exhibit complex and non-convex shapes. InstanSeg provides only semantic segmentation without assigning classes to the segmented cells.

%\hfill

In contrast, our approach offers an alternative integration strategy. It utilizes the paquo library to directly interact with QuPath’s internal application programming interface from within Python. This enables data exchange and processing without the need for intermediate conversion steps and provides greater control over model customization, retraining, and the incorporation of custom processing steps.

The integration of our custom model with QuPath underscores its potential to significantly enhance the diagnostic process by reducing the time burden on pathologists and enabling them to focus on more complex interpretative tasks using familiar software. Leveraging a tool that is already well-established among pathologists increases the likelihood of its adoption into daily clinical workflows. The quantitative data generated through the automated workflow is critical for both clinical decision-making and research, facilitating more accurate biomarker analysis, enabling robust statistical evaluations, and supporting hypothesis generation and testing. Additionally, by streamlining cell segmentation and classification, the tool enhances the scalability and reproducibility of pathological assessments, ultimately contributing to improved diagnostic accuracy and patient outcomes.

\section{Conclusion and future work}

In this study, we address critical challenges in digital pathology and tackle the usability and deployment issues of the developed models in standard computing environments without the need for high-performance computing systems. Our multi-faceted approach encompasses data refinement through cross-relabeling, leveraging foundation models for robust cell segmentation and classification, optimizing model performance via knowledge distillation, and integrating the optimized model into the QuPath software for practical application. This approach is used to construct a capable, versatile, and adjustable model for cell segmentation and classification, with enhanced performance and usability.

\begin{sloppypar}
While our approach shows potential in the field of computational pathology, certain limitations persist. 
For example, our implementation currently exhibits lower performance in detecting macrophages. 
This serves as an instance of the broader challenge of accurately identifying complex cell types. In order to address this issue, extending our approach to incorporate additional data sources, exploring alternative modeling approaches, and integrating other imaging modalities such as immunohistochemical staining may help improve detection accuracy. Moreover, although the distilled model reduces computational demands, integrating advanced deep learning models into clinical practice requires addressing technological gaps and potential resistance to adopting new tools within established diagnostic processes.
\end{sloppypar}

Future work could focus on several key areas to refine the proposed approach and facilitate its adoption in clinical environments. Enhancing the cell-relabeling process with additional datasets \cite{Graham_Jahanifar_etal._2021} could improve the representation of underrepresented cell types and enhance overall model performance. Also, incorporating additional data sources, such as multi-modal imaging or complementary staining methods, may address limitations related to cell type differentiation and class imbalance. Exploring other foundation models \cite{Vorontsov_Bozkurt_etal._2024,Zimmermann_Vorontsov_etal._2024} or introducing additional modalities \cite{Ding_Wagner_etal._2024,Vaidya_Zhang_etal._2025} may provide alternative architectures better suited to specific tasks or offer improved efficiency. Implementing more complex knowledge distillation techniques \cite{Houyon_Cioppa_etal._2023,Zhang_Song_etal._2019} could further optimize the model's performance and adaptability. Additionally, deeper integration with QuPath or other digital pathology software could provide pathologists more control over cell quantification analysis directly within the QuPath interface, thereby increasing accessibility and usability. Such enhancements would not only refine model performance but also ensure greater adaptability and scalability within various clinical environments. Finally, extensive validation of the model by pathologists and benchmarking against independent datasets are essential steps toward establishing the model's reliability and fostering confidence in its clinical utility.

\section*{Acknowledgments} 
This work was funded in part by the Research Council of Norway grant no. 309439 SFI Visual Intelligence, and the North Norwegian Health Authority grant no. HNF1521-20.

\bibliographystyle{IEEEtran}
\begin{sloppypar}
\begin{thebibliography}{99}

\bibitem{chaplot2020neural} Chaplot, Devendra Singh, et al. "Neural topological slam for visual navigation." Proceedings of the IEEE/CVF conference on computer vision and pattern recognition. 2020.

\bibitem{maksymets2021thda} Maksymets, Oleksandr, et al. "Thda: Treasure hunt data augmentation for semantic navigation." Proceedings of the IEEE/CVF International Conference on Computer Vision. 2021.

\bibitem{mezghan2022memory} Mezghan, Lina, et al. "Memory-augmented reinforcement learning for image-goal navigation." 2022 IEEE/RSJ International Conference on Intelligent Robots and Systems (IROS). IEEE, 2022.

\bibitem{al2022zero} Al-Halah, Ziad, Santhosh Kumar Ramakrishnan, and Kristen Grauman. "Zero experience required: Plug \& play modular transfer learning for semantic visual navigation." Proceedings of the IEEE/CVF Conference on Computer Vision and Pattern Recognition. 2022.

\bibitem{ye2021auxiliary} Ye, Joel, et al. "Auxiliary tasks and exploration enable objectgoal navigation." Proceedings of the IEEE/CVF international conference on computer vision. 2021.

\bibitem{chaplot2020object} Chaplot, Devendra Singh, et al. "Object goal navigation using goal-oriented semantic exploration." Advances in Neural Information Processing Systems 33 (2020)

\bibitem{ramakrishnan2022poni} Ramakrishnan, Santhosh Kumar, et al. "Poni: Potential functions for objectgoal navigation with interaction-free learning." Proceedings of the IEEE/CVF Conference on Computer Vision and Pattern Recognition. 2022.

\bibitem{ramrakhya2022habitat} Ramrakhya, Ram, et al. "Habitat-web: Learning embodied object-search strategies from human demonstrations at scale." Proceedings of the IEEE/CVF Conference on Computer Vision and Pattern Recognition. 2022.

\bibitem{mousavian2019visual} Mousavian, Arsalan, et al. "Visual representations for semantic target driven navigation." 2019 International Conference on Robotics and Automation (ICRA). IEEE, 2019.

\bibitem{dhariwal2021diffusion} Dhariwal, Prafulla, and Alexander Nichol. "Diffusion models beat gans on image synthesis." Advances in neural information processing systems 34 (2021)

\bibitem{ho2022classifier} Ho, Jonathan, and Tim Salimans. "Classifier-free diffusion guidance." arXiv preprint arXiv:2207.12598 (2022).

\bibitem{nichol2021glide} Nichol, Alex, et al. "Glide: Towards photorealistic image generation and editing with text-guided diffusion models." arXiv preprint arXiv:2112.10741 (2021)

\bibitem{brooks2023instructpix2pix} Brooks, Tim, Aleksander Holynski, and Alexei A. Efros. "Instructpix2pix: Learning to follow image editing instructions." Proceedings of the IEEE/CVF Conference on Computer Vision and Pattern Recognition. 2023.

\bibitem{fu2023guiding} Fu, Tsu-Jui, et al. "Guiding instruction-based image editing via multimodal large language models." arXiv preprint arXiv:2309.17102 (2023).

\bibitem{geng2024instructdiffusion} Geng, Zigang, et al. "Instructdiffusion: A generalist modeling interface for vision tasks." Proceedings of the IEEE/CVF Conference on Computer Vision and Pattern Recognition. 2024.

\bibitem{zhou2024minedreamer} Zhou, Enshen, et al. "Minedreamer: Learning to follow instructions via chain-of-imagination for simulated-world control." arXiv preprint arXiv:2403.12037 (2024).

\bibitem{zhou2023esc} Zhou, Kaiwen, et al. "Esc: Exploration with soft commonsense constraints for zero-shot object navigation." International Conference on Machine Learning. PMLR, 2023.

\bibitem{yu2023l3mvn} Yu, Bangguo, Hamidreza Kasaei, and Ming Cao. "L3mvn: Leveraging large language models for visual target navigation." 2023 IEEE/RSJ International Conference on Intelligent Robots and Systems (IROS). IEEE, 2023.

\bibitem{gadre2023cows} Gadre, Samir Yitzhak, et al. "Cows on pasture: Baselines and benchmarks for language-driven zero-shot object navigation." Proceedings of the IEEE/CVF Conference on Computer Vision and Pattern Recognition. 2023.

\bibitem{shah2023navigation} Shah, Dhruv, et al. "Navigation with large language models: Semantic guesswork as a heuristic for planning." Conference on Robot Learning. PMLR, 2023.

\bibitem{cai2024bridging} Cai, Wenzhe, et al. "Bridging zero-shot object navigation and foundation models through pixel-guided navigation skill." 2024 IEEE International Conference on Robotics and Automation (ICRA). IEEE, 2024.

\bibitem{yu2023co} Yu, Bangguo, Hamidreza Kasaei, and Ming Cao. "Co-NavGPT: Multi-robot cooperative visual semantic navigation using large language models." arXiv preprint arXiv:2310.07937 (2023).

\bibitem{wu2024voronav} Wu, Pengying, et al. "Voronav: Voronoi-based zero-shot object navigation with large language model." arXiv preprint arXiv:2401.02695 (2024).

\bibitem{qin2023mp5} Qin, Yiran, et al. "Mp5: A multi-modal open-ended embodied system in minecraft via active perception." arXiv preprint arXiv:2312.07472 (2023).

\bibitem{du2024learning} Du, Yilun, et al. "Learning universal policies via text-guided video generation." Advances in Neural Information Processing Systems 36 (2024).

\bibitem{ajay2024compositional} Ajay, Anurag, et al. "Compositional foundation models for hierarchical planning." Advances in Neural Information Processing Systems 36 (2024).

\bibitem{liang2024skilldiffuser} Liang, Zhixuan, et al. "Skilldiffuser: Interpretable hierarchical planning via skill abstractions in diffusion-based task execution." Proceedings of the IEEE/CVF Conference on Computer Vision and Pattern Recognition. 2024.

\bibitem{heusel2017gans} Heusel, Martin, et al. "Gans trained by a two time-scale update rule converge to a local nash equilibrium." Advances in neural information processing systems 30 (2017).

\bibitem{zhang2018unreasonable} Zhang, Richard, et al. "The unreasonable effectiveness of deep features as a perceptual metric." Proceedings of the IEEE conference on computer vision and pattern recognition. 2018.

\bibitem{brown2020language} Brown, Tom B. "Language models are few-shot learners." arXiv preprint arXiv:2005.14165 (2020).

\bibitem{podell2023sdxl} Podell, Dustin, et al. "Sdxl: Improving latent diffusion models for high-resolution image synthesis." arXiv preprint arXiv:2307.01952 (2023).

\bibitem{brohan2022rt} Brohan, Anthony, et al. "Rt-1: Robotics transformer for real-world control at scale." arXiv preprint arXiv:2212.06817 (2022).

\bibitem{brohan2023rt} Brohan, Anthony, et al. "Rt-2: Vision-language-action models transfer web knowledge to robotic control." arXiv preprint arXiv:2307.15818 (2023).

\bibitem{li2024manipllm} Li, Xiaoqi, et al. "Manipllm: Embodied multimodal large language model for object-centric robotic manipulation." Proceedings of the IEEE/CVF Conference on Computer Vision and Pattern Recognition. 2024.

\bibitem{shah2023vint} Shah, Dhruv, et al. "ViNT: A foundation model for visual navigation." arXiv preprint arXiv:2306.14846 (2023).

\bibitem{liu2024visual} Liu, Haotian, et al. "Visual instruction tuning." Advances in neural information processing systems 36 (2024).

\bibitem{hu2021lora} Hu, Edward J., et al. "Lora: Low-rank adaptation of large language models." arXiv preprint arXiv:2106.09685 (2021).

\bibitem{qin2023supfusion} Qin, Yiran, et al. "SupFusion: Supervised LiDAR-camera fusion for 3D object detection." Proceedings of the IEEE/CVF International Conference on Computer Vision. 2023.

\bibitem{qin2024worldsimbench} Qin, Yiran, et al. "Worldsimbench: Towards video generation models as world simulators." arXiv preprint arXiv:2410.18072 (2024).

\bibitem{yu2025gamefactory} Yu, Jiwen, et al. "GameFactory: Creating New Games with Generative Interactive Videos." arXiv preprint arXiv:2501.08325 (2025).

\bibitem{zhou2024code} Zhou, Enshen, et al. "Code-as-Monitor: Constraint-aware Visual Programming for Reactive and Proactive Robotic Failure Detection." arXiv preprint arXiv:2412.04455 (2024).

\bibitem{zhang2024ad} Zhang, Zaibin, et al. "AD-H: Autonomous Driving with Hierarchical Agents." arXiv preprint arXiv:2406.03474 (2024).

\bibitem{wang2024toward} Wang, Chaoqun, et al. "Toward Accurate Camera-based 3D Object Detection via Cascade Depth Estimation and Calibration." arXiv preprint arXiv:2402.04883 (2024).

\bibitem{huang2024story3d} Huang, Yuzhou, et al. "Story3d-agent: Exploring 3d storytelling visualization with large language models." arXiv preprint arXiv:2408.11801 (2024).

\bibitem{savinov2018semi} Savinov, Nikolay, Alexey Dosovitskiy, and Vladlen Koltun. "Semi-parametric topological memory for navigation." arXiv preprint arXiv:1803.00653 (2018).

\bibitem{majumdar2022zson} Majumdar, Arjun, et al. "Zson: Zero-shot object-goal navigation using multimodal goal embeddings." Advances in Neural Information Processing Systems 35 (2022): 32340-32352.

\bibitem{yadav2023offline} Yadav, Karmesh, et al. "Offline visual representation learning for embodied navigation." Workshop on Reincarnating Reinforcement Learning at ICLR 2023. 2023.

\bibitem{yadav2023ovrl} Yadav, Karmesh, et al. "Ovrl-v2: A simple state-of-art baseline for imagenav and objectnav." arXiv preprint arXiv:2303.07798 (2023).

\bibitem{sun2024fgprompt} Sun, Xinyu, et al. "FGPrompt: fine-grained goal prompting for image-goal navigation." Advances in Neural Information Processing Systems 36 (2024).

\bibitem{zhu2017target} Zhu, Yuke, et al. "Target-driven visual navigation in indoor scenes using deep reinforcement learning." 2017 IEEE international conference on robotics and automation (ICRA). IEEE, 2017.

\bibitem{koh2024generating} Koh, Jing Yu, Daniel Fried, and Russ R. Salakhutdinov. "Generating images with multimodal language models." Advances in Neural Information Processing Systems 36 (2024).

\bibitem{krantz2022instance} Krantz, Jacob, et al. "Instance-specific image goal navigation: Training embodied agents to find object instances." arXiv preprint arXiv:2211.15876 (2022).

\bibitem{schulman2017proximal} Schulman, John, et al. "Proximal policy optimization algorithms." arXiv preprint arXiv:1707.06347 (2017).

\bibitem{anderson2018evaluation} Anderson, Peter, et al. "On evaluation of embodied navigation agents." arXiv preprint arXiv:1807.06757 (2018).

\bibitem{lin2024navcot} Lin, Bingqian, et al. "NavCoT: Boosting LLM-Based Vision-and-Language Navigation via Learning Disentangled Reasoning." arXiv preprint arXiv:2403.07376 (2024).

\bibitem{NavGPT} Zhou, Gengze, Yicong Hong, and Qi Wu. "Navgpt: Explicit reasoning in vision-and-language navigation with large language models." Proceedings of the AAAI Conference on Artificial Intelligence.

\bibitem{hahn2021no} Hahn, Meera, et al. "No rl, no simulation: Learning to navigate without navigating." Advances in Neural Information Processing Systems 34 (2021): 26661-26673.

\bibitem{li2025t2isafety} Li, Lijun, et al. "T2ISafety: Benchmark for Assessing Fairness, Toxicity, and Privacy in Image Generation." arXiv preprint arXiv:2501.12612 (2025).

\bibitem{an2024agfsync} An, Jingkun, et al. "AGFSync: Leveraging AI-Generated Feedback for Preference Optimization in Text-to-Image Generation." arXiv preprint arXiv:2403.13352 (2024).


\end{thebibliography}
\end{sloppypar}

\clearpage
\beginsupplement
\section*{Appendix}
\renewcommand{\thesubsection}{S\arabic{subsection}}

\subsection{\label{chap:S1}PanNuke and MoNuSAC preprocessing}
The PanNuke dataset comprises a set of 7,901 RGB patches, each with dimensions of $256 \times 256$ pixels, which we set as the standard patch size for our analysis. In contrast, the MoNuSAC dataset encompasses 294 images of heterogeneous dimensions. To standardize the MoNuSAC images with our experiments, we implement a standardization protocol. Specifically, for images exceeding the dimensions of $256 \times 256$ pixels, we segment them into equal-sized patches and apply mirror padding to the remaining portions to avoid information loss at the peripherals. Patches with dimensions less than $128 \times 128$ pixels are excluded from the dataset due to the insufficient resolution to capture relevant cellular details. For patches where either dimension falls between 128 and 256 pixels, we employ upsampling to achieve the standard patch size. As a result, we obtain a total of 2,823 RGB patches derived from the MoNuSAC dataset for subsequent analysis. For additional details on the MoNuSAC data preparation process, refer to the source code \cite{Shvetsov_2025a}.
\clearpage

\subsection{\label{chap:S2}Data usage for the methodology}

\counterwithin{figure}{subsection}
\renewcommand{\thefigure}{S\arabic{subsection}}

\begin{figure}[h!]
    \centering
    \includegraphics[width=\textwidth, height=0.85\textheight, keepaspectratio]{images/A2.pdf}
    \caption{Overview of the methodology for cross-labeling, dataset refinement, and model comparison. (1) Cross-relabeling - training and testing cell classification models, (2) Cross-relabeling - using cell classification models to create refined dataset, (3) Fine-tuning and training models for comparison, (4) Student knowledge distillation with refined dataset}
    \label{fig:S2}
\end{figure}
\clearpage

\subsection{\label{chap:S3}Confusion matrices for classification models}
\counterwithin{figure}{subsection}
\renewcommand{\thefigure}{S\arabic{subsection}.\arabic{figure}}

\begin{figure}[h!]
    \centering
    \includegraphics[width=\textwidth, height=0.4\textheight, keepaspectratio]{images/A3_1.pdf}
    \caption{Confusion matrix for PanNuke trained model}
    \label{fig:S3.1}
\end{figure}

\begin{figure}[h!]
    \centering
    \includegraphics[width=\textwidth, height=0.4\textheight, keepaspectratio]{images/A3_2.pdf}
    \caption{Confusion matrix for MoNuSAC trained model}
    \label{fig:S3.2}
\end{figure}

\clearpage

\subsection{\label{chap:S4}Datasets cell counts}

\counterwithin{table}{subsection}
\renewcommand{\thetable}{S\arabic{subsection}}

\begin{table}[h!]
\renewcommand{\arraystretch}{2.0}
\centering
\caption{\label{tab:S4}Cell counts for PanNuke, MoNuSAC and refined datasets. Numbers in parentheses indicate preprocessed cell counts for cell classifier models training and testing.}
%\adjustbox{max width=\textwidth}{%
\begin{tabular}{|l|c|c|c|}
\hline
%\rowcolor{gray!30}
Cell type & PanNuke & MoNuSAC & Refined \\
\hline
Neoplastic & 77,403 (68,031) & - & 105,451 \\
\hline
Epithelial & 26,572 (23,207) & - & 29,926 \\
\hline
Epithelial (benign and malignant) & - & 31,402 & - \\
\hline
Inflammatory & 32,276 & - & - \\
\hline
Lymphocytes & - & 37,045 (33,104) & 65,275 \\
\hline
Neutrophils & - & 1,355 (1,252) & 3,833 \\
\hline
Macrophage & - & 1,842 (1,695) & 3,410 \\
\hline
Dead & 2,908 & - & 2,908 \\
\hline
Connective & 50,585 & - & 50,585 \\
\hline
\end{tabular}
%
%}
\end{table}



\clearpage

\subsection{\label{chap:S5}Definition of validation metrics}
\counterwithin{equation}{subsection}
\renewcommand{\theequation}{\arabic{equation}}

\subsubsection{\label{chap:S5.1}R\textsuperscript{2}}
The coefficient of determination, denoted as $R^2$, is a statistical measure that represents the proportion of variance in the dependent variable that is predictable from the independent variables. In the context of cell quantification in pathology, $R^2$ is used to assess how well the predicted quantities of different cell types in a patch align with the actual quantities observed in the ground truth data, with higher values representing more accurate quantification. $R^2$ is defined as
\begin{equation*}
R^2 = 1 - \frac{\sum_{i=1}^n (y_i - \hat{y}_i)^2}{\sum_{i=1}^n (y_i - \bar{y})^2},
\end{equation*}
where $y_i$ represents the actual number of cells of a specific type in the $i$-th image, $\hat{y}_i$ represents the predicted number of cells of that type in the $i$-th image, $\bar{y}$ is the mean of the actual numbers across all images, and $n$ is the total number of images in the dataset.

The $R^2$ metric has a range of $(-\infty, 1]$. An $R^2$ of 1 indicates perfect prediction, where all predicted values exactly match the actual values. An $R^2$ of 0 suggests that the model explains none of the variability of the response data around its mean. If $R^2$ is negative, it indicates that the model performs worse than a model that simply predicts the mean of the actual values for all observations.

\subsubsection{\label{chap:S5.2}PQ}
Panoptic Quality ($PQ$) is a comprehensive metric used to evaluate the performance of segmentation models in tasks that require both instance segmentation and classification. $PQ$ provides a single score that encapsulates both the detection accuracy (i.e., how many objects were correctly identified) and the segmentation quality (i.e., how accurately the objects' boundaries were delineated). This metric is particularly useful in multiclass scenarios where each pixel is classified into distinct categories, such as different cell types in pathology images.

$PQ$ is calculated as the product of two terms: Detection Quality ($DQ$) and Segmentation Quality ($SQ$). It can be expressed as
\begin{equation*}
PQ = DQ \cdot SQ,
\end{equation*}
where
\begin{equation*}
DQ = \frac{TP}{TP + 0.5\, FP + 0.5\, FN},
\end{equation*}
\begin{equation*}
SQ = \frac{\sum_{(p, g) \in \mathcal{M}} IoU(p, g)}{TP}.
\end{equation*}
In these formulas, $TP$ denotes the number of correctly matched instances between ground truth and prediction, $FP$ denotes the predicted instances that have no corresponding ground truth, $FN$ denotes the ground truth instances that were not detected, $IoU(p, g)$ is the Intersection over Union for a pair of matched instances $p$ (prediction) and $g$ (ground truth), and $\mathcal{M}$ is the set of matched pairs.

The $PQ$ metric is calculated for each class and is averaged across classes to provide a global performance measure.

The $PQ$ score has a range of $[0, 1.0]$, where a higher score indicates better performance in both detecting and segmenting the instances correctly. A $PQ$ of 1 signifies perfect identification and segmentation of all instances, whereas a $PQ$ of 0 indicates that no instances were correctly identified and segmented.

\clearpage

\subsection{\label{chap:S6}Segmentation and Detection quality metrics for teacher and student models}

\begin{table}[h!]
\renewcommand{\arraystretch}{2.0}
\centering
\caption{Segmentation and detection quality for student and teacher models (CI 95\%)}
\label{tab:S6}
%\adjustbox{max width=\textwidth}{%
\begin{tabular}{|l|c|c|}
\hline
%\rowcolor{gray!30}
Metric & Teacher & Student \\
\hline
$SQ_{neoplastic}$ & 0.819 (0.815--0.823) & 0.824 (0.819--0.828) \\
\hline
$SQ_{lymphocyte}$ & 0.795 (0.788--0.802) & 0.790 (0.783--0.796) \\
\hline
$SQ_{connective}$ & 0.770 (0.762--0.776) & 0.780 (0.772--0.786) \\
\hline
$SQ_{dead}$ & 0.659 (0.623--0.688) & 0.657 (0.624--0.695) \\
\hline
$SQ_{epithelial}$ & 0.780 (0.770--0.790) & 0.788 (0.779--0.797) \\
\hline
$SQ_{macrophage}$ & 0.788 (0.760--0.810) & 0.757 (0.730--0.783) \\
\hline
$SQ_{neutrofil}$ & 0.782 (0.761--0.801) & 0.775 (0.759--0.792) \\
\hline
$DQ_{neoplastic}$ & 0.706 (0.692--0.719) & 0.727 (0.712--0.741) \\
\hline
$DQ_{lymphocyte}$ & 0.675 (0.656--0.698) & 0.713 (0.691--0.734) \\
\hline
$DQ_{connective}$ & 0.566 (0.546--0.584) & 0.583 (0.565--0.602) \\
\hline
$DQ_{dead}$ & 0.410 (0.361--0.465) & 0.435 (0.306--0.561) \\
\hline
$DQ_{epithelial}$ & 0.668 (0.639--0.694) & 0.673 (0.644--0.702) \\
\hline
$DQ_{macrophage}$ & 0.657 (0.583--0.727) & 0.615 (0.531--0.703) \\
\hline
$DQ_{neutrofil}$ & 0.691 (0.625--0.753) & 0.729 (0.679--0.778) \\
\hline
\end{tabular}
%
%}
\end{table}

\clearpage

\subsection{\label{chap:S7}QuPath integration method}
We adopt an integration strategy leveraging the paquo \cite{Bayer_AG} library, a Python package that enables direct interaction with QuPath’s internal API, thereby facilitating seamless data exchange without intermediate conversion steps. The data processing pipeline (\hyperref[fig:S7]{Appendix Figure S7}) begins with the acquisition of WSIs and their associated annotations from QuPath, which are represented as Shapely \cite{Gillies_Wel_etal._2024} polygons. Utilizing paquo, we directly read, create, and modify these annotations and detections within a QuPath project in the Python environment. Images are then cropped using these polygons and processed by cell segmentation and classification models employing standard vision processing toolkits such as OpenCV, pyvips, and PyTorch. Additionally, QuPath employs Groovy scripts to initiate a Python process that starts the entire pipeline from QuPath graphical interface: fetching polygons, extracting images from them, and running deep learning model inference on the cropped images. 
The results are returned to QuPath, leveraging paquo's Python bindings to manipulate QuPath data while minimizing the computational overhead typically associated with cross-environment communication.

\counterwithin{figure}{subsection}
\renewcommand{\thefigure}{S\arabic{subsection}}

\begin{figure}[h!]
    \centering
    \includegraphics[width=\textwidth]{images/A7.pdf}
    \caption{QuPath integration workflow using Python environment}
    \label{fig:S7}
\end{figure}

Compared to traditional workflows that involve exporting annotations as GeoJSON, classifying them in Python, and reimporting them into QuPath, our approach offers several advantages. We eliminate the need to switch between programming languages, providing a cohesive and streamlined development process entirely within QuPath software and removing the necessity to use other tools. Meanwhile, we avoid storing annotations as intermediate JSON files unless required for external use or archiving. By conducting the entire inference and post-processing workflow within the Python environment, we leverage the power and flexibility of Python libraries for image processing and machine learning. This approach also enables adjustments to any set of labels and models, thereby improving its applicability.

%\hfill

The distilled model and QuPath integration code are packaged into a Docker container, enabling streamlined execution with the Docker engine. Detailed integration code and deployment instructions can be found in the GitHub repository \cite{Shvetsov_2025b}.

Despite these benefits, we acknowledge that the paquo library is a proof‑of‑concept project in its early development stage and has not been tested across all versions of QuPath.

\clearpage

\subsection{\label{chap:S8}Data and code availability statement}
All datasets, models, and code used in this study are publicly available and can be obtained from the repositories listed below. 
The PanNuke \cite{Gamper_Koohbanani_etal._2019} and MoNuSAC \cite{Verma_Kumar_etal._2021} datasets are publicly accessible, and download information along with detailed descriptions can be found in their respective articles. Preprocessing scripts for PanNuke and MoNuSAC data, as well as individual cell extraction scripts, are available on GitHub \cite{Shvetsov_2025a}. The H-Optimus foundation model used in our experiments can be downloaded from the HuggingFace repository \cite{hoptimus2024}, and model information is available on GitHub \cite{Saillard_Jenatton_etal._2024}. In addition, the integration code for QuPath and the distilled model packaged in a Docker container are provided in the repository \cite{Shvetsov_2025b}, and paquo Python library is available from the authors GitHub repository \cite{Bayer_AG}.
\clearpage

\end{document}


% Appendix
\ifnum\Includeappendix=1{
\appendix
%\addtocontents{toc}{\setcounter{tocdepth}{1}}
\section{Omitted Proofs from Section~\ref{sec:uniform}}\label{appendix:uni}
\subsection{Upper Bound}
\begin{proof}[Proof of Proposition~\ref{prop:envy is good SG}]
First, we note that the envy is a martingale.
\begin{observation}\label{envy is martingale}
For every $i,j\in[N]$, the sequence $\left(\env_{i,j}^t \right)_{t=1}^T$ is a martingale. 
\end{observation}
Furthermore, since $\dift$ is symmetric as Remark~\ref{remark: symmetric dif} hints, we can use the following result to connect its tail behavior and its variance. 
\begin{proposition}\label{thm: symmetric bounded sg}
Let $Y$ be a bounded random variable symmetric around $0$, i.e., for any $y \in \mathbb R$ it holds that $\prb{Y \geq y} = \prb{Y \leq -y}$. Then, $Y$ is $\sqrt{\var{Y}}$-subgaussian.
\end{proposition}
We prove Proposition~\ref{thm: symmetric bounded sg} after the end of this proof. Proposition~\ref{thm: symmetric bounded sg} suggests that $\dift$ is $\sqrt{\var{\dift}}$-subgaussian. Ultimately, Lemma~\ref{sg of sup of martingale} analyzes the subgaussianity parameter of the maximum of the martingale $(\env_{i,j}^t )_t$.
\begin{lemma}\label{sg of sup of martingale}
    Let $M^1, M^2,\dots M^T$ be a martingale with increments $Y^1,Y^2,\dots Y^T$, such that $Y^t \mid M^{t-1}$ is $\sigma_t$-SG. Then $\max_t M^t$ is $\left(\sqrt{\sum_{t=1}^T \sigma_t^2}\right)$-SG.
\end{lemma}
Lemma~\ref{sg of sup of martingale} suggests that $\max_{1\leq t\leq T} \env_{i,j}^t $ is 
$\left(\sqrt{\sum_{l=1}^t \var{\dift}}\right)$-SG, thereby concluding the proof of  Proposition~\ref{prop:envy is good SG}.
\end{proof}


\begin{proof}[Proof of Observation~\ref{envy is martingale}]
Recall that $\env_{i,j}^t = \sum^{t}_{l=1}{\Del{l}{i,j}}$. Since the order of selection at time $t$ is independent of $\ordr_t$, it holds that $\E{\env_{i,j}^{t+1}\mid \env_{i,j}^t} =\env_{i,j}^t$. Moreover, since $\E{\abs{\env_{i,j}^t}} \leq  T < \infty$, the stochastic process $\left(\env_{i,j}^t \right)_{t=1}^T$ is a martingale. 
\end{proof}

\begin{proof}[Proposition \ref{thm: symmetric bounded sg}]
    We begin by examining the Taylor polynomial of the function $f(z)=e^z$ around $0$.
    \[
    e^z=  1 + z + \frac{z^2}{2!} + \frac{z^3}{3!} e^{\xi_z},
    \]
    where the last term is the  Lagrange form of the remainder for some $\xi_z \in [-\abs{z}, \abs{z}]$.
    Using this expansion with $z=\lambda y$ and $\xi_{\lambda y} \in [-\abs{\lambda y}, \abs{\lambda y}]$ gets
    \[
    e^{\lambda y} = 1 + \lambda y + \frac{(\lambda y)^2}{2!} + \frac{(\lambda y)^3}{3!} e^{\xi_{\lambda y}}.
    \]
    If $-a<y<a$ then,
    \[
    e^{\lambda y} = 1 + \lambda y + \frac{(\lambda y)^2}{2!} + \frac{(\lambda y)^3}{3!} e^{\abs{\lambda a}}.
    \]
    Thus,
    \[
    \E{e^{\lambda Y}} \leq \E{1 + \lambda Y + \frac{(\lambda Y)^2}{2} + \frac{(\lambda Y)^3}{3} e^{|\lambda a|}} = 
    1 + \lambda \E{Y} + \frac{\lambda^2}{2}\E{Y^2} +\frac{\lambda^3 e^{|\lambda a|}}{3!} \E{Y^3}.
    \]
    Recall that $Y$ is symmetric around $0$ and hence $\E{Y} = \E{Y^3} = 0$, $\var{Y} = \E{Y^2}$.
    Combining these observations with the above we get
    \[
    \E{e^{\lambda Y}} \leq 1 + \frac{\lambda^2}{2}\var{Y} \leq e^{\frac{\lambda^2 \var{Y}}{2}},
    \]
    where the last inequality is due to that $1+z \leq e^z$ for all $z$.
    That sums up the proof of Lemma~\ref{thm: symmetric bounded sg}.
\end{proof}




\begin{proof}[Proof of Lemma~\ref{sg of sup of martingale}]
We begin by proving that $M^t$ is $\left(\sqrt{\sum_{l=1}^t \sigma_l^2}\right)$-SG, and then address $\max_t M^t$


The base case is $t=1$, where we have $M^1=Y^1$ and $Y^1$ is $\sigma_1$-SG by definition. Next, assume that the statement holds for $t=k-1$. It holds that 
\begin{align*}
\E{e^{\lambda M^{k}}}&=\E{\E{e^{\lambda M^{k}}\mid M^{k-1}}}=\E{\E{e^{\lambda (M^{k-1}+Y^k)}\mid M^{k-1}}} = \E{e^{\lambda M^{k-1}} \E{e^{\lambda Y^k}\mid M^{k-1}}}\\
& \leq \E{e^{\lambda M^{k-1}} e^{\frac{\lambda^2 \sigma_k^2}{2}}} \leq e^{\frac{\lambda^2 \sum_{l=1}^{k-1} \sigma_l^2}{2} }e^{\frac{\lambda^2 \sigma_k^2}{2}} = e^{\frac{\lambda^2 \sum_{l=1}^{k} \sigma_l^2}{2} },
\end{align*}
where we used total expectation, the fact that $Y^k\mid M^{k-1}$ is $\sigma_k$-SG and the inductive assumption.

Next, let $\tau$ denote the r.v. for which $M^\tau = \max_t M^t$. It holds that
\begin{align*}
    \E{e^{\lambda \max_t M^t}} &= \E{e^{\lambda M^\tau}} = \E{\E{e^{\lambda M^\tau} \mid \tau}} \leq \E{\E{e^{\lambda M^\tau} \mid \tau}} \leq \E{e^{\frac{\lambda^2 \sum_{l=1}^{\tau} \sigma_l^2}{2} }} \leq e^{\frac{\lambda^2 \sum_{l=1}^{T} \sigma_l^2}{2} },
\end{align*}
where the second to last step follows from the fact that $M^t$ is $\left(\sqrt{\sum_{l=1}^t \sigma_l^2}\right)$-SG. This completes the proof of Lemma~\ref{sg of sup of martingale}.
\end{proof}

\begin{proof}[Proof of Claim~\ref{claim: sg max}]
Fix $a \in \mathbb{R}$ and denote $Y_{\textnormal{max}} = \max_{i \in [n]}{\{Y_i\}}$ for convenience.
We begin by examining $e^{a\E{Y_{\textnormal{max}} }}$.
Using Jensen’s inequality, we have that
\begin{align}\label{eq:asdfrsdgvbsdfg}
e^{a\E{Y_{\textnormal{max}} }} \leq \E{e^{a Y_{\textnormal{max}}}}= \E{\max_{i \in [n]}{\left\{ e^{a Y_i } \right\}}} \leq
\sum_{i=1}^{n}{\E{e^{a Y_i}}} \leq \sum_{i=1}^{n}{ e^{\frac{a^2\sigma^2}{2}} } = ne^{\frac{a^2\sigma^2}{2}}.    
\end{align}
Taking $\ln$ on both sides of Inequality~\eqref{eq:asdfrsdgvbsdfg},
\[
a\E{Y_{\textnormal{max}}}\leq \ln(n) + \frac{a^2\sigma^2}{2}.
\]
Diving by $a$, we obtain
\begin{align}\label{eq:dasolgbdf}
\E{Y_{\textnormal{max}}}\leq \frac{\ln(n)}{a} + \frac{a\sigma^2}{2}.
\end{align}
Inequality~\eqref{eq:dasolgbdf} holds for all $a\in \mathbb{R}$ and specifically for the minimizer of $\frac{\ln{(n)}}{a}\!+\!\frac{a\sigma^2}{2}$, which is $a=\frac{\sqrt{2 \ln{(n)}}}{\sigma}$.
We complete the proof of the claim by substituting $a$ with this value in Inequality~\eqref{eq:dasolgbdf}.
\end{proof}
\begin{proof}[Proof of Proposition~\ref{thm: threshold var}]
    First, since $\dift \leq 1$ almost surely, $\var{\dift} \leq 1$ holds trivially. For the more challenging expression, the proof relies on the fact that when executing an explore-first algorithm, after at most $K$ sessions of any round, all remaining $N-K$ agents must receive the same reward. Hence, at the end of every round $t$, we can find at least $\binom{N-K+1}{2}$ pairs of agents $i,j$ that satisfy $\dif^t_{i,j} = 0$. In other words, for any $i,j$, it holds that 
    \[
    \prb{{\dif^t_{i,j}}=0}\geq \frac{\binom{N-K+1}{2}}{\binom{N}{2}},
    \]
    where the randomness is taken over the stochasticity of the rewards and the arrival order. From that,
    \begin{align*}
        \var{\dift} & = \E{{\dift}^2} - \E{\dift}^2 = \E{{\dift}^2 \mid {\dift}^2 \neq 0}\cdot \prb{{\dift}^2 \neq 0} - \E{\dift}^2
        \nonumber \\&\leq
        1\cdot \left( 1 - \frac{\binom{N-K+1}{2}}{\binom{N}{2}} \right) - 0 =
        1 - \frac{ \left( N-K+1 \right)\left( N-K \right) }{N\left( N-1 \right)}.
    \end{align*}
    With the help of some algebraic operations, we can simplify the expression.
    \begin{align*}
        \var{\dift} & \leq
        1 - \frac{ \left( N-K+1 \right)\left( N-K \right) }{N\left( N-1 \right)} =
        \frac{ N^2 - N - \left( N^2 - 2NK + K^2 + N - K \right) }{ N\left( N-1 \right)} 
        \\&=
        \frac{2NK -2N + K -K^2 }{ N\left( N-1 \right)} =
        \frac{2N(K-1) - K(K-1) }{N\left( N-1 \right)} =
        \frac{(2N - K)(K-1) }{N\left( N-1 \right)}.
    \end{align*}
    This concludes the proof of Proposition~\ref{thm: threshold var}.
\end{proof}
\begin{proof}[Proof of Corollary~\ref{thm: sqrt TK N}]
The corollary holds since 
\begin{align*}
\E{\max_{1\leq t \leq T} \env^t} &\leq
2\sqrt{\ln{(N)}\sum^T_{t=1}{\var{\dift}}}\leq 
2\sqrt{\ln{(N)}\sum^T_{t=1}{\min\left\{1,\frac{2(K-1)}{N-1} \right\}}}\\
&\leq 2\sqrt{T\ln{(N)}{\frac{2(K-1)}{N-1}}}.    
\end{align*}
\end{proof}

\subsection{Lower Bound}\label{appendix:lower bound}
\begin{claim}\label{claim: uni suff}
For the execution in Example~\ref{example: uni suff}, it holds that $\var{\dift} = \frac{1}{12}$.
\end{claim}
\begin{proof}[Proof of Claim~\ref{claim: uni suff}]
We prove the claim by using the definition of variance.
\begin{align}\label{eq:adfgsfhsaagh}
\var{\dift} & = \E{({\dift})^2}-\E{{\dift}}^2\overset{(1)}{=}\E{\left(\rt{(1)} - \rt{(2)}\right)^2} = \E{{\rt{(1)}}^2} - 2\E{\rt{(1)}\rt{(2)}} + \E{{\rt{(2)}}^2}
\end{align}
Next, we apply the threshold structure and properties of the uniform distribution to Equation~\eqref{eq:adfgsfhsaagh}.
{\thinmuskip=0mu
\medmuskip=0mu plus 0mu minus 0mu
\thickmuskip=1mu plus 1mu minus 1mu
\begin{align*}
\textnormal{Eq. }\eqref{eq:adfgsfhsaagh} &= 
\E{{X_1}^2} - \frac{1}{2}\cdot 2\E{X_1 X_1 \mid X_1 \geq \frac{1}{2}} - \frac{1}{2}\cdot 2\E{ X_1 X_2\mid X_1 < \frac{1}{2}}+ \frac{1}{2}\cdot \E{{X_1}^2 \mid X_1 \geq \frac{1}{2}} + \frac{1}{2}\cdot \E{{X_2}^2 \mid X_1 < \frac{1}{2}} \\
&=
\E{{X_1}^2} - \frac{1}{2} \E{{X_1}^2 \mid X_1 \geq \frac{1}{2}} - \E{ X_1\mid X_1 < \frac{1}{2}} \E{ X_2} + \frac{1}{2} \E{{X_2}^2}
\\&=
\frac{3}{2}\E{{\uni{0,1}}^2} - \frac{1}{2} \E{{\uni{\frac{1}{2},1}}^2} - \E{\uni{0,\frac{1}{2}}}= \frac{3}{2} \cdot \frac{1}{3}- \frac{1}{2} \cdot \frac{7}{12} - \frac{1}{4} \cdot \frac{1}{2} \\
&= \frac{1}{12},
\end{align*}}%
where we have used the independence of $X_1$ and $X_2$.
\end{proof}


\begin{claim}\label{claim:example ber suff}
For the execution in Example~\ref{example: ber suff}, it holds that 
\begin{enumerate}
    \item $\var{\Delta^t} \geq \frac{2p(1 - p)}{N}$.
    \item $\var{\Delta^t} \leq  2(1-p)$.
    \item $\var{\Delta^t} \leq  2pK$.
\end{enumerate}
\end{claim}
\begin{proof}[Proof of Claim~\ref{claim:example ber suff}]
Fix two arbitrary agents $i$ and $j$. Recall that Remark~\ref{remark: symmetric dif} ensures that $\Delta_{i,j}^t$ is identically distributed, regardless of the indexes $i$ and $j$. Let the event $B_q$ indicate that the number of arms that realize a value of 0 is $q$, for $q \in \{0, \ldots, K\}$. Further, let $E$ denote the event that $i$ and $j$ receive different rewards. Then we have:
\[
\var{\Delta_{i,j}^t} = \E{(\Delta_{i,j}^t)^2} 
= \sum_{q=0}^K \bigl( \E{(\Delta_{i,j}^t)^2 \mid B_q, E}\Pr(B_q, E) 
+ \E{(\Delta_{i,j}^t)^2 \mid B_q, \overline{E}}\Pr(B_q, \overline{E}) \bigr).
\]
Since $(\Delta_{i,j}^t)^2$ takes the value 1 under event $E$ and 0 otherwise, we get:
\begin{equation}\label{eq:gknmhmgf}
\var{\Delta_{i,j}^t} = \sum_{q=0}^K (1 \cdot \Pr(B_q, E) + 0 \cdot \Pr(B_q, \overline{E})) 
= \sum_{q=1}^K \Pr(B_q, E).    
\end{equation}
Fix any $q\in[K]$. It holds that 
\begin{align}\label{eq:develop for q}
\Pr(B_q, E) = \Pr(B_q)\Pr(E \mid B_q) =  p (1 - p)^{q}\cdot \frac{2\binom{N-2}{q-1}}{\binom{N}{q}} =  2 p (1 - p)^{q}\frac{q(N-q)}{N(N-1)}.
\end{align}
Combining Equations~\eqref{eq:gknmhmgf} and~\eqref{eq:develop for q}, we get
\begin{align}\label{eq:var def}
\var{\Delta_{i,j}^t} = \sum_{q=1}^K 2 p (1 - p)^{q}\frac{q(N-q)}{N(N-1)}.
\end{align}
Since all summands are positive, we obtain the first part of the claim by bounding from below using only the $q=1$ term. That is, we obtain $\var{\Delta_{i,j}^t} \geq  \frac{2p(1 - p)}{N}$. 

For the other parts of the claim, observe that for every $q \in [K]$, $ \frac{q(N-q)}{N(N-1)} \leq 1$; hence, Equation~\eqref{eq:var def} implies that 
\begin{align}\label{eq:gdhfghsdfg}
\var{\Delta_{i,j}^t} \leq \sum_{q=1}^K 2 p (1 - p)^{q}.
\end{align}
From here, we use Inequality~\eqref{eq:gdhfghsdfg} to obtain the second and third parts of the claim. First,
\begin{align*}
\sum_{q=1}^K 2 p (1 - p)^{q} \leq 2 p (1 - p) \sum_{q=0}^\infty (1 - p)^{q} = \frac{ 2 p (1 - p)}{1-(1-p)} = 2(1 - p);
\end{align*}
thus, $\var{\Delta_{i,j}^t} \leq  2(1-p)$ as the second part of the claim implies. Using a different approach to upper bound Inequality~\eqref{eq:gdhfghsdfg}, we get
\begin{align*}
\sum_{q=1}^K 2 p (1 - p)^{q} \leq 2p \sum_{q=1}^K 1^{q}=2pK
\end{align*}
As the third part of the claim asserts. This completes the proof of Claim~\ref{claim:example ber suff}.
\end{proof}



\iffalse %This is for the "sophisticated bound
Next, we move to the third part of the claim. Applying another approach to upper bound the right-hand-side of Equation~\eqref{eq:var def}, we obtain
\begin{align}\label{eq:var upper}
\sum_{q=1}^K 2 p (1 - p)^{q}\frac{q(N-q)}{N(N-1)} &=  \frac{2p}{N (N-1)} \sum_{q=1}^K (1 - p)^{q}q(N-q) \leq \frac{2p}{N (N-1)} \sum_{q=1}^K (1 - p)^{q}q (N -1) 
\nonumber \\
& = \frac{2p}{N} \sum_{q=1}^K (1 - p)^{q}q   \leq \frac{2p}{N}  \sum_{q=0}^\infty (1 - p)^{q}q.
\end{align}
Due to  Observation~\ref{obs:geo and mul} below,
\[
\sum_{q=0}^\infty (1 - p)^{q}q \leq \frac{1-p}{p^2}.
\]
Combining this with Inequality~\eqref{eq:var upper}, we ultimately obtain 
\[
\var{\Delta_{i,j}^t} \leq \frac{2p}{N} \frac{1-p}{p^2} = \frac{2(1-p)}{N p}.
\]
This completes the proof of Claim~\ref{claim:example ber suff}.
\end{proof}
\begin{observation}\label{obs:geo and mul}
For any $x\in (0,1)$, it holds that 
\[
\sum_{q=0}^\infty q x^{q} \leq \frac{x}{(1-x)^2}.
\]
\end{observation}
\begin{proof}[Proof of Observation~\ref{obs:geo and mul}]
Starting from $\sum_{n=0}^\infty x^n = \frac{1}{1-x}$, we differentiate both sides by $x$ to obtain
\[
\frac{d}{dx}\left(\sum_{n=0}^\infty x^n \right) = \frac{d}{dx}\left( \frac{1}{1-x} \right)\Leftrightarrow \sum_{n=0}^\infty n x^{n-1}  = \frac{1}{(1-x)^2} \Leftrightarrow \sum_{n=0}^\infty n x^{n}  = \frac{x}{(1-x)^2},
\]
where the last transition follows from multiplying both sides by $x$.
\end{proof}
\fi


\begin{proof}[Proof of Proposition~\ref{thm: board}]
    The proof of Proposition~\ref{thm: board} relies on the following algebraic inequality, which we prove after this proof.
    \begin{observation}\label{obs: algebric}
        For any $a\geq 0$, it holds that $2a \geq 3a^2 - a^4$.
    \end{observation}
    
    Recall, $Y$ is a non-negative random variable and therefore $\sqrt{\frac{Y}{\E{Y}}}$ is non-negative as well.
    When setting $a=\sqrt{\frac{Y}{\E{Y}}}$ we have
    \begin{align*}
         \frac{2\sqrt{Y}}{\sqrt{\E{Y}}} \geq \frac{3Y}{\E{Y}} - \frac{Y^2}{\E{Y}^2},
    \end{align*}
    for any value $Y$ can take.
    
    Notice that $\E{Y}\geq 0$, hence, by multiplying each side of the inequality by $\frac{\sqrt{\E{Y}}}{2}$ we get
    \begin{align*}
        \sqrt{Y} \geq \sqrt{\E{Y}}\left(\frac{3Y\E{Y} - Y^2}{2\E{Y}^2}\right).
    \end{align*}
    Taking expectation on both sides yields
    \begin{align*}
        \E{\sqrt{Y}} & \geq
        \E{\sqrt{\E{Y}}\left(\frac{3Y\E{Y} - Y^2}{2\E{Y}^2}\right)} \overset{(1)}{=}
        \sqrt{\E{Y}}\left(\E{\frac{2Y\E{Y}}{2\E{Y}^2}} - \E{\frac{Y^2 - Y\E{Y}}{2\E{Y}^2}}\right)
        \\&\overset{(2)}{=}
        \sqrt{\E{Y}}\left(\frac{2\E{Y}\E{Y}}{2\E{Y}^2} - \frac{\E{Y^2} - \E{Y}\E{Y}}{2\E{Y}^2}\right) \overset{(3)}{=}
        \sqrt{\E{Y}}\left(1 - \frac{\var{Y}}{2\E{Y}^2}   \right) ,
    \end{align*}
    where $(1)$ and $(2)$ hold due to linearity of expectation and $(3)$ is by the definition of variance. This concludes the proof of Proposition~\ref{thm: board}.
\end{proof}

\begin{proof}[Proof of Observation~\ref{obs: algebric}]
    To prove the inequality, it is sufficient to prove that the function $f(a) = a^3 - 3a^2 -2$ is non-negative for all $a\geq 0$.
    It holds that $f'(a) = 3 a^2 - 3$; thus
    \begin{align*}
        f'(a) > 0, & \text{ when } 0 \leq a < 1 \\
        f'(a) = 0, & \text{ when } a = 1 \\
        f'(a) < 0, & \text{ when } 1 \leq a.
    \end{align*}
    I.e., $f(a)$ is monotonically decreasing for $0\leq a <1$ and monotonically increasing for $1 \leq a$. 
    Hence, for all $a \geq 0$ it holds that $f(1) \geq f(a) = 0$
\end{proof}

\begin{proof}[Proof of Proposition~\ref{prop:insufficient}]
We prove the proposition by reiterating the proof of Theorem~\ref{thm: uni lower-bound} while avoiding using the definition of sufficient execution. It suffices to show that the left-hand side of Inequality~\eqref{eq:m,bnhjikw} is constant. Starting with the numerator,
\begin{align}\label{eq:lpods}
\var{ \sum^T_{t=1}{ \dift^2} } &= \sum^T_{t=1} \var{ \dift^2} =  \sum^T_{t=1} \left(\E{(\dift)^4}-\E{(\dift)^2}^2 \right) = \sum^T_{t=1} \left( \var{\dift} - \var{\dift}^2 \right) \nonumber \\
&= T\left( \var{\dift} - \var{\dift}^2 \right),
\end{align}
where we have used the fact that the algorithm is stationary over rounds and that $\dift$ only takes values in the $\{0,1\}$ set. We keep using the superscript $t$ in Equation~\eqref{eq:m,bnhjikw} to ease readability, and it could be any arbitrary $t \in [T]$.

Next, we consider the denominator of the left-hand side of Inequality~\eqref{eq:m,bnhjikw}.
\begin{align}\label{eq:ydots}
2 \E{ \sum^T_{t=1}{(\dift)^2} }^2 =2  \left(\sum^T_{t=1}{\E{(\dift)^2} }\right)^2 = 2  \left(\sum^T_{t=1}{\var{\dift} }\right)^2 =  2 \left(T{\var{\dift} }\right)^2 = 2T^2 \var{\dift}^2
\end{align}
Combining Equations~\eqref{eq:lpods} and~\eqref{eq:ydots}, we get
\begin{align}\label{eq:prea}
\frac{T\left( \var{\dift} - \var{\dift}^2 \right)}{2T^2 \var{\dift}^2}=\frac{\left( 1 - \var{\dift} \right)}{2T \var{\dift}} \leq 
\frac{1}{2T \var{\dift}} \leq \frac{1}{2T}\frac{N}{2p(1-p)},
\end{align}
where the last inequality follows from Claim~\ref{claim:example ber suff}. 
Furthermore, recall that the proposition assumption guarantees that $p\in \left[\frac{N}{cT}, 1-\frac{N}{cT}\right]$ for $c\geq 2$, suggesting that
\[
p (1-p)\geq \frac{N}{cT}\left(1- \frac{N}{cT}\right) \geq \frac{N}{cT}\left(1- \frac{1}{c}\right) \geq \frac{N}{cT}\left(\frac{c-1}{c}\right) \geq \frac{N}{cT}\cdot \frac{1}{2}.
\]
Plugging this into the right-hand-side of Inequality~\eqref{eq:prea},
\begin{align}
 \frac{1}{2T}\frac{N}{2p(1-p)} \leq \frac{N}{4T} \cdot \frac{1}{p(1-p)} \leq \frac{N}{4T} \cdot \frac{2c T}{N}  = \frac{c}{2}.
\end{align}
Having observed that the left-hand-side of Inequality~\eqref{eq:m,bnhjikw} is bounded by a constant w.r.t. $T$, we complete the proof by plugging this constant into Inequality~\eqref{thm: mp uni lower-bound 2}.
\end{proof}



\section{Omitted Proofs from Section~\ref{sec:nudge}}\label{appendix:nudge}

\begin{proof}[Proof of Proposition~\ref{prop G less than M}]
We prove the claim by induction over $\tau$. The first round in the excursion $D(t)$ and our base case is $\tau=\underline{t}+1$. Since the rewards are in the $[0,1]$ interval and $G_i^{\underline{t}} \leq 1$, we know that $G_i^{\underline{t}+1} \leq 2 = M_i^{\underline{t}+1}$. 



Next, assume the claim holds for $\tau$; thus, $G_i^\tau \leq M_i^\tau$. Recall that we are guaranteed that $\tau \in D(t)$. Without loss of generality, assume that at time $\tau$ agents are ordered lexicographically. Particularly,  ${\sigma^\tau(i)}=i,{\sigma^\tau(i+1)}=i+1$  and $G_i^\tau = R^\tau_{\sigma^\tau(i)}- R^\tau_{\sigma^\tau(i+1)}= R^\tau_i- R^\tau_{i+1}$. 
Next, observe that 
\begin{equation}\label{eq:asdgndsfjghm}
    R^{\tau+1}_{\sigma^{\tau+1}(i)} = \min_{j\in [i]}\left\{ 
    R^{\tau}_{j}+r^{\tau+1}_{j} 
    \right\}\leq R^{\tau}_{i}+r^{\tau+1}_{i}.
\end{equation}
Inequality~\eqref{eq:asdgndsfjghm} holds due to our assumption that the rewards are ordered according to agent indices at round $\tau$.  and since no agent in the set $[N]\setminus[i]$ could obtain a higher cumulative reward that agents $[i]$ at round $\tau + 1$ since all rewards are bounded by 1 and $G^\tau_i > 1$. Similarly,
\begin{equation}\label{eq:dsasdhhgtnt}
    R^{\tau+1}_{\sigma^{\tau+1}(i+1)} = \max_{j\in [N]\setminus[i]}\left\{ 
    R^{\tau}_{j}+r^{\tau+1}_{j} 
    \right\}\geq R^{\tau}_{i+1}+r^{\tau+1}_{i+1}.
\end{equation}
Combining Inequalities~\eqref{eq:asdgndsfjghm} and~\eqref{eq:dsasdhhgtnt}, we derive that
\begin{align*}
G_i^{\tau+1} &= R^{\tau+1}_{\sigma^{\tau+1}(i)} - R^{\tau+1}_{\sigma^{\tau+1}(i+1)} \leq R^{\tau}_{i}+r^{\tau+1}_{i} - R^{\tau}_{i+1}-r^{\tau+1}_{i+1} \\
& =G_i^\tau + r^{\tau+1}_{i} - r^{\tau+1}_{i+1} \leq M_i^\tau + r^{\tau+1}_{i} - r^{\tau+1}_{i+1} \\
&= M_i^\tau + r^{\tau+1}_{(i)} - r^{\tau+1}_{(i+1)} = M_i^{\tau+1},
\end{align*}
where we have used the inductive assumption and the fact that nudged arrival order sorts agents in a non-increasing order of rewards (Algorithm~\ref{alg: sugg arr}, Line~\ref{line:mapping}). This completes the proof of Proposition~\ref{prop G less than M}.
\end{proof}





\begin{proof}[Proof of Proposition~\ref{prop: sugg-m concentration}]
The recursive definition of $M^\tau$ implies that for every $\tau \in D(t)$, $M^\tau = 2+ \sum_{n= \underline{t}+2 }^{\tau} r^n_{(i)}-r^n_{(i+1)}$; thus, 
\begin{align}\label{eq:fghbdfgh}
\prb{M^t > n}  = \prb{\sum_{l= \underline{t}+2 }^{\tau} r^l_{(i)}-r^l_{(i+1)}> n-2}
\end{align}
Next, let $B^l$ denote the event that $r^l_{(i)}-r^l_{(i+1)} \neq 0$. Furthermore, let $B(\tau)$ denote the (random) set of rounds for which the event $B^l$ occurs between $\underline{t}+2$ and $\tau$. That is,
\[
B(\tau) = \{ l\mid \underline{t}+2 \leq l \leq \tau, \ind{B^l} \}
\]
As a result, due to Property~\ref{prop:nudge} and the definition of $\tdif$ in Equation~\eqref{eq def tdif},
\begin{equation}\label{eq:sgdjfndb}
%\E{r^l_{(i)}-r^l_{(i+1)}} = \E{r^l_{(i)}-r^l_{(i+1)} \mid B^l}\prb{B^l} \leq 
\E{r^l_{(i)}-r^l_{(i+1)} \mid B^l} \leq - \delta \tdif.
\end{equation}
Rewriting Equation~\eqref{eq:fghbdfgh},
\begin{align}\label{eq:hbngaersd}
\prb{M^t > n}  &= \prb{\sum_{l \in B(\tau)} r^l_{(i)}-r^l_{(i+1)}> n-2} \nonumber \\
& =\sum_{b\subseteq \{\underline{t}+2,\dots \tau \}}\prb{\sum_{l \in b} r^l_{(i)}-r^l_{(i+1)}> n-2 \mid  B(\tau) = b}\prb{B(\tau) = b} \nonumber\\
& \stackrel{*}{=} \sum_{b\subseteq \{\underline{t}+2,\dots \tau \}, \abs{b}\geq n-2}\prb{\sum_{l \in b} r^l_{(i)}-r^l_{(i+1)}> n-2 \mid  B(\tau) = b}\prb{B(\tau) = b} \nonumber\\
& \leq \max_{b\subseteq \{\underline{t}+2,\dots \tau \}, \abs{b}\geq n-2} \prb{\sum_{l \in b} r^l_{(i)}-r^l_{(i+1)}> n-2 \mid  B(\tau) = b} \nonumber \\
& = \max_{b\subseteq \{\underline{t}+2,\dots \tau \}, \abs{b}\geq n-2} \prb{\sum_{l \in b} r^l_{(i)}-r^l_{(i+1)} + \delta \tdif \abs{b}> n-2+\delta \tdif \abs{b} \mid  B(\tau) = b},
\end{align}
where the change in the set over which we sum in $*$ follows since $\abs{r^l_{(i)}-r^l_{(i+1)}}\leq 1$ almost surely. Striving to bound the above, notice that, conditioned on $B(\tau) = b$, $\sum_{l \in b} \left( r^l_{(i)}-r^l_{(i+1)} + \delta \tdif \right)$ forms a super-martingale. Using Azuma-Hoeffding inequality,
\begin{align*}
\textnormal{Inequality }\eqref{eq:hbngaersd} &\leq \max_{b\subseteq \{\underline{t}+2,\dots \tau \}, \abs{b}\geq n-2} \exp\left\{-\frac{(n-2+\delta \tdif \abs{b})^2}{2 \sum_{l\in b} (1+\delta \tdif)^2 }\right\} \nonumber \\
& \stackrel{\delta \tdif \leq 1}{\leq}  \max_{b\subseteq \{\underline{t}+2,\dots \tau \}, \abs{b}\geq n-2} \exp\left\{-\frac{(n-2)^2+(n-2)(\delta \tdif \abs{b})+(\delta \tdif \abs{b})^2}{8\abs{b}}\right\} \nonumber
\\
%& =  \max_{b\subseteq \{\underline{t}+2,\dots \tau \}, \abs{b}\geq n-2} \exp\left\{-\frac{(n-2)^2+(n-2)(\delta \tdif \abs{b})+(\delta \tdif \abs{b})^2}{8\abs{b}}\right\} \nonumber \\
& =  \max_{b\subseteq \{\underline{t}+2,\dots \tau \}, \abs{b}\geq n-2} \exp\Bigg\{-\frac{(n-2)(\delta \tdif)}{8}-\underbrace{\frac{(n-2)^2+(\delta \tdif \abs{b})^2}{8\abs{b}}}_{\geq 0}\Bigg\} \nonumber \\
& \leq  \max_{b\subseteq \{\underline{t}+2,\dots \tau \}, \abs{b}\geq n-2} \exp\left\{\frac{-(n-2)\delta \tdif}{8}\right\} \nonumber \\
& = \exp\left\{-\frac{(n-2)(\delta \tdif)}{8}\right\}.
\end{align*}
This completes the proof of Proposition~\ref{prop: sugg-m concentration}.
\end{proof}



\section{Models Captured by the Nudged Arrival Property}\label{appendix:nudge-models}
%\omer{I do not feel comfortable with this...}
\subsection*{Mallows Model~\cite{mallows1957non}}

The Mallows model prioritizes rankings close to a reference order \( \sigma \). The probability of a sampled ranking \( \pi \) is proportional to \( e^{-\beta d(\pi, \sigma)} \), where \( d(\pi, \sigma) \) is the Kendall's tau distance between \( \pi \) and \( \sigma \), and \( \beta \geq 0 \) is the concentration parameter. For any pair of agents \( i, j \) such that \( \sigma^{-1}(i) < \sigma^{-1}(j) \), the probability that \( i \) precedes \( j \) satisfies (following the detailed argument of \cite[Section 2]{lu2014effective}):
\[
\Pr_{\pi \sim \text{Mallows}}(\pi^{-1}(i) < \pi^{-1}(j)) = \frac{e^{-\beta}}{1 + e^{-\beta}}.
\]
By arithmetic manipulation we get $\delta = \frac{1 - e^{-\beta}}{1 + e^{-\beta}}$. 


\subsection*{Plackett-Luce Model~\cite{marden1996analyzing}}
In the Plackett-Luce model, each agent \( i \) is assigned a positive score \( w_i > 0 \), and the probability of observing a ranking \( \pi \) is given by:
\[
\Pr(\pi) = \prod_{k=1}^{N} \frac{w_{\pi(k)}}{\sum_{j=k}^{N} w_{\pi(j)}}.
\]
For any pair \( i, j \), the probability that \( i \) precedes \( j \) is:
\[
\Pr(\pi^{-1}(i) < \pi^{-1}(j)) = \frac{w_i}{w_i + w_j},
\]
which holds since the model satisfies Luce's choice axiom, which guarantees the independence of the pairwise ranking probabilities from the presence of other options \cite{luce1959individual}. Thus, by setting the scores such that \( \frac{w_i}{w_i + w_j} \geq \frac{1+\delta}{2} \) for every $i$ so that $i$ precedes $j$ in the optimal permutation, the nudged arrival property is satisfied. One way to do it is at each round $t$, recursively set $w_{\pi(1)}^{t} = 1, w_{\pi(i+1)}^{t} = \frac{1+\delta}{1 - \delta} w_{\pi(i)}^{t}$. We thus assume that the weights are not global across rounds but are round-dependent and are adjusted based on the accumulated rewards. This can be interpreted as either the designer nudging different agents more forcefully, or alternatively, in a behavioral approach, users that benefited more from the system in the past are willing to cooperate more with its nudges. 
%\omer{how do we 'set' $w_i$?}


% Bradley-Terry is irrelvant, it's only for n=2 in and of itself. Its importance comes from relation to Mallows and Placket-Luce, but we discuss them independently. 

%\subsection*{Bradley-Terry Model}

%This model directly models pairwise comparisons. Each agent \( i \) is assigned a latent score \( w_i > 0 \). The probability that \( i \) is ranked above \( j \) is:
%\[
%\Pr(\pi^{-1}(i) < \pi^{-1}(j)) = \frac{w_i}{w_i + w_j}.
%\]
%By choosing \( \frac{w_i}{w_i + w_j} = \frac{1+\delta}{2} \), the Bradley-Terry model adheres to the nudged arrival property.

% Noisy sorting is too general, there is no direct implication, though it could be relevant in some way. 

%\subsection*{Noisy Sorting Models}

%Noisy sorting models introduce randomness to the reference order \( \sigma \). The probability of sampling a ranking \( \pi \) decreases as its Kendall tau distance from \( \sigma \) increases. For a pair \( i, j \) with \( \sigma^{-1}(i) < \sigma^{-1}(j) \), the pairwise probability is:
%\[
%\Pr(\pi^{-1}(i) < \pi^{-1}(j)) \geq \frac{1+\delta}{2},
%\]
%provided the noise level is appropriately low. This ensures compliance with the nudged arrival property.

\subsection*{Thurstone-Mosteller Model \cite{ThurstoneModel}}

In the Thurstone-Mosteller model, each agent \( i \) is assigned an independently drawn latent cardinal value \( v_i \sim \mathcal{N}(\mu_i, s^2) \). Thus, the probability that \( i \) is ranked above \( j \) is independent of all other draws besides the pairwise draws, and is exactly the probability that drawing from $i$'s normal variable exceeds drawing from $j$'s normal variable. The difference of two normal variables is normal by itself, with mean $\mu_i - \mu_j$ (the means difference) and variance $2s^2$ (the sum of variances), and we are interested in the probability that this variable is above $0$. We can normalize and shift the mean, and get that the probability that \( i \) is ranked above \( j \) is:
\[
\Pr(\pi^{-1}(i) < \pi^{-1}(j)) = \Phi\left(\frac{\mu_i - \mu_j}{\sqrt{2}s}\right),
\]
where \( \Phi \) is the CDF of the standard normal distribution $N(0,1)$. We can thus tune $\mu_i, \mu_j$ (with some globally-set $s$) so that for every $i,j$,  \( \Pr(\pi^{-1}(i) < \pi^{-1}(j)) \geq \frac{1+\delta}{2} \), and the nudged arrival property is satisfied. One way to do that is by calculating the constant shift of the mean $\delta\mu$ that satisfies \( \Pr(\pi^{-1}(i) < \pi^{-1}(j)) = \frac{1+\delta}{2}\), and have
\[
\mu_{\pi(i)}^{t} = \mu_{\pi(1)}^{t} + (i-1)\cdot \delta\mu. 
\]

%\omer{constant shifts of the means}
%---

%Together, these examples demonstrate that the nudged arrival property is sufficiently general to subsume a wide range of ranking models while providing a coherent and interpretable foundation for reasoning about orderings.

\section{Omitted Proofs from Section~\ref{sec:extensions}}
\newcommand{\at}[1]{a_{#1}^t}
\paragraph{Additional notation for this section} The analysis in this section makes extensive use of the notation $\at{(q)}$ for $q \in\{1,2\}$ and $t\in [T]$, denoting the arm pulled in session $i$ of round $t$. 
\begin{proof}[Proof of Theorem~\ref{thm:ef1evny+sw}]
The first part of the proof is given in the body of the paper; hence, we move to the second part, i.e., showing that $\sw = (1+\frac{1}{16})T$.

Fix any $t\in[T]$. We analyze the expected sum of rewards obtained in round $t$, $\E{\rt{(1)}+\rt{(2)}}$.
Notice that $\E{\rt{(1)}} = \E{a_1} = \frac{1}{2}$.
As for $\rt{(2)}$, we are uncertain about the arm the algorithm pulls, but can use total expectation:
\begin{align}\label{eq: ef11}
    \E{\rt{(2)}} & = \E{\rt{(2)} \mid \rt{(1)} > \frac{1}{2}}\cdot \prb{\rt{(1)} > \frac{1}{2}} +  \E{\rt{(2)} \mid \rt{(1)} \leq \frac{1}{2}}\cdot \prb{\rt{(1)} \leq \frac{1}{2}}
    \nonumber\\&= \frac{3}{8} + \frac{1}{2}\cdot \E{\rt{(2)} \mid \rt{(1)} \leq \frac{1}{2}},
\end{align}
where we have used Line~\ref{efclin:pull_a1_again} of $\efc$ for replacing $\E{\rt{(2)} \mid \rt{(1)} > \frac{1}{2}}$ with $\frac{3}{4}$, since arm $a_1$ is pulled for the second session as well. Simplifying the term $\E{\rt{(2)} \mid \rt{(1)} \leq \frac{1}{2}}$ and using the fact that $\E{\rt{(r)} \mid \rt{(1)} \leq \frac{1}{2}}=\frac{1}{4}$, we get,
\begin{align*}
    \E{\rt{(2)} \mid \rt{(1)} \leq \frac{1}{2}} &
    = \E{\rt{(2)} \mid \at{(2)} = \at{(1)}, \rt{(1)}} \cdot \prb{\at{(2)} = \at{(1)} \mid \rt{(1)} \leq \frac{1}{2}}
    \\& \qquad + \E{\rt{(2)} \mid \at{(2)} \neq \at{(1)}, \rt{(1)}} \cdot \prb{\at{(2)} \neq \at{(1)} \mid \rt{(1)} \leq \frac{1}{2}}
    \\&= \frac{1}{4} \cdot \prb{\at{(2)} = \at{(1)} \mid \rt{(1)} \leq \frac{1}{2}} + \frac{1}{2} \cdot \prb{\at{(2)} \neq \at{(1)} \mid \rt{(1)} \leq \frac{1}{2}}
    \\& = \frac{1}{4} + \frac{1}{4}\cdot \prb{\at{(2)} \neq \at{(1)} \mid \rt{(1)} \leq \frac{1}{2}}.
\end{align*}
Consequently, all that is left is to understand how often $\efc$ pulls the second arm when the first arm yields a low reward. Using Proposition~\ref{prop:ef1 open arm}, we obtain
\[
\E{\rt{(2)} \mid \rt{(1)} \leq \frac{1}{2}} =
\frac{1}{4} + \frac{1}{4} \cdot \prb{\at{(2)} \neq \at{(1)} \mid \rt{(1)} \leq \frac{1}{2}} \geq \frac{3}{8}.
\]
Using the above inequality and Equation~\eqref{eq: ef11}, we get
\[
\E{\rt{(2)}} \geq \frac{3}{8} + \frac{1}{2}\cdot \frac{3}{8}= \frac{9}{16}.
\]
Since this holds for any arbitrary $t$, by summing over all rounds, we get
\[
SW(EF1) = \E{\sum_{t=1}^{T}{ \rt{(1)} + \rt{(2)} }} = \sum_{t=1}^{T}{ \E{ \rt{(1)} + \rt{(2)} } } \geq \sum_{t=1}^{T}{ \frac{1}{2} + \frac{9}{16}} = \left( 1+ \frac{1}{16} \right)T.
\]
This concludes the proof of Theorem~\ref{thm:ef1evny+sw}.
\end{proof}

\begin{proof}[Proof of Proposition~\ref{prop:ef1 uni dominance}]
We prove the claim with induction over the round index $t$.
The base step, i.e., $t=0$, is straightforward. Fix any $x\in [0,1]$, and observe that
\[
\prb{\env^0 \leq x} = \prb{0 \leq x} = 1 \geq x.
\]
We move forward to the inductive step. Assume the claim holds for round $t-1$, and let us prove the claim for~$t$. First, notice that if $\at{(2)} = \at{(1)}$, then $\env^t=\env^{t-1}$.
Based on the inductive assumption, the distribution of $\env^{t-1}$ is stochastically dominated by $\uni{0,1}$, and thus so is the distribution of $\env^t$.

Otherwise, from here on we assume $\at{(2)} \neq \at{(1)}$. We continue with an extensive case analysis. We define the following six events $A_1,\dots, A_6$. Each event consists of the conditions that cause the algorithm to pull a different arm in the second session and the outcome of that round:
\begin{align*}
    &A_1 := \left( \rt{(1)} \leq \frac{1}{2} \right) \wedge \left(R^{t-1}_{(1)} = R^{t-1}_{(2)}\right) \wedge  \left( R^{t-1}_{(1)} + \rt{(1)} \geq R^{t-1}_{(2)} + \rt{(2)} \right), \\
    &A_2 := \left( \rt{(1)} \leq \frac{1}{2} \right) \wedge \left(R^{t-1}_{(1)} = R^{t-1}_{(2)}\right) \wedge  \left( R^{t-1}_{(2)} + \rt{(2)} > R^{t-1}_{(1)} + \rt{(1)} \right), \\
    &A_3 := \left( \rt{(1)} \leq \frac{1}{2} \right) \wedge \left(R^{t-1}_{(1)} > R^{t-1}_{(2)} \right) \wedge \left( R^{t-1}_{(1)} - R^{t-1}_{(2)} \leq 1 - \rt{(1)} \right) \wedge \left( R^{t-1}_{(1)} + \rt{(1)} \geq R^{t-1}_{(2)} + \rt{(2)} \right), \\
    &A_4 := \left( \rt{(1)} \leq \frac{1}{2} \right) \wedge \left(R^{t-1}_{(1)} > R^{t-1}_{(2)} \right) \wedge \left( R^{t-1}_{(1)} - R^{t-1}_{(2)} \leq 1 - \rt{(1)} \right) \wedge \left( R^{t-1}_{(2)} + \rt{(2)} > R^{t-1}_{(1)} + \rt{(1)} \right), \\
    &A_5 := \left( \rt{(1)} \leq \frac{1}{2} \right) \wedge \left( R^{t-1}_{(2)} > R^{t-1}_{(1)} \right) \wedge  \left( R^{t-1}_{(2)} - R^{t-1}_{(1)} \leq \rt{(1)} \right) \wedge \left( R^{t-1}_{(1)} + \rt{(1)} \geq R^{t-1}_{(2)} + \rt{(2)} \right), \\
    &A_6 := \left( \rt{(1)} \leq \frac{1}{2} \right) \wedge \left( R^{t-1}_{(2)} > R^{t-1}_{(1)} \right) \wedge  \left( R^{t-1}_{(2)} - R^{t-1}_{(1)} \leq \rt{(1)} \right) \wedge \left( R^{t-1}_{(2)} + \rt{(2)} > R^{t-1}_{(1)} + \rt{(1)} \right).
\end{align*}
Notice that 
\begin{observation}\label{obs:partition}
Given $\at{(2)} \neq \at{(1)}$, the events $A_1, \dots, A_6$ partition the space of all options for $R^{t-1}_{(1)}, R^{t-1}_{(2)}, r^t_{(1)}, r^t_{(2)}$.
\end{observation}
Equipped with Observation~\ref{obs:partition}, we turn to analyze $\env^t$ under $A_1, \dots, A_6$. Fix any arbitrary $x \in [0,1]$.
%In each case $i$, $i\in \{1,\dots,6\}$ we condition the random variables $\env_t,R^{t-1}_1, R^{t-1}_2, r^t_{(1)}, r^t_{(2)}$ on $A_i$.
%; hence, under this event, $r^t_{(1)} \sim \uni{0,\frac{1}{2}}$ 
\begin{itemize}
    \item Case $A_1$.
    Under the conditions of event $A_1$ we have that $\rt{(1)} \geq  \rt{(2)}$.
    Recall that $\rt{(2)} \sim \uni{0,1}$, but considering the latter we know that $ \rt{(2)}\mid A_1 \sim \uni{0,\rt{(1)}}$.
    Given that $R^{t-1}_{(1)} + \rt{(1)} \geq R^{t-1}_{(2)} + \rt{(2)}$, we know the envy at the end of round $t$ is exactly $\env^t = R^{t-1}_{(1)} + \rt{(1)} - R^{t-1}_{(2)} - \rt{(2)} = \rt{(1)} -\rt{(2)}$;
    hence, $\env^{t}\mid A_1 \sim \uni{\rt{(1)} - \rt{(1)},\rt{(1)} - 0}$, i.e., $\env^{t}\mid A_1 \sim \uni{0,\rt{(1)}}$. Therefore,
    \begin{align*}
        \prb{\env^t \leq x \mid A_1} & = \prb{\uni{0, r^t_{(1)}}\leq x \mid r^t_{(1)} \leq \frac{1}{2}} \geq \prb{\uni{0, \frac{1}{2}} \leq x}
        \\& \geq \prb{\uni{0, 1} \leq x} = x.
    \end{align*}
    %Because $\rt{(1)} \leq \frac{1}{2}$, in the worst case $\env^{t}\mid A_1 \sim \uni{0,\frac{1}{2}}$.

    \item Case $A_2$.
    This case is similar to that of $A_1$, only now $ \rt{(2)}\mid A_2 \sim \uni{\rt{(1)}, 1}$ and $\env^t = \rt{(2)} -\rt{(1)}$, resulting with $\env^{t}\mid A_2 \sim \uni{\rt{(1)} - \rt{(1)},1 -\rt{(1)}}$, i.e., $\env^{t}\mid A_2 \sim \uni{0,1 -\rt{(1)}}$.
    Hence,
    \begin{align*}
        \prb{\env^t \leq x \mid A_2 } \geq \prb{\uni{0, 1} \leq x} = x.
    \end{align*}

    \item Case $A_3$.
    Under the conditions of $A_3$ we have that the envy after $t$ rounds is exactly $\env^t = R^{t-1}_{(1)} + \rt{(1)} -\left(R^{t-1}_{(2)} + \rt{(2)}\right)$.
    Since $R^{t-1}_{(1)} + \rt{(1)} \geq R^{t-1}_{(2)} + \rt{(2)}$, $\rt{(2)}$ is now a uniform random variable between $0$ and the minimum between $\left\{1,  R^{t-1}_{(1)} + \rt{(1)} - R^{t-1}_{(2)}\right\}$.
    Due to the guarantee $R^{t-1}_{(1)} - R^{t-1}_{(2)} \leq 1 - \rt{(1)}$ we can finally see that
    \[
    - \rt{(2)} \mid A_3 \sim \uni{-\left(R^{t-1}_{(1)} + \rt{(1)} - R^{t-1}_{(2)}\right), 0};
    \]
    thus,
    \[
    \env^t \mid A_3 \sim \uni{R^{t-1}_{(1)} + \rt{(1)} -R^{t-1}_{(2)} -\left(R^{t-1}_{(1)} + \rt{(1)} - R^{t-1}_{(2)}\right),  R^{t-1}_{(1)} + \rt{(1)} -R^{t-1}_{(2)}}.
    \]
    Finally,
    \begin{align*}
        \prb{\env^t \leq x | A_3} & = \prb{\uni{0, R^{t-1}_{(1)} - R^{t-1}_{(2)}+ \rt{(1)}} \leq x | A_3}
        \\& \geq \prb{\uni{0, 1} \leq x} = x.
    \end{align*}

    \item Case $A_4$.
    Under the conditions of $A_4$ we have that $\rt{(2)}$ is a uniform random variable distributed between $R^{t-1}_{(1)} - R^{t-1}_{(2)} + \rt{(1)}$ and $1$.
    The envy after round $t$ is exactly $R^{t-1}_{(2)} +\rt{(2)}- R^{t-1}_{(1)}-\rt{(1)}$ and thus it is a uniform random variable between $R^{t-1}_{(2)} - R^{t-1}_{(1)}-\rt{(1)} +\left( R^{t-1}_{(1)} - R^{t-1}_{(2)} + \rt{(1)} \right)$ and $R^{t-1}_{(2)} - R^{t-1}_{(1)}-\rt{(1)} +1$.
    I.e., $\env^t \mid A_4 \sim \uni{0,R^{t-1}_{(2)} - R^{t-1}_{(1)}-\rt{(1)} +1}$. Recall $A_4$ suggests $0 > R^{t-1}_{(2)} -R^{t-1}_{(1)} $; finally,
    \begin{align*}
        \prb{\env^t \leq x | A_4} & = \prb{\uni{0, R^{t-1}_{(2)} - R^{t-1}_{(1)}-\rt{(1)} +1} \leq x | A_4}  \\& \geq
        \prb{\uni{0, 0 - \rt{(1)} +1} \leq x } \geq \prb{\uni{0, 1} \leq x} = x,
    \end{align*}
    as $\rt{(1)}\geq 0$.
    Note that under $A_4$ it holds that $R^{t-1}_{(1)} - R^{t-1}_{(2)} \leq 1 - \rt{(1)} $ and thus $0 \leq R^{t-1}_{(2)} - R^{t-1}_{(1)}-\rt{(1)} +1$ almost surely.
    
    \item Case $A_5$.
    Under the conditions of $A_5$, similarly to $A_3$, the envy at the end of round $t$ is exactly $\env^t = R^{t-1}_{(1)} + \rt{(1)} - R^{t-1}_{(2)} + \rt{(2)}$ and
    \[
    - \rt{(2)} \mid A_5 \sim \uni{-\left(R^{t-1}_{(1)} + \rt{(1)} - R^{t-1}_{(2)}\right), 0};
    \]
    thus,
    \[
    \env^t \mid A_5 \sim \uni{R^{t-1}_{(1)} + \rt{(1)} -R^{t-1}_{(2)} -\left(R^{t-1}_{(1)} + \rt{(1)} - R^{t-1}_{(2)}\right),  R^{t-1}_{(1)} + \rt{(1)} -R^{t-1}_{(2)}}.
    \]
    Finally,
    \begin{align*}
        \prb{\env^t \leq x | A_5} & = \prb{\uni{0, R^{t-1}_{(1)} - R^{t-1}_{(2)}+ \rt{(1)}} \leq x | A_5}
        \\& \geq \prb{\uni{0, 0 + \rt{(1)}} \leq x | A_5}
        \geq \prb{\uni{0, \frac{1}{2}} \leq x} 
        \\& \geq \prb{\uni{0, 1} \leq x} = x.
    \end{align*}

    \item Case $A_6$.
    Under the conditions of $A_6$, similarly to $A_4$, the envy at the end of round $t$ is exactly $\env^t = R^{t-1}_{(2)} + \rt{(2)} - R^{t-1}_{(1)} - \rt{(1)}$ and
    \[
    \rt{(2)} \mid A_6 \sim \uni{R^{t-1}_{(1)}+\rt{(1)} -R^{t-1}_{(2)}, 1};
    \]
    thus,
    \[
    \env^t \mid A_6 \sim \uni{R^{t-1}_{(2)} - R^{t-1}_{(1)} - \rt{(1)} + R^{t-1}_{(1)}+\rt{(1)} -R^{t-1}_{(2)}, R^{t-1}_{(2)} - R^{t-1}_{(1)} - \rt{(1)} + 1}.
    \]
    Finally,
    \begin{align*}
        \prb{\env^t \leq x | A_6} & = \prb{\uni{0,R^{t-1}_{(2)} - R^{t-1}_{(1)} - \rt{(1)} + 1} \leq x | A_6}
        \\& \geq \prb{\uni{0,1} \leq x} = x,
    \end{align*}
    where the inequality holds due to $R^{t-1}_{(2)} - R^{t-1}_{(1)} \leq \rt{(1)}$.    
\end{itemize}
We have shown that the inductive step holds under all cases; thereby, the proof of Proposition~\ref{prop:ef1 uni dominance} is complete.
\end{proof}
\begin{proof}[Proof of Proposition~\ref{prop:ef1 open arm}]
We prove the statement using case analysis. We partition the space of events $a^t_{(2)} \neq a^t_{(1)}$ conditioning on $\rt{(1)} \leq \frac{1}{2}$:
\begin{align*}
    &B_1 := R^{t-1}_{(1)} = R^{t-1}_{(2)} \\
    &B_2 := \left(R^{t-1}_{(1)} > R^{t-1}_{(2)} \right) \wedge \left( R^{t-1}_{(1)} - R^{t-1}_{(2)} \leq 1 - \rt{(1)} \right) \\
    &B_3 := \left( R^{t-1}_{(2)} > R^{t-1}_{(1)} \right) \wedge  \left( R^{t-1}_{(2)} - R^{t-1}_{(1)} \leq \rt{(1)} \right).
\end{align*}
Therefore
\begin{align*}
    \prb{a^t_{(2)} \neq a^t_{(1)} \mid \rt{(1)} \leq \frac{1}{2}} & = \prb{ B_1\vee B_2 \vee B_3  \mid \rt{(1)} \leq \frac{1}{2}}
    \\& = \prb{B_1 \mid \rt{(1)} \leq \frac{1}{2} } + \prb{B_2 \mid \rt{(1)} \leq \frac{1}{2} } + \prb{B_3 \mid \rt{(1)} \leq \frac{1}{2} }.
\end{align*}
We prove each part separately, beginning with the event $B_1$.

Since the distributions of the rewards are continuous, event $B_1$ occurs if and only if until round $t$ both agents receive the same rewards from the same arm.
In this case, the algorithm pulls the same arm in both sessions if it yields a reward greater than $\frac{1}{2}$.
Therefore, we must have
\[
    \prb{r_{(1)}^\tau  = r_{(2)}^\tau} = \prb{r_{(1)}^\tau  > \frac{1}{2}}
\]
for all $\tau < t$; hence,
\begin{align}\label{B1}
\prb{B_1 \mid \rt{(1)} > \frac{1}{2}} =\prb{B_1} = \prb{\forall \tau<t : r_{(2)}^\tau > \frac{1}{2}} = \left(\frac{1}{2}\right)^{t-1}.
\end{align}

Next, we examine event $B_2$. Using Bayes formula,
\begin{align*}
    &\prb{B_2 \mid \rt{(1)} \leq \frac{1}{2} } =
    \prb{ \left(R^{t-1}_{(1)} > R^{t-1}_{(2)} \right) \wedge \left( R^{t-1}_{(1)} - R^{t-1}_{(2)} \leq 1 - \rt{(1)} \right) \mid \rt{(1)} \leq \frac{1}{2} }
    \\& = \prb{R^{t-1}_{(1)} - R^{t-1}_{(2)} \leq 1 - \rt{(1)} \mid R^{t-1}_{(1)} > R^{t-1}_{(2)}, \rt{(1)} \leq \frac{1}{2} }\cdot \prb{R^{t-1}_{(1)} > R^{t-1}_{(2)} \mid \rt{(1)} \leq \frac{1}{2}}.
\end{align*}
Notice that given $\rt{(1)} \leq \frac{1}{2}$, the random variable $1- \rt{(1)}$ is $\uni{\frac{1}{2}, 1}$ distributed. Similar arguments holds for $\rt{(1)} \mid \rt{(1)} > \frac{1}{2}$; thus, 
\begin{align*}
&\prb{R^{t-1}_{(1)} - R^{t-1}_{(2)} \leq 1 - \rt{(1)} \mid R^{t-1}_{(1)} > R^{t-1}_{(2)}, \rt{(1)} \leq \frac{1}{2} }
\\&=
\prb{R^{t-1}_{(1)} - R^{t-1}_{(2)} \leq \rt{(1)} \mid R^{t-1}_{(1)} > R^{t-1}_{(2)}, \rt{(1)} > \frac{1}{2} }.
\end{align*}
Simplifying the above,
\begin{align}\label{B2}
    & \prb{B_2 \mid \rt{(1)} \leq \frac{1}{2} }
    \nonumber\\&
    = \prb{R^{t-1}_{(1)} - R^{t-1}_{(2)} \leq \rt{(1)} \mid R^{t-1}_{(1)} > R^{t-1}_{(2)}, \rt{(1)} > \frac{1}{2} }\cdot \prb{R^{t-1}_{(1)} > R^{t-1}_{(2)} \mid \rt{(1)} \leq \frac{1}{2}}\nonumber \\
    & = \prb{R^{t-1}_{(1)} - R^{t-1}_{(2)} \leq \rt{(1)} \mid R^{t-1}_{(1)} > R^{t-1}_{(2)}, \rt{(1)} > \frac{1}{2} }\cdot \prb{R^{t-1}_{(1)} > R^{t-1}_{(2)}}
    .
\end{align}

As for event $B_3$, recall that the arrival order is uniform. As a result, $R^{t-1}_{(1)}, R^{t-1}_{(2)}$ are independent in $\rt{(1)}$. Leveraging this fact,
\begin{align}\label{B3}
    & \prb{B_3 \mid \rt{(1)} \leq \frac{1}{2} } =
    \prb{\left( R^{t-1}_{(2)} > R^{t-1}_{(1)} \right) \wedge  \left( R^{t-1}_{(2)} - R^{t-1}_{(1)} \leq \rt{(1)} \right) \mid \rt{(1)}> \frac{1}{2}}\nonumber \\
    & =  \prb{\left( R^{t-1}_{(1)} > R^{t-1}_{(2)} \right) \wedge  \left( R^{t-1}_{(1)} - R^{t-1}_{(2)} \leq \rt{(1)} \right) \mid \rt{(1)}> \frac{1}{2}}\nonumber \\
    & = \prb{R^{t-1}_{(1)} - R^{t-1}_{(2)} \leq \rt{(1)} \mid R^{t-1}_{(1)} > R^{t-1}_{(2)}, \rt{(1)} \leq \frac{1}{2} }\cdot \prb{R^{t-1}_{(1)} > R^{t-1}_{(2)} \mid \rt{(1)} \leq \frac{1}{2}}\nonumber\\
    & = \prb{R^{t-1}_{(1)} - R^{t-1}_{(2)} \leq \rt{(1)} \mid R^{t-1}_{(1)} > R^{t-1}_{(2)}, \rt{(1)} \leq \frac{1}{2} }\cdot \prb{R^{t-1}_{(1)} > R^{t-1}_{(2)}},
\end{align}
where the last two equalities hold from the same arguments as in the analysis of event $B_2$. Combining Equalities \eqref{B2} and \eqref{B3}, we get
\begin{align*}
 \prb{B_2 \mid \rt{(1)} \leq \frac{1}{2} } & + \prb{B_3 \mid \rt{(1)} \leq \frac{1}{2} }\\ &= \prb{R^{t-1}_{(1)} > R^{t-1}_{(2)}} \cdot\left(
 \prb{R^{t-1}_{(1)} - R^{t-1}_{(2)}   \leq \rt{(1)} \mid R^{t-1}_{(1)} > R^{t-1}_{(2)}, \rt{(1)} > \frac{1}{2} } \right. \\& + \left.
  \prb{R^{t-1}_{(1)} - R^{t-1}_{(2)} \leq \rt{(1)} \mid R^{t-1}_{(1)} > R^{t-1}_{(2)}, \rt{(1)} \leq \frac{1}{2} }\right) \\ & = 
  \frac{1}{2} \left(1 - \frac{1}{2^{t-1}} \right) \cdot\left(
 \prb{R^{t-1}_{(1)} - R^{t-1}_{(2)}   \leq \rt{(1)} \mid R^{t-1}_{(1)} > R^{t-1}_{(2)}, \rt{(1)} > \frac{1}{2} } \right. \\& + \left.
  \prb{R^{t-1}_{(1)} - R^{t-1}_{(2)} \leq \rt{(1)} \mid R^{t-1}_{(1)} > R^{t-1}_{(2)}, \rt{(1)} \leq \frac{1}{2} }\right),
\end{align*}
where the last equality is due to Equation~\eqref{B1}. Next, notice that 
\begin{align*}
\frac{1}{2} \left(1 - \frac{1}{2^{t-1}} \right) &\left(
\prb{R^{t-1}_{(1)} - R^{t-1}_{(2)}   \leq \rt{(1)} \mid R^{t-1}_{(1)} > R^{t-1}_{(2)}, \rt{(1)} > \frac{1}{2} } \right. \\+ & \left.
\prb{R^{t-1}_{(1)} - R^{t-1}_{(2)} \leq \rt{(1)} \mid R^{t-1}_{(1)} > R^{t-1}_{(2)}, \rt{(1)} \leq \frac{1}{2} }\right)\\=
\frac{1}{2} \left(1 - \frac{1}{2^{t-1}} \right) &\left(
\prb{R^{t-1}_{(1)} - R^{t-1}_{(2)}   \leq \rt{(1)} \mid R^{t-1}_{(1)} > R^{t-1}_{(2)}, \rt{(1)} > \frac{1}{2} } \frac{\Pr\left(r_{(1)}^t > \frac{1}{2}\Big| R_1^{t-1} > R_2^{t-1} \right)}{\Pr\left(r_{(1)}^t > \frac{1}{2}\Big| R_1^{t-1} > R_2^{t-1} \right)} \right. \\+ & \left.
\prb{R^{t-1}_{(1)} - R^{t-1}_{(2)} \leq \rt{(1)} \mid R^{t-1}_{(1)} > R^{t-1}_{(2)}, \rt{(1)} \leq \frac{1}{2} } \frac{\Pr\left(r_{(1)}^t\leq \frac{1}{2}\Big| R_1^{t-1} > R_2^{t-1} \right)}{\Pr\left(r_{(1)}^t\leq \frac{1}{2}\Big| R_1^{t-1} > R_2^{t-1} \right)} \right)
\\ = \frac{1}{2}  \left(1 - \frac{1}{2^{t-1}} \right) &\cdot \frac{1}{\frac{1}{2}} \cdot 
\prb{R^{t-1}_{(1)} - R^{t-1}_{(2)} \leq \rt{(1)} \mid R^{t-1}_{(1)} > R^{t-1}_{(2)} },
\end{align*}
where the last equation is based on the law of total probability and the fact that 
\[
\prb{\rt{(1)} > \frac{1}{2} \mid R^{t-1}_{(1)} > R^{t-1}_{(2)}} = \prb{\rt{(1)} \leq \frac{1}{2} \mid R^{t-1}_{(1)} > R^{t-1}_{(2)}} = \frac{1}{2}.
\]
Combining the latter with Equation~\eqref{B1}, we have
\begin{align*}
\prb{a^t_{(2)} \neq a^t_{(1)} \mid \rt{(1)} \leq \frac{1}{2}} & = \left(\frac{1}{2}\right)^{t-1}+ \left(1 - \frac{1}{2^{t-1}} \right)  
\prb{R^{t-1}_{(1)} - R^{t-1}_{(2)} \leq \rt{(1)} \mid R^{t-1}_{(1)} > R^{t-1}_{(2)} }\\
& = \left(\frac{1}{2}\right)^{t-1}+ \left(1 - \frac{1}{2^{t-1}} \right)  
\prb{\abs{R^{t-1}_{(1)} - R^{t-1}_{(2)}} \leq \rt{(1)} \mid R^{t-1}_{(1)} > R^{t-1}_{(2)} }\\
& = \left(\frac{1}{2}\right)^{t-1}+ \left(1 - \frac{1}{2^{t-1}} \right) 
\prb{\env^{t-1} \leq \rt{(1)} }\geq \prb{\env^{t-1} \leq \rt{(1)} }.
\end{align*}
To finish the proof we use Proposition~\ref{prop:ef1 uni dominance}, which implies that
\begin{align*}
    \prb{\env^{t-1} \leq \rt{(1)} } & = \int_{0}^{1} \prb{\env^{t-1} \leq u}\cdot f_{\uni{0,1}} \,du
    \geq \int_{0}^{1} u\cdot f_{\uni{0,1}} \,du
    \\&=\E{\uni{0,1}} = \frac{1}{2};
\end{align*}
thus,
\begin{align*}
\prb{a^t_{(2)} \neq a^t_{(1)} \mid \rt{(1)} \leq \frac{1}{2}} \geq 
\prb{\env^{t-1} \leq \rt{(1)} } \geq
\frac{1}{2}.
\end{align*}
This concludes the proof of Proposition~\ref{prop:ef1 open arm}.
\end{proof}

\section{Average Envy}
\label{sec: avg envy}
In this section, we examine another way to define envy: The average reward disparity between the agents. We define the \emph{average envy}, denoted $\envavg^T$, as 
\[
\envavg^T = \frac{1}{\binom{N}{2}}\sum_{1\leq i<j\leq N}{\abs{\env_{i,j}^T}}.
\]
For the special cases of $N=2$ agents, the definition of maximal envy $\env$ and average envy $\envavg$ coincide. 


Since $\envavg^t \leq \env^t$ for all $t$  almost surely, any upper bound on the maximal envy can be applied to the average envy. Particularly, Theorem~\ref{thm: uni upper-bound} provides an immediate upper bound on $\E{\envavg^T}$ of $O\left(\sqrt{\ln (N) \sum^T_{t=1} \var{\dift}} \right)$. Using a slightly more careful analysis, we can eliminate the $\sqrt{\ln{(N)}}$ factor.
\begin{proposition}\label{prop: avg upper-bound}
When executing any algorithm, it holds that
\[\E{\max_{1\leq t \leq T} \envavg^t (\uniord)} \leq 2\sqrt{\ln{(N)} \sum^{T}_{t=1}{\var{\dift}} }.\]
\end{proposition}
\begin{proof}[Proof of Proposition~\ref{prop: avg upper-bound}]
Much like the proof of Theorem~\ref{thm: uni upper-bound}, this proof is based on the fact that $\dift$ are subgaussian random variables. From the linearity of expectation, we get
\begin{align}\label{avg 1}
\E{\max_{1\leq t \leq T} \envavg^t } &=
\E{\max_{1\leq t \leq T} \frac{1}{\binom{N}{2}}\sum_{1\leq i<j\leq N}{\abs{\env_{i,j}^t}}} 
\leq \E{\frac{1}{\binom{N}{2}}\sum_{1\leq i<j\leq N}{\max_{1\leq t \leq T} 
 \abs{\env_{i,j}^t}}} \nonumber\\
&=\frac{1}{\binom{N}{2}}\sum_{1\leq i<j\leq N}\E{\max_{1\leq t \leq T}  \abs{\sum^t_{\tau=1}{\adif{i}{j}^\tau}} }.
\end{align}
For every pair of agents $i,j \in [N]$, it holds that
\begin{align}\label{avg 2}
\E{\max_{1\leq t \leq T}  \abs{\sum^t_{\tau=1}{\adif{i}{j}^\tau}} } =
\E{\max_{1\leq t \leq T, \sigma\in \{-1,1\}}\left\{\sigma\sum^t_{\tau=1}{\adif{i}{j}^\tau}\right\}} \leq
\sqrt{2\ln{(2)}\sum^T_{t=1}{\var{\dift}}},
\end{align}
where the last inequality is due to Proposition~\ref{prop:envy is good SG} and Claim~\ref{claim: sg max}. By combining Inequalities~\eqref{avg 1} and \eqref{avg 2} we can conclude that
\begin{align*}
\E{\max_{1\leq t \leq T} \envavg^t } \leq
\frac{1}{\binom{N}{2}}\sum_{1\leq i<j\leq N}{\sqrt{2\ln{(2)}\sum^T_{t=1}{\var{\dift}}}} =
\sqrt{2\ln{(2)}\sum^T_{t=1}{\var{\dift}}},
\end{align*}
which concludes the proof of Proposition~\ref{prop: avg upper-bound}. 
\end{proof}
Next, we craft a lower bound for the average envy.
\begin{proposition}\label{prop: mp avg lower-bound}
    For a large enough $T$, a sufficiently random execution with a symmetric, memory-free algorithm yields
    \[\E{\envavg^T} \geq c\sqrt{ \sum^{T}_{t=1}{\var{\dif^t}}},\]
    where $c> 0$ is a global constant.    
\end{proposition}


\begin{proof}[Proof of Proposition~\ref{prop: mp avg lower-bound}]
The proof of Proposition~\ref{prop: mp avg lower-bound} is almost identical to the proof of Theorem~\ref{thm: uni lower-bound}. In that proof, we bounded the (maximal) envy from below using the envy between a specific couple of agents.
Since, the algorithm is symmetric, the bound we showed is valid for every two agents;
thus, for every $i,j$ it holds that
\begin{align*}
\E{\env_{i,j}^T} \geq \frac{A_1}{2}\sqrt{ \sum^{T}_{t=1}{\var{\dif^t}} },
\end{align*}
where $A_1$ is the constant from Theorem~\ref{thm:BDG}. Using linearity of expectation, we get
\begin{align*}
\E{\envavg^T} & =
\E{\frac{1}{\binom{N}{2}}\sum_{i,j \in [N]^2}{\env_{i,j}^T}} =
\frac{1}{\binom{N}{2}}\sum_{i,j \in [N]^2}{\E{\env_{i,j}^T}}
\geq
\frac{1}{\binom{N}{2}}\sum_{i,j \in [N]^2} {\frac{A_1}{2}\sqrt{ \sum^{T}_{t=1}{\var{\dif^t}} }}
\\&=
\frac{A_1}{2}\sqrt{ \sum^{T}_{t=1}{\var{\dif^t}} }.
\end{align*}
This concludes the proof of Proposition~\ref{prop: mp avg lower-bound}.
\end{proof}
We finalize this section by mentioning that the upper bound on nudged arrival and maximal envy holds trivially for the average envy due to the fact that $\envavg^t \leq \env^t$ for all $t$  almost surely. Future work could seek a tighter bound for the average envy.


\section{Socially Optimal Algorithms}\label{appendix:sociallyopt}
In this section, we consider the task of devising socially optimal algorithms. First, in Subsection~\ref{subsec:sw N=2}, we address the two-agent case. Then, in Subsection~\ref{subsec:sopt for N>2}, we develop algorithms for the $N>2$ case.

To ease readability, we make the assumption that reward distributions are stationary, i.e., $\mathcal{D}^1_i, \mathcal{D}^2_i, \dots \mathcal{D}^T_i$ are identical for every arm $a_i \in A$. Consequently,  $X^1_i, X^2_i, \dots X^T_i$ are i.i.d. and we drop the superscript. We stress that our results can also be easily extended to the non-stationary case. Furthermore, we let $\mu_i = \mathbb{E}[X_i^t]$ denote the expected reward of arm $a_i$.

\subsection{Social Welfare for $N=2$}
\label{subsec:sw N=2}

\begin{algorithm}[t]
\caption{Two-agents Socially Optimal Algorithm ($\sopt$)}
\label{alg: sopt}
\LinesNumbered
\DontPrintSemicolon 
\KwIn{horizon $T$, reward distributions $\mathcal{D}_1, \ldots, \mathcal{D}_K$}
Compute $(\is, \js)$ such that
\label{alg: sopt compute}
\begin{align}\label{eq: picking i,j star}
(\is, \js) \in \argmax_{(i,j) \in A^2} \left\{ \mu_i  + \prb{X_i < \mu_j}\mu_j+\prb{X_i \geq \mu_j}\E{X_i \mid X_i \geq \mu_j}  \right\}.
\end{align}\\
\For{round $t = 1$ to $T$}{\label{alg: sopt for}
    Pull $a_{\is}$ \label{alg: sopt 3}\\
    \lIf{$x^t_\is \geq \mu_\js$}{
        Pull $a_{\is}$ \label{alg: sopt if}
    }
    \lElse{
        Pull $a_{\js}$ \label{alg: sopt else}
    }
}
\end{algorithm}
In this section, we design $\sopt$, a socially optimal algorithm for the two-agent case, which we implement in Algorithm~\ref{alg: sopt}. $\sopt$ has Bayesian information, as it receives the reward distributions as inputs. In Line~\ref{alg: sopt compute}, it selects two arms $a_{\is}, a_{\js}$ according to Equation~\eqref{eq: picking i,j star}. As we prove formally, this selection maximizes $\E{{r^t_{(1)}}+{r^t_{(2)}}}$ for any $t$. Arms $a_{\is}, a_{\js}$ are the only arms the algorithm pulls during its execution.

In Line~\ref{alg: sopt for}, $\sopt$ begins interacting with the agents for $T$ rounds.
Notice $\sopt$ does not address the arrival of the agents at all: While the arrival function is crucial for measuring envy, it does not influence the SW.

$\sopt$ pulls arm $a_{\is}$ for the agent that arrives in the first session, 
and observes the realized reward $x^t_{\is}$ (Line~\ref{alg: sopt 3}). In Lines~\ref{alg: sopt if}--\ref{alg: sopt else}, $\sopt$ decides whether to pull the same arm in the second session or $a_{\js}$ instead, based on the realized $x^t_{\is}$:
If $x^t_{\is} \geq \mu_{\js}$ (Line~\ref{alg: sopt if}), i.e., we expect the reward of $a_{\js}$ to be less than or equal to the observed reward, $\sopt$ pulls $a_{\is}$ again.
Otherwise (Line~\ref{alg: sopt else}), we expect the reward of $a_{\js}$  to be greater than that of $a_{\is}$, so the algorithm pulls arm $a_{\js}$. Next, we prove the optimality of $\sopt$.

Before we prove the optimality of $\sopt$, present two propositions that assist in understanding its crux.
\begin{proposition}\label{prop:is}
$\sopt$ does not always pull the arm with the highest expected reward in the first session. That is, $\is$ is not necessarily $\argmax_{i\in \{1, \ldots, K \}}{\mu_i}$.
\end{proposition}

\begin{proof}[Proof of Proposition~\ref{prop:is}]
We show an example satisfying $\is \neq \argmax_{i\in \{1, \ldots, K \}}{\mu_i}$.
Consider $K=2$ arms, with the following distributions.
\begin{align*}
    X_1 \sim
    \begin{cases}
    0.75 & w.p. \ \  0.5 \\
    0.55 & w.p.\  \ 0.5
    \end{cases},
    X_2 \sim
    \begin{cases}
    1 & w.p. \ \  0.6 \\
    0 & w.p.\  \ 0.4
    \end{cases}.
\end{align*}
Observe that $\mu_1 = 0.65 > \mu_2=0.6$. Yet,
\begin{align*}
\mu_1  +\mu_2 \prb{X_1 < \mu_2}+\E{X_1 \mid X_1 \geq \mu_2}  \prb{X_1 \geq \mu_2}=
0.65  + 0.6 \cdot 0.5+ 0.75 \cdot 0.5 = 1.325,
\end{align*}
whereas,
\begin{align*}
\mu_2  +\mu_1 \prb{X_2 < \mu_1}+\E{X_2 \mid X_2 \geq \mu_1}  \prb{X_2 \geq \mu_1}=
0.6  + 0.65 \cdot 0.4 + 1 \cdot 0.6 = 1.46.
\end{align*}
Consequently, $\sopt$ chooses $(\is,\js) = (2,1)$.
This concludes the proof of Proposition~\ref{prop:is}.
\end{proof}
To get the intuition behind Proposition~\ref{prop:is}, recall that with the help of the information exposed by the first agent, the algorithm can make a better decision in the second session.
Therefore, we must find the perfect trade-off between the first agent's welfare (exploitation) and the leverage of the information they provide (exploration).
In contrast, we expect nothing but pure exploitation in the second session.
\begin{proposition}\label{prop:js}
Arm $a_\js$ has the highest expected reward among the remaining arms. I.e., $\js = \argmax_{j\in K\setminus\{\is\}}{\mu_j}$.
\end{proposition}
In other words, in the second session, the algorithm makes the choice that is the most rewarding.
\begin{proof}[Proof of Proposition~\ref{prop:js}]
Let $(\is, \js)$ be a pair of arms that maximizes Equation~\eqref{eq: picking i,j star}.
Suppose, for the sake of contradiction, that there exists ${j'}\notin \{\is,\js\}$ such that $\mu_{j'} > \mu_{\js}$;
thus, $\E{\max{\{X_{\is}, \mu_{{j'}}\}}} \geq \E{\max{\{X_{\is}, \mu_{\js}\}}}$.
I.e.,
\begin{align*}
    \E{X_{\is}} + \E{\max{\{X_{\is}, \mu_{{j'}}\}}} \geq \E{X_{\is}} + \E{\max{\{X_{\is}, \mu_{\js}\}}},
\end{align*}
where equality can occur if and only if $\prb{\max{\{X_{\is}, \mu_{\js}\}} = \mu_{\js}} = 0$.
In this case, the algorithm always pulls arm $a_{\is}$ in the second session, and arm $a_{\js}$ is irrelevant.
Hence, we assume strong inequality.
Notice that the left-hand side is exactly Equation~\eqref{eq: picking i,j star} with $(i,j) = (\is, {j'})$, and the right-hand side is exactly Equation~\eqref{eq: picking i,j star} with $(i,j) = (\is, \js)$.
Hence, we have obtained a contradiction to $(\is, \js)$ being a pair that maximizes Equation~\eqref{eq: picking i,j star}. This concludes the proof of Proposition~\ref{prop:js}.
\end{proof}
We are ready to prove the optimality of $\sopt$.
\begin{theorem}\label{thm:sopt is opt}
Fix any instance with $N=2$ and arbitrary reward distributions. For any algorithm $ALG$ with Bayesian information, it holds that
\[\sw\left(ALG\right) \leq \sw\left(\sopt\right).\]    
\end{theorem}
\begin{proof}[Proof of Theorem~\ref{thm:sopt is opt}]
Fix any instance with $N=2$ and arbitrary reward distributions. The social welfare of an algorithm $ALG$ over $T$ rounds is
\[
\sw(ALG) 
=\E{ \sum^2_{i=1}{R_i^T}}= \E{\sum_{t=1}^T \bigl(r^t_{(1)} + r^t_{(2)}\bigr)}
= T \, \E{r^1_{(1)} + r^1_{(2)}},
\]
where the last equality uses the fact that the rewards in each round are independent.

Thus, to show that $\sopt$ maximizes social welfare, it suffices to show that no algorithm can exceed its expected reward \emph{within a single round}. By the revelation principle~\cite{peters2001common}, there is an optimal \emph{threshold} algorithm: After observing the first-session reward, it decides in the second session by comparing the observed reward against the expected reward of any other arm.

Since $\sopt$ enumerates all pairs of arms $(i,j)$ in Equation~\eqref{eq: picking i,j star} and applies a greedy rule for the second session (choosing either the same arm $i$ or the other arm $j$ based on which is expected to yield a higher reward), it achieves the maximum expected reward per round. Multiplying by $T$ completes the proof.
\end{proof}

\subsection{Socially Optimal Algorithm for $N >2$}\label{subsec:sopt for N>2}
Next, we consider the problem of finding a socially optimal algorithm for $N >2$ agents. First, we present a socially optimal algorithm for the special case of Bernoulli rewards.

\begin{proposition}\label{prop:bernoulli sw optimal}
    Assume that $X_i \sim \ber{p_i}$ for every $i \in [K]$. An algorithm that pulls arms in descending order of $p_i$ until it realizes a reward of 1 is socially optimal for any number of agents $N$.
\end{proposition}
\begin{proof}[Proof of Proposition~\ref{prop:bernoulli sw optimal}]
    Since random algorithms are just a distribution over deterministic algorithms, we know there is an optimal algorithm that is deterministic. Additionally, as in the proof of Theorem~\ref{thm:sopt is opt}, it suffices to focus on a single and arbitrary round $t$.
    
    Notice that after we observe an arm such that $x^t_i =1$, it is optimal to pull it for all remaining agents.
    Similarly,  if we observe arm $a_i$ with $x^t_i =0$ it is strictly sub-optimal to pull it again, unless all arms yielded a reward of 0.
    Thus, the only thing left to prove is the optimality of the order in which the algorithm pulls arms.

    In this special case, we can use a reduction to a Pandora's Box (PB) problem~\cite{weitzman1978optimal}. We first describe the reduction, then characterize the optimal solution for the PB instance, and finally show the equivalence. 
       
    Let the PB instance include $K$ Bernoulli arms with success probability $p_i$ for every arm $i$ and costs $c_i = \frac{1}{B}$, where $B$ is a constant such that $B > \max_{i \in [K]} \frac{1}{p_i}$. Due to Weitzman's seminal result~\cite{weitzman1978optimal}, there exist indices $(\theta_i)_{i=1}^K$ such that the optimal sequence is descending in the index. Each index  $\theta_i$ is the solution to $\E{\max\left\{X_i - \theta_i,0 \right\}}  =c_i =\frac{1}{B}$. Thus, $\theta_i = 1- \frac{1}{B\cdot p_i}$. The optimal solution maximizes $\E{1-\frac{S}{B}}$, where $S$ is a r.v. that counts the number of useless arms pulled (arms with a realized reward of 0). Note that $S$ depends solely on the pulling order.

    Similarly, for our original problem, maximizing the social welfare amounts to maximizing $\E{N-S}$. Since $S$ is distributed identically in both problems, the optimal order for PB is optimal for the original problem as well.
\end{proof}
    % satisfy this condition;
    % hence, the index of each arm is $1- \frac{1}{p_i}$. 
    % Any optimal algorithm for the PB instance maximizes the reward minus the sum of costs; thus, it orders the arms to find a positive reward as quickly as possible. Using the same order ensures that we maximize the social welfare in our problem. 
    % Recall that in the PB instance,  we need to find for each arm $a_i$ an index $\theta_i$ that satisfies $\E{\max\left\{X_i - \theta_i,0 \right\}}  =c_i =1$.
    % It is easy to see that $\theta_i = 1- \frac{1}{p_i}$ satisfy this condition;
    % hence, the index of each arm is $1- \frac{1}{p_i}$. 
    % the optimal sequence for the PB instance is descending in the index.

% \omer{USE THE BELOW AS THE BASIS FOR THE DYNAMIC PROGRAMMING}
% Fix an arbitrary round $t$, and assume that all the rewards are supported in a finite set $V$. We now describe a dynamic programming procedure that finds the optimal algorithm in $O(\abs{V} N2^K)$. Let $B$ be a subset of arms $B \subseteq A$, $n$ denote a number of agents $n\in \{0,1,\dots,N$, and $v$ denote an arbitrary reward $v \in V$. We define the following function $f$:
% \begin{equation}\label{eq:dp f}
% f(n,B,v) =  \max \left\{v\cdot n, \max_{a\in B} \mathbb E\left[X_a + f\left(n-1,B\setminus \{a\}, \max\{v,X_a\}\right) \right] \right\}    
% \end{equation}

Next, we move beyond Bernoulli rewards. Fix an arbitrary round $t$, and assume that all rewards are supported in a finite set $V$ (we later explain how to relax this assumption). We now describe a dynamic programming procedure that finds the optimal algorithm with a computational complexity of $O(\abs{V} \cdot N \cdot 2^K)$. 

We consider the following parameters: $B \subseteq A$, representing a subset of available arms; $n \in \{0,1,\dots,N\}$, denoting the number of remaining agents; and $v \in V$, an arbitrary current reward that models the maximal reward of all observed arms in $A \setminus B$.


We define the function $f(n, B, v)$, representing the maximum expected social welfare achievable given the parameters $(n,B,v)$. To apply this dynamic programming approach, we first establish the base cases for the function $f$:
\begin{itemize}
    \item \textbf{No agents remaining ($n = 0$):} If there are no agents left to assign rewards, the maximum achievable reward is zero regardless of the subset of arms $B$ and the current reward $v$. Formally, 
    \[
    f(0, B, v) = 0 \quad \forall \, B \subseteq A, \, v \in V.
    \]
    
    \item \textbf{No available arms ($B = \emptyset$):} If there are no arms left to pull, the only option is to assign the current reward $v$ to all remaining agents. Thus, the maximum reward in this scenario is the product of $v$ and the number of remaining agents $n$. Formally, 
    \[
    f(n, \emptyset, v) = v \cdot n \quad \forall \, n \in \{0,1,\dots,N\}, \, v \in V.
    \]
\end{itemize}

We move on to the recursive definition of $f$:
\begin{equation}\label{eq:dp_f}
f(n, B, v) = \max \left\{ v \cdot n, \max_{a \in B} \mathbb{E}\left[ X_a^t + f\left(n - 1, B \setminus \{a\}, \max\{v, X_a^t\} \right) \right] \right\}
\end{equation}
The recursive relation in Equation~\eqref{eq:dp_f} considers two scenarios at each step:
\begin{enumerate}
    \item \textbf{Assigning the Current Reward ($v \cdot n$):} In this scenario, the algorithm assigns the current reward $v$ to all remaining $n$ agents without pulling any additional arms.
    
    \item \textbf{Pulling an Arm ($\max_{a \in B} \mathbb{E}\left[ X_a^t + f\left(n - 1, B \setminus \{a\}, \max\{v, X_a^t\} \right) \right]$):} Here, the algorithm selects an arm $a \in B$, observes the stochastic reward $X_a^t$, and then recursively assigns rewards to the remaining $n - 1$ agents. The subset of available arms is updated to $B \setminus \{a\}$, and the current reward is updated to $\max\{v, X_a^t\}$.
\end{enumerate}

The optimal value for round $t$ is obtained by evaluating the function $f$ at the initial conditions where all agents and arms are available, and the starting reward is zero:
\[
f(n = N, B = A, v = 0).
\]
This value represents the maximum expected social welfare achievable by the optimal algorithm for round $t$. 

\begin{theorem}\label{thm:optimality_runtime}
    The dynamic programming procedure defined by the function $f(n, B, v)$ in Equation~\eqref{eq:dp_f} correctly computes the maximum expected total social welfare achievable for round $t$. Furthermore, the procedure operates with a runtime complexity of $O(\abs{V} \cdot N\cdot K\cdot 2^K)$.
\end{theorem}

\begin{proof}
    The optimality of the procedure follows from standard dynamic programming arguments, relying on the fact that the overall problem can be constructed from optimal solutions to its subproblems. The function $f(n, B, v)$ is designed to represent the maximum expected total social welfare achievable given the state $(n, B, v)$. By taking the maximum of the two options (recall Equation~\eqref{eq:dp_f}, $f(n, B, v)$ ensures that the optimal decision is made at each state, thereby maximizing the expected total social welfare.

To determine the runtime complexity, we analyze the number of possible states and the computation required for each state. The function $f(n, B, v)$ is parameterized by the number of remaining agents $n$, which can take $N+1$ possible values; the subset of available arms $B$, which has $2^K$ possible subsets; and the current reward $v$, which can take $\abs{V}$ possible values. Therefore, the total number of states is $O(\abs{V} \cdot N \cdot 2^K)$. For each state $(n, B, v)$, the dynamic programming procedure performs a constant-time computation to calculate $v \cdot n$ and iterates over all arms in $B$, performing constant-time computations for each arm. Given that there are up to $K$ arms in $B$, the computation per state is $O(K)$. Consequently, the overall runtime complexity is $O(\abs{V} \cdot N \cdot K \cdot 2^K )$. 
\end{proof}


Finally, we can relax the finite-support assumption. In scenarios where rewards are arbitrary within the continuous interval $[0,1]$, we discretize the reward space by selecting a finite set $V$ that approximates the continuous range of rewards. By choosing an appropriate granularity for the discretization, we can control the trade-off between the accuracy of the approximation and the computational complexity of the algorithm. While a finer discretization yields a closer approximation to the true optimal solution, it simultaneously increases the size of the state space, thereby enhancing the computational burden. Conversely, a coarser discretization reduces computational requirements at the expense of approximation precision. As this approach is standard, we omit the details. 


Unfortunately, the procedure above is inefficient in the number of arms $K$. We leave the tasks of finding an optimal algorithm, proving hardness, and finding approximately optimal algorithms as open problems.
%, i.e., an $ALG'\in \mathcal A$ with runtime $poly\left(\frac{1}{\epsilon} \right)$ such that $\E{SW(ALG')} \geq \max_{ALG\in \mathcal A} \E{SW(ALG)} - \epsilon$, 
\section{Simulations}
\label{sec:simulations}
\newcommand{\uins}{I_U}
\newcommand{\bins}{I_B}
\newcommand{\sins}{I_S}
In this section, we report the results of extensive simulations to empirically validate our theoretical results and test our conjectures. Specifically, we devote Subsection~\ref{subsec:dep t n} to verify the behavior of uniform, nudged, and adversarial arrival as functions of the horizon $T$ and number of agents $N$. In Subsection~\ref{subsec:nudge sensitive}, we provide a sensitivity analysis of nudged arrival. Subsection~\ref{subsec:sim efc} validates our results from Section~\ref{sec:extensions}, as well as test Conjecture~\ref{thm: efc sw}.

\paragraph{Simulation setup}
For analyses where the time horizon $T$ and the number of agents $N$ are not explicitly specified, we set $T = 10{,}000$ rounds and $N=2$ agents by default. We report the average results over $1{,}000$ independent runs, with the shaded areas indicating three standard deviations from the mean. All simulations were conducted on a standard Lenovo laptop, with the total execution time amounting to a couple of hours.

We used two instances in most of the experiments, 
\begin{itemize}
    \item Uniform instance ($\uins$): $K=4$ arms, all with $\uni{0,1}$ reward distributions. The algorithm explores arms until it finds one with a value greater than $\frac{3}{4}$. 
    \item Bernoulli instance ($\bins$): $K=3$ Bernoulli arms with success parameters of $p_1 = 0.6, p_2=0.4, p_3=0.2$. The algorithm explores arms in a descending order of $p_i$, until a reward of 1 is materialized.
\end{itemize}

\subsection{Dependence on $T$ and $N$}\label{subsec:dep t n}
\begin{figure}
    \centering
    \includegraphics[width=\linewidth]{figures/Figure_1.pdf}
    \caption{
    $\env^t$ as a function of $t$ for both the uniform instance ($\uins$, left panel) and the Bernoulli instance ($\bins$, right panel), each with $N=2$. 
    The three arrival functions shown are $\uniord$, $\sugord$ (with $\delta=\frac 1 2$), and $\advord$. Green `X' markers represent the maximum likelihood estimates (MLE) for the linear model $y = c \cdot x$, while orange circles indicate the MLE for the square-root model $y = c \cdot \sqrt{x}$. The perfect alignment of the simulated data with these curves confirms our theoretical predictions.}
    \label{fig: envy-func-t}
\end{figure}
In this subsection, we validate our results regarding the dependency on $T$ and $N$. Figure~\ref{fig: envy-func-t} shows the cumulative envy as a function of time for all three arrival functions: $\advord$ (green dashed), $\uniord$ (orange dotted), and $\sugord$ (blue smooth). Each plot shows the cumulative envy over time on a logarithmic vertical axis, reflecting the distinct asymptotic behaviors of the arrival functions. The left panel presents the uniform instance $\uins$, and the right panel the Bernoulli instance $\bins$.  Due to the logarithmic scale, the shaded regions indicating three standard deviations are barely distinguishable; we provide further details in Table~\ref{table} in Subsection~\ref{appendix: simulations}.

For both instances, we see that the cumulative envy of $\advord$ and $\uniord$ increases over time, whereas the cumulative envy of $\sugord$ remains nearly constant (subject to some noise). As time progresses, we observe substantially different growth rates in envy across the three arrival functions, consistent with our theoretical analysis. 

The green `X' markers represent the maximum likelihood estimates (MLE) for the linear function $y = c \cdot x$, which closely match the green (dashed) curve for the envy under $\advord$, thereby confirming the linear growth predicted by Proposition~\ref{thm: adv-envy}. The orange (dotted) curve corresponds to $\uniord$, and the orange circles depict the MLE for the square-root function $y = c \cdot \sqrt{x}$. Their close alignment supports Corollary~\ref{cor: uni-envy}. Finally, Theorem~\ref{thm: sugg-envy} asserts that envy under $\sugord$ remains bounded when both the instance parameters and the number of agents are fixed. This is precisely what we observe in the blue (smooth) curves of both panels in Figure~\ref{fig: envy-func-t}.
\begin{figure}
    \centering
    \includegraphics[width=\linewidth]{figures/Figure_2.pdf}
    \caption{$\env^T$ as a function of $N$ for both $\uins$ and $\bins$, under $\uniord$ (left panel) and $\sugord$ with $\delta=\nicefrac{1}{2}$ (right panel).}
    \label{fig: envy-T-func-N}
\end{figure}

We proceed to examine how envy depends on the number of agents, focusing on $\uniord$ and $\sugord$ with $\delta = \frac{1}{2}$. In Figure~\ref{fig: envy-T-func-N}, we plot the cumulative envy after $T = 10^4$ rounds as a function of the number of agents $N$, with $N$ ranging from $2$ to $20$. The left panel depicts the envy under $\uniord$ for both instances, $\bins$ (blue smooth) and $\uins$ (orange dotted). In each instance, envy initially rises for small $N$ and then declines, matching the intuition from Corollary~\ref{thm: sqrt TK N}: as $N$ grows, there are increasingly more agents exploiting rather than exploring (given that these instances have $K\in\{3,4\}$ arms). The right panel in Figure~\ref{fig: envy-T-func-N} considers nudged arrival $\sugord$. Here, envy increases with the number of agents, as Theorem~\ref{thm: sugg-envy} hints. However, this increase is not linear in $N$, but rather milder. We conjecture that the dependence of $\sugord$ is essentially sub-linear in $N$, leaving a precise characterization for future work.
\subsection{Sensitivity Analysis for Nudged Arrival}\label{subsec:nudge sensitive}
\begin{figure}
    \centering
    \begin{subfigure}{0.49\textwidth}
        \includegraphics[width=\linewidth]{figures/Figure_3a.pdf}
        \caption{Envy as a function of $\delta$ for $\uins$ and $\bins$.
        %$\env^T$ as a function of $\delta$ for both Uniform and Bernoulli instances, for $\sugord$.
        }\label{fig: envy_at_T_as_func_delta}
    \end{subfigure}
    \hfill
    \begin{subfigure}{0.49\textwidth}
        \includegraphics[width=\linewidth]{figures/Figure_3b.pdf}
        \caption{Envy as a function of $\tdif$ for $\sins$.
        %$\env^T$ as a function of $T$ for an instances with $\Tilde{\dif}$ that decreases as a function of $T$ under $\ordname_{\delta=\nicefrac{1}{4}}$. 
        }\label{fig: envy_at_T_delta_func_of_T}
    \end{subfigure}
    \caption{Sensitivity Analysis for $\sugord$.}
\end{figure}
In this subsection, we provide a sensitivity analysis for $\sugord$. Figure~\ref{fig: envy_at_T_as_func_delta} shows the cumulative envy as a function of the parameter $\delta$, which reflects how strongly the system can influence the agents' arrival. 
For both $\bins$ (blue smooth) and $\uins$ (orange dotted), we observe that envy decreases as $\delta$ increases. 
Moreover, this reduction appears consistent with $\nicefrac{1}{\delta}$, aligning with the prediction from Theorem~\ref{thm: sugg-envy}.

To examine the dependence of envy on $\tdif$, we introduce reward distributions that explicitly involve $T$, unlike $\uins$ and $\bins$. 
Specifically, we define a new instance, $\sins$, with $K = 2$. 
In this instance,
\begin{align*}
    X_1 \sim
    \begin{cases}
    1 & w.p. \ \  \frac{1}{2} \\
    \frac{1}{4} & w.p.\  \ \frac{1}{2}
    \end{cases},\quad 
    X_2 \sim
    \begin{cases}
    1 & w.p. \ \  \frac{1}{4} + \frac{2}{\sqrt{T}} \\
    0 & w.p.\  \ \frac{3}{4} - \frac{2}{\sqrt{T}}
    \end{cases}.
\end{align*}
Furthermore, we consider the following algorithm: In the first session of every round, it pulls $a_1$. If the observed reward is $1$, it pulls $a_1$ again in the second session; otherwise, it pulls $a_2$. Note that 
\[
\E{\rt{(1)}}=\frac{1+0.25}{2}=\frac{5}{8},
\]
while
\[
\E{\rt{(2)}}= \prb{X_1 = 1}\cdot 1 + \prb{X_1 = 0.25}\cdot \E{X_2} = \frac{1}{2} + \frac{1}{2} \cdot \left( \frac{1}{4} + \frac{2}{\sqrt{T}} \right) = \frac{5}{8} +\frac{1}{\sqrt{T}};
\]
thus, $\tilde{\dif}= \E{\rt{(2)}-\rt{(1)}}= \frac{1}{\sqrt{T}}$.

We now examine how envy evolves in the instance $\sins$. 
Figure~\ref{fig: envy_at_T_delta_func_of_T} displays the cumulative envy $\env^T$ at the final round $T$ for various values of $T$. 
As $T$ grows, $\tilde{\dif}$ decreases, causing envy to increase accordingly. 
Moreover, since $\tfrac{1}{\tilde{\dif}} = \sqrt{T}$, Theorem~\ref{thm: sugg-envy} implies that envy scales proportionally to $\sqrt{T}$, which is consistent with the empirical results.

\subsection{Analysis of EFC}\label{subsec:sim efc}
In this subsection, we examine the theoretical results from Section~\ref{sec:extensions}. Naturally, as our results in that section apply to the special case of Example~\ref{example 1}, all simulations were performed with $N=2$ agents, $K=2$ arms $X_1, X_2 \sim \uni{0,1}$ under uniform arrival and the $\efc$ algorithm. %We remind the reader that the social welfare in any round $t$ is given by $R_1^t+R_2^t$.

%As can be seen, all empirical results align with our theoretical results including Theorem~\ref{??}.
\begin{figure}
    \centering
        \includegraphics[scale=0.6]{figures/Figure_4.pdf}
        \caption{$\Rt{1}+\Rt{2}$ under the $\efc$ algorithm with $C=1$, as a function of $t$.% compared to $\left(1 + \frac{1}{16}\right)t$.
        }\label{fig: sw ef1}
\end{figure}

\begin{figure}
    \centering
        \includegraphics[width=\linewidth]{figures/Figure_5.pdf}
        \caption{$\frac{\Rt{1}+\Rt{2}}{t}$ under the $\efc$ algorithm, as a function of $t$ compared to $1 + \frac{1}{8}\cdot \frac{2C-1}{2C}$. and $1+\frac{1}{8}$}\label{fig: sw efc}
\end{figure}

Figure~\ref{fig: sw ef1} illustrates the social welfare for $\efc$ with $C=1$, plotting 
$\sw_t := R_1^t + R_2^t$ as the orange (smooth) curve. 
The green `X' markers correspond to the line $y=(1+\frac{1}{16})x$, 
the social welfare guaranteed by Theorem~\ref{thm:ef1evny+sw}; 
their perfect alignment confirms the theoretical prediction.


Figure~\ref{fig: sw efc} evaluates Conjecture~\ref{thm: efc sw}, asserting that for any $C \ge 1$, 
we have $\sw_t \geq (1+\frac{1}{8}-\frac{1}{16C})t$. 
To facilitate comparison, the vertical axis in Figure~\ref{fig: sw efc} depicts the average welfare, $\tfrac{\sw_t}{t}$, 
versus the round number $t$ on the horizontal axis. 
We ran $\efc$ for $C \in \{1,2,3,4,5,10,20,40\}$, adding circle markers in corresponding colors to highlight 
the values of $1 + \tfrac{2C - 1}{2C} \cdot \tfrac{1}{8}$. 
Additionally, the star marker shows the horizontal line $1 + \tfrac{1}{8}$, 
representing the maximum achievable welfare in this setting, as noted in Observation~\ref{obs:opt for tradeoff}. 
For each $C$, the corresponding curve lies above the dotted line, 
consistent with our conjectured lower bound. 
When $1 \leq C \leq 10$, the average welfare nearly coincides with the conjectured bound, 
whereas for $C = 20$ and $C = 40$, the welfare is strictly higher than the markers. 
The reason for the latter is that such high envy states occur rarely when $T = 10^4$, 
so the constraint in Line~\ref{efclin:pull_a1_cond} in $\efc$ is seldom activated in practice.

\subsection{Standard Deviations for Figure~\ref{fig: envy-func-t}}\label{appendix: simulations}
\begin{table}[!ht]
    \centering
    \caption{Three standard deviations for Figure~\ref{fig: envy-func-t}.}\label{table}
    \begin{tabular}{|l|l|l|l|l|l|l|}
    \hline
        $t$ & $\uins, \advord$ & $\uins, \uniord$ & $\uins, \sugord$ & $\bins, \advord$ & $\bins, \uniord$ &  $\bins, \sugord$ \\ \hline
        1000 & 0.79 & 0.54 & 0.13 & 1.12 & 0.72 & 0.18 \\ \hline
        2000 & 1.11 & 0.74 & 0.12 & 1.59 & 1.02 & 0.18 \\ \hline
        3000 & 1.34 & 0.87 & 0.12 & 1.92 & 1.27 & 0.19 \\ \hline
        4000 & 1.52 & 0.99 & 0.12 & 2.22 & 1.38 & 0.19 \\ \hline
        5000 & 1.72 & 1.09 & 0.13 & 2.50 & 1.59 & 0.18 \\ \hline
        6000 & 1.91 & 1.23 & 0.12 & 2.72 & 1.73 & 0.19 \\ \hline
        7000 & 2.11 & 1.34 & 0.12 & 2.91 & 1.89 & 0.19 \\ \hline
        8000 & 2.30 & 1.42 & 0.12 & 3.16 & 2.08 & 0.18 \\ \hline
        9000 & 2.43 & 1.52 & 0.13 & 3.37 & 2.22 & 0.21 \\ \hline
        10000 & 2.58 & 1.59 & 0.13 & 3.55 & 2.35 & 0.18 \\ \hline
    \end{tabular}
\end{table}

}%
\fi

\end{document}
