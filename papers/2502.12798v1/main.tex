\documentclass[11pt,final]{article}

%Added after EC
\usepackage{xcolor}


%Make sure the order of begin document, title, author, abstract, make title is correct
%%%%%%%%%%%%%%%%


\usepackage{fullpage}
\usepackage{amsmath,amssymb,amsthm}
\usepackage{lmodern}



\usepackage[numbers,sort&compress]{natbib} 
\setcitestyle{numbers}

\usepackage[colorlinks=true]{hyperref}
\hypersetup{
     colorlinks  = true,
     urlcolor    = teal,
	 citecolor   = teal,
	 linkcolor   = red
}




\usepackage[ruled,noend,linesnumbered]{algorithm2e} % For algorithms
\renewcommand{\algorithmcfname}{ALGORITHM}
\SetAlFnt{\small}
\SetAlCapFnt{\small}
\SetAlCapNameFnt{\small}
\SetAlCapHSkip{0pt}
\IncMargin{-\parindent}



\usepackage[utf8]{inputenc} % allow utf-8 input
\usepackage[T1]{fontenc}    % use 8-bit T1 fonts
\usepackage{lmodern}
\usepackage{url}            % simple URL typesetting
\usepackage{amsfonts}       % blackboard math symbols
\usepackage{nicefrac}       % compact symbols for 1/2, etc.
\usepackage{microtype}      % microtypography
\usepackage{booktabs}       % professional-quality tables
%\usepackage[linesnumbered,ruled,noend]{algorithm2e}

%Make sure this appears AFTER ams packages, etc.
\newtheorem{theorem}{Theorem}
\newtheorem{lemma}{Lemma}
\newtheorem{proposition}{Proposition}
\newtheorem{corollary}{Corollary}
\newtheorem{conjecture}{Conjecture}
\newtheorem{definition}{Definition}
%\newtheorem{example}{Example}
\newtheorem{remark}{Remark}
\newtheorem{property}{Property}
\newtheorem{claim}{Claim}
\newtheorem{observation}{Observation}
\newtheorem{assumption}{Assumption}

\theoremstyle{definition} % Ensures normal (non-italic) text
\newtheorem{example}{Example}


\newcount\Includeappendix %include appendices or not
\Includeappendix=1
\newcommand{\appnx}[1]{{\ifnum\Includeappendix=1{#1}\else{the appendix}\fi}}


% Choose a citation style by commenting/uncommenting the appropriate line:
%\setcitestyle{acmnumeric}
% Lists
\usepackage[inline]{enumitem}
\newlist{enuminline}{enumerate*}{1}
\setlist[enuminline]{label=(\roman*)}

% Math
\newcommand{\vol}{\mathrm{vol}}
\newcommand{\E}{\mathbb E}
\newcommand{\var}{\mathrm{var}}
\newcommand{\cov}{\mathrm{cov}}
\newcommand{\Normal}{\mathcal N}
\newcommand{\slowvar}{\mathcal L}
\newcommand{\bbR}{\mathbb R}
\newcommand{\bbE}{\mathbb E}
\newcommand{\bbP}{\mathbb P}
\newcommand{\calC}{\mathcal C}
\newcommand{\calR}{\mathcal R}
\newcommand{\calT}{\mathcal T}
\newcommand{\loA}{\underline{A}}
\newcommand{\upA}{\overline{A}} 
\newcommand{\cv}{\mathrm{cv}} 
\newcommand{\pp}{\mathrm{pp}}
\newcommand{\HS}{\mathrm{HS}}
\newcommand{\erfi}{\mathrm{erfi}}
\newcommand{\tX}{\widetilde X}
\newcommand{\OneFOne}{{}_1F_1}
\newcommand{\PP}{\mathrm{\texttt{PP}}}
\newcommand{\PPpp}{\mathrm{\texttt{PP+}}}
\newcommand{\BPP}{\mathrm{\texttt{BPP}}}
\newcommand{\BPPpp}{\mathrm{\texttt{BPP+}}}
\newcommand{\FABPPI}{\mathrm{\texttt{FABPP}}}
\newcommand{\asto}{\overset{\text{a.s.}}{\to}}
% Title. Note the optional short title for running heads. In the interest of anonymization, please do not include any acknowledgements.
\begin{document}

\title{Envious Explore and Exploit}

% Anonymized submission.
\author{
Omer Ben{-}Porat%
\thanks{%
    {Technion---Israel Institute of Technology (\url{omerbp@technion.ac.il})}, corresponding author}
\and Yotam Ganfi%
\thanks{%
    {Weizmann Institute (\url{yotam.gafni@gmail.com})}}
\and Or Markovetzki%
\thanks{%
    {Technion---Israel Institute of Technology (\url{ormar@campus.technion.ac.il})}}
}
\maketitle
% Abstract. Note that this must come before \maketitle.
\begin{abstract}
Explore-and-exploit tradeoffs play a key role in recommendation systems (RSs), aiming at serving users better by learning from previous interactions. Despite their commercial success, the societal effects of explore-and-exploit mechanisms are not well understood, especially regarding the utility discrepancy they generate between different users.
In this work, we measure such discrepancy using the economic notion of envy. We present a multi-armed bandit-like model in which every round consists of several sessions, and rewards are realized once per round. We call the latter property \emph{reward consistency}, and show that the RS can leverage this property for better societal outcomes. On the downside, doing so also generates envy, as late-to-arrive users enjoy the information gathered by early-to-arrive users. We examine the generated envy under several arrival order mechanisms and virtually any \emph{anonymous} algorithm, i.e., any algorithm that treats all similar users similarly without leveraging their identities. We provide tight envy bounds on uniform arrival and upper bound the envy for nudged arrival, in which the RS can affect the order of arrival by nudging its users. Furthermore, we study the efficiency-fairness trade-off by devising an algorithm that allows constant envy and approximates the optimal welfare in restricted settings. Finally, we validate our theoretical results empirically using simulations.
\end{abstract}


% Paper body
\section{Introduction}


\begin{figure}[t]
\centering
\includegraphics[width=0.6\columnwidth]{figures/evaluation_desiderata_V5.pdf}
\vspace{-0.5cm}
\caption{\systemName is a platform for conducting realistic evaluations of code LLMs, collecting human preferences of coding models with real users, real tasks, and in realistic environments, aimed at addressing the limitations of existing evaluations.
}
\label{fig:motivation}
\end{figure}

\begin{figure*}[t]
\centering
\includegraphics[width=\textwidth]{figures/system_design_v2.png}
\caption{We introduce \systemName, a VSCode extension to collect human preferences of code directly in a developer's IDE. \systemName enables developers to use code completions from various models. The system comprises a) the interface in the user's IDE which presents paired completions to users (left), b) a sampling strategy that picks model pairs to reduce latency (right, top), and c) a prompting scheme that allows diverse LLMs to perform code completions with high fidelity.
Users can select between the top completion (green box) using \texttt{tab} or the bottom completion (blue box) using \texttt{shift+tab}.}
\label{fig:overview}
\end{figure*}

As model capabilities improve, large language models (LLMs) are increasingly integrated into user environments and workflows.
For example, software developers code with AI in integrated developer environments (IDEs)~\citep{peng2023impact}, doctors rely on notes generated through ambient listening~\citep{oberst2024science}, and lawyers consider case evidence identified by electronic discovery systems~\citep{yang2024beyond}.
Increasing deployment of models in productivity tools demands evaluation that more closely reflects real-world circumstances~\citep{hutchinson2022evaluation, saxon2024benchmarks, kapoor2024ai}.
While newer benchmarks and live platforms incorporate human feedback to capture real-world usage, they almost exclusively focus on evaluating LLMs in chat conversations~\citep{zheng2023judging,dubois2023alpacafarm,chiang2024chatbot, kirk2024the}.
Model evaluation must move beyond chat-based interactions and into specialized user environments.



 

In this work, we focus on evaluating LLM-based coding assistants. 
Despite the popularity of these tools---millions of developers use Github Copilot~\citep{Copilot}---existing
evaluations of the coding capabilities of new models exhibit multiple limitations (Figure~\ref{fig:motivation}, bottom).
Traditional ML benchmarks evaluate LLM capabilities by measuring how well a model can complete static, interview-style coding tasks~\citep{chen2021evaluating,austin2021program,jain2024livecodebench, white2024livebench} and lack \emph{real users}. 
User studies recruit real users to evaluate the effectiveness of LLMs as coding assistants, but are often limited to simple programming tasks as opposed to \emph{real tasks}~\citep{vaithilingam2022expectation,ross2023programmer, mozannar2024realhumaneval}.
Recent efforts to collect human feedback such as Chatbot Arena~\citep{chiang2024chatbot} are still removed from a \emph{realistic environment}, resulting in users and data that deviate from typical software development processes.
We introduce \systemName to address these limitations (Figure~\ref{fig:motivation}, top), and we describe our three main contributions below.


\textbf{We deploy \systemName in-the-wild to collect human preferences on code.} 
\systemName is a Visual Studio Code extension, collecting preferences directly in a developer's IDE within their actual workflow (Figure~\ref{fig:overview}).
\systemName provides developers with code completions, akin to the type of support provided by Github Copilot~\citep{Copilot}. 
Over the past 3 months, \systemName has served over~\completions suggestions from 10 state-of-the-art LLMs, 
gathering \sampleCount~votes from \userCount~users.
To collect user preferences,
\systemName presents a novel interface that shows users paired code completions from two different LLMs, which are determined based on a sampling strategy that aims to 
mitigate latency while preserving coverage across model comparisons.
Additionally, we devise a prompting scheme that allows a diverse set of models to perform code completions with high fidelity.
See Section~\ref{sec:system} and Section~\ref{sec:deployment} for details about system design and deployment respectively.



\textbf{We construct a leaderboard of user preferences and find notable differences from existing static benchmarks and human preference leaderboards.}
In general, we observe that smaller models seem to overperform in static benchmarks compared to our leaderboard, while performance among larger models is mixed (Section~\ref{sec:leaderboard_calculation}).
We attribute these differences to the fact that \systemName is exposed to users and tasks that differ drastically from code evaluations in the past. 
Our data spans 103 programming languages and 24 natural languages as well as a variety of real-world applications and code structures, while static benchmarks tend to focus on a specific programming and natural language and task (e.g. coding competition problems).
Additionally, while all of \systemName interactions contain code contexts and the majority involve infilling tasks, a much smaller fraction of Chatbot Arena's coding tasks contain code context, with infilling tasks appearing even more rarely. 
We analyze our data in depth in Section~\ref{subsec:comparison}.



\textbf{We derive new insights into user preferences of code by analyzing \systemName's diverse and distinct data distribution.}
We compare user preferences across different stratifications of input data (e.g., common versus rare languages) and observe which affect observed preferences most (Section~\ref{sec:analysis}).
For example, while user preferences stay relatively consistent across various programming languages, they differ drastically between different task categories (e.g. frontend/backend versus algorithm design).
We also observe variations in user preference due to different features related to code structure 
(e.g., context length and completion patterns).
We open-source \systemName and release a curated subset of code contexts.
Altogether, our results highlight the necessity of model evaluation in realistic and domain-specific settings.




 %WIP
\section{Model}
\label{sec:model}
Let $[N] = \{1, 2, \dots, N \}$ be a set of $N$ agents.
We examine an environment in which a system interacts with the agents over $T$ rounds.
Every round $t\leq T$ comprises $N$ \emph{sessions}, each session represents an encounter of the system with exactly one agent, and each agent interacts exactly once with the system every round.
I.e., in each round $t$ the agents arrive sequentially. 


\paragraph{Arrival order} The \emph{arrival order} of round $t$, denoted as $\ordv_t=(\ord_t(1),\dots, \ord_t(N))$, is an element from set of all permutations of $[N]$. Each entry $q$ in $\ordv_t$ is the index of the agent that arrives in the $q^{\text{th}}$ session of round $t$.
For example, if $\ord_t(1) = 2$, then agent $2$ arrives in the first session of round $t$.
Correspondingly, $\ord_t^{-1}(i)=q$ implies that agent $i$ arrives in the $q^{\text{th}}$ session of round $t$. 

As we demonstrate later, the arrival order has an immediate impact on agent rewards. We call the mechanism by which the arrival order is set \emph{arrival function} and denote it by $\ordname$. Throughout the paper, we consider several arrival functions such as the \emph{uniform arrival} function, denoted by $\uniord$, and the \emph{nudged arrival} $\sugord$; we introduce those formally in Sections~\ref{sec:uniform} and~\ref{sec:nudge}, respectively.

%We elaborate more on this concept in Section~\ref{sec: arrival}.


\paragraph{Arms} We consider a set of $K \geq 2$ arms, $A = \{a_1, \ldots, a_K\}$. The reward of arm $a_i$ in round $t$ is a random variable $X_i^t \sim \mathcal{D}^t_i$, where the rewards $(X_i^t)_{i,t}$ are mutually independent and bounded within the interval $[0,1]$. The reward distribution $\mathcal{D}^t_i$ of arm $a_i$, $i\in [K]$ at round $t\in T$ is assumed to be non-stationary but independent across arms and rounds. We denote the realized reward of arm $a_i$ in round $t$ by $x_i^t$. We assume \emph{reward consistency}, meaning that rewards may vary between rounds but remain constant within the sessions of a single round. Specifically, if an arm $a_i$ is selected multiple times during round~$t$, each selection yields the same reward $x_i^t$, where the superscript $t$ indicates its dependence on the round rather than the session. This consistency enables the system to leverage information obtained from earlier sessions to make more informed decisions in later sessions within the same round. We provide further details on this principle in Subsection~\ref{subsec:information}.


\paragraph{Algorithms} An algorithm is a mapping from histories to actions. We typically expect algorithms to maximize some aggregated agent metric like social welfare. Let $\mathcal H^{t,q}$ denote the information observed during all sessions of rounds $1$ to $t-1$ and sessions $1$ to $q-1$ in round $t$.  The history $\mathcal H^{t,q}$ is an element from $(A \times [0,1])^{(t-1) \cdot N +q-1}$, consisting of pairs of the form (pulled arm, realized reward). Notice that we restrict our attention to \emph{anonymous} algorithms, i.e., algorithms that do not distinguish between agents based on their identities. Instead, they only respond to the history of arms pulled and rewards observed, without conditioning on which specific agent performed each action.
%In the most general case, algorithms make decisions at session $q$ of round $t$  based on the entire history $\mathcal H^{t,q}$ and the index of the arriving agent $\ord_t(q)$. %Furthermore, we sometimes assume that algorithms have Bayesian information, i.e., algorithms are aware of the distributions $(\mathcal D_i)^K_{i=1}$. 
Furthermore, we sometimes assume that algorithms have Bayesian information, meaning they are aware of the reward distributions $(\mathcal{D}^t_i)_{i,t}$. If such an assumption is required to derive a result, we make it explicit. %Otherwise, we do not assume any additional knowledge about the algorithm’s information. %This distinction allows us to analyze both general algorithms without prior distributional knowledge and specialized algorithms that leverage Bayesian information.


\paragraph{Rewards} Let $\rt{i}$ denote the reward received by agent $i \in [N]$ at round $t$, and let $\Rt{i}$ denote her cumulative reward at the end of round $t$, i.e., $\Rt{i} = \sum_{\tau=1}^{t}{r^{\tau}_{i}}$. We further denote the \emph{social welfare} as the sum of the rewards all agents receive after $T$ rounds. Formally, $\sw=\sum^{N}_{i=1}{R^T_i}$. We emphasize that social welfare is independent of the arrival order. 


\paragraph{Envy}
We denote by $\adift{i}{j}$ the reward discrepancy of agents $i$ and $j$ in round $t$; namely, $\adift{i}{j}= \rt{i} - \rt{j}$. %We call this term \omer{name??} reward discrepancy in round $t$. 
The (cumulative) \emph{envy} between two agents at round $t$ is the difference in their cumulative rewards. Formally, $\env_{i,j}^t= \Rt{i} - \Rt{j}$ is the envy after $t$ rounds between agent $i$ and $j$. We can also formulate envy as the sum of reward discrepancies, $\env_{i,j}^t= \sum^{t}_{\tau=1}{\adif{i}{j}^\tau}$. Notice that envy is a signed quantity and can be either positive or negative. Specifically, if $\env_{i,j}^t < 0$, we say that agent $i$ envies agent $j$, and if $\env_{i,j}^t > 0$, agent $j$ envies agent $i$. The main goal of this paper is to investigate the behavior of the \emph{maximal envy}, defined as
\[
\env^t = \max_{i,j \in [N]} \env^t_{i,j}.
\]
For clarity, the term \emph{envy} will refer to the maximal envy.\footnote{ We address alternative definitions of envy in Section~\ref{sec:discussion}.} % Envy can also be defined in alternative ways, such as by averaging pairwise envy across all agents. We address average envy in Section~\ref{sec:avg_envy}.}
Note that $\env_{i,j}^t$ are random variables that depend on the decision-making algorithm, realized rewards, and the arrival order, and therefore, so is $\env^t$. If a result we obtain regarding envy depends on the arrival order $\ordname$, we write $\env^t(\ordname)$. Similarly, to ease notation, if $\ordname$ can be understood from the context, it is omitted.



\paragraph{Further Notation} We use the subscript $(q)$ to address elements of the $q^{\text{th}}$ session, for $q\in [N]$.
That is, we use the notation $\rt{(q)}$ to denote the reward granted to the agent that arrives in the $q^{\text{th}}$ session of round $t$ and $\Rt{(q)}$ to denote her cumulative reward. %Additionally, we introduce the notation $\at{(q)}$ to denote the arm pulled in that session.
Correspondingly, $\sdift{q}{w} = \rt{(q)} - \rt{(w)}$ is the reward discrepancy of the agents arriving in the $q^{\text{th}}$ and $w^{\text{th}}$ sessions of round $t$, respectively. 
To distinguish agents, arms, sessions and rounds, we use the letters $i,j$ to mark agents and arms, $q,w$ for sessions, and $t,\tau$ for rounds.


\subsection{Example}
\label{sec: example}
To illustrate the proposed setting and notation, we present the following example, which serves as a running example throughout the paper.

\begin{table}[t]
\centering
\begin{tabular}{|c|c|c|c|}
\hline
$t$ (round) & $\ordv_t$ (arrival order) & $x_1^t$ & $x_2^t$ \\ \hline
1           & 2, 1                     & 0.6     & 0.92    \\ \hline
2           & 1, 2                     & 0.48    & 0.1     \\ \hline
3           & 2, 1                     & 0.15    & 0.8     \\ \hline
\end{tabular}
\caption{
    Data for Example~\ref{example 1}.
}
\label{tbl: example}
\end{table}

\begin{algorithm}[t]
\caption{Algorithm for Example~\ref{example 1}}
\label{alguni}
\DontPrintSemicolon 
\For{round $t = 1$ to $T$}{
    pull $a_{1}$ in the first session\label{alguniexample: first}\\
    \lIf{$x^t_1 \geq \frac{1}{2}$}{pull $a_{1}$ again in second session \label{alguniexample: pulling a again}}
    \lElse{pull $a_{2}$ in second session \label{alguniexample: sopt else}}
}
\end{algorithm}


\begin{example}\label{example 1}
We consider $K=2$ uniform arms, $X_1,X_2 \sim \uni{0,1}$, and $N=2$ for some $T\geq 3$. We shall assume arm decision are made by Algorithm~\ref{alguni}: In the first session, the algorithm pulls $a_1$; if it yields a reward greater than $\nicefrac{1}{2}$, the algorithm pulls it again in the second session (the ``if'' clause). Otherwise, it pulls $a_2$.



We further assume that the arrival orders and rewards are as specified in Table~\ref{tbl: example}. Specifically, agent 2 arrives in the first session of round $t=1$, and pulling arm $a_2$ in this round would yield a reward of $x^1_2 = 0.92$. Importantly, \emph{this information is not available to the decision-making algorithm in advance} and is only revealed when or if the corresponding arms are pulled.




In the first round, $\boldsymbol{\eta}^1 = \left(2,1\right)$; thus, agent 2 arrives in the first session.
The algorithm pulls arm $a_1$, which means, $a^1_{(1)} = a_1$, and the agent receives $r_{2}^1=r_{(1)}^1=x_1^1=0.6$.
Later that round, in the second session, agent 1 arrives, and the algorithm pulls the same arm again since $x^1_1 = 0.6 \geq \nicefrac{1}{2}$ due to the ``if'' clause.
I.e., $a^1_{(2)} = a_1$ and $r_{1}^1 = r_{(2)}^1 = x_1^1 = 0.6$.
Even though the realized reward of arm $a_2$ in that round is higher ($0.92$), the algorithm is not aware of that value.
At the end of the first round, $R^1_1 = R^1_{(2)} = R^1_2 = R^1_{(1)} = 0.6$. The reward discrepancy is thus $\adif{1}{2}^1 = \adif{2}{1}^1= \sdif{2}{1}^1 = 0.6 - 0.6 =0$. 



In the second round, agent 1 arrives first, followed by agent 2.
Firstly, the algorithm pulls arm $a_1$ and agent 1 receives a reward of $r_{1}^2 = r_{(1)}^2 = x_1^2 = 0.48$.
Because the reward is lower than $\nicefrac{1}{2}$, in the second session the algorithm pulls the other arm ($a^2_{(2)} = a_2$), granting agent 2 a reward of $r_{2}^2 = r_{(2)}^2 = x_2^2 = 0.1$.
At the end of the second round, $R^2_1 = R^2_{(1)} = 0.6 + 0.48 = 1.08$ and $R^2_2 = R^2_{(2)} = 0.6 + 0.1 = 0.7$. Furthermore, $\sdif{2}{1}^2 = \adif{2}{1}^2 = r^2_{2} - r^2_{1} = 0.1 - 0.48 = -0.38$.

In the third and final round, agent 2 arrives first again, and receives a reward  of $0.15$ from $a_1$. When agent 1 arrives in the second session, the algorithm pulls arm $a_2$, and she receives a reward of $0.8$. As for the reward discrepancy, $\sdif{2}{1}^3 = \adif{2}{1}^3 = r^3_{2} - r^3_{1} = 0.15 - 0.8 = -0.75$. 

Finally, agent 1 has a cumulative reward of $R^3_1 = R^3_{(2)} = 0.6 + 0.48 + 0.8 = 1.88$, whereas agent~2 has a cumulative reward of $R^3_2 = R^3_{(1)} = 0.6 + 0.1 + 0.15 = 0.85$. In terms of envy, $\env^1_{1,2}= \adif{1}{2}^1 =0$, $\env^2_{1,2}=\adif{1}{2}^1+\adif{1}{2}^2= 0.38$, and $\env^3_{1,2} = -\env^3_{2,1} = R^3_1-R^3_2 = 1.88-0.85 = 1.03$, and consequently the envy in round 3 is $\env^3 = 1.03$.
\end{example}


\subsection{Information Exploitation}
\label{subsec:information}

In this subsection, we explain how algorithms can exploit intra-round information.
Since rewards are consistent in the sessions of each round, acquiring information in each session can be used to increase the reward of the following sessions.
In other words, the earlier sessions can be used for exploration, and we generally expect agents arriving in later sessions to receive higher rewards.
Taken to the extreme, an agent that arrives after all arms have been pulled could potentially obtain the highest reward of that round, depending on how the algorithm operates.

To further demonstrate the advantage of late arrival, we reconsider Example~\ref{example 1} and Algorithm~\ref{alguni}. 
The expected reward for the agent in the first session of round $t$ is $\E{\rt{(1)}}=\mu_1=\frac{1}{2}$, yet the expected reward of the agent in the second session is
\begin{align*}
\E{\rt{(2)}}=\E{\rt{(2)}\mid X^t_1 \geq \frac{1}{2} }\prb{X^t_1 \geq \frac{1}{2}} + \E{\rt{(2)}\mid X^t_1 < \frac{1}{2} }\prb{X^t_1 < \frac{1}{2}};
\end{align*}
thus, $\E{\rt{(2)}} =\E{X^t_1\mid X^t_1 \geq \frac{1}{2} }\cdot \frac{1}{2} + \mu_2\cdot\frac{1}{2} = \frac{5}{8}$.
Consequently, the expected welfare per round is $\E{\rt{(1)}+\rt{(2)}}=1+\frac{1}{8}$, and the benefit of arriving in the second session of any round $t$ is $\E{\rt{(2)} - \rt{(1)}} = \frac{1}{8}$. This gap creates envy over time, which we aim to measure and understand.
%This discrepancy generates envy over time, and our paper aims to better understand it.
\subsection{Socially Optimal Algorithms}
\label{sec: sw}
Since our model is novel, particularly in its focus on the reward consistency element, studying social welfare maximizing algorithms represents an important extension of our work. While the primary focus of this paper is to analyze envy under minimal assumptions about algorithmic operations, we also make progress in the direction of social welfare optimization. See more details in Section~\ref{sec:discussion}.%Due to space limitations, we defer the discussion on socially optimal algorithms to  \ifnum\Includeappendix=0{the appendix}\else{Section~\ref{appendix:sociallyopt}}\fi.




% Since our model is novel and specifically the reward consistency element, it might be interesting to study social welfare optimization. While the main focus of our paper is to study envy under minimal assumptions on how the algorithm operates, we take steps toward this direction as well. Due to space limitations, we defer the discussion on socially optimal algorithms to  \ifnum\Includeappendix=0{the appendix}\else{Section~\ref{appendix:sociallyopt}}\fi.  We devise a socially optimal algorithm for the two-agent case, offer efficient and optimal algorithms for special cases of $N>2$ agents, and an inefficient and approximately optimal algorithm for any instance with $N>2$. Moreover, we address the welfare-envy tradeoff in Section~\ref{sec:extensions}.


% Social welfare, unlike envy, is entirely independent of the arrival order. While the main focus of our paper is to study envy under minimal assumptions on how the algorithm operates, socially optimal algorithms might also be of interest. Due to space limitations, we defer the discussion on socially optimal algorithms to  \ifnum\Includeappendix=0{the appendix}\else{Section~\ref{appendix:sociallyopt}}\fi. We devise a socially optimal algorithm for the two-agent case, offer efficient and optimal algorithms for special cases of $N>2$ agents, and an inefficient and approximately optimal algorithm for any instance with $N>2$. %Furthermore, we treat the welfare-envy tradeoff of the special case of Example~\ref{example 1}.


 
\subsection{Defining the Uniform Choice Model}
We introduce optimal menus first through a simple scenario where choice models have uniform probability of selection, with perfect information on preferences. 
We define a choice model through the function $Y_{i,j}(\mathbf{Z}_{t})$. 
First, consider a parameter, $0 \leq p \leq 1$. 
This parameter dictates the probability that a patient matches one of the providers in the system. 
We consider $p$ to be the same across all patients, and $p=0$ indicates that we achieve poor objectives in any scenario. 

Next, consider $Y_{i,j}(\mathbf{Z}_{t}) = Y_{i,j}(\{Z_{t,1}, \cdots, Z_{t,P})$. 
Recall that $r_{i,1} = \argmax_{j} \theta_{i,j}$. 
Similarly, define $s_{i,t} = \argmax_{j} \theta_{i,j} Z_{t,j}$. 
Then $Y_{i,j}(\{Z_{t,1}, \cdots, Z_{t,P}) = p$ if $j = s_{i,t}$, and is $0$ otherwise. 
Such a scenario represents the simplest choice model, where all patients select their highest available provider. 

 
\section{Nudged and Adversarial Arrival}
\label{sec:nudge}
In this section, we address nudged arrival: We assume an exogenous arrival mechanism can influence the order in which agents arrive. Practically, this captures scenarios where the system sends push notifications or otherwise encourages some agents to arrive earlier or later. Our goal is to analyze the envy of arbitrary algorithms without changing the way they select arms. We stress that our analysis still assumes anonymous algorithms: The decision-making process is not affected by agent identities.

% \omer{USE THIS SOMEHOW
% In this section, we examine the envy dynamics of algorithms that \emph{ignore} agent identities \emph{during} the sessions. That is, \omer{we try to minimize envy by making high rewraded agents arrive ealier} the algorithm can affect the arrival order at the beginning of the round, but does not condition the decision on which arm to choose in each session on the arriving agent. \omer{this is good but not perfect}}
Subsection~\ref{subsec:nudged prot} introduces the nudged arrival protocol, $\sugord$, and the accompanying assumptions. Later,  Subsection~\ref{Envy Analysis} presents our main result of the section: An upper bound of $\env^T({\sugord})$ for a broad class of algorithms. Interestingly, we show that the bound depends on the instance parameters but not on the horizon $T$. Finally, we complete this section by adopting a complementary approach, where an adversary can pick the worst arrival order in terms of envy, and show that an $\Omega(T)$ regret is inevitable. For clarity, we remind the reader that for $q,i\in [N]$, $\ordv_t (q)=i$ implies that agent $i$ has arrive in the $q$'th session in round $t$ (similarly for $\ordv_t^{-1}(i)=q$). 
\subsection{Nudged Arrival Protocol}\label{subsec:nudged prot}
\begin{algorithm}[t]
\caption{Nudged Arrival}
\label{alg: sugg arr}
\SetAlgoLined
\LinesNumbered
\KwIn{horizon $T$, nudge parameter $\delta$} \label{line:input}
\For{round $t = 1$ to $T$}{ \label{line:for_loop} 
    let $\sigv_t$ be an $[N]\rightarrow [N]$ mapping such that 
    \[
    R^{t-1}_{\sigma_t(N)} \leq R^{t-1}_{\sigma_t(N-1)} \leq \dots \leq R^{t-1}_{\sigma_t(2)} \leq R^{t-1}_{\sigma_t(1)}.    
    \]
    \label{line:mapping}\\
    sample an arrival order $\ordv_t \sim \sugord(\sigma_t,\delta)$ \label{line:request}
}
\end{algorithm}
We describe the arrival protocol in Algorithm~\ref{alg: sugg arr}. It receives the horizon $T$ and a nudge parameter~$\delta$ as input, and interacts with any recommender algorithm through the horizon. In each round in the for loop of Line~\ref{line:for_loop}, we pick an \textit{ideal permutation} $\sigma_t$ that orders the agents according to their cumulative rewards.  The mapping $\sigma_t$ prioritizes agents according to their cumulative rewards in previous rounds, from the most rewarded one to the least rewarded one. In our notation, $\sigma_t : \{1,\dots,n\} \to [N]$, so that $\sigma_t(i)$ is the name of the agent in position $i$. Hence, $\sigma_t(1)$ is the agent with the highest cumulative reward so far, and $\sigma_t(N)$ is the agent with the lowest. While we assume the agents can be nudged toward this ordering $\sigma_t$, we do not claim that it is implemented or enforced in its exact form. Instead, in Line~\ref{line:request}, we sample an arrival order $\ordv$ from   $\sugord(\sigma_t,\delta)$, which is a distribution over permutations of $N$. Particularly, we assume that $\sugord$ satisfies the \emph{nudged arrival} property.

\begin{property}[Nudged Arrival]\label{prop:nudge}
Given a scalar \( \delta \in (0,1) \) and a mapping \( \sigv : [N] \rightarrow [N] \),  for every two agents \( i,j \) with \( \sigma^{-1}(i) < \sigma^{-1}(j) \) the distribution $\sugord(\sigv,\delta)$ satisfies
\begin{equation}\label{def: nudged} 
\Pr_{\ordv \sim \sugord(\sigma,\delta)}\left(\ordv^{-1}(i) < \ordv^{-1}(j)\right) \geq \frac{1+\delta}{2}. 
\end{equation}  
\end{property}
In other words, Algorithm~\ref{alg: sugg arr} introduces a structured form of randomness in the sequence of agent arrivals. Each round allows for the selection of a permutation $\sigma_t$, representing a preferred (but not guaranteed) ordering of agents. For any two agents $i$ and $j$, if the permutation prioritizes agent $i$'s arrival over agent $j$'s arrival, i.e., $\sigma_t^{-1}(i) < \sigma_t^{-1}(j)$, then agent $i$ is more likely to arrive before agent $j$. Particularly, the probability of agent $i$ preceding agent $j$ in the realized arrival order $\ordv_t$ is at least $\frac{1}{2} + \frac{\delta}{2}$, ensuring a consistent bias $\delta$ toward the preferred ordering. Property~\ref{prop:nudge} is inspired by several well-known models of stochastic ranking, like Mallows model~\cite{mallows1957non}, Plackett-Luce~\cite{marden1996analyzing} and also noisy comparison models~\cite{braverman2007noisy}. In \ifnum\Includeappendix=0{the appendix}\else{Section~\ref{appendix:nudge-models}}\fi, we describe how to derive $\delta$ from each of these models. By specifying Property~\ref{prop:nudge} rather than the full underlying ordering model, we preserve flexibility in how global orderings are derived.  
Next, we illustrate nudged arrival and the role it plays in envy dynamics.
\begin{example}[Nudged Arrival and Envy Dynamics]  \label{example: envy with sugg}
We reconsider the setting of Example~\ref{example 1}, with $K=2$ arms with  $\uni{0,1}$ rewards, $N=2$ agents, Algorithm~\ref{alguni}, but with the nudged arrival ${\sugord}$. To ease readability, we keep using $\env^T$ to denote $\env^T({\sugord})$, omitting the dependence on  ${\sugord}$. Suppose that after $t-1$ rounds, for some arbitrary $t \in [T]$, agent rewards satisfy $\env^{t-1} =\env^{t-1}_{1,2} = R^{t-1}_1 - R^{t-1}_2 > 1$, indicating that agent 2 envies agent 1.


In this scenario, the expected envy after round $t$ is given by:  
{  
\thinmuskip=2mu  
\medmuskip=3mu plus 2mu minus 3mu  
\thickmuskip=4mu plus 5mu minus 2mu  
\begin{align}\label{eq:env_update}  
\mathbb{E}[\env^t \mid \env^{t-1}_{1,2} > 1]  
&= \mathbb{E}\left[\left| \env^{t-1}_{1,2} + \Delta^t_{1,2} \right| \mid \env^{t-1}_{1,2} > 1 \right] = \env^{t-1}_{1,2} + \mathbb{E}\left[\Delta^t_{1,2} \mid\env^{t-1}_{1,2} > 1 \right]. 
\end{align}  
}  



The term $\mathbb{E}[\Delta^t_{1,2} \mid \env^{t-1}_{1,2} > 1 ]$ can be expressed as:  
{  
\thinmuskip=2mu  
\medmuskip=3mu plus 2mu minus 3mu  
\thickmuskip=4mu plus 5mu minus 2mu  
\begin{align}\label{eq:delta_expression}  
\E{\Delta^t_{1,2} \mid \env^{t-1}_{1,2} > 1} &= \E{r_{(2)}^t - r_{(1)}^t \mid \env^{t-1}_{1,2} > 1, \ordv_t = (2,1)}\prb{\ordv_t = (2,1) \mid \env^{t-1}_{1,2} > 1} \nonumber \\  
& \qquad + \E{r_{(1)}^t - r_{(2)}^t \mid \env^{t-1}_{1,2} > 1, \ordv_t  = (1,2)}\prb{\ordv_t = (1,2) \mid \env^{t-1}_{1,2} > 1} \nonumber\\
&= \frac{1}{8} \left[\prb{\ordv_t = (2,1) \mid \env^{t-1}_{1,2} > 1} - \prb{\ordv_t = (1,2) \mid \env^{t-1}_{1,2} > 1} \right],  
\end{align}  
}%
where we have used the fact that $\E{\rt{(2)} - \rt{(1)}} = \frac{1}{8}$ as Subsection~\ref{subsec:information} suggests and the fact that the rewards $r_{(i)}^t$ for $i\in \{1,2 \}$ are independent of the arriving agent's identity.

Under nudged arrival, the ideal permutation is $\sigma_t = (1,2)$, prioritizing agent $1$'s arrival over agent $2$'s arrival;therefore, the permutation $(1,2)$ is more likely than $(2,1)$, resulting in
\[
\prb{\ordv_t = (2,1) \mid \env^{t-1}_{1,2} > 1} - \prb{\ordv_t = (1,2) \mid \env^{t-1}_{1,2} > 1} \leq -\delta.
\]  
Thus, rewriting Equation~\eqref{eq:delta_expression}, we have $\E{\Delta^t_{1,2} \mid \env^{t-1}_{1,2} > 1}  \leq -\frac{\delta}{8}$. To conclude this example, we plug this result into Equation~\eqref{eq:env_update} and obtain
\[
\mathbb{E}[\env^t \mid \env^{t-1}_{1,2} > 1] \leq \env^{t-1}_{1,2}-\frac{\delta}{8},
\]
suggesting that the cumulative envy $\env^t$ is likely to decrease in round $t$ by a non-negligible value.
\end{example}
To be able to analyze envy dynamics and show that it cannot grow too much, we need to have some regularity assumptions on the way algorithms we analyze operate. Indeed, the envy reduction in Example~\ref{example: envy with sugg} relies heavily on the fact that $\E{\rt{(2)} - \rt{(1)}} = \frac{1}{8} > 0$. This inequality ensures that, in expectation, arriving second leads to a higher reward. While we aim to analyze any arbitrary algorithm, we need to ensure that the ordering $\sigma_t$ in Line~\ref{line:mapping} reduces envy in expectation.
The following natural assumption generalizes the behavior of Algorithm~\ref{alguni} in Example~\ref{example: envy with sugg}.\begin{assumption}\label{assumption: nudge alg ref}
In every round $t \in [T]$, the algorithm picks arms so that
\[
    \E{r_{(1)}^t} \leq \E{r_{(2)}^t}\leq \cdots \leq \E{r_{(N)}^t}.
\]
\end{assumption}
To satisfy this assumption,\footnote{In fact, all of our results holds for the much broader case where there exists a permutation $\sigma:N \rightarrow N$ and the algorithm picks arms so that 
$\E{r_{(\sigma(1))}^t} \leq \E{r_{(\sigma(2))}^t}\leq \cdots \leq \E{r_{(\sigma(N))}^t}$. In such a case, we would pick $\sigma_t$ in Line~\ref{line:mapping} of Algorithm~\ref{alg: sugg arr} so that $\left(R^{t-1}_{\sigma_t(\sigma(i))}\right)_i$ is an increasing series.} the algorithm at hand must have some information about the expected rewards; Bayesian information is a sufficient condition, although not necessary.


Furthermore, without loss of generality, we shall assume that $\prb{\Delta^t_{(N),(1)} \neq 0}>0$ in every round $t$. This is indeed without loss of generality, as Assumption~\ref{assumption: nudge alg ref} already guarantees that $\E{\Delta^t_{(N),(1)}}=\E{r^t_{(N)}-r^t_{(1)}} \geq 0 $, and any round in which $\prb{\Delta^t_{(N),(1)} \neq 0}=0$ does not affect envy and can be disregarded. Additionally, to simplify our analysis, we introduce the new notation $\tilde{\dif}$, denoting
\begin{equation}\label{eq def tdif}
\tdif = \min_{1\leq i <j \leq N, t\in [T]:  \prb{\Delta^t_{(j),(i)} \neq 0}>0 }\left\{ \E{\Delta^t_{(j),(i)} \mid \Delta^t_{(j),(i)} \neq 0} \right\}    
\end{equation}
The quantity $\tdif >0$ %, which is well defined as long as since we assume that $\prb{\Delta^t_{(N),(1)} \neq 0}>0$ for every $t$, 
represents a lower bound on our ability to decrease envy from round to round. Recall that Assumption~\ref{assumption: nudge alg ref} implies that $\E{\Delta^t_{(j),(i)}}=\E{r^t_{(j)}-r^t_{(i)}} \geq 0 $ for $j>i$. However, $\Delta^t_{(j),(i)}$ can take 0 sometimes,\footnote{For instance, if $N > K+1$, any explore-first algorithm would have $\Delta^t_{(N-1),(N)}=0$ as the algorithm will pick the same arm for both sessions.} which means there is no scope for nudged arrival to further reduce envy. As long as Assumption~\ref{assumption: nudge alg ref} holds and $ \prb{\Delta^t_{(j),(i)} \neq 0}>0$, we know that
\[
\E{\Delta^t_{(j),(i)} \mid \Delta^t_{(j),(i)} \neq 0}=\frac{ \E{\Delta^t_{(j),(i)}}}{\prb{\Delta^t_{(j),(i)} \neq 0}};
\] thus, we expect $\tdif$ to be significant. To illustrate, recall that in Example~\ref{example: envy with sugg} it holds that $\E{\Delta^t_{(2),(1)}}=\frac{1}{8}$, whereas $\E{\Delta^t_{(2),(1)} \mid \Delta^t_{(2),(1)} \neq 0}=\frac{1}{4}$.
\subsection{Envy Analysis}\label{Envy Analysis}
We are ready to present the main result of the section: Upper bounding the envy under nudged arrival.
\begin{theorem}\label{thm: sugg-envy}
    When executing any algorithm that satisfies Assumption~\ref{assumption: nudge alg ref} with nudged arrival, the expected envy is
    \[\E{\env^T({\sugord})}\leq (N-1)\left(2+\frac{128}{15\delta \tdif}\right) .\]
\end{theorem}
Notice that this upper bound does not depend on the horizon $T$. Intuitively, under nudged arrival, envy behaves like a random walk with a drift toward zero. Although each round may introduce a discrepancy (akin to a random fluctuation), the nudging mechanism consistently pushes the cumulative difference back toward zero. Furthermore, the bound is inversely proportional to $\delta$ and $\tdif$: As $\delta$ decreases, the nudging effect weakens and nudged arrival increasingly resembles uniform arrival. We suspect the terms in the bound are not tight; we discuss it in Section~\ref{sec:discussion}.
\begin{proof}[Proof of Theorem~\ref{thm: sugg-envy}]
Fix any arbitrary algorithm satisfying  Assumption~\ref{assumption: nudge alg ref} and any arbitrary~$t \in [T]$. The proof is outlined as follows:
\begin{enumerate}
    \item Step 1 introduces envy gaps as stochastic processes and the concept of envy excursions.
    \item Demonstrating that envy gaps are nontrivial to analyze, Step 2 presents a more friendly stochastic process that we prove to upper bound the envy gap almost surely.
    \item Step 3 leverages Property~\ref{prop:nudge} and concentration inequalities to upper bound large deviations of the friendly stochastic process.
    \item Lastly, Step 4 uses the tail formula and the concentration from Step 3 to bound to cumulative envy.
\end{enumerate}
\textbf{Step 1: Envy Gap and Excursion}
For every $t$, let $\sigma_t: [N] \rightarrow [N]$ be the ideal permutation from Line~\ref{line:mapping}, i.e.,  
\[
R^{t-1}_{\sigma_t(N)} \leq  R^{t-1}_{\sigma_t(N-1)} \leq \cdots \leq R^{t-1}_{\sigma_t(2)} \leq R^{t-1}_{\sigma_t(1)} .
\]
For every $i, 1\leq i \leq N-1$, we define the \emph{envy gap} $G_i^t = R^t_{\sigma_t(i)} - R^t_{\sigma_t(i+1)}$, representing the envy between the agent with the $i$-th highest reward and the agent with the $(i+1)$-th highest reward.
Consequently, we can define $\env^t$ using the envy gap sequence $(G^t_i)_i$ by
\begin{equation}\label{eq:jknbvfbjj}
\env^t=R^t_{\sigma_t(1)}-R^t_{\sigma_t(N)}=\sum_{i=1}^{N-1} R^t_{\sigma_t(i)}-R^t_{\sigma_t(i+1)} = \sum_{i=1}^{N-1} G_i^t.   
\end{equation}

Next, fix any arbitrary $i$ in the range. We continue by analyzing \emph{excursions} from low envy to high envy and showing they are relatively short, meaning that the expected envy $\E{G_i^t}$ is low. We define an excursion as a sequence of consecutive rounds during which the gap $G_i^\cdot$ exceeds $1$. 
Let $\underline{t} = \max{\left\{ \tau \mid 1 \leq \tau \leq t, G_i^{\tau} \leq 1 \right\}}$ denote last round $\tau$ before $t$ that $G_i^{\tau}$ was less than 1. Similarly,  let $\bar{t} = \min{\left\{ \tau \mid t\leq \tau \leq T, G_i^{\tau} \leq 1 \right\}}$ be the first round $\tau$ after $t$ where $G_i^{\tau}$ is less than 1. For the extreme case where $\bar{t}$ is undefined, we set $\bar{t} = T + 1$. Furthermore, let $D(t)$ denote $t$'s excursion, namely, the set of all the consecutive rounds $\tau$ that includes $t$ during which $G_i^\tau \geq 1$. Formally, $D(t) = \{\tau | \underline{t} < \tau < \bar{t}\}$. Notice that $D(t)$ is an empty set if and only if $G_i^t \leq 1$.

\textbf{Step 2: Auxiliary Stochastic Process}
The sequence $\{G_i^\tau\}_{\tau \in D(t)}$ is challenging to work with because the agents occupying the $i$-th and $(i+1)$-th highest reward position may change from round to round.  We refer to these changes as \emph{rank swaps}, which cause increments like  $G_i^{\tau+1} - G_i^\tau$ to lack a consistent structure. To address this complexity, we introduce the stochastic process $(M^\tau)_\tau$, which is easier to analyze.
%$G_i^\tau$ for $\tau \in D(t)$ is challenging to work with. To demonstrate, notice that the identities of the agent with the $i$-th highest reward and the agent with the $(i+1)$-th highest reward can change from round to round; thus, the increment $G_i^{\tau+1}-G_i^{\tau}$ has no clear structure. 
%To that end, we introduce the stochastic process $(M^\tau)_\tau$, which is easier to analyze. We define it as follows:
\begin{align*}
    M^\tau =
    \begin{cases}
        2 & \text{if } \tau = \underline{t}+1 \\
        M^{\tau-1}+r^\tau_{(i)}-r^\tau_{(i+1)} & \text{else}
        \end{cases}.
\end{align*}
Unlike $G_i^{\tau+1}-G_i^{\tau}$, the increments $M^{\tau+1}-M^{\tau} = r^\tau_{(i)}-r^\tau_{(i+1)}$ are more straightforward and negative in expectation due to Assumption~\ref{assumption: nudge alg ref}. The next proposition demonstrates that $M^\tau_i$ can assist when analyzing $G^\tau_i$.
\begin{proposition}\label{prop G less than M}
For every $\tau \in D(t)$, it holds that $G^\tau_i \leq M^\tau_i$ almost surely.
\end{proposition}
The proof of Proposition~\ref{prop G less than M} appears in \ifapp{Section~\ref{appendix:nudge}}{the appendix}. The main argument enabling this statement is that rank swaps can only decrease the increments $G_i^{\tau+1}-G_i^{\tau}$, but do not affect the increments $M^{\tau+1}-M^{\tau}$.

\textbf{Step 3: Concentration}
The recursive definition of $M^\tau$ implies that for every $\tau \in D(t)$, $M^\tau = 2+ \sum_{n= \underline{t}+2 }^{\tau} r^n_{(i)}-r^n_{(i+1)}$. The next proposition bounds large deviations of $M^\tau$.
\begin{proposition}\label{prop: sugg-m concentration}
    For any $n \in \mathbb{N}$ and $\tau \in D(t)$, it holds that
    \begin{align*}
    \prb{M^\tau > n}\leq \exp\left\{-\frac{(n-2)(\delta \tdif)}{8}\right\}.
    \end{align*}
\end{proposition}
%Intuitively, the bound in Proposition~\ref{prop: sugg-m concentration}  
The proof of Proposition~\ref{prop: sugg-m concentration} appears in \ifapp{Section~\ref{appendix:nudge}}{the appendix}. It leverages the Azuma-Hoeffding inequality and several algebraic tricks to obtain the bound. As we expect, as $\delta$ and $\tdif$ increase, the right-hand side becomes more significant. Alternatively, if the term $\delta \tdif$ approach zero, this bound become irrelevant as $\exp\{0\}=1$.

\textbf{Step 4: Tail Sum}
To finalize the proof, we use the tail-sum formula. Since $G_i^t$ is non-negative,
\begin{align*}
\E{G_i^t} &=
    \int_{x=0}^{\infty} \prb{G_i^t > x}dx \leq
    \sum_{n=0}^{\infty}{\int_{x=n}^{n+1} \prb{G_i^t > n} \,dx}=
    \sum_{n=0}^{\infty}{\prb{G_i^t > n}}.
\end{align*}    
Next, by applying Propositions~\ref{prop G less than M} and~\ref{prop: sugg-m concentration}, we conclude that
\begin{align}\label{eq:tail formula}
    \sum_{n=0}^{\infty}{\prb{G_i^t > n}}& \stackrel{\textnormal{Prop. \ref{prop G less than M}}}{\leq} \sum_{n=0}^{\infty}{\prb{M_i^t > n}} \leq 2+ \sum_{n=2}^{\infty}{\prb{M_i^t > n}}   \stackrel{\textnormal{Prop. \ref{prop: sugg-m concentration}}}{\leq} 2+ \sum_{n=2}^{\infty}\exp{\left\{-\frac{(n-2)\delta \tdif}{8} \right\}}\nonumber \\
    &= 2+ \sum_{n=0}^{\infty}\left(e^{-\frac{\delta \tdif}{8}} \right)^n = 2+\frac{1}{1-\exp{\left(\frac{-\delta \tdif}{8} \right)}} \stackrel{e^{-x}\leq 1-x+\frac{x^2}{2}}{\leq}
    2+\frac{1}{\frac{\delta \tdif}{8}-\frac{(\delta \tdif)^2}{128}}
 \nonumber \\
    & \stackrel{\delta \tdif \leq 1}{\leq} 2+\frac{1}{\frac{\delta \tdif}{8}-\frac{\delta \tdif}{128}} = 2+\frac{128}{15\delta \tdif}.
\end{align}
Ultimately, recall that Inequality~\eqref{eq:tail formula} applies to $\E{G_i^t}$ for every $i$; therefore, Equation~\eqref{eq:jknbvfbjj} ensures that $\E{\env^t}=\E{\sum_{i=1}^{N-1} G_i^t} \leq (N-1)\left(2+\frac{128}{15\delta \tdif}\right) $. This concludes the proof of Theorem~\ref{thm: sugg-envy}.
\end{proof}

{\color{green}




}
\subsection{Adversarial Arrival}
\label{sec: advord}
We end this section by focusing on the adversarial arrival order $\advord$. Intuitively, an adversary seeking to maximize envy would reverse the ideal permutation, placing agents in descending order of their current cumulative rewards. That is, sets the order $\ordv_t$ in round $t$ such that
\[
R^{t-1}_{\ordv_t(N)} \geq R^{t-1}_{\ordv_t(N-1)} \geq \dots \geq R^{t-1}_{\ordv_t(2)} \geq R^{t-1}_{\ordv_t(1)}.
\]
Indeed, it is easy to see that:
\begin{proposition}\label{thm: adv-envy}
When executing any algorithm that satisfies Assumption~\ref{assumption: nudge alg ref} with $\advord$, the expected envy is
    \[\E{\env^T ({\advord})}\geq \tilde{\dif}T.\]
\end{proposition}
\begin{proof}[Proof of Proposition~\ref{thm: adv-envy}]
Assume that the adversary picks agent 1 to be the first and agent $N$ to be the last, i.e., $\ordv_t(1)=1$ and $\ordv_t(N)=N$ for all $t\in [T]$. In such a case,
\begin{align*}
\E{\env^T({\advord})} &=
\E{\max_{i,j\in [N]}{\left\{ \sum^T_{t=1}{\adift{i}{j}} \right\} }} \geq \E{\sum^T_{t=1}{\dif_{N,1}^t}} = \E{\sum^T_{t=1}{\dif^t_{(N),(1)}}}  \geq T\E{\min_{1 \leq t \leq T}{\left\{\sdift{N}{1} \right\} }} \\
&\geq \tdif T.    
\end{align*}
This concludes the proof of Proposition~\ref{thm: adv-envy}.
\end{proof}


\section{Extension: Trading Envy and Welfare}
\label{sec:extensions}








In the previous section, we have shown that coordinating agents' arrival order alone can significantly reduce envy, without affecting the algorithm's core decision-making process. In this section, we take an initial step toward understanding the efficiency-fairness tradeoff, a well-established concept in the literature on fair allocation~\cite{varian1973equity,bertsimas2012efficiency}  and fair classification~\cite{menon2018cost,zafar2017fairness}. 
Specifically, we extend the definition of algorithms from Section~\ref{sec:model} to allow agent‐specific treatment. In other words, algorithms can now observe agent identities and maintain agents accumulated rewards in their memory. Formally, the relevant histories contain triples of the form (agent index, pulled arm, realized reward). We hope to leverage this additional capability to balance social welfare and envy. 


%Misusing agent-specific information can trivially drive envy to $\Theta(T)$. For example, an algorithm that discriminates against agent~$N$ by always assigning it the arm with the lowest expected reward accrues a constant reward gap each round, leading to $\Theta(T)$ envy. Hence, we must proceed cautiously, 

We focus on the special case of our running example (Example~\ref{example 1}): $N=2$ agents, $K=2$ arms with rewards drawn from the uniform distribution, $X_1, X_2 \sim \uni{0,1}$, and the uniform arrival $\uniord$. Furthermore, we assume Bayesian information, i.e., the prior distributions are known. We stress that our results are preliminary, albeit non-trivial. In Subsection~\ref{subsec:ext welfare}, we analyze the socially optimal algorithm from a welfare and envy perspective. Later, in Subsection~\ref{subsec:ext efc}, we develop $\efc$, our welfare-envy balancing algorithm. 
\subsection{Optimal Welfare and Optimal Envy}\label{subsec:ext welfare}
We first analyze the maximal social welfare for this setting. As it turns out, Algorithm~\ref{alguni} is a special case of the optimal two-agent algorithm, as we prove in \ifapp{Section~\ref{appendix:sociallyopt}}{the appendix}. Along with our results from Section~\ref{sec:uniform}, we conclude that:
\begin{observation}\label{obs:opt for tradeoff}
When executing Algorithm~\ref{alguni} on the instance of Example~\ref{example 1} and $\uniord$, it achieves an expected social welfare of (1+$\frac{1}{8}) T$ and induces an expected envy of $\env^T(\uniord)=\tilde \Theta(\sqrt T)$. Furthermore, this is the optimal welfare.
\end{observation}
%The observation suggests that Algorithm~\ref{alguni} is positioned on an extreme point of the Pareto frontier of welfare-envy XXXX, with maximal welfare and high envy. 
This observation indicates that Algorithm~\ref{alguni} occupies an extreme point on the Pareto frontier of the welfare-envy tradeoff: It achieves maximum welfare but also incurs high envy. Another point on the frontier is the algorithm we call $NE$ (\textbf{N}o \textbf{E}nvy), guaranteeing $\env^t=0$ in every round $t$ almost surely. $NE$ draws the same arm in both sessions of every round, as this is essential to maintain zero envy (since the rewards are uniformly distributed and stochastic by nature). Of course, $NE$ has an expected social welfare of $\sw= T$. 


A compelling way to address the social welfare of any algorithm is by examining its ability to exploit the information obtained in the earlier sessions. For example, comparing the performance of $NE$ and Algorithm~\ref{alguni} highlights this difference: $NE$ does not utilize information from the first session, whereas Algorithm~\ref{alguni} leverages it to secure a better reward in the second round. This strategic use of information by Algorithm~\ref{alguni} results in a welfare increase of $\frac{1}{8}$ each round, but also generates envy. This is a key element in the algorithm we propose next.
\subsection{Envy-freeness up to $C$}\label{subsec:ext efc}
%The algorithmic2e version!
% \begin{algorithm}[t]
% \caption{Envy-freeness up to 1 ($\ef$)}
% \label{alg: ef1}
% \begin{algorithmic}[1]
% \Require {horizon $T$}
% \For{round $t = 1 \ldots T$}\label{alg: ef1 1}
%     \State{pull $a_{1}$}\label{alg: ef1 2}
%     \If{$x^t_1 > \frac{1}{2}$}\label{alg: ef1 3}
%         \State{pull $a_{1}$}\label{alg: ef1 4}
%     \Else\label{alg: ef1 5}
%         \If{$\abs{R^{t-1}_{(1)} + \rt{(1)} - R^{t-1}_{(2)}} \leq 1$ and $\abs{R^{t-1}_{(1)} + \rt{(1)} - R^{t-1}_{(2)} - 1} \leq 1$}\label{alg: ef1 6}
%             \State{pull $a_{2}$}\label{alg: ef1 7}
%         \Else\label{alg: ef1 8}
%             \State{pull $a_{1}$}\label{alg: ef1 9}
%         \EndIf
%     \EndIf
% \EndFor
% \end{algorithmic}
% \end{algorithm}
\begin{algorithm}[t]
\caption{Envy-freeness up to $C$ ($\efc$)}
\label{alg:efc}
\SetAlgoLined
\DontPrintSemicolon
\LinesNumbered
\KwIn{horizon $T$, envy bound $C$}
\For{round $t = 1$ to $T$\label{efcline:for}}{
    pull $a_{1}$\label{efclin:pull_a1}\\
    \lIf{$x^t_1 > \frac{1}{2}$}{pull $a_{1}$\label{efclin:pull_a1_again}}
    \Else{\label{efclin:else}
        \lIf{\textnormal{there exists $r\in [0,1]$ such that $\abs{R^{t-1}_{(1)} + r^t_{(1)} - R^{t-1}_{(2)}-r} > C$}}{pull $a_{1}$\label{efclin:pull_a1_cond}}
        %\lIf{$\abs{R^{t-1}_{(1)} + \rt{(1)} - R^{t-1}_{(2)}} \leq C$ and $\abs{R^{t-1}_{(1)} + \rt{(1)} - R^{t-1}_{(2)} - 1} \leq C$}{Pull $a_{2}$\label{efclin:pull_a2}}\textbf{}
        \lElse{pull $a_{2}$\label{efclin:pull_a2_cond_else}}
    }
}
\end{algorithm}

In what follows, we introduce $\efc$, which is an abbreviation of \textbf{E}nvy-\textbf{F}reeness up to $\textbf{C}$, and is
implemented in Algorithm~\ref{alg:efc}. $\efc$ operates by selectively limiting the exploitation of information when the gap between agents' rewards could potentially exceed a predefined envy threshold $C$. This mechanism enforces envy-freeness up to $C$, allowing for better welfare compared to $NE$ while maintaining low envy.

We now describe how $\efc$ works. It interacts with agents for $T$ rounds (Line~\ref{efcline:for}). In every round~$t$, $\efc$ pulls arm $a_1$ for the agent arriving in the first session (Line~\ref{efclin:pull_a1}). The decision to pull $a_1$ first is arbitrary since both arms are identically distributed. If $a_1$ realizes a high reward, i.e., $r^t_{(1)}=x^t_1 > \frac{1}{2}$, $\efc$ pulls it again for the agent in the second session (Line~\ref{efclin:pull_a1_again}). Otherwise, we enter the ``else'' clause in Line~\ref{efclin:else}.

If $a_1$ yields a low reward, the welfare-wise correct action is to pull $a_2$; however, recall that $\efc$ aims to keep the envy lower than $C$. As a result, it ensures that the envy $\abs{R^t_{(1)}-R^t_{(2)}}$ at the end of round $t$ is lower or equal to $C$. Specifically, the ``if'' clause in Line~\ref{efclin:pull_a1_cond} asks whether there exists a realization $r^t_{(2)}=r$ for which the envy would exceed $C$ by the end of the round. If such a realization exists, it pulls $a_1$. Otherwise, it pulls $a_2$ in Line~\ref{efclin:pull_a2_cond_else}. We term $\ef$ the special case of $\efc$ for $C=1$. 
\begin{theorem}\label{thm:ef1evny+sw}
When executing $\efc$ on the instance of Example~\ref{example 1} and $\uniord$, the following hold:
\begin{enumerate}
    \item For all $t$, $\env^t \leq C$ almost surely.
    \item For $C=1$, the social welfare is $\sw \geq \left(1 + \frac{1}{16}\right)T$.
\end{enumerate}
\end{theorem}
Interestingly, Theorem~\ref{thm:ef1evny+sw} implies that $\ef$ recovers half of the social welfare increase due to exploiting information, $(1+\nicefrac{1}{16})T$ versus $(1+\nicefrac{1}{8})T$ for Algorithm~\ref{alguni} and $T$ for $NE$, while limiting the maximal envy to 1 almost surely. We provide a proof sketch below and defer the full proof to \ifapp{Section~\ref{appendix:sociallyopt}}{the appendix}.
\begin{proof}[Proof sketch of Theorem~\ref{thm:ef1evny+sw}]
The first part of the theorem follows directly, as Line~\ref{efclin:pull_a1_cond} allows the envy at $t$ to change only if
\[
\prb{\env^t> C\middle| R^{t-1}_{(1)}, R^{t-1}_{(2)},r^t_{(1)}}=
\prb{\abs{R^{t-1}_{(1)} + r^t_{(1)} - R^{t-1}_{(2)}-r^t_{(2)}} > C\middle| R^{t-1}_{(1)}, R^{t-1}_{(2)},r^t_{(1)}}=0.
\]
The second part of the theorem requires a more detailed argument. We need to understand how often $\efc$ can exploit the information of the first session and enter the ``if'' clause in Line~\ref{efclin:pull_a1_cond}. Since the probability of entering the clause depends on the current level of envy, we must first understand how envy behaves over time. The main technical ingredient we use is the following proposition.
\begin{proposition}\label{prop:ef1 uni dominance}%{thm: ef1 stochastic dominance}
In every round $t$, the distribution of $\env^t$ is stochastically dominated by the uniform distribution over $[0,1]$. I.e., for any $x \in [0,1]$, it holds that $\prb{\env^t \leq x} \geq x$.
\end{proposition}
Despite its intuitive nature, proving this claim demands careful and thorough case analysis. Equipped with Proposition~\ref{prop:ef1 uni dominance}, we turn to analyze how often $\ef$ pulls $a_2$ after observing a low reward $r^t_{(1)} \leq \frac{1}{2}$ in the first session.
\begin{proposition}\label{prop:ef1 open arm}
%\label{thm: ef1 open arm}
        In every round $t$ with $\rt{(1)}\ \leq \frac{1}{2}$, it holds that $\prb{a^t_{(2)} \neq a^t_{(1)} \mid \rt{(1)} \leq \frac{1}{2}} \geq \frac{1}{2}$. 
\end{proposition}
We complete the proof by computing $\sw(\ef)$ via the law of total expectation, using Proposition~\ref{prop:ef1 open arm} to show that the welfare in every round is $1+\frac{1}{16}$, matching the statement of the theorem.
\end{proof}

\subsection{Beyond $C=1$}
The envy analysis in Theorem~\ref{thm:ef1evny+sw} concerns $\ef$, which is a special case of $\efc$ with $C=1$. Unfortunately, our techniques rely heavily on this fact, and extending it would require a different approach. Our preliminary investigation has led us to the following conjecture.
\begin{conjecture}\label{thm: efc sw}
When executing $\efc$ on the instance of Example~\ref{example 1} with any $C \geq 1$ and $\uniord$, the expected social welfare of at least $\sw \geq (1+\frac{1}{8}-\frac{1}{16C})T$.
\end{conjecture}
Simulations we conducted and appear in \ifapp{Section~\ref{sec:simulations}}{the appendix} suggest that Conjecture~\ref{thm: efc sw} holds, and we hope future work could formally prove it.
%While we could not prove this conjecture, we validate it empirically in Chapter~\ref{chap: simulations}.


\section{Discussion of Assumptions}\label{sec:discussion}
In this paper, we have made several assumptions for the sake of clarity and simplicity. In this section, we discuss the rationale behind these assumptions, the extent to which these assumptions hold in practice, and the consequences for our protocol when these assumptions hold.

\subsection{Assumptions on the Demand}

There are two simplifying assumptions we make about the demand. First, we assume the demand at any time is relatively small compared to the channel capacities. Second, we take the demand to be constant over time. We elaborate upon both these points below.

\paragraph{Small demands} The assumption that demands are small relative to channel capacities is made precise in \eqref{eq:large_capacity_assumption}. This assumption simplifies two major aspects of our protocol. First, it largely removes congestion from consideration. In \eqref{eq:primal_problem}, there is no constraint ensuring that total flow in both directions stays below capacity--this is always met. Consequently, there is no Lagrange multiplier for congestion and no congestion pricing; only imbalance penalties apply. In contrast, protocols in \cite{sivaraman2020high, varma2021throughput, wang2024fence} include congestion fees due to explicit congestion constraints. Second, the bound \eqref{eq:large_capacity_assumption} ensures that as long as channels remain balanced, the network can always meet demand, no matter how the demand is routed. Since channels can rebalance when necessary, they never drop transactions. This allows prices and flows to adjust as per the equations in \eqref{eq:algorithm}, which makes it easier to prove the protocol's convergence guarantees. This also preserves the key property that a channel's price remains proportional to net money flow through it.

In practice, payment channel networks are used most often for micro-payments, for which on-chain transactions are prohibitively expensive; large transactions typically take place directly on the blockchain. For example, according to \cite{river2023lightning}, the average channel capacity is roughly $0.1$ BTC ($5,000$ BTC distributed over $50,000$ channels), while the average transaction amount is less than $0.0004$ BTC ($44.7k$ satoshis). Thus, the small demand assumption is not too unrealistic. Additionally, the occasional large transaction can be treated as a sequence of smaller transactions by breaking it into packets and executing each packet serially (as done by \cite{sivaraman2020high}).
Lastly, a good path discovery process that favors large capacity channels over small capacity ones can help ensure that the bound in \eqref{eq:large_capacity_assumption} holds.

\paragraph{Constant demands} 
In this work, we assume that any transacting pair of nodes have a steady transaction demand between them (see Section \ref{sec:transaction_requests}). Making this assumption is necessary to obtain the kind of guarantees that we have presented in this paper. Unless the demand is steady, it is unreasonable to expect that the flows converge to a steady value. Weaker assumptions on the demand lead to weaker guarantees. For example, with the more general setting of stochastic, but i.i.d. demand between any two nodes, \cite{varma2021throughput} shows that the channel queue lengths are bounded in expectation. If the demand can be arbitrary, then it is very hard to get any meaningful performance guarantees; \cite{wang2024fence} shows that even for a single bidirectional channel, the competitive ratio is infinite. Indeed, because a PCN is a decentralized system and decisions must be made based on local information alone, it is difficult for the network to find the optimal detailed balance flow at every time step with a time-varying demand.  With a steady demand, the network can discover the optimal flows in a reasonably short time, as our work shows.

We view the constant demand assumption as an approximation for a more general demand process that could be piece-wise constant, stochastic, or both (see simulations in Figure \ref{fig:five_nodes_variable_demand}).
We believe it should be possible to merge ideas from our work and \cite{varma2021throughput} to provide guarantees in a setting with random demands with arbitrary means. We leave this for future work. In addition, our work suggests that a reasonable method of handling stochastic demands is to queue the transaction requests \textit{at the source node} itself. This queuing action should be viewed in conjunction with flow-control. Indeed, a temporarily high unidirectional demand would raise prices for the sender, incentivizing the sender to stop sending the transactions. If the sender queues the transactions, they can send them later when prices drop. This form of queuing does not require any overhaul of the basic PCN infrastructure and is therefore simpler to implement than per-channel queues as suggested by \cite{sivaraman2020high} and \cite{varma2021throughput}.

\subsection{The Incentive of Channels}
The actions of the channels as prescribed by the DEBT control protocol can be summarized as follows. Channels adjust their prices in proportion to the net flow through them. They rebalance themselves whenever necessary and execute any transaction request that has been made of them. We discuss both these aspects below.

\paragraph{On Prices}
In this work, the exclusive role of channel prices is to ensure that the flows through each channel remains balanced. In practice, it would be important to include other components in a channel's price/fee as well: a congestion price  and an incentive price. The congestion price, as suggested by \cite{varma2021throughput}, would depend on the total flow of transactions through the channel, and would incentivize nodes to balance the load over different paths. The incentive price, which is commonly used in practice \cite{river2023lightning}, is necessary to provide channels with an incentive to serve as an intermediary for different channels. In practice, we expect both these components to be smaller than the imbalance price. Consequently, we expect the behavior of our protocol to be similar to our theoretical results even with these additional prices.

A key aspect of our protocol is that channel fees are allowed to be negative. Although the original Lightning network whitepaper \cite{poon2016bitcoin} suggests that negative channel prices may be a good solution to promote rebalancing, the idea of negative prices in not very popular in the literature. To our knowledge, the only prior work with this feature is \cite{varma2021throughput}. Indeed, in papers such as \cite{van2021merchant} and \cite{wang2024fence}, the price function is explicitly modified such that the channel price is never negative. The results of our paper show the benefits of negative prices. For one, in steady state, equal flows in both directions ensure that a channel doesn't loose any money (the other price components mentioned above ensure that the channel will only gain money). More importantly, negative prices are important to ensure that the protocol selectively stifles acyclic flows while allowing circulations to flow. Indeed, in the example of Section \ref{sec:flow_control_example}, the flows between nodes $A$ and $C$ are left on only because the large positive price over one channel is canceled by the corresponding negative price over the other channel, leading to a net zero price.

Lastly, observe that in the DEBT control protocol, the price charged by a channel does not depend on its capacity. This is a natural consequence of the price being the Lagrange multiplier for the net-zero flow constraint, which also does not depend on the channel capacity. In contrast, in many other works, the imbalance price is normalized by the channel capacity \cite{ren2018optimal, lin2020funds, wang2024fence}; this is shown to work well in practice. The rationale for such a price structure is explained well in \cite{wang2024fence}, where this fee is derived with the aim of always maintaining some balance (liquidity) at each end of every channel. This is a reasonable aim if a channel is to never rebalance itself; the experiments of the aforementioned papers are conducted in such a regime. In this work, however, we allow the channels to rebalance themselves a few times in order to settle on a detailed balance flow. This is because our focus is on the long-term steady state performance of the protocol. This difference in perspective also shows up in how the price depends on the channel imbalance. \cite{lin2020funds} and \cite{wang2024fence} advocate for strictly convex prices whereas this work and \cite{varma2021throughput} propose linear prices.

\paragraph{On Rebalancing} 
Recall that the DEBT control protocol ensures that the flows in the network converge to a detailed balance flow, which can be sustained perpetually without any rebalancing. However, during the transient phase (before convergence), channels may have to perform on-chain rebalancing a few times. Since rebalancing is an expensive operation, it is worthwhile discussing methods by which channels can reduce the extent of rebalancing. One option for the channels to reduce the extent of rebalancing is to increase their capacity; however, this comes at the cost of locking in more capital. Each channel can decide for itself the optimum amount of capital to lock in. Another option, which we discuss in Section \ref{sec:five_node}, is for channels to increase the rate $\gamma$ at which they adjust prices. 

Ultimately, whether or not it is beneficial for a channel to rebalance depends on the time-horizon under consideration. Our protocol is based on the assumption that the demand remains steady for a long period of time. If this is indeed the case, it would be worthwhile for a channel to rebalance itself as it can make up this cost through the incentive fees gained from the flow of transactions through it in steady state. If a channel chooses not to rebalance itself, however, there is a risk of being trapped in a deadlock, which is suboptimal for not only the nodes but also the channel.

\section{Conclusion}
This work presents DEBT control: a protocol for payment channel networks that uses source routing and flow control based on channel prices. The protocol is derived by posing a network utility maximization problem and analyzing its dual minimization. It is shown that under steady demands, the protocol guides the network to an optimal, sustainable point. Simulations show its robustness to demand variations. The work demonstrates that simple protocols with strong theoretical guarantees are possible for PCNs and we hope it inspires further theoretical research in this direction. 


\section*{Acknowledgments}\label{sec:Acknowledgments}
The work of O. Ben-Porat was supported by the Israel Science Foundation (ISF; Grant No. 3079/24). 

\bibliographystyle{plainnat}
% This must be in the first 5 lines to tell arXiv to use pdfLaTeX, which is strongly recommended.
\pdfoutput=1
% In particular, the hyperref package requires pdfLaTeX in order to break URLs across lines.

\documentclass[11pt]{article}

% Change "review" to "final" to generate the final (sometimes called camera-ready) version.
% Change to "preprint" to generate a non-anonymous version with page numbers.
\usepackage{acl}

% Standard package includes
\usepackage{times}
\usepackage{latexsym}

% Draw tables
\usepackage{booktabs}
\usepackage{multirow}
\usepackage{xcolor}
\usepackage{colortbl}
\usepackage{array} 
\usepackage{amsmath}

\newcolumntype{C}{>{\centering\arraybackslash}p{0.07\textwidth}}
% For proper rendering and hyphenation of words containing Latin characters (including in bib files)
\usepackage[T1]{fontenc}
% For Vietnamese characters
% \usepackage[T5]{fontenc}
% See https://www.latex-project.org/help/documentation/encguide.pdf for other character sets
% This assumes your files are encoded as UTF8
\usepackage[utf8]{inputenc}

% This is not strictly necessary, and may be commented out,
% but it will improve the layout of the manuscript,
% and will typically save some space.
\usepackage{microtype}
\DeclareMathOperator*{\argmax}{arg\,max}
% This is also not strictly necessary, and may be commented out.
% However, it will improve the aesthetics of text in
% the typewriter font.
\usepackage{inconsolata}

%Including images in your LaTeX document requires adding
%additional package(s)
\usepackage{graphicx}
% If the title and author information does not fit in the area allocated, uncomment the following
%
%\setlength\titlebox{<dim>}
%
% and set <dim> to something 5cm or larger.

\title{Wi-Chat: Large Language Model Powered Wi-Fi Sensing}

% Author information can be set in various styles:
% For several authors from the same institution:
% \author{Author 1 \and ... \and Author n \\
%         Address line \\ ... \\ Address line}
% if the names do not fit well on one line use
%         Author 1 \\ {\bf Author 2} \\ ... \\ {\bf Author n} \\
% For authors from different institutions:
% \author{Author 1 \\ Address line \\  ... \\ Address line
%         \And  ... \And
%         Author n \\ Address line \\ ... \\ Address line}
% To start a separate ``row'' of authors use \AND, as in
% \author{Author 1 \\ Address line \\  ... \\ Address line
%         \AND
%         Author 2 \\ Address line \\ ... \\ Address line \And
%         Author 3 \\ Address line \\ ... \\ Address line}

% \author{First Author \\
%   Affiliation / Address line 1 \\
%   Affiliation / Address line 2 \\
%   Affiliation / Address line 3 \\
%   \texttt{email@domain} \\\And
%   Second Author \\
%   Affiliation / Address line 1 \\
%   Affiliation / Address line 2 \\
%   Affiliation / Address line 3 \\
%   \texttt{email@domain} \\}
% \author{Haohan Yuan \qquad Haopeng Zhang\thanks{corresponding author} \\ 
%   ALOHA Lab, University of Hawaii at Manoa \\
%   % Affiliation / Address line 2 \\
%   % Affiliation / Address line 3 \\
%   \texttt{\{haohany,haopengz\}@hawaii.edu}}
  
\author{
{Haopeng Zhang$\dag$\thanks{These authors contributed equally to this work.}, Yili Ren$\ddagger$\footnotemark[1], Haohan Yuan$\dag$, Jingzhe Zhang$\ddagger$, Yitong Shen$\ddagger$} \\
ALOHA Lab, University of Hawaii at Manoa$\dag$, University of South Florida$\ddagger$ \\
\{haopengz, haohany\}@hawaii.edu\\
\{yiliren, jingzhe, shen202\}@usf.edu\\}



  
%\author{
%  \textbf{First Author\textsuperscript{1}},
%  \textbf{Second Author\textsuperscript{1,2}},
%  \textbf{Third T. Author\textsuperscript{1}},
%  \textbf{Fourth Author\textsuperscript{1}},
%\\
%  \textbf{Fifth Author\textsuperscript{1,2}},
%  \textbf{Sixth Author\textsuperscript{1}},
%  \textbf{Seventh Author\textsuperscript{1}},
%  \textbf{Eighth Author \textsuperscript{1,2,3,4}},
%\\
%  \textbf{Ninth Author\textsuperscript{1}},
%  \textbf{Tenth Author\textsuperscript{1}},
%  \textbf{Eleventh E. Author\textsuperscript{1,2,3,4,5}},
%  \textbf{Twelfth Author\textsuperscript{1}},
%\\
%  \textbf{Thirteenth Author\textsuperscript{3}},
%  \textbf{Fourteenth F. Author\textsuperscript{2,4}},
%  \textbf{Fifteenth Author\textsuperscript{1}},
%  \textbf{Sixteenth Author\textsuperscript{1}},
%\\
%  \textbf{Seventeenth S. Author\textsuperscript{4,5}},
%  \textbf{Eighteenth Author\textsuperscript{3,4}},
%  \textbf{Nineteenth N. Author\textsuperscript{2,5}},
%  \textbf{Twentieth Author\textsuperscript{1}}
%\\
%\\
%  \textsuperscript{1}Affiliation 1,
%  \textsuperscript{2}Affiliation 2,
%  \textsuperscript{3}Affiliation 3,
%  \textsuperscript{4}Affiliation 4,
%  \textsuperscript{5}Affiliation 5
%\\
%  \small{
%    \textbf{Correspondence:} \href{mailto:email@domain}{email@domain}
%  }
%}

\begin{document}
\maketitle
\begin{abstract}
Recent advancements in Large Language Models (LLMs) have demonstrated remarkable capabilities across diverse tasks. However, their potential to integrate physical model knowledge for real-world signal interpretation remains largely unexplored. In this work, we introduce Wi-Chat, the first LLM-powered Wi-Fi-based human activity recognition system. We demonstrate that LLMs can process raw Wi-Fi signals and infer human activities by incorporating Wi-Fi sensing principles into prompts. Our approach leverages physical model insights to guide LLMs in interpreting Channel State Information (CSI) data without traditional signal processing techniques. Through experiments on real-world Wi-Fi datasets, we show that LLMs exhibit strong reasoning capabilities, achieving zero-shot activity recognition. These findings highlight a new paradigm for Wi-Fi sensing, expanding LLM applications beyond conventional language tasks and enhancing the accessibility of wireless sensing for real-world deployments.
\end{abstract}

\section{Introduction}

In today’s rapidly evolving digital landscape, the transformative power of web technologies has redefined not only how services are delivered but also how complex tasks are approached. Web-based systems have become increasingly prevalent in risk control across various domains. This widespread adoption is due their accessibility, scalability, and ability to remotely connect various types of users. For example, these systems are used for process safety management in industry~\cite{kannan2016web}, safety risk early warning in urban construction~\cite{ding2013development}, and safe monitoring of infrastructural systems~\cite{repetto2018web}. Within these web-based risk management systems, the source search problem presents a huge challenge. Source search refers to the task of identifying the origin of a risky event, such as a gas leak and the emission point of toxic substances. This source search capability is crucial for effective risk management and decision-making.

Traditional approaches to implementing source search capabilities into the web systems often rely on solely algorithmic solutions~\cite{ristic2016study}. These methods, while relatively straightforward to implement, often struggle to achieve acceptable performances due to algorithmic local optima and complex unknown environments~\cite{zhao2020searching}. More recently, web crowdsourcing has emerged as a promising alternative for tackling the source search problem by incorporating human efforts in these web systems on-the-fly~\cite{zhao2024user}. This approach outsources the task of addressing issues encountered during the source search process to human workers, leveraging their capabilities to enhance system performance.

These solutions often employ a human-AI collaborative way~\cite{zhao2023leveraging} where algorithms handle exploration-exploitation and report the encountered problems while human workers resolve complex decision-making bottlenecks to help the algorithms getting rid of local deadlocks~\cite{zhao2022crowd}. Although effective, this paradigm suffers from two inherent limitations: increased operational costs from continuous human intervention, and slow response times of human workers due to sequential decision-making. These challenges motivate our investigation into developing autonomous systems that preserve human-like reasoning capabilities while reducing dependency on massive crowdsourced labor.

Furthermore, recent advancements in large language models (LLMs)~\cite{chang2024survey} and multi-modal LLMs (MLLMs)~\cite{huang2023chatgpt} have unveiled promising avenues for addressing these challenges. One clear opportunity involves the seamless integration of visual understanding and linguistic reasoning for robust decision-making in search tasks. However, whether large models-assisted source search is really effective and efficient for improving the current source search algorithms~\cite{ji2022source} remains unknown. \textit{To address the research gap, we are particularly interested in answering the following two research questions in this work:}

\textbf{\textit{RQ1: }}How can source search capabilities be integrated into web-based systems to support decision-making in time-sensitive risk management scenarios? 
% \sq{I mention ``time-sensitive'' here because I feel like we shall say something about the response time -- LLM has to be faster than humans}

\textbf{\textit{RQ2: }}How can MLLMs and LLMs enhance the effectiveness and efficiency of existing source search algorithms? 

% \textit{\textbf{RQ2:}} To what extent does the performance of large models-assisted search align with or approach the effectiveness of human-AI collaborative search? 

To answer the research questions, we propose a novel framework called Auto-\
S$^2$earch (\textbf{Auto}nomous \textbf{S}ource \textbf{Search}) and implement a prototype system that leverages advanced web technologies to simulate real-world conditions for zero-shot source search. Unlike traditional methods that rely on pre-defined heuristics or extensive human intervention, AutoS$^2$earch employs a carefully designed prompt that encapsulates human rationales, thereby guiding the MLLM to generate coherent and accurate scene descriptions from visual inputs about four directional choices. Based on these language-based descriptions, the LLM is enabled to determine the optimal directional choice through chain-of-thought (CoT) reasoning. Comprehensive empirical validation demonstrates that AutoS$^2$-\ 
earch achieves a success rate of 95–98\%, closely approaching the performance of human-AI collaborative search across 20 benchmark scenarios~\cite{zhao2023leveraging}. 

Our work indicates that the role of humans in future web crowdsourcing tasks may evolve from executors to validators or supervisors. Furthermore, incorporating explanations of LLM decisions into web-based system interfaces has the potential to help humans enhance task performance in risk control.






\section{Related Work}
\label{sec:relatedworks}

% \begin{table*}[t]
% \centering 
% \renewcommand\arraystretch{0.98}
% \fontsize{8}{10}\selectfont \setlength{\tabcolsep}{0.4em}
% \begin{tabular}{@{}lc|cc|cc|cc@{}}
% \toprule
% \textbf{Methods}           & \begin{tabular}[c]{@{}c@{}}\textbf{Training}\\ \textbf{Paradigm}\end{tabular} & \begin{tabular}[c]{@{}c@{}}\textbf{$\#$ PT Data}\\ \textbf{(Tokens)}\end{tabular} & \begin{tabular}[c]{@{}c@{}}\textbf{$\#$ IFT Data}\\ \textbf{(Samples)}\end{tabular} & \textbf{Code}  & \begin{tabular}[c]{@{}c@{}}\textbf{Natural}\\ \textbf{Language}\end{tabular} & \begin{tabular}[c]{@{}c@{}}\textbf{Action}\\ \textbf{Trajectories}\end{tabular} & \begin{tabular}[c]{@{}c@{}}\textbf{API}\\ \textbf{Documentation}\end{tabular}\\ \midrule 
% NexusRaven~\citep{srinivasan2023nexusraven} & IFT & - & - & \textcolor{green}{\CheckmarkBold} & \textcolor{green}{\CheckmarkBold} &\textcolor{red}{\XSolidBrush}&\textcolor{red}{\XSolidBrush}\\
% AgentInstruct~\citep{zeng2023agenttuning} & IFT & - & 2k & \textcolor{green}{\CheckmarkBold} & \textcolor{green}{\CheckmarkBold} &\textcolor{red}{\XSolidBrush}&\textcolor{red}{\XSolidBrush} \\
% AgentEvol~\citep{xi2024agentgym} & IFT & - & 14.5k & \textcolor{green}{\CheckmarkBold} & \textcolor{green}{\CheckmarkBold} &\textcolor{green}{\CheckmarkBold}&\textcolor{red}{\XSolidBrush} \\
% Gorilla~\citep{patil2023gorilla}& IFT & - & 16k & \textcolor{green}{\CheckmarkBold} & \textcolor{green}{\CheckmarkBold} &\textcolor{red}{\XSolidBrush}&\textcolor{green}{\CheckmarkBold}\\
% OpenFunctions-v2~\citep{patil2023gorilla} & IFT & - & 65k & \textcolor{green}{\CheckmarkBold} & \textcolor{green}{\CheckmarkBold} &\textcolor{red}{\XSolidBrush}&\textcolor{green}{\CheckmarkBold}\\
% LAM~\citep{zhang2024agentohana} & IFT & - & 42.6k & \textcolor{green}{\CheckmarkBold} & \textcolor{green}{\CheckmarkBold} &\textcolor{green}{\CheckmarkBold}&\textcolor{red}{\XSolidBrush} \\
% xLAM~\citep{liu2024apigen} & IFT & - & 60k & \textcolor{green}{\CheckmarkBold} & \textcolor{green}{\CheckmarkBold} &\textcolor{green}{\CheckmarkBold}&\textcolor{red}{\XSolidBrush} \\\midrule
% LEMUR~\citep{xu2024lemur} & PT & 90B & 300k & \textcolor{green}{\CheckmarkBold} & \textcolor{green}{\CheckmarkBold} &\textcolor{green}{\CheckmarkBold}&\textcolor{red}{\XSolidBrush}\\
% \rowcolor{teal!12} \method & PT & 103B & 95k & \textcolor{green}{\CheckmarkBold} & \textcolor{green}{\CheckmarkBold} & \textcolor{green}{\CheckmarkBold} & \textcolor{green}{\CheckmarkBold} \\
% \bottomrule
% \end{tabular}
% \caption{Summary of existing tuning- and pretraining-based LLM agents with their training sample sizes. "PT" and "IFT" denote "Pre-Training" and "Instruction Fine-Tuning", respectively. }
% \label{tab:related}
% \end{table*}

\begin{table*}[ht]
\begin{threeparttable}
\centering 
\renewcommand\arraystretch{0.98}
\fontsize{7}{9}\selectfont \setlength{\tabcolsep}{0.2em}
\begin{tabular}{@{}l|c|c|ccc|cc|cc|cccc@{}}
\toprule
\textbf{Methods} & \textbf{Datasets}           & \begin{tabular}[c]{@{}c@{}}\textbf{Training}\\ \textbf{Paradigm}\end{tabular} & \begin{tabular}[c]{@{}c@{}}\textbf{\# PT Data}\\ \textbf{(Tokens)}\end{tabular} & \begin{tabular}[c]{@{}c@{}}\textbf{\# IFT Data}\\ \textbf{(Samples)}\end{tabular} & \textbf{\# APIs} & \textbf{Code}  & \begin{tabular}[c]{@{}c@{}}\textbf{Nat.}\\ \textbf{Lang.}\end{tabular} & \begin{tabular}[c]{@{}c@{}}\textbf{Action}\\ \textbf{Traj.}\end{tabular} & \begin{tabular}[c]{@{}c@{}}\textbf{API}\\ \textbf{Doc.}\end{tabular} & \begin{tabular}[c]{@{}c@{}}\textbf{Func.}\\ \textbf{Call}\end{tabular} & \begin{tabular}[c]{@{}c@{}}\textbf{Multi.}\\ \textbf{Step}\end{tabular}  & \begin{tabular}[c]{@{}c@{}}\textbf{Plan}\\ \textbf{Refine}\end{tabular}  & \begin{tabular}[c]{@{}c@{}}\textbf{Multi.}\\ \textbf{Turn}\end{tabular}\\ \midrule 
\multicolumn{13}{l}{\emph{Instruction Finetuning-based LLM Agents for Intrinsic Reasoning}}  \\ \midrule
FireAct~\cite{chen2023fireact} & FireAct & IFT & - & 2.1K & 10 & \textcolor{red}{\XSolidBrush} &\textcolor{green}{\CheckmarkBold} &\textcolor{green}{\CheckmarkBold}  & \textcolor{red}{\XSolidBrush} &\textcolor{green}{\CheckmarkBold} & \textcolor{red}{\XSolidBrush} &\textcolor{green}{\CheckmarkBold} & \textcolor{red}{\XSolidBrush} \\
ToolAlpaca~\cite{tang2023toolalpaca} & ToolAlpaca & IFT & - & 4.0K & 400 & \textcolor{red}{\XSolidBrush} &\textcolor{green}{\CheckmarkBold} &\textcolor{green}{\CheckmarkBold} & \textcolor{red}{\XSolidBrush} &\textcolor{green}{\CheckmarkBold} & \textcolor{red}{\XSolidBrush}  &\textcolor{green}{\CheckmarkBold} & \textcolor{red}{\XSolidBrush}  \\
ToolLLaMA~\cite{qin2023toolllm} & ToolBench & IFT & - & 12.7K & 16,464 & \textcolor{red}{\XSolidBrush} &\textcolor{green}{\CheckmarkBold} &\textcolor{green}{\CheckmarkBold} &\textcolor{red}{\XSolidBrush} &\textcolor{green}{\CheckmarkBold}&\textcolor{green}{\CheckmarkBold}&\textcolor{green}{\CheckmarkBold} &\textcolor{green}{\CheckmarkBold}\\
AgentEvol~\citep{xi2024agentgym} & AgentTraj-L & IFT & - & 14.5K & 24 &\textcolor{red}{\XSolidBrush} & \textcolor{green}{\CheckmarkBold} &\textcolor{green}{\CheckmarkBold}&\textcolor{red}{\XSolidBrush} &\textcolor{green}{\CheckmarkBold}&\textcolor{red}{\XSolidBrush} &\textcolor{red}{\XSolidBrush} &\textcolor{green}{\CheckmarkBold}\\
Lumos~\cite{yin2024agent} & Lumos & IFT  & - & 20.0K & 16 &\textcolor{red}{\XSolidBrush} & \textcolor{green}{\CheckmarkBold} & \textcolor{green}{\CheckmarkBold} &\textcolor{red}{\XSolidBrush} & \textcolor{green}{\CheckmarkBold} & \textcolor{green}{\CheckmarkBold} &\textcolor{red}{\XSolidBrush} & \textcolor{green}{\CheckmarkBold}\\
Agent-FLAN~\cite{chen2024agent} & Agent-FLAN & IFT & - & 24.7K & 20 &\textcolor{red}{\XSolidBrush} & \textcolor{green}{\CheckmarkBold} & \textcolor{green}{\CheckmarkBold} &\textcolor{red}{\XSolidBrush} & \textcolor{green}{\CheckmarkBold}& \textcolor{green}{\CheckmarkBold}&\textcolor{red}{\XSolidBrush} & \textcolor{green}{\CheckmarkBold}\\
AgentTuning~\citep{zeng2023agenttuning} & AgentInstruct & IFT & - & 35.0K & - &\textcolor{red}{\XSolidBrush} & \textcolor{green}{\CheckmarkBold} & \textcolor{green}{\CheckmarkBold} &\textcolor{red}{\XSolidBrush} & \textcolor{green}{\CheckmarkBold} &\textcolor{red}{\XSolidBrush} &\textcolor{red}{\XSolidBrush} & \textcolor{green}{\CheckmarkBold}\\\midrule
\multicolumn{13}{l}{\emph{Instruction Finetuning-based LLM Agents for Function Calling}} \\\midrule
NexusRaven~\citep{srinivasan2023nexusraven} & NexusRaven & IFT & - & - & 116 & \textcolor{green}{\CheckmarkBold} & \textcolor{green}{\CheckmarkBold}  & \textcolor{green}{\CheckmarkBold} &\textcolor{red}{\XSolidBrush} & \textcolor{green}{\CheckmarkBold} &\textcolor{red}{\XSolidBrush} &\textcolor{red}{\XSolidBrush}&\textcolor{red}{\XSolidBrush}\\
Gorilla~\citep{patil2023gorilla} & Gorilla & IFT & - & 16.0K & 1,645 & \textcolor{green}{\CheckmarkBold} &\textcolor{red}{\XSolidBrush} &\textcolor{red}{\XSolidBrush}&\textcolor{green}{\CheckmarkBold} &\textcolor{green}{\CheckmarkBold} &\textcolor{red}{\XSolidBrush} &\textcolor{red}{\XSolidBrush} &\textcolor{red}{\XSolidBrush}\\
OpenFunctions-v2~\citep{patil2023gorilla} & OpenFunctions-v2 & IFT & - & 65.0K & - & \textcolor{green}{\CheckmarkBold} & \textcolor{green}{\CheckmarkBold} &\textcolor{red}{\XSolidBrush} &\textcolor{green}{\CheckmarkBold} &\textcolor{green}{\CheckmarkBold} &\textcolor{red}{\XSolidBrush} &\textcolor{red}{\XSolidBrush} &\textcolor{red}{\XSolidBrush}\\
API Pack~\cite{guo2024api} & API Pack & IFT & - & 1.1M & 11,213 &\textcolor{green}{\CheckmarkBold} &\textcolor{red}{\XSolidBrush} &\textcolor{green}{\CheckmarkBold} &\textcolor{red}{\XSolidBrush} &\textcolor{green}{\CheckmarkBold} &\textcolor{red}{\XSolidBrush}&\textcolor{red}{\XSolidBrush}&\textcolor{red}{\XSolidBrush}\\ 
LAM~\citep{zhang2024agentohana} & AgentOhana & IFT & - & 42.6K & - & \textcolor{green}{\CheckmarkBold} & \textcolor{green}{\CheckmarkBold} &\textcolor{green}{\CheckmarkBold}&\textcolor{red}{\XSolidBrush} &\textcolor{green}{\CheckmarkBold}&\textcolor{red}{\XSolidBrush}&\textcolor{green}{\CheckmarkBold}&\textcolor{green}{\CheckmarkBold}\\
xLAM~\citep{liu2024apigen} & APIGen & IFT & - & 60.0K & 3,673 & \textcolor{green}{\CheckmarkBold} & \textcolor{green}{\CheckmarkBold} &\textcolor{green}{\CheckmarkBold}&\textcolor{red}{\XSolidBrush} &\textcolor{green}{\CheckmarkBold}&\textcolor{red}{\XSolidBrush}&\textcolor{green}{\CheckmarkBold}&\textcolor{green}{\CheckmarkBold}\\\midrule
\multicolumn{13}{l}{\emph{Pretraining-based LLM Agents}}  \\\midrule
% LEMUR~\citep{xu2024lemur} & PT & 90B & 300.0K & - & \textcolor{green}{\CheckmarkBold} & \textcolor{green}{\CheckmarkBold} &\textcolor{green}{\CheckmarkBold}&\textcolor{red}{\XSolidBrush} & \textcolor{red}{\XSolidBrush} &\textcolor{green}{\CheckmarkBold} &\textcolor{red}{\XSolidBrush}&\textcolor{red}{\XSolidBrush}\\
\rowcolor{teal!12} \method & \dataset & PT & 103B & 95.0K  & 76,537  & \textcolor{green}{\CheckmarkBold} & \textcolor{green}{\CheckmarkBold} & \textcolor{green}{\CheckmarkBold} & \textcolor{green}{\CheckmarkBold} & \textcolor{green}{\CheckmarkBold} & \textcolor{green}{\CheckmarkBold} & \textcolor{green}{\CheckmarkBold} & \textcolor{green}{\CheckmarkBold}\\
\bottomrule
\end{tabular}
% \begin{tablenotes}
%     \item $^*$ In addition, the StarCoder-API can offer 4.77M more APIs.
% \end{tablenotes}
\caption{Summary of existing instruction finetuning-based LLM agents for intrinsic reasoning and function calling, along with their training resources and sample sizes. "PT" and "IFT" denote "Pre-Training" and "Instruction Fine-Tuning", respectively.}
\vspace{-2ex}
\label{tab:related}
\end{threeparttable}
\end{table*}

\noindent \textbf{Prompting-based LLM Agents.} Due to the lack of agent-specific pre-training corpus, existing LLM agents rely on either prompt engineering~\cite{hsieh2023tool,lu2024chameleon,yao2022react,wang2023voyager} or instruction fine-tuning~\cite{chen2023fireact,zeng2023agenttuning} to understand human instructions, decompose high-level tasks, generate grounded plans, and execute multi-step actions. 
However, prompting-based methods mainly depend on the capabilities of backbone LLMs (usually commercial LLMs), failing to introduce new knowledge and struggling to generalize to unseen tasks~\cite{sun2024adaplanner,zhuang2023toolchain}. 

\noindent \textbf{Instruction Finetuning-based LLM Agents.} Considering the extensive diversity of APIs and the complexity of multi-tool instructions, tool learning inherently presents greater challenges than natural language tasks, such as text generation~\cite{qin2023toolllm}.
Post-training techniques focus more on instruction following and aligning output with specific formats~\cite{patil2023gorilla,hao2024toolkengpt,qin2023toolllm,schick2024toolformer}, rather than fundamentally improving model knowledge or capabilities. 
Moreover, heavy fine-tuning can hinder generalization or even degrade performance in non-agent use cases, potentially suppressing the original base model capabilities~\cite{ghosh2024a}.

\noindent \textbf{Pretraining-based LLM Agents.} While pre-training serves as an essential alternative, prior works~\cite{nijkamp2023codegen,roziere2023code,xu2024lemur,patil2023gorilla} have primarily focused on improving task-specific capabilities (\eg, code generation) instead of general-domain LLM agents, due to single-source, uni-type, small-scale, and poor-quality pre-training data. 
Existing tool documentation data for agent training either lacks diverse real-world APIs~\cite{patil2023gorilla, tang2023toolalpaca} or is constrained to single-tool or single-round tool execution. 
Furthermore, trajectory data mostly imitate expert behavior or follow function-calling rules with inferior planning and reasoning, failing to fully elicit LLMs' capabilities and handle complex instructions~\cite{qin2023toolllm}. 
Given a wide range of candidate API functions, each comprising various function names and parameters available at every planning step, identifying globally optimal solutions and generalizing across tasks remains highly challenging.



\section{Preliminaries}
\label{Preliminaries}
\begin{figure*}[t]
    \centering
    \includegraphics[width=0.95\linewidth]{fig/HealthGPT_Framework.png}
    \caption{The \ourmethod{} architecture integrates hierarchical visual perception and H-LoRA, employing a task-specific hard router to select visual features and H-LoRA plugins, ultimately generating outputs with an autoregressive manner.}
    \label{fig:architecture}
\end{figure*}
\noindent\textbf{Large Vision-Language Models.} 
The input to a LVLM typically consists of an image $x^{\text{img}}$ and a discrete text sequence $x^{\text{txt}}$. The visual encoder $\mathcal{E}^{\text{img}}$ converts the input image $x^{\text{img}}$ into a sequence of visual tokens $\mathcal{V} = [v_i]_{i=1}^{N_v}$, while the text sequence $x^{\text{txt}}$ is mapped into a sequence of text tokens $\mathcal{T} = [t_i]_{i=1}^{N_t}$ using an embedding function $\mathcal{E}^{\text{txt}}$. The LLM $\mathcal{M_\text{LLM}}(\cdot|\theta)$ models the joint probability of the token sequence $\mathcal{U} = \{\mathcal{V},\mathcal{T}\}$, which is expressed as:
\begin{equation}
    P_\theta(R | \mathcal{U}) = \prod_{i=1}^{N_r} P_\theta(r_i | \{\mathcal{U}, r_{<i}\}),
\end{equation}
where $R = [r_i]_{i=1}^{N_r}$ is the text response sequence. The LVLM iteratively generates the next token $r_i$ based on $r_{<i}$. The optimization objective is to minimize the cross-entropy loss of the response $\mathcal{R}$.
% \begin{equation}
%     \mathcal{L}_{\text{VLM}} = \mathbb{E}_{R|\mathcal{U}}\left[-\log P_\theta(R | \mathcal{U})\right]
% \end{equation}
It is worth noting that most LVLMs adopt a design paradigm based on ViT, alignment adapters, and pre-trained LLMs\cite{liu2023llava,liu2024improved}, enabling quick adaptation to downstream tasks.


\noindent\textbf{VQGAN.}
VQGAN~\cite{esser2021taming} employs latent space compression and indexing mechanisms to effectively learn a complete discrete representation of images. VQGAN first maps the input image $x^{\text{img}}$ to a latent representation $z = \mathcal{E}(x)$ through a encoder $\mathcal{E}$. Then, the latent representation is quantized using a codebook $\mathcal{Z} = \{z_k\}_{k=1}^K$, generating a discrete index sequence $\mathcal{I} = [i_m]_{m=1}^N$, where $i_m \in \mathcal{Z}$ represents the quantized code index:
\begin{equation}
    \mathcal{I} = \text{Quantize}(z|\mathcal{Z}) = \arg\min_{z_k \in \mathcal{Z}} \| z - z_k \|_2.
\end{equation}
In our approach, the discrete index sequence $\mathcal{I}$ serves as a supervisory signal for the generation task, enabling the model to predict the index sequence $\hat{\mathcal{I}}$ from input conditions such as text or other modality signals.  
Finally, the predicted index sequence $\hat{\mathcal{I}}$ is upsampled by the VQGAN decoder $G$, generating the high-quality image $\hat{x}^\text{img} = G(\hat{\mathcal{I}})$.



\noindent\textbf{Low Rank Adaptation.} 
LoRA\cite{hu2021lora} effectively captures the characteristics of downstream tasks by introducing low-rank adapters. The core idea is to decompose the bypass weight matrix $\Delta W\in\mathbb{R}^{d^{\text{in}} \times d^{\text{out}}}$ into two low-rank matrices $ \{A \in \mathbb{R}^{d^{\text{in}} \times r}, B \in \mathbb{R}^{r \times d^{\text{out}}} \}$, where $ r \ll \min\{d^{\text{in}}, d^{\text{out}}\} $, significantly reducing learnable parameters. The output with the LoRA adapter for the input $x$ is then given by:
\begin{equation}
    h = x W_0 + \alpha x \Delta W/r = x W_0 + \alpha xAB/r,
\end{equation}
where matrix $ A $ is initialized with a Gaussian distribution, while the matrix $ B $ is initialized as a zero matrix. The scaling factor $ \alpha/r $ controls the impact of $ \Delta W $ on the model.

\section{HealthGPT}
\label{Method}


\subsection{Unified Autoregressive Generation.}  
% As shown in Figure~\ref{fig:architecture}, 
\ourmethod{} (Figure~\ref{fig:architecture}) utilizes a discrete token representation that covers both text and visual outputs, unifying visual comprehension and generation as an autoregressive task. 
For comprehension, $\mathcal{M}_\text{llm}$ receives the input joint sequence $\mathcal{U}$ and outputs a series of text token $\mathcal{R} = [r_1, r_2, \dots, r_{N_r}]$, where $r_i \in \mathcal{V}_{\text{txt}}$, and $\mathcal{V}_{\text{txt}}$ represents the LLM's vocabulary:
\begin{equation}
    P_\theta(\mathcal{R} \mid \mathcal{U}) = \prod_{i=1}^{N_r} P_\theta(r_i \mid \mathcal{U}, r_{<i}).
\end{equation}
For generation, $\mathcal{M}_\text{llm}$ first receives a special start token $\langle \text{START\_IMG} \rangle$, then generates a series of tokens corresponding to the VQGAN indices $\mathcal{I} = [i_1, i_2, \dots, i_{N_i}]$, where $i_j \in \mathcal{V}_{\text{vq}}$, and $\mathcal{V}_{\text{vq}}$ represents the index range of VQGAN. Upon completion of generation, the LLM outputs an end token $\langle \text{END\_IMG} \rangle$:
\begin{equation}
    P_\theta(\mathcal{I} \mid \mathcal{U}) = \prod_{j=1}^{N_i} P_\theta(i_j \mid \mathcal{U}, i_{<j}).
\end{equation}
Finally, the generated index sequence $\mathcal{I}$ is fed into the decoder $G$, which reconstructs the target image $\hat{x}^{\text{img}} = G(\mathcal{I})$.

\subsection{Hierarchical Visual Perception}  
Given the differences in visual perception between comprehension and generation tasks—where the former focuses on abstract semantics and the latter emphasizes complete semantics—we employ ViT to compress the image into discrete visual tokens at multiple hierarchical levels.
Specifically, the image is converted into a series of features $\{f_1, f_2, \dots, f_L\}$ as it passes through $L$ ViT blocks.

To address the needs of various tasks, the hidden states are divided into two types: (i) \textit{Concrete-grained features} $\mathcal{F}^{\text{Con}} = \{f_1, f_2, \dots, f_k\}, k < L$, derived from the shallower layers of ViT, containing sufficient global features, suitable for generation tasks; 
(ii) \textit{Abstract-grained features} $\mathcal{F}^{\text{Abs}} = \{f_{k+1}, f_{k+2}, \dots, f_L\}$, derived from the deeper layers of ViT, which contain abstract semantic information closer to the text space, suitable for comprehension tasks.

The task type $T$ (comprehension or generation) determines which set of features is selected as the input for the downstream large language model:
\begin{equation}
    \mathcal{F}^{\text{img}}_T =
    \begin{cases}
        \mathcal{F}^{\text{Con}}, & \text{if } T = \text{generation task} \\
        \mathcal{F}^{\text{Abs}}, & \text{if } T = \text{comprehension task}
    \end{cases}
\end{equation}
We integrate the image features $\mathcal{F}^{\text{img}}_T$ and text features $\mathcal{T}$ into a joint sequence through simple concatenation, which is then fed into the LLM $\mathcal{M}_{\text{llm}}$ for autoregressive generation.
% :
% \begin{equation}
%     \mathcal{R} = \mathcal{M}_{\text{llm}}(\mathcal{U}|\theta), \quad \mathcal{U} = [\mathcal{F}^{\text{img}}_T; \mathcal{T}]
% \end{equation}
\subsection{Heterogeneous Knowledge Adaptation}
We devise H-LoRA, which stores heterogeneous knowledge from comprehension and generation tasks in separate modules and dynamically routes to extract task-relevant knowledge from these modules. 
At the task level, for each task type $ T $, we dynamically assign a dedicated H-LoRA submodule $ \theta^T $, which is expressed as:
\begin{equation}
    \mathcal{R} = \mathcal{M}_\text{LLM}(\mathcal{U}|\theta, \theta^T), \quad \theta^T = \{A^T, B^T, \mathcal{R}^T_\text{outer}\}.
\end{equation}
At the feature level for a single task, H-LoRA integrates the idea of Mixture of Experts (MoE)~\cite{masoudnia2014mixture} and designs an efficient matrix merging and routing weight allocation mechanism, thus avoiding the significant computational delay introduced by matrix splitting in existing MoELoRA~\cite{luo2024moelora}. Specifically, we first merge the low-rank matrices (rank = r) of $ k $ LoRA experts into a unified matrix:
\begin{equation}
    \mathbf{A}^{\text{merged}}, \mathbf{B}^{\text{merged}} = \text{Concat}(\{A_i\}_1^k), \text{Concat}(\{B_i\}_1^k),
\end{equation}
where $ \mathbf{A}^{\text{merged}} \in \mathbb{R}^{d^\text{in} \times rk} $ and $ \mathbf{B}^{\text{merged}} \in \mathbb{R}^{rk \times d^\text{out}} $. The $k$-dimension routing layer generates expert weights $ \mathcal{W} \in \mathbb{R}^{\text{token\_num} \times k} $ based on the input hidden state $ x $, and these are expanded to $ \mathbb{R}^{\text{token\_num} \times rk} $ as follows:
\begin{equation}
    \mathcal{W}^\text{expanded} = \alpha k \mathcal{W} / r \otimes \mathbf{1}_r,
\end{equation}
where $ \otimes $ denotes the replication operation.
The overall output of H-LoRA is computed as:
\begin{equation}
    \mathcal{O}^\text{H-LoRA} = (x \mathbf{A}^{\text{merged}} \odot \mathcal{W}^\text{expanded}) \mathbf{B}^{\text{merged}},
\end{equation}
where $ \odot $ represents element-wise multiplication. Finally, the output of H-LoRA is added to the frozen pre-trained weights to produce the final output:
\begin{equation}
    \mathcal{O} = x W_0 + \mathcal{O}^\text{H-LoRA}.
\end{equation}
% In summary, H-LoRA is a task-based dynamic PEFT method that achieves high efficiency in single-task fine-tuning.

\subsection{Training Pipeline}

\begin{figure}[t]
    \centering
    \hspace{-4mm}
    \includegraphics[width=0.94\linewidth]{fig/data.pdf}
    \caption{Data statistics of \texttt{VL-Health}. }
    \label{fig:data}
\end{figure}
\noindent \textbf{1st Stage: Multi-modal Alignment.} 
In the first stage, we design separate visual adapters and H-LoRA submodules for medical unified tasks. For the medical comprehension task, we train abstract-grained visual adapters using high-quality image-text pairs to align visual embeddings with textual embeddings, thereby enabling the model to accurately describe medical visual content. During this process, the pre-trained LLM and its corresponding H-LoRA submodules remain frozen. In contrast, the medical generation task requires training concrete-grained adapters and H-LoRA submodules while keeping the LLM frozen. Meanwhile, we extend the textual vocabulary to include multimodal tokens, enabling the support of additional VQGAN vector quantization indices. The model trains on image-VQ pairs, endowing the pre-trained LLM with the capability for image reconstruction. This design ensures pixel-level consistency of pre- and post-LVLM. The processes establish the initial alignment between the LLM’s outputs and the visual inputs.

\noindent \textbf{2nd Stage: Heterogeneous H-LoRA Plugin Adaptation.}  
The submodules of H-LoRA share the word embedding layer and output head but may encounter issues such as bias and scale inconsistencies during training across different tasks. To ensure that the multiple H-LoRA plugins seamlessly interface with the LLMs and form a unified base, we fine-tune the word embedding layer and output head using a small amount of mixed data to maintain consistency in the model weights. Specifically, during this stage, all H-LoRA submodules for different tasks are kept frozen, with only the word embedding layer and output head being optimized. Through this stage, the model accumulates foundational knowledge for unified tasks by adapting H-LoRA plugins.

\begin{table*}[!t]
\centering
\caption{Comparison of \ourmethod{} with other LVLMs and unified multi-modal models on medical visual comprehension tasks. \textbf{Bold} and \underline{underlined} text indicates the best performance and second-best performance, respectively.}
\resizebox{\textwidth}{!}{
\begin{tabular}{c|lcc|cccccccc|c}
\toprule
\rowcolor[HTML]{E9F3FE} &  &  &  & \multicolumn{2}{c}{\textbf{VQA-RAD \textuparrow}} & \multicolumn{2}{c}{\textbf{SLAKE \textuparrow}} & \multicolumn{2}{c}{\textbf{PathVQA \textuparrow}} &  &  &  \\ 
\cline{5-10}
\rowcolor[HTML]{E9F3FE}\multirow{-2}{*}{\textbf{Type}} & \multirow{-2}{*}{\textbf{Model}} & \multirow{-2}{*}{\textbf{\# Params}} & \multirow{-2}{*}{\makecell{\textbf{Medical} \\ \textbf{LVLM}}} & \textbf{close} & \textbf{all} & \textbf{close} & \textbf{all} & \textbf{close} & \textbf{all} & \multirow{-2}{*}{\makecell{\textbf{MMMU} \\ \textbf{-Med}}\textuparrow} & \multirow{-2}{*}{\textbf{OMVQA}\textuparrow} & \multirow{-2}{*}{\textbf{Avg. \textuparrow}} \\ 
\midrule \midrule
\multirow{9}{*}{\textbf{Comp. Only}} 
& Med-Flamingo & 8.3B & \Large \ding{51} & 58.6 & 43.0 & 47.0 & 25.5 & 61.9 & 31.3 & 28.7 & 34.9 & 41.4 \\
& LLaVA-Med & 7B & \Large \ding{51} & 60.2 & 48.1 & 58.4 & 44.8 & 62.3 & 35.7 & 30.0 & 41.3 & 47.6 \\
& HuatuoGPT-Vision & 7B & \Large \ding{51} & 66.9 & 53.0 & 59.8 & 49.1 & 52.9 & 32.0 & 42.0 & 50.0 & 50.7 \\
& BLIP-2 & 6.7B & \Large \ding{55} & 43.4 & 36.8 & 41.6 & 35.3 & 48.5 & 28.8 & 27.3 & 26.9 & 36.1 \\
& LLaVA-v1.5 & 7B & \Large \ding{55} & 51.8 & 42.8 & 37.1 & 37.7 & 53.5 & 31.4 & 32.7 & 44.7 & 41.5 \\
& InstructBLIP & 7B & \Large \ding{55} & 61.0 & 44.8 & 66.8 & 43.3 & 56.0 & 32.3 & 25.3 & 29.0 & 44.8 \\
& Yi-VL & 6B & \Large \ding{55} & 52.6 & 42.1 & 52.4 & 38.4 & 54.9 & 30.9 & 38.0 & 50.2 & 44.9 \\
& InternVL2 & 8B & \Large \ding{55} & 64.9 & 49.0 & 66.6 & 50.1 & 60.0 & 31.9 & \underline{43.3} & 54.5 & 52.5\\
& Llama-3.2 & 11B & \Large \ding{55} & 68.9 & 45.5 & 72.4 & 52.1 & 62.8 & 33.6 & 39.3 & 63.2 & 54.7 \\
\midrule
\multirow{5}{*}{\textbf{Comp. \& Gen.}} 
& Show-o & 1.3B & \Large \ding{55} & 50.6 & 33.9 & 31.5 & 17.9 & 52.9 & 28.2 & 22.7 & 45.7 & 42.6 \\
& Unified-IO 2 & 7B & \Large \ding{55} & 46.2 & 32.6 & 35.9 & 21.9 & 52.5 & 27.0 & 25.3 & 33.0 & 33.8 \\
& Janus & 1.3B & \Large \ding{55} & 70.9 & 52.8 & 34.7 & 26.9 & 51.9 & 27.9 & 30.0 & 26.8 & 33.5 \\
& \cellcolor[HTML]{DAE0FB}HealthGPT-M3 & \cellcolor[HTML]{DAE0FB}3.8B & \cellcolor[HTML]{DAE0FB}\Large \ding{51} & \cellcolor[HTML]{DAE0FB}\underline{73.7} & \cellcolor[HTML]{DAE0FB}\underline{55.9} & \cellcolor[HTML]{DAE0FB}\underline{74.6} & \cellcolor[HTML]{DAE0FB}\underline{56.4} & \cellcolor[HTML]{DAE0FB}\underline{78.7} & \cellcolor[HTML]{DAE0FB}\underline{39.7} & \cellcolor[HTML]{DAE0FB}\underline{43.3} & \cellcolor[HTML]{DAE0FB}\underline{68.5} & \cellcolor[HTML]{DAE0FB}\underline{61.3} \\
& \cellcolor[HTML]{DAE0FB}HealthGPT-L14 & \cellcolor[HTML]{DAE0FB}14B & \cellcolor[HTML]{DAE0FB}\Large \ding{51} & \cellcolor[HTML]{DAE0FB}\textbf{77.7} & \cellcolor[HTML]{DAE0FB}\textbf{58.3} & \cellcolor[HTML]{DAE0FB}\textbf{76.4} & \cellcolor[HTML]{DAE0FB}\textbf{64.5} & \cellcolor[HTML]{DAE0FB}\textbf{85.9} & \cellcolor[HTML]{DAE0FB}\textbf{44.4} & \cellcolor[HTML]{DAE0FB}\textbf{49.2} & \cellcolor[HTML]{DAE0FB}\textbf{74.4} & \cellcolor[HTML]{DAE0FB}\textbf{66.4} \\
\bottomrule
\end{tabular}
}
\label{tab:results}
\end{table*}
\begin{table*}[ht]
    \centering
    \caption{The experimental results for the four modality conversion tasks.}
    \resizebox{\textwidth}{!}{
    \begin{tabular}{l|ccc|ccc|ccc|ccc}
        \toprule
        \rowcolor[HTML]{E9F3FE} & \multicolumn{3}{c}{\textbf{CT to MRI (Brain)}} & \multicolumn{3}{c}{\textbf{CT to MRI (Pelvis)}} & \multicolumn{3}{c}{\textbf{MRI to CT (Brain)}} & \multicolumn{3}{c}{\textbf{MRI to CT (Pelvis)}} \\
        \cline{2-13}
        \rowcolor[HTML]{E9F3FE}\multirow{-2}{*}{\textbf{Model}}& \textbf{SSIM $\uparrow$} & \textbf{PSNR $\uparrow$} & \textbf{MSE $\downarrow$} & \textbf{SSIM $\uparrow$} & \textbf{PSNR $\uparrow$} & \textbf{MSE $\downarrow$} & \textbf{SSIM $\uparrow$} & \textbf{PSNR $\uparrow$} & \textbf{MSE $\downarrow$} & \textbf{SSIM $\uparrow$} & \textbf{PSNR $\uparrow$} & \textbf{MSE $\downarrow$} \\
        \midrule \midrule
        pix2pix & 71.09 & 32.65 & 36.85 & 59.17 & 31.02 & 51.91 & 78.79 & 33.85 & 28.33 & 72.31 & 32.98 & 36.19 \\
        CycleGAN & 54.76 & 32.23 & 40.56 & 54.54 & 30.77 & 55.00 & 63.75 & 31.02 & 52.78 & 50.54 & 29.89 & 67.78 \\
        BBDM & {71.69} & {32.91} & {34.44} & 57.37 & 31.37 & 48.06 & \textbf{86.40} & 34.12 & 26.61 & {79.26} & 33.15 & 33.60 \\
        Vmanba & 69.54 & 32.67 & 36.42 & {63.01} & {31.47} & {46.99} & 79.63 & 34.12 & 26.49 & 77.45 & 33.53 & 31.85 \\
        DiffMa & 71.47 & 32.74 & 35.77 & 62.56 & 31.43 & 47.38 & 79.00 & {34.13} & {26.45} & 78.53 & {33.68} & {30.51} \\
        \rowcolor[HTML]{DAE0FB}HealthGPT-M3 & \underline{79.38} & \underline{33.03} & \underline{33.48} & \underline{71.81} & \underline{31.83} & \underline{43.45} & {85.06} & \textbf{34.40} & \textbf{25.49} & \underline{84.23} & \textbf{34.29} & \textbf{27.99} \\
        \rowcolor[HTML]{DAE0FB}HealthGPT-L14 & \textbf{79.73} & \textbf{33.10} & \textbf{32.96} & \textbf{71.92} & \textbf{31.87} & \textbf{43.09} & \underline{85.31} & \underline{34.29} & \underline{26.20} & \textbf{84.96} & \underline{34.14} & \underline{28.13} \\
        \bottomrule
    \end{tabular}
    }
    \label{tab:conversion}
\end{table*}

\noindent \textbf{3rd Stage: Visual Instruction Fine-Tuning.}  
In the third stage, we introduce additional task-specific data to further optimize the model and enhance its adaptability to downstream tasks such as medical visual comprehension (e.g., medical QA, medical dialogues, and report generation) or generation tasks (e.g., super-resolution, denoising, and modality conversion). Notably, by this stage, the word embedding layer and output head have been fine-tuned, only the H-LoRA modules and adapter modules need to be trained. This strategy significantly improves the model's adaptability and flexibility across different tasks.


\section{Experiment}
\label{s:experiment}

\subsection{Data Description}
We evaluate our method on FI~\cite{you2016building}, Twitter\_LDL~\cite{yang2017learning} and Artphoto~\cite{machajdik2010affective}.
FI is a public dataset built from Flickr and Instagram, with 23,308 images and eight emotion categories, namely \textit{amusement}, \textit{anger}, \textit{awe},  \textit{contentment}, \textit{disgust}, \textit{excitement},  \textit{fear}, and \textit{sadness}. 
% Since images in FI are all copyrighted by law, some images are corrupted now, so we remove these samples and retain 21,828 images.
% T4SA contains images from Twitter, which are classified into three categories: \textit{positive}, \textit{neutral}, and \textit{negative}. In this paper, we adopt the base version of B-T4SA, which contains 470,586 images and provides text descriptions of the corresponding tweets.
Twitter\_LDL contains 10,045 images from Twitter, with the same eight categories as the FI dataset.
% 。
For these two datasets, they are randomly split into 80\%
training and 20\% testing set.
Artphoto contains 806 artistic photos from the DeviantArt website, which we use to further evaluate the zero-shot capability of our model.
% on the small-scale dataset.
% We construct and publicly release the first image sentiment analysis dataset containing metadata.
% 。

% Based on these datasets, we are the first to construct and publicly release metadata-enhanced image sentiment analysis datasets. These datasets include scenes, tags, descriptions, and corresponding confidence scores, and are available at this link for future research purposes.


% 
\begin{table}[t]
\centering
% \begin{center}
\caption{Overall performance of different models on FI and Twitter\_LDL datasets.}
\label{tab:cap1}
% \resizebox{\linewidth}{!}
{
\begin{tabular}{l|c|c|c|c}
\hline
\multirow{2}{*}{\textbf{Model}} & \multicolumn{2}{c|}{\textbf{FI}}  & \multicolumn{2}{c}{\textbf{Twitter\_LDL}} \\ \cline{2-5} 
  & \textbf{Accuracy} & \textbf{F1} & \textbf{Accuracy} & \textbf{F1}  \\ \hline
% (\rownumber)~AlexNet~\cite{krizhevsky2017imagenet}  & 58.13\% & 56.35\%  & 56.24\%& 55.02\%  \\ 
% (\rownumber)~VGG16~\cite{simonyan2014very}  & 63.75\%& 63.08\%  & 59.34\%& 59.02\%  \\ 
(\rownumber)~ResNet101~\cite{he2016deep} & 66.16\%& 65.56\%  & 62.02\% & 61.34\%  \\ 
(\rownumber)~CDA~\cite{han2023boosting} & 66.71\%& 65.37\%  & 64.14\% & 62.85\%  \\ 
(\rownumber)~CECCN~\cite{ruan2024color} & 67.96\%& 66.74\%  & 64.59\%& 64.72\% \\ 
(\rownumber)~EmoVIT~\cite{xie2024emovit} & 68.09\%& 67.45\%  & 63.12\% & 61.97\%  \\ 
(\rownumber)~ComLDL~\cite{zhang2022compound} & 68.83\%& 67.28\%  & 65.29\% & 63.12\%  \\ 
(\rownumber)~WSDEN~\cite{li2023weakly} & 69.78\%& 69.61\%  & 67.04\% & 65.49\% \\ 
(\rownumber)~ECWA~\cite{deng2021emotion} & 70.87\%& 69.08\%  & 67.81\% & 66.87\%  \\ 
(\rownumber)~EECon~\cite{yang2023exploiting} & 71.13\%& 68.34\%  & 64.27\%& 63.16\%  \\ 
(\rownumber)~MAM~\cite{zhang2024affective} & 71.44\%  & 70.83\% & 67.18\%  & 65.01\%\\ 
(\rownumber)~TGCA-PVT~\cite{chen2024tgca}   & 73.05\%  & 71.46\% & 69.87\%  & 68.32\% \\ 
(\rownumber)~OEAN~\cite{zhang2024object}   & 73.40\%  & 72.63\% & 70.52\%  & 69.47\% \\ \hline
(\rownumber)~\shortname  & \textbf{79.48\%} & \textbf{79.22\%} & \textbf{74.12\%} & \textbf{73.09\%} \\ \hline
\end{tabular}
}
\vspace{-6mm}
% \end{center}
\end{table}
% 

\subsection{Experiment Setting}
% \subsubsection{Model Setting.}
% 
\textbf{Model Setting:}
For feature representation, we set $k=10$ to select object tags, and adopt clip-vit-base-patch32 as the pre-trained model for unified feature representation.
Moreover, we empirically set $(d_e, d_h, d_k, d_s) = (512, 128, 16, 64)$, and set the classification class $L$ to 8.

% 

\textbf{Training Setting:}
To initialize the model, we set all weights such as $\boldsymbol{W}$ following the truncated normal distribution, and use AdamW optimizer with the learning rate of $1 \times 10^{-4}$.
% warmup scheduler of cosine, warmup steps of 2000.
Furthermore, we set the batch size to 32 and the epoch of the training process to 200.
During the implementation, we utilize \textit{PyTorch} to build our entire model.
% , and our project codes are publicly available at https://github.com/zzmyrep/MESN.
% Our project codes as well as data are all publicly available on GitHub\footnote{https://github.com/zzmyrep/KBCEN}.
% Code is available at \href{https://github.com/zzmyrep/KBCEN}{https://github.com/zzmyrep/KBCEN}.

\textbf{Evaluation Metrics:}
Following~\cite{zhang2024affective, chen2024tgca, zhang2024object}, we adopt \textit{accuracy} and \textit{F1} as our evaluation metrics to measure the performance of different methods for image sentiment analysis. 



\subsection{Experiment Result}
% We compare our model against the following baselines: AlexNet~\cite{krizhevsky2017imagenet}, VGG16~\cite{simonyan2014very}, ResNet101~\cite{he2016deep}, CECCN~\cite{ruan2024color}, EmoVIT~\cite{xie2024emovit}, WSCNet~\cite{yang2018weakly}, ECWA~\cite{deng2021emotion}, EECon~\cite{yang2023exploiting}, MAM~\cite{zhang2024affective} and TGCA-PVT~\cite{chen2024tgca}, and the overall results are summarized in Table~\ref{tab:cap1}.
We compare our model against several baselines, and the overall results are summarized in Table~\ref{tab:cap1}.
We observe that our model achieves the best performance in both accuracy and F1 metrics, significantly outperforming the previous models. 
This superior performance is mainly attributed to our effective utilization of metadata to enhance image sentiment analysis, as well as the exceptional capability of the unified sentiment transformer framework we developed. These results strongly demonstrate that our proposed method can bring encouraging performance for image sentiment analysis.

\setcounter{magicrownumbers}{0} 
\begin{table}[t]
\begin{center}
\caption{Ablation study of~\shortname~on FI dataset.} 
% \vspace{1mm}
\label{tab:cap2}
\resizebox{.9\linewidth}{!}
{
\begin{tabular}{lcc}
  \hline
  \textbf{Model} & \textbf{Accuracy} & \textbf{F1} \\
  \hline
  (\rownumber)~Ours (w/o vision) & 65.72\% & 64.54\% \\
  (\rownumber)~Ours (w/o text description) & 74.05\% & 72.58\% \\
  (\rownumber)~Ours (w/o object tag) & 77.45\% & 76.84\% \\
  (\rownumber)~Ours (w/o scene tag) & 78.47\% & 78.21\% \\
  \hline
  (\rownumber)~Ours (w/o unified embedding) & 76.41\% & 76.23\% \\
  (\rownumber)~Ours (w/o adaptive learning) & 76.83\% & 76.56\% \\
  (\rownumber)~Ours (w/o cross-modal fusion) & 76.85\% & 76.49\% \\
  \hline
  (\rownumber)~Ours  & \textbf{79.48\%} & \textbf{79.22\%} \\
  \hline
\end{tabular}
}
\end{center}
\vspace{-5mm}
\end{table}


\begin{figure}[t]
\centering
% \vspace{-2mm}
\includegraphics[width=0.42\textwidth]{fig/2dvisual-linux4-paper2.pdf}
\caption{Visualization of feature distribution on eight categories before (left) and after (right) model processing.}
% 
\label{fig:visualization}
\vspace{-5mm}
\end{figure}

\subsection{Ablation Performance}
In this subsection, we conduct an ablation study to examine which component is really important for performance improvement. The results are reported in Table~\ref{tab:cap2}.

For information utilization, we observe a significant decline in model performance when visual features are removed. Additionally, the performance of \shortname~decreases when different metadata are removed separately, which means that text description, object tag, and scene tag are all critical for image sentiment analysis.
Recalling the model architecture, we separately remove transformer layers of the unified representation module, the adaptive learning module, and the cross-modal fusion module, replacing them with MLPs of the same parameter scale.
In this way, we can observe varying degrees of decline in model performance, indicating that these modules are indispensable for our model to achieve better performance.

\subsection{Visualization}
% 


% % 开始使用minipage进行左右排列
% \begin{minipage}[t]{0.45\textwidth}  % 子图1宽度为45%
%     \centering
%     \includegraphics[width=\textwidth]{2dvisual.pdf}  % 插入图片
%     \captionof{figure}{Visualization of feature distribution.}  % 使用captionof添加图片标题
%     \label{fig:visualization}
% \end{minipage}


% \begin{figure}[t]
% \centering
% \vspace{-2mm}
% \includegraphics[width=0.45\textwidth]{fig/2dvisual.pdf}
% \caption{Visualization of feature distribution.}
% \label{fig:visualization}
% % \vspace{-4mm}
% \end{figure}

% \begin{figure}[t]
% \centering
% \vspace{-2mm}
% \includegraphics[width=0.45\textwidth]{fig/2dvisual-linux3-paper.pdf}
% \caption{Visualization of feature distribution.}
% \label{fig:visualization}
% % \vspace{-4mm}
% \end{figure}



\begin{figure}[tbp]   
\vspace{-4mm}
  \centering            
  \subfloat[Depth of adaptive learning layers]   
  {
    \label{fig:subfig1}\includegraphics[width=0.22\textwidth]{fig/fig_sensitivity-a5}
  }
  \subfloat[Depth of fusion layers]
  {
    % \label{fig:subfig2}\includegraphics[width=0.22\textwidth]{fig/fig_sensitivity-b2}
    \label{fig:subfig2}\includegraphics[width=0.22\textwidth]{fig/fig_sensitivity-b2-num.pdf}
  }
  \caption{Sensitivity study of \shortname~on different depth. }   
  \label{fig:fig_sensitivity}  
\vspace{-2mm}
\end{figure}

% \begin{figure}[htbp]
% \centerline{\includegraphics{2dvisual.pdf}}
% \caption{Visualization of feature distribution.}
% \label{fig:visualization}
% \end{figure}

% In Fig.~\ref{fig:visualization}, we use t-SNE~\cite{van2008visualizing} to reduce the dimension of data features for visualization, Figure in left represents the metadata features before model processing, the features are obtained by embedding through the CLIP model, and figure in right shows the features of the data after model processing, it can be observed that after the model processing, the data with different label categories fall in different regions in the space, therefore, we can conclude that the Therefore, we can conclude that the model can effectively utilize the information contained in the metadata and use it to guide the model for classification.

In Fig.~\ref{fig:visualization}, we use t-SNE~\cite{van2008visualizing} to reduce the dimension of data features for visualization.
The left figure shows metadata features before being processed by our model (\textit{i.e.}, embedded by CLIP), while the right shows the distribution of features after being processed by our model.
We can observe that after the model processing, data with the same label are closer to each other, while others are farther away.
Therefore, it shows that the model can effectively utilize the information contained in the metadata and use it to guide the classification process.

\subsection{Sensitivity Analysis}
% 
In this subsection, we conduct a sensitivity analysis to figure out the effect of different depth settings of adaptive learning layers and fusion layers. 
% In this subsection, we conduct a sensitivity analysis to figure out the effect of different depth settings on the model. 
% Fig.~\ref{fig:fig_sensitivity} presents the effect of different depth settings of adaptive learning layers and fusion layers. 
Taking Fig.~\ref{fig:fig_sensitivity} (a) as an example, the model performance improves with increasing depth, reaching the best performance at a depth of 4.
% Taking Fig.~\ref{fig:fig_sensitivity} (a) as an example, the performance of \shortname~improves with the increase of depth at first, reaching the best performance at a depth of 4.
When the depth continues to increase, the accuracy decreases to varying degrees.
Similar results can be observed in Fig.~\ref{fig:fig_sensitivity} (b).
Therefore, we set their depths to 4 and 6 respectively to achieve the best results.

% Through our experiments, we can observe that the effect of modifying these hyperparameters on the results of the experiments is very weak, and the surface model is not sensitive to the hyperparameters.


\subsection{Zero-shot Capability}
% 

% (1)~GCH~\cite{2010Analyzing} & 21.78\% & (5)~RA-DLNet~\cite{2020A} & 34.01\% \\ \hline
% (2)~WSCNet~\cite{2019WSCNet}  & 30.25\% & (6)~CECCN~\cite{ruan2024color} & 43.83\% \\ \hline
% (3)~PCNN~\cite{2015Robust} & 31.68\%  & (7)~EmoVIT~\cite{xie2024emovit} & 44.90\% \\ \hline
% (4)~AR~\cite{2018Visual} & 32.67\% & (8)~Ours (Zero-shot) & 47.83\% \\ \hline


\begin{table}[t]
\centering
\caption{Zero-shot capability of \shortname.}
\label{tab:cap3}
\resizebox{1\linewidth}{!}
{
\begin{tabular}{lc|lc}
\hline
\textbf{Model} & \textbf{Accuracy} & \textbf{Model} & \textbf{Accuracy} \\ \hline
(1)~WSCNet~\cite{2019WSCNet}  & 30.25\% & (5)~MAM~\cite{zhang2024affective} & 39.56\%  \\ \hline
(2)~AR~\cite{2018Visual} & 32.67\% & (6)~CECCN~\cite{ruan2024color} & 43.83\% \\ \hline
(3)~RA-DLNet~\cite{2020A} & 34.01\%  & (7)~EmoVIT~\cite{xie2024emovit} & 44.90\% \\ \hline
(4)~CDA~\cite{han2023boosting} & 38.64\% & (8)~Ours (Zero-shot) & 47.83\% \\ \hline
\end{tabular}
}
\vspace{-5mm}
\end{table}

% We use the model trained on the FI dataset to test on the artphoto dataset to verify the model's generalization ability as well as robustness to other distributed datasets.
% We can observe that the MESN model shows strong competitiveness in terms of accuracy when compared to other trained models, which suggests that the model has a good generalization ability in the OOD task.

To validate the model's generalization ability and robustness to other distributed datasets, we directly test the model trained on the FI dataset, without training on Artphoto. 
% As observed in Table 3, compared to other models trained on Artphoto, we achieve highly competitive zero-shot performance, indicating that the model has good generalization ability in out-of-distribution tasks.
From Table~\ref{tab:cap3}, we can observe that compared with other models trained on Artphoto, we achieve competitive zero-shot performance, which shows that the model has good generalization ability in out-of-distribution tasks.


%%%%%%%%%%%%
%  E2E     %
%%%%%%%%%%%%


\section{Conclusion}
In this paper, we introduced Wi-Chat, the first LLM-powered Wi-Fi-based human activity recognition system that integrates the reasoning capabilities of large language models with the sensing potential of wireless signals. Our experimental results on a self-collected Wi-Fi CSI dataset demonstrate the promising potential of LLMs in enabling zero-shot Wi-Fi sensing. These findings suggest a new paradigm for human activity recognition that does not rely on extensive labeled data. We hope future research will build upon this direction, further exploring the applications of LLMs in signal processing domains such as IoT, mobile sensing, and radar-based systems.

\section*{Limitations}
While our work represents the first attempt to leverage LLMs for processing Wi-Fi signals, it is a preliminary study focused on a relatively simple task: Wi-Fi-based human activity recognition. This choice allows us to explore the feasibility of LLMs in wireless sensing but also comes with certain limitations.

Our approach primarily evaluates zero-shot performance, which, while promising, may still lag behind traditional supervised learning methods in highly complex or fine-grained recognition tasks. Besides, our study is limited to a controlled environment with a self-collected dataset, and the generalizability of LLMs to diverse real-world scenarios with varying Wi-Fi conditions, environmental interference, and device heterogeneity remains an open question.

Additionally, we have yet to explore the full potential of LLMs in more advanced Wi-Fi sensing applications, such as fine-grained gesture recognition, occupancy detection, and passive health monitoring. Future work should investigate the scalability of LLM-based approaches, their robustness to domain shifts, and their integration with multimodal sensing techniques in broader IoT applications.


% Bibliography entries for the entire Anthology, followed by custom entries
%\bibliography{anthology,custom}
% Custom bibliography entries only
\bibliography{main}
\newpage
\appendix

\section{Experiment prompts}
\label{sec:prompt}
The prompts used in the LLM experiments are shown in the following Table~\ref{tab:prompts}.

\definecolor{titlecolor}{rgb}{0.9, 0.5, 0.1}
\definecolor{anscolor}{rgb}{0.2, 0.5, 0.8}
\definecolor{labelcolor}{HTML}{48a07e}
\begin{table*}[h]
	\centering
	
 % \vspace{-0.2cm}
	
	\begin{center}
		\begin{tikzpicture}[
				chatbox_inner/.style={rectangle, rounded corners, opacity=0, text opacity=1, font=\sffamily\scriptsize, text width=5in, text height=9pt, inner xsep=6pt, inner ysep=6pt},
				chatbox_prompt_inner/.style={chatbox_inner, align=flush left, xshift=0pt, text height=11pt},
				chatbox_user_inner/.style={chatbox_inner, align=flush left, xshift=0pt},
				chatbox_gpt_inner/.style={chatbox_inner, align=flush left, xshift=0pt},
				chatbox/.style={chatbox_inner, draw=black!25, fill=gray!7, opacity=1, text opacity=0},
				chatbox_prompt/.style={chatbox, align=flush left, fill=gray!1.5, draw=black!30, text height=10pt},
				chatbox_user/.style={chatbox, align=flush left},
				chatbox_gpt/.style={chatbox, align=flush left},
				chatbox2/.style={chatbox_gpt, fill=green!25},
				chatbox3/.style={chatbox_gpt, fill=red!20, draw=black!20},
				chatbox4/.style={chatbox_gpt, fill=yellow!30},
				labelbox/.style={rectangle, rounded corners, draw=black!50, font=\sffamily\scriptsize\bfseries, fill=gray!5, inner sep=3pt},
			]
											
			\node[chatbox_user] (q1) {
				\textbf{System prompt}
				\newline
				\newline
				You are a helpful and precise assistant for segmenting and labeling sentences. We would like to request your help on curating a dataset for entity-level hallucination detection.
				\newline \newline
                We will give you a machine generated biography and a list of checked facts about the biography. Each fact consists of a sentence and a label (True/False). Please do the following process. First, breaking down the biography into words. Second, by referring to the provided list of facts, merging some broken down words in the previous step to form meaningful entities. For example, ``strategic thinking'' should be one entity instead of two. Third, according to the labels in the list of facts, labeling each entity as True or False. Specifically, for facts that share a similar sentence structure (\eg, \textit{``He was born on Mach 9, 1941.''} (\texttt{True}) and \textit{``He was born in Ramos Mejia.''} (\texttt{False})), please first assign labels to entities that differ across atomic facts. For example, first labeling ``Mach 9, 1941'' (\texttt{True}) and ``Ramos Mejia'' (\texttt{False}) in the above case. For those entities that are the same across atomic facts (\eg, ``was born'') or are neutral (\eg, ``he,'' ``in,'' and ``on''), please label them as \texttt{True}. For the cases that there is no atomic fact that shares the same sentence structure, please identify the most informative entities in the sentence and label them with the same label as the atomic fact while treating the rest of the entities as \texttt{True}. In the end, output the entities and labels in the following format:
                \begin{itemize}[nosep]
                    \item Entity 1 (Label 1)
                    \item Entity 2 (Label 2)
                    \item ...
                    \item Entity N (Label N)
                \end{itemize}
                % \newline \newline
                Here are two examples:
                \newline\newline
                \textbf{[Example 1]}
                \newline
                [The start of the biography]
                \newline
                \textcolor{titlecolor}{Marianne McAndrew is an American actress and singer, born on November 21, 1942, in Cleveland, Ohio. She began her acting career in the late 1960s, appearing in various television shows and films.}
                \newline
                [The end of the biography]
                \newline \newline
                [The start of the list of checked facts]
                \newline
                \textcolor{anscolor}{[Marianne McAndrew is an American. (False); Marianne McAndrew is an actress. (True); Marianne McAndrew is a singer. (False); Marianne McAndrew was born on November 21, 1942. (False); Marianne McAndrew was born in Cleveland, Ohio. (False); She began her acting career in the late 1960s. (True); She has appeared in various television shows. (True); She has appeared in various films. (True)]}
                \newline
                [The end of the list of checked facts]
                \newline \newline
                [The start of the ideal output]
                \newline
                \textcolor{labelcolor}{[Marianne McAndrew (True); is (True); an (True); American (False); actress (True); and (True); singer (False); , (True); born (True); on (True); November 21, 1942 (False); , (True); in (True); Cleveland, Ohio (False); . (True); She (True); began (True); her (True); acting career (True); in (True); the late 1960s (True); , (True); appearing (True); in (True); various (True); television shows (True); and (True); films (True); . (True)]}
                \newline
                [The end of the ideal output]
				\newline \newline
                \textbf{[Example 2]}
                \newline
                [The start of the biography]
                \newline
                \textcolor{titlecolor}{Doug Sheehan is an American actor who was born on April 27, 1949, in Santa Monica, California. He is best known for his roles in soap operas, including his portrayal of Joe Kelly on ``General Hospital'' and Ben Gibson on ``Knots Landing.''}
                \newline
                [The end of the biography]
                \newline \newline
                [The start of the list of checked facts]
                \newline
                \textcolor{anscolor}{[Doug Sheehan is an American. (True); Doug Sheehan is an actor. (True); Doug Sheehan was born on April 27, 1949. (True); Doug Sheehan was born in Santa Monica, California. (False); He is best known for his roles in soap operas. (True); He portrayed Joe Kelly. (True); Joe Kelly was in General Hospital. (True); General Hospital is a soap opera. (True); He portrayed Ben Gibson. (True); Ben Gibson was in Knots Landing. (True); Knots Landing is a soap opera. (True)]}
                \newline
                [The end of the list of checked facts]
                \newline \newline
                [The start of the ideal output]
                \newline
                \textcolor{labelcolor}{[Doug Sheehan (True); is (True); an (True); American (True); actor (True); who (True); was born (True); on (True); April 27, 1949 (True); in (True); Santa Monica, California (False); . (True); He (True); is (True); best known (True); for (True); his roles in soap operas (True); , (True); including (True); in (True); his portrayal (True); of (True); Joe Kelly (True); on (True); ``General Hospital'' (True); and (True); Ben Gibson (True); on (True); ``Knots Landing.'' (True)]}
                \newline
                [The end of the ideal output]
				\newline \newline
				\textbf{User prompt}
				\newline
				\newline
				[The start of the biography]
				\newline
				\textcolor{magenta}{\texttt{\{BIOGRAPHY\}}}
				\newline
				[The ebd of the biography]
				\newline \newline
				[The start of the list of checked facts]
				\newline
				\textcolor{magenta}{\texttt{\{LIST OF CHECKED FACTS\}}}
				\newline
				[The end of the list of checked facts]
			};
			\node[chatbox_user_inner] (q1_text) at (q1) {
				\textbf{System prompt}
				\newline
				\newline
				You are a helpful and precise assistant for segmenting and labeling sentences. We would like to request your help on curating a dataset for entity-level hallucination detection.
				\newline \newline
                We will give you a machine generated biography and a list of checked facts about the biography. Each fact consists of a sentence and a label (True/False). Please do the following process. First, breaking down the biography into words. Second, by referring to the provided list of facts, merging some broken down words in the previous step to form meaningful entities. For example, ``strategic thinking'' should be one entity instead of two. Third, according to the labels in the list of facts, labeling each entity as True or False. Specifically, for facts that share a similar sentence structure (\eg, \textit{``He was born on Mach 9, 1941.''} (\texttt{True}) and \textit{``He was born in Ramos Mejia.''} (\texttt{False})), please first assign labels to entities that differ across atomic facts. For example, first labeling ``Mach 9, 1941'' (\texttt{True}) and ``Ramos Mejia'' (\texttt{False}) in the above case. For those entities that are the same across atomic facts (\eg, ``was born'') or are neutral (\eg, ``he,'' ``in,'' and ``on''), please label them as \texttt{True}. For the cases that there is no atomic fact that shares the same sentence structure, please identify the most informative entities in the sentence and label them with the same label as the atomic fact while treating the rest of the entities as \texttt{True}. In the end, output the entities and labels in the following format:
                \begin{itemize}[nosep]
                    \item Entity 1 (Label 1)
                    \item Entity 2 (Label 2)
                    \item ...
                    \item Entity N (Label N)
                \end{itemize}
                % \newline \newline
                Here are two examples:
                \newline\newline
                \textbf{[Example 1]}
                \newline
                [The start of the biography]
                \newline
                \textcolor{titlecolor}{Marianne McAndrew is an American actress and singer, born on November 21, 1942, in Cleveland, Ohio. She began her acting career in the late 1960s, appearing in various television shows and films.}
                \newline
                [The end of the biography]
                \newline \newline
                [The start of the list of checked facts]
                \newline
                \textcolor{anscolor}{[Marianne McAndrew is an American. (False); Marianne McAndrew is an actress. (True); Marianne McAndrew is a singer. (False); Marianne McAndrew was born on November 21, 1942. (False); Marianne McAndrew was born in Cleveland, Ohio. (False); She began her acting career in the late 1960s. (True); She has appeared in various television shows. (True); She has appeared in various films. (True)]}
                \newline
                [The end of the list of checked facts]
                \newline \newline
                [The start of the ideal output]
                \newline
                \textcolor{labelcolor}{[Marianne McAndrew (True); is (True); an (True); American (False); actress (True); and (True); singer (False); , (True); born (True); on (True); November 21, 1942 (False); , (True); in (True); Cleveland, Ohio (False); . (True); She (True); began (True); her (True); acting career (True); in (True); the late 1960s (True); , (True); appearing (True); in (True); various (True); television shows (True); and (True); films (True); . (True)]}
                \newline
                [The end of the ideal output]
				\newline \newline
                \textbf{[Example 2]}
                \newline
                [The start of the biography]
                \newline
                \textcolor{titlecolor}{Doug Sheehan is an American actor who was born on April 27, 1949, in Santa Monica, California. He is best known for his roles in soap operas, including his portrayal of Joe Kelly on ``General Hospital'' and Ben Gibson on ``Knots Landing.''}
                \newline
                [The end of the biography]
                \newline \newline
                [The start of the list of checked facts]
                \newline
                \textcolor{anscolor}{[Doug Sheehan is an American. (True); Doug Sheehan is an actor. (True); Doug Sheehan was born on April 27, 1949. (True); Doug Sheehan was born in Santa Monica, California. (False); He is best known for his roles in soap operas. (True); He portrayed Joe Kelly. (True); Joe Kelly was in General Hospital. (True); General Hospital is a soap opera. (True); He portrayed Ben Gibson. (True); Ben Gibson was in Knots Landing. (True); Knots Landing is a soap opera. (True)]}
                \newline
                [The end of the list of checked facts]
                \newline \newline
                [The start of the ideal output]
                \newline
                \textcolor{labelcolor}{[Doug Sheehan (True); is (True); an (True); American (True); actor (True); who (True); was born (True); on (True); April 27, 1949 (True); in (True); Santa Monica, California (False); . (True); He (True); is (True); best known (True); for (True); his roles in soap operas (True); , (True); including (True); in (True); his portrayal (True); of (True); Joe Kelly (True); on (True); ``General Hospital'' (True); and (True); Ben Gibson (True); on (True); ``Knots Landing.'' (True)]}
                \newline
                [The end of the ideal output]
				\newline \newline
				\textbf{User prompt}
				\newline
				\newline
				[The start of the biography]
				\newline
				\textcolor{magenta}{\texttt{\{BIOGRAPHY\}}}
				\newline
				[The ebd of the biography]
				\newline \newline
				[The start of the list of checked facts]
				\newline
				\textcolor{magenta}{\texttt{\{LIST OF CHECKED FACTS\}}}
				\newline
				[The end of the list of checked facts]
			};
		\end{tikzpicture}
        \caption{GPT-4o prompt for labeling hallucinated entities.}\label{tb:gpt-4-prompt}
	\end{center}
\vspace{-0cm}
\end{table*}
% \section{Full Experiment Results}
% \begin{table*}[th]
    \centering
    \small
    \caption{Classification Results}
    \begin{tabular}{lcccc}
        \toprule
        \textbf{Method} & \textbf{Accuracy} & \textbf{Precision} & \textbf{Recall} & \textbf{F1-score} \\
        \midrule
        \multicolumn{5}{c}{\textbf{Zero Shot}} \\
                Zero-shot E-eyes & 0.26 & 0.26 & 0.27 & 0.26 \\
        Zero-shot CARM & 0.24 & 0.24 & 0.24 & 0.24 \\
                Zero-shot SVM & 0.27 & 0.28 & 0.28 & 0.27 \\
        Zero-shot CNN & 0.23 & 0.24 & 0.23 & 0.23 \\
        Zero-shot RNN & 0.26 & 0.26 & 0.26 & 0.26 \\
DeepSeek-0shot & 0.54 & 0.61 & 0.54 & 0.52 \\
DeepSeek-0shot-COT & 0.33 & 0.24 & 0.33 & 0.23 \\
DeepSeek-0shot-Knowledge & 0.45 & 0.46 & 0.45 & 0.44 \\
Gemma2-0shot & 0.35 & 0.22 & 0.38 & 0.27 \\
Gemma2-0shot-COT & 0.36 & 0.22 & 0.36 & 0.27 \\
Gemma2-0shot-Knowledge & 0.32 & 0.18 & 0.34 & 0.20 \\
GPT-4o-mini-0shot & 0.48 & 0.53 & 0.48 & 0.41 \\
GPT-4o-mini-0shot-COT & 0.33 & 0.50 & 0.33 & 0.38 \\
GPT-4o-mini-0shot-Knowledge & 0.49 & 0.31 & 0.49 & 0.36 \\
GPT-4o-0shot & 0.62 & 0.62 & 0.47 & 0.42 \\
GPT-4o-0shot-COT & 0.29 & 0.45 & 0.29 & 0.21 \\
GPT-4o-0shot-Knowledge & 0.44 & 0.52 & 0.44 & 0.39 \\
LLaMA-0shot & 0.32 & 0.25 & 0.32 & 0.24 \\
LLaMA-0shot-COT & 0.12 & 0.25 & 0.12 & 0.09 \\
LLaMA-0shot-Knowledge & 0.32 & 0.25 & 0.32 & 0.28 \\
Mistral-0shot & 0.19 & 0.23 & 0.19 & 0.10 \\
Mistral-0shot-Knowledge & 0.21 & 0.40 & 0.21 & 0.11 \\
        \midrule
        \multicolumn{5}{c}{\textbf{4 Shot}} \\
GPT-4o-mini-4shot & 0.58 & 0.59 & 0.58 & 0.53 \\
GPT-4o-mini-4shot-COT & 0.57 & 0.53 & 0.57 & 0.50 \\
GPT-4o-mini-4shot-Knowledge & 0.56 & 0.51 & 0.56 & 0.47 \\
GPT-4o-4shot & 0.77 & 0.84 & 0.77 & 0.73 \\
GPT-4o-4shot-COT & 0.63 & 0.76 & 0.63 & 0.53 \\
GPT-4o-4shot-Knowledge & 0.72 & 0.82 & 0.71 & 0.66 \\
LLaMA-4shot & 0.29 & 0.24 & 0.29 & 0.21 \\
LLaMA-4shot-COT & 0.20 & 0.30 & 0.20 & 0.13 \\
LLaMA-4shot-Knowledge & 0.15 & 0.23 & 0.13 & 0.13 \\
Mistral-4shot & 0.02 & 0.02 & 0.02 & 0.02 \\
Mistral-4shot-Knowledge & 0.21 & 0.27 & 0.21 & 0.20 \\
        \midrule
        
        \multicolumn{5}{c}{\textbf{Suprevised}} \\
        SVM & 0.94 & 0.92 & 0.91 & 0.91 \\
        CNN & 0.98 & 0.98 & 0.97 & 0.97 \\
        RNN & 0.99 & 0.99 & 0.99 & 0.99 \\
        % \midrule
        % \multicolumn{5}{c}{\textbf{Conventional Wi-Fi-based Human Activity Recognition Systems}} \\
        E-eyes & 1.00 & 1.00 & 1.00 & 1.00 \\
        CARM & 0.98 & 0.98 & 0.98 & 0.98 \\
\midrule
 \multicolumn{5}{c}{\textbf{Vision Models}} \\
           Zero-shot SVM & 0.26 & 0.25 & 0.25 & 0.25 \\
        Zero-shot CNN & 0.26 & 0.25 & 0.26 & 0.26 \\
        Zero-shot RNN & 0.28 & 0.28 & 0.29 & 0.28 \\
        SVM & 0.99 & 0.99 & 0.99 & 0.99 \\
        CNN & 0.98 & 0.99 & 0.98 & 0.98 \\
        RNN & 0.98 & 0.99 & 0.98 & 0.98 \\
GPT-4o-mini-Vision & 0.84 & 0.85 & 0.84 & 0.84 \\
GPT-4o-mini-Vision-COT & 0.90 & 0.91 & 0.90 & 0.90 \\
GPT-4o-Vision & 0.74 & 0.82 & 0.74 & 0.73 \\
GPT-4o-Vision-COT & 0.70 & 0.83 & 0.70 & 0.68 \\
LLaMA-Vision & 0.20 & 0.23 & 0.20 & 0.09 \\
LLaMA-Vision-Knowledge & 0.22 & 0.05 & 0.22 & 0.08 \\

        \bottomrule
    \end{tabular}
    \label{full}
\end{table*}




\end{document}


% Appendix
\ifnum\Includeappendix=1{
\appendix
%\addtocontents{toc}{\setcounter{tocdepth}{1}}
\section{Omitted Proofs from Section~\ref{sec:uniform}}\label{appendix:uni}
\subsection{Upper Bound}
\begin{proof}[Proof of Proposition~\ref{prop:envy is good SG}]
First, we note that the envy is a martingale.
\begin{observation}\label{envy is martingale}
For every $i,j\in[N]$, the sequence $\left(\env_{i,j}^t \right)_{t=1}^T$ is a martingale. 
\end{observation}
Furthermore, since $\dift$ is symmetric as Remark~\ref{remark: symmetric dif} hints, we can use the following result to connect its tail behavior and its variance. 
\begin{proposition}\label{thm: symmetric bounded sg}
Let $Y$ be a bounded random variable symmetric around $0$, i.e., for any $y \in \mathbb R$ it holds that $\prb{Y \geq y} = \prb{Y \leq -y}$. Then, $Y$ is $\sqrt{\var{Y}}$-subgaussian.
\end{proposition}
We prove Proposition~\ref{thm: symmetric bounded sg} after the end of this proof. Proposition~\ref{thm: symmetric bounded sg} suggests that $\dift$ is $\sqrt{\var{\dift}}$-subgaussian. Ultimately, Lemma~\ref{sg of sup of martingale} analyzes the subgaussianity parameter of the maximum of the martingale $(\env_{i,j}^t )_t$.
\begin{lemma}\label{sg of sup of martingale}
    Let $M^1, M^2,\dots M^T$ be a martingale with increments $Y^1,Y^2,\dots Y^T$, such that $Y^t \mid M^{t-1}$ is $\sigma_t$-SG. Then $\max_t M^t$ is $\left(\sqrt{\sum_{t=1}^T \sigma_t^2}\right)$-SG.
\end{lemma}
Lemma~\ref{sg of sup of martingale} suggests that $\max_{1\leq t\leq T} \env_{i,j}^t $ is 
$\left(\sqrt{\sum_{l=1}^t \var{\dift}}\right)$-SG, thereby concluding the proof of  Proposition~\ref{prop:envy is good SG}.
\end{proof}


\begin{proof}[Proof of Observation~\ref{envy is martingale}]
Recall that $\env_{i,j}^t = \sum^{t}_{l=1}{\Del{l}{i,j}}$. Since the order of selection at time $t$ is independent of $\ordr_t$, it holds that $\E{\env_{i,j}^{t+1}\mid \env_{i,j}^t} =\env_{i,j}^t$. Moreover, since $\E{\abs{\env_{i,j}^t}} \leq  T < \infty$, the stochastic process $\left(\env_{i,j}^t \right)_{t=1}^T$ is a martingale. 
\end{proof}

\begin{proof}[Proposition \ref{thm: symmetric bounded sg}]
    We begin by examining the Taylor polynomial of the function $f(z)=e^z$ around $0$.
    \[
    e^z=  1 + z + \frac{z^2}{2!} + \frac{z^3}{3!} e^{\xi_z},
    \]
    where the last term is the  Lagrange form of the remainder for some $\xi_z \in [-\abs{z}, \abs{z}]$.
    Using this expansion with $z=\lambda y$ and $\xi_{\lambda y} \in [-\abs{\lambda y}, \abs{\lambda y}]$ gets
    \[
    e^{\lambda y} = 1 + \lambda y + \frac{(\lambda y)^2}{2!} + \frac{(\lambda y)^3}{3!} e^{\xi_{\lambda y}}.
    \]
    If $-a<y<a$ then,
    \[
    e^{\lambda y} = 1 + \lambda y + \frac{(\lambda y)^2}{2!} + \frac{(\lambda y)^3}{3!} e^{\abs{\lambda a}}.
    \]
    Thus,
    \[
    \E{e^{\lambda Y}} \leq \E{1 + \lambda Y + \frac{(\lambda Y)^2}{2} + \frac{(\lambda Y)^3}{3} e^{|\lambda a|}} = 
    1 + \lambda \E{Y} + \frac{\lambda^2}{2}\E{Y^2} +\frac{\lambda^3 e^{|\lambda a|}}{3!} \E{Y^3}.
    \]
    Recall that $Y$ is symmetric around $0$ and hence $\E{Y} = \E{Y^3} = 0$, $\var{Y} = \E{Y^2}$.
    Combining these observations with the above we get
    \[
    \E{e^{\lambda Y}} \leq 1 + \frac{\lambda^2}{2}\var{Y} \leq e^{\frac{\lambda^2 \var{Y}}{2}},
    \]
    where the last inequality is due to that $1+z \leq e^z$ for all $z$.
    That sums up the proof of Lemma~\ref{thm: symmetric bounded sg}.
\end{proof}




\begin{proof}[Proof of Lemma~\ref{sg of sup of martingale}]
We begin by proving that $M^t$ is $\left(\sqrt{\sum_{l=1}^t \sigma_l^2}\right)$-SG, and then address $\max_t M^t$


The base case is $t=1$, where we have $M^1=Y^1$ and $Y^1$ is $\sigma_1$-SG by definition. Next, assume that the statement holds for $t=k-1$. It holds that 
\begin{align*}
\E{e^{\lambda M^{k}}}&=\E{\E{e^{\lambda M^{k}}\mid M^{k-1}}}=\E{\E{e^{\lambda (M^{k-1}+Y^k)}\mid M^{k-1}}} = \E{e^{\lambda M^{k-1}} \E{e^{\lambda Y^k}\mid M^{k-1}}}\\
& \leq \E{e^{\lambda M^{k-1}} e^{\frac{\lambda^2 \sigma_k^2}{2}}} \leq e^{\frac{\lambda^2 \sum_{l=1}^{k-1} \sigma_l^2}{2} }e^{\frac{\lambda^2 \sigma_k^2}{2}} = e^{\frac{\lambda^2 \sum_{l=1}^{k} \sigma_l^2}{2} },
\end{align*}
where we used total expectation, the fact that $Y^k\mid M^{k-1}$ is $\sigma_k$-SG and the inductive assumption.

Next, let $\tau$ denote the r.v. for which $M^\tau = \max_t M^t$. It holds that
\begin{align*}
    \E{e^{\lambda \max_t M^t}} &= \E{e^{\lambda M^\tau}} = \E{\E{e^{\lambda M^\tau} \mid \tau}} \leq \E{\E{e^{\lambda M^\tau} \mid \tau}} \leq \E{e^{\frac{\lambda^2 \sum_{l=1}^{\tau} \sigma_l^2}{2} }} \leq e^{\frac{\lambda^2 \sum_{l=1}^{T} \sigma_l^2}{2} },
\end{align*}
where the second to last step follows from the fact that $M^t$ is $\left(\sqrt{\sum_{l=1}^t \sigma_l^2}\right)$-SG. This completes the proof of Lemma~\ref{sg of sup of martingale}.
\end{proof}

\begin{proof}[Proof of Claim~\ref{claim: sg max}]
Fix $a \in \mathbb{R}$ and denote $Y_{\textnormal{max}} = \max_{i \in [n]}{\{Y_i\}}$ for convenience.
We begin by examining $e^{a\E{Y_{\textnormal{max}} }}$.
Using Jensen’s inequality, we have that
\begin{align}\label{eq:asdfrsdgvbsdfg}
e^{a\E{Y_{\textnormal{max}} }} \leq \E{e^{a Y_{\textnormal{max}}}}= \E{\max_{i \in [n]}{\left\{ e^{a Y_i } \right\}}} \leq
\sum_{i=1}^{n}{\E{e^{a Y_i}}} \leq \sum_{i=1}^{n}{ e^{\frac{a^2\sigma^2}{2}} } = ne^{\frac{a^2\sigma^2}{2}}.    
\end{align}
Taking $\ln$ on both sides of Inequality~\eqref{eq:asdfrsdgvbsdfg},
\[
a\E{Y_{\textnormal{max}}}\leq \ln(n) + \frac{a^2\sigma^2}{2}.
\]
Diving by $a$, we obtain
\begin{align}\label{eq:dasolgbdf}
\E{Y_{\textnormal{max}}}\leq \frac{\ln(n)}{a} + \frac{a\sigma^2}{2}.
\end{align}
Inequality~\eqref{eq:dasolgbdf} holds for all $a\in \mathbb{R}$ and specifically for the minimizer of $\frac{\ln{(n)}}{a}\!+\!\frac{a\sigma^2}{2}$, which is $a=\frac{\sqrt{2 \ln{(n)}}}{\sigma}$.
We complete the proof of the claim by substituting $a$ with this value in Inequality~\eqref{eq:dasolgbdf}.
\end{proof}
\begin{proof}[Proof of Proposition~\ref{thm: threshold var}]
    First, since $\dift \leq 1$ almost surely, $\var{\dift} \leq 1$ holds trivially. For the more challenging expression, the proof relies on the fact that when executing an explore-first algorithm, after at most $K$ sessions of any round, all remaining $N-K$ agents must receive the same reward. Hence, at the end of every round $t$, we can find at least $\binom{N-K+1}{2}$ pairs of agents $i,j$ that satisfy $\dif^t_{i,j} = 0$. In other words, for any $i,j$, it holds that 
    \[
    \prb{{\dif^t_{i,j}}=0}\geq \frac{\binom{N-K+1}{2}}{\binom{N}{2}},
    \]
    where the randomness is taken over the stochasticity of the rewards and the arrival order. From that,
    \begin{align*}
        \var{\dift} & = \E{{\dift}^2} - \E{\dift}^2 = \E{{\dift}^2 \mid {\dift}^2 \neq 0}\cdot \prb{{\dift}^2 \neq 0} - \E{\dift}^2
        \nonumber \\&\leq
        1\cdot \left( 1 - \frac{\binom{N-K+1}{2}}{\binom{N}{2}} \right) - 0 =
        1 - \frac{ \left( N-K+1 \right)\left( N-K \right) }{N\left( N-1 \right)}.
    \end{align*}
    With the help of some algebraic operations, we can simplify the expression.
    \begin{align*}
        \var{\dift} & \leq
        1 - \frac{ \left( N-K+1 \right)\left( N-K \right) }{N\left( N-1 \right)} =
        \frac{ N^2 - N - \left( N^2 - 2NK + K^2 + N - K \right) }{ N\left( N-1 \right)} 
        \\&=
        \frac{2NK -2N + K -K^2 }{ N\left( N-1 \right)} =
        \frac{2N(K-1) - K(K-1) }{N\left( N-1 \right)} =
        \frac{(2N - K)(K-1) }{N\left( N-1 \right)}.
    \end{align*}
    This concludes the proof of Proposition~\ref{thm: threshold var}.
\end{proof}
\begin{proof}[Proof of Corollary~\ref{thm: sqrt TK N}]
The corollary holds since 
\begin{align*}
\E{\max_{1\leq t \leq T} \env^t} &\leq
2\sqrt{\ln{(N)}\sum^T_{t=1}{\var{\dift}}}\leq 
2\sqrt{\ln{(N)}\sum^T_{t=1}{\min\left\{1,\frac{2(K-1)}{N-1} \right\}}}\\
&\leq 2\sqrt{T\ln{(N)}{\frac{2(K-1)}{N-1}}}.    
\end{align*}
\end{proof}

\subsection{Lower Bound}\label{appendix:lower bound}
\begin{claim}\label{claim: uni suff}
For the execution in Example~\ref{example: uni suff}, it holds that $\var{\dift} = \frac{1}{12}$.
\end{claim}
\begin{proof}[Proof of Claim~\ref{claim: uni suff}]
We prove the claim by using the definition of variance.
\begin{align}\label{eq:adfgsfhsaagh}
\var{\dift} & = \E{({\dift})^2}-\E{{\dift}}^2\overset{(1)}{=}\E{\left(\rt{(1)} - \rt{(2)}\right)^2} = \E{{\rt{(1)}}^2} - 2\E{\rt{(1)}\rt{(2)}} + \E{{\rt{(2)}}^2}
\end{align}
Next, we apply the threshold structure and properties of the uniform distribution to Equation~\eqref{eq:adfgsfhsaagh}.
{\thinmuskip=0mu
\medmuskip=0mu plus 0mu minus 0mu
\thickmuskip=1mu plus 1mu minus 1mu
\begin{align*}
\textnormal{Eq. }\eqref{eq:adfgsfhsaagh} &= 
\E{{X_1}^2} - \frac{1}{2}\cdot 2\E{X_1 X_1 \mid X_1 \geq \frac{1}{2}} - \frac{1}{2}\cdot 2\E{ X_1 X_2\mid X_1 < \frac{1}{2}}+ \frac{1}{2}\cdot \E{{X_1}^2 \mid X_1 \geq \frac{1}{2}} + \frac{1}{2}\cdot \E{{X_2}^2 \mid X_1 < \frac{1}{2}} \\
&=
\E{{X_1}^2} - \frac{1}{2} \E{{X_1}^2 \mid X_1 \geq \frac{1}{2}} - \E{ X_1\mid X_1 < \frac{1}{2}} \E{ X_2} + \frac{1}{2} \E{{X_2}^2}
\\&=
\frac{3}{2}\E{{\uni{0,1}}^2} - \frac{1}{2} \E{{\uni{\frac{1}{2},1}}^2} - \E{\uni{0,\frac{1}{2}}}= \frac{3}{2} \cdot \frac{1}{3}- \frac{1}{2} \cdot \frac{7}{12} - \frac{1}{4} \cdot \frac{1}{2} \\
&= \frac{1}{12},
\end{align*}}%
where we have used the independence of $X_1$ and $X_2$.
\end{proof}


\begin{claim}\label{claim:example ber suff}
For the execution in Example~\ref{example: ber suff}, it holds that 
\begin{enumerate}
    \item $\var{\Delta^t} \geq \frac{2p(1 - p)}{N}$.
    \item $\var{\Delta^t} \leq  2(1-p)$.
    \item $\var{\Delta^t} \leq  2pK$.
\end{enumerate}
\end{claim}
\begin{proof}[Proof of Claim~\ref{claim:example ber suff}]
Fix two arbitrary agents $i$ and $j$. Recall that Remark~\ref{remark: symmetric dif} ensures that $\Delta_{i,j}^t$ is identically distributed, regardless of the indexes $i$ and $j$. Let the event $B_q$ indicate that the number of arms that realize a value of 0 is $q$, for $q \in \{0, \ldots, K\}$. Further, let $E$ denote the event that $i$ and $j$ receive different rewards. Then we have:
\[
\var{\Delta_{i,j}^t} = \E{(\Delta_{i,j}^t)^2} 
= \sum_{q=0}^K \bigl( \E{(\Delta_{i,j}^t)^2 \mid B_q, E}\Pr(B_q, E) 
+ \E{(\Delta_{i,j}^t)^2 \mid B_q, \overline{E}}\Pr(B_q, \overline{E}) \bigr).
\]
Since $(\Delta_{i,j}^t)^2$ takes the value 1 under event $E$ and 0 otherwise, we get:
\begin{equation}\label{eq:gknmhmgf}
\var{\Delta_{i,j}^t} = \sum_{q=0}^K (1 \cdot \Pr(B_q, E) + 0 \cdot \Pr(B_q, \overline{E})) 
= \sum_{q=1}^K \Pr(B_q, E).    
\end{equation}
Fix any $q\in[K]$. It holds that 
\begin{align}\label{eq:develop for q}
\Pr(B_q, E) = \Pr(B_q)\Pr(E \mid B_q) =  p (1 - p)^{q}\cdot \frac{2\binom{N-2}{q-1}}{\binom{N}{q}} =  2 p (1 - p)^{q}\frac{q(N-q)}{N(N-1)}.
\end{align}
Combining Equations~\eqref{eq:gknmhmgf} and~\eqref{eq:develop for q}, we get
\begin{align}\label{eq:var def}
\var{\Delta_{i,j}^t} = \sum_{q=1}^K 2 p (1 - p)^{q}\frac{q(N-q)}{N(N-1)}.
\end{align}
Since all summands are positive, we obtain the first part of the claim by bounding from below using only the $q=1$ term. That is, we obtain $\var{\Delta_{i,j}^t} \geq  \frac{2p(1 - p)}{N}$. 

For the other parts of the claim, observe that for every $q \in [K]$, $ \frac{q(N-q)}{N(N-1)} \leq 1$; hence, Equation~\eqref{eq:var def} implies that 
\begin{align}\label{eq:gdhfghsdfg}
\var{\Delta_{i,j}^t} \leq \sum_{q=1}^K 2 p (1 - p)^{q}.
\end{align}
From here, we use Inequality~\eqref{eq:gdhfghsdfg} to obtain the second and third parts of the claim. First,
\begin{align*}
\sum_{q=1}^K 2 p (1 - p)^{q} \leq 2 p (1 - p) \sum_{q=0}^\infty (1 - p)^{q} = \frac{ 2 p (1 - p)}{1-(1-p)} = 2(1 - p);
\end{align*}
thus, $\var{\Delta_{i,j}^t} \leq  2(1-p)$ as the second part of the claim implies. Using a different approach to upper bound Inequality~\eqref{eq:gdhfghsdfg}, we get
\begin{align*}
\sum_{q=1}^K 2 p (1 - p)^{q} \leq 2p \sum_{q=1}^K 1^{q}=2pK
\end{align*}
As the third part of the claim asserts. This completes the proof of Claim~\ref{claim:example ber suff}.
\end{proof}



\iffalse %This is for the "sophisticated bound
Next, we move to the third part of the claim. Applying another approach to upper bound the right-hand-side of Equation~\eqref{eq:var def}, we obtain
\begin{align}\label{eq:var upper}
\sum_{q=1}^K 2 p (1 - p)^{q}\frac{q(N-q)}{N(N-1)} &=  \frac{2p}{N (N-1)} \sum_{q=1}^K (1 - p)^{q}q(N-q) \leq \frac{2p}{N (N-1)} \sum_{q=1}^K (1 - p)^{q}q (N -1) 
\nonumber \\
& = \frac{2p}{N} \sum_{q=1}^K (1 - p)^{q}q   \leq \frac{2p}{N}  \sum_{q=0}^\infty (1 - p)^{q}q.
\end{align}
Due to  Observation~\ref{obs:geo and mul} below,
\[
\sum_{q=0}^\infty (1 - p)^{q}q \leq \frac{1-p}{p^2}.
\]
Combining this with Inequality~\eqref{eq:var upper}, we ultimately obtain 
\[
\var{\Delta_{i,j}^t} \leq \frac{2p}{N} \frac{1-p}{p^2} = \frac{2(1-p)}{N p}.
\]
This completes the proof of Claim~\ref{claim:example ber suff}.
\end{proof}
\begin{observation}\label{obs:geo and mul}
For any $x\in (0,1)$, it holds that 
\[
\sum_{q=0}^\infty q x^{q} \leq \frac{x}{(1-x)^2}.
\]
\end{observation}
\begin{proof}[Proof of Observation~\ref{obs:geo and mul}]
Starting from $\sum_{n=0}^\infty x^n = \frac{1}{1-x}$, we differentiate both sides by $x$ to obtain
\[
\frac{d}{dx}\left(\sum_{n=0}^\infty x^n \right) = \frac{d}{dx}\left( \frac{1}{1-x} \right)\Leftrightarrow \sum_{n=0}^\infty n x^{n-1}  = \frac{1}{(1-x)^2} \Leftrightarrow \sum_{n=0}^\infty n x^{n}  = \frac{x}{(1-x)^2},
\]
where the last transition follows from multiplying both sides by $x$.
\end{proof}
\fi


\begin{proof}[Proof of Proposition~\ref{thm: board}]
    The proof of Proposition~\ref{thm: board} relies on the following algebraic inequality, which we prove after this proof.
    \begin{observation}\label{obs: algebric}
        For any $a\geq 0$, it holds that $2a \geq 3a^2 - a^4$.
    \end{observation}
    
    Recall, $Y$ is a non-negative random variable and therefore $\sqrt{\frac{Y}{\E{Y}}}$ is non-negative as well.
    When setting $a=\sqrt{\frac{Y}{\E{Y}}}$ we have
    \begin{align*}
         \frac{2\sqrt{Y}}{\sqrt{\E{Y}}} \geq \frac{3Y}{\E{Y}} - \frac{Y^2}{\E{Y}^2},
    \end{align*}
    for any value $Y$ can take.
    
    Notice that $\E{Y}\geq 0$, hence, by multiplying each side of the inequality by $\frac{\sqrt{\E{Y}}}{2}$ we get
    \begin{align*}
        \sqrt{Y} \geq \sqrt{\E{Y}}\left(\frac{3Y\E{Y} - Y^2}{2\E{Y}^2}\right).
    \end{align*}
    Taking expectation on both sides yields
    \begin{align*}
        \E{\sqrt{Y}} & \geq
        \E{\sqrt{\E{Y}}\left(\frac{3Y\E{Y} - Y^2}{2\E{Y}^2}\right)} \overset{(1)}{=}
        \sqrt{\E{Y}}\left(\E{\frac{2Y\E{Y}}{2\E{Y}^2}} - \E{\frac{Y^2 - Y\E{Y}}{2\E{Y}^2}}\right)
        \\&\overset{(2)}{=}
        \sqrt{\E{Y}}\left(\frac{2\E{Y}\E{Y}}{2\E{Y}^2} - \frac{\E{Y^2} - \E{Y}\E{Y}}{2\E{Y}^2}\right) \overset{(3)}{=}
        \sqrt{\E{Y}}\left(1 - \frac{\var{Y}}{2\E{Y}^2}   \right) ,
    \end{align*}
    where $(1)$ and $(2)$ hold due to linearity of expectation and $(3)$ is by the definition of variance. This concludes the proof of Proposition~\ref{thm: board}.
\end{proof}

\begin{proof}[Proof of Observation~\ref{obs: algebric}]
    To prove the inequality, it is sufficient to prove that the function $f(a) = a^3 - 3a^2 -2$ is non-negative for all $a\geq 0$.
    It holds that $f'(a) = 3 a^2 - 3$; thus
    \begin{align*}
        f'(a) > 0, & \text{ when } 0 \leq a < 1 \\
        f'(a) = 0, & \text{ when } a = 1 \\
        f'(a) < 0, & \text{ when } 1 \leq a.
    \end{align*}
    I.e., $f(a)$ is monotonically decreasing for $0\leq a <1$ and monotonically increasing for $1 \leq a$. 
    Hence, for all $a \geq 0$ it holds that $f(1) \geq f(a) = 0$
\end{proof}

\begin{proof}[Proof of Proposition~\ref{prop:insufficient}]
We prove the proposition by reiterating the proof of Theorem~\ref{thm: uni lower-bound} while avoiding using the definition of sufficient execution. It suffices to show that the left-hand side of Inequality~\eqref{eq:m,bnhjikw} is constant. Starting with the numerator,
\begin{align}\label{eq:lpods}
\var{ \sum^T_{t=1}{ \dift^2} } &= \sum^T_{t=1} \var{ \dift^2} =  \sum^T_{t=1} \left(\E{(\dift)^4}-\E{(\dift)^2}^2 \right) = \sum^T_{t=1} \left( \var{\dift} - \var{\dift}^2 \right) \nonumber \\
&= T\left( \var{\dift} - \var{\dift}^2 \right),
\end{align}
where we have used the fact that the algorithm is stationary over rounds and that $\dift$ only takes values in the $\{0,1\}$ set. We keep using the superscript $t$ in Equation~\eqref{eq:m,bnhjikw} to ease readability, and it could be any arbitrary $t \in [T]$.

Next, we consider the denominator of the left-hand side of Inequality~\eqref{eq:m,bnhjikw}.
\begin{align}\label{eq:ydots}
2 \E{ \sum^T_{t=1}{(\dift)^2} }^2 =2  \left(\sum^T_{t=1}{\E{(\dift)^2} }\right)^2 = 2  \left(\sum^T_{t=1}{\var{\dift} }\right)^2 =  2 \left(T{\var{\dift} }\right)^2 = 2T^2 \var{\dift}^2
\end{align}
Combining Equations~\eqref{eq:lpods} and~\eqref{eq:ydots}, we get
\begin{align}\label{eq:prea}
\frac{T\left( \var{\dift} - \var{\dift}^2 \right)}{2T^2 \var{\dift}^2}=\frac{\left( 1 - \var{\dift} \right)}{2T \var{\dift}} \leq 
\frac{1}{2T \var{\dift}} \leq \frac{1}{2T}\frac{N}{2p(1-p)},
\end{align}
where the last inequality follows from Claim~\ref{claim:example ber suff}. 
Furthermore, recall that the proposition assumption guarantees that $p\in \left[\frac{N}{cT}, 1-\frac{N}{cT}\right]$ for $c\geq 2$, suggesting that
\[
p (1-p)\geq \frac{N}{cT}\left(1- \frac{N}{cT}\right) \geq \frac{N}{cT}\left(1- \frac{1}{c}\right) \geq \frac{N}{cT}\left(\frac{c-1}{c}\right) \geq \frac{N}{cT}\cdot \frac{1}{2}.
\]
Plugging this into the right-hand-side of Inequality~\eqref{eq:prea},
\begin{align}
 \frac{1}{2T}\frac{N}{2p(1-p)} \leq \frac{N}{4T} \cdot \frac{1}{p(1-p)} \leq \frac{N}{4T} \cdot \frac{2c T}{N}  = \frac{c}{2}.
\end{align}
Having observed that the left-hand-side of Inequality~\eqref{eq:m,bnhjikw} is bounded by a constant w.r.t. $T$, we complete the proof by plugging this constant into Inequality~\eqref{thm: mp uni lower-bound 2}.
\end{proof}



\section{Omitted Proofs from Section~\ref{sec:nudge}}\label{appendix:nudge}

\begin{proof}[Proof of Proposition~\ref{prop G less than M}]
We prove the claim by induction over $\tau$. The first round in the excursion $D(t)$ and our base case is $\tau=\underline{t}+1$. Since the rewards are in the $[0,1]$ interval and $G_i^{\underline{t}} \leq 1$, we know that $G_i^{\underline{t}+1} \leq 2 = M_i^{\underline{t}+1}$. 



Next, assume the claim holds for $\tau$; thus, $G_i^\tau \leq M_i^\tau$. Recall that we are guaranteed that $\tau \in D(t)$. Without loss of generality, assume that at time $\tau$ agents are ordered lexicographically. Particularly,  ${\sigma^\tau(i)}=i,{\sigma^\tau(i+1)}=i+1$  and $G_i^\tau = R^\tau_{\sigma^\tau(i)}- R^\tau_{\sigma^\tau(i+1)}= R^\tau_i- R^\tau_{i+1}$. 
Next, observe that 
\begin{equation}\label{eq:asdgndsfjghm}
    R^{\tau+1}_{\sigma^{\tau+1}(i)} = \min_{j\in [i]}\left\{ 
    R^{\tau}_{j}+r^{\tau+1}_{j} 
    \right\}\leq R^{\tau}_{i}+r^{\tau+1}_{i}.
\end{equation}
Inequality~\eqref{eq:asdgndsfjghm} holds due to our assumption that the rewards are ordered according to agent indices at round $\tau$.  and since no agent in the set $[N]\setminus[i]$ could obtain a higher cumulative reward that agents $[i]$ at round $\tau + 1$ since all rewards are bounded by 1 and $G^\tau_i > 1$. Similarly,
\begin{equation}\label{eq:dsasdhhgtnt}
    R^{\tau+1}_{\sigma^{\tau+1}(i+1)} = \max_{j\in [N]\setminus[i]}\left\{ 
    R^{\tau}_{j}+r^{\tau+1}_{j} 
    \right\}\geq R^{\tau}_{i+1}+r^{\tau+1}_{i+1}.
\end{equation}
Combining Inequalities~\eqref{eq:asdgndsfjghm} and~\eqref{eq:dsasdhhgtnt}, we derive that
\begin{align*}
G_i^{\tau+1} &= R^{\tau+1}_{\sigma^{\tau+1}(i)} - R^{\tau+1}_{\sigma^{\tau+1}(i+1)} \leq R^{\tau}_{i}+r^{\tau+1}_{i} - R^{\tau}_{i+1}-r^{\tau+1}_{i+1} \\
& =G_i^\tau + r^{\tau+1}_{i} - r^{\tau+1}_{i+1} \leq M_i^\tau + r^{\tau+1}_{i} - r^{\tau+1}_{i+1} \\
&= M_i^\tau + r^{\tau+1}_{(i)} - r^{\tau+1}_{(i+1)} = M_i^{\tau+1},
\end{align*}
where we have used the inductive assumption and the fact that nudged arrival order sorts agents in a non-increasing order of rewards (Algorithm~\ref{alg: sugg arr}, Line~\ref{line:mapping}). This completes the proof of Proposition~\ref{prop G less than M}.
\end{proof}





\begin{proof}[Proof of Proposition~\ref{prop: sugg-m concentration}]
The recursive definition of $M^\tau$ implies that for every $\tau \in D(t)$, $M^\tau = 2+ \sum_{n= \underline{t}+2 }^{\tau} r^n_{(i)}-r^n_{(i+1)}$; thus, 
\begin{align}\label{eq:fghbdfgh}
\prb{M^t > n}  = \prb{\sum_{l= \underline{t}+2 }^{\tau} r^l_{(i)}-r^l_{(i+1)}> n-2}
\end{align}
Next, let $B^l$ denote the event that $r^l_{(i)}-r^l_{(i+1)} \neq 0$. Furthermore, let $B(\tau)$ denote the (random) set of rounds for which the event $B^l$ occurs between $\underline{t}+2$ and $\tau$. That is,
\[
B(\tau) = \{ l\mid \underline{t}+2 \leq l \leq \tau, \ind{B^l} \}
\]
As a result, due to Property~\ref{prop:nudge} and the definition of $\tdif$ in Equation~\eqref{eq def tdif},
\begin{equation}\label{eq:sgdjfndb}
%\E{r^l_{(i)}-r^l_{(i+1)}} = \E{r^l_{(i)}-r^l_{(i+1)} \mid B^l}\prb{B^l} \leq 
\E{r^l_{(i)}-r^l_{(i+1)} \mid B^l} \leq - \delta \tdif.
\end{equation}
Rewriting Equation~\eqref{eq:fghbdfgh},
\begin{align}\label{eq:hbngaersd}
\prb{M^t > n}  &= \prb{\sum_{l \in B(\tau)} r^l_{(i)}-r^l_{(i+1)}> n-2} \nonumber \\
& =\sum_{b\subseteq \{\underline{t}+2,\dots \tau \}}\prb{\sum_{l \in b} r^l_{(i)}-r^l_{(i+1)}> n-2 \mid  B(\tau) = b}\prb{B(\tau) = b} \nonumber\\
& \stackrel{*}{=} \sum_{b\subseteq \{\underline{t}+2,\dots \tau \}, \abs{b}\geq n-2}\prb{\sum_{l \in b} r^l_{(i)}-r^l_{(i+1)}> n-2 \mid  B(\tau) = b}\prb{B(\tau) = b} \nonumber\\
& \leq \max_{b\subseteq \{\underline{t}+2,\dots \tau \}, \abs{b}\geq n-2} \prb{\sum_{l \in b} r^l_{(i)}-r^l_{(i+1)}> n-2 \mid  B(\tau) = b} \nonumber \\
& = \max_{b\subseteq \{\underline{t}+2,\dots \tau \}, \abs{b}\geq n-2} \prb{\sum_{l \in b} r^l_{(i)}-r^l_{(i+1)} + \delta \tdif \abs{b}> n-2+\delta \tdif \abs{b} \mid  B(\tau) = b},
\end{align}
where the change in the set over which we sum in $*$ follows since $\abs{r^l_{(i)}-r^l_{(i+1)}}\leq 1$ almost surely. Striving to bound the above, notice that, conditioned on $B(\tau) = b$, $\sum_{l \in b} \left( r^l_{(i)}-r^l_{(i+1)} + \delta \tdif \right)$ forms a super-martingale. Using Azuma-Hoeffding inequality,
\begin{align*}
\textnormal{Inequality }\eqref{eq:hbngaersd} &\leq \max_{b\subseteq \{\underline{t}+2,\dots \tau \}, \abs{b}\geq n-2} \exp\left\{-\frac{(n-2+\delta \tdif \abs{b})^2}{2 \sum_{l\in b} (1+\delta \tdif)^2 }\right\} \nonumber \\
& \stackrel{\delta \tdif \leq 1}{\leq}  \max_{b\subseteq \{\underline{t}+2,\dots \tau \}, \abs{b}\geq n-2} \exp\left\{-\frac{(n-2)^2+(n-2)(\delta \tdif \abs{b})+(\delta \tdif \abs{b})^2}{8\abs{b}}\right\} \nonumber
\\
%& =  \max_{b\subseteq \{\underline{t}+2,\dots \tau \}, \abs{b}\geq n-2} \exp\left\{-\frac{(n-2)^2+(n-2)(\delta \tdif \abs{b})+(\delta \tdif \abs{b})^2}{8\abs{b}}\right\} \nonumber \\
& =  \max_{b\subseteq \{\underline{t}+2,\dots \tau \}, \abs{b}\geq n-2} \exp\Bigg\{-\frac{(n-2)(\delta \tdif)}{8}-\underbrace{\frac{(n-2)^2+(\delta \tdif \abs{b})^2}{8\abs{b}}}_{\geq 0}\Bigg\} \nonumber \\
& \leq  \max_{b\subseteq \{\underline{t}+2,\dots \tau \}, \abs{b}\geq n-2} \exp\left\{\frac{-(n-2)\delta \tdif}{8}\right\} \nonumber \\
& = \exp\left\{-\frac{(n-2)(\delta \tdif)}{8}\right\}.
\end{align*}
This completes the proof of Proposition~\ref{prop: sugg-m concentration}.
\end{proof}



\section{Models Captured by the Nudged Arrival Property}\label{appendix:nudge-models}
%\omer{I do not feel comfortable with this...}
\subsection*{Mallows Model~\cite{mallows1957non}}

The Mallows model prioritizes rankings close to a reference order \( \sigma \). The probability of a sampled ranking \( \pi \) is proportional to \( e^{-\beta d(\pi, \sigma)} \), where \( d(\pi, \sigma) \) is the Kendall's tau distance between \( \pi \) and \( \sigma \), and \( \beta \geq 0 \) is the concentration parameter. For any pair of agents \( i, j \) such that \( \sigma^{-1}(i) < \sigma^{-1}(j) \), the probability that \( i \) precedes \( j \) satisfies (following the detailed argument of \cite[Section 2]{lu2014effective}):
\[
\Pr_{\pi \sim \text{Mallows}}(\pi^{-1}(i) < \pi^{-1}(j)) = \frac{e^{-\beta}}{1 + e^{-\beta}}.
\]
By arithmetic manipulation we get $\delta = \frac{1 - e^{-\beta}}{1 + e^{-\beta}}$. 


\subsection*{Plackett-Luce Model~\cite{marden1996analyzing}}
In the Plackett-Luce model, each agent \( i \) is assigned a positive score \( w_i > 0 \), and the probability of observing a ranking \( \pi \) is given by:
\[
\Pr(\pi) = \prod_{k=1}^{N} \frac{w_{\pi(k)}}{\sum_{j=k}^{N} w_{\pi(j)}}.
\]
For any pair \( i, j \), the probability that \( i \) precedes \( j \) is:
\[
\Pr(\pi^{-1}(i) < \pi^{-1}(j)) = \frac{w_i}{w_i + w_j},
\]
which holds since the model satisfies Luce's choice axiom, which guarantees the independence of the pairwise ranking probabilities from the presence of other options \cite{luce1959individual}. Thus, by setting the scores such that \( \frac{w_i}{w_i + w_j} \geq \frac{1+\delta}{2} \) for every $i$ so that $i$ precedes $j$ in the optimal permutation, the nudged arrival property is satisfied. One way to do it is at each round $t$, recursively set $w_{\pi(1)}^{t} = 1, w_{\pi(i+1)}^{t} = \frac{1+\delta}{1 - \delta} w_{\pi(i)}^{t}$. We thus assume that the weights are not global across rounds but are round-dependent and are adjusted based on the accumulated rewards. This can be interpreted as either the designer nudging different agents more forcefully, or alternatively, in a behavioral approach, users that benefited more from the system in the past are willing to cooperate more with its nudges. 
%\omer{how do we 'set' $w_i$?}


% Bradley-Terry is irrelvant, it's only for n=2 in and of itself. Its importance comes from relation to Mallows and Placket-Luce, but we discuss them independently. 

%\subsection*{Bradley-Terry Model}

%This model directly models pairwise comparisons. Each agent \( i \) is assigned a latent score \( w_i > 0 \). The probability that \( i \) is ranked above \( j \) is:
%\[
%\Pr(\pi^{-1}(i) < \pi^{-1}(j)) = \frac{w_i}{w_i + w_j}.
%\]
%By choosing \( \frac{w_i}{w_i + w_j} = \frac{1+\delta}{2} \), the Bradley-Terry model adheres to the nudged arrival property.

% Noisy sorting is too general, there is no direct implication, though it could be relevant in some way. 

%\subsection*{Noisy Sorting Models}

%Noisy sorting models introduce randomness to the reference order \( \sigma \). The probability of sampling a ranking \( \pi \) decreases as its Kendall tau distance from \( \sigma \) increases. For a pair \( i, j \) with \( \sigma^{-1}(i) < \sigma^{-1}(j) \), the pairwise probability is:
%\[
%\Pr(\pi^{-1}(i) < \pi^{-1}(j)) \geq \frac{1+\delta}{2},
%\]
%provided the noise level is appropriately low. This ensures compliance with the nudged arrival property.

\subsection*{Thurstone-Mosteller Model \cite{ThurstoneModel}}

In the Thurstone-Mosteller model, each agent \( i \) is assigned an independently drawn latent cardinal value \( v_i \sim \mathcal{N}(\mu_i, s^2) \). Thus, the probability that \( i \) is ranked above \( j \) is independent of all other draws besides the pairwise draws, and is exactly the probability that drawing from $i$'s normal variable exceeds drawing from $j$'s normal variable. The difference of two normal variables is normal by itself, with mean $\mu_i - \mu_j$ (the means difference) and variance $2s^2$ (the sum of variances), and we are interested in the probability that this variable is above $0$. We can normalize and shift the mean, and get that the probability that \( i \) is ranked above \( j \) is:
\[
\Pr(\pi^{-1}(i) < \pi^{-1}(j)) = \Phi\left(\frac{\mu_i - \mu_j}{\sqrt{2}s}\right),
\]
where \( \Phi \) is the CDF of the standard normal distribution $N(0,1)$. We can thus tune $\mu_i, \mu_j$ (with some globally-set $s$) so that for every $i,j$,  \( \Pr(\pi^{-1}(i) < \pi^{-1}(j)) \geq \frac{1+\delta}{2} \), and the nudged arrival property is satisfied. One way to do that is by calculating the constant shift of the mean $\delta\mu$ that satisfies \( \Pr(\pi^{-1}(i) < \pi^{-1}(j)) = \frac{1+\delta}{2}\), and have
\[
\mu_{\pi(i)}^{t} = \mu_{\pi(1)}^{t} + (i-1)\cdot \delta\mu. 
\]

%\omer{constant shifts of the means}
%---

%Together, these examples demonstrate that the nudged arrival property is sufficiently general to subsume a wide range of ranking models while providing a coherent and interpretable foundation for reasoning about orderings.

\section{Omitted Proofs from Section~\ref{sec:extensions}}
\newcommand{\at}[1]{a_{#1}^t}
\paragraph{Additional notation for this section} The analysis in this section makes extensive use of the notation $\at{(q)}$ for $q \in\{1,2\}$ and $t\in [T]$, denoting the arm pulled in session $i$ of round $t$. 
\begin{proof}[Proof of Theorem~\ref{thm:ef1evny+sw}]
The first part of the proof is given in the body of the paper; hence, we move to the second part, i.e., showing that $\sw = (1+\frac{1}{16})T$.

Fix any $t\in[T]$. We analyze the expected sum of rewards obtained in round $t$, $\E{\rt{(1)}+\rt{(2)}}$.
Notice that $\E{\rt{(1)}} = \E{a_1} = \frac{1}{2}$.
As for $\rt{(2)}$, we are uncertain about the arm the algorithm pulls, but can use total expectation:
\begin{align}\label{eq: ef11}
    \E{\rt{(2)}} & = \E{\rt{(2)} \mid \rt{(1)} > \frac{1}{2}}\cdot \prb{\rt{(1)} > \frac{1}{2}} +  \E{\rt{(2)} \mid \rt{(1)} \leq \frac{1}{2}}\cdot \prb{\rt{(1)} \leq \frac{1}{2}}
    \nonumber\\&= \frac{3}{8} + \frac{1}{2}\cdot \E{\rt{(2)} \mid \rt{(1)} \leq \frac{1}{2}},
\end{align}
where we have used Line~\ref{efclin:pull_a1_again} of $\efc$ for replacing $\E{\rt{(2)} \mid \rt{(1)} > \frac{1}{2}}$ with $\frac{3}{4}$, since arm $a_1$ is pulled for the second session as well. Simplifying the term $\E{\rt{(2)} \mid \rt{(1)} \leq \frac{1}{2}}$ and using the fact that $\E{\rt{(r)} \mid \rt{(1)} \leq \frac{1}{2}}=\frac{1}{4}$, we get,
\begin{align*}
    \E{\rt{(2)} \mid \rt{(1)} \leq \frac{1}{2}} &
    = \E{\rt{(2)} \mid \at{(2)} = \at{(1)}, \rt{(1)}} \cdot \prb{\at{(2)} = \at{(1)} \mid \rt{(1)} \leq \frac{1}{2}}
    \\& \qquad + \E{\rt{(2)} \mid \at{(2)} \neq \at{(1)}, \rt{(1)}} \cdot \prb{\at{(2)} \neq \at{(1)} \mid \rt{(1)} \leq \frac{1}{2}}
    \\&= \frac{1}{4} \cdot \prb{\at{(2)} = \at{(1)} \mid \rt{(1)} \leq \frac{1}{2}} + \frac{1}{2} \cdot \prb{\at{(2)} \neq \at{(1)} \mid \rt{(1)} \leq \frac{1}{2}}
    \\& = \frac{1}{4} + \frac{1}{4}\cdot \prb{\at{(2)} \neq \at{(1)} \mid \rt{(1)} \leq \frac{1}{2}}.
\end{align*}
Consequently, all that is left is to understand how often $\efc$ pulls the second arm when the first arm yields a low reward. Using Proposition~\ref{prop:ef1 open arm}, we obtain
\[
\E{\rt{(2)} \mid \rt{(1)} \leq \frac{1}{2}} =
\frac{1}{4} + \frac{1}{4} \cdot \prb{\at{(2)} \neq \at{(1)} \mid \rt{(1)} \leq \frac{1}{2}} \geq \frac{3}{8}.
\]
Using the above inequality and Equation~\eqref{eq: ef11}, we get
\[
\E{\rt{(2)}} \geq \frac{3}{8} + \frac{1}{2}\cdot \frac{3}{8}= \frac{9}{16}.
\]
Since this holds for any arbitrary $t$, by summing over all rounds, we get
\[
SW(EF1) = \E{\sum_{t=1}^{T}{ \rt{(1)} + \rt{(2)} }} = \sum_{t=1}^{T}{ \E{ \rt{(1)} + \rt{(2)} } } \geq \sum_{t=1}^{T}{ \frac{1}{2} + \frac{9}{16}} = \left( 1+ \frac{1}{16} \right)T.
\]
This concludes the proof of Theorem~\ref{thm:ef1evny+sw}.
\end{proof}

\begin{proof}[Proof of Proposition~\ref{prop:ef1 uni dominance}]
We prove the claim with induction over the round index $t$.
The base step, i.e., $t=0$, is straightforward. Fix any $x\in [0,1]$, and observe that
\[
\prb{\env^0 \leq x} = \prb{0 \leq x} = 1 \geq x.
\]
We move forward to the inductive step. Assume the claim holds for round $t-1$, and let us prove the claim for~$t$. First, notice that if $\at{(2)} = \at{(1)}$, then $\env^t=\env^{t-1}$.
Based on the inductive assumption, the distribution of $\env^{t-1}$ is stochastically dominated by $\uni{0,1}$, and thus so is the distribution of $\env^t$.

Otherwise, from here on we assume $\at{(2)} \neq \at{(1)}$. We continue with an extensive case analysis. We define the following six events $A_1,\dots, A_6$. Each event consists of the conditions that cause the algorithm to pull a different arm in the second session and the outcome of that round:
\begin{align*}
    &A_1 := \left( \rt{(1)} \leq \frac{1}{2} \right) \wedge \left(R^{t-1}_{(1)} = R^{t-1}_{(2)}\right) \wedge  \left( R^{t-1}_{(1)} + \rt{(1)} \geq R^{t-1}_{(2)} + \rt{(2)} \right), \\
    &A_2 := \left( \rt{(1)} \leq \frac{1}{2} \right) \wedge \left(R^{t-1}_{(1)} = R^{t-1}_{(2)}\right) \wedge  \left( R^{t-1}_{(2)} + \rt{(2)} > R^{t-1}_{(1)} + \rt{(1)} \right), \\
    &A_3 := \left( \rt{(1)} \leq \frac{1}{2} \right) \wedge \left(R^{t-1}_{(1)} > R^{t-1}_{(2)} \right) \wedge \left( R^{t-1}_{(1)} - R^{t-1}_{(2)} \leq 1 - \rt{(1)} \right) \wedge \left( R^{t-1}_{(1)} + \rt{(1)} \geq R^{t-1}_{(2)} + \rt{(2)} \right), \\
    &A_4 := \left( \rt{(1)} \leq \frac{1}{2} \right) \wedge \left(R^{t-1}_{(1)} > R^{t-1}_{(2)} \right) \wedge \left( R^{t-1}_{(1)} - R^{t-1}_{(2)} \leq 1 - \rt{(1)} \right) \wedge \left( R^{t-1}_{(2)} + \rt{(2)} > R^{t-1}_{(1)} + \rt{(1)} \right), \\
    &A_5 := \left( \rt{(1)} \leq \frac{1}{2} \right) \wedge \left( R^{t-1}_{(2)} > R^{t-1}_{(1)} \right) \wedge  \left( R^{t-1}_{(2)} - R^{t-1}_{(1)} \leq \rt{(1)} \right) \wedge \left( R^{t-1}_{(1)} + \rt{(1)} \geq R^{t-1}_{(2)} + \rt{(2)} \right), \\
    &A_6 := \left( \rt{(1)} \leq \frac{1}{2} \right) \wedge \left( R^{t-1}_{(2)} > R^{t-1}_{(1)} \right) \wedge  \left( R^{t-1}_{(2)} - R^{t-1}_{(1)} \leq \rt{(1)} \right) \wedge \left( R^{t-1}_{(2)} + \rt{(2)} > R^{t-1}_{(1)} + \rt{(1)} \right).
\end{align*}
Notice that 
\begin{observation}\label{obs:partition}
Given $\at{(2)} \neq \at{(1)}$, the events $A_1, \dots, A_6$ partition the space of all options for $R^{t-1}_{(1)}, R^{t-1}_{(2)}, r^t_{(1)}, r^t_{(2)}$.
\end{observation}
Equipped with Observation~\ref{obs:partition}, we turn to analyze $\env^t$ under $A_1, \dots, A_6$. Fix any arbitrary $x \in [0,1]$.
%In each case $i$, $i\in \{1,\dots,6\}$ we condition the random variables $\env_t,R^{t-1}_1, R^{t-1}_2, r^t_{(1)}, r^t_{(2)}$ on $A_i$.
%; hence, under this event, $r^t_{(1)} \sim \uni{0,\frac{1}{2}}$ 
\begin{itemize}
    \item Case $A_1$.
    Under the conditions of event $A_1$ we have that $\rt{(1)} \geq  \rt{(2)}$.
    Recall that $\rt{(2)} \sim \uni{0,1}$, but considering the latter we know that $ \rt{(2)}\mid A_1 \sim \uni{0,\rt{(1)}}$.
    Given that $R^{t-1}_{(1)} + \rt{(1)} \geq R^{t-1}_{(2)} + \rt{(2)}$, we know the envy at the end of round $t$ is exactly $\env^t = R^{t-1}_{(1)} + \rt{(1)} - R^{t-1}_{(2)} - \rt{(2)} = \rt{(1)} -\rt{(2)}$;
    hence, $\env^{t}\mid A_1 \sim \uni{\rt{(1)} - \rt{(1)},\rt{(1)} - 0}$, i.e., $\env^{t}\mid A_1 \sim \uni{0,\rt{(1)}}$. Therefore,
    \begin{align*}
        \prb{\env^t \leq x \mid A_1} & = \prb{\uni{0, r^t_{(1)}}\leq x \mid r^t_{(1)} \leq \frac{1}{2}} \geq \prb{\uni{0, \frac{1}{2}} \leq x}
        \\& \geq \prb{\uni{0, 1} \leq x} = x.
    \end{align*}
    %Because $\rt{(1)} \leq \frac{1}{2}$, in the worst case $\env^{t}\mid A_1 \sim \uni{0,\frac{1}{2}}$.

    \item Case $A_2$.
    This case is similar to that of $A_1$, only now $ \rt{(2)}\mid A_2 \sim \uni{\rt{(1)}, 1}$ and $\env^t = \rt{(2)} -\rt{(1)}$, resulting with $\env^{t}\mid A_2 \sim \uni{\rt{(1)} - \rt{(1)},1 -\rt{(1)}}$, i.e., $\env^{t}\mid A_2 \sim \uni{0,1 -\rt{(1)}}$.
    Hence,
    \begin{align*}
        \prb{\env^t \leq x \mid A_2 } \geq \prb{\uni{0, 1} \leq x} = x.
    \end{align*}

    \item Case $A_3$.
    Under the conditions of $A_3$ we have that the envy after $t$ rounds is exactly $\env^t = R^{t-1}_{(1)} + \rt{(1)} -\left(R^{t-1}_{(2)} + \rt{(2)}\right)$.
    Since $R^{t-1}_{(1)} + \rt{(1)} \geq R^{t-1}_{(2)} + \rt{(2)}$, $\rt{(2)}$ is now a uniform random variable between $0$ and the minimum between $\left\{1,  R^{t-1}_{(1)} + \rt{(1)} - R^{t-1}_{(2)}\right\}$.
    Due to the guarantee $R^{t-1}_{(1)} - R^{t-1}_{(2)} \leq 1 - \rt{(1)}$ we can finally see that
    \[
    - \rt{(2)} \mid A_3 \sim \uni{-\left(R^{t-1}_{(1)} + \rt{(1)} - R^{t-1}_{(2)}\right), 0};
    \]
    thus,
    \[
    \env^t \mid A_3 \sim \uni{R^{t-1}_{(1)} + \rt{(1)} -R^{t-1}_{(2)} -\left(R^{t-1}_{(1)} + \rt{(1)} - R^{t-1}_{(2)}\right),  R^{t-1}_{(1)} + \rt{(1)} -R^{t-1}_{(2)}}.
    \]
    Finally,
    \begin{align*}
        \prb{\env^t \leq x | A_3} & = \prb{\uni{0, R^{t-1}_{(1)} - R^{t-1}_{(2)}+ \rt{(1)}} \leq x | A_3}
        \\& \geq \prb{\uni{0, 1} \leq x} = x.
    \end{align*}

    \item Case $A_4$.
    Under the conditions of $A_4$ we have that $\rt{(2)}$ is a uniform random variable distributed between $R^{t-1}_{(1)} - R^{t-1}_{(2)} + \rt{(1)}$ and $1$.
    The envy after round $t$ is exactly $R^{t-1}_{(2)} +\rt{(2)}- R^{t-1}_{(1)}-\rt{(1)}$ and thus it is a uniform random variable between $R^{t-1}_{(2)} - R^{t-1}_{(1)}-\rt{(1)} +\left( R^{t-1}_{(1)} - R^{t-1}_{(2)} + \rt{(1)} \right)$ and $R^{t-1}_{(2)} - R^{t-1}_{(1)}-\rt{(1)} +1$.
    I.e., $\env^t \mid A_4 \sim \uni{0,R^{t-1}_{(2)} - R^{t-1}_{(1)}-\rt{(1)} +1}$. Recall $A_4$ suggests $0 > R^{t-1}_{(2)} -R^{t-1}_{(1)} $; finally,
    \begin{align*}
        \prb{\env^t \leq x | A_4} & = \prb{\uni{0, R^{t-1}_{(2)} - R^{t-1}_{(1)}-\rt{(1)} +1} \leq x | A_4}  \\& \geq
        \prb{\uni{0, 0 - \rt{(1)} +1} \leq x } \geq \prb{\uni{0, 1} \leq x} = x,
    \end{align*}
    as $\rt{(1)}\geq 0$.
    Note that under $A_4$ it holds that $R^{t-1}_{(1)} - R^{t-1}_{(2)} \leq 1 - \rt{(1)} $ and thus $0 \leq R^{t-1}_{(2)} - R^{t-1}_{(1)}-\rt{(1)} +1$ almost surely.
    
    \item Case $A_5$.
    Under the conditions of $A_5$, similarly to $A_3$, the envy at the end of round $t$ is exactly $\env^t = R^{t-1}_{(1)} + \rt{(1)} - R^{t-1}_{(2)} + \rt{(2)}$ and
    \[
    - \rt{(2)} \mid A_5 \sim \uni{-\left(R^{t-1}_{(1)} + \rt{(1)} - R^{t-1}_{(2)}\right), 0};
    \]
    thus,
    \[
    \env^t \mid A_5 \sim \uni{R^{t-1}_{(1)} + \rt{(1)} -R^{t-1}_{(2)} -\left(R^{t-1}_{(1)} + \rt{(1)} - R^{t-1}_{(2)}\right),  R^{t-1}_{(1)} + \rt{(1)} -R^{t-1}_{(2)}}.
    \]
    Finally,
    \begin{align*}
        \prb{\env^t \leq x | A_5} & = \prb{\uni{0, R^{t-1}_{(1)} - R^{t-1}_{(2)}+ \rt{(1)}} \leq x | A_5}
        \\& \geq \prb{\uni{0, 0 + \rt{(1)}} \leq x | A_5}
        \geq \prb{\uni{0, \frac{1}{2}} \leq x} 
        \\& \geq \prb{\uni{0, 1} \leq x} = x.
    \end{align*}

    \item Case $A_6$.
    Under the conditions of $A_6$, similarly to $A_4$, the envy at the end of round $t$ is exactly $\env^t = R^{t-1}_{(2)} + \rt{(2)} - R^{t-1}_{(1)} - \rt{(1)}$ and
    \[
    \rt{(2)} \mid A_6 \sim \uni{R^{t-1}_{(1)}+\rt{(1)} -R^{t-1}_{(2)}, 1};
    \]
    thus,
    \[
    \env^t \mid A_6 \sim \uni{R^{t-1}_{(2)} - R^{t-1}_{(1)} - \rt{(1)} + R^{t-1}_{(1)}+\rt{(1)} -R^{t-1}_{(2)}, R^{t-1}_{(2)} - R^{t-1}_{(1)} - \rt{(1)} + 1}.
    \]
    Finally,
    \begin{align*}
        \prb{\env^t \leq x | A_6} & = \prb{\uni{0,R^{t-1}_{(2)} - R^{t-1}_{(1)} - \rt{(1)} + 1} \leq x | A_6}
        \\& \geq \prb{\uni{0,1} \leq x} = x,
    \end{align*}
    where the inequality holds due to $R^{t-1}_{(2)} - R^{t-1}_{(1)} \leq \rt{(1)}$.    
\end{itemize}
We have shown that the inductive step holds under all cases; thereby, the proof of Proposition~\ref{prop:ef1 uni dominance} is complete.
\end{proof}
\begin{proof}[Proof of Proposition~\ref{prop:ef1 open arm}]
We prove the statement using case analysis. We partition the space of events $a^t_{(2)} \neq a^t_{(1)}$ conditioning on $\rt{(1)} \leq \frac{1}{2}$:
\begin{align*}
    &B_1 := R^{t-1}_{(1)} = R^{t-1}_{(2)} \\
    &B_2 := \left(R^{t-1}_{(1)} > R^{t-1}_{(2)} \right) \wedge \left( R^{t-1}_{(1)} - R^{t-1}_{(2)} \leq 1 - \rt{(1)} \right) \\
    &B_3 := \left( R^{t-1}_{(2)} > R^{t-1}_{(1)} \right) \wedge  \left( R^{t-1}_{(2)} - R^{t-1}_{(1)} \leq \rt{(1)} \right).
\end{align*}
Therefore
\begin{align*}
    \prb{a^t_{(2)} \neq a^t_{(1)} \mid \rt{(1)} \leq \frac{1}{2}} & = \prb{ B_1\vee B_2 \vee B_3  \mid \rt{(1)} \leq \frac{1}{2}}
    \\& = \prb{B_1 \mid \rt{(1)} \leq \frac{1}{2} } + \prb{B_2 \mid \rt{(1)} \leq \frac{1}{2} } + \prb{B_3 \mid \rt{(1)} \leq \frac{1}{2} }.
\end{align*}
We prove each part separately, beginning with the event $B_1$.

Since the distributions of the rewards are continuous, event $B_1$ occurs if and only if until round $t$ both agents receive the same rewards from the same arm.
In this case, the algorithm pulls the same arm in both sessions if it yields a reward greater than $\frac{1}{2}$.
Therefore, we must have
\[
    \prb{r_{(1)}^\tau  = r_{(2)}^\tau} = \prb{r_{(1)}^\tau  > \frac{1}{2}}
\]
for all $\tau < t$; hence,
\begin{align}\label{B1}
\prb{B_1 \mid \rt{(1)} > \frac{1}{2}} =\prb{B_1} = \prb{\forall \tau<t : r_{(2)}^\tau > \frac{1}{2}} = \left(\frac{1}{2}\right)^{t-1}.
\end{align}

Next, we examine event $B_2$. Using Bayes formula,
\begin{align*}
    &\prb{B_2 \mid \rt{(1)} \leq \frac{1}{2} } =
    \prb{ \left(R^{t-1}_{(1)} > R^{t-1}_{(2)} \right) \wedge \left( R^{t-1}_{(1)} - R^{t-1}_{(2)} \leq 1 - \rt{(1)} \right) \mid \rt{(1)} \leq \frac{1}{2} }
    \\& = \prb{R^{t-1}_{(1)} - R^{t-1}_{(2)} \leq 1 - \rt{(1)} \mid R^{t-1}_{(1)} > R^{t-1}_{(2)}, \rt{(1)} \leq \frac{1}{2} }\cdot \prb{R^{t-1}_{(1)} > R^{t-1}_{(2)} \mid \rt{(1)} \leq \frac{1}{2}}.
\end{align*}
Notice that given $\rt{(1)} \leq \frac{1}{2}$, the random variable $1- \rt{(1)}$ is $\uni{\frac{1}{2}, 1}$ distributed. Similar arguments holds for $\rt{(1)} \mid \rt{(1)} > \frac{1}{2}$; thus, 
\begin{align*}
&\prb{R^{t-1}_{(1)} - R^{t-1}_{(2)} \leq 1 - \rt{(1)} \mid R^{t-1}_{(1)} > R^{t-1}_{(2)}, \rt{(1)} \leq \frac{1}{2} }
\\&=
\prb{R^{t-1}_{(1)} - R^{t-1}_{(2)} \leq \rt{(1)} \mid R^{t-1}_{(1)} > R^{t-1}_{(2)}, \rt{(1)} > \frac{1}{2} }.
\end{align*}
Simplifying the above,
\begin{align}\label{B2}
    & \prb{B_2 \mid \rt{(1)} \leq \frac{1}{2} }
    \nonumber\\&
    = \prb{R^{t-1}_{(1)} - R^{t-1}_{(2)} \leq \rt{(1)} \mid R^{t-1}_{(1)} > R^{t-1}_{(2)}, \rt{(1)} > \frac{1}{2} }\cdot \prb{R^{t-1}_{(1)} > R^{t-1}_{(2)} \mid \rt{(1)} \leq \frac{1}{2}}\nonumber \\
    & = \prb{R^{t-1}_{(1)} - R^{t-1}_{(2)} \leq \rt{(1)} \mid R^{t-1}_{(1)} > R^{t-1}_{(2)}, \rt{(1)} > \frac{1}{2} }\cdot \prb{R^{t-1}_{(1)} > R^{t-1}_{(2)}}
    .
\end{align}

As for event $B_3$, recall that the arrival order is uniform. As a result, $R^{t-1}_{(1)}, R^{t-1}_{(2)}$ are independent in $\rt{(1)}$. Leveraging this fact,
\begin{align}\label{B3}
    & \prb{B_3 \mid \rt{(1)} \leq \frac{1}{2} } =
    \prb{\left( R^{t-1}_{(2)} > R^{t-1}_{(1)} \right) \wedge  \left( R^{t-1}_{(2)} - R^{t-1}_{(1)} \leq \rt{(1)} \right) \mid \rt{(1)}> \frac{1}{2}}\nonumber \\
    & =  \prb{\left( R^{t-1}_{(1)} > R^{t-1}_{(2)} \right) \wedge  \left( R^{t-1}_{(1)} - R^{t-1}_{(2)} \leq \rt{(1)} \right) \mid \rt{(1)}> \frac{1}{2}}\nonumber \\
    & = \prb{R^{t-1}_{(1)} - R^{t-1}_{(2)} \leq \rt{(1)} \mid R^{t-1}_{(1)} > R^{t-1}_{(2)}, \rt{(1)} \leq \frac{1}{2} }\cdot \prb{R^{t-1}_{(1)} > R^{t-1}_{(2)} \mid \rt{(1)} \leq \frac{1}{2}}\nonumber\\
    & = \prb{R^{t-1}_{(1)} - R^{t-1}_{(2)} \leq \rt{(1)} \mid R^{t-1}_{(1)} > R^{t-1}_{(2)}, \rt{(1)} \leq \frac{1}{2} }\cdot \prb{R^{t-1}_{(1)} > R^{t-1}_{(2)}},
\end{align}
where the last two equalities hold from the same arguments as in the analysis of event $B_2$. Combining Equalities \eqref{B2} and \eqref{B3}, we get
\begin{align*}
 \prb{B_2 \mid \rt{(1)} \leq \frac{1}{2} } & + \prb{B_3 \mid \rt{(1)} \leq \frac{1}{2} }\\ &= \prb{R^{t-1}_{(1)} > R^{t-1}_{(2)}} \cdot\left(
 \prb{R^{t-1}_{(1)} - R^{t-1}_{(2)}   \leq \rt{(1)} \mid R^{t-1}_{(1)} > R^{t-1}_{(2)}, \rt{(1)} > \frac{1}{2} } \right. \\& + \left.
  \prb{R^{t-1}_{(1)} - R^{t-1}_{(2)} \leq \rt{(1)} \mid R^{t-1}_{(1)} > R^{t-1}_{(2)}, \rt{(1)} \leq \frac{1}{2} }\right) \\ & = 
  \frac{1}{2} \left(1 - \frac{1}{2^{t-1}} \right) \cdot\left(
 \prb{R^{t-1}_{(1)} - R^{t-1}_{(2)}   \leq \rt{(1)} \mid R^{t-1}_{(1)} > R^{t-1}_{(2)}, \rt{(1)} > \frac{1}{2} } \right. \\& + \left.
  \prb{R^{t-1}_{(1)} - R^{t-1}_{(2)} \leq \rt{(1)} \mid R^{t-1}_{(1)} > R^{t-1}_{(2)}, \rt{(1)} \leq \frac{1}{2} }\right),
\end{align*}
where the last equality is due to Equation~\eqref{B1}. Next, notice that 
\begin{align*}
\frac{1}{2} \left(1 - \frac{1}{2^{t-1}} \right) &\left(
\prb{R^{t-1}_{(1)} - R^{t-1}_{(2)}   \leq \rt{(1)} \mid R^{t-1}_{(1)} > R^{t-1}_{(2)}, \rt{(1)} > \frac{1}{2} } \right. \\+ & \left.
\prb{R^{t-1}_{(1)} - R^{t-1}_{(2)} \leq \rt{(1)} \mid R^{t-1}_{(1)} > R^{t-1}_{(2)}, \rt{(1)} \leq \frac{1}{2} }\right)\\=
\frac{1}{2} \left(1 - \frac{1}{2^{t-1}} \right) &\left(
\prb{R^{t-1}_{(1)} - R^{t-1}_{(2)}   \leq \rt{(1)} \mid R^{t-1}_{(1)} > R^{t-1}_{(2)}, \rt{(1)} > \frac{1}{2} } \frac{\Pr\left(r_{(1)}^t > \frac{1}{2}\Big| R_1^{t-1} > R_2^{t-1} \right)}{\Pr\left(r_{(1)}^t > \frac{1}{2}\Big| R_1^{t-1} > R_2^{t-1} \right)} \right. \\+ & \left.
\prb{R^{t-1}_{(1)} - R^{t-1}_{(2)} \leq \rt{(1)} \mid R^{t-1}_{(1)} > R^{t-1}_{(2)}, \rt{(1)} \leq \frac{1}{2} } \frac{\Pr\left(r_{(1)}^t\leq \frac{1}{2}\Big| R_1^{t-1} > R_2^{t-1} \right)}{\Pr\left(r_{(1)}^t\leq \frac{1}{2}\Big| R_1^{t-1} > R_2^{t-1} \right)} \right)
\\ = \frac{1}{2}  \left(1 - \frac{1}{2^{t-1}} \right) &\cdot \frac{1}{\frac{1}{2}} \cdot 
\prb{R^{t-1}_{(1)} - R^{t-1}_{(2)} \leq \rt{(1)} \mid R^{t-1}_{(1)} > R^{t-1}_{(2)} },
\end{align*}
where the last equation is based on the law of total probability and the fact that 
\[
\prb{\rt{(1)} > \frac{1}{2} \mid R^{t-1}_{(1)} > R^{t-1}_{(2)}} = \prb{\rt{(1)} \leq \frac{1}{2} \mid R^{t-1}_{(1)} > R^{t-1}_{(2)}} = \frac{1}{2}.
\]
Combining the latter with Equation~\eqref{B1}, we have
\begin{align*}
\prb{a^t_{(2)} \neq a^t_{(1)} \mid \rt{(1)} \leq \frac{1}{2}} & = \left(\frac{1}{2}\right)^{t-1}+ \left(1 - \frac{1}{2^{t-1}} \right)  
\prb{R^{t-1}_{(1)} - R^{t-1}_{(2)} \leq \rt{(1)} \mid R^{t-1}_{(1)} > R^{t-1}_{(2)} }\\
& = \left(\frac{1}{2}\right)^{t-1}+ \left(1 - \frac{1}{2^{t-1}} \right)  
\prb{\abs{R^{t-1}_{(1)} - R^{t-1}_{(2)}} \leq \rt{(1)} \mid R^{t-1}_{(1)} > R^{t-1}_{(2)} }\\
& = \left(\frac{1}{2}\right)^{t-1}+ \left(1 - \frac{1}{2^{t-1}} \right) 
\prb{\env^{t-1} \leq \rt{(1)} }\geq \prb{\env^{t-1} \leq \rt{(1)} }.
\end{align*}
To finish the proof we use Proposition~\ref{prop:ef1 uni dominance}, which implies that
\begin{align*}
    \prb{\env^{t-1} \leq \rt{(1)} } & = \int_{0}^{1} \prb{\env^{t-1} \leq u}\cdot f_{\uni{0,1}} \,du
    \geq \int_{0}^{1} u\cdot f_{\uni{0,1}} \,du
    \\&=\E{\uni{0,1}} = \frac{1}{2};
\end{align*}
thus,
\begin{align*}
\prb{a^t_{(2)} \neq a^t_{(1)} \mid \rt{(1)} \leq \frac{1}{2}} \geq 
\prb{\env^{t-1} \leq \rt{(1)} } \geq
\frac{1}{2}.
\end{align*}
This concludes the proof of Proposition~\ref{prop:ef1 open arm}.
\end{proof}

\section{Average Envy}
\label{sec: avg envy}
In this section, we examine another way to define envy: The average reward disparity between the agents. We define the \emph{average envy}, denoted $\envavg^T$, as 
\[
\envavg^T = \frac{1}{\binom{N}{2}}\sum_{1\leq i<j\leq N}{\abs{\env_{i,j}^T}}.
\]
For the special cases of $N=2$ agents, the definition of maximal envy $\env$ and average envy $\envavg$ coincide. 


Since $\envavg^t \leq \env^t$ for all $t$  almost surely, any upper bound on the maximal envy can be applied to the average envy. Particularly, Theorem~\ref{thm: uni upper-bound} provides an immediate upper bound on $\E{\envavg^T}$ of $O\left(\sqrt{\ln (N) \sum^T_{t=1} \var{\dift}} \right)$. Using a slightly more careful analysis, we can eliminate the $\sqrt{\ln{(N)}}$ factor.
\begin{proposition}\label{prop: avg upper-bound}
When executing any algorithm, it holds that
\[\E{\max_{1\leq t \leq T} \envavg^t (\uniord)} \leq 2\sqrt{\ln{(N)} \sum^{T}_{t=1}{\var{\dift}} }.\]
\end{proposition}
\begin{proof}[Proof of Proposition~\ref{prop: avg upper-bound}]
Much like the proof of Theorem~\ref{thm: uni upper-bound}, this proof is based on the fact that $\dift$ are subgaussian random variables. From the linearity of expectation, we get
\begin{align}\label{avg 1}
\E{\max_{1\leq t \leq T} \envavg^t } &=
\E{\max_{1\leq t \leq T} \frac{1}{\binom{N}{2}}\sum_{1\leq i<j\leq N}{\abs{\env_{i,j}^t}}} 
\leq \E{\frac{1}{\binom{N}{2}}\sum_{1\leq i<j\leq N}{\max_{1\leq t \leq T} 
 \abs{\env_{i,j}^t}}} \nonumber\\
&=\frac{1}{\binom{N}{2}}\sum_{1\leq i<j\leq N}\E{\max_{1\leq t \leq T}  \abs{\sum^t_{\tau=1}{\adif{i}{j}^\tau}} }.
\end{align}
For every pair of agents $i,j \in [N]$, it holds that
\begin{align}\label{avg 2}
\E{\max_{1\leq t \leq T}  \abs{\sum^t_{\tau=1}{\adif{i}{j}^\tau}} } =
\E{\max_{1\leq t \leq T, \sigma\in \{-1,1\}}\left\{\sigma\sum^t_{\tau=1}{\adif{i}{j}^\tau}\right\}} \leq
\sqrt{2\ln{(2)}\sum^T_{t=1}{\var{\dift}}},
\end{align}
where the last inequality is due to Proposition~\ref{prop:envy is good SG} and Claim~\ref{claim: sg max}. By combining Inequalities~\eqref{avg 1} and \eqref{avg 2} we can conclude that
\begin{align*}
\E{\max_{1\leq t \leq T} \envavg^t } \leq
\frac{1}{\binom{N}{2}}\sum_{1\leq i<j\leq N}{\sqrt{2\ln{(2)}\sum^T_{t=1}{\var{\dift}}}} =
\sqrt{2\ln{(2)}\sum^T_{t=1}{\var{\dift}}},
\end{align*}
which concludes the proof of Proposition~\ref{prop: avg upper-bound}. 
\end{proof}
Next, we craft a lower bound for the average envy.
\begin{proposition}\label{prop: mp avg lower-bound}
    For a large enough $T$, a sufficiently random execution with a symmetric, memory-free algorithm yields
    \[\E{\envavg^T} \geq c\sqrt{ \sum^{T}_{t=1}{\var{\dif^t}}},\]
    where $c> 0$ is a global constant.    
\end{proposition}


\begin{proof}[Proof of Proposition~\ref{prop: mp avg lower-bound}]
The proof of Proposition~\ref{prop: mp avg lower-bound} is almost identical to the proof of Theorem~\ref{thm: uni lower-bound}. In that proof, we bounded the (maximal) envy from below using the envy between a specific couple of agents.
Since, the algorithm is symmetric, the bound we showed is valid for every two agents;
thus, for every $i,j$ it holds that
\begin{align*}
\E{\env_{i,j}^T} \geq \frac{A_1}{2}\sqrt{ \sum^{T}_{t=1}{\var{\dif^t}} },
\end{align*}
where $A_1$ is the constant from Theorem~\ref{thm:BDG}. Using linearity of expectation, we get
\begin{align*}
\E{\envavg^T} & =
\E{\frac{1}{\binom{N}{2}}\sum_{i,j \in [N]^2}{\env_{i,j}^T}} =
\frac{1}{\binom{N}{2}}\sum_{i,j \in [N]^2}{\E{\env_{i,j}^T}}
\geq
\frac{1}{\binom{N}{2}}\sum_{i,j \in [N]^2} {\frac{A_1}{2}\sqrt{ \sum^{T}_{t=1}{\var{\dif^t}} }}
\\&=
\frac{A_1}{2}\sqrt{ \sum^{T}_{t=1}{\var{\dif^t}} }.
\end{align*}
This concludes the proof of Proposition~\ref{prop: mp avg lower-bound}.
\end{proof}
We finalize this section by mentioning that the upper bound on nudged arrival and maximal envy holds trivially for the average envy due to the fact that $\envavg^t \leq \env^t$ for all $t$  almost surely. Future work could seek a tighter bound for the average envy.


\section{Socially Optimal Algorithms}\label{appendix:sociallyopt}
In this section, we consider the task of devising socially optimal algorithms. First, in Subsection~\ref{subsec:sw N=2}, we address the two-agent case. Then, in Subsection~\ref{subsec:sopt for N>2}, we develop algorithms for the $N>2$ case.

To ease readability, we make the assumption that reward distributions are stationary, i.e., $\mathcal{D}^1_i, \mathcal{D}^2_i, \dots \mathcal{D}^T_i$ are identical for every arm $a_i \in A$. Consequently,  $X^1_i, X^2_i, \dots X^T_i$ are i.i.d. and we drop the superscript. We stress that our results can also be easily extended to the non-stationary case. Furthermore, we let $\mu_i = \mathbb{E}[X_i^t]$ denote the expected reward of arm $a_i$.

\subsection{Social Welfare for $N=2$}
\label{subsec:sw N=2}

\begin{algorithm}[t]
\caption{Two-agents Socially Optimal Algorithm ($\sopt$)}
\label{alg: sopt}
\LinesNumbered
\DontPrintSemicolon 
\KwIn{horizon $T$, reward distributions $\mathcal{D}_1, \ldots, \mathcal{D}_K$}
Compute $(\is, \js)$ such that
\label{alg: sopt compute}
\begin{align}\label{eq: picking i,j star}
(\is, \js) \in \argmax_{(i,j) \in A^2} \left\{ \mu_i  + \prb{X_i < \mu_j}\mu_j+\prb{X_i \geq \mu_j}\E{X_i \mid X_i \geq \mu_j}  \right\}.
\end{align}\\
\For{round $t = 1$ to $T$}{\label{alg: sopt for}
    Pull $a_{\is}$ \label{alg: sopt 3}\\
    \lIf{$x^t_\is \geq \mu_\js$}{
        Pull $a_{\is}$ \label{alg: sopt if}
    }
    \lElse{
        Pull $a_{\js}$ \label{alg: sopt else}
    }
}
\end{algorithm}
In this section, we design $\sopt$, a socially optimal algorithm for the two-agent case, which we implement in Algorithm~\ref{alg: sopt}. $\sopt$ has Bayesian information, as it receives the reward distributions as inputs. In Line~\ref{alg: sopt compute}, it selects two arms $a_{\is}, a_{\js}$ according to Equation~\eqref{eq: picking i,j star}. As we prove formally, this selection maximizes $\E{{r^t_{(1)}}+{r^t_{(2)}}}$ for any $t$. Arms $a_{\is}, a_{\js}$ are the only arms the algorithm pulls during its execution.

In Line~\ref{alg: sopt for}, $\sopt$ begins interacting with the agents for $T$ rounds.
Notice $\sopt$ does not address the arrival of the agents at all: While the arrival function is crucial for measuring envy, it does not influence the SW.

$\sopt$ pulls arm $a_{\is}$ for the agent that arrives in the first session, 
and observes the realized reward $x^t_{\is}$ (Line~\ref{alg: sopt 3}). In Lines~\ref{alg: sopt if}--\ref{alg: sopt else}, $\sopt$ decides whether to pull the same arm in the second session or $a_{\js}$ instead, based on the realized $x^t_{\is}$:
If $x^t_{\is} \geq \mu_{\js}$ (Line~\ref{alg: sopt if}), i.e., we expect the reward of $a_{\js}$ to be less than or equal to the observed reward, $\sopt$ pulls $a_{\is}$ again.
Otherwise (Line~\ref{alg: sopt else}), we expect the reward of $a_{\js}$  to be greater than that of $a_{\is}$, so the algorithm pulls arm $a_{\js}$. Next, we prove the optimality of $\sopt$.

Before we prove the optimality of $\sopt$, present two propositions that assist in understanding its crux.
\begin{proposition}\label{prop:is}
$\sopt$ does not always pull the arm with the highest expected reward in the first session. That is, $\is$ is not necessarily $\argmax_{i\in \{1, \ldots, K \}}{\mu_i}$.
\end{proposition}

\begin{proof}[Proof of Proposition~\ref{prop:is}]
We show an example satisfying $\is \neq \argmax_{i\in \{1, \ldots, K \}}{\mu_i}$.
Consider $K=2$ arms, with the following distributions.
\begin{align*}
    X_1 \sim
    \begin{cases}
    0.75 & w.p. \ \  0.5 \\
    0.55 & w.p.\  \ 0.5
    \end{cases},
    X_2 \sim
    \begin{cases}
    1 & w.p. \ \  0.6 \\
    0 & w.p.\  \ 0.4
    \end{cases}.
\end{align*}
Observe that $\mu_1 = 0.65 > \mu_2=0.6$. Yet,
\begin{align*}
\mu_1  +\mu_2 \prb{X_1 < \mu_2}+\E{X_1 \mid X_1 \geq \mu_2}  \prb{X_1 \geq \mu_2}=
0.65  + 0.6 \cdot 0.5+ 0.75 \cdot 0.5 = 1.325,
\end{align*}
whereas,
\begin{align*}
\mu_2  +\mu_1 \prb{X_2 < \mu_1}+\E{X_2 \mid X_2 \geq \mu_1}  \prb{X_2 \geq \mu_1}=
0.6  + 0.65 \cdot 0.4 + 1 \cdot 0.6 = 1.46.
\end{align*}
Consequently, $\sopt$ chooses $(\is,\js) = (2,1)$.
This concludes the proof of Proposition~\ref{prop:is}.
\end{proof}
To get the intuition behind Proposition~\ref{prop:is}, recall that with the help of the information exposed by the first agent, the algorithm can make a better decision in the second session.
Therefore, we must find the perfect trade-off between the first agent's welfare (exploitation) and the leverage of the information they provide (exploration).
In contrast, we expect nothing but pure exploitation in the second session.
\begin{proposition}\label{prop:js}
Arm $a_\js$ has the highest expected reward among the remaining arms. I.e., $\js = \argmax_{j\in K\setminus\{\is\}}{\mu_j}$.
\end{proposition}
In other words, in the second session, the algorithm makes the choice that is the most rewarding.
\begin{proof}[Proof of Proposition~\ref{prop:js}]
Let $(\is, \js)$ be a pair of arms that maximizes Equation~\eqref{eq: picking i,j star}.
Suppose, for the sake of contradiction, that there exists ${j'}\notin \{\is,\js\}$ such that $\mu_{j'} > \mu_{\js}$;
thus, $\E{\max{\{X_{\is}, \mu_{{j'}}\}}} \geq \E{\max{\{X_{\is}, \mu_{\js}\}}}$.
I.e.,
\begin{align*}
    \E{X_{\is}} + \E{\max{\{X_{\is}, \mu_{{j'}}\}}} \geq \E{X_{\is}} + \E{\max{\{X_{\is}, \mu_{\js}\}}},
\end{align*}
where equality can occur if and only if $\prb{\max{\{X_{\is}, \mu_{\js}\}} = \mu_{\js}} = 0$.
In this case, the algorithm always pulls arm $a_{\is}$ in the second session, and arm $a_{\js}$ is irrelevant.
Hence, we assume strong inequality.
Notice that the left-hand side is exactly Equation~\eqref{eq: picking i,j star} with $(i,j) = (\is, {j'})$, and the right-hand side is exactly Equation~\eqref{eq: picking i,j star} with $(i,j) = (\is, \js)$.
Hence, we have obtained a contradiction to $(\is, \js)$ being a pair that maximizes Equation~\eqref{eq: picking i,j star}. This concludes the proof of Proposition~\ref{prop:js}.
\end{proof}
We are ready to prove the optimality of $\sopt$.
\begin{theorem}\label{thm:sopt is opt}
Fix any instance with $N=2$ and arbitrary reward distributions. For any algorithm $ALG$ with Bayesian information, it holds that
\[\sw\left(ALG\right) \leq \sw\left(\sopt\right).\]    
\end{theorem}
\begin{proof}[Proof of Theorem~\ref{thm:sopt is opt}]
Fix any instance with $N=2$ and arbitrary reward distributions. The social welfare of an algorithm $ALG$ over $T$ rounds is
\[
\sw(ALG) 
=\E{ \sum^2_{i=1}{R_i^T}}= \E{\sum_{t=1}^T \bigl(r^t_{(1)} + r^t_{(2)}\bigr)}
= T \, \E{r^1_{(1)} + r^1_{(2)}},
\]
where the last equality uses the fact that the rewards in each round are independent.

Thus, to show that $\sopt$ maximizes social welfare, it suffices to show that no algorithm can exceed its expected reward \emph{within a single round}. By the revelation principle~\cite{peters2001common}, there is an optimal \emph{threshold} algorithm: After observing the first-session reward, it decides in the second session by comparing the observed reward against the expected reward of any other arm.

Since $\sopt$ enumerates all pairs of arms $(i,j)$ in Equation~\eqref{eq: picking i,j star} and applies a greedy rule for the second session (choosing either the same arm $i$ or the other arm $j$ based on which is expected to yield a higher reward), it achieves the maximum expected reward per round. Multiplying by $T$ completes the proof.
\end{proof}

\subsection{Socially Optimal Algorithm for $N >2$}\label{subsec:sopt for N>2}
Next, we consider the problem of finding a socially optimal algorithm for $N >2$ agents. First, we present a socially optimal algorithm for the special case of Bernoulli rewards.

\begin{proposition}\label{prop:bernoulli sw optimal}
    Assume that $X_i \sim \ber{p_i}$ for every $i \in [K]$. An algorithm that pulls arms in descending order of $p_i$ until it realizes a reward of 1 is socially optimal for any number of agents $N$.
\end{proposition}
\begin{proof}[Proof of Proposition~\ref{prop:bernoulli sw optimal}]
    Since random algorithms are just a distribution over deterministic algorithms, we know there is an optimal algorithm that is deterministic. Additionally, as in the proof of Theorem~\ref{thm:sopt is opt}, it suffices to focus on a single and arbitrary round $t$.
    
    Notice that after we observe an arm such that $x^t_i =1$, it is optimal to pull it for all remaining agents.
    Similarly,  if we observe arm $a_i$ with $x^t_i =0$ it is strictly sub-optimal to pull it again, unless all arms yielded a reward of 0.
    Thus, the only thing left to prove is the optimality of the order in which the algorithm pulls arms.

    In this special case, we can use a reduction to a Pandora's Box (PB) problem~\cite{weitzman1978optimal}. We first describe the reduction, then characterize the optimal solution for the PB instance, and finally show the equivalence. 
       
    Let the PB instance include $K$ Bernoulli arms with success probability $p_i$ for every arm $i$ and costs $c_i = \frac{1}{B}$, where $B$ is a constant such that $B > \max_{i \in [K]} \frac{1}{p_i}$. Due to Weitzman's seminal result~\cite{weitzman1978optimal}, there exist indices $(\theta_i)_{i=1}^K$ such that the optimal sequence is descending in the index. Each index  $\theta_i$ is the solution to $\E{\max\left\{X_i - \theta_i,0 \right\}}  =c_i =\frac{1}{B}$. Thus, $\theta_i = 1- \frac{1}{B\cdot p_i}$. The optimal solution maximizes $\E{1-\frac{S}{B}}$, where $S$ is a r.v. that counts the number of useless arms pulled (arms with a realized reward of 0). Note that $S$ depends solely on the pulling order.

    Similarly, for our original problem, maximizing the social welfare amounts to maximizing $\E{N-S}$. Since $S$ is distributed identically in both problems, the optimal order for PB is optimal for the original problem as well.
\end{proof}
    % satisfy this condition;
    % hence, the index of each arm is $1- \frac{1}{p_i}$. 
    % Any optimal algorithm for the PB instance maximizes the reward minus the sum of costs; thus, it orders the arms to find a positive reward as quickly as possible. Using the same order ensures that we maximize the social welfare in our problem. 
    % Recall that in the PB instance,  we need to find for each arm $a_i$ an index $\theta_i$ that satisfies $\E{\max\left\{X_i - \theta_i,0 \right\}}  =c_i =1$.
    % It is easy to see that $\theta_i = 1- \frac{1}{p_i}$ satisfy this condition;
    % hence, the index of each arm is $1- \frac{1}{p_i}$. 
    % the optimal sequence for the PB instance is descending in the index.

% \omer{USE THE BELOW AS THE BASIS FOR THE DYNAMIC PROGRAMMING}
% Fix an arbitrary round $t$, and assume that all the rewards are supported in a finite set $V$. We now describe a dynamic programming procedure that finds the optimal algorithm in $O(\abs{V} N2^K)$. Let $B$ be a subset of arms $B \subseteq A$, $n$ denote a number of agents $n\in \{0,1,\dots,N$, and $v$ denote an arbitrary reward $v \in V$. We define the following function $f$:
% \begin{equation}\label{eq:dp f}
% f(n,B,v) =  \max \left\{v\cdot n, \max_{a\in B} \mathbb E\left[X_a + f\left(n-1,B\setminus \{a\}, \max\{v,X_a\}\right) \right] \right\}    
% \end{equation}

Next, we move beyond Bernoulli rewards. Fix an arbitrary round $t$, and assume that all rewards are supported in a finite set $V$ (we later explain how to relax this assumption). We now describe a dynamic programming procedure that finds the optimal algorithm with a computational complexity of $O(\abs{V} \cdot N \cdot 2^K)$. 

We consider the following parameters: $B \subseteq A$, representing a subset of available arms; $n \in \{0,1,\dots,N\}$, denoting the number of remaining agents; and $v \in V$, an arbitrary current reward that models the maximal reward of all observed arms in $A \setminus B$.


We define the function $f(n, B, v)$, representing the maximum expected social welfare achievable given the parameters $(n,B,v)$. To apply this dynamic programming approach, we first establish the base cases for the function $f$:
\begin{itemize}
    \item \textbf{No agents remaining ($n = 0$):} If there are no agents left to assign rewards, the maximum achievable reward is zero regardless of the subset of arms $B$ and the current reward $v$. Formally, 
    \[
    f(0, B, v) = 0 \quad \forall \, B \subseteq A, \, v \in V.
    \]
    
    \item \textbf{No available arms ($B = \emptyset$):} If there are no arms left to pull, the only option is to assign the current reward $v$ to all remaining agents. Thus, the maximum reward in this scenario is the product of $v$ and the number of remaining agents $n$. Formally, 
    \[
    f(n, \emptyset, v) = v \cdot n \quad \forall \, n \in \{0,1,\dots,N\}, \, v \in V.
    \]
\end{itemize}

We move on to the recursive definition of $f$:
\begin{equation}\label{eq:dp_f}
f(n, B, v) = \max \left\{ v \cdot n, \max_{a \in B} \mathbb{E}\left[ X_a^t + f\left(n - 1, B \setminus \{a\}, \max\{v, X_a^t\} \right) \right] \right\}
\end{equation}
The recursive relation in Equation~\eqref{eq:dp_f} considers two scenarios at each step:
\begin{enumerate}
    \item \textbf{Assigning the Current Reward ($v \cdot n$):} In this scenario, the algorithm assigns the current reward $v$ to all remaining $n$ agents without pulling any additional arms.
    
    \item \textbf{Pulling an Arm ($\max_{a \in B} \mathbb{E}\left[ X_a^t + f\left(n - 1, B \setminus \{a\}, \max\{v, X_a^t\} \right) \right]$):} Here, the algorithm selects an arm $a \in B$, observes the stochastic reward $X_a^t$, and then recursively assigns rewards to the remaining $n - 1$ agents. The subset of available arms is updated to $B \setminus \{a\}$, and the current reward is updated to $\max\{v, X_a^t\}$.
\end{enumerate}

The optimal value for round $t$ is obtained by evaluating the function $f$ at the initial conditions where all agents and arms are available, and the starting reward is zero:
\[
f(n = N, B = A, v = 0).
\]
This value represents the maximum expected social welfare achievable by the optimal algorithm for round $t$. 

\begin{theorem}\label{thm:optimality_runtime}
    The dynamic programming procedure defined by the function $f(n, B, v)$ in Equation~\eqref{eq:dp_f} correctly computes the maximum expected total social welfare achievable for round $t$. Furthermore, the procedure operates with a runtime complexity of $O(\abs{V} \cdot N\cdot K\cdot 2^K)$.
\end{theorem}

\begin{proof}
    The optimality of the procedure follows from standard dynamic programming arguments, relying on the fact that the overall problem can be constructed from optimal solutions to its subproblems. The function $f(n, B, v)$ is designed to represent the maximum expected total social welfare achievable given the state $(n, B, v)$. By taking the maximum of the two options (recall Equation~\eqref{eq:dp_f}, $f(n, B, v)$ ensures that the optimal decision is made at each state, thereby maximizing the expected total social welfare.

To determine the runtime complexity, we analyze the number of possible states and the computation required for each state. The function $f(n, B, v)$ is parameterized by the number of remaining agents $n$, which can take $N+1$ possible values; the subset of available arms $B$, which has $2^K$ possible subsets; and the current reward $v$, which can take $\abs{V}$ possible values. Therefore, the total number of states is $O(\abs{V} \cdot N \cdot 2^K)$. For each state $(n, B, v)$, the dynamic programming procedure performs a constant-time computation to calculate $v \cdot n$ and iterates over all arms in $B$, performing constant-time computations for each arm. Given that there are up to $K$ arms in $B$, the computation per state is $O(K)$. Consequently, the overall runtime complexity is $O(\abs{V} \cdot N \cdot K \cdot 2^K )$. 
\end{proof}


Finally, we can relax the finite-support assumption. In scenarios where rewards are arbitrary within the continuous interval $[0,1]$, we discretize the reward space by selecting a finite set $V$ that approximates the continuous range of rewards. By choosing an appropriate granularity for the discretization, we can control the trade-off between the accuracy of the approximation and the computational complexity of the algorithm. While a finer discretization yields a closer approximation to the true optimal solution, it simultaneously increases the size of the state space, thereby enhancing the computational burden. Conversely, a coarser discretization reduces computational requirements at the expense of approximation precision. As this approach is standard, we omit the details. 


Unfortunately, the procedure above is inefficient in the number of arms $K$. We leave the tasks of finding an optimal algorithm, proving hardness, and finding approximately optimal algorithms as open problems.
%, i.e., an $ALG'\in \mathcal A$ with runtime $poly\left(\frac{1}{\epsilon} \right)$ such that $\E{SW(ALG')} \geq \max_{ALG\in \mathcal A} \E{SW(ALG)} - \epsilon$, 
\section{Simulations}
\label{sec:simulations}
\newcommand{\uins}{I_U}
\newcommand{\bins}{I_B}
\newcommand{\sins}{I_S}
In this section, we report the results of extensive simulations to empirically validate our theoretical results and test our conjectures. Specifically, we devote Subsection~\ref{subsec:dep t n} to verify the behavior of uniform, nudged, and adversarial arrival as functions of the horizon $T$ and number of agents $N$. In Subsection~\ref{subsec:nudge sensitive}, we provide a sensitivity analysis of nudged arrival. Subsection~\ref{subsec:sim efc} validates our results from Section~\ref{sec:extensions}, as well as test Conjecture~\ref{thm: efc sw}.

\paragraph{Simulation setup}
For analyses where the time horizon $T$ and the number of agents $N$ are not explicitly specified, we set $T = 10{,}000$ rounds and $N=2$ agents by default. We report the average results over $1{,}000$ independent runs, with the shaded areas indicating three standard deviations from the mean. All simulations were conducted on a standard Lenovo laptop, with the total execution time amounting to a couple of hours.

We used two instances in most of the experiments, 
\begin{itemize}
    \item Uniform instance ($\uins$): $K=4$ arms, all with $\uni{0,1}$ reward distributions. The algorithm explores arms until it finds one with a value greater than $\frac{3}{4}$. 
    \item Bernoulli instance ($\bins$): $K=3$ Bernoulli arms with success parameters of $p_1 = 0.6, p_2=0.4, p_3=0.2$. The algorithm explores arms in a descending order of $p_i$, until a reward of 1 is materialized.
\end{itemize}

\subsection{Dependence on $T$ and $N$}\label{subsec:dep t n}
\begin{figure}
    \centering
    \includegraphics[width=\linewidth]{figures/Figure_1.pdf}
    \caption{
    $\env^t$ as a function of $t$ for both the uniform instance ($\uins$, left panel) and the Bernoulli instance ($\bins$, right panel), each with $N=2$. 
    The three arrival functions shown are $\uniord$, $\sugord$ (with $\delta=\frac 1 2$), and $\advord$. Green `X' markers represent the maximum likelihood estimates (MLE) for the linear model $y = c \cdot x$, while orange circles indicate the MLE for the square-root model $y = c \cdot \sqrt{x}$. The perfect alignment of the simulated data with these curves confirms our theoretical predictions.}
    \label{fig: envy-func-t}
\end{figure}
In this subsection, we validate our results regarding the dependency on $T$ and $N$. Figure~\ref{fig: envy-func-t} shows the cumulative envy as a function of time for all three arrival functions: $\advord$ (green dashed), $\uniord$ (orange dotted), and $\sugord$ (blue smooth). Each plot shows the cumulative envy over time on a logarithmic vertical axis, reflecting the distinct asymptotic behaviors of the arrival functions. The left panel presents the uniform instance $\uins$, and the right panel the Bernoulli instance $\bins$.  Due to the logarithmic scale, the shaded regions indicating three standard deviations are barely distinguishable; we provide further details in Table~\ref{table} in Subsection~\ref{appendix: simulations}.

For both instances, we see that the cumulative envy of $\advord$ and $\uniord$ increases over time, whereas the cumulative envy of $\sugord$ remains nearly constant (subject to some noise). As time progresses, we observe substantially different growth rates in envy across the three arrival functions, consistent with our theoretical analysis. 

The green `X' markers represent the maximum likelihood estimates (MLE) for the linear function $y = c \cdot x$, which closely match the green (dashed) curve for the envy under $\advord$, thereby confirming the linear growth predicted by Proposition~\ref{thm: adv-envy}. The orange (dotted) curve corresponds to $\uniord$, and the orange circles depict the MLE for the square-root function $y = c \cdot \sqrt{x}$. Their close alignment supports Corollary~\ref{cor: uni-envy}. Finally, Theorem~\ref{thm: sugg-envy} asserts that envy under $\sugord$ remains bounded when both the instance parameters and the number of agents are fixed. This is precisely what we observe in the blue (smooth) curves of both panels in Figure~\ref{fig: envy-func-t}.
\begin{figure}
    \centering
    \includegraphics[width=\linewidth]{figures/Figure_2.pdf}
    \caption{$\env^T$ as a function of $N$ for both $\uins$ and $\bins$, under $\uniord$ (left panel) and $\sugord$ with $\delta=\nicefrac{1}{2}$ (right panel).}
    \label{fig: envy-T-func-N}
\end{figure}

We proceed to examine how envy depends on the number of agents, focusing on $\uniord$ and $\sugord$ with $\delta = \frac{1}{2}$. In Figure~\ref{fig: envy-T-func-N}, we plot the cumulative envy after $T = 10^4$ rounds as a function of the number of agents $N$, with $N$ ranging from $2$ to $20$. The left panel depicts the envy under $\uniord$ for both instances, $\bins$ (blue smooth) and $\uins$ (orange dotted). In each instance, envy initially rises for small $N$ and then declines, matching the intuition from Corollary~\ref{thm: sqrt TK N}: as $N$ grows, there are increasingly more agents exploiting rather than exploring (given that these instances have $K\in\{3,4\}$ arms). The right panel in Figure~\ref{fig: envy-T-func-N} considers nudged arrival $\sugord$. Here, envy increases with the number of agents, as Theorem~\ref{thm: sugg-envy} hints. However, this increase is not linear in $N$, but rather milder. We conjecture that the dependence of $\sugord$ is essentially sub-linear in $N$, leaving a precise characterization for future work.
\subsection{Sensitivity Analysis for Nudged Arrival}\label{subsec:nudge sensitive}
\begin{figure}
    \centering
    \begin{subfigure}{0.49\textwidth}
        \includegraphics[width=\linewidth]{figures/Figure_3a.pdf}
        \caption{Envy as a function of $\delta$ for $\uins$ and $\bins$.
        %$\env^T$ as a function of $\delta$ for both Uniform and Bernoulli instances, for $\sugord$.
        }\label{fig: envy_at_T_as_func_delta}
    \end{subfigure}
    \hfill
    \begin{subfigure}{0.49\textwidth}
        \includegraphics[width=\linewidth]{figures/Figure_3b.pdf}
        \caption{Envy as a function of $\tdif$ for $\sins$.
        %$\env^T$ as a function of $T$ for an instances with $\Tilde{\dif}$ that decreases as a function of $T$ under $\ordname_{\delta=\nicefrac{1}{4}}$. 
        }\label{fig: envy_at_T_delta_func_of_T}
    \end{subfigure}
    \caption{Sensitivity Analysis for $\sugord$.}
\end{figure}
In this subsection, we provide a sensitivity analysis for $\sugord$. Figure~\ref{fig: envy_at_T_as_func_delta} shows the cumulative envy as a function of the parameter $\delta$, which reflects how strongly the system can influence the agents' arrival. 
For both $\bins$ (blue smooth) and $\uins$ (orange dotted), we observe that envy decreases as $\delta$ increases. 
Moreover, this reduction appears consistent with $\nicefrac{1}{\delta}$, aligning with the prediction from Theorem~\ref{thm: sugg-envy}.

To examine the dependence of envy on $\tdif$, we introduce reward distributions that explicitly involve $T$, unlike $\uins$ and $\bins$. 
Specifically, we define a new instance, $\sins$, with $K = 2$. 
In this instance,
\begin{align*}
    X_1 \sim
    \begin{cases}
    1 & w.p. \ \  \frac{1}{2} \\
    \frac{1}{4} & w.p.\  \ \frac{1}{2}
    \end{cases},\quad 
    X_2 \sim
    \begin{cases}
    1 & w.p. \ \  \frac{1}{4} + \frac{2}{\sqrt{T}} \\
    0 & w.p.\  \ \frac{3}{4} - \frac{2}{\sqrt{T}}
    \end{cases}.
\end{align*}
Furthermore, we consider the following algorithm: In the first session of every round, it pulls $a_1$. If the observed reward is $1$, it pulls $a_1$ again in the second session; otherwise, it pulls $a_2$. Note that 
\[
\E{\rt{(1)}}=\frac{1+0.25}{2}=\frac{5}{8},
\]
while
\[
\E{\rt{(2)}}= \prb{X_1 = 1}\cdot 1 + \prb{X_1 = 0.25}\cdot \E{X_2} = \frac{1}{2} + \frac{1}{2} \cdot \left( \frac{1}{4} + \frac{2}{\sqrt{T}} \right) = \frac{5}{8} +\frac{1}{\sqrt{T}};
\]
thus, $\tilde{\dif}= \E{\rt{(2)}-\rt{(1)}}= \frac{1}{\sqrt{T}}$.

We now examine how envy evolves in the instance $\sins$. 
Figure~\ref{fig: envy_at_T_delta_func_of_T} displays the cumulative envy $\env^T$ at the final round $T$ for various values of $T$. 
As $T$ grows, $\tilde{\dif}$ decreases, causing envy to increase accordingly. 
Moreover, since $\tfrac{1}{\tilde{\dif}} = \sqrt{T}$, Theorem~\ref{thm: sugg-envy} implies that envy scales proportionally to $\sqrt{T}$, which is consistent with the empirical results.

\subsection{Analysis of EFC}\label{subsec:sim efc}
In this subsection, we examine the theoretical results from Section~\ref{sec:extensions}. Naturally, as our results in that section apply to the special case of Example~\ref{example 1}, all simulations were performed with $N=2$ agents, $K=2$ arms $X_1, X_2 \sim \uni{0,1}$ under uniform arrival and the $\efc$ algorithm. %We remind the reader that the social welfare in any round $t$ is given by $R_1^t+R_2^t$.

%As can be seen, all empirical results align with our theoretical results including Theorem~\ref{??}.
\begin{figure}
    \centering
        \includegraphics[scale=0.6]{figures/Figure_4.pdf}
        \caption{$\Rt{1}+\Rt{2}$ under the $\efc$ algorithm with $C=1$, as a function of $t$.% compared to $\left(1 + \frac{1}{16}\right)t$.
        }\label{fig: sw ef1}
\end{figure}

\begin{figure}
    \centering
        \includegraphics[width=\linewidth]{figures/Figure_5.pdf}
        \caption{$\frac{\Rt{1}+\Rt{2}}{t}$ under the $\efc$ algorithm, as a function of $t$ compared to $1 + \frac{1}{8}\cdot \frac{2C-1}{2C}$. and $1+\frac{1}{8}$}\label{fig: sw efc}
\end{figure}

Figure~\ref{fig: sw ef1} illustrates the social welfare for $\efc$ with $C=1$, plotting 
$\sw_t := R_1^t + R_2^t$ as the orange (smooth) curve. 
The green `X' markers correspond to the line $y=(1+\frac{1}{16})x$, 
the social welfare guaranteed by Theorem~\ref{thm:ef1evny+sw}; 
their perfect alignment confirms the theoretical prediction.


Figure~\ref{fig: sw efc} evaluates Conjecture~\ref{thm: efc sw}, asserting that for any $C \ge 1$, 
we have $\sw_t \geq (1+\frac{1}{8}-\frac{1}{16C})t$. 
To facilitate comparison, the vertical axis in Figure~\ref{fig: sw efc} depicts the average welfare, $\tfrac{\sw_t}{t}$, 
versus the round number $t$ on the horizontal axis. 
We ran $\efc$ for $C \in \{1,2,3,4,5,10,20,40\}$, adding circle markers in corresponding colors to highlight 
the values of $1 + \tfrac{2C - 1}{2C} \cdot \tfrac{1}{8}$. 
Additionally, the star marker shows the horizontal line $1 + \tfrac{1}{8}$, 
representing the maximum achievable welfare in this setting, as noted in Observation~\ref{obs:opt for tradeoff}. 
For each $C$, the corresponding curve lies above the dotted line, 
consistent with our conjectured lower bound. 
When $1 \leq C \leq 10$, the average welfare nearly coincides with the conjectured bound, 
whereas for $C = 20$ and $C = 40$, the welfare is strictly higher than the markers. 
The reason for the latter is that such high envy states occur rarely when $T = 10^4$, 
so the constraint in Line~\ref{efclin:pull_a1_cond} in $\efc$ is seldom activated in practice.

\subsection{Standard Deviations for Figure~\ref{fig: envy-func-t}}\label{appendix: simulations}
\begin{table}[!ht]
    \centering
    \caption{Three standard deviations for Figure~\ref{fig: envy-func-t}.}\label{table}
    \begin{tabular}{|l|l|l|l|l|l|l|}
    \hline
        $t$ & $\uins, \advord$ & $\uins, \uniord$ & $\uins, \sugord$ & $\bins, \advord$ & $\bins, \uniord$ &  $\bins, \sugord$ \\ \hline
        1000 & 0.79 & 0.54 & 0.13 & 1.12 & 0.72 & 0.18 \\ \hline
        2000 & 1.11 & 0.74 & 0.12 & 1.59 & 1.02 & 0.18 \\ \hline
        3000 & 1.34 & 0.87 & 0.12 & 1.92 & 1.27 & 0.19 \\ \hline
        4000 & 1.52 & 0.99 & 0.12 & 2.22 & 1.38 & 0.19 \\ \hline
        5000 & 1.72 & 1.09 & 0.13 & 2.50 & 1.59 & 0.18 \\ \hline
        6000 & 1.91 & 1.23 & 0.12 & 2.72 & 1.73 & 0.19 \\ \hline
        7000 & 2.11 & 1.34 & 0.12 & 2.91 & 1.89 & 0.19 \\ \hline
        8000 & 2.30 & 1.42 & 0.12 & 3.16 & 2.08 & 0.18 \\ \hline
        9000 & 2.43 & 1.52 & 0.13 & 3.37 & 2.22 & 0.21 \\ \hline
        10000 & 2.58 & 1.59 & 0.13 & 3.55 & 2.35 & 0.18 \\ \hline
    \end{tabular}
\end{table}

}%
\fi

\end{document}
