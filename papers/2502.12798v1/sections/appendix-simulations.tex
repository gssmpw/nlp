\section{Simulations}
\label{sec:simulations}
\newcommand{\uins}{I_U}
\newcommand{\bins}{I_B}
\newcommand{\sins}{I_S}
In this section, we report the results of extensive simulations to empirically validate our theoretical results and test our conjectures. Specifically, we devote Subsection~\ref{subsec:dep t n} to verify the behavior of uniform, nudged, and adversarial arrival as functions of the horizon $T$ and number of agents $N$. In Subsection~\ref{subsec:nudge sensitive}, we provide a sensitivity analysis of nudged arrival. Subsection~\ref{subsec:sim efc} validates our results from Section~\ref{sec:extensions}, as well as test Conjecture~\ref{thm: efc sw}.

\paragraph{Simulation setup}
For analyses where the time horizon $T$ and the number of agents $N$ are not explicitly specified, we set $T = 10{,}000$ rounds and $N=2$ agents by default. We report the average results over $1{,}000$ independent runs, with the shaded areas indicating three standard deviations from the mean. All simulations were conducted on a standard Lenovo laptop, with the total execution time amounting to a couple of hours.

We used two instances in most of the experiments, 
\begin{itemize}
    \item Uniform instance ($\uins$): $K=4$ arms, all with $\uni{0,1}$ reward distributions. The algorithm explores arms until it finds one with a value greater than $\frac{3}{4}$. 
    \item Bernoulli instance ($\bins$): $K=3$ Bernoulli arms with success parameters of $p_1 = 0.6, p_2=0.4, p_3=0.2$. The algorithm explores arms in a descending order of $p_i$, until a reward of 1 is materialized.
\end{itemize}

\subsection{Dependence on $T$ and $N$}\label{subsec:dep t n}
\begin{figure}
    \centering
    \includegraphics[width=\linewidth]{figures/Figure_1.pdf}
    \caption{
    $\env^t$ as a function of $t$ for both the uniform instance ($\uins$, left panel) and the Bernoulli instance ($\bins$, right panel), each with $N=2$. 
    The three arrival functions shown are $\uniord$, $\sugord$ (with $\delta=\frac 1 2$), and $\advord$. Green `X' markers represent the maximum likelihood estimates (MLE) for the linear model $y = c \cdot x$, while orange circles indicate the MLE for the square-root model $y = c \cdot \sqrt{x}$. The perfect alignment of the simulated data with these curves confirms our theoretical predictions.}
    \label{fig: envy-func-t}
\end{figure}
In this subsection, we validate our results regarding the dependency on $T$ and $N$. Figure~\ref{fig: envy-func-t} shows the cumulative envy as a function of time for all three arrival functions: $\advord$ (green dashed), $\uniord$ (orange dotted), and $\sugord$ (blue smooth). Each plot shows the cumulative envy over time on a logarithmic vertical axis, reflecting the distinct asymptotic behaviors of the arrival functions. The left panel presents the uniform instance $\uins$, and the right panel the Bernoulli instance $\bins$.  Due to the logarithmic scale, the shaded regions indicating three standard deviations are barely distinguishable; we provide further details in Table~\ref{table} in Subsection~\ref{appendix: simulations}.

For both instances, we see that the cumulative envy of $\advord$ and $\uniord$ increases over time, whereas the cumulative envy of $\sugord$ remains nearly constant (subject to some noise). As time progresses, we observe substantially different growth rates in envy across the three arrival functions, consistent with our theoretical analysis. 

The green `X' markers represent the maximum likelihood estimates (MLE) for the linear function $y = c \cdot x$, which closely match the green (dashed) curve for the envy under $\advord$, thereby confirming the linear growth predicted by Proposition~\ref{thm: adv-envy}. The orange (dotted) curve corresponds to $\uniord$, and the orange circles depict the MLE for the square-root function $y = c \cdot \sqrt{x}$. Their close alignment supports Corollary~\ref{cor: uni-envy}. Finally, Theorem~\ref{thm: sugg-envy} asserts that envy under $\sugord$ remains bounded when both the instance parameters and the number of agents are fixed. This is precisely what we observe in the blue (smooth) curves of both panels in Figure~\ref{fig: envy-func-t}.
\begin{figure}
    \centering
    \includegraphics[width=\linewidth]{figures/Figure_2.pdf}
    \caption{$\env^T$ as a function of $N$ for both $\uins$ and $\bins$, under $\uniord$ (left panel) and $\sugord$ with $\delta=\nicefrac{1}{2}$ (right panel).}
    \label{fig: envy-T-func-N}
\end{figure}

We proceed to examine how envy depends on the number of agents, focusing on $\uniord$ and $\sugord$ with $\delta = \frac{1}{2}$. In Figure~\ref{fig: envy-T-func-N}, we plot the cumulative envy after $T = 10^4$ rounds as a function of the number of agents $N$, with $N$ ranging from $2$ to $20$. The left panel depicts the envy under $\uniord$ for both instances, $\bins$ (blue smooth) and $\uins$ (orange dotted). In each instance, envy initially rises for small $N$ and then declines, matching the intuition from Corollary~\ref{thm: sqrt TK N}: as $N$ grows, there are increasingly more agents exploiting rather than exploring (given that these instances have $K\in\{3,4\}$ arms). The right panel in Figure~\ref{fig: envy-T-func-N} considers nudged arrival $\sugord$. Here, envy increases with the number of agents, as Theorem~\ref{thm: sugg-envy} hints. However, this increase is not linear in $N$, but rather milder. We conjecture that the dependence of $\sugord$ is essentially sub-linear in $N$, leaving a precise characterization for future work.
\subsection{Sensitivity Analysis for Nudged Arrival}\label{subsec:nudge sensitive}
\begin{figure}
    \centering
    \begin{subfigure}{0.49\textwidth}
        \includegraphics[width=\linewidth]{figures/Figure_3a.pdf}
        \caption{Envy as a function of $\delta$ for $\uins$ and $\bins$.
        %$\env^T$ as a function of $\delta$ for both Uniform and Bernoulli instances, for $\sugord$.
        }\label{fig: envy_at_T_as_func_delta}
    \end{subfigure}
    \hfill
    \begin{subfigure}{0.49\textwidth}
        \includegraphics[width=\linewidth]{figures/Figure_3b.pdf}
        \caption{Envy as a function of $\tdif$ for $\sins$.
        %$\env^T$ as a function of $T$ for an instances with $\Tilde{\dif}$ that decreases as a function of $T$ under $\ordname_{\delta=\nicefrac{1}{4}}$. 
        }\label{fig: envy_at_T_delta_func_of_T}
    \end{subfigure}
    \caption{Sensitivity Analysis for $\sugord$.}
\end{figure}
In this subsection, we provide a sensitivity analysis for $\sugord$. Figure~\ref{fig: envy_at_T_as_func_delta} shows the cumulative envy as a function of the parameter $\delta$, which reflects how strongly the system can influence the agents' arrival. 
For both $\bins$ (blue smooth) and $\uins$ (orange dotted), we observe that envy decreases as $\delta$ increases. 
Moreover, this reduction appears consistent with $\nicefrac{1}{\delta}$, aligning with the prediction from Theorem~\ref{thm: sugg-envy}.

To examine the dependence of envy on $\tdif$, we introduce reward distributions that explicitly involve $T$, unlike $\uins$ and $\bins$. 
Specifically, we define a new instance, $\sins$, with $K = 2$. 
In this instance,
\begin{align*}
    X_1 \sim
    \begin{cases}
    1 & w.p. \ \  \frac{1}{2} \\
    \frac{1}{4} & w.p.\  \ \frac{1}{2}
    \end{cases},\quad 
    X_2 \sim
    \begin{cases}
    1 & w.p. \ \  \frac{1}{4} + \frac{2}{\sqrt{T}} \\
    0 & w.p.\  \ \frac{3}{4} - \frac{2}{\sqrt{T}}
    \end{cases}.
\end{align*}
Furthermore, we consider the following algorithm: In the first session of every round, it pulls $a_1$. If the observed reward is $1$, it pulls $a_1$ again in the second session; otherwise, it pulls $a_2$. Note that 
\[
\E{\rt{(1)}}=\frac{1+0.25}{2}=\frac{5}{8},
\]
while
\[
\E{\rt{(2)}}= \prb{X_1 = 1}\cdot 1 + \prb{X_1 = 0.25}\cdot \E{X_2} = \frac{1}{2} + \frac{1}{2} \cdot \left( \frac{1}{4} + \frac{2}{\sqrt{T}} \right) = \frac{5}{8} +\frac{1}{\sqrt{T}};
\]
thus, $\tilde{\dif}= \E{\rt{(2)}-\rt{(1)}}= \frac{1}{\sqrt{T}}$.

We now examine how envy evolves in the instance $\sins$. 
Figure~\ref{fig: envy_at_T_delta_func_of_T} displays the cumulative envy $\env^T$ at the final round $T$ for various values of $T$. 
As $T$ grows, $\tilde{\dif}$ decreases, causing envy to increase accordingly. 
Moreover, since $\tfrac{1}{\tilde{\dif}} = \sqrt{T}$, Theorem~\ref{thm: sugg-envy} implies that envy scales proportionally to $\sqrt{T}$, which is consistent with the empirical results.

\subsection{Analysis of EFC}\label{subsec:sim efc}
In this subsection, we examine the theoretical results from Section~\ref{sec:extensions}. Naturally, as our results in that section apply to the special case of Example~\ref{example 1}, all simulations were performed with $N=2$ agents, $K=2$ arms $X_1, X_2 \sim \uni{0,1}$ under uniform arrival and the $\efc$ algorithm. %We remind the reader that the social welfare in any round $t$ is given by $R_1^t+R_2^t$.

%As can be seen, all empirical results align with our theoretical results including Theorem~\ref{??}.
\begin{figure}
    \centering
        \includegraphics[scale=0.6]{figures/Figure_4.pdf}
        \caption{$\Rt{1}+\Rt{2}$ under the $\efc$ algorithm with $C=1$, as a function of $t$.% compared to $\left(1 + \frac{1}{16}\right)t$.
        }\label{fig: sw ef1}
\end{figure}

\begin{figure}
    \centering
        \includegraphics[width=\linewidth]{figures/Figure_5.pdf}
        \caption{$\frac{\Rt{1}+\Rt{2}}{t}$ under the $\efc$ algorithm, as a function of $t$ compared to $1 + \frac{1}{8}\cdot \frac{2C-1}{2C}$. and $1+\frac{1}{8}$}\label{fig: sw efc}
\end{figure}

Figure~\ref{fig: sw ef1} illustrates the social welfare for $\efc$ with $C=1$, plotting 
$\sw_t := R_1^t + R_2^t$ as the orange (smooth) curve. 
The green `X' markers correspond to the line $y=(1+\frac{1}{16})x$, 
the social welfare guaranteed by Theorem~\ref{thm:ef1evny+sw}; 
their perfect alignment confirms the theoretical prediction.


Figure~\ref{fig: sw efc} evaluates Conjecture~\ref{thm: efc sw}, asserting that for any $C \ge 1$, 
we have $\sw_t \geq (1+\frac{1}{8}-\frac{1}{16C})t$. 
To facilitate comparison, the vertical axis in Figure~\ref{fig: sw efc} depicts the average welfare, $\tfrac{\sw_t}{t}$, 
versus the round number $t$ on the horizontal axis. 
We ran $\efc$ for $C \in \{1,2,3,4,5,10,20,40\}$, adding circle markers in corresponding colors to highlight 
the values of $1 + \tfrac{2C - 1}{2C} \cdot \tfrac{1}{8}$. 
Additionally, the star marker shows the horizontal line $1 + \tfrac{1}{8}$, 
representing the maximum achievable welfare in this setting, as noted in Observation~\ref{obs:opt for tradeoff}. 
For each $C$, the corresponding curve lies above the dotted line, 
consistent with our conjectured lower bound. 
When $1 \leq C \leq 10$, the average welfare nearly coincides with the conjectured bound, 
whereas for $C = 20$ and $C = 40$, the welfare is strictly higher than the markers. 
The reason for the latter is that such high envy states occur rarely when $T = 10^4$, 
so the constraint in Line~\ref{efclin:pull_a1_cond} in $\efc$ is seldom activated in practice.

\subsection{Standard Deviations for Figure~\ref{fig: envy-func-t}}\label{appendix: simulations}
\begin{table}[!ht]
    \centering
    \caption{Three standard deviations for Figure~\ref{fig: envy-func-t}.}\label{table}
    \begin{tabular}{|l|l|l|l|l|l|l|}
    \hline
        $t$ & $\uins, \advord$ & $\uins, \uniord$ & $\uins, \sugord$ & $\bins, \advord$ & $\bins, \uniord$ &  $\bins, \sugord$ \\ \hline
        1000 & 0.79 & 0.54 & 0.13 & 1.12 & 0.72 & 0.18 \\ \hline
        2000 & 1.11 & 0.74 & 0.12 & 1.59 & 1.02 & 0.18 \\ \hline
        3000 & 1.34 & 0.87 & 0.12 & 1.92 & 1.27 & 0.19 \\ \hline
        4000 & 1.52 & 0.99 & 0.12 & 2.22 & 1.38 & 0.19 \\ \hline
        5000 & 1.72 & 1.09 & 0.13 & 2.50 & 1.59 & 0.18 \\ \hline
        6000 & 1.91 & 1.23 & 0.12 & 2.72 & 1.73 & 0.19 \\ \hline
        7000 & 2.11 & 1.34 & 0.12 & 2.91 & 1.89 & 0.19 \\ \hline
        8000 & 2.30 & 1.42 & 0.12 & 3.16 & 2.08 & 0.18 \\ \hline
        9000 & 2.43 & 1.52 & 0.13 & 3.37 & 2.22 & 0.21 \\ \hline
        10000 & 2.58 & 1.59 & 0.13 & 3.55 & 2.35 & 0.18 \\ \hline
    \end{tabular}
\end{table}