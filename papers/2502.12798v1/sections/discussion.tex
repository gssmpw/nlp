\section{Discussion and Future Work}
\label{sec:discussion}
In this work, we have advanced the understanding of envy dynamics in explore-and-exploit systems. Our stylized model, which assumes reward consistency and sequential agent interactions, assists in characterizing envy under uniform and nudged arrival mechanisms, revealing envy dynamics and accumulation. Under uniform arrival, our results highlight that besides pathologic cases, any algorithm generates an unavoidable expected envy of roughly $\sqrt T$. In contrast, if agent arrival could be nudged, the envy ceases to depend on the horizon $T$. Our results highlight that strategic manipulation of arrival orders through nudging can substantially mitigate envy without altering algorithmic decisions. Furthermore, our preliminary investigation into the welfare-envy tradeoff in Section~\ref{sec:extensions} suggests that algorithms can balance social welfare and envy. 

Due to space limitations, we deferred two important analyses to the appendix. First, we examine the \emph{average envy}, defined as $\envavg^T = \frac{1}{\binom{N}{2}}\sum_{1\leq i<j \leq N}{\abs{\env_{i,j}^T}}$. We show how to leverage our results to obtain meaningful bounds for this measure of envy. Second, we conduct extensive simulations, aiming to empirically validate our theoretical results and perform sensitivity analysis. 


We see considerable scope for future work. First, from a technical standpoint, our results contain several gaps. For instance, in nudged arrival, Theorem~\ref{thm: sugg-envy} provides an upper bound of $O\left(\frac{N}{\delta \tdif} \right)$. However, we conjecture that the dependence on the number of agents $N$ should only be logarithmic, if any. Indeed, our simulations support this view. Furthermore, our welfare-envy analysis is preliminary, and future work could extend it to multiple agents, non-uniform distributions, other arrival functions, etc. Another aspect that this paper did not fully address is social welfare maximization under Bayesian information and consistent rewards. Indeed, our goal was to analyze envy in algorithms without heavy assumptions about the way they operate. We consider this problem in \ifapp{Section~\ref{appendix:sociallyopt}}{the appendix}. We draw some interesting connections to Pandora's box and prophet inequalities. We design a socially optimal algorithm for the two-agent case, develop efficient and optimal algorithms for special cases with $N>2$ agents, and propose an inefficient but approximately optimal algorithm for general instances with $N>2$. We suspect that finding a socially optimal algorithm under Bayesian information is NP-hard, as it involves an optimal ordering of arms, which was proved hard for prophet inequalities~\cite{agrawal2020optimal}. Finding optimal algorithms, efficient approximations, or proving hardness remains an open problem.

From a conceptual standpoint, we see several exciting opportunities to deepen our understanding of envy in explore-and-exploit systems. Recall that we assume reward consistency: Rewards are realized only once per round and remain constant across the sessions within that round. This property is crucial, as envy primarily arises when the algorithm exploits information obtained from one agent to benefit another. However, this assumption may be overly restrictive, as it imposes an abrupt shift in reward dynamics across rounds while ignoring potential variability within rounds. A more general modeling approach could introduce structured reward dynamics such as those found in Markov Decision Processes (MDPs), allowing rewards to change gradually from session to session. This would still capture the essence of envy---since agents engaged in exploration would continue to be disadvantaged---but would reflect more realistic environments. 

Another intriguing conceptual direction is to relax the assumption that every agent arrives precisely once in every round. Indeed, if arrival is more chaotic, envy dynamics change remarkably. Agents might arrive multiple times in a round or skip rounds entirely, introducing new forms of informational asymmetry and potentially amplifying or mitigating envy in unexpected ways. Future work could pursue this challenge, exploring how arrival patterns interact with exploration strategies and fairness considerations. %\omer{to add: contextual rewards?}
