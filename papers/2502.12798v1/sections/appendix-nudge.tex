\section{Omitted Proofs from Section~\ref{sec:nudge}}\label{appendix:nudge}

\begin{proof}[Proof of Proposition~\ref{prop G less than M}]
We prove the claim by induction over $\tau$. The first round in the excursion $D(t)$ and our base case is $\tau=\underline{t}+1$. Since the rewards are in the $[0,1]$ interval and $G_i^{\underline{t}} \leq 1$, we know that $G_i^{\underline{t}+1} \leq 2 = M_i^{\underline{t}+1}$. 



Next, assume the claim holds for $\tau$; thus, $G_i^\tau \leq M_i^\tau$. Recall that we are guaranteed that $\tau \in D(t)$. Without loss of generality, assume that at time $\tau$ agents are ordered lexicographically. Particularly,  ${\sigma^\tau(i)}=i,{\sigma^\tau(i+1)}=i+1$  and $G_i^\tau = R^\tau_{\sigma^\tau(i)}- R^\tau_{\sigma^\tau(i+1)}= R^\tau_i- R^\tau_{i+1}$. 
Next, observe that 
\begin{equation}\label{eq:asdgndsfjghm}
    R^{\tau+1}_{\sigma^{\tau+1}(i)} = \min_{j\in [i]}\left\{ 
    R^{\tau}_{j}+r^{\tau+1}_{j} 
    \right\}\leq R^{\tau}_{i}+r^{\tau+1}_{i}.
\end{equation}
Inequality~\eqref{eq:asdgndsfjghm} holds due to our assumption that the rewards are ordered according to agent indices at round $\tau$.  and since no agent in the set $[N]\setminus[i]$ could obtain a higher cumulative reward that agents $[i]$ at round $\tau + 1$ since all rewards are bounded by 1 and $G^\tau_i > 1$. Similarly,
\begin{equation}\label{eq:dsasdhhgtnt}
    R^{\tau+1}_{\sigma^{\tau+1}(i+1)} = \max_{j\in [N]\setminus[i]}\left\{ 
    R^{\tau}_{j}+r^{\tau+1}_{j} 
    \right\}\geq R^{\tau}_{i+1}+r^{\tau+1}_{i+1}.
\end{equation}
Combining Inequalities~\eqref{eq:asdgndsfjghm} and~\eqref{eq:dsasdhhgtnt}, we derive that
\begin{align*}
G_i^{\tau+1} &= R^{\tau+1}_{\sigma^{\tau+1}(i)} - R^{\tau+1}_{\sigma^{\tau+1}(i+1)} \leq R^{\tau}_{i}+r^{\tau+1}_{i} - R^{\tau}_{i+1}-r^{\tau+1}_{i+1} \\
& =G_i^\tau + r^{\tau+1}_{i} - r^{\tau+1}_{i+1} \leq M_i^\tau + r^{\tau+1}_{i} - r^{\tau+1}_{i+1} \\
&= M_i^\tau + r^{\tau+1}_{(i)} - r^{\tau+1}_{(i+1)} = M_i^{\tau+1},
\end{align*}
where we have used the inductive assumption and the fact that nudged arrival order sorts agents in a non-increasing order of rewards (Algorithm~\ref{alg: sugg arr}, Line~\ref{line:mapping}). This completes the proof of Proposition~\ref{prop G less than M}.
\end{proof}





\begin{proof}[Proof of Proposition~\ref{prop: sugg-m concentration}]
The recursive definition of $M^\tau$ implies that for every $\tau \in D(t)$, $M^\tau = 2+ \sum_{n= \underline{t}+2 }^{\tau} r^n_{(i)}-r^n_{(i+1)}$; thus, 
\begin{align}\label{eq:fghbdfgh}
\prb{M^t > n}  = \prb{\sum_{l= \underline{t}+2 }^{\tau} r^l_{(i)}-r^l_{(i+1)}> n-2}
\end{align}
Next, let $B^l$ denote the event that $r^l_{(i)}-r^l_{(i+1)} \neq 0$. Furthermore, let $B(\tau)$ denote the (random) set of rounds for which the event $B^l$ occurs between $\underline{t}+2$ and $\tau$. That is,
\[
B(\tau) = \{ l\mid \underline{t}+2 \leq l \leq \tau, \ind{B^l} \}
\]
As a result, due to Property~\ref{prop:nudge} and the definition of $\tdif$ in Equation~\eqref{eq def tdif},
\begin{equation}\label{eq:sgdjfndb}
%\E{r^l_{(i)}-r^l_{(i+1)}} = \E{r^l_{(i)}-r^l_{(i+1)} \mid B^l}\prb{B^l} \leq 
\E{r^l_{(i)}-r^l_{(i+1)} \mid B^l} \leq - \delta \tdif.
\end{equation}
Rewriting Equation~\eqref{eq:fghbdfgh},
\begin{align}\label{eq:hbngaersd}
\prb{M^t > n}  &= \prb{\sum_{l \in B(\tau)} r^l_{(i)}-r^l_{(i+1)}> n-2} \nonumber \\
& =\sum_{b\subseteq \{\underline{t}+2,\dots \tau \}}\prb{\sum_{l \in b} r^l_{(i)}-r^l_{(i+1)}> n-2 \mid  B(\tau) = b}\prb{B(\tau) = b} \nonumber\\
& \stackrel{*}{=} \sum_{b\subseteq \{\underline{t}+2,\dots \tau \}, \abs{b}\geq n-2}\prb{\sum_{l \in b} r^l_{(i)}-r^l_{(i+1)}> n-2 \mid  B(\tau) = b}\prb{B(\tau) = b} \nonumber\\
& \leq \max_{b\subseteq \{\underline{t}+2,\dots \tau \}, \abs{b}\geq n-2} \prb{\sum_{l \in b} r^l_{(i)}-r^l_{(i+1)}> n-2 \mid  B(\tau) = b} \nonumber \\
& = \max_{b\subseteq \{\underline{t}+2,\dots \tau \}, \abs{b}\geq n-2} \prb{\sum_{l \in b} r^l_{(i)}-r^l_{(i+1)} + \delta \tdif \abs{b}> n-2+\delta \tdif \abs{b} \mid  B(\tau) = b},
\end{align}
where the change in the set over which we sum in $*$ follows since $\abs{r^l_{(i)}-r^l_{(i+1)}}\leq 1$ almost surely. Striving to bound the above, notice that, conditioned on $B(\tau) = b$, $\sum_{l \in b} \left( r^l_{(i)}-r^l_{(i+1)} + \delta \tdif \right)$ forms a super-martingale. Using Azuma-Hoeffding inequality,
\begin{align*}
\textnormal{Inequality }\eqref{eq:hbngaersd} &\leq \max_{b\subseteq \{\underline{t}+2,\dots \tau \}, \abs{b}\geq n-2} \exp\left\{-\frac{(n-2+\delta \tdif \abs{b})^2}{2 \sum_{l\in b} (1+\delta \tdif)^2 }\right\} \nonumber \\
& \stackrel{\delta \tdif \leq 1}{\leq}  \max_{b\subseteq \{\underline{t}+2,\dots \tau \}, \abs{b}\geq n-2} \exp\left\{-\frac{(n-2)^2+(n-2)(\delta \tdif \abs{b})+(\delta \tdif \abs{b})^2}{8\abs{b}}\right\} \nonumber
\\
%& =  \max_{b\subseteq \{\underline{t}+2,\dots \tau \}, \abs{b}\geq n-2} \exp\left\{-\frac{(n-2)^2+(n-2)(\delta \tdif \abs{b})+(\delta \tdif \abs{b})^2}{8\abs{b}}\right\} \nonumber \\
& =  \max_{b\subseteq \{\underline{t}+2,\dots \tau \}, \abs{b}\geq n-2} \exp\Bigg\{-\frac{(n-2)(\delta \tdif)}{8}-\underbrace{\frac{(n-2)^2+(\delta \tdif \abs{b})^2}{8\abs{b}}}_{\geq 0}\Bigg\} \nonumber \\
& \leq  \max_{b\subseteq \{\underline{t}+2,\dots \tau \}, \abs{b}\geq n-2} \exp\left\{\frac{-(n-2)\delta \tdif}{8}\right\} \nonumber \\
& = \exp\left\{-\frac{(n-2)(\delta \tdif)}{8}\right\}.
\end{align*}
This completes the proof of Proposition~\ref{prop: sugg-m concentration}.
\end{proof}


