\section{Omitted Proofs from Section~\ref{sec:extensions}}
\newcommand{\at}[1]{a_{#1}^t}
\paragraph{Additional notation for this section} The analysis in this section makes extensive use of the notation $\at{(q)}$ for $q \in\{1,2\}$ and $t\in [T]$, denoting the arm pulled in session $i$ of round $t$. 
\begin{proof}[Proof of Theorem~\ref{thm:ef1evny+sw}]
The first part of the proof is given in the body of the paper; hence, we move to the second part, i.e., showing that $\sw = (1+\frac{1}{16})T$.

Fix any $t\in[T]$. We analyze the expected sum of rewards obtained in round $t$, $\E{\rt{(1)}+\rt{(2)}}$.
Notice that $\E{\rt{(1)}} = \E{a_1} = \frac{1}{2}$.
As for $\rt{(2)}$, we are uncertain about the arm the algorithm pulls, but can use total expectation:
\begin{align}\label{eq: ef11}
    \E{\rt{(2)}} & = \E{\rt{(2)} \mid \rt{(1)} > \frac{1}{2}}\cdot \prb{\rt{(1)} > \frac{1}{2}} +  \E{\rt{(2)} \mid \rt{(1)} \leq \frac{1}{2}}\cdot \prb{\rt{(1)} \leq \frac{1}{2}}
    \nonumber\\&= \frac{3}{8} + \frac{1}{2}\cdot \E{\rt{(2)} \mid \rt{(1)} \leq \frac{1}{2}},
\end{align}
where we have used Line~\ref{efclin:pull_a1_again} of $\efc$ for replacing $\E{\rt{(2)} \mid \rt{(1)} > \frac{1}{2}}$ with $\frac{3}{4}$, since arm $a_1$ is pulled for the second session as well. Simplifying the term $\E{\rt{(2)} \mid \rt{(1)} \leq \frac{1}{2}}$ and using the fact that $\E{\rt{(r)} \mid \rt{(1)} \leq \frac{1}{2}}=\frac{1}{4}$, we get,
\begin{align*}
    \E{\rt{(2)} \mid \rt{(1)} \leq \frac{1}{2}} &
    = \E{\rt{(2)} \mid \at{(2)} = \at{(1)}, \rt{(1)}} \cdot \prb{\at{(2)} = \at{(1)} \mid \rt{(1)} \leq \frac{1}{2}}
    \\& \qquad + \E{\rt{(2)} \mid \at{(2)} \neq \at{(1)}, \rt{(1)}} \cdot \prb{\at{(2)} \neq \at{(1)} \mid \rt{(1)} \leq \frac{1}{2}}
    \\&= \frac{1}{4} \cdot \prb{\at{(2)} = \at{(1)} \mid \rt{(1)} \leq \frac{1}{2}} + \frac{1}{2} \cdot \prb{\at{(2)} \neq \at{(1)} \mid \rt{(1)} \leq \frac{1}{2}}
    \\& = \frac{1}{4} + \frac{1}{4}\cdot \prb{\at{(2)} \neq \at{(1)} \mid \rt{(1)} \leq \frac{1}{2}}.
\end{align*}
Consequently, all that is left is to understand how often $\efc$ pulls the second arm when the first arm yields a low reward. Using Proposition~\ref{prop:ef1 open arm}, we obtain
\[
\E{\rt{(2)} \mid \rt{(1)} \leq \frac{1}{2}} =
\frac{1}{4} + \frac{1}{4} \cdot \prb{\at{(2)} \neq \at{(1)} \mid \rt{(1)} \leq \frac{1}{2}} \geq \frac{3}{8}.
\]
Using the above inequality and Equation~\eqref{eq: ef11}, we get
\[
\E{\rt{(2)}} \geq \frac{3}{8} + \frac{1}{2}\cdot \frac{3}{8}= \frac{9}{16}.
\]
Since this holds for any arbitrary $t$, by summing over all rounds, we get
\[
SW(EF1) = \E{\sum_{t=1}^{T}{ \rt{(1)} + \rt{(2)} }} = \sum_{t=1}^{T}{ \E{ \rt{(1)} + \rt{(2)} } } \geq \sum_{t=1}^{T}{ \frac{1}{2} + \frac{9}{16}} = \left( 1+ \frac{1}{16} \right)T.
\]
This concludes the proof of Theorem~\ref{thm:ef1evny+sw}.
\end{proof}

\begin{proof}[Proof of Proposition~\ref{prop:ef1 uni dominance}]
We prove the claim with induction over the round index $t$.
The base step, i.e., $t=0$, is straightforward. Fix any $x\in [0,1]$, and observe that
\[
\prb{\env^0 \leq x} = \prb{0 \leq x} = 1 \geq x.
\]
We move forward to the inductive step. Assume the claim holds for round $t-1$, and let us prove the claim for~$t$. First, notice that if $\at{(2)} = \at{(1)}$, then $\env^t=\env^{t-1}$.
Based on the inductive assumption, the distribution of $\env^{t-1}$ is stochastically dominated by $\uni{0,1}$, and thus so is the distribution of $\env^t$.

Otherwise, from here on we assume $\at{(2)} \neq \at{(1)}$. We continue with an extensive case analysis. We define the following six events $A_1,\dots, A_6$. Each event consists of the conditions that cause the algorithm to pull a different arm in the second session and the outcome of that round:
\begin{align*}
    &A_1 := \left( \rt{(1)} \leq \frac{1}{2} \right) \wedge \left(R^{t-1}_{(1)} = R^{t-1}_{(2)}\right) \wedge  \left( R^{t-1}_{(1)} + \rt{(1)} \geq R^{t-1}_{(2)} + \rt{(2)} \right), \\
    &A_2 := \left( \rt{(1)} \leq \frac{1}{2} \right) \wedge \left(R^{t-1}_{(1)} = R^{t-1}_{(2)}\right) \wedge  \left( R^{t-1}_{(2)} + \rt{(2)} > R^{t-1}_{(1)} + \rt{(1)} \right), \\
    &A_3 := \left( \rt{(1)} \leq \frac{1}{2} \right) \wedge \left(R^{t-1}_{(1)} > R^{t-1}_{(2)} \right) \wedge \left( R^{t-1}_{(1)} - R^{t-1}_{(2)} \leq 1 - \rt{(1)} \right) \wedge \left( R^{t-1}_{(1)} + \rt{(1)} \geq R^{t-1}_{(2)} + \rt{(2)} \right), \\
    &A_4 := \left( \rt{(1)} \leq \frac{1}{2} \right) \wedge \left(R^{t-1}_{(1)} > R^{t-1}_{(2)} \right) \wedge \left( R^{t-1}_{(1)} - R^{t-1}_{(2)} \leq 1 - \rt{(1)} \right) \wedge \left( R^{t-1}_{(2)} + \rt{(2)} > R^{t-1}_{(1)} + \rt{(1)} \right), \\
    &A_5 := \left( \rt{(1)} \leq \frac{1}{2} \right) \wedge \left( R^{t-1}_{(2)} > R^{t-1}_{(1)} \right) \wedge  \left( R^{t-1}_{(2)} - R^{t-1}_{(1)} \leq \rt{(1)} \right) \wedge \left( R^{t-1}_{(1)} + \rt{(1)} \geq R^{t-1}_{(2)} + \rt{(2)} \right), \\
    &A_6 := \left( \rt{(1)} \leq \frac{1}{2} \right) \wedge \left( R^{t-1}_{(2)} > R^{t-1}_{(1)} \right) \wedge  \left( R^{t-1}_{(2)} - R^{t-1}_{(1)} \leq \rt{(1)} \right) \wedge \left( R^{t-1}_{(2)} + \rt{(2)} > R^{t-1}_{(1)} + \rt{(1)} \right).
\end{align*}
Notice that 
\begin{observation}\label{obs:partition}
Given $\at{(2)} \neq \at{(1)}$, the events $A_1, \dots, A_6$ partition the space of all options for $R^{t-1}_{(1)}, R^{t-1}_{(2)}, r^t_{(1)}, r^t_{(2)}$.
\end{observation}
Equipped with Observation~\ref{obs:partition}, we turn to analyze $\env^t$ under $A_1, \dots, A_6$. Fix any arbitrary $x \in [0,1]$.
%In each case $i$, $i\in \{1,\dots,6\}$ we condition the random variables $\env_t,R^{t-1}_1, R^{t-1}_2, r^t_{(1)}, r^t_{(2)}$ on $A_i$.
%; hence, under this event, $r^t_{(1)} \sim \uni{0,\frac{1}{2}}$ 
\begin{itemize}
    \item Case $A_1$.
    Under the conditions of event $A_1$ we have that $\rt{(1)} \geq  \rt{(2)}$.
    Recall that $\rt{(2)} \sim \uni{0,1}$, but considering the latter we know that $ \rt{(2)}\mid A_1 \sim \uni{0,\rt{(1)}}$.
    Given that $R^{t-1}_{(1)} + \rt{(1)} \geq R^{t-1}_{(2)} + \rt{(2)}$, we know the envy at the end of round $t$ is exactly $\env^t = R^{t-1}_{(1)} + \rt{(1)} - R^{t-1}_{(2)} - \rt{(2)} = \rt{(1)} -\rt{(2)}$;
    hence, $\env^{t}\mid A_1 \sim \uni{\rt{(1)} - \rt{(1)},\rt{(1)} - 0}$, i.e., $\env^{t}\mid A_1 \sim \uni{0,\rt{(1)}}$. Therefore,
    \begin{align*}
        \prb{\env^t \leq x \mid A_1} & = \prb{\uni{0, r^t_{(1)}}\leq x \mid r^t_{(1)} \leq \frac{1}{2}} \geq \prb{\uni{0, \frac{1}{2}} \leq x}
        \\& \geq \prb{\uni{0, 1} \leq x} = x.
    \end{align*}
    %Because $\rt{(1)} \leq \frac{1}{2}$, in the worst case $\env^{t}\mid A_1 \sim \uni{0,\frac{1}{2}}$.

    \item Case $A_2$.
    This case is similar to that of $A_1$, only now $ \rt{(2)}\mid A_2 \sim \uni{\rt{(1)}, 1}$ and $\env^t = \rt{(2)} -\rt{(1)}$, resulting with $\env^{t}\mid A_2 \sim \uni{\rt{(1)} - \rt{(1)},1 -\rt{(1)}}$, i.e., $\env^{t}\mid A_2 \sim \uni{0,1 -\rt{(1)}}$.
    Hence,
    \begin{align*}
        \prb{\env^t \leq x \mid A_2 } \geq \prb{\uni{0, 1} \leq x} = x.
    \end{align*}

    \item Case $A_3$.
    Under the conditions of $A_3$ we have that the envy after $t$ rounds is exactly $\env^t = R^{t-1}_{(1)} + \rt{(1)} -\left(R^{t-1}_{(2)} + \rt{(2)}\right)$.
    Since $R^{t-1}_{(1)} + \rt{(1)} \geq R^{t-1}_{(2)} + \rt{(2)}$, $\rt{(2)}$ is now a uniform random variable between $0$ and the minimum between $\left\{1,  R^{t-1}_{(1)} + \rt{(1)} - R^{t-1}_{(2)}\right\}$.
    Due to the guarantee $R^{t-1}_{(1)} - R^{t-1}_{(2)} \leq 1 - \rt{(1)}$ we can finally see that
    \[
    - \rt{(2)} \mid A_3 \sim \uni{-\left(R^{t-1}_{(1)} + \rt{(1)} - R^{t-1}_{(2)}\right), 0};
    \]
    thus,
    \[
    \env^t \mid A_3 \sim \uni{R^{t-1}_{(1)} + \rt{(1)} -R^{t-1}_{(2)} -\left(R^{t-1}_{(1)} + \rt{(1)} - R^{t-1}_{(2)}\right),  R^{t-1}_{(1)} + \rt{(1)} -R^{t-1}_{(2)}}.
    \]
    Finally,
    \begin{align*}
        \prb{\env^t \leq x | A_3} & = \prb{\uni{0, R^{t-1}_{(1)} - R^{t-1}_{(2)}+ \rt{(1)}} \leq x | A_3}
        \\& \geq \prb{\uni{0, 1} \leq x} = x.
    \end{align*}

    \item Case $A_4$.
    Under the conditions of $A_4$ we have that $\rt{(2)}$ is a uniform random variable distributed between $R^{t-1}_{(1)} - R^{t-1}_{(2)} + \rt{(1)}$ and $1$.
    The envy after round $t$ is exactly $R^{t-1}_{(2)} +\rt{(2)}- R^{t-1}_{(1)}-\rt{(1)}$ and thus it is a uniform random variable between $R^{t-1}_{(2)} - R^{t-1}_{(1)}-\rt{(1)} +\left( R^{t-1}_{(1)} - R^{t-1}_{(2)} + \rt{(1)} \right)$ and $R^{t-1}_{(2)} - R^{t-1}_{(1)}-\rt{(1)} +1$.
    I.e., $\env^t \mid A_4 \sim \uni{0,R^{t-1}_{(2)} - R^{t-1}_{(1)}-\rt{(1)} +1}$. Recall $A_4$ suggests $0 > R^{t-1}_{(2)} -R^{t-1}_{(1)} $; finally,
    \begin{align*}
        \prb{\env^t \leq x | A_4} & = \prb{\uni{0, R^{t-1}_{(2)} - R^{t-1}_{(1)}-\rt{(1)} +1} \leq x | A_4}  \\& \geq
        \prb{\uni{0, 0 - \rt{(1)} +1} \leq x } \geq \prb{\uni{0, 1} \leq x} = x,
    \end{align*}
    as $\rt{(1)}\geq 0$.
    Note that under $A_4$ it holds that $R^{t-1}_{(1)} - R^{t-1}_{(2)} \leq 1 - \rt{(1)} $ and thus $0 \leq R^{t-1}_{(2)} - R^{t-1}_{(1)}-\rt{(1)} +1$ almost surely.
    
    \item Case $A_5$.
    Under the conditions of $A_5$, similarly to $A_3$, the envy at the end of round $t$ is exactly $\env^t = R^{t-1}_{(1)} + \rt{(1)} - R^{t-1}_{(2)} + \rt{(2)}$ and
    \[
    - \rt{(2)} \mid A_5 \sim \uni{-\left(R^{t-1}_{(1)} + \rt{(1)} - R^{t-1}_{(2)}\right), 0};
    \]
    thus,
    \[
    \env^t \mid A_5 \sim \uni{R^{t-1}_{(1)} + \rt{(1)} -R^{t-1}_{(2)} -\left(R^{t-1}_{(1)} + \rt{(1)} - R^{t-1}_{(2)}\right),  R^{t-1}_{(1)} + \rt{(1)} -R^{t-1}_{(2)}}.
    \]
    Finally,
    \begin{align*}
        \prb{\env^t \leq x | A_5} & = \prb{\uni{0, R^{t-1}_{(1)} - R^{t-1}_{(2)}+ \rt{(1)}} \leq x | A_5}
        \\& \geq \prb{\uni{0, 0 + \rt{(1)}} \leq x | A_5}
        \geq \prb{\uni{0, \frac{1}{2}} \leq x} 
        \\& \geq \prb{\uni{0, 1} \leq x} = x.
    \end{align*}

    \item Case $A_6$.
    Under the conditions of $A_6$, similarly to $A_4$, the envy at the end of round $t$ is exactly $\env^t = R^{t-1}_{(2)} + \rt{(2)} - R^{t-1}_{(1)} - \rt{(1)}$ and
    \[
    \rt{(2)} \mid A_6 \sim \uni{R^{t-1}_{(1)}+\rt{(1)} -R^{t-1}_{(2)}, 1};
    \]
    thus,
    \[
    \env^t \mid A_6 \sim \uni{R^{t-1}_{(2)} - R^{t-1}_{(1)} - \rt{(1)} + R^{t-1}_{(1)}+\rt{(1)} -R^{t-1}_{(2)}, R^{t-1}_{(2)} - R^{t-1}_{(1)} - \rt{(1)} + 1}.
    \]
    Finally,
    \begin{align*}
        \prb{\env^t \leq x | A_6} & = \prb{\uni{0,R^{t-1}_{(2)} - R^{t-1}_{(1)} - \rt{(1)} + 1} \leq x | A_6}
        \\& \geq \prb{\uni{0,1} \leq x} = x,
    \end{align*}
    where the inequality holds due to $R^{t-1}_{(2)} - R^{t-1}_{(1)} \leq \rt{(1)}$.    
\end{itemize}
We have shown that the inductive step holds under all cases; thereby, the proof of Proposition~\ref{prop:ef1 uni dominance} is complete.
\end{proof}
\begin{proof}[Proof of Proposition~\ref{prop:ef1 open arm}]
We prove the statement using case analysis. We partition the space of events $a^t_{(2)} \neq a^t_{(1)}$ conditioning on $\rt{(1)} \leq \frac{1}{2}$:
\begin{align*}
    &B_1 := R^{t-1}_{(1)} = R^{t-1}_{(2)} \\
    &B_2 := \left(R^{t-1}_{(1)} > R^{t-1}_{(2)} \right) \wedge \left( R^{t-1}_{(1)} - R^{t-1}_{(2)} \leq 1 - \rt{(1)} \right) \\
    &B_3 := \left( R^{t-1}_{(2)} > R^{t-1}_{(1)} \right) \wedge  \left( R^{t-1}_{(2)} - R^{t-1}_{(1)} \leq \rt{(1)} \right).
\end{align*}
Therefore
\begin{align*}
    \prb{a^t_{(2)} \neq a^t_{(1)} \mid \rt{(1)} \leq \frac{1}{2}} & = \prb{ B_1\vee B_2 \vee B_3  \mid \rt{(1)} \leq \frac{1}{2}}
    \\& = \prb{B_1 \mid \rt{(1)} \leq \frac{1}{2} } + \prb{B_2 \mid \rt{(1)} \leq \frac{1}{2} } + \prb{B_3 \mid \rt{(1)} \leq \frac{1}{2} }.
\end{align*}
We prove each part separately, beginning with the event $B_1$.

Since the distributions of the rewards are continuous, event $B_1$ occurs if and only if until round $t$ both agents receive the same rewards from the same arm.
In this case, the algorithm pulls the same arm in both sessions if it yields a reward greater than $\frac{1}{2}$.
Therefore, we must have
\[
    \prb{r_{(1)}^\tau  = r_{(2)}^\tau} = \prb{r_{(1)}^\tau  > \frac{1}{2}}
\]
for all $\tau < t$; hence,
\begin{align}\label{B1}
\prb{B_1 \mid \rt{(1)} > \frac{1}{2}} =\prb{B_1} = \prb{\forall \tau<t : r_{(2)}^\tau > \frac{1}{2}} = \left(\frac{1}{2}\right)^{t-1}.
\end{align}

Next, we examine event $B_2$. Using Bayes formula,
\begin{align*}
    &\prb{B_2 \mid \rt{(1)} \leq \frac{1}{2} } =
    \prb{ \left(R^{t-1}_{(1)} > R^{t-1}_{(2)} \right) \wedge \left( R^{t-1}_{(1)} - R^{t-1}_{(2)} \leq 1 - \rt{(1)} \right) \mid \rt{(1)} \leq \frac{1}{2} }
    \\& = \prb{R^{t-1}_{(1)} - R^{t-1}_{(2)} \leq 1 - \rt{(1)} \mid R^{t-1}_{(1)} > R^{t-1}_{(2)}, \rt{(1)} \leq \frac{1}{2} }\cdot \prb{R^{t-1}_{(1)} > R^{t-1}_{(2)} \mid \rt{(1)} \leq \frac{1}{2}}.
\end{align*}
Notice that given $\rt{(1)} \leq \frac{1}{2}$, the random variable $1- \rt{(1)}$ is $\uni{\frac{1}{2}, 1}$ distributed. Similar arguments holds for $\rt{(1)} \mid \rt{(1)} > \frac{1}{2}$; thus, 
\begin{align*}
&\prb{R^{t-1}_{(1)} - R^{t-1}_{(2)} \leq 1 - \rt{(1)} \mid R^{t-1}_{(1)} > R^{t-1}_{(2)}, \rt{(1)} \leq \frac{1}{2} }
\\&=
\prb{R^{t-1}_{(1)} - R^{t-1}_{(2)} \leq \rt{(1)} \mid R^{t-1}_{(1)} > R^{t-1}_{(2)}, \rt{(1)} > \frac{1}{2} }.
\end{align*}
Simplifying the above,
\begin{align}\label{B2}
    & \prb{B_2 \mid \rt{(1)} \leq \frac{1}{2} }
    \nonumber\\&
    = \prb{R^{t-1}_{(1)} - R^{t-1}_{(2)} \leq \rt{(1)} \mid R^{t-1}_{(1)} > R^{t-1}_{(2)}, \rt{(1)} > \frac{1}{2} }\cdot \prb{R^{t-1}_{(1)} > R^{t-1}_{(2)} \mid \rt{(1)} \leq \frac{1}{2}}\nonumber \\
    & = \prb{R^{t-1}_{(1)} - R^{t-1}_{(2)} \leq \rt{(1)} \mid R^{t-1}_{(1)} > R^{t-1}_{(2)}, \rt{(1)} > \frac{1}{2} }\cdot \prb{R^{t-1}_{(1)} > R^{t-1}_{(2)}}
    .
\end{align}

As for event $B_3$, recall that the arrival order is uniform. As a result, $R^{t-1}_{(1)}, R^{t-1}_{(2)}$ are independent in $\rt{(1)}$. Leveraging this fact,
\begin{align}\label{B3}
    & \prb{B_3 \mid \rt{(1)} \leq \frac{1}{2} } =
    \prb{\left( R^{t-1}_{(2)} > R^{t-1}_{(1)} \right) \wedge  \left( R^{t-1}_{(2)} - R^{t-1}_{(1)} \leq \rt{(1)} \right) \mid \rt{(1)}> \frac{1}{2}}\nonumber \\
    & =  \prb{\left( R^{t-1}_{(1)} > R^{t-1}_{(2)} \right) \wedge  \left( R^{t-1}_{(1)} - R^{t-1}_{(2)} \leq \rt{(1)} \right) \mid \rt{(1)}> \frac{1}{2}}\nonumber \\
    & = \prb{R^{t-1}_{(1)} - R^{t-1}_{(2)} \leq \rt{(1)} \mid R^{t-1}_{(1)} > R^{t-1}_{(2)}, \rt{(1)} \leq \frac{1}{2} }\cdot \prb{R^{t-1}_{(1)} > R^{t-1}_{(2)} \mid \rt{(1)} \leq \frac{1}{2}}\nonumber\\
    & = \prb{R^{t-1}_{(1)} - R^{t-1}_{(2)} \leq \rt{(1)} \mid R^{t-1}_{(1)} > R^{t-1}_{(2)}, \rt{(1)} \leq \frac{1}{2} }\cdot \prb{R^{t-1}_{(1)} > R^{t-1}_{(2)}},
\end{align}
where the last two equalities hold from the same arguments as in the analysis of event $B_2$. Combining Equalities \eqref{B2} and \eqref{B3}, we get
\begin{align*}
 \prb{B_2 \mid \rt{(1)} \leq \frac{1}{2} } & + \prb{B_3 \mid \rt{(1)} \leq \frac{1}{2} }\\ &= \prb{R^{t-1}_{(1)} > R^{t-1}_{(2)}} \cdot\left(
 \prb{R^{t-1}_{(1)} - R^{t-1}_{(2)}   \leq \rt{(1)} \mid R^{t-1}_{(1)} > R^{t-1}_{(2)}, \rt{(1)} > \frac{1}{2} } \right. \\& + \left.
  \prb{R^{t-1}_{(1)} - R^{t-1}_{(2)} \leq \rt{(1)} \mid R^{t-1}_{(1)} > R^{t-1}_{(2)}, \rt{(1)} \leq \frac{1}{2} }\right) \\ & = 
  \frac{1}{2} \left(1 - \frac{1}{2^{t-1}} \right) \cdot\left(
 \prb{R^{t-1}_{(1)} - R^{t-1}_{(2)}   \leq \rt{(1)} \mid R^{t-1}_{(1)} > R^{t-1}_{(2)}, \rt{(1)} > \frac{1}{2} } \right. \\& + \left.
  \prb{R^{t-1}_{(1)} - R^{t-1}_{(2)} \leq \rt{(1)} \mid R^{t-1}_{(1)} > R^{t-1}_{(2)}, \rt{(1)} \leq \frac{1}{2} }\right),
\end{align*}
where the last equality is due to Equation~\eqref{B1}. Next, notice that 
\begin{align*}
\frac{1}{2} \left(1 - \frac{1}{2^{t-1}} \right) &\left(
\prb{R^{t-1}_{(1)} - R^{t-1}_{(2)}   \leq \rt{(1)} \mid R^{t-1}_{(1)} > R^{t-1}_{(2)}, \rt{(1)} > \frac{1}{2} } \right. \\+ & \left.
\prb{R^{t-1}_{(1)} - R^{t-1}_{(2)} \leq \rt{(1)} \mid R^{t-1}_{(1)} > R^{t-1}_{(2)}, \rt{(1)} \leq \frac{1}{2} }\right)\\=
\frac{1}{2} \left(1 - \frac{1}{2^{t-1}} \right) &\left(
\prb{R^{t-1}_{(1)} - R^{t-1}_{(2)}   \leq \rt{(1)} \mid R^{t-1}_{(1)} > R^{t-1}_{(2)}, \rt{(1)} > \frac{1}{2} } \frac{\Pr\left(r_{(1)}^t > \frac{1}{2}\Big| R_1^{t-1} > R_2^{t-1} \right)}{\Pr\left(r_{(1)}^t > \frac{1}{2}\Big| R_1^{t-1} > R_2^{t-1} \right)} \right. \\+ & \left.
\prb{R^{t-1}_{(1)} - R^{t-1}_{(2)} \leq \rt{(1)} \mid R^{t-1}_{(1)} > R^{t-1}_{(2)}, \rt{(1)} \leq \frac{1}{2} } \frac{\Pr\left(r_{(1)}^t\leq \frac{1}{2}\Big| R_1^{t-1} > R_2^{t-1} \right)}{\Pr\left(r_{(1)}^t\leq \frac{1}{2}\Big| R_1^{t-1} > R_2^{t-1} \right)} \right)
\\ = \frac{1}{2}  \left(1 - \frac{1}{2^{t-1}} \right) &\cdot \frac{1}{\frac{1}{2}} \cdot 
\prb{R^{t-1}_{(1)} - R^{t-1}_{(2)} \leq \rt{(1)} \mid R^{t-1}_{(1)} > R^{t-1}_{(2)} },
\end{align*}
where the last equation is based on the law of total probability and the fact that 
\[
\prb{\rt{(1)} > \frac{1}{2} \mid R^{t-1}_{(1)} > R^{t-1}_{(2)}} = \prb{\rt{(1)} \leq \frac{1}{2} \mid R^{t-1}_{(1)} > R^{t-1}_{(2)}} = \frac{1}{2}.
\]
Combining the latter with Equation~\eqref{B1}, we have
\begin{align*}
\prb{a^t_{(2)} \neq a^t_{(1)} \mid \rt{(1)} \leq \frac{1}{2}} & = \left(\frac{1}{2}\right)^{t-1}+ \left(1 - \frac{1}{2^{t-1}} \right)  
\prb{R^{t-1}_{(1)} - R^{t-1}_{(2)} \leq \rt{(1)} \mid R^{t-1}_{(1)} > R^{t-1}_{(2)} }\\
& = \left(\frac{1}{2}\right)^{t-1}+ \left(1 - \frac{1}{2^{t-1}} \right)  
\prb{\abs{R^{t-1}_{(1)} - R^{t-1}_{(2)}} \leq \rt{(1)} \mid R^{t-1}_{(1)} > R^{t-1}_{(2)} }\\
& = \left(\frac{1}{2}\right)^{t-1}+ \left(1 - \frac{1}{2^{t-1}} \right) 
\prb{\env^{t-1} \leq \rt{(1)} }\geq \prb{\env^{t-1} \leq \rt{(1)} }.
\end{align*}
To finish the proof we use Proposition~\ref{prop:ef1 uni dominance}, which implies that
\begin{align*}
    \prb{\env^{t-1} \leq \rt{(1)} } & = \int_{0}^{1} \prb{\env^{t-1} \leq u}\cdot f_{\uni{0,1}} \,du
    \geq \int_{0}^{1} u\cdot f_{\uni{0,1}} \,du
    \\&=\E{\uni{0,1}} = \frac{1}{2};
\end{align*}
thus,
\begin{align*}
\prb{a^t_{(2)} \neq a^t_{(1)} \mid \rt{(1)} \leq \frac{1}{2}} \geq 
\prb{\env^{t-1} \leq \rt{(1)} } \geq
\frac{1}{2}.
\end{align*}
This concludes the proof of Proposition~\ref{prop:ef1 open arm}.
\end{proof}
