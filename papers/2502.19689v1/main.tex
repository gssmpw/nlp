%% 
%% Copyright 2019-2021 Elsevier Ltd
%% 
%% This file is part of the 'CAS Bundle'.
%% --------------------------------------
%% 
%% It may be distributed under the conditions of the LaTeX Project Public
%% License, either version 1.2 of this license or (at your option) any
%% later version.  The latest version of this license is in
%%    http://www.latex-project.org/lppl.txt
%% and version 1.2 or later is part of all distributions of LaTeX
%% version 1999/12/01 or later.
%% 
%% The list of all files belonging to the 'CAS Bundle' is
%% given in the file `manifest.txt'.
%% 
%% Template article for cas-sc documentclass for 
%% single column output.

\documentclass[a4paper]{cas-sc}
\usepackage{graphicx}   
\usepackage{subcaption} 
% If the frontmatter runs over more than one page
% use the longmktitle option.

%\documentclass[a4paper,fleqn,longmktitle]{cas-sc}

\usepackage[numbers]{natbib}
%\usepackage[authoryear]{natbib}
%\usepackage[authoryear,longnamesfirst]{natbib}

%%%Author macros
\def\tsc#1{\csdef{#1}{\textsc{\lowercase{#1}}\xspace}}
\tsc{WGM}
\tsc{QE}
%%%

% Uncomment and use as if needed
%\newtheorem{theorem}{Theorem}
%\newtheorem{lemma}[theorem]{Lemma}
%\newdefinition{rmk}{Remark}
%\newproof{pf}{Proof}
%\newproof{pot}{Proof of Theorem \ref{thm}}

\begin{document}
\let\WriteBookmarks\relax
\def\floatpagepagefraction{1}
\def\textpagefraction{.001}
\let\printorcid\relax 

% Short title
% \shorttitle{<short title of the paper for running head>}    
\shorttitle{3D Trajectory Reconstruction of Moving Points Based on a Monocular Camera}   

%Short author
%\shortauthors{<short author list for running head>} 
%\shortauthors{}
\shortauthors{Huang et al.}

% Main title of the paper
\title[mode = title]{3D Trajectory Reconstruction of Moving Points Based on a Monocular Camera}  

\author[1,2]{Huayu Huang}

\author[1,2]{Banglei Guan}

\author[1,2]{Yang Shang}

\author[1,2]{Qifeng Yu}

\address[1]{College of Aerospace Science and Engineering, National University of Defense Technology, Changsha, China}
\address[2]{Hunan Provincial Key Laboratory of Image Measurement and Vision Navigation, National University of Defense Technology, Changsha, China}

% Title footnote mark
% eg: \tnotemark[1]
% \tnotemark[<tnote number>] 


% Title footnote 1.
% eg: \tnotetext[1]{Title footnote text}
% \tnotetext[<tnote number>]{<tnote text>} 

% First author
%
% Options: Use if required
% eg: \author[1,3]{Author Name}[type=editor,
%       style=chinese,
%       auid=000,
%       bioid=1,
%       prefix=Sir,
%       orcid=0000-0000-0000-0000,
%       facebook=<facebook id>,
%       twitter=<twitter id>,
%       linkedin=<linkedin id>,
%       gplus=<gplus id>]

% \author[<aff no>]{<author name>}[<options>]

% Corresponding author indication
% \cormark[<corr mark no>]

% Footnote of the first author
% \fnmark[<footnote mark no>]

% Email id of the first author
% \ead{<email address>}

% URL of the first author
% \ead[url]{<URL>}

% Credit authorship
% eg: \credit{Conceptualization of this study, Methodology, Software}
% \credit{<Credit authorship details>}

% Address/affiliation
% \affiliation[<aff no>]{organization={},
%             addressline={}, 
%             city={},
% %          citysep={}, % Uncomment if no comma needed between city and postcode
%             postcode={}, 
%             state={},
%             country={}}

% \author[<aff no>]{<author name>}[<options>]

% Footnote of the second author
% \fnmark[2]

% Email id of the second author
% \ead{}

% URL of the second author
% \ead[url]{}

% Credit authorship
% \credit{}

% Address/affiliation
% \affiliation[<aff no>]{organization={},
%             addressline={}, 
%             city={},
% %          citysep={}, % Uncomment if no comma needed between city and postcode
%             postcode={}, 
%             state={},
%             country={}}

% Corresponding author text
% \cortext[1]{Corresponding author}

% Footnote text
% \fntext[1]{}

% For a title note without a number/mark
%\nonumnote{}

% Here goes the abstract
\begin{abstract}


The choice of representation for geographic location significantly impacts the accuracy of models for a broad range of geospatial tasks, including fine-grained species classification, population density estimation, and biome classification. Recent works like SatCLIP and GeoCLIP learn such representations by contrastively aligning geolocation with co-located images. While these methods work exceptionally well, in this paper, we posit that the current training strategies fail to fully capture the important visual features. We provide an information theoretic perspective on why the resulting embeddings from these methods discard crucial visual information that is important for many downstream tasks. To solve this problem, we propose a novel retrieval-augmented strategy called RANGE. We build our method on the intuition that the visual features of a location can be estimated by combining the visual features from multiple similar-looking locations. We evaluate our method across a wide variety of tasks. Our results show that RANGE outperforms the existing state-of-the-art models with significant margins in most tasks. We show gains of up to 13.1\% on classification tasks and 0.145 $R^2$ on regression tasks. All our code and models will be made available at: \href{https://github.com/mvrl/RANGE}{https://github.com/mvrl/RANGE}.

\end{abstract}

 

% Use if graphical abstract is present
%\begin{graphicalabstract}
%\includegraphics{}
%\end{graphicalabstract}

% Keywords
% Each keyword is seperated by \sep
\begin{keywords}
dynamic 3D reconstruction \sep 
monocular vision \sep 
trajectory intersection \sep 
ridge estimation \sep 
reconstructibility
\end{keywords}

\maketitle

% Main text
\section{Introduction}
Backdoor attacks pose a concealed yet profound security risk to machine learning (ML) models, for which the adversaries can inject a stealth backdoor into the model during training, enabling them to illicitly control the model's output upon encountering predefined inputs. These attacks can even occur without the knowledge of developers or end-users, thereby undermining the trust in ML systems. As ML becomes more deeply embedded in critical sectors like finance, healthcare, and autonomous driving \citep{he2016deep, liu2020computing, tournier2019mrtrix3, adjabi2020past}, the potential damage from backdoor attacks grows, underscoring the emergency for developing robust defense mechanisms against backdoor attacks.

To address the threat of backdoor attacks, researchers have developed a variety of strategies \cite{liu2018fine,wu2021adversarial,wang2019neural,zeng2022adversarial,zhu2023neural,Zhu_2023_ICCV, wei2024shared,wei2024d3}, aimed at purifying backdoors within victim models. These methods are designed to integrate with current deployment workflows seamlessly and have demonstrated significant success in mitigating the effects of backdoor triggers \cite{wubackdoorbench, wu2023defenses, wu2024backdoorbench,dunnett2024countering}.  However, most state-of-the-art (SOTA) backdoor purification methods operate under the assumption that a small clean dataset, often referred to as \textbf{auxiliary dataset}, is available for purification. Such an assumption poses practical challenges, especially in scenarios where data is scarce. To tackle this challenge, efforts have been made to reduce the size of the required auxiliary dataset~\cite{chai2022oneshot,li2023reconstructive, Zhu_2023_ICCV} and even explore dataset-free purification techniques~\cite{zheng2022data,hong2023revisiting,lin2024fusing}. Although these approaches offer some improvements, recent evaluations \cite{dunnett2024countering, wu2024backdoorbench} continue to highlight the importance of sufficient auxiliary data for achieving robust defenses against backdoor attacks.

While significant progress has been made in reducing the size of auxiliary datasets, an equally critical yet underexplored question remains: \emph{how does the nature of the auxiliary dataset affect purification effectiveness?} In  real-world  applications, auxiliary datasets can vary widely, encompassing in-distribution data, synthetic data, or external data from different sources. Understanding how each type of auxiliary dataset influences the purification effectiveness is vital for selecting or constructing the most suitable auxiliary dataset and the corresponding technique. For instance, when multiple datasets are available, understanding how different datasets contribute to purification can guide defenders in selecting or crafting the most appropriate dataset. Conversely, when only limited auxiliary data is accessible, knowing which purification technique works best under those constraints is critical. Therefore, there is an urgent need for a thorough investigation into the impact of auxiliary datasets on purification effectiveness to guide defenders in  enhancing the security of ML systems. 

In this paper, we systematically investigate the critical role of auxiliary datasets in backdoor purification, aiming to bridge the gap between idealized and practical purification scenarios.  Specifically, we first construct a diverse set of auxiliary datasets to emulate real-world conditions, as summarized in Table~\ref{overall}. These datasets include in-distribution data, synthetic data, and external data from other sources. Through an evaluation of SOTA backdoor purification methods across these datasets, we uncover several critical insights: \textbf{1)} In-distribution datasets, particularly those carefully filtered from the original training data of the victim model, effectively preserve the model’s utility for its intended tasks but may fall short in eliminating backdoors. \textbf{2)} Incorporating OOD datasets can help the model forget backdoors but also bring the risk of forgetting critical learned knowledge, significantly degrading its overall performance. Building on these findings, we propose Guided Input Calibration (GIC), a novel technique that enhances backdoor purification by adaptively transforming auxiliary data to better align with the victim model’s learned representations. By leveraging the victim model itself to guide this transformation, GIC optimizes the purification process, striking a balance between preserving model utility and mitigating backdoor threats. Extensive experiments demonstrate that GIC significantly improves the effectiveness of backdoor purification across diverse auxiliary datasets, providing a practical and robust defense solution.

Our main contributions are threefold:
\textbf{1) Impact analysis of auxiliary datasets:} We take the \textbf{first step}  in systematically investigating how different types of auxiliary datasets influence backdoor purification effectiveness. Our findings provide novel insights and serve as a foundation for future research on optimizing dataset selection and construction for enhanced backdoor defense.
%
\textbf{2) Compilation and evaluation of diverse auxiliary datasets:}  We have compiled and rigorously evaluated a diverse set of auxiliary datasets using SOTA purification methods, making our datasets and code publicly available to facilitate and support future research on practical backdoor defense strategies.
%
\textbf{3) Introduction of GIC:} We introduce GIC, the \textbf{first} dedicated solution designed to align auxiliary datasets with the model’s learned representations, significantly enhancing backdoor mitigation across various dataset types. Our approach sets a new benchmark for practical and effective backdoor defense.


 
\section{Background}
% \begin{tcolorbox}[simplebox]
% We first formally define the problem and highlight its challenge. 
% Then we present an EM approach to address this challenge. 
% \end{tcolorbox}
% \vspace{-0.3cm}
% \subsection{Problem Statement }\label{sec_ps}

% Here’s a polished and enriched version of your problem formulation section, with improved clarity, precision, and academic tone:

% ---
\begin{figure}[t]
    \centering % Center the figure
    \includegraphics[width=\linewidth]{figs/example.pdf} % Include the figure
    \caption{\small \textbf{Example of Autonomous Code Integration.} \small We aim to enable LLMs to determine tool-usage strategies
based on their own capability boundaries. In the example, the model write code to solve the problem that demand special tricks, strategically bypassing its inherent limitations.} 
    \label{fig_example}
    \vspace{-0.2cm}
\end{figure}
\textbf{Problem Statement.} Modern tool-augmented language models address mathematical problems \( x_q \in \mathcal{X}_Q \) by generating step-by-step solutions that interleave natural language reasoning with executable Python code (Fig.~\ref{fig_example}). Formally, given a problem \( x_q \), a model \( \mathcal{M}_\theta \) iteratively constructs a solution \( y_a = \{y_1, \dots, y_T\} \) by sampling components \( y_t \sim p(y_t | y_{<t}, x_q) \), where \( y_{<t} \) encompasses both prior reasoning steps, code snippets and execution results \( \mathbf{e}_t \) from a Python interpreter. The process terminates upon generating an end token, and the solution is evaluated via a binary reward \( r(y_a,x_q) = \mathbb{I}(y_a \equiv y^*) \) indicating equivalence to the ground truth \( y^* \). The learning objective is formulated as:
\[
\max_{\theta} \mathbb{E}_{x_q \sim \mathcal{X}_Q} \left[r(y_a, x_q) \right]
\]

\noindent\textbf{Challenge and Motivation.} Developing autonomous code integration (AutoCode) strategies poses unique challenges, as optimal tool-usage behaviors must dynamically adapt to a model's intrinsic capabilities and problem-solving contexts. While traditional supervised fine-tuning (SFT) relies on imitation learning from expert demonstrations, this paradigm fundamentally limits the emergence of self-directed tool-usage strategies. Unfortunately, current math LLMs predominantly employ SFT to orchestrate tool integration~\citep{mammoth, tora, dsmath, htl}, their rigid adherence to predefined reasoning templates therefore struggles with the dynamic interplay between a model’s evolving problem-solving competencies and the adaptive tool-usage strategies required for diverse mathematical contexts.

Reinforcement learning (RL) offers a promising alternative by enabling trial-and-error discovery of autonomous behaviors. Recent work like DeepSeek-R1~\citep{dsr1} demonstrates RL's potential to enhance reasoning without expert demonstrations. However, we observe that standard RL methods (e.g., PPO~\cite{ppo}) suffer from a critical inefficiency (see Sec.~\ref{sec_ablation}): Their tendency to exploit local policy neighborhoods leads to insufficient exploration of the vast combinatorial space of code-integrated reasoning paths, especially when only given a terminal reward in mathematical problem-solving.

To bridge this gap, we draw inspiration from human metacognition -- the iterative process where learners refine tool-use strategies through deliberate exploration, outcome analysis, and belief updates. A novice might initially attempt manual root-finding via algebraic methods, observe computational bottlenecks or inaccuracies, and therefore prompting the usage of calculators. Through systematic reflection on these experiences, they internalize the contextual efficacy of external tools, gradually forming stable heuristics that balance reasoning with judicious tool invocation. 


To this end, \emph{our focus diverges from standard agentic tool-use frameworks~\citep{agentr}}, which merely prioritize successful tool execution. Instead, \emph{we aim to instill \emph{human-like metacognition} in LLMs, enabling them to (1) determine tool-usage based on their own capability boundaries (see the analysis in Sec.~\ref{sec_ablation}), and (2) dynamically adapt tool-usage strategies as their reasoning abilities evolve (via our EM framework).}
% For instance, while an LLM might solve a combinatorics problem via CoT alone, it should autonomously invoke code for eigenvalue calculations in linear algebra where symbolic computations are error-prone. Achieving this requires models to \emph{jointly optimize} their reasoning and tool-integration policies in a mutually reinforcing manner.


% Mirroring this metacognitive cycle, we propose an Expectation-Maximization (EM) framework that allows LLMs to develop AutoCode strategies via guided exploration (the E-step) and self-refinement (the M-step).


% \vspace{-0.3cm}
\section{Methodology}

Inspired by human metacognitive processes, we introduce an Expectation-Maximization (EM) framework that trains LLMs for autonomous code integration (AutoCode) through alternations (Fig.~\ref{fig_overview}):

\begin{enumerate}[leftmargin=0.5cm,topsep=1pt,itemsep=0pt,parsep=0pt]
    \item \emph{Guided Exploration (E-step):} Identifies high-potential code-integrated solutions by systematically probing the model's inherent capabilities.
\item \emph{Self-Refinement (M-step):} Optimizes the model's tool-usage strategy and chain-of-thought reasoning using curated trajectories from the E-step.
\end{enumerate}


\begin{figure*}[t]
    \centering
    \includegraphics[width=\linewidth]{figs/overview.pdf}
    \caption{\small \textbf{Method Overview.} \small (Left) shows an overview for the EM framework, which alternates between finding a reference strategy for guided exploration (E-step) and off-policy RL (M-step). (Right) shows the data curation for guided exploration. We generate \(K\) rollouts, estimate values of code-triggering decisions and subsample the initial data with sampling weights per Eq.~\ref{eq_sampling}.}
    \label{fig_overview}
\end{figure*}

\subsection{The EM Framework for AutoCode}

A central challenge in AutoCode lies in the code triggering decisions, represented by the binary decision \(c \in \{0, 1\}\).  While supervised fine-tuning (SFT) suffers from missing ground truth for these decisions, standard reinforcement learning (RL) struggles with the combinatorial explosion of code-integrated reasoning paths. Our innovation bridges these approaches through systematic exploration of both code-enabled (\(c=1\)) and non-code (\(c=0\)) solution paths, constructing reference decisions for policy optimization.

We formalize this idea within a maximum likelihood estimation (MLE) framework. Let \( P (r=1 | x_q;\theta\) denote the probability of generating a correct response to query \( x_q \) under model \(\mathcal{M}_\theta\). Our objective becomes:
\begin{align}
    \mathcal{J}_{\mathrm{MLE}}(\theta) \doteq \log P(r=1 | x_q; \theta) \label{eq_mle}
\end{align}
This likelihood depends on two latent factors: (1) the code triggering decision \(\pi_\theta(c | x_q)\) and (2) the solution generation process \(\pi_\theta(y_a | x_q, c)\). Here, for notation-wise clarity, we consider  code-triggering decision at a solution's beginning (\( c\) following \(x_q\) immediately). We show generalization to mid-reasoning code integration in Sec.~\ref{sec_impl}.

The EM framework provides a principled way to optimize this MLE objective in the presence of latent variables~\cite{prml}. We derive the evidence lower bound (ELBO): \( \mathcal{J}_{\mathrm{ELBO}}(s, \theta) \doteq \)
\begin{align}
    % \mathcal{J}_{\mathrm{MLE}}(\theta) &
    % \ge 
    \mathbb{E}_{s(c | x_q)}\left[\log \frac{\pi_\theta(c | x_q) \cdot P(r=1 | c, x_q; \theta)}{s(c | x_q)}\right] 
    % \\
     \label{eq_elbo}
\end{align}
where \(s(c | x_q)\) serves as a surrogate distribution approximating optimal code triggering strategies. It is also considered as the reference decisions for code integration. 

\noindent\textbf{E-step: Guided Exploration}  computes the reference strategy \(s(c | x_q)\) by maximizing the ELBO, equivalent to minimizing the KL-divergence: \( \max_s \mathcal{J}_{\mathrm{ELBO}}(s, \theta) = \)
\begin{align}
     - \mathrm{D_{KL}}\left(s(c | x_q) \| P(r=1, c | x_q; \theta)\right) \label{eq_estep}
\end{align}

The reference strategy \(s(c | x_q)\) thus approximates the posterior distribution over code-triggering decisions \(c\) that maximize correctness, i.e., \(P(r=1, c | x_q; \theta)\).  Intuitively, it guides exploration by prioritizing decisions with high potential: if decision \(c\) is more likely to lead to correct solutions, the reference strategy assigns higher probability mass to it, providing guidance for the subsequent RL procedure.

\noindent\textbf{M-step: Self-Refinement } updates the model parameters \(\theta\) through a composite objective:
\begin{multline}
\max_\theta \mathcal{J}_{\mathrm{ELBO}}(s, \theta) =\mathbb{E}_{\substack{c \sim s(c|x_q) \\ y_a \sim \pi_\theta(y_a|x_q, c)}} \Big[ r(x_q, y_a) \Big] \\- \mathcal{CE}\Big(s(c|x_q) \,\|\, \pi_\theta(c|x_q)\Big)\label{eq_mstep}
\end{multline}
The first term implements reward-maximizing policy gradient updates for solution generation, while while the second aligns native code triggering with reference strategies through cross-entropy minimization (see Fig.~\ref{fig_overview} for an illustration of the optimization). This dual optimization jointly enhances both tool-usage policies and reasoning capabilities.



\subsection{Practical Implementation}\label{sec_impl}
In the above EM framework, we alternate between finding a reference strategy \( s \) for code-triggering decisions  in the E-step, and perform reinforcement learning under the guidance from \( s \) in the M-step. We implement this framework through an iterative process of offline data curation and off-policy RL.

\noindent\textbf{Offline Data Curation.} We implement the E-step through Monte Carlo rollouts and subsampling. For each problem \(x_q\), we estimate the reference strategy as an energy distribution: 
\begin{equation}
    s^\ast(c | x_q)  = \frac{\exp\left(\alpha\cdot \pi_\theta(c | x_q) Q(x_q,c;\theta)\right)}{Z(x_q)}.\label{eq_sampling}
\end{equation}
where \( Q(x_q,c;\theta)\) estimates the expected value through \( K \) rollouts per decision, \(\pi_\theta(c|x_q) \) represents the model's current prior and the \( Z(x_q) \) is the partition function to ensure normalization. Intuitively, the strategy will assign higher probability mass to the decision \( c \) that has higher expected value \( Q(x_q,c;\theta)\) meanwhile balancing its intrinsic preference \( \pi_\theta(c|x_q)\). 

Our curation pipeline proceeds through: 
\begin{itemize}[leftmargin=0.5cm,topsep=1pt,itemsep=0pt,parsep=0pt]
\item Generate \(K\) rollouts for \(c=0\) (pure reasoning) and \(c=1\) (code integration), creating candidate dataset \(\mathcal{D}\).  
\item Compute \(Q(x_q,c)\) as the expected success rate across rollouts for each pair \((x_q,c)\).  
\item Subsample \(\mathcal{D}_{\text{train}}\) from \(\mathcal{D}\) using importance weights according to Eq.~\ref{eq_sampling}.  
\end{itemize}

To explicitly probe code-integrated solutions, we employ prefix-guided generation -- e.g., prepending prompts like \texttt{``Let’s first analyze the problem, then consider if python code could help''} -- to bias generations toward free-form code-reasoning patterns.

 This pipeline enables guided exploration by focusing on high-potential code-integrated trajectories identified by the reference strategy, contrasting with standard RL’s reliance on local policy neighborhoods. As demonstrated in Sec.~\ref{sec_ablation}, this strategic data curation significantly improves training efficiency by shaping the exploration space.





\noindent\textbf{Off-Policy RL.}
To mitigate distributional shifts caused by mismatches between offline data and the policy, we optimize a clipped off-policy RL objective. The refined M-step (Eq.~\ref{eq_mstep}) becomes:
\begin{multline}
    % \max_\theta 
    \underset{(x_q,y_a)}{\mathbb{E}}\left[
\text{clip}\left(\frac{\pi_\theta(y_a|x_q)}{\pi_{\text{ref}}(y_a|x_q)},1-\epsilon,1+\epsilon\right)\cdot A\right]
\\-\mathbb{E}_{(x_q,c)}\Big[\log \pi_\theta(c|x_q) \Big]\label{eq_finalm}
\end{multline}
where  \( (x_q, c, y_a) \) is sampled from the dataset \( \mathcal{D}_{\text{train}} \). The importance weight \(\frac{\pi_\theta(y_a|x_q)}{\pi_{\text{ref}}(y_a|x_q)}\) accounts for off-policy correction with PPO-like clipping. The advantage function \(A(x_q,y_a)\) is computed via query-wise reward normalization~\cite{ppo}. 

\noindent\textbf{Generalizing to Mid-Reasoning Code Integration.} Our method extends to mid-reasoning code integration by initiating Monte Carlo rollouts from partial solutions \((x_q, y_{<t})\). Notably, we observe emergence of mid-reasoning code triggers after initial warm-up with prefix-probed solutions. Thus, our implementation requires only two initial probing strategies: explicit prefix prompting for code integration and vanilla generation for pure reasoning, which jointly seed diverse mid-reasoning code usage in later iterations.
 
\section{Simulated experiments}\label{sec3}

In this section, we evaluate the performance of the proposed method on simulated data. Section \ref{sec3.1} conducts simulated experiments under a low noise level to evaluate the feasibility of the three general types of methods for the issues studied in this paper. The experimental results demonstrate the feasibility of using temporal polynomials to describe the target trajectory. Further, we analyze the reasons for the inapplicability of the other two types of methods. Section \ref{sec3.2} evaluates the accuracy and robustness of the proposed method on the simulated data under limited observation conditions. Section \ref{sec3.3} evaluates the performance of the proposed $K$ selection algorithm. Section \ref{sec3.4} evaluates the performance of the proposed method under missing data situations.

\subsection{Feasibility}\label{sec3.1}

In this section, we evaluate the feasibility of the three general types of methods using simulated data. This paper primarily addresses the 3D trajectory reconstruction of moving targets such as vehicles and ships and aims to provide precise 3D trajectory reconstruction results in a short time. There are three general types of methods based on motion assumptions of the moving points. The previous works of trajectory intersection methods have proven that they can effectively measure the trajectory of vehicles or ships using a monocular camera by representing the target trajectory as temporal polynomials \cite{Yu2009,Li2014,Chen2019}. However, when solving these motions, the reconstruction accuracy and computational efficiency of the trajectory triangulation methods and the DCT trajectory basis vectors methods are inadequate.

\begin{figure}[htbp]
\centering
\includegraphics[width=2.5in]{fig/Fig9.pdf}
\caption{Minimal number of linear equations need to be solved in different orders of polynomial.}
\label{fig9}
\end{figure}

Although the state-of-the-art trajectory triangulation algorithm \cite{Kaminski2004} (referred to as TT) can utilize linear equations to reconstruct diverse trajectories expressible as polynomials, it requires many more equations at the same order of the polynomial. Under the same $K$ value, our algorithm requires at least $\lfloor\frac{3}{2}(K+1)\rfloor$ equations. However, the TT algorithm requires solving at least \(N_{d}=\begin{pmatrix}d+5\\d\end{pmatrix}-\begin{pmatrix}d+3\\d-2\end{pmatrix}-1\) equations. where \(\begin{pmatrix}\cdot\\\cdot\end{pmatrix}\) represents the combinatorial number. Therefore, the minimal number of linear equations of the TT algorithm increases exponentially while that of our algorithm increases linearly, as shown in Figure \ref{fig9}. Some TT methods even require the optimization of nonlinear equations and can only reconstruct the targets moving along a line \cite{Avidan1999,Avidan2000}. These methods acquire a good initial value. Thus, the TT methods are not suitable for accurately recovering the 3D trajectory of the target within a short period of time.

As for the DCT trajectory basis vectors based method \cite{Park2015} (referred to as DCT), theoretically, the DCT trajectory basis vectors can represent any object trajectory without prior information \cite{Akhter2011}. Thus, this method can reconstruct arbitrary trajectories. However, considering the problem addressed in this paper, for vehicle and ship targets, the most common motion patterns are uniform linear motion, and uniform accelerated motion. Referring to the motion patterns of vehicles and ships, the simulated motions of the point targets are set as: 
\[\begin{cases}X=10+5t\\Y=5t\\Z=t\end{cases},
\begin{cases}X=10+t^2\\Y=13+2t^2\\Z=0.5t^2\end{cases},\]
where the simulated moving points are in uniform linear motion and uniform accelerated motion, respectively. Let the observation platform equipped with a monocular camera perform circular motion to ensure a definite solution for the system. The trajectory of the camera's optical center is: 
\[  \begin{cases}X_C=100\sin(\frac{t}{10\pi})\\Y_C=100-100\cos(\frac{t}{10\pi})\\Z_C=100\end{cases}.\]

\begin{figure}[htbp]
\centering
\subfloat[]{\includegraphics[width=3in]{fig/Fig10a.pdf}%
\label{fig10a}}
\hfil
\subfloat[]{\includegraphics[width=3in]{fig/Fig10b.pdf}%
\label{fig10b}}
\caption{We evaluate the DCT method \cite{Park2015} and our algorithm on simulated data under a low noise level. (a) The result on uniform linear motion data. (b) The result on uniform accelerated motion data.}
\label{fig10}
\end{figure}

The trajectory of the camera's optical center and the direction of the sight-ray at each observation time are known. In the simulated data, a low level of system noise is introduced. The minimal solution of the proposed method only needs 3 observations for uniform linear motion and 5 for uniform accelerated motion. The total number of observations in the simulated data is 60, far greater than the minimum required observations. When reconstructing the trajectories of targets moving in uniform linear motion and uniform accelerated motion using the DCT method, there is a significant degeneracy even when the number of observations reaches 60. As shown in Figure \ref{fig10}, the estimated trajectories by the DCT method significantly deviate from the ground truths. However, our algorithm can accurately reconstruct the target trajectory under a low noise level and sufficient observation conditions. This may be because while DCT trajectory basis vectors can easily represent arbitrarily complex motions, simple motions like uniform linear motion require a large number of DCT trajectory basis vectors to represent them. When solving simple motions, DCT method tends to overfit the measurement noise. Thus, the DCT method exhibits significant degeneracy. The DCT method is more applicable to human motion but is unsuitable for reconstructing vehicle and ship trajectories. For the problem researched in this paper, representing the target trajectory as temporal polynomials is the optimal method.

\subsection{Accuracy and robustness}\label{sec3.2}

In this section, we evaluate the accuracy and robustness of the proposed method on the simulated data. The previous works of trajectory intersection methods have proven that they can effectively measure the trajectory of vehicles, ships, and other moving targets using a monocular camera by representing the target trajectory as temporal polynomials \cite{Yu2009,Li2014,Chen2019}. However, under limited observation conditions such as insufficient observations, long distance, high observation error of platform, and low motion complexity of the observation platform, the previous methods are unstable. They can even cause degeneracies because of the severe ill-conditioning of the least squares equation system caused by limited observation conditions. Therefore, this paper introduces ridge estimation to the trajectory intersection method to mitigate the ill-conditioned problem and improve the stability under limited observation conditions.

\begin{figure}[htbp] 
\centering
\subfloat[]{\includegraphics[width=2.5in]{fig/Fig11a.pdf}%
\label{fig11a}}
\hfil
\subfloat[]{\includegraphics[width=2.5in]{fig/Fig11b.pdf}%
\label{fig11b}}
\caption{We evaluate the TI method \cite{Yu2009}, the LSSVM method \cite{Li2014}, and our algorithm on uniform linear motion and uniform accelerated motion data. (a) The result of uniform linear motion data. (b) The result of uniform accelerated motion data.}
\label{fig11}
\end{figure}

To quantitatively evaluate our 3D trajectory reconstruction method, the trajectory of the point targets and the camera’s optical center are set as Section \ref{sec3.1}. The target trajectory is estimated by the trajectory intersection method \cite{Yu2009} (referred to as TI), the LSSVM method \cite{Li2014}, and the proposed method, respectively. However, we introduce a high level of variety noises satisfying the normal distribution with a mean of zero, including the position systematic noise of the camera's optical center with a standard deviation of 1 m, the position random noise of the camera's optical center with a standard deviation of 1 m, the angle systematic noise of the sight-ray with a standard deviation of 0.3°, and the angle random noise of the sight-ray with a standard deviation of 0.3°. The frame rate is set at 10 Hz and 1000 independent experiments are conducted under the observation times from 1 to 6 s. Estimate the target trajectory using the TI method \cite{Yu2009}, the LSSVM method \cite{Li2014} and our algorithm respectively, and calculate the mean root mean square (RMS) error of the target position at each total observation time. The experimental results are shown in Figure \ref{fig11}. 

It can be seen from Figure \ref{fig11a}, when the target is in uniform linear motion, the LSSVM method exhibits higher accuracy when the number of observations is insufficient. However, when sufficient observations are available, the accuracy of the LSSVM method is lower. When the target is moving in uniform accelerated motion, the LSSVM method demonstrates higher accuracy, as shown in Figure \ref{fig11b}. This may be attributed to that the trajectory estimated by the LSSVM method does not strictly adhere to the determined order. For linear motion, it is more likely to cause significant errors. However, as shown in Figure \ref{fig11}, in the both two motion patterns of the targets, the proposed algorithm achieved the highest accuracy, especially under the ill-conditioning of a small number of observations. As shown in Figure \ref{fig11a}, when the number of observations is relatively low, the accuracy of our algorithm is significantly higher than that of the TI method. As the number of observations increases, the accuracy of both methods gradually approaches. However, although it appears to be very close in Figure \ref{fig11a}, our algorithm consistently outperforms the TI method.

\begin{figure}[htbp]  
\centering
\subfloat[]{\includegraphics[width=3in]{fig/Fig12a.pdf}%
\label{fig12a}}
\hfil
\subfloat[]{\includegraphics[width=3in]{fig/Fig12b.pdf}%
\label{fig12b}}
\caption{Illustration of experimental results under low \textit{reconstructability} condition. (a) The result of uniform linear motion data with a total time of 2 s. (b) The result of uniform accelerated motion data with a total time of 3.5 s.}
\label{fig12}
\end{figure}

Theoretically, to obtain a definite solution for the target trajectory, the number of observations $N$ only needs to satisfy the condition $2N \geq 3(K+1)$. However, due to various observation noises in practice, more observations are usually required to ensure the estimation accuracy. By introducing ridge estimation, our algorithm has significantly improved the estimation accuracy under fewer observations. Our algorithm only requires an observation time of 1 s to achieve almost the same estimation accuracy as the TI method with 3 s when the target moves in uniform linear motion, as Figure \ref{fig11a}. As shown in Figure \ref{fig11b}, when the total observation time is less than 5 s, the TI result occurs degeneracy, but our algorithm can still reconstruct the target trajectory accurately. Figure \ref{fig12a} shows the situation at the total observation time of 2 s in Figure \ref{fig11a} and Figure \ref{fig12b} shows the situation at the total time of 3.5 s in Figure \ref{fig11b}. The targets are in uniform linear motion and uniform accelerated motion, respectively. Figure \ref{fig12a} shows that the trajectories reconstructed by the TI and the LSSVM methods exhibit low degradation accuracy. However, our algorithm can accurately reconstruct the target trajectory. The average RMS errors for the TI, LSSVM, and our methods are 13.33 m, 21.47 m, and 2.46 m, respectively. It can be seen in Figure \ref{fig12b} that the trajectory reconstructed by the TI method degrades to be close to the camera trajectory as the proved situation in Section \ref{sec2.4}. Although the trajectory reconstructed by the LSSVM method does not degrade to be close to the camera trajectory as the TI method, its shape still exhibits degeneracy, showing significant differences from the ground truth. However, our algorithm can accurately reconstruct the target trajectory. The average RMS errors for the three methods are 106.89 m, 17.81 m, and 3.13 m, respectively. The simulated experimental results demonstrate that our algorithm can mitigate the ill-conditioned problem and improve the accuracy and robustness.

\subsection{Performance of the $K$ selection method}\label{sec3.3}

\begin{figure}[htbp]  
\centering
\subfloat[]{\includegraphics[width=2.5in]{fig/Fig13a.pdf}%
\label{fig13a}}
\hfil
\subfloat[]{\includegraphics[width=2.5in]{fig/Fig13b.pdf}%
\label{fig13b}}
\caption{3D reconstruction errors and reprojection errors of sight-rays calculated using deferent orders of temporal polynomials, $K$. (a) The result of uniform linear motion. (b) The result of uniform accelerated motion.}
\label{fig13}
\end{figure}

In this section, we evaluate the performance of the proposed $K$ selection algorithm on simulated data. The target points are set to be moving in uniform linear motion and uniform accelerated motion. The corresponding $K$ values are 1 and 2. The camera's optical center is moving in a circular motion. For each motion pattern, various noises as Section \ref{sec3.2} are added. The method proposed in Section \ref{sec2.3} is used to select the value of $K$. As shown in Figure \ref{fig13}, when the reprojection error of sight-rays is minimized, the 3D reconstruction error estimated using the corresponding $K$ value is also minimal. This demonstrates that the K value selected by the proposed method achieves the highest 3D reconstruction accuracy. We perform 1000 simulated experiments for each motion pattern and count the accuracy rate of $K$ selection. The experimental results are shown in Table \ref{tab1}. 

\begin{table}[htbp]
    \centering
    \caption{Selection accuracy and calculation speed of our selection algorithm}
    \label{tab1}
    \begin{tabular}{ccccc}
        \toprule
      Ground truth of $K$  & 1 & 2  \\
      \midrule
       Selection accuracy  & 98.1\% & 99.6\%\\
       Average time/s  & $1.57\times 10^{-3}$ & $1.61\times 10^{-3}$ \\
    \bottomrule 
    \end{tabular}
\end{table}

\begin{figure}[htbp] 
\centering
\includegraphics[width=2.5in]{fig/Fig14.pdf}
\caption{Required time of DCT algorithm \cite{Park2015} to select the number of DCT trajectory basis vectors and our algorithm to select the order of temporal polynomials.}
\label{fig14}
\end{figure}

As shown in Table \ref{tab1}, the proposed $K$ selection algorithm in this paper can accurately determine the correct value of $K$. At a high noise level, the selection accuracy is above 98\%. Due to temporal polynomials' significant physical meaning, in a period of time, the order of temporal polynomials for moving vehicles or ships is usually 1 or 2. Therefore, compared with the selection algorithm in reference \cite{Park2015}, our algorithm uses much less time to select $K$. Take the uniform linear motion as an example. As shown in Table \ref{tab1}, the time required for our algorithm to select $K$ is approximately $1.57\times10^{-3}$ s. However, under the same conditions, the DCT algorithm \cite{Park2015} requires $1.42\times10^{-1}$ s to select the number of DCT trajectory basis vectors, which is nearly 90 times that of our algorithm. More seriously, as the number of observations increases, the consumption time of their algorithm exhibits quadratic growth, as shown in Figure \ref{fig14}. This demonstrates the high efficiency and accuracy of the proposed selection algorithm.

\subsection{Performance of handling missing data}\label{sec3.4}

\begin{figure}[htbp] 
\centering
\includegraphics[width=2.5in]{fig/Fig15.pdf}
\caption{Illustration of the experimental result under different occlusion conditions.}
\label{fig15}
\end{figure}

In this section, we evaluate the performance of the proposed method under missing data situations. In real-world scenarios, missing data usually occurs from factors such as motion blur, self-occlusion, or exceeding the field of view. Our algorithm can also handle these missing data situations. To test for the effects of missing data of our algorithm, the simulated experiments are conducted under conditions of high \textit{reconstructability}. We set the position systematic noise of the camera's optical center with a standard deviation of 0.1 m, the position random noise of the camera's optical center with a standard deviation of 0.1 m, the angle systematic noise of the sight-rays with a standard deviation of 0.1°, and the angle random noise of the line of sight with a standard deviation of 0.05°. Occlude 0, 20, 40, and 60\% of the observation data, respectively, and use our algorithm to reconstruct the target trajectory. The mean RMS error under occlusion conditions is shown in Figure \ref{fig15}. The experimental results demonstrate that our method exhibits strong robustness under the weak observation conditions of missing data. It does not degenerate even under up to 60\% occlusion. 
\section{Real-world experiments}\label{sec4}

In this section, we evaluate our method on the real-world data. In the real-world data, the UAV observation platform observes vehicles traveling on the highway. The flight platform is equipped with a monocular RGB camera, which captures images at a spatial resolution of 1280 × 720 pixels and a temporal resolution of 25 Hz. The camera exhibits a constrained field of view, with both its horizontal and vertical angular extents not exceeding 2°. The observation range of the flight platform extends from 10 to 20 km. The moving target travels along the highway in approximately uniform straight-line motion, with a speed of approximately 60 km/h. The position of the camera's optical center and the ground truth of the target trajectory are provided by satellite positioning. Due to the long observation distance, the inclination angle is large. The Kernelized Correlation Filters (KCF) \cite{Henriques2015} algorithm is used to track the target. Under such observation conditions, the target on the road can be regarded as a point target. We are only concerned with the trajectory and motion parameters of the target, regardless of its attitude. The center of the bounding box is taken as the image point of the target. Our purpose is to utilize images captured by aerial platforms equipped with only a monocular camera to reconstruct the target's motion parameters and trajectory, as shown in Figure \ref{fig16}.

\begin{figure}[htbp]
\centering
\includegraphics[width=5.5in]{fig/Fig16.pdf}
\caption{Equipped with a monocular camera only, the UAV tracks the moving target and reconstructs its trajectory.}
\label{fig16}
\end{figure}

Three sequences of continuous observational images are selected, each spanning approximately 30 seconds. Therefore, each sequence contains 750 observations, sufficient to solve for the target's uniform linear motion. The flight platform's trajectories are approximated as straight linear over the first two sequences' observation time. Consequently, these conditions approach a degenerate case for the TI method with a low \textit{reconstructability}. In contrast, the flight platform's trajectory is a curve in the last sequence. So, the last observational dataset is conducted under conditions of high \textit{reconstructability}. The values of the \textit{reconstructability} for the three observational datasets are 0.29, 0.85, and 8.05, respectively. The TI method \cite{Yu2009} and the proposed algorithm are employed to reconstruct the target trajectory. Figure \ref{fig17} illustrates the reconstruction results for all three observational datasets. The localization errors are presented in Table \ref{tab2}, \ref{tab3}, and \ref{tab4}, respectively.

\begin{figure}[htbp] 
\centering
\subfloat[]{\includegraphics[width=2.1in]{fig/Fig17a.pdf}%
\label{fig17a}}
\hfil
\subfloat[]{\includegraphics[width=2.1in]{fig/Fig17b.pdf}%
\label{fig17b}}
\hfil
\subfloat[]{\includegraphics[width=2.1in]{fig/Fig17c.pdf}%
\label{fig17c}}
\caption{Illustration of real-world experimental results. (a) Real-world experiment under a low \textit{reconstructability} of 0.29. (b) Real-world experiment under a low \textit{reconstructability} of 0.85. (c) Real-world experiment under a high \textit{reconstructability} of 8.05.}
\label{fig17}
\end{figure}

\begin{table}[htbp]
    \centering
    \caption{Localization error of Experiment \textbf{a} under the condition of \textit{reconstructability} $\eta=0.29$}
    \label{tab2}
    \begin{tabular}{cccccc}
        \toprule
      Method  & $\mathrm{\sigma_x(m)}$ & $\mathrm{\sigma_y(m)}$ & $\mathrm{\sigma_z(m)}$ & $\mathrm{\sigma(m)}$ \\
      \midrule
      TI  & 4534.25 & 8754.78 & 13741.21 & 16912.32 \\
      Ours  & 15.37 & 30.46 & 49.81 & \textbf{60.37} \\
    \bottomrule 
    \end{tabular}
\end{table}

\begin{table}[htbp] 
    \centering
    \caption{Localization error of Experiment \textbf{b} under the condition of \textit{reconstructability} $\eta=0.85$}
    \label{tab3}
    \begin{tabular}{cccccc}
        \toprule
      Method  & $\mathrm{\sigma_x(m)}$ & $\mathrm{\sigma_y(m)}$ & $\mathrm{\sigma_z(m)}$ & $\mathrm{\sigma(m)}$ \\
      \midrule
      TI  & 10669.07 & 7772.67 & 1241.33 & 13258.37 \\
      Ours  & 37.89 & 27.19 & 4.48 & \textbf{46.85} \\
    \bottomrule 
    \end{tabular}
\end{table}

\begin{table}[htbp] 
    \centering
    \caption{Localization error of Experiment \textbf{c} under the condition of \textit{reconstructability} $\eta=8.05$}
    \label{tab4}
    \begin{tabular}{cccccc}
       \toprule
      Method  & $\mathrm{\sigma_x(m)}$ & $\mathrm{\sigma_y(m)}$ & $\mathrm{\sigma_z(m)}$ & $\mathrm{\sigma(m)}$ \\
      \midrule
      TI  & 148.17 & 540.78 & 714.64 & 908.36 \\
      Ours  & 10.43 & 33.80 & 47.01 & \textbf{58.83} \\
    \bottomrule 
    \end{tabular}
\end{table}

Consistent with the conclusions drawn in Section \ref{sec2.4}, under conditions of low \textit{reconstructability} in the observational dataset, the TI method exhibits degeneracy, with the reconstructed target trajectory closely approximating the camera trajectory. And the smaller the value of \textit{reconstructibility}, the closer the reconstructed trajectory is to the camera trajectory. However, the introduction of ridge estimation effectively mitigates the problem of ill-conditioning, ensuring that our algorithm can still accurately reconstruct the trajectory of the target under conditions of low \textit{reconstructability}. Moreover, under conditions of high \textit{reconstructability} in the observational dataset, the reconstruction accuracy of the TI method remains relatively low due to factors such as long observation distance, narrow field of view, and high noise levels, as shown in Figure \ref{fig17c} and Table \ref{tab4}. In contrast, our algorithm demonstrates superior robustness, enabling the high-precision reconstruction of the target trajectory under such limited observation conditions.

Experiments employing real-world data corroborate the efficacy of our proposed algorithm under limited observation conditions of long observation distance, low \textit{reconstructability}, limited field of view, and high noise levels. The results demonstrate that our method achieves significantly higher accuracy than the conventional trajectory intersection method. Moreover, our method exhibits remarkable stability, maintaining consistent performance without degeneracy. Thus our method shows superior robustness. This experiment also validates the analysis in Section \ref{sec2.4}. As the trajectory of the flight platform becomes more complex, transitioning from straight lines to curves, the \textit{reconstructability} of the system increases, leading to improved reconstruction accuracy.

In practical applications, we recommended maneuvering the UAV in response to the motion of the target points. By adjusting the UAV's steering, acceleration, and deceleration, the complexity of its moving trajectory can be enhanced, thereby enhancing the \textit{reconstructability} of the system. This approach can significantly improve the reconstruction accuracy of the point's trajectory. 
\section{Conclusion}
We introduce a novel approach, \algo, to reduce human feedback requirements in preference-based reinforcement learning by leveraging vision-language models. While VLMs encode rich world knowledge, their direct application as reward models is hindered by alignment issues and noisy predictions. To address this, we develop a synergistic framework where limited human feedback is used to adapt VLMs, improving their reliability in preference labeling. Further, we incorporate a selective sampling strategy to mitigate noise and prioritize informative human annotations.

Our experiments demonstrate that this method significantly improves feedback efficiency, achieving comparable or superior task performance with up to 50\% fewer human annotations. Moreover, we show that an adapted VLM can generalize across similar tasks, further reducing the need for new human feedback by 75\%. These results highlight the potential of integrating VLMs into preference-based RL, offering a scalable solution to reducing human supervision while maintaining high task success rates. 

\section*{Impact Statement}
This work advances embodied AI by significantly reducing the human feedback required for training agents. This reduction is particularly valuable in robotic applications where obtaining human demonstrations and feedback is challenging or impractical, such as assistive robotic arms for individuals with mobility impairments. By minimizing the feedback requirements, our approach enables users to more efficiently customize and teach new skills to robotic agents based on their specific needs and preferences. The broader impact of this work extends to healthcare, assistive technology, and human-robot interaction. One possible risk is that the bias from human feedback can propagate to the VLM and subsequently to the policy. This can be mitigated by personalization of agents in case of household application or standardization of feedback for industrial applications. 

% Numbered list
% Use the style of numbering in square brackets.
% If nothing is used, default style will be taken.
%\begin{enumerate}[a)]
%\item 
%\item 
%\item 
%\end{enumerate}  

% Unnumbered list
%\begin{itemize}
%\item 
%\item 
%\item 
%\end{itemize}  

% Description list
%\begin{description}
%\item[]
%\item[] 
%\item[] 
%\end{description}  

% Figure
% \begin{figure}[<options>]
% 	\centering
% 		\includegraphics[<options>]{}
% 	  \caption{}\label{fig1}
% \end{figure}


% \begin{table}[<options>]
% \caption{}\label{tbl1}
% \begin{tabular*}{\tblwidth}{@{}LL@{}}
% \toprule
%   &  \\ % Table header row
% \midrule
%  & \\
%  & \\
%  & \\
%  & \\
% \bottomrule
% \end{tabular*}
% \end{table}

% Uncomment and use as the case may be
%\begin{theorem} 
%\end{theorem}

% Uncomment and use as the case may be
%\begin{lemma} 
%\end{lemma}

%% The Appendices part is started with the command \appendix;
%% appendix sections are then done as normal sections
%% \appendix

% To print the credit authorship contribution details
% \printcredits

%% Loading bibliography style file
%\bibliographystyle{model1-num-names}
\bibliographystyle{elsarticle-num}

% Loading bibliography database
\bibliography{cas-refs}

% Biography
% \bio{}
% Here goes the biography details.
% \endbio

% \bio{pic1}
% Here goes the biography details.
% \endbio

\end{document}

