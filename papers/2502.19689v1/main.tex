%% 
%% Copyright 2019-2021 Elsevier Ltd
%% 
%% This file is part of the 'CAS Bundle'.
%% --------------------------------------
%% 
%% It may be distributed under the conditions of the LaTeX Project Public
%% License, either version 1.2 of this license or (at your option) any
%% later version.  The latest version of this license is in
%%    http://www.latex-project.org/lppl.txt
%% and version 1.2 or later is part of all distributions of LaTeX
%% version 1999/12/01 or later.
%% 
%% The list of all files belonging to the 'CAS Bundle' is
%% given in the file `manifest.txt'.
%% 
%% Template article for cas-sc documentclass for 
%% single column output.

\documentclass[a4paper]{cas-sc}
\usepackage{graphicx}   
\usepackage{subcaption} 
% If the frontmatter runs over more than one page
% use the longmktitle option.

%\documentclass[a4paper,fleqn,longmktitle]{cas-sc}

\usepackage[numbers]{natbib}
%\usepackage[authoryear]{natbib}
%\usepackage[authoryear,longnamesfirst]{natbib}

%%%Author macros
\def\tsc#1{\csdef{#1}{\textsc{\lowercase{#1}}\xspace}}
\tsc{WGM}
\tsc{QE}
%%%

% Uncomment and use as if needed
%\newtheorem{theorem}{Theorem}
%\newtheorem{lemma}[theorem]{Lemma}
%\newdefinition{rmk}{Remark}
%\newproof{pf}{Proof}
%\newproof{pot}{Proof of Theorem \ref{thm}}

\begin{document}
\let\WriteBookmarks\relax
\def\floatpagepagefraction{1}
\def\textpagefraction{.001}
\let\printorcid\relax 

% Short title
% \shorttitle{<short title of the paper for running head>}    
\shorttitle{3D Trajectory Reconstruction of Moving Points Based on a Monocular Camera}   

%Short author
%\shortauthors{<short author list for running head>} 
%\shortauthors{}
\shortauthors{Huang et al.}

% Main title of the paper
\title[mode = title]{3D Trajectory Reconstruction of Moving Points Based on a Monocular Camera}  

\author[1,2]{Huayu Huang}

\author[1,2]{Banglei Guan}

\author[1,2]{Yang Shang}

\author[1,2]{Qifeng Yu}

\address[1]{College of Aerospace Science and Engineering, National University of Defense Technology, Changsha, China}
\address[2]{Hunan Provincial Key Laboratory of Image Measurement and Vision Navigation, National University of Defense Technology, Changsha, China}

% Title footnote mark
% eg: \tnotemark[1]
% \tnotemark[<tnote number>] 


% Title footnote 1.
% eg: \tnotetext[1]{Title footnote text}
% \tnotetext[<tnote number>]{<tnote text>} 

% First author
%
% Options: Use if required
% eg: \author[1,3]{Author Name}[type=editor,
%       style=chinese,
%       auid=000,
%       bioid=1,
%       prefix=Sir,
%       orcid=0000-0000-0000-0000,
%       facebook=<facebook id>,
%       twitter=<twitter id>,
%       linkedin=<linkedin id>,
%       gplus=<gplus id>]

% \author[<aff no>]{<author name>}[<options>]

% Corresponding author indication
% \cormark[<corr mark no>]

% Footnote of the first author
% \fnmark[<footnote mark no>]

% Email id of the first author
% \ead{<email address>}

% URL of the first author
% \ead[url]{<URL>}

% Credit authorship
% eg: \credit{Conceptualization of this study, Methodology, Software}
% \credit{<Credit authorship details>}

% Address/affiliation
% \affiliation[<aff no>]{organization={},
%             addressline={}, 
%             city={},
% %          citysep={}, % Uncomment if no comma needed between city and postcode
%             postcode={}, 
%             state={},
%             country={}}

% \author[<aff no>]{<author name>}[<options>]

% Footnote of the second author
% \fnmark[2]

% Email id of the second author
% \ead{}

% URL of the second author
% \ead[url]{}

% Credit authorship
% \credit{}

% Address/affiliation
% \affiliation[<aff no>]{organization={},
%             addressline={}, 
%             city={},
% %          citysep={}, % Uncomment if no comma needed between city and postcode
%             postcode={}, 
%             state={},
%             country={}}

% Corresponding author text
% \cortext[1]{Corresponding author}

% Footnote text
% \fntext[1]{}

% For a title note without a number/mark
%\nonumnote{}

% Here goes the abstract
\begin{abstract}

% Recent works to jointly reconstruct 3D human and object from a single RGB image, are mostly model-based, that fail to capture the fine details of the clothed human body and object surface. In this paper, we introduce ReCHOR, a novel, model-free, first-method to produce realistic clothed human-object reconstructions from a monocular view. This is extremely challenging due to human-object occlusions, diverse interactions and depth ambiguity, as it needs to infer both 3D spatial awareness and high resolution details. Our core idea is based on estimating neural implicit representations for human and object respectively by an attention-based neural implicit model that attends to pixel-aligned features from both the global human-object image for spatial awareness and  the local separate view of human and object images for high quality details. Additionally, the network is conditioned on semantic features from an initial estimated human-object pose prior and a generative diffusion model that inpaints occluded regions, thus enabling the retrieval of details from them.
% We also propose a synthetic dataset with rendered scenes of diverse, inter-occluded 3D human and object scans, to train our network. We evaluate our method on the synthetic and real world BEHAVE dataset. Our experiments show that our method outperforms the SOTA in achieving realistic clothed human-object reconstructions.
Recent approaches to jointly reconstruct 3D humans and objects from a single RGB image represent 3D shapes with template-based or coarse models, which fail to capture details of loose clothing on human bodies. In this paper, we introduce a novel implicit approach for jointly reconstructing realistic 3D clothed humans and objects from a monocular view. For the first time, we model both the human and the object with an implicit representation, allowing to capture more realistic details such as clothing. This task is extremely challenging due to human-object occlusions and the lack of 3D information in 2D images, often leading to poor detail reconstruction and depth ambiguity. To address these problems, we propose a novel attention-based neural implicit model that leverages image pixel alignment from both the input human-object image for a global understanding of the human-object scene and from local separate views of the human and object images to improve realism with, for example, clothing details. Additionally, the network is conditioned on semantic features derived from an estimated human-object pose prior, which provides 3D spatial information about the shared space of humans and objects. To handle human occlusion caused by objects, we use a generative diffusion model that inpaints the occluded regions, recovering otherwise lost details. For training and evaluation, we introduce a synthetic dataset featuring rendered scenes of inter-occluded 3D human scans and diverse objects. Extensive evaluation on both synthetic and real-world datasets demonstrates the superior quality of the proposed human-object reconstructions over competitive methods.
\end{abstract} 

% Use if graphical abstract is present
%\begin{graphicalabstract}
%\includegraphics{}
%\end{graphicalabstract}

% Keywords
% Each keyword is seperated by \sep
\begin{keywords}
dynamic 3D reconstruction \sep 
monocular vision \sep 
trajectory intersection \sep 
ridge estimation \sep 
reconstructibility
\end{keywords}

\maketitle

% Main text
\section{Introduction}
\label{sec:intro}
% Image editing methods in diffusion models depend on user-defined control directions - users can unlock their creativity using these methods by specifying the desired manipulation through prompts~\cite{gandikota2023concept}, reference images~\cite{ruiz2022dreambooth, kumari2022customdiffusion, gal2022image, chen2024trainingfreeregionalpromptingdiffusion}, or attribute vectors~\cite{parmar2023zero,hertz2022prompt}. In this work, we ask a fundamentally different question: \emph{Can we automatically discover the underlying visual structure of a concept within diffusion model's knowledge?} %Rather than requiring user-specified controls, we aim to decompose the model's internal knowledge into meaningful directions.

% This question touches on a fundamental limitation in how we interact with diffusion models. Current control methods ~\cite{zhang2023addingconditionalcontroltexttoimage, gandikota2023concept, ye2023ipadaptertextcompatibleimage,ye2023ipadaptertextcompatibleimage, hertz2024stylealignedimagegeneration, li2023photomaker, shi2024instantbooth, chen2024trainingfreeregionalpromptingdiffusion} require users to specify their desired manipulations in advance, limiting interactive creativity. This contrasts with natural human artistic workflows, where creators dynamically explore creative ideas while jointly refining them toward meaningful artistic outcomes~\cite{hoffmann2016modeling}. This synergy between specification and exploration is not new to generative models. Early GAN architectures naturally developed disentangled latent spaces that enabled continuous\cite{harkonen2020ganspace,radford2015unsupervised, wu2021stylespace, shen2020interfacegan}, compositional control over generated images. Users could explore these spaces to discover interesting variations that would be difficult to describe in words~\cite{wu2021stylespace}, then combine them to achieve their creative goals~\cite{grabe2022towards}. 


% While diffusion models have largely superseded GANs in conditional image synthesis~\cite{dhariwal2021diffusion},  their underlying structure remains less understood. Diffusion models achieve remarkable diversity through high-dimensional latents, unlike GANs' compact latent spaces.  With a single prompt, diffusion models can generate radically different variations through different random initializations of input noise. We ask - Is it possible to discover interpretable structure within this vast space of variations?

Text-to-image diffusion models are capable of generating remarkable visual variations from a single prompt through different random initializations. However, this vast creative potential remains largely opaque to users---while we can generate diverse images, we lack understanding of the underlying structure of these variations. This presents a fundamental challenge: how can we discover and expose the latent visual capabilities encoded within these models?

\let\thefootnote\relax \footnote{$^{*}$Correspondence to \texttt{gandikota.ro@northeastern.edu}}

The challenge touches on a key limitation in how we interact with diffusion models today. Current control methods require users to explicitly specify their desired edits in advance through prompts~\cite{gandikota2023concept}, reference images~\cite{zhang2023addingconditionalcontroltexttoimage, chen2024trainingfreeregionalpromptingdiffusion, ruiz2022dreambooth,kumari2022customdiffusion, Ryu_lora, hu2021lora}, or attribute vectors~\cite{ye2023ipadaptertextcompatibleimage, hertz2024stylealignedimagegeneration, li2023photomaker, shi2024instantbooth,parmar2023zero,hertz2022prompt}. That contrasts sharply with natural human creative workflows, where artists dynamically explore creative ideas and jointly refine them toward meaningful artistic outcomes~\cite{hoffmann2016modeling}. The need for pre-specified controls creates a barrier between users and the full creative potential of these models.

Interestingly, earlier generative models like GANs~\cite{gans,karras2019style,brock2018large} naturally developed more interpretable internal structures. Their compact latent spaces often exhibited emergent disentanglement~\cite{harkonen2020ganspace,radford2015unsupervised, wu2021stylespace, shen2020interfacegan}, enabling continuous and compositional control over generated images. Users could explore these spaces to discover interesting variations that would be difficult to describe in words~\cite{wu2021stylespace}, then combine them to achieve their creative goals~\cite{grabe2022towards}.

Diffusion models have largely superseded GANs in conditional image synthesis~\cite{dhariwal2021diffusion}, achieving greater diversity through much higher-dimensional latents. And yet an understanding of the underlying structure of these larger latent spaces has remained elusive. In this work, we ask a fundamental question: \emph{Can we automatically discover the visual structure within a diffusion model's knowledge of a concept?} Rather than requiring user-specified controls, we aim to decompose the model's internal representations into expressive directions that users can explore and combine.

To address these needs, we present \textbf{SliderSpace}, a framework that brings systematic explorability to diffusion models. Given just a text prompt, SliderSpace discovers a canonical set of meaningful, diverse, and controllable directions within the model's knowledge of that concept. Each direction is implemented as a low-rank adapter~\cite{hu2021lora} that can be scaled and composed with others, allowing users to explore and smoothly combine different aspects of variation, as shown in Figure~\ref{fig:intro}.

We ground SliderSpace discovery in three key requirements for meaningful decomposition of a diffusion model's visual manifold: 
\begin{enumerate}
    \item \textbf{Unsupervised Discovery:} The decomposition process should emerge from the intrinsic structure of the model's learned representation, rather than being guided by predefined attributes. This ensures we capture the true topology of the model's knowledge space rather than projecting our assumptions onto it.
    
    \item \textbf{Semantic Orthogonality:} Each discovered control must represent a distinct semantic direction. This is enforced in a semantic feature space, like CLIP, where every slider has an orthogonal effect in embeddings. This prevents discovering multiple controls that create similar semantic effects, making the system more efficient and easier.
    
    \item \textbf{Distribution Consistency:} Directions must induce consistent transformations across both random seeds and prompt variations. 
\end{enumerate}

These requirements naturally lead to our proposed framework, which we formalize in Section~\ref{sec:method}. As we show in our experiments, SliderSpace is architecture-agnostic, working with both conventional U-Net based models like Stable Diffusion~\cite{rombach2022high, rombach2022sd20, podell2023sdxl, turbo, dmd} and recent transformer-based architectures like Flux~\cite{flux}.

We demonstrate the expressiveness of SliderSpace through three applications: First, we show how SliderSpace can decompose high-level concepts into diverse and expressive components, revealing the natural axes of variation in the model's understanding. Second, we explore artistic style variation, where SliderSpace discovers directions that match or exceed the diversity of manually curated artist lists while being judged more useful by human evaluators. Finally, we show how SliderSpace can help reverse the mode collapse commonly observed in distilled diffusion models, restoring diversity while maintaining generation speed.

Beyond providing practical creative control, SliderSpace opens new avenues for understanding and utilizing the latent capabilities of diffusion models. By mapping these models' visual potential into intuitive, composable directions, we take a step toward making their creative possibilities more accessible and interpretable to users.

% Image editing methods in diffusion models unlock the creativity of users. In this work we ask an alternate question: \emph{Can we organize and expose what of the diffusion model is already capable of?}.
% Existing methods for controlling image generation typically require users to manually specify edit directions for desired changes. This process is time-consuming, requires technical expertise, and limits the spontaneity of the creative process. For instance, if a user wants to adjust the smile of a generated person, they must explicitly request this edit, often through imprecise prompt engineering or model fine-tuning. This approach of predefined controls or manual specifications restricts users from fully exploring the latent capabilities of the model. There may be interesting stylistic variations or attributes that the model can generate, but users have no easy way to discover or utilize these.

% Natural visual disentanglement was an emergent property in the latent space of Generative Adversarial Models (GANs) \cite{harkonen2020ganspace,radford2015unsupervised, wu2021stylespace, shen2020interfacegan}. In particular, it has been observed that StyleGAN~\cite{karras2019style} stylespace neurons offer detailed control over many meaningful aspects of images that would be difficult to describe in words~\cite{wu2021stylespace}. However, diffusion models do not share such a compact latent space~\cite{park2023unsupervised}; and efforts to uncover such a space in the semantic embeddings of the text conditioning have met with limited success \nik{Nick - is there a specific citation you were thinking about?}.

% In this work we introduce \textbf{SliderSpace}, which takes a step towards uncovering an analogous low dimensional representation of diffusion models' visual breadth; in essence treating the diffusion model as many generators sharing parameters, where a particular generator is defined by a specific prompt. For a given prompt we sample many random seeds (and optionally prompt expansions using an LLM), generate the corresponding images, and apply an off the shelf feature extractor (in this work CLIP, but our method can be applied to any differentiable feature extractor). We use PCA to analyze these features, and for each of the leading $k$ principal components we train a LoRA \cite{} which causes the diffusion model to produces images which increase the feature magnitude along that component when passed back through the same feature extractor. This leads to a 'Slider' for each principal component, because each LoRA can be scaled and applied to the original diffusion model, continuously varying those visual features in the generated results (as measured, in our case, by CLIP).

% There are many other works that enhance the controllability of diffusion models. One common approach is enabling users to add spatial constraints to a generation either manually, or via a reference image \cite{zhang2023addingconditionalcontroltexttoimage, chen2024trainingfreeregionalpromptingdiffusion}, a second is leveraging more abstract embeddings (e.g. identity, style) extracted from a reference image \cite{ye2023ipadaptertextcompatibleimage, hertz2024stylealignedimagegeneration, li2023photomaker, shi2024instantbooth}, a third is finetuning a foundation model to better generate a concept important to the user \cite{ruiz2022dreambooth, kumari2022customdiffusion, Ryu_lora, hu2021lora}, and a fourth (most relevant to this work) is finding low-rank adaptors of the model based on a prompt or small training set which can be scaled to provide continous control over one aspect of generated image (e.g. night vs day, basic vs luxury, etc.) \cite{gandikota2023concept}. SliderSpace is complementary to all of these methods and offers something distinct. All of the other methods we are aware require the user (and / or model designer) to know in advance what type of control they want. In contrast SliderSpace assists users in discovering and controlling hidden capabilities present in the diffusion model's distribution of possible generations.

%We propose that truly intuitive creative control in a text-to-image model should meet three key criteria: \emph{discoverability}, \emph{intuitiveness}, and \emph{specificity}. The model should reveal controllable attributes that may not be immediately obvious, offer controls that are easy to understand and manipulate, and ensure each control affects a distinct attribute of the generated image.

% We demonstrate the utility and power of SliderSpace using three applications built on top of SDXL-DMD \cite{dmd}, because its fast generation speed lends itself well to the continuous control offered by SliderSpace.

% First, we study concept decomposition (Section \ref{sec:concept_exp}), where we learn sliders for a specific concept (e.g. 'monster', 'waterfall', 'car'). Through quantitative metrics of diversity and text alignment we demonstrate that the learned sliders dramatically boost the diversity of generations when randomly applied without harming text alignment; we also ask humans to qualitatively judge these results in a user study where they find the SliderSpace results to be more 'Diverse', 'Useful', and 'Creative' than our baselines.

% Second, we attempt to compare the automatic discoveries of SliderSpace to a large scale manual study of artistic styles (Section \ref{sec:art_exp}), open-sourced by ParrotZone \cite{parrotzone}. In this study SDXL was prompted with over 4300 artist names,  and based on visual inspection the cases of successful stylistic mimicry recorded. Quantitatively SliderSpace more closely matches the distribution of artistic variation discovered by ParrotZone than other baselines, and in our user studies was judged to be significantly more 'Diverse' and 'Useful' than the baselines. To our surprise humans even judged SliderSpace results to be slightly more 'Diverse' than the results generated by the manually discovered artist names of \cite{parrotzone}.

% Third, we attempt to use SliderSpace to reverse the mode collapse commonly observed in distilled few-step diffusion models relative to the original teacher model (Section \ref{sec:diverse_exp}). We quantitatively demonstrate that applying SliderSpace to SDXL-DMD leads to more closely matching the distribution of images by the original teacher, SDXL.

%Through extensive experiments on various state-of-the-art text-to-image models, we demonstrate that SliderSpace significantly enhances user control and creative expression in AI-assisted image generation tasks. Our method enables a range of applications, including concept decomposition and control, diversity improvement in generated images, customization dissection and edits, and the exploration of artistic styles inherent in the model.

% SliderSpace goes beyond providing a practical tool for enhanced creative control. By mapping the visual potential of diffusion models it can open new avenues for generative creativity and deepens our understanding of each model's hidden potential. 
% !TEX root = ../main.tex

\section{Scene Graph Construction for Videos}
\label{sec:scene}
\begin{figure*}[t]
    \centering
    \includegraphics[width=\textwidth]{figures/overview.pdf}
    \vspace{-4mm}
    \caption{An overview of our zero-shot video caption generation pipeline. The pipeline consists of (a) frame-level caption generation using image VLMs, (b) textual scene graph parsing for each frame caption, (c) merging of scene graphs into a unified graph, and (d) video-level caption generation through our graph-to-text model. Our proposed framework leverages frame-level scene graphs to produce detailed and coherent video captions.
    }
    \label{fig:framework}
\end{figure*}

Our objective is to effectively extend the capabilities of image-based vision-language models (VLMs) to the video domain without relying on video-text training. 
To this end, we introduce a novel video captioning framework that combines image VLMs with scene graph structures, as shown in Figure~\ref{fig:framework}.
The proposed method consists of four key steps: 1) generating captions for each frame using an image VLM, 2) converting these captions into scene graphs, 3) consolidating the scene graphs from all frames into a unified graph, and 4) generating comprehensive descriptions from this unified graph. 
This algorithm enables the generation of coherent and detailed video captions, bridging the gap between image and video understanding.

\subsection{Generating image-level captions}
\label{sub:generating}
We obtain image-level captions from a set of sparsely sampled frames using the open-source image VLM, LLAVA-NEXT-7B~\cite{liu2024llavanext}.
This model is selected for its strong performance across multiple benchmarks.
Our approach, however, is flexible and can incorporate any image-based VLM, including proprietary, closed-source models, as long as APIs are accessible.
The model is prompted to generate sentences optimized for scene graph construction, which are subsequently parsed into scene graphs.

\subsection{Parsing captions into scene graphs}
A scene graph $G = (\mathcal{O}, \mathcal{E})$ is defined by a set of objects, $\mathcal{O} = \{o_1, o_2, \ldots \}$, and a set of edges, $\mathcal{E}$.
Each object $o_i = (c_i, \mathcal{A}_i)$ consists of an object class $c_i \in \mathcal{C}$ and a set of attributes $\mathcal{A}_i \subseteq A$, where $\mathcal{C}$ is a set of object classes and $\mathcal{A}$ is a set of all possible attributes.
A directed edge, $e_{i,j} \equiv (o_i, o_j) \in \mathcal{E}$, has a label $r \in \mathcal{R}$, specifying the relationship from one object to the other.
All the values of object classes, attributes, and relationship labels, are text strings.

We convert the generated caption from each frame into a scene graph, providing more structured understanding of individual frames. 
By expressing the visual content in each frame using a graph based on detected objects and their relationships, we can apply a graph merging technique to produce a holistic representation of the entire input video.
We parse a caption into a scene graph using a textual scene graph parser, specifically the FACTUAL-MR parser~\cite{li-etal-2023-factual} in our implementation.

\subsection{Scene graph consolidation}
\label{sub:scene}
% !TEX root = ../main.tex

\begin{algorithm}[t]
\caption{Scene graph merging}
\label{alg:hierarchical_graph_merge}
\begin{algorithmic}[1]

  \STATE \textbf{Input:} 
  \STATE \quad $\mathcal{Q} = [ G_1, G_2, \dots, G_n ]$: a priority queue with frame-level scene graphs
  \STATE \quad $\phi(\cdot)$: a graph encoder
  \STATE \quad $\psi_i(\cdot)$: a function returning the $i^\text{th}$ object in a graph
  \STATE \quad $\pi$: a permutation function
  \STATE \quad $\tau$: a threshold

  \STATE \textbf{Output:} $G_{\text{video}}$: a video-level scene graph

  \WHILE{$|\mathcal{Q}| > 1$}
    \STATE $G^s = (\mathcal{O}^s, \mathcal{E}^s) \gets \text{dequeue}(\mathcal{Q})$
    \STATE $G^t = (\mathcal{O}^t, \mathcal{E}^t) \gets \text{dequeue}(\mathcal{Q})$
    \STATE $G^m = (\mathcal{O}^m, \mathcal{E}^m) \gets (\mathcal{O}^s \cup \mathcal{O}^t, \mathcal{E}^s \cup \mathcal{E}^t)$

    \STATE $\pi^* \gets \displaystyle \arg\max_{\pi \in \Pi} \sum_{i} 
      \frac{\psi_i(\phi(G^s))}{\lVert \psi_i(\phi(G^s)) \rVert} \; \cdot \;
      \frac{\psi_i(\phi(G_{\pi}^t))}{\lVert \psi_i(\phi(G_{\pi}^t)) \rVert}$

    \FOR{$(p, q) \in \mathcal{M}$ such that $s_{p, q} > \tau$}
      \STATE $\hat{c} \gets \text{update\_class}(c^s_p, c^t_q)$
      \STATE $\hat{o} \gets (\hat{c}, \mathcal{A}^s_p \cup \mathcal{A}^t_q)$
      \STATE $\mathcal{O}^m \gets \{\hat{o}\} \cup \bigl(\mathcal{O}^m \setminus \{o^s_p, o^t_q\}\bigr)$
      \STATE \textbf{for each} $(o_x, o_y) \in \mathcal{E}^m$:
      \STATE \quad $(o_x, o_y) \mapsto 
        \begin{cases}
           (\hat{o}, o_y), & \text{if } o_x \in \{ o_p^s, o_q^t \}; \\
           (o_x, \hat{o}), & \text{if } o_y \in \{ o_p^s, o_q^t \}; \\
           (o_x, o_y), & \text{otherwise.}
        \end{cases}$
    \ENDFOR

    \STATE $\mathcal{Q} \gets \text{enqueue}(\mathcal{Q}, G^m)$
  \ENDWHILE

  \STATE $G_{\text{video}} \gets \text{dequeue}(\mathcal{Q})$
  \STATE \textbf{return} $G_{\text{video}}$

\end{algorithmic}
\end{algorithm} 
The scene graph consolidation step combines all frame-level scene graphs into a single graph that captures the overall visual content of the video. 
We outline our graph merging procedure, followed by a subgraph extraction technique for more focused video caption generation.

\subsubsection{Video-level graph integration}

Given two scene graphs, $G^s = (\mathcal{O}^s, \mathcal{E}^s)$ and $G^t = (\mathcal{O}^t, \mathcal{E}^t)$, constructed from two different frames, we perform the Hungarian matching between their object sets, $\mathcal{O}^s$ and $\mathcal{O}^t$.
The Hungarian algorithm aims to find the maximum matching between the objects in $\mathcal{O}^s$ and $\mathcal{O}^t$, which is given by
%
\begin{equation}
	\pi^* = \underset{\pi \in \Pi}{\arg\max} \sum_{i} \frac{ \psi_i(\phi(G^s))}{\| \psi_i(\phi(G^s)) \|} \cdot \frac{\psi_i(\phi(G_\pi^t)) }{\| \psi_i(\phi(G_\pi^t)) \|},
\end{equation}
%
where $\phi(\cdot)$ denotes the graph encoder, $\psi_i(\cdot)$ is the function to extract the $i^\text{th}$ object from an embedded graph, and $\pi \in \Pi$ indicates a permutation of objects in a graph.
Note that we introduce dummy objects  to deal with different numbers of objects for matching.

After identifying a set of matching object pairs, $\mathcal{M}$, \eg, $(p, q)$, where $o_p^s \in \mathcal{O}^s$ and $o_q^t \in \mathcal{O}^t$, using their cosine similarity with a predefined threshold, $\tau$, we merge the matched objects into a new one $\hat{o} \in \hat{\mathcal{O}}$, which is given by
%
\begin{equation}
    \hat{o} = (\hat{c} , \mathcal{A}^s_p \cup \mathcal{A}^t_q) \in \hat{\mathcal{O}},
\end{equation}
%
where $\hat{c}$ represents a class of the merged objects and $\hat{\mathcal{O}}$ denotes a set of new objects from all legitimate matching pairs.

Using this, we construct a new merged scene graph, $G^m$, which replaces each pair of merged objects with a new object $\hat{o}$, as follows:
%
\begin{equation}
	G^m = (\mathcal{O}^m, \mathcal{E}^m),
\end{equation}
%
where $\mathcal{O}^{m} =\mathcal{O}^s \cup \mathcal{O}^t \cup \hat{\mathcal{O}} ~ \setminus \bigcup_{(p, q) \in \mathcal{M}} \{o^s_p, o^t_q\}$, and the edge set $\mathcal{E}^m$ is also updated to reflect the changes in the object configuration.
Formally, each matching pair $(p, q) \in \mathcal{M}$ incurs the merge of the two objects and the construction of a new object $\hat{o}$, which results in the update of the edge set as $\mathcal{E}^m \equiv \mathcal{E}^s \cup \mathcal{E}^t$, which is formally given by
%
\begin{equation}
	(o_x, o_y) \in \mathcal{E}^m \rightarrow 
	\begin{cases}
		(\hat{o}, o_y) & \text{if } o_x \in \{o_p^s, o_q^t \}, \\
		(o_x, \hat{o}) & \text{if } o_y \in \{o_p^s, o_q^t \}, \\
		(o_x, o_y) & \text{otherwise.}
	\end{cases}
\end{equation}

We perform graph merging using a priority queue, where pairs of graphs are prioritized for merging based on their embedding similarity. 
In each iteration, the two most similar graphs are dequeued, merged, and the resulting graph is enqueued back into the priority queue.
This process is repeated until only one scene graph remains.
The final scene graph provides a comprehensive representation of the video, preserving frame-level details often overlooked by standard captioning models.
Algorithm~\ref{alg:hierarchical_graph_merge} describes the detailed procedure of our graph merging strategy.
  
\subsubsection{Prioritized subgraph extraction}
To generate concise and focused video captions, we apply subgraph extraction to retain only the most contextually relevant information. 
During the graph merging process, we track each node's merge count as a measure of its significance within the consolidated graph. 
We then identify the top $k$ nodes with the highest merge counts and extract their corresponding subgraphs. 
This approach prioritizes objects that consistently appear across multiple frames, as they often represent key entities in the scene. 
By emphasizing these essential elements and filtering out less relevant details, our method constructs a compact scene graph to generate a more focused video caption.

\section{Video Captioning}
\label{sec:videocaption}
To generate video-level descriptions that accurately reflect visual content, we developed a model that takes scene graphs as input and produce natural language descriptions.
This model is designed to effectively capture key components and relationships within the scene graph in generated text.

\vspace{-2mm}
\paragraph{Architecture}
We employ a modified encoder-decoder transformer architecture.
To prepare the input sequence for the graph encoder, each node, edge, and attribute in the graph, represented as a word or phrase, is tokenized into NLP tokens. 
These tokens are mapped to their embeddings via an embedding lookup.
For nodes consisting of multiple NLP tokens, their embeddings are averaged to form a single vector representation.
Additionally, a [CLS] token is appended as a global node to prevent isolation among disconnected components and ensure coherence. 
The adjacency matrix serves as an attention mask, incorporating graph topology into the attention mechanism. 
The graph encoder's output is then used as key and value inputs for the cross-attention layers of the text decoder, which generates the final outputs.

\vspace{-2mm}
\paragraph{Dataset}
For training, we collected approximately 2.5M text corpora that cover diverse visual scene contexts from various sources, including image caption datasets such as  MS-COCO~\cite{chen2015microsoft}, Flickr30k~\cite{young2014image}, TextCaps~\cite{sidorov2020textcaps}, Visual Genome~\cite{krishna2017visual}, and Visual Genome paragraph captioning~\cite{krause2016paragraphs}.
To further enhance the dataset, we incorporated model-generated captions for Kinetics-400~\cite{kay2017kinetics} dataset, with four uniformly sampled frames per video.
Note that neither the datasets nor the image VLMs used for generating frame captions are related to the target video captioning benchmarks.


\vspace{-2mm}
\paragraph{Training}
The model is trained using a next-token prediction objective, aiming to reconstruct the source text conditioned on the scene graph:
%
\begin{equation}
\mathcal{L}(\theta) = \sum_{i=1}^{N} \log P_{\theta}(t_i \mid t_{1:i-1}, G),
\end{equation}  
%
where $t_i$ represents the $i^\text{th}$ token in the source text, and $N$ denotes the total number of tokens.


\vspace{-2mm}
\paragraph{Video caption generation}
After constructing the video-level scene graph as described in Section~\ref{sec:scene}, we generate a video caption using the trained graph-to-text decoder, which conveys the overall narrative of the video.
 
\section{Simulated experiments}\label{sec3}

In this section, we evaluate the performance of the proposed method on simulated data. Section \ref{sec3.1} conducts simulated experiments under a low noise level to evaluate the feasibility of the three general types of methods for the issues studied in this paper. The experimental results demonstrate the feasibility of using temporal polynomials to describe the target trajectory. Further, we analyze the reasons for the inapplicability of the other two types of methods. Section \ref{sec3.2} evaluates the accuracy and robustness of the proposed method on the simulated data under limited observation conditions. Section \ref{sec3.3} evaluates the performance of the proposed $K$ selection algorithm. Section \ref{sec3.4} evaluates the performance of the proposed method under missing data situations.

\subsection{Feasibility}\label{sec3.1}

In this section, we evaluate the feasibility of the three general types of methods using simulated data. This paper primarily addresses the 3D trajectory reconstruction of moving targets such as vehicles and ships and aims to provide precise 3D trajectory reconstruction results in a short time. There are three general types of methods based on motion assumptions of the moving points. The previous works of trajectory intersection methods have proven that they can effectively measure the trajectory of vehicles or ships using a monocular camera by representing the target trajectory as temporal polynomials \cite{Yu2009,Li2014,Chen2019}. However, when solving these motions, the reconstruction accuracy and computational efficiency of the trajectory triangulation methods and the DCT trajectory basis vectors methods are inadequate.

\begin{figure}[htbp]
\centering
\includegraphics[width=2.5in]{fig/Fig9.pdf}
\caption{Minimal number of linear equations need to be solved in different orders of polynomial.}
\label{fig9}
\end{figure}

Although the state-of-the-art trajectory triangulation algorithm \cite{Kaminski2004} (referred to as TT) can utilize linear equations to reconstruct diverse trajectories expressible as polynomials, it requires many more equations at the same order of the polynomial. Under the same $K$ value, our algorithm requires at least $\lfloor\frac{3}{2}(K+1)\rfloor$ equations. However, the TT algorithm requires solving at least \(N_{d}=\begin{pmatrix}d+5\\d\end{pmatrix}-\begin{pmatrix}d+3\\d-2\end{pmatrix}-1\) equations. where \(\begin{pmatrix}\cdot\\\cdot\end{pmatrix}\) represents the combinatorial number. Therefore, the minimal number of linear equations of the TT algorithm increases exponentially while that of our algorithm increases linearly, as shown in Figure \ref{fig9}. Some TT methods even require the optimization of nonlinear equations and can only reconstruct the targets moving along a line \cite{Avidan1999,Avidan2000}. These methods acquire a good initial value. Thus, the TT methods are not suitable for accurately recovering the 3D trajectory of the target within a short period of time.

As for the DCT trajectory basis vectors based method \cite{Park2015} (referred to as DCT), theoretically, the DCT trajectory basis vectors can represent any object trajectory without prior information \cite{Akhter2011}. Thus, this method can reconstruct arbitrary trajectories. However, considering the problem addressed in this paper, for vehicle and ship targets, the most common motion patterns are uniform linear motion, and uniform accelerated motion. Referring to the motion patterns of vehicles and ships, the simulated motions of the point targets are set as: 
\[\begin{cases}X=10+5t\\Y=5t\\Z=t\end{cases},
\begin{cases}X=10+t^2\\Y=13+2t^2\\Z=0.5t^2\end{cases},\]
where the simulated moving points are in uniform linear motion and uniform accelerated motion, respectively. Let the observation platform equipped with a monocular camera perform circular motion to ensure a definite solution for the system. The trajectory of the camera's optical center is: 
\[  \begin{cases}X_C=100\sin(\frac{t}{10\pi})\\Y_C=100-100\cos(\frac{t}{10\pi})\\Z_C=100\end{cases}.\]

\begin{figure}[htbp]
\centering
\subfloat[]{\includegraphics[width=3in]{fig/Fig10a.pdf}%
\label{fig10a}}
\hfil
\subfloat[]{\includegraphics[width=3in]{fig/Fig10b.pdf}%
\label{fig10b}}
\caption{We evaluate the DCT method \cite{Park2015} and our algorithm on simulated data under a low noise level. (a) The result on uniform linear motion data. (b) The result on uniform accelerated motion data.}
\label{fig10}
\end{figure}

The trajectory of the camera's optical center and the direction of the sight-ray at each observation time are known. In the simulated data, a low level of system noise is introduced. The minimal solution of the proposed method only needs 3 observations for uniform linear motion and 5 for uniform accelerated motion. The total number of observations in the simulated data is 60, far greater than the minimum required observations. When reconstructing the trajectories of targets moving in uniform linear motion and uniform accelerated motion using the DCT method, there is a significant degeneracy even when the number of observations reaches 60. As shown in Figure \ref{fig10}, the estimated trajectories by the DCT method significantly deviate from the ground truths. However, our algorithm can accurately reconstruct the target trajectory under a low noise level and sufficient observation conditions. This may be because while DCT trajectory basis vectors can easily represent arbitrarily complex motions, simple motions like uniform linear motion require a large number of DCT trajectory basis vectors to represent them. When solving simple motions, DCT method tends to overfit the measurement noise. Thus, the DCT method exhibits significant degeneracy. The DCT method is more applicable to human motion but is unsuitable for reconstructing vehicle and ship trajectories. For the problem researched in this paper, representing the target trajectory as temporal polynomials is the optimal method.

\subsection{Accuracy and robustness}\label{sec3.2}

In this section, we evaluate the accuracy and robustness of the proposed method on the simulated data. The previous works of trajectory intersection methods have proven that they can effectively measure the trajectory of vehicles, ships, and other moving targets using a monocular camera by representing the target trajectory as temporal polynomials \cite{Yu2009,Li2014,Chen2019}. However, under limited observation conditions such as insufficient observations, long distance, high observation error of platform, and low motion complexity of the observation platform, the previous methods are unstable. They can even cause degeneracies because of the severe ill-conditioning of the least squares equation system caused by limited observation conditions. Therefore, this paper introduces ridge estimation to the trajectory intersection method to mitigate the ill-conditioned problem and improve the stability under limited observation conditions.

\begin{figure}[htbp] 
\centering
\subfloat[]{\includegraphics[width=2.5in]{fig/Fig11a.pdf}%
\label{fig11a}}
\hfil
\subfloat[]{\includegraphics[width=2.5in]{fig/Fig11b.pdf}%
\label{fig11b}}
\caption{We evaluate the TI method \cite{Yu2009}, the LSSVM method \cite{Li2014}, and our algorithm on uniform linear motion and uniform accelerated motion data. (a) The result of uniform linear motion data. (b) The result of uniform accelerated motion data.}
\label{fig11}
\end{figure}

To quantitatively evaluate our 3D trajectory reconstruction method, the trajectory of the point targets and the camera’s optical center are set as Section \ref{sec3.1}. The target trajectory is estimated by the trajectory intersection method \cite{Yu2009} (referred to as TI), the LSSVM method \cite{Li2014}, and the proposed method, respectively. However, we introduce a high level of variety noises satisfying the normal distribution with a mean of zero, including the position systematic noise of the camera's optical center with a standard deviation of 1 m, the position random noise of the camera's optical center with a standard deviation of 1 m, the angle systematic noise of the sight-ray with a standard deviation of 0.3°, and the angle random noise of the sight-ray with a standard deviation of 0.3°. The frame rate is set at 10 Hz and 1000 independent experiments are conducted under the observation times from 1 to 6 s. Estimate the target trajectory using the TI method \cite{Yu2009}, the LSSVM method \cite{Li2014} and our algorithm respectively, and calculate the mean root mean square (RMS) error of the target position at each total observation time. The experimental results are shown in Figure \ref{fig11}. 

It can be seen from Figure \ref{fig11a}, when the target is in uniform linear motion, the LSSVM method exhibits higher accuracy when the number of observations is insufficient. However, when sufficient observations are available, the accuracy of the LSSVM method is lower. When the target is moving in uniform accelerated motion, the LSSVM method demonstrates higher accuracy, as shown in Figure \ref{fig11b}. This may be attributed to that the trajectory estimated by the LSSVM method does not strictly adhere to the determined order. For linear motion, it is more likely to cause significant errors. However, as shown in Figure \ref{fig11}, in the both two motion patterns of the targets, the proposed algorithm achieved the highest accuracy, especially under the ill-conditioning of a small number of observations. As shown in Figure \ref{fig11a}, when the number of observations is relatively low, the accuracy of our algorithm is significantly higher than that of the TI method. As the number of observations increases, the accuracy of both methods gradually approaches. However, although it appears to be very close in Figure \ref{fig11a}, our algorithm consistently outperforms the TI method.

\begin{figure}[htbp]  
\centering
\subfloat[]{\includegraphics[width=3in]{fig/Fig12a.pdf}%
\label{fig12a}}
\hfil
\subfloat[]{\includegraphics[width=3in]{fig/Fig12b.pdf}%
\label{fig12b}}
\caption{Illustration of experimental results under low \textit{reconstructability} condition. (a) The result of uniform linear motion data with a total time of 2 s. (b) The result of uniform accelerated motion data with a total time of 3.5 s.}
\label{fig12}
\end{figure}

Theoretically, to obtain a definite solution for the target trajectory, the number of observations $N$ only needs to satisfy the condition $2N \geq 3(K+1)$. However, due to various observation noises in practice, more observations are usually required to ensure the estimation accuracy. By introducing ridge estimation, our algorithm has significantly improved the estimation accuracy under fewer observations. Our algorithm only requires an observation time of 1 s to achieve almost the same estimation accuracy as the TI method with 3 s when the target moves in uniform linear motion, as Figure \ref{fig11a}. As shown in Figure \ref{fig11b}, when the total observation time is less than 5 s, the TI result occurs degeneracy, but our algorithm can still reconstruct the target trajectory accurately. Figure \ref{fig12a} shows the situation at the total observation time of 2 s in Figure \ref{fig11a} and Figure \ref{fig12b} shows the situation at the total time of 3.5 s in Figure \ref{fig11b}. The targets are in uniform linear motion and uniform accelerated motion, respectively. Figure \ref{fig12a} shows that the trajectories reconstructed by the TI and the LSSVM methods exhibit low degradation accuracy. However, our algorithm can accurately reconstruct the target trajectory. The average RMS errors for the TI, LSSVM, and our methods are 13.33 m, 21.47 m, and 2.46 m, respectively. It can be seen in Figure \ref{fig12b} that the trajectory reconstructed by the TI method degrades to be close to the camera trajectory as the proved situation in Section \ref{sec2.4}. Although the trajectory reconstructed by the LSSVM method does not degrade to be close to the camera trajectory as the TI method, its shape still exhibits degeneracy, showing significant differences from the ground truth. However, our algorithm can accurately reconstruct the target trajectory. The average RMS errors for the three methods are 106.89 m, 17.81 m, and 3.13 m, respectively. The simulated experimental results demonstrate that our algorithm can mitigate the ill-conditioned problem and improve the accuracy and robustness.

\subsection{Performance of the $K$ selection method}\label{sec3.3}

\begin{figure}[htbp]  
\centering
\subfloat[]{\includegraphics[width=2.5in]{fig/Fig13a.pdf}%
\label{fig13a}}
\hfil
\subfloat[]{\includegraphics[width=2.5in]{fig/Fig13b.pdf}%
\label{fig13b}}
\caption{3D reconstruction errors and reprojection errors of sight-rays calculated using deferent orders of temporal polynomials, $K$. (a) The result of uniform linear motion. (b) The result of uniform accelerated motion.}
\label{fig13}
\end{figure}

In this section, we evaluate the performance of the proposed $K$ selection algorithm on simulated data. The target points are set to be moving in uniform linear motion and uniform accelerated motion. The corresponding $K$ values are 1 and 2. The camera's optical center is moving in a circular motion. For each motion pattern, various noises as Section \ref{sec3.2} are added. The method proposed in Section \ref{sec2.3} is used to select the value of $K$. As shown in Figure \ref{fig13}, when the reprojection error of sight-rays is minimized, the 3D reconstruction error estimated using the corresponding $K$ value is also minimal. This demonstrates that the K value selected by the proposed method achieves the highest 3D reconstruction accuracy. We perform 1000 simulated experiments for each motion pattern and count the accuracy rate of $K$ selection. The experimental results are shown in Table \ref{tab1}. 

\begin{table}[htbp]
    \centering
    \caption{Selection accuracy and calculation speed of our selection algorithm}
    \label{tab1}
    \begin{tabular}{ccccc}
        \toprule
      Ground truth of $K$  & 1 & 2  \\
      \midrule
       Selection accuracy  & 98.1\% & 99.6\%\\
       Average time/s  & $1.57\times 10^{-3}$ & $1.61\times 10^{-3}$ \\
    \bottomrule 
    \end{tabular}
\end{table}

\begin{figure}[htbp] 
\centering
\includegraphics[width=2.5in]{fig/Fig14.pdf}
\caption{Required time of DCT algorithm \cite{Park2015} to select the number of DCT trajectory basis vectors and our algorithm to select the order of temporal polynomials.}
\label{fig14}
\end{figure}

As shown in Table \ref{tab1}, the proposed $K$ selection algorithm in this paper can accurately determine the correct value of $K$. At a high noise level, the selection accuracy is above 98\%. Due to temporal polynomials' significant physical meaning, in a period of time, the order of temporal polynomials for moving vehicles or ships is usually 1 or 2. Therefore, compared with the selection algorithm in reference \cite{Park2015}, our algorithm uses much less time to select $K$. Take the uniform linear motion as an example. As shown in Table \ref{tab1}, the time required for our algorithm to select $K$ is approximately $1.57\times10^{-3}$ s. However, under the same conditions, the DCT algorithm \cite{Park2015} requires $1.42\times10^{-1}$ s to select the number of DCT trajectory basis vectors, which is nearly 90 times that of our algorithm. More seriously, as the number of observations increases, the consumption time of their algorithm exhibits quadratic growth, as shown in Figure \ref{fig14}. This demonstrates the high efficiency and accuracy of the proposed selection algorithm.

\subsection{Performance of handling missing data}\label{sec3.4}

\begin{figure}[htbp] 
\centering
\includegraphics[width=2.5in]{fig/Fig15.pdf}
\caption{Illustration of the experimental result under different occlusion conditions.}
\label{fig15}
\end{figure}

In this section, we evaluate the performance of the proposed method under missing data situations. In real-world scenarios, missing data usually occurs from factors such as motion blur, self-occlusion, or exceeding the field of view. Our algorithm can also handle these missing data situations. To test for the effects of missing data of our algorithm, the simulated experiments are conducted under conditions of high \textit{reconstructability}. We set the position systematic noise of the camera's optical center with a standard deviation of 0.1 m, the position random noise of the camera's optical center with a standard deviation of 0.1 m, the angle systematic noise of the sight-rays with a standard deviation of 0.1°, and the angle random noise of the line of sight with a standard deviation of 0.05°. Occlude 0, 20, 40, and 60\% of the observation data, respectively, and use our algorithm to reconstruct the target trajectory. The mean RMS error under occlusion conditions is shown in Figure \ref{fig15}. The experimental results demonstrate that our method exhibits strong robustness under the weak observation conditions of missing data. It does not degenerate even under up to 60\% occlusion. 
\section{Real-world experiments}\label{sec4}

In this section, we evaluate our method on the real-world data. In the real-world data, the UAV observation platform observes vehicles traveling on the highway. The flight platform is equipped with a monocular RGB camera, which captures images at a spatial resolution of 1280 × 720 pixels and a temporal resolution of 25 Hz. The camera exhibits a constrained field of view, with both its horizontal and vertical angular extents not exceeding 2°. The observation range of the flight platform extends from 10 to 20 km. The moving target travels along the highway in approximately uniform straight-line motion, with a speed of approximately 60 km/h. The position of the camera's optical center and the ground truth of the target trajectory are provided by satellite positioning. Due to the long observation distance, the inclination angle is large. The Kernelized Correlation Filters (KCF) \cite{Henriques2015} algorithm is used to track the target. Under such observation conditions, the target on the road can be regarded as a point target. We are only concerned with the trajectory and motion parameters of the target, regardless of its attitude. The center of the bounding box is taken as the image point of the target. Our purpose is to utilize images captured by aerial platforms equipped with only a monocular camera to reconstruct the target's motion parameters and trajectory, as shown in Figure \ref{fig16}.

\begin{figure}[htbp]
\centering
\includegraphics[width=5.5in]{fig/Fig16.pdf}
\caption{Equipped with a monocular camera only, the UAV tracks the moving target and reconstructs its trajectory.}
\label{fig16}
\end{figure}

Three sequences of continuous observational images are selected, each spanning approximately 30 seconds. Therefore, each sequence contains 750 observations, sufficient to solve for the target's uniform linear motion. The flight platform's trajectories are approximated as straight linear over the first two sequences' observation time. Consequently, these conditions approach a degenerate case for the TI method with a low \textit{reconstructability}. In contrast, the flight platform's trajectory is a curve in the last sequence. So, the last observational dataset is conducted under conditions of high \textit{reconstructability}. The values of the \textit{reconstructability} for the three observational datasets are 0.29, 0.85, and 8.05, respectively. The TI method \cite{Yu2009} and the proposed algorithm are employed to reconstruct the target trajectory. Figure \ref{fig17} illustrates the reconstruction results for all three observational datasets. The localization errors are presented in Table \ref{tab2}, \ref{tab3}, and \ref{tab4}, respectively.

\begin{figure}[htbp] 
\centering
\subfloat[]{\includegraphics[width=2.1in]{fig/Fig17a.pdf}%
\label{fig17a}}
\hfil
\subfloat[]{\includegraphics[width=2.1in]{fig/Fig17b.pdf}%
\label{fig17b}}
\hfil
\subfloat[]{\includegraphics[width=2.1in]{fig/Fig17c.pdf}%
\label{fig17c}}
\caption{Illustration of real-world experimental results. (a) Real-world experiment under a low \textit{reconstructability} of 0.29. (b) Real-world experiment under a low \textit{reconstructability} of 0.85. (c) Real-world experiment under a high \textit{reconstructability} of 8.05.}
\label{fig17}
\end{figure}

\begin{table}[htbp]
    \centering
    \caption{Localization error of Experiment \textbf{a} under the condition of \textit{reconstructability} $\eta=0.29$}
    \label{tab2}
    \begin{tabular}{cccccc}
        \toprule
      Method  & $\mathrm{\sigma_x(m)}$ & $\mathrm{\sigma_y(m)}$ & $\mathrm{\sigma_z(m)}$ & $\mathrm{\sigma(m)}$ \\
      \midrule
      TI  & 4534.25 & 8754.78 & 13741.21 & 16912.32 \\
      Ours  & 15.37 & 30.46 & 49.81 & \textbf{60.37} \\
    \bottomrule 
    \end{tabular}
\end{table}

\begin{table}[htbp] 
    \centering
    \caption{Localization error of Experiment \textbf{b} under the condition of \textit{reconstructability} $\eta=0.85$}
    \label{tab3}
    \begin{tabular}{cccccc}
        \toprule
      Method  & $\mathrm{\sigma_x(m)}$ & $\mathrm{\sigma_y(m)}$ & $\mathrm{\sigma_z(m)}$ & $\mathrm{\sigma(m)}$ \\
      \midrule
      TI  & 10669.07 & 7772.67 & 1241.33 & 13258.37 \\
      Ours  & 37.89 & 27.19 & 4.48 & \textbf{46.85} \\
    \bottomrule 
    \end{tabular}
\end{table}

\begin{table}[htbp] 
    \centering
    \caption{Localization error of Experiment \textbf{c} under the condition of \textit{reconstructability} $\eta=8.05$}
    \label{tab4}
    \begin{tabular}{cccccc}
       \toprule
      Method  & $\mathrm{\sigma_x(m)}$ & $\mathrm{\sigma_y(m)}$ & $\mathrm{\sigma_z(m)}$ & $\mathrm{\sigma(m)}$ \\
      \midrule
      TI  & 148.17 & 540.78 & 714.64 & 908.36 \\
      Ours  & 10.43 & 33.80 & 47.01 & \textbf{58.83} \\
    \bottomrule 
    \end{tabular}
\end{table}

Consistent with the conclusions drawn in Section \ref{sec2.4}, under conditions of low \textit{reconstructability} in the observational dataset, the TI method exhibits degeneracy, with the reconstructed target trajectory closely approximating the camera trajectory. And the smaller the value of \textit{reconstructibility}, the closer the reconstructed trajectory is to the camera trajectory. However, the introduction of ridge estimation effectively mitigates the problem of ill-conditioning, ensuring that our algorithm can still accurately reconstruct the trajectory of the target under conditions of low \textit{reconstructability}. Moreover, under conditions of high \textit{reconstructability} in the observational dataset, the reconstruction accuracy of the TI method remains relatively low due to factors such as long observation distance, narrow field of view, and high noise levels, as shown in Figure \ref{fig17c} and Table \ref{tab4}. In contrast, our algorithm demonstrates superior robustness, enabling the high-precision reconstruction of the target trajectory under such limited observation conditions.

Experiments employing real-world data corroborate the efficacy of our proposed algorithm under limited observation conditions of long observation distance, low \textit{reconstructability}, limited field of view, and high noise levels. The results demonstrate that our method achieves significantly higher accuracy than the conventional trajectory intersection method. Moreover, our method exhibits remarkable stability, maintaining consistent performance without degeneracy. Thus our method shows superior robustness. This experiment also validates the analysis in Section \ref{sec2.4}. As the trajectory of the flight platform becomes more complex, transitioning from straight lines to curves, the \textit{reconstructability} of the system increases, leading to improved reconstruction accuracy.

In practical applications, we recommended maneuvering the UAV in response to the motion of the target points. By adjusting the UAV's steering, acceleration, and deceleration, the complexity of its moving trajectory can be enhanced, thereby enhancing the \textit{reconstructability} of the system. This approach can significantly improve the reconstruction accuracy of the point's trajectory. 
\section{Conclusion}

%In this paper, w
We propose a new PEFT method called DiffoRA, which enables efficient and adaptive LLM fine-tuning based on LoRA. 
Instead of adjusting every interior rank, 
%of the decomposition matrices 
%of all modules, 
we argue that adopting LoRA module-wisely is sufficient. 
To achieve this, we construct a DAM to select the modules that are most suitable and essential to fine-tune. We theoretically analyze how the DAM impacts the convergence rate and generalization capability.
%of the pre-trained model. 
Furthermore, we adopt continuous relaxation and discretization to establish DAM.
%for each task. 
To alleviate the issue of discretization discrepancy, we utilize the weight-sharing strategy for optimization. 
%We fully implement our method and t
The experimental results demonstrate that our DiffoRA works consistently better than the baselines across all benchmarks. 

% Numbered list
% Use the style of numbering in square brackets.
% If nothing is used, default style will be taken.
%\begin{enumerate}[a)]
%\item 
%\item 
%\item 
%\end{enumerate}  

% Unnumbered list
%\begin{itemize}
%\item 
%\item 
%\item 
%\end{itemize}  

% Description list
%\begin{description}
%\item[]
%\item[] 
%\item[] 
%\end{description}  

% Figure
% \begin{figure}[<options>]
% 	\centering
% 		\includegraphics[<options>]{}
% 	  \caption{}\label{fig1}
% \end{figure}


% \begin{table}[<options>]
% \caption{}\label{tbl1}
% \begin{tabular*}{\tblwidth}{@{}LL@{}}
% \toprule
%   &  \\ % Table header row
% \midrule
%  & \\
%  & \\
%  & \\
%  & \\
% \bottomrule
% \end{tabular*}
% \end{table}

% Uncomment and use as the case may be
%\begin{theorem} 
%\end{theorem}

% Uncomment and use as the case may be
%\begin{lemma} 
%\end{lemma}

%% The Appendices part is started with the command \appendix;
%% appendix sections are then done as normal sections
%% \appendix

% To print the credit authorship contribution details
% \printcredits

%% Loading bibliography style file
%\bibliographystyle{model1-num-names}
\bibliographystyle{elsarticle-num}

% Loading bibliography database
\bibliography{cas-refs}

% Biography
% \bio{}
% Here goes the biography details.
% \endbio

% \bio{pic1}
% Here goes the biography details.
% \endbio

\end{document}

