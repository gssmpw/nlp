\section{Methods}\label{sec2}

This section proposes the 3D trajectory reconstruction method of moving points, as well as the automatic selection method of the order parameter. Then a detailed geometric analysis is conducted on the reconstructability of the algorithm.

\subsection{Linear reconstruction of a 3D point trajectory}\label{sec2.1}

\begin{figure}[htbp]
\centering
\includegraphics[width=3in]{fig/Fig2.pdf}
\caption{Illustration of 3D trajectory reconstruction of a moving target based on a flight platform equipped with a monocular camera.}
\label{fig2}
\end{figure}

Figure \ref{fig2} illustrates the scenario where a flight platform, equipped with a monocular camera, is observing a moving target continually. The target can be regarded as a point target. Assuming that at the time $t_i$, the position of the camera is \(\mathbf{C}_i=\begin{bmatrix}X_{Ci}&Y_{Ci}&Z_{Ci}\end{bmatrix}^\mathrm{T}\), which is accurately known. The position of the target is \(\mathbf{P}_i=\begin{bmatrix}X_i&Y_i&Z_i\end{bmatrix}^\mathrm{T}\). Let $\mathbf{P}_{i}$ be imaged as \(\mathbf{p}_i=\begin{bmatrix}x_i&y_i\end{bmatrix}^\mathrm{T}\). The internal parameter $\mathbf{K}$ is calibrated in advance\cite{Hartley2004,Liang2024}. The rotation matrix $\mathbf{R}_{i}$ of the camera can be calculated employing the technique of structure from motion within a given scene \cite{Guan2023,Yu2024_2}. Therefore, we can calculate the direction of the observation sight-ray \(\mathbf{l}_i=\begin{bmatrix}l_{Xi}&l_{Yi}&l_{Zi}\end{bmatrix}^\mathrm{T}\),
\begin{equation}
    \mathbf{l}_i=\frac{\mathbf{R}_i^\mathrm{T}\mathbf{K}^{-1}\mathbf{p}_i}{\|\mathbf{R}_i^\mathrm{T}\mathbf{K}^{-1}\mathbf{p}_i\|}.
    \label{eq1}
\end{equation} 

\begin{figure}[htbp]
\centering
\includegraphics[width=2.2in]{fig/Fig3.pdf}
\caption{Illustration of residual error.}
\label{fig3}
\end{figure}

As shown in Figure \ref{fig3}, the residual error $\mathbf{e}_{i}$ between the true position and the ideal position of the target can be represented as: 
\begin{equation}  
    \mathbf{e}_{i} = (\mathbf{I} - \mathbf{l}_{i}\mathbf{l}_{i}^\mathrm{T} )(\mathbf{C}_{i} - \mathbf{P}_{i}),
    \label{eq2}
\end{equation}  
where $\mathbf{I}$ represents the identity matrix. Under the criterion of minimizing the sum of squared residuals, we can establish the following equation: 
\begin{equation}
    (\mathbf{I}-\mathbf{l}_{i}\mathbf{l}_{i}^\mathrm{T})\mathbf{P}_{i}=(\mathbf{I}-\mathbf{l}_{i}\mathbf{l}_{i}^\mathrm{T})\mathbf{C}_{i},
    \label{eq3}
\end{equation} 
where the rank of this equation is 2. Therefore, by combining Eq. (\ref{eq3}) from $N$ observations, the number of unknowns is 3$N$, while the number of independent equations is $2N$. For a specific set of measurements, there exists an infinite array of solutions.

To calculate the position of the target, additional constraints must be imposed on its trajectory. We achieve this by expressing the target trajectory as polynomials of time $t_i$:
\begin{equation}
    \left(X_{i}=\sum\limits_{k=0}^{K}a_{k}t_i^{k}\quad Y_{i}=\sum\limits_{k=0}^{K}b_{k}t_i^{k}\quad Z_{i}=\sum\limits_{k=0}^{K}c_{k}t_i^{k}\right),
    \label{eq4}
\end{equation} 
where $K$ is the order of the polynomials. $a_{k}(k=0, 1,\cdots, K)$, $b_{k}(k=0, 1,\cdots, K)$, $c_{k}(k=0, 1,\cdots, K)$ are the motion parameters of the target that need to be solved. Based on Eq. (\ref{eq3}) and Eq. (\ref{eq4}), the subsequent set of equations can be formulated:
\begin{equation}
    (\mathbf{I}-\mathbf{l}_{i}\mathbf{l}_{i}^\mathrm{T})\begin{bmatrix}\sum\limits_{k=0}^{K}a_{k}t_i^{k}\\\sum\limits_{k=0}^{K}b_{k}t_i^{k}\\\sum\limits_{k=0}^{K}c_{k}t_i^{k}\end{bmatrix}=(\mathbf{I}-\mathbf{l}_{i}\mathbf{l}_{i}^\mathrm{T})\mathbf{C}_{i}.
    \label{eq5}
\end{equation} 


Solving Eq. (\ref{eq5}) from $N$ observations for the motion parameters of the target,  the number of unknowns is $3(K+1)$, while the number of independent equations is $2N$. It is a linear least squares system if $2N \geq 3(K+1)$. The estimated motion parameters $a_{k}$, $b_{k}$, and $c_{k}$ can be obtained by solving the linear least equations. The reconstructed trajectory of the target is a trajectory that passes through all sight-rays and is represented by linear polynomials of time $t_i$. Then the moving trajectory of the target can be reconstructed by Eq. (\ref{eq4}).

\subsection{Trajectory reconstruction under ridge estimation}\label{sec2.2}

As proved in Section \ref{sec2.1}, when the target's motion is represented as temporal polynomials and the number of observations $N$ satisfies condition $2N \geq 3(K+1)$, there is a least squares solution for the target trajectory. As the number of observations increases, the reconstructed trajectory approaches the ground truth of the target trajectory. However, under limited observation conditions such as insufficient observations, long distance, and high observation error of platform, the 3D reconstruction error of trajectory intersection can be substantial or even cause degeneracies to occur. Long observation distance is a common limited observation condition in practical applications. Measurement errors caused by camera pose errors tend to amplify as the distance increases. Moreover, as shown in Figure \ref{fig4}, long-distance observation can also result in a large inclination angle.

\begin{figure}[htbp]
\centering
\includegraphics[width=5in]{fig/Fig4.pdf}
\caption{Imaging maps of different inclination angles.}
\label{fig4}
\end{figure}

Generally, the downward-looking is the optimal observation. However, in complex three-dimensional environments, observations of distant target areas often need to be conducted at a certain altitude. This will lead to the inclination angle. As the observation distance increases, the observation inclination angle becomes larger. As shown in Figure \ref{fig4}, under a large inclination angle, the geometric shape of the image undergoes significant distortion, and the angular error of the sight-ray increases notably. Therefore, the limited observations can cause severe ill-conditioning to the least squares system, i.e., small observation noise can lead to a significant measurement error. The least squares estimation results are unstable.

To improve the stability of target trajectory reconstruction under limited observation conditions, we introduce ridge estimation. By adding a penalty function to the least squares estimation equations, the proposed method mitigates the ill-conditioning, thereby improving the accuracy and stability of the system.

From Eq. (\ref{eq4}), the position of the target can be expressed as 
\begin{equation}
    \mathbf{P}_{i}=\tilde{\boldsymbol{\beta}}\tilde{\mathbf{\Theta}}_i,
    \label{eq6}
\end{equation} 
where 
\begin{equation}
    \tilde{\boldsymbol{\beta}}=\begin{bmatrix}a_0&a_1&\cdots&a_K\\b_0&b_1&\cdots&b_K\\c_0&c_1&\cdots&c_K\end{bmatrix}=\begin{bmatrix}\beta_a\\\beta_b\\\beta_c\end{bmatrix},
    \label{eq7}
\end{equation}

\begin{equation}
   \tilde{\mathbf{\Theta}}_i=\begin{bmatrix}t_i^0&t_i^1&\cdots&t_i^K\end{bmatrix}^\mathrm{T}.
   \label{eq8}
\end{equation}

Further, $\mathbf{P}_{i}$ can be rewritten as 
\begin{equation}
    \mathbf{P}_{i}=\mathbf{\Theta}_i\boldsymbol{\beta},
    \label{eq9}
\end{equation}
where 
\begin{equation}
    \mathbf{\Theta}_i=\mathbf{I}_{3\times3}\otimes\tilde{{\mathbf{\Theta}}}_i^\mathrm{T},
    \label{eq10}
\end{equation}

\begin{equation}
\boldsymbol{\beta}=\begin{bmatrix}\beta_a&\beta_b&\beta_c\end{bmatrix}^\mathrm{T},
\label{eq11}
\end{equation}
where $\otimes$ represents Kronecker Product. Let \(\mathbf{V}_i=(\mathbf{I}-\mathbf{l}_i\mathbf{l}_i^\mathrm{T})\), \(\mathbf{B}_i=(\mathbf{I}-\mathbf{l}_i\mathbf{l}_i^\mathrm{T})\mathbf{C_i}\) and \(\mathbf{A}_i=\mathbf{V}_i\mathbf{\Theta}_i\). Eq. (\ref{eq5}) can be written as a set of linear equations related to the target's motion parameters $\boldsymbol{\beta}$ : 
\begin{equation}
    \mathbf{A}_{i}\boldsymbol{\beta}=\mathbf{B}_{i},
    \label{eq12}
\end{equation} 
where only $\boldsymbol{\beta}$ is unknown. Combining the linear equations in Eq. (\ref{eq12}) of $N$ observations simultaneously, we can obtain the linear equations group:  
\begin{equation}
    \mathbf{A}\boldsymbol{\beta}=\mathbf{B},
    \label{eq13}
\end{equation} 
where \(\mathbf{A}\in\mathbb{R}^{3N\times3(K+1)}\), \(\mathbf{B}\in\mathbb{R}^{3N\times1}\). When $2N \geq 3(K+1)$, the least squares solution of the equation system Eq. (\ref{eq13}) is: 
\begin{equation}
 \boldsymbol{\beta}=(\mathbf{A}^\mathrm{T}\mathbf{A})^{-1}\mathbf{A}^\mathrm{T}\mathbf{B}.
 \label{eq14}
\end{equation}

However, under limited observation conditions, the 3D reconstruction error of the least squares estimation can be substantial or even cause degeneracies to occur. In such scenarios, the correlation of the observational data becomes significant, resulting in ill-conditioning of the least squares system. This ill-conditioning severely undermines the accuracy and robustness of the estimation results. To address this issue and enhance the accuracy and stability of reconstruction results under limited observation conditions, we introduce ridge estimation. 

Ridge estimation is an improved least squares estimation method. It improves the stability of the estimated result by sacrificing some precision to reduce the mean squared error while giving up the unbiasedness of least squares estimation. Ridge estimation is widely applied in data analysis across fields such as economics, engineering, and biomedicine. It performs very well in handling data with multicollinearity. Given the ill-conditioning problems, this paper introduces ridge estimation and adds a penalty function to the least squares estimation objective function. By appropriately reducing the accuracy, and sacrificing the unbiasedness of least squares, more robust and more realistic regression coefficients are obtained to solve the problem of serious multicollinearity of design matrix column vectors.

After adding the regularization term to the diagonal of the normal matrix of Eq. (\ref{eq14}), the ridge estimation can be expressed as:
\begin{equation}
    \boldsymbol{\beta}=(\mathbf{A}^\mathrm{T}\mathbf{A}-{r}\mathbf{I})^{-1}\mathbf{A}^\mathrm{T}\mathbf{B},
    \label{eq15}
\end{equation}
where $r$ is the ridge parameter, and \(\mathbf{I}\in\mathbb{R}^{3(K+1)\times3(K+1)}\) is the identity matrix. The ridge estimation is a linear transformation of the least squares estimation. Under limited observation conditions, there exists a $r>0$ such that the ridge estimation is better than the least squares estimation.

The selection of the ridge parameter $r$ is crucial for ridge estimation. Various methods exist to select the ridge parameters \cite{Golub1979,Lee1985}. In this paper, to obtain results quickly, we adopt the Hoerl-Kennard-Baldwin ridge parameter selection method \cite{Hoerl1975}. It is an efficient approach that does not require fitting or iteration. In practice, this method demonstrates good performance. We choose this method after considering both computational efficiency and accuracy. The calculation of the ridge parameter is as follows:
\begin{equation}
    r=\frac{t\delta_0^2}{\hat{\boldsymbol{\beta}}^\mathrm{T}\mathbf{A}^\mathrm{T}\mathbf{A}\hat{\boldsymbol{\beta}}},
    \label{eq16}
\end{equation}
where $t$ is the rank of matrix $\mathbf{A}$. When the number of observations $N$ satisfies $2N \geq 3(K+1)$, $t=3(K+1)$. $\hat{\boldsymbol{\beta}}$ is the result of the target's motion parameter calculated by the least squares estimate, and $\delta_0^2$ is calculated by:
\begin{equation}
    \delta_0^2=\frac{\mathbf{B}^\mathrm{T}\left[\mathbf{I}-\mathbf{A}(\mathbf{A}^\mathrm{T}\mathbf{A})^{-1}\mathbf{A}^\mathrm{T}\right]\mathbf{B}}{n-t}.
    \label{eq17}
\end{equation}

\subsection{Automatic selection method for the order parameter}\label{sec2.3}

Our approach requires selecting the order of the temporal polynomials, $K$. This parameter was manually adjusted in previous trajectory intersection methods based on experience. The order of the temporal polynomials controls the complexity of the reconstructed trajectory and the number of parameters to be estimated. Selecting the order of the temporal polynomials correctly is crucial to the accuracy of the 3D reconstruction of the target trajectory. If the value of the selected $K$ is excessively high, the algorithm tends to overfit the measurement noise. On the other hand, if the value is too low, the trajectory reconstructed by the algorithm fails to capture the intricacies of the target's motion. In this section, we present an algorithm to automatically determine the optimal value of $K$ rather than manually setting a value.

To automatically select the order of the temporal polynomials, we construct an objective function based on geometric error. Specifically, for each observation time $t_j$, we compute the direction vector from the optical center of the camera to the reconstructed target, denoted as $\hat{\mathbf{l}}_j$. Subsequently, for each value $K_i$, we calculate the sum of the squared sight-ray errors at all $t_j$. Then, we construct the following objective function:
\begin{equation}
    K_i^*=\underset{K_i}{\text{argmin}}\sum_{j=1}^N\lVert \hat{\mathbf{l}}_j-{\mathbf{l}}_j \rVert_2,
    \label{eq18}
\end{equation}
where $K_i=0,1,2,3$. Because within a certain period of time, the ground target's motion usually follows certain physical laws, such as static state, uniform linear motion, or uniform accelerated motion. Taking $K_i$ as 0, 1, 2, or 3 can accurately reconstruct the target trajectory and achieve high efficiency simultaneously. It is worth mentioning that in the method of reference \cite{Park2015}, the automatic selection of the number of DCT trajectory basis vectors $K^{DCT}$ requires calculating the reprojection error for \(K_i^{CDT}=1,2,\ldots,\lfloor2F/3\rfloor \), where $\lfloor\cdot\rfloor$  is the floor operator (the largest integer not greater than $\cdot$ ). Compared to our method, the reference method \cite{Park2015} requires a significant amount of time to automatically select the value of $K$. It is an advantage of temporal polynomials that they carry significant physical meaning. When $K_i$ is high, the trajectory tends to fit the measurement noise excessively, leading to an increased reprojection error for the sight-rays. In contrast, when $K_i$ is low, the reprojection error for the sight-rays remains high because of the restricted expressiveness of the temporal polynomials. Therefore, when minimizing the objective function in Eq. (\ref{eq18}), we only need four iterations to determine the optimal order of temporal polynomials.

\subsection{3D trajectory reconstructability analysis}\label{sec2.4}

\begin{figure}[htbp] 
\centering
\subfloat[]{\includegraphics[width=2.1in]{fig/Fig5a.pdf}%
\label{fig5a}}
\hfil
\subfloat[]{\includegraphics[width=2.1in]{fig/Fig5b.pdf}%
\label{fig5b}}
\hfil
\subfloat[]{\includegraphics[width=2.1in]{fig/Fig5c.pdf}%
\label{fig5c}}
\caption{Degeneracy situations. (a) The order of the temporal polynomial representing the camera motion is lower than that of the target's motion. (b) All sight-rays intersect at the same point. (c) All sight-rays are parallel.}
\label{fig5}
\end{figure}

This section performs a geometric analysis of the camera motion, target motion, and temporal polynomials in our algorithm. As previous work proved \cite{Yu2009,Zhou2015}, there are two situations where a definite solution cannot be obtained, as shown in Figure \ref{fig5}: 

(i) The order of the temporal polynomials representing the camera motion is equal to or lower than that of the target motion, as shown in Figure \ref{fig5a}. 

(ii) All sight-rays intersect at the same point (all sight-rays being parallel is a special case of this situation, where they intersect at the infinite point), as shown in Figure \ref{fig5b} and Figure \ref{fig5c}.

However, in practical applications, it is noted that the reconstructed trajectories of the target closely align with the ground truth when the camera motion relative to the target motion is sufficiently significant in practice. Conversely, if the camera motion is simple compared to the target motion, the solution is likely to deviate from the ground truth. Park et al.\cite{Park2015} proposed an index called \textit{reconstructability} for DCT trajectory basis vectors to measure the reconstruction accuracy of solvable systems. The \textit{reconstructability} is extended from DCT trajectory basis vectors to temporal polynomials by analyzing the geometric relationship among camera motion, target motion, and temporal polynomials. The definition of \textit{reconstructability} enables us to quantify the reconstruction accuracy. Why the estimated trajectory degenerates to be close to the camera trajectory can be explained using this definition. The \textit{reconstructability} based on temporal polynomials is defined as follows.

\begin{figure}[htbp]
\centering
\includegraphics[width=3in]{fig/Fig6.pdf}
\caption{Geometric illustration of the least squares solution, when the order of the temporal polynomials representing the camera motion is higher than that of the target's motion.}
\label{fig6}
\end{figure}

As shown in Figure \ref{fig6}, let $\mathbf{C}$ and $\mathbf{P}$ be the trajectory of the camera and the ground truth of the target trajectory represented by temporal polynomials, respectively. The plane $p=\operatorname{col}(\mathbf{\Theta})$ is the subspace spanned by different orders of time $t$. Thus $\mathrm{col}(\mathbf{\Theta}^{\perp})$ represents the null space of the matrix $\mathbf{\Theta}$. It can be seen from Figure \ref{fig6} that the estimated position of the target lies in plane $p$ and the line $l$ that connects $\mathbf{C}$ and $\mathbf{P}$, simultaneously. Note that the line and the plane are a conceptual 3D vector space representation for the 3$N$-dimensional space. Let the trajectory of the camera $\mathbf{C}$ and the ground truth of the target $\mathbf{P}$ be projected onto the plane $p$ as $\mathbf{\Theta}{\boldsymbol{\beta}}_\mathbf{C}$ and $\mathbf{\Theta}{\boldsymbol{\beta}}_\mathbf{P}$, respectively. Then $\mathbf{C}$ and $\mathbf{P}$ can be decomposed into the column space of $\mathbf{\Theta}$ and that of the null space, $\mathbf{\Theta}^{\perp}$ as follows: 
\begin{equation}
\mathbf{C}=\mathbf{\Theta}\boldsymbol{\beta}_{\mathbf{C}}+\mathbf{\Theta}^{\perp}\boldsymbol{\beta}_{\mathbf{C}}^{\perp}
\label{eq19}
\end{equation}
\begin{equation}\mathbf{P}=\mathbf{\Theta}\boldsymbol{\beta}_{\mathbf{P}}+\mathbf{\Theta}^{\perp}\boldsymbol{\beta}_{\mathbf{P}}^{\perp}
\label{eq20}
\end{equation}
where $\boldsymbol{\beta}^{\perp}$ is the coefficient vector for the null space. \(\mathbf{\Theta}^{\perp}\boldsymbol{\beta}_{\mathbf{C}}^{\perp}\) and \(\mathbf{\Theta}^{\perp}\boldsymbol{\beta}_{\mathbf{P}}^{\perp}\) are the component of the null space  $\mathbf{\Theta}^{\perp}$, i.e. the part that cannot be represented by the temporal polynomials. Thus, the \textit{reconstructability} $\eta$ of the target trajectory for temporal polynomials can be defined as:
\begin{equation}
\eta\left(\boldsymbol{\Theta}\right)=\frac{\left\|\boldsymbol{\Theta}^\perp\boldsymbol{\beta}_\mathbf{C}^\perp\right\|}{\left\|\boldsymbol{\Theta}^\perp\boldsymbol{\beta}_\mathbf{P}^\perp\right\|}\simeq\frac{\text{How poorly }\boldsymbol{\Theta} \text{ describes }\mathbf{C}}{\text{How poorly }\boldsymbol{\Theta}\text{ describes }\mathbf{P}}.
\label{eq21}
\end{equation}

From Eq. (\ref{eq21}) and Figure \ref{fig6}, it can be seen that the larger the value of $\eta$, the higher the reconstruction accuracy of the target trajectory. And when \(\eta\rightarrow\infty\), the estimated trajectory is equal to the ground truth. This shows the importance of the correct selection of the temporal polynomials' order $K$.

\begin{figure}[htbp] 
\centering
\subfloat[]{\includegraphics[width=3in]{fig/Fig7a.pdf}%
\label{fig7a}}
\hfil
\subfloat[]{\includegraphics[width=3in]{fig/Fig7b.pdf}%
\label{fig7b}}
\caption{Geometric illustration of degeneracy situations. (a) The order of the temporal polynomials representing the camera motion is lower than that of the target's motion. (b) The order of the temporal polynomials representing the camera motion is consistent with the selected order.}
\label{fig7}
\end{figure}

The reconstruction accuracy can be described quantitatively based on the \textit{reconstructability}. A more detailed geometric analysis of the measurement system can also be conducted. For example, from Eq. (\ref{eq21}) and Figure \ref{fig6}, the first situation of unsolvable systems that were proposed in previous work can be more intuitively proved. Moreover, why the estimated trajectory degenerates to be close to the camera trajectory in practice can be explained. As shown in Figure \ref{fig7a}, when the order of the temporal polynomials representing the camera motion is lower than that of the target's motion, we can have \(\left\|\boldsymbol{\Theta}^\perp\boldsymbol{\beta}_\mathbf{C}^\perp\right\|<\left\|\boldsymbol{\Theta}^\perp\boldsymbol{\beta}_\mathbf{P}^\perp\right\| \). Thus, the intersection of line $l$ and plane $p$ is closer to $\mathbf{C}$ but far away from $\mathbf{P}$ and this leads to incorrect estimation result $\hat{\mathbf{P}}$. As shown in Figure \ref{fig7b}, when the selected order parameter $K$ can well describe the camera trajectory $\mathbf{C}$, the estimated result $\hat{\mathbf{P}}$ is identical to the camera trajectory. This can explain why the estimated trajectory degenerates to be close to the camera trajectory in practice when the camera motion is relatively simple. 

\begin{figure}[htbp]
\centering
\includegraphics[width=3in]{fig/Fig8.pdf}
\caption{Geometric illustration of a kind of limited observation condition.}
\label{fig8}
\end{figure}

Another kind of limited observation condition can also be analyzed using the definition of \textit{reconstructability}. As shown in Figure \ref{fig8}, the complexity of the camera trajectory is similar to that of the target trajectory, i.e.,  \(\left\|\boldsymbol{\Theta}^\perp\boldsymbol{\beta}_\mathbf{C}^\perp\right\|\approx\left\|\boldsymbol{\Theta}^\perp\boldsymbol{\beta}_\mathbf{P}^\perp\right\| \). If there is no noise, the camera trajectory is more complex, and the chosen order of temporal polynomial $K$ is appropriate, the target trajectory can theoretically be solved. However, in practical applications, the estimation results often degrade due to the influence of system noise. As shown in Figure \ref{fig8}, the measurement error can be significant under such conditions. Under such conditions, a small observation error can lead to substantial measurement errors, or even cause degradation. Therefore, such limited observation conditions can also cause ill-conditioning to the least squares solution system. Ridge estimation can be introduced to mitigate the ill-conditioning.

Through the analysis of \textit{reconstructability}, the geometric relationship among camera trajectory, target trajectory, and temporal polynomials have been more intuitively determined. We have also analyzed another kind of limited observation condition.