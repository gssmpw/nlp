\section{Simulated experiments}\label{sec3}

In this section, we evaluate the performance of the proposed method on simulated data. Section \ref{sec3.1} conducts simulated experiments under a low noise level to evaluate the feasibility of the three general types of methods for the issues studied in this paper. The experimental results demonstrate the feasibility of using temporal polynomials to describe the target trajectory. Further, we analyze the reasons for the inapplicability of the other two types of methods. Section \ref{sec3.2} evaluates the accuracy and robustness of the proposed method on the simulated data under limited observation conditions. Section \ref{sec3.3} evaluates the performance of the proposed $K$ selection algorithm. Section \ref{sec3.4} evaluates the performance of the proposed method under missing data situations.

\subsection{Feasibility}\label{sec3.1}

In this section, we evaluate the feasibility of the three general types of methods using simulated data. This paper primarily addresses the 3D trajectory reconstruction of moving targets such as vehicles and ships and aims to provide precise 3D trajectory reconstruction results in a short time. There are three general types of methods based on motion assumptions of the moving points. The previous works of trajectory intersection methods have proven that they can effectively measure the trajectory of vehicles or ships using a monocular camera by representing the target trajectory as temporal polynomials \cite{Yu2009,Li2014,Chen2019}. However, when solving these motions, the reconstruction accuracy and computational efficiency of the trajectory triangulation methods and the DCT trajectory basis vectors methods are inadequate.

\begin{figure}[htbp]
\centering
\includegraphics[width=2.5in]{fig/Fig9.pdf}
\caption{Minimal number of linear equations need to be solved in different orders of polynomial.}
\label{fig9}
\end{figure}

Although the state-of-the-art trajectory triangulation algorithm \cite{Kaminski2004} (referred to as TT) can utilize linear equations to reconstruct diverse trajectories expressible as polynomials, it requires many more equations at the same order of the polynomial. Under the same $K$ value, our algorithm requires at least $\lfloor\frac{3}{2}(K+1)\rfloor$ equations. However, the TT algorithm requires solving at least \(N_{d}=\begin{pmatrix}d+5\\d\end{pmatrix}-\begin{pmatrix}d+3\\d-2\end{pmatrix}-1\) equations. where \(\begin{pmatrix}\cdot\\\cdot\end{pmatrix}\) represents the combinatorial number. Therefore, the minimal number of linear equations of the TT algorithm increases exponentially while that of our algorithm increases linearly, as shown in Figure \ref{fig9}. Some TT methods even require the optimization of nonlinear equations and can only reconstruct the targets moving along a line \cite{Avidan1999,Avidan2000}. These methods acquire a good initial value. Thus, the TT methods are not suitable for accurately recovering the 3D trajectory of the target within a short period of time.

As for the DCT trajectory basis vectors based method \cite{Park2015} (referred to as DCT), theoretically, the DCT trajectory basis vectors can represent any object trajectory without prior information \cite{Akhter2011}. Thus, this method can reconstruct arbitrary trajectories. However, considering the problem addressed in this paper, for vehicle and ship targets, the most common motion patterns are uniform linear motion, and uniform accelerated motion. Referring to the motion patterns of vehicles and ships, the simulated motions of the point targets are set as: 
\[\begin{cases}X=10+5t\\Y=5t\\Z=t\end{cases},
\begin{cases}X=10+t^2\\Y=13+2t^2\\Z=0.5t^2\end{cases},\]
where the simulated moving points are in uniform linear motion and uniform accelerated motion, respectively. Let the observation platform equipped with a monocular camera perform circular motion to ensure a definite solution for the system. The trajectory of the camera's optical center is: 
\[  \begin{cases}X_C=100\sin(\frac{t}{10\pi})\\Y_C=100-100\cos(\frac{t}{10\pi})\\Z_C=100\end{cases}.\]

\begin{figure}[htbp]
\centering
\subfloat[]{\includegraphics[width=3in]{fig/Fig10a.pdf}%
\label{fig10a}}
\hfil
\subfloat[]{\includegraphics[width=3in]{fig/Fig10b.pdf}%
\label{fig10b}}
\caption{We evaluate the DCT method \cite{Park2015} and our algorithm on simulated data under a low noise level. (a) The result on uniform linear motion data. (b) The result on uniform accelerated motion data.}
\label{fig10}
\end{figure}

The trajectory of the camera's optical center and the direction of the sight-ray at each observation time are known. In the simulated data, a low level of system noise is introduced. The minimal solution of the proposed method only needs 3 observations for uniform linear motion and 5 for uniform accelerated motion. The total number of observations in the simulated data is 60, far greater than the minimum required observations. When reconstructing the trajectories of targets moving in uniform linear motion and uniform accelerated motion using the DCT method, there is a significant degeneracy even when the number of observations reaches 60. As shown in Figure \ref{fig10}, the estimated trajectories by the DCT method significantly deviate from the ground truths. However, our algorithm can accurately reconstruct the target trajectory under a low noise level and sufficient observation conditions. This may be because while DCT trajectory basis vectors can easily represent arbitrarily complex motions, simple motions like uniform linear motion require a large number of DCT trajectory basis vectors to represent them. When solving simple motions, DCT method tends to overfit the measurement noise. Thus, the DCT method exhibits significant degeneracy. The DCT method is more applicable to human motion but is unsuitable for reconstructing vehicle and ship trajectories. For the problem researched in this paper, representing the target trajectory as temporal polynomials is the optimal method.

\subsection{Accuracy and robustness}\label{sec3.2}

In this section, we evaluate the accuracy and robustness of the proposed method on the simulated data. The previous works of trajectory intersection methods have proven that they can effectively measure the trajectory of vehicles, ships, and other moving targets using a monocular camera by representing the target trajectory as temporal polynomials \cite{Yu2009,Li2014,Chen2019}. However, under limited observation conditions such as insufficient observations, long distance, high observation error of platform, and low motion complexity of the observation platform, the previous methods are unstable. They can even cause degeneracies because of the severe ill-conditioning of the least squares equation system caused by limited observation conditions. Therefore, this paper introduces ridge estimation to the trajectory intersection method to mitigate the ill-conditioned problem and improve the stability under limited observation conditions.

\begin{figure}[htbp] 
\centering
\subfloat[]{\includegraphics[width=2.5in]{fig/Fig11a.pdf}%
\label{fig11a}}
\hfil
\subfloat[]{\includegraphics[width=2.5in]{fig/Fig11b.pdf}%
\label{fig11b}}
\caption{We evaluate the TI method \cite{Yu2009}, the LSSVM method \cite{Li2014}, and our algorithm on uniform linear motion and uniform accelerated motion data. (a) The result of uniform linear motion data. (b) The result of uniform accelerated motion data.}
\label{fig11}
\end{figure}

To quantitatively evaluate our 3D trajectory reconstruction method, the trajectory of the point targets and the camera’s optical center are set as Section \ref{sec3.1}. The target trajectory is estimated by the trajectory intersection method \cite{Yu2009} (referred to as TI), the LSSVM method \cite{Li2014}, and the proposed method, respectively. However, we introduce a high level of variety noises satisfying the normal distribution with a mean of zero, including the position systematic noise of the camera's optical center with a standard deviation of 1 m, the position random noise of the camera's optical center with a standard deviation of 1 m, the angle systematic noise of the sight-ray with a standard deviation of 0.3°, and the angle random noise of the sight-ray with a standard deviation of 0.3°. The frame rate is set at 10 Hz and 1000 independent experiments are conducted under the observation times from 1 to 6 s. Estimate the target trajectory using the TI method \cite{Yu2009}, the LSSVM method \cite{Li2014} and our algorithm respectively, and calculate the mean root mean square (RMS) error of the target position at each total observation time. The experimental results are shown in Figure \ref{fig11}. 

It can be seen from Figure \ref{fig11a}, when the target is in uniform linear motion, the LSSVM method exhibits higher accuracy when the number of observations is insufficient. However, when sufficient observations are available, the accuracy of the LSSVM method is lower. When the target is moving in uniform accelerated motion, the LSSVM method demonstrates higher accuracy, as shown in Figure \ref{fig11b}. This may be attributed to that the trajectory estimated by the LSSVM method does not strictly adhere to the determined order. For linear motion, it is more likely to cause significant errors. However, as shown in Figure \ref{fig11}, in the both two motion patterns of the targets, the proposed algorithm achieved the highest accuracy, especially under the ill-conditioning of a small number of observations. As shown in Figure \ref{fig11a}, when the number of observations is relatively low, the accuracy of our algorithm is significantly higher than that of the TI method. As the number of observations increases, the accuracy of both methods gradually approaches. However, although it appears to be very close in Figure \ref{fig11a}, our algorithm consistently outperforms the TI method.

\begin{figure}[htbp]  
\centering
\subfloat[]{\includegraphics[width=3in]{fig/Fig12a.pdf}%
\label{fig12a}}
\hfil
\subfloat[]{\includegraphics[width=3in]{fig/Fig12b.pdf}%
\label{fig12b}}
\caption{Illustration of experimental results under low \textit{reconstructability} condition. (a) The result of uniform linear motion data with a total time of 2 s. (b) The result of uniform accelerated motion data with a total time of 3.5 s.}
\label{fig12}
\end{figure}

Theoretically, to obtain a definite solution for the target trajectory, the number of observations $N$ only needs to satisfy the condition $2N \geq 3(K+1)$. However, due to various observation noises in practice, more observations are usually required to ensure the estimation accuracy. By introducing ridge estimation, our algorithm has significantly improved the estimation accuracy under fewer observations. Our algorithm only requires an observation time of 1 s to achieve almost the same estimation accuracy as the TI method with 3 s when the target moves in uniform linear motion, as Figure \ref{fig11a}. As shown in Figure \ref{fig11b}, when the total observation time is less than 5 s, the TI result occurs degeneracy, but our algorithm can still reconstruct the target trajectory accurately. Figure \ref{fig12a} shows the situation at the total observation time of 2 s in Figure \ref{fig11a} and Figure \ref{fig12b} shows the situation at the total time of 3.5 s in Figure \ref{fig11b}. The targets are in uniform linear motion and uniform accelerated motion, respectively. Figure \ref{fig12a} shows that the trajectories reconstructed by the TI and the LSSVM methods exhibit low degradation accuracy. However, our algorithm can accurately reconstruct the target trajectory. The average RMS errors for the TI, LSSVM, and our methods are 13.33 m, 21.47 m, and 2.46 m, respectively. It can be seen in Figure \ref{fig12b} that the trajectory reconstructed by the TI method degrades to be close to the camera trajectory as the proved situation in Section \ref{sec2.4}. Although the trajectory reconstructed by the LSSVM method does not degrade to be close to the camera trajectory as the TI method, its shape still exhibits degeneracy, showing significant differences from the ground truth. However, our algorithm can accurately reconstruct the target trajectory. The average RMS errors for the three methods are 106.89 m, 17.81 m, and 3.13 m, respectively. The simulated experimental results demonstrate that our algorithm can mitigate the ill-conditioned problem and improve the accuracy and robustness.

\subsection{Performance of the $K$ selection method}\label{sec3.3}

\begin{figure}[htbp]  
\centering
\subfloat[]{\includegraphics[width=2.5in]{fig/Fig13a.pdf}%
\label{fig13a}}
\hfil
\subfloat[]{\includegraphics[width=2.5in]{fig/Fig13b.pdf}%
\label{fig13b}}
\caption{3D reconstruction errors and reprojection errors of sight-rays calculated using deferent orders of temporal polynomials, $K$. (a) The result of uniform linear motion. (b) The result of uniform accelerated motion.}
\label{fig13}
\end{figure}

In this section, we evaluate the performance of the proposed $K$ selection algorithm on simulated data. The target points are set to be moving in uniform linear motion and uniform accelerated motion. The corresponding $K$ values are 1 and 2. The camera's optical center is moving in a circular motion. For each motion pattern, various noises as Section \ref{sec3.2} are added. The method proposed in Section \ref{sec2.3} is used to select the value of $K$. As shown in Figure \ref{fig13}, when the reprojection error of sight-rays is minimized, the 3D reconstruction error estimated using the corresponding $K$ value is also minimal. This demonstrates that the K value selected by the proposed method achieves the highest 3D reconstruction accuracy. We perform 1000 simulated experiments for each motion pattern and count the accuracy rate of $K$ selection. The experimental results are shown in Table \ref{tab1}. 

\begin{table}[htbp]
    \centering
    \caption{Selection accuracy and calculation speed of our selection algorithm}
    \label{tab1}
    \begin{tabular}{ccccc}
        \toprule
      Ground truth of $K$  & 1 & 2  \\
      \midrule
       Selection accuracy  & 98.1\% & 99.6\%\\
       Average time/s  & $1.57\times 10^{-3}$ & $1.61\times 10^{-3}$ \\
    \bottomrule 
    \end{tabular}
\end{table}

\begin{figure}[htbp] 
\centering
\includegraphics[width=2.5in]{fig/Fig14.pdf}
\caption{Required time of DCT algorithm \cite{Park2015} to select the number of DCT trajectory basis vectors and our algorithm to select the order of temporal polynomials.}
\label{fig14}
\end{figure}

As shown in Table \ref{tab1}, the proposed $K$ selection algorithm in this paper can accurately determine the correct value of $K$. At a high noise level, the selection accuracy is above 98\%. Due to temporal polynomials' significant physical meaning, in a period of time, the order of temporal polynomials for moving vehicles or ships is usually 1 or 2. Therefore, compared with the selection algorithm in reference \cite{Park2015}, our algorithm uses much less time to select $K$. Take the uniform linear motion as an example. As shown in Table \ref{tab1}, the time required for our algorithm to select $K$ is approximately $1.57\times10^{-3}$ s. However, under the same conditions, the DCT algorithm \cite{Park2015} requires $1.42\times10^{-1}$ s to select the number of DCT trajectory basis vectors, which is nearly 90 times that of our algorithm. More seriously, as the number of observations increases, the consumption time of their algorithm exhibits quadratic growth, as shown in Figure \ref{fig14}. This demonstrates the high efficiency and accuracy of the proposed selection algorithm.

\subsection{Performance of handling missing data}\label{sec3.4}

\begin{figure}[htbp] 
\centering
\includegraphics[width=2.5in]{fig/Fig15.pdf}
\caption{Illustration of the experimental result under different occlusion conditions.}
\label{fig15}
\end{figure}

In this section, we evaluate the performance of the proposed method under missing data situations. In real-world scenarios, missing data usually occurs from factors such as motion blur, self-occlusion, or exceeding the field of view. Our algorithm can also handle these missing data situations. To test for the effects of missing data of our algorithm, the simulated experiments are conducted under conditions of high \textit{reconstructability}. We set the position systematic noise of the camera's optical center with a standard deviation of 0.1 m, the position random noise of the camera's optical center with a standard deviation of 0.1 m, the angle systematic noise of the sight-rays with a standard deviation of 0.1°, and the angle random noise of the line of sight with a standard deviation of 0.05°. Occlude 0, 20, 40, and 60\% of the observation data, respectively, and use our algorithm to reconstruct the target trajectory. The mean RMS error under occlusion conditions is shown in Figure \ref{fig15}. The experimental results demonstrate that our method exhibits strong robustness under the weak observation conditions of missing data. It does not degenerate even under up to 60\% occlusion.