\section{Real-world experiments}\label{sec4}

In this section, we evaluate our method on the real-world data. In the real-world data, the UAV observation platform observes vehicles traveling on the highway. The flight platform is equipped with a monocular RGB camera, which captures images at a spatial resolution of 1280 × 720 pixels and a temporal resolution of 25 Hz. The camera exhibits a constrained field of view, with both its horizontal and vertical angular extents not exceeding 2°. The observation range of the flight platform extends from 10 to 20 km. The moving target travels along the highway in approximately uniform straight-line motion, with a speed of approximately 60 km/h. The position of the camera's optical center and the ground truth of the target trajectory are provided by satellite positioning. Due to the long observation distance, the inclination angle is large. The Kernelized Correlation Filters (KCF) \cite{Henriques2015} algorithm is used to track the target. Under such observation conditions, the target on the road can be regarded as a point target. We are only concerned with the trajectory and motion parameters of the target, regardless of its attitude. The center of the bounding box is taken as the image point of the target. Our purpose is to utilize images captured by aerial platforms equipped with only a monocular camera to reconstruct the target's motion parameters and trajectory, as shown in Figure \ref{fig16}.

\begin{figure}[htbp]
\centering
\includegraphics[width=5.5in]{fig/Fig16.pdf}
\caption{Equipped with a monocular camera only, the UAV tracks the moving target and reconstructs its trajectory.}
\label{fig16}
\end{figure}

Three sequences of continuous observational images are selected, each spanning approximately 30 seconds. Therefore, each sequence contains 750 observations, sufficient to solve for the target's uniform linear motion. The flight platform's trajectories are approximated as straight linear over the first two sequences' observation time. Consequently, these conditions approach a degenerate case for the TI method with a low \textit{reconstructability}. In contrast, the flight platform's trajectory is a curve in the last sequence. So, the last observational dataset is conducted under conditions of high \textit{reconstructability}. The values of the \textit{reconstructability} for the three observational datasets are 0.29, 0.85, and 8.05, respectively. The TI method \cite{Yu2009} and the proposed algorithm are employed to reconstruct the target trajectory. Figure \ref{fig17} illustrates the reconstruction results for all three observational datasets. The localization errors are presented in Table \ref{tab2}, \ref{tab3}, and \ref{tab4}, respectively.

\begin{figure}[htbp] 
\centering
\subfloat[]{\includegraphics[width=2.1in]{fig/Fig17a.pdf}%
\label{fig17a}}
\hfil
\subfloat[]{\includegraphics[width=2.1in]{fig/Fig17b.pdf}%
\label{fig17b}}
\hfil
\subfloat[]{\includegraphics[width=2.1in]{fig/Fig17c.pdf}%
\label{fig17c}}
\caption{Illustration of real-world experimental results. (a) Real-world experiment under a low \textit{reconstructability} of 0.29. (b) Real-world experiment under a low \textit{reconstructability} of 0.85. (c) Real-world experiment under a high \textit{reconstructability} of 8.05.}
\label{fig17}
\end{figure}

\begin{table}[htbp]
    \centering
    \caption{Localization error of Experiment \textbf{a} under the condition of \textit{reconstructability} $\eta=0.29$}
    \label{tab2}
    \begin{tabular}{cccccc}
        \toprule
      Method  & $\mathrm{\sigma_x(m)}$ & $\mathrm{\sigma_y(m)}$ & $\mathrm{\sigma_z(m)}$ & $\mathrm{\sigma(m)}$ \\
      \midrule
      TI  & 4534.25 & 8754.78 & 13741.21 & 16912.32 \\
      Ours  & 15.37 & 30.46 & 49.81 & \textbf{60.37} \\
    \bottomrule 
    \end{tabular}
\end{table}

\begin{table}[htbp] 
    \centering
    \caption{Localization error of Experiment \textbf{b} under the condition of \textit{reconstructability} $\eta=0.85$}
    \label{tab3}
    \begin{tabular}{cccccc}
        \toprule
      Method  & $\mathrm{\sigma_x(m)}$ & $\mathrm{\sigma_y(m)}$ & $\mathrm{\sigma_z(m)}$ & $\mathrm{\sigma(m)}$ \\
      \midrule
      TI  & 10669.07 & 7772.67 & 1241.33 & 13258.37 \\
      Ours  & 37.89 & 27.19 & 4.48 & \textbf{46.85} \\
    \bottomrule 
    \end{tabular}
\end{table}

\begin{table}[htbp] 
    \centering
    \caption{Localization error of Experiment \textbf{c} under the condition of \textit{reconstructability} $\eta=8.05$}
    \label{tab4}
    \begin{tabular}{cccccc}
       \toprule
      Method  & $\mathrm{\sigma_x(m)}$ & $\mathrm{\sigma_y(m)}$ & $\mathrm{\sigma_z(m)}$ & $\mathrm{\sigma(m)}$ \\
      \midrule
      TI  & 148.17 & 540.78 & 714.64 & 908.36 \\
      Ours  & 10.43 & 33.80 & 47.01 & \textbf{58.83} \\
    \bottomrule 
    \end{tabular}
\end{table}

Consistent with the conclusions drawn in Section \ref{sec2.4}, under conditions of low \textit{reconstructability} in the observational dataset, the TI method exhibits degeneracy, with the reconstructed target trajectory closely approximating the camera trajectory. And the smaller the value of \textit{reconstructibility}, the closer the reconstructed trajectory is to the camera trajectory. However, the introduction of ridge estimation effectively mitigates the problem of ill-conditioning, ensuring that our algorithm can still accurately reconstruct the trajectory of the target under conditions of low \textit{reconstructability}. Moreover, under conditions of high \textit{reconstructability} in the observational dataset, the reconstruction accuracy of the TI method remains relatively low due to factors such as long observation distance, narrow field of view, and high noise levels, as shown in Figure \ref{fig17c} and Table \ref{tab4}. In contrast, our algorithm demonstrates superior robustness, enabling the high-precision reconstruction of the target trajectory under such limited observation conditions.

Experiments employing real-world data corroborate the efficacy of our proposed algorithm under limited observation conditions of long observation distance, low \textit{reconstructability}, limited field of view, and high noise levels. The results demonstrate that our method achieves significantly higher accuracy than the conventional trajectory intersection method. Moreover, our method exhibits remarkable stability, maintaining consistent performance without degeneracy. Thus our method shows superior robustness. This experiment also validates the analysis in Section \ref{sec2.4}. As the trajectory of the flight platform becomes more complex, transitioning from straight lines to curves, the \textit{reconstructability} of the system increases, leading to improved reconstruction accuracy.

In practical applications, we recommended maneuvering the UAV in response to the motion of the target points. By adjusting the UAV's steering, acceleration, and deceleration, the complexity of its moving trajectory can be enhanced, thereby enhancing the \textit{reconstructability} of the system. This approach can significantly improve the reconstruction accuracy of the point's trajectory.