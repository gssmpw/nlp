\section{The Query Agent}
\label{sec:query-agent}
Query formulation is an opaque process in the lens of the search system: it only has access to the query with no insight into how it was constructed.
With the introduction of query agents, the interplay between user intent and the agent interpretation thereof also plays an important role.
This could introduce new complexities, biases, and challenges for the search engine~\cite{ranking-manipulation-conversational}.
Agents can generate queries that are of a different nature than those generated by humans~\cite{llm-query-variations}, and have different characteristics~\cite{query-generation-personality}. 
This necessitates developing query agents that formulate queries that reflect human intent while also accounting for potential biases. 

Query agents can be integrated with other agents. For example, an LLM-based conversational agent could invoke a lexical module to generate queries in order to mitigate the potential bias of ranker agents to certain query agent types. 

The ranker agent may operate with varying levels of knowledge about query generation~\cite{query-rewrite-for-rag}.
It could be aware of the query agent and the inputs (e.g., information need description if the query agent is part of the search system), it could have knowledge of the agent type (e.g., via API call signatures or query patterns), or it could remain entirely isolated from the generation process of the query. It is interesting to explore methods 
methods for making inference about the query agent under different knowledge states. 

\myparagraph{Decentralizing the ranker}
A query agent can search simultaneously with multiple rankers and/or multiple query variations and aggregate results based on different criteria.  
The merits of retrieving with more than one query were demonstrated~\cite{mult-queries-benefit}. 
This paradigm shift enables post-ranking interventions such as relevance feedback and fusion of ranked lists. 
Consequently, the ranking process becomes somewhat decentralized in the sense that the outputs of the ranker need not be the final output the user obtains. 

This creates opportunities for research on developing methods for the query agent to refine and aggregate search results through post-processing techniques.
Among the topics for future exploration are: (i) how to select ranker agents; (ii) how to formulate a query for each ranker agent; and (iii) how to fuse results retrieved by different rankers so as to satisfy the information need. Note that in contrast to standard fusion, which is often applied by meta-search engines that have no information about the underlying information need except for the query \cite{Kurland+Culpepper:18a}, here the query agent can know the information need. 

\myparagraph{Learning to Search} Unlike human users, who iteratively refine queries through manual reformulation, query agents can generate large-scale query variations and evaluate them in a systematic way. These agents can learn from past user interactions and generalize over users to refine query generation. 

\subsection{Strategic Query Agents}

\myparagraph{Matching the ranker agent} 
The query agent could employ different strategies. We show in Section~\ref{sec:effectiveness-experiment} that the alignment between the query and the ranker agents has a significant impact on retrieval effectiveness.
The query agent could employ methods to align with the ranker agent or a specific document agent. For example, when interacting with a lexical ranker agent, the query agent can estimate IDF values of terms~\cite{Yossef+Gurevich:08a} and formulate the query accordingly.


\myparagraph{Impact on publishers dynamics}
Both theoretical and empirical evidence suggests that the query agent has the potential to impact the dynamics between publishers~\cite{MultQueries}. Our analysis in Section~\ref{sec:online-experiment} further reinforces these findings. For example, prior work~\cite{MultQueries} demonstrated that an increased number of query variations makes equilibrium states in competitive retrieval unattainable. 







