\section{Empirical Exploration}
\label{sec:experimental-settings}

The main goal of the experiments presented below is to demonstrate the effects on the retrieval setting of interactions between different query, document, and ranker agents.
To this end, we conduct a series of three experiments. 
In the first experiment, we evaluate the effectiveness of different rankers
with various query and document agents (Section~\ref{sec:effectiveness-experiment}).
In the second experiment, we explore the interplay between the document and ranker agent, specifically when document agents compete against human-authored documents for rank promotion (Section~\ref{sec:offline-experiment}).
In the third experiment, we
study how ranker and query agents influence the competitive dynamics among different document agents (Section~\ref{sec:online-experiment}).

We now turn to describe our methodology. First, we detail the datasets used for evaluation (Section~\ref{sec:data}). Then, in Section~\ref{sec:agents-implementation}, we introduce the implementations of the ranker, query, and document agents, each categorized into three types: lexical, semantic, and LLM-based. Additionally, for query and document agents, we also consider human agents, i.e., queries and documents generated by humans.
Several of the agents we present are novel to this study.

% overview
\subsection{Data}
\label{sec:data}

We utilize datasets from ranking competitions in which human participants competed against automated agents ~\cite{Greg-bot, Niv, MultQueries}. These competitions were structured to simulate realistic retrieval environments where documents evolved across multiple ranking rounds due to ranking incentives. The datasets consist of documents that were iteratively modified by human participants attempting to improve their ranking, alongside automated agents that applied ranking-incentivized modifications.

% details of competitions
We use four publicly available datasets~\cite{Greg-bot, Niv, MultQueries} with recordings of content-based ranking competitions held in IR courses where students acted as publishers. The students were given an initial document and were tasked with promoting it in rankings induced by an undisclosed ranking function over several rounds. At the
beginning of the competition, all students received the same example of a relevant document for each topic. After each round, they were presented with a ranking of their documents.
Their objective was to modify their documents in an effort to improve their ranking in the subsequent round. All documents were plaintext and limited to a maximum of $150$ terms. The competitions were incentivized through performance-based course-grade bonuses and were conducted with the approval of ethics committees.


\begin{table}[h] \footnotesize  \centering\resizebox{0.48\textwidth}{!}{\begin{tabular}{c|l|c|c|c}
\toprule
\textbf{Task} & \textbf{Dataset} & \textbf{N-shot} & \multirowcell{\textbf{Train texts} \\ \textbf{for STMD}} & \multirowcell{\textbf{Evaluation} \\ \textbf{texts}} \\
\midrule
\multirow{3}{*}{\multirowcell{Text \\ Summarization}} & CNN/DailyMail & 0 & 2,000 & 2,000 \\
& XSum & 0 & 2,000 & 2,000 \\
& SamSum & 0 & 2,000 & 819 \\
\midrule
\multirow{4}{*}{\multirowcell{QA \\ Long answer}} & PubMedQA & 0 & 2,000 & 2,000 \\
& MedQUAD & 5 & 2,000 & 2,000 \\
& TruthfulQA & 5 & 408 & 409 \\
& GSM8k & 5 & 2,000 & 1,319 \\
\midrule
\multirow{4}{*}{\multirowcell{QA \\ Short answer}} & SciQ & 0 & 5,000 & 1,000 \\
& CoQA & \multirowcell{all preceding \\ questions} & 5,000 & 2,000 \\
& TriviaQA & 5 & 5,000 & 2,000 \\
\midrule
\multirow{1}{*}{\multirowcell{MCQA}} & MMLU & 5 & 5,000 & 2,000 \\
\bottomrule
\end{tabular}
}\caption{\label{tab:dataset_stat} The statistics of the datasets used for evaluation.}
\end{table}

% the datasets
% greg data
The competitions conducted by~\citet{Greg-bot} employed a \lambdamart
ranker \cite{wu2010adapting} with $25$ features: $24$ lexical
features, and a single document quality score.  The competitions were held for
$15$ queries (topic titles) from TREC's ClueWeb09 dataset.

% niv dataset 
The competitions held by~\citet{Niv} employed the \efive-based ~\cite{e5} ranking function\footnote{intfloat/e5-large-unsupervised}. Two competitions were held for each of the $15$ same queries as Goren et al.'s~\cite{Greg-bot} competition.

% mult dataset
As discussed in Section~\ref{sec:query-agent}, the query agent might consider different query variations, generated by either one or a variety of query agent types. 
To explore the effects of query variations on the competition dynamics, we utilize the dataset from \citet{MultQueries}. In these competitions students were requested to improve their rankings for three different query variations per topic. We use two of their competitions: the ones where students were not allowed to use \llm tools.
In the first competition, the ranker was a \bert~\cite{BERT} model fine-tuned on the MS-MARCO dataset.
The second competition employed LambdaMART trained specifically for competitive setings~\cite{Ziv-Ranker}.
The features were all lexical except for one semantic feature based on the score of the same \bert model used in the first competition.  
Both competitions were conducted for $30$ topics from TREC's ClueWeb12 collection, with query variations from the dataset reported in~\cite{uqv100}.

% single vs. multi
We use \firstmention{\single} to refer to competitions where students competed for a single query and \firstmention{\multi} 
to refer to competitions with multiple query variations. Similarly, \firstmention{\ltr} denotes competitions using a LambdaMART,
while \firstmention{\neu} refers to those using a neural ranker (\bert / \efive). 

All competitions featured a total of five participants per topic. However, not all of them were students, as some of the competing documents were planted. 
In addition, each competition included different \botagents that competed alongside human participants. The students were unaware that they were competing against automated agents. The \botagent in \GregDataset utilized both lexical and semantic features, whereas in the other competitions, the \botagent was entirely \llm-based.
Table~\ref{tab:dataset} summarizes the details of all four datasets used in our experiments.


\subsection{Agents Implementation}
\label{sec:agents-implementation}
% \section{LLM-empowered Agent}\label{sec:agent}
In this section, we delve into the potential of LLMs as intelligent agents in FLE. LLMs can act as catalysts for personalized learning, addressing the long-standing scalability, adaptability, and inclusivity challenges in traditional teaching paradigms.

\subsection{Fundamental Abilities}
This section highlights five key abilities of LLM-empowered agents that enable them to function as adaptive tutors.

\paragraph{Knowledge Integration.} LLMs excel at merging structured educational knowledge graphs~\cite{abu2024knowledge,hu2024foke} with unstructured textual data~\cite{li2024supporting,modran2024llm}, providing rich, contextualized information on linguistic constructs and cultural nuances. Their ability to perform real-time knowledge editing~\cite{wang2024knowledge,zhang2024comprehensive} ensures learners receive content aligned with evolving language usage, addressing the inherent limitations of static materials.

\paragraph{Pedagogical Alignment.} LLMs require embedding with pedagogical principles to facilitate genuine learning experiences~\cite{carroll1965contributions,taneja1995educational}. Recent work incorporates theoretical frameworks, such as Bloom’s taxonomy~\cite{bloom1956taxonomy}, to guide LLMs in systematically addressing different cognitive levels~\cite{jiang2024llms}. Approaches like \textit{Pedagogical Chain of Thought}~\cite{jiang2024llms} and \textit{preference learning}~\cite{sonkar-etal-2024-pedagogical,rafailov2024direct} focus on aligning model responses with educational objectives.

% \paragraph{Pedagogical Alignment.} To truly function as effective educational agents, LLMs must be imbued with pedagogical modes of thought~\cite{carroll1965contributions,taneja1995educational}, which can be achieved through pedagogical alignment~\cite{razafinirina2024pedagogical}. Inspired by Bloom Cognitive Model~\cite{bloom1956taxonomy}, which categorizes student abilities into six cognitive levels—Remember, Understand, Apply, Analyze, Evaluate, and Create—\citet{jiang2024llms} proposes \textit{Pedagogical Chain of Thought} to enhance mistake correction of LLMs without requiring fine-turning. \citet{sonkar-etal-2024-pedagogical} develop a synthetic preference dataset embedded with pedagogical principles and explore whether preference learning techniques~\cite{rafailov2024direct,azar2024general,ethayarajh2024kto} can align LLMs with educational objectives. The conversational CLASS~\cite{sonkar-etal-2023-class} framework equips intelligent tutoring systems with tutor-like step-by-step guiding strategies rather than merely providing direct answers to students.


\paragraph{Planning.} By assisting in crafting teaching objectives and lesson designs, LLMs can handle complex tasks such as differentiated instruction~\cite{hu2024teaching}. LessonPlanner~\cite{fan2024lessonplanner} has been proposed to assist novice teachers in preparing lesson plans, with expert interviews confirming its effectiveness. \citet{zheng2024automatic} propose a three-stage process to produce customized lesson plans, using Retrieval-Augmented Generation (RAG), self-critique, and subsequent refinement.

\paragraph{Memory.} Effective tutoring systems track learner histories and tailor subsequent interactions accordingly~\cite{jiang2024ai,chen2024empowering}. When serving as memory-augmented agents, LLMs can retain individualized data—such as repeated grammar mistakes or overlooked vocabulary—thereby improving continuity and enabling consistent scaffolding of future learning tasks.

\paragraph{Tool Using.} Beyond textual interactions, LLM-based agents can integrate specialized tools to streamline the educational ecosystem, from cognitive diagnosis modules~\cite{ma2019cognitive} to report generators~\cite{zhou2025study}. By orchestrating these resources, LLMs seamlessly unify diverse utilities under a single interface, enhancing learner experience and instructional efficiency.

\paragraph{Discussion.} Existing research often overlooks the interplay among listening, speaking, reading, and writing in real-world language learning~\cite{caines2023application,shetye2024evaluation}. Most systems concentrate on text-oriented features, lacking robust benchmarks and methodologies to evaluate integrated, multimodal interactions. Recent work on pedagogical alignment~\cite{razafinirina2024pedagogical} largely addresses textual data, leaving out real-time speaking and listening dynamics that demand complex, rapid feedback. Similarly, while memory modules can track repeated written errors, their effectiveness in monitoring and improving learning efficacy remains underexplored.


% \paragraph{Planning.} Planning is a cornerstone of successful education, involving setting instructional objectives, identifying teaching priorities, organizing teaching activities, articulating subject content, and selecting methods and strategies~\cite{hu2024teaching}. The design of teaching plans entails an abundant knowledge base and rich teaching experiences to tailor to learners' diverse needs,  with the constraint of available resources. Recent advances in LLMs demonstrate significant potential in planning complex tasks~\cite{huang2024understanding}. LessonPlanner~\cite{fan2024lessonplanner} has been proposed to assist novice teachers in preparing lesson plans, with expert interviews confirming its effectiveness. \citet{zheng2024automatic} propose a three-stage process to produce customized lesson plans, using Retrieval-Augmented Generation (RAG), self-critique, and subsequent refinement.

% \paragraph{Memory.} The ability to retain and recall information about a learner's progress is essential for fostering consistent and personalized education. LLMs can serve as memory-augmented agents~\cite{jiang2024ai}, maintaining detailed records of a learner's strengths, weaknesses, and historical performance. By leveraging this memory, LLMs can provide targeted feedback, revisit challenging topics, and build on prior knowledge in a coherent manner~\cite{chen2024empowering}. This continuity not only enhances the learner's experience but also ensures that educational interventions are data-driven and evidence-based.

% \paragraph{Tool Using.} LLMs can further extend their utility by interfacing with a variety of educational tools, such as cognitive diagnosis~\cite{ma2019cognitive} and report generation~\cite{zhou2025study}. LLMs can integrate these tools into a cohesive learning ecosystem by serving as a central orchestrator, streamlining the learner's experience. This ability to seamlessly incorporate external resources amplifies the effectiveness of LLM-driven educational systems.


\subsection{Applications}
Although still in its early stages, LLM-empowered agents have already started to show promising applications in FLE.
% These applications leverage the capabilities of LLMs to create more interactive and personalized learning environments, offering new ways to address traditional challenges in language instruction. Despite being in development, these applications have the potential to transform how language learners engage with content and interact with tutoring systems.


\paragraph{Classroom Simulation.} Classroom simulation leverages LLM-empowered agents to recreate complex, interactive learning settings without the logistical hurdles of organizing physical classrooms~\cite{zhang2024simulating}. By simulating virtual students and tutors, researchers can study pedagogical strategies at scale, generate diverse learner interactions, and refine teaching techniques. Moreover, this virtual data can be used to fine-tune LLMs for specific educational contexts and learner profiles~\cite{liusocraticlm}, offering a cost-effective and adaptable approach to language instruction.

\paragraph{Intelligent Tutoring System (ITS).} LLM-based agents have demonstrated the capacity to provide dynamic, personalized tutoring experiences~\cite{kwon-etal-2024-biped}, effectively identifying learner weaknesses through large-scale linguistic analysis~\cite{caines2023application}. This makes them promising for delivering individualized instruction at scale. Although current ITS applications in mathematics~\cite{pal2024autotutor} and science~\cite{stamper2024enhancing} have shown success, the extension to FLE requires nuanced handling of cultural and contextual elements, as well as the unpredictability of human language usage.

\paragraph{Discussion.} Despite the promise of these applications, critical challenges remain. Existing classroom simulation frameworks often \textit{lack standardized benchmarks for FLE}, making it difficult to assess the efficacy and generalizability of developed systems~\cite{zhang2024simulating}. In addition, evaluating language-specific tutoring strategies, including real-time conversational practice and holistic skill integration, remains an underexplored frontier. Addressing these gaps requires \textit{new datasets and metrics} centered on holistic skill development, as well as interdisciplinary collaboration.

\begin{tcolorbox}[top=1pt, bottom=1pt, left=1pt, right=1pt]
\textbf{Our position.} We argue that to overcome current limitations, \textit{future research} should focus on the integration of multimodal learning tasks~\cite{sonlu2024effects} and the development of standardized frameworks for evaluating FLE simulations. Moreover, LLMs should evolve beyond text-based capabilities to provide real-time, context-sensitive feedback, particularly in speaking and listening. Interdisciplinary collaboration and the creation of new datasets tailored to FLE are crucial for refining these systems and ensuring their scalability and inclusivity in language instruction. Additionally, addressing the complexities of cultural context and learner variability will be key to the success of LLMs as effective agents in FLE.
\end{tcolorbox}


% \paragraph{Classroom Simulation.} Classroom simulation is a powerful application of LLM-empowered agents that addresses the limitations of traditional language learning environments. By simulating interactions among virtual students and tutors, multi-agent setups create realistic, dynamic classroom scenarios that reflect real-world complexities. This approach not only facilitates the study of pedagogical strategies but also accelerates educational research by eliminating the logistical challenges of organizing physical classrooms~\cite{zhang2024simulating}. Additionally, simulated classroom data can be used to fine-tune LLMs, enhancing their ability to adapt to diverse teaching contexts and learner profiles~\cite{liusocraticlm}. These simulations offer scalability, cost-efficiency, and the opportunity to test various instructional methods in controlled yet realistic environments.

% \paragraph{Intelligent Tutoring System (ITS).} LLM-powered agents have the potential to create an immersive learning environment and provide a personalized alternative to traditional one-on-one tutoring~\cite{kwon-etal-2024-biped}. Furthermore, their ability to process vast amounts of linguistic data enables them to identify and address specific learner weaknesses~\cite{caines2023application}, making them particularly effective in delivering tailored instruction at scale. However, recent efforts are mainly on non-language subjects like mathematics~\cite{pal2024autotutor} and science~\cite{stamper2024enhancing} which hold more objective learning goals.

\myparagraph{Ranker}
We use
three types of agents (lexical, semantic and LLM) each with two rankers.
\subsubsection*{\lexicalagent}
% TF-IDF weights 
We use \bm and \tfidf\footnote{The retrieval score is the sum of \tfidf weights of query terms in the document.}  with Indri's default hyperparameter values\footnote{\url{https://www.lemurproject.org/indri}}.
Collection statistics are based on ClueWeb09 for \GregDataset and \NivDataset and ClueWeb12 for \MultiB and \MultiD.

\subsubsection*{\semanticagent}
We use \contriever~\cite{contriever} and \efive~\cite{e5}, both fine-tuned on the MS-MARCO~\cite{msmarco} dataset\footnote{intfloat/e5-base-v2, nthakur/contriever-base-msmarco}. The document retrieval score is the cosine between the query and document embedding vectors.

\subsubsection*{LLM} 
We adopt a pointwise relevance generation model~\cite{liang2022holistic} for an LLM ranker.
We experiment with two lightweight ($<10$B parameters) instruction-tuned open-source LLMs: \llamaWithVersion~\cite{dubey_llama_2024} and \gemmaWithVersion~\cite{gemma_team_gemma_nodate} from the Hugging Face repository\footnote{meta-llama/Meta-Llama-3.1-8B-Instruct, google/gemma-2-9b-it}. 

All the pre-trained models described above are used consistently across all experiments with their Hugging Face default hyperparameter values, unless specified otherwise.

\myparagraph{Query}
To
study the impact of different query formulation techniques, we implemented several query agents that generate queries based on the backstories provided in the UQV100 dataset~\cite{uqv100}.


%In our experiments,
Herein, we treat each query variation independently as a query, i.e., without directly considering the combined effect of multiple queries simultaneously generated by the query agent.
Previous work~\cite{MultQueries} explored the dynamics when humans and document agents compete for multiple (human) queries simultaneously. 
%multiple queries --- all human generated.  
The exploration of the dynamics when document agents compete for multiple queries generated by different query agents is beyond the scope of this paper and remains an interesting direction for future work. 


\subsubsection*{Human} 
We utilize the human-generated query variations from the UQV100 dataset, which was created via crowdsourcing~\cite{uqv100}. For each topic, we selected the five most frequent query variations. 

\subsubsection*{Lexical}
We applied the YAKE keyword extraction method~\cite{YAKE} to identify key phrases in the backstories. Each extracted phrase was considered a potential query and was ranked w.r.t. the backstory using BM25. The five highest-scoring phrases per topic were selected.

\subsubsection*{Semantic} 
We adopt the doc2query method~\cite{doc2query} to generate for each topic a pool of $1,000$ query variations from the backstory.
We use two representations: \efive and \contriever.
For each of them, the final five variations for each topic were selected based on the cosine similarity of the embedding vector of the variation and that of the backstory. 


\subsubsection*{LLM}
We utilized an existing dataset containing query variations generated by \gptWithVersion~\cite{llm-query-variations} and replicated its methodology to generate additional variations with \llama~\cite{dubey_llama_2024} and \gemma~\cite{gemma_team_gemma_nodate}, all set with a temperature of $1$.
To ensure well-formed outputs, we added to the prompt an instruction to return plain lists of queries. For each topic, the first five variations generated by each LLM were selected.





\myparagraph{Document}
% How we defined the agents for the experiment
We implement different document agents that apply ranking-incentivized modifications to documents. 
The \botagents operate as follows: first, they observe the ranking for a given query. Then, they apply modifications to a document to improve its ranking in the next round.

Two
ranking-incentivized \botagents for document modifications were recently presented.
~Gorent et al.'s \cite{Greg-bot} \botagent replaces a sentence \SrcSentence from the
document with a candidate sentence \TargetSentence. 
The \botagent selects the pair $(\SrcSentenceMath, \TargetSentenceMath)$ using a learning-to-rank approach with a small set of lexical and semantic features. 
Our goal is to compare \textit{lexical} and \textit{semantic} agents, making it unsuitable to adopt their method directly.
Instead, we develop a \botagent inspired by their approach.
Bardas et al. ~\cite{Niv} used LLM-based agents for document modification. We adopt their approach to develop our LLM \botagent.


\subsubsection*{Lexical}

Inspired by~\citet{Greg-bot}, the lexical \botagent modifies documents
by replacing a sentence from the original document, \SrcSentence (source), with
a candidate sentence \TargetSentence (target).  Candidate sentences are
extracted from documents published by other publishers in the round.
The \botagent developed by \citet{Greg-bot} relies on a set of lexical and semantic features.  In particular, it employs two lexical
features, assuming that higher feature values indicate increased
retrieval score assigned by the undisclosed ranking function:
\firstmention{\QryTerm}, which is the fraction of query term occurrences in the sentence, and \firstmention{\SimTop}, which represents the cosine similarity between the sentence's \tfidf vector and the centroid of the top \numhighestranked ranked documents in the current ranking.

Each feature is computed for both the source (\QryTermSrc, \SimSrcTop) and target (\QryTermTarget, \SimTargetTop) sentences. The final score is:
\begin{equation*}
    score_{lex}(\SrcSentenceMath, \TargetSentenceMath) = \interpolationMath \cdot (\QryTermTargetMath - \QryTermSrcMath) + (1-\interpolationMath) \cdot (\SimTargetTopMath-\SimSrcTopMath).
\end{equation*}
The highest scoring sentence pair is selected to perform the replacement.
To ensure that modifications do not significantly alter
the document, we only consider a pair $(\SrcSentenceMath, \TargetSentenceMath)$ if the \tfidf-based cosine similarity between \SrcSentence and \TargetSentence exceeds a predefined threshold \nlithreshold.
The parameter values are: $\interpolationMath \in [0, 0.1, \ldots,1]$, $\numhighestrankedMath \in \{2, 3, 4\}$ and $\nlithresholdMath \in [0,0.1,\ldots,0,5]$.

\subsubsection*{Semantic}
The semantic document agent, novel to this study, modifies documents in a similar manner to the lexical document agent, but relies on semantic (embedding) representations rather than lexical features. 
We
use three strategies to select the candidate target sentences to use
for replacement in the original document.
The first strategy, \firstmention{\all}, considers sentences from all
other documents as candidates.  The \firstmention{\better} strategy
selects only sentences from documents ranked higher than the original
document.  The \firstmention{\best} strategy picks sentences
exclusively from the highest-ranked document.  For each pair of
sentences $(\SrcSentenceMath, \TargetSentenceMath)$ and query $q$, we
compute the cosine similarity between the embedding vectors
representing \SrcSentence and \TargetSentence and the similarity between \TargetSentence and $q$: 
\begin{equation*}
    score_{sem}(\SrcSentenceMath, \TargetSentenceMath) = \interpolationMath \cdot cos(\SrcSentenceMath, \TargetSentenceMath) + (1-\interpolationMath) \cdot cos(\SrcSentenceMath, q)
\end{equation*}
We use this method to balance the presumed relevance of the candidate (target)
sentences, and accordingly their contribution to the retrieval score, and their
contextual fit within the document.  To preserve document coherence,
we use an NLI model~\cite{NLI} to ensure that the new sentence is entailed
by the candidate sentence. A candidate sentence is considered only if
it entails the original sentence, which is considered to be the case
if the probability assigned by the NLI model is higher
than \firstmention{\nlithreshold}.  In our experiments, we use two
dense pretrained text embedding models before
fine-tuning: \efive~\cite{e5}
and \contriever~\cite{contriever}\footnote{intfloat/e5-base-unsupervised,
facebook/contriever}.
The value of \interpolation was selected from $[0, 0.1,\ldots,1]$ and the value of \nlithreshold was selected from $[0,0.1,\ldots,0.5]$. 

\subsubsection*{LLM}

The \textit{LLM-based} \botagent introduced by \citet{Niv} modifies documents using different prompting strategies.
%that guide the agent's behavior. 
Document agents are defined by a specific LLM model and the prompting strategy.
The prompts include the rules and constraints for the document format, the original document, and the assigned query.
Additionally, they include information about past documents and their rankings. 
We use two prompting strategies: \feedbackPair, which includes two randomly selected documents and their rankings, and \feedbackAll, which provides the full ranked list. 
To mitigate copying behavior observed in human-driven ranking competitions~\cite{Nimrod, Greg-Herding}, the \nocopy\xspace flag ensures that the \botagent is also explicitly warned against directly copying content from other documents. 
Both \llama and \gemma were used with $top_p=0.95$ and $temperature=0.7$.

Unless otherwise specified, the free-parameter values of all document agents were set per ranker to maximize \firstmention{scaled rank promotion}: the raw rank change of a document, normalized by the maximum possible promotion or demotion based on its position. Higher values indicate increased improvement.
