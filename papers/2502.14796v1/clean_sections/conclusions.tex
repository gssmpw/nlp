\section{Conclusions}
The progress with large language models (LLMs) has opened the era of agents, specifically, in the information retrieval (IR) field \cite{Chen+al:21a,Shah+White:24a,White:25a,Zhai:24a,Zhang+al:24a}. We argued that the most fundamental task in the information retrieval field, namely, ad hoc retrieval, should now be re-visited in the face of multi-agent retrieval settings.

Document authors use agents to generate content. Users of search engines use agents as assistants for their information seeking tasks. There are novel opportunities for agent-based retrieval models. We argued that existing retrieval frameworks and paradigms, as well as evaluation, should be re-considered in the agent era. We also discussed fundamental research directions we believe are important for the study of multi-agent retrieval settings. In addition, we empirically analyzed an elaborated suite of multi-agent retrieval settings. Our findings shed light on the far reaching implications of the inter-play between document, user and ranker agents.


