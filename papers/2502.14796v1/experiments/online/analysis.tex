
\begin{table}
    \caption{The average rank position (over all competitions and rounds) of the static document agent's document.}
    \centering
    \scriptsize
    % \renewcommand{\arraystretch}{1.3}
    % \setlength{\tabcolsep}{6pt}
    \begin{tabular}{l|ccc}
        \toprule
        & \multicolumn{3}{c}{\textbf{Ranker Agent}} \\
        \cmidrule(lr){2-4}
        \textbf{Query Agent} & \textbf{\lexicalagent} & \textbf{\semanticagent} & \textbf{\llm} \\
        \midrule
        human    & 2.79 & 3.09 & 3.19 \\
        lexical  & 3.26 & 3.45 & 3.32 \\
        semantic & 3.10 & 3.33 & 3.17 \\
        llm      & 2.99 & 3.19 & 3.26 \\
        \bottomrule
    \end{tabular}
    \label{tab:static_player}
      % \vspace{-6mm}
\end{table} 

We begin by evaluating the performance of the static document agent. The variance of its performance across different query agents indicates fundamental differences in the settings' dynamics. We see in Table~\ref{tab:static_player} that the static \botagent demonstrates superior performance when the query agent type and ranker agent type are not aligned.
For example, the average rank of the static \botagent is $3.26$ when both the query agent and the ranker agent are lexical, but the average rank is $2.99$ when the query agent type is LLM. 
This means that when the \highlight{query and ranker types are mismatched, the ability of publishers to perform ranking-incentivized document manipulations decreases.}  This finding is in line with that presented above about the impact of the alignment between the document agent and the ranker agent type on the document agent rank-promotion.

\begin{figure}
    \centering
    \scriptsize
    \begin{subfigure}{0.3\textwidth}
        \centering
\includegraphics[width=\linewidth]{experiments/online/figures/legend.png}
    \end{subfigure}
    \begin{subfigure}{0.45\textwidth}
        \centering
        \includegraphics[width=0.9\linewidth]{experiments/online/figures/e5_by_query_agent.png}
        \caption{Ranker Agent: \efive.}
        \label{fig:e5_misalignment}
    \end{subfigure}
    \hfill
    \begin{subfigure}{0.45\textwidth}
        \centering
        \includegraphics[width=0.9\linewidth]{experiments/online/figures/bm25_by_query_agent.png}
        \caption{Ranker Agent: \bm.}
        \label{fig:bm25_misalignment}
    \end{subfigure}
    \caption{The average rank of document agents in each round across different ranker agents and query agents.}
\label{fig:query_ranker_misalignment}
% \vspace{-1mm}
\end{figure}


Diving deeper into the missaligment between the ranker agent type and the query agent type, we see in Figure~\ref{fig:query_ranker_misalignment} that the performance of \botagents is highly affected by the query agent type and its alignment with the ranker agent type. 
When the ranker agent and query agent types are aligned, the difference between different \botagents is higher (recall that this is a zero-sum game; the success of one \botagent comes at the expense of the other). 
In Figure~\ref{fig:e5_misalignment} we also observe the resultant rank sensitivity of certain document agents to the query agent. For example, when queries were generated by \efive, the \llama agent was the second-best \botagent. However, when queries were generated by humans, its performance declined and resulted in a substantially lower average rank.

\paragraph{Implications for the Ranker Agent}
We already observed in Section~\ref{sec:effectiveness-experiment} that when the query and document agent types are mismatched, the effectiveness of the ranker diminishes. 
However, an open question remains: how does this misalignment specifically impact modification strategies and the performance of \botagents in competitive settings? Understanding this effect is important for designing ranker agents that account for adversarial adaptation and strategic dynamics. 


\paragraph{Implications for the Document Agent}
The results of the offline experiment (Section~\ref{sec:offline-experiment}) suggest that the alignment between the document agent and the ranker agent has high influence on the ability of the publisher to promote her document. Here, the results show that the alignment between the document and the ranker agent is not the only factor that affects the document agent's performance. 
We also showed that the dynamics are highly affected by the query agent. 
An interesting direction for future research is designing strategic query agents that promote desired dynamic outcomes in terms of search effectiveness and corpus effects.
