
\begin{table}
  \caption{The scaled rank promotion of the
    %different
    \botagents, averaged across the \GregDataset and \NivDataset datasets.
    Boldface: the highest promotion in a column.}
    \centering
    % \renewcommand{\arraystretch}{1.15} % Adjust row spacing
    % \setlength{\tabcolsep}{4pt} % Adjust column spacing
    \scriptsize
    \begin{tabular}{l|l|cc|cc|cc}
        \toprule
        \multicolumn{2}{c}{ } &\multicolumn{6}{|c}{\textbf{Ranker Agent}} \\
        \cmidrule(lr){3-8}
        %\midrule
        \multicolumn{2}{c|}{\textbf{Document Agent}} 
        & \multicolumn{2}{c|}{\textbf{Lexical}} 
        & \multicolumn{2}{c|}{\textbf{Semantic}} 
        & \multicolumn{2}{c}{\textbf{LLM}} \\
        \hline
        \textbf{Type} & \textbf{Model} 
        & \textbf{\bm} & \textbf{\tfidf} 
        & \textbf{\contriever} & \textbf{\efive} 
        & \textbf{\gemma} & \textbf{\llama} \\
        \hline
        Human    & -         & 0.071  & 0.074  & 0.179  & 0.183  & 0.117  & 0.078  \\
        Lexical   & -         & \textbf{0.430}  & \textbf{0.433}  & 0.253  & 0.236  & 0.158  & 0.049  \\
        Semantic  & \contriever & 0.269  & 0.267  & \textbf{0.363}  & 0.289  & 0.226  & 0.094  \\
        Semantic  & \efive   & 0.229  & 0.233  & 0.308  & \textbf{0.329}  & 0.183  & 0.092  \\
        LLM       & \gemma     & 0.217  & 0.216  & 0.273  & 0.264  & \textbf{0.497}  & 0.280  \\
        LLM       & \llama     & 0.084  & 0.089  & 0.268  & 0.240  & 0.354  & \textbf{0.640}  \\
        \bottomrule
    \end{tabular}
    \label{tab:bot_performance_large}
    \vspace{-4mm}
\end{table}
  


Table~\ref{tab:bot_performance_large} presents the average scaled rank promotion of the different \botagents. For each ranker agent and \botagent, we present the average scaled promotion of the most effective document agent, i.e., the one whose hyperparameter values were set to optimize scaled rank promotion. Table~\ref{tab:bot_performance_large} shows that the average scaled promotion of all \botagents is positive, which means that, on average, \botagents are always successful in promoting their documents compared to humans, regardless of their type and the ranker agent's type. Specifically, the novel zero-shot lexical and semantic \botagents we developed consistently achieve higher scaled rank promotion values than humans across all ranker agents (\textbf{RQ4}). 
The same finding holds for the LLM document agents as recently shown \cite{Niv}.
This shows that all rankers are vulnerable to ranking-incentivized document manipulations by document agents. 

Table~\ref{tab:bot_performance_large} shows an additional clear trend: \highlight{A mismatch between the document agent type and the ranker agent type leads to a substantial decrease in the ability of the \botagent to improve the ranking of its document}. 
Conversly, when the document agent and ranker agent types are aligned, 
the document agent's ability to achieve higher rankings improves markedly (\textbf{RQ5}).
For example, the \gemma \botagent achieved a scaled rank promotion of $0.497$ when the ranker agent was also \gemma, but only $0.354$ when the ranker agent was \llama. When the ranker agent type was of a different type (i.e., not an LLM), the maximal scaled promotion achieved was only $0.226$.
Similar trends are observed for the other document agents. 
Rank promotion is highest when the document agent and the ranker agent are the same.  
The second highest rank promotion is attained when the ranker agent is of the same type as the document agent.
This indicates that the alignment between the ranker agent and the document agent is crucial from the publishers' perspective for rank promotion strategies.
Moreover, the findings suggest that the publisher can be highly effective even with knowledge of only the ranker agent type (i.e., lexical, semantic, or \llm) (\textbf{RQ7}).
