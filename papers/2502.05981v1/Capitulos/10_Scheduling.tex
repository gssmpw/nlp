\section{Scheduling Optimization Problems}
In this section we will address scheduling problems, commonly related to real industrial problems, and used for performance improvement.

\subsection{Task Scheduling}
We aim to address the challenge of distributing tasks across a set of machines while adhering to specific rules regarding multiple task sets. That is, we have a set of $m$ machines and on $i$-th machine there are $P_i$ possible tasks to perform. The task $j$ of machine $i$ has an execution time $T_{ij}$. Additionally, we have a set of directed rules constraining these task combinations. An example of rules is: “If machine 1 performs task 3 and machine 2 performs task 4, machine 0 must perform task 1”. We seek a task allocation on machines that adheres to the rules while minimizing execution time. If the chosen variables are $x_i$ as the task performed by the $i$-th machine, the cost function is
\begin{equation}
    C(\vec{x}) = \sum_i T_{i,x_i}.
\end{equation}

This problem is deeply analyzed in~\cite{Task_TN}, mixing this MeLoCoToN with genetic algorithms and an iterative Motion Onion. In summary, the problem is solved by a tensor network that minimizes the cost function with initialization tensors with imaginary time evolution, and a series of constraint layers that filter the states. Each layer is composed of a series of control tensors that determine whether the rule constraint is satisfied, and if it is, the projector tensor is told to project the state of the target qudit to the imposed value. In addition, such rules can be condensed under certain conditions to avoid such an exponential scaling with the number of rules. The tensor network is shown in Fig.~\ref{fig: Task Scheduling}.
\begin{figure}[h]
    \centering
    \includegraphics[width=0.7\linewidth]{Images/Task_Scheduling.pdf}
    \caption{Task Scheduling Problem Tensor Network with 6 machines and 3 rules.}
    \label{fig: Task Scheduling}
\end{figure}

In addition, the paper shows how to start the problem without constraints. It is solved until a combination is found that does not comply with a rule, in which case the constraint layer corresponding to that rule and all those that can be condensed with it are added. In this way, iteratively only the strictly necessary part of the solution space is eliminated, avoiding disproportionate scaling.


\subsection{Flow Shop Scheduling Problem}
{\color{red} Subsection not available due to paper pending publication.}







\subsection{Job Shop Scheduling Problem}
{\color{red} Subsection not available due to paper pending publication.}













