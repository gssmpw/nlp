\section{Integer Programming}
In this section we will deal with two general combinatorial optimization formalisms with constraints that are well known and widely used. Integer programming is a whole branch of constrained optimization. It consists in optimization with integer variables, and is usually referred to as integer linear programming (ILP), where the cost function and constraints are linear functions.


\subsection{Integer Linear Programming}
In the case of the ILP, the problems are formulated as follows
\begin{equation}
    \begin{gathered}
        \text{maximize }C(\vec{x}) = \sum_i c_i x_i,\\
        \text{subject to } \sum_{j} A_{ij}  x_j \leq b_i \ \forall i,\\
        x_i \in \mathbb{Z}^+
    \end{gathered}
\end{equation}

In general, this kind of problems can be solved if $A$ and $\vec{b}$ consists of only positive integers. We can solve it with a tensor network similar to the presented for the systems of linear equations in Fig.~\ref{fig: Lineal Solver}. This tensor network makes the same process as in the linear solver case, but we change the post-selection nodes for $\vec{b}$. In this case, instead of using vectors with project into $b_i$ values, we use vectors with project into all values equal or less than $b_i$. This means, we change the $\delta^{b_i}$ projection vector by a step vector $s^{b_i}$ with elements
\begin{equation}
    s^{b_i}_j = 1 - H(j-b_i).
\end{equation}
This allows to take into account the restriction. To impose the maximization of the linear cost function, the superposition plus vectors are connected with a imaginary time evolution layer of matrices $S^i$ for the $i$-th variable with elements
\begin{equation}
    S^i_{j,k} = e^{\tau c_i j}\delta_{j,k}
\end{equation}
This way we can impose the restriction through the matrix multiplication tensor network, and the maximization through the imaginary time evolution layer.

\subsection{Integer Quadratic Programming}
In this case, the problem to be solved is the minimization of the cost function
\begin{equation}
    C(\vec{x}) = \sum_{i,j} Q_{ij} x_ix_j + \sum_i c_i x_i
\end{equation}
subject to the constraint
\begin{equation}\label{eq: restrict integer}
    \sum_{j} A_{ij}  x_j \leq b_i \ \forall i.
\end{equation}

In this case, we only have to connect the tensor network that solves the QUDO problem with the tensor network that imposes the constraint, so that our resulting tensor network is the one represented in Fig.~\ref{fig: IQP}.

\begin{figure}[h]
    \centering
    \includegraphics[width=0.9\linewidth]{Images/Quadratic_Programming.pdf}
    \caption{Tensor network to solve the Integer Quadratic Programming.}
    \label{fig: IQP}
\end{figure}

\subsection{Integer Polynomial Programming}
In this case, the cost function can be expressed as in \ref{eq: cost HOBO}, with the constraint \ref{eq: restrict integer}. This way, we only need to connect the TLC of the HODO problem with the TLC of the linear solver, in the same way as in Fig.~\ref{fig: IQP}.


