\section{Related Work}
\subsection{Vision Transformers}
The vision transformer design is adapted from Transformer~\cite{trnasformer}, which was designed for natural language processing (NLP). This adaptation makes it suitable for computer vision tasks such as image classification~\cite{vit, DeiT, swin}, object detection~\cite{detr, detr2}, and semantic segmentation~\cite{setr, setr2, segmenter}. 

\subsubsection{Class Token \& Global Average Pooling}
ViT~\cite{vit} conducts ablation studies comparing the class token and GAP. Additionally, there are other studies on the use of GAP and class tokens in vision transformers~\cite{gap_cls, cpvt}. Studies such as CeiT~\cite{ceit} and T2T-ViT~\cite{t2t} use class token, while others like Swin Transformer~\cite{swin} and CPVT~\cite{cpvt} adapt GAP. CPVT achieves performance improvements by using GAP instead of the class token. Although the class token is not inherently translation-invariant, it can become so through training. By adopting GAP, which is inherently translation-invariant, better improvements in image classification tasks are achieved~\cite{cpvt}. Furthermore, GAP results in even less computational complexity because it eliminates the need to compute the attention interaction between the class token and the image patches. 

\subsection{Position Embeddings in Vision Transformers}

\subsubsection{Absolute Position Embedding} In the transformer, absolute position embedding is generated through a sinusoidal function and added to the input token embedding~\cite{trnasformer}. The sinusoidal functions are designed to give the position embedding locally consistent similarity, which helps vision transformers focus more effectively on tokens that are close to each other in the input sequence. This local consistency enhances the model's ability to capture spatial relationships and patterns~\cite{trnasformer}.

Besides sinusoidal positional embedding, position embedding can also be learnable. Learnable position embedding is created through training parameters, which are initialized with a fixed-dimensional tensor and updated along with the model's parameters during training. Recently, many models have adopted absolute position embedding due to their effectiveness in encoding positional information~\cite{vit, DeiT, swin}.

\subsubsection{Relative Position Embedding} In addition to absolute position embedding, there is also relative position embedding~\cite{rpe}. Relative PE encodes the relative position information between tokens. The first to propose relative PE in computer vision was \cite{rpe}. Furthermore, \cite{2drpe} proposed a 2-D relative position encoding for image classification that showed superior performance compared to traditional 2-D sinusoidal embedding. This relative encoding captures spatial relationships between tokens more effectively. In related research, iRPE \cite{irpe} improves relative PE by incorporating query interactions and relative distance modeling in self-attention. RoPE \cite{rope} introduces flexible sequence lengths, decaying inter-token dependency, and relative position encoding in linear self-attention.

\begin{figure*}[t]
\centering
\includegraphics[width=2.0\columnwidth]{Figure/Figure3.pdf}
\caption{The overview of the various methods. (a) ViT. (b) LaPE~\cite{lape}. (c) PVG, an improved Layer-wise structure. Specifically, we adopt a structure where the token embedding and PE are added before entering layer 0 and a hierarchical structure for delivering PE, excluding layer 0. (d) MPVG. The main difference from PVG is whether the initial PE is delivered to the Last LN.} \label{Figure3}

\end{figure*}