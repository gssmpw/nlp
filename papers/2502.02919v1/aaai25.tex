%File: formatting-instructions-latex-2025.tex
%release 2025.0
\documentclass[letterpaper]{article} % DO NOT CHANGE THIS
\usepackage{aaai25}  % DO NOT CHANGE THIS
\usepackage{times}  % DO NOT CHANGE THIS
\usepackage{helvet}  % DO NOT CHANGE THIS
\usepackage{courier}  % DO NOT CHANGE THIS
\usepackage[hyphens]{url}  % DO NOT CHANGE THIS
\usepackage{graphicx} % DO NOT CHANGE THIS
\urlstyle{rm} % DO NOT CHANGE THIS
\def\UrlFont{\rm}  % DO NOT CHANGE THIS
\usepackage{natbib}  % DO NOT CHANGE THIS AND DO NOT ADD ANY OPTIONS TO IT
\usepackage{caption} % DO NOT CHANGE THIS AND DO NOT ADD ANY OPTIONS TO IT
\frenchspacing  % DO NOT CHANGE THIS
\setlength{\pdfpagewidth}{8.5in}  % DO NOT CHANGE THIS
\setlength{\pdfpageheight}{11in}  % DO NOT CHANGE THIS
%
% These are recommended to typeset algorithms but not required. See the subsubsection on algorithms. Remove them if you don't have algorithms in your paper.
\usepackage{algorithm}
\usepackage{algorithmic}
\usepackage{multirow}
\usepackage{makecell}
\usepackage{amsmath}
\usepackage{amsfonts}
\usepackage{amssymb}
\usepackage{comment}
\usepackage{bm}
\usepackage{pifont}
\usepackage{subcaption}


%
% These are are recommended to typeset listings but not required. See the subsubsection on listing. Remove this block if you don't have listings in your paper.
\usepackage{newfloat}
\usepackage{listings}
\DeclareCaptionStyle{ruled}{labelfont=normalfont,labelsep=colon,strut=off} % DO NOT CHANGE THIS
\lstset{%
	basicstyle={\footnotesize\ttfamily},% footnotesize acceptable for monospace
	numbers=left,numberstyle=\footnotesize,xleftmargin=2em,% show line numbers, remove this entire line if you don't want the numbers.
	aboveskip=0pt,belowskip=0pt,%
	showstringspaces=false,tabsize=2,breaklines=true}
\floatstyle{ruled}
\newfloat{listing}{tb}{lst}{}
\floatname{listing}{Listing}
%
% Keep the \pdfinfo as shown here. There's no need
% for you to add the /Title and /Author tags.
\pdfinfo{
/TemplateVersion (2025.1)
}

% DISALLOWED PACKAGES
% \usepackage{authblk} -- This package is specifically forbidden
% \usepackage{balance} -- This package is specifically forbidden
% \usepackage{color (if used in text)
% \usepackage{CJK} -- This package is specifically forbidden
% \usepackage{float} -- This package is specifically forbidden
% \usepackage{flushend} -- This package is specifically forbidden
% \usepackage{fontenc} -- This package is specifically forbidden
% \usepackage{fullpage} -- This package is specifically forbidden
% \usepackage{geometry} -- This package is specifically forbidden
% \usepackage{grffile} -- This package is specifically forbidden
% \usepackage{hyperref} -- This package is specifically forbidden
% \usepackage{navigator} -- This package is specifically forbidden
% (or any other package that embeds links such as navigator or hyperref)
% \indentfirst} -- This package is specifically forbidden
% \layout} -- This package is specifically forbidden
% \multicol} -- This package is specifically forbidden
% \nameref} -- This package is specifically forbidden
% \usepackage{savetrees} -- This package is specifically forbidden
% \usepackage{setspace} -- This package is specifically forbidden
% \usepackage{stfloats} -- This package is specifically forbidden
% \usepackage{tabu} -- This package is specifically forbidden
% \usepackage{titlesec} -- This package is specifically forbidden
% \usepackage{tocbibind} -- This package is specifically forbidden
% \usepackage{ulem} -- This package is specifically forbidden
% \usepackage{wrapfig} -- This package is specifically forbidden
% DISALLOWED COMMANDS
% \nocopyright -- Your paper will not be published if you use this command
% \addtolength -- This command may not be used
% \balance -- This command may not be used
% \baselinestretch -- Your paper will not be published if you use this command
% \clearpage -- No page breaks of any kind may be used for the final version of your paper
% \columnsep -- This command may not be used
% \newpage -- No page breaks of any kind may be used for the final version of your paper
% \pagebreak -- No page breaks of any kind may be used for the final version of your paperr
% \pagestyle -- This command may not be used
% \tiny -- This is not an acceptable font size.
% \vspace{- -- No negative value may be used in proximity of a caption, figure, table, section, subsection, subsubsection, or reference
% \vskip{- -- No negative value may be used to alter spacing above or below a caption, figure, table, section, subsection, subsubsection, or reference

\setcounter{secnumdepth}{0} %May be changed to 1 or 2 if section numbers are desired.

% The file aaai25.sty is the style file for AAAI Press
% proceedings, working notes, and technical reports.
%

% Title

% Your title must be in mixed case, not sentence case.
% That means all verbs (including short verbs like be, is, using,and go),
% nouns, adverbs, adjectives should be capitalized, including both words in hyphenated terms, while
% articles, conjunctions, and prepositions are lower case unless they
% directly follow a colon or long dash
\title{Maximizing the Position Embedding for Vision Transformers \\ with Global Average Pooling}
\author{
    Wonjun Lee\textsuperscript{\rm 1,\rm 2}, 
    Bumsub Ham\textsuperscript{\rm 1}, 
    Suhyun Kim\textsuperscript{\rm 2}\equalcontrib
}
\affiliations{
    %Afiliations
    \textsuperscript{\rm 1}Yonsei University, Republic of Korea \\
    \textsuperscript{\rm 2}Korea Institute of Science and Technology, Republic of Korea \\
    % email address must be in roman text type, not monospace or sans serif
    \{velpegor, bumsub.ham\}@yonsei.ac.kr, dr.suhyun.kim@gmail.com
%
% See more examples next
}

%Example, Single Author, ->> remove \iffalse,\fi and place them surrounding AAAI title to use it
\iffalse
\title{My Publication Title --- Single Author}
\author {
    Author Name
}
\affiliations{
    Affiliation\\
    Affiliation Line 2\\
    name@example.com
}
\fi

\iffalse
%Example, Multiple Authors, ->> remove \iffalse,\fi and place them surrounding AAAI title to use it
\title{My Publication Title --- Multiple Authors}
\author {
    % Authors
    First Author Name\textsuperscript{\rm 1,\rm 2},
    Second Author Name\textsuperscript{\rm 2},
    Third Author Name\textsuperscript{\rm 1}
}
\affiliations {
    % Affiliations
    \textsuperscript{\rm 1}Affiliation 1\\
    \textsuperscript{\rm 2}Affiliation 2\\
    firstAuthor@affiliation1.com, secondAuthor@affilation2.com, thirdAuthor@affiliation1.com
}
\fi


% REMOVE THIS: bibentry
% This is only needed to show inline citations in the guidelines document. You should not need it and can safely delete it.
\usepackage{bibentry}
% END REMOVE bibentry

\begin{document}

\maketitle

\begin{abstract}
In vision transformers, position embedding (PE) plays a crucial role in capturing the order of tokens. However, in vision transformer structures, there is a limitation in the expressiveness of PE due to the structure where position embedding is simply added to the token embedding. A layer-wise method that delivers PE to each layer and applies independent Layer Normalizations for token embedding and PE has been adopted to overcome this limitation. In this paper, we identify the conflicting result that occurs in a layer-wise structure when using the global average pooling (GAP) method instead of the class token. To overcome this problem, we propose MPVG, which maximizes the effectiveness of PE in a layer-wise structure with GAP. Specifically, we identify that PE counterbalances token embedding values at each layer in a layer-wise structure. Furthermore, we recognize that the counterbalancing role of PE is insufficient in the layer-wise structure, and we address this by maximizing the effectiveness of PE through MPVG. Through experiments, we demonstrate that PE performs a counterbalancing role and that maintaining this counterbalancing directionality significantly impacts vision transformers. As a result, the experimental results show that MPVG outperforms existing methods across vision transformers on various tasks.
\end{abstract}

% Uncomment the following to link to your code, datasets, an extended version or similar.
%
% \begin{links}
%     \link{Code}{https://aaai.org/example/code}
%     \link{Datasets}{https://aaai.org/example/datasets}
%     \link{Extended version}{https://aaai.org/example/extended-version}
% \end{links}

\section{Introduction}

Recently, vision transformers have become essential architecture in the field of computer vision due to their superior performance, surpassing CNNs in various tasks such as image classification, object detection, and semantic segmentation. This superiority has led to extensive research into numerous elements of vision transformer architecture, starting with ViT \cite{vit}.

Among the research on vision transformers, image representation methods for class prediction have been studied. In ViT, the class token is used to perform image representation, and the output of this token is then used to make class predictions via Multi-Layer Perceptron (MLP)~\cite {vit}. However, in several vision transformers, global average pooling (GAP) has been preferred over the class token method due to its translation-invariant characteristics and superior performance~\cite{cpvt}. As a result, the GAP method has been widely adopted in vision transformers~\cite{swin, twins, STViT, scaling}.

\begin{figure}[t]
\centering
\includegraphics[width=1.0\columnwidth]{Figure/Figure1.pdf}
\caption{
The conflicting result between the GAP method and the Layer-wise method. In DeiT-Ti, using the GAP method and the Layer-wise method separately results in performance improvements, but combining these two methods leads to a decrease in performance. As a result, MPVG resolves this phenomenon between the GAP and Layer-wise structure, maximizing the effect of PE.} \label{Figure1}

\end{figure}

Another research topic in vision transformers is position embedding (PE). PE plays a crucial role in providing positional information of tokens in the vision transformer, as the self-attention mechanism has an inherent deficiency in capturing the ordering of input tokens \cite{irpe, permutation}. In the original vision transformer, the expressiveness of the PE is limited due to its structure, where PE is simply added to the token embedding before being input into the first layer. To address this problem, each layer has independent Layer Normalizations (LNs) for the token embedding and PE, with PE being gradually delivered across all the layers~\cite{lape}. We refer to this method as "Layer-wise". The Layer-wise structure resolves the existing limitations and enhances the expressiveness of PE.

Interestingly, as shown in Fig~\ref{Figure1}, we observed results that differed from our expectations between the class token and GAP methods with PE delivered in the Layer-wise structure. On image classification, the GAP method demonstrates superior performance compared to the class token method~\cite{cpvt}. Additionally, the Layer-wise structure also improves the performance of vision transformers by enhancing the expressiveness of PE~\cite{lape}. However, we observed a conflicting result where performance decreased when the GAP and Layer-wise structure were applied together. Therefore, to overcome the conflicting results, we propose a method to maximize the effectiveness of PE in the GAP approach.

We observe that PE exhibits distinct characteristics at each layer in the Layer-wise structure, which are not seen in the original vision transformer. As shown in Fig~\ref{Figure2}, we find that PE tends to counterbalance the values of token embedding passing through the layers in the Layer-wise structure. Additionally, we observe that this tendency becomes more pronounced as the layers deepen. We also discover that in the Layer-wise structure, while the token embedding values maintain the counterbalanced effect by PE as they progress through the layers, as shown in Fig~\ref{Figure2}-(b), the directional balance of the token embedding is not adequately compensated by PE after passing through the last layer, even though it is still maintained. Through this observation, we establish two hypotheses: (1) in the Layer-wise structure, PE initially provides position information, but as the layers deepen, PE plays a role in counterbalancing the values of token embedding; (2) after the last layer, it is beneficial for vision transformers to maintain the directional balance by counterbalancing the token embedding values with PE.

To validate these hypotheses, we simply add PE to the Layer Normalization (LN) that exists outside the layers and before the MLP head. We call this LN as "Last LN". We refer to the method that uses an improved Layer-wise structure, different from the conventional Layer-wise structure, and does not deliver PE to the Last LN as PVG. Additionally, we refer to the method that maximizes the role of PE by delivering it to the Last LN as MPVG. By comparing PVG and MPVG, we demonstrate that MPVG effectively maximizes PE and that maintaining the counterbalancing directionality of PE is beneficial for vision transformers. Our experiments validate our hypothesis and demonstrate that MPVG outperforms other methods. The results demonstrate that MPVG consistently performs well for vision transformers.



In this paper, our contributions are as follows:

\begin{enumerate}
    \item We propose a simple yet effective method called MPVG, which maximizes the effect of PE in the GAP method. We show that MPVG leads to better performance in vision transformers.
    
    \item We provide an analysis of the phenomenon observed in PE when using the Layer-wise structure and offer insights into the role of PE.
    
    \item Through experiments, we verify that MPVG is generally effective for vision transformers on various tasks such as image classification, semantic segmentation, and object detection.
\end{enumerate}

\begin{figure*}[ht!]
\centering
\includegraphics[width=2.0\columnwidth]{Figure/Figure2.pdf}
\caption{The heatmaps depict the characteristics of each layer in both the original structure and the Layer-wise structure with the GAP method. For the Layer-wise structure, the heatmaps illustrate cases both with and without PE in the Last LN. For each heatmap based on DeiT-Ti, the x-axis represents the dimension of DeiT-Ti (256), and the y-axis represents the number of tokens (196). In both (a) and the top row (token embedding) of (b), the heatmaps represent the average value of token embedding in each layer, while the bottom row of (b) shows the heatmap of PE. The correlation in (b) refers to the correlation coefficient between token embedding and position embedding.}
\label{Figure2}
\end{figure*}

\section{Related Work}

\subsection{Vision Transformers}
The vision transformer design is adapted from Transformer~\cite{trnasformer}, which was designed for natural language processing (NLP). This adaptation makes it suitable for computer vision tasks such as image classification~\cite{vit, DeiT, swin}, object detection~\cite{detr, detr2}, and semantic segmentation~\cite{setr, setr2, segmenter}. 

\subsubsection{Class Token \& Global Average Pooling}
ViT~\cite{vit} conducts ablation studies comparing the class token and GAP. Additionally, there are other studies on the use of GAP and class tokens in vision transformers~\cite{gap_cls, cpvt}. Studies such as CeiT~\cite{ceit} and T2T-ViT~\cite{t2t} use class token, while others like Swin Transformer~\cite{swin} and CPVT~\cite{cpvt} adapt GAP. CPVT achieves performance improvements by using GAP instead of the class token. Although the class token is not inherently translation-invariant, it can become so through training. By adopting GAP, which is inherently translation-invariant, better improvements in image classification tasks are achieved~\cite{cpvt}. Furthermore, GAP results in even less computational complexity because it eliminates the need to compute the attention interaction between the class token and the image patches. 

\subsection{Position Embeddings in Vision Transformers}

\subsubsection{Absolute Position Embedding} In the transformer, absolute position embedding is generated through a sinusoidal function and added to the input token embedding~\cite{trnasformer}. The sinusoidal functions are designed to give the position embedding locally consistent similarity, which helps vision transformers focus more effectively on tokens that are close to each other in the input sequence. This local consistency enhances the model's ability to capture spatial relationships and patterns~\cite{trnasformer}.

Besides sinusoidal positional embedding, position embedding can also be learnable. Learnable position embedding is created through training parameters, which are initialized with a fixed-dimensional tensor and updated along with the model's parameters during training. Recently, many models have adopted absolute position embedding due to their effectiveness in encoding positional information~\cite{vit, DeiT, swin}.

\subsubsection{Relative Position Embedding} In addition to absolute position embedding, there is also relative position embedding~\cite{rpe}. Relative PE encodes the relative position information between tokens. The first to propose relative PE in computer vision was \cite{rpe}. Furthermore, \cite{2drpe} proposed a 2-D relative position encoding for image classification that showed superior performance compared to traditional 2-D sinusoidal embedding. This relative encoding captures spatial relationships between tokens more effectively. In related research, iRPE \cite{irpe} improves relative PE by incorporating query interactions and relative distance modeling in self-attention. RoPE \cite{rope} introduces flexible sequence lengths, decaying inter-token dependency, and relative position encoding in linear self-attention.

\begin{figure*}[t]
\centering
\includegraphics[width=2.0\columnwidth]{Figure/Figure3.pdf}
\caption{The overview of the various methods. (a) ViT. (b) LaPE~\cite{lape}. (c) PVG, an improved Layer-wise structure. Specifically, we adopt a structure where the token embedding and PE are added before entering layer 0 and a hierarchical structure for delivering PE, excluding layer 0. (d) MPVG. The main difference from PVG is whether the initial PE is delivered to the Last LN.} \label{Figure3}

\end{figure*}


\section{Methodology}
In this section, we first explain the absolute position embedding and then provide a detailed overview of the Layer-wise structure~\cite{lape}. Next, we introduce PVG, an improved Layer-wise structure, along with MPVG, which effectively leverages the characteristics of PE in the Layer-wise structure.

\subsection{Preliminary: Absolute Position Embedding}
The method of absolute position embedding used in vision transformers is as follows. As shown in Fig~\ref{Figure3}-(a), PE is added to the token embedding before they are input into the layer. This can be expressed as follows:
\begin{equation}
x_0 = [x_{\text{cls}}; \; p^1; \;p^2; \; \ldots \; p^N;] + pos, \tag{1}
\label{equation1}
\end{equation} 

\noindent where $\mathit{p}$ and $\mathit{pos}$ represent the patch and position embedding, respectively. \(N\) represents the number of patches, calculated as \(HW / P^2\), where \(H\) and \(W\) are the height and width of the image, and \(P \times P\) is the resolution of each patch. The combined token embedding and PE, denoted as $\mathit{x}$, can be expressed in a layer as follows:

\begin{equation}
x'_l = \text{MSA}(\text{LN}_{l}(x_{l})) + x_{l} \quad (l = 0 \ldots L), \tag{2}
\label{equation2}
\end{equation}

\begin{equation}
x_{l+1} = \text{MLP}(\text{LN}^{'}_{l}(x'_l)) + x'_l \quad (l = 0 \ldots L), \tag{3}
\label{equation3}
\end{equation}

\begin{equation}
y = \text{LN}(x_{L+1}) \tag{4}
\label{equation4}
\end{equation}

\noindent where LN, LN', and LN'' represent different Layer Normalizations, Multi-head Self-Attention is denoted as MSA, and Multi-Layer Perceptron is denoted as MLP. $x_{L+1}$ refers to the value after passing through the last layer $L$.

\subsection{Preliminary: Layer-wise Structure}


LaPE~\cite{lape} points out problems with the joining method that position embedding and token embedding in the vision transformers. As shown in Eq.~(\ref{equation1}), when PE is added to the token embedding before the first layer, and the same LN is applied to both the token embedding and PE as in Eq.~(\ref{equation2}), they share the same affine parameters in LN. This method limits the expressiveness of PE. Therefore, the Layer-wise structure is used to resolve these problems. This can be expressed as follows:

\begin{equation}
x_0 = [x_{\text{cls}}; \; p^1; \;p^2; \; \ldots \; p^N;], \tag{5}
\label{equation5}
\end{equation} 

\begin{equation}
x'_l = \text{MSA}(\text{LN}_{l}(x_{l}) + \text{LN}^{'}_{l}(pos_{l})) + x_{l} \tag{6}
\label{equation6}
\end{equation}

\noindent Eq.~(\ref{equation1}) is modified to Eq.~(\ref{equation5}) and Eq.~(\ref{equation2}) to Eq.~(\ref{equation6}). In Eq.~(\ref{equation6}), the Layer-wise structure uses independent LN for token embedding($\mathit{x}$) and PE. PE is delivered in each layer as follows :

\begin{equation} 
\left\{
\begin{aligned}
    pos_{0} &= pos \\
    pos_{l} &= \text{LN}^{'}_{l-1}(pos_{l-1}) \quad (l = 1 \ldots L)
\end{aligned}
\right.
\tag{7}
\label{equation7}
\end{equation}



\subsection{Maximizing the Position Embedding with GAP}

In this section, we propose two methods, MPVG and PVG, to validate our hypothesis. In Fig \ref{Figure2}-(b), we observed that, in Layer-wise structure, the effect of PE in counterbalancing the values of token embedding($\mathit{x}$) becomes more pronounced as the layers deepen, as evidenced by the correlation between the two. However, in Layer-wise structure, although the directionality of the token embedding is maintained outside the layer, there is no PE to counterbalance that value. Therefore, we validate our hypothesis by comparing MPVG, which delivers PE to the Last LN, with PVG, which does not.

We remove the class token as we adapt the Global Average Pooling (GAP) method. Although we use the Layer-wise structure, we modify specific details. Specifically, we combine two structural approaches: (1) adding token embedding and PE before inputting the layer. (2) delivering PE to each layer except the 0th layer. We call this method as PVG. In PVG method, as shown in Figure~\ref{Figure3}-(c), is as follows:

\begin{equation}
x_0 = [p^1; \;p^2; \; \ldots \; p^N;] + pos, \tag{8}
\label{equation8}
\end{equation} 

\begin{equation}
x_l' = 
\begin{cases} 
\text{MSA}(\text{LN}_{0}(x_0)) + x_0 & \text{if } l = 0 \\ 
\text{MSA}(\text{LN}_{l}(x_{l}) + \text{LN}'_{l}(pos_{l-1})) + x_{l} & \text{if } 1 \leq l \leq L \tag{9}
\end{cases}
\label{equation9}
\end{equation}

\begin{equation} 
\left\{
\begin{aligned}
    pos_{0} &= pos \\
    pos_{l} &= \text{LN}^{'}_{l}(pos_{l-1}) \quad (l = 1 \ldots L)
\end{aligned}
\right.
\tag{10}
\label{equation10}
\end{equation}


\noindent The subsequent process is the same as in Eq.~(\ref{equation3}) and Eq.~(\ref{equation4}). In MPVG, we modify Eq.~(\ref{equation4}) as follows after going through the process of PVG:

\begin{equation}
y = \text{LN}(x_{L+1}) + \text{LN}^{'}(pos_0) \tag{11}
\label{equation11}
\end{equation}

To verify whether maintaining the counterbalance effect of PE is beneficial, we deliver PE to the Last LN in PVG, as shown in Eq.~(\ref{equation11}). We refer to this method as MPVG. In the next section, we verify the superiority of MPVG by comparing the two methods. Also, we show that MPVG outperforms previous approaches through experiments across various vision transformers and datasets.


\section{Experiment}


\subsubsection{Training Settings} All experiments are conducted on an RTX 4090 with 4 GPUs using AdamW optimizer~\cite{adamw}, while DeiT-B is trained on an RTX 4090 with 8 GPUs. 

\subsection{Image Classification}

We evaluate the performance of our methods on ImageNet-1K~\cite{imagenet} and CIFAR-100~\cite{cifar100}. On ImageNet-1K, we conduct experiments with DeiT~\cite{DeiT}, Swin~\cite{swin}, CeiT~\cite{ceit}, and T2T-ViT~\cite{t2t}. In the case of Swin, due to its staged architecture that generates hierarchical representations with the same feature map resolution as convolutional networks, both PVG and MPVG exceptionally include layer 0. All vision transformers are trained on 224×224 resolution images for 300 epochs, except T2T-ViT-7, which is trained for 310 epochs. 

On CIFAR-100, we conduct experiments using ViT-Lite~\cite{cct} and T2T-ViT-7~\cite{t2t}. ViT-Lite was trained for 310 epochs on 32×32 resolution images with a batch size of 128. In the case of T2T-ViT-7, we transfer our pretrained T2T-ViT to downstream datasets such as CIFAR-100 and finetune the pretrained T2T-ViT-7 for 60 epochs with a batch size of 128.
\begin{table}[t]
\renewcommand\arraystretch{1.2}
\resizebox{\columnwidth}{!}{

\begin{tabular}{cccc}
\hline 
Model                                                                                                             & Method        & \#Params (M) & \begin{tabular}[c]{@{}c@{}}Top-1 \\ Acc (\%)\end{tabular} \\ \hline \hline
\multicolumn{1}{c|}{\multirow{4}{*}{\begin{tabular}[c]{@{}c@{}}DeiT-Ti\\ \cite{DeiT}\end{tabular}}}  & Default       & 5.717        & 72.14                                                     \\
\multicolumn{1}{c|}{}                                                                                             & LaPE          & 5.721        & 72.94                                                     \\
\multicolumn{1}{c|}{}                                                                                             & PVG           & 5.721        & 73.17                                                     \\
\multicolumn{1}{c|}{}                                                                                             & \textbf{MPVG} & 5.721        & \textbf{73.51}                                            \\ \hline
\multicolumn{1}{c|}{\multirow{4}{*}{\begin{tabular}[c]{@{}c@{}}DeiT-S\\ \cite{DeiT}\end{tabular}}}   & Default       & 22.050       & 79.81                                                     \\
\multicolumn{1}{c|}{}                                                                                             & LaPE          & 22.059       & 80.39                                                     \\
\multicolumn{1}{c|}{}                                                                                             & PVG           & 22.058       & 80.38                                                     \\
\multicolumn{1}{c|}{}                                                                                             & \textbf{MPVG} & 22.059       & \textbf{80.61}                                            \\ \hline
\multicolumn{1}{c|}{\multirow{4}{*}{\begin{tabular}[c]{@{}c@{}}DeiT-B\\ \cite{DeiT}\end{tabular}}}   & Default       & 86.567       & 81.85                                                     \\
\multicolumn{1}{c|}{}                                                                                             & LaPE          & 86.586       &  82.15                                                         \\
\multicolumn{1}{c|}{}                                                                                             & PVG           & 86.583             &  82.21                                                         \\
\multicolumn{1}{c|}{}                                                                                             & \textbf{MPVG} &  86.584            &   \textbf{82.42}                                                        \\ \hline
\multicolumn{1}{c|}{\multirow{4}{*}{\begin{tabular}[c]{@{}c@{}}Swin-Ti\\ \cite{swin}\end{tabular}}}  & Default       & 28.589       & 81.37                                                     \\
\multicolumn{1}{c|}{}                                                                                             & LaPE          & 28.599       & 81.48                                                     \\
\multicolumn{1}{c|}{}                                                                                             & PVG           & 28.598       & 81.52                                                     \\
\multicolumn{1}{c|}{}                                                                                             & \textbf{MPVG} & 28.599       & \textbf{81.64}                                            \\ \hline
\multicolumn{1}{c|}{\multirow{4}{*}{\begin{tabular}[c]{@{}c@{}}CeiT-Ti\\ \cite{ceit}\end{tabular}}}  & Default       & 6.356        & 76.62                                                     \\
\multicolumn{1}{c|}{}                                                                                             & LaPE          & 6.361        & 76.89                                                     \\
\multicolumn{1}{c|}{}                                                                                             & PVG           & 6.361        & 77.14                                                     \\
\multicolumn{1}{c|}{}                                                                                             & \textbf{MPVG} & 6.361        & \textbf{77.20}                                            \\ \hline
\multicolumn{1}{c|}{\multirow{4}{*}{\begin{tabular}[c]{@{}c@{}}T2T-ViT-7\\ \cite{t2t}\end{tabular}}} & Default       & 4.310        & 71.76                                                     \\
\multicolumn{1}{c|}{}                                                                                             & LaPE          & 4.313        & 72.01                                                     \\
\multicolumn{1}{c|}{}                                                                                             & PVG           & 4.312        & 71.91                                                     \\
\multicolumn{1}{c|}{}                                                                                             & \textbf{MPVG} & 4.313        & \textbf{72.28}                                            \\ \hline
\end{tabular}
}
\caption{Top-1 accuracy comparison with various methods, using DeiT-T, DeiT-S, DeiT-B, Swin-Ti, CeiT-Ti, T2T-ViT-7 on ImageNet-1K.}
\label{table1}
\end{table}

\begin{table}[t!]
\resizebox{\columnwidth}{!}{
\begin{tabular}{cccc}
\hline 
Model                                                                                                             & \begin{tabular}[c]{@{}c@{}}Method\end{tabular} & \#Param (M) & \begin{tabular}[c]{@{}c@{}}Top-1\\ Acc (\%)\end{tabular} \\ \hline \hline
\multicolumn{1}{c|}{\multirow{4}{*}{\begin{tabular}[c]{@{}c@{}}ViT-Lite\\ \cite{cct}\end{tabular}}} & Default                                             & 3.740       & 74.90                                                    \\
\multicolumn{1}{c|}{}                                                                                             & LaPE                                                & 3.744       & 75.52                                                    \\
\multicolumn{1}{c|}{}                                                                                             & PVG                                                 & 3.742       & 76.67                                                    \\
\multicolumn{1}{c|}{}                                                                                             & \textbf{MPVG}                                       & 3.743       & \textbf{76.87}                                           \\ \hline
\multicolumn{1}{c|}{\multirow{4}{*}{\begin{tabular}[c]{@{}c@{}}T2T-ViT-7\\ \cite{t2t}\end{tabular}}} & Default                                             &  4.078    &   83.22                                              \\
\multicolumn{1}{c|}{}                                                                                             & LaPE                                                &  4.082     &    83.41                                               \\
\multicolumn{1}{c|}{}                                                                                             & PVG                                                 &  4.081    &   83.39                                                  \\
\multicolumn{1}{c|}{}                                                                                             & \textbf{MPVG}                                       &  4.081     & \textbf{83.51}                                           \\ \hline
\end{tabular}
}
\caption{Top-1 accuracy comparison with various methods, using ViT-Lite and T2T-ViT-7 on CIFAR-100. In the case of T2T-ViT, the results are based on fine-tuning the pretrained model on the downstream dataset, CIFAR-100.}
\label{table2}
\end{table}

As shown in Table~\ref{table1}, For MPVG, the performance on DeiT-Ti improved from 72.14\% to 73.51\%, representing an increase of approximately 1.37\%. For DeiT-S, the performance improved from 79.81\% to 80.61\%, an increase of approximately 0.80\%. Additionally, there were performance improvements of 0.57\% in DeiT-B, 0.27\% in Swin-Ti, 0.58\% in CeiT, and 0.52\% in T2T-ViT. Overall, MPVG outperforms the existing methods in all cases. Moreover, we confirm that MPVG consistently demonstrates superior performance compared to PVG across various vision transformers. 


As shown in Table~\ref{table2}, MVPG achieves overall performance improvements on CIFAR-100. Specifically, MPVG improves the performance of ViT-Lite by 1.97\%, from 74.90\% to 76.87\%, and enhances the performance of T2T-ViT-7 by 0.29\% over the default. Additionally, MPVG shows a 0.2\% and 0.12\% improvement over PVG for ViT-Lite and T2T-ViT-7, respectively. Overall, MPVG outperforms existing methods across all cases on CIFAR-100.

\subsection{Object Detection}
On object detection, we evaluate our methods on COCO 2017~\cite{coco}. To demonstrate the effectiveness of our method on object detection tasks, we select the ViT-Adapter-Ti~\cite{vitadapter} model based on Mask R-CNN~\cite{mask-rcnn} in MMDetection framework~\cite{mmdet}. Additionally, we use the default settings and train it for 36 epochs using the 3x+MS(multi-scale training) schedule. As shown in Table~\ref{tabel3}, MPVG achieves improvements of +0.6 in box AP and +0.5 in mask AP compared to the default setting. MPVG, in particular, demonstrates superior performance with an increase of +0.5 in box AP and +0.4 in mask AP over PVG.

\subsection{Semantic Segmentation}
On semantic segmentation, we evaluate our methods on ADE20K~\cite{ade20k}. We select the ViT-Adapter-Ti~\cite{vitadapter} model based on UperNet~\cite{upernet} in MMsegmentation framework~\cite{mmseg} and train it using the default settings. As shown in Table~\ref{table4}, MPVG achieves an improvement of +1.14 mIoU compared to the default. Furthermore, MPVG outperforms PVG, achieving a performance improvement of +0.62 mIoU.

\begin{table}[t]
\resizebox{\columnwidth}{!}{
\begin{tabular}{cccc}
\hline
Model                                                & Pre-trained                                      & Method & AP$^{\text{box}}$ / AP$^{\text{mask}}$ \\ \hline\hline
\multicolumn{1}{c|}{\multirow{4}{*}{ViT-Adapter-Ti}} & \multicolumn{1}{c|}{\multirow{4}{*}{DeiT-Ti}} & \multicolumn{1}{c}{Default}   & 45.9 / 41.0                       \\
\multicolumn{1}{c|}{}                                & \multicolumn{1}{c|}{}                         & \multicolumn{1}{c}{LaPE}      & 46.2 / 41.2                                \\ 
\multicolumn{1}{c|}{}                                & \multicolumn{1}{c|}{}                         & \multicolumn{1}{c}{PVG}       & 46.1 / 41.2                                  \\
\multicolumn{1}{c|}{}                                & \multicolumn{1}{c|}{}                         & \multicolumn{1}{c}{\textbf{MPVG}}      &  \textbf{46.5} / \textbf{41.4}                                 \\ \hline
\end{tabular}
}
\caption{Performance comparison of Object Detection on COCO2017. For comparison, DeiT-Ti model pretrained on ImageNet-1K with each method is used.}
\label{tabel3}
\end{table}


\begin{table}[t]
\centering{
\begin{tabular}{cccc}
\hline
Model                                                & Pre-trained                                      & Method & mIoU \\ \hline\hline
\multicolumn{1}{c|}{\multirow{4}{*}{ViT-Adapter-Ti}} & \multicolumn{1}{c|}{\multirow{4}{*}{DeiT-Ti}} & \multicolumn{1}{c}{Default}   & 40.55 \\
\multicolumn{1}{c|}{}                                & \multicolumn{1}{c|}{}                         & \multicolumn{1}{c}{LaPE}      & 41.42 \\ 
\multicolumn{1}{c|}{}                                & \multicolumn{1}{c|}{}                         & \multicolumn{1}{c}{PVG}       & 41.07 \\
\multicolumn{1}{c|}{}                                & \multicolumn{1}{c|}{}                         & \multicolumn{1}{c}{\textbf{MPVG}}      & \textbf{41.69} \\ \hline
\end{tabular}
}
\caption{Performance comparison of Semantic Segmentation on ADE20K. For comparison, DeiT-Ti model pretrained on ImageNet-1K with each method is used.}
\label{table4}
\end{table}

\begin{figure}[t!]
\centering
\includegraphics[width=1.0\columnwidth]{Figure/Figure4.pdf}
\caption{Correlation coefficient between token embedding and position embedding in Layer-wise. Each token embedding and position embedding is based on the values after applying LN. DeiT-Ti, DeiT-S, and CeiT-Ti each have a total of 12 layers, but T2T-ViT-7 has 7 layers.
 } \label{Figure4}

\end{figure}

\begin{figure}[t]
\centering
\includegraphics[width=1.0\columnwidth]{Figure/Figure5.pdf}
\caption{Comparison of two methods on DeiT-Ti. (a) Structure with only GAP applied, showing 72.40\% performance; and (b) Structure with GAP and position embedding added to the Last LN in a non-Layer-wise structure, also showing 72.14\% performance. 
 } \label{Figure5}
%그림에 숫자 좀 더 키울 것
\end{figure}

\subsection{Analysis}

Through experiments on image classification, object detection, and semantic segmentation, we demonstrate the effectiveness of MPVG. In all tasks, MPVG not only outperforms the baseline but also achieves the best performance among all methods. This validates our hypothesis and proves that our method is an effective approach to maximizing PE in the GAP method. Fig~\ref{Figure4} shows that in Layer-wise structure, token embedding and position embedding exhibit increasingly opposing directions as the layers deepen. This suggests that PE not only provides positional information in the initial layers but also may play a counterbalancing role that becomes more pronounced in deeper layers. To further explore this, we compare PVG and MPVG to confirm that PE has a counterbalancing effect. This comparison proves that maintaining the counterbalancing role of PE impacts the performance of vision transformers.

In conclusion, several key points can be identified: (1) In the initial layers, PE primarily provides positional information, enabling the model to understand the spatial relationships between tokens. However, as the layers deepen, PE plays a role in counterbalancing the token embedding. (2) This counterbalancing effect of PE has a significant impact on the performance of vision transformers. Therefore, MPVG demonstrates that maintaining this direction is beneficial for vision transformers and proves that PE can perform additional roles to sustain this effect.

\begin{table}[t!]
\resizebox{\columnwidth}{!}{
\begin{tabular}{cccc}
\hline 
Model                                         & PE Method                                  & Last LN                              & Top-1 Accuracy (\%) \\ \hline\hline
\multicolumn{1}{c|}{\multirow{4}{*}{DeiT-Ti}} & \multicolumn{1}{c|}{\multirow{4}{*}{MPVG}} & \multicolumn{1}{c|}{$pos_{11}$} & 73.30      \\
\multicolumn{1}{c|}{}                         & \multicolumn{1}{c|}{}                      & \multicolumn{1}{c|}{$pos_{8}$}          &       73.38              \\
\multicolumn{1}{c|}{}                         & \multicolumn{1}{c|}{}                      & \multicolumn{1}{c|}{$pos_{5}$}          &     73.39                \\
\multicolumn{1}{c|}{}                         & \multicolumn{1}{c|}{}                      & \multicolumn{1}{c|}{\bm{$pos_{0}$}}         & \textbf{73.51}              \\ \hline
\end{tabular}
}
\caption{Comparison of the value of PE added to the Last LN in MPVG. $pos_{0}$ refers to the initial position embedding, and $pos_{11}$ represents the position embedding after applying LN in the last layer. $pos_{N}$ (=$\text{LN}^{'}_{N}(pos_{N-1})$) indicates the PE input for the $\mathit{(N+1)}$th layer.}
\label{table5}
\end{table}

\subsection{Effect of PE in Last LN}
We conduct additional experiments to validate our hypothesis. Specifically, we aim to confirm that adding PE to the Last LN effectively maintains the counterbalancing role of PE in Layer-wise structure. We compare the method using only GAP with the method that adds PE to the Last LN in a non-Layer-wise structure while using GAP. We perform these experiments with DeiT-Ti on ImageNet-1K.

In Fig 5, we compare (a), where only GAP is applied, with (b), where PE is delivered to the Last LN in a non-Layer-wise structure with GAP. Fig~\ref{Figure5}-(a) shows a 72.40\% performance, while Fig~\ref{Figure5}-(b) shows a decreased performance of 72.14\%. This indicates that adding PE to the Last LN is only effective in Layer-wise structure where PE is delivered to each layer. In other words, these experimental results prove that in Layer-wise structure, the token embedding contains values that are counterbalanced by PE. Specifically, in Fig 5-(b) structure, the token embedding that passes through the layers does not contain the directional values, which is counterbalanced by PE. Thus, adding PE to the Last LN not only has no effect but actually leads to a decrease in performance. As a result, as shown in Fig 2-(b), in Layer-wise structure, the token embedding progresses while retaining values that are meant to be counterbalanced by PE, but after passing through the final layer, this directional value is not adequately compensated by PE. This proves that PE is necessary to perform this additional counterbalancing role in the Last LN.

\subsection{Ablation Study}

\subsubsection{The impact of PE values delivered to the Last LN} We conduct experiments to investigate the impact of varying the PE values passed to the Last LN in MPVG. In Table~\ref{table5}, \(\mathit{pos}_{N}\) represents the value of \(\text{LN}^{'}_{N}(\mathit{pos}_{N-1})\). Since MPVG does not deliver PE to layer 0, \(N\) ranges from 1 to \(L-1\), where \(L\) is the number of layers. Experiments show that MPVG consistently outperforms PVG, which achieves a performance of 73.17\%, regardless of the PE values passed to the Last LN. This suggests that delivering PE in the Last LN has a significant positive impact on the performance of vision transformers. Furthermore, it demonstrates that maintaining the role of PE in the Last LN is generally effective. MPVG adopts \(\mathit{pos}_{0}\), which shows the best performance by comparing various PE values delivered to the Last LN.

\begin{table}[t]
\renewcommand\arraystretch{1.0}
\resizebox{\columnwidth}{!}{
\begin{tabular}{ccccc}
\hline 
\multirow{2}{*}{Model}                     & \multicolumn{3}{c}{Structure} & \multirow{2}{*}{\begin{tabular}[c]{@{}c@{}}Top-1\\ Acc (\%)\end{tabular}} \\
                                           & Layer 0  & Hierarchical  & ($\mathit{x}$+PE) &                                                                           \\ \hline\hline
\multicolumn{1}{c|}{\multirow{4}{*}{MPVG}} & \ding{55}       & \checkmark             & \checkmark       &    73.31                                                                       \\
\multicolumn{1}{c|}{}                      & \checkmark       & \ding{55}             & \checkmark       &       73.48                                                                    \\
\multicolumn{1}{c|}{}                      & \checkmark       & \checkmark             & \ding{55}       &       73.28                                                                    \\
\multicolumn{1}{c|}{}                      & \checkmark       & \checkmark             & \checkmark       & \textbf{73.51}                                                            \\ \hline
\end{tabular}
}
\caption{Structural Differences in MPVG. "Layer 0" denotes whether layer 0 is included when delivering PE to layers. ``Hierarchical'' denotes whether \(\mathit{pos}_l\) is \(\text{LN}^{'}_{l}(\mathit{pos}_{l-1})\) or \(\text{LN}^{'}_{l}(\mathit{pos}_{0})\).
 "($\mathit{x}$+PE)" denotes whether PE is added to the token embedding($\mathit{x}$) before entering layer 0 or not.}
\label{table6}
\end{table}


\subsubsection{Structural differences in MPVG} We also experiment by varying the architecture structure in MPVG. Table 6 presents the ablations for the differences in architecture within MPVG. The experiments are conducted on DeiT-Ti using the ImageNet-1K. Through this experiment, we adopt an improved Layer-wise structure that differs from the conventional Layer-wise structure.

Specifically, we conduct comparative experiments on three structural differences: (1) Our methods exclude layer 0 when delivering PE. Through our experiments, we find that delivering PE to layer 0, which was previously included, is not only unnecessary but also improves the performance of vision transformers when excluded. (2) We add PE to the token embedding before it enters layer 0. Unlike LaPE where token embedding $\mathit{x}$ and PE are separated before entering the first layer, our methods add PE to $\mathit{x}$ before entering layer 0. This structure does not limit the expressiveness of PE because independent LN is applied to both the token embedding and PE in each layer, and PE is delivered in a Layer-wise structure. Moreover, the ($\mathit{x}$+PE) structure boosts performance by approximately 0.23\%. (3) We observe that the performance is similar between hierarchical and non-hierarchical structures. However, in non-hierarchical structures, performance often declines in small or large vision transformers due to overfitting~\cite{lape}. Through Table 6, we demonstrate that our methods represent the optimal structure in the Layer-wise structure.

\section{Conclusion}
We reveal that position embedding can play additional roles in vision transformers using the GAP method. Specifically, in a Layer-wise structure, PE has a counterbalancing effect on the values of token embedding, and maintaining this directional balance by PE is beneficial for vision transformers. Based on these observations, we propose a simple yet effective method, MPVG. MPVG utilizes the characteristics of PE observed in the Layer-wise structure to maximize the PE. Through extensive experiments, we demonstrate that MPVG is generally effective on vision transformers, outperforming previous methods. However, MPVG has a potential limitation in that it is incompatible with the class token method. Through these limitations, we will further explore the broader applicability of MPVG and the effects of PE's counterbalancing as part of our future work. In this paper, we demonstrate that MPVG effectively addresses the issues arising in GAP and Layer-wise structures, providing a significantly more meaningful approach. Through this, we look forward to MPVG offering a broader perspective on position embedding.

\section{Acknowledgments}
This research was supported by the MSIT(Ministry of Science and ICT), Korea, under the ITRC(Information Technology Research Center) support program(IITP-2024-RS-2023-00258649, 80\%) supervised by the IITP(Institute for Information \& Communications Technology Planning \& Evaluation) and was partly supported by the IITP grant funded by the Korea government (MSIT) (No.RS-2022-00143524, Development of Fundamental Technology and Integrated Solution for Next-Generation Automatic Artificial Intelligence System) and (No.RS2023-00225630, Development of Artificial Intelligence for Text-based 3D Movie Generation).


\bibliography{aaai25}

%% bare_jrnl_compsoc.tex
%% V1.4b
%% 2015/08/26
%% by Michael Shell
%% See:
%% http://www.michaelshell.org/
%% for current contact information.
%%
%% This is a skeleton file demonstrating the use of IEEEtran.cls
%% (requires IEEEtran.cls version 1.8b or later) with an IEEE
%% Computer Society journal paper.
%%
%% Support sites:
%% http://www.michaelshell.org/tex/ieeetran/
%% http://www.ctan.org/pkg/ieeetran
%% and
%% http://www.ieee.org/

%%*************************************************************************
%% Legal Notice:
%% This code is offered as-is without any warranty either expressed or
%% implied; without even the implied warranty of MERCHANTABILITY or
%% FITNESS FOR A PARTICULAR PURPOSE! 
%% User assumes all risk.
%% In no event shall the IEEE or any contributor to this code be liable for
%% any damages or losses, including, but not limited to, incidental,
%% consequential, or any other damages, resulting from the use or misuse
%% of any information contained here.
%%
%% All comments are the opinions of their respective authors and are not
%% necessarily endorsed by the IEEE.
%%
%% This work is distributed under the LaTeX Project Public License (LPPL)
%% ( http://www.latex-project.org/ ) version 1.3, and may be freely used,
%% distributed and modified. A copy of the LPPL, version 1.3, is included
%% in the base LaTeX documentation of all distributions of LaTeX released
%% 2003/12/01 or later.
%% Retain all contribution notices and credits.
%% ** Modified files should be clearly indicated as such, including  **
%% ** renaming them and changing author support contact information. **
%%*************************************************************************


% *** Authors should verify (and, if needed, correct) their LaTeX system  ***
% *** with the testflow diagnostic prior to trusting their LaTeX platform ***
% *** with production work. The IEEE's font choices and paper sizes can   ***
% *** trigger bugs that do not appear when using other class files.       ***                          ***
% The testflow support page is at:
% http://www.michaelshell.org/tex/testflow/


\documentclass[10pt,journal,compsoc]{IEEEtran}
%
% If IEEEtran.cls has not been installed into the LaTeX system files,
% manually specify the path to it like:
% \documentclass[10pt,journal,compsoc]{../sty/IEEEtran}





% Some very useful LaTeX packages include:
% (uncomment the ones you want to load)


% *** MISC UTILITY PACKAGES ***
%
%\usepackage{ifpdf}
% Heiko Oberdiek's ifpdf.sty is very useful if you need conditional
% compilation based on whether the output is pdf or dvi.
% usage:
% \ifpdf
%   % pdf code
% \else
%   % dvi code
% \fi
% The latest version of ifpdf.sty can be obtained from:
% http://www.ctan.org/pkg/ifpdf
% Also, note that IEEEtran.cls V1.7 and later provides a builtin
% \ifCLASSINFOpdf conditional that works the same way.
% When switching from latex to pdflatex and vice-versa, the compiler may
% have to be run twice to clear warning/error messages.






% *** CITATION PACKAGES ***
%
\ifCLASSOPTIONcompsoc
  % IEEE Computer Society needs nocompress option
  % requires cite.sty v4.0 or later (November 2003)
  \usepackage[nocompress]{cite}
\else
  % normal IEEE
  \usepackage{cite}
\fi
% cite.sty was written by Donald Arseneau
% V1.6 and later of IEEEtran pre-defines the format of the cite.sty package
% \cite{} output to follow that of the IEEE. Loading the cite package will
% result in citation numbers being automatically sorted and properly
% "compressed/ranged". e.g., [1], [9], [2], [7], [5], [6] without using
% cite.sty will become [1], [2], [5]--[7], [9] using cite.sty. cite.sty's
% \cite will automatically add leading space, if needed. Use cite.sty's
% noadjust option (cite.sty V3.8 and later) if you want to turn this off
% such as if a citation ever needs to be enclosed in parenthesis.
% cite.sty is already installed on most LaTeX systems. Be sure and use
% version 5.0 (2009-03-20) and later if using hyperref.sty.
% The latest version can be obtained at:
% http://www.ctan.org/pkg/cite
% The documentation is contained in the cite.sty file itself.
%
% Note that some packages require special options to format as the Computer
% Society requires. In particular, Computer Society  papers do not use
% compressed citation ranges as is done in typical IEEE papers
% (e.g., [1]-[4]). Instead, they list every citation separately in order
% (e.g., [1], [2], [3], [4]). To get the latter we need to load the cite
% package with the nocompress option which is supported by cite.sty v4.0
% and later. Note also the use of a CLASSOPTION conditional provided by
% IEEEtran.cls V1.7 and later.





% *** GRAPHICS RELATED PACKAGES ***
%
\ifCLASSINFOpdf
  % \usepackage[pdftex]{graphicx}
  % declare the path(s) where your graphic files are
  % \graphicspath{{../pdf/}{../jpeg/}}
  % and their extensions so you won't have to specify these with
  % every instance of \includegraphics
  % \DeclareGraphicsExtensions{.pdf,.jpeg,.png}
\else
  % or other class option (dvipsone, dvipdf, if not using dvips). graphicx
  % will default to the driver specified in the system graphics.cfg if no
  % driver is specified.
  % \usepackage[dvips]{graphicx}
  % declare the path(s) where your graphic files are
  % \graphicspath{{../eps/}}
  % and their extensions so you won't have to specify these with
  % every instance of \includegraphics
  % \DeclareGraphicsExtensions{.eps}
\fi
% graphicx was written by David Carlisle and Sebastian Rahtz. It is
% required if you want graphics, photos, etc. graphicx.sty is already
% installed on most LaTeX systems. The latest version and documentation
% can be obtained at: 
% http://www.ctan.org/pkg/graphicx
% Another good source of documentation is "Using Imported Graphics in
% LaTeX2e" by Keith Reckdahl which can be found at:
% http://www.ctan.org/pkg/epslatex
%
% latex, and pdflatex in dvi mode, support graphics in encapsulated
% postscript (.eps) format. pdflatex in pdf mode supports graphics
% in .pdf, .jpeg, .png and .mps (metapost) formats. Users should ensure
% that all non-photo figures use a vector format (.eps, .pdf, .mps) and
% not a bitmapped formats (.jpeg, .png). The IEEE frowns on bitmapped formats
% which can result in "jaggedy"/blurry rendering of lines and letters as
% well as large increases in file sizes.
%
% You can find documentation about the pdfTeX application at:
% http://www.tug.org/applications/pdftex






% *** MATH PACKAGES ***
%
%\usepackage{amsmath}
% A popular package from the American Mathematical Society that provides
% many useful and powerful commands for dealing with mathematics.
%
% Note that the amsmath package sets \interdisplaylinepenalty to 10000
% thus preventing page breaks from occurring within multiline equations. Use:
%\interdisplaylinepenalty=2500
% after loading amsmath to restore such page breaks as IEEEtran.cls normally
% does. amsmath.sty is already installed on most LaTeX systems. The latest
% version and documentation can be obtained at:
% http://www.ctan.org/pkg/amsmath





% *** SPECIALIZED LIST PACKAGES ***
%
%\usepackage{algorithmic}
% algorithmic.sty was written by Peter Williams and Rogerio Brito.
% This package provides an algorithmic environment fo describing algorithms.
% You can use the algorithmic environment in-text or within a figure
% environment to provide for a floating algorithm. Do NOT use the algorithm
% floating environment provided by algorithm.sty (by the same authors) or
% algorithm2e.sty (by Christophe Fiorio) as the IEEE does not use dedicated
% algorithm float types and packages that provide these will not provide
% correct IEEE style captions. The latest version and documentation of
% algorithmic.sty can be obtained at:
% http://www.ctan.org/pkg/algorithms
% Also of interest may be the (relatively newer and more customizable)
% algorithmicx.sty package by Szasz Janos:
% http://www.ctan.org/pkg/algorithmicx




% *** ALIGNMENT PACKAGES ***
%
%\usepackage{array}
% Frank Mittelbach's and David Carlisle's array.sty patches and improves
% the standard LaTeX2e array and tabular environments to provide better
% appearance and additional user controls. As the default LaTeX2e table
% generation code is lacking to the point of almost being broken with
% respect to the quality of the end results, all users are strongly
% advised to use an enhanced (at the very least that provided by array.sty)
% set of table tools. array.sty is already installed on most systems. The
% latest version and documentation can be obtained at:
% http://www.ctan.org/pkg/array


% IEEEtran contains the IEEEeqnarray family of commands that can be used to
% generate multiline equations as well as matrices, tables, etc., of high
% quality.




% *** SUBFIGURE PACKAGES ***
%\ifCLASSOPTIONcompsoc
%  \usepackage[caption=false,font=footnotesize,labelfont=sf,textfont=sf]{subfig}
%\else
%  \usepackage[caption=false,font=footnotesize]{subfig}
%\fi
% subfig.sty, written by Steven Douglas Cochran, is the modern replacement
% for subfigure.sty, the latter of which is no longer maintained and is
% incompatible with some LaTeX packages including fixltx2e. However,
% subfig.sty requires and automatically loads Axel Sommerfeldt's caption.sty
% which will override IEEEtran.cls' handling of captions and this will result
% in non-IEEE style figure/table captions. To prevent this problem, be sure
% and invoke subfig.sty's "caption=false" package option (available since
% subfig.sty version 1.3, 2005/06/28) as this is will preserve IEEEtran.cls
% handling of captions.
% Note that the Computer Society format requires a sans serif font rather
% than the serif font used in traditional IEEE formatting and thus the need
% to invoke different subfig.sty package options depending on whether
% compsoc mode has been enabled.
%
% The latest version and documentation of subfig.sty can be obtained at:
% http://www.ctan.org/pkg/subfig




% *** FLOAT PACKAGES ***
%
%\usepackage{fixltx2e}
% fixltx2e, the successor to the earlier fix2col.sty, was written by
% Frank Mittelbach and David Carlisle. This package corrects a few problems
% in the LaTeX2e kernel, the most notable of which is that in current
% LaTeX2e releases, the ordering of single and double column floats is not
% guaranteed to be preserved. Thus, an unpatched LaTeX2e can allow a
% single column figure to be placed prior to an earlier double column
% figure.
% Be aware that LaTeX2e kernels dated 2015 and later have fixltx2e.sty's
% corrections already built into the system in which case a warning will
% be issued if an attempt is made to load fixltx2e.sty as it is no longer
% needed.
% The latest version and documentation can be found at:
% http://www.ctan.org/pkg/fixltx2e


%\usepackage{stfloats}
% stfloats.sty was written by Sigitas Tolusis. This package gives LaTeX2e
% the ability to do double column floats at the bottom of the page as well
% as the top. (e.g., "\begin{figure*}[!b]" is not normally possible in
% LaTeX2e). It also provides a command:
%\fnbelowfloat
% to enable the placement of footnotes below bottom floats (the standard
% LaTeX2e kernel puts them above bottom floats). This is an invasive package
% which rewrites many portions of the LaTeX2e float routines. It may not work
% with other packages that modify the LaTeX2e float routines. The latest
% version and documentation can be obtained at:
% http://www.ctan.org/pkg/stfloats
% Do not use the stfloats baselinefloat ability as the IEEE does not allow
% \baselineskip to stretch. Authors submitting work to the IEEE should note
% that the IEEE rarely uses double column equations and that authors should try
% to avoid such use. Do not be tempted to use the cuted.sty or midfloat.sty
% packages (also by Sigitas Tolusis) as the IEEE does not format its papers in
% such ways.
% Do not attempt to use stfloats with fixltx2e as they are incompatible.
% Instead, use Morten Hogholm'a dblfloatfix which combines the features
% of both fixltx2e and stfloats:
%
% \usepackage{dblfloatfix}
% The latest version can be found at:
% http://www.ctan.org/pkg/dblfloatfix




%\ifCLASSOPTIONcaptionsoff
%  \usepackage[nomarkers]{endfloat}
% \let\MYoriglatexcaption\caption
% \renewcommand{\caption}[2][\relax]{\MYoriglatexcaption[#2]{#2}}
%\fi
% endfloat.sty was written by James Darrell McCauley, Jeff Goldberg and 
% Axel Sommerfeldt. This package may be useful when used in conjunction with 
% IEEEtran.cls'  captionsoff option. Some IEEE journals/societies require that
% submissions have lists of figures/tables at the end of the paper and that
% figures/tables without any captions are placed on a page by themselves at
% the end of the document. If needed, the draftcls IEEEtran class option or
% \CLASSINPUTbaselinestretch interface can be used to increase the line
% spacing as well. Be sure and use the nomarkers option of endfloat to
% prevent endfloat from "marking" where the figures would have been placed
% in the text. The two hack lines of code above are a slight modification of
% that suggested by in the endfloat docs (section 8.4.1) to ensure that
% the full captions always appear in the list of figures/tables - even if
% the user used the short optional argument of \caption[]{}.
% IEEE papers do not typically make use of \caption[]'s optional argument,
% so this should not be an issue. A similar trick can be used to disable
% captions of packages such as subfig.sty that lack options to turn off
% the subcaptions:
% For subfig.sty:
% \let\MYorigsubfloat\subfloat
% \renewcommand{\subfloat}[2][\relax]{\MYorigsubfloat[]{#2}}
% However, the above trick will not work if both optional arguments of
% the \subfloat command are used. Furthermore, there needs to be a
% description of each subfigure *somewhere* and endfloat does not add
% subfigure captions to its list of figures. Thus, the best approach is to
% avoid the use of subfigure captions (many IEEE journals avoid them anyway)
% and instead reference/explain all the subfigures within the main caption.
% The latest version of endfloat.sty and its documentation can obtained at:
% http://www.ctan.org/pkg/endfloat
%
% The IEEEtran \ifCLASSOPTIONcaptionsoff conditional can also be used
% later in the document, say, to conditionally put the References on a 
% page by themselves.




% *** PDF, URL AND HYPERLINK PACKAGES ***
%
%\usepackage{url}
% url.sty was written by Donald Arseneau. It provides better support for
% handling and breaking URLs. url.sty is already installed on most LaTeX
% systems. The latest version and documentation can be obtained at:
% http://www.ctan.org/pkg/url
% Basically, \url{my_url_here}.





% *** Do not adjust lengths that control margins, column widths, etc. ***
% *** Do not use packages that alter fonts (such as pslatex).         ***
% There should be no need to do such things with IEEEtran.cls V1.6 and later.
% (Unless specifically asked to do so by the journal or conference you plan
% to submit to, of course. )


% correct bad hyphenation here
\hyphenation{op-tical net-works semi-conduc-tor}

\usepackage[utf8]{inputenc} % allow utf-8 input
\usepackage[T1]{fontenc}    % use 8-bit T1 fonts
\usepackage{hyperref}       % hyperlinks
\usepackage{url}            % simple URL typesetting
\usepackage{booktabs}       % professional-quality tables
\usepackage{amsfonts}       % blackboard math symbols
\usepackage{nicefrac}       % compact symbols for 1/2, etc.
\usepackage{microtype}      % microtypography
\usepackage{xcolor}         % colors
\usepackage{amsmath}
\usepackage{multirow}
\usepackage{float}
\usepackage{colortbl}
\usepackage{graphicx} 
\usepackage{subcaption}
\def\etal{\emph{et al.}}
\usepackage{graphicx}
\usepackage{arydshln}
\usepackage{caption}
\usepackage{tabularx}
\usepackage{adjustbox}
\usepackage{algorithmic}
\usepackage{mathtools}
\usepackage[linesnumbered,boxed,ruled,commentsnumbered]{algorithm2e}
\newcommand{\LCS}[1]{\textcolor{orange}{[LCS: #1]}}
\begin{document}
\renewcommand\arraystretch{1.5}
%
% paper title
% Titles are generally capitalized except for words such as a, an, and, as,
% at, but, by, for, in, nor, of, on, or, the, to and up, which are usually
% not capitalized unless they are the first or last word of the title.
% Linebreaks \\ can be used within to get better formatting as desired.
% Do not put math or special symbols in the title.
\title{\title{Supplementary Materials of  \\ \emph{Robust Disentangled Counterfactual Learning for Physical Audiovisual Commonsense Reasoning}}}
%
%
% author names and IEEE memberships
% note positions of commas and nonbreaking spaces ( ~ ) LaTeX will not break
% a structure at a ~ so this keeps an author's name from being broken across
% two lines.
% use \thanks{} to gain access to the first footnote area
% a separate \thanks must be used for each paragraph as LaTeX2e's \thanks
% was not built to handle multiple paragraphs
%
%
%\IEEEcompsocitemizethanks is a special \thanks that produces the bulleted
% lists the Computer Society journals use for "first footnote" author
% affiliations. Use \IEEEcompsocthanksitem which works much like \item
% for each affiliation group. When not in compsoc mode,
% \IEEEcompsocitemizethanks becomes like \thanks and
% \IEEEcompsocthanksitem becomes a line break with idention. This
% facilitates dual compilation, although admittedly the differences in the
% desired content of \author between the different types of papers makes a
% one-size-fits-all approach a daunting prospect. For instance, compsoc 
% journal papers have the author affiliations above the "Manuscript
% received ..."  text while in non-compsoc journals this is reversed. Sigh.

\author{
        Mengshi Qi,~\IEEEmembership{Member,~IEEE},
        Changsheng Lv,
        Huadong Ma,~\IEEEmembership{Fellow,~IEEE}
\thanks{This work is partly supported by the Funds for the NSFC Project under Grant 62202063, Beijing Natural Science Foundation (L243027). (\emph{Corresponding author: Mengshi Qi~(email:~qms@bupt.edu.cn)})}
\thanks{M. Qi, C. Lv, and H. Ma are with the State Key Laboratory of Networking and Switching Technology, Beijing University of Posts and Telecommunications, China.}
}
% note the % following the last \IEEEmembership and also \thanks - 
% these prevent an unwanted space from occurring between the last author name
% and the end of the author line. i.e., if you had this:
% 
% \author{....lastname \thanks{...} \thanks{...} }
%                     ^------------^------------^----Do not want these spaces!
%
% a space would be appended to the last name and could cause every name on that
% line to be shifted left slightly. This is one of those "LaTeX things". For
% instance, "\textbf{A} \textbf{B}" will typeset as "A B" not "AB". To get
% "AB" then you have to do: "\textbf{A}\textbf{B}"
% \thanks is no different in this regard, so shield the last } of each \thanks
% that ends a line with a % and do not let a space in before the next \thanks.
% Spaces after \IEEEmembership other than the last one are OK (and needed) as
% you are supposed to have spaces between the names. For what it is worth,
% this is a minor point as most people would not even notice if the said evil
% space somehow managed to creep in.



% The paper headers
\markboth{Transactions on Pattern Analysis and Machine Intelligence}%
{Shell \MakeLowercase{\textit{et al.}}: Bare Demo of IEEEtran.cls for Computer Society Journals}
% The only time the second header will appear is for the odd numbered pages
% after the title page when using the twoside option.
% 
% *** Note that you probably will NOT want to include the author's ***
% *** name in the headers of peer review papers.                   ***
% You can use \ifCLASSOPTIONpeerreview for conditional compilation here if
% you desire.



% The publisher's ID mark at the bottom of the page is less important with
% Computer Society journal papers as those publications place the marks
% outside of the main text columns and, therefore, unlike regular IEEE
% journals, the available text space is not reduced by their presence.
% If you want to put a publisher's ID mark on the page you can do it like
% this:
%\IEEEpubid{0000--0000/00\$00.00~\copyright~2015 IEEE}
% or like this to get the Computer Society new two part style.
%\IEEEpubid{\makebox[\columnwidth]{\hfill 0000--0000/00/\$00.00~\copyright~2015 IEEE}%
%\hspace{\columnsep}\makebox[\columnwidth]{Published by the IEEE Computer Society\hfill}}
% Remember, if you use this you must call \IEEEpubidadjcol in the second
% column for its text to clear the IEEEpubid mark (Computer Society jorunal
% papers don't need this extra clearance.)



% use for special paper notices
%\IEEEspecialpapernotice{(Invited Paper)}



% for Computer Society papers, we must declare the abstract and index terms
% PRIOR to the title within the \IEEEtitleabstractindextext IEEEtran
% command as these need to go into the title area created by \maketitle.
% As a general rule, do not put math, special symbols or citations
% in the abstract or keywords.
\IEEEtitleabstractindextext{%
% \begin{abstract}
% In this paper, we propose a Disentangled Counterfactual Learning~(DCL) approach for physical audiovisual commonsense reasoning. The task aims to infer objects' physics commonsense based on both video and audio input, with the main challenge is how to imitate the reasoning ability of humans. Most of the current methods fail to take full advantage of different characteristics in multi-modal data, and lacking causal reasoning ability in models impedes the progress of implicit physical knowledge inferring. 
% To address these issues, our proposed DCL method decouples videos into static (time-invariant) and dynamic (time-varying) factors in the latent space by the disentangled sequential encoder, which adopts a variational autoencoder (VAE) to maximize the mutual information with a contrastive loss function. Furthermore, we introduce a counterfactual learning module to augment the model's reasoning ability by modeling physical knowledge relationships among different objects under counterfactual intervention. Our proposed method is a plug-and-play module that can be incorporated into any baseline. In experiments, we show that our proposed method improves baseline methods and achieves state-of-the-art performance. 
% \end{abstract}

% Note that keywords are not normally used for peerreview papers.
\begin{IEEEkeywords}
Physical Commonsense Reasoning, Robust Multimodal Learning, Disentangled Representation, Counterfactual Analysis.
\end{IEEEkeywords}}


% make the title area
\maketitle


% To allow for easy dual compilation without having to reenter the
% abstract/keywords data, the \IEEEtitleabstractindextext text will
% not be used in maketitle, but will appear (i.e., to be "transported")
% here as \IEEEdisplaynontitleabstractindextext when the compsoc 
% or transmag modes are not selected <OR> if conference mode is selected 
% - because all conference papers position the abstract like regular
% papers do.
\IEEEdisplaynontitleabstractindextext
% \IEEEdisplaynontitleabstractindextext has no effect when using
% compsoc or transmag under a non-conference mode.



% For peer review papers, you can put extra information on the cover
% page as needed:
% \ifCLASSOPTIONpeerreview
% \begin{center} \bfseries EDICS Category: 3-BBND \end{center}
% \fi
%
% For peerreview papers, this IEEEtran command inserts a page break and
% creates the second title. It will be ignored for other modes.
\IEEEpeerreviewmaketitle


In this supplementary material, we provide a comprehensive algorithm underlying our proposed model, encompassing both the DCL and RDCL in Section~\ref{Sec. Algorithm of DCL and RDCL}. Section~\ref{Sec: Drivations and more experimental results} includes derivations and supplementary experimental results. Additionally, Section~\ref{Sec: VLM-Assisted Reasoning Dataset} presents more samples and statistical analyses of the VLM-Assisted Reasoning Dataset introduced in our main paper.
% Computer Society journal (but not conference!) papers do something unusual
% with the very first section heading (almost always called "Introduction").
% They place it ABOVE the main text! IEEEtran.cls does not automatically do
% this for you, but you can achieve this effect with the provided
% \IEEEraisesectionheading{} command. Note the need to keep any \label that
% is to refer to the section immediately after \section in the above as
% \IEEEraisesectionheading puts \section within a raised box.




% The very first letter is a 2 line initial drop letter followed
% by the rest of the first word in caps (small caps for compsoc).
% 
% form to use if the first word consists of a single letter:
% \IEEEPARstart{A}{demo} file is ....
% 
% form to use if you need the single drop letter followed by
% normal text (unknown if ever used by the IEEE):
% \IEEEPARstart{A}{}demo file is ....
% 
% Some journals put the first two words in caps:
% \IEEEPARstart{T}{his demo} file is ....
% 
% Here we have the typical use of a "T" for an initial drop letter
% and "HIS" in caps to complete the first word.
\section{Algorithm of DCL and RDCL}
In this section, we introduce the detailed algorithms for Disentangled Counterfactual
Learning (DCL) in Section~\ref{Sec. Algorithm of DCL} and Robust Disentangled Counterfactual
Learning (RDCL) in Section~\ref{Sec. Algorithm of RDCL} for Physical Commonsense Reasoning.  
\label{Sec. Algorithm of DCL and RDCL}
\subsection{DCL}
\label{Sec. Algorithm of DCL}
The overall framework of the proposed DCL algorithm is outlined in Algorithm~\ref{algorithm: DCL}. The model takes as input a training batch consisting of paired video-audio data along with associated physical knowledge questions. It outputs the final prediction, denoted as $\hat{Y}_{TIE}$.  

\subsection{RDCL}
\label{Sec. Algorithm of RDCL}
Unlike DCL, which processes complete multimodal inputs, RDCL is designed to handle incomplete modalities. As an illustrative example, we consider scenarios where audio data are missing. The corresponding algorithm is presented in Algorithm~\ref{algorithm: RDCL}.
\begin{algorithm}[ht]
\caption{Disentangled Counterfactual Learning~(DCL) Batch-Wise Training}
\label{algorithm: DCL}
\SetAlgoLined
\KwIn{
    Training batch $\{ \langle v_1, v_2 \rangle_i, \langle a_1, a_2 \rangle_i, q_i \}_{i=1}^{B}$, \\
    Batch size $B$, \\
    Pretrained image encoder $\mathcal{E}_{\text{img}}(\theta)$, \\
    Pretrained audio encoder $\mathcal{E}_{\text{aud}}(\theta)$, \\
    Pretrained text encoder $\mathcal{E}_{\text{text}}(\theta)$, \\
    Labels $\{Y_{GT,i}\}_{i=1}^B$, \\
    Number of frames $T$
}
\KwOut{
    Predicted answers $\{\hat{Y}_{TIE,i}\}_{i=1}^B$
}


Encode features: \\
  \For{$j \in \{1, 2\}$}{ 
   \quad $X^{v_j} = \{X_1^{v_j}, X_2^{v_j}, \cdots, X_T^{v_j}\} \gets \mathcal{E}_{\text{img}}(v_j)$ \\ 
   \quad $X^{a_j} \gets \mathcal{E}_{\text{aud}}(a_j)$
   }
   $X^t \gets \mathcal{E}_{\text{text}}(q)$ \\

\For{each sample in the batch}{
    Disentangle static factors $X^v_s$ and dynamic factors $X^v_z$ from $X^v$ via DSE in Section 4.2. \\
}

Compute the adjacency matrix $A_X$ using Eq. (15), (16), and (17). \\

Obtain the fused feature $F_1$, $F_2$ using Eq.(14). \\

Construct intervened features $X^*$ using Eq. (20), and compute the intervened adjacency matrix ${A}^*$ using Eqs. (15), (16) and (17). \\

Predict the $\hat{Y}_{X, A_X}$ and $\hat{Y}_{X^*, A_{X^*}}$ using Eq.(18)\\

Use $\hat{Y}_{TIE}$ obtained from Eq.(19) as the output.
\end{algorithm}

\section{Drivations and more experimental results}
\label{Sec: Drivations and more experimental results}
\subsection{Approximate Estimation of the Objective Function}

In Section 4.2 Disentangled Sequential Encoder of our main paper, our goal is to maximize the log-likelihood of \( x_{1:T} \). However, due to the computational complexity associated with high-dimensional integrals, directly obtaining \( \log p(x_{1:T}) \) is challenging. To address this issue, we employ the Evidence Lower Bound (ELBO) as an approximation to the log-likelihood. 

{\bf For the input sequence $x_{1:T}$}, Eq.(~\ref{eq:elbo_derivation}) in Figure~\ref{Seq-ELBO} shown adapted from the standard VAE framework \cite{bai2021contrastively}, noticing that either the prior or the approximate posterior factorizes over $s$ and $z_{1:T}$.
{\bf For the entire dataset}, let \( p_D \) represent the empirical data distribution, which assigns a probability mass of \( 1/N \) to each of the \( N \) training sequences in \( D \). The aggregated posteriors are defined as shown in Eq.(~\ref{Eq.2}),  Eq.(~\ref{Eq.3}), and  Eq.(~\ref{Eq.4}) in Figure~\ref{proof_3}. By rearranging terms and applying similar operations to \( x \), we arrive at Eq.~(\ref {Eq.6}) and Eq.~(\ref{Eq.7}) in Figure~\ref{proof_3}. Finally, integrating the above derivations, we obtain the dataset ELBO by subtracting a distinct KL divergence from the data log-likelihood, as illustrated in Eq.~\ref{eq:elbo} in Figure~\ref{proof-4}.
\begin{figure*}
\begin{equation}
\begin{aligned}
& \log p(x_{1:T}) \\
\ge& -KL[q(s,z_{1:T}|x_{1:T})||p(s,z_{1:T}|x_{1:T})]+\log p(x_{1:T})\\
=&\mathbb E_{q(s, z_{1:T}|x_{1:T})} \left[ \log p(s,z_{1:T}|x_{1:T}) - \log q(s,z_{1:T}|x_{1:T}) + \log p(x_{1:T}) \right] \\
=&\mathbb E_{q(s,z_{1:T}|x_{1:T})}[\log p(x_{1:T}|s, z_{1:T})-\log q(s,z_{1:T}|x_{1:T})+\log p(s,z_{1:T})]\\
=&\mathbb E_{q(s,z_{1:T}|x_{1:T})}[\log p(x_{1:T}|s, z_{1:T})-\log q(s|x_{1:T}) - \log p(z_{1:T}|x_{1:T})+\log p(s) + \log p(z_{1:T})]\\
=&\mathbb E_{q(z_{1:T}, s|x_{1:T})} \left[ \underbrace{\log p(x_{1:T}|s, z_{1:T})}_{\text{Reconstruction term}}-\underbrace{KL[q(s|x_{1:T})||p(s)]}_{s\text{-regression}}-\underbrace{KL[q(z_{1:T}|x_{1:T})||p(z_{1:T})]}_{z\text{-regression}} \right].
\end{aligned}  
\label{eq:elbo_derivation}
\end{equation}
\caption{The ELBO derivation for the input sequence \( x_{1:T} \).}
\label{Seq-ELBO}
\end{figure*}

\begin{figure*}
\centering
\begin{align}
q(s) & = \mathbb E_{x_{1:T}\sim p_D} [q(s|x_{1:T})] = \frac{1}{N} \sum_{x_{1:T}\in D} q(s|x_{1:T}), \label{Eq.2} \\
q(z_{1:T}) & = \mathbb E_{x_{1:T}\sim p_D} [q(z_{1:T}|x_{1:T})] = \frac{1}{N} \sum_{x_{1:T}\in D} q(z_{1:T}|x_{1:T}), \label{Eq.3} \\
q(s,z_{1:T}) & = \mathbb E_{x_{1:T}\sim p_D} [q(s|x_{1:T}) q(z_{1:T}|x_{1:T})] = \frac{1}{N} \sum_{x_{1:T}\in D} q(s|x_{1:T}) q(z_{1:T}|x_{1:T}). \label{Eq.4}
\end{align}
\begin{equation}
\begin{aligned}
& \mathbb E_{x_{1:T}\sim p_D}[KL[q(s|x_{1:T})||p(s)]] \\
=& \mathbb E_{x_{1:T}\sim p_D}\mathbb E_{q(s|x_{1:T})}[\log q(s|x_{1:T}) - \log q(s) + \log q(s) - \log p(s)] \\
=& \mathbb E_{q(x_{1:T}, s)} \log \left[ \frac{q(s|x_{1:T})}{q(s)} \right] + \mathbb E_{q(x_{1:T}, s)} [\log q(s)-\log p(s)] \\
=& I_q (x_{1:T};s) + KL [q(s)||p(s)]. 
\end{aligned}
\label{}
\end{equation}
\begin{equation}
KL [q(s)||p(s)] = \mathbb E_{x_{1:T}\sim p_D}[KL[q(s|x_{1:T})||p(s)]] - I_q(x_{1:T};s). 
\label{Eq.6}
\end{equation}
\begin{gather}
KL[q(z_{1:T})||p(z_{1:T})] = \mathbb E_{x_{1:T}\sim p_D}[KL[q(z_{1:T}|x_{1:T})||p(z_{1:T})]] - I_q(x_{1:T};z_{1:T}). 
\label{Eq.7}
\end{gather}
\caption{Aggregated equations and their relationships.}
\label{proof_3}
\end{figure*}

\begin{figure*}
\begin{equation}
\begin{aligned} 
&\frac{1}{N} \sum_{x_{1:T}\in D} \log p(x_{1:T}) = \mathbb E_{x_{1:T}\sim p_D}[\log p(x_{1:T})] \\
\ge& \mathbb E_{x_{1:T}\sim p_D}[\log p(x_{1:T}) - KL[q(s, z_{1:T})||p(s, z_{1:T}|x_{1:T})]] \\
=& \mathbb E_{x_{1:T}\sim p_D}[\mathbb E_{q(s, z_{1:T}|x_{1:T})}[\log p(x_{1:T}) - \log q(s, z_{1:T}) + \log p(s, z_{1:T}|x_{1:T})]] \\
=& \mathbb E_{x_{1:T}\sim p_D}[\mathbb E_{q(s, z_{1:T}|x_{1:T})}[\log p(x_{1:T}) - \log q(s, z_{1:T}) + \log p(x_{1:T}|s, z_{1:T}) + \log p(s, z_{1:T}) - \log p(x_{1:T})]] \\
=& \mathbb E_{x_{1:T}\sim p_D}[\mathbb E_{q(s, z_{1:T}|x_{1:T})}[\log p(x_{1:T}|s, z_{1:T}) - \log q(s, z_{1:T}) + \log p(s, z_{1:T})]] \\
=& \mathbb E_{x_{1:T}\sim p_D}[\mathbb E_{q(s, z_{1:T}|x_{1:T})}[\log p(x_{1:T}|s, z_{1:T})]] - \mathbb E_{x_{1:T}\sim p_D}[\mathbb E_{q(s, z_{1:T}|x_{1:T})}[\log q(s, z_{1:T}) - \log p(s, z_{1:T})]] \\
=& \mathbb E_{x_{1:T}\sim p_D}[\mathbb E_{q(s, z_{1:T}|x_{1:T})}[\log p(x_{1:T}|s, z_{1:T})]] - KL[q(s, z_{1:T})||p(s, z_{1:T})] \\
=& \mathbb E_{x_{1:T}\sim p_D}[\mathbb E_{q(s, z_{1:T}|x_{1:T})}[\log p(x_{1:T}|s, z_{1:T})]] - I_q(s;z_{1:T}) - KL[q(s)||p(s)] - KL[q(z_{1:T})||p(z_{1:T})] \\
=& \mathbb E_{x_{1:T}\sim p_D}[\mathbb E_{q(s, z_{1:T}|x_{1:T})}[\log p(x_{1:T}|s, z_{1:T})]] \\
&\hspace{5em} - \mathbb E_{x_{1:T}\sim p_D}[KL[q(s|x_{1:T})||p(s)]] - \mathbb E_{x_{1:T}\sim p_D}[KL[q(z_{1:T}|x_{1:T})||p(z_{1:T})]] \\
&\hspace{5em} + I_q(s;x_{1:T}) + I_q(z_{1:T};x_{1:T}) - I_q(s;z_{1:T}).
\end{aligned}
\label{eq:elbo}
\end{equation}
\caption{Derivation of the ELBO for a dataset by subtracting a KL-divergence term from the data log-likelihood.}
\label{proof-4}
\end{figure*}

\subsection{Sensitivity Analysis of Parameters}  
We conducted a sensitivity analysis on the parameters $\gamma$ and $\theta$ as defined in Eq. (6) of the main paper, with results presented in Figure~\ref{fig: hyperparameters}. Specifically, we evaluated $\gamma$ over the range $\{0.01, 0.1, 1, 10\}$ and $\theta$ over the range $\{0.5, 5, 50, 500\}$. The results denote that our proposed DCL method exhibits strong robustness to variations in both $\gamma$ and $\theta$, achieving consistent and stable performance across all tested parameter configurations.
\subsection{Analysis of dynamic factors}
Our DSE+ method separates video features into static (time-invariant) and dynamic (time-varying) factors. Figure~\ref{fig: t-sne} shows t-SNE visualizations of these factors alongside raw video features. While raw features appear scattered, dynamic factors extracted by DSE+ exhibit clear clustering, as highlighted by red circles. For example, Figure~\ref{fig: t-sne}(b) shows objects with similar dynamic characteristics, such as small size and lightweight, positioned adjacently. The upper portion of Figure~\ref{fig: t-sne} illustrates a cluster where actions consistently depict a hand grasping and striking the object, reflecting their lightweight characteristics. In contrast to raw features, DSE+ successfully captures this dynamic information. Similarly, the lower section highlights another cluster with shared thickness-related properties, further demonstrating DSE+’s ability to extract dynamic physical characteristics.  
\subsection{Additional Qualitative Results}
As shown in Figure~\ref{fig: Qualitative Results_1}, we present more visualization results comparing our proposed method with other baseline models. It can be seen from the figures that our proposed DCL method outperforms the original process.
\subsection{Impact of visual bias}
As illustrated in Figure~\ref{fig: visual bias}, we show the absolute accuracy differences for specific object pairs. Accuracy for pairs in the lowest 25\% of occurrence frequency improves notably after applying DCL, demonstrating its effectiveness in reducing visual bias for less frequent pairs. However, for some high-frequency pairs ({\it e.g.}, ``paper-foam'' and ``paper-textiles''), a slight accuracy decrease occurs after DCL. This is because dominant visual bias previously led to correct but unreliable predictions, while DCL mitigates this bias, revealing the model’s robust performance.  

\section{VLM-Assisted Reasoning Dataset}
\label{Sec: VLM-Assisted Reasoning Dataset}
\subsection{More examples of VLM-Assisted Reasoning Dataset}
Figure \ref{fig: VLM} illustrates the prompts used for the Vision-Language Model (VLM), with subfigures (a)–(f) showcasing its assisted reasoning outputs across various samples. A failure case is evident in Figure~\ref{fig: VLM}(f), where the VLM excessively emphasizes object-specific details ({\it e.g.}, identifying the type of wine) while overlooking the physical characteristics of the glass bottle. Future work will focus on developing more targeted prompting strategies to address such limitations.
\subsection{Dataset Statistics}
We obtained corresponding VLM descriptions for each object in the PACS dataset, resulting in 1,526 descriptions. The average length of these descriptions is 74.05 words, with a maximum length of 118 words and a minimum length of 41 words. The corresponding word cloud is illustrated in Figure~\ref{fig: Word Cloud} and the Top-50 Material Types in the Object Pair are shown in Figure~\ref{fig: Frequency of Material Types}. The generated data is available at https://github.com/MICLAB-BUPT/DCL.
\begin{figure}[ht]
    \centering
    \includegraphics[width=0.9\linewidth]{Figures/Fig_11.pdf}
    \caption{Absolute differences in accuracy scores between two configurations: AudioCLIP with DCL using solely video input (V w/ DCL) and AudioCLIP utilizing only video input (V). The parenthetical value indicates the frequency of occurrence, measured at 11.5 instances within the final 25\% of the training dataset.}
    \label{fig: visual bias}
\end{figure}
\begin{figure}[ht]
    \centering
    \includegraphics[width=0.9\linewidth]{Figures/T-SNE.pdf}
    \caption{T-SNE visualization of video features before applying DSE+ (a), along with dynamic factors (b) and static factors (c) obtained after DSE+. The \textcolor{red}{red circles} indicate clusters that have been manually identified as containing samples with similar physical properties. We provide examples of these clusters based on shared attributes, including weight and thickness.}
    \label{fig: t-sne}
\end{figure}
\begin{figure}[b]
    \centering
    \includegraphics[width=1.0\linewidth]{Figures/Fig_12.pdf}
   \caption{Word Cloud for VLM-Assisted Reasoning}
    \label{fig: Word Cloud}
\end{figure}
\begin{algorithm}[ht]
\caption{Robust Disentangled Counterfactual Learning (RDCL) Batch-Wise Training}
\label{algorithm: RDCL}
\SetAlgoLined
\KwIn{
    Training batch $\{ \langle v_1, v_2 \rangle_i, \langle a_1, a_2 \rangle_i, q_i \}_{i=1}^{B}$, \\
    Batch size $B$, \\
    Pretrained image encoder $\mathcal{E}_{\text{img}}(\theta)$, \\
    Pretrained audio encoder $\mathcal{E}_{\text{aud}}(\theta)$, \\
    Pretrained text encoder $\mathcal{E}_{\text{text}}(\theta)$, \\
    Labels $\{Y_{GT,i}\}_{i=1}^B$, \\
    Number of frames $T$, proportion of missing data in object 1's video $a_{v1}.$ 
}
\KwOut{
    Predicted answers $\{\hat{Y}_{TIE,i}\}_{i=1}^B$
}

\textbf{Encode features:} \\
\For{$j \in \{1, 2\}$}{ 
   \quad $X^{v_j} = \{X_1^{v_j}, X_2^{v_j}, \cdots, X_T^{v_j}\} \gets \mathcal{E}_{\text{img}}(v_j)$ \\ 
   \quad $X^{a_j} \gets \mathcal{E}_{\text{aud}}(a_j)$
}
$X^t \gets \mathcal{E}_{\text{text}}(q)$ \\

Obtain the set of missing data set $B_{{miss}}$ and the complete data set $B_{{com}}$ using Eq.(27). \\

For each sample $i$ in the batch:

\If{$i \in B_{com}$}{
    \For{each sample in the $B_{{com}}$}{
        Disentangle static factors $X^v_s$ and dynamic factors $X^v_z$ from $X^v$ via DSE in Section 4.2. \\
    }
    
    Use unique encoder and shared encoder to encode $X^v_s$, $X^v_z$, and $X^a$ using Eqs.(24) and (25), obtaining $r_m^{unique}$ and $r_m^{share}, m \in \{a,z,s\}$.
}
\Else{
    Use Eqs. (29) and (30) to complete the missing information.
}

Compute the adjacency matrix $A_X$ using Eqs. (15), (16), and (17). \\

Obtain the fused features $F_1$ and $F_2$ using Eq.(14). \\

Construct the intervened features $X^*$ using Eq. (20), and compute the intervened adjacency matrix ${A}^*$ using Eqs. (15), (16), and (17). \\

Predict $\hat{Y}_{X, A_X}$ and $\hat{Y}_{X^*, A_{X^*}}$ using Eq.(18). \\

Use $\hat{Y}_{TIE}$ obtained from Eq.(19) as the output.
\end{algorithm}

\begin{figure}[ht]
    \centering
    \includegraphics[width=1.0\linewidth]{Figures/Fig_13.pdf}
    \caption{Frequency of Material Types for Object Pairs in the PACS Training Set.}
    \label{fig: Frequency of Material Types}
\end{figure}
\begin{figure*}[ht]
    \centering
    \includegraphics[width=1.0\linewidth]{Figures/Fig_7.pdf}
    \caption{Performance comparisons of various hyperparameters in Eq. (6) are presented. Figures (a) and (b) display the performance of AudioCLIP with different values of $\gamma$ in $\mathcal{L}_{DSE}$ on the PACS and PACS-Material datasets. Figures (c) and (d) show the performance of AudioCLIP with varying $\theta$ in $\mathcal{L}_{DSE}$ on the same datasets.}
    \label{fig: hyperparameters}
\end{figure*}
\begin{figure*}
    \centering
    \includegraphics[width=0.9\linewidth]{Figures/Fig_8.pdf}
    \caption{Qualitative Results of baseline w/ and w/o our proposed method, where “Material” refers to the material of the object. The correct answers are depicted in green while the incorrect ones are depicted in red.}
    \label{fig: Qualitative Results_1}
\end{figure*}
\begin{figure*}
    \centering
    \includegraphics[width=0.9\linewidth]{Figures/Fig_9.pdf}
    \caption{Qualitative Results of baseline w/ and w/o our proposed method, where “Material” refers to the material of the object. The correct answers are depicted in green while the incorrect ones are depicted in red.}
    \label{fig: Qualitative Results_2}
\end{figure*}
\begin{figure*}
    \centering
    \includegraphics[width=0.75\linewidth]{Figures/Fig_10.pdf}
    \caption{Prompt Text, Input Image, and Corresponding Response of the VLM ({\it i.e.}, Doubao-1.5-Vision-Pro)}
    \label{fig: VLM}
\end{figure*}


\ifCLASSOPTIONcaptionsoff
  \newpage
\fi


\bibliographystyle{IEEEtran}
\bibliography{reference}
\end{document}




\end{document}
