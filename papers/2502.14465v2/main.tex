%%
%% This is file `sample-acmsmall.tex',
%% generated with the docstrip utility.
%%
%% The original source files were:
%%
%% samples.dtx  (with options: `all,journal,bibtex,acmsmall')
%% 
%% IMPORTANT NOTICE:
%% 
%% For the copyright see the source file.
%% 
%% Any modified versions of this file must be renamed
%% with new filenames distinct from sample-acmsmall.tex.
%% 
%% For distribution of the original source see the terms
%% for copying and modification in the file samples.dtx.
%% 
%% This generated file may be distributed as long as the
%% original source files, as listed above, are part of the
%% same distribution. (The sources need not necessarily be
%% in the same archive or directory.)
%%
%%
%% Commands for TeXCount
%TC:macro \cite [option:text,text]
%TC:macro \citep [option:text,text]
%TC:macro \citet [option:text,text]
%TC:envir table 0 1
%TC:envir table* 0 1
%TC:envir tabular [ignore] word
%TC:envir displaymath 0 word
%TC:envir math 0 word
%TC:envir comment 0 0
%%
%%
%% The first command in your LaTeX source must be the \documentclass
%% command.
%%
%% For submission and review of your manuscript please change the
%% command to \documentclass[manuscript, screen, review]{acmart}.
%%
%% When submitting camera ready or to TAPS, please change the command
%% to \documentclass[sigconf]{acmart} or whichever template is required
%% for your publication.
%%
%%
\PassOptionsToPackage{dvipsnames}{xcolor}
\documentclass[acmsmall, nonacm, screen]{acmart}

\usepackage{listings}
\usepackage{comment}
\usepackage{wrapfig}
\usepackage{tabularx}
\usepackage[inline]{enumitem}
\usepackage{multicol}
\usepackage{algorithm}
\usepackage{algpseudocode}
\usepackage{mdframed}
\algnewcommand\Failure{\textbf{Failure}}
\algnewcommand\True{\textbf{True}}
\algnewcommand\False{\textbf{False}}
\algnewcommand\Land{\textbf{and }}
\algnewcommand\Or{\textbf{or }}
\algnewcommand\algorithmicsmatch{\textbf{match}}
\algnewcommand\algorithmiccase{\textbf{case}}
\algnewcommand\algorithmicdefault{\textbf{default}}
\algdef{SE}[MATCH]{Match}{EndMatch}[1]{\algorithmicsmatch\ #1:}{\algorithmicend\ \algorithmicsmatch}%
\algdef{SE}[CASE]{Case}{EndCase}[1]{\algorithmiccase\ #1 \ \algorithmicdo}{\algorithmicend\ \algorithmiccase}%
\algdef{SE}[DEFAULT]{Default}{EndCase}{\algorithmicdefault\ \algorithmicdo}{\algorithmicend\ \algorithmicdefault}%
\algtext*{EndMatch}%
\algtext*{EndCase}%

\definecolor{typecolor}{HTML}{B00040}
\definecolor{actioncolor}{rgb}{0.58, 0.0, 0.82}
\newcommand{\typekw}[1]{\textbf{\textcolor{typecolor}{#1}}}
\newcommand{\typeact}[1]{\textbf{\textcolor{actioncolor}{#1}}}
\newcommand{\bind}[0]{\textcolor{brown}{\textsc{Bind}}}
\newcommand{\subhist}[0]{\textcolor{brown}{\textsc{SubHist}}}
\newcommand{\subconc}[0]{\textcolor{brown}{\textsc{SubConc}}}
\newcommand{\histconj}[0]{\textcolor{brown}{\textsc{HistConj}}}
\newcommand{\overa}[1]{\textcolor{red}{\{}#1\textcolor{red}{\}}}
\newcommand{\under}[1]{\textcolor{blue}{[}#1\textcolor{blue}{]}}
\newcommand{\nextd}[0]{\textcolor{brown}{\textsc{new}}}
\newcommand{\bluet}[1]{\textcolor{blue}{#1}}
\newcommand{\oranget}[1]{\textcolor{orange}{#1}}
\newcommand{\greent}[1]{\textcolor{green}{#1}}

\theoremstyle{definition}
\newtheorem{definition}{Definition}[section]
\newtheorem{notation}{Notation}[section]
\newtheorem{example}{Example}[section]
\newtheorem{lemma}{Lemma}[section]
\newtheorem{theorem}{Theorem}[section]
\newtheorem{note}{Note}[section]

%%
%% \BibTeX command to typeset BibTeX logo in the docs
\AtBeginDocument{%
  \providecommand\BibTeX{{%
    Bib\TeX}}}

%% Rights management information.  This information is sent to you
%% when you complete the rights form.  These commands have SAMPLE
%% values in them; it is your responsibility as an author to replace
%% the commands and values with those provided to you when you
%% complete the rights form.
% commented
%\setcopyright{acmlicensed}
%\copyrightyear{2024}
%\acmYear{2024}
%\acmDOI{XXXXXXX.XXXXXXX}


%%
%% These commands are for a JOURNAL article.
% commented
%\acmJournal{JACM}
%\acmVolume{37}
%\acmNumber{4}
%\acmArticle{111}
%\acmMonth{8}

%%
%% Submission ID.
%% Use this when submitting an article to a sponsored event. You'll
%% receive a unique submission ID from the organizers
%% of the event, and this ID should be used as the parameter to this command.
% commented
%\acmSubmissionID{123-A56-BU3}

%%
%% For managing citations, it is recommended to use bibliography
%% files in BibTeX format.
%%
%% You can then either use BibTeX with the ACM-Reference-Format style,
%% or BibLaTeX with the acmnumeric or acmauthoryear sytles, that include
%% support for advanced citation of software artefact from the
%% biblatex-software package, also separately available on CTAN.
%%
%% Look at the sample-*-biblatex.tex files for templates showcasing
%% the biblatex styles.
%%

%%
%% The majority of ACM publications use numbered citations and
%% references.  The command \citestyle{authoryear} switches to the
%% "author year" style.
%%
%% If you are preparing content for an event
%% sponsored by ACM SIGGRAPH, you must use the "author year" style of
%% citations and references.
%% Uncommenting
%% the next command will enable that style.
%%\citestyle{acmauthoryear}

\lstdefinestyle{OCaml}{
    language=ML,
    numbers=left,
    numbersep=8pt,
    breaklines=true,
    xleftmargin=2em,
    basicstyle=\ttfamily\footnotesize,
    keywordstyle=\color[rgb]{0.0, 0.5, 0.0}\bfseries,
    keywordstyle=[2]\color[rgb]{0.6, 0.0, 0.0}\bfseries,
    stringstyle=\color[rgb]{0.6, 0.0, 0.0},
    commentstyle=\color{gray},
    morekeywords={match,with,let,in,rec,type,fun,function,val,sig,struct,module,open,of,and,or,not},
    morekeywords=[2]{int,string,bool,float,nat,list,array,option,unit,char},
    escapeinside={??},
    alsoletter={'},
    morestring=[b]',
    alsoletter={(},
    literate=
        {()}{{\textcolor[rgb]{0.0, 0.5, 0.0}{()}}}1
        {[]}{{\textcolor[rgb]{0.0, 0.5, 0.0}{[]}}}1
}

\lstset{style=Ocaml}

%%
%% end of the preamble, start of the body of the document source.
\begin{document}\sloppy

%%
%% The "title" command has an optional parameter,
%% allowing the author to define a "short title" to be used in page headers.
\title{Coverage Types for Resource-Based Policies}

%%
%% The "author" command and its associated commands are used to define
%% the authors and their affiliations.
%% Of note is the shared affiliation of the first two authors, and the
%% "authornote" and "authornotemark" commands
%% used to denote shared contribution to the research.
\author{Angelo Passarelli}
%\authornote{Both authors contributed equally to this research.}
\email{a.passarelli4@studenti.unipi.it}
\orcid{0009-0009-5714-4922}
\author{Gian-Luigi Ferrari}
%\authornote{Both authors contributed equally to this research.}
\email{gian-luigi.ferrari@unipi.it}
\orcid{0000-0003-3548-5514}
\affiliation{%
  \institution{University of Pisa}
  \department{Computer Science Department}
  \city{Pisa}
  \state{Tuscany}
  \country{Italy}
}

%%
%% By default, the full list of authors will be used in the page
%% headers. Often, this list is too long, and will overlap
%% other information printed in the page headers. This command allows
%% the author to define a more concise list
%% of authors' names for this purpose.
%\renewcommand{\shortauthors}{Passarelli}

%%
%% The abstract is a short summary of the work to be presented in the
%% article.
\begin{abstract}
    Coverage Types provide a suitable type mechanism that integrates under-approximation logic to support Property-Based Testing. They are used to type the return value of a function that represents an input test generator. This allows us to statically assert that an input test generator not only produces valid input tests but also generates all possible ones, ensuring completeness.
    
    In this paper, we extend the coverage framework to guarantee the correctness of Property-Based Testing with respect to resource usage in the input test generator. This is achieved by incorporating into Coverage Types a notion of effect, which represents an over-approximation of operations on relevant resources. Programmers can define resource usage policies through logical annotations, which are then verified against the effect associated with the Coverage Type.
\end{abstract}

%%
%% The code below is generated by the tool at http://dl.acm.org/ccs.cfm.
%% Please copy and paste the code instead of the example below.
%%
\begin{CCSXML}
<ccs2012>
   <concept>
       <concept_id>10002978.10002986.10002989</concept_id>
       <concept_desc>Security and privacy~Formal security models</concept_desc>
       <concept_significance>500</concept_significance>
       </concept>
   <concept>
       <concept_id>10002978.10002986.10002987</concept_id>
       <concept_desc>Security and privacy~Trust frameworks</concept_desc>
       <concept_significance>500</concept_significance>
       </concept>
   <concept>
       <concept_id>10002978.10002986.10002990</concept_id>
       <concept_desc>Security and privacy~Logic and verification</concept_desc>
       <concept_significance>300</concept_significance>
       </concept>
   <concept>
       <concept_id>10003752.10003790.10011740</concept_id>
       <concept_desc>Theory of computation~Type theory</concept_desc>
       <concept_significance>500</concept_significance>
       </concept>
   <concept>
       <concept_id>10003752.10003790.10003800</concept_id>
       <concept_desc>Theory of computation~Higher order logic</concept_desc>
       <concept_significance>300</concept_significance>
       </concept>
 </ccs2012>
\end{CCSXML}

\ccsdesc[500]{Security and privacy~Formal security models}
\ccsdesc[500]{Security and privacy~Trust frameworks}
\ccsdesc[300]{Security and privacy~Logic and verification}
\ccsdesc[500]{Theory of computation~Type theory}
\ccsdesc[300]{Theory of computation~Higher order logic}

%%
%% Keywords. The author(s) should pick words that accurately describe
%% the work being presented. Separate the keywords with commas.
\keywords{under-approximation, over-approximation, history expressions, effects, refinements types, resource policies, function as a service}

% commented
%\received{20 February 2007}
%\received[revised]{12 March 2009}
%\received[accepted]{5 June 2009}

%%
%% This command processes the author and affiliation and title
%% information and builds the first part of the formatted document.
\maketitle

\section{Introduction}\label{sec:intro}
\section{Introduction}
\label{sec:intro}

\begin{figure*}[tb]
    \centering
    \includegraphics[width=0.848\linewidth]{figs/circuitnn.pdf} 
    \caption{Illustration of differentiable CircuitNN. CircuitNN is designed based on differentiable NAND gates. After DAS is guided by PI and PO pairs of the truth table, CircuitNN can get the precise circuit architecture logic equivalent to the truth table.}
    \label{fig:circuitnn}
\end{figure*}

% 1. Describe the importance of logic synthesis
% 2. Existing Problems
% (a) Neural Architecture Search: Unstable, Predefined Setting, etc.
% (b) Circuit Generation: Probabilistic Model, Logic Equivalence

With the rapid advancement of technology, the scale of integrated circuits (ICs) has expanded exponentially. 
This expansion has introduced significant challenges in chip manufacturing, particularly concerning power and area metrics.
A primary objective in IC design is achieving the same circuit function with fewer transistors, thereby reducing power usage and area occupancy.

Logic synthesis~\cite{hachtel2005logicsynth}, a critical step in electronic design automation (EDA), transforms behavioral-level circuit designs into optimized gate-level circuits, ultimately yielding the final IC layout. 
The primary goal of logic synthesis is to identify the physical implementation with the fewest gates for a given circuit function. 
This task constitutes a challenging NP-hard combinatorial optimization problem. 
Current logic synthesis tools~\cite{brayton2010abc, wolf2013yosys} rely on human-designed heuristics, often leading to sub-optimal outcomes.

Differentiable architecture search (DAS) techniques~\cite{liu2018darts, chu2020darts} offer novel perspectives on addressing challenges in this problem.
Circuit functions can be represented through truth tables, which map binary inputs to their corresponding outputs. 
Truth tables provide a precise representation of input-output relationships, ensuring the design of functionally equivalent circuits.
Inspired by this, researchers~\cite{deepmind2024ai4sys, wang2024tnet} have begun exploring the application of DAS to synthesize circuits directly from truth tables.
Specifically, \citet{deepmind2024ai4sys} proposed CircuitNN, a framework that learns differentiable connection structures with logic gates, enabling the automatic generation of logic circuits from truth tables.
This approach significantly reduces the complexity of traditional circuit generation. 
Building on this, \citet{wang2024tnet} introduced T-Net, a triangle-shaped variant of CircuitNN, incorporating regularization techniques to enhance the efficiency of DAS.

Despite these advancements, several challenges remain. 
The computational complexity of DAS grows quadratically with the number of gates, posing scalability issues.
Although triangle-shaped architecture~\cite{wang2024tnet} partially mitigates this problem, redundancy persists. 
%Additionally, DAS is susceptible to converging to local optima, limiting the ability to search architectures that satisfy the given truth tables~\cite{liu2018darts}. 
%Furthermore, hyperparameters (network depth and layer width) require extensive searches, introducing complexity and prolonging the synthesis process. 
Additionally, DAS is susceptible to converging to local optima~\cite{liu2018darts} and hyperparameters (network depth and layer width) require extensive searches. 
The challenges arise from the vast search space in DAS. 
% Even with predefined settings for CircuitNN, finding a configuration that meets the truth table requires extensive trial and error during the DAS process. 
Intuitively, limiting the search space through predefined parameters (network depth, gates per layer, and connection probabilities) can significantly reduce the complexity.

Recent advances~\cite{openai2023gpt4, abramson2024alphafold3, esser2024sd3, li2024mar} in conditional generative models have demonstrated remarkable performance across language, vision, and graph generation tasks. 
Motivated by these developments, we propose a novel approach to circuit generation that generates preliminary circuit structures to guide DAS in generating refined circuits matching specified truth tables. 
Firstly, we introduce CircuitVQ, a tokenizer with a discrete codebook for circuit tokenization. 
Built upon our Circuit AutoEncoder framework~\cite{hou2022graphmae,li2023maskgae,wu2025mgvga}, CircuitVQ is trained through a circuit reconstruction task. 
Specifically, the CircuitVQ encoder encodes input circuits into discrete tokens using a learnable codebook, while the decoder reconstructs the circuit adjacency matrix based on these tokens.
Subsequently, the CircuitVQ encoder serves as a circuit tokenizer for CircuitAR pretraining, which employs a masked autoregressive modeling paradigm~\cite{chang2022maskgit, li2023mage}. 
In this process, the discrete codes function as supervision signals. 
After training, CircuitAR can generate discrete tokens progressively, which can be decoded into initial circuit structures by the decoder of the CircuitVQ. 
These prior insights can guide DAS in producing refined circuits that match the target truth tables precisely.

Our key contributions can be summarized as follows:
\begin{itemize}
\item We introduce CircuitVQ, a circuit tokenizer that facilitates graph autoregressive modeling for circuit generation, based on our Circuit AutoEncoder framework;
\item Develop CircuitAR, a model trained using masked autoregressive modeling, which generates initial circuit structures conditioned on given truth tables;
\item Propose a refinement framework that integrates differentiable architecture search to produce functionally equivalent circuits guided by target truth tables;
\item Comprehensive experiments demonstrating the scalability and capability emergence of our CircuitAR and the superior performance of the proposed circuit generation approach.
\end{itemize}

% Motivation
% (a) Diffusion (Vision, Graph), Autoregressive (Language, Vision)
% (b) Circuit Generation for Predefined Setting
% (c) Neural Architecture Search for Strict Logic Equivalence

% Contribution
% (a) Circuit Tokenizer (new transformer arch, training strategy)
% (b) CircuitAR (train and gen strategies, post-ar strategy)
% (c) Extensive Evaluation including BitD (Bit Distance) for Scalability


\section{Overview}\label{sec:overview}
\begin{figure*}[t]
\begin{center}
\includegraphics[width=.85\linewidth]{fig_overview_v3.pdf}
\end{center}
\caption{
FastAtlas Overview: In each frame, we compute charts spanning fully or partially visible triangles (a), determine texture space bounding boxes for the visible portions of the view-space projections of each chart, and tightly pack these boxes into atlases (b, here $2K \times 2K$). We simultaneously bijectively parameterize and shade the charts into their atlas boxes, obtaining high quality texture space shading (c), and use this shading to render the shaded frames (d).}
\label{fig:overview}
\label{fig:alg_overview}
\end{figure*}

\section{Overview}
\label{sec:overview}
Our work has two core contributions: a real-time, GPU-based algorithm for tight packing of general parameterized charts into compact atlases; and a real-time TSS method that
utilizes this packing.  

\paragraph*{FastAtlas Packing.}
FastAtlas runs entirely on the GPU as a series of compute shaders. It takes the bounding boxes of parameterized charts as input, and packs them into an atlas (Fig~\ref{fig:overview}b, Sec.~\ref{sec:pack}). As such, the only input it requires are the dimensions of the bounding boxes.
Its outputs are deterministic; identical input charts are packed into identical atlases. This is critical for TSS and similar applications, as it ensures that consecutive frames taken from the same camera view have the same shading. Even minute shading differences across such frames can cause sampling jitter, leading to undesirable flicker \cite{baker2012rock}. 
While prior methods such as \cite{mueller2018shading,hladky2019tessellated,hladky2021snakebinning,Neff2022MSA} cap the dimensions of the charts that can be packed as-is for a given atlas size, and scale down all charts that exceed these dimensions, we scale all charts by the same factor, and do so only when strictly necessary to achieve packing success (Figs~\ref{fig:atlas},~\ref{fig:sas_issues}). 

\paragraph*{TSS using FastAtlas.}
Our end-to-end TSS atlas generation method combines the packing method above with a novel approach for computing seamless per-frame charts. 
We define our charts as the connected components of the visible surfaces in each frame (Fig.~\ref{fig:overview}a), and efficiently compute them using a parallel union-find algorithm (Sec.~\ref{sec:visible}). Since the boundaries of these charts coincide with the contours of the rendered surface, they are {\em invisible} to the viewer. This approach 
eliminates the artifacts caused by shading discontinuities along visible seams (Fig.~\ref{fig:seams}). 

\begin{parWithWrapFigure}
\begin{wrapfigure}{l}{.27\columnwidth}%
\includegraphics[width=\linewidth]{fig_inset_view_plane.pdf}%
\end{wrapfigure}
We bijectively parametrize the {\em visible portions} of our charts by projecting them to view space (inset). This maps a constant number of texels to each pixel in the final rendered output, evenly distributing residual undersampling error across all image pixels. While conceptually straightforward, efficiently parameterizing charts containing partially visible triangles using viewspace projection is non-trivial, as the visible portions may no longer be triangular (e.g. green triangle in the inset); applying naive projection to triangles with vertices behind the camera may produce ill-posed results. Clipping triangles before projection is both computationally expensive and significantly complicates downstream operations. We avoid explicit clipping by observing that all that is required for atlas packing is the dimensions of, potentially conservative, bounding boxes of these projected visible portions. We compute such bounding boxes without explicit chart clipping by adapting a conservative screen coverage estimator \shortcite{Blinn:CalculatingScreenCoverage} (Sec.~\ref{sec:box}). We then pack the computed boxes using FastAtlas. 
\end{parWithWrapFigure}

Finally, we shade the visible portion of each chart into its corresponding atlas bounding box (Fig~\ref{fig:overview}c). 
The resulting texture is then used during rasterization as a standard texture map (Fig. ~\ref{fig:overview}d). 
Our framework is compatible with all existing approaches for texture space shading, including forward shading, raytraced illumination, or deferred shading in texture space \cite{baker:2016}. In the examples shown, we use the standard forward shading based rendering pipeline included in the G3D Innovation Engine \cite{G3D17}, a commercial grade renderer.


\section{History Expressions}\label{sec:history}
In this section, we introduce History Expressions and outline their semantics.

\begin{definition}[History Expression]
    A \textbf{\emph{History Expression}}, denoted $H$, is defined by the following grammar:
    \begin{equation}
        \begin{split}
            H, H' ::=& \;Val_H \;|\; Exp_H \;|\; H \cdot H' \;|\; H + H' \\
            Exp_{H} ::=& \;\alpha(\overline{b{:}\phi}) \;|\; F(\overline{a{:}(b{:}\phi})) \;|\; call(\phi; \;\overline{a{:}(b{:}\psi})) \;|\; \mu F(\overline{a{:}(b{:}\phi}))(H_F) \\
            Val_{H} ::=& \;\epsilon \;|\; \alpha(\overline{v}) \;|\; F(\overline{v}) \;|\; new_r(X) \;|\; get(F)
        \end{split}
        \label{eq:hist_gramm}
    \end{equation}
\end{definition}

The notation $\alpha(\overline{b{:}\phi})$ represents the application of the action named $\alpha$ on a resource, operating over a set of values with base type $b$, specifically those that satisfy the predicate $\phi$. A similar notation is used for the application of an external function named $F$. In this case, each parameter (represented by the base type-qualifier pair) is linked to the corresponding argument in the function signature.

We first comment on expressions. The expression $call$ represents the operation of invoking external functions. However, this expression supports a form of call by property, meaning that the function being called is one that satisfies the predicate $\phi$. Lastly, the $\mu$ notation denotes the declaration of a recursive function, with $H_F$ representing its latent effect.

We move move to values. Values consists of the following:
    \begin{itemize}
    \item The empty history $\epsilon$, indicating the occurrence of no relevant action;
    \item The application of an external function on a set of values; 
    \item The creation of a new resource of type $r$ with the identifier $X$ bound to it; 
    \item The value $get(F)$, which is a constant bound to the API described by the name $F$.
\end{itemize}

Intuitively, the inclusion of type qualifiers in actions and API calls allows for a more refined over-approximation. This topic will be explored further when we present the type system in Section \ref{sec:type_system}.

We can introduce the substitution operator $[\cdot/\cdot]$ on identifiers in History Expressions. This will be mainly useful in local \emph{non} recursive functions, as in these the creation of resources will be allowed, but when the latent effect of a function becomes active it will be necessary to change all the identifiers of the resources created within them, because if there were another call to that same function, the same identifiers would be reused!

\newpage

The substitution $[\cdot/\cdot]$ is then defined as follows:

\begin{equation}
    \begin{split}
        (H \cdot H')[Y/X] =&\; H[Y/X] \cdot H'[Y/X] \\
        (H + H')[Y/X] =&\; H[Y/X] + H'[Y/X] \\
        \alpha(\overline{b_i{:}\phi_i})[Y/X] = &\; \alpha(\overline{b_i{:}\phi_i[Y/X]}) \\
        call(\phi; \;\overline{a_i{:}(b_i{:}\psi_i)})[Y/X] = &\; call(\phi[Y/X]; \;\overline{a_i{:}(b_i{:}\psi_i[Y/X])}) \\
        (\mu F(\overline{a_i{:}({b_i{:}\phi_i})})(H_F) =& \;\mu(F[Y/X])(\overline{a_i{:}({b_i{:}\phi_i[Y/X]})})(H_F[Y/X]) \\
        F(\overline{a_i{:}(b_i{:}\phi_i}))[Y/X] =& \; F[Y/X](\overline{a_i{:}(b_i{:}\phi_i[Y/X]})) \\
        \epsilon[Y/X] =& \;\epsilon \\
        new_r(Z)[Y/X] = &\; new_r(Z[Y/X]) \\
        \alpha(\overline{v})[Y/X] =& \;\alpha(\overline{v[Y/X]}) \\
        F(\overline{v})[Y/X] =& \;F[Y/X](\overline{v[Y/X]}) \\
        get(F)[Y/X] =& \;get(F[Y/X])
    \end{split}
    \label{eq:subst}
\end{equation}

The $\mu$ construct serves as a variable binder for recursive function identifiers, hence this requires defining the identity of History Expressions up-to alpha conversion.

\subsection{$\alpha$-conversion History Expression}\label{subsec:alpha}

We can define the following two rules for the $\alpha$-conversion of History Expressions:
\begin{enumerate}
    \item $H' \cdot \mu F(\overline{a{:}(b{:}\phi)})(H_F) \cdot H'' \;=_{\alpha}\; H' \cdot \mu G(\overline{a{:}(b{:}\phi)})(H_F[G/F]) \cdot H''[G/F]$ \\
    $\text{if} \quad H' \cdot \mu F(\overline{a{:}(b{:}\phi)})(H_F) \cdot H''$ is the longest History Expression containing $\mu F$ and \\
    $G \notin \textsc{Bound}(H') \cup \textsc{Bound}(H_F) \cup \textsc{Bound}(H'')$ where $\textsc{Bound}(H)$ represents all the identifiers used within $H$.
    \item $H' \cdot \mu F(\overline{a{:}(b{:}\phi)})(H_{rec}) \cdot H'' \cdot H''' \;=_{\alpha}\;  H' \cdot H'' \cdot \mu F(\overline{a{:}(b{:}\phi)})(H_{rec}) \cdot H'''$ \\
    $\text{if} \quad \forall \;call(\psi; \;\overline{c{:}(b{:}\theta)}) \in H'', \; \neg \psi[v \mapsto F] \;\land\; \forall\; G(\overline{d{:}(b{:}{\sigma})}) \in H'', \; G \neq F$
\end{enumerate}

Rule (1) allows the identifier of a recursive function to be changed if it has not been used globally in the history in which it is contained.
Rule (2), on the other hand, permits a recursive function declaration to be shifted to the right, provided that the subsequent history ($H''$) contains no API type calls to the function.

\subsection{Equality Relation}
We introduce an equational theory of History Expressions. The equality relation $=$ over History Expressions is the least congruence including $\alpha$-conversion such that:

\begin{figure}[ht]
    \begin{equation*}
        \begin{gathered}
            \epsilon \cdot H = H = H \cdot \epsilon \quad \quad H = H + H \quad \quad H + H' = H' + H \\
            H \cdot (H' \cdot H'') = (H \cdot H') \cdot H'' \quad \quad H + (H' + H'') = (H + H') + H'' \\
            H \cdot (H' + H'') = (H \cdot H') + (H \cdot H'') \quad \quad (H' + H'') \cdot H = (H' \cdot H) + (H'' \cdot H)
        \end{gathered}
    \end{equation*}
    \caption{Equality Axioms of History Expressions}
    \label{fig:hist_eq}
\end{figure}

\subsection{Denotation}

We now focus on the denotation of History Expressions. We start by noting that the management of resource creations within recursive functions and APIs will not be taken into account in the definition of denotation. These constraints will be introduced in Section \ref{sec:type_system}, where the type system is presented, but we will discuss the motivations at the end of this paragraph.

We first introduce an auxiliary function that allows a variable to be bound within the qualifiers of the history. The function, \bind, is inductively defined on the structure of $H$, and its definition is shown in Figure \ref{eq:bind}.

With $\Phi(\tau_x)$ we indicate that only the qualifier of the passed type is taken. In the definition of this function, we note that the discourse about type independence made earlier applies: when we bind a variable in the history, we do not care if the type associated with it is an under- or over-approximation, but only the qualifier is taken into account.

\begin{figure}[H]
    \begin{equation*}
        \begin{split}
            \bind(x{:}\tau_x, \;H \cdot H') =& \; \bind(x{:}\tau_x, \;H) \cdot \bind(x{:}\tau_x, \;H') \\
            \bind(x{:}\tau_x, \;H + H') =& \; \bind(x{:}\tau_x, \;H) + \bind(x{:}\tau_x, \;H') \\
            \bind(x{:}\tau_x, \;\alpha(\overline{b_i{:}\phi_i})) =& \; \alpha(\overline{b_i{:}\exists x. \Phi(\tau_x)[v \mapsto x] \;\land\; \phi_i}) \\
            \bind(x{:}\tau_x, \;call(\phi; \;\overline{a_i{:}(b_i{:}\psi_i}))) =& \; call(\exists x. \Phi(\tau_x) \;\land\; \\
            & \phi; \;\overline{a_i{:}(b_i{:}\exists x. \Phi(\tau_x)[v \mapsto x] \;\land\; \psi_i})) \\
            \bind(x{:}\tau_x, \;\mu F(\overline{a_i{:}({b_i{:}\phi_i})})(H_F)) =& \;\mu F(\overline{a_i{:}({b_i{:}\exists x. \Phi(\tau_x)[v \mapsto x] \;\land\; \phi_i})}) \\
            & (\bind(x{:}\tau_x, \;H_F)) \\
            \bind(x{:}\tau_x, \;F(\overline{a_i{:}(b_i{:}\phi_i)})) =& \;F(\overline{a_i{:}(b_i{:}\exists x.\phi_i)}) \\
            \bind(x{:}\tau_x, \;\epsilon) =& \;\epsilon \\
            \bind(x{:}\tau_x, \;new_r(X)) =& \; new_r(X) \\
            \bind(x{:}\tau_x, \;\alpha(\overline{v})) =& \;\alpha(\overline{v}) \\
            \bind(x{:}\tau_x, \;F(\overline{v_i})) =& \;F(\overline{v_i}) \\
            \bind(x{:}\tau_x, \;get(F)) =& \;get(F)
        \end{split}
        \end{equation*}
    \caption{Definition of function \bind\ which aims to bind a variable within the qualifiers in the history.}
    \label{eq:bind}
\end{figure}

Furthermore, it is possible to execute the \bind\ function on several types by introducing the following equality:

\begin{equation}
    \bind(x_1{:}\tau_1, \dots, x_n{:}\tau_n, \;H) \doteq \bind(x_1{:}\tau_1, \dots, x_{n-1}{:}\tau_{n-1}, \;\bind(x_n{:}\tau_n, \;H))
\end{equation}

Below we are going to present a number of definitions and notations that will be much used throughout the presentation of the paper.

\begin{notation}[$Rid$]
    With the set $Rid$, we are going to denote the set of all possible resource identifiers, both local and remote.
\end{notation}

\begin{definition}[Resource Context]
    The \emph{resource context} is defined as follows:
    \begin{equation*}
        \Delta \subset Rid \cup (Rid \mapsto \tau)
    \end{equation*}
    This set will play two roles:
    \begin{enumerate*}[label=(\roman*)]
        \item makes it possible to keep track of resource identifiers already in use, that is, those that are already associated with resources that have already been created, so that when a new resource is created (through the \verb|new|), it has a new identifier;
        \item the second function is to contain the association between the resource identifiers used to represent the API, and the signature, that is, the $\tau$ type of the external function.
    \end{enumerate*}
    Obviously, in normal contexts, at first $\Delta$ is populated only by the mapping $Rid \mapsto \tau$ for each API that will be available within the program.
\end{definition}

\begin{notation}[$\eta$]
    With the symbol $\eta$, we shall indicate a history that is no longer reducible, which can only be composed of concatenations of the values $Val_H$ presented in \ref{eq:hist_gramm}.
\end{notation}

\begin{notation}[$\uparrow$]
    The notation $\uparrow$, associated with a History Expression $H$, allows each $new_r(X)$ present in $H$ to be assigned the symbol $\uparrow$: this is necessary because when calculating the denotation of $H$ we will have to take into account, in a set, the identifiers of the resources on which the \verb|new| was actually called, and since this in our language is a terminal value, we use this special symbol to indicate that we must first add the identifier to the set and then stop with the reduction.
    For example, if $H$ were equal to:
    \begin{equation}
        (new_r(X) + new_r(Y)) \cdot \alpha(r: v = X \;\lor\; v = Y)
    \end{equation}
    Applying $\uparrow$ to $H$ would return:
    \begin{equation}
        (new_r(X)^{\uparrow} + new_r(Y)^{\uparrow}) \cdot \alpha(r: v = X \;\lor\; v = Y)
    \end{equation}
    In this way, having to choose between the creation of $X$ and that of $Y$, assuming we choose the former, the identifier $X$ will be inserted in our set, and then during the reduction of the expression $\alpha$ we will know that only $X$ and not $Y$ will have to be taken into account, thus generating only the value $\alpha(X)$.

    The symbol $\uparrow$ is also associated with any expression indicating a call of an external function, of the form $F(\overline{a_i{:}(b_i{:}\phi_i}))$: in this case, however, the semantics of the symbol will indicate that the latent effect of $F$ has not yet been inserted into the main history. 
    
    The application of the $\uparrow$ symbol does not take place within the effects of the recursive functions linked to the $\mu$ construct: $\uparrow$ will be applied each time the effect becomes active.
\end{notation}

\begin{definition}[History Expression's Denotation without a Context]\label{def:history_den}
    The denotation $\llbracket \cdot \rrbracket$ of an \emph{History Expression} $H$, in a context of resources $\Delta$, is the set of all histories $\eta$ such that $H^{\uparrow}$ is reduced (in one or more steps) to $\eta$, and $\eta$ is terminal, i.e. it is no longer reducible.
    \begin{equation}
        \llbracket H \rrbracket = \{ \eta \;|\; (\varnothing, \;\varnothing, \;H^{\uparrow}) \rightarrow^* (\Omega, \;\Upsilon, \;\eta) \;\land\; (\Omega, \;\Upsilon, \;\eta) \nrightarrow \}
    \end{equation}
\end{definition}

We note how the reduction consists of a triple of the form:
\begin{equation}
    (\Omega, \;\Upsilon, \;H)
\end{equation}
Where $\Omega$ will denote the set in which the identifiers of the resources actually created during the calculation of a single history $\eta$ will be held. While with $\Upsilon$ we shall denote a list of substitutions specified by the notation $\{\cdot/\cdot\}$ that concern only the identifiers of external functions that actually represent recursive functions. This list will be the construct that will allow true recursion, as it will allow each identifier representing a recursive call to be substituted for the behaviour of the recursive function itself.

The relation $\rightarrow$ for $H$ is defined in Figure \ref{fig:denot_history}. Rule No. 6 makes it possible to add to the set $\Omega$ the identifier of a resource that has just been created and which may be used in actions following its creation. The two rules below constraint that in a concatenation, the whole of the term on the left must first be reduced and then the term on the right. This is necessary because histories have a property of temporal ordering between actions and, referring back to the previous rule concerning the \verb|new|, in order to be able to use a resource it will be necessary for it to have first been created and then added to the $\Omega$ set.

\begin{figure}[H]
    \begin{equation*}
        \begin{gathered}
            (\Omega, \;\Upsilon, \;\epsilon) \nrightarrow \quad \quad 
            (\Omega, \;\Upsilon, \;\alpha(\overline{v})) \nrightarrow \quad \quad
            (\Omega, \;\Upsilon, \;get(F)) \nrightarrow \quad \quad
            (\Omega, \;\Upsilon, \;F(\overline{v})) \nrightarrow \\ \\
            (\Omega, \;\Upsilon, \;new_r(X)) \nrightarrow \quad \quad
            \frac{
                (\Omega, \;\Upsilon, \;\eta) \nrightarrow \quad (\Omega, \;\Upsilon, \;\eta') \nrightarrow
            }{
                (\Omega, \;\Upsilon, \;\eta \cdot \eta') \nrightarrow
            } \\ \\
            (\Omega, \;\Upsilon, \;new_r(X)^{\uparrow}) \rightarrow (\Omega \cup \{X\}, \;\Upsilon, \;new_r(X)) \\ \\
            \frac{
                (\Omega, \;\Upsilon, \;H) \rightarrow (\Omega', \;\Upsilon', \;H'')
            }{
                (\Omega, \;\Upsilon, \;H \cdot H') \rightarrow (\Omega', \;\Upsilon', \;H'' \cdot H')
            } \quad \quad
            \frac{
                (\Omega, \;\Upsilon, \;\eta) \nrightarrow \quad (\Omega, \;\Upsilon, \;H) \rightarrow (\Omega', \;\Upsilon', \;H')
            }{
                (\Omega, \;\Upsilon, \;\eta \cdot H) \rightarrow (\Omega', \;\Upsilon', \;\eta \cdot H')
            } \\ \\
            (\Omega, \;\Upsilon, \;H + H') \rightarrow (\Omega, \;\Upsilon, \;H) \quad \quad
            (\Omega, \;\Upsilon, \;H + H') \rightarrow (\Omega, \;\Upsilon, \;H') \\ \\
            \frac{
                \begin{gathered}
                    \forall i, Val_i = \{u: b_i \;|\; \phi_i[v \mapsto u] \;\land\; (u \in Rid \Longrightarrow u \in \Omega \;\lor\; \Delta(u)\downarrow)\} \\
                    H = \begin{cases}
                        \underset{\overline{u} \in \underset{i}{\prod} Val_i}{\bigoplus} \alpha(\overline{u}) & \text{if}\;\forall i, Val_i \neq \varnothing \\
                        \epsilon & \text{otherwise}
                    \end{cases}
                \end{gathered}
            }{
                (\Omega, \;\Upsilon, \;\alpha(\overline{b_i{:}\phi_i})) \rightarrow (\Omega, \;\Upsilon, \;H)
            } \\ \\
            \frac{
                Api = \{ F \;|\; \phi[v \mapsto F] \;\land\; \Delta(F)\downarrow \} \quad H = \begin{cases}
                    (\underset{F \in Api}{\bigoplus} F(\overline{a_i{:}(b_i{:}\psi_i}))^{\uparrow}) & \text{if}\;Api \neq \varnothing \\
                    \epsilon & \text{otherwise}
                \end{cases}
            }{
                (\Omega, \;\Upsilon, \;call(\phi; \;\overline{a_i{:}(b_i{:}\psi_i))}) \rightarrow (\Omega, \;\Upsilon, \;H)
            } \\ \\
            \frac{
                \Delta(F) = \overline{\tau} \rightarrow (\tau_F, \; H_F) \quad H_F^{\star} = \bind(\overline{a_i: \under{v: b_i \;|\; \psi_i}}, \;H_F)
            }{
                (\Omega, \;\Upsilon, \;F(\overline{a_i{:}(b_i{:}\psi_i}))^{\uparrow}) \rightarrow (\Omega, \;\Upsilon, \;(F\Upsilon)(\overline{a_i{:}(b_i{:}\psi_i})) \cdot {H_F^{\star}}^{\uparrow})
            } \\ \\
            \frac{
                \begin{gathered}
                    \forall i, Val_i = \{u: b_i \;|\; \phi_i[v \mapsto u] \;\land\; (u \in Rid \Longrightarrow u \in \Omega \;\lor\; \Delta(u)\downarrow)\} \\
                    H = \begin{cases}
                        \underset{\overline{u} \in \underset{i}{\prod} Val_i}{\bigoplus} \alpha(\overline{u}) & \text{if}\;\forall i, Val_i \neq \varnothing \\
                        \epsilon & \text{otherwise}
                    \end{cases}             
                \end{gathered}
            }{
                (\Omega, \;\Upsilon, \;F(\overline{a_i{:}(b_i{:}\phi_i)})) \rightarrow (\Omega, \;\Upsilon, \;H)
            } \\ \\
            (\Omega, \;\Upsilon, \;H(\overline{a_i{:}(b_i{:}\psi_i)}) \rightarrow (\Omega, \;\Upsilon, \;{\bind(\overline{a_i{:}(b_i{:}\psi_i)}, \;H)}^{\uparrow}) \\ \\
            (\Omega, \;\Upsilon, \;\mu F(\overline{a_i{:}({b_i{:}\phi_i})})(H_F)) \rightarrow (\Omega, \;\Upsilon\{H_F/F\}, \;\epsilon)
        \end{gathered}
    \end{equation*}
    \caption{Reduction Relation for History Expressions}
    \label{fig:denot_history}
\end{figure}

After the two rules for nondeterministic choice, we present the reduction rule for invoking an action on \textbf{all} those values satisfying each predicate $\phi_i$. We note how not only are put into a set, for each argument $i$ of the action, all those values of type $b_i$ that satisfy $\phi_i$; but in the case where the value under consideration is an identifier of a resource - and thus $b_i$ will be a resource type - this either must be present in $\Omega$ - it was created by means of a \verb|new| - or in the case where the identifier represents an external function, this must be defined in $\Delta$. Finally, if each parameter $i$ has associated at least one value on which to perform the action (so each $Val_i$ is different from the empty set), we create a vector $\overline{u}$ for each possible combination between the values of each parameter $i$ and place them in nondeterministic choice. Otherwise, if there is at least one parameter that does not have any of the admissible values associated with it, it means that the action within the programme can never occur (this is guaranteed by the type qualifiers, which represent a correct over-approximation for each parameter). The same mechanism was used for APIs call reduction as can be seen in rule No. 15.

The next rule allows the calls of a set of external functions invoked on the same parameters to be unpacked into as many non-deterministic choices as possible. The APIs taken into consideration, in addition to having to satisfy the predicate $\phi$, must also be defined in the context of the resources $\Delta$.

Below, we find the rule that allows the latent effect of an external function to be active: the effect is then taken from $\Delta$ and all the parameters passed to the call are bound to it. In the case where the identifier $F$ is in fact associated with a recursive function, one (and only one) of the substitutions in $\Gamma$ will be successful (and as we will see in the type system rules in Section \ref{sec:type_system} the latent effect associated with a recursive function in $\Delta$ will be $\epsilon$).

The last two rules handle recursion. The term to be reduced in the first of the two is a direct consequence of the reduction of external functions with no active latent effect: thus $H$ will be the latent effect of a recursive function, inserted thanks to one of the reductions in $\Gamma$, and will be consequently bound to the parameters on which the recursive function was called. The last rule simply adds the substitution between the latent effect and the identifier of the recursive function declared through the $\mu$ construct to the $\Gamma$ list.

\begin{definition}[History Expression's Denotation under a Context]
    The denotation of a history $H$, taking into account already having a context $\Gamma$ to bind the non-quantified variables in the action qualifiers, is inductively defined on $\Gamma$ as follows:
    \begin{equation}
        \begin{gathered}
            {\llbracket H \rrbracket}_{\varnothing} = \llbracket H \rrbracket \\
            {\llbracket H \rrbracket}_{x: \tau_x,\Gamma} = {\llbracket \bind(H, \;x{:}\tau_x) \rrbracket}_{\Gamma}
        \end{gathered}
    \end{equation}
\end{definition}

That is, for the empty context, the denotation is the same as the context-free one defined above since the variables will all be bound; whereas, for the inductive step, if the context consists of an association $x:\tau_x$ and another set of associations contained in $\Gamma$, we evaluate the denotation in $\Gamma$, but of the history in which all the qualifiers in the actions are bound existentially to a variable $x$ of type $\tau_x$.

\begin{theorem}[Correctness of Denotation to Type Qualifiers]
    Given a History Expression $H$ and an interpretation $\mathcal{I}$ that maps each qualifier in $H$ with the value assigned by the interpretation itself, it holds that:
    \begin{equation}
        \forall \eta, \;\Gamma, \;H. \;\eta \in \llbracket H \rrbracket_{\Gamma} \Longrightarrow \exists \mathcal{I}. \;\mathcal{I} \models \Phi(H) \;\land\; \eta \in \llbracket H(\mathcal{I}) \rrbracket_{\Gamma}
    \end{equation}
    This property tells us that for each terminal history $\eta$ that belongs to a History Expression $H$, there exists at least one interpretation $\mathcal{I}$ that satisfies the qualifiers taken into account by $H$ such that, by applying the substitution within $H$, $\eta$ belongs to the denotation of $H$ with inside instead of type qualifiers the values, surely correct since they belong to a valid interpretation. Consequently, the values in $\eta$ will also be correct with respect to the type qualifiers as this belongs to a denotation (the second) which will not use the reduction rules on the calculation of values, this because there are directly within $H$. 
    
    \paragraph{Interpretation} Since the qualifiers have no variable name or explicit identifier associated with them, we will use a number as an identifier in the definition of the interpretations, which will represent the index of the position of the qualifier itself within the History Expression with respect to the others, starting from left to right.
    
    Applying an interpretation $\mathcal{I} = \{\overline{i = v_i}\}$ to a History Expression $H$ replaces each qualifier indexed by $i$ with the expression $v = v_i$. This actually contradicts what was said earlier about the rules on calculating qualifiers not being used, because in the end $v = v_i$ is still a qualifier. In fact in reality these rules will be used, but the type qualifier becomes trivial, the only value that satisfies $v = v_i$ is $v_i$ itself!
    
    Finally, it should be pointed out that the only qualifiers that will not require mapping are those encapsulated within the $\mu$ recursion construct, since as already mentioned this is not used for the semantic purposes of History Expressions, but only to verify well-formedness in the type system rules as we shall see later.
\end{theorem}

\begin{proof}
    Having used inference rules for the definition of denotation, the demonstration can proceed by induction:
    \begin{itemize}
        \item For the first five rules, the demonstration is immediate as we are already operating on values that are not altered. For the next rule (concatenation of two terminal histories) the same.
        \item The event of creation of a resource is not of interest to us, as it does not encapsulate any type qualifiers, but directly presents a value (the identifier of the new resource) that is preserved in the reduction, and is added to the set $\Omega$. This last piece of information is important for some of the following points.
        \item For the following two concatenation rules we simply use the inductive hypothesis on the reduction of the History Expressions $H$ into $H''$ and $H'$ respectively, since they are in the premises. We conclude by stating that the property on the qualifiers is respected.
        \item On the other hand, for non-deterministic choice rules, here we are simply dropping a history so within the chosen interpretation $\mathcal{I}$ for the qualifiers in it can be assigned any value, still respecting the type qualifier, but independent of the terminal history $\eta$.
        \item Let us move on to the first notable rule, which concerns the reduction of the action $\alpha$. We must first differentiate between the case in which the type of values is a resource and the case in which it is not. In the latter case, what we compute in the premises is precisely the set of \textbf{all} values satisfying each predicate, and then we place \textbf{any} of these values (actually a vector of values in the case where the action has more than one parameter, but the transition to this case is w.l.g.) in non-deterministic choice. Proceeding in the reduction we will then have to choose only one of these values $v_i$ (because of the rules on the operator $+$) and this will belong to the terminal history $\eta$ (encapsulated in the event $\alpha$). Within the interpretation $\mathcal{I}$ we can then choose as the value for the qualifier at the position of $\alpha$ just $v_i$ since it satisfies it. For resource types, when we go to compute the set of values, we simply add another condition to the qualifier $\phi$ (which allows us to have identifiers that actually exist or APIs that have been defined), only restricting the starting set.

        In the case where the value generated is $\epsilon$, it will mean that there will be at least one qualifier that represents a contradiction, so there can be no value that satisfies it: as the value for the interpretation $\mathcal{I}$ we will choose the special symbol $\bot$, which in the semantics of the interpretation allows us to skip checking the qualifier in question.
        
        \paragraph{Note} In practice, when we will integrate History Expressions with the Coverage Types type system, the variables can never have a type with a contradictory qualifier associated with them, as this would lead to a context not being in good form \cite{coverage}. On the merits, such a qualifier (e.g. trivially $v = 1 \;\land\; v = 2$) is possible to find during the typing phase (just think of the function application of a parameter that does not respect the type of the argument), but since it will represent the presence of a type-matching error, it will lead to an error by failing to infer a type. \\

        It is worth noting, as this last discussion makes us reflect on why in the theorem the implication does not also hold the other way around, i.e. that for every valid interpretation there is a terminal history that belongs to the denotation: this is not possible with resource types in mind, in fact one could take in $\mathcal{I}$ identifiers that satisfy the type qualifier, but which have not been defined or created.
        \item For the remaining rules using qualifiers, such as the one on the single API call, the demonstration is identical, as the premises are the same. For the multiple call of APIs the calculation is also similar, as done above we restrict the set of values that satisfy the qualifier, in this case with those that also represent defined APIs. Basically, as a value for each qualifier in $\mathcal{I}$ we always take the one chosen during the reduction.
        \item Finally, for the last three rules that do not perform calculations on the qualifiers, we note how these, regardless of the operations they perform, which are irrelevant for the choice of values in $\mathcal{I}$, always leave the type qualifiers unchanged by only dragging them into other constructs or sets, such as $\Omega$ for recursion, or by binding them into other histories through \bind.
    \end{itemize}
\end{proof}

\begin{note}[Creation of resources in recursive functions and APIs]
    As previously announced, there will be constraints on the latent effects of APIs and recursive functions for which the creation of any type of resource within them will not be permitted. This implementation choice was made for two reasons:
    \begin{enumerate}[label=(\roman*)]
        \item The former is mainly related to recursive functions, and the reason for this is that unfolding a recursive function that at each iteration \emph{could} create new resources is dangerous: we do not know how many times the recursive call could be invoked, statically the number of invocations is potentially infinite! Despite the fact that in the type system of $\lambda^{\textbf{TG}}$ there is a control on recursion that imposes a decreasing ordering relation between the first parameter of the current function and the one present in each recursive call \cite{coverage}, however, the resources we have at our disposal are always limited in quantity, so even if the recursive function after $n$ calls terminates, this $n$ could be sufficiently large to create, for example, a disproportionate number of files, or network connections, realising, deliberately or otherwise, a \emph{DOS} \cite{dos} attack. The same goes for APIs: due to the principle of compositionality, an API could call other APIs within it, perhaps using recursive functions, and encounter the same problem.
        Let us imagine that we have the following recursive History Expression $H$ associated with any function\footnote{In our type system this syntax is \textbf{illegal} both because in Section \ref{sec:type_system} we will see that a recursive function with this history cannot be typed, and also because if we were to calculate the denotation of this history this \textbf{not} would be a correct over-approximation of the function.}:
        \begin{equation}
            H \equiv \mu F(n{:}(v > 0))(new_{file}(X) \cdot F(n{:}(v = n - 1)))
        \end{equation}
        Although, as already mentioned, the rules guarantee that the recursion is finite - in fact $n$ descends by $1$ at each iteration and when it is less than or equal to $0$ the recursion stops - from the point of view of resources and their limited availability this represents a problem. Consider a potential call of $F$ with $n$ equal to $10,000$: although this is feasible, it would lead to the creation of $10,000$ files, which could cause problems on the memory device!
        \item The second reason, is related to a \emph{ROP} (Return Oriented Programming) \cite{rop} attack, which a potential attacker could carry out on our code. Should this be able to find an entry point in the program to inject and execute malicious code, such as through \emph{buffer overflow} \cite{buffer}, by checking the stack it can manage and change the address of the return function, subsequently creating a fairly long chain of function calls, in which for instance it always returns to the same function, which may be the one that creates one or more resources. In spite of this, in order not to limit the expressiveness of the language too much, we will still allow the creation of resources within local non-recursive functions, while for recursive functions, or for API calls, instead of creating them directly within them, it will be possible to use the \verb|new| before their invocation and pass the desired resources as parameters.
    \end{enumerate}
\end{note}

\begin{proof}[Correctness of $\alpha$-conversion rules] After having introduced the denotation of History Expressions, we can demonstrate the two rules concerning $\alpha$-conversion presented above:
    \begin{itemize}
        \item For rule (1), it is necessary to make the assumption that the identifiers associated with the recursive functions \textbf{not} can be used explicitly within programmes, but must only be used by the compiler. This constraint will actually be present in our language and will be formally defined in the well-formedness rules of the type system in Section \ref{sec:type_system}. Thus, it is not possible for these identifiers to be present within type qualifiers, except in the qualifier of the expression $call$ and consequently also as an identifier in the single API call $F(\overline{a_i{:}(b_i{:}\phi_i)})$.
        The first part of the History Expression, $H'$, is identical. The second part, the recursive construct, although the identifier is changed from $F$ to $G$, this will be reduced to $\epsilon$. But there is the side effect that adds the substitution $\{H_F/F\}$ to the list $\Omega$. So according to the denotation in Figure \ref{fig:denot_history}, if $F$ in $H''$ is a recursive function, then the substitution in $\Omega$ will be applied by replacing the identifier $F$ with the History Expressions associated with the body of the recursive function, and in any terminal history $\eta$ it will not be possible to find any reference to $F$ in the values, due to the assumption made at the beginning. Consequently it is possible to change the identifier $F$ in all occurrences in $H''$ - and in the body of $H_F$ since it can become active in $H''$ through substitutions - to another $G$ such that it is globally fresh. For $H'$, however, if it is correct, there can be no reference to $F$.
        \item For rule (2), the demonstration is simple. Since the recursive construct is reduced to $\epsilon$, we will have that:
        \begin{equation}
            H' \cdot \epsilon \cdot H'' \cdot H''' = H' \cdot H'' \cdot \epsilon \cdot H'''
        \end{equation}
        According to the rules of equality in Figure \ref{fig:hist_eq}. The only denotation that might change, however, is only that of H‘’ since in its evaluation it will now no longer have in $\Omega$ the recursive function identified by $F$. But if there are no calls to $F$ in it, this substitution will never be used.
    \end{itemize}
\end{proof}

\section{Policies}\label{sec:policies}

In this section, we explore different approaches to defining policies and introduce two widely applicable relationships.

\begin{definition}[Ordering between events]
    The relation $<_{\eta}$ defines an ordering between two events in the terminal history $\eta$, The relation is defined as follows:
    \begin{equation}
        \alpha(X) <_{\eta} \beta(Y) \Longleftrightarrow \exists \; \eta_1, \eta_2, \eta_3. \; \eta = \eta_1 \cdot \alpha(X) \cdot \eta_2 \cdot \beta(Y) \cdot \eta_3
    \end{equation}
\end{definition}

\begin{definition}[Ownership of an event]
    The ownership relation of an event in a terminal history $\eta$ is defined by:

    \begin{equation}
        \alpha(X) \in \eta \Longleftrightarrow \exists \; \eta_1, \eta_2. \; \eta = \eta_1 \cdot \alpha(X) \cdot \eta_2
    \end{equation}
\end{definition}

\begin{example}
    For example, the policy, discussed at the beginning of the introduction on reading and writing files, is defined by making use of the relationships presented above:
    
    \begin{equation}
        \begin{gathered}
            \forall \; \eta \in \llbracket H \rrbracket .\; \forall \; read(X) \in \eta .\; \exists \; open(X) . \; open(X) <_{\eta} read(X) \;\land\; \\ 
            \nexists \; close(X) \in \eta .\; \; open(X) <_{\eta} close(X) <_{\eta} read(X)
        \end{gathered}
    \end{equation}

  One can use the same pattern to represent the same policy but on write operations (\verb|write|).
\end{example}

\section{Language \& May-Types with Must-Types}\label{sec:tuple_type}
\begin{figure*}[t]
\centering
\begin{center} 
  \includegraphics[width=.63\textwidth]{figures/fidelity_legend.pdf}
\end{center}
\begin{subfigure}[b]{.3\textwidth}
  \centering
  \includegraphics[width=\linewidth]{figures/sentiment_fidelity3_small.pdf}
  \caption{\emph{Sentiment} $n\in [32,63]$}
  \label{fig:fidelity_sentiment}
\end{subfigure}%
\begin{subfigure}[b]{.3\textwidth}
  \centering
  \includegraphics[width=\linewidth]{figures/drop_fidelity_small.pdf}
  \caption{\emph{DROP} $n\in [32,63]$}
  \label{fig:fidelity_drop}
\end{subfigure}%
\begin{subfigure}[b]{.3\textwidth}
  \centering
  \includegraphics[width=\linewidth]{figures/hotpot_fidelity_small.pdf}
  \caption{\emph{HotpotQA} $n\in [32,63]$}
  \label{fig:fidelity_hotpot}
\end{subfigure}
\begin{subfigure}[b]{.3\textwidth}
  \centering
  \includegraphics[width=\linewidth]{figures/sentiment_fidelity3_large.pdf}
  \caption{\emph{Sentiment} $n\in [64,127]$}
  \label{fig:fidelity_sentiment2}
\end{subfigure}%
\begin{subfigure}[b]{.3\textwidth}
  \centering
  \includegraphics[width=\linewidth]{figures/drop_fidelity_large.pdf}
  \caption{\emph{DROP} $n\in [64,127]$}
  \label{fig:fidelity_drop2}
\end{subfigure}%
\begin{subfigure}[b]{.3\textwidth}
  \centering
  \includegraphics[width=\linewidth]{figures/hotpot_fidelity_large.pdf}
  \caption{\emph{HotpotQA} $n\in [64,127]$}
  \label{fig:fidelity_hotpot2}
\end{subfigure}
\vspace{-5pt}
\caption{On the removal task, \SpecExp{} performs competitively with 2\textsuperscript{nd} order methods on the \emph{Sentiment} dataset, and out-performs all approaches on \emph{DROP} and  \emph{HotpotQA} dataset for $n \in [32,63]$. When $n$ is too large to compute other interaction indices, we outperform marginal methods.}
\label{fig:fidelity}
\vspace{-12pt}
\end{figure*}

\vspace{-14pt}
\section{Experiments}
\label{sec:language}

\paragraph{Datasets} 
We use three popular datasets that require the LLM to understand interactions between features. 
\begin{enumerate}[ topsep=0pt, itemsep=0pt, leftmargin=*]
\item \emph{Sentiment} is primarily composed of the \emph{Large Movie Review Dataset} \cite{maas-EtAl:2011:ACL-HLT2011}, which contains both positive and negative IMDb movie reviews. The dataset is augmented with examples from the \emph{SST} dataset \cite{ socher2013recursive} to ensure coverage for small $n$. We treat the words of the reviews as the input features.
\item{\emph{HotpotQA} \cite{yang2018hotpotqa} is a question-answering dataset requiring multi-hop reasoning over multiple Wikipedia articles to answer complex questions. We use the sentences of the articles as the input features.}
\item{\emph{Discrete Reasoning Over Paragraphs} (DROP)} \cite{dua2019drop} is a comprehension benchmark requiring discrete reasoning operations like addition, counting, and sorting over paragraph-level content to answer questions. We use the words of the paragraphs as the input features. 
\end{enumerate}
%
%\emph{DROP} and \emph{HotpotQA} require , while \emph{Sentiment} is encoder-only. 
%
\vspace{-7pt}
\paragraph{Models} For \textit{DROP} and \textit{HotpotQA}, (generative question-answering tasks) we use \texttt{Llama-3.2-3B-Instruct} \cite{grattafiori2024llama3herdmodels} with $8$-bit quantization. For \emph{Sentiment} (classification), we use the encoder-only fine-tuned \texttt{DistilBERT} model \cite{Sanh2019DistilBERTAD,sentimentBert}.
%

\vspace{-7pt}
\paragraph{Baselines} We compare against popular marginal metrics LIME, SHAP, and the Banzhaf value. 
%
For interaction indices, we consider Faith-Shapley, Faith-Banzhaf, and the Shapley-Taylor Index. We compute all benchmarks where computationally feasible. That is, we always compute marginal attributions and interaction indices when $n$ is sufficiently small. In figures, we only show the best performing baselines. Results and implementation details for all baselines can be found in 
Appendix~\ref{apdx:experiments}.

\vspace{-6pt}
\paragraph{Hyperparameters} \SpecExp{} has several parameters to determine the number of model inferences (masks). We choose $C=3$, informed by \citet{li2015spright} under a simplified sparse Fourier setting. We fix $t = 5$, which is the error correction capability of \SpecExp{} and serves as an approximate bound on the maximum degree. 
%
We also set $b=8$; the total collected samples are $\approx C2^bt \log(n)$. 
%
For $\ell_1$ regression-based interaction indices, we choose the regularization parameter via $5$-fold cross-validation. 




\vspace{-3pt}
\subsection{Metrics}


We compare \SpecExp{} to other methods across a variety of well-established metrics to assess performance.
%\Efe{How about textbf rather than emph here?}

\textbf{Faithfulness}: To characterize how well the surrogate function $\hat{f}$ approximates the true function, we define \emph{faithfulness} \cite{zhang2023trade}:
\vspace{-3pt}
\begin{equation}
    R^2 = 1 -  \frac{\lVert \hat{f} - f \rVert^2}{\left\lVert f - \bar{f} \right\rVert^2},
\end{equation}
where $\left\lVert f  \right\rVert^2 = \sum_{\bbm \in \bbF_2^n}f(\bbm)^2$ and $\bar{f} = \frac{1}{2^n} \sum_{\bbm \in \bbF_2^n}f(\bbm)$.

Faithfulness measures the ability of different explanation methods to predict model output when masking random inputs. 
%
We measure faithfulness over 10,000 random \emph{test} masks per-sample, and report average $R^2$ across samples. 
%

\textbf{Top-$r$ Removal}: We measure the ability of methods to identify the top $r$ influential features to model output:
\vspace{-2pt}
\begin{align}
\begin{split}
    \mathrm{Rem}(r) = \frac{|f(\boldsymbol{1}) - f(\bbm^*)|}{|f(\boldsymbol{1})|}, \\
    \;\bbm^* = \argmax \limits_{\abs{\bbm} = n-r}|\hat{f}(\boldsymbol{1}) - \hat{f}(\bbm)|.
\end{split}
\end{align}
\vspace{-8pt}


\textbf{Recovery Rate@$r$:} 
%
Each question in \emph{HotpotQA} contains human-labeled annotations for the sentences required to correctly answer the question. 
%
We measure the ability of interaction indices to recover these human-labeled annotations. 
%
Let $S_{r^*} \subseteq [n]$ denote human-annotated sentence indices. %corresponding to the human-annotated sentences containing the answer. 
Let $S_{i}$ denote feature indices of the $i^{\text{th}}$ most important interaction for a given interaction index.
%
Define the recovery ability at $r$ for each method as follows
\vspace{-2pt}
\begin{equation}
\label{eq:recovery_k}
    \text{Recovery@}r = 
    \frac{1}{r}\sum^r_{i=1}\frac{\abs{S_r^*\cap S_i}}{|S_{i}|}.
\end{equation}
\vspace{-8pt}

Intuitively, \eqref{eq:recovery_k} measures how well interaction indices capture features that align with human-labels.   


\begin{figure*}[t]
\centering
\hfill
\begin{subfigure}[b]{.5\textwidth}
  \centering
    \hspace{0.82cm}\includegraphics[width=0.75\textwidth]{figures/recall_legend.pdf}
  \includegraphics[width=.9\linewidth]{figures/hotpot_recall.pdf}
  \caption{Recovery rate$@10$ for \emph{HotpotQA} }
  \label{fig:recovery_hotpot}
\end{subfigure}%
\hfill % To ensure space between the figures
\begin{subfigure}[b]{.46\textwidth}
  \centering
    \includegraphics[width=1\textwidth]{figures/hotpot.pdf}
  \caption{Human-labeled interaction identified by \SpecExp{}.}
  \label{fig:hotpot_additional}
\end{subfigure}
\hfill
\caption{(a) \SpecExp{} recovers more human-labeled features with significantly fewer training masks as compared to other methods. (b) For a long-context example ($n = 128$ sentences), \SpecExp{} identifies the three human-labeled sentences as the most important third order interaction while ignoring unimportant contextual information.}
\vspace{-8pt}
\end{figure*}

\vspace{-8pt}
\subsection{Faithfulness and Runtime}
\vspace{-3pt}

Fig.~\ref{fig:faith} shows the faithfulness of \SpecExp{} compared to other methods. We also plot the runtime of all approaches for the \emph{Sentiment} dataset for different values of $n$. 
%
All attribution methods are learned over a fixed number of training masks.
% 

\textbf{Comparison to Interaction Indices } \SpecExp{} maintains competitive performance with the best-performing interaction indices across datasets. 
%
Recall these indices enumerate \emph{all possible interactions}, whereas \SpecExp{} does not. 
%
This difference is reflected in the runtimes of Fig.~\ref{fig:faith}(a).
%
The runtime of other interaction indices explodes as $n$ increases while \SpecExp{} does not suffer any increase in runtime. 

\vspace{-2pt}
\textbf{Comparison to Marginal Attributions } For input lengths $n$ too large to run interaction indices, \SpecExp{} is significantly more faithful than marginal attribution approaches across all three datasets.

\vspace{-2pt}
\textbf{Varying number of training masks } Results in Appendix ~\ref{apdx:experiments} show that \SpecExp{} continues to out-perform other approaches as we vary the number of training masks. 

\vspace{-2pt}
\textbf{Sparsity of \SpecExp{} Surrogate Function} Results in Appendix ~\ref{apdx:experiments}, Table~\ref{tab:faith} show 
surrogate functions learned by \SpecExp{} have Fourier representations where only $\sim 10^{-100}$ percent of coefficients are non-zero. 


\vspace{-6pt}
\subsection{Removal}
\label{subsec:removal}

Fig.~\ref{fig:fidelity} plots the change in model output as we mask the top $r$ features for different regimes of $n$. 
%

\vspace{-2pt}
\textbf{Small $n$ } \SpecExp{} is competitive with other interaction indices for \textit{Sentiment}, and out-performs them for \textit{HotpotQA} and \textit{DROP}. 
%
Performance of \SpecExp{} in this task is particularly notable since Shapley-based methods are designed to identify a small set of influential features. 
%
On the other hand, \SpecExp{} does not optimize for this metric, but instead learns the function $f(\cdot)$ over all possible $2^n$ masks. 
%

\textbf{Large $n$ } \SpecExp{} out-performs all marginal approaches, indicating the utility of considering interactions.
%

\vspace{-10pt}
\subsection{Recovery Rate of Human-Labeled Interactions}

%
We compare the recovery rate \eqref{eq:recovery_k} for $r = 10$ of \SpecExp{} against third order Faith-Banzhaf and Faith-Shap interaction indices. 
%
We choose third order interaction indices because all examples 
are answerable with information from at most three sentences, i.e., maximum degree $d = 3$.
%
Recovery rate is measured as we vary the number of training masks. 

Results are shown in Fig.~\ref{fig:recovery_hotpot}, where \SpecExp{} has the highest recovery rate of all interaction indices across all sample sizes. 
%
Further, \SpecExp{} achieves close to its maximum performance with few samples, other approaches require many more samples to approach the recovery rate of \SpecExp{}. 

\textbf{Example of Learned Interaction by \SpecExp{}} Fig.~\ref{fig:hotpot_additional} displays a long-context example (128 sentences) from \emph{HotpotQA} whose answer is contained in the three highlighted sentences. 
%
\SpecExp{} identifies the three human-labeled sentences as the most important third order interaction while ignoring unimportant contextual information. 
%
Other third order methods are not computable at this length. 
%

\begin{figure*}[t]
    \centering
    \includegraphics[width=0.9\linewidth]{figures/case_studies.pdf}
    \caption{SHAP provides marginal feature attributions. Feature interaction attributions computed by SPEX provide a more comprehensive understanding of (above) words interactions that cause the model to answer incorrectly and (below) interactions between image patches that informed the model's output.}
    \label{fig:caseStudies}
\end{figure*}



\section{A Static Type System for Resource Policies}\label{sec:type_system}
\section{Typing Rules}\label{app:type-system}

The following rules are in addition to those presented in Section \ref{sec:type_system}.

\begin{figure}[ht]
    \centering
    \noindent
    \begin{minipage}{0.3\textwidth}
        \begin{equation}
            \frac{
                \Gamma \vdash^{\textbf{WF}} (\text{Ty}(op), \; \epsilon)
            }{
                \Gamma \vdash op : (\text{Ty}(op), \; \epsilon)
            }
            \tag{\textsc{TOp}}
        \end{equation}
    \end{minipage}
    \hfill
    \begin{minipage}{0.6\textwidth}
        \begin{equation}
            \frac{
                \begin{gathered}
                    \Gamma \vdash e_x : (\tau_x, \; H_{e_x}) \quad \Gamma, x{:}\tau_x \vdash e : (\tau, \; H_e) \\
                    \Gamma \vdash^{\textbf{WF}} (\tau, \; H_{e_x} \cdot H_e)
                \end{gathered}  
            }{
                \Gamma \vdash \texttt{\small{let}}\;x = e_x\;\texttt{\small{in}}\;e : (\tau, \; H_{e_x} \cdot H_e)
            }
            \tag{\textsc{TLetE}}
        \end{equation}
    \end{minipage}

    \vspace{20pt}

    \begin{equation}
        \frac{
            \begin{gathered}
                \Gamma \vdash op : (\overline{a_i{:}\overa{v: b_i \;|\; \phi_i}} \rightarrow \tau_x, \; H_{op} ) \quad \forall i, \Gamma \vdash u_i : (\under{v: b_i \;|\; \phi_i}, \; H_{u_i}) \\
                \Gamma, x{:}\tau_x\overline{[a_i \mapsto u_i]} \vdash e : (\tau, \; H_e) \quad \Gamma \vdash^{\textbf{WF}} (\tau, \; H_{op} \cdot ( \underset{i}{\bullet} \; H_{u_i} ) \cdot H_e)
            \end{gathered}
        }{
            \Gamma \vdash \texttt{\small{let}}\;x = op\;\overline{u_i}\;\texttt{\small{in}}\;e : (\tau, \; H_{op} \cdot ( \underset{i}{\bullet} \; H_{u_i} ) \cdot H_e)
        }
        \tag{\textsc{TAppOp}}
    \end{equation}

    \vspace{20pt}

    \begin{equation}
        \frac{
            \begin{gathered}
                \neg(\kappa_x = \pi) \quad \Gamma \vdash v_1 : ((\tau_1 \rightarrow \kappa_1) \rightarrow \kappa_x, \; H_{v_1}) \quad \Gamma \vdash v_2: (\tau_1 \rightarrow \kappa_1, \; H_{v_2}) \\
                \Gamma, x : \kappa_x \vdash e: (\tau, \; H_e) \quad \Gamma \vdash^{\textbf{WF}} (\tau, \; H_{v_1} \cdot H_{v_2} \cdot H_e)
            \end{gathered}
        }{
            \Gamma \vdash \texttt{\small{let}}\;x = v_1\;v_2\;\texttt{\small{in}}\;e : (\tau, \; H_{v_1} \cdot H_{v_2} \cdot H_e)
        }
        \tag{\textsc{TAppFunMulti}}
    \end{equation}

    \vspace{20pt}

    \begin{equation}
        \frac{
            \begin{gathered}
                \Gamma \vdash v_1 : ((\tau_1 \rightarrow \kappa_1) \rightarrow (\tau_{v_1}, \; H_{\tau_{v_1}}), \; H_{v_1}) \quad \Gamma \vdash v_2: (\tau_1 \rightarrow \kappa_1, \; H_{v_2}) \\
                \Gamma, x : \kappa_x \vdash e: (\tau, \; H_e) \quad \Gamma \vdash^{\textbf{WF}} (\tau, \; H_{v_1} \cdot H_{v_2} \cdot H_{\tau_{v_1}} \cdot H_e)
            \end{gathered}
        }{
            \Gamma \vdash \texttt{\small{let}}\;x = v_1\;v_2\;\texttt{\small{in}}\;e : (\tau, \; H_{v_1} \cdot H_{v_2} \cdot H_{\tau_{v_1}} \cdot H_e)
        }
        \tag{\textsc{TAppFunLast}}
    \end{equation}
\end{figure}

\section{Algorithmic Constructions \& Properties}\label{sec:algorithm}
\section{The general case: Proof of \texorpdfstring{\Cref{thm:main-decomp}}{Theorem 1.6}}\label{sec:algo}

First, we show that data structure of \Cref{l:max_min_query} can be used to compute distances witnessed by shortest paths that pass through a constant-size separator.

\begin{lemma}\label{l:single_adhesion}
Fix a constant $k \in \mathbb{N}$. There exists an algorithm which as the input receives an edge-weighted graph $G$ on $n$ vertices and $m$ edges together with a partition of its vertices into three sets $A, B, C$ such that $|B| \leq k$ and there are no edges between $A$ and $C$, and as the output computes $\max_{c \in C} \dist(a, c)$ for every $a \in A$. The running time is $\Oh(m \log n + n \log^{k - 1} n)$.
\end{lemma}

\begin{proof}
Let $B = \{b_1, \ldots, b_k\}$. For any $a \in A, c \in C$, we have $\dist(a, c) = \min_{i \in [k]} \dist(a, b_i) + \dist(c, b_i)$. First, we run Dijkstra's algorithm from every vertex in $B$ to find $\dist(v, b_i)$ for every $v \in V(G)$ and $i \in [k]$. Next, we use \Cref{l:max_min_query} to construct a data structure $\mathbb{D}$ for the point set $\{(\dist(c, b_1), \dots, \dist(c, b_k))\colon c\in C\}\subseteq \mathbb{R}^k$. Now, the value $\max_{c \in C} \dist(a, c)$ for any given $a$ is equal to the answer of $\mathbb{D}$ to the query with argument $(\dist(a, b_1), \dots, \dist(a, b_k))$.
\end{proof}

After computing the distances over a constant-size separator, we will use the following observation to simplify one of the sides of the separation.

\begin{lemma}\label{l:inserting_paths}
Let $G$ be a edge-weighted connected graph and let $A, B, C$ be a partition of its vertices such that there are no edges between $A$ and $C$. For every pair of vertices $u, v \in B$, let $P_{u, v}$ be any shortest path from $u$ to $v$ with all internal vertices in $C$ (assuming such a path exists).

Let $G'$ denote a graph obtained from $G[A \cup B]$ by adding an edge from $u$ to $v$ of weight equal to the length of $P_{u, v}$, for all $u, v \in B$ for which $P_{u, v}$ exists. Then,  $$\dist_G(s, t) = \dist_{G'}(s, t)\qquad\textrm{for all }s,t\in A\cup B.$$
\end{lemma}
\begin{proof}
Let $G''$ be the graph obtained by adding new edges of $G'$ to $G$.
Fix any $s, t \in A \cup B$ and let $P$ denote the shortest path from $s$ to $t$ in $G''$ which minimizes the number of vertices from $C$ visited. Naturally, the weight of $P$ is equal $\dist_G(s, t)$. Assume that such path visits at least one vertex of $C$. Then, the path $P$ is of the form $s \xrightarrow{P_1} x \xrightarrow{P_2} y \xrightarrow{P_3} t$, where $x, y \in B$ and all the internal vertices of $P_2$ are in $C$. By the construction of $G'$, $P_2$ can be replaced with a direct edge from $x$ to $y$ of the same weight. We obtain a same weight path with a smaller number of vertices of $C$ visited, which is a contradiction. Therefore, $P$ is entirely contained in $A \cup B$, hence it exists in $G'$. This shows that $\dist_G(s, t) = \dist_{G'}(s, t)$.
\end{proof}


The next lemma encapsulates the main algorithmic content of the proof of \Cref{thm:main-decomp}. The algorithm will split the tree decomposition provided on input into smaller parts for which the eccentricities are easier to calculate. We use the following lemma to handle a single such part.
\begin{lemma}\label{l:star}
Fix constants $k, g \in \mathbb{N}, 0 < \delta < \frac{1}{54}$. Assume we are given $n \in \mathbb{N}$, an edge-weighted graph $G$ on at most $n$ vertices with a weight function $w \colon E(G) \to \mathbb{N}$, a vertex subset $A$ and a collection of non-empty vertex subsets $V_0, V_1, \dots, V_\ell$ satisfying the following conditions:
\begin{itemize}[nosep]
	\item The sum of weights of all the edges in $G$ is bounded by $\Oh(n)$.
	\item $V(G) \setminus A = V_0 \cup V_1 \cup \dots \cup V_\ell$.
	\item $|A| \leq k$.
	\item For every $i \in [\ell]$, $G[V_i \setminus V_0]$ is connected, $N_G(V_i \setminus V_0) = V_i \cap V_0$, $|V_i| = \Oh(n^\delta)$, and $|V_0 \cap V_i| \leq 4$.
	\item For all $i, j \in [\ell], i \neq j$, $V_i \setminus V_0$ and $V_j \setminus V_0$ are disjoint and non-adjacent in $G$.
	\item Every edge $uv \in E(G)$ with $u, v \not\in A$ is contained in $G[V_i]$ for some $i\in \{0,1,\ldots,\ell\}$.
	\item The graph obtained by taking $G[V_0]$ and adding a clique on $V_0 \cap V_i$ for every $i \in [\ell]$ has Euler genus bounded by $g$.
\end{itemize}
Then, we can compute the eccentricity of every vertex of $G$ in time $\Oh \left( n^{1 + \frac{150 + 54 \delta}{151}} \log^k n \right)$.
\end{lemma}

\begin{proof}
Fix $\delta' = \frac{1 + 97 \delta}{151}$; we have $\delta' - \delta = \frac{1 - 54\delta}{151} > 0$.
Let $E_i$ denote the set of edges with one endpoint in $V_i$ and the other endpoint in $V_i \setminus V_0$. For $i \in [\ell]$, we shall say that $V_i$ is {\em{heavy}} if the sum of weights of $E_i$ is larger than $n^{\delta'}$. Since the sets $E_i$ are pairwise disjoint and the total sum of weights of all the edges is bounded by $\Oh(n)$, the number of heavy subsets is bounded by $\Oh(n^{1 - \delta'})$. Without loss of generality, we may assume that $V_{\ell' + 1}, \dots, V_\ell$ are heavy and $V_1, \dots, V_{\ell'}$ are not, for some $\ell'\in \{0,\ldots,\ell\}$.


For any source vertex $s$, we can calculate distances from $s$ to every vertex of $G$  using breadth first search in time $\Oh(\sum_{e \in E(G)} w(e)) = \Oh(n)$.
In particular, for every $\ell' < i \leq \ell$, we can compute the distances from every vertex of $V_i$ to every vertex of $G$ in total time $\Oh(n^{2 - \delta' + \delta})$, because $$|V_{\ell'+1}\cup \ldots\cup V_{\ell}|\leq n^{1-\delta'}\cdot \Oh(n^\delta)=\Oh(n^{1-\delta'+
\delta}).$$
Additionally, we calculate distances $\dist_G(a, v)$ for every $a \in A, v \in V(G)$ in time $O(n)$.

For every $i \in [\ell]$ and $u,v \in V_0 \cap V_i$, there exists a shortest path $P_{i,u,v}$ from $u$ to $v$ with all internal vertices belonging to $V_i - V_0$ due to the assumption that $G[V_i - V_0]$ is connected and $N_G(V_i - V_0) = V_i \cap V_0$. Therefore, the distance from $u$ to $v$ is bounded by the sum of weights of edges in $E_i$. In particular, for $i \in [\ell']$, $\dist_G(u, v) \leq n^{\delta'}$.

We define $\widetilde{G}$ to be the graph obtained by taking $G[A \cup V_0 \cup \dots \cup V_{\ell'}]$ and applying the following operation for every $i \in \{\ell' + 1, \dots, \ell\}$:
for each pair of vertices $u, v \in A \cup (V_0 \cap V_i)$, add an edge in $\widetilde{G}$ between $u$ and $v$ with weight equal to the total weight of $P_{i,u,v}$. For a fixed $i, u$, we can find $P_{i, u, v}$ for all $v$ using breadth first search in time $\Oh(n)$. Taking a sum over all $i, u$, we get that $\tilde{G}$ can be computed in total time $\Oh(n^{2 - \delta'})$.


\begin{claim}\label{cl:wG}
The sum of the edge weights in $\widetilde{G}$ is $\Oh(n)$. Moreover, for all $u, v \in V(\widetilde{G})$, we have $\dist_{\widetilde{G}}(u, v) = \dist_{G}(u, v)$.
\end{claim}

\begin{proof}
Consider $i \in \{\ell' + 1, \dots, \ell\}$ and any $u, v \in A \cup (V_0 \cap V_i)$ for which we added an edge. Its weight is bounded by the sum of weights of edges in $E_i$. Therefore, the total weight of all edges added is at most
$$
\sum_{i \in \{\ell' + 1, \dots, \ell\}} \left( |A \cup (V_0 \cap V_i)|^2 \sum_{e \in E_i} w(e) \right) \leq (4 + k)^2 \sum_{e \in E(G)} w(e) = \Oh(n).
$$
This proves the first part of the claim.

For the second part of the claim, consider any $i \in \{\ell' + 1, \dots, \ell \}$ and observe that by our assumptions, $A \cup (V_0 \cap V_i)$ separates $(V_0 \cup \dots \cup V_{\ell'} \cup V_{i + 1} \cup \dots \cup V_\ell) \setminus V_i$ from $V_i \setminus V_0$. Hence it suffices to repeatedly apply \Cref{l:inserting_paths}.
\end{proof}

For every $u \in V(\widetilde{G})$, we have $\ecc_G(u) = \max(\ecc_{\widetilde{G}}(v), \max_{v \in V(G) \setminus V(\widetilde{G})} \dist_G(u, v))$. Note, that we already know all the distances $\dist_G(u, v)$ for $v \in V(G) \setminus V(\widetilde{G})$. Similarly, we can already compute $\ecc_G(u)$ for every $u \in V(G) \setminus V(\widetilde{G})$. Therefore, it remains to compute $\ecc_{\widetilde{G}}(v)$ for each $v \in V(\widetilde{G})$. Our goal is to show that this can be done efficiently using \Cref{l:main_ecc}.

Now, let $G'$ be the graph obtained from $\tilde{G}$ by replacing every edge $e$ non-indicent to $A$ with $w(e)\geq 2$ with a path of length $w(e)$ consisting of unit-weight edges. This operation again preserves the distances. Since the sum of edge weights in $\tilde{G}$ is of $\Oh(n)$, the total number of vertices in $G'$ is of $\Oh(n)$. For $0 \leq i \leq \ell'$, we write $V'_i$ to denote the set $V_i$ together with all the vertices added as a part of a path between two endpoints in $V_i$.
As $V_i$ is not heavy for each $i\in [\ell']$, we have
$$
|V'_i \setminus V'_0| \leq |V_i| + \sum_{e \in E_i} w(e) = \Oh(n^{\delta'})\qquad \textrm{for all }i\in [\ell'].
$$

Let $G_0$ denote the graph $G'[V'_0]$ and let $G_0^*$ denote the graph $G'- A$ with $V'_i - V'_0$ contracted to a single vertex $v_i^*$, for each $i \in [\ell']$; note that, all edges of $G_0$ and $G_0^*$ have unit weight.

\begin{claim}
	The graph $G_0^*$ is does not contain $K_{t}$ as a minor, where $t = \Oh(\sqrt{g})$.
\end{claim}

\begin{proof}
Let $\bar{G}_0$ denote the graph obtained by taking $G_0$ and adding a clique on $V_0 \cap V_i$ for every $i \in [\ell']$.
By lemma assumptions and the fact that subdividing edges does not increase the Euler genus, $\bar{G}_0$ has Euler genus at most $g$. In particular, $\bar{G}_0$ is $K_{t'}$-minor-free for some $t' = \Oh(\sqrt{g})$, because the Euler genus of $K_{t'}$ is $\Omega({t'}^2)$.

Similarly, let $\bar{G}_0^*$ be the graph obtained by taking $G_0^*$ and adding a clique on each $V_0 \cap V_i$.
Note, that $\bar{G}_0^* - \{v_1^*, \dots, v_{\ell'}^*\}$ is precisely $\bar{G}_0$. Let $t = \max(t', 6)$.
Recall that a minor model of a clique $K_t$ consists of $t$ pairwise vertex-disjoint connected subgraphs, called
branch sets, such that there is at least one edge between each pair of the branch sets.
Consider a minor model $\varphi$ of $K_{t}$ in $\bar{G}^*_0$.
Note that $\varphi$ cannot contain any singleton branch set of the form $\{v^*_i\}$, for the degree of $v^*_i$ in $\bar{G}^*_0$ is at most $4 < t - 1$. Furthermore, since $N_{\bar{G}^*_0}(v^*_i) = V_0 \cap V_i$, any branch set containing $v^*_i$ and at least one other vertex contains some $u \in V_0 \cap V_i$, and $N_{\bar{G}^*_0}(v^*_i)\subseteq N_{\bar{G}^*_0}(u)$, hence removing $v^*_i$ from this branch set preserves the model. Therefore, we can assume without loss of generality that all branch sets of $\varphi$ are disjoint from $\{v^*_1, \dots, v^*_{\ell'}\}$, hence $\varphi$ is a minor model of $K_{t}$ in $\bar{G}_0$. This is a contradiction, as $t \geq t'$ and $\bar{G}_0$ is $K_{t'}$-minor-free. Therefore, $\bar{G}_0^*$ is $K_t$-minor-free, hence $G_0^*$ also.
\end{proof}

Let $\rho' = \frac{2 - 108 \delta}{151} > 0$. The graph $G^*_0$ is a unit-weight graph and is $K_{t}$-minor-free.
Hence, by applying \Cref{t:r_division} to $G^*_0$ (with $\varepsilon = \rho'/2$)
we obtain an $n^{\rho'}$-division $\mathcal{R}_0$ in time $\Oh(n^{1 + \rho'})$.
We extend it to $G' - A$ by mapping every contracted vertex $v^*_i$ to $N_{G' - A}[V'_i - V'_0] = (V'_i - V'_0) \cup (V_0 \cap V_i)$. Formally, we put $V''_i \coloneqq N_{G' - A}[V'_i - V'_0]$ and 
$$
\mathcal{R} \coloneqq \left\{ (R_0 \cap V'_0) \cup \bigcup_{i \colon v^*_i \in R_0} V''_i \colon R_0 \in \mathcal{R}_0 \right\}.
$$

Now, we argue that $\mathcal{R}$ is a reasonable division of $G' - A$. Clearly, all sets $R \in \mathcal{R}$ are connected in $G' - A$. Pick any $R \in \mathcal{R}$ and let $R_0$ be its corresponding set in $\mathcal{R}_0$.
Every vertex $v^*_i$ is mapped to a set of size $\Oh(n^{\delta'})$, therefore
$$|R| \leq |R_0| \cdot \Oh(n^{\delta'}) = \Oh(n^{\rho' + \delta'}).$$

By our construction, for every $i \in [\ell']$, $R$ is either disjoint from $V'_i - V'_0$ or contains whole $N_{G' - A}[V'_i - V'_0]$. This means that no vertex belonging to any $V'_i - V'_0$ can be in $\partial R$, hence $\partial R \subseteq V'_0$.

Pick any $u \in \partial R \cap R_0$. Assume that $u \not\in \partial R_0$. Then every vertex of $N_{G_0^*}(u)$ must be in $R_0$, hence $N_{G - A'}(u) \subseteq R$, which is a contradiction. This means that $\partial R \cap R_0 \subseteq \partial R_0$.

Pick any $u \in \partial R - R_0$. Then, $u \in V_0 \cap V_i$ for some $i \in [\ell']$ such that $v_i^* \in R_0$. Moreover, $v_i^* \in \partial R_0$ and is adjacent to $u$ in $G_0^*$. The number of such $u$ is bounded by $4 |\partial R_0 \cap \{ v_1^*, \dots, v_{\ell'}^* \}|$.

Putting two cases together, we obtain:
$$
\sum_{R \in \mathcal{R}} |\partial R| = \sum_{R \in \mathcal{R}} \left(|\partial R \cap R_0| + |\partial R - R_0|\right) \leq \sum_{R_0 \in \mathcal{R}_0} \left(|\partial R_0| + 4 |\partial R_0 \cap \{ v_1^*, \dots, v_{\ell'}^* \}|\right) = \Oh(n^{1 - \frac{1}{2}\rho'}).
$$

It remains to show the following claim.

\begin{claim}
Pick any $R \in \mathcal{R}, s_R \in R$. The number of different distance profiles on $R$ relative to $s_R$ in $G' - A$ is of $\Oh(n^{48\rho' + 54\delta'})$.
\end{claim}
\begin{proof}
We look at every vertex $v \in V(G') \setminus A$ and consider three cases: $v \in R$, $v \in V'_0$, and $v \in V'_i \setminus (V'_0 \cup R)$ for some $i \in [\ell']$. By our construction, $R \cap V'_0$ is non-empty, hence w.l.o.g. we can assume that $s_R \in V'_0$ as whether two vertices have the same profile on $R$ is independent of the choice of the pivot vertex.

In the first case, there are at most $|R| = \Oh(n^{\rho' + \delta'})$ such vertices, hence they realise at most that many profiles.

In the second case, we want to observe that profile of any vertex $v \in V'_0$ on $R$ depends only on its profile on $R \cap V'_0$ (relative to $s_R$). Pick any $t \in R - V'_0$. Then $t \in V'_i - V'_0$ for some $i \in [\ell']$, $V_i \cap V_0 \subseteq R \cap V'_0$, and every path from $v$ to $t$ intersects $V_i \cap V_0$. In particular, distances from $v$ to vertices of $V_i \cap V_0$ determine its distance to $t$, which proves the observation.

Let $\tilde{G}_0$ denote the graph obtained by taking $G'[V'_0]$ and for every $i \in [\ell'], u, v \in V_0 \cap V_i$ adding a disjoint path from $u$ to $v$ of length $\dist(u, v)$. Let $P_i$ denote the vertex set of paths added between $V_0 \cap V_i$. For every $t \in V'_0$ we have $\dist_{G' - A}(v, t) = \dist_{\tilde{G}_0}(v, t)$, so it suffices to bound the number of profiles on $R \cap V'_0$ in $\tilde{G}_0$. By our assumptions, $\tilde{G}_0$ has Euler genus bounded by $g$ and all $P_i$ are of size $\Oh(n^{\delta'})$.

Let $R_0$ be the set of $\mathcal{R}_0$ corresponding to $R$. Let $\tilde{R}_0$ denote the set $(R \cap V'_0) \cup \bigcup_{i : v^*_i \in R_0} P_i$. Such set is connected in $\tilde{G}_0$. Moreover, similarly to $R$, its size is $\Oh(n^{\rho' + \delta'})$. Applying \Cref{thm:distprofiles}, we get that the number of distance profiles on $\tilde{R}_0$ in $\tilde{G}_0$ is $\Oh(n^{12(\rho' + \delta')})$, which also bounds the number of profiles on $R$ in $G' - A$ realised by $V'_0$.

For the third case, assume $v \in V'_i \setminus (V'_0 \cup R)$ for some $i\in [\ell']$. Every path from $v$ to any vertex of $R$ in $G' - A$ intersects $V_i \cap V_0$. Let $v_1, \dots v_p$ be the vertices of $V_i \cap V_0$, where $p \leq 4$. The profile of $v$ on $R$ is then determined by the following:
\begin{itemize}[nosep]
 \item[(a)] the profile of each $v_j$ on $R$,
 \item[(b)] $\dist_{G' - A}(v, v_j) - \dist_{G' - A}(v, v_1)$ for each $2 \leq j \leq p$, and
 \item[(c)] $\dist_{G' - A}(s_R, v_j) - \dist_{G' - A}(s_R, v_1)$ for each $2 \leq j \leq p$ where $s_R$ is some pivot vertex of $R$.
\end{itemize}
By the previous case, the number of distance profiles of each $v_j$ is $\Oh(n^{12(\rho' + \delta')})$. The distances between $v$ and $v_j$ are bounded by $|V'_i|$, hence each quantity described in (b) can take $\Oh(n^{\delta'})$ different possible values. Similarly, since $v_1$ and $v_j$ are connected via $V'_i$, $|\dist_{G' - A}(s_R, v_j) - \dist_{G' - A}(s_R, v_1)| \leq \Oh(n^{\delta'})$. The number of different possible profiles of such $v$ is therefore bounded by $\Oh(n^{48(\rho' + \delta') + 6\delta'}) = \Oh(n^{48\rho' + 54\delta'})$. This finishes the proof of the claim.
\end{proof}

Now we can apply \Cref{l:main_ecc} to graph $G'$ with apex set $A$, $X = V(\widetilde{G})$, and the following constants: $$\rho = \rho' + \delta',\qquad \gamma = 1 - \frac{1}{2}\rho',\quad \textrm{and}\quad \alpha = 48\rho' + 54 \delta'.$$ This allows us to calculate all $V(\widetilde{G})$-eccentricities in $G'$ in time
$$
\Oh \left( \left(
	n^{ 2 - \frac{1}{2} \rho' } +
	n^{ 1 + 48\rho' + 54 \delta' }
\right) \log^k n \right) =
\Oh \left( n^{1 + \frac{150 + 54 \delta}{151}} \log^k n \right).
$$
Since for each $v\in V(\widetilde{G})$ we have $\ecc_{\widetilde{G}}(v) = \max_{u \in V(\widetilde{G})} \dist_{\widetilde{G}}(v, u) = \max_{u \in V(\widetilde{G})} \dist_{G'}(v, u)$, this means that we have successfully computed all the eccentricities in $\widetilde{G}$; and as we argued, this is enough to compute all the eccentricities in $G$ as well.

Finally, the total running time of the algorithm is
$$
\Oh \left( n^{1 + \frac{150 + 54 \delta}{151}} \log^k n + n^{2 - \delta' + \delta} \right) =
\Oh \left( n^{1 + \frac{150 + 54 \delta}{151}} \log^k n \right).
$$\qedhere
\end{proof}


\begin{lemma}\label{l:star2}
Fix constants $k, g \in \mathbb{N}, 0 < \delta < \frac{1}{54}$. Assume we are given $n \in \mathbb{N}$, an edge-weighted graph $G$ on at most $n$ vertices with a weight function $w \colon E(G) \to \mathbb{N}$, a vertex subset $A$ and a collection of non-empty vertex subsets $V_0, V_1, \dots, V_\ell$ satisfying the same conditions as in \Cref{l:star} with the following differences:
\begin{itemize}
	\item we don't require $G[V_i - V_0]$ to be connected and $V_i - V_0$ to be adjacent to whole $V_i \cap V_0$;
	\item instead of $|V_0 \cap V_i| \leq 4$, we require $|V_0 \cap V_i| \leq k$.
\end{itemize}
Then, we can compute the eccentricity of every vertex of $G$ in time $\Oh \left( n^{1 + \frac{150 + 54 \delta}{151}} \log^{k + 5g} n \right)$.
\end{lemma}

\begin{proof}
We will reduce our input to one which will satisfy the conditions of \Cref{l:star}. We start by addressing the adhesions $V_0 \cap V_i$ containing too many vertices.

Let $G_0$ denote the graph $G[V_0]$ with cliques placed at $V_0 \cap V_i$ for every $i \in [\ell]$.
For every $i \in [\ell]$ we repeat the following procedure: while $|V_0 \cap V_i| > 4$,
remove arbitrary $5$ vertices from $V_0 \cap V_i$. Since $|V_0 \cap V_i| \leq k$ for each $i\in [\ell]$,
this procedure can be implemented in total time $\Oh(n)$. As a result, at the end we have $|V_0 \cap V_i| \leq 4$ for all $i \in [\ell]$. Let $M$ be the set of all the removed vertices. By our assumptions, $G_0$ has Euler genus bounded by $g$, hence it cannot contain $g + 1$ pairwise disjoint copies of $K_5$
(as the Euler genus of a graph is the sum of the Euler genera of its 2-connected components~\cite{StahlB77} and $K_5$ is not planar). Each removed quintiple of vertices induces a $K_5$ in $G_0$, hence we have $|M| \leq 5g$. We set $A' = A \cup M$ and may thus assume that $V_i$ is disjoint from $A'$ for all $0 \leq i \leq \ell$.

Now, fix $i \in [\ell]$. Let $C^i_1, \dots, C^i_{r_i}$ denote the connected components of $V_i - V_0$ in $G - A'$. We define $W^i_j := N_{G - A'}[C^i_j]$ for every $j \in [r_i]$. Clearly, all $W^i_j$ induce a connected subgraph of $G$ and satisfy $N_{G - A'}(W^i_j - V_0) = W^i_j \cap V_0$. We put $V'_0 := V_0$ and enumerate
$$
\{V'_1, V'_2, \dots V'_{\ell'}\} := \{ W^i_j \colon i \in [\ell], j \in [r_i] \}.
$$
It is easy to verify that the sets $A'$ and $V'_0, V'_1, \dots, V'_{\ell'}$ satisfy the conditions of \Cref{l:star}. We apply said lemma to calculate the eccentricity of every vertex of $G$ in the desired time.
\end{proof}



The next statement is a reformulation of \Cref{thm:main-decomp}.

\begin{theorem}
Fix constants $k, g \in \mathbb{N}$. Assume we are given a graph $G$ on $n$ vertices together with its tree decomposition $(T, \beta)$ and a set of private apices $A_t \subseteq \beta(t)$ for each node $t\in V(T)$ such that the following conditions hold:
\begin{itemize}[nosep]
 \item For every node $t \in V(T)$, we have $|A_t| \leq k$.
 \item For every edge $st \in E(T)$,  we have $|\beta(v) \cap \beta(u)|\leq k$.
 \item For every node $t \in V(T)$, graph obtained by taking $G[\beta(t)] - A_t$ and turning  $(\beta(t) \cap \beta(s))\setminus A_t$ into a clique for every edge $st \in E(T)$ has Euler genus bounded by $g$.
\end{itemize}
Then, we can compute the eccentricity of every vertex of $G$ in time $\Oh \left( n^{1 + \frac{355}{356}} \log^{k + 5g} n \right)$.
\end{theorem}

\begin{proof}
We may assume that $|V(T)|\leq n$, for every tree decomposition with no two bags comparable by inclusion has this property; and adjacent comparable bags can be merged by contracting the edge between them.

For a node $t\in V(T)$, by the {\em{weight}} of $t$ we mean the size of the corresponding bag, that is, $|\beta(t)|$. For any subset of nodes $S \subseteq V(T)$, we define $\beta(S) \coloneqq \bigcup_{t \in S} \beta(t)$ By the {\em{weight}} of $S$, we mean the total weight of the elements of $S$, that is, $\sum_{t\in S} |\beta(t)|$. 

\begin{claim}\label{cl:weight-T}
The weight of $V(T)$ is of $\Oh(n)$.
\end{claim}

\begin{proof}
The sets $\beta'(t) := \beta(t) - \bigcup_{s \in N_T(t)} \beta(s)$ are pairwise disjoint. We have
$$
\sum_{t \in V(T)} |\beta(t)| =
\sum_{t \in V(T)} |\beta'(t)| + 2 \cdot \sum_{st \in E(T)} |\beta(s) \cap \beta(t)| \leq
|V(T)| + 2k|E(T)| = \Oh(n).
$$
\end{proof}

Since every bag induces a graph of bounded Euler genus, the number of edges contained in a bag is linear in its size. In particular, this implies that the total number of edges of $G$ is also bounded by $\Oh(n)$.

We set $$\delta \coloneqq \frac{1}{356}\qquad\textrm{and}\qquad \Delta \coloneqq \frac{355}{356}.$$ Root the tree $T$ in an arbitrarily chosen node; this naturally imposes an ancestor-descendant relation in $T$ (for convenience, every node is considered its own ancestor and descendant).

We start by partitioning $T$ into connected subtrees using the following procedure.
We proceed bottom-up over $T$, processing nodes in any order so that a node is processed after all its strict descendants have been processed. Along the way, we mark some nodes and split the edges of $T$ into heavy and light. Let $t \in V(T)$ be the currently processed non-root node of $T$ and let $e \in E(T)$ be the edge connecting $t$ with its parent. If the total weight of all the unmarked nodes that are descendants of $t$ is at least $n^\delta$ (recall that this includes $t$ itself as well), then we declare $e$ heavy and mark all the descendants of $t$ that were unmarked so far. Otherwise, the edge $e$ is declared light and the procedure proceeds to further nodes of $T$.

Observe that
removing all heavy edges splits $T$ into connected subtrees, say $T'_1, \cdots T'_m$. All of the subtrees, except for possibly the subtree containing the root node, are of weight at least $n^\delta$. In particular, the number of subtrees $m$, and therefore the number of heavy edges, is  bounded by $\Oh(n^{1 - \delta})$. Moreover, in every subtree $T'_i$, removing the node closest to the root splits $T'_i$ into smaller components, each of weight less than $n^\delta$.

Fix a heavy edge $e$ and let $T^e_1$ and $T^e_2$ be the two subtrees into which $T$ splits after removing~$e$. Let $X^e_i = \beta(T^e_i)$ for $i \in \{1, 2\}$. Put $A_e = X^e_1 \setminus X^e_2$, $C_e = X^e_2 \setminus X^e_1$, and $B_e = X^e_1 \cap X^e_2$. By the properties of tree decompositions, such choice of $A_e, B_e, C_e$ satisfies the conditions of \Cref{l:single_adhesion}, hence in time $\Oh(n \log^{k - 1} n)$ we can compute $\max_{v \in X^e_2} \dist_G(u,v)$ for every $u \in X^e_1$, and $\max_{u \in X^e_1} \dist_G(u,v)$ for every $v \in X^e_2$. Computing this for every heavy edge $e$ takes total time $\Oh(n^{2 - \delta} \log^{k - 1} n)$.

Fix any subtree $T'=T'_j$. Let $e_1 = t^{e_1}_1t^{e_1}_2, e_2 = t^{e_2}_1 t^{e_2}_2, \dots, e_\ell = t^{e_\ell}_1 t^{e_\ell}_2$ denote the heavy edges incident to $T'$, where $t^{e_i}_1 \in V(T')$ and $V(T') \subseteq V(T_1^{e_i})$ for every $i \in [\ell]$.
For a vertex $v \in \beta(T')$, let
$$d_0(v) = \max_{u \in \beta(T')} \dist_G(v, u)\qquad\textrm{and}\qquad d_i(v) = \max_{u \in X_2^{e_i}}\dist_G(v,u),\quad\textrm{for } i \in [\ell].$$ We have $\ecc(v) = \max \{ d_i(v)\colon i\in \{0,1,\ldots,\ell\}\}$.The values of $d_i(v)$ are already calculated for all $i\in [\ell]$, hence it remains to compute $d_0(v)$.

For every $i \in [\ell]$ and every pair of vertices $u, v \in \beta(t^{e_i}_1) \cap \beta(t^{e_i}_2)$ we find a shortest path between $u$ and $v$ with all internal vertices inside $X^{e_i}_2$ (or determine that it doesn't exist). For a fixed $u, v$ this can be done in time $\Oh(n)$. Since in total we perform this step at most $2k^2$ times per heavy edge, it takes $\Oh(n^{2 - \delta})$ time in total. Let $P_{i, u, v}$ denote such path, assuming it exists.

Let $G'$ denote the graph obtained from $G[\beta(T')]$ by taking every $i, u, v$ for which $P_{i, u, v}$ exists and adding an edge between $u$ and $v$ of weight equal to the total weight of $P_{i, u, v}$.
The weight of every edge inserted in $\beta(t^{e_i}_1) \cap \beta(t^{e_i}_2)$ is bounded by $|X^{e_i}_2|+1$. The total weight of all edges inserted is therefore at most
$$
\sum_{i \in [\ell]} |\beta(t^{e_i}_1) \cap \beta(t^{e_i}_2)|^2 \cdot (|X^{e_i}_2|+1) \leq
k^2 \sum_{i \in [\ell]} (|X^{e_i}_2|+1) = \Oh(n),
$$
where the last equality follows from the fact that all the trees $T^{e_i}_2$ are pairwise disjoint.
By \Cref{l:inserting_paths}, we have $\dist_{G'}(u, v) = \dist_G(u, v)$ for each $u, v \in \beta(T')$. Hence, computing $d_0(v)$ for every $v \in \beta(T')$ is equivalent to computing the eccentricity of every vertex in $G'$.

If the size of $\beta(T')$ is smaller than $n^\Delta$, we compute the eccentricities naively in time $\Oh(|\beta(T')|^2)$, 
noting that $G'$ has $\Oh(|\beta(T')|)$ edges (thanks to Claim~\ref{cl:weight-T} and bounded genus assumption 
of the last bullet of the theorem statement). Otherwise, we argue that we can use the algorithm in \Cref{l:star} as follows.

Let $t$ be the node of $T'$ closest to the root. Let $s_1, \dots, s_p$ be the children of $t$ in $T$ and let $T''_i$ denote the connected component of $T' - \{t\}$ containing $s_i$. Set $V_0 = \beta(t)$ and $V_i = \beta(T''_i)$ for $i \in [p]$.

It is now easy to verify that $G'$ and sets $A, \{V_i\colon 0\leq i\leq p\}$ selected this way satisfy the assumptions of \Cref{l:star2}. This allows us to use it to compute the eccentricities in $G'$ in time
$$
\Oh \left( n^{1 + \frac{150 + 54\delta}{151}} \log^{k + 5g} n \right) =
\Oh \left( n^{1 + \frac{354}{356}} \log^{k + 5g} n \right).
$$
As we argued, from these eccentricities, we may easily compute all the eccentricities in $G$.

Now, let us analyse the total running time of the whole algorithm. We invoke \Cref{l:star} $\Oh(n^{1 - \Delta})$ times, since we apply it only to subtrees $T'_i$ of size at least $n^\Delta$. The total running time of those applications is hence
$$
\Oh \left( n^{2 + \frac{354}{356} - \Delta} \log^{k + 5g} n \right) =
\Oh \left( n^{1 + \frac{355}{356}} \log^{k + 5g} n \right).
$$
We compute the eccentricities naively for subtrees smaller than $n^\Delta$, hence the total running time of this computation is
$$
\sum_{i \in [m] \colon |\beta(T'_i)| \leq n^\Delta} |\beta(T'_i)|^2 \leq
n^\Delta \cdot \sum_{i \in m} |\beta(T'_i)| = \Oh(n^{1 + \Delta})=\Oh\left(n^{1+\frac{355}{356}}\right).
$$
The rest of computation can be done in $\Oh(n^{2 - \delta} \log^k n)$. Therefore, the whole algorithm runs in time $\Oh \left( n^{1 + \frac{355}{356}} \log^{k + 5g} n \right)$.
\end{proof}


\section{Discussion}\label{sec:discussion}
This work identifies signal collapse as a critical bottleneck in one-shot neural network pruning. Performance loss in pruned networks is due to \textbf{signal collapse} in addition to the removal of critical parameters. We propose \textbf{REFLOW} (\textbf{Re}storing \textbf{F}low of \textbf{Low}-variance signals), a simple yet effective method that mitigates signal collapse without computationally expensive weight updates. By focusing on signal preservation, REFLOW highlights the importance of mitigating signal collapse in sparse networks and enables magnitude pruning to match or surpass state-of-the-art one-shot pruning methods such as CHITA, CBS, and WF.

REFLOW consistently achieves state-of-the-art accuracy across diverse architectures, restoring ResNeXt-101 from under 4.1\% to 78.9\% top-1 accuracy at 80\% sparsity on ImageNet. Its lightweight design makes it a practical solution for both research and deployment, delivering high-quality sparse models without the overhead of traditional approaches. These findings challenge the traditional emphasis on weight selection strategies and underscore the critical role of signal propagation for achieving high-quality sparse networks in the context of one-shot pruning.




\section{Conclusion Remarks}\label{sec:conclusion}
\section*{Conclusion}
This paper aims to enhance our understanding of the computational complexity of computing various Shapley value variants. We found that for various ML models --- including decision trees, regression tree ensembles, weighted automata, and linear regression --- both local and global interventional and baseline SHAP can be computed in polynomial time under HMM modeled distributions. This extends popular algorithms, such as TreeSHAP, beyond their empirical distributional scope. We also establish strict complexity gaps between the various SHAP variants (baseline, interventional, and conditional) and prove the intractability of computing SHAP for tree ensembles and neural networks in simplified scenarios. Overall, we present SHAP as a versatile framework whose complexity depends on four key factors: \begin{inparaenum}[(i)] \item model type, \item SHAP variant, \item distribution modeling approach, \item and local vs. global explanations\end{inparaenum}. We believe this perspective provides deeper insight into the computational complexity of SHAP, paving the way for future work.




%We believe that our framework provides a more intricate understanding of SHAP computation complexity across different models, distributions, and variants, paving the way for further research.

Our work opens promising directions for future research. First, expanding our computational analysis to other SHAP-related metrics, such as asymmetric SHAP~\citep{frye20} and SAGE~\citep{covert2020understanding}, would be valuable. Additionally, we aim to explore more expressive distribution classes and relaxed assumptions beyond those in Section \ref{sec:tractable} while maintaining tractable SHAP computation. Finally, when exact computation is intractable (Section \ref{sec:intractable}), investigating the approximability of SHAP metrics through approximation and parameterized complexity theory~\citep{downey2012parameterized} is an important direction.

%Our work opens several promising avenues for future research on the computational properties of explainable AI methods, with a particular focus on SHAP. First, it would be interesting to broaden the computational analysis conducted in this work to include other popular SHAP-related metrics in the literature, such as asymmetric SHAP \cite{frye20} and SAGE \cite{covert2020understanding}. Also, in the future, we aim to explore more expressive distribution classes and relaxed distributional assumptions—extending beyond those examined in Section \ref{sec:tractable} —that still yield tractable SHAP computation. Finally, when exact computation proves intractable (Section \ref{sec:intractable}), it is worthwhile to theoretically investigate the question of the approximability of computing the SHAP metrics across various configurations, through the lens of approximation and parametrized complexity theory \cite{arora2009computational}.

%This paper aims to deepen our understanding of the computational complexity involved in obtaining different Shapley value variants. We found that for a variety of ML models, including decision trees, tree ensembles for regression, weighted automata, and linear regression models — computing both local and global interventional and baseline SHAP can be done in polynomial time when distributions are modeled by HMMs. This extends the distributional scope of popular algorithms like TreeSHAP, which is limited to empirical distributions. Additionally, we demonstrate a strict complexity gap between SHAP variants, showing that interventional and baseline SHAP can be strictly easier to compute than conditional SHAP. Despite these positive results, we uncovered intractability for various SHAP variants in neural networks and tree ensembles. Finally, we provided generalized complexity relations across SHAP variants. We believe that our framework offers a deeper understanding of the complexity involved in computing SHAP across various variants, models, distributions, as well as in both local and global computations, laying the groundwork for future research.

%%
%% The acknowledgments section is defined using the "acks" environment
%% (and NOT an unnumbered section). This ensures the proper
%% identification of the section in the article metadata, and the
%% consistent spelling of the heading.
%\begin{acks}
%\end{acks}

%%
%% The next two lines define the bibliography style to be used, and
%% the bibliography file.
\bibliographystyle{ACM-Reference-Format}
\bibliography{bibliography}

\newpage

\appendix

\section{Typing Rules}\label{app:type-system}

The following rules are in addition to those presented in Section \ref{sec:type_system}.

\begin{figure}[ht]
    \centering
    \noindent
    \begin{minipage}{0.3\textwidth}
        \begin{equation}
            \frac{
                \Gamma \vdash^{\textbf{WF}} (\text{Ty}(op), \; \epsilon)
            }{
                \Gamma \vdash op : (\text{Ty}(op), \; \epsilon)
            }
            \tag{\textsc{TOp}}
        \end{equation}
    \end{minipage}
    \hfill
    \begin{minipage}{0.6\textwidth}
        \begin{equation}
            \frac{
                \begin{gathered}
                    \Gamma \vdash e_x : (\tau_x, \; H_{e_x}) \quad \Gamma, x{:}\tau_x \vdash e : (\tau, \; H_e) \\
                    \Gamma \vdash^{\textbf{WF}} (\tau, \; H_{e_x} \cdot H_e)
                \end{gathered}  
            }{
                \Gamma \vdash \texttt{\small{let}}\;x = e_x\;\texttt{\small{in}}\;e : (\tau, \; H_{e_x} \cdot H_e)
            }
            \tag{\textsc{TLetE}}
        \end{equation}
    \end{minipage}

    \vspace{20pt}

    \begin{equation}
        \frac{
            \begin{gathered}
                \Gamma \vdash op : (\overline{a_i{:}\overa{v: b_i \;|\; \phi_i}} \rightarrow \tau_x, \; H_{op} ) \quad \forall i, \Gamma \vdash u_i : (\under{v: b_i \;|\; \phi_i}, \; H_{u_i}) \\
                \Gamma, x{:}\tau_x\overline{[a_i \mapsto u_i]} \vdash e : (\tau, \; H_e) \quad \Gamma \vdash^{\textbf{WF}} (\tau, \; H_{op} \cdot ( \underset{i}{\bullet} \; H_{u_i} ) \cdot H_e)
            \end{gathered}
        }{
            \Gamma \vdash \texttt{\small{let}}\;x = op\;\overline{u_i}\;\texttt{\small{in}}\;e : (\tau, \; H_{op} \cdot ( \underset{i}{\bullet} \; H_{u_i} ) \cdot H_e)
        }
        \tag{\textsc{TAppOp}}
    \end{equation}

    \vspace{20pt}

    \begin{equation}
        \frac{
            \begin{gathered}
                \neg(\kappa_x = \pi) \quad \Gamma \vdash v_1 : ((\tau_1 \rightarrow \kappa_1) \rightarrow \kappa_x, \; H_{v_1}) \quad \Gamma \vdash v_2: (\tau_1 \rightarrow \kappa_1, \; H_{v_2}) \\
                \Gamma, x : \kappa_x \vdash e: (\tau, \; H_e) \quad \Gamma \vdash^{\textbf{WF}} (\tau, \; H_{v_1} \cdot H_{v_2} \cdot H_e)
            \end{gathered}
        }{
            \Gamma \vdash \texttt{\small{let}}\;x = v_1\;v_2\;\texttt{\small{in}}\;e : (\tau, \; H_{v_1} \cdot H_{v_2} \cdot H_e)
        }
        \tag{\textsc{TAppFunMulti}}
    \end{equation}

    \vspace{20pt}

    \begin{equation}
        \frac{
            \begin{gathered}
                \Gamma \vdash v_1 : ((\tau_1 \rightarrow \kappa_1) \rightarrow (\tau_{v_1}, \; H_{\tau_{v_1}}), \; H_{v_1}) \quad \Gamma \vdash v_2: (\tau_1 \rightarrow \kappa_1, \; H_{v_2}) \\
                \Gamma, x : \kappa_x \vdash e: (\tau, \; H_e) \quad \Gamma \vdash^{\textbf{WF}} (\tau, \; H_{v_1} \cdot H_{v_2} \cdot H_{\tau_{v_1}} \cdot H_e)
            \end{gathered}
        }{
            \Gamma \vdash \texttt{\small{let}}\;x = v_1\;v_2\;\texttt{\small{in}}\;e : (\tau, \; H_{v_1} \cdot H_{v_2} \cdot H_{\tau_{v_1}} \cdot H_e)
        }
        \tag{\textsc{TAppFunLast}}
    \end{equation}
\end{figure}
\section{The general case: Proof of \texorpdfstring{\Cref{thm:main-decomp}}{Theorem 1.6}}\label{sec:algo}

First, we show that data structure of \Cref{l:max_min_query} can be used to compute distances witnessed by shortest paths that pass through a constant-size separator.

\begin{lemma}\label{l:single_adhesion}
Fix a constant $k \in \mathbb{N}$. There exists an algorithm which as the input receives an edge-weighted graph $G$ on $n$ vertices and $m$ edges together with a partition of its vertices into three sets $A, B, C$ such that $|B| \leq k$ and there are no edges between $A$ and $C$, and as the output computes $\max_{c \in C} \dist(a, c)$ for every $a \in A$. The running time is $\Oh(m \log n + n \log^{k - 1} n)$.
\end{lemma}

\begin{proof}
Let $B = \{b_1, \ldots, b_k\}$. For any $a \in A, c \in C$, we have $\dist(a, c) = \min_{i \in [k]} \dist(a, b_i) + \dist(c, b_i)$. First, we run Dijkstra's algorithm from every vertex in $B$ to find $\dist(v, b_i)$ for every $v \in V(G)$ and $i \in [k]$. Next, we use \Cref{l:max_min_query} to construct a data structure $\mathbb{D}$ for the point set $\{(\dist(c, b_1), \dots, \dist(c, b_k))\colon c\in C\}\subseteq \mathbb{R}^k$. Now, the value $\max_{c \in C} \dist(a, c)$ for any given $a$ is equal to the answer of $\mathbb{D}$ to the query with argument $(\dist(a, b_1), \dots, \dist(a, b_k))$.
\end{proof}

After computing the distances over a constant-size separator, we will use the following observation to simplify one of the sides of the separation.

\begin{lemma}\label{l:inserting_paths}
Let $G$ be a edge-weighted connected graph and let $A, B, C$ be a partition of its vertices such that there are no edges between $A$ and $C$. For every pair of vertices $u, v \in B$, let $P_{u, v}$ be any shortest path from $u$ to $v$ with all internal vertices in $C$ (assuming such a path exists).

Let $G'$ denote a graph obtained from $G[A \cup B]$ by adding an edge from $u$ to $v$ of weight equal to the length of $P_{u, v}$, for all $u, v \in B$ for which $P_{u, v}$ exists. Then,  $$\dist_G(s, t) = \dist_{G'}(s, t)\qquad\textrm{for all }s,t\in A\cup B.$$
\end{lemma}
\begin{proof}
Let $G''$ be the graph obtained by adding new edges of $G'$ to $G$.
Fix any $s, t \in A \cup B$ and let $P$ denote the shortest path from $s$ to $t$ in $G''$ which minimizes the number of vertices from $C$ visited. Naturally, the weight of $P$ is equal $\dist_G(s, t)$. Assume that such path visits at least one vertex of $C$. Then, the path $P$ is of the form $s \xrightarrow{P_1} x \xrightarrow{P_2} y \xrightarrow{P_3} t$, where $x, y \in B$ and all the internal vertices of $P_2$ are in $C$. By the construction of $G'$, $P_2$ can be replaced with a direct edge from $x$ to $y$ of the same weight. We obtain a same weight path with a smaller number of vertices of $C$ visited, which is a contradiction. Therefore, $P$ is entirely contained in $A \cup B$, hence it exists in $G'$. This shows that $\dist_G(s, t) = \dist_{G'}(s, t)$.
\end{proof}


The next lemma encapsulates the main algorithmic content of the proof of \Cref{thm:main-decomp}. The algorithm will split the tree decomposition provided on input into smaller parts for which the eccentricities are easier to calculate. We use the following lemma to handle a single such part.
\begin{lemma}\label{l:star}
Fix constants $k, g \in \mathbb{N}, 0 < \delta < \frac{1}{54}$. Assume we are given $n \in \mathbb{N}$, an edge-weighted graph $G$ on at most $n$ vertices with a weight function $w \colon E(G) \to \mathbb{N}$, a vertex subset $A$ and a collection of non-empty vertex subsets $V_0, V_1, \dots, V_\ell$ satisfying the following conditions:
\begin{itemize}[nosep]
	\item The sum of weights of all the edges in $G$ is bounded by $\Oh(n)$.
	\item $V(G) \setminus A = V_0 \cup V_1 \cup \dots \cup V_\ell$.
	\item $|A| \leq k$.
	\item For every $i \in [\ell]$, $G[V_i \setminus V_0]$ is connected, $N_G(V_i \setminus V_0) = V_i \cap V_0$, $|V_i| = \Oh(n^\delta)$, and $|V_0 \cap V_i| \leq 4$.
	\item For all $i, j \in [\ell], i \neq j$, $V_i \setminus V_0$ and $V_j \setminus V_0$ are disjoint and non-adjacent in $G$.
	\item Every edge $uv \in E(G)$ with $u, v \not\in A$ is contained in $G[V_i]$ for some $i\in \{0,1,\ldots,\ell\}$.
	\item The graph obtained by taking $G[V_0]$ and adding a clique on $V_0 \cap V_i$ for every $i \in [\ell]$ has Euler genus bounded by $g$.
\end{itemize}
Then, we can compute the eccentricity of every vertex of $G$ in time $\Oh \left( n^{1 + \frac{150 + 54 \delta}{151}} \log^k n \right)$.
\end{lemma}

\begin{proof}
Fix $\delta' = \frac{1 + 97 \delta}{151}$; we have $\delta' - \delta = \frac{1 - 54\delta}{151} > 0$.
Let $E_i$ denote the set of edges with one endpoint in $V_i$ and the other endpoint in $V_i \setminus V_0$. For $i \in [\ell]$, we shall say that $V_i$ is {\em{heavy}} if the sum of weights of $E_i$ is larger than $n^{\delta'}$. Since the sets $E_i$ are pairwise disjoint and the total sum of weights of all the edges is bounded by $\Oh(n)$, the number of heavy subsets is bounded by $\Oh(n^{1 - \delta'})$. Without loss of generality, we may assume that $V_{\ell' + 1}, \dots, V_\ell$ are heavy and $V_1, \dots, V_{\ell'}$ are not, for some $\ell'\in \{0,\ldots,\ell\}$.


For any source vertex $s$, we can calculate distances from $s$ to every vertex of $G$  using breadth first search in time $\Oh(\sum_{e \in E(G)} w(e)) = \Oh(n)$.
In particular, for every $\ell' < i \leq \ell$, we can compute the distances from every vertex of $V_i$ to every vertex of $G$ in total time $\Oh(n^{2 - \delta' + \delta})$, because $$|V_{\ell'+1}\cup \ldots\cup V_{\ell}|\leq n^{1-\delta'}\cdot \Oh(n^\delta)=\Oh(n^{1-\delta'+
\delta}).$$
Additionally, we calculate distances $\dist_G(a, v)$ for every $a \in A, v \in V(G)$ in time $O(n)$.

For every $i \in [\ell]$ and $u,v \in V_0 \cap V_i$, there exists a shortest path $P_{i,u,v}$ from $u$ to $v$ with all internal vertices belonging to $V_i - V_0$ due to the assumption that $G[V_i - V_0]$ is connected and $N_G(V_i - V_0) = V_i \cap V_0$. Therefore, the distance from $u$ to $v$ is bounded by the sum of weights of edges in $E_i$. In particular, for $i \in [\ell']$, $\dist_G(u, v) \leq n^{\delta'}$.

We define $\widetilde{G}$ to be the graph obtained by taking $G[A \cup V_0 \cup \dots \cup V_{\ell'}]$ and applying the following operation for every $i \in \{\ell' + 1, \dots, \ell\}$:
for each pair of vertices $u, v \in A \cup (V_0 \cap V_i)$, add an edge in $\widetilde{G}$ between $u$ and $v$ with weight equal to the total weight of $P_{i,u,v}$. For a fixed $i, u$, we can find $P_{i, u, v}$ for all $v$ using breadth first search in time $\Oh(n)$. Taking a sum over all $i, u$, we get that $\tilde{G}$ can be computed in total time $\Oh(n^{2 - \delta'})$.


\begin{claim}\label{cl:wG}
The sum of the edge weights in $\widetilde{G}$ is $\Oh(n)$. Moreover, for all $u, v \in V(\widetilde{G})$, we have $\dist_{\widetilde{G}}(u, v) = \dist_{G}(u, v)$.
\end{claim}

\begin{proof}
Consider $i \in \{\ell' + 1, \dots, \ell\}$ and any $u, v \in A \cup (V_0 \cap V_i)$ for which we added an edge. Its weight is bounded by the sum of weights of edges in $E_i$. Therefore, the total weight of all edges added is at most
$$
\sum_{i \in \{\ell' + 1, \dots, \ell\}} \left( |A \cup (V_0 \cap V_i)|^2 \sum_{e \in E_i} w(e) \right) \leq (4 + k)^2 \sum_{e \in E(G)} w(e) = \Oh(n).
$$
This proves the first part of the claim.

For the second part of the claim, consider any $i \in \{\ell' + 1, \dots, \ell \}$ and observe that by our assumptions, $A \cup (V_0 \cap V_i)$ separates $(V_0 \cup \dots \cup V_{\ell'} \cup V_{i + 1} \cup \dots \cup V_\ell) \setminus V_i$ from $V_i \setminus V_0$. Hence it suffices to repeatedly apply \Cref{l:inserting_paths}.
\end{proof}

For every $u \in V(\widetilde{G})$, we have $\ecc_G(u) = \max(\ecc_{\widetilde{G}}(v), \max_{v \in V(G) \setminus V(\widetilde{G})} \dist_G(u, v))$. Note, that we already know all the distances $\dist_G(u, v)$ for $v \in V(G) \setminus V(\widetilde{G})$. Similarly, we can already compute $\ecc_G(u)$ for every $u \in V(G) \setminus V(\widetilde{G})$. Therefore, it remains to compute $\ecc_{\widetilde{G}}(v)$ for each $v \in V(\widetilde{G})$. Our goal is to show that this can be done efficiently using \Cref{l:main_ecc}.

Now, let $G'$ be the graph obtained from $\tilde{G}$ by replacing every edge $e$ non-indicent to $A$ with $w(e)\geq 2$ with a path of length $w(e)$ consisting of unit-weight edges. This operation again preserves the distances. Since the sum of edge weights in $\tilde{G}$ is of $\Oh(n)$, the total number of vertices in $G'$ is of $\Oh(n)$. For $0 \leq i \leq \ell'$, we write $V'_i$ to denote the set $V_i$ together with all the vertices added as a part of a path between two endpoints in $V_i$.
As $V_i$ is not heavy for each $i\in [\ell']$, we have
$$
|V'_i \setminus V'_0| \leq |V_i| + \sum_{e \in E_i} w(e) = \Oh(n^{\delta'})\qquad \textrm{for all }i\in [\ell'].
$$

Let $G_0$ denote the graph $G'[V'_0]$ and let $G_0^*$ denote the graph $G'- A$ with $V'_i - V'_0$ contracted to a single vertex $v_i^*$, for each $i \in [\ell']$; note that, all edges of $G_0$ and $G_0^*$ have unit weight.

\begin{claim}
	The graph $G_0^*$ is does not contain $K_{t}$ as a minor, where $t = \Oh(\sqrt{g})$.
\end{claim}

\begin{proof}
Let $\bar{G}_0$ denote the graph obtained by taking $G_0$ and adding a clique on $V_0 \cap V_i$ for every $i \in [\ell']$.
By lemma assumptions and the fact that subdividing edges does not increase the Euler genus, $\bar{G}_0$ has Euler genus at most $g$. In particular, $\bar{G}_0$ is $K_{t'}$-minor-free for some $t' = \Oh(\sqrt{g})$, because the Euler genus of $K_{t'}$ is $\Omega({t'}^2)$.

Similarly, let $\bar{G}_0^*$ be the graph obtained by taking $G_0^*$ and adding a clique on each $V_0 \cap V_i$.
Note, that $\bar{G}_0^* - \{v_1^*, \dots, v_{\ell'}^*\}$ is precisely $\bar{G}_0$. Let $t = \max(t', 6)$.
Recall that a minor model of a clique $K_t$ consists of $t$ pairwise vertex-disjoint connected subgraphs, called
branch sets, such that there is at least one edge between each pair of the branch sets.
Consider a minor model $\varphi$ of $K_{t}$ in $\bar{G}^*_0$.
Note that $\varphi$ cannot contain any singleton branch set of the form $\{v^*_i\}$, for the degree of $v^*_i$ in $\bar{G}^*_0$ is at most $4 < t - 1$. Furthermore, since $N_{\bar{G}^*_0}(v^*_i) = V_0 \cap V_i$, any branch set containing $v^*_i$ and at least one other vertex contains some $u \in V_0 \cap V_i$, and $N_{\bar{G}^*_0}(v^*_i)\subseteq N_{\bar{G}^*_0}(u)$, hence removing $v^*_i$ from this branch set preserves the model. Therefore, we can assume without loss of generality that all branch sets of $\varphi$ are disjoint from $\{v^*_1, \dots, v^*_{\ell'}\}$, hence $\varphi$ is a minor model of $K_{t}$ in $\bar{G}_0$. This is a contradiction, as $t \geq t'$ and $\bar{G}_0$ is $K_{t'}$-minor-free. Therefore, $\bar{G}_0^*$ is $K_t$-minor-free, hence $G_0^*$ also.
\end{proof}

Let $\rho' = \frac{2 - 108 \delta}{151} > 0$. The graph $G^*_0$ is a unit-weight graph and is $K_{t}$-minor-free.
Hence, by applying \Cref{t:r_division} to $G^*_0$ (with $\varepsilon = \rho'/2$)
we obtain an $n^{\rho'}$-division $\mathcal{R}_0$ in time $\Oh(n^{1 + \rho'})$.
We extend it to $G' - A$ by mapping every contracted vertex $v^*_i$ to $N_{G' - A}[V'_i - V'_0] = (V'_i - V'_0) \cup (V_0 \cap V_i)$. Formally, we put $V''_i \coloneqq N_{G' - A}[V'_i - V'_0]$ and 
$$
\mathcal{R} \coloneqq \left\{ (R_0 \cap V'_0) \cup \bigcup_{i \colon v^*_i \in R_0} V''_i \colon R_0 \in \mathcal{R}_0 \right\}.
$$

Now, we argue that $\mathcal{R}$ is a reasonable division of $G' - A$. Clearly, all sets $R \in \mathcal{R}$ are connected in $G' - A$. Pick any $R \in \mathcal{R}$ and let $R_0$ be its corresponding set in $\mathcal{R}_0$.
Every vertex $v^*_i$ is mapped to a set of size $\Oh(n^{\delta'})$, therefore
$$|R| \leq |R_0| \cdot \Oh(n^{\delta'}) = \Oh(n^{\rho' + \delta'}).$$

By our construction, for every $i \in [\ell']$, $R$ is either disjoint from $V'_i - V'_0$ or contains whole $N_{G' - A}[V'_i - V'_0]$. This means that no vertex belonging to any $V'_i - V'_0$ can be in $\partial R$, hence $\partial R \subseteq V'_0$.

Pick any $u \in \partial R \cap R_0$. Assume that $u \not\in \partial R_0$. Then every vertex of $N_{G_0^*}(u)$ must be in $R_0$, hence $N_{G - A'}(u) \subseteq R$, which is a contradiction. This means that $\partial R \cap R_0 \subseteq \partial R_0$.

Pick any $u \in \partial R - R_0$. Then, $u \in V_0 \cap V_i$ for some $i \in [\ell']$ such that $v_i^* \in R_0$. Moreover, $v_i^* \in \partial R_0$ and is adjacent to $u$ in $G_0^*$. The number of such $u$ is bounded by $4 |\partial R_0 \cap \{ v_1^*, \dots, v_{\ell'}^* \}|$.

Putting two cases together, we obtain:
$$
\sum_{R \in \mathcal{R}} |\partial R| = \sum_{R \in \mathcal{R}} \left(|\partial R \cap R_0| + |\partial R - R_0|\right) \leq \sum_{R_0 \in \mathcal{R}_0} \left(|\partial R_0| + 4 |\partial R_0 \cap \{ v_1^*, \dots, v_{\ell'}^* \}|\right) = \Oh(n^{1 - \frac{1}{2}\rho'}).
$$

It remains to show the following claim.

\begin{claim}
Pick any $R \in \mathcal{R}, s_R \in R$. The number of different distance profiles on $R$ relative to $s_R$ in $G' - A$ is of $\Oh(n^{48\rho' + 54\delta'})$.
\end{claim}
\begin{proof}
We look at every vertex $v \in V(G') \setminus A$ and consider three cases: $v \in R$, $v \in V'_0$, and $v \in V'_i \setminus (V'_0 \cup R)$ for some $i \in [\ell']$. By our construction, $R \cap V'_0$ is non-empty, hence w.l.o.g. we can assume that $s_R \in V'_0$ as whether two vertices have the same profile on $R$ is independent of the choice of the pivot vertex.

In the first case, there are at most $|R| = \Oh(n^{\rho' + \delta'})$ such vertices, hence they realise at most that many profiles.

In the second case, we want to observe that profile of any vertex $v \in V'_0$ on $R$ depends only on its profile on $R \cap V'_0$ (relative to $s_R$). Pick any $t \in R - V'_0$. Then $t \in V'_i - V'_0$ for some $i \in [\ell']$, $V_i \cap V_0 \subseteq R \cap V'_0$, and every path from $v$ to $t$ intersects $V_i \cap V_0$. In particular, distances from $v$ to vertices of $V_i \cap V_0$ determine its distance to $t$, which proves the observation.

Let $\tilde{G}_0$ denote the graph obtained by taking $G'[V'_0]$ and for every $i \in [\ell'], u, v \in V_0 \cap V_i$ adding a disjoint path from $u$ to $v$ of length $\dist(u, v)$. Let $P_i$ denote the vertex set of paths added between $V_0 \cap V_i$. For every $t \in V'_0$ we have $\dist_{G' - A}(v, t) = \dist_{\tilde{G}_0}(v, t)$, so it suffices to bound the number of profiles on $R \cap V'_0$ in $\tilde{G}_0$. By our assumptions, $\tilde{G}_0$ has Euler genus bounded by $g$ and all $P_i$ are of size $\Oh(n^{\delta'})$.

Let $R_0$ be the set of $\mathcal{R}_0$ corresponding to $R$. Let $\tilde{R}_0$ denote the set $(R \cap V'_0) \cup \bigcup_{i : v^*_i \in R_0} P_i$. Such set is connected in $\tilde{G}_0$. Moreover, similarly to $R$, its size is $\Oh(n^{\rho' + \delta'})$. Applying \Cref{thm:distprofiles}, we get that the number of distance profiles on $\tilde{R}_0$ in $\tilde{G}_0$ is $\Oh(n^{12(\rho' + \delta')})$, which also bounds the number of profiles on $R$ in $G' - A$ realised by $V'_0$.

For the third case, assume $v \in V'_i \setminus (V'_0 \cup R)$ for some $i\in [\ell']$. Every path from $v$ to any vertex of $R$ in $G' - A$ intersects $V_i \cap V_0$. Let $v_1, \dots v_p$ be the vertices of $V_i \cap V_0$, where $p \leq 4$. The profile of $v$ on $R$ is then determined by the following:
\begin{itemize}[nosep]
 \item[(a)] the profile of each $v_j$ on $R$,
 \item[(b)] $\dist_{G' - A}(v, v_j) - \dist_{G' - A}(v, v_1)$ for each $2 \leq j \leq p$, and
 \item[(c)] $\dist_{G' - A}(s_R, v_j) - \dist_{G' - A}(s_R, v_1)$ for each $2 \leq j \leq p$ where $s_R$ is some pivot vertex of $R$.
\end{itemize}
By the previous case, the number of distance profiles of each $v_j$ is $\Oh(n^{12(\rho' + \delta')})$. The distances between $v$ and $v_j$ are bounded by $|V'_i|$, hence each quantity described in (b) can take $\Oh(n^{\delta'})$ different possible values. Similarly, since $v_1$ and $v_j$ are connected via $V'_i$, $|\dist_{G' - A}(s_R, v_j) - \dist_{G' - A}(s_R, v_1)| \leq \Oh(n^{\delta'})$. The number of different possible profiles of such $v$ is therefore bounded by $\Oh(n^{48(\rho' + \delta') + 6\delta'}) = \Oh(n^{48\rho' + 54\delta'})$. This finishes the proof of the claim.
\end{proof}

Now we can apply \Cref{l:main_ecc} to graph $G'$ with apex set $A$, $X = V(\widetilde{G})$, and the following constants: $$\rho = \rho' + \delta',\qquad \gamma = 1 - \frac{1}{2}\rho',\quad \textrm{and}\quad \alpha = 48\rho' + 54 \delta'.$$ This allows us to calculate all $V(\widetilde{G})$-eccentricities in $G'$ in time
$$
\Oh \left( \left(
	n^{ 2 - \frac{1}{2} \rho' } +
	n^{ 1 + 48\rho' + 54 \delta' }
\right) \log^k n \right) =
\Oh \left( n^{1 + \frac{150 + 54 \delta}{151}} \log^k n \right).
$$
Since for each $v\in V(\widetilde{G})$ we have $\ecc_{\widetilde{G}}(v) = \max_{u \in V(\widetilde{G})} \dist_{\widetilde{G}}(v, u) = \max_{u \in V(\widetilde{G})} \dist_{G'}(v, u)$, this means that we have successfully computed all the eccentricities in $\widetilde{G}$; and as we argued, this is enough to compute all the eccentricities in $G$ as well.

Finally, the total running time of the algorithm is
$$
\Oh \left( n^{1 + \frac{150 + 54 \delta}{151}} \log^k n + n^{2 - \delta' + \delta} \right) =
\Oh \left( n^{1 + \frac{150 + 54 \delta}{151}} \log^k n \right).
$$\qedhere
\end{proof}


\begin{lemma}\label{l:star2}
Fix constants $k, g \in \mathbb{N}, 0 < \delta < \frac{1}{54}$. Assume we are given $n \in \mathbb{N}$, an edge-weighted graph $G$ on at most $n$ vertices with a weight function $w \colon E(G) \to \mathbb{N}$, a vertex subset $A$ and a collection of non-empty vertex subsets $V_0, V_1, \dots, V_\ell$ satisfying the same conditions as in \Cref{l:star} with the following differences:
\begin{itemize}
	\item we don't require $G[V_i - V_0]$ to be connected and $V_i - V_0$ to be adjacent to whole $V_i \cap V_0$;
	\item instead of $|V_0 \cap V_i| \leq 4$, we require $|V_0 \cap V_i| \leq k$.
\end{itemize}
Then, we can compute the eccentricity of every vertex of $G$ in time $\Oh \left( n^{1 + \frac{150 + 54 \delta}{151}} \log^{k + 5g} n \right)$.
\end{lemma}

\begin{proof}
We will reduce our input to one which will satisfy the conditions of \Cref{l:star}. We start by addressing the adhesions $V_0 \cap V_i$ containing too many vertices.

Let $G_0$ denote the graph $G[V_0]$ with cliques placed at $V_0 \cap V_i$ for every $i \in [\ell]$.
For every $i \in [\ell]$ we repeat the following procedure: while $|V_0 \cap V_i| > 4$,
remove arbitrary $5$ vertices from $V_0 \cap V_i$. Since $|V_0 \cap V_i| \leq k$ for each $i\in [\ell]$,
this procedure can be implemented in total time $\Oh(n)$. As a result, at the end we have $|V_0 \cap V_i| \leq 4$ for all $i \in [\ell]$. Let $M$ be the set of all the removed vertices. By our assumptions, $G_0$ has Euler genus bounded by $g$, hence it cannot contain $g + 1$ pairwise disjoint copies of $K_5$
(as the Euler genus of a graph is the sum of the Euler genera of its 2-connected components~\cite{StahlB77} and $K_5$ is not planar). Each removed quintiple of vertices induces a $K_5$ in $G_0$, hence we have $|M| \leq 5g$. We set $A' = A \cup M$ and may thus assume that $V_i$ is disjoint from $A'$ for all $0 \leq i \leq \ell$.

Now, fix $i \in [\ell]$. Let $C^i_1, \dots, C^i_{r_i}$ denote the connected components of $V_i - V_0$ in $G - A'$. We define $W^i_j := N_{G - A'}[C^i_j]$ for every $j \in [r_i]$. Clearly, all $W^i_j$ induce a connected subgraph of $G$ and satisfy $N_{G - A'}(W^i_j - V_0) = W^i_j \cap V_0$. We put $V'_0 := V_0$ and enumerate
$$
\{V'_1, V'_2, \dots V'_{\ell'}\} := \{ W^i_j \colon i \in [\ell], j \in [r_i] \}.
$$
It is easy to verify that the sets $A'$ and $V'_0, V'_1, \dots, V'_{\ell'}$ satisfy the conditions of \Cref{l:star}. We apply said lemma to calculate the eccentricity of every vertex of $G$ in the desired time.
\end{proof}



The next statement is a reformulation of \Cref{thm:main-decomp}.

\begin{theorem}
Fix constants $k, g \in \mathbb{N}$. Assume we are given a graph $G$ on $n$ vertices together with its tree decomposition $(T, \beta)$ and a set of private apices $A_t \subseteq \beta(t)$ for each node $t\in V(T)$ such that the following conditions hold:
\begin{itemize}[nosep]
 \item For every node $t \in V(T)$, we have $|A_t| \leq k$.
 \item For every edge $st \in E(T)$,  we have $|\beta(v) \cap \beta(u)|\leq k$.
 \item For every node $t \in V(T)$, graph obtained by taking $G[\beta(t)] - A_t$ and turning  $(\beta(t) \cap \beta(s))\setminus A_t$ into a clique for every edge $st \in E(T)$ has Euler genus bounded by $g$.
\end{itemize}
Then, we can compute the eccentricity of every vertex of $G$ in time $\Oh \left( n^{1 + \frac{355}{356}} \log^{k + 5g} n \right)$.
\end{theorem}

\begin{proof}
We may assume that $|V(T)|\leq n$, for every tree decomposition with no two bags comparable by inclusion has this property; and adjacent comparable bags can be merged by contracting the edge between them.

For a node $t\in V(T)$, by the {\em{weight}} of $t$ we mean the size of the corresponding bag, that is, $|\beta(t)|$. For any subset of nodes $S \subseteq V(T)$, we define $\beta(S) \coloneqq \bigcup_{t \in S} \beta(t)$ By the {\em{weight}} of $S$, we mean the total weight of the elements of $S$, that is, $\sum_{t\in S} |\beta(t)|$. 

\begin{claim}\label{cl:weight-T}
The weight of $V(T)$ is of $\Oh(n)$.
\end{claim}

\begin{proof}
The sets $\beta'(t) := \beta(t) - \bigcup_{s \in N_T(t)} \beta(s)$ are pairwise disjoint. We have
$$
\sum_{t \in V(T)} |\beta(t)| =
\sum_{t \in V(T)} |\beta'(t)| + 2 \cdot \sum_{st \in E(T)} |\beta(s) \cap \beta(t)| \leq
|V(T)| + 2k|E(T)| = \Oh(n).
$$
\end{proof}

Since every bag induces a graph of bounded Euler genus, the number of edges contained in a bag is linear in its size. In particular, this implies that the total number of edges of $G$ is also bounded by $\Oh(n)$.

We set $$\delta \coloneqq \frac{1}{356}\qquad\textrm{and}\qquad \Delta \coloneqq \frac{355}{356}.$$ Root the tree $T$ in an arbitrarily chosen node; this naturally imposes an ancestor-descendant relation in $T$ (for convenience, every node is considered its own ancestor and descendant).

We start by partitioning $T$ into connected subtrees using the following procedure.
We proceed bottom-up over $T$, processing nodes in any order so that a node is processed after all its strict descendants have been processed. Along the way, we mark some nodes and split the edges of $T$ into heavy and light. Let $t \in V(T)$ be the currently processed non-root node of $T$ and let $e \in E(T)$ be the edge connecting $t$ with its parent. If the total weight of all the unmarked nodes that are descendants of $t$ is at least $n^\delta$ (recall that this includes $t$ itself as well), then we declare $e$ heavy and mark all the descendants of $t$ that were unmarked so far. Otherwise, the edge $e$ is declared light and the procedure proceeds to further nodes of $T$.

Observe that
removing all heavy edges splits $T$ into connected subtrees, say $T'_1, \cdots T'_m$. All of the subtrees, except for possibly the subtree containing the root node, are of weight at least $n^\delta$. In particular, the number of subtrees $m$, and therefore the number of heavy edges, is  bounded by $\Oh(n^{1 - \delta})$. Moreover, in every subtree $T'_i$, removing the node closest to the root splits $T'_i$ into smaller components, each of weight less than $n^\delta$.

Fix a heavy edge $e$ and let $T^e_1$ and $T^e_2$ be the two subtrees into which $T$ splits after removing~$e$. Let $X^e_i = \beta(T^e_i)$ for $i \in \{1, 2\}$. Put $A_e = X^e_1 \setminus X^e_2$, $C_e = X^e_2 \setminus X^e_1$, and $B_e = X^e_1 \cap X^e_2$. By the properties of tree decompositions, such choice of $A_e, B_e, C_e$ satisfies the conditions of \Cref{l:single_adhesion}, hence in time $\Oh(n \log^{k - 1} n)$ we can compute $\max_{v \in X^e_2} \dist_G(u,v)$ for every $u \in X^e_1$, and $\max_{u \in X^e_1} \dist_G(u,v)$ for every $v \in X^e_2$. Computing this for every heavy edge $e$ takes total time $\Oh(n^{2 - \delta} \log^{k - 1} n)$.

Fix any subtree $T'=T'_j$. Let $e_1 = t^{e_1}_1t^{e_1}_2, e_2 = t^{e_2}_1 t^{e_2}_2, \dots, e_\ell = t^{e_\ell}_1 t^{e_\ell}_2$ denote the heavy edges incident to $T'$, where $t^{e_i}_1 \in V(T')$ and $V(T') \subseteq V(T_1^{e_i})$ for every $i \in [\ell]$.
For a vertex $v \in \beta(T')$, let
$$d_0(v) = \max_{u \in \beta(T')} \dist_G(v, u)\qquad\textrm{and}\qquad d_i(v) = \max_{u \in X_2^{e_i}}\dist_G(v,u),\quad\textrm{for } i \in [\ell].$$ We have $\ecc(v) = \max \{ d_i(v)\colon i\in \{0,1,\ldots,\ell\}\}$.The values of $d_i(v)$ are already calculated for all $i\in [\ell]$, hence it remains to compute $d_0(v)$.

For every $i \in [\ell]$ and every pair of vertices $u, v \in \beta(t^{e_i}_1) \cap \beta(t^{e_i}_2)$ we find a shortest path between $u$ and $v$ with all internal vertices inside $X^{e_i}_2$ (or determine that it doesn't exist). For a fixed $u, v$ this can be done in time $\Oh(n)$. Since in total we perform this step at most $2k^2$ times per heavy edge, it takes $\Oh(n^{2 - \delta})$ time in total. Let $P_{i, u, v}$ denote such path, assuming it exists.

Let $G'$ denote the graph obtained from $G[\beta(T')]$ by taking every $i, u, v$ for which $P_{i, u, v}$ exists and adding an edge between $u$ and $v$ of weight equal to the total weight of $P_{i, u, v}$.
The weight of every edge inserted in $\beta(t^{e_i}_1) \cap \beta(t^{e_i}_2)$ is bounded by $|X^{e_i}_2|+1$. The total weight of all edges inserted is therefore at most
$$
\sum_{i \in [\ell]} |\beta(t^{e_i}_1) \cap \beta(t^{e_i}_2)|^2 \cdot (|X^{e_i}_2|+1) \leq
k^2 \sum_{i \in [\ell]} (|X^{e_i}_2|+1) = \Oh(n),
$$
where the last equality follows from the fact that all the trees $T^{e_i}_2$ are pairwise disjoint.
By \Cref{l:inserting_paths}, we have $\dist_{G'}(u, v) = \dist_G(u, v)$ for each $u, v \in \beta(T')$. Hence, computing $d_0(v)$ for every $v \in \beta(T')$ is equivalent to computing the eccentricity of every vertex in $G'$.

If the size of $\beta(T')$ is smaller than $n^\Delta$, we compute the eccentricities naively in time $\Oh(|\beta(T')|^2)$, 
noting that $G'$ has $\Oh(|\beta(T')|)$ edges (thanks to Claim~\ref{cl:weight-T} and bounded genus assumption 
of the last bullet of the theorem statement). Otherwise, we argue that we can use the algorithm in \Cref{l:star} as follows.

Let $t$ be the node of $T'$ closest to the root. Let $s_1, \dots, s_p$ be the children of $t$ in $T$ and let $T''_i$ denote the connected component of $T' - \{t\}$ containing $s_i$. Set $V_0 = \beta(t)$ and $V_i = \beta(T''_i)$ for $i \in [p]$.

It is now easy to verify that $G'$ and sets $A, \{V_i\colon 0\leq i\leq p\}$ selected this way satisfy the assumptions of \Cref{l:star2}. This allows us to use it to compute the eccentricities in $G'$ in time
$$
\Oh \left( n^{1 + \frac{150 + 54\delta}{151}} \log^{k + 5g} n \right) =
\Oh \left( n^{1 + \frac{354}{356}} \log^{k + 5g} n \right).
$$
As we argued, from these eccentricities, we may easily compute all the eccentricities in $G$.

Now, let us analyse the total running time of the whole algorithm. We invoke \Cref{l:star} $\Oh(n^{1 - \Delta})$ times, since we apply it only to subtrees $T'_i$ of size at least $n^\Delta$. The total running time of those applications is hence
$$
\Oh \left( n^{2 + \frac{354}{356} - \Delta} \log^{k + 5g} n \right) =
\Oh \left( n^{1 + \frac{355}{356}} \log^{k + 5g} n \right).
$$
We compute the eccentricities naively for subtrees smaller than $n^\Delta$, hence the total running time of this computation is
$$
\sum_{i \in [m] \colon |\beta(T'_i)| \leq n^\Delta} |\beta(T'_i)|^2 \leq
n^\Delta \cdot \sum_{i \in m} |\beta(T'_i)| = \Oh(n^{1 + \Delta})=\Oh\left(n^{1+\frac{355}{356}}\right).
$$
The rest of computation can be done in $\Oh(n^{2 - \delta} \log^k n)$. Therefore, the whole algorithm runs in time $\Oh \left( n^{1 + \frac{355}{356}} \log^{k + 5g} n \right)$.
\end{proof}


\end{document}
\endinput
%%
%% End of file `sample-acmsmall.tex'.
