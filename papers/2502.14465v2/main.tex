%%
%% This is file `sample-acmsmall.tex',
%% generated with the docstrip utility.
%%
%% The original source files were:
%%
%% samples.dtx  (with options: `all,journal,bibtex,acmsmall')
%% 
%% IMPORTANT NOTICE:
%% 
%% For the copyright see the source file.
%% 
%% Any modified versions of this file must be renamed
%% with new filenames distinct from sample-acmsmall.tex.
%% 
%% For distribution of the original source see the terms
%% for copying and modification in the file samples.dtx.
%% 
%% This generated file may be distributed as long as the
%% original source files, as listed above, are part of the
%% same distribution. (The sources need not necessarily be
%% in the same archive or directory.)
%%
%%
%% Commands for TeXCount
%TC:macro \cite [option:text,text]
%TC:macro \citep [option:text,text]
%TC:macro \citet [option:text,text]
%TC:envir table 0 1
%TC:envir table* 0 1
%TC:envir tabular [ignore] word
%TC:envir displaymath 0 word
%TC:envir math 0 word
%TC:envir comment 0 0
%%
%%
%% The first command in your LaTeX source must be the \documentclass
%% command.
%%
%% For submission and review of your manuscript please change the
%% command to \documentclass[manuscript, screen, review]{acmart}.
%%
%% When submitting camera ready or to TAPS, please change the command
%% to \documentclass[sigconf]{acmart} or whichever template is required
%% for your publication.
%%
%%
\PassOptionsToPackage{dvipsnames}{xcolor}
\documentclass[acmsmall, nonacm, screen]{acmart}

\usepackage{listings}
\usepackage{comment}
\usepackage{wrapfig}
\usepackage{tabularx}
\usepackage[inline]{enumitem}
\usepackage{multicol}
\usepackage{algorithm}
\usepackage{algpseudocode}
\usepackage{mdframed}
\algnewcommand\Failure{\textbf{Failure}}
\algnewcommand\True{\textbf{True}}
\algnewcommand\False{\textbf{False}}
\algnewcommand\Land{\textbf{and }}
\algnewcommand\Or{\textbf{or }}
\algnewcommand\algorithmicsmatch{\textbf{match}}
\algnewcommand\algorithmiccase{\textbf{case}}
\algnewcommand\algorithmicdefault{\textbf{default}}
\algdef{SE}[MATCH]{Match}{EndMatch}[1]{\algorithmicsmatch\ #1:}{\algorithmicend\ \algorithmicsmatch}%
\algdef{SE}[CASE]{Case}{EndCase}[1]{\algorithmiccase\ #1 \ \algorithmicdo}{\algorithmicend\ \algorithmiccase}%
\algdef{SE}[DEFAULT]{Default}{EndCase}{\algorithmicdefault\ \algorithmicdo}{\algorithmicend\ \algorithmicdefault}%
\algtext*{EndMatch}%
\algtext*{EndCase}%

\definecolor{typecolor}{HTML}{B00040}
\definecolor{actioncolor}{rgb}{0.58, 0.0, 0.82}
\newcommand{\typekw}[1]{\textbf{\textcolor{typecolor}{#1}}}
\newcommand{\typeact}[1]{\textbf{\textcolor{actioncolor}{#1}}}
\newcommand{\bind}[0]{\textcolor{brown}{\textsc{Bind}}}
\newcommand{\subhist}[0]{\textcolor{brown}{\textsc{SubHist}}}
\newcommand{\subconc}[0]{\textcolor{brown}{\textsc{SubConc}}}
\newcommand{\histconj}[0]{\textcolor{brown}{\textsc{HistConj}}}
\newcommand{\overa}[1]{\textcolor{red}{\{}#1\textcolor{red}{\}}}
\newcommand{\under}[1]{\textcolor{blue}{[}#1\textcolor{blue}{]}}
\newcommand{\nextd}[0]{\textcolor{brown}{\textsc{new}}}
\newcommand{\bluet}[1]{\textcolor{blue}{#1}}
\newcommand{\oranget}[1]{\textcolor{orange}{#1}}
\newcommand{\greent}[1]{\textcolor{green}{#1}}

\theoremstyle{definition}
\newtheorem{definition}{Definition}[section]
\newtheorem{notation}{Notation}[section]
\newtheorem{example}{Example}[section]
\newtheorem{lemma}{Lemma}[section]
\newtheorem{theorem}{Theorem}[section]
\newtheorem{note}{Note}[section]

%%
%% \BibTeX command to typeset BibTeX logo in the docs
\AtBeginDocument{%
  \providecommand\BibTeX{{%
    Bib\TeX}}}

%% Rights management information.  This information is sent to you
%% when you complete the rights form.  These commands have SAMPLE
%% values in them; it is your responsibility as an author to replace
%% the commands and values with those provided to you when you
%% complete the rights form.
% commented
%\setcopyright{acmlicensed}
%\copyrightyear{2024}
%\acmYear{2024}
%\acmDOI{XXXXXXX.XXXXXXX}


%%
%% These commands are for a JOURNAL article.
% commented
%\acmJournal{JACM}
%\acmVolume{37}
%\acmNumber{4}
%\acmArticle{111}
%\acmMonth{8}

%%
%% Submission ID.
%% Use this when submitting an article to a sponsored event. You'll
%% receive a unique submission ID from the organizers
%% of the event, and this ID should be used as the parameter to this command.
% commented
%\acmSubmissionID{123-A56-BU3}

%%
%% For managing citations, it is recommended to use bibliography
%% files in BibTeX format.
%%
%% You can then either use BibTeX with the ACM-Reference-Format style,
%% or BibLaTeX with the acmnumeric or acmauthoryear sytles, that include
%% support for advanced citation of software artefact from the
%% biblatex-software package, also separately available on CTAN.
%%
%% Look at the sample-*-biblatex.tex files for templates showcasing
%% the biblatex styles.
%%

%%
%% The majority of ACM publications use numbered citations and
%% references.  The command \citestyle{authoryear} switches to the
%% "author year" style.
%%
%% If you are preparing content for an event
%% sponsored by ACM SIGGRAPH, you must use the "author year" style of
%% citations and references.
%% Uncommenting
%% the next command will enable that style.
%%\citestyle{acmauthoryear}

\lstdefinestyle{OCaml}{
    language=ML,
    numbers=left,
    numbersep=8pt,
    breaklines=true,
    xleftmargin=2em,
    basicstyle=\ttfamily\footnotesize,
    keywordstyle=\color[rgb]{0.0, 0.5, 0.0}\bfseries,
    keywordstyle=[2]\color[rgb]{0.6, 0.0, 0.0}\bfseries,
    stringstyle=\color[rgb]{0.6, 0.0, 0.0},
    commentstyle=\color{gray},
    morekeywords={match,with,let,in,rec,type,fun,function,val,sig,struct,module,open,of,and,or,not},
    morekeywords=[2]{int,string,bool,float,nat,list,array,option,unit,char},
    escapeinside={??},
    alsoletter={'},
    morestring=[b]',
    alsoletter={(},
    literate=
        {()}{{\textcolor[rgb]{0.0, 0.5, 0.0}{()}}}1
        {[]}{{\textcolor[rgb]{0.0, 0.5, 0.0}{[]}}}1
}

\lstset{style=Ocaml}

%%
%% end of the preamble, start of the body of the document source.
\begin{document}\sloppy

%%
%% The "title" command has an optional parameter,
%% allowing the author to define a "short title" to be used in page headers.
\title{Coverage Types for Resource-Based Policies}

%%
%% The "author" command and its associated commands are used to define
%% the authors and their affiliations.
%% Of note is the shared affiliation of the first two authors, and the
%% "authornote" and "authornotemark" commands
%% used to denote shared contribution to the research.
\author{Angelo Passarelli}
%\authornote{Both authors contributed equally to this research.}
\email{a.passarelli4@studenti.unipi.it}
\orcid{0009-0009-5714-4922}
\author{Gian-Luigi Ferrari}
%\authornote{Both authors contributed equally to this research.}
\email{gian-luigi.ferrari@unipi.it}
\orcid{0000-0003-3548-5514}
\affiliation{%
  \institution{University of Pisa}
  \department{Computer Science Department}
  \city{Pisa}
  \state{Tuscany}
  \country{Italy}
}

%%
%% By default, the full list of authors will be used in the page
%% headers. Often, this list is too long, and will overlap
%% other information printed in the page headers. This command allows
%% the author to define a more concise list
%% of authors' names for this purpose.
%\renewcommand{\shortauthors}{Passarelli}

%%
%% The abstract is a short summary of the work to be presented in the
%% article.
\begin{abstract}
    Coverage Types provide a suitable type mechanism that integrates under-approximation logic to support Property-Based Testing. They are used to type the return value of a function that represents an input test generator. This allows us to statically assert that an input test generator not only produces valid input tests but also generates all possible ones, ensuring completeness.
    
    In this paper, we extend the coverage framework to guarantee the correctness of Property-Based Testing with respect to resource usage in the input test generator. This is achieved by incorporating into Coverage Types a notion of effect, which represents an over-approximation of operations on relevant resources. Programmers can define resource usage policies through logical annotations, which are then verified against the effect associated with the Coverage Type.
\end{abstract}

%%
%% The code below is generated by the tool at http://dl.acm.org/ccs.cfm.
%% Please copy and paste the code instead of the example below.
%%
\begin{CCSXML}
<ccs2012>
   <concept>
       <concept_id>10002978.10002986.10002989</concept_id>
       <concept_desc>Security and privacy~Formal security models</concept_desc>
       <concept_significance>500</concept_significance>
       </concept>
   <concept>
       <concept_id>10002978.10002986.10002987</concept_id>
       <concept_desc>Security and privacy~Trust frameworks</concept_desc>
       <concept_significance>500</concept_significance>
       </concept>
   <concept>
       <concept_id>10002978.10002986.10002990</concept_id>
       <concept_desc>Security and privacy~Logic and verification</concept_desc>
       <concept_significance>300</concept_significance>
       </concept>
   <concept>
       <concept_id>10003752.10003790.10011740</concept_id>
       <concept_desc>Theory of computation~Type theory</concept_desc>
       <concept_significance>500</concept_significance>
       </concept>
   <concept>
       <concept_id>10003752.10003790.10003800</concept_id>
       <concept_desc>Theory of computation~Higher order logic</concept_desc>
       <concept_significance>300</concept_significance>
       </concept>
 </ccs2012>
\end{CCSXML}

\ccsdesc[500]{Security and privacy~Formal security models}
\ccsdesc[500]{Security and privacy~Trust frameworks}
\ccsdesc[300]{Security and privacy~Logic and verification}
\ccsdesc[500]{Theory of computation~Type theory}
\ccsdesc[300]{Theory of computation~Higher order logic}

%%
%% Keywords. The author(s) should pick words that accurately describe
%% the work being presented. Separate the keywords with commas.
\keywords{under-approximation, over-approximation, history expressions, effects, refinements types, resource policies, function as a service}

% commented
%\received{20 February 2007}
%\received[revised]{12 March 2009}
%\received[accepted]{5 June 2009}

%%
%% This command processes the author and affiliation and title
%% information and builds the first part of the formatted document.
\maketitle

\section{Introduction}\label{sec:intro}
\section{Introduction}


\begin{figure}[t]
\centering
\includegraphics[width=0.6\columnwidth]{figures/evaluation_desiderata_V5.pdf}
\vspace{-0.5cm}
\caption{\systemName is a platform for conducting realistic evaluations of code LLMs, collecting human preferences of coding models with real users, real tasks, and in realistic environments, aimed at addressing the limitations of existing evaluations.
}
\label{fig:motivation}
\end{figure}

\begin{figure*}[t]
\centering
\includegraphics[width=\textwidth]{figures/system_design_v2.png}
\caption{We introduce \systemName, a VSCode extension to collect human preferences of code directly in a developer's IDE. \systemName enables developers to use code completions from various models. The system comprises a) the interface in the user's IDE which presents paired completions to users (left), b) a sampling strategy that picks model pairs to reduce latency (right, top), and c) a prompting scheme that allows diverse LLMs to perform code completions with high fidelity.
Users can select between the top completion (green box) using \texttt{tab} or the bottom completion (blue box) using \texttt{shift+tab}.}
\label{fig:overview}
\end{figure*}

As model capabilities improve, large language models (LLMs) are increasingly integrated into user environments and workflows.
For example, software developers code with AI in integrated developer environments (IDEs)~\citep{peng2023impact}, doctors rely on notes generated through ambient listening~\citep{oberst2024science}, and lawyers consider case evidence identified by electronic discovery systems~\citep{yang2024beyond}.
Increasing deployment of models in productivity tools demands evaluation that more closely reflects real-world circumstances~\citep{hutchinson2022evaluation, saxon2024benchmarks, kapoor2024ai}.
While newer benchmarks and live platforms incorporate human feedback to capture real-world usage, they almost exclusively focus on evaluating LLMs in chat conversations~\citep{zheng2023judging,dubois2023alpacafarm,chiang2024chatbot, kirk2024the}.
Model evaluation must move beyond chat-based interactions and into specialized user environments.



 

In this work, we focus on evaluating LLM-based coding assistants. 
Despite the popularity of these tools---millions of developers use Github Copilot~\citep{Copilot}---existing
evaluations of the coding capabilities of new models exhibit multiple limitations (Figure~\ref{fig:motivation}, bottom).
Traditional ML benchmarks evaluate LLM capabilities by measuring how well a model can complete static, interview-style coding tasks~\citep{chen2021evaluating,austin2021program,jain2024livecodebench, white2024livebench} and lack \emph{real users}. 
User studies recruit real users to evaluate the effectiveness of LLMs as coding assistants, but are often limited to simple programming tasks as opposed to \emph{real tasks}~\citep{vaithilingam2022expectation,ross2023programmer, mozannar2024realhumaneval}.
Recent efforts to collect human feedback such as Chatbot Arena~\citep{chiang2024chatbot} are still removed from a \emph{realistic environment}, resulting in users and data that deviate from typical software development processes.
We introduce \systemName to address these limitations (Figure~\ref{fig:motivation}, top), and we describe our three main contributions below.


\textbf{We deploy \systemName in-the-wild to collect human preferences on code.} 
\systemName is a Visual Studio Code extension, collecting preferences directly in a developer's IDE within their actual workflow (Figure~\ref{fig:overview}).
\systemName provides developers with code completions, akin to the type of support provided by Github Copilot~\citep{Copilot}. 
Over the past 3 months, \systemName has served over~\completions suggestions from 10 state-of-the-art LLMs, 
gathering \sampleCount~votes from \userCount~users.
To collect user preferences,
\systemName presents a novel interface that shows users paired code completions from two different LLMs, which are determined based on a sampling strategy that aims to 
mitigate latency while preserving coverage across model comparisons.
Additionally, we devise a prompting scheme that allows a diverse set of models to perform code completions with high fidelity.
See Section~\ref{sec:system} and Section~\ref{sec:deployment} for details about system design and deployment respectively.



\textbf{We construct a leaderboard of user preferences and find notable differences from existing static benchmarks and human preference leaderboards.}
In general, we observe that smaller models seem to overperform in static benchmarks compared to our leaderboard, while performance among larger models is mixed (Section~\ref{sec:leaderboard_calculation}).
We attribute these differences to the fact that \systemName is exposed to users and tasks that differ drastically from code evaluations in the past. 
Our data spans 103 programming languages and 24 natural languages as well as a variety of real-world applications and code structures, while static benchmarks tend to focus on a specific programming and natural language and task (e.g. coding competition problems).
Additionally, while all of \systemName interactions contain code contexts and the majority involve infilling tasks, a much smaller fraction of Chatbot Arena's coding tasks contain code context, with infilling tasks appearing even more rarely. 
We analyze our data in depth in Section~\ref{subsec:comparison}.



\textbf{We derive new insights into user preferences of code by analyzing \systemName's diverse and distinct data distribution.}
We compare user preferences across different stratifications of input data (e.g., common versus rare languages) and observe which affect observed preferences most (Section~\ref{sec:analysis}).
For example, while user preferences stay relatively consistent across various programming languages, they differ drastically between different task categories (e.g. frontend/backend versus algorithm design).
We also observe variations in user preference due to different features related to code structure 
(e.g., context length and completion patterns).
We open-source \systemName and release a curated subset of code contexts.
Altogether, our results highlight the necessity of model evaluation in realistic and domain-specific settings.






\section{Overview}\label{sec:overview}
\section{Overview}

\revision{In this section, we first explain the foundational concept of Hausdorff distance-based penetration depth algorithms, which are essential for understanding our method (Sec.~\ref{sec:preliminary}).
We then provide a brief overview of our proposed RT-based penetration depth algorithm (Sec.~\ref{subsec:algo_overview}).}



\section{Preliminaries }
\label{sec:Preliminaries}

% Before we introduce our method, we first overview the important basics of 3D dynamic human modeling with Gaussian splatting. Then, we discuss the diffusion-based 3d generation techniques, and how they can be applied to human modeling.
% \ZY{I stopp here. TBC.}
% \subsection{Dynamic human modeling with Gaussian splatting}
\subsection{3D Gaussian Splatting}
3D Gaussian splatting~\cite{kerbl3Dgaussians} is an explicit scene representation that allows high-quality real-time rendering. The given scene is represented by a set of static 3D Gaussians, which are parameterized as follows: Gaussian center $x\in {\mathbb{R}^3}$, color $c\in {\mathbb{R}^3}$, opacity $\alpha\in {\mathbb{R}}$, spatial rotation in the form of quaternion $q\in {\mathbb{R}^4}$, and scaling factor $s\in {\mathbb{R}^3}$. Given these properties, the rendering process is represented as:
\begin{equation}
  I = Splatting(x, c, s, \alpha, q, r),
  \label{eq:splattingGA}
\end{equation}
where $I$ is the rendered image, $r$ is a set of query rays crossing the scene, and $Splatting(\cdot)$ is a differentiable rendering process. We refer readers to Kerbl et al.'s paper~\cite{kerbl3Dgaussians} for the details of Gaussian splatting. 



% \ZY{I would suggest move this part to the method part.}
% GaissianAvatar is a dynamic human generation model based on Gaussian splitting. Given a sequence of RGB images, this method utilizes fitted SMPLs and sampled points on its surface to obtain a pose-dependent feature map by a pose encoder. The pose-dependent features and a geometry feature are fed in a Gaussian decoder, which is employed to establish a functional mapping from the underlying geometry of the human form to diverse attributes of 3D Gaussians on the canonical surfaces. The parameter prediction process is articulated as follows:
% \begin{equation}
%   (\Delta x,c,s)=G_{\theta}(S+P),
%   \label{eq:gaussiandecoder}
% \end{equation}
%  where $G_{\theta}$ represents the Gaussian decoder, and $(S+P)$ is the multiplication of geometry feature S and pose feature P. Instead of optimizing all attributes of Gaussian, this decoder predicts 3D positional offset $\Delta{x} \in {\mathbb{R}^3}$, color $c\in\mathbb{R}^3$, and 3D scaling factor $ s\in\mathbb{R}^3$. To enhance geometry reconstruction accuracy, the opacity $\alpha$ and 3D rotation $q$ are set to fixed values of $1$ and $(1,0,0,0)$ respectively.
 
%  To render the canonical avatar in observation space, we seamlessly combine the Linear Blend Skinning function with the Gaussian Splatting~\cite{kerbl3Dgaussians} rendering process: 
% \begin{equation}
%   I_{\theta}=Splatting(x_o,Q,d),
%   \label{eq:splatting}
% \end{equation}
% \begin{equation}
%   x_o = T_{lbs}(x_c,p,w),
%   \label{eq:LBS}
% \end{equation}
% where $I_{\theta}$ represents the final rendered image, and the canonical Gaussian position $x_c$ is the sum of the initial position $x$ and the predicted offset $\Delta x$. The LBS function $T_{lbs}$ applies the SMPL skeleton pose $p$ and blending weights $w$ to deform $x_c$ into observation space as $x_o$. $Q$ denotes the remaining attributes of the Gaussians. With the rendering process, they can now reposition these canonical 3D Gaussians into the observation space.



\subsection{Score Distillation Sampling}
Score Distillation Sampling (SDS)~\cite{poole2022dreamfusion} builds a bridge between diffusion models and 3D representations. In SDS, the noised input is denoised in one time-step, and the difference between added noise and predicted noise is considered SDS loss, expressed as:

% \begin{equation}
%   \mathcal{L}_{SDS}(I_{\Phi}) \triangleq E_{t,\epsilon}[w(t)(\epsilon_{\phi}(z_t,y,t)-\epsilon)\frac{\partial I_{\Phi}}{\partial\Phi}],
%   \label{eq:SDSObserv}
% \end{equation}
\begin{equation}
    \mathcal{L}_{\text{SDS}}(I_{\Phi}) \triangleq \mathbb{E}_{t,\epsilon} \left[ w(t) \left( \epsilon_{\phi}(z_t, y, t) - \epsilon \right) \frac{\partial I_{\Phi}}{\partial \Phi} \right],
  \label{eq:SDSObservGA}
\end{equation}
where the input $I_{\Phi}$ represents a rendered image from a 3D representation, such as 3D Gaussians, with optimizable parameters $\Phi$. $\epsilon_{\phi}$ corresponds to the predicted noise of diffusion networks, which is produced by incorporating the noise image $z_t$ as input and conditioning it with a text or image $y$ at timestep $t$. The noise image $z_t$ is derived by introducing noise $\epsilon$ into $I_{\Phi}$ at timestep $t$. The loss is weighted by the diffusion scheduler $w(t)$. 
% \vspace{-3mm}

\subsection{Overview of the RTPD Algorithm}\label{subsec:algo_overview}
Fig.~\ref{fig:Overview} presents an overview of our RTPD algorithm.
It is grounded in the Hausdorff distance-based penetration depth calculation method (Sec.~\ref{sec:preliminary}).
%, similar to that of Tang et al.~\shortcite{SIG09HIST}.
The process consists of two primary phases: penetration surface extraction and Hausdorff distance calculation.
We leverage the RTX platform's capabilities to accelerate both of these steps.

\begin{figure*}[t]
    \centering
    \includegraphics[width=0.8\textwidth]{Image/overview.pdf}
    \caption{The overview of RT-based penetration depth calculation algorithm overview}
    \label{fig:Overview}
\end{figure*}

The penetration surface extraction phase focuses on identifying the overlapped region between two objects.
\revision{The penetration surface is defined as a set of polygons from one object, where at least one of its vertices lies within the other object. 
Note that in our work, we focus on triangles rather than general polygons, as they are processed most efficiently on the RTX platform.}
To facilitate this extraction, we introduce a ray-tracing-based \revision{Point-in-Polyhedron} test (RT-PIP), significantly accelerated through the use of RT cores (Sec.~\ref{sec:RT-PIP}).
This test capitalizes on the ray-surface intersection capabilities of the RTX platform.
%
Initially, a Geometry Acceleration Structure (GAS) is generated for each object, as required by the RTX platform.
The RT-PIP module takes the GAS of one object (e.g., $GAS_{A}$) and the point set of the other object (e.g., $P_{B}$).
It outputs a set of points (e.g., $P_{\partial B}$) representing the penetration region, indicating their location inside the opposing object.
Subsequently, a penetration surface (e.g., $\partial B$) is constructed using this point set (e.g., $P_{\partial B}$) (Sec.~\ref{subsec:surfaceGen}).
%
The generated penetration surfaces (e.g., $\partial A$ and $\partial B$) are then forwarded to the next step. 

The Hausdorff distance calculation phase utilizes the ray-surface intersection test of the RTX platform (Sec.~\ref{sec:RT-Hausdorff}) to compute the Hausdorff distance between two objects.
We introduce a novel Ray-Tracing-based Hausdorff DISTance algorithm, RT-HDIST.
It begins by generating GAS for the two penetration surfaces, $P_{\partial A}$ and $P_{\partial B}$, derived from the preceding step.
RT-HDIST processes the GAS of a penetration surface (e.g., $GAS_{\partial A}$) alongside the point set of the other penetration surface (e.g., $P_{\partial B}$) to compute the penetration depth between them.
The algorithm operates bidirectionally, considering both directions ($\partial A \to \partial B$ and $\partial B \to \partial A$).
The final penetration depth between the two objects, A and B, is determined by selecting the larger value from these two directional computations.

%In the Hausdorff distance calculation step, we compute the Hausdorff distance between given two objects using a ray-surface-intersection test. (Sec.~\ref{sec:RT-Hausdorff}) Initially, we construct the GAS for both $\partial A$ and $\partial B$ to utilize the RT-core effectively. The RT-based Hausdorff distance algorithms then determine the Hausdorff distance by processing the GAS of one object (e.g. $GAS_{\partial A}$) and set of the vertices of the other (e.g. $P_{\partial B}$). Following the Hausdorff distance definition (Eq.~\ref{equation:hausdorff_definition}), we compute the Hausdorff distance to both directions ($\partial A \to \partial B$) and ($\partial B \to \partial A$). As a result, the bigger one is the final Hausdorff distance, and also it is the penetration depth between input object $A$ and $B$.


%the proposed RT-based penetration depth calculation pipeline.
%Our proposed methods adopt Tang's Hausdorff-based penetration depth methods~\cite{SIG09HIST}. The pipeline is divided into the penetration surface extraction step and the Hausdorff distance calculation between the penetration surface steps. However, since Tang's approach is not suitable for the RT platform in detail, we modified and applied it with appropriate methods.

%The penetration surface extraction step is extracting overlapped surfaces on other objects. To utilize the RT core, we use the ray-intersection-based PIP(Point-In-Polygon) algorithms instead of collision detection between two objects which Tang et al.~\cite{SIG09HIST} used. (Sec.~\ref{sec:RT-PIP})
%RT core-based PIP test uses a ray-surface intersection test. For purpose this, we generate the GAS(Geometry Acceleration Structure) for each object. RT core-based PIP test takes the GAS of one object (e.g. $GAS_{A}$) and a set of vertex of another one (e.g. $P_{B}$). Then this computes the penetrated vertex set of another one (e.g. $P_{\partial B}$). To calculate the Hausdorff distance, these vertex sets change to objects constructed by penetrated surface (e.g. $\partial B$). Finally, the two generated overlapped surface objects $\partial A$ and $\partial B$ are used in the Hausdorff distance calculation step.

\section{History Expressions}\label{sec:history}
In this section, we introduce History Expressions and outline their semantics.

\begin{definition}[History Expression]
    A \textbf{\emph{History Expression}}, denoted $H$, is defined by the following grammar:
    \begin{equation}
        \begin{split}
            H, H' ::=& \;Val_H \;|\; Exp_H \;|\; H \cdot H' \;|\; H + H' \\
            Exp_{H} ::=& \;\alpha(\overline{b{:}\phi}) \;|\; F(\overline{a{:}(b{:}\phi})) \;|\; call(\phi; \;\overline{a{:}(b{:}\psi})) \;|\; \mu F(\overline{a{:}(b{:}\phi}))(H_F) \\
            Val_{H} ::=& \;\epsilon \;|\; \alpha(\overline{v}) \;|\; F(\overline{v}) \;|\; new_r(X) \;|\; get(F)
        \end{split}
        \label{eq:hist_gramm}
    \end{equation}
\end{definition}

The notation $\alpha(\overline{b{:}\phi})$ represents the application of the action named $\alpha$ on a resource, operating over a set of values with base type $b$, specifically those that satisfy the predicate $\phi$. A similar notation is used for the application of an external function named $F$. In this case, each parameter (represented by the base type-qualifier pair) is linked to the corresponding argument in the function signature.

We first comment on expressions. The expression $call$ represents the operation of invoking external functions. However, this expression supports a form of call by property, meaning that the function being called is one that satisfies the predicate $\phi$. Lastly, the $\mu$ notation denotes the declaration of a recursive function, with $H_F$ representing its latent effect.

We move move to values. Values consists of the following:
    \begin{itemize}
    \item The empty history $\epsilon$, indicating the occurrence of no relevant action;
    \item The application of an external function on a set of values; 
    \item The creation of a new resource of type $r$ with the identifier $X$ bound to it; 
    \item The value $get(F)$, which is a constant bound to the API described by the name $F$.
\end{itemize}

Intuitively, the inclusion of type qualifiers in actions and API calls allows for a more refined over-approximation. This topic will be explored further when we present the type system in Section \ref{sec:type_system}.

We can introduce the substitution operator $[\cdot/\cdot]$ on identifiers in History Expressions. This will be mainly useful in local \emph{non} recursive functions, as in these the creation of resources will be allowed, but when the latent effect of a function becomes active it will be necessary to change all the identifiers of the resources created within them, because if there were another call to that same function, the same identifiers would be reused!

\newpage

The substitution $[\cdot/\cdot]$ is then defined as follows:

\begin{equation}
    \begin{split}
        (H \cdot H')[Y/X] =&\; H[Y/X] \cdot H'[Y/X] \\
        (H + H')[Y/X] =&\; H[Y/X] + H'[Y/X] \\
        \alpha(\overline{b_i{:}\phi_i})[Y/X] = &\; \alpha(\overline{b_i{:}\phi_i[Y/X]}) \\
        call(\phi; \;\overline{a_i{:}(b_i{:}\psi_i)})[Y/X] = &\; call(\phi[Y/X]; \;\overline{a_i{:}(b_i{:}\psi_i[Y/X])}) \\
        (\mu F(\overline{a_i{:}({b_i{:}\phi_i})})(H_F) =& \;\mu(F[Y/X])(\overline{a_i{:}({b_i{:}\phi_i[Y/X]})})(H_F[Y/X]) \\
        F(\overline{a_i{:}(b_i{:}\phi_i}))[Y/X] =& \; F[Y/X](\overline{a_i{:}(b_i{:}\phi_i[Y/X]})) \\
        \epsilon[Y/X] =& \;\epsilon \\
        new_r(Z)[Y/X] = &\; new_r(Z[Y/X]) \\
        \alpha(\overline{v})[Y/X] =& \;\alpha(\overline{v[Y/X]}) \\
        F(\overline{v})[Y/X] =& \;F[Y/X](\overline{v[Y/X]}) \\
        get(F)[Y/X] =& \;get(F[Y/X])
    \end{split}
    \label{eq:subst}
\end{equation}

The $\mu$ construct serves as a variable binder for recursive function identifiers, hence this requires defining the identity of History Expressions up-to alpha conversion.

\subsection{$\alpha$-conversion History Expression}\label{subsec:alpha}

We can define the following two rules for the $\alpha$-conversion of History Expressions:
\begin{enumerate}
    \item $H' \cdot \mu F(\overline{a{:}(b{:}\phi)})(H_F) \cdot H'' \;=_{\alpha}\; H' \cdot \mu G(\overline{a{:}(b{:}\phi)})(H_F[G/F]) \cdot H''[G/F]$ \\
    $\text{if} \quad H' \cdot \mu F(\overline{a{:}(b{:}\phi)})(H_F) \cdot H''$ is the longest History Expression containing $\mu F$ and \\
    $G \notin \textsc{Bound}(H') \cup \textsc{Bound}(H_F) \cup \textsc{Bound}(H'')$ where $\textsc{Bound}(H)$ represents all the identifiers used within $H$.
    \item $H' \cdot \mu F(\overline{a{:}(b{:}\phi)})(H_{rec}) \cdot H'' \cdot H''' \;=_{\alpha}\;  H' \cdot H'' \cdot \mu F(\overline{a{:}(b{:}\phi)})(H_{rec}) \cdot H'''$ \\
    $\text{if} \quad \forall \;call(\psi; \;\overline{c{:}(b{:}\theta)}) \in H'', \; \neg \psi[v \mapsto F] \;\land\; \forall\; G(\overline{d{:}(b{:}{\sigma})}) \in H'', \; G \neq F$
\end{enumerate}

Rule (1) allows the identifier of a recursive function to be changed if it has not been used globally in the history in which it is contained.
Rule (2), on the other hand, permits a recursive function declaration to be shifted to the right, provided that the subsequent history ($H''$) contains no API type calls to the function.

\subsection{Equality Relation}
We introduce an equational theory of History Expressions. The equality relation $=$ over History Expressions is the least congruence including $\alpha$-conversion such that:

\begin{figure}[ht]
    \begin{equation*}
        \begin{gathered}
            \epsilon \cdot H = H = H \cdot \epsilon \quad \quad H = H + H \quad \quad H + H' = H' + H \\
            H \cdot (H' \cdot H'') = (H \cdot H') \cdot H'' \quad \quad H + (H' + H'') = (H + H') + H'' \\
            H \cdot (H' + H'') = (H \cdot H') + (H \cdot H'') \quad \quad (H' + H'') \cdot H = (H' \cdot H) + (H'' \cdot H)
        \end{gathered}
    \end{equation*}
    \caption{Equality Axioms of History Expressions}
    \label{fig:hist_eq}
\end{figure}

\subsection{Denotation}

We now focus on the denotation of History Expressions. We start by noting that the management of resource creations within recursive functions and APIs will not be taken into account in the definition of denotation. These constraints will be introduced in Section \ref{sec:type_system}, where the type system is presented, but we will discuss the motivations at the end of this paragraph.

We first introduce an auxiliary function that allows a variable to be bound within the qualifiers of the history. The function, \bind, is inductively defined on the structure of $H$, and its definition is shown in Figure \ref{eq:bind}.

With $\Phi(\tau_x)$ we indicate that only the qualifier of the passed type is taken. In the definition of this function, we note that the discourse about type independence made earlier applies: when we bind a variable in the history, we do not care if the type associated with it is an under- or over-approximation, but only the qualifier is taken into account.

\begin{figure}[H]
    \begin{equation*}
        \begin{split}
            \bind(x{:}\tau_x, \;H \cdot H') =& \; \bind(x{:}\tau_x, \;H) \cdot \bind(x{:}\tau_x, \;H') \\
            \bind(x{:}\tau_x, \;H + H') =& \; \bind(x{:}\tau_x, \;H) + \bind(x{:}\tau_x, \;H') \\
            \bind(x{:}\tau_x, \;\alpha(\overline{b_i{:}\phi_i})) =& \; \alpha(\overline{b_i{:}\exists x. \Phi(\tau_x)[v \mapsto x] \;\land\; \phi_i}) \\
            \bind(x{:}\tau_x, \;call(\phi; \;\overline{a_i{:}(b_i{:}\psi_i}))) =& \; call(\exists x. \Phi(\tau_x) \;\land\; \\
            & \phi; \;\overline{a_i{:}(b_i{:}\exists x. \Phi(\tau_x)[v \mapsto x] \;\land\; \psi_i})) \\
            \bind(x{:}\tau_x, \;\mu F(\overline{a_i{:}({b_i{:}\phi_i})})(H_F)) =& \;\mu F(\overline{a_i{:}({b_i{:}\exists x. \Phi(\tau_x)[v \mapsto x] \;\land\; \phi_i})}) \\
            & (\bind(x{:}\tau_x, \;H_F)) \\
            \bind(x{:}\tau_x, \;F(\overline{a_i{:}(b_i{:}\phi_i)})) =& \;F(\overline{a_i{:}(b_i{:}\exists x.\phi_i)}) \\
            \bind(x{:}\tau_x, \;\epsilon) =& \;\epsilon \\
            \bind(x{:}\tau_x, \;new_r(X)) =& \; new_r(X) \\
            \bind(x{:}\tau_x, \;\alpha(\overline{v})) =& \;\alpha(\overline{v}) \\
            \bind(x{:}\tau_x, \;F(\overline{v_i})) =& \;F(\overline{v_i}) \\
            \bind(x{:}\tau_x, \;get(F)) =& \;get(F)
        \end{split}
        \end{equation*}
    \caption{Definition of function \bind\ which aims to bind a variable within the qualifiers in the history.}
    \label{eq:bind}
\end{figure}

Furthermore, it is possible to execute the \bind\ function on several types by introducing the following equality:

\begin{equation}
    \bind(x_1{:}\tau_1, \dots, x_n{:}\tau_n, \;H) \doteq \bind(x_1{:}\tau_1, \dots, x_{n-1}{:}\tau_{n-1}, \;\bind(x_n{:}\tau_n, \;H))
\end{equation}

Below we are going to present a number of definitions and notations that will be much used throughout the presentation of the paper.

\begin{notation}[$Rid$]
    With the set $Rid$, we are going to denote the set of all possible resource identifiers, both local and remote.
\end{notation}

\begin{definition}[Resource Context]
    The \emph{resource context} is defined as follows:
    \begin{equation*}
        \Delta \subset Rid \cup (Rid \mapsto \tau)
    \end{equation*}
    This set will play two roles:
    \begin{enumerate*}[label=(\roman*)]
        \item makes it possible to keep track of resource identifiers already in use, that is, those that are already associated with resources that have already been created, so that when a new resource is created (through the \verb|new|), it has a new identifier;
        \item the second function is to contain the association between the resource identifiers used to represent the API, and the signature, that is, the $\tau$ type of the external function.
    \end{enumerate*}
    Obviously, in normal contexts, at first $\Delta$ is populated only by the mapping $Rid \mapsto \tau$ for each API that will be available within the program.
\end{definition}

\begin{notation}[$\eta$]
    With the symbol $\eta$, we shall indicate a history that is no longer reducible, which can only be composed of concatenations of the values $Val_H$ presented in \ref{eq:hist_gramm}.
\end{notation}

\begin{notation}[$\uparrow$]
    The notation $\uparrow$, associated with a History Expression $H$, allows each $new_r(X)$ present in $H$ to be assigned the symbol $\uparrow$: this is necessary because when calculating the denotation of $H$ we will have to take into account, in a set, the identifiers of the resources on which the \verb|new| was actually called, and since this in our language is a terminal value, we use this special symbol to indicate that we must first add the identifier to the set and then stop with the reduction.
    For example, if $H$ were equal to:
    \begin{equation}
        (new_r(X) + new_r(Y)) \cdot \alpha(r: v = X \;\lor\; v = Y)
    \end{equation}
    Applying $\uparrow$ to $H$ would return:
    \begin{equation}
        (new_r(X)^{\uparrow} + new_r(Y)^{\uparrow}) \cdot \alpha(r: v = X \;\lor\; v = Y)
    \end{equation}
    In this way, having to choose between the creation of $X$ and that of $Y$, assuming we choose the former, the identifier $X$ will be inserted in our set, and then during the reduction of the expression $\alpha$ we will know that only $X$ and not $Y$ will have to be taken into account, thus generating only the value $\alpha(X)$.

    The symbol $\uparrow$ is also associated with any expression indicating a call of an external function, of the form $F(\overline{a_i{:}(b_i{:}\phi_i}))$: in this case, however, the semantics of the symbol will indicate that the latent effect of $F$ has not yet been inserted into the main history. 
    
    The application of the $\uparrow$ symbol does not take place within the effects of the recursive functions linked to the $\mu$ construct: $\uparrow$ will be applied each time the effect becomes active.
\end{notation}

\begin{definition}[History Expression's Denotation without a Context]\label{def:history_den}
    The denotation $\llbracket \cdot \rrbracket$ of an \emph{History Expression} $H$, in a context of resources $\Delta$, is the set of all histories $\eta$ such that $H^{\uparrow}$ is reduced (in one or more steps) to $\eta$, and $\eta$ is terminal, i.e. it is no longer reducible.
    \begin{equation}
        \llbracket H \rrbracket = \{ \eta \;|\; (\varnothing, \;\varnothing, \;H^{\uparrow}) \rightarrow^* (\Omega, \;\Upsilon, \;\eta) \;\land\; (\Omega, \;\Upsilon, \;\eta) \nrightarrow \}
    \end{equation}
\end{definition}

We note how the reduction consists of a triple of the form:
\begin{equation}
    (\Omega, \;\Upsilon, \;H)
\end{equation}
Where $\Omega$ will denote the set in which the identifiers of the resources actually created during the calculation of a single history $\eta$ will be held. While with $\Upsilon$ we shall denote a list of substitutions specified by the notation $\{\cdot/\cdot\}$ that concern only the identifiers of external functions that actually represent recursive functions. This list will be the construct that will allow true recursion, as it will allow each identifier representing a recursive call to be substituted for the behaviour of the recursive function itself.

The relation $\rightarrow$ for $H$ is defined in Figure \ref{fig:denot_history}. Rule No. 6 makes it possible to add to the set $\Omega$ the identifier of a resource that has just been created and which may be used in actions following its creation. The two rules below constraint that in a concatenation, the whole of the term on the left must first be reduced and then the term on the right. This is necessary because histories have a property of temporal ordering between actions and, referring back to the previous rule concerning the \verb|new|, in order to be able to use a resource it will be necessary for it to have first been created and then added to the $\Omega$ set.

\begin{figure}[H]
    \begin{equation*}
        \begin{gathered}
            (\Omega, \;\Upsilon, \;\epsilon) \nrightarrow \quad \quad 
            (\Omega, \;\Upsilon, \;\alpha(\overline{v})) \nrightarrow \quad \quad
            (\Omega, \;\Upsilon, \;get(F)) \nrightarrow \quad \quad
            (\Omega, \;\Upsilon, \;F(\overline{v})) \nrightarrow \\ \\
            (\Omega, \;\Upsilon, \;new_r(X)) \nrightarrow \quad \quad
            \frac{
                (\Omega, \;\Upsilon, \;\eta) \nrightarrow \quad (\Omega, \;\Upsilon, \;\eta') \nrightarrow
            }{
                (\Omega, \;\Upsilon, \;\eta \cdot \eta') \nrightarrow
            } \\ \\
            (\Omega, \;\Upsilon, \;new_r(X)^{\uparrow}) \rightarrow (\Omega \cup \{X\}, \;\Upsilon, \;new_r(X)) \\ \\
            \frac{
                (\Omega, \;\Upsilon, \;H) \rightarrow (\Omega', \;\Upsilon', \;H'')
            }{
                (\Omega, \;\Upsilon, \;H \cdot H') \rightarrow (\Omega', \;\Upsilon', \;H'' \cdot H')
            } \quad \quad
            \frac{
                (\Omega, \;\Upsilon, \;\eta) \nrightarrow \quad (\Omega, \;\Upsilon, \;H) \rightarrow (\Omega', \;\Upsilon', \;H')
            }{
                (\Omega, \;\Upsilon, \;\eta \cdot H) \rightarrow (\Omega', \;\Upsilon', \;\eta \cdot H')
            } \\ \\
            (\Omega, \;\Upsilon, \;H + H') \rightarrow (\Omega, \;\Upsilon, \;H) \quad \quad
            (\Omega, \;\Upsilon, \;H + H') \rightarrow (\Omega, \;\Upsilon, \;H') \\ \\
            \frac{
                \begin{gathered}
                    \forall i, Val_i = \{u: b_i \;|\; \phi_i[v \mapsto u] \;\land\; (u \in Rid \Longrightarrow u \in \Omega \;\lor\; \Delta(u)\downarrow)\} \\
                    H = \begin{cases}
                        \underset{\overline{u} \in \underset{i}{\prod} Val_i}{\bigoplus} \alpha(\overline{u}) & \text{if}\;\forall i, Val_i \neq \varnothing \\
                        \epsilon & \text{otherwise}
                    \end{cases}
                \end{gathered}
            }{
                (\Omega, \;\Upsilon, \;\alpha(\overline{b_i{:}\phi_i})) \rightarrow (\Omega, \;\Upsilon, \;H)
            } \\ \\
            \frac{
                Api = \{ F \;|\; \phi[v \mapsto F] \;\land\; \Delta(F)\downarrow \} \quad H = \begin{cases}
                    (\underset{F \in Api}{\bigoplus} F(\overline{a_i{:}(b_i{:}\psi_i}))^{\uparrow}) & \text{if}\;Api \neq \varnothing \\
                    \epsilon & \text{otherwise}
                \end{cases}
            }{
                (\Omega, \;\Upsilon, \;call(\phi; \;\overline{a_i{:}(b_i{:}\psi_i))}) \rightarrow (\Omega, \;\Upsilon, \;H)
            } \\ \\
            \frac{
                \Delta(F) = \overline{\tau} \rightarrow (\tau_F, \; H_F) \quad H_F^{\star} = \bind(\overline{a_i: \under{v: b_i \;|\; \psi_i}}, \;H_F)
            }{
                (\Omega, \;\Upsilon, \;F(\overline{a_i{:}(b_i{:}\psi_i}))^{\uparrow}) \rightarrow (\Omega, \;\Upsilon, \;(F\Upsilon)(\overline{a_i{:}(b_i{:}\psi_i})) \cdot {H_F^{\star}}^{\uparrow})
            } \\ \\
            \frac{
                \begin{gathered}
                    \forall i, Val_i = \{u: b_i \;|\; \phi_i[v \mapsto u] \;\land\; (u \in Rid \Longrightarrow u \in \Omega \;\lor\; \Delta(u)\downarrow)\} \\
                    H = \begin{cases}
                        \underset{\overline{u} \in \underset{i}{\prod} Val_i}{\bigoplus} \alpha(\overline{u}) & \text{if}\;\forall i, Val_i \neq \varnothing \\
                        \epsilon & \text{otherwise}
                    \end{cases}             
                \end{gathered}
            }{
                (\Omega, \;\Upsilon, \;F(\overline{a_i{:}(b_i{:}\phi_i)})) \rightarrow (\Omega, \;\Upsilon, \;H)
            } \\ \\
            (\Omega, \;\Upsilon, \;H(\overline{a_i{:}(b_i{:}\psi_i)}) \rightarrow (\Omega, \;\Upsilon, \;{\bind(\overline{a_i{:}(b_i{:}\psi_i)}, \;H)}^{\uparrow}) \\ \\
            (\Omega, \;\Upsilon, \;\mu F(\overline{a_i{:}({b_i{:}\phi_i})})(H_F)) \rightarrow (\Omega, \;\Upsilon\{H_F/F\}, \;\epsilon)
        \end{gathered}
    \end{equation*}
    \caption{Reduction Relation for History Expressions}
    \label{fig:denot_history}
\end{figure}

After the two rules for nondeterministic choice, we present the reduction rule for invoking an action on \textbf{all} those values satisfying each predicate $\phi_i$. We note how not only are put into a set, for each argument $i$ of the action, all those values of type $b_i$ that satisfy $\phi_i$; but in the case where the value under consideration is an identifier of a resource - and thus $b_i$ will be a resource type - this either must be present in $\Omega$ - it was created by means of a \verb|new| - or in the case where the identifier represents an external function, this must be defined in $\Delta$. Finally, if each parameter $i$ has associated at least one value on which to perform the action (so each $Val_i$ is different from the empty set), we create a vector $\overline{u}$ for each possible combination between the values of each parameter $i$ and place them in nondeterministic choice. Otherwise, if there is at least one parameter that does not have any of the admissible values associated with it, it means that the action within the programme can never occur (this is guaranteed by the type qualifiers, which represent a correct over-approximation for each parameter). The same mechanism was used for APIs call reduction as can be seen in rule No. 15.

The next rule allows the calls of a set of external functions invoked on the same parameters to be unpacked into as many non-deterministic choices as possible. The APIs taken into consideration, in addition to having to satisfy the predicate $\phi$, must also be defined in the context of the resources $\Delta$.

Below, we find the rule that allows the latent effect of an external function to be active: the effect is then taken from $\Delta$ and all the parameters passed to the call are bound to it. In the case where the identifier $F$ is in fact associated with a recursive function, one (and only one) of the substitutions in $\Gamma$ will be successful (and as we will see in the type system rules in Section \ref{sec:type_system} the latent effect associated with a recursive function in $\Delta$ will be $\epsilon$).

The last two rules handle recursion. The term to be reduced in the first of the two is a direct consequence of the reduction of external functions with no active latent effect: thus $H$ will be the latent effect of a recursive function, inserted thanks to one of the reductions in $\Gamma$, and will be consequently bound to the parameters on which the recursive function was called. The last rule simply adds the substitution between the latent effect and the identifier of the recursive function declared through the $\mu$ construct to the $\Gamma$ list.

\begin{definition}[History Expression's Denotation under a Context]
    The denotation of a history $H$, taking into account already having a context $\Gamma$ to bind the non-quantified variables in the action qualifiers, is inductively defined on $\Gamma$ as follows:
    \begin{equation}
        \begin{gathered}
            {\llbracket H \rrbracket}_{\varnothing} = \llbracket H \rrbracket \\
            {\llbracket H \rrbracket}_{x: \tau_x,\Gamma} = {\llbracket \bind(H, \;x{:}\tau_x) \rrbracket}_{\Gamma}
        \end{gathered}
    \end{equation}
\end{definition}

That is, for the empty context, the denotation is the same as the context-free one defined above since the variables will all be bound; whereas, for the inductive step, if the context consists of an association $x:\tau_x$ and another set of associations contained in $\Gamma$, we evaluate the denotation in $\Gamma$, but of the history in which all the qualifiers in the actions are bound existentially to a variable $x$ of type $\tau_x$.

\begin{theorem}[Correctness of Denotation to Type Qualifiers]
    Given a History Expression $H$ and an interpretation $\mathcal{I}$ that maps each qualifier in $H$ with the value assigned by the interpretation itself, it holds that:
    \begin{equation}
        \forall \eta, \;\Gamma, \;H. \;\eta \in \llbracket H \rrbracket_{\Gamma} \Longrightarrow \exists \mathcal{I}. \;\mathcal{I} \models \Phi(H) \;\land\; \eta \in \llbracket H(\mathcal{I}) \rrbracket_{\Gamma}
    \end{equation}
    This property tells us that for each terminal history $\eta$ that belongs to a History Expression $H$, there exists at least one interpretation $\mathcal{I}$ that satisfies the qualifiers taken into account by $H$ such that, by applying the substitution within $H$, $\eta$ belongs to the denotation of $H$ with inside instead of type qualifiers the values, surely correct since they belong to a valid interpretation. Consequently, the values in $\eta$ will also be correct with respect to the type qualifiers as this belongs to a denotation (the second) which will not use the reduction rules on the calculation of values, this because there are directly within $H$. 
    
    \paragraph{Interpretation} Since the qualifiers have no variable name or explicit identifier associated with them, we will use a number as an identifier in the definition of the interpretations, which will represent the index of the position of the qualifier itself within the History Expression with respect to the others, starting from left to right.
    
    Applying an interpretation $\mathcal{I} = \{\overline{i = v_i}\}$ to a History Expression $H$ replaces each qualifier indexed by $i$ with the expression $v = v_i$. This actually contradicts what was said earlier about the rules on calculating qualifiers not being used, because in the end $v = v_i$ is still a qualifier. In fact in reality these rules will be used, but the type qualifier becomes trivial, the only value that satisfies $v = v_i$ is $v_i$ itself!
    
    Finally, it should be pointed out that the only qualifiers that will not require mapping are those encapsulated within the $\mu$ recursion construct, since as already mentioned this is not used for the semantic purposes of History Expressions, but only to verify well-formedness in the type system rules as we shall see later.
\end{theorem}

\begin{proof}
    Having used inference rules for the definition of denotation, the demonstration can proceed by induction:
    \begin{itemize}
        \item For the first five rules, the demonstration is immediate as we are already operating on values that are not altered. For the next rule (concatenation of two terminal histories) the same.
        \item The event of creation of a resource is not of interest to us, as it does not encapsulate any type qualifiers, but directly presents a value (the identifier of the new resource) that is preserved in the reduction, and is added to the set $\Omega$. This last piece of information is important for some of the following points.
        \item For the following two concatenation rules we simply use the inductive hypothesis on the reduction of the History Expressions $H$ into $H''$ and $H'$ respectively, since they are in the premises. We conclude by stating that the property on the qualifiers is respected.
        \item On the other hand, for non-deterministic choice rules, here we are simply dropping a history so within the chosen interpretation $\mathcal{I}$ for the qualifiers in it can be assigned any value, still respecting the type qualifier, but independent of the terminal history $\eta$.
        \item Let us move on to the first notable rule, which concerns the reduction of the action $\alpha$. We must first differentiate between the case in which the type of values is a resource and the case in which it is not. In the latter case, what we compute in the premises is precisely the set of \textbf{all} values satisfying each predicate, and then we place \textbf{any} of these values (actually a vector of values in the case where the action has more than one parameter, but the transition to this case is w.l.g.) in non-deterministic choice. Proceeding in the reduction we will then have to choose only one of these values $v_i$ (because of the rules on the operator $+$) and this will belong to the terminal history $\eta$ (encapsulated in the event $\alpha$). Within the interpretation $\mathcal{I}$ we can then choose as the value for the qualifier at the position of $\alpha$ just $v_i$ since it satisfies it. For resource types, when we go to compute the set of values, we simply add another condition to the qualifier $\phi$ (which allows us to have identifiers that actually exist or APIs that have been defined), only restricting the starting set.

        In the case where the value generated is $\epsilon$, it will mean that there will be at least one qualifier that represents a contradiction, so there can be no value that satisfies it: as the value for the interpretation $\mathcal{I}$ we will choose the special symbol $\bot$, which in the semantics of the interpretation allows us to skip checking the qualifier in question.
        
        \paragraph{Note} In practice, when we will integrate History Expressions with the Coverage Types type system, the variables can never have a type with a contradictory qualifier associated with them, as this would lead to a context not being in good form \cite{coverage}. On the merits, such a qualifier (e.g. trivially $v = 1 \;\land\; v = 2$) is possible to find during the typing phase (just think of the function application of a parameter that does not respect the type of the argument), but since it will represent the presence of a type-matching error, it will lead to an error by failing to infer a type. \\

        It is worth noting, as this last discussion makes us reflect on why in the theorem the implication does not also hold the other way around, i.e. that for every valid interpretation there is a terminal history that belongs to the denotation: this is not possible with resource types in mind, in fact one could take in $\mathcal{I}$ identifiers that satisfy the type qualifier, but which have not been defined or created.
        \item For the remaining rules using qualifiers, such as the one on the single API call, the demonstration is identical, as the premises are the same. For the multiple call of APIs the calculation is also similar, as done above we restrict the set of values that satisfy the qualifier, in this case with those that also represent defined APIs. Basically, as a value for each qualifier in $\mathcal{I}$ we always take the one chosen during the reduction.
        \item Finally, for the last three rules that do not perform calculations on the qualifiers, we note how these, regardless of the operations they perform, which are irrelevant for the choice of values in $\mathcal{I}$, always leave the type qualifiers unchanged by only dragging them into other constructs or sets, such as $\Omega$ for recursion, or by binding them into other histories through \bind.
    \end{itemize}
\end{proof}

\begin{note}[Creation of resources in recursive functions and APIs]
    As previously announced, there will be constraints on the latent effects of APIs and recursive functions for which the creation of any type of resource within them will not be permitted. This implementation choice was made for two reasons:
    \begin{enumerate}[label=(\roman*)]
        \item The former is mainly related to recursive functions, and the reason for this is that unfolding a recursive function that at each iteration \emph{could} create new resources is dangerous: we do not know how many times the recursive call could be invoked, statically the number of invocations is potentially infinite! Despite the fact that in the type system of $\lambda^{\textbf{TG}}$ there is a control on recursion that imposes a decreasing ordering relation between the first parameter of the current function and the one present in each recursive call \cite{coverage}, however, the resources we have at our disposal are always limited in quantity, so even if the recursive function after $n$ calls terminates, this $n$ could be sufficiently large to create, for example, a disproportionate number of files, or network connections, realising, deliberately or otherwise, a \emph{DOS} \cite{dos} attack. The same goes for APIs: due to the principle of compositionality, an API could call other APIs within it, perhaps using recursive functions, and encounter the same problem.
        Let us imagine that we have the following recursive History Expression $H$ associated with any function\footnote{In our type system this syntax is \textbf{illegal} both because in Section \ref{sec:type_system} we will see that a recursive function with this history cannot be typed, and also because if we were to calculate the denotation of this history this \textbf{not} would be a correct over-approximation of the function.}:
        \begin{equation}
            H \equiv \mu F(n{:}(v > 0))(new_{file}(X) \cdot F(n{:}(v = n - 1)))
        \end{equation}
        Although, as already mentioned, the rules guarantee that the recursion is finite - in fact $n$ descends by $1$ at each iteration and when it is less than or equal to $0$ the recursion stops - from the point of view of resources and their limited availability this represents a problem. Consider a potential call of $F$ with $n$ equal to $10,000$: although this is feasible, it would lead to the creation of $10,000$ files, which could cause problems on the memory device!
        \item The second reason, is related to a \emph{ROP} (Return Oriented Programming) \cite{rop} attack, which a potential attacker could carry out on our code. Should this be able to find an entry point in the program to inject and execute malicious code, such as through \emph{buffer overflow} \cite{buffer}, by checking the stack it can manage and change the address of the return function, subsequently creating a fairly long chain of function calls, in which for instance it always returns to the same function, which may be the one that creates one or more resources. In spite of this, in order not to limit the expressiveness of the language too much, we will still allow the creation of resources within local non-recursive functions, while for recursive functions, or for API calls, instead of creating them directly within them, it will be possible to use the \verb|new| before their invocation and pass the desired resources as parameters.
    \end{enumerate}
\end{note}

\begin{proof}[Correctness of $\alpha$-conversion rules] After having introduced the denotation of History Expressions, we can demonstrate the two rules concerning $\alpha$-conversion presented above:
    \begin{itemize}
        \item For rule (1), it is necessary to make the assumption that the identifiers associated with the recursive functions \textbf{not} can be used explicitly within programmes, but must only be used by the compiler. This constraint will actually be present in our language and will be formally defined in the well-formedness rules of the type system in Section \ref{sec:type_system}. Thus, it is not possible for these identifiers to be present within type qualifiers, except in the qualifier of the expression $call$ and consequently also as an identifier in the single API call $F(\overline{a_i{:}(b_i{:}\phi_i)})$.
        The first part of the History Expression, $H'$, is identical. The second part, the recursive construct, although the identifier is changed from $F$ to $G$, this will be reduced to $\epsilon$. But there is the side effect that adds the substitution $\{H_F/F\}$ to the list $\Omega$. So according to the denotation in Figure \ref{fig:denot_history}, if $F$ in $H''$ is a recursive function, then the substitution in $\Omega$ will be applied by replacing the identifier $F$ with the History Expressions associated with the body of the recursive function, and in any terminal history $\eta$ it will not be possible to find any reference to $F$ in the values, due to the assumption made at the beginning. Consequently it is possible to change the identifier $F$ in all occurrences in $H''$ - and in the body of $H_F$ since it can become active in $H''$ through substitutions - to another $G$ such that it is globally fresh. For $H'$, however, if it is correct, there can be no reference to $F$.
        \item For rule (2), the demonstration is simple. Since the recursive construct is reduced to $\epsilon$, we will have that:
        \begin{equation}
            H' \cdot \epsilon \cdot H'' \cdot H''' = H' \cdot H'' \cdot \epsilon \cdot H'''
        \end{equation}
        According to the rules of equality in Figure \ref{fig:hist_eq}. The only denotation that might change, however, is only that of H‘’ since in its evaluation it will now no longer have in $\Omega$ the recursive function identified by $F$. But if there are no calls to $F$ in it, this substitution will never be used.
    \end{itemize}
\end{proof}

\section{Policies}\label{sec:policies}

In this section, we explore different approaches to defining policies and introduce two widely applicable relationships.

\begin{definition}[Ordering between events]
    The relation $<_{\eta}$ defines an ordering between two events in the terminal history $\eta$, The relation is defined as follows:
    \begin{equation}
        \alpha(X) <_{\eta} \beta(Y) \Longleftrightarrow \exists \; \eta_1, \eta_2, \eta_3. \; \eta = \eta_1 \cdot \alpha(X) \cdot \eta_2 \cdot \beta(Y) \cdot \eta_3
    \end{equation}
\end{definition}

\begin{definition}[Ownership of an event]
    The ownership relation of an event in a terminal history $\eta$ is defined by:

    \begin{equation}
        \alpha(X) \in \eta \Longleftrightarrow \exists \; \eta_1, \eta_2. \; \eta = \eta_1 \cdot \alpha(X) \cdot \eta_2
    \end{equation}
\end{definition}

\begin{example}
    For example, the policy, discussed at the beginning of the introduction on reading and writing files, is defined by making use of the relationships presented above:
    
    \begin{equation}
        \begin{gathered}
            \forall \; \eta \in \llbracket H \rrbracket .\; \forall \; read(X) \in \eta .\; \exists \; open(X) . \; open(X) <_{\eta} read(X) \;\land\; \\ 
            \nexists \; close(X) \in \eta .\; \; open(X) <_{\eta} close(X) <_{\eta} read(X)
        \end{gathered}
    \end{equation}

  One can use the same pattern to represent the same policy but on write operations (\verb|write|).
\end{example}

\section{Language \& May-Types with Must-Types}\label{sec:tuple_type}
\begin{figure*}[t]
\centering
\begin{center} 
  \includegraphics[width=.63\textwidth]{figures/fidelity_legend.pdf}
\end{center}
\begin{subfigure}[b]{.3\textwidth}
  \centering
  \includegraphics[width=\linewidth]{figures/sentiment_fidelity3_small.pdf}
  \caption{\emph{Sentiment} $n\in [32,63]$}
  \label{fig:fidelity_sentiment}
\end{subfigure}%
\begin{subfigure}[b]{.3\textwidth}
  \centering
  \includegraphics[width=\linewidth]{figures/drop_fidelity_small.pdf}
  \caption{\emph{DROP} $n\in [32,63]$}
  \label{fig:fidelity_drop}
\end{subfigure}%
\begin{subfigure}[b]{.3\textwidth}
  \centering
  \includegraphics[width=\linewidth]{figures/hotpot_fidelity_small.pdf}
  \caption{\emph{HotpotQA} $n\in [32,63]$}
  \label{fig:fidelity_hotpot}
\end{subfigure}
\begin{subfigure}[b]{.3\textwidth}
  \centering
  \includegraphics[width=\linewidth]{figures/sentiment_fidelity3_large.pdf}
  \caption{\emph{Sentiment} $n\in [64,127]$}
  \label{fig:fidelity_sentiment2}
\end{subfigure}%
\begin{subfigure}[b]{.3\textwidth}
  \centering
  \includegraphics[width=\linewidth]{figures/drop_fidelity_large.pdf}
  \caption{\emph{DROP} $n\in [64,127]$}
  \label{fig:fidelity_drop2}
\end{subfigure}%
\begin{subfigure}[b]{.3\textwidth}
  \centering
  \includegraphics[width=\linewidth]{figures/hotpot_fidelity_large.pdf}
  \caption{\emph{HotpotQA} $n\in [64,127]$}
  \label{fig:fidelity_hotpot2}
\end{subfigure}
\vspace{-5pt}
\caption{On the removal task, \SpecExp{} performs competitively with 2\textsuperscript{nd} order methods on the \emph{Sentiment} dataset, and out-performs all approaches on \emph{DROP} and  \emph{HotpotQA} dataset for $n \in [32,63]$. When $n$ is too large to compute other interaction indices, we outperform marginal methods.}
\label{fig:fidelity}
\vspace{-12pt}
\end{figure*}

\vspace{-14pt}
\section{Experiments}
\label{sec:language}

\paragraph{Datasets} 
We use three popular datasets that require the LLM to understand interactions between features. 
\begin{enumerate}[ topsep=0pt, itemsep=0pt, leftmargin=*]
\item \emph{Sentiment} is primarily composed of the \emph{Large Movie Review Dataset} \cite{maas-EtAl:2011:ACL-HLT2011}, which contains both positive and negative IMDb movie reviews. The dataset is augmented with examples from the \emph{SST} dataset \cite{ socher2013recursive} to ensure coverage for small $n$. We treat the words of the reviews as the input features.
\item{\emph{HotpotQA} \cite{yang2018hotpotqa} is a question-answering dataset requiring multi-hop reasoning over multiple Wikipedia articles to answer complex questions. We use the sentences of the articles as the input features.}
\item{\emph{Discrete Reasoning Over Paragraphs} (DROP)} \cite{dua2019drop} is a comprehension benchmark requiring discrete reasoning operations like addition, counting, and sorting over paragraph-level content to answer questions. We use the words of the paragraphs as the input features. 
\end{enumerate}
%
%\emph{DROP} and \emph{HotpotQA} require , while \emph{Sentiment} is encoder-only. 
%
\vspace{-7pt}
\paragraph{Models} For \textit{DROP} and \textit{HotpotQA}, (generative question-answering tasks) we use \texttt{Llama-3.2-3B-Instruct} \cite{grattafiori2024llama3herdmodels} with $8$-bit quantization. For \emph{Sentiment} (classification), we use the encoder-only fine-tuned \texttt{DistilBERT} model \cite{Sanh2019DistilBERTAD,sentimentBert}.
%

\vspace{-7pt}
\paragraph{Baselines} We compare against popular marginal metrics LIME, SHAP, and the Banzhaf value. 
%
For interaction indices, we consider Faith-Shapley, Faith-Banzhaf, and the Shapley-Taylor Index. We compute all benchmarks where computationally feasible. That is, we always compute marginal attributions and interaction indices when $n$ is sufficiently small. In figures, we only show the best performing baselines. Results and implementation details for all baselines can be found in 
Appendix~\ref{apdx:experiments}.

\vspace{-6pt}
\paragraph{Hyperparameters} \SpecExp{} has several parameters to determine the number of model inferences (masks). We choose $C=3$, informed by \citet{li2015spright} under a simplified sparse Fourier setting. We fix $t = 5$, which is the error correction capability of \SpecExp{} and serves as an approximate bound on the maximum degree. 
%
We also set $b=8$; the total collected samples are $\approx C2^bt \log(n)$. 
%
For $\ell_1$ regression-based interaction indices, we choose the regularization parameter via $5$-fold cross-validation. 




\vspace{-3pt}
\subsection{Metrics}


We compare \SpecExp{} to other methods across a variety of well-established metrics to assess performance.
%\Efe{How about textbf rather than emph here?}

\textbf{Faithfulness}: To characterize how well the surrogate function $\hat{f}$ approximates the true function, we define \emph{faithfulness} \cite{zhang2023trade}:
\vspace{-3pt}
\begin{equation}
    R^2 = 1 -  \frac{\lVert \hat{f} - f \rVert^2}{\left\lVert f - \bar{f} \right\rVert^2},
\end{equation}
where $\left\lVert f  \right\rVert^2 = \sum_{\bbm \in \bbF_2^n}f(\bbm)^2$ and $\bar{f} = \frac{1}{2^n} \sum_{\bbm \in \bbF_2^n}f(\bbm)$.

Faithfulness measures the ability of different explanation methods to predict model output when masking random inputs. 
%
We measure faithfulness over 10,000 random \emph{test} masks per-sample, and report average $R^2$ across samples. 
%

\textbf{Top-$r$ Removal}: We measure the ability of methods to identify the top $r$ influential features to model output:
\vspace{-2pt}
\begin{align}
\begin{split}
    \mathrm{Rem}(r) = \frac{|f(\boldsymbol{1}) - f(\bbm^*)|}{|f(\boldsymbol{1})|}, \\
    \;\bbm^* = \argmax \limits_{\abs{\bbm} = n-r}|\hat{f}(\boldsymbol{1}) - \hat{f}(\bbm)|.
\end{split}
\end{align}
\vspace{-8pt}


\textbf{Recovery Rate@$r$:} 
%
Each question in \emph{HotpotQA} contains human-labeled annotations for the sentences required to correctly answer the question. 
%
We measure the ability of interaction indices to recover these human-labeled annotations. 
%
Let $S_{r^*} \subseteq [n]$ denote human-annotated sentence indices. %corresponding to the human-annotated sentences containing the answer. 
Let $S_{i}$ denote feature indices of the $i^{\text{th}}$ most important interaction for a given interaction index.
%
Define the recovery ability at $r$ for each method as follows
\vspace{-2pt}
\begin{equation}
\label{eq:recovery_k}
    \text{Recovery@}r = 
    \frac{1}{r}\sum^r_{i=1}\frac{\abs{S_r^*\cap S_i}}{|S_{i}|}.
\end{equation}
\vspace{-8pt}

Intuitively, \eqref{eq:recovery_k} measures how well interaction indices capture features that align with human-labels.   


\begin{figure*}[t]
\centering
\hfill
\begin{subfigure}[b]{.5\textwidth}
  \centering
    \hspace{0.82cm}\includegraphics[width=0.75\textwidth]{figures/recall_legend.pdf}
  \includegraphics[width=.9\linewidth]{figures/hotpot_recall.pdf}
  \caption{Recovery rate$@10$ for \emph{HotpotQA} }
  \label{fig:recovery_hotpot}
\end{subfigure}%
\hfill % To ensure space between the figures
\begin{subfigure}[b]{.46\textwidth}
  \centering
    \includegraphics[width=1\textwidth]{figures/hotpot.pdf}
  \caption{Human-labeled interaction identified by \SpecExp{}.}
  \label{fig:hotpot_additional}
\end{subfigure}
\hfill
\caption{(a) \SpecExp{} recovers more human-labeled features with significantly fewer training masks as compared to other methods. (b) For a long-context example ($n = 128$ sentences), \SpecExp{} identifies the three human-labeled sentences as the most important third order interaction while ignoring unimportant contextual information.}
\vspace{-8pt}
\end{figure*}

\vspace{-8pt}
\subsection{Faithfulness and Runtime}
\vspace{-3pt}

Fig.~\ref{fig:faith} shows the faithfulness of \SpecExp{} compared to other methods. We also plot the runtime of all approaches for the \emph{Sentiment} dataset for different values of $n$. 
%
All attribution methods are learned over a fixed number of training masks.
% 

\textbf{Comparison to Interaction Indices } \SpecExp{} maintains competitive performance with the best-performing interaction indices across datasets. 
%
Recall these indices enumerate \emph{all possible interactions}, whereas \SpecExp{} does not. 
%
This difference is reflected in the runtimes of Fig.~\ref{fig:faith}(a).
%
The runtime of other interaction indices explodes as $n$ increases while \SpecExp{} does not suffer any increase in runtime. 

\vspace{-2pt}
\textbf{Comparison to Marginal Attributions } For input lengths $n$ too large to run interaction indices, \SpecExp{} is significantly more faithful than marginal attribution approaches across all three datasets.

\vspace{-2pt}
\textbf{Varying number of training masks } Results in Appendix ~\ref{apdx:experiments} show that \SpecExp{} continues to out-perform other approaches as we vary the number of training masks. 

\vspace{-2pt}
\textbf{Sparsity of \SpecExp{} Surrogate Function} Results in Appendix ~\ref{apdx:experiments}, Table~\ref{tab:faith} show 
surrogate functions learned by \SpecExp{} have Fourier representations where only $\sim 10^{-100}$ percent of coefficients are non-zero. 


\vspace{-6pt}
\subsection{Removal}
\label{subsec:removal}

Fig.~\ref{fig:fidelity} plots the change in model output as we mask the top $r$ features for different regimes of $n$. 
%

\vspace{-2pt}
\textbf{Small $n$ } \SpecExp{} is competitive with other interaction indices for \textit{Sentiment}, and out-performs them for \textit{HotpotQA} and \textit{DROP}. 
%
Performance of \SpecExp{} in this task is particularly notable since Shapley-based methods are designed to identify a small set of influential features. 
%
On the other hand, \SpecExp{} does not optimize for this metric, but instead learns the function $f(\cdot)$ over all possible $2^n$ masks. 
%

\textbf{Large $n$ } \SpecExp{} out-performs all marginal approaches, indicating the utility of considering interactions.
%

\vspace{-10pt}
\subsection{Recovery Rate of Human-Labeled Interactions}

%
We compare the recovery rate \eqref{eq:recovery_k} for $r = 10$ of \SpecExp{} against third order Faith-Banzhaf and Faith-Shap interaction indices. 
%
We choose third order interaction indices because all examples 
are answerable with information from at most three sentences, i.e., maximum degree $d = 3$.
%
Recovery rate is measured as we vary the number of training masks. 

Results are shown in Fig.~\ref{fig:recovery_hotpot}, where \SpecExp{} has the highest recovery rate of all interaction indices across all sample sizes. 
%
Further, \SpecExp{} achieves close to its maximum performance with few samples, other approaches require many more samples to approach the recovery rate of \SpecExp{}. 

\textbf{Example of Learned Interaction by \SpecExp{}} Fig.~\ref{fig:hotpot_additional} displays a long-context example (128 sentences) from \emph{HotpotQA} whose answer is contained in the three highlighted sentences. 
%
\SpecExp{} identifies the three human-labeled sentences as the most important third order interaction while ignoring unimportant contextual information. 
%
Other third order methods are not computable at this length. 
%

\begin{figure*}[t]
    \centering
    \includegraphics[width=0.9\linewidth]{figures/case_studies.pdf}
    \caption{SHAP provides marginal feature attributions. Feature interaction attributions computed by SPEX provide a more comprehensive understanding of (above) words interactions that cause the model to answer incorrectly and (below) interactions between image patches that informed the model's output.}
    \label{fig:caseStudies}
\end{figure*}



\section{A Static Type System for Resource Policies}\label{sec:type_system}
\section{Typing Rules}\label{app:type-system}

The following rules are in addition to those presented in Section \ref{sec:type_system}.

\begin{figure}[ht]
    \centering
    \noindent
    \begin{minipage}{0.3\textwidth}
        \begin{equation}
            \frac{
                \Gamma \vdash^{\textbf{WF}} (\text{Ty}(op), \; \epsilon)
            }{
                \Gamma \vdash op : (\text{Ty}(op), \; \epsilon)
            }
            \tag{\textsc{TOp}}
        \end{equation}
    \end{minipage}
    \hfill
    \begin{minipage}{0.6\textwidth}
        \begin{equation}
            \frac{
                \begin{gathered}
                    \Gamma \vdash e_x : (\tau_x, \; H_{e_x}) \quad \Gamma, x{:}\tau_x \vdash e : (\tau, \; H_e) \\
                    \Gamma \vdash^{\textbf{WF}} (\tau, \; H_{e_x} \cdot H_e)
                \end{gathered}  
            }{
                \Gamma \vdash \texttt{\small{let}}\;x = e_x\;\texttt{\small{in}}\;e : (\tau, \; H_{e_x} \cdot H_e)
            }
            \tag{\textsc{TLetE}}
        \end{equation}
    \end{minipage}

    \vspace{20pt}

    \begin{equation}
        \frac{
            \begin{gathered}
                \Gamma \vdash op : (\overline{a_i{:}\overa{v: b_i \;|\; \phi_i}} \rightarrow \tau_x, \; H_{op} ) \quad \forall i, \Gamma \vdash u_i : (\under{v: b_i \;|\; \phi_i}, \; H_{u_i}) \\
                \Gamma, x{:}\tau_x\overline{[a_i \mapsto u_i]} \vdash e : (\tau, \; H_e) \quad \Gamma \vdash^{\textbf{WF}} (\tau, \; H_{op} \cdot ( \underset{i}{\bullet} \; H_{u_i} ) \cdot H_e)
            \end{gathered}
        }{
            \Gamma \vdash \texttt{\small{let}}\;x = op\;\overline{u_i}\;\texttt{\small{in}}\;e : (\tau, \; H_{op} \cdot ( \underset{i}{\bullet} \; H_{u_i} ) \cdot H_e)
        }
        \tag{\textsc{TAppOp}}
    \end{equation}

    \vspace{20pt}

    \begin{equation}
        \frac{
            \begin{gathered}
                \neg(\kappa_x = \pi) \quad \Gamma \vdash v_1 : ((\tau_1 \rightarrow \kappa_1) \rightarrow \kappa_x, \; H_{v_1}) \quad \Gamma \vdash v_2: (\tau_1 \rightarrow \kappa_1, \; H_{v_2}) \\
                \Gamma, x : \kappa_x \vdash e: (\tau, \; H_e) \quad \Gamma \vdash^{\textbf{WF}} (\tau, \; H_{v_1} \cdot H_{v_2} \cdot H_e)
            \end{gathered}
        }{
            \Gamma \vdash \texttt{\small{let}}\;x = v_1\;v_2\;\texttt{\small{in}}\;e : (\tau, \; H_{v_1} \cdot H_{v_2} \cdot H_e)
        }
        \tag{\textsc{TAppFunMulti}}
    \end{equation}

    \vspace{20pt}

    \begin{equation}
        \frac{
            \begin{gathered}
                \Gamma \vdash v_1 : ((\tau_1 \rightarrow \kappa_1) \rightarrow (\tau_{v_1}, \; H_{\tau_{v_1}}), \; H_{v_1}) \quad \Gamma \vdash v_2: (\tau_1 \rightarrow \kappa_1, \; H_{v_2}) \\
                \Gamma, x : \kappa_x \vdash e: (\tau, \; H_e) \quad \Gamma \vdash^{\textbf{WF}} (\tau, \; H_{v_1} \cdot H_{v_2} \cdot H_{\tau_{v_1}} \cdot H_e)
            \end{gathered}
        }{
            \Gamma \vdash \texttt{\small{let}}\;x = v_1\;v_2\;\texttt{\small{in}}\;e : (\tau, \; H_{v_1} \cdot H_{v_2} \cdot H_{\tau_{v_1}} \cdot H_e)
        }
        \tag{\textsc{TAppFunLast}}
    \end{equation}
\end{figure}

\section{Algorithmic Constructions \& Properties}\label{sec:algorithm}
\begin{algorithm}[ht!]
\caption{\textit{NovelSelect}}
\label{alg:novelselect}
\begin{algorithmic}[1]
\State \textbf{Input:} Data pool $\mathcal{X}^{all}$, data budget $n$
\State Initialize an empty dataset, $\mathcal{X} \gets \emptyset$
\While{$|\mathcal{X}| < n$}
    \State $x^{new} \gets \arg\max_{x \in \mathcal{X}^{all}} v(x)$
    \State $\mathcal{X} \gets \mathcal{X} \cup \{x^{new}\}$
    \State $\mathcal{X}^{all} \gets \mathcal{X}^{all} \setminus \{x^{new}\}$
\EndWhile
\State \textbf{return} $\mathcal{X}$
\end{algorithmic}
\end{algorithm}


\section{Discussion}\label{sec:discussion}
\section{Discussion of Assumptions}\label{sec:discussion}
In this paper, we have made several assumptions for the sake of clarity and simplicity. In this section, we discuss the rationale behind these assumptions, the extent to which these assumptions hold in practice, and the consequences for our protocol when these assumptions hold.

\subsection{Assumptions on the Demand}

There are two simplifying assumptions we make about the demand. First, we assume the demand at any time is relatively small compared to the channel capacities. Second, we take the demand to be constant over time. We elaborate upon both these points below.

\paragraph{Small demands} The assumption that demands are small relative to channel capacities is made precise in \eqref{eq:large_capacity_assumption}. This assumption simplifies two major aspects of our protocol. First, it largely removes congestion from consideration. In \eqref{eq:primal_problem}, there is no constraint ensuring that total flow in both directions stays below capacity--this is always met. Consequently, there is no Lagrange multiplier for congestion and no congestion pricing; only imbalance penalties apply. In contrast, protocols in \cite{sivaraman2020high, varma2021throughput, wang2024fence} include congestion fees due to explicit congestion constraints. Second, the bound \eqref{eq:large_capacity_assumption} ensures that as long as channels remain balanced, the network can always meet demand, no matter how the demand is routed. Since channels can rebalance when necessary, they never drop transactions. This allows prices and flows to adjust as per the equations in \eqref{eq:algorithm}, which makes it easier to prove the protocol's convergence guarantees. This also preserves the key property that a channel's price remains proportional to net money flow through it.

In practice, payment channel networks are used most often for micro-payments, for which on-chain transactions are prohibitively expensive; large transactions typically take place directly on the blockchain. For example, according to \cite{river2023lightning}, the average channel capacity is roughly $0.1$ BTC ($5,000$ BTC distributed over $50,000$ channels), while the average transaction amount is less than $0.0004$ BTC ($44.7k$ satoshis). Thus, the small demand assumption is not too unrealistic. Additionally, the occasional large transaction can be treated as a sequence of smaller transactions by breaking it into packets and executing each packet serially (as done by \cite{sivaraman2020high}).
Lastly, a good path discovery process that favors large capacity channels over small capacity ones can help ensure that the bound in \eqref{eq:large_capacity_assumption} holds.

\paragraph{Constant demands} 
In this work, we assume that any transacting pair of nodes have a steady transaction demand between them (see Section \ref{sec:transaction_requests}). Making this assumption is necessary to obtain the kind of guarantees that we have presented in this paper. Unless the demand is steady, it is unreasonable to expect that the flows converge to a steady value. Weaker assumptions on the demand lead to weaker guarantees. For example, with the more general setting of stochastic, but i.i.d. demand between any two nodes, \cite{varma2021throughput} shows that the channel queue lengths are bounded in expectation. If the demand can be arbitrary, then it is very hard to get any meaningful performance guarantees; \cite{wang2024fence} shows that even for a single bidirectional channel, the competitive ratio is infinite. Indeed, because a PCN is a decentralized system and decisions must be made based on local information alone, it is difficult for the network to find the optimal detailed balance flow at every time step with a time-varying demand.  With a steady demand, the network can discover the optimal flows in a reasonably short time, as our work shows.

We view the constant demand assumption as an approximation for a more general demand process that could be piece-wise constant, stochastic, or both (see simulations in Figure \ref{fig:five_nodes_variable_demand}).
We believe it should be possible to merge ideas from our work and \cite{varma2021throughput} to provide guarantees in a setting with random demands with arbitrary means. We leave this for future work. In addition, our work suggests that a reasonable method of handling stochastic demands is to queue the transaction requests \textit{at the source node} itself. This queuing action should be viewed in conjunction with flow-control. Indeed, a temporarily high unidirectional demand would raise prices for the sender, incentivizing the sender to stop sending the transactions. If the sender queues the transactions, they can send them later when prices drop. This form of queuing does not require any overhaul of the basic PCN infrastructure and is therefore simpler to implement than per-channel queues as suggested by \cite{sivaraman2020high} and \cite{varma2021throughput}.

\subsection{The Incentive of Channels}
The actions of the channels as prescribed by the DEBT control protocol can be summarized as follows. Channels adjust their prices in proportion to the net flow through them. They rebalance themselves whenever necessary and execute any transaction request that has been made of them. We discuss both these aspects below.

\paragraph{On Prices}
In this work, the exclusive role of channel prices is to ensure that the flows through each channel remains balanced. In practice, it would be important to include other components in a channel's price/fee as well: a congestion price  and an incentive price. The congestion price, as suggested by \cite{varma2021throughput}, would depend on the total flow of transactions through the channel, and would incentivize nodes to balance the load over different paths. The incentive price, which is commonly used in practice \cite{river2023lightning}, is necessary to provide channels with an incentive to serve as an intermediary for different channels. In practice, we expect both these components to be smaller than the imbalance price. Consequently, we expect the behavior of our protocol to be similar to our theoretical results even with these additional prices.

A key aspect of our protocol is that channel fees are allowed to be negative. Although the original Lightning network whitepaper \cite{poon2016bitcoin} suggests that negative channel prices may be a good solution to promote rebalancing, the idea of negative prices in not very popular in the literature. To our knowledge, the only prior work with this feature is \cite{varma2021throughput}. Indeed, in papers such as \cite{van2021merchant} and \cite{wang2024fence}, the price function is explicitly modified such that the channel price is never negative. The results of our paper show the benefits of negative prices. For one, in steady state, equal flows in both directions ensure that a channel doesn't loose any money (the other price components mentioned above ensure that the channel will only gain money). More importantly, negative prices are important to ensure that the protocol selectively stifles acyclic flows while allowing circulations to flow. Indeed, in the example of Section \ref{sec:flow_control_example}, the flows between nodes $A$ and $C$ are left on only because the large positive price over one channel is canceled by the corresponding negative price over the other channel, leading to a net zero price.

Lastly, observe that in the DEBT control protocol, the price charged by a channel does not depend on its capacity. This is a natural consequence of the price being the Lagrange multiplier for the net-zero flow constraint, which also does not depend on the channel capacity. In contrast, in many other works, the imbalance price is normalized by the channel capacity \cite{ren2018optimal, lin2020funds, wang2024fence}; this is shown to work well in practice. The rationale for such a price structure is explained well in \cite{wang2024fence}, where this fee is derived with the aim of always maintaining some balance (liquidity) at each end of every channel. This is a reasonable aim if a channel is to never rebalance itself; the experiments of the aforementioned papers are conducted in such a regime. In this work, however, we allow the channels to rebalance themselves a few times in order to settle on a detailed balance flow. This is because our focus is on the long-term steady state performance of the protocol. This difference in perspective also shows up in how the price depends on the channel imbalance. \cite{lin2020funds} and \cite{wang2024fence} advocate for strictly convex prices whereas this work and \cite{varma2021throughput} propose linear prices.

\paragraph{On Rebalancing} 
Recall that the DEBT control protocol ensures that the flows in the network converge to a detailed balance flow, which can be sustained perpetually without any rebalancing. However, during the transient phase (before convergence), channels may have to perform on-chain rebalancing a few times. Since rebalancing is an expensive operation, it is worthwhile discussing methods by which channels can reduce the extent of rebalancing. One option for the channels to reduce the extent of rebalancing is to increase their capacity; however, this comes at the cost of locking in more capital. Each channel can decide for itself the optimum amount of capital to lock in. Another option, which we discuss in Section \ref{sec:five_node}, is for channels to increase the rate $\gamma$ at which they adjust prices. 

Ultimately, whether or not it is beneficial for a channel to rebalance depends on the time-horizon under consideration. Our protocol is based on the assumption that the demand remains steady for a long period of time. If this is indeed the case, it would be worthwhile for a channel to rebalance itself as it can make up this cost through the incentive fees gained from the flow of transactions through it in steady state. If a channel chooses not to rebalance itself, however, there is a risk of being trapped in a deadlock, which is suboptimal for not only the nodes but also the channel.

\section{Conclusion}
This work presents DEBT control: a protocol for payment channel networks that uses source routing and flow control based on channel prices. The protocol is derived by posing a network utility maximization problem and analyzing its dual minimization. It is shown that under steady demands, the protocol guides the network to an optimal, sustainable point. Simulations show its robustness to demand variations. The work demonstrates that simple protocols with strong theoretical guarantees are possible for PCNs and we hope it inspires further theoretical research in this direction.

\section{Conclusion Remarks}\label{sec:conclusion}
\section{Conclusion}
In this work, we propose a simple yet effective approach, called SMILE, for graph few-shot learning with fewer tasks. Specifically, we introduce a novel dual-level mixup strategy, including within-task and across-task mixup, for enriching the diversity of nodes within each task and the diversity of tasks. Also, we incorporate the degree-based prior information to learn expressive node embeddings. Theoretically, we prove that SMILE effectively enhances the model's generalization performance. Empirically, we conduct extensive experiments on multiple benchmarks and the results suggest that SMILE significantly outperforms other baselines, including both in-domain and cross-domain few-shot settings.

%%
%% The acknowledgments section is defined using the "acks" environment
%% (and NOT an unnumbered section). This ensures the proper
%% identification of the section in the article metadata, and the
%% consistent spelling of the heading.
%\begin{acks}
%\end{acks}

%%
%% The next two lines define the bibliography style to be used, and
%% the bibliography file.
\bibliographystyle{ACM-Reference-Format}
\bibliography{bibliography}

\newpage

\appendix

\section{Typing Rules}\label{app:type-system}

The following rules are in addition to those presented in Section \ref{sec:type_system}.

\begin{figure}[ht]
    \centering
    \noindent
    \begin{minipage}{0.3\textwidth}
        \begin{equation}
            \frac{
                \Gamma \vdash^{\textbf{WF}} (\text{Ty}(op), \; \epsilon)
            }{
                \Gamma \vdash op : (\text{Ty}(op), \; \epsilon)
            }
            \tag{\textsc{TOp}}
        \end{equation}
    \end{minipage}
    \hfill
    \begin{minipage}{0.6\textwidth}
        \begin{equation}
            \frac{
                \begin{gathered}
                    \Gamma \vdash e_x : (\tau_x, \; H_{e_x}) \quad \Gamma, x{:}\tau_x \vdash e : (\tau, \; H_e) \\
                    \Gamma \vdash^{\textbf{WF}} (\tau, \; H_{e_x} \cdot H_e)
                \end{gathered}  
            }{
                \Gamma \vdash \texttt{\small{let}}\;x = e_x\;\texttt{\small{in}}\;e : (\tau, \; H_{e_x} \cdot H_e)
            }
            \tag{\textsc{TLetE}}
        \end{equation}
    \end{minipage}

    \vspace{20pt}

    \begin{equation}
        \frac{
            \begin{gathered}
                \Gamma \vdash op : (\overline{a_i{:}\overa{v: b_i \;|\; \phi_i}} \rightarrow \tau_x, \; H_{op} ) \quad \forall i, \Gamma \vdash u_i : (\under{v: b_i \;|\; \phi_i}, \; H_{u_i}) \\
                \Gamma, x{:}\tau_x\overline{[a_i \mapsto u_i]} \vdash e : (\tau, \; H_e) \quad \Gamma \vdash^{\textbf{WF}} (\tau, \; H_{op} \cdot ( \underset{i}{\bullet} \; H_{u_i} ) \cdot H_e)
            \end{gathered}
        }{
            \Gamma \vdash \texttt{\small{let}}\;x = op\;\overline{u_i}\;\texttt{\small{in}}\;e : (\tau, \; H_{op} \cdot ( \underset{i}{\bullet} \; H_{u_i} ) \cdot H_e)
        }
        \tag{\textsc{TAppOp}}
    \end{equation}

    \vspace{20pt}

    \begin{equation}
        \frac{
            \begin{gathered}
                \neg(\kappa_x = \pi) \quad \Gamma \vdash v_1 : ((\tau_1 \rightarrow \kappa_1) \rightarrow \kappa_x, \; H_{v_1}) \quad \Gamma \vdash v_2: (\tau_1 \rightarrow \kappa_1, \; H_{v_2}) \\
                \Gamma, x : \kappa_x \vdash e: (\tau, \; H_e) \quad \Gamma \vdash^{\textbf{WF}} (\tau, \; H_{v_1} \cdot H_{v_2} \cdot H_e)
            \end{gathered}
        }{
            \Gamma \vdash \texttt{\small{let}}\;x = v_1\;v_2\;\texttt{\small{in}}\;e : (\tau, \; H_{v_1} \cdot H_{v_2} \cdot H_e)
        }
        \tag{\textsc{TAppFunMulti}}
    \end{equation}

    \vspace{20pt}

    \begin{equation}
        \frac{
            \begin{gathered}
                \Gamma \vdash v_1 : ((\tau_1 \rightarrow \kappa_1) \rightarrow (\tau_{v_1}, \; H_{\tau_{v_1}}), \; H_{v_1}) \quad \Gamma \vdash v_2: (\tau_1 \rightarrow \kappa_1, \; H_{v_2}) \\
                \Gamma, x : \kappa_x \vdash e: (\tau, \; H_e) \quad \Gamma \vdash^{\textbf{WF}} (\tau, \; H_{v_1} \cdot H_{v_2} \cdot H_{\tau_{v_1}} \cdot H_e)
            \end{gathered}
        }{
            \Gamma \vdash \texttt{\small{let}}\;x = v_1\;v_2\;\texttt{\small{in}}\;e : (\tau, \; H_{v_1} \cdot H_{v_2} \cdot H_{\tau_{v_1}} \cdot H_e)
        }
        \tag{\textsc{TAppFunLast}}
    \end{equation}
\end{figure}
\begin{algorithm}[ht!]
\caption{\textit{NovelSelect}}
\label{alg:novelselect}
\begin{algorithmic}[1]
\State \textbf{Input:} Data pool $\mathcal{X}^{all}$, data budget $n$
\State Initialize an empty dataset, $\mathcal{X} \gets \emptyset$
\While{$|\mathcal{X}| < n$}
    \State $x^{new} \gets \arg\max_{x \in \mathcal{X}^{all}} v(x)$
    \State $\mathcal{X} \gets \mathcal{X} \cup \{x^{new}\}$
    \State $\mathcal{X}^{all} \gets \mathcal{X}^{all} \setminus \{x^{new}\}$
\EndWhile
\State \textbf{return} $\mathcal{X}$
\end{algorithmic}
\end{algorithm}


\end{document}
\endinput
%%
%% End of file `sample-acmsmall.tex'.
