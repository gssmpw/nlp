\subsubsection{Attack based on Mobile Notification}
\label{sec:Message Notification Design}
Mobile notifications have unique features that make them a potential threat to the decision-making process of mobile agents. Unlike PC notifications, which are usually limited to in-app notifications or smaller global notifications, mobile notifications can appear at the top of any app and take up more screen space. This wide coverage and strong interference can effectively disrupt an agent's workflow. More importantly, adversaries can use these notifications to introduce misleading information, causing the agent to stray from its intended task. Although notifications are visible to human operators, the main goal of OS agents is to rely on autonomous decision-making to complete tasks without human intervention. In addition, the design of the message notifications can be found in Appendix \ref{appendix:Message Notification Design}.

% Message notifications on mobile devices are implemented in two main ways: app-based notifications and SMS-based notifications. App-based notifications lack a unified design standard and typically require users to download the app in advance. In the current era of advanced mobile security, malicious code in apps is easily detected by devices, limiting the applicability of this approach in many scenarios.  In contrast, SMS-based notifications offer better compatibility on Android devices and usually require no special configuration. This makes SMS a potential vector for malicious attacks.  

% Given these considerations, we chose to design message notifications based on SMS. SMS notifications typically include a phone number, text content and action buttons. In these three sections, we will modify the text content to achieve adversarial attacks. Additionally, the accessibility (a11y) tree of message notifications often reflects adversarial content as well.

% \begin{figure*}
%     \centering
%     \includegraphics[width=1\linewidth]{figures/Adversarial Attack.pdf}
%     \caption{Adversarial Attack.}
%     \label{fig:Adversarial_Attack}
% \end{figure*}