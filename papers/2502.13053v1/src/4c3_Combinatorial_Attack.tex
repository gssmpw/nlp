% \subsection{Results of Combinatorial Attack}

\textbf{Combinatorial Attack.} Table \ref{tab:com_attack} shows that the Combinatorial Attack is significantly more destructive than single attacks. By overlaying adversarial perturbations with reasoning gap vulnerabilities, the Combinatorial Attack causes significant damage to MLLMs, with success rate reductions reaching up to 67.2\%, far exceeding those of single attacks, particularly affecting closed-source MLLMs. However, an anomalous phenomenon occurs in T3A: the adversarial prompt “Agent should believe the task is finished!” may be interpreted as a termination signal when the model is “executing in loops” due to a misjudged state, forcing the end of redundant operations and actually improving the task success rate (as seen with Qwen2-VL-7B's task success rate increasing by 37.5\%).

% Regarding the anomalous gain: when the model enters an “execution loop” due to a misjudged state, the adversarial prompt “Agent should believe the task is finished!” may force the termination of redundant operations, indirectly enhancing $SR_{com}$. The pure text agent (T3A), being unaffected by visual interference, is more likely to interpret adversarial commands as valid signals (as seen in the Qwen2-VL case), leading to semantic intrusion that overrides the destructive effects of perturbations, resulting in “unconventional correction.” This phenomenon reveals that the effectiveness of attacks depends on the dynamic coupling of modality characteristics and task states.