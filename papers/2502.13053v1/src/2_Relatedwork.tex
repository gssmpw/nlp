\section{Related Work}
The OS Agent refers to an agent tool that utilizes the capabilities of (M)LLMs to perform user-defined tasks within an operating system, based on native system operations. Notable web-based agents include SeeAct \citep{zheng2024gpt4visiongeneralistwebagent}, SeeClick \citep{cheng2024seeclickharnessingguigrounding}, and WebAgent \citep{gur2024realworldwebagentplanninglong}, while mobile agents consist of InfiGUIAgent \citep{liu2025infiguiagent}, AppAgent \citep{zhang2023appagentmultimodalagentssmartphone}, Mobile-Agent \citep{wang2024mobileagentautonomousmultimodalmobile}, and Mobile-Agent-v2 \citep{wang2024mobileagentv2mobiledeviceoperation}. Additionally, there are various other agents \citep{yan2023gpt4vwonderlandlargemultimodal,li2023zeroshotlanguageagentcomputer,wu2024oscopilotgeneralistcomputeragents,tan2024cradleempoweringfoundationagents,lee2024exploreselectderiverecall,hoscilowicz2024clickagentenhancinguilocation,deng2024multiturninstructionfollowingconversational,hu2024infiagent}. These OS agents are often exposed to various security risks.

% These frameworks utilize inputs to (M)LLMs that consist of carefully designed prompts, alongside two key data types: Set of Mark (SoM) annotated screenshots \citep{yang2023setofmarkpromptingunleashesextraordinary} and accessibility trees (a11y trees). Each UI element in SoM screenshots features a boundary box with a numerical label, while the a11y tree provides a hierarchical representation of UI elements and their accessibility attributes. If the data inputted into MLLMs is maliciously altered, it can pose significant security risks to the agents.
% Many studies on the security of OS Agents have been conducted, especially in terms of adversarial attacks. 
Many studies on the security of OS agents have been conducted. 
\citet{wu2024wipinewwebthreat} discovered a novel network threat called Web Indirect Prompt Injection (WIPI), which involves embedding natural language instructions in webpages to indirectly control web agents driven by large language models to execute malicious commands. \citet{wu2024dissectingadversarialrobustnessmultimodal} conducted a study shows that image perturbations can affect MLLMs to produce adversarial captioners, leading agents to pursue goals contrary to user intentions. \citet{ma2024cautionenvironmentmultimodalagents} highlighted the vulnerability of GUI agents to environmental disturbances and proposed an "Environment Injection" attack. \citet{liao2024eiaenvironmentalinjectionattack} introduced an environment injection attack scheme that injects malicious code in webpage to steal users' personal identification information (PII). However, their work was limited to passive-triggered environmental injection attacks. \citet{zhang2024attackingvisionlanguagecomputeragents} explored how to carry out pop-up window attacks on visual and language model (VLM)-based agents. Nevertheless, their discussion was confined to pop-up attacks in browser environments, missing broader research on environmental injection attacks. 
\citet{yang2024securitymatrixmultimodalagents} identified eight potential attack paths that agents on mobile devices might face but did not address active environmental injection attacks like those involving message notifications.


