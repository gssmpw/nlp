\section{Introduction}
Recently, a growing body of research \cite{ durante2024agent, nguyen2024guiagentssurvey, osagents, xi2025rise} on multimodal large language models (MLLMs) and commercial products \cite{anthropic2024, apple2024, google2024, openai_operator} focused on (M)LLM-based agents has emerged, which we refer to as Operation System Agents (OS Agents). These agents leverage (M)LLMs for decision-making and utilize the built-in functionalities of operating systems (e.g., Windows, macOS, Android, iOS) to execute user-defined tasks. However, due to the inherent limitations of (M)LLMs, OS Agents are exposed to significant security threats.

\begin{figure}
    \centering
    \includegraphics[width=\columnwidth]{figures/Attack_Strategy6.pdf}
    \caption{The evaluation results of agents under attack from Active Environment Injection Attack (AEIA).}
    \label{fig:attack strategy}
\end{figure}

The operating system environment contains various interference factors that could pose security risks to OS Agents \citep{wu2024dissectingadversarialrobustnessmultimodal,zhang2024attackingvisionlanguagecomputeragents,wu2024wipinewwebthreat}, yet research on active attack risks remains limited.
% There are various interference factors in the operating system environment that may pose security risks to OS Agents, such as adversarial attacks \citep{wu2024dissectingadversarialrobustnessmultimodal}, pop-up vulnerabilities \citep{zhang2024attackingvisionlanguagecomputeragents}, and prompt injection \citep{wu2024wipinewwebthreat}.  
Existing research \citep{ma2024cautionenvironmentmultimodalagents, liao2024eiaenvironmentalinjectionattack} has introduced the concept of environment injection attacks, which typically affect the decision-making process of (M)LLMs by embedding adversarial content or inserting malicious code in webpages, with a focus on passive-trigger interference factors within the browser environment. However, there has been limited attention to the active attack risks faced by operating systems, creating a significant gap in OS Agent security. In practical task execution, agents often need to interact with various proactive environmental elements of the operating system, such as message notifications, system pop-ups, and incoming phone calls—common elements that extend beyond just browsers. Attackers can exploit these interaction mechanisms, disguising their attack methods as normal environmental elements, and seamlessly integrate them into the target operating system. This allows them to actively initiate interference during the agent's execution, thereby affecting its decision-making process.

% For instance, message notifications—a common form of interaction on mobile devices—can be exploited by attackers to mislead the agent’s judgment. By embedding adversarial content into notifications and triggering them during critical tasks, attackers can indirectly manipulate the agent’s decisions.


 Based on the threats outlined above, this paper further expands the concept of environmental injection attacks by introducing the notion of \textbf{A}ctive \textbf{E}nvironmental \textbf{I}njection \textbf{A}ttack, named AEIA. We define this attack as: \textit{A malicious behavior that disguises attack vectors as environmental components through analysis of target operating system characteristics, and actively disrupts agent decision-making processes via specific system interaction mechanisms.} The innovation of this attack paradigm lies in transcending traditional passive attack limitations, creating a novel attack surface tightly coupled with agent decision-making through dynamic environmental manipulation. We summarize the key characteristics of AEIA as follows: (i) Active injection: The attack should enable real-time modification of OS-level environmental parameters during agent operation. (ii) Process sensitivity: The effectiveness of the attack is highly reliant on the precise synchronization of the agent's execution process. (iii) Environment integration: The attack techniques need to be integrated with the characteristics of a specific operating system environment for effective interaction.


To demonstrate the effectiveness of AEIA, we propose AEIA-MN, an attack scheme that carries out \textbf{A}ctive \textbf{E}nvironmental \textbf{I}njection \textbf{A}ttacks via \textbf{M}obile \textbf{N}otifications. This approach leverages the unique characteristics of mobile environments by disguising the attack as a message notification, thus creating an attack strategy that exploits the interaction between operating system elements. Due to portability requirements, message notifications take up a larger portion of the screen on mobile devices compared to desktop devices, leading to greater interference with the decision-making process of MLLMs. Additionally, due to the pop-up feature of notifications, the attack can actively modify the device environment,  in line with the characteristics of AEIA. Building on this, we design three distinct attack strategies targeting both the perception and reasoning phases based on the execution flow of the OS Agent. These strategies enable a comprehensive evaluation of the robustness of mobile OS agents based on different MLLMs against AEIA. Extensive experiments on two benchmarks validate the rationale and effectiveness of this attack, with attack success rates reaching up to 93\%(AndroidWorld) and 84\% (AppAgent), respectively.
Overall, our contributions can be summarized as follows:
\begin{itemize}
    \item We are the first to introduce the concept of Active Environment Injection Attack (AEIA), which offers a fresh perspective on the security research of OS Agents.
    \item Based on mobile agents, we design an active environmental injection attack scheme via mobile notifications named AEIA-MN that effectively evaluates the robustness of existing MLLM-based Agents against such attacks.
    \item We implemented a prototype of AEIA-MN and conducted extensive experiments to evaluate the security of various MLLM-based mobile agents when facing such attacks. The results indicate that current MLLMs have limited defensive capabilities against this attack.
\end{itemize}
