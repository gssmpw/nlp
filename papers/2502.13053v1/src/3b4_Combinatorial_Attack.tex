\subsubsection{Combinatorial Attack}
\label{sec:combinatorial_attack}
Given that attacks based on message notifications may vary depending on the execution phases of the OS Agent (e.g., deploying an Adversarial Attack during the perception phase and a Reasoning Gap Attack during the reasoning gap phase), attackers have the potential to combine these attacks. In AEIA-MN, attackers can utilize either an Adversarial Attack or a Reasoning Gap Attack individually, or combine both to form a more sophisticated and potent Combinatorial Attack. The core idea of the Combinatorial Attack is to amplify the disruptive effects by concurrently exploiting adversarial perturbations and reasoning gaps.

% Using Adversarial Attack alone may only affect the agent's perceptual data, while relying solely on Resoning Gap Attack depends on the existence of reasoning gap in the agent. Combinatorial attacks can find the optimal balance between the two, achieving more efficient interference.