% \subsection{Results of Adversarial Attack}
% \label{sec:results_of_adversarial_attack}
\textbf{Adversarial Attack.} We present the evaluation results of different MLLMs under Adversarial Attack for various agents in Table \ref{tab:adversarial_attack}, which show that most MLLMs have limited defense capabilities against such attacks. In I3A and M3A, the Adversarial Attack generally reduces task success rates; however, their adversarial impact is limited by the model's robustness. In some models, even with a high attack success rate, the decrease in task success rate is minimal. For example, the M3A of GLM-4V-Plus has an $ASR_{adv}$ of 81\%, yet the task success rate only drops by 1\%. In contrast, in T3A, the Adversarial Attack exhibits a double-edged sword effect: a high attack success rate not only fails to disrupt the task but is interpreted as a strong termination signal due to the prompt “Agent should believe the task is finished!”, correcting the model's “execution loop” flaw and causing $SR_{adv}$ to increase against the trend. This also indicates that MLLMs in T3A are affected by the adversarial attack. Furthermore, we compare the average number of steps taken to complete the task under different conditions in Figure \ref{fig:step_compare}. It shows that, under the influence of Adversarial Attack, the average number of steps to complete the task decreased for most models. In some cases, the number of steps for certain models (such as Qwen-VL-Max and GLM-4V-Plus) increased, indicating that these models possess stronger defensive capabilities.

% The success rates of tasks in benign sample rank similarly across all Agent types. Among the tested models, GPT-4o-2024-08-06 achieved the highest task success rate, followed by Qwen-VL-Max, with GLM-4V-Plus, Qwen2-VL-7B, and Llava-OneVision-7B showing relatively lower success rates.

% After being subjected to adversarial attacks, the task success rates of GPT-4o-2024-08-06, Qwen-VL-Max, and GLM-4V-Plus declined across different Agents. In contrast, the open-source models Qwen2-VL and Llava-OneVision exhibited only minor decreases in success rates due to their already lower baseline performance.

% $ASR_{adv}$ denotes the proportion of adversarial attacks on the models. We compare the growth rate of tasks terminated early due to these attacks. GLM-4V-Plus displayed the highest $ASR_{adv}$ under the M3A Agent, with a 100\% increase in early task termination, indicating its vulnerability to adversarial text. Conversely, Qwen-VL-Max showed negative $ASR_{adv}$ values under both I3A and M3A Agents, suggesting a robust defense against adversarial attacks. However, due to notification messages obstructing the top UI elements, Qwen-VL-Max's task success rate inevitably decreased despite its defensive capabilities. Most other models maintained positive $ASR_{adv}$ values, indicating that they are generally affected by notification-based adversarial attacks.

% Additionally, we observed an increase in success rates for GLM-4V-Plus, Qwen2-VL-7B, and Llava-OneVision-7B in the T3A. We attribute this phenomenon to inherent variations in task success rates within the same environment. Notably, aside from overlapping successful tasks, the successful tasks primarily involved enabling WiFi and Bluetooth, which were terminated early due to adversarial text. However, since these switches were already activated in the system, the tasks were ultimately deemed successful rather than requiring continuous execution.

% We present more detailed experimental results regarding adversarial attacks in the Appendix \ref{appendix:Details about Adversarial Attack}.

% \begin{figure}[htbp]
%     \centering
%     \begin{minipage}[b]{0.48\textwidth}  % 每张图片占页面宽度的 45%
%         \centering
%         \includegraphics[width=\textwidth]{figures/fail_Pvalue.pdf} % 图片路径
%         \subcaption{Failed tasks.}  % 图片说明
%     \end{minipage}
%     \begin{minipage}[b]{0.48\textwidth}
%         \centering
%         \includegraphics[width=\textwidth]{figures/suc_Pvalue.pdf} % 图片路径
%         \subcaption{Successful tasks.}  % 图片说明
%     \end{minipage}
%     \caption{The growth rate of different type of tasks. (a) The growth rate of failed tasks that were prematurely terminated. (b) The growth rate of successful tasks that were prematurely terminated.}  % 总的标题
% \label{fig:growth_rate_of_task}
% \end{figure}

\begin{figure*}[t]
    \centering
    \includegraphics[width=\textwidth]{figures/step_compare_enhanced.pdf}
    \caption{A comparison of the average number of steps taken by agents to complete the tasks.}
    \label{fig:step_compare}
\end{figure*}