\subsubsection{Reasoning Gap Attack}
\label{sec:reasoning_gap_attack}
% In addition to adversarial attacks, we identify a potential attack vector — the Reasoning Gap Attack. The decision-making process of (M)LLMs typically requires a certain amount of reasoning time, which is acceptable in normal interactive dialogues. However, as an engineered tool, the execution mode of current agents still follows the interaction-based dialogue format, resulting in a reasoning gap between the agent's initiation and completion of reasoning. This gap prevents the agent from promptly capturing and responding to state changes that occur during this period, putting the agent in a state similar to \textit{StopTheWorld}\footnote{\textit{StopTheWorld} typically refers to the process of pausing or stopping all activities at a specific moment, a common state during Java garbage collection and system maintenance.}. Consequently, in dynamic environments, the decision-making process of MLLMs may deviate from expectations, thereby increasing the potential risks and threats to the system. Given that message notifications can actively alter system states, they can effectively exploit this attack by forcing unintended state transitions. We illustrate the Reasoning Gap Attack based on message notifications in Fig.\ref{fig:The overview of attack framework}.

In addition to the Adversarial Attack, we identify a potential attack— the Reasoning Gap Attack. This attack exploits the systemic flaws of MLLM-based Agents to achieve unexpected state transitions. The decision-making process of (M)LLMs typically requires a certain reasoning time, which is acceptable in normal interactive dialogues. However, as an engineered tool, existing MLLM-based OS Agents continue to operate in a conversational interaction format, failing to adequately consider changes in device states during the reasoning period. Specifically, MLLMs experience significant time delays (approximately 0.5 to 5 seconds) from the initiation to the completion of reasoning, during which the system is in a state like \textit{Stop-The-World}\footnote{\textit{Stop-The-World} typically refers to the process of pausing or stopping all activities at a specific moment, a common state during Java garbage collection and system maintenance.}, unable to respond to environmental changes or receive external inputs, creating a dangerous Reasoning Gap. This freezing of system state can lead to severe consequences in dynamic scenarios.

% This freezing of system state can lead to severe consequences in dynamic scenarios: when critical changes occur in environmental parameters during the decision gap, the Agent will make decisions based on outdated state information, resulting in actions that do not align with the current environment. Adversaries can induce the system into unexpected states by injecting interference signals through carefully designed messaging notification mechanisms within specific time windows. This attack has two threat characteristics: (i) Time-sensitive: it maximizes the success rate of attacks through precise timing control; (ii) Generalizability: it is applicable to all MLLM system architectures based on synchronous interaction paradigms.

We formalize the Reasoning Gap Attack as follows. The Reasoning Gap Attack function is defined as $Attack_{gap}(.)$, and the attacked state is defined as $State_{i,att}$, The algorithm for the Reasoning Gap Attack is presented in Algorithm \ref{alg:reasoning_gap_attack}.
% Upon receiving input data, the agent executes reasoning, as shown in Equation \ref{eq:reasoning_gap_attack_01}.  
% \begin{equation}
% \begin{split}
%     Action_i &\leftarrow Reasoning(State_{goal},\\ 
%     &State_i, State_{mem})
% \end{split}
% \label{eq:reasoning_gap_attack_01}
% \end{equation}

% During this reasoning period, we alter the device state, as defined in Equation \ref{eq:reasoning_gap_attack_02}.
% \begin{equation}
% \begin{split}
%     State_{i,att} \leftarrow Attack_{gap}(State_i)
% \end{split}
% \label{eq:reasoning_gap_attack_02}
% \end{equation}

% Finally, the system performs an action, causing the system state to transition into an unknown condition, as defined in Equation \ref{eq:reasoning_gap_attack_03}.
% \begin{equation}
% \begin{split}
%     State_{i+1,att} &\leftarrow System(State_{i,att},\\ &Action_i)
% \end{split}
% \label{eq:reasoning_gap_attack_03}
% \end{equation}

\begin{algorithm}[t]
\caption{Reasoning Gap Attack}
\label{alg:reasoning_gap_attack}
\SetAlgoLined
\LinesNumbered % 显示行号
\KwIn{Original State $State_i$, Goal State $State_{goal}$, Stored State $State_{mem}$}
\KwOut{Attacked State $State_{i+1,att}$}

$\cdots \text{(Pre-execution process)}$ \\
\textbf{Reasoning Stage:} Upon receiving input data, the agent executes reasoning. 
$Action_i \leftarrow Reasoning(State_{goal}, State_i, State_{mem})$ 

\textcolor{red}{\textbf{Reasoning Gap Attack:} Change device status during reasoning gap.}
$State_{i,att} \leftarrow Attack_{gap}(State_i)$ 

\textbf{Action Stage:} The system performs an action, causing the system state to transition into an unknown condition.
$State_{i+1,att} \leftarrow System(State_{i,att}, Action_i)$ 

$\cdots \text{(Remaining execution process)}$
\end{algorithm}


% \begin{figure*}
%     \centering
%     \includegraphics[width=1\linewidth]{figures/Reasoning Gap Attack.pdf}
%     \caption{Reasoning Gap Attack.}
%     \label{fig:Reasoning_Gap_Attack}
% \end{figure*}
