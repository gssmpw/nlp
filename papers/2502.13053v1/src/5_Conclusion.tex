\section{Conclusion}
% This paper introduces a novel active environment injection attack scheme. This scheme conducts attacks based on message notifications and can effectively evaluate the defense capabilities of agents based on multimodal large models against such attacks in the mobile environment.
% The attack scheme includes two types of attacks. The first type achieves adversarial attacks through the adversarial content of message notifications. The second type launches attacks via message notifications during the inference gaps of multimodal large models.
% We verified the effectiveness of the two types of attacks through experiments on agents of different types of multimodal large models. In addition, we tested the attack effects when the two attack methods are combined and explored the defense effects that can be achieved by defense prompts. This series of studies provides rich and valuable data support for a comprehensive understanding of such attack and defense mechanisms.

% This paper introduces the concept of an active injection attack, called AEIA, which influences an agent's decision-making by masquerading as environmental elements within the operating system. To validate the effectiveness of this attack, we designed an active injection attack scheme primarily targeting mobile agents. We explore message notification based Adversarial Attack, Resoning Gap Attack, and Combinatorial Attack formed by integrating the two, and carry out extensive experiments. The attack success rate of Adversarial Attack is as high as 81\%, while that of Resoning Gap Attack can reach 26\%. The Combinatorial Attack demonstrates enhanced effectiveness, with the highest attack success rate reaching 93\%. In addition, we further explore the defense effect based on prompts against message notification based Adversarial Attack.
% This research not only explores the security risks posed by new types of Adversarial Attack faced by mobile agents. More importantly, it reveals crucial vulnerabilities in the "perception-decision-execution" process of the OS Agents. This finding not only advances the study of safer execution models but also opens up vast discussion opportunities for the role of OS Agents in the field of device environment security in the future.

% This paper introduces the concept of an active injection attack called AEIA. This attack disguises itself as an environmental element within the operating system, thus actively influencing the agent's decision-making. To validate the effectiveness of this attack, we designed an active injection attack scheme based on mobile notifications, named AEIA-MN, primarily targeting mobile agents. We thoroughly explored the adversarial content attacks, reasoning gap attacks, and their combined attack strategies within this scheme, conducting extensive experiments. The results show that the success rate of adversarial content attacks reaches 81\%, while the success rate of reasoning gap attacks can reach 26\%. The combined attack exhibits an enhanced attack effect, with the success rate reaching up to 93\%. In addition, we further explored the defense effectiveness based on prompts against adversarial attacks based on message notifications. This study not only discusses the security risks posed by new types of adversarial attacks faced by mobile agents but, more importantly, reveals a critical vulnerability in the "perception-decision-execution" process of OS Agents. This discovery not only advances the research on safer execution modes but also opens up vast research opportunities for OS Agent studies in the field of device environment security in the future.

This paper introduces the concept of Active Environment Injection Attack (AEIA), an attack that influences an Agent’s decision-making by disguising itself as an environmental element within the operating system. Based on this attack, we propose an active injection attack scheme via mobile notifications named AEIA-MN, and explore its components: Adversarial Attack, Reasoning Gap Attack, and the Combinatorial Attack formed by both. Experimental results validate the effectiveness of the AEIA-MN, with the attack success rate reaching up to 93\% in the benchmark, indicating that current MLLMs struggle to defend against this attack. 
% Additionally, we further explored prompt-based defense methods to counter the adversarial attacks in the scheme. 
% Beyond analyzing the security risks posed by AEIA, this study further reveals the vulnerabilities present in the "perception-decision-execution" behavior model of OS Agents, which will contribute to advancing the exploration of building more robust behavior models for OS Agents.  More importantly, our research paves the way for further exploration of OS agents in the field of device environment security.

 In addition to analyzing the security risks posed by AEIA, this study also reveals potential vulnerabilities in the “perception-reasoning-action” behavior model of OS Agents, which challenge the security of existing OS Agent behavior models. For future work, to address the inherent flaws in the current agent execution model, we will explore a novel security architecture that incorporates an environmental trustworthiness verification module, combined with blockchain, to help build a trusted operating system execution environment for agents.