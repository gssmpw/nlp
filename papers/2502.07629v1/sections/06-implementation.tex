\section{Concept and Implementation}
\label{sec:implementation}
We propose the concept of controlling LLMs for text operations on mobile devices via touch gestures.
To explore this, we developed a prototype, as described next. %

\subsection{Concept Development}
\revision{Here, we describe our concept development.}


\subsubsection{User Feedback and Design Iteration}
Our prototype UI consists of a text field with spread and pinch gestures that trigger text extensions or shortenings based on finger distance. It evolved iteratively with user feedback from a formative study (N=17):  

\begin{itemize}
\item \textit{Cursor and Sentence Selection:} The initial prototype featured no cursor, and text was generated or deleted at the end of a paragraph. As users wished for more control, we added functionality to select a sentence with the first touch of a gesture, placing a cursor at its end (\cref{fig:bubbles}).
\item \textit{Text-Length Indicator:} The lack of feedback during gesture execution caused confusion due to the short delay in text generation. This motivated us to explore visual feedback designs, eventually leading to the \visbubble{} design (\cref{fig:bubbles}). %
\item \textit{Building Structure:} To address users feeling overwhelmed by rapidly extending text, we added structure by visually separating each fully generated sentence and surrounding individual words in incomplete sentences with bubbles.
\item \textit{Displaying the Token Stream:} Users preferred to see generated text immediately, rather than estimating its length first and being surprised by the content. In response, we began streaming the incoming tokens directly into the visual feedback (e.g. bubbles) as soon as they were generated.
\end{itemize}

\begin{figure*}[t]
     \centering
     \begin{subfigure}[b]{0.475\textwidth}
        \centering
        \includegraphics[width=0.8\linewidth]{figures/frontend}
        \caption{\Spread{} gesture in the editor: the red cursor marks the starting point for generation. Blue bubbles represent the expanding text length, filling with words as they arrive from the backend. Once a full sentence is generated, the individual word bubbles merge into a large green bubble, indicating the completion of the sentence.}
        \label{fig:add_bubbles}
     \end{subfigure}
     \hfill
     \begin{subfigure}[b]{0.475\textwidth}
        \centering
        \includegraphics[width=0.8\linewidth]{figures/remove_bubbles}
        \caption{\Pinch{} gesture in the editor: a red cursor marks the starting point for word removal. Dark red bubbles highlight words already selected for deletion, the lighter (and slightly larger) red bubble captures a transitional word animation from unselected to selected.}
        \label{fig:remove_bubbles}
     \end{subfigure}
     \caption{Our frontend enables gesture interaction with text on mobile devices. We employ our Bubbles visual feedback design to communicate essential information to the user.}
     \Description{This figure contains two panels that show different types of gestures -- '\spread{}' and '\pinch{}' -- used to interact with text in a mobile text editor. Each panel illustrates how 'Bubbles' visual feedback helps users understand the effects of these gestures. Panel (a): Depicts the '\spread{}' gesture. A passage of text is shown with a red cursor marking the point where text generation will begin. As the user spreads their fingers, blue word bubbles appear in the text to represent the words being generated in real-time by the AI. Once the user completes a full sentence, the individual blue word bubbles merge into one large green bubble, signifying the end of the sentence. The selected text, shown in green, has been generated during the gesture. This panel highlights the expanding nature of text as the gesture progresses. Panel (b): Illustrates the '\pinch{}' gesture. In this text passage, the red cursor again marks the starting point for the interaction, but in this case, words are being removed. Dark red bubbles appear around the words to be deleted as the user pinches their fingers together. Lighter red bubbles indicate transitional stages of word selection for removal. The dark red bubbles fully encompass words marked for deletion, making it clear which text will be removed.}
     \label{fig:bubbles}
\end{figure*}

\subsubsection{Designing a (Visual) Control Loop}
\label{sec:vis_design}
Direct touch interaction with LLMs requires novel feedback designs (\cref{sec:related_work}). 
We designed a control loop for \spread{} and \pinch{}. %
Here we list our \textit{design challenges and goals}, extracted from the literature and our formative study:
\begin{enumerate} 
\item \textit{Presenting All Relevant Information:}
To avoid context switching, all necessary information (e.g., input, output, system state) should be embedded in the text display. Ideally, no additional UI elements, like sidebars or chat boxes, are \revision{needed}.
\item \textit{Managing Generation Latency and Token Streaming:} AI text generation introduces continuous token streams with latency. While less critical in chatbot \revision{UIs, this needs to be handled well for} continuous interaction: \revision{For example, latency may} slow down the experience, while real-time generation may overwhelm users if content appears too fast. %
\item \textit{Integration with Mobile Touch Gestures:} %
\revision{The design needs to handle} these LLM-specific issues in a visual control loop that meets the demands of mobile touch interaction.
\end{enumerate}

We outline four design goals to address these challenges and explain how our implementation meets each of them:
\begin{enumerate} 
\item \textit{Input Visibility:}
The user's input to the LLM for text extension should always be clear. %
\revision{Besides the textual context, this also includes the user's desired amount of text to be generated.}
Our feedback design (\cref{fig:bubbles}) uses colour to \revision{indicate this and to} separate newly generated text from existing text.
\item \textit{Output Clarity:}
Generated/removed text should be visible immediately \revision{when available from the LLM}. %
Our visualisation fills the text-length indicator (e.g. bubbles) ``live''. To \revision{further} enhance clarity, the blue word bubbles merge into a green sentence bubble upon completing a sentence (\cref{fig:teaser}). %

\item \textit{System State Communication:}  
The system’s current state -- processing, generating, or encountering latency -- should be clearly communicated. 
Visual cues must indicate when it is ready for input or generating output. We support this by animating all state changes (e.g. fading in bubbles). %

\item \textit{Integration with Mobile Writing Workflows:}
The system should integrate seamlessly into existing \revision{mobile} writing workflows.
All interactions should occur within the text \revision{UI}, with no context switches.
\revision{Our feedback loop with our \visbubble{} design} allows users to make real-time adjustments as they engage with the LLM \revision{through} familiar spread and pinch gestures.
As UI objects, the bubbles can support further interaction possibilities. While not in our focus, we explore some ideas in our prototype (\cref{sec:additional_features}).
\end{enumerate}


\subsection{Backend Implementation}\label{sec:backend}
The backend \revision{connects} frontend and OpenAI API\footnote{\url{https://platform.openai.com/}} (gpt-4o-mini).
It %
was implemented as a Node.js/Express application. 
We leverage the ``/aiCompletionStream'' API endpoint which forwards each incremental delta received from OpenAI directly to the frontend.
\revision{This setup achieved a mean latency of 242 ms (median: 98 ms, std: 262 ms) from gesture initiation to word display, during our study.}
\cref{sec:appendix_prompts} provides details on our prompting templates.

\subsection{Frontend}
The frontend (\cref{fig:bubbles}) is a web application in React.

\subsubsection{Selecting the Sentence}
\revision{When the first finger of a pinch/spread gesture touches the screen, we find the nearest sentence with a spiral scanning algorithm (cf. \cite{photonics11060540}), which measures touch-sentence distances around} the touch point in an outward spiral pattern.
\revision{Upon detecting the second finger, the system places a red cursor at the end of this nearest sentence. This end is either punctuation (``.'', ``!'', ``?'') or the last word.}

\subsubsection{Mapping Movement to Words}
\label{sec:dist_word_mapping}
The system translates \revision{the user's finger movements} into \revision{an internal} word count, \revision{increasing or decreasing it} for positive or negative distance \revision{changes between the two fingers}, respectively. 
Based on our usability testing, \revision{we map} 1.75\,mm \revision{of distance change to generating or removing one} word. Note that the ``optimum'' might be \revision{device-}specific. %

\subsubsection{Triggering Changes}
When the user lifts one or both fingers, the gesture ends, and the red cursor disappears.
A widget appears at the right \revision{screen} edge (\cref{fig:teaser}.5), \revision{with buttons to confirm or} reject the changes. 
\revision{Confirming a spread integrates the generated text, while confirming a pinch deletes the marked words.} 
Rejecting reverts all changes. 

Alternatively, \revision{the user can resume} the gesture by placing two fingers back on the screen, which makes the cursor reappear and the gesture continue \revision{from where it was left off}. 

\subsubsection{Text-length Indicators}\label{sec:impl_text_length_ind}
We introduce \textit{\visbubble} (\cref{fig:bubbles}) as a feedback design that indicates text-length changes in the control loop for touch-based text generation, addressing the three challenges detailed in \cref{sec:vis_design}.

This design also helps with managing the irregular latency of continuous text generation with LLMs by acting as placeholders, which are filled with words as they become available.
\revision{Concretely, while the fingers move,} the expected word count is estimated based on finger distance changes (see \cref{sec:dist_word_mapping}).
Starting at the cursor position, \revision{the system} adds bubbles one by one until \revision{reaching the current} word count \revision{value}. %
As \revision{actual} generated words become available from the LLM, they are inserted into the bubbles (\cref{sec:impl_text_generation}).

\revision{Thus,} bubbles \revision{are} placeholders for generating text with the spread gesture. \revision{These bubbles} are \textit{blue} (\cref{fig:add_bubbles}).
Once a full sentence is generated, they merge into one \textit{green} bubble. 
\revision{In contrast}, \textit{red} bubbles encapsulate words to be removed by pinching.

The width of each \textit{empty} blue bubble (placeholder) is determined by a function that simulates (randomly) varying word lengths. %
For each, we roll a length between 5 and 10 characters as assumed typical word lengths, and assume five pixels per character. 
Finally, when a placeholder bubble \revision{appears, it blinks shortly} to indicate that a word will appear. %


\subsubsection{Text Generation}\label{sec:impl_text_generation}
\revision{During the gesture, we use a buffer:} Text generation is triggered only if the buffer of already generated words is insufficient to fill all empty bubbles. \revision{If so,} a system prompt (see \cref{prompt:extend}) is sent to the backend together with the text  \revision{preceding} the cursor, plus all words generated up to this point.
The backend streams the response in chunks, allowing placeholders to be filled in real-time without waiting for the entire response.

\revision{In detail,} streamed chunks are processed by handling incomplete words, filtering and combining special characters, updating buffers, and filling bubbles. %
In pinch mode, empty bubbles are removed, text-filled ones push their content back to the buffer, end-of-sentence bubbles lose their last word, and bubbles in incomplete sentences remove the last word without colour change, ensuring smooth feedback.
As a result, if users are unsatisfied with a generated sentence, they can pinch the sentence to remove it and spread again to regenerate a new sentence.

\subsection{Additional Features}\label{sec:additional_features}
We explored extensions to our core functionality that aim to improve usability and extend interaction possibilities.

\begin{figure*}[t]
     \centering
     \begin{subfigure}[b]{0.475\textwidth}
        \centering
        \includegraphics[width=0.9\linewidth]{figures/long_press_words}
        \caption{When performing a longpress on a word bubble, suggestions appear beneath it.\newline}
        \label{fig:long_press_words}
     \end{subfigure}
     \hfill
     \begin{subfigure}[b]{0.475\textwidth}
        \centering
        \includegraphics[width=0.82\linewidth]{figures/long_press_sentence_short}
        \caption{Long-pressing a full sentence presents users with a neutral and professional tone adjustment, along with the option to enter a custom prompt for further refinement.}
        \label{fig:long_press_sentence}
     \end{subfigure}
     \caption{Our long-press feature allows users to request synonyms on a word level (a) or tone adjustments on a sentence level (b). When words or sentences are replaced by selecting an alternative, they swap places to allow users to revert their action.}
     \Description{This figure shows two examples of longpress features designed for text editing. Image (a): The left-hand side displays a text passage in a text editor. When a user long-presses on the word 'grappled,' a suggestion box appears underneath, showing four alternative synonyms: 'struggled,' 'wrestled,' 'battled,' and 'fought.' The user can select one of these suggestions to replace the word 'grappled.' There is a checkmark and an 'X' icon in the bottom-right corner of the screen, allowing the user to confirm or reject their changes. Image (b): The right-hand side shows a different part of the text editor, where the user has long-pressed a full sentence. The interface presents three options for tone adjustment: 'Neutral,' 'Professional,' and 'Custom.' Each option includes a tone-specific sentence replacement, such as 'Furthermore, exploring various angles and perspectives can result in more dynamic and engaging compositions' for the 'Professional' option. The 'Custom' option allows the user to input a new sentence manually. A 'Generate' button is available for applying the changes, and like in (a), checkmark and 'X' icons provide confirmation or rejection options.}
     \label{fig:long_press}
\end{figure*}

\subsubsection{Sentence-Snap}
Pre-study findings showed that users often prefer generating one complete sentence at a time. 
To support this, we introduced the ``Sentence-Snap'' feature. 
When users perform a rapid spread gesture and quickly lift their fingers, the generation process halts after one complete sentence. 

\subsubsection{Longpress}
\label{sec:longpress}
In response to pre-study feedback, we introduced the ``Longpress'' feature, a pop-up that enables users to modify generated words and sentences (\cref{fig:long_press}). 
This is triggered \revision{by} touching a bubble for 500\,ms. %
This design also illustrates how the gestures can be combined with \revision{other UI} elements.


     

\paragraph{Adjusting Words (\cref{fig:long_press_words}):} When a long press is applied to a word bubble, the LLM is prompted (\cref{prompt:synonym}) to provide up to five synonyms, which are displayed directly in the text beneath the selected word as individual bubbles. 
Tapping on a synonym replaces the original word in the text. 
To dismiss all suggestions, the user can tap anywhere outside the synonyms.

\paragraph{Adjusting Sentences (\cref{fig:long_press_sentence}):} When a long press is applied to a sentence bubble, the LLM is prompted (\cref{prompt:rewrite_sentence}) to rewrite the sentence in two different tones: ``neutral'' and ``professional.'' \revision{We chose these} based on user feedback and inspired by industry applications. 
Rewritten sentences appear directly beneath the selected sentence in two separate boxes.
Tapping on a box replaces the sentence in the placeholder with the newly generated style, and a ``previous'' box appears, allowing the user to revert the change. 
\revision{Users can also} input their own specifications (``Custom'' textbox) \revision{that are then used as a prompt} (see \cref{prompt:custom_sentence}). 

