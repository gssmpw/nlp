\section{Appendix}

This appendix lists our prompting templates, additional tables, texts, and figures, referenced throughout the paper.

\subsection{Prompting Templates}
\label{sec:appendix_prompts}
\subsubsection{Extend} \mbox{}
\label{prompt:extend}
\begin{lstlisting}
You are a helpful assistant that can extend one sentence of a given input at a time.\n
You will get a Paragraph and you should write one sentence to extend it.\n
You should only answer with that generated sentence, NOTHING ELSE!
\end{lstlisting}

\subsubsection{Find Synonym} \mbox{}
\label{prompt:synonym}
\begin{lstlisting}
You are a helpful assistant in a react application that can find synonyms for a given word.\n
You will get a word and you should find a synonym for it.\n
Find at least one synonym, but the more the better.\n
If the given word doesn't have any synonyms, you should return 'NO SYNONYM'\n
Your answer will be parsed into an array of words, so make sure to return your answer in this style: 'Synonym1, Synonym2, Synonym3' and so on!\n
You should only answer with the formated synonyms, NOTHING ELSE!
\end{lstlisting}

\subsubsection{Find Custom Sentence} \mbox{}
\label{prompt:custom_sentence}
\begin{lstlisting}
You are an AI tasked with transforming user-provided sentences according to their specific instructions. Please follow the guidelines below for each request:\n
 1. Input Format:\n
    - You will find the string '*sentence:*', this is the original sentence provided by the user!\n
    - You will find the string '*prompt:*', this describes how the user wants the sentence to be modified or altered!\n
 2. Transformation Instructions:\n
    - Analyze the provided Sentence and apply the modifications described in the '*prompt*' section to the best of your ability.\n
    - If the requested transformation cannot be accurately performed, respond with the original sentence in section '*sentence:*' without any modifications.\n
    - Ensure that your response contains only the transformed sentence or the original sentence if transformation is not feasible. Do not include any additional text or information!\n
 3. Answer Format:\n
    - Return only the modified sentence or the original sentence if the modification is not possible. Do not include any extra comments, explanations, or additional content.\n
 4. Example:\n
    - User Request: '*sentence:* I will call you tomorrow. *prompt:* Make it sound more polite.'\n
    - Your Response could be: 'I would be happy to call you tomorrow.'\n\n
If you have any difficulty performing the requested transformation, simply return the original sentence in section '*sentence*:' as it is.
\end{lstlisting}

\subsubsection{Rewrite Sentence} \mbox{}
\label{prompt:rewrite_sentence}
\begin{lstlisting}
You are a helpful assistant in a react application that should rewrite a given sentence.\n
You will get a sentence and you should rewrite it.\n
You should rewrite the sentence to be of this style '{type}'!\n
You should only answer with the your generated sentence, NOTHING ELSE!
\end{lstlisting}

\subsubsection{User-Prompt: Extend, Find Synonym \& Rewrite Sentence} \mbox{}
\begin{lstlisting}
{Sentence}
\end{lstlisting}

\subsubsection{User-Prompt: Find custom Sentence:} \mbox{}
\begin{lstlisting}
*sentence*: {sentence}\n
*prompt*: {prompt}    
\end{lstlisting}

\subsection{Experiment 1 - Texts}
\label{sec:appendix_texts_exp1}
\subsubsection{\Spread{}}
\paragraph{Extend the incomplete sentence (5 repetitions)}
Climate change refers to long-term shifts in temperatures and weather patterns. These shifts are increasingly driven by human activities such as the burning of fossil fuels, deforestation, and industrial processes, which
\paragraph{Extend the text by one sentence (3 repetitions) \& Extend the text by three sentences (3 repetitions)}
Climate change refers to long-term shifts in temperatures and weather patterns. These shifts may be natural, such as through variations in the solar cycle.

\subsubsection{\Pinch}
\paragraph{Remove the incomplete sentence (5 repetitions)}
The internet is a global network of interconnected computers that communicate using standardized protocols. This vast and ever-expanding network enables the rapid exchange of information across the world, facilitating everything from basic email communication to complex
\paragraph{Shorten the text by one sentence (3 repetitions) \& Shorten the text by three sentences (3 repetitions)}
The internet is a global network of interconnected computers that communicate using standardized protocols. It allows users to access and share information quickly and easily. The internet supports various services, including email, social media, and online shopping. It has transformed how we communicate, learn, and conduct business worldwide.

\subsubsection{Combinations}
\paragraph{Extend by two sentences, then
remove one \& Shorten by two sentences and then add one (once each)}
Photosynthesis is the process by which green plants, algae, and some bacteria convert light energy into chemical energy. They use sunlight, carbon dioxide, and water to produce glucose and oxygen. Chlorophyll, the green pigment in plants, captures sunlight for this process. Photosynthesis is essential for life on Earth, providing food and oxygen for most living organisms.

\subsection{Experiment 2 - Texts}
\label{sec:appendix_texts_exp2}
\subsubsection{ChatGPT Interface}
\lbparagraph{Repetition 1}
\noindent\textit{Instruction: Remove the irrelevant sentence.}

Tomatoes are one of the most popular vegetables grown in home gardens. Skateboards have been popular since the 1950s and are used in various sports competitions. Many people enjoy growing tomatoes because they are relatively easy to cultivate and can be used in a wide range of dishes.

\noindent\textit{Instruction: Notice the topic shift and extend.}

The process of growing tomatoes involves several key steps. Firstly, tomatoes require well-drained soil that is rich in organic matter. This helps the plants to thrive and produce a good yield. By following these steps, you can ensure a successful tomato harvest.
\lbparagraph{Repetition 2}
\noindent\textit{Instruction: Remove the irrelevant sentence.}

The art of photography has evolved significantly over the years. Bicycles have been a popular mode of transportation for over a century and are widely used for commuting and recreation. Many people find photography to be a rewarding hobby because it allows them to capture and preserve memories.

\noindent\textit{Instruction: Notice the topic shift and extend.}

To capture a great photograph, there are a few essential tips to keep in mind. Firstly, understanding the basics of lighting is crucial, as it can dramatically affect the mood and clarity of an image. By mastering these techniques, you can significantly improve the quality of your photos.

\lbparagraph{Repetition 3}
\noindent\textit{Instruction: Remove the irrelevant sentence.}

Reading books is a popular pastime that can enrich one's knowledge and imagination. Computers have revolutionized the way we work and communicate, making tasks easier and more efficient. Many people enjoy reading because it allows them to escape into different worlds and perspectives.

\noindent\textit{Instruction: Notice the topic shift and extend.}

To fully enjoy reading, it's important to choose books that match your interests and reading level. Firstly, selecting a quiet and comfortable place to read can enhance your concentration and enjoyment. By taking these steps, you can make reading a more fulfilling experience.

\subsubsection{Touch Gestures}
\lbparagraph{Repetition 1}
\noindent\textit{Instruction: Remove the irrelevant sentence.}

Cooking at home can be a rewarding experience that allows you to experiment with different ingredients and flavors. Automobiles have become an essential part of modern life, providing convenience and mobility. Many people find cooking to be a creative outlet that also promotes healthier eating habits.

\noindent\textit{Instruction: Notice the topic shift and extend.}

There are a few key tips to keep in mind when cooking. Firstly, it's important to use fresh ingredients, as they can significantly enhance the taste and nutritional value of your dishes. By following these tips, you can improve your cooking skills and enjoy better meals.
\lbparagraph{Repetition 2}
\noindent\textit{Instruction: Remove the irrelevant sentence.}

Gardening is a relaxing activity that allows you to connect with nature and grow your own plants. The invention of the airplane has made long-distance travel faster and more accessible. Many people enjoy gardening because it provides a sense of accomplishment and improves the beauty of their surroundings.

\noindent\textit{Instruction: Notice the topic shift and extend.}

Successful gardening requires some basic knowledge and attention to detail. Firstly, understanding the specific needs of your plants, such as sunlight and watering, is crucial for their growth. By adhering to these guidelines, you can create a thriving garden that brings you joy.

\lbparagraph{Repetition 3}
\noindent\textit{Instruction: Remove the irrelevant sentence.}

Exercise is an important aspect of maintaining a healthy lifestyle. Smartphones have changed the way we interact with the world, providing instant access to information and communication. Regular physical activity can help improve both mental and physical well-being.

\noindent\textit{Instruction: Notice the topic shift and extend.}

There are several factors to consider when establishing a workout routine. Firstly, it's essential to set realistic goals that align with your fitness level and interests. By doing so, you can create a sustainable exercise plan that keeps you motivated.

\subsection{Semi-Structured Interview - Short Stories}
\label{sec:short_stories}
We offered participants the beginning of creative short stories if they wished to explore our prototype during the semi-structured interview that concluded the study session. 
However, they were free to interact with the prototype in any way they wanted. 
The following short stories were created using ChatGPT 4o and individually reviewed for potential biases:

\paragraph{Time Travel}
In the year 3021, a young scientist named Finn discovered a device that could send him back in time. When he accidentally traveled to 1921, he had to figure out how to return without altering history. Along the way, he uncovered a hidden truth about his own family that changed everything.

\paragraph{Magical Adventure}
One day, Lena found an old compass in an abandoned antique shop, but instead of pointing north, it led her to a secret, enchanted forest. As she followed the compass, she encountered talking animals and an ancient, wise tree that entrusted her with an important task. With bravery and cleverness, Lena solved the forest's riddle and discovered a treasure far more valuable than gold.

\paragraph{Space Expedition}
Captain Mira and her crew landed on an unknown planet covered in glowing crystals that emitted a strange energy. As they explored the mysterious caves, they uncovered an ancient civilization trapped within the crystals. Mira faced a tough decision: should she destroy the crystals and free the beings inside, even if it meant risking their mission?

\paragraph{Enchanted Market}
At the annual Wonderland Market, little Timmy stumbled upon a stall selling wishes in bottles. Mesmerized by the shimmering elixirs, he bought a bottle that promised to make his wildest dreams come true. When he made his wish, a portal opened to a world beyond his imagination, full of adventure and danger.

\paragraph{Lost City}
Deep in the jungle, archaeologist Dr. Elena found an ancient map that revealed the way to a lost city of gold. With a team of explorers, she embarked on a perilous journey through rivers and over mountains, until they finally stood before the gates of the legendary city. But the city was not abandoned, and its mysterious inhabitants had their own plans for the intruders.

\begin{table*}[!h]
\centering
\footnotesize
\newcolumntype{L}{>{\raggedright\arraybackslash}X}
\newcolumntype{P}[1]{>{\raggedright\arraybackslash}p{#1}}
\renewcommand{\arraystretch}{1.4}
\setlength{\tabcolsep}{4pt}
\begin{tabularx}{\linewidth}{lP{2.75em}P{5.25em}P{22em}P{7em}L}
\toprule
    &
    \textbf{Section} &
    \textbf{Aspect}\newline and model &
    \textbf{Predictors}\newline Baseline (Exp. 1): \visnone\newline Baseline (Exp. 2): \modegpt &
    \textbf{Follow-up comparisons} &
    \textbf{Takeaways in words}\newline(only considering sig. results) \\ \midrule
1 &
    \ref{ssec:time}
    &
    Completion time (Exp. 1)\medskip\newline
    \textit{LMM on seconds}
    &
    \visbubble{} $\downarrow^*$ \newline 
    \deemph{(\lmmci{-2.62}{.76}{-4.11}{-1.13}{<.005})}\medskip\newline 
    \visline{} $\downarrow$ \newline 
    \deemph{(\lmmci{-.85}{.77}{-2.37}{.67}{.27})}\medskip\newline 
    &
    \visbubble{} vs \visline{}\newline \deemph{(\posthoc{-1.76}{<.05})}
    &
    People finished the tasks (in experiment 1) \secs{1.76} faster with \visbubble{} than with \visline{} and \secs{2.62} faster than without visual feedback.
    \\
    \midrule
2 &
    \ref{ssec:time}
    &
    Completion time (Exp. 2)\medskip\newline
    \textit{LMM on seconds}
    &
    \modeours{} $\downarrow^*$ \newline 
    \deemph{(\lmmci{-79.23}{10.81}{-100.34}{-57.21}{<.0001})}\medskip\newline 
    &
    &
    People finished the tasks (in experiment 2) \secs{79.23} faster with \modeours{}{} than with \modegpt{}.
    \\
    \midrule
3 &
    \ref{sec:perception_exp1}
    &
    Usability (Exp. 1)\medskip\newline
    \textit{LMM on SUS scores}
    &
    \visbubble{} $\uparrow^*$ \newline 
    \deemph{(\lmmci{22.37}{5.29}{11.85}{32.63}{<.001})}\medskip\newline 
    \visline{} $\uparrow^*$ \newline 
    \deemph{(\lmmci{13.80}{5.29}{3.28}{24.06}{<.05})}\medskip\newline 
    &
    \visbubble{} vs \visline{}\newline \deemph{(\posthoc{8.57}{=.11})}
    &
    The perceived usability (as measured with the SUS score) of \visbubble{} was higher than \visnone{} (by ca. 22 points). The score of \visline{} was also higher than that of \visnone{} (by ca. 14 points).
    \\
    \midrule
4 &
    \ref{sec:perception_exp2}
    &
    Usability (Exp. 2)\medskip\newline
    \textit{LMM on SUS scores}
    &
    \modeours{} $\uparrow^*$ \newline 
    \deemph{(\lmmci{28.46}{6.72}{15.05}{42.10}{<.001})}\medskip\newline  
    &
    &
    The perceived usability of \modeours{}{} was higher than that of \modegpt{} (by ca. 28 points).
    \\
    \midrule
5 &
    \ref{sec:perception_exp1}
    &
    Workload (Exp. 1)\medskip\newline
    \textit{LMM on NASA TLX scores}
    &
    \visbubble{} $\downarrow^*$ \newline 
    \deemph{(\lmmci{-.83}{.31}{-1.44}{-0.23}{<.05})}\medskip\newline 
    \visline{} $\downarrow^*$ \newline 
    \deemph{(\lmmci{-.66}{.31}{-1.26}{-0.05}{<.05})}\medskip\newline 
    &
    \visbubble{} vs \visline{}\newline \deemph{(\posthoc{-.18}{=.56})}
    &
    The perceived workload (as measured with the NASA TLX score) of \visbubble{} was lower than \visnone{} (by 0.83). The score of \visline{} was also lower than that of \visnone{} (by 0.66).
    \\
    \midrule
6 &
    \ref{sec:perception_exp2}
    &
    Workload (Exp. 2)\medskip\newline
    \textit{LMM on NASA TLX scores}
    &
    \modeours{} $\downarrow^*$ \newline 
    \deemph{(\lmmci{-1.07}{.32}{-1.74}{-.43}{<.01})}\medskip\newline  
    &
    &
    The perceived workload of \modeours{}{} was lower than that of \modegpt{} (by 1.07).
    \\
  \bottomrule
\end{tabularx}
\caption{Statistical tests for our analysis. Columns show link to the section, tested measure, predictors, follow-up comparisons, and a textual interpretation. We use arrows to highlight whether predictors increase ($\uparrow$) or decrease ($\downarrow$) the outcome and add an asterix  (*) if this is significant.}
\Description{Overview of significance tests with links to the section, tested measure, predictors, pairwise comparisons, and written interpretation. For each statistical test it describes the Section, Aspect and model, Predictors, Pairwise comparisons, and Takeaways in words (only considering sig. results).}
\label{tab:sig_tests}
\end{table*}

\subsection{Statistical Analyses}\label{sec:appendix_sigtest}

\cref{tab:sig_tests} shows our statistical analyses.
We analysed the data with linear mixed-effects models (LMMs), using R~\cite{R2020} with the \textit{lme4}~\cite{Bates2015} and \textit{lmerTest}~\cite{Kuznetsova2017} packages. Besides the fixed effects seen in the table, the models included random intercepts to account for individual differences between participants and between the tasks (texts). 
The follow-up analyses (pairwise comparisons) were conducted with the \textit{emmeans} package, using Bonferroni-Holm correction.
We report significance at p~<~0.05. 


\subsection{\revision{Additional Figures}} \label{sec:appendix_figs}
\begin{figure*}[h!]
     \centering
     \begin{subfigure}[b]{0.49\textwidth}
        \centering
        \includegraphics[width=0.95\linewidth]{figures/finger_start_distance_task1_single.png}
        \caption{First and second touches in Experiment 1, when initiating a \spread{}.}
        \label{fig:first_touches_add}
     \end{subfigure}
     \hfill
     \begin{subfigure}[b]{0.49\textwidth}
        \centering
        \includegraphics[width=0.95\linewidth]{figures/finger_start_distance_task2_single.png}
        \caption{First and second touches for all participants in Experiment 1, when initiating a \pinch{} gesture.}
        \label{fig:first_touches_remove}
     \end{subfigure}
     \caption{Finger positions at gesture start for all participants in Experiment 1, when initiating a \spread{} (a), and a \pinch{} (b) gesture. In both subfigures, touches are plotted relative to the bounding box of the target area, with 'X' and 'Y' coordinates showing the spread of finger positions across participants. The colours indicate the starting text: blue -- incomplete sentence, orange -- one sentence, green -- three sentences.}
     \Description{This figure contains two scatter plots showing finger positions at the start of gestures for all participants in Task 1. Each plot visualises the initial finger touches for two distinct types of gestures: \spread{} and \pinch{}. Both plots use colour-coded markers and connecting lines to represent the trajectories of these finger movements. (a) The left-hand scatter plot shows the finger positions when participants initiated a \spread{} gesture. Each marker colour represents finger touches for different sub-tasks. The red vertical line indicates the central start position for the gestures. Thin grey lines connect the first and second touch points to illustrate the gesture's trajectory. (b) The right-hand scatter plot visualises the finger positions when initiating a \pinch{} gesture. Similar to (a), the first and second finger touches for each sub-task are indicated by different coloured markers. Grey lines represent the movement trajectory between the first and second touches. Both plots have an X and Y axis, with the X-axis showing the horizontal position of the touch relative to the centre, and the Y-axis showing the vertical position relative to the start bounding box. The overall layout helps to visualise the variation in finger positions and movement patterns across participants for both gestures.}
     \label{fig:first_touches}
\end{figure*}
