\documentclass[sigconf]{acmart}

\usepackage{tabularx}
\usepackage{booktabs}
\usepackage{makecell}
\usepackage[capitalize, noabbrev]{cleveref}
\usepackage{subcaption}
\usepackage{placeins}
\usepackage{siunitx}
\usepackage{enumitem}
\usepackage{calc}
\usepackage{listings}
\usepackage{float}
\lstset{
basicstyle=\small\ttfamily,
columns=flexible,
breaklines=true
}

\AtBeginDocument{%
  \providecommand\BibTeX{{%
    \normalfont B\kern-0.5em{\scshape i\kern-0.25em b}\kern-0.8em\TeX}}}

\setcopyright{acmlicensed}
\copyrightyear{2025} 
\acmYear{2025} 
\setcopyright{cc}
\setcctype{by}
\acmConference[CHI '25]{CHI Conference on Human Factors in Computing Systems}{April 26-May 1, 2025}{Yokohama, Japan}
\acmBooktitle{CHI Conference on Human Factors in Computing Systems (CHI '25), April 26-May 1, 2025, Yokohama, Japan}\acmDOI{10.1145/3706598.3713554}
\acmISBN{979-8-4007-1394-1/25/04}






\begin{document}

\title{Exploring Mobile Touch Interaction with Large Language Models}

\author{Tim Zindulka}
\email{tim.zindulka@uni-bayreuth.de}
\orcid{0009-0009-1972-351X}
\affiliation{%
  \institution{University of Bayreuth}
  \city{Bayreuth}
  \country{Germany}
}

\author{Jannek Sekowski}
\email{jannek.sekowski@uni-bayreuth.de}
\orcid{0009-0006-3324-7837}
\affiliation{%
  \institution{University of Bayreuth}
  \city{Bayreuth}
  \country{Germany}
}

\author{Florian Lehmann}
\email{florian.lehmann@uni-bayreuth.de}
\orcid{0000-0003-0201-867X}
\affiliation{%
  \institution{University of Bayreuth}
  \city{Bayreuth}
  \country{Germany}
}

\author{Daniel Buschek}
\email{daniel.buschek@uni-bayreuth.de}
\orcid{0000-0002-0013-715X}
\affiliation{%
  \institution{University of Bayreuth}
  \city{Bayreuth}
  \country{Germany}
}

\renewcommand{\shortauthors}{Zindulka et al.}

\definecolor{TimsColor}{rgb}{0.1,0.5,0.8}
\newcommand{\tim}[1]{\textsf{\textbf{\textcolor{TimsColor}{[Tim: #1]}}}}
\definecolor{JanneksColor}{rgb}{0.5,0.8,0.5}
\newcommand{\jannek}[1]{\textsf{\textbf{\textcolor{JanneksColor}{[Sven: #1]}}}}
\definecolor{FlosColor}{rgb}{0.9,0.1,0.8}
\newcommand{\flo}[1]{\textsf{\textbf{\textcolor{FlosColor}{[Flo: #1]}}}}
\definecolor{DanielsColor}{rgb}{0.9,0.6,0.1}
\newcommand{\daniel}[1]{\textsf{\textbf{\textcolor{DanielsColor}{[Daniel: #1]}}}}


\newcommand{\minsec}[2]{\SI{#1}{\minute} \SI{#2}{\second}}
\newcommand{\mins}[1]{\SI{#1}{\minute}}
\newcommand{\secs}[1]{\SI{#1}{\second}}
\newcommand{\pct}[1]{\ifnum\pdfstrcmp{#1}{X}=0
        X\% 
    \else\SI{#1}{\percent}\fi
}
\newcommand{\lbparagraph}[1]{\paragraph{#1}\mbox{}\\}



\newcommand{\lmmci}[5]{$\beta$=#1, SE=#2, CI$_{95\%}$=[#3, #4], p#5}
\newcommand{\posthoc}[2]{#1, p#2}

\definecolor{deemphColor}{rgb}{0.4,0.4,0.4}
\newcommand{\deemph}[1]{\textcolor{deemphColor}{#1}}

\newcommand{\pinch}{pinch-to-shorten}
\newcommand{\Pinch}{Pinch-to-shorten}
\newcommand{\spread}{spread-to-generate}
\newcommand{\Spread}{Spread-to-generate}
\newcommand{\visbubble}{Bubbles}
\newcommand{\visline}{Lines}
\newcommand{\visnone}{NoVis}
\newcommand{\modeours}{Gestures}
\newcommand{\modegpt}{ChatGPT}


\newcommand\revision[1]{\textcolor{black}{#1}}


\begin{abstract}
Interacting with Large Language Models (LLMs) for text editing on mobile devices currently requires users to break out of their writing environment and switch to a conversational AI interface. 
In this paper, we propose to control the LLM via touch gestures performed directly on the text.
We first chart a design space that covers fundamental touch input and text transformations.
In this space, we then concretely explore two control mappings: \spread{} and \pinch{}, with visual feedback loops.
We evaluate this concept in a user study (N=14) that compares three feedback designs: no visualisation, text length indicator, and length + word indicator. 
The results demonstrate that touch-based control of LLMs is both feasible and user-friendly, with the length + word indicator proving most effective for managing text generation. 
This work lays the foundation for further research into gesture-based interaction with LLMs on touch devices.
\end{abstract}



\begin{CCSXML}
<ccs2012>
   <concept>
       <concept_id>10003120.10003121.10011748</concept_id>
       <concept_desc>Human-centered computing~Empirical studies in HCI</concept_desc>
       <concept_significance>500</concept_significance>
       </concept>
   <concept>
       <concept_id>10003120.10003121.10003128.10011753</concept_id>
       <concept_desc>Human-centered computing~Text input</concept_desc>
       <concept_significance>500</concept_significance>
       </concept>
   <concept>
       <concept_id>10010147.10010178.10010179</concept_id>
       <concept_desc>Computing methodologies~Natural language processing</concept_desc>
       <concept_significance>500</concept_significance>
       </concept>
 </ccs2012>
\end{CCSXML}

\ccsdesc[500]{Human-centered computing~Empirical studies in HCI}
\ccsdesc[500]{Human-centered computing~Text input}
\ccsdesc[500]{Computing methodologies~Natural language processing}



\keywords{Writing assistance, Large language models, Human-AI interaction, Mobile interaction, Touch interaction, Direct manipulation}


\begin{teaserfigure}
  \centering
  \includegraphics[width=\textwidth]{figures/teaser}
  \caption{Our \textit{\spread} touch gesture for controlling generative AI on mobile devices. Touches are marked in orange. \textit{(1)} Placing two fingers on the screen sets the cursor (red) to the end of the sentence at the first touch (here: top touch). \textit{(2-3)} Spreading the two fingers fades in blue ``word bubbles'', which indicate estimates of length and number of words to be generated. In the background, an LLM generates text and streams it to the UI, where it is inserted into empty bubbles as it becomes available. \textit{(4)} Reaching the end of a sentence turns the word bubbles into one green sentence bubble. Further spreading the fingers starts generating another sentence. \textit{(5)} A confirmation widget is shown when lifting the fingers. Tapping the check mark accepts the generated text for \textit{(6)} the final result.}
  \label{fig:teaser}
  \Description{This figure shows six stages of the '\spread' gesture for controlling a generative AI system on mobile devices. The gesture is used to generate and insert text into a passage, and key touch points are highlighted in orange circles. Panel 1: The user places two fingers on the screen, and the cursor (shown in red) is positioned at the end of the sentence, where text generation will begin. Panel 2: As the user begins spreading their fingers apart, blue 'word bubbles' start appearing between the cursor and the point of finger contact. These bubbles represent placeholders for the AI-generated words that will be inserted into the sentence. Panel 3: As the spreading gesture continues, more 'word bubbles' are generated, reflecting additional text the AI will insert. The bubbles indicate the length and number of words that are likely to be generated. Panel 4: Once the AI has completed generating text for the current sentence, all the blue 'word bubbles' turn into a single green sentence bubble. This indicates that the AI has fully generated the sentence. Panel 5: A confirmation widget appears in the bottom right corner with a checkmark and 'X' button after the user lifts their fingers. This allows the user to either accept or reject the generated text. Panel 6: The user taps the checkmark to confirm and accept the AI-generated text, which is then finalized and inserted into the document. Each panel shows the progression of the gesture, from starting the text generation process to confirming the final result.}
\end{teaserfigure}



\maketitle


The increasing reliance on LLMs for multimodal tasks across far-reaching sectors such as healthcare, finance, and manufacturing underscores the need to assess the accuracy and reliability of the information they generate. Vision-Language Models (VLM) have achieved state-of-the-art (SoTA) performance on Visual Question-Answering (VQA) benchmarks, and these models often utilize Retrieval-Augmented Generation (RAG) to maintain factual accuracy and relevance in a dynamic information environment. However, this has led to uncertainty in the information the LLM bases its answer on, as it may choose between parametric memory and retrieved sources. When models rely on memorized information instead of dynamically retrieving information, they may inadvertently propagate outdated or incorrect information, causing serious legal and ethical risks and undermining trust and reliability in AI systems \citep{huang2023survey}.
% The ability to strike a balance between generalization and specialization in AI systems is therefore crucial for ensuring the safe, reliable use of these technologies in real-world applications.

Despite these concerns, the way that Vision-Language models (VLMs) memorize and retrieve information, particularly in complex multimodal tasks, remains under-explored. Current research often focuses on either the general capabilities of large language models (LLMs) or the specialized retrieval mechanisms in retrieval augmented generation systems (RAG) \citep{incontext_rag,chen_murag_2022,liu_universal_2023}. Particularly in the context of multimodal retrieval and multihop reasoning, few studies analyze the tradeoff between finetuning for specialized tasks and zero-shot prompting for general-purpose vision-language capabilities. A lack of consensus on how to approach this tradeoff motivates the development of measures to quantify reliance on parametric memory, as well as metrics for quantifying the potential performance impact of extending LLMs with RAG systems.

To address this gap, we investigate how multimodal QA models balance accuracy with memorization on the WebQA benchmark. We compare finetuned multimodal systems against zero-shot VLMs, analyzing how retrieval performance influences QA accuracy. In particular, we focus on cases where retrieval fails, allowing us to measure reliance on parametric memory through two proposed metrics---the \ppr (\PPR) which quantifies how much model accuracy is influenced by retrieval quality, contrasting performance in best-case versus worst-case retrieval scenarios, and the \ucr (\UCR) which measures how often correct QA responses are generated when the retriever fails, providing a proxy for memorization.

To enable this analysis, we make several methodological contributions. For the finetuned QA models, we investigate Vision-Transformer (ViT) architectures, which allow for multihop reasoning over multiple sources. To investigate the impact of retrieval performance on trained LMs, we propose a variable-input Fusion-in-Decoder (FiD) model \cite{tanaka_slidevqa_2023, nlvr2}, building upon the VoLTA architecture \citep{pramanick_volta_2023}. For the zero-shot case, we build upon previous research on In-Context Retrieval \citep{incontext_rag} by demonstrating that LLMs such as GPT-4o are capable of performing the final ranking step of the retrieval process. In doing so, we find that GPT-4o, a general-purpose LLM, achieves SoTA performance on the WebQA task, outperforming existing finetuned RAG models by a significant margin (7\% higher accuracy). 

Crucially, our results reveal that while retrieval-augmented models reduce memorization, the training paradigm plays an important role. Finetuned models exhibit higher reliance on parametric memory, whereas zero-shot RAG approaches have lower memorization scores at the cost of accuracy. This suggests that while retrieval modules may mitigate the risks associated with outdated or incorrect information, SoTA performance requires that they be coupled with specialized QA models. Our memorization measures contribute to the development of transparent and reliable AI systems, particularly in applications where the sourcing of up-to-date, factual information is critical.



% We investigate the impact of question complexity on the ability of these models to integrate multiple data sources—such as images, text, and external retrievers—and produce coherent and accurate answers. We also explore whether in-context retrieval can be a viable alternative to traditional retrieval-augmented systems, offering a more streamlined approach to multimodal QA.

% To achieve this, we first compare zero-shot prompting multimodal LLMs with finetuned multimodal systems. We evaluate both types of models on the WebQA benchmark, a dataset designed for complex question answering that requires reasoning across both image and text sources. For the finetuned models, we use a Fusion-in-Decoder (FiD) architecture, which allows for multihop reasoning over multiple sources. Additionally, we introduce the concept of In-Context Retrieval Language Modeling (RLM), where the LLM itself performs retrieval tasks without the need for external retrievers. This method builds upon existing research in in-context learning  and aims to explore the viability of LLMs retrieving relevant sources and generating accurate answers directly from their context window.

% In order to investigate source utilization in finetuned multimodal models and LLMs, three lines of inquiry are established; 
% \begin{itemize}
%     \item Study 1: retrieval vs QA performance on webQA (motivating example, does QA answer correctly even with incorrect sources?)
%     \item Study 2: performance on adversarial examples where parametric knowledge would be incorrect by design
%     \item Study 3: improving performance on adversarial examples by fine-tuning (i.e model robustness)
% \end{itemize}

% Note, there is one weakness in this plan which is tying in the work we've already done. 
% If we added something from adversarial generation to the retrieval experiment (like a combination of study 1 + 3) it would be complete. So for instance we could try fine-tuning the retriever with adversarial examples (and not just the QA model)

% \begin{figure}
%     \centering
%     \includegraphics[width=0.95\linewidth]{figures/segmentation/webqa_segment_infill.png}
%     \caption{Example of the segmentation substitution pipeline from the WebQA task.}
%     % d5c76d760dba11ecb1e81171463288e9
%     \label{fig:seg_sub_pipeline}
% \end{figure}



% Retrieval augmented generation (RAG) with zero-shot prompting and fine-tuning Large Language Models (LLMs) have become the go-to methods for tasks relying on information retrieval and text generation. In many cases the LLMs parametric memory can sufficiently generalize to answer questions without being provided with retrieval mechanisms for out-of-domain knowledge. However, LLMs often hallucinate and provide wrong information in certain scenarios. This problem is amplified even further on open-domain Question Answering (QA) tasks involving multiple modalities. Grounded text generation using retrieved sources \citep{lewis2021retrievalaugmented} has been extensively studied for text-to-text QA tasks, but its application in multimodal settings has not been studied as much.


% Multimodal reasoning and question answering have gained prominence in recent research endeavors, with an increasing emphasis on handling various forms of data, particularly text and images. In this study, we address a specific gap in the existing literature by focusing on the development of a versatile multihop model capable of accommodating varying numbers of input images.

% Our motivation for this research lies in the growing complexity of answering questions using information on the web, where the challenge of navigating the open-domain setting is further complicated by the presence of multiple modalities and sometimes requires reasoning over multiple sources. WebQA is an ideal dataset on which to compare performance of finetuned RAG systems against general purpose LLMs; it is multimodal, with correct answers requiring reasoning over image and text sources. It is multihop, requiring a complex reasoning process over multiple sources. Finally, WebQA questions from different categories can be broken down into subdomains to analyze performance over domains of varying cardinality.

% Motivated by the real-world challenges of building retrieval and question answering (QA) systems, we design and finetune a closed domain, multimodal, multihop QA model, that is capable of reasoning over a varying number of sources taken as input from an external retriever module. This research contributes to the relatively underexplored domain of multihop reasoning across various input sources and modalities. Our goal is to explore the challenges posed by these scenarios and develop strategies that enable QA models to retrieve relevant information, conduct logical or numerical reasoning across diverse modalities, and generate coherent responses in natural language. To our knowledge, this is the first application of the Fusion-in-Decoder (FiD) architecture \cite{tanaka_slidevqa_2023, nlvr2} that is shown to work with a variable number of inputs, enabling multi-hop reasoning over sources.

% In-Context Learning refers to the ability of LLMs to perform any task by simply providing examples in the input prompt \citep{dong2022survey,min2022rethinking}. Inspired by this research, we propose a method to use the LLM itself as a multimodal retriever, potentially eschewing the requirement of a distinct retrieval module, thereby allowing the design of simpler retrieval-augmented QA systems. We dub this method In-Context Retrieval Language Modeling (RLM). To the best of the authors knowledge, In-Content RLM is disparate from other retrieval augmented approaches which utilize external retrieval modules \citep{incontext_rag,chen_murag_2022,liu_universal_2023}. Despite being a natural extension of In-Context learning, In-Context RLM has not yet been studied empirically.

% To expand on our contribution of In-Context Retrieval, this stems from the well-researched in-context learning of LLMs. In-context learning is the ability of a model to perform any task given a sufficient context window \citep{dong2022survey,min2022rethinking}. Such tasks could include retrieval and ranking, but typically, the go-to solution for tasks requiring retrieval has been RAG. To the best of the authors knowledge, In-Context Retrieval is distinct from In-Context Retrieval Augmented Language Modelling (RALM), and despite being a natural extension of In-Context learning, In-Context Retrieval has not yet been shown empirically.

% Finally, we explore the tradeoff between using zero-shot prompting LLMs and the fine-tuning approach. While we find that, overall, GPT-4o obtains SoTA performance on the WebQA task, outperforming the accuracy of existing finetuned RAG approaches by 7\%, finetuned approaches still perform better on more restricted subdomains\footnote{``In-Context RLM" @ \url{https://eval.ai/web/challenges/challenge-page/1255/leaderboard/3168}}. Finally, we validate that GPT-4o is relying on retrieval abilities to solve the task; we find that GPT-4o is capable of retrieving relevant sources in the presence of distractors and furthermore, when GPT-4o fails to retrieve correct sources, it answers incorrectly 75\% of the time, meaning that it is not relying on parametric memory for this task.

% \paragraph{Contributions}
% Based on our experimentation and analysis on the WebQA benchmark, we make the following contributions:
% \begin{itemize}
%     \item Propose a new architecture for multimodal multihop QA that takes variable number of input sources inspired by the Fusion-in-Decoder method.
%     \item Comparison of general purpose LLMs vs specialized models on the WebQA benchmark.
%     \item Observation of In-Context Multimodal Retrieval abilities of GPT-4o and that it does not rely on parametric memory for multimodal QA.
%     \item Analysis of relationship between retrieval and QA task performance.
%     \item Analysis of task and query complexity on the performance of retrieval and QA tasks.
% \end{itemize}
















% Throughout this paper, we will present our methodology, experiments, and findings, emphasizing our approach to multihop reasoning over varying numbers of input images. We believe that our work contributes to a deeper understanding of multimodal reasoning and has the potential to enhance the capabilities of question-answering systems in the intricate, multimodal landscape of web-based information.
\section{Related Work}\label{sec:related_work}

\subsection{\revision{Writing and AI}}
\revision{
The integration of AI has become a central topic in recent HCI research on writing assistant systems. 
\citet{Lee2024dsiiwa} recently summarized trends and mapped out a comprehensive design space for this area of work.
Many of their surveyed AI tools exhibit a ``fragmentation'' of text and UIs, as described by \citet{Buschek2024collage}: These tools introduce snippets, cards, pop-ups, and so on, to make space for prompt input and AI output. %
}

\revision{
However, the limited space on mobile devices makes these fragmented interfaces an unsuitable design choice for writing on the go.
Traditionally, mobile text entry research integrated ``intelligent'' features into the keyboard and text (e.g. word suggestions, auto-corrections~\cite{Palin2019, Banovic2019, Quinn2016}). However, today, LLMs are capable of much larger text contributions, beyond individual words or short phrases that had found their place at the top of mobile keyboards. 
Notably, none of the 115 tools surveyed by \citet{Lee2024dsiiwa} is explicitly designed for mobile touch use (beyond keyboard apps), highlighting a gap in research and practical applicability for this context.}

\revision{At the outset of our research, this motivated us to first chart a novel design space for mobile touch interaction with LLMs. We then used this space to inform our new design of gesture-controlled text generation specifically tailored for writing on mobile devices. Concretely, our resulting design supports users in triggering text generation, anticipating its length, and potentially curating it (cutting, regenerating) -- in one continuous interaction via familiar touch gestures (spread, pinch). At the same time, it supports users in handling the irregular system latency of text generation with a new visual control loop design (feedforward/feedback). %
}

\subsection{Beyond Prompts and Towards Continuous Interaction}
With \textit{DirectGPT}, \citet{directGPT} recently explored direct manipulation paradigms for interacting with LLMs, fundamentally inspiring our work. 
We extend this to touch gestures and mobile devices. \revision{In the future, we see} the potential for combining these concepts, such as integrating our gestures for text generation with the drag-and-drop interactions by \citet{directGPT} to refer to objects beyond text.
\revision{\textit{DirectGPT} operates in a discrete, turn-based interaction model: While the user's direct manipulation actions might be continuous (e.g. drag and drop), they ultimately result in a prompt that is ``done and ready'' to send to the LLM. The user then evaluates the resulting output to inform their next prompting actions.}
Similarly, \textit{TaleBrush} by \citet{Chung2022taleBrush} lets users steer story character narratives via drawn lines, using continuous (mouse) input. It also generates text only after input completion.
\revision{In contrast, our approach aims for continuous direct interaction, where continuous user actions (e.g. spreading fingers) are automatically and continuously sent to the LLM in the background, and output is continuously updated in the UI in parallel, approaching a new ``closed'' direct manipulation control loop.} 








Current systems like ChatGPT already use continuous displays for output, that is, adding one word at a time, to simulate an impression of ``writing'' and to handle computational latency in the UI.
Building on this, \citet{Lehmann2022suggVsCont} evaluated a design where users intervene in an LLM's continuous output stream for co-writing on smartphones. This pushed users into an editing role, requiring them to delete and modify text to align it with their intentions.

This role distribution -- AI drafts, user edits -- may suit cases where users prefer avoiding manual typing to generate initial drafts. Mobile devices, where text entry can be cumbersome~\cite{Kristensson2014inviscid}, are a prime example. 
However, research shows that AI-generated text often fails to meet user expectations, particularly in interpersonal communication~\cite{Fu2024texttoself, Liu2022aimailperception, Robertson2021cantreply}, and tends to be verbose~\cite{Fu2024texttoself}. These limitations highlight that while adding text to drafts is relatively simple, interacting with generated text is far more challenging, despite clear motivations for doing so.
\revision{We explore continuous direct interactions to address these challenges in the space-constrained context of mobile touch devices.} %


\subsection{Designing Mobile Touch Gesture Controls}
One \revision{existing} gesture-based UI for mobile text entry is the word-gesture keyboard~\cite{Zhai2012gesturekbMagazine}. It has inspired gestures for formatting~\cite{Alvina2017commandboard}, and can be combined with gestures for triggering word corrections~\cite{Cui2020justcorrect, Zhang2019typthencorrect}. \revision{These gestures are typically visualised via finger trajectories. More broadly, visual feedback} is a key component of designing gesture controls.


\subsubsection{Feedback via Additional Visual Elements}
The well-known \textit{marking menus}~\cite{Kurtenbach1993markingmenus} \revision{draw} the pointer trajectory as a line, adding a line segment per submenu the user \revision{moves} through. This guides learning of gesture mappings and can then be ignored by expert users~\cite{Kurtenbach1994markingmenus}. 
With \textit{Fieldward} and \textit{Pathward}, \citet{Malloch2017fieldpathward} designed visual feedback that is continuously updated while moving the finger, to guide users in defining a distinguishable set of gestures. Many gestures in our context of interacting with an LLM are also likely to benefit from visual feedback that is updated ``live'', to create a closed control loop.
Concrete requirements for visual feedback differ in our case: For example, visualising finger trajectories may not be suitable, since it might occlude the text the user is interacting with. 
Occlusion has been addressed in concepts for mobile text selection, such as a pop-up (callout) displayed above the finger position~\cite{Ishii2016callout, Vogel2007shift}. This is designed for focusing on the word or character at the cursor. In contrast, many LLM-based applications are likely \revision{interested in larger scopes of text} (e.g. generating or revising a paragraph or whole draft).



\subsubsection{Feedback without Additional Visual Elements}
Other established mobile touch gestures do not need additional visual markup as feedback, because their triggered transformations are already visual in nature. Examples include scrolling by swiping with the finger vertically, and zooming in and out of a text or image via spreading and pinching fingers. The resulting change in content size and/or location \textit{is} the feedback. \revision{In contrast,} we are not interested in geometric zooming or offsets, but in gestures that change the text content in some way.


\subsubsection{Summary and Implications for our Research}
In summary, related literature and industry standards present two kinds of design for gesture control loops on mobile devices -- those that add extra visual elements and those that do not. 
Notably, text is usually in the latter category or text-related visual feedback elements are local in nature. However, we require a novel combination here -- feedback for continuously manipulating (larger scope) text content with a gesture. \revision{We next describe this open design space in more detail.}























\section{A Design Space for Mobile Touch Interaction with LLMs}
\label{sec:design_space}

We developed a design space for touch-based LLM interaction through focused brainstorming sessions and affinity diagramming \cite{hartson2012ux} in our research team. We gathered high-level ideas, examples from research and industry, and explored various mappings of LLM capabilities to touch gestures. 
\revision{The resulting space} has four dimensions (\cref{fig:design_space}):
(1) Input,
(2) Referential Interaction,
(3) Textual Interaction,
(4) Output.

\begin{figure*}
    \centering
    \includegraphics[width=1\linewidth]{figures/design_space}
    \caption{Overview of our design space for mobile touch interaction with generative AI for text, with four dimensions (left to right) and subdimensions (values at vertical lines). Coloured streams indicate the design choices for the two concrete touch gesture controls that we designed, implemented, and evaluated in this paper.}
    \Description{This figure represents a design space for mobile touch interaction with generative AI for text. It spans four dimensions: Input, Referential Interaction, Text Interaction, and Output, each further divided into subdimensions. Coloured streams (blue for ``\Spread{}'' and red for ``\Pinch{}'') highlight the specific design choices evaluated in the paper. Both gestures have the following choices highlighted: ``direct'', ``continuous'', and ``feedback'' for the ``input'' dimension, ``select sentence on touch down'' for the ``referential interaction'' dimension, ``diegetic'' and ``additive'' is marked for spread in the ``text interaction'' dimension whereas pinch has ``diegetic'' and ``subtractive'' highlighted. In the output dimension both have ``audit'', ``diegetic'', and ``feedback marked''.}
    \label{fig:design_space}
\end{figure*}

\subsection{Input}
This dimension captures three fundamental choices for interaction designs with touch input. %

\subsubsection{Direct vs Indirect}
Indirect input for interaction with an LLM requires additional UI elements (e.g. virtual keyboard, floating objects, buttons) and/or uses an external application beyond the main writing interface, such as when copy-pasting in text generated with ChatGPT.
In contrast, direct input refers to touch gestures directly on the text that the LLM should operate on. Mobile device users are used to \revision{typical gestures such as} tap, swipe, pinch, spread, and drag. 

\subsubsection{Discrete vs Continuous}
The most common type of touch input today is a discrete event -- tapping a button. Beyond this, there are touch gestures that unfold continuously on the screen over time, such as a swipe or pinch.

\subsubsection{Feedback} 
This captures if and how feedback is given to the user leading up to the final input and/or after it has been completed. For example, this might include a visual trajectory line while performing a gesture. As another example, text selection feedback can serve in this role, such as when highlighting the current selection with a coloured background, updated ``live'' while swiping over it.



\subsection{Referential Interaction} 
Inspired by work on referential and verbal acts in HCI~\cite{wolff1998acting}, we separate these aspects in our design space. Our referential dimension captures \textit{how users interact with text to set the context}, scope, reference, and so on, as a precursor to LLM operations on that text. Visual (UI) changes resulting from these interactions are typically ephemeral and not meant to become a lasting part of the text document.
\revision{As a common example, text selection} may involve (long-)taps, moving the finger over the text, and/or dragging UI instruments~\cite{BeaudouinLafon2000instrumental} (e.g. start/end markers). We decided to keep this dimension at a high level, as it is not our main focus. Possible subdimensions to explore in the future include the design of selection procedures and markers \revision{for} ``pointing'' an LLM at text parts (cf.~\cite{directGPT}).


\subsection{Textual Interaction}
This captures \textit{how users interact with the content (semantics) and style of the text}. These interactions are performed to trigger and parametrise the LLM to operate on the text in a lasting way.
Examples from research on AI writing support~\cite{Lee2024dsiiwa} include extending or shortening text, summarising parts of it, and rewriting, possibly with a specific \revision{goal} (e.g. adjusting tone or level of formality).
To structure this space of possible \revision{AI} text operations, here we focus on two aspects we identified as most fundamental \revision{in} our context.

\subsubsection{Diegetic vs Non-diegetic Text}
This captures whether text involved in interaction (e.g. text selected with a preceding referential interaction) is \textit{diegetic}, \textit{non-diegetic}, or a mix of both (cf.~\cite{Buschek2024collage, Dang2023choice}). Diegetic text \revision{is intended to be in} the final text (e.g. a sentence in a story to improve with AI), while non-diegetic text is not (e.g. a ``todo'' note or prompt for the LLM).

\subsubsection{Additive vs Subtractive vs Transformative}
This subdimension captures whether the AI text operation involved in the interaction is \textit{additive}, \textit{subtractive}, \textit{transformative} or a mix. Additive operations add words, such as when extending a draft. Subtractive refers to the opposite, such as when the system removes words to shorten a text semantically (e.g. by summarising parts of it) or mechanically (e.g. cutting the last 3 sentences). Finally, transformative refers to AI operations that change text without clearly adding or removing \revision{words}. For example, a user might use AI to revise their draft with the goal of achieving a consistently formal tone. This might involve both cutting informal phrases as well as adding parts to adhere to formal expectations.


\subsection{Output}
This dimension captures how the results of the interaction -- the result of the AI's text operation -- is presented.

\subsubsection{Audit vs Edit} 
\revision{Text changes can be} direct edits by the AI (e.g. fixing spelling mistakes) or ``audits'', typically suggestions for users to accept or reject (cf. \cite{Cooper2014aboutface}). 

\subsubsection{Diegetic vs Non-diegetic Text}
AI text resulting from the interaction \revision{can be} diegetic or non-diegetic, or a mix of both. For example, an added sentence suggestion is diegetic, as it is intended to become a part of the text. In contrast, an added feedback comment is non-diegetic AI output (cf.~\cite{Buschek2024collage}).

\subsubsection{Feedback} 
This subdimension captures if and how feedback is given on the way to the final result of an AI text operation and/or after it has been completed. For example, a loading indicator \revision{might show that the} LLM is computing the output. Text itself might be used \revision{for this}, such as in ChatGPT, which adds one word at a time.


\subsection{Applying the design space: \Spread{}, \Pinch{}}
\label{sec:ds_applied}

In this paper, we explore two concrete designs arising from our design space. \cref{sec:implementation} describes \revision{their implementation.} Here, we situate them in our design space to illustrate how the space can be used to generate designs and systematically describe them. %

\subsubsection{\Spread{}}
This touch interaction uses \textit{direct} and \textit{continuous} input (spread gesture with two fingers, as known from zooming) on \textit{diegetic} text (user's draft) to control an \textit{additive} LLM capability with \textit{diegetic} output (text generation) that users need to confirm (\textit{audit}). 
The \textit{referential interaction} is realised by interpreting the first touch event of the two fingers as a selection of the sentence, after which the generated text should be added.
The \textit{feedback} for both input and output is combined into one visual element -- an indicator of the length of the generated text, extended ``live'' while spreading the fingers. \revision{Our study (\cref{sec:method}) compares} two variants of this design (lines, bubbles) with a baseline (no feedback).


\subsubsection{\Pinch{}}
This interaction is the complement to the above: It uses \textit{direct} and \textit{continuous} input (pinch gesture with two fingers) on \textit{diegetic} text (user's draft) to control a \textit{subtractive} operation with a \textit{diegetic} result (text deletion) that users need to confirm (\textit{audit}). 
\textit{Referential interaction} and \textit{feedback} are realised as described above. However, our visual design differs to make it easier for users to perceive the effect at a glance (e.g. red for delete, blue for extend). Our implementation of \pinch{} does not require an LLM.






\section{Concept and Implementation}
\label{sec:implementation}
We propose the concept of controlling LLMs for text operations on mobile devices via touch gestures.
To explore this, we developed a prototype, as described next. %

\subsection{Concept Development}
\revision{Here, we describe our concept development.}


\subsubsection{User Feedback and Design Iteration}
Our prototype UI consists of a text field with spread and pinch gestures that trigger text extensions or shortenings based on finger distance. It evolved iteratively with user feedback from a formative study (N=17):  

\begin{itemize}
\item \textit{Cursor and Sentence Selection:} The initial prototype featured no cursor, and text was generated or deleted at the end of a paragraph. As users wished for more control, we added functionality to select a sentence with the first touch of a gesture, placing a cursor at its end (\cref{fig:bubbles}).
\item \textit{Text-Length Indicator:} The lack of feedback during gesture execution caused confusion due to the short delay in text generation. This motivated us to explore visual feedback designs, eventually leading to the \visbubble{} design (\cref{fig:bubbles}). %
\item \textit{Building Structure:} To address users feeling overwhelmed by rapidly extending text, we added structure by visually separating each fully generated sentence and surrounding individual words in incomplete sentences with bubbles.
\item \textit{Displaying the Token Stream:} Users preferred to see generated text immediately, rather than estimating its length first and being surprised by the content. In response, we began streaming the incoming tokens directly into the visual feedback (e.g. bubbles) as soon as they were generated.
\end{itemize}

\begin{figure*}[t]
     \centering
     \begin{subfigure}[b]{0.475\textwidth}
        \centering
        \includegraphics[width=0.8\linewidth]{figures/frontend}
        \caption{\Spread{} gesture in the editor: the red cursor marks the starting point for generation. Blue bubbles represent the expanding text length, filling with words as they arrive from the backend. Once a full sentence is generated, the individual word bubbles merge into a large green bubble, indicating the completion of the sentence.}
        \label{fig:add_bubbles}
     \end{subfigure}
     \hfill
     \begin{subfigure}[b]{0.475\textwidth}
        \centering
        \includegraphics[width=0.8\linewidth]{figures/remove_bubbles}
        \caption{\Pinch{} gesture in the editor: a red cursor marks the starting point for word removal. Dark red bubbles highlight words already selected for deletion, the lighter (and slightly larger) red bubble captures a transitional word animation from unselected to selected.}
        \label{fig:remove_bubbles}
     \end{subfigure}
     \caption{Our frontend enables gesture interaction with text on mobile devices. We employ our Bubbles visual feedback design to communicate essential information to the user.}
     \Description{This figure contains two panels that show different types of gestures -- '\spread{}' and '\pinch{}' -- used to interact with text in a mobile text editor. Each panel illustrates how 'Bubbles' visual feedback helps users understand the effects of these gestures. Panel (a): Depicts the '\spread{}' gesture. A passage of text is shown with a red cursor marking the point where text generation will begin. As the user spreads their fingers, blue word bubbles appear in the text to represent the words being generated in real-time by the AI. Once the user completes a full sentence, the individual blue word bubbles merge into one large green bubble, signifying the end of the sentence. The selected text, shown in green, has been generated during the gesture. This panel highlights the expanding nature of text as the gesture progresses. Panel (b): Illustrates the '\pinch{}' gesture. In this text passage, the red cursor again marks the starting point for the interaction, but in this case, words are being removed. Dark red bubbles appear around the words to be deleted as the user pinches their fingers together. Lighter red bubbles indicate transitional stages of word selection for removal. The dark red bubbles fully encompass words marked for deletion, making it clear which text will be removed.}
     \label{fig:bubbles}
\end{figure*}

\subsubsection{Designing a (Visual) Control Loop}
\label{sec:vis_design}
Direct touch interaction with LLMs requires novel feedback designs (\cref{sec:related_work}). 
We designed a control loop for \spread{} and \pinch{}. %
Here we list our \textit{design challenges and goals}, extracted from the literature and our formative study:
\begin{enumerate} 
\item \textit{Presenting All Relevant Information:}
To avoid context switching, all necessary information (e.g., input, output, system state) should be embedded in the text display. Ideally, no additional UI elements, like sidebars or chat boxes, are \revision{needed}.
\item \textit{Managing Generation Latency and Token Streaming:} AI text generation introduces continuous token streams with latency. While less critical in chatbot \revision{UIs, this needs to be handled well for} continuous interaction: \revision{For example, latency may} slow down the experience, while real-time generation may overwhelm users if content appears too fast. %
\item \textit{Integration with Mobile Touch Gestures:} %
\revision{The design needs to handle} these LLM-specific issues in a visual control loop that meets the demands of mobile touch interaction.
\end{enumerate}

We outline four design goals to address these challenges and explain how our implementation meets each of them:
\begin{enumerate} 
\item \textit{Input Visibility:}
The user's input to the LLM for text extension should always be clear. %
\revision{Besides the textual context, this also includes the user's desired amount of text to be generated.}
Our feedback design (\cref{fig:bubbles}) uses colour to \revision{indicate this and to} separate newly generated text from existing text.
\item \textit{Output Clarity:}
Generated/removed text should be visible immediately \revision{when available from the LLM}. %
Our visualisation fills the text-length indicator (e.g. bubbles) ``live''. To \revision{further} enhance clarity, the blue word bubbles merge into a green sentence bubble upon completing a sentence (\cref{fig:teaser}). %

\item \textit{System State Communication:}  
The system’s current state -- processing, generating, or encountering latency -- should be clearly communicated. 
Visual cues must indicate when it is ready for input or generating output. We support this by animating all state changes (e.g. fading in bubbles). %

\item \textit{Integration with Mobile Writing Workflows:}
The system should integrate seamlessly into existing \revision{mobile} writing workflows.
All interactions should occur within the text \revision{UI}, with no context switches.
\revision{Our feedback loop with our \visbubble{} design} allows users to make real-time adjustments as they engage with the LLM \revision{through} familiar spread and pinch gestures.
As UI objects, the bubbles can support further interaction possibilities. While not in our focus, we explore some ideas in our prototype (\cref{sec:additional_features}).
\end{enumerate}


\subsection{Backend Implementation}\label{sec:backend}
The backend \revision{connects} frontend and OpenAI API\footnote{\url{https://platform.openai.com/}} (gpt-4o-mini).
It %
was implemented as a Node.js/Express application. 
We leverage the ``/aiCompletionStream'' API endpoint which forwards each incremental delta received from OpenAI directly to the frontend.
\revision{This setup achieved a mean latency of 242 ms (median: 98 ms, std: 262 ms) from gesture initiation to word display, during our study.}
\cref{sec:appendix_prompts} provides details on our prompting templates.

\subsection{Frontend}
The frontend (\cref{fig:bubbles}) is a web application in React.

\subsubsection{Selecting the Sentence}
\revision{When the first finger of a pinch/spread gesture touches the screen, we find the nearest sentence with a spiral scanning algorithm (cf. \cite{photonics11060540}), which measures touch-sentence distances around} the touch point in an outward spiral pattern.
\revision{Upon detecting the second finger, the system places a red cursor at the end of this nearest sentence. This end is either punctuation (``.'', ``!'', ``?'') or the last word.}

\subsubsection{Mapping Movement to Words}
\label{sec:dist_word_mapping}
The system translates \revision{the user's finger movements} into \revision{an internal} word count, \revision{increasing or decreasing it} for positive or negative distance \revision{changes between the two fingers}, respectively. 
Based on our usability testing, \revision{we map} 1.75\,mm \revision{of distance change to generating or removing one} word. Note that the ``optimum'' might be \revision{device-}specific. %

\subsubsection{Triggering Changes}
When the user lifts one or both fingers, the gesture ends, and the red cursor disappears.
A widget appears at the right \revision{screen} edge (\cref{fig:teaser}.5), \revision{with buttons to confirm or} reject the changes. 
\revision{Confirming a spread integrates the generated text, while confirming a pinch deletes the marked words.} 
Rejecting reverts all changes. 

Alternatively, \revision{the user can resume} the gesture by placing two fingers back on the screen, which makes the cursor reappear and the gesture continue \revision{from where it was left off}. 

\subsubsection{Text-length Indicators}\label{sec:impl_text_length_ind}
We introduce \textit{\visbubble} (\cref{fig:bubbles}) as a feedback design that indicates text-length changes in the control loop for touch-based text generation, addressing the three challenges detailed in \cref{sec:vis_design}.

This design also helps with managing the irregular latency of continuous text generation with LLMs by acting as placeholders, which are filled with words as they become available.
\revision{Concretely, while the fingers move,} the expected word count is estimated based on finger distance changes (see \cref{sec:dist_word_mapping}).
Starting at the cursor position, \revision{the system} adds bubbles one by one until \revision{reaching the current} word count \revision{value}. %
As \revision{actual} generated words become available from the LLM, they are inserted into the bubbles (\cref{sec:impl_text_generation}).

\revision{Thus,} bubbles \revision{are} placeholders for generating text with the spread gesture. \revision{These bubbles} are \textit{blue} (\cref{fig:add_bubbles}).
Once a full sentence is generated, they merge into one \textit{green} bubble. 
\revision{In contrast}, \textit{red} bubbles encapsulate words to be removed by pinching.

The width of each \textit{empty} blue bubble (placeholder) is determined by a function that simulates (randomly) varying word lengths. %
For each, we roll a length between 5 and 10 characters as assumed typical word lengths, and assume five pixels per character. 
Finally, when a placeholder bubble \revision{appears, it blinks shortly} to indicate that a word will appear. %


\subsubsection{Text Generation}\label{sec:impl_text_generation}
\revision{During the gesture, we use a buffer:} Text generation is triggered only if the buffer of already generated words is insufficient to fill all empty bubbles. \revision{If so,} a system prompt (see \cref{prompt:extend}) is sent to the backend together with the text  \revision{preceding} the cursor, plus all words generated up to this point.
The backend streams the response in chunks, allowing placeholders to be filled in real-time without waiting for the entire response.

\revision{In detail,} streamed chunks are processed by handling incomplete words, filtering and combining special characters, updating buffers, and filling bubbles. %
In pinch mode, empty bubbles are removed, text-filled ones push their content back to the buffer, end-of-sentence bubbles lose their last word, and bubbles in incomplete sentences remove the last word without colour change, ensuring smooth feedback.
As a result, if users are unsatisfied with a generated sentence, they can pinch the sentence to remove it and spread again to regenerate a new sentence.

\subsection{Additional Features}\label{sec:additional_features}
We explored extensions to our core functionality that aim to improve usability and extend interaction possibilities.

\begin{figure*}[t]
     \centering
     \begin{subfigure}[b]{0.475\textwidth}
        \centering
        \includegraphics[width=0.9\linewidth]{figures/long_press_words}
        \caption{When performing a longpress on a word bubble, suggestions appear beneath it.\newline}
        \label{fig:long_press_words}
     \end{subfigure}
     \hfill
     \begin{subfigure}[b]{0.475\textwidth}
        \centering
        \includegraphics[width=0.82\linewidth]{figures/long_press_sentence_short}
        \caption{Long-pressing a full sentence presents users with a neutral and professional tone adjustment, along with the option to enter a custom prompt for further refinement.}
        \label{fig:long_press_sentence}
     \end{subfigure}
     \caption{Our long-press feature allows users to request synonyms on a word level (a) or tone adjustments on a sentence level (b). When words or sentences are replaced by selecting an alternative, they swap places to allow users to revert their action.}
     \Description{This figure shows two examples of longpress features designed for text editing. Image (a): The left-hand side displays a text passage in a text editor. When a user long-presses on the word 'grappled,' a suggestion box appears underneath, showing four alternative synonyms: 'struggled,' 'wrestled,' 'battled,' and 'fought.' The user can select one of these suggestions to replace the word 'grappled.' There is a checkmark and an 'X' icon in the bottom-right corner of the screen, allowing the user to confirm or reject their changes. Image (b): The right-hand side shows a different part of the text editor, where the user has long-pressed a full sentence. The interface presents three options for tone adjustment: 'Neutral,' 'Professional,' and 'Custom.' Each option includes a tone-specific sentence replacement, such as 'Furthermore, exploring various angles and perspectives can result in more dynamic and engaging compositions' for the 'Professional' option. The 'Custom' option allows the user to input a new sentence manually. A 'Generate' button is available for applying the changes, and like in (a), checkmark and 'X' icons provide confirmation or rejection options.}
     \label{fig:long_press}
\end{figure*}

\subsubsection{Sentence-Snap}
Pre-study findings showed that users often prefer generating one complete sentence at a time. 
To support this, we introduced the ``Sentence-Snap'' feature. 
When users perform a rapid spread gesture and quickly lift their fingers, the generation process halts after one complete sentence. 

\subsubsection{Longpress}
\label{sec:longpress}
In response to pre-study feedback, we introduced the ``Longpress'' feature, a pop-up that enables users to modify generated words and sentences (\cref{fig:long_press}). 
This is triggered \revision{by} touching a bubble for 500\,ms. %
This design also illustrates how the gestures can be combined with \revision{other UI} elements.


     

\paragraph{Adjusting Words (\cref{fig:long_press_words}):} When a long press is applied to a word bubble, the LLM is prompted (\cref{prompt:synonym}) to provide up to five synonyms, which are displayed directly in the text beneath the selected word as individual bubbles. 
Tapping on a synonym replaces the original word in the text. 
To dismiss all suggestions, the user can tap anywhere outside the synonyms.

\paragraph{Adjusting Sentences (\cref{fig:long_press_sentence}):} When a long press is applied to a sentence bubble, the LLM is prompted (\cref{prompt:rewrite_sentence}) to rewrite the sentence in two different tones: ``neutral'' and ``professional.'' \revision{We chose these} based on user feedback and inspired by industry applications. 
Rewritten sentences appear directly beneath the selected sentence in two separate boxes.
Tapping on a box replaces the sentence in the placeholder with the newly generated style, and a ``previous'' box appears, allowing the user to revert the change. 
\revision{Users can also} input their own specifications (``Custom'' textbox) \revision{that are then used as a prompt} (see \cref{prompt:custom_sentence}). 


\section{Method}\label{sec:method}
We conducted a within-subject \revision{lab} study to investigate touch gesture interaction for LLM-powered text length modification. \revision{Concretely, we compared} different feedback designs and evaluated \revision{the gestures} against a baseline chatbot interface. \cref{fig:method_overview} shows the study design and procedure.


\begin{figure*}
    \centering
    \includegraphics[width=\linewidth]{figures/method_overview}
    \caption{Overview of our user study design and procedure.}
    \Description{This figure provides an overview of the user study design and procedure, highlighting key steps, experiments, and tasks performed during the study, as they will be described in the following sections. The figure is structured into four main sections: Intro, Experiment 1, Experiment 2, and Conclusion. For each, it illustrates the content using screenshots and icons.}
    \label{fig:method_overview}
\end{figure*}


\subsection{Experimental Design}
\revision{The study consisted of} two experiments and a ``hands on'' semi-structured interview.

\subsubsection{Experiment 1: Comparing Visual Feedback Loop Designs}
The goals for Experiment 1 were to observe how participants perceive the general concept, collect data on \spread{} and \pinch{} gesture execution, and compare three feedback designs.


\paragraph{Conditions}
The \textit{independent variable} was ``Feedback'' with three conditions: 
\visbubble{} -- the design shown in \cref{fig:teaser} and described in the previous section; 
\visline{} -- an alternative inspired by typical text selection markup, i.e. colouring the background of the text line (incl. without text when used in the placeholder role, cf.~\cref{sec:impl_text_length_ind}); 
and \visnone{} -- a baseline with no visual feedback beyond the text itself.
We selected these designs based on their ability to provide varying levels of feedback on text generation and modification (none vs length with \visline{} vs length + word count with \visbubble).
Their order was counterbalanced.


\paragraph{Measures}
As \textit{dependent variables}, we logged interaction metrics and collected subjective feedback through think-aloud protocols and questionnaires.


\paragraph{Tasks}
Both text extension and shortening were tested using short, neutral and unopinionated texts provided to participants as starting points (see \cref{sec:appendix_texts_exp1}).
We used the same texts for all conditions as the focus was not on reading and text comprehension, but repetitive execution of the gestures.

To explore different text length modifications, we asked participants to complete/remove one incomplete sentence (five repetitions each), extend/shorten the text by one full sentence (three repetitions each), extend/shorten the text by three sentences (three repetitions each), and perform a combination of extension and shortening tasks (twice).
The number of repetitions was chosen to balance task complexity with participant fatigue and potential learning effects.
To avoid additional confounds, we excluded the long-press feature (\cref{sec:longpress}) and disabled the phone's integrated keyboard.



\subsubsection{Experiment 2: Comparing Direct Touch Interaction Against a Chatbot UI}
This experiment aimed to provide deeper insights into how participants perceive the concept of controlling AI with touch gestures for text generation, compared to a typical chatbot UI. %

\paragraph{Conditions}
Interaction technique was the \textit{independent variable}, with two conditions: 
Direct Touch Interaction -- our proposed gesture-based text length modification using the Bubbles visualisation;
Chatbot Interaction -- a baseline imitating traditional chatbot-like interaction (\cref{fig:gpt_interface}).
\revision{We selected this baseline as a widely adopted, state-of-the-art approach for interacting with LLMs today, especially on mobile devices (e.g. see apps for ChatGPT\footnote{\url{https://openai.com/chatgpt/download/}} and Copilot\footnote{\url{https://play.google.com/store/apps/details?id=com.microsoft.copilot}})}. 
\revision{Suggestions and autocorrection were included via the normal phone keyboard.}
We counterbalanced the two conditions.

\paragraph{Measures}
As \textit{dependent variables}, we logged quantitative data (e.g. interaction logs) and qualitative data (e.g., observations, feedback, questionnaires).

\paragraph{Tasks}
We prepared short, unopinionated texts for participants to work with (\cref{sec:appendix_texts_exp2}).
They \revision{included} sentences clearly out of place (e.g. a topic shift from gardening to skating) or had missing parts (e.g. starting an enumeration but listing only one item).

Instead of giving quantifiable instructions on text length as in Experiment 1 (i.e. ``Remove \textit{one} sentence.''), this approach allowed us to give more abstract instructions. These required participants to read and understand the text (i.e. ``Remove the \textit{irrelevant} sentence.'').
Here we used different texts for the two conditions.
As in Experiment 1, tasks were considered completed when participants accepted the text length modification.

To explore more natural workflows in this more open task, we enabled the built-in mobile keyboard, and participants had access to the long-press feature.





\subsection{Apparatus and Materials}

\begin{figure*}[t]
     \centering
     \includegraphics[height=5.5cm]{figures/alternative_conditions}
     \caption{Examples of the UI in Experiment 1: In the \visline{} condition (left) coloured lines provide visual feedback on the change of text length. The design of the \visnone{} condition (right) offered no visual feedback beyond the text itself.}
     \Description{This figure shows two panels showing examples of the 'Lines' and 'NoVis' experimental conditions during Experiment 1 of the user study. Each panel demonstrates a moment of interaction where participants are tasked with editing text using different feedback methods. Panel (a): Lines Condition –- This panel shows the 'Lines' condition, where a coloured line is used to visualise changes in text length. The participant is tasked with adding three sentences to a text about climate change. The red cursor indicates the current insertion point. Green lines appear under newly generated text, highlighting the length of the text extension. The visual feedback helps the user keep track of the amount of text being added in real-time. Panel (b): NoVis Condition –- This panel illustrates the 'NoVis' condition, where no visual feedback is provided other than the text itself. The task asks the participant to add two sentences, accept the changes, and then remove one sentence. The red cursor shows where the participant is editing the text on photosynthesis. Without additional visual cues, participants must rely solely on the text to track their changes.}
     \label{fig:vis_conditions}
\end{figure*}

\subsubsection{Comparative Designs} %
\paragraph{Experiment 1}
For Experiment 1, we implemented two additional designs:
\visline{} \revision{indicated text-length as} a continuous green bar for generation, and a red bar \revision{for removing text}, but it had no placeholders for \revision{individual} words (\cref{fig:vis_conditions} left).
NoVis showed no text-length indicators at all (\cref{fig:vis_conditions} right).
In all three visualisations, generated text appeared as soon as the token stream arrived.


\paragraph{Experiment 2}
This experiment used our prototype with \visbubble.
For comparison, the same task was also completed using a chatbot-like UI, similar to the ChatGPT app.
To ensure consistent data logging across conditions and avoid differences in latency and logging frequency, we implemented the chatbot UI directly in our web app (\cref{fig:gpt_interface}), using the same LLM as for the gestures (\cref{sec:backend}). 
Participants could easily switch between the text editor and chatbot via a button at the top, making this ``simulated'' context switch faster than actual switching between separate apps.

\begin{figure}
    \centering
    \includegraphics[height=7cm]{figures/gpt_interface.png}
    \caption{Our imitation of a chatbot-like interface, resembling the ChatGPT app. The green text bubble contains the user's prompt. The chatbot returns a grey text bubble, where the irrelevant sentence has been removed. Users can copy the revised text using the blue 'Copy' button. A chat input field and 'Send' button allow further interaction.}
    \Description{This figure shows a mobile interface designed to imitate a chatbot-like interaction, resembling the ChatGPT app. The interface includes two sections: Upper section: A green text bubble provides a task prompt, instructing the chatbot to 'remove the irrelevant sentence from the text.' The text to be edited follows, discussing tomatoes and skateboards. The instruction asks the chatbot to remove the sentence about skateboards, which is not relevant to the topic about growing tomatoes. Lower section: After processing the request, the chatbot returns a grey text bubble with the revised text. The irrelevant sentence about skateboards has been removed, and the corrected sentence about tomatoes is provided. A blue 'Copy' button is displayed below the revised text, allowing the user to copy the corrected version. At the bottom, there is a standard chat input box, where the user can type a message, along with a 'Send' button to interact with the chatbot.}
    \label{fig:gpt_interface}
\end{figure}

\subsubsection{Study Environment}
We conducted the study in person at our university lab in a neutral office environment with minimal distractions. 
Participants were seated comfortably, with both arms resting on a table while holding the device (\cref{fig:lab_study}).

We used an iPhone 14 with a screen resolution of 2532 × 1170 pixels and a screen size of 6.06\,in (153.924\,mm) running iOS 17.6.1. 
At the start of each experiment, the experimenter opened our web app using the Chrome Browser, and started the screen recording.

\subsubsection{Experimental Software and Web Application} %
We hosted our prototype (\cref{sec:implementation}) locally on our server. 
It was embedded into a framework that handled the study logic, including briefings and in-app questionnaires, and managed counterbalancing and study progression (\cref{fig:study_elements}). %

\begin{figure}
    \centering
    \includegraphics[height=7cm]{figures/study_elements.png}
    \caption{Two study elements showing the screen participants saw during the study with our web app opened inside a browser (URL blurred for anonymity), and a screenshot of one part of our questionnaire with UI elements for study progression. A digital clock in red indicates that screen recording was active.}
    \Description{This figure displays two screenshots representing key elements of the user study interface. Left panel: The left-hand image shows the screen participants viewed during the study while interacting with the web app inside a browser (with the URL blurred for anonymity). The app is presented in a Google Chrome mobile browser and displays a text-editing task from the prototype. In this specific instance, the participant is engaged with a semi-structured interview where they modify a passage about time travel by interacting with highlighted text. The red-highlighted digital clock at the top indicates that screen recording is active. Other UI elements, including task progression indicators and buttons for skipping to the next task or resetting the prototype, are also visible. Right panel: The right-hand image shows a section of the usability questionnaire that participants filled out as part of the study. The UI includes a rating scale from 1 (strongly disagree) to 5 (strongly agree) for the statement 'I think that I would like to use this system frequently.' Navigation buttons ('Previous' and 'Next') allow participants to move between sections of the questionnaire.}
    \label{fig:study_elements}
\end{figure}





\begin{table*}
\footnotesize
\begin{tabular}{cccccccc}
\toprule
\textbf{P} & \textbf{Age} & \textbf{Gender} & \textbf{Occupation} & \textbf{WPM} & \makecell{\textbf{Experience with} \\ \textbf{Generated Texts}} & \textbf{Frequency of AI Usage} & \makecell{\textbf{Familiarity with LLMs} \\ \textbf{(Likert scale, 1/low to 5/high)}}  \\
\midrule
P1 & 22 & Male & Student & 30 & Yes (ChatGPT) & Occasionally & 5 \\
P2 & 20 & Male & Apprentice & 36 & Yes (ChatGPT) & Rarely & 4 \\
P3 & 20 & Male & Student & 38 & No & Rarely & 2 \\
P4 & 20 & Male & Student & 26 & No & Rarely & 1 \\
P5 & 22 & Male & Student & 53 & Yes (study related) & Occasionally & 5 \\
P6 & 20 & Male & Student & 37 & No & Rarely & 2 \\
P7 & 24 & Male & Student & 44 & Yes (ChatGPT) & Occasionally & 5 \\
P8 & 20 & Female & Student & 29 & Yes (ChatGPT) & Occasionally & 4 \\
P9 & 21 & Female & Student & 35 & No & Rarely & 1 \\
P10 & 48 & Female & Technical Assistant & 22 & No & Rarely & 1 \\
P11 & 60 & Female & Technical Assistant & 16 & No & Rarely & 1 \\
P12 & 19 & Male & HS Graduate & 34 & Yes (ChatGPT) & \revision{Rarely} & 2 \\
P13 & 59 & Female & Midwife & 10 & No & Rarely & 1 \\
P14 & 21 & Male & Student & 50 & Yes (ChatGPT) & Occasionally & 5 \\
\bottomrule
\end{tabular}
\caption{Overview of the participants.}
\Description{This table gives an overview of the participants.}
\label{tab:participants}
\end{table*}

\subsection{Participants}
We recruited 14 participants (5 female, 9 male) through our university network. 
Criteria for participation were high English proficiency and the ability to perform touch gestures on mobile devices. 
To ensure a more diverse range of perspectives and assess the accessibility of our concept across different demographics, we selected participants from distinct age groups and occupational backgrounds.
All participants were right-handed. %
While writing on mobile devices, most participants reported to use both hands and thumbs (11), two write one-handed with the thumb, and one with their index finger. 
All use their mobile devices (11 iOS, 3 Android) at least several times a day, mainly for writing messages (13), notes, comments and TODOs (3), and browsing the internet (3).
\cref{tab:participants} provides an overview. %

\subsection{Procedure}

\begin{figure}
    \centering
    \includegraphics[height=5cm]{figures/lab_study.jpg}
    \caption{A participant performing the \spread{} gesture using our prototype web app. The participant provided consent to publish this figure.}
    \Description{This figure shows a participant interacting with a mobile device, performing the '\spread{}' gesture using the prototype web app. The participant is seated at a desk in an office environment with computer monitors and other office supplies in the background. The mobile device is being held with both hands, and the participant's fingers are spread apart on the screen, indicating the gesture used to control text generation. A bottle of water and a sheet of paper with study-related materials are visible on the desk.}
    \label{fig:lab_study}
\end{figure}

Participants \revision{received} study documents in accordance with our university's regulations. %
After giving informed consent, participants filled out a demographic questionnaire.

We explained the experiment and demonstrated interaction with our prototype. 
This covered using gestures to adjust text length and the experimental conditions.
We encouraged participants to interact with the device in whichever way felt most comfortable. %
Furthermore, we asked them to ``think aloud'' while interacting. %
The phone was cleaned with disinfectant wipes between participants.

To \revision{assess} manual typing speed and help participants familiarize themselves with the device, they first completed a one-minute typing test, typing random English words. 
We answered all questions that came up.

The study took around 74 minutes on average.
Participants were compensated at a rate of €\,5 per (started) 20 minutes. %


\subsubsection{Experiment 1} %
\label{sec:procedure_exp1}
The same three tasks (extend, shorten, combination) were performed on the same texts (see \cref{sec:appendix_texts_exp1}) for all feedback designs, structured as follows:

First, participants extended the text by completing one incomplete sentence (5 repetitions), adding one sentence (3 repetitions), and adding three sentences (3 repetitions).
Second, they removed one incomplete sentence, one full sentence, and three sentences, repeating each task as previously.
Finally, participants performed combinations: 
In the first, they added two sentences and then removed one; 
in the second, they removed two sentences and then added one. 
These combination subtasks were performed only once.

After the final subtask \revision{per condition}, participants answered the in-app questionnaires and put the phone down \revision{for a} short break.
They then repeated the tasks with the next counterbalanced condition. 
\revision{Finally,} Experiment 1 concluded with a questionnaire and an extended break. %




\subsubsection{Experiment 2} %
\label{sec:procedure_exp2}
Here we instructed participants to edit a provided text by either removing sentences that did not fit the surrounding context or extending the text to fill logical or semantic gaps.
Beyond these abstract instructions, we did not specify the location of the text modifications or the amount of text to be removed or added. 
Thus, participants had to determine for themselves when they had changed the text to their satisfaction.
\revision{This was repeated for} three different texts per condition (see \cref{sec:appendix_texts_exp2}).
After the first condition, we asked participants to answer in-app questionnaires and put the phone down \revision{for} a short break, \revision{before continuing} with the second condition.
\revision{At the end, participants answered} a questionnaire regarding their experience with both conditions.

\subsubsection{Semi-Structured Interview}
\revision{The study sessions concluded with} a semi-structured interview to gather detailed feedback. 
During the interview, participants were free to explore our prototype (with long-press and keyboard enabled) without any prescribed tasks.
They could also switch to the chatbot UI.
If they wished, we provided a selection of short creative story texts (see \cref{sec:short_stories}) as inspiration, though they were free to write anything they wanted.
This allowed them to directly demonstrate what they were referring to during the interview. %

\section{Results}
Here we report on the study results.
Across both experiments, we recorded 1706 gestures from the 14 participants.

\subsection{Time}
\label{ssec:time}
In Experiment 1 we asked participants to repeatedly perform spread and pinch gestures to generate and remove text.
The mean time to complete this task (i.e. performing all gestures), was 16.38\,s when no visualisation was used, \secs{16.30} for Lines and \secs{14.41} for Bubbles.
These differences were significant as follows (\cref{tab:sig_tests}, row 1): 
\revision{With Bubbles,} participants completed the tasks significantly faster than with Lines (-\secs{1.76}) or without a visualisation (-\secs{2.62}).
\cref{fig:exp1_times} shows each subtask as a box blot.


\begin{figure*}[t]
    \centering
    \includegraphics[width=1\linewidth]{figures/task_completion_exp1.pdf}
    \caption{Task completion times for all subtasks in Experiment 1, segmented by visualisation condition (NoVis, Line, and Bubbles). Each subtask is represented along the x-axis, with the y-axis showing the completion time in seconds. Box plots display the completion times for each visualisation condition, with individual data points plotted as dots.}
    \Description{This figure shows a box plot illustrating task completion times across different tasks, subtasks, and visualisation conditions from Experiment 1. The x-axis represents the task and subtask combinations, while the y-axis shows completion time in seconds. Each subtask is labelled along the x-axis (e.g., extend - incomplete sentence, extend - 1 sentence, etc.), representing various tasks the participants completed during the study. The y-axis measures the time taken to complete each task, ranging from 0 to 70 seconds. Three colours represent the different visualisation conditions: Red represents NoVis, Green represents Lines visualisation, Blue represents Bubbles. Each box plot indicates the distribution of completion times for each visualisation across the different subtasks. Individual data points are plotted as dots to provide a clearer view of task variability.}
    \label{fig:exp1_times}
\end{figure*}

\revision{In Experiment 2}, with a mean completion time of \secs{56.35}, Bubbles was twice as fast as the chatbot UI, which took people \secs{134.86}.
This difference is significant (\cref{tab:sig_tests}, row 2).

The mean execution time of one extend gesture was \secs{5.23} for NoVis, \secs{4.59} for Lines, and \secs{3.88} for Bubbles.
Gestures to remove text took an average of \secs{2.73} for NoVis, \secs{2.59} for Lines, and \secs{2.52} for Bubbles.
\revision{See \cref{tab:time} for more details.}


\begin{table}[t]
\footnotesize
\centering
\begin{tabular}{llccc}
\toprule
\textbf{Task} & \textbf{Variable} & \textbf{Time} & \textbf{SD} & \textbf{Median} \\
\midrule
\multicolumn{5}{c}{\textbf{Experiment 1 - Visualisation}} \medskip\\
Overall & No Vis & 16.38 & 11.80 & 12.95 \\
 & Lines & 16.30 & 12.36 & 12.16 \\
 & Bubbles & 14.41 & 8.96 & 11.53 \medskip\\
Extend & No Vis & 18.48 & 11.53 & 13.89 \\
 & Lines & 17.16 & 12.22 & 13.24 \\
 & Bubbles & 15.58 & 8.91 & 12.87 \medskip\\
Shorten & No Vis & 12.27 & 9.87 & 9.52 \\
 & Lines & 12.79 & 10.36 & 9.78 \\
 & Bubbles & 12.02 & 8.54 & 9.24 \medskip\\
Combination & No Vis & 31.48 & 14.26 & 29.60 \\
 & Lines & 33.18 & 14.44 & 30.76 \\
 & Bubbles & 25.12 & 12.04 & 23.03 \medskip\\
\midrule
\multicolumn{5}{c}{\textbf{Experiment 2 - Interaction}} \medskip\\
Combined Task & Gesture & 56.35 & 30.87 & 51.03 \\
 & CUI & 134.86 & 58.05 & 125.94 \medskip\\
\midrule
\multicolumn{5}{c}{\textbf{Gesture Execution Time}} \medskip\\
Extend & No Vis & 5.23 & 3.13 & 4.71 \\
 & Lines & 4.59 & 2.45 & 4.33 \\
 & Bubbles & 3.88 & 2.20 & 3.61 \medskip\\
Shorten & No Vis & 2.73 & 1.45 & 2.22 \\
 & Lines & 2.59 & 1.51 & 2.00 \\
 & Bubbles & 2.52 & 0.87 & 2.41 \\
\bottomrule
\end{tabular}
\caption{Times measured in the study.}
\Description{This table gives an overview of the task times during the user study.}
\label{tab:time}
\end{table}








\subsection{\revision{Use of Space}}
The average finger distance when generating one sentence with the spread gesture was 229 pixel or 38.1\,mm. %
The average with the pinch gesture was 130 pixel or 21.6\,mm. %
\cref{fig:finger_kde} shows \revision{the finger locations.}
Participants used 1.88 (SD 1.77, median 1) subgestures (defined as taking the fingers off the screen to readjust) to complete one sentence, 1.73 (SD 1.35, median 1.0) to add a full sentence, and 3.89 (SD 2.70, median 3) to add three.
Similarly, they used 2.30 (SD 2.55, median 2) subgestures to remove one incomplete sentence, 1.38 (SD 0.64, median 1) to remove one sentence, and 2.92 (SD 1.68, median 2) to remove three.

\begin{figure}[t]
    \centering
    \includegraphics[height=7cm]{figures/finger_movement_overall_new}
    \caption{Density plot of all finger positions that were logged during the user study.}
    \Description{This figure is a Kernel Density Estimate (KDE) plot, visualising the density of all finger positions during gesture movements in the user study. The plot contains two density regions, each representing the positions of the first and second fingers during gesture input. The blue contour represents the distribution of the first finger’s touch positions, which are concentrated in the upper-left portion of the plot, around the coordinates (100, 150) in X and Y pixels, respectively. The orange contour indicates the density of the second finger’s positions, which are located in the lower-right part of the plot, around the coordinates (250, 500).The X-axis represents the horizontal position of the touches in pixels, and the Y-axis represents the vertical position. Darker areas within each density region indicate where finger positions are more concentrated during the gestures.}
    \label{fig:finger_kde}
\end{figure}


\begin{figure}[t]
    \centering
    \includegraphics[width=1\linewidth]{figures/distance_avg_all_versions_just_lines}
    \caption{The progression of normalized distance (relative to the target) over time for the visualisations in Experiment 1. The dashed red line represents the target distance, which is defined as the distance the fingers needed to be moved apart to create the intended amount of text. The pink, green, and blue lines represent the average distance at each point in time for \visnone{}, \visline{}, and \visbubble, respectively. The thickness of the lines indicates the number of samples included. Individual traces are displayed in light background lines.}
    \Description{This line graph illustrates how the distance between fingers, normalized relative to a target value, changes over time under different visual conditions tested in Experiment 1. The x-axis represents time in arbitrary units, while the y-axis represents normalized distance (relative to a target value of 1.0). A dashed red line indicates the target distance, which corresponds to the ideal finger movement required to achieve the desired text change. Three average distance lines are drawn: a pink line for the NoVis condition, a green line for the Lines condition, and a blue line for the Bubbles condition. The thickness of the lines indicate the number of samples included, which decreases gradually. Numerous faint individual traces are shown in the background, representing the variations in gesture distances for different trials under each condition. The lines for Bubbles reaches the dashed red line the fastest. NoVis reaches this target line the latest and overshoots.}
    \label{fig:normdist_time_all_avg}
\end{figure}

\subsection{Anatomy of Touch Gesture Text Generation}



\revision{To initiate the gestures, participants placed one finger at either the end or middle of a sentence, with the second finger positioned closer for the \spread{} gesture (280 px on average, SD 77.5, median 274) than for the \pinch{} gesture (343 px on average, SD 70.5, median 351). Also see \cref{fig:first_touches} in \cref{sec:appendix_figs}.}


An analysis inspired by human pointing dynamics \cite{mouse_mueller2017} \revision{revealed} distinct phases during the \spread{} gesture (\cref{fig:normdist_time_all_avg}): %
Participants rapidly spread their fingers to generate text, \revision{with 80\% of the target distance (where the task was completed) covered in the first half of the movement time.}
Afterward, the movement slowed down as the text approached the intended length.



Without visual feedback, participants often overshot the intended text length, going beyond the distance needed to complete the task.
This overshooting was less prominent with the Lines visualisation and entirely absent with Bubbles, as shown in \cref{fig:overshoot}. 

For Bubbles, the target was approached with a more consistent velocity.
Note that in \cref{fig:overshoot}, both the distance to the target and the time have been normalized\revision{, making the} curve for Bubbles appear to show a delayed target reach compared to the others. 
However, in absolute terms, Bubbles \revision{achieved} the shortest execution time, as detailed in \cref{ssec:time} and \cref{fig:normdist_time_all_avg}.

Finally, regarding hand postures, many participants intuitively used both thumbs for the gestures, though some adjusted their technique depending on the text's position, switching between thumbs and index fingers. 

\begin{figure*}[t]
     \centering
     \begin{subfigure}[t]{0.49\textwidth}
        \centering
        \includegraphics[width=0.95\linewidth]{figures/distance_normalized_time_all_versions_scaled.pdf}
        \caption{The complete average movement to generate the targeted amount of text with the \spread{} gesture as a function of distance (relative to the target text length) over normalised time.}
        \label{fig:overshoot1} 
     \end{subfigure}
     \hfill
     \begin{subfigure}[t]{0.49\textwidth}
        \centering
        \includegraphics[width=0.9555\linewidth]{figures/distance_normalized_time_all_versions_zoomed.pdf}
        \caption{The movement from (a) zoomed in on the end of the movement. At around 80\% of the total movement time a visible overshoot occurs for both \visnone and \visline.}
        \label{fig:overshoot2}
     \end{subfigure}
     \caption{The average movement to generate the target amount of text with \spread{} gives insights into how this gesture was executed. The distance is normalised relative to the targeted amount of text (red dashed line) and the normalized time. Points above y=1 indicate an overshoot, where more text was being generated than intended. We obtain this intention directly from users, as they had to confirm the text length modification after performing the gesture.}
     \Description{This figure contains two line graphs comparing the performance of different visualisation conditions ('NoVis', 'Lines', and 'Bubbles') while using the \spread{} gesture to generate text. Left graph (a): This graph plots the complete average movement required to generate the targeted text length as a function of normalized distance (relative to the target text length) over normalized time. The x-axis represents normalized time (0.0 to 1.0), and the y-axis represents normalized distance (relative to the target distance of 1.0). The three visualisation conditions ('NoVis', 'Lines', and 'Bubbles') are compared, with each shown as a distinct line. The target distance is represented by a red dashed line, while the other lines show the performance of each condition. The graph demonstrates how each condition approaches the target distance over time. Right graph (b): This graph zooms in on the final portion of the movement, focusing on the last 20\% of the total movement time. The same three conditions ('NoVis', 'Lines', and 'Bubbles') are plotted, showing the movement beyond 0.70 normalized time. Around 80\% of the total movement time, a visible overshoot occurs in both 'NoVis' and 'Lines' conditions, where the lines exceed the target distance.}
     \label{fig:overshoot}
\end{figure*}






\subsection{Perception of Visualisation Techniques (Experiment 1)}
\label{sec:perception_exp1}
After each condition in Experiment 1, participants rated the system's usability with the SUS questionnaire, their perceived workload using the NASA-TLX, and nine additional 5-point Likert items.

These results show that Bubbles achieved the highest usability rating of 85.54, indicating ``excellent'' usability, whereas Lines achieved a ``good'' score of 76.96, and NoVis performed significantly (\cref{tab:sig_tests}, row 3) worse than both with an ``OK'' score of 63.46 \cite{susBangor2009}.
Similarly, participants perceived the lowest workload using Bubbles with a NASA-TLX Raw score of 1.976, followed by 2.154 for Lines, and 2.794 for NoVis, which was significantly higher than both visualisations (\cref{tab:sig_tests}, row 5).

Further Likert items in Experiment 1 (see \cref{fig:own_likert_exp1}) overall indicate that participants perceived their interaction with the system as fast and without delays, regardless of the visualisation.
Comparatively, NoVis was clearly rated the lowest across all items.
While this is expected for questions regarding the support offered by visual representations, it is worth noting that both Lines and Bubbles also %
considerably increased participants' enjoyment of the system.
Comparing Bubbles and Lines, Bubbles was rated higher across all items, except for participants' perceived level of control over text length, where both visualisations were rated equally high. 

\begin{figure*}
    \centering
    \includegraphics[width=.65\linewidth]{figures/exp1_likert_comparison.pdf}
    \caption{Likert results on participants' perception of interaction with our prototype and the visual presentations, rated after each condition in Experiment 1. These rating overall indicate that participants felt that their interaction with the system was fast and without delays, regardless of the visualisation technique, but comparatively, NoVis was clearly rated the lowest across all items.}
    \Description{This figure shows divergent bar charts for the Likert results on participants' perception of interaction with our prototype and the visual presentations, rated after each condition in Experiment 1. These rating overall indicate that participants felt that their interaction with the system was fast and without delays, regardless of the visualisation technique, but comparatively, NoVis was clearly rated the lowest across all items.}
    \label{fig:own_likert_exp1}
\end{figure*}


\subsection{Perception of Interaction (Experiment 2)}
\label{sec:perception_exp2}
In Experiment 2, participants rated the system's usability significantly higher (\cref{tab:sig_tests}, row 4) for the touch gestures (81, ``good'') compared to using the CUI designed to mimic ChatGPT (52.5, ``OK'').
Similarly, their perceived workload was significantly (\cref{tab:sig_tests}, row 6) lower when using touch gestures (2.060) compared to using ChatGPT (3.153). %

This trend is further reflected in the nine Likert-items for Experiment 2 (see \cref{fig:own_likert_exp2}), which participants rated after each condition.
Our gesture-based interaction approach scored higher than the ChatGPT-like CUI across all items.

\begin{figure*}
    \centering
    \includegraphics[width=.65\linewidth]{figures/exp2_likert_comparison.pdf}
    \caption{Likert results on participants' perception of interaction with our touch-based prototype compared to our CUI implementation, rated after each condition in Experiment 2. Most participants rated almost all items overwhelmingly positive when using our approach.}
    \Description{This figure shows divergent bar charts for the likert results on participants' perception of interaction with our touch-based prototype compared to our CUI implementation, rated after each condition in Experiment 2. Most participants rated almost all items overwhelmingly positive when using our approach. Only for the item that asked participants whether they felt like the author of the text, while still rating our mode higher, the responses were more mixed.}
    \label{fig:own_likert_exp2}
\end{figure*}


\subsection{Subjective Feedback}
\label{sec:ss_interview_perception}
After both experiments, participants rated four Likert items on their overall experience with the gestures.

All but one ``neutral'' participant agreed (7) or strongly agreed (6), that ``the gesture controls felt intuitive.''
Similar, all but one ``neutral'' participant agreed (5) and strongly agreed (8), that ``the gesture controls felt natural''.
They ``would use this gesture-based text control of text generation feature in [their] daily tasks'' (4 ``neutral'', 8 ``agree'', 2 ``strongly agree'') and ``would recommend this gesture-based text control feature to others'' (1 ``neutral'', 3 ``agree'', 10 ``strongly agree'').
No one disagreed with any of these statements.


As the last part, we conducted semi-structured interviews. %
All but one participant %
picked Bubbles as their favourite, explicitly confirming the bigger picture observed in the preceding sections.




When asked about challenges, six participants reported no issues, one mentioned occlusion, and seven said they sometimes struggled to select the intended sentence. 
\revision{Difficulties include ``fat finger'' issues on the smaller screen (some were accustomed to larger devices), misunderstanding the interaction pattern (placing the cursor instead of tapping the sentence), and lack of individual adjustments for visual feedback delay and touch target offset. %
}
However, P8 added, ``you'd get into it over time''\revision{, which we also observed in other participants as the study progressed}.

\revision{Participants’ feedback on what they liked most about using gesture controls emphasized its intuitive, ``extremely smooth'' (P14) and natural feel.
They preferred the Bubbles visualisation for its clear separation of words, sentences, and deletions.}
In \cref{sec:perception_exp2}, we reported mixed feelings about authorship (\cref{fig:own_likert_exp2}). 
However, when reminded about the long-press feature in the interviews, 
\revision{11 out of 14 participants said this would enhance their sense of authorship.}

Multiple participants also noted that the visualisation effectively managed latency, with P7 remarking that they had ``hardly any feeling of latency''. %
The interaction as a whole -- but especially with Bubbles -- was described as satisfying, with P13 noting, ``It was simply fun to watch!''








\section{Discussion}


The value of hybrid paper-digital interfaces that augment physical documents with digital content has been well established~\cite{han_hybrid_2021}. To realize hybrid documents, we propose a novel watermarking method that uses IR-based printing to store digital content as an intrinsic part of the document. Unlike previous approaches that rely on a network connection and external storage for the digital assets, our approach with \systemName~ maintains the document format as the single container of both physical and digital contents. 
% \hl{This approach provides additional benefits that interactions are network-free and more privacy-preserving.}
In this section, we discuss the limitations of using \systemName~ from the perspectives of authoring, printing, detection, and consumption. We also discuss future work to improve the approach.



\subsection{Towards Multimodal Content}
\systemName's interface currently supports only a rudimentary form of AR that only permits 2D visual content and audio and does not leverage multimodal content to its full potential. We hope that we will enable authors to embed more intricate objects and interactions in the future. Adding support for multimedia assets will pave the way to expanding on previously examined use cases ~\cite{alessandrini_audio-augmented_2014, rajaram_paper_2022}. The most impactful addition would be that of 3D assets since that would allow users to utilize the document as a spatial anchor. 



\subsection{Effects on Printed IR Ink}




\new{Our experiments investigated human and machine detectability under controlled indoor lighting and viewing conditions.}
\newCameraReady{However, the perception and reliability of printed IR ink can vary significantly depending on external factors such as illumination levels, viewing angles, and the type of paper used.
One key limitation is that low-light conditions may reduce the effectiveness of human detectability due to insufficient IR reflection, while excessive illumination, such as direct sunlight, could lead to overexposure, affecting both machine readability and ink longevity. Future work could explore adaptive imaging techniques or optimized illumination setups to mitigate these issues. 
Regarding \textit{paper type}, we anticipate that glossy or coated paper may introduce reflection artifacts, which could interfere with machine-based recognition systems by creating specular highlights that obscure ink contrast.
Conversely, highly porous or rough-textured paper might cause ink diffusion, potentially reducing print sharpness and affecting detection accuracy.
A promising direction is to systematically evaluate these effects using an experimental framework similar to Xu et al.'s methodology, where QR code robustness was tested under varying lighting conditions, scanning angles, and environmental factors \cite{xu_art-up_2021}. Applying a similar protocol could help quantify how different conditions impact IR ink detection and visibility.}



As with any other dye-based ink, UV-dependent fading is an inherent limitation~\cite{maxmax_-_llewellyn_data_processing_ir_2022} that is important to consider for document reliability and permanence. Constant exposure to heavy UV sources such as the sun can cause the inks to fade over time.
\citet{willis_hideout_2013} showed how different IR ink types can be used in conjunction with UV-resistant coatings to achieve 98\% contrast preservation under office lighting conditions. We envision that for commercial use cases where long-term preservation is desired, a similar coating can be applied, which is sufficient for indoor applications.


\subsection{Invisible IR Ink}
We used a psychophysical experiment to determine conservative estimates for the IR ink densities that remain invisible to users for a wide range of different background colors. As with most experiments, we were limited by the number of background colors that we could include in the experiment. \new{Therefore, we do not know how our results generalize to other background colors, combinations of colors, and different graphic patterns. Nevertheless, we developed a system that facilitates embedding with invisible IR ink for any RGB color. Note that our method is very conservative and favors invisibility over machine detection. The actual DTs are likely much higher, which should further improve machine detectability. Nevertheless, we contribute a methodology that designers can apply to determine the "sweet spot" between visibility, data capacity, and machine detectability.}




We also want to highlight that the gradient IR ink and background color bars used in our psychophysical experiment differ from QR codes which may have had an effect on our estimated invisible IR ink DTs. We decided against using QR codes in the experiment because (1) it would have only been possible to test QR codes at predefined densities (e.g., at 20\%, 40\%, ...), limiting the precision of the study, and (2) the limited number of samples, which would not reflect the variability of QR codes of different sizes. As a result, our experiment also aims to provide general estimates for shapes other than QR codes and may, therefore, be used as a starting point for any type of invisible IR-printed marker.   



\subsection{Discoverability and Practicality}
\label{NIR_AR_FormFactor}

\new{
To demonstrate \systemName, we used NIR-based fabrication and detection tools. While NIR cameras are getting popular in many handheld devices (e.g., \textit{iPhone} and \textit{iPad} use it for facial recognition and LIDAR 3D scanning), not all platforms currently give 3rd-party developers access to the raw NIR stream (e.g., only the processed depth map can be accessed on \textit{iOS}). We argue the interest in NIR applications will increase as more use cases are demonstrated by future projects.}
% And we want to add this to AR headsets in the future, which already come with NIR cameras (conventionally used for depth sensing).

In our current implementation, \systemName~ documents can optionally be marked with a small icon or visual label on the document in our embedding tool
to allow users to discover embedded AR content.
\new{We envision that next-generation AR hardware can more fully leverage \systemName's utility. Compared to using a handheld device, always-on AR smart glasses such as \textit{Meta} \textit{Orion} could be constantly scanning the environment for hidden AR content~\cite{di_gioia_investigating_2022, campos_zamora_moirewidgets_2024}.
Currently, most AR glasses already leverage NIR cameras, but mainly for localizing the user and mapping their environment for 3D tracking purposes. With head-worn AR glasses with integrated NIR cameras used for \systemName~ sensing, the user would not have to manually capture objects using the phone during the interaction. We recommend future research to focus on implementing this on AR glasses.}



\section{Conclusion}


We introduced \systemName, a watermarking technique that embeds computer-readable information while remaining invisible to the human eye. Unlike previous methods, our approach utilizes the entire document, including white spaces, and works regardless of background color. Through a psychophysical experiment, we determined the maximum ink that can be embedded without being detected. We developed tools to support users in applying IR ink technology, including software for efficient information embedding and a universal camera module for capturing \systemName~watermarks. Our open-source ML pipeline processes these images for robust use with standard QR code readers. We demonstrated various use cases, highlighting the potential of invisible IR content for hybrid paper-digital interfaces and advancing watermarking techniques.






\begin{acks}
Funded by the Deutsche Forschungsgemeinschaft (DFG, German Research Foundation) -- 525037874.
This project is funded by the Bavarian State Ministry of Science and the Arts and coordinated by the Bavarian Research Institute for Digital Transformation (bidt).
\end{acks}

\bibliographystyle{ACM-Reference-Format}
\bibliography{bibliography}


\appendix

% \autoref{fig:text_bias_example} and \autoref{fig:query_shortening_examples} give examples of incorrect response from \citet{chang_webqa_2021} along with rationale for the mistakes.

% \begin{figure*}[!h]
%     \centering
%     \includegraphics[width=18cm]{figures/text_bias_example.png}
%     \caption{Examples of an incorrect VQA response from the baseline \citep{chang_webqa_2021} where the image source is not used, and instead the most likely answer based on word co-occurance is given.}
%     \label{fig:text_bias_example}
% \end{figure*}


% \begin{figure*}[!h]
%     \centering
%     \includegraphics{figures/query_shortening_examples.png}
%     \caption{Examples of incorrect VQA responses from the baseline \citep{chang_webqa_2021} where the query could be shortened post-retrieval to remove confounding terms that are only useful for retrieval, such as specific nouns.}
%     \label{fig:query_shortening_examples}
% \end{figure*}

\subsection{Model Selection Results}
\label{sec:model_selection}
We explore baseline methods for the QA task on the WebQA validation set. \autoref{tab:baselines} gives results for the baseline models. The MH-VoLTA model outperforms all baseline and zero-shot models on the validation set image questions. However, the extension of the VoLTA model for variable input multi-hop tasks risks a regression in performance on traditional VQA tasks which have fixed-input where the number of input images is constant. To determine MH-VoLTA generalizes from fixed to variable input tasks, we compare performance between two variants of the original VoLTA model, finetuned on one and two image subsets of WebQA, with MH-VoLTA. We find that MH-VoLTA is capable of reasoning over both one and two-image image questions, and it's performance is on-par with VoLTA variants trained on one and two image sources separately\autoref{tab:baselines}. See \autoref{sec:one_vs_two_image_volta} for more details on the one and two image VoLTA variants, as well as a breakdown of model performance by question category (\autoref{fig:multihop_volta_res}). See \autoref{sec:baselines} for a description of the baseline models used.

% GPT-4o and GPT-3.5 models outperform both VLP and GIT baselines. This is particularly impressive for GPT-3.5, which as a unimodal model can only take image caption as input. Despite having mulitmodal inputs, BLIP-2 underperforms GPT-3.5 which we attribute to the difference in decoder quality. We adopt the best performing general purpose LLM (GPT-4o) and specialized VQA model (MH-VoLTA) for all futher experiments.


% Given our focus on question categories that can be addressed using images, \autoref{tab:baselines} splits the baseline results by performance into questions that require one and two image sources respectively. 

\paragraph{VQAv2 and NLVR2}
\label{sec:vqav2}
In addition to WebQA, we evaluate models on two fixed-input VQA datasets---VQAv2 \citep{goyal2017making}, a multi-class, single-image VQA dataset, and \citep{nlvr2}, a binary classification, two-image VQA dataset. These datasets are well-suited to VoLTA classifier architecture. In particular, question categories in VQAv2, along with the associated answer-domains, match well with WebQA, with a substantial portion of both datasets focusing on color, shape, number, and yes/no questions.

\subsection{Multihop VoLTA on one vs two image sources}
\label{sec:one_vs_two_image_volta}
The results for finetuning VoLTA and MH-VoLTA on the WebQA dataset experiments are provided in \autoref{tab:multihop_volta_res}. We explored the application of Multihop-VoLTA in addressing queries based on single images, questions involving two images, and a combination of both single and two-image queries (referred to as multiple images, \autoref{fig:model_perf_qcate}). 

We find that the variable Multihop-VoLTA model (\autoref{fig:multihop_volta_res}) is en-par with the fixed-input one and two-image VoLTA model variants (\autoref{fig:model_perf_qcate}). This underscores the stability of our finetuning approach for MH-VoLTA across both training paradigms. The MH-VoLTA models have on the order of 100M parameters, of which 10M are trainable after applying LoRA. All models are trained for 80 epochs on a Nvidia A6000. 


% We find that the distribution of question types is very different between questions with one and two image sources. Questions with two image sources have more questions in categories with higher random accuracy like Choose and YesNo, whereas questions with one image source have more questions in the harder categories like shape, color, and number, where random accuracy is much lower given the larger answer domain.


% \begin{table}[!h]
%     \centering
%     \begin{tabular}{cccc}
%     \toprule
%          &Accuracy& Fluency  &GPT-Fluency\\
%          \midrule
%          Single Image&0.764&   0.003&\\
%          Two Images&0.851&   0.002&\\
%          Multiple Images&0.799&   0.002&\\
%          \bottomrule
%     \end{tabular}
%     \caption{Multihop VoLTA Results}
%     \label{tab:multihop_volta_res}
% \end{table}


\begin{table}
    \centering
    \caption{Model selection results on WebQA validation set (further broken into 1 and 2 image input categories), and the VQAv2 and NLVR2 (NLV) test sets. MH-V denotes MH-VoLTA. See \autoref{sec:baselines} for model descriptions.}% including Cross-Category accuracy by source image count.}
    \label{tab:baselines}
    \begin{tabular}{clllll}
    \toprule
     & \multicolumn{3}{c}{\makecell{WebQA Acc}} & VQA & NLV \\
        \midrule
         Model & All & 1 img & 2 img & Acc & Acc \\
         \midrule
         \hyperref[sec:mh-volta]{MH-VoL} & \textbf{0.71} & 0.72 & 0.70 & 73.9 & 76.5 \\
         VoLTA\textsubscript{1}  & -- & \textbf{0.77} & -- & \textbf{74.6} & -- \\
         VoLTA\textsubscript{2}  & -- & -- & \textbf{0.84} & -- & \textbf{76.7} \\
         \midrule
         GPT-4o & 0.56 & \textbf{0.58} & 0.69 &  -- & --\\ % 0.77 &
         \hyperref[sec:gpt3.5]{GPT-3.5} & 0.53 &  0.41 & 0.45 & -- & --\\ % 0.47 &
         \hyperref[sec:VLP]{VLP} & 0.50 & 0.40 & 0.42 & -- & -- \\ % fl 0.48 & 
         \hyperref[sec:GIT]{GIT}  & 0.42 & 0.43 & 0.35 & -- & --\\ % 0.19 & 
         \hyperref[sec:blip]{BLIP-2} & 0.40 &  0.37 & 0.44 & -- & --\\ % 0.20 &
         % GIT & Original & No & 0.222 & 0.025 &  \\
         % BLIP-2 & Original & No & 0.400 & 0.126 & \\
         % BLIP & Simple & No & & & \\
         % BLIP-2 & Simple & Yes & 0.357 & 0.133 & \\
         % BLIP-2 + GPT-3.5 & Original & No & & & \\
         % BLIP-GPT-3.5 & Original & Yes & & & \\
         % BLIP-GPT-3.5 & Simple & No & & & \\
         % BLIP-GPT-3.5 & Simple & Yes & & & \\
         \bottomrule
    \end{tabular}
\end{table}

\begin{table}
    \centering
    \caption{MH-VoLTA results and dataset breakdown}
    \label{tab:multihop_volta_res}
    \begin{tabular}{ccc}
    \toprule
         & No. of Samples&Accuracy\\
         \midrule
         Single Image& 760&0.764\\
         Two Images& 576&0.851\\
         Multiple Images& 1336&0.799\\
         \bottomrule
    \end{tabular}
\end{table}

% The results presented in \autoref{fig:model_perf_qcate} demonstrate the robustness of our model when trained independently for single-image and two-image questions. \autoref{fig:gpt_vs_volta_by_category} illustrates a comparable graph for joint finetuning of VoLTA on both single-image and two-image sources. Interestingly, there is virtually no distinction in accuracy, even up to the second decimal place, between individual and joint training. 

% Nevertheless, a notable decline is observed in the 'shape' question category when compared to both single-image and two-image questions. This decrease can be attributed to the model predominantly predicting the 'h' shape for the majority of two-image questions. While this behavior is consistent with single-image questions, it is particularly pronounced in the 'shape' category, where there are 65 questions with a single positive image source and only 8 questions with two positive image sources. This disparity in the distribution of questions with varying positive image sources contributes to the observed performance difference in the 'shape' category.

% The 'YesNo' question category exhibits the highest performance compared to other question categories, which aligns with expectations given its binary classification nature. However, our model shows a slight susceptibility to frequency bias, indicating a tendency to be influenced by the prevalence of certain classes in the training data. Addressing this bias is important for achieving a more balanced and accurate predictive performance across all question categories.


% \begin{figure*}
%     \centering
%     \subfigure[Performance of the MH-VoLTA classifier by question category and image count.]{%
%         \includegraphics[width=0.3\textwidth, trim={0 0 0 3cm},clip]{figures/results/model_perf_acc_comparison_per_qtype_joint.png}
%         \label{fig:multihop_volta_res} 
%         % model_perf_qcate_joint
%     }\hfill
%     \subfigure[Performance of fixed input single and two-image VoLTA classifiers.]{%
%         \includegraphics[width=0.3\textwidth]{figures/results/model_perf_acc_comparison_per_qtype.png}
%         \label{fig:model_perf_qcate}
%     }\hfill
%     \subfigure[Convergence of the VoLTA loss function on the WebQA dev set across several MH-VoLTA training runs.]{%
%         \includegraphics[width=0.33\textwidth]{figures/results/mhvolta_loss.png}
%         \label{fig:loss_convergence}
%     }
%     \caption{Comparison of the variable MH-VoLTA model (left) vs fixed input VoLTA models (center) across different question categories, ordered by the number of image sources per question. Models converge after ~80 epochs (right).}
% \end{figure*}

\begin{figure*}
    \centering
    \begin{subfigure}[b]{0.3\textwidth}
        \includegraphics[width=\linewidth, trim={0 0 0 3cm},clip]{figures/results/model_perf_acc_comparison_per_qtype_joint.png}
        \caption{Performance of the MH-VoLTA classifier by question category and image count.}
        \label{fig:multihop_volta_res}
    \end{subfigure}\hfill
    \begin{subfigure}[b]{0.3\textwidth}
        \includegraphics[width=\linewidth]{figures/results/model_perf_acc_comparison_per_qtype.png}
        \caption{Performance of fixed input single and two-image VoLTA classifiers.}
        \label{fig:model_perf_qcate}
    \end{subfigure}\hfill
    \begin{subfigure}[b]{0.33\textwidth}
        \includegraphics[width=\linewidth]{figures/results/mhvolta_loss.png}
        \caption{Convergence of the VoLTA loss function on the WebQA dev set across several MH-VoLTA training runs.}
        \label{fig:loss_convergence}
    \end{subfigure}
    \caption{Comparison of the variable MH-VoLTA model (left) vs fixed input VoLTA models (center) across different question categories, ordered by the number of image sources per question. Models converge after ~80 epochs (right).}
\end{figure*}


\subsection{Performance by Question Category}
\label{sec:category_perf}
We report the mean accuracy per question category for Multihop-VoLTA in \autoref{fig:multihop_volta_res} using source retrieval oracles. We find that performance is dependent upon the level of training data available, with the shape category having the least number of samples in the dataset. Question counts per category are as follows; Yes/No (n = 7,320), color (n = 1,830), number (n = 2,118), shape (n = 565). The similarity in results across different question categories reinforces the reliability and stability of our model's performance. For a breakdown of labels per question category, see \autoref{sec:categories}.


% \paragraph{CLIP} Contrastive Language-Image Pre-training \cite{radford2021learning} simultaneously trains both an image encoder and a text encoder. Their task is to predict the correct associations within a batch of (image, text) training examples. However, for the challenging task of full-scale retrieval without any prior exposure in WebQA, it's important to note that running VLP-based retrieval across the entire source collection would be prohibitively time-consuming, estimated at three years. To overcome this limitation, the WebQA approach opted to use CLIP for dense retrieval and BM25 \cite{robertson2009probabilistic} for a more coarse-grained retrieval process.

% \paragraph{VLP} The unified Vision-Language Pre-training (VLP) model \cite{zhou_unified_2020}, is a versatile multimodal generative transformer. This model can be customized through finetuning to excel in either vision-language generation tasks, like generating captions for images, or comprehension tasks, such as answering questions based on visual content. What sets it apart is its utilization of a single multi-layer transformer network for both encoding and decoding, which distinguishes it from numerous other approaches that employ distinct models for these two functions.

% We follow \cite{chang_webqa_2021} in using VLP a baseline model. However, we also adapt two other Vision Language models as baselines, namely  GIT \citep{wang2022git} and BLIP2 \citep{li2023blip2}, which are introduced more formally in later sections.

\subsection{GPT-4o Retrieval Prompt}
\begin{framed}
\label{frame:labeling_prompt}
\textbf{system}: Answer the question in one word. Then list the Fact\_ID or Image\_ID of all facts used to derive the answer in square brackets.

\textbf{human}: Question: <query>

\textbf{human}: Text Facts:
[fact\_id\_1: fact\_1, ..., id\_n: fact\_n]

\textbf{human}: Image\_ID: img\_id\_1, \\
Caption: img\_caption\_1

\textbf{human}: [Input\_type=image] \\
image\_url=url\_1

...

\textbf{human}: Image\_ID: img\_id\_m, \\
Caption: img\_caption\_m

\textbf{human}: [Input\_type=image] \\
image\_url=url\_m

\end{framed}

\subsection{Question Complexity Analysis Metrics}
\label{appendix:complexity_analysis_metrics}

The Flesch-Kincaid Grade Level is a readability metric that evaluates the difficulty of a text based on the length of its words and sentences \citep{flesch2007flesch}, and is defined as;
\begin{equation}
\begin{split}
  \text{FKGL} = 0.39 \left( \frac{\text{Total Words}}{\text{Total Sentences}} \right) \\ + 11.8 \left( \frac{\text{Total Syllables}}{\text{Total Words}} \right) - 15.59  
\end{split}
\end{equation}

The Gunning Fog Index is a readability test used in linguistics to assess the complexity of English writing \citep{gunning1952technique}, and is defined as;
\begin{equation}
\begin{split}
    \text{GFI} = \frac{0.4 \times \text{Total Words}}{\text{Total Sentences}} \\
    + \frac{40 \times \text{Total Complex Words}}{\text{Total Words}}
\end{split}
\end{equation}


% \footnote{The Flesch-Kincaid Grade Level is a readability metric that evaluates the difficulty of a text based on the length of its words and sentences.}

% \footnote{The Gunning Fog Index is a readability test used in linguistics to assess the complexity of English writing.}



% $$


\subsection{Question Category Domain Lists}
\label{sec:categories}
\begin{figure*}
\begin{lstlisting}
yesno_set = {'yes', 'no'}
color_set = {
    'orangebrown', 'spot', 'yellow', 'blue', 'rainbow', 'ivory', 
    'brown', 'gray', 'teal', 'bluewhite', 'orangepurple', 'black', 
    'white', 'gold', 'redorange', 'pink', 'blonde', 'tan', 'turquoise', 
    'grey', 'beige', 'golden', 'orange', 'bronze', 'maroon', 'purple', 
    'bluere', 'red', 'rust', 'violet', 'transparent', 'yes', 'silver', 
    'chrome', 'green', 'aqua'
}
shape_set = {
    'globular', 'octogon', 'ring', 'hoop', 'octagon', 'concave', 'flat', 
    'wavy', 'shamrock', 'cross', 'cylinder', 'cylindrical', 'pentagon', 
    'point', 'pyramidal', 'crescent', 'rectangular', 'hook', 'tube', 
    'cone', 'bell', 'spiral', 'ball', 'convex', 'square', 'arch', 'h', 
    'cuboid', 'step', 'rectangle', 'dot', 'oval', 'circle', 'star', 
    'crosse', 'crest', 'octagonal', 'cube', 'triangle', 'semicircle', 
    'domeshape', 'obelisk', 'corkscrew', 'curve', 'circular', 'xs', 
    'slope', 'pyramid', 'round', 'bow', 'straight', 'triangular', 
    'heart', 'fork', 'teardrop', 'fold', 'curl', 'spherical', 
    'diamond', 'keyhole', 'conical', 'dome', 'sphere', 'bellshaped', 
    'rounded', 'hexagon', 'flower', 'globe', 'torus'
}   
\end{lstlisting}
\end{figure*}

\subsection{Baseline Models}
\label{sec:baselines}

\paragraph{VLP}
\label{sec:VLP}

% A pretrained object detector is used to extract image regions $r_i$, region features $R_i$, region object labels $C_i$, and region geometric features $G_i$. Regions are embedded according to the following network, where [.|.] is the concatenation of features;
% \begin{equation}
%     r_i = W_r R_i + W_p[LayerNorm(W_cC_i)|LayerNorm(W_gG_i]
% \end{equation}

% Special tokens [CLS], [SEP], and [STOP] are special input tokens that represent the start of visual input, the boundary between visual and text input, and the end of the sequence. The weight matrices are trained according to two masked language modeling objectives, where 15\% of input text tokens are replaced (80\% with a [MASK] token, 10\% with a random token and 10\% with the original token). 

% The two objectives are the BERT bidirectional objective and the seq2seq objective which satisfies the auto-regressive property that is desired by the generative setting of VQA. Notably, VLP objectives do not include Next Sentence Prediction as in BERT, which was found to lower performance \citep{zhou_unified_2020}. The seq2seq objective is implemented with a simple self-attention mask M. 
The VLP transformer model consists of a unified encoder and decoder \citep{zhou_unified_2020}. The VLP architecture is made up of 12 layers of transformer blocks trained according to the BERT bidirectional and the seq2seq objectives where the self-attention module in the transformer block are defined as;

\begin{equation}
    A^l = softmax(\frac{Q^TK}{\sqrt{d}} + M)V^T
\end{equation}

where $V = W^l_VH^l-1, Q = W^l_QH^l-1, K = W^l_KH^l-1$. As in \cite{vaswani2017attention}, a feedforward layer (with residual) maps $A^l$ to $H^l$. The model is trained on image caption pairs, and then finetuned for the VQA task. Finetuning follows by taking the hidden states from the final layer and feeding them to a multi-layer perceptron. The model used has been finetuned twice, once on the VQA dataset (as described by \cite{yu_unified_2023}), and again on the WebQA dataset. 

% In the later case, only seq2seq masking is applied, but more importantly, the model is trained on questions that require two image sources as described in the dataset section (section 5.1). Of all the baselines, this is the only model which has been explicitly trained for multihop VQA. During inference, five variants are generated using beam search and the most confident output is used.

\paragraph{GIT}
\label{sec:GIT}
To contrast with VLP, a pretrained multihop VQA model, we use a pre-trained Generative Image-to-Text Transformer (GIT) \citep{wang2022git}.
GIT employs a simplified VQA architecture with one encoder for images and one decoder for text. As such, the model is explicitly incapable for multihop VQA between text and images, so it serves as a baseline for pre-trained models that do not utilize image descriptions, and so we concatenate image sources if there are more than one. 

% The GIT model is finetuned on VQA-2 \citep{goyal2017making}, a typical VQA paradigm where questions do not require knowledge external to the image. By inspection, we also find that questions are simpler (and shorter) in the VQA tasks for which it is trained. We therefore expect that simplified questions will perform better on this model and so the GIT model may be illustrative of where WebQA departs from typical VQA tasks.

GIT is pre-trained using the language modeling task (as opposed to MLM which is used by VLP) where the model learns to predict captions in an auto-regressive manner. For VQA finetuning, the text input is swapped to the query, so that answers are predicted. 

\paragraph{BLIP-2}
\label{sec:blip}
Similar to VLP, the Bootstrapping Language-Image Pre-training model (BLIP) is a unified vision language pre-trained model \citep{li2022blip}. It relies on a visual transformer which is less computationally demanding and is pre-trained on over 100 Million image-caption pairs using a contrastive loss (ITC) for image-text contrastive alignment and image-text matching (ITM). 
% The authors show that it achieves state of the art performance on various VQA tasks and follows the architecture described in Figure \ref{fig:blip_vqa}. The authors of \cite{li2023blip2} introduced BLIP-2 which is based on a similar architecture as BLIP but it introduces a Q-transformer as the encoder and an LLM decoder. 
In addition to the ITC and ITM losses, the authors introduce an additional Image-grounded text generation (ITG) loss that trains the Q-former encoder to generate texts, given input images as the condition. 

% The final objective to minimize is defined as;
% \begin{equation}
%     \textit{Loss} = \text{ITC + ITG + ITM}
% \end{equation}
% \begin{figure}[h]
%   \centering
%   \includegraphics[width=\columnwidth]{figures/architectures/blip-vqa.png}
%   \caption{BLIP VQA finetuning architecture \cite{li2022blip}}
%   \label{fig:blip_vqa}
% \end{figure}

\paragraph{GPT-3.5 Turbo}
\label{sec:gpt3.5}
Throughout the dataset, a consistent challenge emerges: the model must focus on details, understand them, and accurately respond to questions, even after the provision of positive source images. This challenge has led to the exploration of an image-to-text approach, where the task involves generating descriptive captions for the images. This transforms the problem into a unimodal text retrieval and generation task. Using this method, the SOLAR model has had success on the WebQA task \cite{alibaba_text}. Accordingly, we include a zero-shot oracle baseline, passing queries and image captions to gpt-turbo-3.5 \citep{brown2020language}.


% \subsection{Vin+VLP Baseline Results}
% \label{appendix:baseline_results}
% \begin{figure}[h]
%   \centering
%   \includegraphics[width=\textwidth]{figures/results/histogram_accs_qcate.png}
%   \caption{A histogram depicting the distribution of datapoint-wise accuracy scores for the Vin+VLP baseline on the validation dataset for each type of question category, as assessed using the official evaluation script}
%   \label{fig:hist_accs_qcate}
% \end{figure}

% \begin{figure}[!h]
%   \centering
%   \begin{minipage}{0.48\textwidth}
%     \centering
%     \includegraphics[width=\textwidth]{figures/results/histogram_accuracy_1img.png}
%     \caption{A histogram similar to Fig \ref{fig:hist_accs_qcate}, but specifically for questions with a single positive image source}
%     \label{fig:hist_1img}
%   \end{minipage}\hfill
%   \begin{minipage}{0.48\textwidth}
%     \centering
%     \includegraphics[width=\textwidth]{figures/results/histogram_accuracy_2img.png}
%     \caption{A histogram similar to Fig \ref{fig:hist_accs_qcate}, but specifically for questions with a two positive image sources}
%     \label{fig:hist_2img}
%   \end{minipage}
% \end{figure}

% \begin{figure*}[!h]
%     \centering
%     \includegraphics[width=0.49\linewidth]{figures/examples/guitar.png}
%     \includegraphics[width=0.49\linewidth]{figures/examples/guitar.png}
    
%     \caption{
%     \textit{Guid: d5be98180dba11ecb1e81171463288e9} \\ 
%     \textit{Question Category: "choose"} \\
%     \textit{Question: "Which instrument usually requires a bow to play it; A violin or Fernandes Monterey Deluxe?"} \\ 
%     \textit{Ground Truth: "A violin requires a bow to play it, but the Fernandes Monterey Deluxe does not."} \\ 
%     \textit{Prediction: "A violin"} \\ 
%     \\
%     Even though the model answered the question correctly, neither of the provided images in this modified sample contains a violin. The model simply answers this question based on its pretraining knowledge.}
%     \label{fig:blip_example_1}
% \end{figure*}
% \vspace{100mm}
% \begin{figure}[!h]
%   \centering
%   \includegraphics[width=0.4\columnwidth]{figures/results/histogram_accuracy.png}
%   \vspace{-5mm}
%   \caption{A histogram depicting the distribution of datapoint-wise accuracy scores for the Vin+VLP baseline on the validation dataset, as assessed using the official evaluation script}
%   \label{fig:vin_vlp_hist_acc}
% \end{figure}

% The data presented in Figure \ref{fig:vin_vlp_hist_acc} reveals a significant trend: the accuracy scores predominantly cluster around either 0 or 1. This pattern aligns with the nature of the evaluation methodology, which relies on category-aware lexical overlap for assessment. It's important to note that the evaluation script selects the top candidate answer from the model predictions for these assessments.

% \vspace{8mm}


% \section{Error Analysis - Examples}
% \label{sec:appendix_error_analysis_examples}

% \begin{figure*}[!h]
%     \centering
%     \vspace{-5mm}
%     \includegraphics[width=0.35\linewidth]{figures/examples/empire_tower.jpg}
%     \includegraphics[width=0.35\linewidth]{figures/examples/empire_tower_2.jpg}
%     \vspace{-2mm}
%     \caption{
%     \textit{Guid: d5bc623c0dba11ecb1e81171463288e9} \hspace{20mm} \textit{Question Category: "YesNo"} \\
%     \textit{Question: "Was there a building constructed after 2007 that can now be seen in the distance behind the Empire State Building and mimics its shape?"} \\ 
%     \textit{Ground Truth: "There was a building constructed after 2007 that can now be seen in the distance behind the Empire State Building that mimics its shape"} \\ 
%     \textit{Prediction: "Yes , there was a building constructed after 2007 ..."} \\ 
%     % \\
%     Even though the model answered the question correctly, neither the images nor their captions provide any information of the year being 2007 or later. As the answer does contain a Yes/No keyword, the prediction can never be wrong.
%     }
%     \label{fig:example_3}
% \end{figure*}

% \begin{figure}[!h]
%   \centering
%   \begin{minipage}{0.48\textwidth}
%     \centering
%     \includegraphics[width=\linewidth]{figures/examples/Barcelona-St-Josep-La-Boqueria.jpg}
%     \vspace{-7mm}
%     \caption{
%       \textit{\\Guid: d5bc2b6e0dba11ecb1e81171463288e9} \\ 
%       \textit{Question Category: "choose"} \\
%       \textit{Question: "Is the word on the lollipops at Market St Joseph La Boqueria in Barcelona written in print or cursive?"} \\ 
%       \textit{Ground Truth: "The word on the lollipops is written in cursive."} \\ 
%       \textit{Prediction: "The word on the lollipops at Market St Joseph La Boqueria in Barcelona is written in print ."} \\ 
%       \\
%       Answering this question requires a human to closely examine the image due to its low resolution and the small portion relevant to the answer (The white annotation has been added by us for improved visualization and is not part of the original image)}
%     \label{fig:example_1}
%   \end{minipage}
%   \hfill
%   \begin{minipage}{0.48\textwidth}
%     \centering
%     \includegraphics[width=\linewidth]{figures/examples/Washington_Square_Arch-Isabella.jpg}
%     \vspace{-7mm}
%     \caption{
%       \textit{\\Guid: d5bcd3700dba11ecb1e81171463288e9} \\ 
%       \textit{Question Category: "shape"} \\
%       \textit{Question: "What shape is the fountain near the arch in Washington Square Park?"} \\ 
%       \textit{Ground Truth: "There is a circle shaped fountain near the arch in Washington Square Park."} \\ 
%       \textit{Prediction: "The fountain near the arch in Washington Square Park is a circle ."} \\ 
%       \\
%       While the model provides a correct answer to this question, the accuracy score is 0.5, which should ideally be 1. This issue arises from the model's inability to distinguish between a square shape and a place like 'Washington Square Park.' The evaluation metric interprets the model's response as 'square' and 'circle' instead of just 'circle.'}
%     \label{fig:example_2}
%   \end{minipage}
% \end{figure}


\end{document}
