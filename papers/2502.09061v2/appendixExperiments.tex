


\begin{table*}[t]
    \centering
    \caption{Comparison of \Tool{} and baselines with various models on GSM-Symbolic based on accuracy, number of tokens, and average time.}
    \begin{tabular}{llcccr}
        \toprule
        \textbf{Model} & \textbf{k} & \textbf{Method} & \textbf{Acc. (\%)} & \textbf{Parse (\%)} &  \textbf{Tokens}\\
        
\midrule

   &  & \stdUnconstrained{} & 20 & 98 & 18.23 \\
 & & \stdConstrained{} & 21 & 95 & 34.28 \\
 Qwen2.5-1.5B-Instruct & 2 & \cotUnconstrained{} & 22 & 90 & 130.74 \\
 & & \textbf{\Tool{}} & \textbf{28} & 96 & 140.52 \\

\midrule

   &  & \stdUnconstrained{} & 18 & 95 & 18.23 \\
 & & \stdConstrained{} & 18 & 96 & 34.28 \\
 Qwen2.5-1.5B-Instruct & 4 & \cotUnconstrained{} & 24 & 94 & 130.74 \\
 & & \textbf{\Tool{}} & \textbf{30} & 98 & 140.52 \\

\midrule
   &  & \stdUnconstrained{} & 21 & 97 & 23.34  \\
 & & \stdConstrained{} & 22 & 97 & 25.29  \\
  Qwen2.5-1.5B-Instruct  & 8 & \cotUnconstrained{} & 26 & 90 & 128.97  \\
 & & \textbf{\Tool{}} & \textbf{31} & 100 & 131.3  \\

\midrule

    & & \stdUnconstrained{} & 37 & 96 & 17.22 \\
 & & \stdConstrained{} & 36 & 99 & 18.61  \\
 Qwen2.5-Coder-7B-Instruct & 2 & \cotUnconstrained{} & 32 & 84 & 148.87 \\
 & & \textbf{\Tool{}} & \textbf{37} & 96 & 155.65\\

\midrule

    & & \stdUnconstrained{} & 36 & 96 & 16.89 \\
 & & \stdConstrained{} & 36 & 100 & 18.81  \\
 Qwen2.5-Coder-7B-Instruct & 4 & \cotUnconstrained{} & 35 & 89 & 151.29 \\
 & & \textbf{\Tool{}} & \textbf{37} & 97 & 163.21\\

\midrule

    & & \stdUnconstrained{} & 36 & 94 & 17.92 \\
 & & \stdConstrained{} & 35 & 99 & 25.28  \\
 Qwen2.5-Coder-7B-Instruct & 8 & \cotUnconstrained{} & 37 & 88 & 138.38 \\
 & & \textbf{\Tool{}} & \textbf{39} & 94 & 155.32\\

\midrule

     & & \stdUnconstrained{} & 20 & 66 & 115.22 \\
&  & \stdConstrained{} & 26 & 95 & 26.99 \\
 Qwen2.5-Math-7B-Instruct & 2 & \cotUnconstrained{} & 28 & 72 & 190.51 \\
 & & \textbf{\Tool{}} & \textbf{32} & 89 & 195.65 \\

 \midrule

    &  & \stdUnconstrained{} & 22 & 83 & 47 \\
 & & \stdConstrained{} & 29 & 98 & 27.08 \\
 Qwen2.5-Math-7B-Instruct & 4 & \cotUnconstrained{} & 28 & 76 & 184.35 \\
 & & \textbf{\Tool{}} & \textbf{37} & 88 & 194.77  \\

 \midrule

     & & \stdUnconstrained{} & 27 & 89 & 25.7 \\
 & & \stdConstrained{} & 29 & 99 & 26.81 \\
 Qwen2.5-Math-7B-Instruct & 8 &  \cotUnconstrained{} & 29 & 82 & 155.26\\
 & & \textbf{\Tool{}} & \textbf{38} & 94 & 158.86 \\

 \midrule

     & & \stdUnconstrained{} & 19 & 61 & 157.36 \\
 & & \stdConstrained{} & 23 & 95 & 45.58  \\
 Llama-3.1-8B-Instruct & 2 & \cotUnconstrained{} & 29 & 84 & 198.64 \\
 & & \textbf{\Tool{}} & \textbf{35} & 94 & 206.85 \\

 \midrule
     & & \stdUnconstrained{} & 18 & 68 & 131.5 \\
 & & \stdConstrained{} & 24 & 96 & 37.38  \\
 Llama-3.1-8B-Instruct & 4 & \cotUnconstrained{} & 26 & 92 & 172.21 \\
 & & \textbf{\Tool{}} & \textbf{30} & 97 & 179.95 \\

  \midrule

     & & \stdUnconstrained{} & 21 & 73 & 128.38 \\
 & & \stdConstrained{} & 26 & 98 & 35.97  \\
 Llama-3.1-8B-Instruct & 8 & \cotUnconstrained{} & 30 & 95 & 163.55 \\
 & & \textbf{\Tool{}} & \textbf{33} & 95 & 170.22 \\


\bottomrule
    \end{tabular}
    \label{tab:gsm_symbolic_comparison_k_shot}
    \vspace{-.2in}
\end{table*}


\subsection{GSM-Symbolic Examples and Prompt}
\label{sec:gsm_info}
\textbf{GSM-Symbolic Problem Solution Examples:}



% \lstdefinestyle{myGrammarStyle}{
%     basicstyle=\scriptsize\ttfamily, % Reduce font size
%     commentstyle=\color{gray},
%     keywordstyle=\color{blue},
%     stringstyle=\color{orange},
%     numbers=left, % Line numbers on left
%     numberstyle=\tiny\color{gray}, % Line numbers styling
%     breaklines=true, % Wrap long lines
%     frame=single, % Frame around the code
%     framesep=3pt, % Adjust frame separation
%     xleftmargin=5pt, % Adjust left margin
%     xrightmargin=5pt, % Adjust right margin
%     backgroundcolor=\color{blue!4}, % Background color
%     tabsize=2, % Tab size
%     captionpos=b, % Caption position bottom
%     aboveskip=5pt, % Reduce space above the listing
%     belowskip=5pt, % Reduce space below the listing
%     linewidth=0.9\linewidth, % Set custom line width to less than text width
%     escapeinside={(*@}{@*)}, % for escaping to LaTeX
% }

\begin{lstlisting}[style=myGrammarStyle, caption=Problem Solution Examples for GSM-Symbolic]
Question: A fog bank rolls in from the ocean to cover a city. It takes {t} minutes to cover every {d} miles of the city. If the city is {y} miles across from oceanfront to the opposite inland edge, how many minutes will it take for the fog bank to cover the whole city?

Answer: y//d*t

Question: {name} makes {drink} using teaspoons of sugar and cups of water in the ratio of {m}:{n}. If she used a total of {x} teaspoons of sugar and cups of water, calculate the number of teaspoonfuls of sugar she used.

Answer: ((m*x)//(m+n))
\end{lstlisting}
\label{gram:gsm_example}
\textbf{GSM-Symbolic Prompt:}



% \lstdefinestyle{myGrammarStyle}{
%     basicstyle=\scriptsize\ttfamily, % Reduce font size
%     commentstyle=\color{gray},
%     keywordstyle=\color{blue},
%     stringstyle=\color{orange},
%     numbers=left, % Line numbers on left
%     numberstyle=\tiny\color{gray}, % Line numbers styling
%     breaklines=true, % Wrap long lines
%     frame=single, % Frame around the code
%     framesep=3pt, % Adjust frame separation
%     xleftmargin=5pt, % Adjust left margin
%     xrightmargin=5pt, % Adjust right margin
%     backgroundcolor=\color{green!4}, % Background color
%     tabsize=2, % Tab size
%     captionpos=b, % Caption position bottom
%     aboveskip=5pt, % Reduce space above the listing
%     belowskip=5pt, % Reduce space below the listing
%     linewidth=0.9\linewidth, % Set custom line width to less than text width
%     escapeinside={(*@}{@*)}, % for escaping to LaTeX
% }

\begin{lstlisting}[style=myGrammarStyle, caption=CoT Prompt Template For GSM-Symbolic Evaluation]
You are an expert in solving grade school math tasks. You will be presented with a grade-school math word problem with symbolic variables and be asked to solve it.

Before answering you should reason about the problem (using the <reasoning> field in the response described below). Intermediate symbolic expressions generated during reasoning should be wrapped in << >>.

Then, output the symbolic expression wrapped in << >> that answers the question. The expressions must use numbers as well as the variables defined in the question. You are only allowed to use the following operations: +, -, /, //, %, (), and int().

You will always respond in the format described below: 
Let's think step by step. <reasoning> The final answer is <<symbolic expression>>

There are {t} trees in the {g}. {g} workers will plant trees in the {g} today. After they are done, there will be {tf} trees. How many trees did the {g} workers plant today?

Let's think step by step. Initially, there are {t} trees. After planting, there are {tf} trees. The number of trees planted is <<tf - t>>. The final answer is <<tf - t>>.

If there are {c} cars in the parking lot and {nc} more cars arrive, how many cars are in the parking lot?

Let's think step by step. Initially, there are {c} cars. {nc} more cars arrive, so the total becomes <<c + nc>>. The final answer is <<c + nc>>.

{p1} had {ch1} {o1} and {p2} had {ch2} {o1}. If they ate {a} {o1}, how many pieces do they have left in total?

Let's think step by step. Initially, {p1} had {ch1} {o1}, and {p2} had {ch2} {o1}, making a total of <<ch1 + ch2>>. After eating {a} {o1}, the remaining total is <<ch1 + ch2 - a>>. The final answer is <<ch1 + ch2 - a>>.

{p1} had {l1} {o1}. {p1} gave {g} {o1} to {p2}. How many {o1} does {p1} have left?

Let's think step by step. {p1} started with {l1} {o1}. After giving {g} {o1} to {p2}, {p1} has <<l1 - g>> {o1} left. The final answer is <<l1 - g>>.

{p1} has {t} {o1}. For Christmas, {p1} got {tm} {o1} from {p2} and {td} {o1} from {p3}. How many {o1} does {p1} have now?

Let's think step by step. {p1} started with {t} {o1}. {p1} received {tm} {o1} from {p2} and {td} {o1} from {p3}. The total is <<t + tm + td>>. The final answer is <<t + tm + td>>.

There were {c} {o1} in the server room. {nc} more {o1} were installed each day, from {d1} to {d2}. How many {o1} are now in the server room?

Let's think step by step. Initially, there were {c} {o1}. {nc} {o1} were added each day for <<d2 - d1 + 1>> days, which is <<nc * (d2 - d1 + 1)>>. The total is <<c + nc * (d2 - d1 + 1)>>. The final answer is <<c + nc * (d2 - d1 + 1)>>.

{p1} had {gb1} {o1}. On {day1}, {p1} lost {l1} {o1}. On {day2}, {p1} lost {l2} more. How many {o1} does {p1} have at the end of {day2}?

Let's think step by step. Initially, {p1} had {gb1} {o1}. After losing {l1} {o1} on {day1}, {p1} had <<gb1 - l1>>. After losing {l2} {o1} on {day2}, the total is <<gb1 - l1 - l2>>. The final answer is <<gb1 - l1 - l2>>.

{p1} has ${m}. {p1} bought {q} {o1} for ${p} each. How much money does {p1} have left?

Let's think step by step. Initially, {p1} had ${m}. {p1} spent <<q * p>> on {q} {o1}. The remaining money is <<m - q * p>>. The final answer is <<m - q * p>>.

{question}
\end{lstlisting}
\label{gram:gsm_prompt}



% \lstdefinestyle{myGrammarStyle}{
%     basicstyle=\scriptsize\ttfamily, % Reduce font size
%     commentstyle=\color{gray},
%     keywordstyle=\color{blue},
%     stringstyle=\color{orange},
%     numbers=left, % Line numbers on left
%     numberstyle=\tiny\color{gray}, % Line numbers styling
%     breaklines=true, % Wrap long lines
%     frame=single, % Frame around the code
%     framesep=3pt, % Adjust frame separation
%     xleftmargin=5pt, % Adjust left margin
%     xrightmargin=5pt, % Adjust right margin
%     backgroundcolor=\color{green!4}, % Background color
%     tabsize=2, % Tab size
%     captionpos=b, % Caption position bottom
%     aboveskip=5pt, % Reduce space above the listing
%     belowskip=5pt, % Reduce space below the listing
%     linewidth=0.9\linewidth, % Set custom line width to less than text width
%     escapeinside={(*@}{@*)}, % for escaping to LaTeX
% }

\begin{lstlisting}[style=myGrammarStyle, caption= Prompt Template For GSM-Symbolic Evaluation Without CoT]
You are an expert in solving grade school math tasks. You will be presented with a grade-school math word problem with symbolic variables and be asked to solve it.

Only output the symbolic expression wrapped in << >> that answers the question. The expression must use numbers as well as the variables defined in the question. You are only allowed to use the following operations: +, -, /, //, %, (), and int().

You will always respond in the format described below: 
<<symbolic expression>>

There are {t} trees in the {g}. {g} workers will plant trees in the {g} today. After they are done, there will be {tf} trees. How many trees did the {g} workers plant today?

<<tf - t>>

If there are {c} cars in the parking lot and {nc} more cars arrive, how many cars are in the parking lot?

<<c + nc>>

{p1} had {ch1} {o1} and {p2} had {ch2} {o1}. If they ate {a} {o1}, how many pieces do they have left in total?

<<ch1 + ch2 - a>>

{p1} had {l1} {o1}. {p1} gave {g} {o1} to {p2}. How many {o1} does {p1} have left?

<<l1 - g>>

{p1} has {t} {o1}. For Christmas, {p1} got {tm} {o1} from {p2} and {td} {o1} from {p3}. How many {o1} does {p1} have now?

<<t + tm + td>>

There were {c} {o1} in the {loc}. {nc} more {o1} were installed each day, from {d1} to {d2}. How many {o1} are now in the {loc}?

<<c + nc * (d2 - d1 + 1)>>

{p1} had {gb1} {o1}. On {day1}, {p1} lost {l1} {o1}. On {day2}, {p1} lost {l2} more. How many {o1} does {p1} have at the end of {day2}?

<<gb1 - l1 - l2>>

{p1} has ${m}. {p1} bought {q} {o1} for ${p} each. How much money does {p1} have left?

<<m - q * p>>

{question}
\end{lstlisting}
\label{gram:gsm_prompt_no_cot}

\subsection{FOLIO Examples and Prompt}
\label{sec:folio_info}
\textbf{FOLIO Problem Solution Examples:}

\begin{lstlisting}[style=myGrammarStyle, caption=Problem Solution Examples for FOLIO]
Question: 
People in this club who perform in school talent shows often attend and are very engaged with school events.
People in this club either perform in school talent shows often or are inactive and disinterested community members.
People in this club who chaperone high school dances are not students who attend the school.
All people in this club who are inactive and disinterested members of their community chaperone high school dances.
All young children and teenagers in this club who wish to further their academic careers and educational opportunities are students who attend the school. 
Bonnie is in this club and she either both attends and is very engaged with school events and is a student who attends the school or is not someone who both attends and is very engaged with school events and is not a student who attends the school.
Based on the above information, is the following statement true, false, or uncertain? Bonnie performs in school talent shows often.
###

FOL Solution: 
Predicates:
InClub(x) ::: x is a member of the club.
Perform(x) ::: x performs in school talent shows.
Attend(x) ::: x attends school events.
Engaged(x) ::: x is very engaged with school events.
Inactive(x) ::: x is an inactive and disinterested community member.
Chaperone(x) ::: x chaperones high school dances.
Student(x) ::: x is a student who attends the school.
Wish(x) ::: x wishes to further their academic careers and educational opportunities.
Premises:
{forall} x (InClub(x) {and} Attend(x) {and} Engaged(x) {implies} Attend(x)) ::: People in this club who perform in school talent shows often attend and are very engaged with school events.
{forall} x (InClub(x) {implies} (Perform(x) {xor} Inactive(x))) ::: People in this club either perform in school talent shows often or are inactive and disinterested community members.
{forall} x (InClub(x) {and} Chaperone(x) {implies} {not}Student(x)) ::: People in this club who chaperone high school dances are not students who attend the school.
{forall} x (InClub(x) {and} Inactive(x) {implies} Chaperone(x)) ::: All people in this club who are inactive and disinterested members of their community chaperone high school dances.
{forall} x (InClub(x) {and} (Young(x) {or} Teenager(x)) {and} Wish(x) {implies} Student(x)) ::: All young children and teenagers in this club who wish to further their academic careers and educational opportunities are students who attend the school.
{forall} x (InClub(x) {implies} (Attend(x) {and} Engaged(x)) {xor} {not}(Attend(x) {and} Engaged(x)) {and} {not}Student(x) {xor} Student(x)) ::: Bonnie is in this club and she either both attends and is very engaged with school events and is a student who attends the school or is not someone who both attends and is very engaged with school events and is not a student who attends the school.
Conclusion:
InClub(bonnie) {and} Perform(bonnie) ::: Bonnie performs in school talent shows often.

Answer: Uncertain

\end{lstlisting}
\label{gram:folio_example}
\textbf{FOLIO Prompt:}
% \lstdefinestyle{myGrammarStyle}{
%     basicstyle=\scriptsize\ttfamily, % Reduce font size
%     commentstyle=\color{gray},
%     keywordstyle=\color{blue},
%     stringstyle=\color{orange},
%     numbers=left, % Line numbers on left
%     numberstyle=\tiny\color{gray}, % Line numbers styling
%     breaklines=true, % Wrap long lines
%     frame=single, % Frame around the code
%     framesep=3pt, % Adjust frame separation
%     xleftmargin=5pt, % Adjust left margin
%     xrightmargin=5pt, % Adjust right margin
%     backgroundcolor=\color{green!4}, % Background color
%     tabsize=2, % Tab size
%     captionpos=b, % Caption position bottom
%     aboveskip=5pt, % Reduce space above the listing
%     belowskip=5pt, % Reduce space below the listing
%     linewidth=0.9\linewidth, % Set custom line width to less than text width
%     escapeinside={(*@}{@*)}, % for escaping to LaTeX
% }

\begin{lstlisting}[style=myGrammarStyle, caption=Prompt Template Used For FOLIO Evaluation]
Given a problem description and a question. The task is to parse the problem and the question into first-order logic formulas.
The grammar of the first-order logic formula is defined as follows:
1) logical conjunction of expr1 and expr2: expr1 {and} expr2
2) logical disjunction of expr1 and expr2: expr1 {or} expr2
3) logical exclusive disjunction of expr1 and expr2: expr1 {xor} expr2
4) logical negation of expr1: {not}expr1
5) expr1 implies expr2: expr1 {implies} expr2
6) expr1 if and only if expr2: expr1 {iff} expr2
7) logical universal quantification: {forall} x
8) logical existential quantification: {exists} x. These are the ONLY operations in the grammar.
------

Answer the question EXACTLY like the examples.

Problem:
All people who regularly drink coffee are dependent on caffeine. People either regularly drink coffee or joke about being addicted to caffeine. No one who jokes about being addicted to caffeine is unaware that caffeine is a drug. Rina is either a student and unaware that caffeine is a drug, or neither a student nor unaware that caffeine is a drug. If Rina is not a person dependent on caffeine and a student, then Rina is either a person dependent on caffeine and a student, or neither a person dependent on caffeine nor a student.
Question:
Based on the above information, is the following statement true, false, or uncertain? Rina is either a person who jokes about being addicted to caffeine or is unaware that caffeine is a drug.
###

We take three steps: first, we define the necessary predicates and premises, and finally, we encode the question `Rina is either a person who jokes about being addicted to caffeine or is unaware that caffeine is a drug.` in the conclusion. Now, we will write only the logic program, nothing else.
Predicates:
Dependent(x) ::: x is a person dependent on caffeine.
Drinks(x) ::: x regularly drinks coffee.
Jokes(x) ::: x jokes about being addicted to caffeine.
Unaware(x) ::: x is unaware that caffeine is a drug.
Student(x) ::: x is a student.
Premises:
{forall} x (Drinks(x) {implies} Dependent(x)) ::: All people who regularly drink coffee are dependent on caffeine.
{forall} x (Drinks(x) {xor} Jokes(x)) ::: People either regularly drink coffee or joke about being addicted to caffeine.
{forall} x (Jokes(x) {implies} {not}Unaware(x)) ::: No one who jokes about being addicted to caffeine is unaware that caffeine is a drug. 
(Student(rina) {and} Unaware(rina)) {xor} {not}(Student(rina) {or} Unaware(rina)) ::: Rina is either a student and unaware that caffeine is a drug, or neither a student nor unaware that caffeine is a drug.
Conclusion:
Jokes(rina) {xor} Unaware(rina) ::: Rina is either a person who jokes about being addicted to caffeine or is unaware that caffeine is a drug.
------

Problem:
Miroslav Venhoda was a Czech choral conductor who specialized in the performance of Renaissance and Baroque music. Any choral conductor is a musician. Some musicians love music. Miroslav Venhoda published a book in 1946 called Method of Studying Gregorian Chant.
Question:
Based on the above information, is the following statement true, false, or uncertain? Miroslav Venhoda loved music.
###

We take three steps: first, we define the necessary predicates and premises, and finally, we encode the question `Miroslav Venhoda loved music.` in the conclusion. Now, we will write only the logic program, nothing else.
Predicates:
Czech(x) ::: x is a Czech person.
ChoralConductor(x) ::: x is a choral conductor.
Musician(x) ::: x is a musician.
Love(x, y) ::: x loves y.
Author(x, y) ::: x is the author of y.
Book(x) ::: x is a book.
Publish(x, y) ::: x is published in year y.
Specialize(x, y) ::: x specializes in y.
Premises:
Czech(miroslav) {and} ChoralConductor(miroslav) {and} Specialize(miroslav, renaissance) {and} Specialize(miroslav, baroque) ::: Miroslav Venhoda was a Czech choral conductor who specialized in the performance of Renaissance and Baroque music.
{forall} x (ChoralConductor(x) {implies} Musician(x)) ::: Any choral conductor is a musician.
{exists} x (Musician(x) {and} Love(x, music)) ::: Some musicians love music.
Book(methodOfStudyingGregorianChant) {and} Author(miroslav, methodOfStudyingGregorianChant) {and} Publish(methodOfStudyingGregorianChant, year1946) ::: Miroslav Venhoda published a book in 1946 called Method of Studying Gregorian Chant.
Conclusion:
Love(miroslav, music) ::: Miroslav Venhoda loved music.
------

{question}
\end{lstlisting}
\label{gram:folio_prompt}


\newpage
\subsection{Case Study For GSM-Symbolic}



\lstdefinestyle{myGrammarStyle}{
    basicstyle=\scriptsize\ttfamily, % Reduce font size
    commentstyle=\color{gray},
    keywordstyle=\color{blue},
    stringstyle=\color{orange},
    numbers=left, % Line numbers on left
    numberstyle=\tiny\color{gray}, % Line numbers styling
    breaklines=true, % Wrap long lines
    frame=single, % Frame around the code
    framesep=3pt, % Adjust frame separation
    xleftmargin=5pt, % Adjust left margin
    xrightmargin=5pt, % Adjust right margin
    backgroundcolor=\color{gray!4}, % Background color
    tabsize=2, % Tab size
    captionpos=b, % Caption position bottom
    aboveskip=5pt, % Reduce space above the listing
    belowskip=5pt, % Reduce space below the listing
    linewidth=0.9\linewidth, % Set custom line width to less than text width
    escapeinside={(*@}{@*)}, % for escaping to LaTeX
}

\begin{lstlisting}[style=myGrammarStyle, caption=Case Study for GSM-Symbolic]
Question: {name} hires a {vehicle} from {start_hour}  to {end_hour}. He gets {free_hours} hours free. The first paid hour is ${first_hour_cost} and each hour after that is {multiplier} the cost. How much did he pay?

Constrained: <<(int(end_hour - start_hour) - free_hours) * first_hour_cost + free_hours * first_hour_cost + (int(end_hour - start_hour) - free_hours - 1) * multiplier * first_hour_cost>>

Unconstrained With CoT:  Let's think step by step. The total time hired is from {start_hour} to {end_hour}, which is <<int((end_hour - start_hour).total_seconds() / 3600)>> hours. Subtracting the {free_hours} free hours, the paid hours are <<int((end_hour - start_hour).total_seconds() / 3600) - free_hours>>. The final answer is <<first_hour_cost + (int((end_hour - start_hour).total_seconds() / 3600) - free_hours - 1) * multiplier * first_hour_cost>>.

CRANE: Let's think step by step. The total time hired is from {start_hour} to {end_hour}, totaling <<int(end_hour - start_hour)>> hours. Subtracting {free_hours} free hours, the paid hours are <<int(end_hour - start_hour) - free_hours>>. The first hour costs {first_hour_cost}, and each additional hour costs {multiplier} times that. The final answer is <<first_hour_cost + (int(end_hour - start_hour) - free_hours - 1) * multiplier * first_hour_cost>>.


\end{lstlisting}
\label{gram:gsm_case_study}

\Tool{} effectively alternates between constrained and unconstrained generation to produce intermediate expressions, the final answer, and to maintain the reasoning capabilities of the LLM. In contrast, unconstrained generation with CoT results in a syntactically incorrect expression, while constrained generation produces a syntactically valid but incorrect expression.

\subsection{Sampling Ablation for GSM-Symbolic}
In our GSM-Symbolic case study, we use IterGen as the constrained generation baseline and initialize \Tool{} with IterGen. Both IterGen and \Tool{} employ selective rejection sampling to filter tokens that do not satisfy semantic constraints. For comparison, we also run unconstrained generation using temperature sampling and evaluate its performance against \Tool{}. Specifically, for Qwen2.5-1.5B-Instruct and Llama-3.1-8B-Instruct, we generate three samples with unconstrained generation at a temperature of \( t = 0.7 \) and compute pass@1/2/3 metrics. 

As shown in Table ~\ref{tab:rejection_sample_gsm}, \Tool{} with greedy decoding achieves higher accuracy than pass@1/2/3 for unconstrained generation with Chain-of-Thought (CoT) and temperature sampling on Qwen2.5-1.5B-Instruct. Although, for Llama-3.1-8B-Instruct, unconstrained generation with CoT and temperature sampling achieves a pass@3 accuracy of 35\%—2\% higher than \Tool{}—it generates approximately 4 times as many tokens as \Tool{}.

\begin{table*}[t]
    \centering
    \small
    \caption{Comparison of \Tool{} and greedy and sampling baselines with different models on GSM-Symbolic.}
    \begin{tabular}{llccr}
        \toprule
        \textbf{Model} & \textbf{Method} & \textbf{pass@1/2/3 (\%)} & \textbf{Parse (\%)} &  \textbf{Tokens} \\
        
\midrule
     & \stdUnconstrained{} (Greedy) & 21 & 97 & 23.34\\
     & \stdUnconstrained{} (t = 0.7) & 15/19/22 & 88/96/98 & 20.19/39.76/60.57\\
 & \stdConstrained{} (Greedy) & 22 & 97 & 25.29 \\
 Qwen2.5-1.5B-Instruct & \cotUnconstrained{} (Greedy) & 26 & 90 & 128.97\\
 & \cotUnconstrained{} (t = 0.7) & 21/25/30 & 78/91/96 & 146.22/292.96/444.61\\
 & \textbf{\Tool{}} & \textbf{31} & 100 & 131.3\\

\midrule

     & \stdUnconstrained{} (Greedy) & 21 & 73 & 128.38\\
     & \stdUnconstrained{} (t = 0.7) & 15/21/25  & 51/74/84 & 106.88/232.75/369.86\\
 & \stdConstrained{} (Greedy) & 26 & 98 & 35.97 \\
 Llama-3.1-8B-Instruct & \cotUnconstrained{} (Greedy) & 30 & 95 & 163.55 \\
 & \cotUnconstrained{} (t = 0.7) & 24/29/\textbf{35} & 89/98/98 & 196.01/403.68/607.7\\
 & \textbf{\Tool{}} (Greedy) & 33 & 95 & 170.22 \\



\bottomrule
    \end{tabular}
    \label{tab:rejection_sample_gsm}
    \vspace{-.2in}
\end{table*}



\begin{table*}[t]
    \centering
    \small
    \caption{Comparison of \Tool{} and greedy and sampling baselines with different models on FOLIO.}
    \begin{tabular}{llccr}
        \toprule
        \textbf{Model} & \textbf{Method} & \textbf{pass@1/2/3 (\%)} & \textbf{Compile (\%)} &  \textbf{Tokens} \\
    \midrule
    & \cotUnconstrained{} (Greedy) & 36.95 & 70.94 & 350.64  \\
    & \cotUnconstrained{} (t = 0.7) & 16.75/28.57/34.98 & 35.96/55.67/68.47 & 401.5/800.19/1219.33  \\
 Qwen2.5-7B-Instruct & \stdConstrained{} (Greedy) & 37.44 & 87.68 & 775.62 \\
 & \textbf{\Tool{}} (Greedy) & \textbf{42.36} & 87.68 & 726.88  \\

% Qwen2.5-7B & \cotConstrained{} & 40.39 & 774.18 & 25.74 \\
 \midrule
     & \cotUnconstrained{} (Greedy) & 32.02 & 57.14 & 371.52  \\
     & \cotUnconstrained{} (t = 0.7) & 14.29/22.66/29.06 & 33.99/46.8/57.64 & 435.35/877.33/1307.45  \\
 Llama-3.1-8B-Instruct & \stdConstrained{} (Greedy) & 39.41 & 86.21 & 549.75  \\
 & \textbf{\Tool{}} (Greedy) & \textbf{46.31} & 85.71 & 449.77  \\




\bottomrule
    \end{tabular}
    \label{tab:rejection_sample_fol}
    \vspace{-.2in}
\end{table*}


\subsection{Grammars}
\subsubsection{GSM-Symbolic Grammar}
\label{sec:gsm_grammar}

\lstdefinestyle{myGrammarStyle}{
    basicstyle=\scriptsize\ttfamily, % Reduce font size
    commentstyle=\color{gray},
    keywordstyle=\color{blue},
    stringstyle=\color{orange},
    numbers=left, % Line numbers on left
    numberstyle=\tiny\color{gray}, % Line numbers styling
    breaklines=true, % Wrap long lines
    frame=single, % Frame around the code
    framesep=3pt, % Adjust frame separation
    xleftmargin=5pt, % Adjust left margin
    xrightmargin=5pt, % Adjust right margin
    backgroundcolor=\color{yellow!4}, % Background color
    tabsize=2, % Tab size
    captionpos=b, % Caption position bottom
    aboveskip=5pt, % Reduce space above the listing
    belowskip=5pt, % Reduce space below the listing
    linewidth=0.9\linewidth, % Set custom line width to less than text width
    escapeinside={(*@}{@*)}, % for escaping to LaTeX
}

\begin{lstlisting}[style=myGrammarStyle, caption=GSM-Symbolic Grammar]
start: space? "<" "<" space? expr space? ">" ">" space?

expr: expr space? "+" space? term   
     | expr space? "-" space? term   
     | term

term: term space? "*" space? factor 
     | term space? "/" space? factor 
     | term space? "//" space? factor 
     | term space? "%" space? factor  
     | factor space?

factor: "-" space? factor    
       | TYPE "(" space? expr space? ")" 
       | primary space?

primary: NUMBER        
        | VARIABLE      
        | "(" space? expr space? ")"

TYPE.4: "int"

space: " "

%import common.CNAME -> VARIABLE
%import common.NUMBER
\end{lstlisting}
\label{gram:gsm_grammar}



\subsubsection{Prover9 Grammar}
\label{sec:prover9_grammar}

\lstdefinestyle{myGrammarStyle}{
    basicstyle=\scriptsize\ttfamily, % Reduce font size
    commentstyle=\color{gray},
    keywordstyle=\color{blue},
    stringstyle=\color{orange},
    numbers=left, % Line numbers on left
    numberstyle=\tiny\color{gray}, % Line numbers styling
    breaklines=true, % Wrap long lines
    frame=single, % Frame around the code
    framesep=3pt, % Adjust frame separation
    xleftmargin=5pt, % Adjust left margin
    xrightmargin=5pt, % Adjust right margin
    backgroundcolor=\color{yellow!4}, % Background color
    tabsize=2, % Tab size
    captionpos=b, % Caption position bottom
    aboveskip=5pt, % Reduce space above the listing
    belowskip=5pt, % Reduce space below the listing
    linewidth=0.9\linewidth, % Set custom line width to less than text width
    escapeinside={(*@}{@*)}, % for escaping to LaTeX
}

\begin{lstlisting}[style=myGrammarStyle, caption=Prover9 Grammar]
start: predicate_section premise_section conclusion_section

predicate_section: "Predicates:" predicate_definition+
premise_section: "Premises:" premise+
conclusion_section: "Conclusion:" conclusion+

predicate_definition: PREDICATE "(" VAR ("," VAR)* ")" COMMENT  -> define_predicate
premise: quantified_expr COMMENT -> define_premise
conclusion: quantified_expr COMMENT -> define_conclusion

quantified_expr: quantifier VAR "(" expression ")" | expression
quantifier: "{forall}" -> forall | "{exists}" -> exists

expression: bimplication_expr

?bimplication_expr: implication_expr ("{iff}" bimplication_expr)?  -> iff
?implication_expr: xor_expr ("{implies}" implication_expr)?  -> imply
?xor_expr: or_expr ("{xor}" xor_expr)?                -> xor
?or_expr: and_expr ("{or}" or_expr)?                -> or
?and_expr: neg_expr ("{and}" and_expr)?              -> and
?neg_expr: "{not}" quantified_expr                   -> neg 
        | atom

?atom: PREDICATE "(" VAR ("," VAR)* ")" -> predicate 
    | "(" quantified_expr ")" 

// Variable names begin with a lowercase letter
VAR.-1: /[a-z][a-zA-Z0-9_]*/  | /[0-9]+/

// Predicate names begin with a capital letter
PREDICATE.-1: /[A-Z][a-zA-Z0-9]*/

COMMENT: /:::.*\n/

%import common.WS
%ignore WS
\end{lstlisting}
\label{gram:prover9_grammar}



