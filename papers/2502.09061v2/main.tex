%%%%%%%% ICML 2024 EXAMPLE LATEX SUBMISSION FILE %%%%%%%%%%%%%%%%%

\documentclass{article}

% Recommended, but optional, packages for figures and better typesetting:
\usepackage{microtype}
\usepackage[final]{graphicx}
\usepackage{subfigure}
\usepackage{booktabs} % for professional tables
\usepackage{enumitem}
\usepackage{caption}
\usepackage{diagbox}
\usepackage{multirow}
\usepackage[utf8]{inputenc}
\usepackage{algorithm}
\usepackage[noend]{algpseudocode}
\usepackage{ifthen}
\newboolean{icml}
\setboolean{icml}{false}

% hyperref makes hyperlinks in the resulting PDF.
% If your build breaks (sometimes temporarily if a hyperlink spans a page)
% please comment out the following usepackage line and replace
% \usepackage{icml2024} with \usepackage[nohyperref]{icml2024} above.
\usepackage{hyperref}


% Attempt to make hyperref and algorithmic work together better:
% \newcommand{\theHalgorithm}{\arabic{algorithm}}

% Use the following line for the initial blind version submitted for review:
% \usepackage{icml2024}

% If accepted, instead use the following line for the camera-ready submission:

\ifthenelse{\boolean{icml}}
{
\usepackage{icml2025}
}
{
\usepackage[accepted]{icml2025}
}

% For theorems and such
\usepackage{amsmath}
\usepackage{amssymb}
\usepackage{mathtools}
\usepackage{amsthm}
\usepackage{xspace}
\usepackage[]{xcolor}
\usepackage{thm-restate}
\usepackage{subcaption}

% For code blocks
\usepackage{listings}
\lstdefinestyle{myGrammarStyle}{
    basicstyle=\scriptsize\ttfamily, % Reduce font size
    commentstyle=\color{gray},
    keywordstyle=\color{blue},
    stringstyle=\color{orange},
    numbers=left, % Line numbers on left
    numberstyle=\tiny\color{gray}, % Line numbers styling
    breaklines=true, % Wrap long lines
    frame=single, % Frame around the code
    framesep=3pt, % Adjust frame separation
    xleftmargin=5pt, % Adjust left margin
    xrightmargin=5pt, % Adjust right margin
    backgroundcolor=\color{blue!4}, % Background color
    tabsize=2, % Tab size
    captionpos=b, % Caption position bottom
    aboveskip=5pt, % Reduce space above the listing
    belowskip=5pt, % Reduce space below the listing
    linewidth=0.9\linewidth, % Set custom line width to less than text width
    escapeinside={(*@}{@*)}, % for escaping to LaTeX
}

% if you use cleveref..
\usepackage[capitalize,noabbrev]{cleveref}

%%%%%%%%%%%%%%%%%%%%%%%%%%%%%%%%
% THEOREMS
%%%%%%%%%%%%%%%%%%%%%%%%%%%%%%%%
\theoremstyle{plain}
\newtheorem{theorem}{Theorem}[section]
\newtheorem{proposition}[theorem]{Proposition}
\newtheorem{lemma}[theorem]{Lemma}
\newtheorem{corollary}[theorem]{Corollary}
\theoremstyle{definition}
\newtheorem{definition}[theorem]{Definition}
\newtheorem{assumption}[theorem]{Assumption}
\theoremstyle{remark}
\newtheorem{remark}[theorem]{Remark}
\DeclareMathOperator*{\argmin}{arg\,min}

% Todonotes is useful during development; simply uncomment the next line
%    and comment out the line below the next line to turn off comments
%\usepackage[disable,textsize=tiny]{todonotes}
\usepackage[textsize=tiny]{todonotes}

% The \icmltitle you define below is probably too long as a header.
% Therefore, a short form for the running title is supplied here:

\ifthenelse{\boolean{icml}}
{
\icmltitlerunning{CRANE: Reasoning with constrained LLM generation}
}
\icmltitlerunning{\icmltitlerunning{CRANE: Reasoning with constrained LLM generation}
}

\begin{document}

\twocolumn[
\ifthenelse{\boolean{icml}}
{
\icmltitle{CRANE: Reasoning with constrained LLM generation}
}
{
\icmltitle{CRANE: Reasoning with constrained LLM generation}

}

% It is OKAY to include author information, even for blind
% submissions: the style file will automatically remove it for you
% unless you've provided the [accepted] option to the icml2024
% package.

% List of affiliations: The first argument should be a (short)
% identifier you will use later to specify author affiliations
% Academic affiliations should list Department, University, City, Region, Country
% Industry affiliations should list Company, City, Region, Country

% You can specify symbols, otherwise they are numbered in order.
% Ideally, you should not use this facility. Affiliations will be numbered
% in order of appearance and this is the preferred way.
\icmlsetsymbol{equal}{*}

\begin{icmlauthorlist}
\icmlauthor{Debangshu Banerjee}{yyy,equal}
\icmlauthor{Tarun Suresh}{yyy,equal}
\icmlauthor{Shubham Ugare}{yyy}
\icmlauthor{Sasa Misailovic}{yyy}
\icmlauthor{Gagandeep Singh}{yyy}
\end{icmlauthorlist}

\icmlaffiliation{yyy}{Department of Computer Science, University of Illinois Urbana-Champaign, USA}
% \icmlaffiliation{comp}{VMware Research, USA}

\icmlcorrespondingauthor{Debangshu Banerjee}{db21@illinois.edu}

% You may provide any keywords that you
% find helpful for describing your paper; these are used to populate
% the "keywords" metadata in the PDF but will not be shown in the document

\icmlkeywords{Machine Learning, ICML}

\vskip 0.3in]

% this must go after the closing bracket ] following \twocolumn[ ...

% This command actually creates the footnote in the first column
% listing the affiliations and the copyright notice.
% The command takes one argument, which is text to display at the start of the footnote.
% The \icmlEqualContribution command is standard text for equal contribution.
% Remove it (just {}) if you do not need this facility.

%\printAffiliationsAndNotice{}  % leave blank if no need to mention equal contribution
\printAffiliationsAndNotice{\icmlEqualContribution} % otherwise use the standard text.

\usepackage{bbm}
\usepackage{graphicx}
\usepackage{amsmath,amssymb,amsthm,amsfonts}

\usepackage{paralist}
\usepackage{bm}
\usepackage{xspace}
\usepackage{url}
\usepackage{prettyref}
\usepackage{boxedminipage}
\usepackage{wrapfig}
\usepackage{ifthen}
\usepackage{color}
\usepackage{xspace}

\newcommand{\ii}{{\sc Indicator-Instance}\xspace}
\newcommand{\midd}{{\sf mid}}


\usepackage{amsmath,amsthm,amsfonts,amssymb}
\usepackage{mathtools}
\usepackage{graphicx}


% \usepackage{fullpage}

\usepackage{nicefrac}

\newtheorem{inftheorem}{Informal Theorem}
\newtheorem{claim}{Claim}
\newtheorem*{definition*}{Definition}
\newtheorem{example}{Example}

\DeclareMathOperator*{\argmax}{arg\,max}
\DeclareMathOperator*{\argmin}{arg\,min}
\usepackage{subcaption}

\newtheorem{problem}{Problem}
\usepackage[utf8]{inputenc}
\newcommand{\rank}{\mathsf{rank}}
\newcommand{\tr}{\mathsf{Tr}}
\newcommand{\tv}{\mathsf{TV}}
\newcommand{\opt}{\mathsf{OPT}}
\newcommand{\rr}{\textsc{R}\space}
\newcommand{\alg}{\textsf{Alg}\space}
\newcommand{\sd}{\textsf{sd}_\lambda}
\newcommand{\lblq}{\mathfrak{lq} (X_1)}
\newcommand{\diag}{\textsf{diag}}
\newcommand{\sign}{\textsf{sgn}}
\newcommand{\BC}{\texttt{BC} }
\newcommand{\MM}{\texttt{MM} }
\newcommand{\Nexp}{N_{\mathrm{exp}}}
\newcommand{\Nrep}{N_{\mathrm{replay}}}
\newcommand{\Drep}{D_{\mathrm{replay}}}
\newcommand{\Nsim}{N_{\mathrm{sim}}}
\newcommand{\piBC}{\pi^{\texttt{BC}}}
\newcommand{\piRE}{\pi^{\texttt{RE}}}
\newcommand{\piEMM}{\pi^{\texttt{MM}}}
\newcommand{\mmd}{\texttt{Mimic-MD} }
\newcommand{\RE}{\texttt{RE} }
\newcommand{\dem}{\pi^E}
\newcommand{\Rlint}{\mathcal{R}_{\mathrm{lin,t}}}
\newcommand{\Rlipt}{\mathcal{R}_{\mathrm{lip,t}}}
\newcommand{\Rlin}{\mathcal{R}_{\mathrm{lin}}}
\newcommand{\Rlip}{\mathcal{R}_{\mathrm{lip}}}
\newcommand{\Rmax}{R_{\mathrm{max}}}
\newcommand{\Rall}{\mathcal{R}_{\mathrm{all}}}
\newcommand{\Rdet}{\mathcal{R}_{\mathrm{det}}}
\newcommand{\Fmax}{F_{\mathrm{max}}}
\newcommand{\Nmax}{\mathcal{N}_{\mathrm{max}}}
\newcommand{\piref}{\pi^{\mathrm{ref}}}
\newcommand{\green}{\text{\color{green!75!black} green}\;}
\newcommand{\thetaBC}{\widehat{\theta}^{\textsf{BC}}}
\newcommand{\ent}{\mathcal{E}_{\Theta,n,\delta}}
\newcommand{\eNt}{\mathcal{E}_{\Theta_t,\Nexp,\delta}}
\newcommand{\eNtH}{\mathcal{E}_{\Theta_t,\Nexp,\delta/H}}

\newcommand{\eref}[1]{(\ref{#1})}
\newcommand{\sref}[1]{Sec. \ref{#1}}
\newcommand{\dr}{\widehat{d}_{\mathrm{replay}}}
\newcommand{\figref}[1]{Fig. \ref{#1}}

\usepackage{xcolor}
\definecolor{expert}{HTML}{008000}
\definecolor{error}{HTML}{f96565}
\newcommand{\GKS}[1]{{\textcolor{violet}{\textbf{GKS: #1}}}}
\newcommand{\Q}[1]{{\textcolor{red}{\textbf{Question #1}}}}
\newcommand{\ZSW}[1]{{\textcolor{orange}{\textbf{ZSW: #1}}}}
\newcommand{\JAB}[1]{{\textcolor{teal}{\textbf{JAB: #1}}}}
\newcommand{\jab}[1]{{\textcolor{teal}{\textbf{JAB: #1}}}}
\newcommand{\SAN}[1]{{\textcolor{blue}{\textbf{SC: #1}}}}
\newcommand{\scnote}[1]{\SAN{#1}}
\newcommand{\norm}[1]{\left\lVert #1 \right\rVert}

\usepackage{color-edits}
\addauthor{sw}{blue}

\usepackage{thmtools}
\usepackage{thm-restate}

\usepackage{tikz}
\usetikzlibrary{arrows,calc} 
\newcommand{\tikzAngleOfLine}{\tikz@AngleOfLine}
\def\tikz@AngleOfLine(#1)(#2)#3{%
\pgfmathanglebetweenpoints{%
\pgfpointanchor{#1}{center}}{%
\pgfpointanchor{#2}{center}}
\pgfmathsetmacro{#3}{\pgfmathresult}%
}

\declaretheoremstyle[
    headfont=\normalfont\bfseries, 
    bodyfont = \normalfont\itshape]{mystyle} 
\declaretheorem[name=Theorem,style=mystyle,numberwithin=section]{thm}

% \usepackage{algorithm}
% \usepackage{algorithmic}
\usepackage[linesnumbered,algoruled,boxed,lined,noend]{algorithm2e}

\usepackage{listings}
\usepackage{amsmath}
\usepackage{amsthm}
\usepackage{tikz}
\usepackage{caption}
\usepackage{mdwmath}
\usepackage{multirow}
\usepackage{mdwtab}
\usepackage{eqparbox}
\usepackage{multicol}
\usepackage{amsfonts}
\usepackage{tikz}
\usepackage{multirow,bigstrut,threeparttable}
\usepackage{amsthm}
\usepackage{bbm}
\usepackage{epstopdf}
\usepackage{mdwmath}
\usepackage{mdwtab}
\usepackage{eqparbox}
\usetikzlibrary{topaths,calc}
\usepackage{latexsym}
\usepackage{cite}
\usepackage{amssymb}
\usepackage{bm}
\usepackage{amssymb}
\usepackage{graphicx}
\usepackage{mathrsfs}
\usepackage{epsfig}
\usepackage{psfrag}
\usepackage{setspace}
\usepackage[%dvips,
            CJKbookmarks=true,
            bookmarksnumbered=true,
            bookmarksopen=true,
%						bookmarks=false,
            colorlinks=true,
            citecolor=red,
            linkcolor=blue,
            anchorcolor=red,
            urlcolor=blue
            ]{hyperref}
%\usepackage{algorithm}
\usepackage[linesnumbered,algoruled,boxed,lined]{algorithm2e}
\usepackage{algpseudocode}
\usepackage{stfloats}
\RequirePackage[numbers]{natbib}

\usepackage{comment}
\usepackage{mathtools}
\usepackage{blkarray}
\usepackage{multirow,bigdelim,dcolumn,booktabs}

\usepackage{xparse}
\usepackage{tikz}
\usetikzlibrary{calc}
\usetikzlibrary{decorations.pathreplacing,matrix,positioning}

\usepackage[T1]{fontenc}
\usepackage[utf8]{inputenc}
\usepackage{mathtools}
\usepackage{blkarray, bigstrut}
\usepackage{gauss}

\newenvironment{mygmatrix}{\def\mathstrut{\vphantom{\big(}}\gmatrix}{\endgmatrix}

\newcommand{\tikzmark}[1]{\tikz[overlay,remember picture] \node (#1) {};}

%% Adapted form https://tex.stackexchange.com/questions/206898/braces-for-cases-in-tabular-environment/207704#207704
\newcommand*{\BraceAmplitude}{0.4em}%
\newcommand*{\VerticalOffset}{0.5ex}%  
\newcommand*{\HorizontalOffset}{0.0em}% 
\newcommand*{\blocktextwid}{3.0cm}%
\NewDocumentCommand{\InsertLeftBrace}{%
	O{} % #1 = draw options
	O{\HorizontalOffset,\VerticalOffset} % #2 = optional brace shift options
	O{\blocktextwid} % #3 = optional text width
	m   % #4 = top tikzmark
	m   % #5 = bottom tikzmark
	m   % #6 = node text
}{%
	\begin{tikzpicture}[overlay,remember picture]
	\coordinate (Brace Top)    at ($(#4.north) + (#2)$);
	\coordinate (Brace Bottom) at ($(#5.south) + (#2)$);
	\draw [decoration={brace, amplitude=\BraceAmplitude}, decorate, thick, draw=black, #1]
	(Brace Bottom) -- (Brace Top) 
	node [pos=0.5, anchor=east, align=left, text width=#3, color=black, xshift=\BraceAmplitude] {#6};
	\end{tikzpicture}%
}%
\NewDocumentCommand{\InsertRightBrace}{%
	O{} % #1 = draw options
	O{\HorizontalOffset,\VerticalOffset} % #2 = optional brace shift options
	O{\blocktextwid} % #3 = optional text width
	m   % #4 = top tikzmark
	m   % #5 = bottom tikzmark
	m   % #6 = node text
}{%
	\begin{tikzpicture}[overlay,remember picture]
	\coordinate (Brace Top)    at ($(#4.north) + (#2)$);
	\coordinate (Brace Bottom) at ($(#5.south) + (#2)$);
	\draw [decoration={brace, amplitude=\BraceAmplitude}, decorate, thick, draw=black, #1]
	(Brace Top) -- (Brace Bottom) 
	node [pos=0.5, anchor=west, align=left, text width=#3, color=black, xshift=\BraceAmplitude] {#6};
	\end{tikzpicture}%
}%
\NewDocumentCommand{\InsertTopBrace}{%
	O{} % #1 = draw options
	O{\HorizontalOffset,\VerticalOffset} % #2 = optional brace shift options
	O{\blocktextwid} % #3 = optional text width
	m   % #4 = top tikzmark
	m   % #5 = bottom tikzmark
	m   % #6 = node text
}{%
	\begin{tikzpicture}[overlay,remember picture]
	\coordinate (Brace Top)    at ($(#4.west) + (#2)$);
	\coordinate (Brace Bottom) at ($(#5.east) + (#2)$);
	\draw [decoration={brace, amplitude=\BraceAmplitude}, decorate, thick, draw=black, #1]
	(Brace Top) -- (Brace Bottom) 
	node [pos=0.5, anchor=south, align=left, text width=#3, color=black, xshift=\BraceAmplitude] {#6};
	\end{tikzpicture}%
}%

\usetikzlibrary{patterns}

\definecolor{cof}{RGB}{219,144,71}
\definecolor{pur}{RGB}{186,146,162}
\definecolor{greeo}{RGB}{91,173,69}
\definecolor{greet}{RGB}{52,111,72}

% provide arXiv number if available:
% \arxiv{cs.IT/1502.00326}

% put your definitions there:

%\newtheorem{remark}{Remark} \def\remref#1{Remark~\ref{#1}}
%\newtheorem{conjecture}{Conjecture} \def\remref#1{Remark~\ref{#1}}
%\newtheorem{example}{Example}

%\theorembodyfont{\itshape}
%\newtheorem{theorem}{Theorem}
%\newtheorem{proposition}{Proposition}
%\newtheorem{lemma}{Lemma} \def\lemref#1{Lemma~\ref{#1}}
%\newtheorem{corollary}{Corollary}


%\theorembodyfont{\rmfamily}
%\newtheorem{definition}{Definition}
%\numberwithin{equation}{section}
% \theoremstyle{plain}
% \newtheorem{theorem}{Theorem}
% \newtheorem{Example}{Example}
% \newtheorem{lemma}{Lemma}
% \newtheorem{remark}{Remark}
% \newtheorem{corollary}{Corollary}
% \newtheorem{definition}{Definition}
% \newtheorem{conjecture}{Conjecture}
% \newtheorem{question}{Question}
% \newtheorem*{induction}{Induction Hypothesis}
% \newtheorem*{folklore}{Folklore}
% \newtheorem{assumption}{Assumption}

\def \by {\bar{y}}
\def \bx {\bar{x}}
\def \bh {\bar{h}}
\def \bz {\bar{z}}
\def \cF {\mathcal{F}}
\def \bP {\mathbb{P}}
\def \bE {\mathbb{E}}
\def \bR {\mathbb{R}}
\def \bF {\mathbb{F}}
\def \cG {\mathcal{G}}
\def \cM {\mathcal{M}}
\def \cB {\mathcal{B}}
\def \cN {\mathcal{N}}
\def \var {\mathsf{Var}}
\def\1{\mathbbm{1}}
\def \FF {\mathbb{F}}


\newenvironment{keywords}
{\bgroup\leftskip 20pt\rightskip 20pt \small\noindent{\bfseries
Keywords:} \ignorespaces}%
{\par\egroup\vskip 0.25ex}
\newlength\aftertitskip     \newlength\beforetitskip
\newlength\interauthorskip  \newlength\aftermaketitskip















%%%%%%%%%%%%%%%%%%%%%%%%%%%% by Wu %%%%%%%%%%%%%%%%%%%%%%%%%%%%
\usepackage{xspace}

\newcommand{\Lip}{\mathrm{Lip}}
\newcommand{\stepa}[1]{\overset{\rm (a)}{#1}}
\newcommand{\stepb}[1]{\overset{\rm (b)}{#1}}
\newcommand{\stepc}[1]{\overset{\rm (c)}{#1}}
\newcommand{\stepd}[1]{\overset{\rm (d)}{#1}}
\newcommand{\stepe}[1]{\overset{\rm (e)}{#1}}
\newcommand{\stepf}[1]{\overset{\rm (f)}{#1}}


\newcommand{\floor}[1]{{\left\lfloor {#1} \right \rfloor}}
\newcommand{\ceil}[1]{{\left\lceil {#1} \right \rceil}}

\newcommand{\blambda}{\bar{\lambda}}
\newcommand{\reals}{\mathbb{R}}
\newcommand{\naturals}{\mathbb{N}}
\newcommand{\integers}{\mathbb{Z}}
\newcommand{\Expect}{\mathbb{E}}
\newcommand{\expect}[1]{\mathbb{E}\left[#1\right]}
\newcommand{\Prob}{\mathbb{P}}
\newcommand{\prob}[1]{\mathbb{P}\left[#1\right]}
\newcommand{\pprob}[1]{\mathbb{P}[#1]}
\newcommand{\intd}{{\rm d}}
\newcommand{\TV}{{\sf TV}}
\newcommand{\LC}{{\sf LC}}
\newcommand{\PW}{{\sf PW}}
\newcommand{\htheta}{\hat{\theta}}
\newcommand{\eexp}{{\rm e}}
\newcommand{\expects}[2]{\mathbb{E}_{#2}\left[ #1 \right]}
\newcommand{\diff}{{\rm d}}
\newcommand{\eg}{e.g.\xspace}
\newcommand{\ie}{i.e.\xspace}
\newcommand{\iid}{i.i.d.\xspace}
\newcommand{\fracp}[2]{\frac{\partial #1}{\partial #2}}
\newcommand{\fracpk}[3]{\frac{\partial^{#3} #1}{\partial #2^{#3}}}
\newcommand{\fracd}[2]{\frac{\diff #1}{\diff #2}}
\newcommand{\fracdk}[3]{\frac{\diff^{#3} #1}{\diff #2^{#3}}}
\newcommand{\renyi}{R\'enyi\xspace}
\newcommand{\lpnorm}[1]{\left\|{#1} \right\|_{p}}
\newcommand{\linf}[1]{\left\|{#1} \right\|_{\infty}}
\newcommand{\lnorm}[2]{\left\|{#1} \right\|_{{#2}}}
\newcommand{\Lploc}[1]{L^{#1}_{\rm loc}}
\newcommand{\hellinger}{d_{\rm H}}
\newcommand{\Fnorm}[1]{\lnorm{#1}{\rm F}}
%% parenthesis
\newcommand{\pth}[1]{\left( #1 \right)}
\newcommand{\qth}[1]{\left[ #1 \right]}
\newcommand{\sth}[1]{\left\{ #1 \right\}}
\newcommand{\bpth}[1]{\Bigg( #1 \Bigg)}
\newcommand{\bqth}[1]{\Bigg[ #1 \Bigg]}
\newcommand{\bsth}[1]{\Bigg\{ #1 \Bigg\}}
\newcommand{\xxx}{\textbf{xxx}\xspace}
\newcommand{\toprob}{{\xrightarrow{\Prob}}}
\newcommand{\tolp}[1]{{\xrightarrow{L^{#1}}}}
\newcommand{\toas}{{\xrightarrow{{\rm a.s.}}}}
\newcommand{\toae}{{\xrightarrow{{\rm a.e.}}}}
\newcommand{\todistr}{{\xrightarrow{{\rm D}}}}
\newcommand{\eqdistr}{{\stackrel{\rm D}{=}}}
\newcommand{\iiddistr}{{\stackrel{\text{\iid}}{\sim}}}
%\newcommand{\var}{\mathsf{var}}
\newcommand\indep{\protect\mathpalette{\protect\independenT}{\perp}}
\def\independenT#1#2{\mathrel{\rlap{$#1#2$}\mkern2mu{#1#2}}}
\newcommand{\Bern}{\text{Bern}}
\newcommand{\Poi}{\mathsf{Poi}}
\newcommand{\iprod}[2]{\left \langle #1, #2 \right\rangle}
\newcommand{\Iprod}[2]{\langle #1, #2 \rangle}
\newcommand{\indc}[1]{{\mathbf{1}_{\left\{{#1}\right\}}}}
\newcommand{\Indc}{\mathbf{1}}
\newcommand{\regoff}[1]{\textsf{Reg}_{\mathcal{F}}^{\text{off}} (#1)}
\newcommand{\regon}[1]{\textsf{Reg}_{\mathcal{F}}^{\text{on}} (#1)}

\definecolor{myblue}{rgb}{.8, .8, 1}
\definecolor{mathblue}{rgb}{0.2472, 0.24, 0.6} % mathematica's Color[1, 1--3]
\definecolor{mathred}{rgb}{0.6, 0.24, 0.442893}
\definecolor{mathyellow}{rgb}{0.6, 0.547014, 0.24}


\newcommand{\red}{\color{red}}
\newcommand{\blue}{\color{blue}}
\newcommand{\nb}[1]{{\sf\blue[#1]}}
\newcommand{\nbr}[1]{{\sf\red[#1]}}

\newcommand{\tmu}{{\tilde{\mu}}}
\newcommand{\tf}{{\tilde{f}}}
\newcommand{\tp}{\tilde{p}}
\newcommand{\tilh}{{\tilde{h}}}
\newcommand{\tu}{{\tilde{u}}}
\newcommand{\tx}{{\tilde{x}}}
\newcommand{\ty}{{\tilde{y}}}
\newcommand{\tz}{{\tilde{z}}}
\newcommand{\tA}{{\tilde{A}}}
\newcommand{\tB}{{\tilde{B}}}
\newcommand{\tC}{{\tilde{C}}}
\newcommand{\tD}{{\tilde{D}}}
\newcommand{\tE}{{\tilde{E}}}
\newcommand{\tF}{{\tilde{F}}}
\newcommand{\tG}{{\tilde{G}}}
\newcommand{\tH}{{\tilde{H}}}
\newcommand{\tI}{{\tilde{I}}}
\newcommand{\tJ}{{\tilde{J}}}
\newcommand{\tK}{{\tilde{K}}}
\newcommand{\tL}{{\tilde{L}}}
\newcommand{\tM}{{\tilde{M}}}
\newcommand{\tN}{{\tilde{N}}}
\newcommand{\tO}{{\tilde{O}}}
\newcommand{\tP}{{\tilde{P}}}
\newcommand{\tQ}{{\tilde{Q}}}
\newcommand{\tR}{{\tilde{R}}}
\newcommand{\tS}{{\tilde{S}}}
\newcommand{\tT}{{\tilde{T}}}
\newcommand{\tU}{{\tilde{U}}}
\newcommand{\tV}{{\tilde{V}}}
\newcommand{\tW}{{\tilde{W}}}
\newcommand{\tX}{{\tilde{X}}}
\newcommand{\tY}{{\tilde{Y}}}
\newcommand{\tZ}{{\tilde{Z}}}

\newcommand{\sfa}{{\mathsf{a}}}
\newcommand{\sfb}{{\mathsf{b}}}
\newcommand{\sfc}{{\mathsf{c}}}
\newcommand{\sfd}{{\mathsf{d}}}
\newcommand{\sfe}{{\mathsf{e}}}
\newcommand{\sff}{{\mathsf{f}}}
\newcommand{\sfg}{{\mathsf{g}}}
\newcommand{\sfh}{{\mathsf{h}}}
\newcommand{\sfi}{{\mathsf{i}}}
\newcommand{\sfj}{{\mathsf{j}}}
\newcommand{\sfk}{{\mathsf{k}}}
\newcommand{\sfl}{{\mathsf{l}}}
\newcommand{\sfm}{{\mathsf{m}}}
\newcommand{\sfn}{{\mathsf{n}}}
\newcommand{\sfo}{{\mathsf{o}}}
\newcommand{\sfp}{{\mathsf{p}}}
\newcommand{\sfq}{{\mathsf{q}}}
\newcommand{\sfr}{{\mathsf{r}}}
\newcommand{\sfs}{{\mathsf{s}}}
\newcommand{\sft}{{\mathsf{t}}}
\newcommand{\sfu}{{\mathsf{u}}}
\newcommand{\sfv}{{\mathsf{v}}}
\newcommand{\sfw}{{\mathsf{w}}}
\newcommand{\sfx}{{\mathsf{x}}}
\newcommand{\sfy}{{\mathsf{y}}}
\newcommand{\sfz}{{\mathsf{z}}}
\newcommand{\sfA}{{\mathsf{A}}}
\newcommand{\sfB}{{\mathsf{B}}}
\newcommand{\sfC}{{\mathsf{C}}}
\newcommand{\sfD}{{\mathsf{D}}}
\newcommand{\sfE}{{\mathsf{E}}}
\newcommand{\sfF}{{\mathsf{F}}}
\newcommand{\sfG}{{\mathsf{G}}}
\newcommand{\sfH}{{\mathsf{H}}}
\newcommand{\sfI}{{\mathsf{I}}}
\newcommand{\sfJ}{{\mathsf{J}}}
\newcommand{\sfK}{{\mathsf{K}}}
\newcommand{\sfL}{{\mathsf{L}}}
\newcommand{\sfM}{{\mathsf{M}}}
\newcommand{\sfN}{{\mathsf{N}}}
\newcommand{\sfO}{{\mathsf{O}}}
\newcommand{\sfP}{{\mathsf{P}}}
\newcommand{\sfQ}{{\mathsf{Q}}}
\newcommand{\sfR}{{\mathsf{R}}}
\newcommand{\sfS}{{\mathsf{S}}}
\newcommand{\sfT}{{\mathsf{T}}}
\newcommand{\sfU}{{\mathsf{U}}}
\newcommand{\sfV}{{\mathsf{V}}}
\newcommand{\sfW}{{\mathsf{W}}}
\newcommand{\sfX}{{\mathsf{X}}}
\newcommand{\sfY}{{\mathsf{Y}}}
\newcommand{\sfZ}{{\mathsf{Z}}}


\newcommand{\calA}{{\mathcal{A}}}
\newcommand{\calB}{{\mathcal{B}}}
\newcommand{\calC}{{\mathcal{C}}}
\newcommand{\calD}{{\mathcal{D}}}
\newcommand{\calE}{{\mathcal{E}}}
\newcommand{\calF}{{\mathcal{F}}}
\newcommand{\calG}{{\mathcal{G}}}
\newcommand{\calH}{{\mathcal{H}}}
\newcommand{\calI}{{\mathcal{I}}}
\newcommand{\calJ}{{\mathcal{J}}}
\newcommand{\calK}{{\mathcal{K}}}
\newcommand{\calL}{{\mathcal{L}}}
\newcommand{\calM}{{\mathcal{M}}}
\newcommand{\calN}{{\mathcal{N}}}
\newcommand{\calO}{{\mathcal{O}}}
\newcommand{\calP}{{\mathcal{P}}}
\newcommand{\calQ}{{\mathcal{Q}}}
\newcommand{\calR}{{\mathcal{R}}}
\newcommand{\calS}{{\mathcal{S}}}
\newcommand{\calT}{{\mathcal{T}}}
\newcommand{\calU}{{\mathcal{U}}}
\newcommand{\calV}{{\mathcal{V}}}
\newcommand{\calW}{{\mathcal{W}}}
\newcommand{\calX}{{\mathcal{X}}}
\newcommand{\calY}{{\mathcal{Y}}}
\newcommand{\calZ}{{\mathcal{Z}}}

\newcommand{\bara}{{\bar{a}}}
\newcommand{\barb}{{\bar{b}}}
\newcommand{\barc}{{\bar{c}}}
\newcommand{\bard}{{\bar{d}}}
\newcommand{\bare}{{\bar{e}}}
\newcommand{\barf}{{\bar{f}}}
\newcommand{\barg}{{\bar{g}}}
\newcommand{\barh}{{\bar{h}}}
\newcommand{\bari}{{\bar{i}}}
\newcommand{\barj}{{\bar{j}}}
\newcommand{\bark}{{\bar{k}}}
\newcommand{\barl}{{\bar{l}}}
\newcommand{\barm}{{\bar{m}}}
\newcommand{\barn}{{\bar{n}}}
\newcommand{\baro}{{\bar{o}}}
\newcommand{\barp}{{\bar{p}}}
\newcommand{\barq}{{\bar{q}}}
\newcommand{\barr}{{\bar{r}}}
\newcommand{\bars}{{\bar{s}}}
\newcommand{\bart}{{\bar{t}}}
\newcommand{\baru}{{\bar{u}}}
\newcommand{\barv}{{\bar{v}}}
\newcommand{\barw}{{\bar{w}}}
\newcommand{\barx}{{\bar{x}}}
\newcommand{\bary}{{\bar{y}}}
\newcommand{\barz}{{\bar{z}}}
\newcommand{\barA}{{\bar{A}}}
\newcommand{\barB}{{\bar{B}}}
\newcommand{\barC}{{\bar{C}}}
\newcommand{\barD}{{\bar{D}}}
\newcommand{\barE}{{\bar{E}}}
\newcommand{\barF}{{\bar{F}}}
\newcommand{\barG}{{\bar{G}}}
\newcommand{\barH}{{\bar{H}}}
\newcommand{\barI}{{\bar{I}}}
\newcommand{\barJ}{{\bar{J}}}
\newcommand{\barK}{{\bar{K}}}
\newcommand{\barL}{{\bar{L}}}
\newcommand{\barM}{{\bar{M}}}
\newcommand{\barN}{{\bar{N}}}
\newcommand{\barO}{{\bar{O}}}
\newcommand{\barP}{{\bar{P}}}
\newcommand{\barQ}{{\bar{Q}}}
\newcommand{\barR}{{\bar{R}}}
\newcommand{\barS}{{\bar{S}}}
\newcommand{\barT}{{\bar{T}}}
\newcommand{\barU}{{\bar{U}}}
\newcommand{\barV}{{\bar{V}}}
\newcommand{\barW}{{\bar{W}}}
\newcommand{\barX}{{\bar{X}}}
\newcommand{\barY}{{\bar{Y}}}
\newcommand{\barZ}{{\bar{Z}}}

\newcommand{\hX}{\hat{X}}
\newcommand{\Ent}{\mathsf{Ent}}
\newcommand{\awarm}{{A_{\text{warm}}}}
\newcommand{\thetaLS}{{\widehat{\theta}^{\text{\rm LS}}}}

\newcommand{\jiao}[1]{\langle{#1}\rangle}
\newcommand{\gaht}{\textsc{GoodActionHypTest}\;}
\newcommand{\iaht}{\textsc{InitialActionHypTest}\;}
\newcommand{\true}{\textsf{True}\;}
\newcommand{\false}{\textsf{False}\;}

% \usepackage[capitalize,noabbrev]{cleveref}
% \crefname{lemma}{Lemma}{Lemmas}
% \Crefname{lemma}{Lemma}{Lemmas}
% \crefname{thm}{Theorem}{Theorems}
% \Crefname{thm}{Theorem}{Theorems}
% \Crefname{assumption}{Assumption}{Assumptions}
% \Crefname{inftheorem}{Informal Theorem}{Informal Theorems}
% \crefformat{equation}{(#2#1#3)}

% % if you use cleveref..
% \usepackage[capitalize,noabbrev]{cleveref}
% \crefname{lemma}{Lemma}{Lemmas}
% \crefname{proposition}{Proposition}{Propositions}
% \crefname{remark}{Remark}{Remarks}
% \crefname{corollary}{Corollary}{Corollaries}
% \crefname{definition}{Definition}{Definitions}
% \crefname{conjecture}{Conjecture}{Conjectures}
% \crefname{figure}{Fig.}{Figures}

%hi
\begin{abstract}
Code generation, symbolic math reasoning, and other tasks require LLMs to produce outputs that are both syntactically and semantically correct. Constrained LLM generation is a promising direction to enforce adherence to formal grammar, but prior works have empirically observed that strict enforcement of formal constraints often diminishes the reasoning capabilities of LLMs. In this work, we first provide a theoretical explanation for why constraining LLM outputs to very restrictive grammars that only allow syntactically valid final answers reduces the reasoning capabilities of the model. Second, we demonstrate that by augmenting the output grammar with carefully designed additional rules, it is always possible to preserve the reasoning capabilities of the LLM while ensuring syntactic and semantic correctness in its outputs. Building on these theoretical insights, we propose a reasoning-augmented constrained decoding algorithm, \Tool, which effectively balances the correctness of constrained generation with the flexibility of unconstrained generation. 
Experiments on multiple open-source LLMs and benchmarks show that \Tool{} significantly outperforms both state-of-the-art
constrained decoding strategies and standard unconstrained decoding, showing up to \upto{} points accuracy improvement over baselines on challenging symbolic reasoning benchmarks GSM-symbolic and FOLIO.

\end{abstract}

\section{Introduction}
\label{sec:introduction}
The business processes of organizations are experiencing ever-increasing complexity due to the large amount of data, high number of users, and high-tech devices involved \cite{martin2021pmopportunitieschallenges, beerepoot2023biggestbpmproblems}. This complexity may cause business processes to deviate from normal control flow due to unforeseen and disruptive anomalies \cite{adams2023proceddsriftdetection}. These control-flow anomalies manifest as unknown, skipped, and wrongly-ordered activities in the traces of event logs monitored from the execution of business processes \cite{ko2023adsystematicreview}. For the sake of clarity, let us consider an illustrative example of such anomalies. Figure \ref{FP_ANOMALIES} shows a so-called event log footprint, which captures the control flow relations of four activities of a hypothetical event log. In particular, this footprint captures the control-flow relations between activities \texttt{a}, \texttt{b}, \texttt{c} and \texttt{d}. These are the causal ($\rightarrow$) relation, concurrent ($\parallel$) relation, and other ($\#$) relations such as exclusivity or non-local dependency \cite{aalst2022pmhandbook}. In addition, on the right are six traces, of which five exhibit skipped, wrongly-ordered and unknown control-flow anomalies. For example, $\langle$\texttt{a b d}$\rangle$ has a skipped activity, which is \texttt{c}. Because of this skipped activity, the control-flow relation \texttt{b}$\,\#\,$\texttt{d} is violated, since \texttt{d} directly follows \texttt{b} in the anomalous trace.
\begin{figure}[!t]
\centering
\includegraphics[width=0.9\columnwidth]{images/FP_ANOMALIES.png}
\caption{An example event log footprint with six traces, of which five exhibit control-flow anomalies.}
\label{FP_ANOMALIES}
\end{figure}

\subsection{Control-flow anomaly detection}
Control-flow anomaly detection techniques aim to characterize the normal control flow from event logs and verify whether these deviations occur in new event logs \cite{ko2023adsystematicreview}. To develop control-flow anomaly detection techniques, \revision{process mining} has seen widespread adoption owing to process discovery and \revision{conformance checking}. On the one hand, process discovery is a set of algorithms that encode control-flow relations as a set of model elements and constraints according to a given modeling formalism \cite{aalst2022pmhandbook}; hereafter, we refer to the Petri net, a widespread modeling formalism. On the other hand, \revision{conformance checking} is an explainable set of algorithms that allows linking any deviations with the reference Petri net and providing the fitness measure, namely a measure of how much the Petri net fits the new event log \cite{aalst2022pmhandbook}. Many control-flow anomaly detection techniques based on \revision{conformance checking} (hereafter, \revision{conformance checking}-based techniques) use the fitness measure to determine whether an event log is anomalous \cite{bezerra2009pmad, bezerra2013adlogspais, myers2018icsadpm, pecchia2020applicationfailuresanalysispm}. 

The scientific literature also includes many \revision{conformance checking}-independent techniques for control-flow anomaly detection that combine specific types of trace encodings with machine/deep learning \cite{ko2023adsystematicreview, tavares2023pmtraceencoding}. Whereas these techniques are very effective, their explainability is challenging due to both the type of trace encoding employed and the machine/deep learning model used \cite{rawal2022trustworthyaiadvances,li2023explainablead}. Hence, in the following, we focus on the shortcomings of \revision{conformance checking}-based techniques to investigate whether it is possible to support the development of competitive control-flow anomaly detection techniques while maintaining the explainable nature of \revision{conformance checking}.
\begin{figure}[!t]
\centering
\includegraphics[width=\columnwidth]{images/HIGH_LEVEL_VIEW.png}
\caption{A high-level view of the proposed framework for combining \revision{process mining}-based feature extraction with dimensionality reduction for control-flow anomaly detection.}
\label{HIGH_LEVEL_VIEW}
\end{figure}

\subsection{Shortcomings of \revision{conformance checking}-based techniques}
Unfortunately, the detection effectiveness of \revision{conformance checking}-based techniques is affected by noisy data and low-quality Petri nets, which may be due to human errors in the modeling process or representational bias of process discovery algorithms \cite{bezerra2013adlogspais, pecchia2020applicationfailuresanalysispm, aalst2016pm}. Specifically, on the one hand, noisy data may introduce infrequent and deceptive control-flow relations that may result in inconsistent fitness measures, whereas, on the other hand, checking event logs against a low-quality Petri net could lead to an unreliable distribution of fitness measures. Nonetheless, such Petri nets can still be used as references to obtain insightful information for \revision{process mining}-based feature extraction, supporting the development of competitive and explainable \revision{conformance checking}-based techniques for control-flow anomaly detection despite the problems above. For example, a few works outline that token-based \revision{conformance checking} can be used for \revision{process mining}-based feature extraction to build tabular data and develop effective \revision{conformance checking}-based techniques for control-flow anomaly detection \cite{singh2022lapmsh, debenedictis2023dtadiiot}. However, to the best of our knowledge, the scientific literature lacks a structured proposal for \revision{process mining}-based feature extraction using the state-of-the-art \revision{conformance checking} variant, namely alignment-based \revision{conformance checking}.

\subsection{Contributions}
We propose a novel \revision{process mining}-based feature extraction approach with alignment-based \revision{conformance checking}. This variant aligns the deviating control flow with a reference Petri net; the resulting alignment can be inspected to extract additional statistics such as the number of times a given activity caused mismatches \cite{aalst2022pmhandbook}. We integrate this approach into a flexible and explainable framework for developing techniques for control-flow anomaly detection. The framework combines \revision{process mining}-based feature extraction and dimensionality reduction to handle high-dimensional feature sets, achieve detection effectiveness, and support explainability. Notably, in addition to our proposed \revision{process mining}-based feature extraction approach, the framework allows employing other approaches, enabling a fair comparison of multiple \revision{conformance checking}-based and \revision{conformance checking}-independent techniques for control-flow anomaly detection. Figure \ref{HIGH_LEVEL_VIEW} shows a high-level view of the framework. Business processes are monitored, and event logs obtained from the database of information systems. Subsequently, \revision{process mining}-based feature extraction is applied to these event logs and tabular data input to dimensionality reduction to identify control-flow anomalies. We apply several \revision{conformance checking}-based and \revision{conformance checking}-independent framework techniques to publicly available datasets, simulated data of a case study from railways, and real-world data of a case study from healthcare. We show that the framework techniques implementing our approach outperform the baseline \revision{conformance checking}-based techniques while maintaining the explainable nature of \revision{conformance checking}.

In summary, the contributions of this paper are as follows.
\begin{itemize}
    \item{
        A novel \revision{process mining}-based feature extraction approach to support the development of competitive and explainable \revision{conformance checking}-based techniques for control-flow anomaly detection.
    }
    \item{
        A flexible and explainable framework for developing techniques for control-flow anomaly detection using \revision{process mining}-based feature extraction and dimensionality reduction.
    }
    \item{
        Application to synthetic and real-world datasets of several \revision{conformance checking}-based and \revision{conformance checking}-independent framework techniques, evaluating their detection effectiveness and explainability.
    }
\end{itemize}

The rest of the paper is organized as follows.
\begin{itemize}
    \item Section \ref{sec:related_work} reviews the existing techniques for control-flow anomaly detection, categorizing them into \revision{conformance checking}-based and \revision{conformance checking}-independent techniques.
    \item Section \ref{sec:abccfe} provides the preliminaries of \revision{process mining} to establish the notation used throughout the paper, and delves into the details of the proposed \revision{process mining}-based feature extraction approach with alignment-based \revision{conformance checking}.
    \item Section \ref{sec:framework} describes the framework for developing \revision{conformance checking}-based and \revision{conformance checking}-independent techniques for control-flow anomaly detection that combine \revision{process mining}-based feature extraction and dimensionality reduction.
    \item Section \ref{sec:evaluation} presents the experiments conducted with multiple framework and baseline techniques using data from publicly available datasets and case studies.
    \item Section \ref{sec:conclusions} draws the conclusions and presents future work.
\end{itemize}

\section{Preliminaries}
\label{sec:prelim}
\label{sec:term}
We define the key terminologies used, primarily focusing on the hidden states (or activations) during the forward pass. 

\paragraph{Components in an attention layer.} We denote $\Res$ as the residual stream. We denote $\Val$ as Value (states), $\Qry$ as Query (states), and $\Key$ as Key (states) in one attention head. The \attlogit~represents the value before the softmax operation and can be understood as the inner product between  $\Qry$  and  $\Key$. We use \Attn~to denote the attention weights of applying the SoftMax function to \attlogit, and ``attention map'' to describe the visualization of the heat map of the attention weights. When referring to the \attlogit~from ``$\tokenB$'' to  ``$\tokenA$'', we indicate the inner product  $\langle\Qry(\tokenB), \Key(\tokenA)\rangle$, specifically the entry in the ``$\tokenB$'' row and ``$\tokenA$'' column of the attention map.

\paragraph{Logit lens.} We use the method of ``Logit Lens'' to interpret the hidden states and value states \citep{belrose2023eliciting}. We use \logit~to denote pre-SoftMax values of the next-token prediction for LLMs. Denote \readout~as the linear operator after the last layer of transformers that maps the hidden states to the \logit. 
The logit lens is defined as applying the readout matrix to residual or value states in middle layers. Through the logit lens, the transformed hidden states can be interpreted as their direct effect on the logits for next-token prediction. 

\paragraph{Terminologies in two-hop reasoning.} We refer to an input like “\Src$\to$\brga, \brgb$\to$\Ed” as a two-hop reasoning chain, or simply a chain. The source entity $\Src$ serves as the starting point or origin of the reasoning. The end entity $\Ed$ represents the endpoint or destination of the reasoning chain. The bridge entity $\Brg$ connects the source and end entities within the reasoning chain. We distinguish between two occurrences of $\Brg$: the bridge in the first premise is called $\brga$, while the bridge in the second premise that connects to $\Ed$ is called $\brgc$. Additionally, for any premise ``$\tokenA \to \tokenB$'', we define $\tokenA$ as the parent node and $\tokenB$ as the child node. Furthermore, if at the end of the sequence, the query token is ``$\tokenA$'', we define the chain ``$\tokenA \to \tokenB$, $\tokenB \to \tokenC$'' as the Target Chain, while all other chains present in the context are referred to as distraction chains. Figure~\ref{fig:data_illustration} provides an illustration of the terminologies.

\paragraph{Input format.}
Motivated by two-hop reasoning in real contexts, we consider input in the format $\bos, \text{context information}, \query, \answer$. A transformer model is trained to predict the correct $\answer$ given the query $\query$ and the context information. The context compromises of $K=5$ disjoint two-hop chains, each appearing once and containing two premises. Within the same chain, the relative order of two premises is fixed so that \Src$\to$\brga~always precedes \brgb$\to$\Ed. The orders of chains are randomly generated, and chains may interleave with each other. The labels for the entities are re-shuffled for every sequence, choosing from a vocabulary size $V=30$. Given the $\bos$ token, $K=5$ two-hop chains, \query, and the \answer~tokens, the total context length is $N=23$. Figure~\ref{fig:data_illustration} also illustrates the data format. 

\paragraph{Model structure and training.} We pre-train a three-layer transformer with a single head per layer. Unless otherwise specified, the model is trained using Adam for $10,000$ steps, achieving near-optimal prediction accuracy. Details are relegated to Appendix~\ref{app:sec_add_training_detail}.


% \RZ{Do we use source entity, target entity, and mediator entity? Or do we use original token, bridge token, end token?}





% \paragraph{Basic notations.} We use ... We use $\ve_i$ to denote one-hot vectors of which only the $i$-th entry equals one, and all other entries are zero. The dimension of $\ve_i$ are usually omitted and can be inferred from contexts. We use $\indicator\{\cdot\}$ to denote the indicator function.

% Let $V > 0$ be a fixed positive integer, and let $\vocab = [V] \defeq \{1, 2, \ldots, V\}$ be the vocabulary. A token $v \in \vocab$ is an integer in $[V]$ and the input studied in this paper is a sequence of tokens $s_{1:T} \defeq (s_1, s_2, \ldots, s_T) \in \vocab^T$ of length $T$. For any set $\mathcal{S}$, we use $\Delta(\mathcal{S})$ to denote the set of distributions over $\mathcal{S}$.

% % to a sequence of vectors $z_1, z_2, \ldots, z_T \in \real^{\dout}$ of dimension $\dout$ and length $T$.

% Let $\mU = [\vu_1, \vu_2, \ldots, \vu_V]^\transpose \in \real^{V\times d}$ denote the token embedding matrix, where the $i$-th row $\vu_i \in \real^d$ represents the $d$-dimensional embedding of token $i \in [V]$. Similarly, let $\mP = [\vp_1, \vp_2, \ldots, \vp_T]^\transpose \in \real^{T\times d}$ denote the positional embedding matrix, where the $i$-th row $\vp_i \in \real^d$ represents the $d$-dimensional embedding of position $i \in [T]$. Both $\mU$ and $\mP$ can be fixed or learnable.

% After receiving an input sequence of tokens $s_{1:T}$, a transformer will first process it using embedding matrices $\mU$ and $\mP$ to obtain a sequence of vectors $\mH = [\vh_1, \vh_2, \ldots, \vh_T] \in \real^{d\times T}$, where 
% \[
% \vh_i = \mU^\transpose\ve_{s_i} + \mP^\transpose\ve_{i} = \vu_{s_i} + \vp_i.
% \]

% We make the following definitions of basic operations in a transformer.

% \begin{definition}[Basic operations in transformers] 
% \label{defn:operators}
% Define the softmax function $\softmax(\cdot): \real^d \to \real^d$ over a vector $\vv \in \real^d$ as
% \[\softmax(\vv)_i = \frac{\exp(\vv_i)}{\sum_{j=1}^d \exp(\vv_j)} \]
% and define the softmax function $\softmax(\cdot): \real^{m\times n} \to \real^{m \times n}$ over a matrix $\mV \in \real^{m\times n}$ as a column-wise softmax operator. For a squared matrix $\mM \in \real^{m\times m}$, the causal mask operator $\mask(\cdot): \real^{m\times m} \to \real^{m\times m}$  is defined as $\mask(\mM)_{ij} = \mM_{ij}$ if $i \leq j$ and  $\mask(\mM)_{ij} = -\infty$ otherwise. For a vector $\vv \in \real^n$ where $n$ is the number of hidden neurons in a layer, we use $\layernorm(\cdot): \real^n \to \real^n$ to denote the layer normalization operator where
% \[
% \layernorm(\vv)_i = \frac{\vv_i-\mu}{\sigma}, \mu = \frac{1}{n}\sum_{j=1}^n \vv_j, \sigma = \sqrt{\frac{1}{n}\sum_{j=1}^n (\vv_j-\mu)^2}
% \]
% and use $\layernorm(\cdot): \real^{n\times m} \to \real^{n\times m}$ to denote the column-wise layer normalization on a matrix.
% We also use $\nonlin(\cdot)$ to denote element-wise nonlinearity such as $\relu(\cdot)$.
% \end{definition}

% The main components of a transformer are causal self-attention heads and MLP layers, which are defined as follows.

% \begin{definition}[Attentions and MLPs]
% \label{defn:attn_mlp} 
% A single-head causal self-attention $\attn(\mH;\mQ,\mK,\mV,\mO)$ parameterized by $\mQ,\mK,\mV \in \real^{{\dqkv\times \din}}$ and $\mO \in \real^{\dout\times\dqkv}$ maps an input matrix $\mH \in \real^{\din\times T}$ to
% \begin{align*}
% &\attn(\mH;\mQ,\mK,\mV,\mO) \\
% =&\mO\mV\layernorm(\mH)\softmax(\mask(\layernorm(\mH)^\transpose\mK^\transpose\mQ\layernorm(\mH))).
% \end{align*}
% Furthermore, a multi-head attention with $M$ heads parameterized by $\{(\mQ_m,\mK_m,\mV_m,\mO_m) \}_{m=1}^M$ is defined as 
% \begin{align*}
%     &\Attn(\mH; \{(\mQ_m,\mK_m,\mV_m,\mO_m) \}_{m\in[M]}) \\ =& \sum_{m=1}^M \attn(\mH;\mQ_m,\mK_m,\mV_m,\mO_m) \in \real^{\dout \times T}.
% \end{align*}
% An MLP layer $\mlp(\mH;\mW_1,\mW_2)$ parameterized by $\mW_1 \in \real^{\dhidden\times \din}$ and $\mW_2 \in \real^{\dout \times \dhidden}$ maps an input matrix $\mH = [\vh_1, \ldots, \vh_T] \in \real^{\din \times T}$ to
% \begin{align*}
%     &\mlp(\mH;\mW_1,\mW_2) = [\vy_1, \ldots, \vy_T], \\ \text{where } &\vy_i = \mW_2\nonlin(\mW_1\layernorm(\vh_i)), \forall i \in [T].
% \end{align*}

% \end{definition}

% In this paper, we assume $\din=\dout=d$ for all attention heads and MLPs to facilitate residual stream unless otherwise specified. Given \Cref{defn:operators,defn:attn_mlp}, we are now able to define a multi-layer transformer.

% \begin{definition}[Multi-layer transformers]
% \label{defn:transformer}
%     An $L$-layer transformer $\transformer(\cdot): \vocab^T \to \Delta(\vocab)$ parameterized by $\mP$, $\mU$, $\{(\mQ_m^{(l)},\mK_m^{(l)},\mV_m^{(l)},\mO_m^{(l)})\}_{m\in[M],l\in[L]}$,  $\{(\mW_1^{(l)},\mW_2^{(l)})\}_{l\in[L]}$ and $\Wreadout \in \real^{V \times d}$ receives a sequence of tokens $s_{1:T}$ as input and predict the next token by outputting a distribution over the vocabulary. The input is first mapped to embeddings $\mH = [\vh_1, \vh_2, \ldots, \vh_T] \in \real^{d\times T}$ by embedding matrices $\mP, \mU$ where 
%     \[
%     \vh_i = \mU^\transpose\ve_{s_i} + \mP^\transpose\ve_{i}, \forall i \in [T].
%     \]
%     For each layer $l \in [L]$, the output of layer $l$, $\mH^{(l)} \in \real^{d\times T}$, is obtained by 
%     \begin{align*}
%         &\mH^{(l)} =  \mH^{(l-1/2)} + \mlp(\mH^{(l-1/2)};\mW_1^{(l)},\mW_2^{(l)}), \\
%         & \mH^{(l-1/2)} = \mH^{(l-1)} + \\ & \quad \Attn(\mH^{(l-1)}; \{(\mQ_m^{(l)},\mK_m^{(l)},\mV_m^{(l)},\mO_m^{(l)}) \}_{m\in[M]}), 
%     \end{align*}
%     where the input $\mH^{(l-1)}$ is the output of the previous layer $l-1$ for $l > 1$ and the input of the first layer $\mH^{(0)} = \mH$. Finally, the output of the transformer is obtained by 
%     \begin{align*}
%         \transformer(s_{1:T}) = \softmax(\Wreadout\vh_T^{(L)})
%     \end{align*}
%     which is a $V$-dimensional vector after softmax representing a distribution over $\vocab$, and $\vh_T^{(L)}$ is the $T$-th column of the output of the last layer, $\mH^{(L)}$.
% \end{definition}



% For each token $v \in \vocab$, there is a corresponding $d_t$-dimensional token embedding vector $\embed(v) \in \mathbb{R}^{d_t}$. Assume the maximum length of the sequence studied in this paper does not exceed $T$. For each position $t \in [T]$, there is a corresponding positional embedding  








\section{Proposed Method}
\textbf{Problem Statement: } Let $\mathcal{X}$ and $\mathcal{Y}$ denote the input and output spaces, respectively, and $D = \{(x_i, y_i)\}_{i=1}^n$ the dataset, where $x_i \in \mathcal{X}$ and $y_i \in \mathcal{Y}$ are the $i^{th}$ question-answer pair. For each $x_i$, the goal is to generate a response $\hat{y}_i$ that maximizes the overall accuracy. The goal is to achieve this using a pool  of $N$ pretrained foundational LLMs without additional training or fine-tuning.


\begin{comment}
    

Let $\mathcal{X}$ and $\mathcal{Y}$ denote input and output spaces respectively and  $\mathcal{D}= \{(x_i,y_i)\}_{i=1}^n$ denote Dataset, where $x_i \in \mathcal{X}$ and $y_i \in \mathcal{Y}$ is the $i^{th}$ question-answer(QA) pair. For each $x_i$ we want to generate a response $\hat{y}_i$  such that overall accuracy denoted by $\frac{1}{n}\sum_{i=1}^n\mathbf{I}(y_i=\hat{y}_i)$ is maximized.  We want to maximize this using ensemble of $N$ LLMs without any additional training or fine-tuning.
\end{comment}
\renewcommand{\algorithmicrequire}{\textbf{Input:}}
\renewcommand{\algorithmicensure}{\textbf{Output:}}

\begin{algorithm}[t]
    \caption{Components of UAF - SELECTOR and FUSER}
    \label{alg:selector}
    
    \begin{algorithmic}
    \REQUIRE $D_{val}$, Pool of LLMs $\mathcal{M}= \{M^j\}_{j=1}^{N}$, Uncertainty function $U_f(.,.,.)$, Ensemble size $K$, Test data point $x_{test}$
    \ENSURE Test data response $\hat{y}_{test}$
    \newline

    \STATE \textbf{procedure }SELECTOR ($D_{val}, \mathcal{M}, U_f, K$)
    \FOR{$M^j \in \mathcal{M}$}
    %\STATE $L_j, U_j = \emptyset,\emptyset$
    \FOR{$(x_i, y_i) \in D_{val}$}
    \STATE $\hat{y}_i^j = M^j(x_i)$   \quad;\quad  $s_i^j = \mathbf{1}(\hat{y}_i^j == y_i)$
    \STATE $u_i^j = U_f(M^j, x_i, \hat{y}_i^j)$
    %\STATE $L_j \leftarrow L_j \cup IsCorrect_i^j$   \quad;\quad    $U_j \leftarrow U_j \cup u_{val_i}^j$
    \ENDFOR

    \STATE $Acc_j = \frac{1}{|D_{val}|}\sum_i s_i^j$  ;\quad $SAH_j = ROC\_AUC\_score(\{s_i^j,u_i^j\}_i)$
    \STATE $Cscore_j = Acc_j \times SAH_j$
    \ENDFOR
    \STATE \textbf{return:} $\mathcal{M}^{sel} =  TopK (\{Cscore_j\}_{j=1}^N)$,  \hfill\COMMENT{//Top $K$ LLMs}
    \STATE \quad  \quad      $Acc^{sel} = \{Acc_j| j \in \mathcal{M}^{sel}\}$ \hfill\COMMENT{//Accuracy of selected $K$ LLMs}
    %\STATE $j_1, j_2, \dots, j_K = \arg\max_{j \in \{1, 2, \dots, N\}} Cscore_j$ 
    %\ENSURE $j_1, j_2, \dots, j_K$    \hfill\COMMENT{//Indices of Top K llms}
    \newline
\STATE \textbf{procedure } FUSER ($x_{test}, \mathcal{M}^{sel}, Acc^{sel}, U_f, K$)
\FOR{$M^k \in \mathcal{M}^{sel}$}
\STATE $\hat{y}_{test}^k = M^k(x_{test})$
\STATE $u_{test}^k = U_f(M^k, x_{test}, \hat{y}_{test}^k)$
\ENDFOR
\STATE $\hat{y}_{test} = \hat{y}_{test}^{k^*} \text{, where } k^* = argmax_{k \in \{1,\dots K\}} Acc_k \times (1-u_{test}^k)$
\STATE \textbf{return:} $\hat{y}_{test}$

\end{algorithmic}
\end{algorithm}

\subsection{Uncertainty Aware Fusion (UAF)}\label{sec:uaf}
Figure \ref{fig:fuser} provides an overview of our UAF framework. At a high level, UAF consists of two modules: SELECTOR and FUSER. Given a specific task, the SELECTOR selects the top $K$ LLMs from a pool of $N$ LLMs based on performance metric. FUSER then combines the outputs of these $K$ LLMs to produce the final response.

\subsubsection{SELECTOR}
Given a pool of $N$ LLMs denoted by $\mathcal{M}$, SELECTOR selects $K$ LLMs (where $K<N$) to optimize computational efficiency and enhance overall factual accuracy by pruning underperforming LLMs. Selection is based on two criteria: (1) task-specific accuracy and (2) self-assessment of hallucinations based on an given uncertainty measure. Given a specified uncertainty measure $U_f(\cdot)$, and  a validation set $D_{val}$, we prompt each LLM with input $x_i$,   obtaining response  $\hat{y}_i^j$  and corresponding uncertainty score $u_i^j$ from the $j^{th}$ LLM $M^j$.  We compute the accuracy $Acc_j$ of $M^j$ as the fraction of correct responses. We also measure the LLMs ability for  self-assessment of hallucinations $SAH_j$ as the area under the ROC curve for the binary classification of truthful vs. hallucinatory responses using uncertainty scores. We then compute a combined score $Cscore_j = Acc_j \times SAH_j$ for each LLM. The top K models with the highest combined scores are selected greedily, where K is a   hyperparameter tuned for specific tasks.

\begin{comment}
To achieve this we sample $10\%$ of examples from $D$ and denote it as  $D_{val}$. We use the rest denoted by $D_{test}$ for evaluation. For each $(xval_i,yval_i)\in D_{val}$ we prompt each of the $N$ LLMs with $xval_i$. Let $\hat{yval}_i^j$ denote the response and $u_i^j$ its corresponding uncertainty score computed with a particular uncertainty method for $j^{th}$ LLM. Accuracy of $j^{th}$ LLM, denoted by $Acc_j$ is defined as percentage of correct responses. We also measure area under the receiver operator characteristic curve  $Unc\_auroc_j$ of $j^{th}$ LLM, by measuring the performance of classifying its own correct(truthful) from incorrect(hallucinatory) responses by varying the thresholds on the set of uncertainty scores $\{u_i^j\}_{i=1}^{nval}$. For each $j^{th}$ LLM we compute a combined score $Cscore_j = Acc_j*Unc\_auroc_j$. We use greedy method to select $K$ LLMs based on top $K$ highest $Cscore$. Here $K$ is the hyperparameter which we tune using $D_{val}$. Algorithm \ref{alg:selector} presents the pseudo-code for this method.
\end{comment}

\subsubsection{FUSER}
Given the selected ensemble of $K$ models $\mathcal{M}^{sel}$, 
%with respective accuracies $\{Acc_1, \dots, Acc_K \}$,
for each unseen example $x_{test}$, we generate outputs from the $K$ LLMs denoted by $\{\hat{y}_{test}^1,\dots,\hat{y}_{test}^K\}$ along with the  corresponding instance-specific uncertainty scores.  denoted by $\{u_{test}^1,\dots,u_{test}^K\}$. 
While there can be several fusion strategies, since we are dealing with natural language responses, the simplest one is to example-specific selection from the candidate outputs, i.e., 
\[
\hat{y}_{test} = \hat{y}_{test}^{k^*}, \quad \text{where} \quad {k^*} = \arg\max_{k \in \{1, \dots, K\}} f^k.
\]
Selection criterion $f^k$ could be based on validation set accuracy alone, inverse uncertainty or some combination of both such as $\text{Acc}_k \cdot (1 - u_{test}^k)$  or $\frac{\text{Acc}_k}{u_{test}^k}$.
The first strategy essentially reduces the ensemble to a single most accurate model, while the second one elevates the most confident one. However, both of these approaches are sub-optimal compared to combined criteria, specifically 
\[
f^k = \text{Acc}_k \cdot (1 - u_{test}^k),
\]
which yields the best performance. Experiments with  other combined selection criteria shows similar behavior to the aforementioned one and hence,  we omit the results for brevity.


%There are multiple ways to choose the final answer from these candidate responses - $\hat{y}_{test} = \hat{y}_{test}^j where j = argmax_{k \in \{1,\dots K\}}f^k$. We can define $f^k$ as $Acc_k$ , $1-u_{test}^k$ ,$Acc_k*(1-u_{test}^k)$ or $Acc_k/u_{test}^k$. First strategy essentially reduces the ensemble to a single model having highest accuracy, while second one picks the final answer as the one with the least uncertainty. Both of these are suboptimal compared to  $f_k = Acc_k*(1-u_{test}^k)$ with which we experiment here in this work. Alternative strategy $f_k = Acc_k/u_{test}^k$ shows similar behaviour as above and we omit it's analysis for brevity.

%Although there are multiple ways to aggregate these candidate responses we propose a simple aggregation technique where we choose the final answer as the one with the least uncertainty ie. $\hat{y}_{test} = \hat{y}_{test}^j where j = argmin_{k \in \{1,\dots K\}}u_{test}^k$. Algorithm \ref{alg:selector} presents the pseudo-code of UAF components.



\section*{Limitations and Ethical Considerations}

\noindent\textbf{Limitations.} The primary limitation of our work is that it extends only the dataset provided by MUSE and employs DeepSeek-v3 for question generation. 
To mitigate this generalization risk, we have released our code and the generated audit suite, allowing researchers to utilize our framework to create additional audit datasets and evaluate their quality. Meanwhile, this is also our future work to extend our framework to other benchmarks.

\noindent\textbf{Ethical Considerations.} Machine unlearning can be employed to mitigate risks associated with LLMs in terms of privacy, security, bias, and copyright. Our work is dedicated to providing a comprehensive evaluation framework to help researchers better understand the unlearning effectiveness of LLMs, which we believe will have a positive impact on society.
\section{Experiments}
\label{sec:experiments}
The experiments are designed to address two key research questions.
First, \textbf{RQ1} evaluates whether the average $L_2$-norm of the counterfactual perturbation vectors ($\overline{||\perturb||}$) decreases as the model overfits the data, thereby providing further empirical validation for our hypothesis.
Second, \textbf{RQ2} evaluates the ability of the proposed counterfactual regularized loss, as defined in (\ref{eq:regularized_loss2}), to mitigate overfitting when compared to existing regularization techniques.

% The experiments are designed to address three key research questions. First, \textbf{RQ1} investigates whether the mean perturbation vector norm decreases as the model overfits the data, aiming to further validate our intuition. Second, \textbf{RQ2} explores whether the mean perturbation vector norm can be effectively leveraged as a regularization term during training, offering insights into its potential role in mitigating overfitting. Finally, \textbf{RQ3} examines whether our counterfactual regularizer enables the model to achieve superior performance compared to existing regularization methods, thus highlighting its practical advantage.

\subsection{Experimental Setup}
\textbf{\textit{Datasets, Models, and Tasks.}}
The experiments are conducted on three datasets: \textit{Water Potability}~\cite{kadiwal2020waterpotability}, \textit{Phomene}~\cite{phomene}, and \textit{CIFAR-10}~\cite{krizhevsky2009learning}. For \textit{Water Potability} and \textit{Phomene}, we randomly select $80\%$ of the samples for the training set, and the remaining $20\%$ for the test set, \textit{CIFAR-10} comes already split. Furthermore, we consider the following models: Logistic Regression, Multi-Layer Perceptron (MLP) with 100 and 30 neurons on each hidden layer, and PreactResNet-18~\cite{he2016cvecvv} as a Convolutional Neural Network (CNN) architecture.
We focus on binary classification tasks and leave the extension to multiclass scenarios for future work. However, for datasets that are inherently multiclass, we transform the problem into a binary classification task by selecting two classes, aligning with our assumption.

\smallskip
\noindent\textbf{\textit{Evaluation Measures.}} To characterize the degree of overfitting, we use the test loss, as it serves as a reliable indicator of the model's generalization capability to unseen data. Additionally, we evaluate the predictive performance of each model using the test accuracy.

\smallskip
\noindent\textbf{\textit{Baselines.}} We compare CF-Reg with the following regularization techniques: L1 (``Lasso''), L2 (``Ridge''), and Dropout.

\smallskip
\noindent\textbf{\textit{Configurations.}}
For each model, we adopt specific configurations as follows.
\begin{itemize}
\item \textit{Logistic Regression:} To induce overfitting in the model, we artificially increase the dimensionality of the data beyond the number of training samples by applying a polynomial feature expansion. This approach ensures that the model has enough capacity to overfit the training data, allowing us to analyze the impact of our counterfactual regularizer. The degree of the polynomial is chosen as the smallest degree that makes the number of features greater than the number of data.
\item \textit{Neural Networks (MLP and CNN):} To take advantage of the closed-form solution for computing the optimal perturbation vector as defined in (\ref{eq:opt-delta}), we use a local linear approximation of the neural network models. Hence, given an instance $\inst_i$, we consider the (optimal) counterfactual not with respect to $\model$ but with respect to:
\begin{equation}
\label{eq:taylor}
    \model^{lin}(\inst) = \model(\inst_i) + \nabla_{\inst}\model(\inst_i)(\inst - \inst_i),
\end{equation}
where $\model^{lin}$ represents the first-order Taylor approximation of $\model$ at $\inst_i$.
Note that this step is unnecessary for Logistic Regression, as it is inherently a linear model.
\end{itemize}

\smallskip
\noindent \textbf{\textit{Implementation Details.}} We run all experiments on a machine equipped with an AMD Ryzen 9 7900 12-Core Processor and an NVIDIA GeForce RTX 4090 GPU. Our implementation is based on the PyTorch Lightning framework. We use stochastic gradient descent as the optimizer with a learning rate of $\eta = 0.001$ and no weight decay. We use a batch size of $128$. The training and test steps are conducted for $6000$ epochs on the \textit{Water Potability} and \textit{Phoneme} datasets, while for the \textit{CIFAR-10} dataset, they are performed for $200$ epochs.
Finally, the contribution $w_i^{\varepsilon}$ of each training point $\inst_i$ is uniformly set as $w_i^{\varepsilon} = 1~\forall i\in \{1,\ldots,m\}$.

The source code implementation for our experiments is available at the following GitHub repository: \url{https://anonymous.4open.science/r/COCE-80B4/README.md} 

\subsection{RQ1: Counterfactual Perturbation vs. Overfitting}
To address \textbf{RQ1}, we analyze the relationship between the test loss and the average $L_2$-norm of the counterfactual perturbation vectors ($\overline{||\perturb||}$) over training epochs.

In particular, Figure~\ref{fig:delta_loss_epochs} depicts the evolution of $\overline{||\perturb||}$ alongside the test loss for an MLP trained \textit{without} regularization on the \textit{Water Potability} dataset. 
\begin{figure}[ht]
    \centering
    \includegraphics[width=0.85\linewidth]{img/delta_loss_epochs.png}
    \caption{The average counterfactual perturbation vector $\overline{||\perturb||}$ (left $y$-axis) and the cross-entropy test loss (right $y$-axis) over training epochs ($x$-axis) for an MLP trained on the \textit{Water Potability} dataset \textit{without} regularization.}
    \label{fig:delta_loss_epochs}
\end{figure}

The plot shows a clear trend as the model starts to overfit the data (evidenced by an increase in test loss). 
Notably, $\overline{||\perturb||}$ begins to decrease, which aligns with the hypothesis that the average distance to the optimal counterfactual example gets smaller as the model's decision boundary becomes increasingly adherent to the training data.

It is worth noting that this trend is heavily influenced by the choice of the counterfactual generator model. In particular, the relationship between $\overline{||\perturb||}$ and the degree of overfitting may become even more pronounced when leveraging more accurate counterfactual generators. However, these models often come at the cost of higher computational complexity, and their exploration is left to future work.

Nonetheless, we expect that $\overline{||\perturb||}$ will eventually stabilize at a plateau, as the average $L_2$-norm of the optimal counterfactual perturbations cannot vanish to zero.

% Additionally, the choice of employing the score-based counterfactual explanation framework to generate counterfactuals was driven to promote computational efficiency.

% Future enhancements to the framework may involve adopting models capable of generating more precise counterfactuals. While such approaches may yield to performance improvements, they are likely to come at the cost of increased computational complexity.


\subsection{RQ2: Counterfactual Regularization Performance}
To answer \textbf{RQ2}, we evaluate the effectiveness of the proposed counterfactual regularization (CF-Reg) by comparing its performance against existing baselines: unregularized training loss (No-Reg), L1 regularization (L1-Reg), L2 regularization (L2-Reg), and Dropout.
Specifically, for each model and dataset combination, Table~\ref{tab:regularization_comparison} presents the mean value and standard deviation of test accuracy achieved by each method across 5 random initialization. 

The table illustrates that our regularization technique consistently delivers better results than existing methods across all evaluated scenarios, except for one case -- i.e., Logistic Regression on the \textit{Phomene} dataset. 
However, this setting exhibits an unusual pattern, as the highest model accuracy is achieved without any regularization. Even in this case, CF-Reg still surpasses other regularization baselines.

From the results above, we derive the following key insights. First, CF-Reg proves to be effective across various model types, ranging from simple linear models (Logistic Regression) to deep architectures like MLPs and CNNs, and across diverse datasets, including both tabular and image data. 
Second, CF-Reg's strong performance on the \textit{Water} dataset with Logistic Regression suggests that its benefits may be more pronounced when applied to simpler models. However, the unexpected outcome on the \textit{Phoneme} dataset calls for further investigation into this phenomenon.


\begin{table*}[h!]
    \centering
    \caption{Mean value and standard deviation of test accuracy across 5 random initializations for different model, dataset, and regularization method. The best results are highlighted in \textbf{bold}.}
    \label{tab:regularization_comparison}
    \begin{tabular}{|c|c|c|c|c|c|c|}
        \hline
        \textbf{Model} & \textbf{Dataset} & \textbf{No-Reg} & \textbf{L1-Reg} & \textbf{L2-Reg} & \textbf{Dropout} & \textbf{CF-Reg (ours)} \\ \hline
        Logistic Regression   & \textit{Water}   & $0.6595 \pm 0.0038$   & $0.6729 \pm 0.0056$   & $0.6756 \pm 0.0046$  & N/A    & $\mathbf{0.6918 \pm 0.0036}$                     \\ \hline
        MLP   & \textit{Water}   & $0.6756 \pm 0.0042$   & $0.6790 \pm 0.0058$   & $0.6790 \pm 0.0023$  & $0.6750 \pm 0.0036$    & $\mathbf{0.6802 \pm 0.0046}$                    \\ \hline
%        MLP   & \textit{Adult}   & $0.8404 \pm 0.0010$   & $\mathbf{0.8495 \pm 0.0007}$   & $0.8489 \pm 0.0014$  & $\mathbf{0.8495 \pm 0.0016}$     & $0.8449 \pm 0.0019$                    \\ \hline
        Logistic Regression   & \textit{Phomene}   & $\mathbf{0.8148 \pm 0.0020}$   & $0.8041 \pm 0.0028$   & $0.7835 \pm 0.0176$  & N/A    & $0.8098 \pm 0.0055$                     \\ \hline
        MLP   & \textit{Phomene}   & $0.8677 \pm 0.0033$   & $0.8374 \pm 0.0080$   & $0.8673 \pm 0.0045$  & $0.8672 \pm 0.0042$     & $\mathbf{0.8718 \pm 0.0040}$                    \\ \hline
        CNN   & \textit{CIFAR-10} & $0.6670 \pm 0.0233$   & $0.6229 \pm 0.0850$   & $0.7348 \pm 0.0365$   & N/A    & $\mathbf{0.7427 \pm 0.0571}$                     \\ \hline
    \end{tabular}
\end{table*}

\begin{table*}[htb!]
    \centering
    \caption{Hyperparameter configurations utilized for the generation of Table \ref{tab:regularization_comparison}. For our regularization the hyperparameters are reported as $\mathbf{\alpha/\beta}$.}
    \label{tab:performance_parameters}
    \begin{tabular}{|c|c|c|c|c|c|c|}
        \hline
        \textbf{Model} & \textbf{Dataset} & \textbf{No-Reg} & \textbf{L1-Reg} & \textbf{L2-Reg} & \textbf{Dropout} & \textbf{CF-Reg (ours)} \\ \hline
        Logistic Regression   & \textit{Water}   & N/A   & $0.0093$   & $0.6927$  & N/A    & $0.3791/1.0355$                     \\ \hline
        MLP   & \textit{Water}   & N/A   & $0.0007$   & $0.0022$  & $0.0002$    & $0.2567/1.9775$                    \\ \hline
        Logistic Regression   &
        \textit{Phomene}   & N/A   & $0.0097$   & $0.7979$  & N/A    & $0.0571/1.8516$                     \\ \hline
        MLP   & \textit{Phomene}   & N/A   & $0.0007$   & $4.24\cdot10^{-5}$  & $0.0015$    & $0.0516/2.2700$                    \\ \hline
       % MLP   & \textit{Adult}   & N/A   & $0.0018$   & $0.0018$  & $0.0601$     & $0.0764/2.2068$                    \\ \hline
        CNN   & \textit{CIFAR-10} & N/A   & $0.0050$   & $0.0864$ & N/A    & $0.3018/
        2.1502$                     \\ \hline
    \end{tabular}
\end{table*}

\begin{table*}[htb!]
    \centering
    \caption{Mean value and standard deviation of training time across 5 different runs. The reported time (in seconds) corresponds to the generation of each entry in Table \ref{tab:regularization_comparison}. Times are }
    \label{tab:times}
    \begin{tabular}{|c|c|c|c|c|c|c|}
        \hline
        \textbf{Model} & \textbf{Dataset} & \textbf{No-Reg} & \textbf{L1-Reg} & \textbf{L2-Reg} & \textbf{Dropout} & \textbf{CF-Reg (ours)} \\ \hline
        Logistic Regression   & \textit{Water}   & $222.98 \pm 1.07$   & $239.94 \pm 2.59$   & $241.60 \pm 1.88$  & N/A    & $251.50 \pm 1.93$                     \\ \hline
        MLP   & \textit{Water}   & $225.71 \pm 3.85$   & $250.13 \pm 4.44$   & $255.78 \pm 2.38$  & $237.83 \pm 3.45$    & $266.48 \pm 3.46$                    \\ \hline
        Logistic Regression   & \textit{Phomene}   & $266.39 \pm 0.82$ & $367.52 \pm 6.85$   & $361.69 \pm 4.04$  & N/A   & $310.48 \pm 0.76$                    \\ \hline
        MLP   &
        \textit{Phomene} & $335.62 \pm 1.77$   & $390.86 \pm 2.11$   & $393.96 \pm 1.95$ & $363.51 \pm 5.07$    & $403.14 \pm 1.92$                     \\ \hline
       % MLP   & \textit{Adult}   & N/A   & $0.0018$   & $0.0018$  & $0.0601$     & $0.0764/2.2068$                    \\ \hline
        CNN   & \textit{CIFAR-10} & $370.09 \pm 0.18$   & $395.71 \pm 0.55$   & $401.38 \pm 0.16$ & N/A    & $1287.8 \pm 0.26$                     \\ \hline
    \end{tabular}
\end{table*}

\subsection{Feasibility of our Method}
A crucial requirement for any regularization technique is that it should impose minimal impact on the overall training process.
In this respect, CF-Reg introduces an overhead that depends on the time required to find the optimal counterfactual example for each training instance. 
As such, the more sophisticated the counterfactual generator model probed during training the higher would be the time required. However, a more advanced counterfactual generator might provide a more effective regularization. We discuss this trade-off in more details in Section~\ref{sec:discussion}.

Table~\ref{tab:times} presents the average training time ($\pm$ standard deviation) for each model and dataset combination listed in Table~\ref{tab:regularization_comparison}.
We can observe that the higher accuracy achieved by CF-Reg using the score-based counterfactual generator comes with only minimal overhead. However, when applied to deep neural networks with many hidden layers, such as \textit{PreactResNet-18}, the forward derivative computation required for the linearization of the network introduces a more noticeable computational cost, explaining the longer training times in the table.

\subsection{Hyperparameter Sensitivity Analysis}
The proposed counterfactual regularization technique relies on two key hyperparameters: $\alpha$ and $\beta$. The former is intrinsic to the loss formulation defined in (\ref{eq:cf-train}), while the latter is closely tied to the choice of the score-based counterfactual explanation method used.

Figure~\ref{fig:test_alpha_beta} illustrates how the test accuracy of an MLP trained on the \textit{Water Potability} dataset changes for different combinations of $\alpha$ and $\beta$.

\begin{figure}[ht]
    \centering
    \includegraphics[width=0.85\linewidth]{img/test_acc_alpha_beta.png}
    \caption{The test accuracy of an MLP trained on the \textit{Water Potability} dataset, evaluated while varying the weight of our counterfactual regularizer ($\alpha$) for different values of $\beta$.}
    \label{fig:test_alpha_beta}
\end{figure}

We observe that, for a fixed $\beta$, increasing the weight of our counterfactual regularizer ($\alpha$) can slightly improve test accuracy until a sudden drop is noticed for $\alpha > 0.1$.
This behavior was expected, as the impact of our penalty, like any regularization term, can be disruptive if not properly controlled.

Moreover, this finding further demonstrates that our regularization method, CF-Reg, is inherently data-driven. Therefore, it requires specific fine-tuning based on the combination of the model and dataset at hand.
\putsec{related}{Related Work}

\noindent \textbf{Efficient Radiance Field Rendering.}
%
The introduction of Neural Radiance Fields (NeRF)~\cite{mil:sri20} has
generated significant interest in efficient 3D scene representation and
rendering for radiance fields.
%
Over the past years, there has been a large amount of research aimed at
accelerating NeRFs through algorithmic or software
optimizations~\cite{mul:eva22,fri:yu22,che:fun23,sun:sun22}, and the
development of hardware
accelerators~\cite{lee:cho23,li:li23,son:wen23,mub:kan23,fen:liu24}.
%
The state-of-the-art method, 3D Gaussian splatting~\cite{ker:kop23}, has
further fueled interest in accelerating radiance field
rendering~\cite{rad:ste24,lee:lee24,nie:stu24,lee:rho24,ham:mel24} as it
employs rasterization primitives that can be rendered much faster than NeRFs.
%
However, previous research focused on software graphics rendering on
programmable cores or building dedicated hardware accelerators. In contrast,
\name{} investigates the potential of efficient radiance field rendering while
utilizing fixed-function units in graphics hardware.
%
To our knowledge, this is the first work that assesses the performance
implications of rendering Gaussian-based radiance fields on the hardware
graphics pipeline with software and hardware optimizations.

%%%%%%%%%%%%%%%%%%%%%%%%%%%%%%%%%%%%%%%%%%%%%%%%%%%%%%%%%%%%%%%%%%%%%%%%%%
\myparagraph{Enhancing Graphics Rendering Hardware.}
%
The performance advantage of executing graphics rendering on either
programmable shader cores or fixed-function units varies depending on the
rendering methods and hardware designs.
%
Previous studies have explored the performance implication of graphics hardware
design by developing simulation infrastructures for graphics
workloads~\cite{bar:gon06,gub:aam19,tin:sax23,arn:par13}.
%
Additionally, several studies have aimed to improve the performance of
special-purpose hardware such as ray tracing units in graphics
hardware~\cite{cho:now23,liu:cha21} and proposed hardware accelerators for
graphics applications~\cite{lu:hua17,ram:gri09}.
%
In contrast to these works, which primarily evaluate traditional graphics
workloads, our work focuses on improving the performance of volume rendering
workloads, such as Gaussian splatting, which require blending a huge number of
fragments per pixel.

%%%%%%%%%%%%%%%%%%%%%%%%%%%%%%%%%%%%%%%%%%%%%%%%%%%%%%%%%%%%%%%%%%%%%%%%%%
%
In the context of multi-sample anti-aliasing, prior work proposed reducing the
amount of redundant shading by merging fragments from adjacent triangles in a
mesh at the quad granularity~\cite{fat:bou10}.
%
While both our work and quad-fragment merging (QFM)~\cite{fat:bou10} aim to
reduce operations by merging quads, our proposed technique differs from QFM in
many aspects.
%
Our method aims to blend \emph{overlapping primitives} along the depth
direction and applies to quads from any primitive. In contrast, QFM merges quad
fragments from small (e.g., pixel-sized) triangles that \emph{share} an edge
(i.e., \emph{connected}, \emph{non-overlapping} triangles).
%
As such, QFM is not applicable to the scenes consisting of a number of
unconnected transparent triangles, such as those in 3D Gaussian splatting.
%
In addition, our method computes the \emph{exact} color for each pixel by
offloading blending operations from ROPs to shader units, whereas QFM
\emph{approximates} pixel colors by using the color from one triangle when
multiple triangles are merged into a single quad.



\section{Conclusion}
In this work, we propose a simple yet effective approach, called SMILE, for graph few-shot learning with fewer tasks. Specifically, we introduce a novel dual-level mixup strategy, including within-task and across-task mixup, for enriching the diversity of nodes within each task and the diversity of tasks. Also, we incorporate the degree-based prior information to learn expressive node embeddings. Theoretically, we prove that SMILE effectively enhances the model's generalization performance. Empirically, we conduct extensive experiments on multiple benchmarks and the results suggest that SMILE significantly outperforms other baselines, including both in-domain and cross-domain few-shot settings.
\clearpage
\newpage
This paper presents work whose goal is to advance the field of Machine Learning. 
There are many potential societal consequences of our work, none which we feel must be specifically highlighted here.
\bibliography{ref}
\bibliographystyle{icml2025}
\appendix
\onecolumn
\section{Turing Machine Computation}
\label{appen:turningDetails}
A Turing machine processes an input string $\vect{x} \in \inset$. Its configuration consists of a finite state set $Q$, an input tape $c_0$, $k$ work tapes $c_1, \dots, c_k$, and an output tape $c_{k+1}$. Additionally, each tape $\tau$ has an associated head position $h_\tau$.  

Initially, the machine starts in the initial state $q_0 \in Q$ with the input tape $c_0^0$ containing $\vect{x}$, positioned at index $0$, and surrounded by infinite blank symbols ($b$). The head on the input tape is set to $h_0^0 = 0$, while all other tapes contain only blank symbols $b$s and have their heads positioned at $0$.  

At each time step $i$, if $q_i \notin F$ ($F$ is a set of halting states), the configuration updates recursively by computing:  
\[
\langle q_{i+1}, \gamma_1^i, \dots, \gamma_{k+1}^i, d_0^i, \dots, d_{k+1}^i \rangle = \delta(q_i, c_0^i[h_0^i], \dots, c_{k+1}^i[h_{k+1}^i])
\]  
where $\delta$ is the transition function. The machine updates each tape $\tau$ by setting $c_\tau^{i+1}[h_\tau^i] = \gamma_\tau^i$, leaving all other tape cells unchanged. The head position for each tape is updated as $h_\tau^{i+1} = h_\tau^i + d_\tau^i$.  If $q_i \in F$, the Turing machine halts and outputs the sequence of tokens on the output tape, starting from the current head position and continuing up to (but not including) the first blank symbol ($b$). A Turing machine can also function as a language recognizer by setting the input alphabet $\Sigma = \{0,1\}$ and interpreting the first output token as either $0$ or $1$.

\section{Thershold Circuit Class}
\label{appen:threshInfo}
$\class$ is a class of computational problems that can be recognized by constant-depth, polynomial-size circuits composed of threshold gates. A threshold gate, such as $\theta_{\leq k}$, outputs 1 if the sum of its input bits is at most $ k $, while $\theta_{\geq k}$ outputs 1 if the sum is at least $ k $. These circuits also include standard logic gates like $\wedge$, $\vee$, and $\neg$ as special cases of threshold functions. Since $\class$ circuits can simulate $AC^{0}$ circuits ( a polysize, constant-depth $\{\wedge, \vee, \neg\}$-circuit family), they are at least as powerful as $AC^{0}$ in the computational hierarchy. The circuit families we have defined above are non-uniform, meaning that there is no requirement for the circuits processing different input sizes to be related in any way. In degenerate cases, non-uniform circuit families can solve undecidable problems making them an unrealizable model of computation \cite{complexityBook}. Intuitively, a uniform circuit family requires that the circuits for different input sizes must be "somewhat similar" to each other. This concept is formalized by stating that there exists a resource-constrained Turing machine that, given the input $ 1^n $, can generate a serialization of the corresponding circuit $ C_n $ for that input size. Specifically, a logspace uniform $\class$ family can be constructed by a logspace-bounded Turing machine from the string $1^n$.

\section{Proofs}
\label{sec:proof}
\begin{lemma}[Constant depth circuit for $\llmFormal$]
\label{lem:constrainLem}
For any log-precision constant layer transformer-based LLM $\llm{}$ with finite vocabulary $V$, a single deterministic auto-regressive step $\llmFormal(x)$ operating on any input of size $n \in \mathbb{N}$ with $\vect{x} \in V^{n}$ can be simulated by a logspace-uniform threshold circuit family of depth $C$ where $C$ is constant.
\end{lemma}
\begin{proof}
The construction is from Theorem~2 in \cite{tc0}.
\end{proof}

\compLimitLemma*
\begin{proof}
The language $ L(\regG) $ is finite; therefore, for any string $ \vect{s} \in L(\regG) $, the length satisfies $ |\vect{s}| \leq N $, where $ N $ is a constant. Consequently, for any input $ \vect{x} $, the output $ \vect{y}_G = \llmG{\vect{x}}{G} $ has a constant length, i.e., $ |\vect{y}_{G}| \leq N $. The number of autoregressive steps is also bounded by $ N $.  

From Lemma~\ref{lem:constrainLem}, each unconstrained autoregressive computation $ \llmFormal(\vect{x}) $ can be simulated by a constant-depth threshold circuit $ C $. This implies that $ \llm_f(\vect{x}, \regG) $ can also be simulated by a constant-depth threshold circuit since it only involves an additional multiplication by a constant-sized precomputed Boolean mask $ \{0, 1\}^{|V|} $ (see Section~\ref{sec:prelims}).  

Given that the number of autoregressive steps is a constant $ N $, and each step can be simulated by a constant-depth circuit $ C $, we can simulate all $ N $ steps using a depth $ N \times C $ circuit by stacking the circuits for each step sequentially. For uniformity, we are just stacking together a constant number of constant depth circuits we can do it in a log-space bounded Turning machine $M$. 

Note that this proof holds only because $ L(\regG) $ allows only constant-size strings in the output.  
\end{proof}
\expressivity*
\begin{proof}
The construction follows from Theorem 2 \cite{expressivity1}.
\end{proof}

In this construction, the deterministic Turing machine run captured by a sequence of $\hatConfig{\gamma_1}, \dots, \hatConfig{\gamma_{t(n)}}$ capturing the state entered, tokens written, and directions moved after each token before generating the output $M(\vect{x})$. Then on any input the $\vect{x}$ the output $\llm_{M}(\vect{x}) = \hatConfig{\gamma_1}, \cdots, \hatConfig{\gamma_{t(n)}}\cdot M(\vect{x})$ (assuming $M$ halts within on $\vect{x}$ within $\runtime{n}$ steps where $n = |\vect{x}|$ and $\runtime{n}$ is a polynomial over $n$).

\expressConstrain*
\begin{proof}
$\llm{}_{M}(\vect{x})) = \hatConfig{\gamma_1}\cdots\hatConfig{\gamma_{\runtime{n}}}\cdot M(\vect{x})$. We show that $\llm{}_{M}(\vect{x}) \in L(\augG)$. 
$\augG \to \rGM G$. Since, $G$ is output grammar of $M$ then $M(\vect{x}) \in \lang{G}$. For all $1 \leq i \leq \runtime{n}$ $\hatConfig{\gamma_{i}} \in \hatConfig{\Gamma}$. Then, $\hatConfig{\gamma_1}\cdots\hatConfig{\gamma_{\runtime{n}}} \in \hatConfig{\Gamma}^{*} \subseteq L(\rGM)$.

Then $\llm{}_{M}(\vect{x}) \in \lang{\augG}$ then under constrained decoding the output $\llm{}_{M}(\vect{x})$ remains unchanged and $\llm{}_{M}(\vect{x}) = \llmMG{\vect{x}}{\augG} = \vect{r} \cdot M(\vect{x})$ where $\vect{r} = \hatConfig{\gamma_1}\cdots\hatConfig{\gamma_{\runtime{n}}}$.
\end{proof}
\clearpage
\newpage



\begin{table*}[t]
    \centering
    \caption{Comparison of \Tool{} and baselines with various models on GSM-Symbolic based on accuracy, number of tokens, and average time.}
    \begin{tabular}{llcccr}
        \toprule
        \textbf{Model} & \textbf{k} & \textbf{Method} & \textbf{Acc. (\%)} & \textbf{Parse (\%)} &  \textbf{Tokens}\\
        
\midrule

   &  & \stdUnconstrained{} & 20 & 98 & 18.23 \\
 & & \stdConstrained{} & 21 & 95 & 34.28 \\
 Qwen2.5-1.5B-Instruct & 2 & \cotUnconstrained{} & 22 & 90 & 130.74 \\
 & & \textbf{\Tool{}} & \textbf{28} & 96 & 140.52 \\

\midrule

   &  & \stdUnconstrained{} & 18 & 95 & 18.23 \\
 & & \stdConstrained{} & 18 & 96 & 34.28 \\
 Qwen2.5-1.5B-Instruct & 4 & \cotUnconstrained{} & 24 & 94 & 130.74 \\
 & & \textbf{\Tool{}} & \textbf{30} & 98 & 140.52 \\

\midrule
   &  & \stdUnconstrained{} & 21 & 97 & 23.34  \\
 & & \stdConstrained{} & 22 & 97 & 25.29  \\
  Qwen2.5-1.5B-Instruct  & 8 & \cotUnconstrained{} & 26 & 90 & 128.97  \\
 & & \textbf{\Tool{}} & \textbf{31} & 100 & 131.3  \\

\midrule

    & & \stdUnconstrained{} & 37 & 96 & 17.22 \\
 & & \stdConstrained{} & 36 & 99 & 18.61  \\
 Qwen2.5-Coder-7B-Instruct & 2 & \cotUnconstrained{} & 32 & 84 & 148.87 \\
 & & \textbf{\Tool{}} & \textbf{37} & 96 & 155.65\\

\midrule

    & & \stdUnconstrained{} & 36 & 96 & 16.89 \\
 & & \stdConstrained{} & 36 & 100 & 18.81  \\
 Qwen2.5-Coder-7B-Instruct & 4 & \cotUnconstrained{} & 35 & 89 & 151.29 \\
 & & \textbf{\Tool{}} & \textbf{37} & 97 & 163.21\\

\midrule

    & & \stdUnconstrained{} & 36 & 94 & 17.92 \\
 & & \stdConstrained{} & 35 & 99 & 25.28  \\
 Qwen2.5-Coder-7B-Instruct & 8 & \cotUnconstrained{} & 37 & 88 & 138.38 \\
 & & \textbf{\Tool{}} & \textbf{39} & 94 & 155.32\\

\midrule

     & & \stdUnconstrained{} & 20 & 66 & 115.22 \\
&  & \stdConstrained{} & 26 & 95 & 26.99 \\
 Qwen2.5-Math-7B-Instruct & 2 & \cotUnconstrained{} & 28 & 72 & 190.51 \\
 & & \textbf{\Tool{}} & \textbf{32} & 89 & 195.65 \\

 \midrule

    &  & \stdUnconstrained{} & 22 & 83 & 47 \\
 & & \stdConstrained{} & 29 & 98 & 27.08 \\
 Qwen2.5-Math-7B-Instruct & 4 & \cotUnconstrained{} & 28 & 76 & 184.35 \\
 & & \textbf{\Tool{}} & \textbf{37} & 88 & 194.77  \\

 \midrule

     & & \stdUnconstrained{} & 27 & 89 & 25.7 \\
 & & \stdConstrained{} & 29 & 99 & 26.81 \\
 Qwen2.5-Math-7B-Instruct & 8 &  \cotUnconstrained{} & 29 & 82 & 155.26\\
 & & \textbf{\Tool{}} & \textbf{38} & 94 & 158.86 \\

 \midrule

     & & \stdUnconstrained{} & 19 & 61 & 157.36 \\
 & & \stdConstrained{} & 23 & 95 & 45.58  \\
 Llama-3.1-8B-Instruct & 2 & \cotUnconstrained{} & 29 & 84 & 198.64 \\
 & & \textbf{\Tool{}} & \textbf{35} & 94 & 206.85 \\

 \midrule
     & & \stdUnconstrained{} & 18 & 68 & 131.5 \\
 & & \stdConstrained{} & 24 & 96 & 37.38  \\
 Llama-3.1-8B-Instruct & 4 & \cotUnconstrained{} & 26 & 92 & 172.21 \\
 & & \textbf{\Tool{}} & \textbf{30} & 97 & 179.95 \\

  \midrule

     & & \stdUnconstrained{} & 21 & 73 & 128.38 \\
 & & \stdConstrained{} & 26 & 98 & 35.97  \\
 Llama-3.1-8B-Instruct & 8 & \cotUnconstrained{} & 30 & 95 & 163.55 \\
 & & \textbf{\Tool{}} & \textbf{33} & 95 & 170.22 \\


\bottomrule
    \end{tabular}
    \label{tab:gsm_symbolic_comparison_k_shot}
    \vspace{-.2in}
\end{table*}


\subsection{GSM-Symbolic Examples and Prompt}
\label{sec:gsm_info}
\textbf{GSM-Symbolic Problem Solution Examples:}



% \lstdefinestyle{myGrammarStyle}{
%     basicstyle=\scriptsize\ttfamily, % Reduce font size
%     commentstyle=\color{gray},
%     keywordstyle=\color{blue},
%     stringstyle=\color{orange},
%     numbers=left, % Line numbers on left
%     numberstyle=\tiny\color{gray}, % Line numbers styling
%     breaklines=true, % Wrap long lines
%     frame=single, % Frame around the code
%     framesep=3pt, % Adjust frame separation
%     xleftmargin=5pt, % Adjust left margin
%     xrightmargin=5pt, % Adjust right margin
%     backgroundcolor=\color{blue!4}, % Background color
%     tabsize=2, % Tab size
%     captionpos=b, % Caption position bottom
%     aboveskip=5pt, % Reduce space above the listing
%     belowskip=5pt, % Reduce space below the listing
%     linewidth=0.9\linewidth, % Set custom line width to less than text width
%     escapeinside={(*@}{@*)}, % for escaping to LaTeX
% }

\begin{lstlisting}[style=myGrammarStyle, caption=Problem Solution Examples for GSM-Symbolic]
Question: A fog bank rolls in from the ocean to cover a city. It takes {t} minutes to cover every {d} miles of the city. If the city is {y} miles across from oceanfront to the opposite inland edge, how many minutes will it take for the fog bank to cover the whole city?

Answer: y//d*t

Question: {name} makes {drink} using teaspoons of sugar and cups of water in the ratio of {m}:{n}. If she used a total of {x} teaspoons of sugar and cups of water, calculate the number of teaspoonfuls of sugar she used.

Answer: ((m*x)//(m+n))
\end{lstlisting}
\label{gram:gsm_example}
\textbf{GSM-Symbolic Prompt:}



% \lstdefinestyle{myGrammarStyle}{
%     basicstyle=\scriptsize\ttfamily, % Reduce font size
%     commentstyle=\color{gray},
%     keywordstyle=\color{blue},
%     stringstyle=\color{orange},
%     numbers=left, % Line numbers on left
%     numberstyle=\tiny\color{gray}, % Line numbers styling
%     breaklines=true, % Wrap long lines
%     frame=single, % Frame around the code
%     framesep=3pt, % Adjust frame separation
%     xleftmargin=5pt, % Adjust left margin
%     xrightmargin=5pt, % Adjust right margin
%     backgroundcolor=\color{green!4}, % Background color
%     tabsize=2, % Tab size
%     captionpos=b, % Caption position bottom
%     aboveskip=5pt, % Reduce space above the listing
%     belowskip=5pt, % Reduce space below the listing
%     linewidth=0.9\linewidth, % Set custom line width to less than text width
%     escapeinside={(*@}{@*)}, % for escaping to LaTeX
% }

\begin{lstlisting}[style=myGrammarStyle, caption=CoT Prompt Template For GSM-Symbolic Evaluation]
You are an expert in solving grade school math tasks. You will be presented with a grade-school math word problem with symbolic variables and be asked to solve it.

Before answering you should reason about the problem (using the <reasoning> field in the response described below). Intermediate symbolic expressions generated during reasoning should be wrapped in << >>.

Then, output the symbolic expression wrapped in << >> that answers the question. The expressions must use numbers as well as the variables defined in the question. You are only allowed to use the following operations: +, -, /, //, %, (), and int().

You will always respond in the format described below: 
Let's think step by step. <reasoning> The final answer is <<symbolic expression>>

There are {t} trees in the {g}. {g} workers will plant trees in the {g} today. After they are done, there will be {tf} trees. How many trees did the {g} workers plant today?

Let's think step by step. Initially, there are {t} trees. After planting, there are {tf} trees. The number of trees planted is <<tf - t>>. The final answer is <<tf - t>>.

If there are {c} cars in the parking lot and {nc} more cars arrive, how many cars are in the parking lot?

Let's think step by step. Initially, there are {c} cars. {nc} more cars arrive, so the total becomes <<c + nc>>. The final answer is <<c + nc>>.

{p1} had {ch1} {o1} and {p2} had {ch2} {o1}. If they ate {a} {o1}, how many pieces do they have left in total?

Let's think step by step. Initially, {p1} had {ch1} {o1}, and {p2} had {ch2} {o1}, making a total of <<ch1 + ch2>>. After eating {a} {o1}, the remaining total is <<ch1 + ch2 - a>>. The final answer is <<ch1 + ch2 - a>>.

{p1} had {l1} {o1}. {p1} gave {g} {o1} to {p2}. How many {o1} does {p1} have left?

Let's think step by step. {p1} started with {l1} {o1}. After giving {g} {o1} to {p2}, {p1} has <<l1 - g>> {o1} left. The final answer is <<l1 - g>>.

{p1} has {t} {o1}. For Christmas, {p1} got {tm} {o1} from {p2} and {td} {o1} from {p3}. How many {o1} does {p1} have now?

Let's think step by step. {p1} started with {t} {o1}. {p1} received {tm} {o1} from {p2} and {td} {o1} from {p3}. The total is <<t + tm + td>>. The final answer is <<t + tm + td>>.

There were {c} {o1} in the server room. {nc} more {o1} were installed each day, from {d1} to {d2}. How many {o1} are now in the server room?

Let's think step by step. Initially, there were {c} {o1}. {nc} {o1} were added each day for <<d2 - d1 + 1>> days, which is <<nc * (d2 - d1 + 1)>>. The total is <<c + nc * (d2 - d1 + 1)>>. The final answer is <<c + nc * (d2 - d1 + 1)>>.

{p1} had {gb1} {o1}. On {day1}, {p1} lost {l1} {o1}. On {day2}, {p1} lost {l2} more. How many {o1} does {p1} have at the end of {day2}?

Let's think step by step. Initially, {p1} had {gb1} {o1}. After losing {l1} {o1} on {day1}, {p1} had <<gb1 - l1>>. After losing {l2} {o1} on {day2}, the total is <<gb1 - l1 - l2>>. The final answer is <<gb1 - l1 - l2>>.

{p1} has ${m}. {p1} bought {q} {o1} for ${p} each. How much money does {p1} have left?

Let's think step by step. Initially, {p1} had ${m}. {p1} spent <<q * p>> on {q} {o1}. The remaining money is <<m - q * p>>. The final answer is <<m - q * p>>.

{question}
\end{lstlisting}
\label{gram:gsm_prompt}



% \lstdefinestyle{myGrammarStyle}{
%     basicstyle=\scriptsize\ttfamily, % Reduce font size
%     commentstyle=\color{gray},
%     keywordstyle=\color{blue},
%     stringstyle=\color{orange},
%     numbers=left, % Line numbers on left
%     numberstyle=\tiny\color{gray}, % Line numbers styling
%     breaklines=true, % Wrap long lines
%     frame=single, % Frame around the code
%     framesep=3pt, % Adjust frame separation
%     xleftmargin=5pt, % Adjust left margin
%     xrightmargin=5pt, % Adjust right margin
%     backgroundcolor=\color{green!4}, % Background color
%     tabsize=2, % Tab size
%     captionpos=b, % Caption position bottom
%     aboveskip=5pt, % Reduce space above the listing
%     belowskip=5pt, % Reduce space below the listing
%     linewidth=0.9\linewidth, % Set custom line width to less than text width
%     escapeinside={(*@}{@*)}, % for escaping to LaTeX
% }

\begin{lstlisting}[style=myGrammarStyle, caption= Prompt Template For GSM-Symbolic Evaluation Without CoT]
You are an expert in solving grade school math tasks. You will be presented with a grade-school math word problem with symbolic variables and be asked to solve it.

Only output the symbolic expression wrapped in << >> that answers the question. The expression must use numbers as well as the variables defined in the question. You are only allowed to use the following operations: +, -, /, //, %, (), and int().

You will always respond in the format described below: 
<<symbolic expression>>

There are {t} trees in the {g}. {g} workers will plant trees in the {g} today. After they are done, there will be {tf} trees. How many trees did the {g} workers plant today?

<<tf - t>>

If there are {c} cars in the parking lot and {nc} more cars arrive, how many cars are in the parking lot?

<<c + nc>>

{p1} had {ch1} {o1} and {p2} had {ch2} {o1}. If they ate {a} {o1}, how many pieces do they have left in total?

<<ch1 + ch2 - a>>

{p1} had {l1} {o1}. {p1} gave {g} {o1} to {p2}. How many {o1} does {p1} have left?

<<l1 - g>>

{p1} has {t} {o1}. For Christmas, {p1} got {tm} {o1} from {p2} and {td} {o1} from {p3}. How many {o1} does {p1} have now?

<<t + tm + td>>

There were {c} {o1} in the {loc}. {nc} more {o1} were installed each day, from {d1} to {d2}. How many {o1} are now in the {loc}?

<<c + nc * (d2 - d1 + 1)>>

{p1} had {gb1} {o1}. On {day1}, {p1} lost {l1} {o1}. On {day2}, {p1} lost {l2} more. How many {o1} does {p1} have at the end of {day2}?

<<gb1 - l1 - l2>>

{p1} has ${m}. {p1} bought {q} {o1} for ${p} each. How much money does {p1} have left?

<<m - q * p>>

{question}
\end{lstlisting}
\label{gram:gsm_prompt_no_cot}

\subsection{FOLIO Examples and Prompt}
\label{sec:folio_info}
\textbf{FOLIO Problem Solution Examples:}

\begin{lstlisting}[style=myGrammarStyle, caption=Problem Solution Examples for FOLIO]
Question: 
People in this club who perform in school talent shows often attend and are very engaged with school events.
People in this club either perform in school talent shows often or are inactive and disinterested community members.
People in this club who chaperone high school dances are not students who attend the school.
All people in this club who are inactive and disinterested members of their community chaperone high school dances.
All young children and teenagers in this club who wish to further their academic careers and educational opportunities are students who attend the school. 
Bonnie is in this club and she either both attends and is very engaged with school events and is a student who attends the school or is not someone who both attends and is very engaged with school events and is not a student who attends the school.
Based on the above information, is the following statement true, false, or uncertain? Bonnie performs in school talent shows often.
###

FOL Solution: 
Predicates:
InClub(x) ::: x is a member of the club.
Perform(x) ::: x performs in school talent shows.
Attend(x) ::: x attends school events.
Engaged(x) ::: x is very engaged with school events.
Inactive(x) ::: x is an inactive and disinterested community member.
Chaperone(x) ::: x chaperones high school dances.
Student(x) ::: x is a student who attends the school.
Wish(x) ::: x wishes to further their academic careers and educational opportunities.
Premises:
{forall} x (InClub(x) {and} Attend(x) {and} Engaged(x) {implies} Attend(x)) ::: People in this club who perform in school talent shows often attend and are very engaged with school events.
{forall} x (InClub(x) {implies} (Perform(x) {xor} Inactive(x))) ::: People in this club either perform in school talent shows often or are inactive and disinterested community members.
{forall} x (InClub(x) {and} Chaperone(x) {implies} {not}Student(x)) ::: People in this club who chaperone high school dances are not students who attend the school.
{forall} x (InClub(x) {and} Inactive(x) {implies} Chaperone(x)) ::: All people in this club who are inactive and disinterested members of their community chaperone high school dances.
{forall} x (InClub(x) {and} (Young(x) {or} Teenager(x)) {and} Wish(x) {implies} Student(x)) ::: All young children and teenagers in this club who wish to further their academic careers and educational opportunities are students who attend the school.
{forall} x (InClub(x) {implies} (Attend(x) {and} Engaged(x)) {xor} {not}(Attend(x) {and} Engaged(x)) {and} {not}Student(x) {xor} Student(x)) ::: Bonnie is in this club and she either both attends and is very engaged with school events and is a student who attends the school or is not someone who both attends and is very engaged with school events and is not a student who attends the school.
Conclusion:
InClub(bonnie) {and} Perform(bonnie) ::: Bonnie performs in school talent shows often.

Answer: Uncertain

\end{lstlisting}
\label{gram:folio_example}
\textbf{FOLIO Prompt:}
% \lstdefinestyle{myGrammarStyle}{
%     basicstyle=\scriptsize\ttfamily, % Reduce font size
%     commentstyle=\color{gray},
%     keywordstyle=\color{blue},
%     stringstyle=\color{orange},
%     numbers=left, % Line numbers on left
%     numberstyle=\tiny\color{gray}, % Line numbers styling
%     breaklines=true, % Wrap long lines
%     frame=single, % Frame around the code
%     framesep=3pt, % Adjust frame separation
%     xleftmargin=5pt, % Adjust left margin
%     xrightmargin=5pt, % Adjust right margin
%     backgroundcolor=\color{green!4}, % Background color
%     tabsize=2, % Tab size
%     captionpos=b, % Caption position bottom
%     aboveskip=5pt, % Reduce space above the listing
%     belowskip=5pt, % Reduce space below the listing
%     linewidth=0.9\linewidth, % Set custom line width to less than text width
%     escapeinside={(*@}{@*)}, % for escaping to LaTeX
% }

\begin{lstlisting}[style=myGrammarStyle, caption=Prompt Template Used For FOLIO Evaluation]
Given a problem description and a question. The task is to parse the problem and the question into first-order logic formulas.
The grammar of the first-order logic formula is defined as follows:
1) logical conjunction of expr1 and expr2: expr1 {and} expr2
2) logical disjunction of expr1 and expr2: expr1 {or} expr2
3) logical exclusive disjunction of expr1 and expr2: expr1 {xor} expr2
4) logical negation of expr1: {not}expr1
5) expr1 implies expr2: expr1 {implies} expr2
6) expr1 if and only if expr2: expr1 {iff} expr2
7) logical universal quantification: {forall} x
8) logical existential quantification: {exists} x. These are the ONLY operations in the grammar.
------

Answer the question EXACTLY like the examples.

Problem:
All people who regularly drink coffee are dependent on caffeine. People either regularly drink coffee or joke about being addicted to caffeine. No one who jokes about being addicted to caffeine is unaware that caffeine is a drug. Rina is either a student and unaware that caffeine is a drug, or neither a student nor unaware that caffeine is a drug. If Rina is not a person dependent on caffeine and a student, then Rina is either a person dependent on caffeine and a student, or neither a person dependent on caffeine nor a student.
Question:
Based on the above information, is the following statement true, false, or uncertain? Rina is either a person who jokes about being addicted to caffeine or is unaware that caffeine is a drug.
###

We take three steps: first, we define the necessary predicates and premises, and finally, we encode the question `Rina is either a person who jokes about being addicted to caffeine or is unaware that caffeine is a drug.` in the conclusion. Now, we will write only the logic program, nothing else.
Predicates:
Dependent(x) ::: x is a person dependent on caffeine.
Drinks(x) ::: x regularly drinks coffee.
Jokes(x) ::: x jokes about being addicted to caffeine.
Unaware(x) ::: x is unaware that caffeine is a drug.
Student(x) ::: x is a student.
Premises:
{forall} x (Drinks(x) {implies} Dependent(x)) ::: All people who regularly drink coffee are dependent on caffeine.
{forall} x (Drinks(x) {xor} Jokes(x)) ::: People either regularly drink coffee or joke about being addicted to caffeine.
{forall} x (Jokes(x) {implies} {not}Unaware(x)) ::: No one who jokes about being addicted to caffeine is unaware that caffeine is a drug. 
(Student(rina) {and} Unaware(rina)) {xor} {not}(Student(rina) {or} Unaware(rina)) ::: Rina is either a student and unaware that caffeine is a drug, or neither a student nor unaware that caffeine is a drug.
Conclusion:
Jokes(rina) {xor} Unaware(rina) ::: Rina is either a person who jokes about being addicted to caffeine or is unaware that caffeine is a drug.
------

Problem:
Miroslav Venhoda was a Czech choral conductor who specialized in the performance of Renaissance and Baroque music. Any choral conductor is a musician. Some musicians love music. Miroslav Venhoda published a book in 1946 called Method of Studying Gregorian Chant.
Question:
Based on the above information, is the following statement true, false, or uncertain? Miroslav Venhoda loved music.
###

We take three steps: first, we define the necessary predicates and premises, and finally, we encode the question `Miroslav Venhoda loved music.` in the conclusion. Now, we will write only the logic program, nothing else.
Predicates:
Czech(x) ::: x is a Czech person.
ChoralConductor(x) ::: x is a choral conductor.
Musician(x) ::: x is a musician.
Love(x, y) ::: x loves y.
Author(x, y) ::: x is the author of y.
Book(x) ::: x is a book.
Publish(x, y) ::: x is published in year y.
Specialize(x, y) ::: x specializes in y.
Premises:
Czech(miroslav) {and} ChoralConductor(miroslav) {and} Specialize(miroslav, renaissance) {and} Specialize(miroslav, baroque) ::: Miroslav Venhoda was a Czech choral conductor who specialized in the performance of Renaissance and Baroque music.
{forall} x (ChoralConductor(x) {implies} Musician(x)) ::: Any choral conductor is a musician.
{exists} x (Musician(x) {and} Love(x, music)) ::: Some musicians love music.
Book(methodOfStudyingGregorianChant) {and} Author(miroslav, methodOfStudyingGregorianChant) {and} Publish(methodOfStudyingGregorianChant, year1946) ::: Miroslav Venhoda published a book in 1946 called Method of Studying Gregorian Chant.
Conclusion:
Love(miroslav, music) ::: Miroslav Venhoda loved music.
------

{question}
\end{lstlisting}
\label{gram:folio_prompt}


\newpage
\subsection{Case Study For GSM-Symbolic}



\lstdefinestyle{myGrammarStyle}{
    basicstyle=\scriptsize\ttfamily, % Reduce font size
    commentstyle=\color{gray},
    keywordstyle=\color{blue},
    stringstyle=\color{orange},
    numbers=left, % Line numbers on left
    numberstyle=\tiny\color{gray}, % Line numbers styling
    breaklines=true, % Wrap long lines
    frame=single, % Frame around the code
    framesep=3pt, % Adjust frame separation
    xleftmargin=5pt, % Adjust left margin
    xrightmargin=5pt, % Adjust right margin
    backgroundcolor=\color{gray!4}, % Background color
    tabsize=2, % Tab size
    captionpos=b, % Caption position bottom
    aboveskip=5pt, % Reduce space above the listing
    belowskip=5pt, % Reduce space below the listing
    linewidth=0.9\linewidth, % Set custom line width to less than text width
    escapeinside={(*@}{@*)}, % for escaping to LaTeX
}

\begin{lstlisting}[style=myGrammarStyle, caption=Case Study for GSM-Symbolic]
Question: {name} hires a {vehicle} from {start_hour}  to {end_hour}. He gets {free_hours} hours free. The first paid hour is ${first_hour_cost} and each hour after that is {multiplier} the cost. How much did he pay?

Constrained: <<(int(end_hour - start_hour) - free_hours) * first_hour_cost + free_hours * first_hour_cost + (int(end_hour - start_hour) - free_hours - 1) * multiplier * first_hour_cost>>

Unconstrained With CoT:  Let's think step by step. The total time hired is from {start_hour} to {end_hour}, which is <<int((end_hour - start_hour).total_seconds() / 3600)>> hours. Subtracting the {free_hours} free hours, the paid hours are <<int((end_hour - start_hour).total_seconds() / 3600) - free_hours>>. The final answer is <<first_hour_cost + (int((end_hour - start_hour).total_seconds() / 3600) - free_hours - 1) * multiplier * first_hour_cost>>.

CRANE: Let's think step by step. The total time hired is from {start_hour} to {end_hour}, totaling <<int(end_hour - start_hour)>> hours. Subtracting {free_hours} free hours, the paid hours are <<int(end_hour - start_hour) - free_hours>>. The first hour costs {first_hour_cost}, and each additional hour costs {multiplier} times that. The final answer is <<first_hour_cost + (int(end_hour - start_hour) - free_hours - 1) * multiplier * first_hour_cost>>.


\end{lstlisting}
\label{gram:gsm_case_study}

\Tool{} effectively alternates between constrained and unconstrained generation to produce intermediate expressions, the final answer, and to maintain the reasoning capabilities of the LLM. In contrast, unconstrained generation with CoT results in a syntactically incorrect expression, while constrained generation produces a syntactically valid but incorrect expression.

\subsection{Sampling Ablation for GSM-Symbolic}
In our GSM-Symbolic case study, we use IterGen as the constrained generation baseline and initialize \Tool{} with IterGen. Both IterGen and \Tool{} employ selective rejection sampling to filter tokens that do not satisfy semantic constraints. For comparison, we also run unconstrained generation using temperature sampling and evaluate its performance against \Tool{}. Specifically, for Qwen2.5-1.5B-Instruct and Llama-3.1-8B-Instruct, we generate three samples with unconstrained generation at a temperature of \( t = 0.7 \) and compute pass@1/2/3 metrics. 

As shown in Table ~\ref{tab:rejection_sample_gsm}, \Tool{} with greedy decoding achieves higher accuracy than pass@1/2/3 for unconstrained generation with Chain-of-Thought (CoT) and temperature sampling on Qwen2.5-1.5B-Instruct. Although, for Llama-3.1-8B-Instruct, unconstrained generation with CoT and temperature sampling achieves a pass@3 accuracy of 35\%—2\% higher than \Tool{}—it generates approximately 4 times as many tokens as \Tool{}.

\begin{table*}[t]
    \centering
    \small
    \caption{Comparison of \Tool{} and greedy and sampling baselines with different models on GSM-Symbolic.}
    \begin{tabular}{llccr}
        \toprule
        \textbf{Model} & \textbf{Method} & \textbf{pass@1/2/3 (\%)} & \textbf{Parse (\%)} &  \textbf{Tokens} \\
        
\midrule
     & \stdUnconstrained{} (Greedy) & 21 & 97 & 23.34\\
     & \stdUnconstrained{} (t = 0.7) & 15/19/22 & 88/96/98 & 20.19/39.76/60.57\\
 & \stdConstrained{} (Greedy) & 22 & 97 & 25.29 \\
 Qwen2.5-1.5B-Instruct & \cotUnconstrained{} (Greedy) & 26 & 90 & 128.97\\
 & \cotUnconstrained{} (t = 0.7) & 21/25/30 & 78/91/96 & 146.22/292.96/444.61\\
 & \textbf{\Tool{}} & \textbf{31} & 100 & 131.3\\

\midrule

     & \stdUnconstrained{} (Greedy) & 21 & 73 & 128.38\\
     & \stdUnconstrained{} (t = 0.7) & 15/21/25  & 51/74/84 & 106.88/232.75/369.86\\
 & \stdConstrained{} (Greedy) & 26 & 98 & 35.97 \\
 Llama-3.1-8B-Instruct & \cotUnconstrained{} (Greedy) & 30 & 95 & 163.55 \\
 & \cotUnconstrained{} (t = 0.7) & 24/29/\textbf{35} & 89/98/98 & 196.01/403.68/607.7\\
 & \textbf{\Tool{}} (Greedy) & 33 & 95 & 170.22 \\



\bottomrule
    \end{tabular}
    \label{tab:rejection_sample_gsm}
    \vspace{-.2in}
\end{table*}



\begin{table*}[t]
    \centering
    \small
    \caption{Comparison of \Tool{} and greedy and sampling baselines with different models on FOLIO.}
    \begin{tabular}{llccr}
        \toprule
        \textbf{Model} & \textbf{Method} & \textbf{pass@1/2/3 (\%)} & \textbf{Compile (\%)} &  \textbf{Tokens} \\
    \midrule
    & \cotUnconstrained{} (Greedy) & 36.95 & 70.94 & 350.64  \\
    & \cotUnconstrained{} (t = 0.7) & 16.75/28.57/34.98 & 35.96/55.67/68.47 & 401.5/800.19/1219.33  \\
 Qwen2.5-7B-Instruct & \stdConstrained{} (Greedy) & 37.44 & 87.68 & 775.62 \\
 & \textbf{\Tool{}} (Greedy) & \textbf{42.36} & 87.68 & 726.88  \\

% Qwen2.5-7B & \cotConstrained{} & 40.39 & 774.18 & 25.74 \\
 \midrule
     & \cotUnconstrained{} (Greedy) & 32.02 & 57.14 & 371.52  \\
     & \cotUnconstrained{} (t = 0.7) & 14.29/22.66/29.06 & 33.99/46.8/57.64 & 435.35/877.33/1307.45  \\
 Llama-3.1-8B-Instruct & \stdConstrained{} (Greedy) & 39.41 & 86.21 & 549.75  \\
 & \textbf{\Tool{}} (Greedy) & \textbf{46.31} & 85.71 & 449.77  \\




\bottomrule
    \end{tabular}
    \label{tab:rejection_sample_fol}
    \vspace{-.2in}
\end{table*}


\subsection{Grammars}
\subsubsection{GSM-Symbolic Grammar}
\label{sec:gsm_grammar}

\lstdefinestyle{myGrammarStyle}{
    basicstyle=\scriptsize\ttfamily, % Reduce font size
    commentstyle=\color{gray},
    keywordstyle=\color{blue},
    stringstyle=\color{orange},
    numbers=left, % Line numbers on left
    numberstyle=\tiny\color{gray}, % Line numbers styling
    breaklines=true, % Wrap long lines
    frame=single, % Frame around the code
    framesep=3pt, % Adjust frame separation
    xleftmargin=5pt, % Adjust left margin
    xrightmargin=5pt, % Adjust right margin
    backgroundcolor=\color{yellow!4}, % Background color
    tabsize=2, % Tab size
    captionpos=b, % Caption position bottom
    aboveskip=5pt, % Reduce space above the listing
    belowskip=5pt, % Reduce space below the listing
    linewidth=0.9\linewidth, % Set custom line width to less than text width
    escapeinside={(*@}{@*)}, % for escaping to LaTeX
}

\begin{lstlisting}[style=myGrammarStyle, caption=GSM-Symbolic Grammar]
start: space? "<" "<" space? expr space? ">" ">" space?

expr: expr space? "+" space? term   
     | expr space? "-" space? term   
     | term

term: term space? "*" space? factor 
     | term space? "/" space? factor 
     | term space? "//" space? factor 
     | term space? "%" space? factor  
     | factor space?

factor: "-" space? factor    
       | TYPE "(" space? expr space? ")" 
       | primary space?

primary: NUMBER        
        | VARIABLE      
        | "(" space? expr space? ")"

TYPE.4: "int"

space: " "

%import common.CNAME -> VARIABLE
%import common.NUMBER
\end{lstlisting}
\label{gram:gsm_grammar}



\subsubsection{Prover9 Grammar}
\label{sec:prover9_grammar}

\lstdefinestyle{myGrammarStyle}{
    basicstyle=\scriptsize\ttfamily, % Reduce font size
    commentstyle=\color{gray},
    keywordstyle=\color{blue},
    stringstyle=\color{orange},
    numbers=left, % Line numbers on left
    numberstyle=\tiny\color{gray}, % Line numbers styling
    breaklines=true, % Wrap long lines
    frame=single, % Frame around the code
    framesep=3pt, % Adjust frame separation
    xleftmargin=5pt, % Adjust left margin
    xrightmargin=5pt, % Adjust right margin
    backgroundcolor=\color{yellow!4}, % Background color
    tabsize=2, % Tab size
    captionpos=b, % Caption position bottom
    aboveskip=5pt, % Reduce space above the listing
    belowskip=5pt, % Reduce space below the listing
    linewidth=0.9\linewidth, % Set custom line width to less than text width
    escapeinside={(*@}{@*)}, % for escaping to LaTeX
}

\begin{lstlisting}[style=myGrammarStyle, caption=Prover9 Grammar]
start: predicate_section premise_section conclusion_section

predicate_section: "Predicates:" predicate_definition+
premise_section: "Premises:" premise+
conclusion_section: "Conclusion:" conclusion+

predicate_definition: PREDICATE "(" VAR ("," VAR)* ")" COMMENT  -> define_predicate
premise: quantified_expr COMMENT -> define_premise
conclusion: quantified_expr COMMENT -> define_conclusion

quantified_expr: quantifier VAR "(" expression ")" | expression
quantifier: "{forall}" -> forall | "{exists}" -> exists

expression: bimplication_expr

?bimplication_expr: implication_expr ("{iff}" bimplication_expr)?  -> iff
?implication_expr: xor_expr ("{implies}" implication_expr)?  -> imply
?xor_expr: or_expr ("{xor}" xor_expr)?                -> xor
?or_expr: and_expr ("{or}" or_expr)?                -> or
?and_expr: neg_expr ("{and}" and_expr)?              -> and
?neg_expr: "{not}" quantified_expr                   -> neg 
        | atom

?atom: PREDICATE "(" VAR ("," VAR)* ")" -> predicate 
    | "(" quantified_expr ")" 

// Variable names begin with a lowercase letter
VAR.-1: /[a-z][a-zA-Z0-9_]*/  | /[0-9]+/

// Predicate names begin with a capital letter
PREDICATE.-1: /[A-Z][a-zA-Z0-9]*/

COMMENT: /:::.*\n/

%import common.WS
%ignore WS
\end{lstlisting}
\label{gram:prover9_grammar}




\end{document}
