%%%%%%%% ICML 2024 EXAMPLE LATEX SUBMISSION FILE %%%%%%%%%%%%%%%%%

\documentclass{article}

% Recommended, but optional, packages for figures and better typesetting:
\usepackage{microtype}
\usepackage[final]{graphicx}
\usepackage{subfigure}
\usepackage{booktabs} % for professional tables
\usepackage{enumitem}
\usepackage{caption}
\usepackage{diagbox}
\usepackage{multirow}
\usepackage[utf8]{inputenc}
\usepackage{algorithm}
\usepackage[noend]{algpseudocode}
\usepackage{ifthen}
\newboolean{icml}
\setboolean{icml}{false}

% hyperref makes hyperlinks in the resulting PDF.
% If your build breaks (sometimes temporarily if a hyperlink spans a page)
% please comment out the following usepackage line and replace
% \usepackage{icml2024} with \usepackage[nohyperref]{icml2024} above.
\usepackage{hyperref}


% Attempt to make hyperref and algorithmic work together better:
% \newcommand{\theHalgorithm}{\arabic{algorithm}}

% Use the following line for the initial blind version submitted for review:
% \usepackage{icml2024}

% If accepted, instead use the following line for the camera-ready submission:

\ifthenelse{\boolean{icml}}
{
\usepackage{icml2025}
}
{
\usepackage[accepted]{icml2025}
}

% For theorems and such
\usepackage{amsmath}
\usepackage{amssymb}
\usepackage{mathtools}
\usepackage{amsthm}
\usepackage{xspace}
\usepackage[]{xcolor}
\usepackage{thm-restate}
\usepackage{subcaption}

% For code blocks
\usepackage{listings}
\lstdefinestyle{myGrammarStyle}{
    basicstyle=\scriptsize\ttfamily, % Reduce font size
    commentstyle=\color{gray},
    keywordstyle=\color{blue},
    stringstyle=\color{orange},
    numbers=left, % Line numbers on left
    numberstyle=\tiny\color{gray}, % Line numbers styling
    breaklines=true, % Wrap long lines
    frame=single, % Frame around the code
    framesep=3pt, % Adjust frame separation
    xleftmargin=5pt, % Adjust left margin
    xrightmargin=5pt, % Adjust right margin
    backgroundcolor=\color{blue!4}, % Background color
    tabsize=2, % Tab size
    captionpos=b, % Caption position bottom
    aboveskip=5pt, % Reduce space above the listing
    belowskip=5pt, % Reduce space below the listing
    linewidth=0.9\linewidth, % Set custom line width to less than text width
    escapeinside={(*@}{@*)}, % for escaping to LaTeX
}

% if you use cleveref..
\usepackage[capitalize,noabbrev]{cleveref}

%%%%%%%%%%%%%%%%%%%%%%%%%%%%%%%%
% THEOREMS
%%%%%%%%%%%%%%%%%%%%%%%%%%%%%%%%
\theoremstyle{plain}
\newtheorem{theorem}{Theorem}[section]
\newtheorem{proposition}[theorem]{Proposition}
\newtheorem{lemma}[theorem]{Lemma}
\newtheorem{corollary}[theorem]{Corollary}
\theoremstyle{definition}
\newtheorem{definition}[theorem]{Definition}
\newtheorem{assumption}[theorem]{Assumption}
\theoremstyle{remark}
\newtheorem{remark}[theorem]{Remark}
\DeclareMathOperator*{\argmin}{arg\,min}

% Todonotes is useful during development; simply uncomment the next line
%    and comment out the line below the next line to turn off comments
%\usepackage[disable,textsize=tiny]{todonotes}
\usepackage[textsize=tiny]{todonotes}

% The \icmltitle you define below is probably too long as a header.
% Therefore, a short form for the running title is supplied here:

\ifthenelse{\boolean{icml}}
{
\icmltitlerunning{CRANE: Reasoning with constrained LLM generation}
}
\icmltitlerunning{\icmltitlerunning{CRANE: Reasoning with constrained LLM generation}
}

\begin{document}

\twocolumn[
\ifthenelse{\boolean{icml}}
{
\icmltitle{CRANE: Reasoning with constrained LLM generation}
}
{
\icmltitle{CRANE: Reasoning with constrained LLM generation}

}

% It is OKAY to include author information, even for blind
% submissions: the style file will automatically remove it for you
% unless you've provided the [accepted] option to the icml2024
% package.

% List of affiliations: The first argument should be a (short)
% identifier you will use later to specify author affiliations
% Academic affiliations should list Department, University, City, Region, Country
% Industry affiliations should list Company, City, Region, Country

% You can specify symbols, otherwise they are numbered in order.
% Ideally, you should not use this facility. Affiliations will be numbered
% in order of appearance and this is the preferred way.
\icmlsetsymbol{equal}{*}

\begin{icmlauthorlist}
\icmlauthor{Debangshu Banerjee}{yyy,equal}
\icmlauthor{Tarun Suresh}{yyy,equal}
\icmlauthor{Shubham Ugare}{yyy}
\icmlauthor{Sasa Misailovic}{yyy}
\icmlauthor{Gagandeep Singh}{yyy}
\end{icmlauthorlist}

\icmlaffiliation{yyy}{Department of Computer Science, University of Illinois Urbana-Champaign, USA}
% \icmlaffiliation{comp}{VMware Research, USA}

\icmlcorrespondingauthor{Debangshu Banerjee}{db21@illinois.edu}

% You may provide any keywords that you
% find helpful for describing your paper; these are used to populate
% the "keywords" metadata in the PDF but will not be shown in the document

\icmlkeywords{Machine Learning, ICML}

\vskip 0.3in]

% this must go after the closing bracket ] following \twocolumn[ ...

% This command actually creates the footnote in the first column
% listing the affiliations and the copyright notice.
% The command takes one argument, which is text to display at the start of the footnote.
% The \icmlEqualContribution command is standard text for equal contribution.
% Remove it (just {}) if you do not need this facility.

%\printAffiliationsAndNotice{}  % leave blank if no need to mention equal contribution
\printAffiliationsAndNotice{\icmlEqualContribution} % otherwise use the standard text.

\newcommand{\ones}{\mathbf 1}
\newcommand{\reals}{{\mbox{\bf R}}}
\newcommand{\integers}{{\mbox{\bf Z}}}
\newcommand{\symm}{{\mbox{\bf S}}}  % symmetric matrices

\newcommand{\nullspace}{{\mathcal N}}
\newcommand{\range}{{\mathcal R}}
\newcommand{\Rank}{\mathop{\bf Rank}}
%\newcommand{\Tr}{\mathop{\bf Tr}}
\newcommand{\diag}{\mathop{\bf diag}}
\newcommand{\card}{\mathop{\bf card}}
\newcommand{\rank}{\mathop{\bf rank}}
\newcommand{\conv}{\mathop{\bf conv}}
\newcommand{\prox}{\mathbf{prox}}

\newcommand{\Expect}{\mathop{\bf E{}}}
\newcommand{\var}{\mathop{\bf var{}}}
\newcommand{\Prob}{\mathop{\bf Prob}}
\newcommand{\Co}{{\mathop {\bf Co}}} % convex hull
\newcommand{\dist}{\mathop{\bf dist{}}}
%\newcommand{\argmin}{\mathop{\rm argmin}}
%\newcommand{\argmax}{\mathop{\rm argmax}}
\newcommand{\epi}{\mathop{\bf epi}} % epigraph
\newcommand{\Vol}{\mathop{\bf vol}}
\newcommand{\dom}{\mathop{\bf dom}} % domain
\newcommand{\intr}{\mathop{\bf int}}
%\newcommand{\sign}{\mathop{\bf sign}}

\newcommand{\cf}{{\it cf.}}
\newcommand{\eg}{{\it e.g.}}
\newcommand{\ie}{{\it i.e.}}
\newcommand{\etc}{{\it etc.}}

\newcommand{\todo}{{\bf TODO}}

\newcommand{\bone}{\boldsymbol{1}}
\newcommand{\balpha}{\boldsymbol{\alpha}}
\newcommand{\bbeta}{\boldsymbol{\beta}}
\newcommand{\bdelta}{\boldsymbol{\delta}}
\newcommand{\bepsilon}{\boldsymbol{\epsilon}}
\newcommand{\blambda}{\boldsymbol{\lambda}}
\newcommand{\bomega}{\boldsymbol{\omega}}
\newcommand{\bpi}{\boldsymbol{\pi}}
\newcommand{\bnu}{\boldsymbol{\nu}}
\newcommand{\bphi}{\boldsymbol{\phi}}
\newcommand{\bvphi}{\boldsymbol{\varphi}}
\newcommand{\bpsi}{\boldsymbol{\psi}}
\newcommand{\bsigma}{\boldsymbol{\sigma}}
\newcommand{\btheta}{\boldsymbol{\theta}}
\newcommand{\bzeta}{\boldsymbol{\zeta}}
\newcommand{\bxi}{\boldsymbol{\xi}}
\newcommand{\ba}{\boldsymbol{a}}
\newcommand{\bb}{\boldsymbol{b}}
\newcommand{\bc}{\boldsymbol{c}}
\newcommand{\bd}{\boldsymbol{d}}
\newcommand{\be}{\boldsymbol{e}}
\newcommand{\boldf}{\boldsymbol{f}}
\newcommand{\bg}{\boldsymbol{g}}
\newcommand{\bh}{\boldsymbol{h}}
\newcommand{\bi}{\boldsymbol{i}}
\newcommand{\bj}{\boldsymbol{j}}
\newcommand{\bk}{\boldsymbol{k}}
\newcommand{\bell}{\boldsymbol{\ell}}
\newcommand{\bp}{\boldsymbol{p}}
\newcommand{\br}{\boldsymbol{r}}
\newcommand{\bs}{\boldsymbol{s}}
\newcommand{\bt}{\boldsymbol{t}}
\newcommand{\bu}{\boldsymbol{u}}
\newcommand{\bv}{\boldsymbol{v}}
\newcommand{\bw}{\boldsymbol{w}}
\newcommand{\bx}{{\boldsymbol{x}}}
\newcommand{\by}{\boldsymbol{y}}
\newcommand{\bz}{\boldsymbol{z}}
\newcommand{\bA}{\boldsymbol{A}}
\newcommand{\bB}{\boldsymbol{B}}
\newcommand{\bC}{\boldsymbol{C}}
\newcommand{\bD}{\boldsymbol{D}}
\newcommand{\bE}{\boldsymbol{E}}
\newcommand{\bF}{\boldsymbol{F}}
\newcommand{\bG}{\boldsymbol{G}}
\newcommand{\bH}{\boldsymbol{H}}
\newcommand{\bI}{\boldsymbol{I}}
\newcommand{\bJ}{\boldsymbol{J}}
\newcommand{\bL}{\boldsymbol{L}}
\newcommand{\bM}{\boldsymbol{M}}
\newcommand{\bP}{\boldsymbol{P}}
\newcommand{\bQ}{\boldsymbol{Q}}
\newcommand{\bR}{\boldsymbol{R}}
\newcommand{\bS}{\boldsymbol{S}}
\newcommand{\bT}{\boldsymbol{T}}
\newcommand{\bU}{\boldsymbol{U}}
\newcommand{\bV}{\boldsymbol{V}}
\newcommand{\bW}{\boldsymbol{W}}
\newcommand{\bX}{\boldsymbol{X}}
\newcommand{\bY}{\boldsymbol{Y}}
\newcommand{\bZ}{\boldsymbol{Z}}

% new theorems
% \newtheorem{theorem}{Theorem}
%\newtheorem*{proof}{Proof}

% usepackages
\usepackage{amsmath}
\usepackage{amsfonts}
\usepackage{textcomp} % for \textlangle and \textrangle macros
\newcommand{\qdist}[1]{\ifmmode\langle#1\rangle\else\textlangle#1\textrangle\fi}
\usepackage{xcolor}
\usepackage{algorithm} % for algorithms
\usepackage{algpseudocode} % for pseudocode
\usepackage{comment} % for large comments
\usepackage{bbm}
\usepackage{dsfont}
\usepackage{subfigure}
\usepackage{bm}
\usepackage{booktabs} % For better table lines
\usepackage{array} % For better column formatting
%\usepackage{appendix}
%\usepackage[english]{babel}
%\usepackage{amsthm}
\usepackage{graphicx} % for graphs




%hi
\begin{abstract}
Code generation, symbolic math reasoning, and other tasks require LLMs to produce outputs that are both syntactically and semantically correct. Constrained LLM generation is a promising direction to enforce adherence to formal grammar, but prior works have empirically observed that strict enforcement of formal constraints often diminishes the reasoning capabilities of LLMs. In this work, we first provide a theoretical explanation for why constraining LLM outputs to very restrictive grammars that only allow syntactically valid final answers reduces the reasoning capabilities of the model. Second, we demonstrate that by augmenting the output grammar with carefully designed additional rules, it is always possible to preserve the reasoning capabilities of the LLM while ensuring syntactic and semantic correctness in its outputs. Building on these theoretical insights, we propose a reasoning-augmented constrained decoding algorithm, \Tool, which effectively balances the correctness of constrained generation with the flexibility of unconstrained generation. 
Experiments on multiple open-source LLMs and benchmarks show that \Tool{} significantly outperforms both state-of-the-art
constrained decoding strategies and standard unconstrained decoding, showing up to \upto{} points accuracy improvement over baselines on challenging symbolic reasoning benchmarks GSM-symbolic and FOLIO.

\end{abstract}

\documentclass[../main.tex]{subfiles}
\graphicspath{{../images/}}
\makeatletter
\def\input@path{{../images/}}
\makeatother
\begin{document}
\section{Introduction}
\begin{figure}
\centering
\begin{tikzpicture}
\node[inner sep=0pt] (ws) at (0, 0) {
\includegraphics[height=.4\textwidth, trim={10cm 0 10cm 0},clip]{world_space.png}};
\node[inner sep=0pt] (cs) at (6,0) {\includegraphics[height=.4\textwidth, trim={10cm 1cm 10cm 4cm},clip]{conf_space.png}};
\end{tikzpicture}
\vspace{-5pt}
\label{fig:pbrm_intro}
\caption{\textbf{Left}: Shows world space obstacles as grey spheres. Robots start and goal configuration is colored red and green, respectively. Configurations along the computed path are colored transparent blue. \textbf{Right:} Mapped world space scenario to configuration space. Obstacle region is the grey mesh. Red spheres are collision-free regions computed by the neural SCDF. The optimized shortest path in the convex corridor is the blue curve.}
\vspace{-25pt}
\end{figure}
Motion planning is the problem of finding a collision-free trajectory that connects a given start and goal configuration. The planning takes place in the configuration space of the robot. For single body robots, like mobile robots or drones, the configuration space and the world space are usually the same. This simplifies the planning, since explicit obstacle representations are available which enables geometrical tools like separating hyperplanes, smallest distance to obstacles etc., to be used when designing motion planning algorithms. For multi-body robots like manipulators, the situation is completely different. The world space obstacles are usually mapped to non-convex regions, and to make the problem even harder, the mapping is usually not known. Forming explicit representations of the obstacle region in the configuration space is usually too expensive or intractable. Despite all of this, sampling based planners are used with great success, which mainly is due to their use of implicit representations of the obstacle region. The basic idea is to construct a graph in the configuration space that covers and connects the collision-free region. From this graph, a path can be extracted that connects a given start and goal configuration. The approach is computationally expensive, since the graph is constructed with the smallest geometrical building block available, points, which represents a collision-check. Furthermore, the extracted paths from the graph are non-smooth and jagged due to the stochastic nature of the approach. This adds an additional post-processing step to the process, where the paths are shortcutted and smoothened, before the path can be used for tracking. Clearly a lot of time is invested to form this graph and produce smooth paths. Thus, if the obstacles start to move, then all of this work is done in no use, since all points that make up this graph need to be re-verified, which is simply too time consuming to be done in real time.
\\\\
In this work, we want to address the existing drawbacks of the sampling based planners. Our main contribution is an improved motion planner where each vertex in the graph covers a collision-free region in the form of a sphere instead of a point and where the edges are formed with neighboring intersecting spheres. This representation has the advantage of instead of returning piecewise linear paths, returning a sequence of overlapping spheres, i.e. a convex corridor, that connects a given start and goal configuration, illustrated in Figure \ref{fig:pbrm_intro}. This convex corridor allows us to use convex optimization to produce smooth trajectories, instead of computationally expensive post-processing methods. The representation further allows us to estimate the coverage of the collision-free space, which gives us awareness and feedback in the offline roadmap construction phase. Finally, our representation is simple to adapt to moving obstacles, simply requery for the new radii and recheck for intersections. 
\\\\
The spherical collision-free regions are formed using a signed distance function (SDF), which is a function that returns the smallest distance from an arbitrary point to the boundary of an obstacle. As the name implies, the distance is signed, thus if the point is inside the obstacle it is negative otherwise positive. If the distance is positive, a sphere with radius equal to the distance is guaranteed to cover a collision-free region. Using an SDF in motion planning is not new, but what is novel about our approach is that we express the distance in the configuration space instead of the world space and by doing so allows us to form these convex collision-free regions. We refer to the resulting SDF as a signed configuration distance function (SCDF). Computing an SCDF analytically is non-trivial, our approach is therefore to parameterize the SCDF with a deep neural network and learn the mapping by supervised learning. Our resulting neural SCDF can compute distances for different parameter values of obstacle shapes and we also show how multiple distances can be combined, thus making our approach flexible.
\section{Related work}
Motion planning algorithms can roughly be divided into three families, grid-based, sampling based and optimization based methods. Grid-based methods (GBM) discretize the planning space from which a graph is then compiled. A standard search method is A$^\star$ \citep{a_star}, which is classified as an \textit{informed} search method, since it employs a heuristic function to speed up the search. A$^\star$ guarantees to return an optimal path at the level of discretization used. GBMs usually discretize the planning space by a regular lattice and this limits the GBMs to problems with low dimensionality due to the curse of dimensionality. Thus, GBMs are usually limited to single-body robots where the degrees of freedom (DOF) are low. To overcome the inherent scaling problem with the GBMs, stochastic methods are usually used for multi-body robots. These methods are termed as sampling-based methods (SBM) and core members within this family are the rapidly-exploring random trees (RRT) \citep{rrt} and the probabilistic roadmap (PRM) \citep{prm}. RRT grows a tree from the start configuration and explores the collision-free region in a rapid way until it is able to connect to the goal region. RRT is usually improved by bi-directional planning \citep{rrt_connect}, i.e. an additional tree is grown from the goal configuration and the trees are tested for connection after any tree has been expanded. RRT is a single-query method, thus it searches for a path from scratch each time it is queried. Contrary to this, PRM is a multi-query method, which solves for multiple queries without starting from scratch. PRM does this by creating a roadmap (graph) that covers the collision-free space as an offline step. The graph is then used to solve for multiple queries. PRMs are used in cases where the environment does not change since the extra offline step is too computationally costly and needs to be re-done if the environment is changed. In our work, we address this inherent issue by using a different roadmap representation. Our vertices in the graph cover a collision-free region in the form of spheres and we form the edges by checking for intersecting spheres. If something in the environment changes, we recompute the spheres radii and recheck the intersections, without relying on collision detection. We use a trained neural network to compute the sphere radius, therefore querying for the radius can be done fast, hence our representation enables the PRM for dynamic environments.
\\\\
In the recent decades, optimization based methods (OBM) \citep{chomp, schulman, itomp, stomp} have been introduced as an alternative to SBM for multi-body robots. Like the SBM, the OBMs scale well to higher dimensional problems and produce smoother motion. It is common to use a SDF in the optimization since it is a smooth function, thus enabling gradient-based methods. However, the standard way of expressing the SDF is in world space. The distance therefore needs to be mapped to the configuration space by the forward kinematics. This mapping makes the optimization problem a non-linear program (NLP), which is computationally expensive to solve. Recently, a different approach has been proposed. In \cite{mp_gcs} motion planning is formulated as a convex optimization problem by using the graph of convex sets framework \citep{gcs}. The underlying idea is to decompose the collision-free space into intersecting convex sets from which a convex optimization problem is formulated. In cases where an explicit representation of the obstacles in the configuration space exists, like for single-body robots, creating collision-free convex regions can be done fast \citep{iris}. For multi-body robots, this is non-trivial. Existing work does this successfully \citep{iris_nlp, iris_c} by an optimization based approach, but the methods are still too time consuming to be used in the presence of moving obstacles. Our approach is instead to use deep learning to learn an SDF expressed in the configuration space. With this, we can query for shortest distances to the collision boundary, which allows us to expand spherical regions which are collision-free. Our approach is fast and therefore enables our suggested roadmap planner to be used in dynamic environments.
\\\\
Recent research has focused on learning collision detection \citep{fk_kernel_distance, diffco, graphdistnet} by predicting the signed distance between the robot links and the surrounding obstacles in the world space. The learned SDF is used in trajectory optimization but since the distance is expressed in the world space, the problem becomes an NLP and therefore takes a long time to solve. We take a novel approach and suggest to instead express the signed distance in the configuration space. This allows us to improve the PRM at the same time as it enables convex optimization for trajectory optimization, which runs faster and is more reliable than NLP solvers. In \cite{cspf} a learned signed distance function in the configuration space is proposed similar to our approach. However, their approach is restricted to point cloud representations, while we propose to represent the obstacles as parameterized geometric shapes, e.g. spheres. Furthermore, we also show how to use our learned SCDF to improve an existing roadmap planner.
\section{Problem formulation}
A robot is located in the world space, $\W \subset \R^3 $. The unique location of the robot is given by its configuration $\q \in \C$, where $\C$ is the configuration space. The set of points covered by the robots bodies at a certain configuration is expressed as $\B(\q) \subset \W$. The robot is surrounded by $\NrObst$ obstacles $\O = \bigcup_{i=1}^{\NrObst} \O_i$, where  $\O_i \subset \W$. The representation of the obstacle in the configuration space is the set $\C\O_i = \{\q \in \C \: |\: \B(\q) \cap \O_i \neq \emptyset \}$. The obstacle space is formed as $\Co = \bigcup_{i=1}^{\NrObst} \C \O_i$. The complement is referred to as the free space, $\Cf = \C \setminus \Co$. The path planning problem is a tuple, ($\Cf$, $\qStart$, $\qGoal$), where we want to connect a query pair, consisting of a start, $\qStart$, and goal configuration, $\qGoal$, with a geometric path, $\q(s): [0, 1] \mapsto \Cf$, such that $\q(0)=\qStart$ and $\q(1)=\qGoal$, or report correctly when such a path does not exist.
\end{document}


\section{Preliminary}

\paragraph{Notation} Consider a sentence of $T$ tokens $\vx=\{\vx_1,\ldots, \vx_T\}\in\gX$, and let $P$ be the unknown target language distribution, $\tilde P(\vx)$ be the empirical distribution of the training data (which is an approximation of $P$), and $Q$ be the distribution of our model at hand. Since our paper is also closely related to RLHF, we will also use $\pi$ to represent the distributions. In particular, we sometimes write $\pi_\theta$ for a distribution that is parameterized by $\theta$, where $\theta$ is usually the set of trainable parameters of the LLM; we write $\pr$ for a reference distribution that should be clear given the context. The next token prediction loss is minimizing the forward-KL between $P$ and $Q$. 





\section{Proposed Method}
\textbf{Problem Statement: } Let $\mathcal{X}$ and $\mathcal{Y}$ denote the input and output spaces, respectively, and $D = \{(x_i, y_i)\}_{i=1}^n$ the dataset, where $x_i \in \mathcal{X}$ and $y_i \in \mathcal{Y}$ are the $i^{th}$ question-answer pair. For each $x_i$, the goal is to generate a response $\hat{y}_i$ that maximizes the overall accuracy. The goal is to achieve this using a pool  of $N$ pretrained foundational LLMs without additional training or fine-tuning.


\begin{comment}
    

Let $\mathcal{X}$ and $\mathcal{Y}$ denote input and output spaces respectively and  $\mathcal{D}= \{(x_i,y_i)\}_{i=1}^n$ denote Dataset, where $x_i \in \mathcal{X}$ and $y_i \in \mathcal{Y}$ is the $i^{th}$ question-answer(QA) pair. For each $x_i$ we want to generate a response $\hat{y}_i$  such that overall accuracy denoted by $\frac{1}{n}\sum_{i=1}^n\mathbf{I}(y_i=\hat{y}_i)$ is maximized.  We want to maximize this using ensemble of $N$ LLMs without any additional training or fine-tuning.
\end{comment}
\renewcommand{\algorithmicrequire}{\textbf{Input:}}
\renewcommand{\algorithmicensure}{\textbf{Output:}}

\begin{algorithm}[t]
    \caption{Components of UAF - SELECTOR and FUSER}
    \label{alg:selector}
    
    \begin{algorithmic}
    \REQUIRE $D_{val}$, Pool of LLMs $\mathcal{M}= \{M^j\}_{j=1}^{N}$, Uncertainty function $U_f(.,.,.)$, Ensemble size $K$, Test data point $x_{test}$
    \ENSURE Test data response $\hat{y}_{test}$
    \newline

    \STATE \textbf{procedure }SELECTOR ($D_{val}, \mathcal{M}, U_f, K$)
    \FOR{$M^j \in \mathcal{M}$}
    %\STATE $L_j, U_j = \emptyset,\emptyset$
    \FOR{$(x_i, y_i) \in D_{val}$}
    \STATE $\hat{y}_i^j = M^j(x_i)$   \quad;\quad  $s_i^j = \mathbf{1}(\hat{y}_i^j == y_i)$
    \STATE $u_i^j = U_f(M^j, x_i, \hat{y}_i^j)$
    %\STATE $L_j \leftarrow L_j \cup IsCorrect_i^j$   \quad;\quad    $U_j \leftarrow U_j \cup u_{val_i}^j$
    \ENDFOR

    \STATE $Acc_j = \frac{1}{|D_{val}|}\sum_i s_i^j$  ;\quad $SAH_j = ROC\_AUC\_score(\{s_i^j,u_i^j\}_i)$
    \STATE $Cscore_j = Acc_j \times SAH_j$
    \ENDFOR
    \STATE \textbf{return:} $\mathcal{M}^{sel} =  TopK (\{Cscore_j\}_{j=1}^N)$,  \hfill\COMMENT{//Top $K$ LLMs}
    \STATE \quad  \quad      $Acc^{sel} = \{Acc_j| j \in \mathcal{M}^{sel}\}$ \hfill\COMMENT{//Accuracy of selected $K$ LLMs}
    %\STATE $j_1, j_2, \dots, j_K = \arg\max_{j \in \{1, 2, \dots, N\}} Cscore_j$ 
    %\ENSURE $j_1, j_2, \dots, j_K$    \hfill\COMMENT{//Indices of Top K llms}
    \newline
\STATE \textbf{procedure } FUSER ($x_{test}, \mathcal{M}^{sel}, Acc^{sel}, U_f, K$)
\FOR{$M^k \in \mathcal{M}^{sel}$}
\STATE $\hat{y}_{test}^k = M^k(x_{test})$
\STATE $u_{test}^k = U_f(M^k, x_{test}, \hat{y}_{test}^k)$
\ENDFOR
\STATE $\hat{y}_{test} = \hat{y}_{test}^{k^*} \text{, where } k^* = argmax_{k \in \{1,\dots K\}} Acc_k \times (1-u_{test}^k)$
\STATE \textbf{return:} $\hat{y}_{test}$

\end{algorithmic}
\end{algorithm}

\subsection{Uncertainty Aware Fusion (UAF)}\label{sec:uaf}
Figure \ref{fig:fuser} provides an overview of our UAF framework. At a high level, UAF consists of two modules: SELECTOR and FUSER. Given a specific task, the SELECTOR selects the top $K$ LLMs from a pool of $N$ LLMs based on performance metric. FUSER then combines the outputs of these $K$ LLMs to produce the final response.

\subsubsection{SELECTOR}
Given a pool of $N$ LLMs denoted by $\mathcal{M}$, SELECTOR selects $K$ LLMs (where $K<N$) to optimize computational efficiency and enhance overall factual accuracy by pruning underperforming LLMs. Selection is based on two criteria: (1) task-specific accuracy and (2) self-assessment of hallucinations based on an given uncertainty measure. Given a specified uncertainty measure $U_f(\cdot)$, and  a validation set $D_{val}$, we prompt each LLM with input $x_i$,   obtaining response  $\hat{y}_i^j$  and corresponding uncertainty score $u_i^j$ from the $j^{th}$ LLM $M^j$.  We compute the accuracy $Acc_j$ of $M^j$ as the fraction of correct responses. We also measure the LLMs ability for  self-assessment of hallucinations $SAH_j$ as the area under the ROC curve for the binary classification of truthful vs. hallucinatory responses using uncertainty scores. We then compute a combined score $Cscore_j = Acc_j \times SAH_j$ for each LLM. The top K models with the highest combined scores are selected greedily, where K is a   hyperparameter tuned for specific tasks.

\begin{comment}
To achieve this we sample $10\%$ of examples from $D$ and denote it as  $D_{val}$. We use the rest denoted by $D_{test}$ for evaluation. For each $(xval_i,yval_i)\in D_{val}$ we prompt each of the $N$ LLMs with $xval_i$. Let $\hat{yval}_i^j$ denote the response and $u_i^j$ its corresponding uncertainty score computed with a particular uncertainty method for $j^{th}$ LLM. Accuracy of $j^{th}$ LLM, denoted by $Acc_j$ is defined as percentage of correct responses. We also measure area under the receiver operator characteristic curve  $Unc\_auroc_j$ of $j^{th}$ LLM, by measuring the performance of classifying its own correct(truthful) from incorrect(hallucinatory) responses by varying the thresholds on the set of uncertainty scores $\{u_i^j\}_{i=1}^{nval}$. For each $j^{th}$ LLM we compute a combined score $Cscore_j = Acc_j*Unc\_auroc_j$. We use greedy method to select $K$ LLMs based on top $K$ highest $Cscore$. Here $K$ is the hyperparameter which we tune using $D_{val}$. Algorithm \ref{alg:selector} presents the pseudo-code for this method.
\end{comment}

\subsubsection{FUSER}
Given the selected ensemble of $K$ models $\mathcal{M}^{sel}$, 
%with respective accuracies $\{Acc_1, \dots, Acc_K \}$,
for each unseen example $x_{test}$, we generate outputs from the $K$ LLMs denoted by $\{\hat{y}_{test}^1,\dots,\hat{y}_{test}^K\}$ along with the  corresponding instance-specific uncertainty scores.  denoted by $\{u_{test}^1,\dots,u_{test}^K\}$. 
While there can be several fusion strategies, since we are dealing with natural language responses, the simplest one is to example-specific selection from the candidate outputs, i.e., 
\[
\hat{y}_{test} = \hat{y}_{test}^{k^*}, \quad \text{where} \quad {k^*} = \arg\max_{k \in \{1, \dots, K\}} f^k.
\]
Selection criterion $f^k$ could be based on validation set accuracy alone, inverse uncertainty or some combination of both such as $\text{Acc}_k \cdot (1 - u_{test}^k)$  or $\frac{\text{Acc}_k}{u_{test}^k}$.
The first strategy essentially reduces the ensemble to a single most accurate model, while the second one elevates the most confident one. However, both of these approaches are sub-optimal compared to combined criteria, specifically 
\[
f^k = \text{Acc}_k \cdot (1 - u_{test}^k),
\]
which yields the best performance. Experiments with  other combined selection criteria shows similar behavior to the aforementioned one and hence,  we omit the results for brevity.


%There are multiple ways to choose the final answer from these candidate responses - $\hat{y}_{test} = \hat{y}_{test}^j where j = argmax_{k \in \{1,\dots K\}}f^k$. We can define $f^k$ as $Acc_k$ , $1-u_{test}^k$ ,$Acc_k*(1-u_{test}^k)$ or $Acc_k/u_{test}^k$. First strategy essentially reduces the ensemble to a single model having highest accuracy, while second one picks the final answer as the one with the least uncertainty. Both of these are suboptimal compared to  $f_k = Acc_k*(1-u_{test}^k)$ with which we experiment here in this work. Alternative strategy $f_k = Acc_k/u_{test}^k$ shows similar behaviour as above and we omit it's analysis for brevity.

%Although there are multiple ways to aggregate these candidate responses we propose a simple aggregation technique where we choose the final answer as the one with the least uncertainty ie. $\hat{y}_{test} = \hat{y}_{test}^j where j = argmin_{k \in \{1,\dots K\}}u_{test}^k$. Algorithm \ref{alg:selector} presents the pseudo-code of UAF components.



One limitation of this study is that it only evaluated LLaVA as the target Vision Language Model (VLM), which may limit the generalizability of the findings to other models. Additionally, the alignment of visual attention heatmaps for non-existing objects was not assessed, indicating that further analysis is needed in this area. 

Moreover, the experiments were conducted solely using the MSCOCO dataset, and future work should expand the evaluation to include additional datasets to ensure the robustness and broader applicability of the results. Furthermore, since datasets that contain both questions and corresponding answers alongside matching segmentation data, which can be used to evaluate object hallucination, are scarce, it may be necessary to develop such datasets.

\section{Experiments: Planning outperforms Heuristics}
\label{sec:experiment}

We begin our empirical demonstrations by showcasing the effectiveness of our planning framework on both synthetic and real datasets. We focus on the simplest planning algorithm, 1-step lookaheads (Algorithm~\ref{alg:complete}), and show that even basic planning can hold great promise. 
We illustrate our framework using two uncertainty quantification modules---GPs and 
\ensembles/ \ensembleplus. 

Throughout this section, we focus on evaluating the mean squared error of 
a regression model $\model$,  and develop adaptive policies that minimize uncertainty on $g(f)$ defined in~\eqref{eqn:l2-g-f}.
When GPs provide a valid model of uncertainty, 
our experiments show that our planning framework significantly outperforms other baselines. 
We further demonstrate that our conceptual framework extends to deep learning-based uncertainty quantification methods such as  \ensembleplus while highlighting computational challenges that need to be resolved in order to scale our ideas. 
For simplicity, we assume a naive predictor, i.e., $\psi(\cdot) \equiv 0$. However, we emphasize that this problem is just as complex as if we were using a sophisticated model $\psi(.)$. The performance gap between the algorithms 
primarily depends
on the level  of uncertainty in our prior beliefs.

To evaluate the performance of our algorithm, we benchmark it against several baselines. 
%Active learning baselines use an acquisition function $\ac$ to select points that have the highest   function value: $X\opt_t \in \argmax_{X \in \xpoolj{t}} \ac({X})$ at every step $t$. These methods may also need an UQ module, which we simply use the same UQ module as in our algorithm, and it  outputs $V(X)$ that measures the the uncertainty of each point $X \in \xpoolj{t}$.
Our first set of baselines are from active learning~\citep{AggarwalKoGuHaPh14}:
\\ % \noindent\textbf{Active Learning Heuristics:} 
\textbf{(1)} 
\textsf{Uncertainty Sampling (Static):}  In this approach, we query the samples for which the model is least certain about. Specifically, we estimate the variance of the latent output $f(X)$ for each $X \in \xpool$ using the UQ module and select the top-$K$ points with the highest uncertainty. \\
\textbf{(2)} \textsf{Uncertainty Sampling (Sequential):} This is a greedy heuristic that sequentially selects the points with the highest uncertainty within a batch, while updating the posterior beliefs using pseudo labels from the current posterior state. Unlike \textsf{Uncertainty Sampling (Static)}, this method takes into account the information gained from each point within batch, and hence tries to diversify the selected points within a batch. 

 
We also compare our approach to the  \textbf{(3)} \textsf{Random Sampling}, which selects each batch uniformly at random from the pool. Additionally, we compare solving the planning problem using  \textsf{REINFORCE}-based policy gradients with   $\mathsf{Smoothed\text{-}Autodiff}$ policy gradients.\footnote{Our code repository is available at
  \url{https://github.com/namkoong-lab/adaptive-labeling}.}
%Detailed experimental setups are provided in Section \ref{sec:details-experiments}.

%We repeat all experiments with 10 random seeds.




\begin{figure}[t]
\centering
\begin{minipage}[b]{0.49\textwidth}
\centering
\includegraphics[width=\textwidth, height=5cm]{figures/original_scale/Var_of_l_2_loss.pdf}
\caption{(Synthetic data) Variance of mean squared loss evaluated through the posterior belief $\mu_t$ at each horizon $t$. This is the objective that policy gradient methods like \textsf{REINFORCE} and $\ouralgo$ optimizes. 1-step lookaheads are surprisingly effective even in long horizons.}
\label{fig:var-l2-sim}
\end{minipage}
\hfill
\begin{minipage}[b]{0.49\textwidth}
\centering \includegraphics[width=\textwidth, height=5cm]{figures/original_scale/Error_of_estimated_model_l_2_loss.pdf}
\caption{(Synthetic data) Error between MSE calculated based on collected data $\mc{D}^{0:T}$ vs. population oracle MSE over $\mc{D}_{\rm eval} \sim P_X$. Reducing uncertainty over posteriors directly leads to better OOD evaluations. 1-step lookaheads significantly outperform active learning heuristics in small horizons.}
\label{fig:mean-l2-sim}
\end{minipage}
%\caption{Simulated data for GPs}
%\label{fig:both_plots}
\end{figure}

\subsection{Planning with Gaussian processes}
\label{sec:experiment-plan-GP}
We now briefly describe the data generation process for the GP experiments,  deferring a more detailed discussion of the dataset generation to Section~\ref{sec:details-experiments}. 
We use both the synthetic data and the real data to test our methodology.
For the \emph{simulated data},  we construct a setting where the general population is distributed across \emph{51 non-overlapping clusters} while the initial labeled data $\dtrain$ just comes from one cluster. In contrast, both $\dpool \defeq (\xpool,\ypool),\deval \defeq (\xeval,\yeval)$ are generated   from all the clusters. 
We begin with a low-dimensional scenario, generating a one-dimensional regression setting using a GP. %Gaussian Process (GP).
Although the data-generating process is not known to the algorithms,  we assume that the GP hyperparameters are known to all the algorithms
to ensure fair comparisons. This can be viewed as a setting where our prior is well-specified, allowing us to isolate the effects
of different policy optimization approaches
 without any concerns about the misspecified priors. We select $10$ batches, each of size $K=5$ across $T = 10$ time horizons.

To examine the robustness of our method against the distributional assumptions made  in the simulated case, we then move to a real dataset where the correct prior is not known. We simulate selection bias from the eICU dataset~\citep{PollardJoRaCeMaBa18}, which contains real-world patient data with in-hospital mortality outcomes. 
We conduct a $k$-means clustering to generate 51 clusters and then select data from those clusters. We view this to be a credible replication of practice, as severe distribution shifts are common due to selection bias in clinical labels.  To convert the binary mortality labels into a regression setting, we train a  random forest classifier and fit a GP on predicted scores, which serves as the UQ module for all the algorithms. As before, the task is to select 10 batches, each consisting of 5 samples, across 10 time horizons.

 In Figures~\ref{fig:var-l2-sim} and~\ref{fig:mean-l2-sim}, we present results for the simulated data. 
Figure~\ref{fig:var-l2-sim} shows the variance of $\ell_2$ loss, and Figure~\ref{fig:mean-l2-sim} presents the error in the estimated $\ell_2$ loss using $\mu_t$ (relative to true $\ell_2$ loss, that is unknown to the algorithm). 
As we can see from these plots, our method one-step lookahead  gives substantial improvements  over active learning baselines and random sampling. In addition,
compared to the one-step lookahead planning approach using \textsf{REINFORCE}-based policy gradients, 
we observe that $\mathsf{Smoothed\text{-}Autodiff}$-based policy gradients provide significantly more robust performance over all horizons.

In Figures~\ref{fig:var-l2-real}~and~\ref{fig:mean-l2-real}, we observe similar findings on the eICU data. We see that planning policies (\textsf{REINFORCE} and $\mathsf{Smoothed\text{-}Autodiff}$) consistently outperform other heuristics by a large margin.  Active learning baselines perform poorly in these small-horizon batched problems and can sometimes be even worse than the random search baselines.  Overall, our results show the importance of careful planning in adaptive labeling for reliable model evaluation. 

We offer some intuition as to why one-step lookahead planning may outperform other heuristic algorithms. 
 First,  \textsf{Uncertainty sampling (Static)} while myopically selects the
 top-$K$ inputs with the highest uncertainty, it fails to consider 
the overlap in information content among the ``best” instances; see \citep{AggarwalKoGuHaPh14} for more details. 
In other words,  it might acquire points from the same region with high uncertainty while failing to induce diversity among the batch.
Although \textsf{Uncertainty Sampling (Sequential)} somewhat addresses the issue of information overlap, a significant drawback of 
this algorithm
is the disconnect between the objective we aim to optimize and the algorithm. For example, it might sample from a region with high uncertainty but very low density. 

\begin{figure}[t]
\centering
\begin{minipage}[b]{0.48\textwidth}
\centering
\includegraphics[width=\textwidth, height=5cm]{figures/original_scale/Var_of_l_2_loss_real.pdf}
\caption{(Real-world eICU data) Variance of mean squared loss evaluated through the posterior belief $\mu_t$ at each horizon $t$. Even 1-step lookaheads are extremely effective planners, and auto-differentiation-based pathwise policy gradients provide a reliable optimization algorithm based on low-variance gradient estimates.}
\label{fig:var-l2-real}
\end{minipage}
\hfill
\begin{minipage}[b]{0.48\textwidth}
\centering \includegraphics[width=\textwidth, height=5cm]{figures/original_scale/Error_of_estimated_model_l_2_loss_real.pdf}
\caption{(Real-world eICU data) Error between MSE calculated based on collected data $\mc{D}^{0:T}$ vs. population oracle MSE over $\mc{D}_{\rm eval} \sim P_X$. Reducing uncertainty over posteriors directly leads to better OOD evaluations. Our method significantly outperforms active learning-based heuristics, and random sampling.}
\label{fig:mean-l2-real}
\end{minipage}
%\caption{Real data for GPs}
\end{figure}
 
%\vspace{-1.5cm}
% \begin{wrapfigure}{r}{.32\columnwidth}
%   \vspace{-.5cm} 
%   \centering
% \includegraphics[scale=.29]{figures/Var of l2l_2 loss.pdf}
%   \vspace{-0.2cm}
%   \caption{Results of GP}
% \label{fig:var-l2-gp}
%   \vspace{-0.1cm}
% \end{wrapfigure}


% Attempts have been made  in the past to address these  drawbacks heuristically  (see \citep{AggarwalKoGuHaPh14}). We give a unified computational framework while approaching the problem in a more principled manner and solving it more optimally.




\subsection{Planning with  neural network-based uncertainty quantification methods ($\ensembleplus$)}


We now provide a proof-of-concept that shows the generalizability of our conceptual framework  to the deep learning-based UQ modules, specifically focusing on $\ensembleplus$ due to their previously observed superior performance~\citep{OsbandWenAsDwIbLuRo23}. Recall that implementing our framework with deep learning-based UQ modules  requires us to retrain the model across multiple possible random actions $\bm{a}(\theta)$ sampled from the current policy $\pi_\theta$.
This requires significant computational resources, in sharp contrast to the GPs where the posteriors are in closed form and can be readily updated and differentiated. 

Due to the computational constraints, we test $\ensembleplus$ on a toy setting to demonstrate the generalizability of our framework. We consider a setting where the general population consists of four clusters, while the initial labeled data only comes from one cluster. Again we generate data using GPs.  The task is to select a batch of 2 points in one horizon. We detail the $\ensembleplus$ architecture in Section \ref{sec:details-experiments}, and we assume prior uncertainty to be large (depends on the scaling of the prior generating functions). 
The results are summarized in the Table~\ref{tab:UQ_ensemble}.

% \begin{table}[H]
% \vspace{-10pt}
% \caption{Performance under \ensembleplus as UQ module}
%     \centering
%     \begin{tabular}{|m{3cm}|m{2.5cm}|m{2cm}|} 
%     \hline
%       Algorithm   & Variance of $\loss_2$ loss estimate & Error of $\loss_2$ loss estimate  \\ \hline Random Sampling 
%          & $1710.9 \pm 1352.1$ & $8.67\pm6.62$ 
%       \\ \hline \ouralgo & $1.30 \pm 0.68$ & $0.91\pm0.25$ \\ \hline
%     \end{tabular}
%     \label{tab:UQ_ensemble}
%     %\vspace{-10pt}
% \end{table}




\begin{table}[h]
\vspace{-10pt}
\caption{Performance under \ensembleplus as the UQ module}
\centering
\begin{tabular}{|l|l|l|}
\hline
Algorithm   & Variance of $\loss_2$ loss estimate & Error of $\loss_2$ loss estimate  \\
\hline
\textsf{Random sampling} & 7129.8 $\pm$ 1027.0 & 136.2 $\pm$ 8.28 \\ \hline
\textsf{Uncertainty sampling (Static)} & 10852 $\pm$ 0.0 & 162.156 $\pm$ 0.0 \\ \hline
\textsf{Uncertainty sampling (Sequential)} & 8585.5 $\pm$ 898.9 & 144 $\pm$ 6.93 \\ \hline
\textsf{REINFORCE} & 1697.1 $\pm$ 0.0 & 45.27 $\pm$ 0.0 \\ \hline
\ouralgo & 1697.1 $\pm$ 0.0 & 45.27 $\pm$ 0.0 \\ \hline
\end{tabular}
%\caption{Comparison of different algorithms based on variance   and   error in $\ell_2$ loss estimation with Ensemble $+$ as the UQ module. Our results demonstrate that {\ouralgo} and REINFORCE outperformthe other active learning based heuristics, confirming the benefits of our MDP formulation for the adaptive labeling problem, as also demonstrated in Section 4.\\
%\footnotesize{Experimental details: We use Gaussian Processes as our data generating process, GP parameters are the same as in Section D.3.  The task is to select a batch of 2 points along one horizon.The marginal distribution $p_X$ has 4 \textit{non-overlapping} clusters. Initial data comes from one cluster, while pool and evaluation points comes from all the clusters. We have $20$ initial labeled data points, $10$ pool points, and $252$ evaluation points.  Training procedures are similar to the one in Section D.3.} }
\label{tab:UQ_ensemble}
\end{table}



% We faced  issues in scaling up these experiments which will be our focus in the future. 





% \begin{itemize}
%     \item Posteriors should be consistent. Two dimensions: even with less training,  
%     \item the inference should be  fast enough
% \end{itemize}


% Potential research directions for uncertainty quantification

% In this section we consider a simple setting We consider a simpler setting and 


% For synthetic dataset generation, we use ...... For real datasets, we use ...... We compare our methodolgy to several baselines ()    This Section is structured as follows:
% \begin{itemize}
%     \item \textbf{GPs, square loss objective} (Section \ref{}): 
%     %the broad aim of the experiments  in this section is to isolate the performance of our methodology without any concerns for the inefficiencies induced due to a mis-specified prior or imperfect posterior inference. To accomplish this we generate synthetic datasets using GPs (detailed later). We use the well specified prior (GPs - with same hyperparameter setting) as our UQ module.   
%      As GPs provide differentaible posterior inference - any errors induced due to imperfect posterior updates are also isolated. We note that under this setting
%      \item In Section\ref{} we demonstrate why our methodology performs better than other baselines - by devising various synthetic experiments ()
%     \item  \textbf{UQ Benchmarking }(Section \ref{}): Before diving into the experiments using $\ensembleplus$ and ENNs,  we showcase our benchmarking experiments in Section \ref{}. We use real datasets We observe that ENNs perform better
%      \item \textbf{Ensemble $+$}, objective: recall, accuracy
%     \item \textbf{ENN}, objective: recall, accuracy
% \end{itemize}




% In Section {}, we test 
% \subsection{Experimental details}

% \begin{itemize}
%     \item UQ methodologies - GPs, ENNs
%     \item Objectives - Recall,  ATE
%     \item Datasets - ATE-synthetic datasets, Recall-synthetic, real datasets
%     \item Baselines - 
%     \begin{itemize}
%         \item Random sampling
%         \item Active learning - Uncertainty based sampling - In regression setting almost all of the 
%         \item Myopic greedy - Greedy Batch based sampling
%         \item Policy Gradient
%     \end{itemize}
    
% \end{itemize}

% \subsection{Experiments}
%     \begin{itemize}
%     \item GPs with square loss
%     \item Benchmarking ENN
%         \item ENNs with ATE
%         \item ENNs with Recall
%     \end{itemize}

% \subsection{Benefits over other algorithms - intuition and experiments}

%Active learning - Myopic greedy / Don't rely on the objective rather some entropy version.


%%% Local Variables:
%%% mode: latex
%%% TeX-master: "main"
%%% End:


\section{Related Work} \label{sec:related}

% \textbf{Adversarial Attack}
\textbf{Attacks on SLAM.} 
%With the rise of machine learning, 
The robustness of computer vision systems is being actively investigated. With the emergence of adversarial images in the digital domain by adding optimized noise directly to images~\cite{szegedy2013intriguing,carlini2017towards}, researchers find that such attacks also exist physically in the real world \cite{eykholt2018robust,song2018physical,zhao2019seeing}. To fill the gap between attacks in the digital and physical worlds, recent studies have demonstrated that attacks on real-world computer vision systems are practical \cite{eykholt2018robust,li2019adversarial,man2020ghostimage,sharif2016accessorize,zhao2019seeing,zhou2018invisible}. However, attacks on traditional computer vision methods such as SLAM are relatively less explored. \cite{yoshida2022adversarial} proposes an attack against the scan matching algorithm in LiDAR-based SLAM, while most SLAMs in AR/VR devices rely on different sensors like RGB/depth cameras and IMUs. \cite{ikram2022perceptual} and \cite{chen2024adversary} mislead visual SLAM by poisoning the images with special patterns, and \cite{wang2021can} causes the camera to fail using infrared light. In our work, we demonstrate attacks on Visual-Inertial SLAM (VI-SLAM) by perturbing the IMU readings, rather than cameras, and showing its impact on XR user experience. 

\textbf{Acoustic Injection Attacks.} Among various physical attacks, acoustic injection attacks are attractive due to their low cost. Son~\etal~\cite{son2015rocking} were the first to introduce acoustic attacks on MEMS gyroscopes, demonstrating how these attacks could lead to sensor denial-of-service and result in drone crashes. WALNUT~\cite{trippel2017walnut} expanded on this by developing output biasing and control attacks that enable precise manipulation of MEMS accelerometer outputs using modulated sound waves. Wang et al.~\cite{wang2017sonic} demonstrated a sonic gun, showcasing the vulnerability of various smart devices (\eg drones and self-balancing vehicles) to acoustic attacks. Tu et al. \cite{tu2018injected} designed side-swing and switching attacks to alter the outputs of MEMS gyroscopes and accelerometers. Furthermore, Ji et al. \cite{ji2021poltergeist} fool the object detectors by applying acoustic attack to the image stabilizers commonly used in modern cameras. However, none of the existing works study the relationship between the acoustic injections and SLAM outputs on recent XR devices. 

% \zijian{Do we need one session about security in AR/VR?}
% \yicheng{TODO}
%\jiasi{cite the AIVR paper (UMass Amherst?) paper is we have not already. They add IMU perturbation but w/o SLAM, iirc} \yicheng{Cited}

\textbf{XR Security and Privacy.} 
%Security and privacy concerns in XR systems have gained significant attention. 
For single-user XR systems, researchers have demonstrated various side-channel attacks to extract sensitive information (\eg keystrokes) through video feeds~\cite{ling2019know}, head movements~\cite{nair2023unique, slocum2023going}, architectural hints~\cite{zhang2023its,shang2020arspy}, power usage~\cite{li2024dangers}, and EM side-channel leakages~\cite{al2021vr}. In multi-user XR systems, Su et al.~\cite{su2024remote} use avatar motion data to infer keystrokes in shared VR environments. Slocum et al.~\cite{slocum2024doesn} reveal vulnerabilities in the shared state frameworks of multi-user AR. Similarly, Lebeck et al.~\cite{lebeck2017securing} highlight risks like deceptive virtual objects and emphasize access control for managing shared physical and virtual spaces. Ruth et al.~\cite{ruth2019secure} further propose a secure multi-user AR framework focusing on content sharing and permissions.
Chandio et al.~\cite{chandio2024stealthy} %introduced a multi-modal spatiotemporal attack that 
simultaneously manipulated visual and inertial sensors to disrupt XR pose estimation. However, their study evaluated the attack using offline datasets and assumed the attacker's capability to manipulate IMU data streams through acoustic means, without real experiments. Ours is the first to demonstrate acoustic injection attacks on recent XR devices, like the Hololens 2, in the real world.
 



\section*{Conclusion}
This paper aims to enhance our understanding of the computational complexity of computing various Shapley value variants. We found that for various ML models --- including decision trees, regression tree ensembles, weighted automata, and linear regression --- both local and global interventional and baseline SHAP can be computed in polynomial time under HMM modeled distributions. This extends popular algorithms, such as TreeSHAP, beyond their empirical distributional scope. We also establish strict complexity gaps between the various SHAP variants (baseline, interventional, and conditional) and prove the intractability of computing SHAP for tree ensembles and neural networks in simplified scenarios. Overall, we present SHAP as a versatile framework whose complexity depends on four key factors: \begin{inparaenum}[(i)] \item model type, \item SHAP variant, \item distribution modeling approach, \item and local vs. global explanations\end{inparaenum}. We believe this perspective provides deeper insight into the computational complexity of SHAP, paving the way for future work.




%We believe that our framework provides a more intricate understanding of SHAP computation complexity across different models, distributions, and variants, paving the way for further research.

Our work opens promising directions for future research. First, expanding our computational analysis to other SHAP-related metrics, such as asymmetric SHAP~\citep{frye20} and SAGE~\citep{covert2020understanding}, would be valuable. Additionally, we aim to explore more expressive distribution classes and relaxed assumptions beyond those in Section \ref{sec:tractable} while maintaining tractable SHAP computation. Finally, when exact computation is intractable (Section \ref{sec:intractable}), investigating the approximability of SHAP metrics through approximation and parameterized complexity theory~\citep{downey2012parameterized} is an important direction.

%Our work opens several promising avenues for future research on the computational properties of explainable AI methods, with a particular focus on SHAP. First, it would be interesting to broaden the computational analysis conducted in this work to include other popular SHAP-related metrics in the literature, such as asymmetric SHAP \cite{frye20} and SAGE \cite{covert2020understanding}. Also, in the future, we aim to explore more expressive distribution classes and relaxed distributional assumptions—extending beyond those examined in Section \ref{sec:tractable} —that still yield tractable SHAP computation. Finally, when exact computation proves intractable (Section \ref{sec:intractable}), it is worthwhile to theoretically investigate the question of the approximability of computing the SHAP metrics across various configurations, through the lens of approximation and parametrized complexity theory \cite{arora2009computational}.

%This paper aims to deepen our understanding of the computational complexity involved in obtaining different Shapley value variants. We found that for a variety of ML models, including decision trees, tree ensembles for regression, weighted automata, and linear regression models — computing both local and global interventional and baseline SHAP can be done in polynomial time when distributions are modeled by HMMs. This extends the distributional scope of popular algorithms like TreeSHAP, which is limited to empirical distributions. Additionally, we demonstrate a strict complexity gap between SHAP variants, showing that interventional and baseline SHAP can be strictly easier to compute than conditional SHAP. Despite these positive results, we uncovered intractability for various SHAP variants in neural networks and tree ensembles. Finally, we provided generalized complexity relations across SHAP variants. We believe that our framework offers a deeper understanding of the complexity involved in computing SHAP across various variants, models, distributions, as well as in both local and global computations, laying the groundwork for future research.
\clearpage
\newpage
\section{Impact and Ethics}
\label{sec:impactStatement}
This paper introduces research aimed at advancing the field of Machine Learning. We do not identify any specific societal consequences of our work that need to be explicitly emphasized here.
\bibliography{ref}
\bibliographystyle{icml2025}
\appendix
\onecolumn
\section{Turing Machine Computation}
\label{appen:turningDetails}
A Turing machine processes an input string $\vect{x} \in \inset$. Its configuration consists of a finite state set $Q$, an input tape $c_0$, $k$ work tapes $c_1, \dots, c_k$, and an output tape $c_{k+1}$. Additionally, each tape $\tau$ has an associated head position $h_\tau$.  

Initially, the machine starts in the initial state $q_0 \in Q$ with the input tape $c_0^0$ containing $\vect{x}$, positioned at index $0$, and surrounded by infinite blank symbols ($b$). The head on the input tape is set to $h_0^0 = 0$, while all other tapes contain only blank symbols $b$s and have their heads positioned at $0$.  

At each time step $i$, if $q_i \notin F$ ($F$ is a set of halting states), the configuration updates recursively by computing:  
\[
\langle q_{i+1}, \gamma_1^i, \dots, \gamma_{k+1}^i, d_0^i, \dots, d_{k+1}^i \rangle = \delta(q_i, c_0^i[h_0^i], \dots, c_{k+1}^i[h_{k+1}^i])
\]  
where $\delta$ is the transition function. The machine updates each tape $\tau$ by setting $c_\tau^{i+1}[h_\tau^i] = \gamma_\tau^i$, leaving all other tape cells unchanged. The head position for each tape is updated as $h_\tau^{i+1} = h_\tau^i + d_\tau^i$.  If $q_i \in F$, the Turing machine halts and outputs the sequence of tokens on the output tape, starting from the current head position and continuing up to (but not including) the first blank symbol ($b$). A Turing machine can also function as a language recognizer by setting the input alphabet $\Sigma = \{0,1\}$ and interpreting the first output token as either $0$ or $1$.

\section{Thershold Circuit Class}
\label{appen:threshInfo}
$\class$ is a class of computational problems that can be recognized by constant-depth, polynomial-size circuits composed of threshold gates. A threshold gate, such as $\theta_{\leq k}$, outputs 1 if the sum of its input bits is at most $ k $, while $\theta_{\geq k}$ outputs 1 if the sum is at least $ k $. These circuits also include standard logic gates like $\wedge$, $\vee$, and $\neg$ as special cases of threshold functions. Since $\class$ circuits can simulate $AC^{0}$ circuits ( a polysize, constant-depth $\{\wedge, \vee, \neg\}$-circuit family), they are at least as powerful as $AC^{0}$ in the computational hierarchy. The circuit families we have defined above are non-uniform, meaning that there is no requirement for the circuits processing different input sizes to be related in any way. In degenerate cases, non-uniform circuit families can solve undecidable problems making them an unrealizable model of computation \cite{complexityBook}. Intuitively, a uniform circuit family requires that the circuits for different input sizes must be "somewhat similar" to each other. This concept is formalized by stating that there exists a resource-constrained Turing machine that, given the input $ 1^n $, can generate a serialization of the corresponding circuit $ C_n $ for that input size. Specifically, a logspace uniform $\class$ family can be constructed by a logspace-bounded Turing machine from the string $1^n$.

\section{Proofs}
\label{sec:proof}
\begin{lemma}[Constant depth circuit for $\llmFormal$]
\label{lem:constrainLem}
For any log-precision constant layer transformer-based LLM $\llm{}$ with finite vocabulary $V$, a single deterministic auto-regressive step $\llmFormal(x)$ operating on any input of size $n \in \mathbb{N}$ with $\vect{x} \in V^{n}$ can be simulated by a logspace-uniform threshold circuit family of depth $C$ where $C$ is constant.
\end{lemma}
\begin{proof}
The construction is from Theorem~2 in \cite{tc0}.
\end{proof}

\compLimitLemma*
\begin{proof}
The language $ L(\regG) $ is finite; therefore, for any string $ \vect{s} \in L(\regG) $, the length satisfies $ |\vect{s}| \leq N $, where $ N $ is a constant. Consequently, for any input $ \vect{x} $, the output $ \vect{y}_G = \llmG{\vect{x}}{G} $ has a constant length, i.e., $ |\vect{y}_{G}| \leq N $. The number of autoregressive steps is also bounded by $ N $.  

From Lemma~\ref{lem:constrainLem}, each unconstrained autoregressive computation $ \llmFormal(\vect{x}) $ can be simulated by a constant-depth threshold circuit $ C $. This implies that $ \llm_f(\vect{x}, \regG) $ can also be simulated by a constant-depth threshold circuit since it only involves an additional multiplication by a constant-sized precomputed Boolean mask $ \{0, 1\}^{|V|} $ (see Section~\ref{sec:prelims}).  

Given that the number of autoregressive steps is a constant $ N $, and each step can be simulated by a constant-depth circuit $ C $, we can simulate all $ N $ steps using a depth $ N \times C $ circuit by stacking the circuits for each step sequentially. For uniformity, we are just stacking together a constant number of constant depth circuits we can do it in a log-space bounded Turning machine $M$. 

Note that this proof holds only because $ L(\regG) $ allows only constant-size strings in the output.  
\end{proof}
\expressivity*
\begin{proof}
The construction follows from Theorem 2 \cite{expressivity1}.
\end{proof}

In this construction, the deterministic Turing machine run captured by a sequence of $\hatConfig{\gamma_1}, \dots, \hatConfig{\gamma_{t(n)}}$ capturing the state entered, tokens written, and directions moved after each token before generating the output $M(\vect{x})$. Then on any input the $\vect{x}$ the output $\llm_{M}(\vect{x}) = \hatConfig{\gamma_1}, \cdots, \hatConfig{\gamma_{t(n)}}\cdot M(\vect{x})$ (assuming $M$ halts within on $\vect{x}$ within $\runtime{n}$ steps where $n = |\vect{x}|$ and $\runtime{n}$ is a polynomial over $n$).

\expressConstrain*
\begin{proof}
$\llm{}_{M}(\vect{x})) = \hatConfig{\gamma_1}\cdots\hatConfig{\gamma_{\runtime{n}}}\cdot M(\vect{x})$. We show that $\llm{}_{M}(\vect{x}) \in L(\augG)$. 
$\augG \to \rGM G$. Since, $G$ is output grammar of $M$ then $M(\vect{x}) \in \lang{G}$. For all $1 \leq i \leq \runtime{n}$ $\hatConfig{\gamma_{i}} \in \hatConfig{\Gamma}$. Then, $\hatConfig{\gamma_1}\cdots\hatConfig{\gamma_{\runtime{n}}} \in \hatConfig{\Gamma}^{*} \subseteq L(\rGM)$.

Then $\llm{}_{M}(\vect{x}) \in \lang{\augG}$ then under constrained decoding the output $\llm{}_{M}(\vect{x})$ remains unchanged and $\llm{}_{M}(\vect{x}) = \llmMG{\vect{x}}{\augG} = \vect{r} \cdot M(\vect{x})$ where $\vect{r} = \hatConfig{\gamma_1}\cdots\hatConfig{\gamma_{\runtime{n}}}$.
\end{proof}
\clearpage
\newpage



\begin{table*}[t]
    \centering
    \caption{Comparison of \Tool{} and baselines with various models on GSM-Symbolic based on accuracy, number of tokens, and average time.}
    \begin{tabular}{llcccr}
        \toprule
        \textbf{Model} & \textbf{k} & \textbf{Method} & \textbf{Acc. (\%)} & \textbf{Parse (\%)} &  \textbf{Tokens}\\
        
\midrule

   &  & \stdUnconstrained{} & 20 & 98 & 18.23 \\
 & & \stdConstrained{} & 21 & 95 & 34.28 \\
 Qwen2.5-1.5B-Instruct & 2 & \cotUnconstrained{} & 22 & 90 & 130.74 \\
 & & \textbf{\Tool{}} & \textbf{28} & 96 & 140.52 \\

\midrule

   &  & \stdUnconstrained{} & 18 & 95 & 18.23 \\
 & & \stdConstrained{} & 18 & 96 & 34.28 \\
 Qwen2.5-1.5B-Instruct & 4 & \cotUnconstrained{} & 24 & 94 & 130.74 \\
 & & \textbf{\Tool{}} & \textbf{30} & 98 & 140.52 \\

\midrule
   &  & \stdUnconstrained{} & 21 & 97 & 23.34  \\
 & & \stdConstrained{} & 22 & 97 & 25.29  \\
  Qwen2.5-1.5B-Instruct  & 8 & \cotUnconstrained{} & 26 & 90 & 128.97  \\
 & & \textbf{\Tool{}} & \textbf{31} & 100 & 131.3  \\

\midrule

    & & \stdUnconstrained{} & 37 & 96 & 17.22 \\
 & & \stdConstrained{} & 36 & 99 & 18.61  \\
 Qwen2.5-Coder-7B-Instruct & 2 & \cotUnconstrained{} & 32 & 84 & 148.87 \\
 & & \textbf{\Tool{}} & \textbf{37} & 96 & 155.65\\

\midrule

    & & \stdUnconstrained{} & 36 & 96 & 16.89 \\
 & & \stdConstrained{} & 36 & 100 & 18.81  \\
 Qwen2.5-Coder-7B-Instruct & 4 & \cotUnconstrained{} & 35 & 89 & 151.29 \\
 & & \textbf{\Tool{}} & \textbf{37} & 97 & 163.21\\

\midrule

    & & \stdUnconstrained{} & 36 & 94 & 17.92 \\
 & & \stdConstrained{} & 35 & 99 & 25.28  \\
 Qwen2.5-Coder-7B-Instruct & 8 & \cotUnconstrained{} & 37 & 88 & 138.38 \\
 & & \textbf{\Tool{}} & \textbf{39} & 94 & 155.32\\

\midrule

     & & \stdUnconstrained{} & 20 & 66 & 115.22 \\
&  & \stdConstrained{} & 26 & 95 & 26.99 \\
 Qwen2.5-Math-7B-Instruct & 2 & \cotUnconstrained{} & 28 & 72 & 190.51 \\
 & & \textbf{\Tool{}} & \textbf{32} & 89 & 195.65 \\

 \midrule

    &  & \stdUnconstrained{} & 22 & 83 & 47 \\
 & & \stdConstrained{} & 29 & 98 & 27.08 \\
 Qwen2.5-Math-7B-Instruct & 4 & \cotUnconstrained{} & 28 & 76 & 184.35 \\
 & & \textbf{\Tool{}} & \textbf{37} & 88 & 194.77  \\

 \midrule

     & & \stdUnconstrained{} & 27 & 89 & 25.7 \\
 & & \stdConstrained{} & 29 & 99 & 26.81 \\
 Qwen2.5-Math-7B-Instruct & 8 &  \cotUnconstrained{} & 29 & 82 & 155.26\\
 & & \textbf{\Tool{}} & \textbf{38} & 94 & 158.86 \\

 \midrule

     & & \stdUnconstrained{} & 19 & 61 & 157.36 \\
 & & \stdConstrained{} & 23 & 95 & 45.58  \\
 Llama-3.1-8B-Instruct & 2 & \cotUnconstrained{} & 29 & 84 & 198.64 \\
 & & \textbf{\Tool{}} & \textbf{35} & 94 & 206.85 \\

 \midrule
     & & \stdUnconstrained{} & 18 & 68 & 131.5 \\
 & & \stdConstrained{} & 24 & 96 & 37.38  \\
 Llama-3.1-8B-Instruct & 4 & \cotUnconstrained{} & 26 & 92 & 172.21 \\
 & & \textbf{\Tool{}} & \textbf{30} & 97 & 179.95 \\

  \midrule

     & & \stdUnconstrained{} & 21 & 73 & 128.38 \\
 & & \stdConstrained{} & 26 & 98 & 35.97  \\
 Llama-3.1-8B-Instruct & 8 & \cotUnconstrained{} & 30 & 95 & 163.55 \\
 & & \textbf{\Tool{}} & \textbf{33} & 95 & 170.22 \\


\bottomrule
    \end{tabular}
    \label{tab:gsm_symbolic_comparison_k_shot}
    \vspace{-.2in}
\end{table*}


\subsection{GSM-Symbolic Examples and Prompt}
\label{sec:gsm_info}
\textbf{GSM-Symbolic Problem Solution Examples:}



% \lstdefinestyle{myGrammarStyle}{
%     basicstyle=\scriptsize\ttfamily, % Reduce font size
%     commentstyle=\color{gray},
%     keywordstyle=\color{blue},
%     stringstyle=\color{orange},
%     numbers=left, % Line numbers on left
%     numberstyle=\tiny\color{gray}, % Line numbers styling
%     breaklines=true, % Wrap long lines
%     frame=single, % Frame around the code
%     framesep=3pt, % Adjust frame separation
%     xleftmargin=5pt, % Adjust left margin
%     xrightmargin=5pt, % Adjust right margin
%     backgroundcolor=\color{blue!4}, % Background color
%     tabsize=2, % Tab size
%     captionpos=b, % Caption position bottom
%     aboveskip=5pt, % Reduce space above the listing
%     belowskip=5pt, % Reduce space below the listing
%     linewidth=0.9\linewidth, % Set custom line width to less than text width
%     escapeinside={(*@}{@*)}, % for escaping to LaTeX
% }

\begin{lstlisting}[style=myGrammarStyle, caption=Problem Solution Examples for GSM-Symbolic]
Question: A fog bank rolls in from the ocean to cover a city. It takes {t} minutes to cover every {d} miles of the city. If the city is {y} miles across from oceanfront to the opposite inland edge, how many minutes will it take for the fog bank to cover the whole city?

Answer: y//d*t

Question: {name} makes {drink} using teaspoons of sugar and cups of water in the ratio of {m}:{n}. If she used a total of {x} teaspoons of sugar and cups of water, calculate the number of teaspoonfuls of sugar she used.

Answer: ((m*x)//(m+n))
\end{lstlisting}
\label{gram:gsm_example}
\textbf{GSM-Symbolic Prompt:}



% \lstdefinestyle{myGrammarStyle}{
%     basicstyle=\scriptsize\ttfamily, % Reduce font size
%     commentstyle=\color{gray},
%     keywordstyle=\color{blue},
%     stringstyle=\color{orange},
%     numbers=left, % Line numbers on left
%     numberstyle=\tiny\color{gray}, % Line numbers styling
%     breaklines=true, % Wrap long lines
%     frame=single, % Frame around the code
%     framesep=3pt, % Adjust frame separation
%     xleftmargin=5pt, % Adjust left margin
%     xrightmargin=5pt, % Adjust right margin
%     backgroundcolor=\color{green!4}, % Background color
%     tabsize=2, % Tab size
%     captionpos=b, % Caption position bottom
%     aboveskip=5pt, % Reduce space above the listing
%     belowskip=5pt, % Reduce space below the listing
%     linewidth=0.9\linewidth, % Set custom line width to less than text width
%     escapeinside={(*@}{@*)}, % for escaping to LaTeX
% }

\begin{lstlisting}[style=myGrammarStyle, caption=CoT Prompt Template For GSM-Symbolic Evaluation]
You are an expert in solving grade school math tasks. You will be presented with a grade-school math word problem with symbolic variables and be asked to solve it.

Before answering you should reason about the problem (using the <reasoning> field in the response described below). Intermediate symbolic expressions generated during reasoning should be wrapped in << >>.

Then, output the symbolic expression wrapped in << >> that answers the question. The expressions must use numbers as well as the variables defined in the question. You are only allowed to use the following operations: +, -, /, //, %, (), and int().

You will always respond in the format described below: 
Let's think step by step. <reasoning> The final answer is <<symbolic expression>>

There are {t} trees in the {g}. {g} workers will plant trees in the {g} today. After they are done, there will be {tf} trees. How many trees did the {g} workers plant today?

Let's think step by step. Initially, there are {t} trees. After planting, there are {tf} trees. The number of trees planted is <<tf - t>>. The final answer is <<tf - t>>.

If there are {c} cars in the parking lot and {nc} more cars arrive, how many cars are in the parking lot?

Let's think step by step. Initially, there are {c} cars. {nc} more cars arrive, so the total becomes <<c + nc>>. The final answer is <<c + nc>>.

{p1} had {ch1} {o1} and {p2} had {ch2} {o1}. If they ate {a} {o1}, how many pieces do they have left in total?

Let's think step by step. Initially, {p1} had {ch1} {o1}, and {p2} had {ch2} {o1}, making a total of <<ch1 + ch2>>. After eating {a} {o1}, the remaining total is <<ch1 + ch2 - a>>. The final answer is <<ch1 + ch2 - a>>.

{p1} had {l1} {o1}. {p1} gave {g} {o1} to {p2}. How many {o1} does {p1} have left?

Let's think step by step. {p1} started with {l1} {o1}. After giving {g} {o1} to {p2}, {p1} has <<l1 - g>> {o1} left. The final answer is <<l1 - g>>.

{p1} has {t} {o1}. For Christmas, {p1} got {tm} {o1} from {p2} and {td} {o1} from {p3}. How many {o1} does {p1} have now?

Let's think step by step. {p1} started with {t} {o1}. {p1} received {tm} {o1} from {p2} and {td} {o1} from {p3}. The total is <<t + tm + td>>. The final answer is <<t + tm + td>>.

There were {c} {o1} in the server room. {nc} more {o1} were installed each day, from {d1} to {d2}. How many {o1} are now in the server room?

Let's think step by step. Initially, there were {c} {o1}. {nc} {o1} were added each day for <<d2 - d1 + 1>> days, which is <<nc * (d2 - d1 + 1)>>. The total is <<c + nc * (d2 - d1 + 1)>>. The final answer is <<c + nc * (d2 - d1 + 1)>>.

{p1} had {gb1} {o1}. On {day1}, {p1} lost {l1} {o1}. On {day2}, {p1} lost {l2} more. How many {o1} does {p1} have at the end of {day2}?

Let's think step by step. Initially, {p1} had {gb1} {o1}. After losing {l1} {o1} on {day1}, {p1} had <<gb1 - l1>>. After losing {l2} {o1} on {day2}, the total is <<gb1 - l1 - l2>>. The final answer is <<gb1 - l1 - l2>>.

{p1} has ${m}. {p1} bought {q} {o1} for ${p} each. How much money does {p1} have left?

Let's think step by step. Initially, {p1} had ${m}. {p1} spent <<q * p>> on {q} {o1}. The remaining money is <<m - q * p>>. The final answer is <<m - q * p>>.

{question}
\end{lstlisting}
\label{gram:gsm_prompt}



% \lstdefinestyle{myGrammarStyle}{
%     basicstyle=\scriptsize\ttfamily, % Reduce font size
%     commentstyle=\color{gray},
%     keywordstyle=\color{blue},
%     stringstyle=\color{orange},
%     numbers=left, % Line numbers on left
%     numberstyle=\tiny\color{gray}, % Line numbers styling
%     breaklines=true, % Wrap long lines
%     frame=single, % Frame around the code
%     framesep=3pt, % Adjust frame separation
%     xleftmargin=5pt, % Adjust left margin
%     xrightmargin=5pt, % Adjust right margin
%     backgroundcolor=\color{green!4}, % Background color
%     tabsize=2, % Tab size
%     captionpos=b, % Caption position bottom
%     aboveskip=5pt, % Reduce space above the listing
%     belowskip=5pt, % Reduce space below the listing
%     linewidth=0.9\linewidth, % Set custom line width to less than text width
%     escapeinside={(*@}{@*)}, % for escaping to LaTeX
% }

\begin{lstlisting}[style=myGrammarStyle, caption= Prompt Template For GSM-Symbolic Evaluation Without CoT]
You are an expert in solving grade school math tasks. You will be presented with a grade-school math word problem with symbolic variables and be asked to solve it.

Only output the symbolic expression wrapped in << >> that answers the question. The expression must use numbers as well as the variables defined in the question. You are only allowed to use the following operations: +, -, /, //, %, (), and int().

You will always respond in the format described below: 
<<symbolic expression>>

There are {t} trees in the {g}. {g} workers will plant trees in the {g} today. After they are done, there will be {tf} trees. How many trees did the {g} workers plant today?

<<tf - t>>

If there are {c} cars in the parking lot and {nc} more cars arrive, how many cars are in the parking lot?

<<c + nc>>

{p1} had {ch1} {o1} and {p2} had {ch2} {o1}. If they ate {a} {o1}, how many pieces do they have left in total?

<<ch1 + ch2 - a>>

{p1} had {l1} {o1}. {p1} gave {g} {o1} to {p2}. How many {o1} does {p1} have left?

<<l1 - g>>

{p1} has {t} {o1}. For Christmas, {p1} got {tm} {o1} from {p2} and {td} {o1} from {p3}. How many {o1} does {p1} have now?

<<t + tm + td>>

There were {c} {o1} in the {loc}. {nc} more {o1} were installed each day, from {d1} to {d2}. How many {o1} are now in the {loc}?

<<c + nc * (d2 - d1 + 1)>>

{p1} had {gb1} {o1}. On {day1}, {p1} lost {l1} {o1}. On {day2}, {p1} lost {l2} more. How many {o1} does {p1} have at the end of {day2}?

<<gb1 - l1 - l2>>

{p1} has ${m}. {p1} bought {q} {o1} for ${p} each. How much money does {p1} have left?

<<m - q * p>>

{question}
\end{lstlisting}
\label{gram:gsm_prompt_no_cot}

\subsection{FOLIO Examples and Prompt}
\label{sec:folio_info}
\textbf{FOLIO Problem Solution Examples:}

\begin{lstlisting}[style=myGrammarStyle, caption=Problem Solution Examples for FOLIO]
Question: 
People in this club who perform in school talent shows often attend and are very engaged with school events.
People in this club either perform in school talent shows often or are inactive and disinterested community members.
People in this club who chaperone high school dances are not students who attend the school.
All people in this club who are inactive and disinterested members of their community chaperone high school dances.
All young children and teenagers in this club who wish to further their academic careers and educational opportunities are students who attend the school. 
Bonnie is in this club and she either both attends and is very engaged with school events and is a student who attends the school or is not someone who both attends and is very engaged with school events and is not a student who attends the school.
Based on the above information, is the following statement true, false, or uncertain? Bonnie performs in school talent shows often.
###

FOL Solution: 
Predicates:
InClub(x) ::: x is a member of the club.
Perform(x) ::: x performs in school talent shows.
Attend(x) ::: x attends school events.
Engaged(x) ::: x is very engaged with school events.
Inactive(x) ::: x is an inactive and disinterested community member.
Chaperone(x) ::: x chaperones high school dances.
Student(x) ::: x is a student who attends the school.
Wish(x) ::: x wishes to further their academic careers and educational opportunities.
Premises:
{forall} x (InClub(x) {and} Attend(x) {and} Engaged(x) {implies} Attend(x)) ::: People in this club who perform in school talent shows often attend and are very engaged with school events.
{forall} x (InClub(x) {implies} (Perform(x) {xor} Inactive(x))) ::: People in this club either perform in school talent shows often or are inactive and disinterested community members.
{forall} x (InClub(x) {and} Chaperone(x) {implies} {not}Student(x)) ::: People in this club who chaperone high school dances are not students who attend the school.
{forall} x (InClub(x) {and} Inactive(x) {implies} Chaperone(x)) ::: All people in this club who are inactive and disinterested members of their community chaperone high school dances.
{forall} x (InClub(x) {and} (Young(x) {or} Teenager(x)) {and} Wish(x) {implies} Student(x)) ::: All young children and teenagers in this club who wish to further their academic careers and educational opportunities are students who attend the school.
{forall} x (InClub(x) {implies} (Attend(x) {and} Engaged(x)) {xor} {not}(Attend(x) {and} Engaged(x)) {and} {not}Student(x) {xor} Student(x)) ::: Bonnie is in this club and she either both attends and is very engaged with school events and is a student who attends the school or is not someone who both attends and is very engaged with school events and is not a student who attends the school.
Conclusion:
InClub(bonnie) {and} Perform(bonnie) ::: Bonnie performs in school talent shows often.

Answer: Uncertain

\end{lstlisting}
\label{gram:folio_example}
\textbf{FOLIO Prompt:}
% \lstdefinestyle{myGrammarStyle}{
%     basicstyle=\scriptsize\ttfamily, % Reduce font size
%     commentstyle=\color{gray},
%     keywordstyle=\color{blue},
%     stringstyle=\color{orange},
%     numbers=left, % Line numbers on left
%     numberstyle=\tiny\color{gray}, % Line numbers styling
%     breaklines=true, % Wrap long lines
%     frame=single, % Frame around the code
%     framesep=3pt, % Adjust frame separation
%     xleftmargin=5pt, % Adjust left margin
%     xrightmargin=5pt, % Adjust right margin
%     backgroundcolor=\color{green!4}, % Background color
%     tabsize=2, % Tab size
%     captionpos=b, % Caption position bottom
%     aboveskip=5pt, % Reduce space above the listing
%     belowskip=5pt, % Reduce space below the listing
%     linewidth=0.9\linewidth, % Set custom line width to less than text width
%     escapeinside={(*@}{@*)}, % for escaping to LaTeX
% }

\begin{lstlisting}[style=myGrammarStyle, caption=Prompt Template Used For FOLIO Evaluation]
Given a problem description and a question. The task is to parse the problem and the question into first-order logic formulas.
The grammar of the first-order logic formula is defined as follows:
1) logical conjunction of expr1 and expr2: expr1 {and} expr2
2) logical disjunction of expr1 and expr2: expr1 {or} expr2
3) logical exclusive disjunction of expr1 and expr2: expr1 {xor} expr2
4) logical negation of expr1: {not}expr1
5) expr1 implies expr2: expr1 {implies} expr2
6) expr1 if and only if expr2: expr1 {iff} expr2
7) logical universal quantification: {forall} x
8) logical existential quantification: {exists} x. These are the ONLY operations in the grammar.
------

Answer the question EXACTLY like the examples.

Problem:
All people who regularly drink coffee are dependent on caffeine. People either regularly drink coffee or joke about being addicted to caffeine. No one who jokes about being addicted to caffeine is unaware that caffeine is a drug. Rina is either a student and unaware that caffeine is a drug, or neither a student nor unaware that caffeine is a drug. If Rina is not a person dependent on caffeine and a student, then Rina is either a person dependent on caffeine and a student, or neither a person dependent on caffeine nor a student.
Question:
Based on the above information, is the following statement true, false, or uncertain? Rina is either a person who jokes about being addicted to caffeine or is unaware that caffeine is a drug.
###

We take three steps: first, we define the necessary predicates and premises, and finally, we encode the question `Rina is either a person who jokes about being addicted to caffeine or is unaware that caffeine is a drug.` in the conclusion. Now, we will write only the logic program, nothing else.
Predicates:
Dependent(x) ::: x is a person dependent on caffeine.
Drinks(x) ::: x regularly drinks coffee.
Jokes(x) ::: x jokes about being addicted to caffeine.
Unaware(x) ::: x is unaware that caffeine is a drug.
Student(x) ::: x is a student.
Premises:
{forall} x (Drinks(x) {implies} Dependent(x)) ::: All people who regularly drink coffee are dependent on caffeine.
{forall} x (Drinks(x) {xor} Jokes(x)) ::: People either regularly drink coffee or joke about being addicted to caffeine.
{forall} x (Jokes(x) {implies} {not}Unaware(x)) ::: No one who jokes about being addicted to caffeine is unaware that caffeine is a drug. 
(Student(rina) {and} Unaware(rina)) {xor} {not}(Student(rina) {or} Unaware(rina)) ::: Rina is either a student and unaware that caffeine is a drug, or neither a student nor unaware that caffeine is a drug.
Conclusion:
Jokes(rina) {xor} Unaware(rina) ::: Rina is either a person who jokes about being addicted to caffeine or is unaware that caffeine is a drug.
------

Problem:
Miroslav Venhoda was a Czech choral conductor who specialized in the performance of Renaissance and Baroque music. Any choral conductor is a musician. Some musicians love music. Miroslav Venhoda published a book in 1946 called Method of Studying Gregorian Chant.
Question:
Based on the above information, is the following statement true, false, or uncertain? Miroslav Venhoda loved music.
###

We take three steps: first, we define the necessary predicates and premises, and finally, we encode the question `Miroslav Venhoda loved music.` in the conclusion. Now, we will write only the logic program, nothing else.
Predicates:
Czech(x) ::: x is a Czech person.
ChoralConductor(x) ::: x is a choral conductor.
Musician(x) ::: x is a musician.
Love(x, y) ::: x loves y.
Author(x, y) ::: x is the author of y.
Book(x) ::: x is a book.
Publish(x, y) ::: x is published in year y.
Specialize(x, y) ::: x specializes in y.
Premises:
Czech(miroslav) {and} ChoralConductor(miroslav) {and} Specialize(miroslav, renaissance) {and} Specialize(miroslav, baroque) ::: Miroslav Venhoda was a Czech choral conductor who specialized in the performance of Renaissance and Baroque music.
{forall} x (ChoralConductor(x) {implies} Musician(x)) ::: Any choral conductor is a musician.
{exists} x (Musician(x) {and} Love(x, music)) ::: Some musicians love music.
Book(methodOfStudyingGregorianChant) {and} Author(miroslav, methodOfStudyingGregorianChant) {and} Publish(methodOfStudyingGregorianChant, year1946) ::: Miroslav Venhoda published a book in 1946 called Method of Studying Gregorian Chant.
Conclusion:
Love(miroslav, music) ::: Miroslav Venhoda loved music.
------

{question}
\end{lstlisting}
\label{gram:folio_prompt}


\newpage
\subsection{Case Study For GSM-Symbolic}



\lstdefinestyle{myGrammarStyle}{
    basicstyle=\scriptsize\ttfamily, % Reduce font size
    commentstyle=\color{gray},
    keywordstyle=\color{blue},
    stringstyle=\color{orange},
    numbers=left, % Line numbers on left
    numberstyle=\tiny\color{gray}, % Line numbers styling
    breaklines=true, % Wrap long lines
    frame=single, % Frame around the code
    framesep=3pt, % Adjust frame separation
    xleftmargin=5pt, % Adjust left margin
    xrightmargin=5pt, % Adjust right margin
    backgroundcolor=\color{gray!4}, % Background color
    tabsize=2, % Tab size
    captionpos=b, % Caption position bottom
    aboveskip=5pt, % Reduce space above the listing
    belowskip=5pt, % Reduce space below the listing
    linewidth=0.9\linewidth, % Set custom line width to less than text width
    escapeinside={(*@}{@*)}, % for escaping to LaTeX
}

\begin{lstlisting}[style=myGrammarStyle, caption=Case Study for GSM-Symbolic]
Question: {name} hires a {vehicle} from {start_hour}  to {end_hour}. He gets {free_hours} hours free. The first paid hour is ${first_hour_cost} and each hour after that is {multiplier} the cost. How much did he pay?

Constrained: <<(int(end_hour - start_hour) - free_hours) * first_hour_cost + free_hours * first_hour_cost + (int(end_hour - start_hour) - free_hours - 1) * multiplier * first_hour_cost>>

Unconstrained With CoT:  Let's think step by step. The total time hired is from {start_hour} to {end_hour}, which is <<int((end_hour - start_hour).total_seconds() / 3600)>> hours. Subtracting the {free_hours} free hours, the paid hours are <<int((end_hour - start_hour).total_seconds() / 3600) - free_hours>>. The final answer is <<first_hour_cost + (int((end_hour - start_hour).total_seconds() / 3600) - free_hours - 1) * multiplier * first_hour_cost>>.

CRANE: Let's think step by step. The total time hired is from {start_hour} to {end_hour}, totaling <<int(end_hour - start_hour)>> hours. Subtracting {free_hours} free hours, the paid hours are <<int(end_hour - start_hour) - free_hours>>. The first hour costs {first_hour_cost}, and each additional hour costs {multiplier} times that. The final answer is <<first_hour_cost + (int(end_hour - start_hour) - free_hours - 1) * multiplier * first_hour_cost>>.


\end{lstlisting}
\label{gram:gsm_case_study}

\Tool{} effectively alternates between constrained and unconstrained generation to produce intermediate expressions, the final answer, and to maintain the reasoning capabilities of the LLM. In contrast, unconstrained generation with CoT results in a syntactically incorrect expression, while constrained generation produces a syntactically valid but incorrect expression.

\subsection{Sampling Ablation for GSM-Symbolic}
In our GSM-Symbolic case study, we use IterGen as the constrained generation baseline and initialize \Tool{} with IterGen. Both IterGen and \Tool{} employ selective rejection sampling to filter tokens that do not satisfy semantic constraints. For comparison, we also run unconstrained generation using temperature sampling and evaluate its performance against \Tool{}. Specifically, for Qwen2.5-1.5B-Instruct and Llama-3.1-8B-Instruct, we generate three samples with unconstrained generation at a temperature of \( t = 0.7 \) and compute pass@1/2/3 metrics. 

As shown in Table ~\ref{tab:rejection_sample_gsm}, \Tool{} with greedy decoding achieves higher accuracy than pass@1/2/3 for unconstrained generation with Chain-of-Thought (CoT) and temperature sampling on Qwen2.5-1.5B-Instruct. Although, for Llama-3.1-8B-Instruct, unconstrained generation with CoT and temperature sampling achieves a pass@3 accuracy of 35\%—2\% higher than \Tool{}—it generates approximately 4 times as many tokens as \Tool{}.

\begin{table*}[t]
    \centering
    \small
    \caption{Comparison of \Tool{} and greedy and sampling baselines with different models on GSM-Symbolic.}
    \begin{tabular}{llccr}
        \toprule
        \textbf{Model} & \textbf{Method} & \textbf{pass@1/2/3 (\%)} & \textbf{Parse (\%)} &  \textbf{Tokens} \\
        
\midrule
     & \stdUnconstrained{} (Greedy) & 21 & 97 & 23.34\\
     & \stdUnconstrained{} (t = 0.7) & 15/19/22 & 88/96/98 & 20.19/39.76/60.57\\
 & \stdConstrained{} (Greedy) & 22 & 97 & 25.29 \\
 Qwen2.5-1.5B-Instruct & \cotUnconstrained{} (Greedy) & 26 & 90 & 128.97\\
 & \cotUnconstrained{} (t = 0.7) & 21/25/30 & 78/91/96 & 146.22/292.96/444.61\\
 & \textbf{\Tool{}} & \textbf{31} & 100 & 131.3\\

\midrule

     & \stdUnconstrained{} (Greedy) & 21 & 73 & 128.38\\
     & \stdUnconstrained{} (t = 0.7) & 15/21/25  & 51/74/84 & 106.88/232.75/369.86\\
 & \stdConstrained{} (Greedy) & 26 & 98 & 35.97 \\
 Llama-3.1-8B-Instruct & \cotUnconstrained{} (Greedy) & 30 & 95 & 163.55 \\
 & \cotUnconstrained{} (t = 0.7) & 24/29/\textbf{35} & 89/98/98 & 196.01/403.68/607.7\\
 & \textbf{\Tool{}} (Greedy) & 33 & 95 & 170.22 \\



\bottomrule
    \end{tabular}
    \label{tab:rejection_sample_gsm}
    \vspace{-.2in}
\end{table*}



\begin{table*}[t]
    \centering
    \small
    \caption{Comparison of \Tool{} and greedy and sampling baselines with different models on FOLIO.}
    \begin{tabular}{llccr}
        \toprule
        \textbf{Model} & \textbf{Method} & \textbf{pass@1/2/3 (\%)} & \textbf{Compile (\%)} &  \textbf{Tokens} \\
    \midrule
    & \cotUnconstrained{} (Greedy) & 36.95 & 70.94 & 350.64  \\
    & \cotUnconstrained{} (t = 0.7) & 16.75/28.57/34.98 & 35.96/55.67/68.47 & 401.5/800.19/1219.33  \\
 Qwen2.5-7B-Instruct & \stdConstrained{} (Greedy) & 37.44 & 87.68 & 775.62 \\
 & \textbf{\Tool{}} (Greedy) & \textbf{42.36} & 87.68 & 726.88  \\

% Qwen2.5-7B & \cotConstrained{} & 40.39 & 774.18 & 25.74 \\
 \midrule
     & \cotUnconstrained{} (Greedy) & 32.02 & 57.14 & 371.52  \\
     & \cotUnconstrained{} (t = 0.7) & 14.29/22.66/29.06 & 33.99/46.8/57.64 & 435.35/877.33/1307.45  \\
 Llama-3.1-8B-Instruct & \stdConstrained{} (Greedy) & 39.41 & 86.21 & 549.75  \\
 & \textbf{\Tool{}} (Greedy) & \textbf{46.31} & 85.71 & 449.77  \\




\bottomrule
    \end{tabular}
    \label{tab:rejection_sample_fol}
    \vspace{-.2in}
\end{table*}


\subsection{Grammars}
\subsubsection{GSM-Symbolic Grammar}
\label{sec:gsm_grammar}

\lstdefinestyle{myGrammarStyle}{
    basicstyle=\scriptsize\ttfamily, % Reduce font size
    commentstyle=\color{gray},
    keywordstyle=\color{blue},
    stringstyle=\color{orange},
    numbers=left, % Line numbers on left
    numberstyle=\tiny\color{gray}, % Line numbers styling
    breaklines=true, % Wrap long lines
    frame=single, % Frame around the code
    framesep=3pt, % Adjust frame separation
    xleftmargin=5pt, % Adjust left margin
    xrightmargin=5pt, % Adjust right margin
    backgroundcolor=\color{yellow!4}, % Background color
    tabsize=2, % Tab size
    captionpos=b, % Caption position bottom
    aboveskip=5pt, % Reduce space above the listing
    belowskip=5pt, % Reduce space below the listing
    linewidth=0.9\linewidth, % Set custom line width to less than text width
    escapeinside={(*@}{@*)}, % for escaping to LaTeX
}

\begin{lstlisting}[style=myGrammarStyle, caption=GSM-Symbolic Grammar]
start: space? "<" "<" space? expr space? ">" ">" space?

expr: expr space? "+" space? term   
     | expr space? "-" space? term   
     | term

term: term space? "*" space? factor 
     | term space? "/" space? factor 
     | term space? "//" space? factor 
     | term space? "%" space? factor  
     | factor space?

factor: "-" space? factor    
       | TYPE "(" space? expr space? ")" 
       | primary space?

primary: NUMBER        
        | VARIABLE      
        | "(" space? expr space? ")"

TYPE.4: "int"

space: " "

%import common.CNAME -> VARIABLE
%import common.NUMBER
\end{lstlisting}
\label{gram:gsm_grammar}



\subsubsection{Prover9 Grammar}
\label{sec:prover9_grammar}

\lstdefinestyle{myGrammarStyle}{
    basicstyle=\scriptsize\ttfamily, % Reduce font size
    commentstyle=\color{gray},
    keywordstyle=\color{blue},
    stringstyle=\color{orange},
    numbers=left, % Line numbers on left
    numberstyle=\tiny\color{gray}, % Line numbers styling
    breaklines=true, % Wrap long lines
    frame=single, % Frame around the code
    framesep=3pt, % Adjust frame separation
    xleftmargin=5pt, % Adjust left margin
    xrightmargin=5pt, % Adjust right margin
    backgroundcolor=\color{yellow!4}, % Background color
    tabsize=2, % Tab size
    captionpos=b, % Caption position bottom
    aboveskip=5pt, % Reduce space above the listing
    belowskip=5pt, % Reduce space below the listing
    linewidth=0.9\linewidth, % Set custom line width to less than text width
    escapeinside={(*@}{@*)}, % for escaping to LaTeX
}

\begin{lstlisting}[style=myGrammarStyle, caption=Prover9 Grammar]
start: predicate_section premise_section conclusion_section

predicate_section: "Predicates:" predicate_definition+
premise_section: "Premises:" premise+
conclusion_section: "Conclusion:" conclusion+

predicate_definition: PREDICATE "(" VAR ("," VAR)* ")" COMMENT  -> define_predicate
premise: quantified_expr COMMENT -> define_premise
conclusion: quantified_expr COMMENT -> define_conclusion

quantified_expr: quantifier VAR "(" expression ")" | expression
quantifier: "{forall}" -> forall | "{exists}" -> exists

expression: bimplication_expr

?bimplication_expr: implication_expr ("{iff}" bimplication_expr)?  -> iff
?implication_expr: xor_expr ("{implies}" implication_expr)?  -> imply
?xor_expr: or_expr ("{xor}" xor_expr)?                -> xor
?or_expr: and_expr ("{or}" or_expr)?                -> or
?and_expr: neg_expr ("{and}" and_expr)?              -> and
?neg_expr: "{not}" quantified_expr                   -> neg 
        | atom

?atom: PREDICATE "(" VAR ("," VAR)* ")" -> predicate 
    | "(" quantified_expr ")" 

// Variable names begin with a lowercase letter
VAR.-1: /[a-z][a-zA-Z0-9_]*/  | /[0-9]+/

// Predicate names begin with a capital letter
PREDICATE.-1: /[A-Z][a-zA-Z0-9]*/

COMMENT: /:::.*\n/

%import common.WS
%ignore WS
\end{lstlisting}
\label{gram:prover9_grammar}




\end{document}
