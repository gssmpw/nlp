


% \lstdefinestyle{myGrammarStyle}{
%     basicstyle=\scriptsize\ttfamily, % Reduce font size
%     commentstyle=\color{gray},
%     keywordstyle=\color{blue},
%     stringstyle=\color{orange},
%     numbers=left, % Line numbers on left
%     numberstyle=\tiny\color{gray}, % Line numbers styling
%     breaklines=true, % Wrap long lines
%     frame=single, % Frame around the code
%     framesep=3pt, % Adjust frame separation
%     xleftmargin=5pt, % Adjust left margin
%     xrightmargin=5pt, % Adjust right margin
%     backgroundcolor=\color{green!4}, % Background color
%     tabsize=2, % Tab size
%     captionpos=b, % Caption position bottom
%     aboveskip=5pt, % Reduce space above the listing
%     belowskip=5pt, % Reduce space below the listing
%     linewidth=0.9\linewidth, % Set custom line width to less than text width
%     escapeinside={(*@}{@*)}, % for escaping to LaTeX
% }

\begin{lstlisting}[style=myGrammarStyle, caption= Prompt Template For GSM-Symbolic Evaluation Without CoT]
You are an expert in solving grade school math tasks. You will be presented with a grade-school math word problem with symbolic variables and be asked to solve it.

Only output the symbolic expression wrapped in << >> that answers the question. The expression must use numbers as well as the variables defined in the question. You are only allowed to use the following operations: +, -, /, //, %, (), and int().

You will always respond in the format described below: 
<<symbolic expression>>

There are {t} trees in the {g}. {g} workers will plant trees in the {g} today. After they are done, there will be {tf} trees. How many trees did the {g} workers plant today?

<<tf - t>>

If there are {c} cars in the parking lot and {nc} more cars arrive, how many cars are in the parking lot?

<<c + nc>>

{p1} had {ch1} {o1} and {p2} had {ch2} {o1}. If they ate {a} {o1}, how many pieces do they have left in total?

<<ch1 + ch2 - a>>

{p1} had {l1} {o1}. {p1} gave {g} {o1} to {p2}. How many {o1} does {p1} have left?

<<l1 - g>>

{p1} has {t} {o1}. For Christmas, {p1} got {tm} {o1} from {p2} and {td} {o1} from {p3}. How many {o1} does {p1} have now?

<<t + tm + td>>

There were {c} {o1} in the {loc}. {nc} more {o1} were installed each day, from {d1} to {d2}. How many {o1} are now in the {loc}?

<<c + nc * (d2 - d1 + 1)>>

{p1} had {gb1} {o1}. On {day1}, {p1} lost {l1} {o1}. On {day2}, {p1} lost {l2} more. How many {o1} does {p1} have at the end of {day2}?

<<gb1 - l1 - l2>>

{p1} has ${m}. {p1} bought {q} {o1} for ${p} each. How much money does {p1} have left?

<<m - q * p>>

{question}
\end{lstlisting}
\label{gram:gsm_prompt_no_cot}