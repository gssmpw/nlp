% \lstdefinestyle{myGrammarStyle}{
%     basicstyle=\scriptsize\ttfamily, % Reduce font size
%     commentstyle=\color{gray},
%     keywordstyle=\color{blue},
%     stringstyle=\color{orange},
%     numbers=left, % Line numbers on left
%     numberstyle=\tiny\color{gray}, % Line numbers styling
%     breaklines=true, % Wrap long lines
%     frame=single, % Frame around the code
%     framesep=3pt, % Adjust frame separation
%     xleftmargin=5pt, % Adjust left margin
%     xrightmargin=5pt, % Adjust right margin
%     backgroundcolor=\color{green!4}, % Background color
%     tabsize=2, % Tab size
%     captionpos=b, % Caption position bottom
%     aboveskip=5pt, % Reduce space above the listing
%     belowskip=5pt, % Reduce space below the listing
%     linewidth=0.9\linewidth, % Set custom line width to less than text width
%     escapeinside={(*@}{@*)}, % for escaping to LaTeX
% }

\begin{lstlisting}[style=myGrammarStyle, caption=Prompt Template Used For FOLIO Evaluation]
Given a problem description and a question. The task is to parse the problem and the question into first-order logic formulas.
The grammar of the first-order logic formula is defined as follows:
1) logical conjunction of expr1 and expr2: expr1 {and} expr2
2) logical disjunction of expr1 and expr2: expr1 {or} expr2
3) logical exclusive disjunction of expr1 and expr2: expr1 {xor} expr2
4) logical negation of expr1: {not}expr1
5) expr1 implies expr2: expr1 {implies} expr2
6) expr1 if and only if expr2: expr1 {iff} expr2
7) logical universal quantification: {forall} x
8) logical existential quantification: {exists} x. These are the ONLY operations in the grammar.
------

Answer the question EXACTLY like the examples.

Problem:
All people who regularly drink coffee are dependent on caffeine. People either regularly drink coffee or joke about being addicted to caffeine. No one who jokes about being addicted to caffeine is unaware that caffeine is a drug. Rina is either a student and unaware that caffeine is a drug, or neither a student nor unaware that caffeine is a drug. If Rina is not a person dependent on caffeine and a student, then Rina is either a person dependent on caffeine and a student, or neither a person dependent on caffeine nor a student.
Question:
Based on the above information, is the following statement true, false, or uncertain? Rina is either a person who jokes about being addicted to caffeine or is unaware that caffeine is a drug.
###

We take three steps: first, we define the necessary predicates and premises, and finally, we encode the question `Rina is either a person who jokes about being addicted to caffeine or is unaware that caffeine is a drug.` in the conclusion. Now, we will write only the logic program, nothing else.
Predicates:
Dependent(x) ::: x is a person dependent on caffeine.
Drinks(x) ::: x regularly drinks coffee.
Jokes(x) ::: x jokes about being addicted to caffeine.
Unaware(x) ::: x is unaware that caffeine is a drug.
Student(x) ::: x is a student.
Premises:
{forall} x (Drinks(x) {implies} Dependent(x)) ::: All people who regularly drink coffee are dependent on caffeine.
{forall} x (Drinks(x) {xor} Jokes(x)) ::: People either regularly drink coffee or joke about being addicted to caffeine.
{forall} x (Jokes(x) {implies} {not}Unaware(x)) ::: No one who jokes about being addicted to caffeine is unaware that caffeine is a drug. 
(Student(rina) {and} Unaware(rina)) {xor} {not}(Student(rina) {or} Unaware(rina)) ::: Rina is either a student and unaware that caffeine is a drug, or neither a student nor unaware that caffeine is a drug.
Conclusion:
Jokes(rina) {xor} Unaware(rina) ::: Rina is either a person who jokes about being addicted to caffeine or is unaware that caffeine is a drug.
------

Problem:
Miroslav Venhoda was a Czech choral conductor who specialized in the performance of Renaissance and Baroque music. Any choral conductor is a musician. Some musicians love music. Miroslav Venhoda published a book in 1946 called Method of Studying Gregorian Chant.
Question:
Based on the above information, is the following statement true, false, or uncertain? Miroslav Venhoda loved music.
###

We take three steps: first, we define the necessary predicates and premises, and finally, we encode the question `Miroslav Venhoda loved music.` in the conclusion. Now, we will write only the logic program, nothing else.
Predicates:
Czech(x) ::: x is a Czech person.
ChoralConductor(x) ::: x is a choral conductor.
Musician(x) ::: x is a musician.
Love(x, y) ::: x loves y.
Author(x, y) ::: x is the author of y.
Book(x) ::: x is a book.
Publish(x, y) ::: x is published in year y.
Specialize(x, y) ::: x specializes in y.
Premises:
Czech(miroslav) {and} ChoralConductor(miroslav) {and} Specialize(miroslav, renaissance) {and} Specialize(miroslav, baroque) ::: Miroslav Venhoda was a Czech choral conductor who specialized in the performance of Renaissance and Baroque music.
{forall} x (ChoralConductor(x) {implies} Musician(x)) ::: Any choral conductor is a musician.
{exists} x (Musician(x) {and} Love(x, music)) ::: Some musicians love music.
Book(methodOfStudyingGregorianChant) {and} Author(miroslav, methodOfStudyingGregorianChant) {and} Publish(methodOfStudyingGregorianChant, year1946) ::: Miroslav Venhoda published a book in 1946 called Method of Studying Gregorian Chant.
Conclusion:
Love(miroslav, music) ::: Miroslav Venhoda loved music.
------

{question}
\end{lstlisting}
\label{gram:folio_prompt}