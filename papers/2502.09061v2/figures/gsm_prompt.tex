


% \lstdefinestyle{myGrammarStyle}{
%     basicstyle=\scriptsize\ttfamily, % Reduce font size
%     commentstyle=\color{gray},
%     keywordstyle=\color{blue},
%     stringstyle=\color{orange},
%     numbers=left, % Line numbers on left
%     numberstyle=\tiny\color{gray}, % Line numbers styling
%     breaklines=true, % Wrap long lines
%     frame=single, % Frame around the code
%     framesep=3pt, % Adjust frame separation
%     xleftmargin=5pt, % Adjust left margin
%     xrightmargin=5pt, % Adjust right margin
%     backgroundcolor=\color{green!4}, % Background color
%     tabsize=2, % Tab size
%     captionpos=b, % Caption position bottom
%     aboveskip=5pt, % Reduce space above the listing
%     belowskip=5pt, % Reduce space below the listing
%     linewidth=0.9\linewidth, % Set custom line width to less than text width
%     escapeinside={(*@}{@*)}, % for escaping to LaTeX
% }

\begin{lstlisting}[style=myGrammarStyle, caption=CoT Prompt Template For GSM-Symbolic Evaluation]
You are an expert in solving grade school math tasks. You will be presented with a grade-school math word problem with symbolic variables and be asked to solve it.

Before answering you should reason about the problem (using the <reasoning> field in the response described below). Intermediate symbolic expressions generated during reasoning should be wrapped in << >>.

Then, output the symbolic expression wrapped in << >> that answers the question. The expressions must use numbers as well as the variables defined in the question. You are only allowed to use the following operations: +, -, /, //, %, (), and int().

You will always respond in the format described below: 
Let's think step by step. <reasoning> The final answer is <<symbolic expression>>

There are {t} trees in the {g}. {g} workers will plant trees in the {g} today. After they are done, there will be {tf} trees. How many trees did the {g} workers plant today?

Let's think step by step. Initially, there are {t} trees. After planting, there are {tf} trees. The number of trees planted is <<tf - t>>. The final answer is <<tf - t>>.

If there are {c} cars in the parking lot and {nc} more cars arrive, how many cars are in the parking lot?

Let's think step by step. Initially, there are {c} cars. {nc} more cars arrive, so the total becomes <<c + nc>>. The final answer is <<c + nc>>.

{p1} had {ch1} {o1} and {p2} had {ch2} {o1}. If they ate {a} {o1}, how many pieces do they have left in total?

Let's think step by step. Initially, {p1} had {ch1} {o1}, and {p2} had {ch2} {o1}, making a total of <<ch1 + ch2>>. After eating {a} {o1}, the remaining total is <<ch1 + ch2 - a>>. The final answer is <<ch1 + ch2 - a>>.

{p1} had {l1} {o1}. {p1} gave {g} {o1} to {p2}. How many {o1} does {p1} have left?

Let's think step by step. {p1} started with {l1} {o1}. After giving {g} {o1} to {p2}, {p1} has <<l1 - g>> {o1} left. The final answer is <<l1 - g>>.

{p1} has {t} {o1}. For Christmas, {p1} got {tm} {o1} from {p2} and {td} {o1} from {p3}. How many {o1} does {p1} have now?

Let's think step by step. {p1} started with {t} {o1}. {p1} received {tm} {o1} from {p2} and {td} {o1} from {p3}. The total is <<t + tm + td>>. The final answer is <<t + tm + td>>.

There were {c} {o1} in the server room. {nc} more {o1} were installed each day, from {d1} to {d2}. How many {o1} are now in the server room?

Let's think step by step. Initially, there were {c} {o1}. {nc} {o1} were added each day for <<d2 - d1 + 1>> days, which is <<nc * (d2 - d1 + 1)>>. The total is <<c + nc * (d2 - d1 + 1)>>. The final answer is <<c + nc * (d2 - d1 + 1)>>.

{p1} had {gb1} {o1}. On {day1}, {p1} lost {l1} {o1}. On {day2}, {p1} lost {l2} more. How many {o1} does {p1} have at the end of {day2}?

Let's think step by step. Initially, {p1} had {gb1} {o1}. After losing {l1} {o1} on {day1}, {p1} had <<gb1 - l1>>. After losing {l2} {o1} on {day2}, the total is <<gb1 - l1 - l2>>. The final answer is <<gb1 - l1 - l2>>.

{p1} has ${m}. {p1} bought {q} {o1} for ${p} each. How much money does {p1} have left?

Let's think step by step. Initially, {p1} had ${m}. {p1} spent <<q * p>> on {q} {o1}. The remaining money is <<m - q * p>>. The final answer is <<m - q * p>>.

{question}
\end{lstlisting}
\label{gram:gsm_prompt}