\section{Introduction}
\subsection{Motivation}
Selfish routing games, also known as competitive routing games, have been the subject of extensive study due to their wide-ranging applications in traffic and communication networks. These games involve multiple agents, such as road users or data packets, competing for limited resources like network paths and bandwidth. When considering road traffic, the idea of modeling with a continuous flow model appears unrealistic. Given that vehicles are indivisible and possess substantial size, the applicability of particles that can be fractionated at will seems dubious with respect to capturing all relevant dynamics. Consequently, there has been a recent surge in the study of atomic, dynamic flow models.

Werth et al.~\cite{WERTH201418} considered such a model with deterministic queues, which also ensures the first-in, first-out (FIFO) property, from a game-theoretic perspective. The authors examine two variants, each pertaining to distinct objective functions. For narrowest flows, the cost of a user is the most expensive resource on her path and users try to minimize these bottlenecks. However, it is NP-hard to determine according flows and the price of anarchy (PoA) is tightly bounded by the number of players. Conversely, when dealing with the \emph{makespan} objective function, where every user strives to minimize their arrival time, equilibria are easy to compute in the single commodity case. However, the PoA has been left open. Since the introduction of this model by Werth et al., numerous variants have been studied, but the PoA in the original model is still an unresolved problem.

 This paper aims to take a step towards addressing this gap. We find a constant upper bound of 2 for the class of linear multigraphs regarding the makespan objective and provide a sequence of linear multigraphs that converge to a price of stability ($\PoS$) and therefore also a PoA of at least $\frac{e}{e-1}$.

\subsection{Informal Model Description}
%\todo[inline]{Beweis Thm 10 in Hauptteil. N checken und durch $>0, \geq 0$ ersetzen. Breaks und Whitespace fixen}
Prior to delving into a comprehensive description of the model in Section~\ref{sec:prelim}, we offer an informal overview that is intended to provide a more effective categorization within the multifaceted realm of routing games.

We are examining a dynamic, competitive routing game involving atomic players on a directed graph $G=(V,E)$. Each edge $e \in E$ is equipped with an integral transit time $\tau(e) \in \N_{>0}$. Each player routes an unsplittable packet of unit size through this network from origin $s$ to destination $d$, that is, the players choose an $s-d$ path, start at time zero and try to reach $d$ as quickly as possible, i.e., players aim to minimize the arrival time at $d$. Furthermore, each edge is equipped with a point queue. At any point in time arbitrarily many packets may enter such a queue, but the queue has a limited outflow rate, that is at most $\nu_e$ packets may leave the queue at a particular time step. These queues operate according to the FIFO principle, ensuring that packets exit in the same sequence they entered. There is no limit to the maximum storage capacity of these queues. Consequently, the total latency on an edge is composed of the transit time and any potential waiting time in the queue.

In particular, we consider so-called uniformly fastest route (UFR) equilibria that have also been analyzed by~\cite{DBLP:journals/ior/ScarsiniST18}. Such a UFR equilibrium is achieved if for every player and for every node $v$ on its route, there exists no other route such that this player can arrive earlier at $v$. In other words, a UFR equilibrium also takes arrival time at intermediate nodes into account.
For the social optimum, we consider the makespan of the problem, that is, assuming all players start at time zero, we minimize the arrival time of the last player.

\subsection{Related Literature}

The diversity of routing games mirrors the breadth of their applications. We will give a brief overview of the origin and development, focusing particularly on the most relevant results for the present work.

A shared characteristic across all games is that each player possesses a utility function, typically characterized by parameters such as latency\footnote{In the literature latency is sometimes also called travel time.}~(see, e.g.,~\cite{braess1968paradoxon,kochskutella2011,pigou,DBLP:journals/ior/ScarsiniST18}, residence time~\cite{cao2022bounding}, bottleneck~\cite{WERTH201418}, or arrival penalties\footnote{ Deviation from the desired arrival time is often taken into account as an additive penalty in the objective of dynamic traffic assignment models (DTA) used in the traffic community.}~\cite{HAN201317} and chooses a path (strategy) from its origin to its destination. A state in the game is referred to as a (pure) Nash equilibrium when no individual player can decrease the private utility function by unilaterally altering their route.

Equilibrium solutions are often compared to a system optimum with respect to some objective function. %The system optimum represents the solution that a central authority would achieve if it optimized all paths to enhance the global efficiency of the traffic network, even if it meant compromising individual latencies. 
Common objectives include minimizing the total latency (sum over all road users)~\cite{harks2018competitive}, the total completion time (also known as makespan)~\cite{WERTH201418}, the (average) delay~\cite{DBLP:journals/ior/ScarsiniST18}, 
or the throughput~\cite{kochskutella2011}. The ratio between the worst equilibrium and the system optimum is referred to as the price of anarchy ($\PoA$). In addition, we also consider the price of stability ($\PoS$), which refers to the ratio between the best equilibrium and the system's optimal state.  

In the categorization of such competitive routing games, it is necessary to make several differentiations. Initially, the games can be classified as either static or dynamic. In a static setup, players choose their path, and interference arises when both paths share the same edge. This interference can be attributed to load-dependent latencies~\cite{roughgarden2005selfish}, or due to the edge capacities that constrain the total flow on an edge~\cite{correa2004selfish}. Pigou's two link network~\cite{pigou} and Braess’ Paradox~\cite{braess1968paradoxon} are pioneering examples in the realm of static games.

Conversely, dynamic games incorporate the element of time, with particles navigating the network in a time-sensitive manner. Typically, the subsequent player may experience delays due to the preceding player, as the former must wait until the latter clears the next edge. Furthermore, it is common to require adherence to the FIFO principle. Moreover, the capacity typically constrains the flow rate (measured in flow units per time unit), rather than the aggregate flow passing through an edge. A notable early and often adapted example is Vickrey’s queuing model~\cite{vickrey1969congestion}, which introduced point queues of no physical space and factored in varying departure times. 
Since then, the game-theoretic aspects of flows over time have been a topic of extensive discussion within the traffic community. From the perspective of algorithmic game theory, Koch and Skutella~\cite{kochskutella2011} have demonstrated that Nash flows over time with the underlying deterministic queuing model can be interpreted as a sequence of specific static flows. %This model is widely adapted. For example, Sering and Vargas~Koch succeeded in adding upper bounds on queue lengths and spillback to this setting~\cite{sering2019spillback}.


The second differentiation pertains to non-atomic versus atomic games. In non-atomic flows, one can envision an infinite number of players, each controlling an infinitesimally small packet. %In this scenario, the decision of a single player does not significantly alter the overall outcome. 
On the other hand, atomic flows, as introduced by Rosenthal~\cite{rosenthal1973network}, are characterized by each entity controlling a substantial portion of the total flow. Typically, these games involve a finite number of players, each with an unsplittable packet of unit size.

Non-atomic models are relatively well-understood due to their amenability to analytical methods. 
For example, in non-atomic models, the value of an equilibrium flow is often unique~\cite{cominetti2011existence}.
On the other hand, the $\PoA$ is still an open question in many dynamic scenarios. A significant advancement was made by Correa et al.~\cite{DBLP:journals/mor/CorreaCO22}, where they established that if one is permitted to restrict the inflow rate of the equilibrium to the optimum flow’s initial inflow rate, the $\PoA$ concerning the makespan is bounded by $\frac{e}{e-1}$. Unfortunately, these findings are contingent upon a monotonicity conjecture~\cite{DBLP:journals/mor/CorreaCO22}. Despite its intuitive appeal, this conjecture has only been validated for relatively straightforward graph classes, such as linear multigraphs, to date.

The suitability of non-atomic models, particularly for depicting scenarios such as road traffic, is a subject of ongoing debate, since it is unclear whether particle size is irrelevant in such applications.
A game-theoretic approach based on a dynamic, discrete model with point queues was introduced by Werth~\cite{WERTH201418}.
Subsequent to this approach, numerous variants of atomic routing games have been explored, given that these models necessitate the incorporation of additional concepts or intricate modeling details. A particular aspect is simultaneity: what happens when two players want to use the same edge at the same time? This necessitates a concept for parallel processing (fair time sharing, see~\cite{hoefer2011competitive}) or tie-breaking rules. For the latter, predefined priorities on players~\cite{harks2018competitive,WERTH201418} or edges~\cite{DBLP:conf/sigecom/CaoCCW17,scheffler2022routing,WERTH201418} are typically employed. %It is worth mentioning that the tie-breaking with player priorities in Harks et al.~\cite{harks2018competitive} and with edge priorities in Scheffler et al.~\cite{scheffler2022routing} is executed as a substitute for the application of the FIFO rule. Nevertheless, players can indirectly reestablish the FIFO property and prevent being overtaken by using a suitable strategy.

%Many other properties of the models can significantly influence the results. For instance, in symmetric games, all players share the same strategy set, which implies that all players have identical origins and destinations (single commodity). Conversely, also problems with multiple sources or sinks are studied (see, e.g., for Nash flows over time). It is worth noting that in (atomic) models with multiple sources and sinks (multi-commodity), pure Nash equilibrium solutions may not even exist~\cite{roughgarden2005selfish,scheffler2022routing,wang2023atomic}. Other influential properties include the specific queuing model, inflow or outflow capacities, the choice of starting time (e.g., all at once, fixed generations, or time windows), full or incomplete information about the route choice of other users, and whether adjustments to the route at intermediate nodes are permissible or not.
Atomic models are often more challenging to analyze since they usually do not have unique equilibria, but some results were obtained for special setting. For example, Scarsini et al.~\cite{DBLP:journals/ior/ScarsiniST18} studied a game where the inflow occurs in generations of players, also emphasizing the relevance of UFR equilibria. Similarly, Cao et al.~\cite{DBLP:conf/sigecom/CaoCCW17} studied a game where the inflow rate never exceeds the network capacity, adding the concept of dynamic route choices.


%Scarsini et al.~\cite{DBLP:journals/ior/ScarsiniST18} studied a game where the inflow occurs in generations of players, that is, a generation of finitely many players enters the network at each time step in an endless process. 
%For a network with two nodes, parallel links, and uniform generations, which have at most as many players as the minimum cut of the network, the queue lengths grow until the latency reaches the length of the longest edge used in the system optimum. Here, the equilibrium is unique and all later generations suffer from the greedy route choices of the first generations. Furthermore, the authors present more complex networks with $n$ nodes with non-unique UFR equilibria, where the ratio between $\PoA$ and $\PoS$ is $n-1$ when considering the total latency of a generation as objective. Moreover, they show Braess paradox-like phenomena. In particular, the equilibrium cost might decrease when the length of an edge is increased or when there are initial queues in the network. 

%Similarly, Cao et al.~\cite{DBLP:conf/sigecom/CaoCCW17} studied a game where the inflow rate never exceeds the network capacity, adding the concept of dynamic route choices, i.e., users choose their path while
%traveling and may reroute at every node. For this game on series-parallel graphs, the authors show that the queue lengths are bounded by a network dependent constant, and, hence, also the $\PoA$ (with respect to the average latency) is bounded. The same authors also studied bounds on queue lengths in a more general setting~\cite{cao2022bounding}. They show that queues in an acyclic network, where the inflow is restricted by the minimum cut, are bounded by $(12m)^m$ under any arbitrary route selection. Here, $m$ represents the total number of edges, each with a unit transit time and unit capacity. This finding also establishes a bound on the $\PoA$, which is independent of the number of players but is impractically large for instances that are relevant in practice.

However, the relationship between discrete and continuous routing models remains partially elusive. When considering system optimal solutions for a single source and a single sink, the similarities are pronounced. Already Ford and Fulkerson~\cite{ford1958constructing} pioneered the computation of dynamic flows for discrete time based on static flow computations. Fleischer and Tardos~\cite{DBLP:journals/orl/FleischerT98} successfully addressed this problem for continuous time. A comprehensive summary is available by Skutella~\cite{DBLP:conf/bonnco/Skutella08}. The situation becomes more intricate when examining equilibria. A Nash equilibrium in the continuous case is an equilibrium in the discrete case, if and only if the continuous flow can be interpreted as a flow of discrete packets, which is not possible in general. Conversely, while every state in the discrete setting can be interpreted as a feasible continuous flow~\cite{DBLP:journals/orl/FleischerT98}, the equilibrium property is often lost. That is, a Nash equilibrium in the discrete case does not necessarily translate to an equilibrium in the continuous case.
%Recently, it was shown that Nash equilibria in an atomic model converge to the equilibrium solution of a non-atomic model when one refines the packet size and increases the number of players accordingly~\cite{olver2024convergence,DBLP:journals/mor/SeringKZ23}.
% Another way to obtain equilibrium solutions is to actually play the game in a multi-agent simulation tool like MATSim~\cite{matsim}. Experimental evidence for a strong connection between the analytical model of flows over time and the co-evolutionary approach of MATSim, when the vehicle size and the step size is reduced, was given in~\cite{ziemke2021flows}.

%We should not fail to mention that calculating the system optimum can often be challenging, too. In static problems, computing the system optimum usually involves a type of min-cost-flow computation, which can be NP-hard in the case of atomic multi-commodity games~\cite{even1975complexity}. 

%In dynamic problems, the system optimum is also a flow over time. For the non-atomic case and constant latencies, we have to deal with quickest flows which can be chosen temporally repeated (see~\cite{DBLP:conf/bonnco/Skutella08} for a sound introduction). Köhler and Skutella studied a flow over time model with load-dependent latencies~\cite{koehler2005flows}. 
%Dynamic atomic flows have been intensively studied in the context of communication networks and parallel computing (see, e.g.,~\cite{cidon1995greedy} for early developments). Due to the need for ultra-fast algorithms, it is a common design choice to consider the path selection and the queuing policy separately~\cite{leighton1988universal}. For the latter, delaying packet release can strategically enhance overall throughput. In~\cite{mansour1991greedy}, the efficiency of greedy scheduling that consistently forwards a packet whenever possible is discussed. This queuing policy can be viewed as the system optimum analogous to UFR.

\subsection{Our Contribution}

We consider an atomic dynamic routing game on linear multigraphs\footnote{This graph class is also known in the literature as chain-of-parallel networks~\cite{DBLP:conf/sigecom/CaoCCW17,DBLP:journals/ior/ScarsiniST18}.}, where we have a linearly ordered node set $V=\{v_1,\dots,v_m\}$ and the edge set consists of multiple edges $(v_j,v_{j+1})$. For this graph class, we prove the $\PoA$ to be at most 2, presenting the first result of a constant $\PoA$ on a graph class completely independent of the number of players or the network size. Specifically, we do not impose artificial restrictions on the inflow rate of the network. Furthermore, we show that whenever players have the choice between multiple quickest paths, the sum of queues in the network is maximized when players greedily opt for the longest queues available.

Moreover, we show that there are instances of linear multigraphs, where the $\PoS$ converges to at least $\frac{e}{e-1}$, indicating that discrete Nash flows even on such a restricted graph class are non-trivial. Furthermore, this result aligns with the upper bound for the $\PoA$ in a non-atomic routing game investigated by Correa et al.~\cite{DBLP:journals/mor/CorreaCO22} where the inflow rate is restricted to that of an optimal flow and a certain monotonicity conjecture holds. %The authors of~\cite{DBLP:journals/mor/CorreaCO22} establish the validity of their conjecture on monotonicity on linear multigraphs. 
We show that our instances can be translated to the setting of Correa et al., thereby establishing a lower bound of $e/(e-1)$ on the $\PoA$ in their context. With the monotonicity conjecture validated for these instances by the authors of~\cite{DBLP:journals/mor/CorreaCO22}, we achieve the first tight bound on the $\PoA$ for a non-trivial class of graphs concerning Nash flows over time.

This paper is organized as follows. Upon establishing the required notation, we proceed to examine system optimal flows within our framework in Section~\ref{sec:understandingopt}. Subsequently, we demonstrate the unique structure of the least favorable UFR equilibrium and establish an upper bound of 2 on the $\PoA$ on linear multigraphs in Section~\ref{sec:poa}. We wrap up our findings in Section~\ref{sec:pos} by formally showing a strong connection to the dynamic model and by establishing a lower bound on the $\PoS$ of $\frac{e}{e-1}$ for this class of graphs. Please note that some proofs have been omitted and are provided in the appendix.