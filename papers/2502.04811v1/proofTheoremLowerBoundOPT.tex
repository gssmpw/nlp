Let $P^1$ denote a shortest $s-d$ path in $G$ and $P^j$ a shortest $s-d$ path in $G$ after deleting paths $P^1, \dots, P^{j-1}$. According to Section \ref{sec:understandingopt}, there is an optimal state $S^*_i$ of $\Gamma_i$ of the following form. $C^{\Gamma_i}(S^*_i)+1-\tau(P^1)$ packets use $P^1$ and $C^{\Gamma_i}(S^*_i)+\delta_j -\tau(P^j)$ packets use $P^j$ with $\delta_j\in\{0,1\}$ for \mbox{$2\leq j \leq k'\leq k$} with \mbox{$\sum_{j=2}^{k'} \delta_j$}$ = $\mbox{$n_i-k'\cdot C^{\Gamma_i}(S^*_i) -1 +\sum_{j=1}^{k'} \tau(P^j)$}. 
% Furthermore, it holds that $\tau(P^{k'}) \leq C^{\Gamma_i}(S^*_i)$ and if $k'<k$ we have $\tau(P^{k'+1}) > C^{\Gamma_i}(S^*_i)$. 
Together with the fact that the optimal state does not use paths $P^{k'+1}, \dots, P^k$, i.e., $\tau(P^k)\geq \dots \geq \tau(P^{k'+1}) > C^{\Gamma_i}(S^*_i)$, we obtain that the total number of packets sent by $S^*_i$ is
\begin{align*}
    n_i &= C^{\Gamma_i}(S^*_i)+1-\tau(P^1) + \sum_{j=2}^{k'} \left(C^{\Gamma_i}(S^*_i)+\delta_j -\tau(P^j)\right)\nonumber\\
    &= k'\cdot C^{\Gamma_i}(S^*_i)+1-\sum_{j=1}^{k'}\tau(P^j) + \sum_{j=2}^{k'} \delta_j \\
    &\geq k'\cdot C^{\Gamma_i}(S^*_i)-\sum_{j=1}^{k'}\tau(P^j)\nonumber\\
    &\geq k\cdot C^{\Gamma_i}(S^*_i)-\sum_{j=1}^{k}\tau(P^j)  \\
    &= k \cdot C^{\Gamma_i}(S^*_i)-  \Biggl(\underbrace{\sum\limits_{j=1}^k j - \sum\limits_{j=1}^{\ell} j }_{\text{standard edges}}+ \underbrace{\sum\limits_{j=2}^{k-\ell} (j-1)\tau_j}_{\text{special edges}}\Biggr)\nonumber\\
    &= k\cdot C^{\Gamma_i}(S^*_i) - \left(\frac{k^2+k}{2} - \frac{\ell^2+\ell}{2} + \sum\limits_{j=2}^{k-\ell} (j-1)\tau_j\right).%\label{eq:opt}
\end{align*}
Rearranging terms implies an upper bound on $C^{\Gamma_i}(S^*_i)$.
\[C^{\Gamma_i}(S^*_i)\leq \frac{1}{k}\left(n_i+\frac{k^2+k}{2} - \frac{\ell^2+\ell}{2} +\sum\limits_{j=2}^{k-\ell} (j-1)\tau_j\right)\]
%
With the help of the telescope sum 
\begin{align*}
    &\sum\limits_{j=2}^{k-\ell} (j-1)\left( \frac{n}{k-j+1} -  \frac{n}{k-j+2}+2\right) \nonumber\\
    = & (k-\ell)(k-\ell-1) + (k-\ell-1)\left( \frac{n}{\ell+1}\right) - \sum_{j=0}^{k-\ell-2} \frac{n}{k-j}% \label{eq:teleskop}
\end{align*}
we get
\begin{align}
    % &\leq 1 + \frac{1}{k} \left( n_i+\frac{k^2+k}{2} - k - \frac{l^2+l}{2} + \sum\limits_{j=2}^{k-\ell} (j-1)\left(\left\lceil \frac{n_i}{k-j+1}\right\rceil - \left\lceil \frac{n_i}{k-j+2}\right\rceil+1\right)\right)\nonumber\\
    C^{\Gamma_i}&(S^*_i) \leq \frac{1}{k}\left(n_i+\frac{k^2+k}{2} - \frac{\ell^2+\ell}{2} + (k-\ell)(k-\ell-1) \right.\nonumber\\
    &\hspace{1.7cm}\left.+ (k-\ell-1)\left( \frac{n_i}{\ell+1}\right) - \sum\limits_{j=0}^{k-\ell-2} \frac{n_i}{k-j}\right)\nonumber\\
    &= \left( \frac{3k^2-4k\ell-k+\ell^2+\ell}{2k}\right) + n_i \cdot \left( \frac{k-\ell-1}{(\ell+1)k} + \frac{1}{k} -\frac{1}{k}\sum\limits_{j=\ell+2}^{k} \frac{1}{j}\right).\label{eq:optUB}
\end{align}
For the j-th harmonic number $H_j$, it is well known that $H_j= \ln(j) + \gamma + \frac{1}{2j} - \varepsilon_j$ with $0\leq \varepsilon_j \leq \frac{1}{8j^2}$ and the Euler-Mascheroni constant $\gamma \approx 0.577$. Thus,
\begin{align}
    \lim\limits_{\ell \rightarrow \infty} \sum\limits_{j=\ell+2}^{\left\lceil e \ell\right\rceil} \frac{1}{j} 
    &= \lim\limits_{\ell \rightarrow \infty} \left(H_{\left\lceil e \ell\right\rceil}-H_{\ell+1}\right)\nonumber \\
    % &= \lim\limits_{\ell \rightarrow \infty} \left(\ln(\left\lceil e \ell\right\rceil) + \frac{1}{2\left\lceil e \ell\right\rceil} - \varepsilon_{\left\lceil e \ell\right\rceil}%  \right. \nonumber\\ &\hspace{1.7cm}\left.
    % - \ln(\ell+1) - \frac{1}{2(\ell+1)} + \varepsilon_{\ell+1}\right) \nonumber\\
    &=  \lim\limits_{\ell \rightarrow \infty} \left(\underbrace{\ln\left(\frac{\left\lceil e \ell\right\rceil}{\ell+1}\right)}_{\rightarrow \ln(e) = 1} + \underbrace{\frac{1}{2\left\lceil e \ell\right\rceil} - \varepsilon_{\left\lceil e \ell\right\rceil} - \frac{1}{2(\ell+1)} + \varepsilon_{\ell+1}}_{\rightarrow 0}\right) = 1 .\label{eq:harmonic}
\end{align}
Since the growth of $n_i$ dominates the term, we obtain 
\begin{align*}
    \PoS(\mathcal{G}) &\geq \lim\limits_{i \rightarrow \infty} \PoS(\Gamma_i) =\lim\limits_{i \rightarrow \infty} \frac{C^{\Gamma_i}(S_i)}{C^{\Gamma_i}(S^*_i)}\\
    &\hspace{-0.25cm}\overset{(\ref{eq:nash}), (\ref{eq:optUB}) }{\geq }\lim\limits_{i \rightarrow \infty} \frac{(k-\ell -1) + n_i\cdot \left(\frac{1}{\ell+1}\right)}{ \left( \frac{3k^2-4k\ell-k+\ell^2+\ell}{2k}\right) + n_i \cdot \left( \frac{k-\ell-1}{(\ell+1)k} + \frac{1}{k} -\frac{1}{k}\sum\limits_{j=\ell+2}^{k} \frac{1}{j}\right)}.
\end{align*}
By simplifying this expression, we get
\begin{align*}
    \PoS(\mathcal{G})&\geq \lim\limits_{i  \rightarrow \infty} \frac{\frac{1}{\ell+1}}{ \frac{k-\ell-1}{(\ell+1)k} + \frac{1}{k} -\frac{1}{k}\sum\limits_{j=\ell+2}^{k} \frac{1}{j}} 
    = \lim\limits_{i \rightarrow \infty} \frac{1}{ 1 -\frac{\ell+1}{k} \sum\limits_{j=\ell+2}^{k} \frac{1}{j}} \\
    &=  \lim\limits_{i \rightarrow \infty} \frac{1}{ 1 -\underbrace{\frac{\ell+1}{\left\lceil e \ell\right\rceil}}_{\rightarrow \frac{1}{e}} \underbrace{\sum\limits_{j=\ell+2}^{\left\lceil e \ell\right\rceil} \frac{1}{j}}_{\rightarrow 1, \text{ due to } (\ref{eq:harmonic})}} 
    = \frac{1}{1-\frac{1}{e}\cdot 1}
    = \frac{1}{\left(\frac{e-1}{e}\right)}
    = \frac{e}{e-1}\;,
\end{align*}
which finishes the proof.