\section{Preliminaries}\label{sec:prelim}
We consider the classic packet routing game introduced by~\cite{WERTH201418}. Formally, a packet routing game $\Gamma$ is given by the tuple $\Gamma =(N, G=(V,E), \tau: E \rightarrow \N_{>0},$ $s \in V, d \in V)$,
where $N=[n]\coloneq \{1, \dots, n\}$ with $n\in \N_{>0}$ is the player set, $G$ is a multigraph with nodes $V$ and directed edges $E$. Additionally, $\tau(e) \in \N_{>0}$ denotes the transit time of an edge $e \in E$. Each player has one packet located at the source node $s \in V$ and aims to send this packet as fast as possible to the destination node $d$.

In this work, we focus on packet routing games on \emph{linear multigraphs} and show the first non-trivial bounds on the $\PoA$ for packet routing games with FIFO policy. A linear multigraph for $s\neq d$ is given by $G=(V,E)$ with \mbox{$V=\lbrace s=v_0, v_1, \ldots , v_m = d\rbrace$}  and $j=j'+1$ for all edges $(v_{j'},v_j)\in E$. We call all edges from $v_{j-1}$ to $v_{j}$ together with the nodes $v_{j-1}$ and $v_{j}$ the $j$-th layer of the graph. 
We assume the edges $e^j_1, e^j_2, \ldots$ in each layer $j$ to be ordered with respect to their transit times $\tau(e^j_1)\leq \tau(e^j_2)\leq \ldots$. 
We denote the set of all linear multigraphs by $\mathcal{G}$.
 
Since each player is associated with a single packet, we will refer to the packet of a player as the player itself. Given $G$, $s$, and $d$, the strategy space $\mathcal{S}\coloneqq \mathcal{P}^n$ for the players is given by all simple $s-d$ paths $\mathcal{P}$ in $G$. A tuple $S=(P_{(1)}, \dots, P_{(n)}) \in \mathcal{S}$ is called a state of the game. Each state $S \in \mathcal{S}$ induces a network loading in the following sense.

\paragraph{Network Loading:}
Given a state $S$ of the game, we define the positions of every player at any point in time by the following algorithm: 
We initialize at time $t=0$ all queues $(q_e)_{e\in E}$ to be empty and position all packets at $s$. We add each player to the queue of the first edge in their paths. Players in queues are ordered according to FIFO, i.e., by their arrival time at $e$, where we break ties according to the player's index in favor of the player with the lower index.
We now iterate over $t$ until all packets have arrived at $d$:
For each edge $e$ we add all players $i$ with $e \in P_{(i)}$ who have left the queue of their previous edge $e'$ on their path at time $t-\tau(e')$ to queue $q_e$. Subsequently, for every edge $e$ with a non-empty queue, we remove the first player from the queue.\footnote{Note that we assume edge capacities to be equal to 1 to simplify notation, but we can extend the model to arbitrary edge capacities as we discuss as we discuss in Appendix~\ref{sec:edgcap}.} Hence, the player arrives at $\tau(e) + t$ at the head of $e$. If a player $i$ is both added to $q_e$ and removed from $q_e$ at the same time $t$, we say \emph{player $i$ does not queue on $e$}. Furthermore, when a player leaves the queue of an edge at time $t$, we refer to this edge as \emph{used (at time $t$)}. Note that this procedure can be turned into a polynomial time algorithm by keeping track and iterating only over relevant times $t$, where players leave queues.

We now introduce some additional notation for the network loading:
The \emph{waiting time} $w_e^i(S)$ of player $i$ on edge $e$ is defined as the time that player $i$ spends in queue $q_e$. The \emph{latency} $l_e^i(S)$ of player $i\in N$ on edge $e \in E$ is defined by $l_e^i(S) \coloneqq \tau(e) + w^i_e(S)$. 
The \emph{workload} $l_{e}(S,t)$ on an edge $e$ at time $t$ under the strategy profile $S$ is defined as the latency that a fictional player entering $e$ at time $t$ would experience on that edge if she had the largest index among all players present at $q_e$ at that time.
For the chosen $s-d$ path $P_{(i)} = (e_1,\ldots, e_m)$ of player $i$, the arrival time $a_{v_j}^i(S)$ of $i$ at the head $v_j$ of the edge $e_j$ is given by $a_{v_j}^i(S) = \sum_{r=1}^{j} l_{e_r}^i(S)$. For a fixed strategy profile $S$, this yields a uniquely defined \emph{arrival pattern} $a_v(S) = (a_v^i(S))_{i \in N}$ for every node $v\in V$, which we interpret as an ordered vector of the arrival times of all players for that node. For $v=s$, we set $a_s(S)= 0^n\coloneq(0,\ldots,0)\in \N_{\geq 0}^n$, since all players start at time $0$ at $s$. We call the number of players that arrive at node $v$ at time $t$ the \emph{inflow of $v$ at time $t$}. We refer to all players with arrival time $t$ at node $v$ as the \emph{generation $t$ at $v$}. We define the \emph{completion time} of player $i\in N$ to be $C_i(S)\coloneq a_{d}^i(S)$ and the (total) \emph{completion time} of the state to be $C(S)\coloneq \max_{i\in N} C_i(S)$. Since the network loading algorithm uniquely determines the positions of all players at any given time, we can compute all these values through a post-processing step.

\begin{figure}[t]
    \centering
        \tikzstyle{vertex}=[circle,fill=black!25,minimum size=20pt,inner sep=0pt]
        \tikzstyle{smallvertex}=[circle,fill=black,minimum size=6pt,inner sep=0pt]
        \tikzstyle{edge} = [draw,thin,->]
        \tikzstyle{weight} = [font=\small]
        \tikzstyle{selected edge} = [draw,line width=3pt,-,red!50]
        \tikzstyle{big edge} = [draw,line width=2pt,->,black]

        \begin{tikzpicture}[scale=0.55,auto,swap]
            \node[vertex](s1) at (0,0){$s$};
            \node[vertex](s2) at (3,0){$v$};
            \node[vertex](s3) at (7,0){$d$};

            \draw [edge] (s1) to[out=45,in=135, distance=1cm ] node[weight,above,black]{$\tau(e_1^1)=1$} (s2);
            \draw [edge] (s1) to[out=-45,in=-135, distance=1cm ] node[weight,below=0.2,black]{$\tau(e_2^1)=2$} (s2);

            \draw [edge] (s2) to[out=0,in=180, distance=1cm ] node[weight,below=0.25,black]{$\tau(e_1^2)=2$} (s3);

            \node[draw, fill=white] (P1) at (3.75,0) {$1$};
            \node[draw, fill=white] (P3) at (0.6,0.6) {$3$};
            \node[draw, fill=white] (P2) at (1.5,-1) {$2$};
        \end{tikzpicture}
%
\hspace{1cm}
        \begin{tikzpicture}[scale=0.55,auto,swap]
            \node[vertex](s1) at (0,0){$s$};
            \node[vertex](s2) at (3,0){$v$};
            \node[vertex](s3) at (7,0){$d$};

            \draw [edge] (s1) to[out=45,in=135, distance=1cm ] node[weight,above,black]{$\tau(e_1^1)=1$} (s2);
            \draw [edge] (s1) to[out=-45,in=-135, distance=1cm ] node[weight,below=0.2,black]{$\tau(e_2^1)=2$} (s2);

            \draw [edge] (s2) to[out=0,in=180, distance=1cm ] node[weight,below=0.25cm,black]{$\tau(e_1^2)=2$} (s3);

            \node[draw, fill=white] (P1) at (5,0) {$1$};
            \node[draw, fill=white] (P2) at (3.75,0) {$2$};
            \node[draw, fill=white] (P3) at (3.75,0.75) {$3$};
        \end{tikzpicture}
    \caption{An example to illustrate the notation. The left figure depicts the positions of players at $t=1$, the right figure for $t=2$.}
    \label{fig:SimpleExample}
\end{figure}

\paragraph{Social Objective:} 
The social objective is to minimize the completion time $C(S)$ among all $S\in \mathcal{S}$.
A state $S^*\in\mathcal{S}$ is called an \emph{optimal state} for the game if $S^* \in \argmin_{S \in \mathcal{S}} C(S)$.

\paragraph{Example:} 
To illustrate notation, we refer the reader to Figure~\ref{fig:SimpleExample}. For the player set $N=[3]$ and the strategies $S=((e_1^1, e_1^2),(e_2^1, e_1^2),(e_1^1, e_1^2))$ we obtain the waiting times $w_{e_1^1}^1(S)=w_{e_2^1}^2(S)=0$, $w_{e_1^1}^3(S)=1$ in the first layer. The latencies in the first layer are $l_{e_1^1}^1(S)= 1, l_{e_2^1}^2(S)=l_{e_1^1}^3(S)=2$ concluding in the workloads of $l_{e_1^1}(S,0)= l_{e_2^1}(S,0)=3$ and the arrival pattern $a_v(S)=(1,2,2)$ at $v$.
 For the second layer we obtain waiting times $w_{e_1^2}^1(S)=w_{e_1^2}^2(S)=0$, $w_{e_1^2}^3(S)=1$, latencies $l_{e_1^2}^1(S)= l_{e_1^2}^2(S)=2$, $l_{e_1^2}^3(S)=3$, workloads $l_{e_1^2}(S,0)=2$, $l_{e_1^2}(S,1)=3$, $l_{e_1^2}(S,2)=4$, $l_{e_1^2}(S,3)=3$ and the arrival pattern at $d$ is $a_d(S)=(3,4,5)$ with completion time $C(S)=5$.

\paragraph{Equilibria:}
A state $S=(P_{(1)}, \dots, P_{(n)}) \in \mathcal{S}$ is called a uniformly fastest route \emph{(UFR) equilibrium} if for every player $i\in N$, $P\in\mathcal{P}$ and node $v\in P_{(i)}\cap P$ it holds that $a_{v}^i(S)\leq a_{v}^i(S')$ for $S'=(P_{(1)}, \dots, P_{(i-1)}, P, P_{(i+1)}, \dots, P_{(n)})$.
 This means that in a UFR equilibrium, no player can achieve a better arrival time at any node $v$ on her chosen path by unilaterally altering her strategy to another path. Following the work of Scarsini et al.~\cite{DBLP:journals/ior/ScarsiniST18}, we believe this definition aptly
reflects the self-interested behavior of players in road traffic and we consider only this kind of equilibria, here\footnote{Details can be found in Appendix \ref{sec:equi}.}. The set of all such UFR equilibria is denoted by $\Nash$. To improve readability, we often drop the term \emph{UFR} and simply call them \emph{equilibria} in the rest of the paper. It should be noted that this set is not empty. Assume all players select their paths sequentially in the order of their indices, starting with the player with the smallest index, and every player chooses a uniform fastest route given the choices of the players with lower indices. Since none of the later players can arrive at an intermediate node before a player with lower index, no player can displace an earlier player and we indeed obtain an equilibrium.

We start by observing that the completion time in any equilibrium of our packet routing game is realized by the player with the highest index. 
\begin{lemma}\label{lemma_order}
    For every packet routing game $\Gamma$ on a linear multigraph and $S\in \Nash$ it holds that $C(S)= C_n(S)$. 
\end{lemma}
\begin{proof}
We will show the even stronger statement that $a_{v}^i(S)\leq a_{v}^{i+1}(S)$ at every node $v\in V$. Assume this would not hold. Let $v_j$ be the node with smallest index with $a_{v_j}^i(S)> a_{v_j}^{i+1}(S)$. Assume that player $i$ would change her strategy only in this layer by choosing the same edge $e$ as player $i+1$. Since $a_{v_{j-1}}^i(S)\leq a_{v_{j-1}}^{i+1}(S)$ the player $i$ enters the queue $q_e$ not later than player $i+1$. Since the state is the same up to node $v_{j-1}$ the time that other players that use edge $e$ arrive at edge $e$ does not change. Hence, player $i$ is removed from the queue not later than player $i+1$ is removed in state $S$ from this queue and thus arrives strictly earlier at $v_j$ than in $S$. This contradicts $S\in \Nash$.
\end{proof}
\noindent In essence, the proof of Lemma~\ref{lemma_order} also shows that all equilibria share a common characteristic with the specially constructed equilibrium above: all players arrive at every node and in particular at the destination $d$ in the order of their indices.

\paragraph{Price of Anarchy and Price of Stability:}
In terms of the social objective, our goal within the game $\Gamma$ is to minimize the completion time $C(S)$ across all $S\in \mathcal{S}$, i.e., the arrival time of the player arriving last at the destination node $d$. Since the players only optimize their own arrival times and not the social objective function, this equilibrium might not be efficient with respect to this social objective function.
Two popular measures of inefficiency are the price of anarchy ($\PoA$) and the price of stability ($\PoS$) which for a given packet routing game $\Gamma$ and an optimal state $S^*$ are defined by
\begin{align*}
    \PoA(\Gamma) \coloneqq \frac{\max\limits_{S \in \Nash} C(S)}{\min\limits_{S \in \mathcal{S}} C(S)} =\frac{\max\limits_{S \in \Nash} C(S)}{C(S^*)}, \; \PoS(\Gamma) \coloneqq \frac{\min\limits_{S \in \Nash} C(S)}{\min\limits_{S \in \mathcal{S}} C(S)} =\frac{\min\limits_{S \in \Nash} C(S)}{C(S^*)}
\end{align*}
and for a set $\mathcal{H}$ of graphs as
\begin{align*}
    \PoA(\mathcal{H}) \coloneqq \sup\limits_{\Gamma: G(\Gamma) \in \mathcal{H}} \PoA(\Gamma), \quad \PoS(\mathcal{H}) \coloneqq \sup\limits_{\Gamma: G(\Gamma) \in \mathcal{H}} \PoS(\Gamma).
\end{align*}
By definition we have $\PoS(\mathcal{H})\leq \PoA(\mathcal{H})$.
If the underlying game $\Gamma$ is not immediately apparent from context, it will be indicated through a superscript to all definitions.