\begin{proof}
It is sufficient to show that the statement holds for starting patterns $a$ and $b$ which differ in only one entry by one. Then, a general $a$ can be transformed into $b$ by step-wise transformations. 

Let $a$ and $b$ be such starting patterns and $\ell$ be the index at which the patterns differ, that is $a(\ell)+1=b(\ell)$. Note that player $\ell$ is the player with the largest index of her generation in $a$ and the player with the smallest index of her generation in $b$. Obviously, $C_{\ell}(\badNash^a)\le C_{\ell}(\badNash^b)$, since all players up to player $\ell-1$ choose the same strategy under the greedy queue policy and $\ell$ could use the same edge under both starting patterns. Note that $C_\ell(\badNash^a) = C_\ell(\badNash^b)$ if and only if $\ell$ has to wait under starting pattern $a$, i.e., $\ell$ uses an old edge with a non-empty queue.

\paragraph{Claim:} If $C_\ell(\badNash^a) = C_\ell(\badNash^b)$, we have $\badNash^a = \badNash^b$.

\noindent \textit{Proof of claim.} Both $\badNash^a$ and $\badNash^b$ arise from the greedy queue policy $\phi$, i.e., is the same for all players up to $\ell-1$ as the graph and the starting times are the same up to player $\ell-1$. Given this setup, $\phi$ allocates the edge with the smallest workload to $\ell$ where ties are broken in favor of longer queues. As observed above, $\ell$ is allocated to some edge $\badNash^a_{(\ell)}$ with queue under $\badNash^a$. The arrival time of $\ell$ on the same edge under $b$ is obviously the same, thus, this is still the fastest edge with the longest queue under $b$. Hence, $\badNash^a_{(\ell)} = \badNash^b_{(\ell)}$ with the same arrival time of $\ell$, i.e., nothing changes for all subsequent players and $\badNash^a = \badNash^b$.\qed

\medskip
%
\noindent This finishes the proof for the case $C_\ell(\badNash^a) = C_\ell(\badNash^b)$.
On the other hand, if player $\ell$ arrives earlier because she starts one time unit earlier at $s$, then she does not queue, i.e., she uses an unused edge. As this edge is also unused when $\ell$ starts one time step later, we observe $C_\ell(\badNash^a) = C_\ell(\badNash^b) -1$. It remains to show that $C_i(\badNash^a) \leq C_i(\badNash^b)$ for all $i > \ell$. 

We add an additional dummy player to the player set with an index between $\ell$ and $\ell+1$, where the starting time of the dummy is the same as of $\ell+1$. As we consider the greedy queue policy and $\ell$ did not queue, the dummy uses the edge of player $\ell$ under $b$. Since the subsequent players now face the exact same situation in our new game with starting pattern $a$ and the additional dummy as in $b$, $C_i(\badNash^a) = C_i(\badNash^b)$ for all $i\geq\ell +1$.

Now, we will carefully modify the priority of the dummy and swap it with the subsequent players generation by generation. While doing so, we observe that the arrival times $C_i(\badNash^a)$ do not increase. For the generation of player $\ell +1$, we sequentially swap the priorities and strategies of the dummy with its subsequent player. This swap only changes the strategies of the dummy and the subsequent player and may decrease the arrival time of the subsequent player. As this swap does not change queue lengths, all other players still behave according to the greedy queue policy. Thus, the new state is an equilibrium according to greedy queue. We can continue this procedure until the dummy player is the player with the largest index of its generation. Note that the swaps might have increased the dummy player's arrival time. 

If the dummy has the largest index in its generation, we have two cases. 

\paragraph{Case 1:} If the dummy player is in a queue, we increment its generation by one, i.e., it is now the player with the smallest index of the next generation and starts one time unit later. Now, the same argument as in the claim holds for the dummy and she chooses the same edge under the policy greedy queue. We continue and iterate the swapping until the dummy is again the player with the largest index of the generation.
\paragraph{Case 2:} If the dummy player uses an edge without queue or is the player with the largest index, we delete the dummy player. In the first subcase, deleting the dummy does not change the queues for the player after the dummy, as this player could even take the dummy's edge without observing any queue. In the second subcase, the dummy was the player with the largest index, and deleting the dummy trivially does not change $C(\badNash^a)$.\\

\noindent Finally, we have started with $C_i(\badNash^a) = C_i(\badNash^b)$ for all $i>\ell$ and we have never increased $C_i(\badNash^a)$ during the dummy swaps. Thus, we have $C_i(\badNash^a)\leq C_i(\badNash^b)$ for all $i\in N$. 
\end{proof}