
\documentclass[sigconf,screen,authorversion]{acmart}
\usepackage[utf8]{inputenc}
% \usepackage{standalone} 
\usepackage{makecell}
 \usepackage{acmart-taps}
\usepackage{subcaption}
%%
%% \BibTeX command to typeset BibTeX logo in the docs
\AtBeginDocument{%
  \providecommand\BibTeX{{%
    Bib\TeX}}}

% %% Rights management information.  This information is sent to you
% %% when you complete the rights form.  These commands have SAMPLE
% %% values in them; it is your responsibility as an author to replace
% %% the commands and values with those provided to you when you
% %% complete the rights form.
% \setcopyright{acmlicensed}
% \copyrightyear{2018}
% \acmYear{2018}
% \acmDOI{XXXXXXX.XXXXXXX}

% %% These commands are for a PROCEEDINGS abstract or paper.
% \acmConference[Conference acronym 'XX]{Make sure to enter the correct
%   conference title from your rights confirmation emai}{June 03--05,
%   2018}{Woodstock, NY}
% %%
% %%  Uncomment \acmBooktitle if the title of the proceedings is different
% %%  from ``Proceedings of ...''!
% %%
% %%\acmBooktitle{Woodstock '18: ACM Symposium on Neural Gaze Detection,
% %%  June 03--05, 2018, Woodstock, NY}
% \acmISBN{978-1-4503-XXXX-X/18/06}
\copyrightyear{}
\acmYear{2025}
\setcopyright{none}
\setcctype{}
\acmConference[CHI '25]{CHI Conference on Human Factors in Computing Systems}{April 26-May 1, 2025}{Yokohama, Japan}
\acmBooktitle{CHI Conference on Human Factors in Computing Systems (CHI '25), April 26-May 1, 2025, Yokohama, Japan}\acmDOI{10.1145/3706598.3713241}
% \acmISBN{979-8-4007-1394-1/25/04}



%%
%% end of the preamble, start of the body of the document source.
\begin{document}

%%
%% The "title" command has an optional parameter,
%% allowing the author to define a "short title" to be used in page headers.
\title[How Users Provide Feedback to Shape Personalized Recommendation Content]{Beyond Explicit and Implicit: How Users Provide Feedback to Shape Personalized Recommendation Content}%Beyond Explicit and Implicit: How Users Leverage Feedback Mechanisms to Influence Personalized Recommendations

%%
%% The "author" command and its associated commands are used to define
%% the authors and their affiliations.
%% Of note is the shared affiliation of the first two authors, and the
%% "authornote" and "authornotemark" commands
%% used to denote shared contribution to the research.
% wenqi ruiqing manyu pengyi qunfang
% \author{Ben Trovato}
% \authornote{Both authors contributed equally to this research.}
% \email{trovato@corporation.com}
% \orcid{1234-5678-9012}
% \author{G.K.M. Tobin}
% \authornotemark[1]
% \email{webmaster@marysville-ohio.com}
% \affiliation{%
%   \institution{Institute for Clarity in Documentation}
%   \city{Dublin}
%   \state{Ohio}
%   \country{USA}
% }

% \author{Lars Th{\o}rv{\"a}ld}
% \affiliation{%
%   \institution{The Th{\o}rv{\"a}ld Group}
%   \city{Hekla}
%   \country{Iceland}}
% \email{larst@affiliation.org}

% \author{Valerie B\'eranger}
% \affiliation{%
%   \institution{Inria Paris-Rocquencourt}
%   \city{Rocquencourt}
%   \country{France}
% }

% \author{Aparna Patel}
% \affiliation{%
%  \institution{Rajiv Gandhi University}
%  \city{Doimukh}
%  \state{Arunachal Pradesh}
%  \country{India}}

% \author{Huifen Chan}
% \affiliation{%
%   \institution{Tsinghua University}
%   \city{Haidian Qu}
%   \state{Beijing Shi}
%   \country{China}}

% \author{Charles Palmer}
% \affiliation{%
%   \institution{Palmer Research Laboratories}
%   \city{San Antonio}
%   \state{Texas}
%   \country{USA}}
% \email{cpalmer@prl.com}

% \author{John Smith}
% \affiliation{%
%   \institution{The Th{\o}rv{\"a}ld Group}
%   \city{Hekla}
%   \country{Iceland}}
% \email{jsmith@affiliation.org}

% \author{Julius P. Kumquat}
% \affiliation{%
%   \institution{The Kumquat Consortium}
%   \city{New York}
%   \country{USA}}
% \email{jpkumquat@consortium.net}
\author{Wenqi Li}
\affiliation{%
  \institution{Department of Information Management, Peking University}
  \city{Beijing}
  \country{China}
}\email{wenqili@pku.edu.cn}

\author{Jui-Ching Kuo}
\affiliation{%
  \institution{National Tsing Hua University}
  \city{Hsinchu}
  \country{Taiwan}}
\email{rosieeee321@gmail.com}

\author{Manyu Sheng}
\affiliation{%
  \institution{University of Chinese Academy of Sciences}
  \city{Beijing}
  \country{China}
}\email{shengmanyu@mail.las.ac.cn}

\author{Pengyi Zhang}
\affiliation{%
  \institution{Department of Information Management, Peking University}
  \city{Beijing}
  \country{China}
}\email{pengyi@pku.edu.cn}

\author{Qunfang Wu}
\affiliation{%
  \institution{Harvard University}
  \city{Cambridge}
  \state{MA}
  \country{USA}
}
\email{qunfangwu@fas.harvard.edu}
%%
%% By default, the full list of authors will be used in the page
%% headers. Often, this list is too long, and will overlap
%% other information printed in the page headers. This command allows
%% the author to define a more concise list
%% of authors' names for this purpose.
\renewcommand{\shortauthors}{Li et al.}

Large language model (LLM)-based agents have shown promise in tackling complex tasks by interacting dynamically with the environment. 
Existing work primarily focuses on behavior cloning from expert demonstrations and preference learning through exploratory trajectory sampling. However, these methods often struggle in long-horizon tasks, where suboptimal actions accumulate step by step, causing agents to deviate from correct task trajectories.
To address this, we highlight the importance of \textit{timely calibration} and the need to automatically construct calibration trajectories for training agents. We propose \textbf{S}tep-Level \textbf{T}raj\textbf{e}ctory \textbf{Ca}libration (\textbf{\model}), a novel framework for LLM agent learning. 
Specifically, \model identifies suboptimal actions through a step-level reward comparison during exploration. It constructs calibrated trajectories using LLM-driven reflection, enabling agents to learn from improved decision-making processes. These calibrated trajectories, together with successful trajectory data, are utilized for reinforced training.
Extensive experiments demonstrate that \model significantly outperforms existing methods. Further analysis highlights that step-level calibration enables agents to complete tasks with greater robustness. 
Our code and data are available at \url{https://github.com/WangHanLinHenry/STeCa}.

%%
%% The code below is generated by the tool at http://dl.acm.org/ccs.cfm.
%% Please copy and paste the code instead of the example below.
%%
\begin{CCSXML}
<ccs2012>
   <concept>
       <concept_id>10003120.10003121.10011748</concept_id>
       <concept_desc>Human-centered computing~Empirical studies in HCI</concept_desc>
       <concept_significance>500</concept_significance>
       </concept>
   <concept>
       <concept_id>10002951.10003260.10003261.10003271</concept_id>
       <concept_desc>Information systems~Personalization</concept_desc>
       <concept_significance>500</concept_significance>
       </concept>
 </ccs2012>
\end{CCSXML}

\ccsdesc[500]{Human-centered computing~Empirical studies in HCI}
\ccsdesc[500]{Information systems~Personalization}

%%
%% Keywords. The author(s) should pick words that accurately describe
%% the work being presented. Separate the keywords with commas.
\keywords{Personalized recommendation algorithm, Explicit feedback, Implicit feedback, User purpose, Semi-structured interview, Xiaohongshu, RedNote, Douyin, TikTok}

% \received{20 February 2007}
% \received[revised]{12 March 2009}
% \received[accepted]{5 June 2009}

%%
%% This command processes the author and affiliation and title
%% information and builds the first part of the formatted document.
\sloppy
\maketitle

\section{Introduction}

Despite the remarkable capabilities of large language models (LLMs)~\cite{DBLP:conf/emnlp/QinZ0CYY23,DBLP:journals/corr/abs-2307-09288}, they often inevitably exhibit hallucinations due to incorrect or outdated knowledge embedded in their parameters~\cite{DBLP:journals/corr/abs-2309-01219, DBLP:journals/corr/abs-2302-12813, DBLP:journals/csur/JiLFYSXIBMF23}.
Given the significant time and expense required to retrain LLMs, there has been growing interest in \emph{model editing} (a.k.a., \emph{knowledge editing})~\cite{DBLP:conf/iclr/SinitsinPPPB20, DBLP:journals/corr/abs-2012-00363, DBLP:conf/acl/DaiDHSCW22, DBLP:conf/icml/MitchellLBMF22, DBLP:conf/nips/MengBAB22, DBLP:conf/iclr/MengSABB23, DBLP:conf/emnlp/YaoWT0LDC023, DBLP:conf/emnlp/ZhongWMPC23, DBLP:conf/icml/MaL0G24, DBLP:journals/corr/abs-2401-04700}, 
which aims to update the knowledge of LLMs cost-effectively.
Some existing methods of model editing achieve this by modifying model parameters, which can be generally divided into two categories~\cite{DBLP:journals/corr/abs-2308-07269, DBLP:conf/emnlp/YaoWT0LDC023}.
Specifically, one type is based on \emph{Meta-Learning}~\cite{DBLP:conf/emnlp/CaoAT21, DBLP:conf/acl/DaiDHSCW22}, while the other is based on \emph{Locate-then-Edit}~\cite{DBLP:conf/acl/DaiDHSCW22, DBLP:conf/nips/MengBAB22, DBLP:conf/iclr/MengSABB23}. This paper primarily focuses on the latter.

\begin{figure}[t]
  \centering
  \includegraphics[width=0.48\textwidth]{figures/demonstration.pdf}
  \vspace{-4mm}
  \caption{(a) Comparison of regular model editing and EAC. EAC compresses the editing information into the dimensions where the editing anchors are located. Here, we utilize the gradients generated during training and the magnitude of the updated knowledge vector to identify anchors. (b) Comparison of general downstream task performance before editing, after regular editing, and after constrained editing by EAC.}
  \vspace{-3mm}
  \label{demo}
\end{figure}

\emph{Sequential} model editing~\cite{DBLP:conf/emnlp/YaoWT0LDC023} can expedite the continual learning of LLMs where a series of consecutive edits are conducted.
This is very important in real-world scenarios because new knowledge continually appears, requiring the model to retain previous knowledge while conducting new edits. 
Some studies have experimentally revealed that in sequential editing, existing methods lead to a decrease in the general abilities of the model across downstream tasks~\cite{DBLP:journals/corr/abs-2401-04700, DBLP:conf/acl/GuptaRA24, DBLP:conf/acl/Yang0MLYC24, DBLP:conf/acl/HuC00024}. 
Besides, \citet{ma2024perturbation} have performed a theoretical analysis to elucidate the bottleneck of the general abilities during sequential editing.
However, previous work has not introduced an effective method that maintains editing performance while preserving general abilities in sequential editing.
This impacts model scalability and presents major challenges for continuous learning in LLMs.

In this paper, a statistical analysis is first conducted to help understand how the model is affected during sequential editing using two popular editing methods, including ROME~\cite{DBLP:conf/nips/MengBAB22} and MEMIT~\cite{DBLP:conf/iclr/MengSABB23}.
Matrix norms, particularly the L1 norm, have been shown to be effective indicators of matrix properties such as sparsity, stability, and conditioning, as evidenced by several theoretical works~\cite{kahan2013tutorial}. In our analysis of matrix norms, we observe significant deviations in the parameter matrix after sequential editing.
Besides, the semantic differences between the facts before and after editing are also visualized, and we find that the differences become larger as the deviation of the parameter matrix after editing increases.
Therefore, we assume that each edit during sequential editing not only updates the editing fact as expected but also unintentionally introduces non-trivial noise that can cause the edited model to deviate from its original semantics space.
Furthermore, the accumulation of non-trivial noise can amplify the negative impact on the general abilities of LLMs.

Inspired by these findings, a framework termed \textbf{E}diting \textbf{A}nchor \textbf{C}ompression (EAC) is proposed to constrain the deviation of the parameter matrix during sequential editing by reducing the norm of the update matrix at each step. 
As shown in Figure~\ref{demo}, EAC first selects a subset of dimension with a high product of gradient and magnitude values, namely editing anchors, that are considered crucial for encoding the new relation through a weighted gradient saliency map.
Retraining is then performed on the dimensions where these important editing anchors are located, effectively compressing the editing information.
By compressing information only in certain dimensions and leaving other dimensions unmodified, the deviation of the parameter matrix after editing is constrained. 
To further regulate changes in the L1 norm of the edited matrix to constrain the deviation, we incorporate a scored elastic net ~\cite{zou2005regularization} into the retraining process, optimizing the previously selected editing anchors.

To validate the effectiveness of the proposed EAC, experiments of applying EAC to \textbf{two popular editing methods} including ROME and MEMIT are conducted.
In addition, \textbf{three LLMs of varying sizes} including GPT2-XL~\cite{radford2019language}, LLaMA-3 (8B)~\cite{llama3} and LLaMA-2 (13B)~\cite{DBLP:journals/corr/abs-2307-09288} and \textbf{four representative tasks} including 
natural language inference~\cite{DBLP:conf/mlcw/DaganGM05}, 
summarization~\cite{gliwa-etal-2019-samsum},
open-domain question-answering~\cite{DBLP:journals/tacl/KwiatkowskiPRCP19},  
and sentiment analysis~\cite{DBLP:conf/emnlp/SocherPWCMNP13} are selected to extensively demonstrate the impact of model editing on the general abilities of LLMs. 
Experimental results demonstrate that in sequential editing, EAC can effectively preserve over 70\% of the general abilities of the model across downstream tasks and better retain the edited knowledge.

In summary, our contributions to this paper are three-fold:
(1) This paper statistically elucidates how deviations in the parameter matrix after editing are responsible for the decreased general abilities of the model across downstream tasks after sequential editing.
(2) A framework termed EAC is proposed, which ultimately aims to constrain the deviation of the parameter matrix after editing by compressing the editing information into editing anchors. 
(3) It is discovered that on models like GPT2-XL and LLaMA-3 (8B), EAC significantly preserves over 70\% of the general abilities across downstream tasks and retains the edited knowledge better.
\section{Related Work}
\subsection{User Perceptions of Personalized Recommendations}
% What is personalized recommendation platforms;
% Xiaohongshu and TikTok
% How existing literature studied users in personalized recommendation? For example, folk theories studies; user strategization.
% What is the gap/challenges left in existing literature?

Personalized recommendation platforms utilize algorithms to tailor content to individual users based on their preferences and behaviors, such as subscriptions, clicks, likes and dislikes, and dwell time ~\cite{adomavicius2005toward, setyani2019exploring,yi2014beyond}. Previously, recommender systems have been widely used in search engines, news consumption, and e-commerce sites \cite{schafer1999recommender,lu2015recommender,raza2022news}. Powered by recommendation algorithms, recommender systems capture and analyze user interactions, such as clicks, purchases, or explicit ratings, to build user models that represent preferences and behaviors of users. Based on these models, recommendation algorithms like collaborative filtering analyze similarities between users or items to generate personalized recommendations that match each user's profile~\cite{zanker2009case}. More recently, social media platforms are increasingly integrating personalized recommendation algorithms~\cite{guy2010social} to keep users engaged for longer periods of time by delivering content aligned closely with their preferences, as well as facilitating users’ content creation and networking ~\cite{seaver2019captivating,van2018networks, kang2022ai}. As a result, personalized recommendation algorithms are gradually taking up editorial roles in shaping what users view and know ~\cite{gillespie2014relevance}. They have reshaped the content consumption ~\cite{bucher2016algorithmic}, content creation ~\cite{devito2018people, bucher2018cleavage}, and online networking ~\cite{eriksson2021algorithmic} in social media.
Platforms like Douyin and Xiaohongshu are gaining tremendous popularity among domestic Chinese users as well as international audiences (known as TikTok and RedNote). These kinds of platforms allow users to directly interact with the content and rely heavily on algorithms to capture these interactions, rather than solely on account follows, to optimize individually customized image or video feeds ~\cite{Huang_2021, klug2021trick, Chen2019}. 

However, due to the opaque nature of the underlying algorithms ~\cite{gillespie2014relevance, pasquale2011restoring}, users have very limited understanding of how the personalized recommendation platforms operate ~\cite{eslami2015always}. This lack of understanding often leads users to develop ``folk theories'' ~\cite{devito2017algorithms, eslami2016first, ngo2022exploring, bucher2016algorithmic} to make sense of how the systems function to tailor contents delivered to them. Klug et al. found that TikTok users, for instance, often assume that video engagement, posting time, and the accumulation of hashtags are key factors that influence the platform's algorithm recommendation~\cite{klug2021trick}. Such folk theories are not static; instead, they evolve as users encounter new experiences and information, which helps them navigate their interactions with algorithmic systems ~\cite{devito2018people}. These folk theories directly influence how users perceive and interact with the algorithms, based on which users would exert control over the algorithm by taking action to improve content personalization ~\cite{haupt2023recommending} or increase their visibility on social media platforms ~\cite{burrell2019users, bucher2018cleavage}. Content creators also share and discuss their experiences with algorithms, namely ``algorithmic gossip,'' to collectively refine their strategies in promoting content consumption~\cite{bishop2019managing}. 

To sum, personalized recommendation platforms have reshaped how users consume, create, and share content online. But, users often lack a clear understanding of how their interactions are processed by the platform. As users form ``folk theories'' to understand the underlying mechanism, it becomes crucial to explore how users interact with these personalized recommendation platforms and provide feedback to shape their recommendation feeds, which leads to the next section of the literature review.


\subsection{User Strategic Interaction with Recommendation Algorithms}
Users are becoming more aware that their interaction behaviors could influence algorithms and further shape their online experiences, sometimes resulting an uncomfortable feeling that personalized recommendation algorithms are ``spying'' on their thoughts~\cite{ellison2020we,klug2021trick}. Such awareness lead to a variety of user behaviors aimed at teaching, resisting, and repurposing algorithms ~\cite{kim2023investigating} as well as personalizing content moderation ~\cite{jhaver2023personalizing}. 

Some of these behaviors are subtle content modification actions, such as ``voldemorting'' (i.e., not mentioning words or names) and ``screenshotting'' (i.e., making content visible without sending its website traffic) to control their online presence~\cite{van2018networks} or ``algospeak'' (i.e., intentionally altering or substituting words when creating or sharing online content) to bypass algorithmic moderation ~\cite{klug2023how}. Additionally, users spend sustained time and effort to combat unwanted recommendation content using various strategies ~\cite{papadamou2022just, ricks2022does}, such as refraining from hitting likes, tapping ``Not interested,'' searching new keywords~\cite{kim2023investigating}, ignoring content they liked~\cite{Cen_Ilyas_Allen_Li_Madry_2024}, and configuring personalized content moderation tools by blocking specific keywords~\cite{jhaver2022designing}. Some research challenged algorithmic recommendation and content moderation systems that perpetuate inequalities and injustices ~\cite{karizat2021algorithmic,bishop2018anxiety,rosenblat2016algorithmic,kasy2021fairness}. For example, TikTok users modified their engagement---such as following users and sharing their contents---to influence their recommendation feeds, align them with their personal identities, and also impact other users’ feeds to resist the suppression of marginalized social identities ~\cite{karizat2021algorithmic}. They also engaged with specific hashtags and likes to curate a desired ``For You'' feed in response to perceived inequalities ~\cite{simpson2021}. To avoid incorrect moderation in social media, users used coded language or stopped using the platforms when they perceived that the platform disproportionately removed
marginalized users’ identity-related content ~\cite{mayworm2024content}, and sometimes they resorted to switching account after being shadowbanned~\cite{yao2024blind}. User-driven algorithm auditing is also leveraged to uncover harmful algorithmic behaviors ~\cite{devos2022toward, shen2021everyday}. 

Thus, algorithms are now shaped not only through users’ organic interactions with the platform, but also through their strategic attempts to influence the recommendation feeds ~\cite{lee2022algorithmic}. For example, Haupt et al. modeled this strategic process as a two-stage noisy signaling game, where users first strategically consume content presented to them in an initial ``cold start'' phase to influence future recommendation feeds, and then, the system refines its suggestions based on these interactions, leading to an equilibrium where user preferences are clearly distinguished ~\cite{haupt2023recommending}. Taylor and Choi extended the human-algorithm interaction by adding that users notice the personalization and perceive the algorithm as responsive to their identity, which further shapes their interactions and outcomes on the platform ~\cite{taylor2022initial}. Some research also refer to users' purposeful manipulation as ``gaming'' the algorithm ~\cite{petre2019gaming,SHEPHERD2020102572, hardt2016strategic}. Content creators may ``play the visibility game'' by leveraging relational and simulated influence to gain commercial benefits~\cite{cotter2019playing}. While gaming could spur innovations and discover new uses for existing platforms ~\cite{bambauer2018algorithm}, the strategic adaptation of behaviors may as well be misinterpreted by the algorithm and degrade its accuracy ~\cite{Cen_Ilyas_Allen_Li_Madry_2024}.

These studies indicate that users consciously employ a variety of behaviors to influence algorithms and shape their recommendation feeds, guided by their perceptions or ``folk theories'' of how the algorithms work. However, limited research has connected these user perceptions with the platforms' underlying feedback mechanisms. Understanding how users' perceptions align with or diverge from the algorithm's intended responses---and how they interact to shape recommendation feeds---can inform improvements in feedback design for personalized recommendation systems.

\subsection{Implicit and Explicit User Feedback}
Feedback mechanisms play an important role in human-computer interaction. System feedback communicates the response to a user's action \cite{perez1996collaborative} , while user feedback can be used as a useful input for system optimization ~\cite{Spink/Losee:96}.  User interactions are captured and used as user feedback to refine recommendation feeds. In fields of information retrieval ~\cite{Spink/Losee:96} and recommender systems ~\cite{oard2001modeling}, these mechanisms are commonly divided into two categories: explicit and implicit feedback. \textit{Explicit feedback} relies on direct user input to specify their preferences, such as specifying keywords, marking documents, rating, or answering questions about their interests ~\cite{kelly2003implicit}. It usually represents a deliberate, unambiguous, and intentional quality assessment by a user ~\cite{jannach2018recommending}. However, explicit feedback mechanisms often impose a higher user cost, demanding additional effort beyond typical search behavior ~\cite{kelly2003implicit, gadanho2007addressing}. In contrast, \textit{implicit feedback} refers to all kinds of user interactions with the systems from which the system can indirectly infer user preferences ~\cite{jannach2018recommending}. It unobtrusively obtain user preferences based on their natural interactions, such as viewing, selecting, saving, and forwarding ~\cite{kelly2003implicit,jannach2018recommending}. 

Observable user behaviors used as feedback were categorized into several types, including examination, retention, social and public action, physical action, creation, and annotation, with explicit feedback behaviors mainly falling under the annotation category ~\cite{jannach2018recommending}. Among these behaviors, clicks have long been used as a major feedback mechanism by search engines as indicators of intent and relevance. Scholars attempted to analyze users' click behavior under search queries~\cite{zhang2011user,shen2012personalized} to optimize the ranking algorithms of search engines. The number of clicks on ads has also become the primary indicator of revenue for commercial search engines ~\cite{kim2011advertiser}. Because of the economic motivation, numerous studies have explored how to predict ad clicks more accurately in search engines ~\cite{jacques2015differentiation,liu2010personalized}. In social media platforms, more diverse traits are used to predict user preferences, such as the user's friend networks~\cite{ellison2020we}, posts~\cite{kim2016click}, likes~\cite{cheung2022influences, hu2022interest}, and bookmarks~\cite{klaisubun2007behavior}. Oard and Kim outlined the \textit{minimum scope} of different behaviors, which is the minimum level at which feedback is applied---whether it impacts a portion of an object (segment level, e.g., viewing comments of a post), an entire object (object level, e.g., collecting a post), or multiple objects (class level, e.g., platform-wide searches) ~\cite{oard2001modeling}. Jawaheer et al. compared different properties between implicit and explicit feedback. For instance, explicit feedback provides transparency but also required cognitive effort, thus only a small percentage of users actively provide it, often leading to bias toward those who are more expressive ~\cite{jawaheer2014modeling}. In terms of \textit{polarity}---the positive (preference) or negative (dislike) orientation of the feedback---explicit feedback contains both whereas implicit feedback is typically considered positive ~\cite{jawaheer2014modeling, hu2008collaborative}. These properties, along with factors like user goals ~\cite{nazari2022choice,liang2023enabling}, digital literacy ~\cite{devito2018people}, and task complexity ~\cite{white2005study}, all have potential influence on users' understanding and adoption of different feedback mechanisms. For example, skilled users rely on platform cues to grasp algorithms, while those with lower web skills depend on external guidance. Users with peacekeeping self-presentation goals tend to limit explicit hashtag use compared to those focused on authenticity ~\cite{devito2017algorithms}. Users are reluctant to provide explicit feedback in complex tasks but are more willing to do so in media and entertainment domains ~\cite{white2005study,jawaheer2014modeling}. Moreover, the interpretation and effectiveness of these feedback is also dependent on contextual factors including user characteristics ~\cite{jawaheer2014modeling} and system functionalities ~\cite{liu2024train}.

While the explicit-implicit dichotomy offers clarity in examining user feedback from the system’s perspective, it may not fully capture the complexity of users' intentions when interacting with personalized recommendation platforms. Users now consciously manipulate behaviors once considered natural and unobtrusive to influence recommendation feeds, blurring the line between explicit and implicit feedback. Particularly, the term ``implicit'' can refer to unintention, background attention, and lack of awareness, causing confusion about when an interaction is truly implicit versus explicit ~\cite{serim2019explicating}. 
This evolving landscape necessitates a combined perspective from both users and systems, as well as a re-examination of how and why users employ different feedback mechanisms in personalized recommendation systems.
\section{Method} \label{section: method}

\begin{figure*}[t]
    \centering
    \includegraphics[width=\textwidth]{figures/overview.pdf}
    \caption{Overview of \ours and cross-token prefetching framework. (a) \textbf{\ours}  formulates the attention history as a spatiotemporal sequence, and predicts the attention at the next step with a pre-trained model. To enhance efficiency, the attention history is updated in a compressed form at each decoding step. (b) \textbf{The cross-token prefetching framework} asynchronously evaluates critical tokens and fetches KV for the next token during the LLM inference, thereby accelerating the decoding stage.}
    \label{fig:prefetch_overview}  
\end{figure*}


In this section, we introduce \ours, the first learning-based method for identifying critical tokens, along with the cross-token prefetch framework for improved cache management. We begin with the problem formulation for attention prediction in Section~\ref{section:formulation}, followed by a description of our novel \ours in Section~\ref{sec:attention_predictor}. Finally, Section~\ref{section: prefetch} presents a cross-token prefetch framework that efficiently hides both evaluation and cache loading latencies.

\subsection{Problem Formulation}
\label{section:formulation}

In the language model decoding stage, we denote $\mathbf{Q}_t \in \mathbb{R}^{1 \times d}$, $\mathbf{K} \in \mathbb{R}^{t \times d}$ as the query tensor and key tensor used for generate token $t$, respectively. Specifically, we denote \( \mathbf{K}_i \in \mathbb{R}^{1 \times d} \), where \( i \in \{1, 2, \dots, t\} \), as the key tensor for token \( i \), and \( \mathbf{K} = \mathbf{K}_{1:t} \) as the complete key tensor. The attention at step $t$ is calculated as:
\begin{equation}
    A_t=\text{Softmax}\left(\frac{1}{\sqrt{d}} \mathbf{Q}_t \mathbf{K}^\top \right), A_t \in \mathbb{R}^{1 \times t}. 
\end{equation}


The sparsity-based KV cache compression seeks to find a subset of keys with budget $B$ that preserves the most important attention values.
Specifically, the set of selectable key positions is $\Gamma=\{\{\mathbf{p}\}=\left\{p_i\right\}_{i=1}^B|p_i\in\{ 1,2,\ldots,t\} ,p_i\neq p_j,\forall i,j=1,2,\ldots,B\}$. 
We define the \textbf{attention recovery rate} as:
\begin{equation}
\label{eq:attention_recovery_score}
R_{rec} = \frac{\sum_{i=0}^{B}{A_{t, p_i}}}{||A_t||_1},
\end{equation}
which reflects the amount of information preserved after compression. A higher recovery rate $R_{rec}$ indicates less information loss caused by KV cache compression.
Therefore, the goal of KV cache compression can be formulated as finding the positions $\mathbf{p}$ that maximize $R_{rec}$, i.e.,
\begin{equation}
\label{eq:find_p}
\underset{\mathbf{p} \in \Gamma }{\max} \,R_{rec}. 
\end{equation}

To determine the positions $\mathbf{p}$, existing methods typically employ heuristic approaches to score the attention at step $t$, represented as $S_t \in \mathbb{R}^{1 \times t}$, and then select the top $B$ positions. 
For example, the well-known method H2O~\citep{zhang2023h2o} accumulates historical attention scores, where $S_t = \sum_{n=1}^{t-1}{A_n}$. 
In this paper, we predict the attention of step $t$ as $\hat{A_t}$ and use it as $S_t$.

After identifying the critical token positions $\mathbf{p}$, the attention is computed sparsely $A^\text{sparse} = \text{Softmax}\left(\frac{1}{\sqrt{d}} \mathbf{Q} {\mathbf{K}^{\text{sparse}}}^\top \right)$, with selected keys $\mathbf{K}^{\text{sparse}} = \text{concate}\{\mathbf{K}_{p_i}\}$.


\subsection{\ours: A Spatiotemporal Predictor}
\label{sec:attention_predictor}

\textbf{Prediction formulation.} We formulate the attention history $A_H \in \mathbb{R}^{t\times t}$ as a spatiotemporal sequence.
The first dimension of $A_H$ corresponds to the time series over the decoding steps,
while the second dimension represents a sparse series over different keys.
We then train a model to predict the attention for step $t$ as $\hat{A}_{t+1} = F(A_H)$, where $F(\cdot)$ denotes the model function.
For efficiency, we limit the time steps of $A_H$ using a hyperparameter $H$, so that the input to the predictor is $A_H \in \mathbb{R}^{H \times t}$.
\begin{figure*}[t]
    \centering
    \includegraphics[width=0.8\textwidth]{figures/prefetch_timeline.pdf}
    \caption{Timeline of our proposed cross-token prefetching. By asynchronously loading the critical KV cache for the next token, our framework hides the token evaluation and transfer latency, accelerating the decoding stage of LLM inference.}
    \label{fig:prefetch_timeline}
\end{figure*}


\textbf{Model design.} To capture spatiotemporal features, we use a convolutional neural network (CNN) composed of two 2D convolution layers followed by a 1D convolution layer. 
The 2D convolutions capture spatiotemporal features at multiple scales, while the 1D convolution focuses on the time dimension, extracting temporal patterns across time steps. By replacing the fully connected layer with a 1D convolution kernel, the model adapts to the increasing spatial dimension, without data segmentation or training multiple models.
Compared to an auto-regressive LSTM~\citep{graves2012lstm}, the CNN is more lightweight and offers faster predictions, maintaining a prediction time shorter than the single-token inference latency. Additionally, when compared to an MLP~\citep{rumelhart1986MLP} on time-series dimension, the CNN is more effective at capturing spatial features, which improves prediction accuracy. 


\textbf{Training strategy.} Our model is both data-efficient and generalizable.
We train the model only on a small subset of attention data, specifically approximately 3\% extracted from the dataset. The model performance on the entire dataset shows our model effectively captures the patterns (see Section \ref{sec:exp_main}). 
Additionally, due to the temporal characteristics of attention inherent in the LLM, a single model can generalize well across various datasets. For example, our model trained on LongBench also performs well on the GSM8K dataset, highlighting the generalization capability of \ours.



\textbf{Block-wise attention compression.}
To speed up prediction, we apply attention compression before computation. 
By taking advantage of the
attention's locality,
\ours predict attention and identify critical tokens in blocks. Inspired by~\citet{tang2024quest}, we use the maximum attention value in each block as its representative.
Specifically, max-pooling is applied on $A$ with a kernel size equal to the block size $b$, as $A_t^{comp} = Maxpooling(A_t,b)$, reducing prediction computation to roughly $\frac{1}{b}$. 

\textbf{Distribution error calibration.}
Due to the sparsity of attention computation, the distribution of attention history $A_H$ used for prediction may deviate from the distribution of dense attention. This deviation tends to accumulate over decoding, particularly as the output length increases. To mitigate this issue and enhance prediction accuracy, we introduce a distribution error calibration technique to correct these deviations. Specifically, we calculate and store the full attention score every $M$ steps, effectively balancing accuracy with computational efficiency.

\textbf{Overall process.}
As shown in \autoref{fig:prefetch_overview} and Algorithm \ref{alg:predict}, \ours prepares an attention history queue in the prefilling stage, and predicts attention during the decoding stage. First, the $A_t$ from the LLM is compressed to $A_t^{comp}$ using block-wise attention compression. Next, $A_H$ is updated with $A_t^{comp}$. The next step attention $\hat{A}_{t+1}$ is then predicted with the pretrained model. From $\hat{A}_{t+1}$, the top-K positions are selected with a budget of $B/b$, since $\hat{A}_{t+1}$ is in compressed form. Finally, the indices are expanded with $b$ to obtain the final critical token positions
$\mathbf{p}$.

\begin{algorithm}[ht!]
   \caption{Identify Critical Tokens}
   \label{alg:predict}
   
    \textbf{Input}: Attention scores $A_t$, Attention history $A_H$, Block size $b$, KV budget $B$
    \\
    \textbf{Output}: Critical KV token positions $\mathbf{p}$
    
    \begin{algorithmic}[1]
    \STATE Pad $A_t$ to the nearest multiple of $b$ with zero
    \STATE $A_t^{comp} \gets \text{MaxPooling}(A_t, b)$
    \STATE $A_H \gets \text{Update}(A_h, A_t^{comp})$
    \STATE $\hat{A}_{t+1} \gets \text{Prediction model}(A_H)$
    \STATE $\text{Positions} \gets \text{Top-K}(\hat{A}_{t+1}, B / b)$
    \STATE $\mathbf{p} \gets \text{Expand}(\text{Positions}, b)$ \\
    \textbf{Return} positions $\mathbf{p} $
    \end{algorithmic}
\end{algorithm}


\subsection{KV Cache Cross-token Prefetching} \label{section: prefetch}

To address the increased memory cost of longer contexts, current LLM systems offload the KV cache to the CPU, but I/O transfer latency becomes the new significant bottleneck in inference. KV cache prefetching offers a solution by asynchronously loading important cache portions in advance, hiding retrieval time. We introduce the cross-token KV cache prefetching framework, which differs from the cross-layer method in Infinigen \citep{lee2024infinigen} by leveraging longer transfer times and enhancing data integration. 
Specifically, our implementation involves a prefetching process for each layer. As illustrated in Figure \ref{fig:prefetch_overview}, during the prefill phase, the computed KV cache is completely offloaded to the CPU without compression. Then, \ours forecasts the critical token indices $\mathbf{p}$ for the next step. The framework then prefetches the KV cache with $\mathbf{p}$ for the next step onto the GPU. Concurrently, the GPU processes inference for other layers, so the maximum time available for prediction and cache loading corresponds to the inference time per token. Subsequently, the GPU utilizes the query for the next step along with the prefetched partial KV cache to calculate the sparse attention. The attention history is then updated with the newly computed attention scores. The timeline of cross-token prefetching can be seen in \autoref{fig:prefetch_timeline}.


\section{Findings}

The inductive analysis across different robotic artists revealed recurrent factors that contribute to artistic creativity in robotic artwork. Here we present four such facets---\textit{Embodiment and Materiality}, \textit{Malfunction}, \textit{Audience's Reaction and Reception}, and \textit{Process of Creation and Exhibition}. Robotic art is unique in each of them. We argue that these factors are salient in the real-world practices of robotic art---uses of robots in artistic or creation activities. By investigating the practice of robotic art, our study contributes empirically to understanding how computing machines are creatively used for artistic and non-pragmatic purposes. Building upon prior works on artistic input to HCI ~\cite{kang2022electronicists}, we advance the discourse by exploring how artistic practices, values, attitudes, and ways of thinking can serve as resources for HCI practitioners studying or designing for creative activities with machines.

\subsection{Embodiment and Materiality}
\label{f:emb}
Embodiment and materiality are key factors in artistic creativity, shaping the design of robotic artworks. As embodied forms, robots interact with physical space, materials, and humans, matching with human cognition through bodily perception~\cite{davis2012embodied}. Their embodiment encompasses physical appearance, movement, and human interaction, aspects crucial for HCI researchers designing robots to engage with their environment~\cite{marshall2013introduction}. For most of our artists (N=7), understanding robots' material and embodied nature deeply influences their creative process, shaping their thinking and inspiring new ideas. While embodiment imposes physical limitations, it also enhances artistic expression, fostering new styles and aesthetics.

\paragraph{Expressivity From Embodiment}
The embodied property of robots produces an important expressivity and artistic style in robotic art that is challenging to replicate without physical embodiment. For example, David compared drawing by physical robots with drawing in computer programs, concluding that the former is more expressive in an artistic sense because the action of drawing by robots is embodied in the physical world rather than being ``simulated'' in computer programs: ``\textit{I use embodiment (embodied action of drawing by robots)... the drawings work because they do real gestures, it (the drawing) is not simulated. So the drawing has this dynamic feel to it because it is really the movements and the gestures and things... there is a certain speed that it (the embodiment) gives this expressivity to the drawing}.'' The embodied drawing by robots adheres to the physical properties of the material and environmental factors (e.g., pencil, paper, table, robotic arm's degree of freedom, humidity, lighting of the scene), making the drawing process complex, and at times, random and uncontrollable. This complexity introduces more possibilities for artistic expression.

The degree of artistic expressivity depends on which specific materials enable the embodiment of drawing by robots. Interestingly, David claimed that industrial robots, though can draw with high precision, produce less expressive drawings than his self-built robots whose robotic arm's movement is not that precise but more dynamic, flexible, and turbulent:

\begin{quote}
    I don't use industrial robots, because industrial robots are pen plotters. They do exactly what you ask. But they (non-industrial robots) are flexible and... not that precise... when it's drawn, it (the drawing by non-industrial robots) has more expressivity because of the embodiment. The embodiment is very important. It's only because I use those types of arms (self-built robotic arms). It would be far less important if I was using industrial robots.
\end{quote}

He also mentioned explicitly that precise drawing is not artistic: ``\textit{But anyway, that (precise drawing robot) is the technology. And it works very nicely, but I couldn't find it artistic. I was actually disappointed when I got it to work.}'' Similarly, Sophie noted that plotting/printing robots create different drawings than painting robots do: ``\textit{I wanted it (the artwork) to be painted and I didn't want it for the visuality of it or the behavior of it. I didn't want it to be plotted or printed, [it] feels different [and] has a different existence.}''

Although both industrial robots and self-built robots draw in embodied ways, the results can appear either precise or dynamic, depending on how the robots are built and programmed---in other words, how the artists configure the material aspects of robots to realize the embodiment. In practice, our robotic artists need to think about ways of utilizing embodiment and properties of robots and all other involved materials to be artistically expressive, to be creative.

\paragraph{Inspiration From Embodiment}
We found that the embodied nature of robotic art often becomes a source of inspiration for new artistic ideas. Linda, an artist-engineer who conducts scholarly research at the intersection of robotics and dance, reflected on how interacting with embodied robots makes her think about the differences between human and robotic bodies:

\begin{quote}
    I've never felt more human. You just feel, you notice, oh, I can fall here and I can get right back up, but it (the robot) falls and it can't get right back up. Or how soft am I? How wet? Like, (patted her face) there's so much water content and squish when I lay on the floor. And it [the robot] doesn't have that... That generates new ideas and helps me be creative.
\end{quote}

She also explained how robotic bodies allow her to examine human movements: ``\textit{The robot is doing something that I can't do on my own body---pure right (her arm was moving toward her right), and [then]... [I] can look at my messy right [movement] next to its [robot's] pure right... that's creative, that's energizing to me to see and play with movement profiles with such a pure tool for decomposing the elements of it, making me notice them}.'' She also shared an anecdote that building a special robot with high degrees of freedom inspired her to explore the differences between human and robotic bodies, enabling her to see new things for her art projects.

\paragraph{Creativity From Embodiment}
Our artists emphasized the embodied nature of creativity and intelligence in general based on their artistic practices, asserting that creativity is inherently embodied rather than disembodied, symbolic, spiritual, and something only happens in the human head. For our robotic artists, creativity is built upon understanding embodied entities in the environment rather than abstract concepts in the mind. Samuel used ChatGPT as an example to argue how the disembodied way of communication between humans and machines limits creative interactions:

\begin{quote}
  I think most of the creativity is coming from non-verbal information flow. So when we are discussing with ChatGPT only through text... the creativity that we can experience is so limited because we do have to sit in front of ChatGPT and we cannot move around or ChatGPT is not going to move around. So our conversation is... very limited... that missing embodiment... is also missing creativity in the conversation with ChatGPT.
\end{quote}

The design of ChatGPT aligns with the mainstream approach to disembodied chatbots running as computer programs. In these designs, symbolic content (e.g., text, images, videos, audio) serves as the communicative medium, but bodily interaction is minimized (i.e., users primarily sit and type). While creativity is arguably rooted in embodied interaction with other material bodies, current interactive agents (e.g., Copilot and Midjourney) designed to support creative work remain largely symbolic and disembodied. Limiting human-machine communication to symbolic channels may lose the benefits of embodiment in acquiring creativity.

According to our artists, one reason for the lack of attention to the embodied dimension of creativity is the historical dichotomy between mind and body, which categorizes creativity as something in the mind:

\begin{quote}
     It (the idea that creativity is disembodied) was very much driven by a view that you can split the body and the mind, and intelligence is happening in a symbolic way, mainly in the brain... [This] led to a large focus on software applications and delayed focus on robotic hardware improvements. And still today, you can see the split of hardware and software... [F]or a lot of organic entities, the integration of bodily capacities with their environment could be seen as more intelligent than the representational capacities... [A]s an artist, I am trained to work with bodies interacting with environments or with other bodies, also this fluent transition from bodily action to semantic questioning. (Daniel)
\end{quote}

This dichotomy, which may have formulated the engineering of computing systems, is rarely compatible with the artist's view that intelligence and creativity can be more richly manifested through bodily interaction and relationships.

Embodiment has been an essential prerequisite of creativity for some artists since their creative production requires understanding embodied entities. To summarize this subsection, embodiment is an important source of creativity for robotic artists. Practically, it yields new artistic expressions and aesthetics whose complexity is difficult to replicate by computer programs. The embodied form of robots, in turn,  inspires creative ideas for artworks. These ideas can arise from understanding the entities embodied in the physical world, whether robots, humans, or other bodies in the environment. The symbolic and disembodied modes of interaction between human creators and machines in creative activities can be complemented and strengthened by embodied interaction.

\subsection{Malfunction: ``Ghost in the Machine''}
Robots, encompassing both mechanical and digital devices, are inherently susceptible to malfunction, with physical robots being more prone to errors, glitches, and noise than virtual agents. These malfunctions are widespread in robotics. In robotic art, such errors hold unique significance, influencing the interpretation and value of the art. Unlike engineers, who aim to fix errors, robotic artists often embrace malfunctions as part of their creative process (N=7).

\paragraph{Embracing Errors and Uncertainty}
Evelyn views machine errors not as obstacles, but as opportunities for unique artistic expression. She embraces the imperfections that arise from machine errors, seeing them as a way to humanize the machine and its output:

\begin{quote}
     I embrace these errors. For me, it is the way to show that using the machine in a way that's very counter-intuitive... celebrating that error instead of trying to perfect it, or slowing down the machine instead of trying to create commodities as fast as we can... what's interesting with the machine [is] to actually turn it upside down and think that the machine is a bit like a human child, and everything it does actually slow, it's imperfect, it's full of mistakes.
\end{quote}

Evelyn’s approach challenges the conventional expectation of machines as flawless and efficient executors. By slowing down the machine and celebrating its errors, she imbues the machine with a human-like quality of imperfection. This perspective turns the machine into something capable of growth and learning, much like a human child. The errors, therefore, potentially become a source of uniqueness and individuality in the artwork, adding depth and complexity to the artistic expression. This ``counterintuitive'' way of viewing error resonates with Alex who contrasts this view with the engineering tendency that strives to be neat, rational, and organized through monitoring and fixing errors: ``\textit{[S]ometimes, it's (error/glitch is) like a source of treasure. Like you find something that you could work on, you find something that people don't really use... But when we are tinkering, we sometimes reach this point of, `ah, okay, now this is visible.'... I think sometimes even just those things (errors/glitches) could be a work of [art]}.''

Alex and others see robotic malfunctions as opportunities to imagine alternative approaches and values. When robotic systems' behaviors deviate from their programs, they often refuse to ``fix'' the unexpected behaviors, instead, they allow the unexpected to unfold as serendipitous events that can inspire new design features. Preserving malfunctions allows the artists to think about the artistic potential of something derailing from the initial plan and make informed adjustments accordingly. These values would not be examined, integrated, or utilized to contribute to creativity if the immediate response to malfunctions was negation and subjecting the malfunctions as inferior to the planned behaviors. As Sophie noted, artistic practices are inherently unpredictable and shaped by the contingencies of the creation process.

Many of our artists described how they perceive, evaluate, and appreciate the unexpectedness of robotic art, revealing new artistic ideas that would not have emerged otherwise: \textit{``[I]nstead of an ink particle, you had a hole in the form of that part... I was like, `Oh, we'll see that the material is saturated, I will not push it (brush).' But the robot doesn't have this understanding and pushes it. And I thought, `Oh, it's actually a good outcome. It's actually both conceptually and aesthetically very pleasing to me'}'' (Sophie).

In this case, the robot performed an action that a human artist normally would not perform---pushing the brush on the canvas. The robot made an unusual decision and breakthrough in expression, called by many artists as ``surprise.'' Once the artist recognizes its artistic value, it may be further explored and developed. Linda shared a similar anecdote where an unexpected jitter from the way the motor pulls a string gives a ``texture'' to the robotic movement, which she sees as creative.

\paragraph{Incorporating Malfunctions as Intended Design}
Our artists deliberately incorporate errors into their artworks. It demonstrates how valuing malfunctions and the unexpected can directly contribute to the work's artistic creativity. Linda articulated the idea that humans are capable not only of learning from mistakes but also of intentionally leveraging these errors to their advantage, echoing insights from our other artists: \textit{``Glitches are 100\% part of the creative and artistic process... It's undeniable that we recover better from mistakes [than computers do], but I think it's more than that. We actually can incorporate mistakes and make them part of an intended design.''}

By making malfunctions part of the intended design, the artists engage with and utilize them to enhance artistic expression or similar ends. Choosing not to fix these issues offers the artists alternatives to designing and realizing their robotic art. For example, David recounted an anecdote about a bug---a flaw in a computer program's software or hardware---that unexpectedly made a line drawing ``beautiful.'' Rather than fixing the bug, he decided to make it an optional feature, allowing him to switch it on or off:

\begin{quote}
    Generally I don't take care of them (glitches). So there are those glitches that give this unpredictable because I like to have drawings that are not predictable... I fix it (bug) and then I have the possibility of using or not using the bug... I'm always surprised by the output... it (bug) creates a surprise for the spectator who is looking at the robot drawing... I just left it (bug) and it's still there. Sometimes I switch it (bug) on, sometimes I switch it off.
\end{quote}

If David had fixed the bug, without retaining it in the program, he would not have possessed such a feature of expression. This shows how differently robotic artists handle technical malfunctions than typical engineers or roboticists. Malfunctions should be avoided in engineering but may yield creative outcomes for robotic art. This is not to claim that the creative value is innate within malfunctions. As our findings have shown, malfunctions are raw materials that can be deliberately utilized by the artists to achieve creativity. When malfunctions are not desirable in art, they may primarily be engineering challenges, as the following examples illustrate.

\paragraph{Avoiding Malfunctions}
The fragility of robots is a widely shared concern among our robotic artists (N=8). Regardless of their origin---self-built, modified, or off-the-shelf---all robots are susceptible to breakdowns in real-world environments, particularly during extended exhibitions without proper maintenance by artists or qualified personnel. For exhibitions, malfunctions are generally unacceptable, and robots ought to \textit{perform flawlessly} when showcasing to the audience. To address malfunctions, the artists came up with different strategies, such as having backup materials for replacement and assembling the robots on-site at the exhibition. One strategy is reducing the complexity of the robotic system, simplifying it to minimize the risk of failure or loss of control. Their approach involves designing robots that resist internal breakdowns and withstand external environmental factors, such as moisture and gravity. Linda explained, ``\textit{If I do build them (robots), I try to keep them simple and I try to make something that will withstand its environment... Sometimes that might be outdoors next to the ocean for six days}.'' David further emphasized that the concern for fragility leads to the need for simplicity in robotic design:

\begin{quote}
    [I]f you're used to do programs that are disembodied, that are only on the computer, you can do very complex things. But as soon as you work with robots, you have to simplify everything... They exist in the same physical world [as us]. Dynamic, the speed, the time, the weight of thing are the same for us. So there are all those limits, which [requires] you to simplify a lot of the programming.
\end{quote}

Mitigating malfunctions and recognizing their artistic potential are not mutually exclusive. Designs that address engineering malfunctions can also yield artistic qualities. As illustrated in the findings, utilizing and mitigating malfunctions occur at different phases of artistic practice. In the production phase within the studios, artists often regard malfunctions not as impediments but as sources of inspiration. By celebrating serendipitous errors and the unexpected, they deliberately integrate these elements into their robotic creations, pushing material and expressive boundaries. In this phase, the primary interaction happens at the individual level---between the artist and the robot(s). In contrast, within exhibition spaces like museums, malfunctions conflict with the expectation that the robots should function flawlessly, risking being disqualified from display. Here, the interaction shifts to a social context, where artists must negotiate with curators and audiences on how to present the robot. This transition from studio to exhibition thus signifies an important change in, context, practice, and actors involved. Hence, next, we highlight the significance of audience reaction and reception that shape creative outcomes.

\subsection{Audience's Reaction and Reception}

\begin{quote}
    I suppose [that] every project I do is a collaboration between me, the machine, and the interactant to some extent. --- Robert
\end{quote}

The artistic and creative value in robotic artwork is determined not just by the work itself but often by the audience’s reactions and interpretations. Our artists (N=7) mentioned that they observe or think about audience reaction, and often incorporate them into subsequent iterations of their work. Alex, for instance, is motivated in the first place by observing how people react to robots, drawing inspiration from their perceptions.

\paragraph{Audience Reaction Shapes Robotic Design}
One of the most direct ways audiences influence the practice of robotic art is through the artists, even when it is unintentional. For instance, after observing that some audience members interact with his robots by squeezing two springs on the robot together---causing a short circuit---Robert decided to revise the material design of the robots to prevent such accidents: ``\textit{I knew darn well that the children were going to squeeze the springs together. So I was very excited to find that even if they did that, I put a kind of a self-healing fuse, polycrystalline that will heal itself... it was an important component of the design}.''

Robert’s response highlights the importance of audience reaction, which he observes and integrates into his robot designs. While in this case, the reaction led to the resolution of a technical issue rather than adding an artistic element,  Alex's experience illustrates how audience interaction can inspire new aesthetics in his work. Alex described how he adapted the environment around his robots based on the audience’s tendency to project personalities onto them:

\begin{quote}
    People project something like animals or themselves or something [on the robots]. And then I got inspiration from that. Then I made a little brighter setup with some objects, a little bit like forest kind of setup. And then people try to imagine more stories. And then I also put some effect to [make the setting] looks like night or daylight or morning. Then people really see [the robots] differently.
\end{quote}

These examples demonstrate how the audience's explicit and implicit feedback (action, projection, and imagination) influences artists’ decisions in designing robots. Audiences are not passive recipients of the artists’ creations; rather, they become part of a collective creative process, leaving their mark on the final work.

\paragraph{Audience Reaction Shapes Robotic Performance}
Linda described how she designed a robotic component for ``\textit{onstage performer[s] as well as audience members to come and interact with [the] robot in a creative way},'' emphasizing the importance of creating a space for audience interaction. Robert further suggested that these interactions during the exhibition possess performative features, which he views as an artwork: ``\textit{I would consider the final product (the drawing by his autonomous robots) as the art. And I would also consider the [audience's] experience of watching them (the robots) paint also as a kind of performative artwork}.'' Robert views robots not as static objects but as responsive entities capable of meaningful interactions with both their environment and the audience. He views robots as possessing ``emergent agency'':

\begin{quote}
    I think that's an agency I would call emergent agency, which is to say that the system software, the physical structure itself in relation to the viewer, interactant creates a kind of emergent behavior where the robot is, and it's designed to some extent to react or respond either with sound or motion in some way to the viewer. And by doing so, it then allows the viewer to see that response, which then reprograms the viewer's response to that. So there's almost a kind of feedback loop that I find happens a lot with robotic art.
\end{quote}

Daniel mentioned a similar idea in the context of live dance performance. The performance benefits from incorporating ``real-time learning interactive systems'' because that makes the performance not solely predefined but ``\textit{[emerged] in the moment of interaction which was not there before [the performance].}'' Without the audience serving as the stimulus, interactive robots in exhibitions would not be perceived as they were. In other words, robots react to the audience, which casts changes in the audience's perception, then robots sense the changes and react again, forming a continuous feedback loop or improvisation between the robots and the audience.

\paragraph{Open Interpretations Make Robotic Art}
Artwork that remains open, undetermined, complex, and vague often invites diverse interpretations~\cite{eco1989open}. The same applies to interaction design where systems may not have a single user interpretation~\cite{sengers2006staying}. Samuel built three humanoid robots with different levels of functionality. The third robot, though technologically more advanced, received less curiosity from the audience than the first, more rudimentary robot:

\begin{quote}
    [For the third-gen robot],... people immediately understand what he (the robot) is doing. So people just leave after five minutes. But [for] the first one (first-gen robot), people tend to spend like 20 [or] 30 minutes because people don't understand what he's doing. But now it [the third-gen robot] is interpretable, so I understand that... giving him too much meaning is dangerous, [when] work[ing] on an art stuff, because people get tired... people are used to those things (technological functions), which [have] tons of meaning [about] what the machine is doing.
\end{quote}

Here, incorporating technical functionalities into the robot assigns clear objectives easily grasped by the audience, making the perceived meanings more rigid and restricting the scope for diversified interpretation.

Beyond the individual level, the way of interpretation is also socially shaped. Samuel made the point that the perception of creativity is also partly a social product because ``\textit{creativity is depending on what kind of society we are in and what kind of people we are interacting with}.'' Mark and Robert extended that the perception of robotic art is culturally conditioned, varying across different societies and generations. They mentioned how the animist cultural tendency of some East Asian societies potentially makes people more willing to accept and interested in robots and non-human entities (e.g., plants and animals) behaving as if intelligent and agentic. The way that the social context of interpretation and perception determines artistic values reiterates our claim that the audience's reception of robotic artwork is one of the key aspects of robotic art practice. It suggests that in achieving certain artistic goals by robotic art, considering the audience's background and ``horizon of expectations''~\cite{jauss1982toward}---the socially and historically conditioned structure by which a person comprehends, interprets, and appraises any text based on cultural codes and lived experiences---may be constructive in refining the work's idea.

\subsection{Process of Creation and Exhibition}

Many of the artists we interviewed (N=6) emphasized, or alluded to, the artistic value in the \textit{process} of making robotic art. Specifically, two types of processes are discussed here---the process of \textit{creation} and the process of \textit{exhibition}, reflecting two salient temporal stages of robotic art practice. We do not, by any means, suggest that process is unique to robotic art; apparently, other forms of art also attend to processes of their art practice. Our intention has been to situate the analysis of process in the emerging, particular context of robotic art and to reveal how process leads to a new understanding of robot's uses and roles in real-world scenarios.

Sophie builds robotic systems capable of physically painting on canvas. She uses these robots to explore the painting process itself rather than to focus on the final product—what she referred to as images instead of paintings. Her case exemplifies that the act of making can become the focal point of artistic interest. In her view, paintings as artifacts are space-and-time bound ``material-based work'' that requires ``interactive practice'' and ``decision making,'' whereas the resulting images are ``merely digital representation[s]'' of this process. The difference between images generated by computer programs and paintings created through human touch underscores her rationale for utilizing robots: to bring the tactile, material process of painting to the forefront.

\begin{quote}
   [I]n the end, if I'm trying to crop everything (all my ideas) together, then it (the commonality) is to make the temporality of the decision making process of painting more visible and present. So I'm not really interested in how the image looks. And we experience an object that actually has a temporal element, how it's been created with layers, with tons of decision making... because I am interested in painting as a process and less [as] a product, I'm trying to use the process of making a painting to reflect a lot of our human creativity, our relationship to machines, questions of agencies, and so on.
\end{quote}

She has been building robotic systems that have ``adaptive behavior[s]'' during the painting process, where the systems are designed to ``\textit{analyze a stroke [on the canvas] and then create a successive one}.'' This design ensures that robots' actions are not exclusively dictated by the pre-programmed instructions but also influenced by the constantly changing ``state of the world,'' which includes factors such as the evolving canvas, environmental conditions, and the interaction between the robot and its surroundings. Consequently, a painting is not just a visual product but represents a series of actions with a temporal dimension.

Another important process for robotic art practice is exhibition. In the exhibition space, robotic artworks often take the forms of performances or improvisations, actively interacting and potentially shaping their environment in real time. 
For example, Alex's robots paint spontaneous color patterns on canvas during the exhibition, transforming the event into a performative art experience that aligns with his intention of foregrounding the painting process. The dynamic nature of live drawing at the exhibition---``making a show live''---has been central to Alex's artistic approach.
Moreover, new qualities of robotic artwork not only emerge by interacting with other entities, such as viewers or environmental factors, but also through the artwork itself as it develops over time. Daniel recounted an instance where a crack in his robotic installation continued to expand, gradually altering the artwork throughout the exhibition:

\begin{quote}
    I used [a] dome as a costume of the robot, and it (the robot) was an interactive real-time installation. The foam [on the dome] got a crack, and I decided to keep it cracking throughout the exhibition for one week. The crack in the costume was tearing down and it created a different artistic situation I could not have planned. It was so strong that it changed the whole work... I want to be sensible to those moments and see them as part of the process... I don't see that (situation) as, `okay, that is now destroying my artwork.' No, it is evolving or creating a new one within.
\end{quote}

This case illustrates how robotic artwork is not fixed but remains malleable even during the exhibition stage; temporal changes within the artwork can introduce new artistic qualities that evolve the work beyond its original design. Highlighting the artwork's temporality here allows for elucidating how the current state of the created artifact and creativity come to be. The practice of robotic art thus extends beyond the creation stage, encompassing the exhibition period. While in many cases the creation process is well planned, and temporal changes during the exhibition are typically unforeseen, both processes reveal that robotic art is in a state of ongoing creation across time. By paying attention to these processes, we unravel the temporal dimension that contributes to the creative values in robotic art.

In this Findings section, we have highlighted four aspects of robotic art practice that contribute to the artistic quality of the work or to achieving some artistic goals. The analysis reveals how various actors—artists, robots, audiences, and environments---are involved in the practice, influencing one another. These interactive patterns explain how creativity in robotic art is distributed within and emerges from the relations of actors. This idea echos with Daniel's reflection, as he noted that he sees robotic artwork as \textit{``a product of a situation of a creative potential that is part of the environment, all the entities involved as well as me,''} emphasizing the distributed and emergent nature of creativity in robotic art.

\section{Discussion}
\subsection{Revisiting the Explicit-Implicit Dichotomy Through Intention}
% 1. Why our findings of the ``intentional implicit feedback'' are important. We can talk about the benefits: identifying the intentional implicit feedback can (1) help with users' understanding and control of the algo and achieve more desirable personal recommendation, (2) reduce user unintended or unexpected outcome (false positive error), and (3) free up users' cognition for other tasks. 
% We can also talk about the problems of the explicit–implicit dichotomy...

The explicit-implicit feedback dichotomy was largely framed from the system's perspective, as Jannach et al. framed it---implicit feedback referred to interactions from which we (the system side) can indirectly infer user preferences ~\cite{jannach2018recommending}. The implicit feedback was supposed to be user's natural interactions with the platform. However, from the user’s perspective, they are not only aware of, but also intentionally, proactively act as feedback inputs to influence the recommendation feeds. For instance, in our findings, when participants ``dwell on'' or ``swipe past'' a post, they might consider it as an explicit expression of their preferences. Yet, the system might take it as implicit feedback and fail to understand or capture it. This highlights a fundamental challenge: feedback categorized as implicit from the system's point of view may be deliberately used as explicit by users.

This misalignment between how users and the system perceive feedback can lead to inaccuracies or failures in capturing user intent. A specific problem is that implicit feedback is often only interpreted as positive feedback~\cite{jawaheer2014modeling,hu2008collaborative}, whereas our findings revealed that users also intentionally employed implicit feedback behaviors to communicate negative feedback, such as ignoring or swiping past a post ($n = 21$), stop using the platform or switch platforms ($n = 12$), and refresh the feed ($n = 2$). Neglecting on interpreting negative implicit feedback could distort the user profile~\cite{hu2008collaborative}. Thus, incorporating both positive and negative implicit feedback is essential for more accurate recommendation. Another related problem is that the platform' focus on implicit positive feedback may lead to instances of false positive interpretation. For instance, several participants mentioned they searched for information as a one-time task, but the platform over interpreted this as a preference, resulting in their feed being flooded with similar content that was no longer relevant or needed. This suggests that implicit feedback should be considered through more nuanced distinctions, such as positive versus negative, intentional versus incidental~\cite{dix2002beyond}, and reactive versus proactive behaviors~\cite{ju2008range}. Particularly, defining explicit–implicit distinction through intention can add greater precision to the interpretation of user feedback~\cite{serim2019explicating}. 
% implicit feedback has traditionally been treated as positive, and the lack of attention to negative signals can lead to misrepresenting user preferences ~\cite{hu2008collaborative}. However, our participants frequently used implicit feedback to express negative signals, with actions like ignoring or swiping past posts being the most common.

%Therefore, the addition of intentional implicit feedback to the implicit-explicit dichotomy in our paper can help platforms better distill users' intention, addressing the above issues to improve the accuracy in predicting user preferences, achieving more desired personalized recommendations.Comparing to unintentional implicit feedback, 
In addition, acknowledging and responding to the intentional implicit feedback can afford users more understanding of the algorithm’s learning process. Users will feel that their interactions are shaping the recommendation feeds, thus foster a greater sense of self-causality and increase user agency~\cite{feng2024mapping}. Comparing to explicit feedback, intentional implicit feedback also frees up users’ cognitive resources. Research has shown that explicit feedback often imposes a higher cognitive load, as users must actively engage with the system ~\cite{kelly2003implicit,gadanho2007addressing}. In contrast, allowing users to guide recommendations through unobtrusive, yet intentional, behaviors can make the interaction more seamless and less mentally taxing. This aligns with findings from previous studies that highlight the importance of minimizing user effort in feedback mechanisms.

\subsection{Aligning Feedback Types with Users' Purposes}
% Discuss why users chose to use different feedback types for different purposes. The nature and characteristics of the feedbacks.
The study identified four purposes in participants' using of personalized recommendation platforms: content consumption, directed information seeking, content creation and promotion, and feed customization. The first three purposes echo previous literature, which has discussed directed and undirected consumption~\cite{feng2024mapping}, as well as self-presentation ~\cite{devito2017algorithms}. We identified feed customization as a purpose that arises when users are dissatisfied with the recommendation contents, which encompasses four specific goals. 

Users choose different feedback types based on their purposes. Explicit feedback (75.9\% instances) is primarily used for feed customization. As shown in~\autoref{tab:feedbackType}, all explicit feedback behaviors were supported by platform features, providing cues to helps users understand the potential outcomes of their actions. Such clarity makes explicit feedback the preferred choice when users seek quick corrections to their recommendation feeds. For instance, our findings indicated that most users opted for explicit feedback, particularly negative explicit feedback, such as marking content as ``Not interested,'' blocking, or reporting, when they had a strong desire to remove inappropriate or irrelevant content from their feeds. Previous studies found that users were reluctant to provide explicit feedback in complex tasks like online shopping but were more willing to do so in media and entertainment domains~\cite{white2005study,jawaheer2014modeling}.
This aligns with our findings that, on personalized recommendation platforms, users did not show a strong preference between explicit (31 instances) and implicit feedback (44 instances) during feed customization.

On the other hand, intentional implicit feedback (88.6\%) was also mostly used for feed customization, but often when users' intent is not particularly strong or urgent---either to increase content diversity or improve recommendation relevance. Intentional implicit feedback allows users to steer the algorithm towards their evolving preferences without overtly signaling their intent or interrupting the flow of content consumption. The choice among various implicit feedback behaviors is usually based on their continuously updated folk theories~\cite{devito2017algorithms}. Since users are not fully certain how the algorithm will interpret their actions ~\cite{eslami2015always}, they tend to experiment and make repeated attempts when necessary. Despite these efforts, our participants sometimes observed minimal or delayed changes in the recommendation feeds, due to the less obvious and plausible nature of implicit feedback~\cite{jannach2018recommending}. This then may then lead to more passive behaviors like stop using the platforms.

Moreover, the design and functionality of the platforms~\cite{liu2010personalized}, such as Douyin’s swipe interaction versus Xiaohongshu’s click-and-select feed layout, also mediate users' specific feedback behaviors. Participants on Xiaohongshu, for example, found selecting and clicking more efficient for improving recommendation relevance, comparing to using the swipe function in Douyin. Therefore, platform design should support a wider range of intentional implicit feedback by providing users with clearer traits of how their actions influence recommendation feeds and more intuitive ways to interact with the platform.

\subsection{Design Implications}
% 1. supporting intentional implicit feedback: (1) provide features to support or prompts, but also remain unobtrusive for natural flow. (2) support negative feedback.Most used implicit and explicit are both negative feedback.
% (3)  The need for users to repeatedly mark content as ``Not interested'' suggests that the algorithm may not be efficiently learning from initial feedback. Enhancing the algorithm's ability to quickly adapt to user preferences after receiving negative feedback could improve the user experience by reducing the frequency of unwanted recommendations. but also not too sensitive to reduce false positive 
% 2. Incorporate user goals. such as balance between relevance and diversity
% 3. transparency in data collected and protect privacy 
The discussion leads to the following design implications for the personalized recommendation platforms to effectively support and respond to user feedback driven by various purposes.

\subsubsection{Supporting Intentional Implicit Feedback Through Recognition and Visualization} 
First, platforms need to support intentional implicit feedback. When users intentionally provide implicit feedback, they expect the algorithm to infer their interest from their ``natural interactions''~\cite{kelly2003implicit} and respond seamlessly. This differs from explicit feedback, which is given through direct user input that interrupts users' natural flow. However, when the subtle feedback goes unnoticed or yields little immediate effect, users may feel a loss of control. Therefore, while most implicit feedback is not supported by platform features (shown in~\autoref{tab:feedbackType}), it's important to inform users that their feedback has been recognized and will influence future recommendation contents. One potential solution is to implement non-intrusive visual cues, such as small icons or pop-ups, that confirm feedback has been registered. For instance, after a user click similar posts several times or search for a topic, a pop-up could confirm that the platform has recognized this interest and will increase similar content recommendations. Another option is to provide progress indicators that allow users to see how their behavior is affecting the feeds over time. This could include a dashboard showing how many times they had skipped certain content and how it has adjusted their recommendation feeds. This can increase transparency and help users understand how the algorithm is learning from their feedback and responding accordingly. 

Additionally, platforms need to improve their handling of negative implicit feedback. The challenge with interpreting negative implicit feedback lies in its subtlety. Unlike explicit feedback like marking as not interested, implicit negative signals are often less direct and harder to capture. To address this issue, platforms need to adopt more sophisticated approaches to detect and validate implicit negative feedback. While some platforms have already implemented ``show less'' features that allow users to indicate their content preferences (such as ~\cite{meta_new_2022}). These features are typically static and presented with all content. What's needed is a more dynamic approach that can capture and respond to users' behavioral patterns in real-time. One potential strategy is explicit user confirmation. For example, when a user repeatedly swipes past posts or consistently ignores content, the system could proactively detect that the user is not interested in this topic and prompt the user with an option like, ``Want to see less of this content?'' to confirm their intent. Such dynamic approaches can capture users' intentional implicit feedback behaviors more effectively and would not cause a significant disruption to users' normal usage.
% acknolwege current available design(https://ai.meta.com/blog/facebook-feed-improvements-ai-show-more-less/#:~:text=The%20challenge%20with%20training%20these,they%20would%20like%20to%20see)

\subsubsection{Designing for Purpose-Oriented Feedback}
Second, we suggest designing for purpose-oriented feedback. Our findings revealed close association between users' purposes and their choice of feedback types. This underscores the importance of incorporating users' diverse and evolving purposes in the design of personalized recommendation platforms in addition to focusing on their behaviors~\cite{liang2023enabling, nazari2022choice}. To address this, platforms can integrate context-aware learning models to infer user purposes. For example, during directed information seeking, algorithm can prioritize relevance and rely more on implicit feedback~\cite{white2005study}; in content consumption, the algorithm can also consider content diversity, as many participants were concerned with information cocoon~\cite{li2022exploratory}. Meanwhile, the platforms can also provide customization options, such as a slider, for users to adjust the balance between relevance and diversity in their recommendation contents. It should be mention that these setting options, such as setting up interested channels, should be made more discoverable, as some participants did not notice the feature~\cite{liu2024train}.

\subsubsection{Transparency in Feedback Data Collection and Protecting Privacy}
Quite a few participants expressed concerns about their privacy, noting that platforms seemed to be monitoring their conversations, chats with friends, and even activities on other platforms. Despite these concerns, only two participants took active steps to protect their privacy by disabling personalization. However, after noticing that the recommendation quality significantly deteriorated, they eventually re-enabled it. This suggests that users have no effective means to combat platforms' invasion of privacy. To address these concerns, platforms should be transparent about the types of data being collected, including behaviors within the platform, cross-platform tracking, and even offline monitoring. Platforms should allow users to customize their privacy settings, giving them the ability to select which types of data they are comfortable sharing for algorithm personalization. This could include options to opt out of specific data types, while still maintaining a personalized recommendation system.

Interestingly, as we mentioned in the results, one participant mentioned deliberately speaking aloud to trigger the platform's monitoring and influence recommendation feeds. This behavior highlights a potential design opportunity: platforms could introduce voice-based feedback to explicitly capture users' preferences, rather than continuously monitoring their voices. By offering users the ability to verbally express their content needs, platforms could create a more transparent and interactive method for users to directly influence recommendation feeds while respecting their privacy.


% 4. What will be the challenges of designing for ``intentional implicit feedback''? For example, ``constructive validity''~\cite{serim2019explicating}---the extent about how intentional implicit feedback translates into design in our case. 
% Other challenges can be users' literacy to use these designs, users' fluid needs, and delayed system reactions. According to Serim and Jacucci~\cite{serim2019explicating}, users' attitudes can be fluid and influenced by situational factors, which poses uncertainties to determining implicitness.
\subsection{Limitations and Future Work}
This study has several limitations that should be noted. First, the recruitment method relied on pre-screening surveys and snowball sampling. This might have introduced selection bias, favoring more engaged platform users and educated users. Second, the feedback behaviors reported by users were neither comprehensive nor reflective of the ground truth. Specifically, the list of unintentional implicit feedback behaviors in ~\autoref{appendix:unintentional} includes only interactions mentioned by users and categorized as implicit feedback in previous literature~\cite{jannach2018recommending}. We do not have access to the complete set of behaviors that platforms use as implicit feedback. 

% \subsection{Future Work}
% 研究什么样的人会用intentional里的所有behavior,和他们的demo characteristics,以及为何选择specific behavior
Our findings shed light on how users leveraged different feedback mechanisms to fulfill their purposes on personalized recommendation platforms, but further investigation is needed to validate and expand upon these insights. One potential future work is to conduct a large-scale survey to confirm the feedback behaviors and patterns identified in our study. Although our interviews touched upon users' motivations for selecting different types of feedback in various contexts and purpose-oriented situations, a survey study would allow us to explore users' rationales when choosing different feedback types in different contexts. Another important direction for future research is to examine how different types of feedback—explicit feedback, unintentional implicit feedback, and intentional implicit feedback—impact user experience and satisfaction with recommendation systems. This could include how each feedback type influences content quality, user engagement, and perceived control over the recommendation system. Additionally, future research could examine the relationships between users' feedback choices and factors such as demographics and algorithm literacy. Understanding how these factors influence users' engagement with feedback mechanisms could support the design of more personalized and effective feedback mechanisms.
\section{Conclusion }
This paper introduces the Latent Radiance Field (LRF), which to our knowledge, is the first work to construct radiance field representations directly in the 2D latent space for 3D reconstruction. We present a novel framework for incorporating 3D awareness into 2D representation learning, featuring a correspondence-aware autoencoding method and a VAE-Radiance Field (VAE-RF) alignment strategy to bridge the domain gap between the 2D latent space and the natural 3D space, thereby significantly enhancing the visual quality of our LRF.
Future work will focus on incorporating our method with more compact 3D representations, efficient NVS, few-shot NVS in latent space, as well as exploring its application with potential 3D latent diffusion models.



%%
%% The acknowledgments section is defined using the "acks" environment
%% (and NOT an unnumbered section). This ensures the proper
%% identification of the section in the article metadata, and the
%% consistent spelling of the heading.
\begin{acks}
We thank the participants for sharing their experiences and insights. We would also like to express our appreciation to Zitong Huang and Yaqi Zhang for their assistance in collecting and analyzing the interview data. Additionally, we extend our thanks to Lu Xian, Zefeng Ben Zhang, and all reviewers for their thoughtful feedback, which greatly contributed to refining this work. This research was supported by the NSFC Grant No.72174014 and funding from the Inequality in America Initiative at Harvard University. 
\end{acks}

%%
%% The next two lines define the bibliography style to be used, and
%% the bibliography file.
\balance
\bibliographystyle{ACM-Reference-Format}
\bibliography{0-references, references_poster}


%%
%% If your work has an appendix, this is the place to put it.
\newpage
\section{Additional Related Works}
\label{sec:app-add-rel-works}
\subsection{Training Data Selection}

\begin{figure*}[!ht]
    \centering
    \includegraphics[width=\textwidth]{figs/per-token-loss-diff.pdf}
    \caption{Histograms of MIA signal of tokens. Each figure depicts a sample. Blue means the member samples while orange represents the non-member samples. We limited the y-axis range to -3 to 3 for better visibility, so it can result in missing several non-significant outliers.}
    \label{fig:add-per-token-loss}
\end{figure*}

Training data selection are methods that filter high-quality data from noisy big data \textit{before training} to improve the model utility and training efficiency. There are several works leveraging reference models~\cite{Coleman2020Selection, xie2023doremi}, prompting LLMs~\cite{li-etal-2024-one}, deduplication~\cite{lee2022deduplicating, kandpal2022deduplicating}, and distribution matching~\cite{kang2024get}. However, we do not aim to cover this data selection approach, as it is orthogonal and can be combined with ours.


\subsection{Selective Training}
Selective training refers to methods that \textit{dynamically choose} specific samples or tokens \textit{during training}. Selective training methods are the most relevant to our work. Generally, sample selection has been widely studied in the context of traditional classification models via online batch selection~\cite{loshchilov2016o, Angelosonl, pmlr-v108-kawaguchi20a}. These batch selection methods replace the naive random mini-batch sampling with mechanisms that consider the importance of each sample mainly via their loss values. ~\citet{2022PrioritizedTraining} indeed choose highly important samples from regular random batches by utilizing a reference model. However, due to the sequential nature of LLMs, which makes the training significantly different from the traditional classification ML, sample-level selection is not effective for language modeling~\cite{kaddour2023no}. \citet{lin2024not} extend the reference model-based framework to select meaningful tokens within batches. All of the previous methods for selective training aim to improve the training performance and compute efficiency. Our work is the first looking at this aspect for defending against MIAs.

\section{Token-level membership inference risk analysis}
Figures~\ref{fig:add-per-token-loss} and~\ref{fig:add-per-token-dynamics} present the analysis for additional samples. Generally, the trends are consistent with the one presented in Section~\ref{sec:analysis}.

\begin{figure*}[!ht]
    \centering
    \includegraphics[width=0.28\textwidth]{figs/mia-ranking_1.png}
    \includegraphics[width=0.28\textwidth]{figs/mia-ranking_2.png}
    \includegraphics[width=0.3\textwidth]{figs/mia-ranking_3.png}    
    \caption{MIA signal ranking of tokens during training. Each figure illustrates a sample.}
    \label{fig:add-per-token-dynamics}
\end{figure*}

\label{sec:app-analysis}

\section{Experiment settings}
\subsection{Implementation details}
\label{sec:app-implementation}
$\bullet$ \textbf{FT}. We implement the conventional fine tuning using Huggingface Trainer. We manually tune the learning rate to make sure no significant underfitting or overfitting. The batch size is selected appropriately to fit the physical memory and comparable with the other methods'.

\noindent $\bullet$ \textbf{Goldfish}. Goldfish is also implemented with Huggingface Trainer, where we custom the \texttt{compute\_loss} function. We implement the deterministic masking version rather than the random masking to make sure the same tokens are masked over epochs, potentially leading to better preventing memorization. The learning rate is also manually tuned, we noticed that the optimal Goldfish learning rate is usually slightly greater than FT's. This can be the gradients of two methods are almost similar, Goldfish just removes some tokens' contribution to the loss calculation. The batch size of FT can set as the same as FT, as Goldfish does not have significant overhead on memory.

\noindent $\bullet$ \textbf{DPSGD}. DPSGD is implemented by FastDP~\cite{bu2023zero}. We implement DPSGD with fastDP~\cite{bu2023zero} which offers state-of-the-art efficiency in terms of memory and training speed. We also use automatic clipping~\cite{bu2023automatic} and a mixed optimization strategy~\cite{mixclipping} between per-layer and per-sample clipping for robust performance and stability.

\noindent $\bullet$ \textbf{\methodname}. We implement \methodname using Huggingface Trainer, same as FT and Goldfish. The learning is reused from FT. The batch size of \methodname is usually smaller than FT and Goldfish when the model becomes large such as Pythia and Llama 2 due to the reference model, which consumes some memory.

For a fair comparison, we aim to implement the same batch size for all methods if feasible. In case of OOM (out of memory), we perform gradient accumulation, so all the methods can have comparable batch sizes. We provide the hyper-parameters of method for GPT2 in Table~\ref{tab:hyperparameter}. For Pythia and Llama 2, the learning rate, batch size, and number of epochs are tuned again, but the hyper-parameters regarding the privacy mechanisms remain the same. To make sure there is no naive overfitting, we evaluate the methods by selecting the best models on a validation set. Moreover, the testing and attack datasets remains identical for evaluating all methods. Additionally, we balance the number of member and non-member samples for MIA evaluation. It is worth noting that for the ablation study and analysis, if not state, the default model architecture and dataset are GPT2 and CC-news.

\begin{table*}[!ht]
    \centering
    \begin{tabular}{c|clc}
    \textbf{LLM} & \textbf{Method} & \textbf{Hyper-parameter} & \textbf{Value}  \\ \hline
     \multirow{22}{*}{\textbf{GPT2}}  &  \multirow{4}{*}{FT} &  Learning rate & 1.75e-5 \\ 
     & & Batch size & 96 \\
     & & Gradient accumulation steps & 1 \\
     & & Number of epochs & 20 \\ \cline{2-4}
       &  \multirow{5}{*}{Goldfish} &  Learning rate & 2e-5 \\ 
     & & Batch size & 96 \\
     & & Grad accumulation steps & 1 \\
     & & Number of epochs & 20 \\
     & & Masking Rate & 25\% \\ \cline{2-4}
           &  \multirow{6}{*}{DPSGD} &  Learning rate & 1.5e-3 \\ 
     & & Batch size & 96 \\
     & & Grad accumulation steps & 1 \\
     & & Number of epochs & 10 \\
     & & Clipping & automatic clipping \\ 
     & & Privacy budget & (8, 1e-5)-DP \\ \cline{2-4}
           &  \multirow{6}{*}{DuoLearn} &  Learning rate & 1.75e-3 \\ 
     & & Batch size & 96 \\
     & & Grad accumulation steps & 1 \\
     & & Number of epochs & 20 \\
     & & $K_h$ & 60\% \\ 
     & & $K_m$ & 20\% \\
     & & $\tau$ & 0 \\
     & & $\alpha$ & 0.8 \\ \hline
    \end{tabular}
    \caption{Hyper-parameters of the methods for GPT2.}
    \label{tab:hyperparameter}
\end{table*}


\section{Additional Results}
\label{sec:app-add-res}

\begin{figure}[!ht]
    \centering
    \includegraphics[width=0.8\linewidth]{figs/add_loss_vs_steps_ft_duolearn.pdf}
    \caption{Breakdown to the cross entropy loss values of FT on the testing set and \methodname on the training set during training.}
    \label{fig:add-overlap-breakdown}
\end{figure}

\subsection{Overall Evaluation}
% \begin{table*}[htp]
%     \centering
%     \begin{tabular}{cl|ccccc|ccccc}
%      \multirow{3}{*}{\textbf{LLM}}  & \multirow{3}{*}{\textbf{Method}} &  \multicolumn{5}{c|}{\textbf{CCNews}} & \multicolumn{5}{c}{\textbf{Wikipedia}} \\ \cmidrule(lr){3-7}  \cmidrule(lr){8-12}
%       &  & PPL & Loss & Ref & min-k & \multicolumn{1}{c|}{zlib} & PPL & Loss & Ref & min-k & zlib \\ \midrule
%       \multirow{4}{*}{GPT2} & \textit{Base} & \textit{29.442} & \textit{0.018} & \textit{0.002} & \textit{0.022} & \textit{0.006} & \textit{34.429} & \textit{0.002} & \textit{0.014} & \textit{0.010} & \textit{0.002} \\ 
%       \multirow{4}{*}{124M} & FT & \textbf{21.861} & 0.030 & 0.026 & 0.016 & 0.016 & \textbf{12.729} & 0.018 & 0.574 & 0.016 & 0.014 \\
%       & Goldfish & 21.902 & 0.030 & 0.024 & 0.028 & 0.016 & 12.853 & 0.018 & 0.632 & 0.016 & 0.010 \\
%       & DPSGD & 26.022 & \textbf{0.018} & \textbf{0.004} & \textbf{0.018} & 0.008 & 18.523 & \textbf{0.004} & 0.036 & 0.018 & 0.006 \\
%       & \methodname & 23.733 & 0.030 & 0.022 & 0.026 & \textbf{0.006} & 13.628 & 0.014 & \textbf{0.010} & \textbf{0.014} & \textbf{0.004} \\ \midrule
      
%       \multirow{4}{*}{Pythia} & \textit{Base} & \textit{13.973} & \textit{0.002} & \textit{0.008} & \textit{0.020} & \textit{0.014} & \textit{10.287} & \textit{0.002} & \textit{0.014} & \textit{0.006} & \textit{0.008} \\ 
%       \multirow{4}{*}{1.4B} & FT & 11.922 & 0.014 & 0.008 & 0.022 & 0.020 & \textbf{6.439} & 0.020 & 0.440 & 0.010 & 0.020 \\
%       & Goldfish & \textbf{11.903} & 0.014 & 0.008 & 0.024 & 0.018 & 6.465 & 0.016 & 0.412 & 0.010 & 0.020 \\
%       & DPSGD & 13.286 & \textbf{0.002} & \textbf{0.004} & \textbf{0.018} & \textbf{0.014} & 7.751 & \textbf{0.004} & \textbf{0.016} & {0.010} & \textbf{0.004} \\
%       & \methodname & 12.670 & 0.004 & 0.020 & \textbf{0.018} & 0.016 & 6.553 & 0.008 & 0.030 & \textbf{0.006} & 0.006 \\ \midrule
      
%       \multirow{4}{*}{Llama-2} & \textit{Base} & \textit{9.364} & \textit{0.006} & \textit{0.006} & \textit{0.024} & \textit{0.006} & \textit{7.014} & \textit{0.006} & \textit{0.016} & \textit{0.016} & \textit{0.010} \\ 
%       \multirow{4}{*}{7B} & FT & \textbf{6.261} & 0.002 & 0.018 & 0.002 & 0.002 & \textbf{3.830} & 0.028 & 0.170 & 0.030 & 0.028 \\
%       & Goldfish & 6.280 & 0.002 & 0.018 & 0.002 & 0.006 & 3.839 & 0.028 & 0.198 & 0.028 & 0.028 \\
%       & DPSGD & 6.777 & 0.008 & 0.026 & 0.016 & 0.010 & 4.490 & \textbf{0.006} & 0.014 & \textbf{0.020} & \textbf{0.010} \\
%       & \methodname & 6.395 & \textbf{0.002} & \textbf{0.020} & \textbf{0.004} & \textbf{0.002} & 4.006 & 0.010 & \textbf{0.002} & 0.028 & 0.012 \\ 
%     \end{tabular}
%     \caption{TPR at FPR of 1\% \textcolor{red}{TODO: check consistency with the main table of MIA AUC scores}}
%     \label{tab:tpr}
% \end{table*}


\begin{table*}[!ht]
  \centering
  \resizebox{\textwidth}{!}{\begin{tabular}{cl|ccccc|ccccc}
   \multirow{3}{*}{\textbf{LLM}}  & \multirow{3}{*}{\textbf{Method}} &  \multicolumn{5}{c|}{\textbf{Wikipedia}} & \multicolumn{5}{c}{\textbf{CC-news}} \\ \cmidrule(lr){3-7}  \cmidrule(lr){8-12}
    &  & PPL & Loss & Ref & min-k & \multicolumn{1}{c|}{zlib} & PPL & Loss & Ref & min-k & zlib \\ \midrule
    \multirow{4}{*}{GPT2} & \textit{Base} & \textit{34.429} & \textit{0.002} & \textit{0.014} & \textit{0.010} & \textit{0.002} & \textit{29.442} & \textit{0.018} & \textit{0.002} & \textit{0.022} & \textit{0.006} \\ 
    \multirow{4}{*}{124M} & FT & \textbf{12.729} & 0.018 & 0.574 & 0.016 & 0.014 & \textbf{21.861} & 0.030 & 0.026 & 0.016 & 0.016 \\
    & Goldfish & 12.853 & 0.018 & 0.632 & 0.016 & 0.010 & 21.902 & 0.030 & 0.024 & 0.028 & 0.016 \\
    & DPSGD & 18.523 & \textbf{0.004} & 0.036 & 0.018 & 0.006 & 26.022 & \textbf{0.018} & \textbf{0.004} & \textbf{0.018} & 0.008 \\
    & \methodname & 13.628 & 0.014 & \textbf{0.010} & \textbf{0.014} & \textbf{0.004} & 23.733 & 0.030 & 0.022 & 0.026 & \textbf{0.006} \\ \midrule
    
    \multirow{4}{*}{Pythia} & \textit{Base} & \textit{10.287} & \textit{0.002} & \textit{0.014} & \textit{0.006} & \textit{0.008} & \textit{13.973} & \textit{0.002} & \textit{0.008} & \textit{0.020} & \textit{0.014} \\ 
    \multirow{4}{*}{1.4B} & FT & \textbf{6.439} & 0.020 & 0.440 & 0.010 & 0.020 & 11.922 & 0.014 & 0.008 & 0.022 & 0.020 \\
    & Goldfish & 6.465 & 0.016 & 0.412 & 0.010 & 0.020 & \textbf{11.903} & 0.014 & 0.008 & 0.024 & 0.018 \\
    & DPSGD & 7.751 & \textbf{0.004} & \textbf{0.016} & {0.010} & \textbf{0.004} & 13.286 & \textbf{0.002} & \textbf{0.004} & \textbf{0.018} & \textbf{0.014} \\
    & \methodname & 6.553 & 0.008 & 0.030 & \textbf{0.006} & 0.006 & 12.670 & 0.004 & 0.020 & \textbf{0.018} & 0.016 \\ \midrule
    
    \multirow{4}{*}{Llama-2} & \textit{Base} & \textit{7.014} & \textit{0.006} & \textit{0.016} & \textit{0.016} & \textit{0.010} & \textit{9.364} & \textit{0.006} & \textit{0.006} & \textit{0.024} & \textit{0.006} \\ 
    \multirow{4}{*}{7B} & FT & \textbf{3.830} & 0.028 & 0.170 & 0.030 & 0.028 & \textbf{6.261} & 0.002 & 0.018 & 0.002 & 0.002 \\
    & Goldfish & 3.839 & 0.028 & 0.198 & 0.028 & 0.028 & 6.280 & 0.002 & 0.018 & 0.002 & 0.006 \\
    & DPSGD & 4.490 & \textbf{0.006} & 0.014 & \textbf{0.020} & \textbf{0.010} & 6.777 & 0.008 & 0.026 & 0.016 & 0.010 \\
    & \methodname & 4.006 & 0.010 & \textbf{0.002} & 0.028 & 0.012 & 6.395 & \textbf{0.002} & \textbf{0.020} & \textbf{0.004} & \textbf{0.002} \\ 
  \end{tabular}}
  \caption{Overall Evaluation: Perplexity (PPL) and TPR at FPR of 1\% scores of the MIAs with different signals (Loss/Ref/Min-k/Zlib). For all metrics, the lower the value, the better the result.}
  \label{tab:tpr}
\end{table*}
Table~\ref{tab:tpr} provides the True Positive Rate (TPR) at low False Positive Rate (FPR) of the overall evaluation. Generally, compared to CC-news, Wikipedia poses a significant higher risk at low FPR. For example, the reference-based attack can achieve a score of 0.57~ on GPT2 if no protection. In general, Goldfish fails to mitigate the risk in this scenario, while both DPSGD and \methodname offer robust protection.

\subsection{Auxiliary dataset}
We investigate the size of the auxiliary dataset which is disjoint with the training data of the target model and the attack model. In this experiment, the other methods are trained with 3K samples. Figure~\ref{fig:aux_size} presents the language modeling performance while varying the auxiliary dataset's size. The result demonstrates that the better reference model, the better language modeling performance. It is worth noting that even with a very small number of samples, \methodname can still outperform DPSGD. Additionally, there is only a little benefit when increasing from 1000 to 3000, this indicates that the reference model is not needed to be perfect, as it just serves as a calibration factor. This phenomena is consistent with previous selective training works~\cite{lin2024not, 2022PrioritizedTraining}.
\begin{figure}
    \centering
    \includegraphics[width=0.8\linewidth]{figs/auxiliary_size.pdf}
    \caption{Language modeling performance while varying the auxiliary dataset's size. Note that the results of FT and Goldfish are significantly overlapping.}
    \label{fig:aux_size}
\end{figure}

\subsection{Training time}
We report the training time for full fine-tuning Pythia 1.4B. We manually increase the batch size that could fit into the GPU's physical memory. As a results, FT and Goldfish can run with a batch size of 48, while DPSGD and \methodname can reach the batch size of 32. We also implement gradient accumulation, so all the methods can have the same virtual batch size.

\begin{table}[!ht]
    \centering
    \begin{tabular}{c|c}
        \textbf{Training Time} & \textbf{\textbf{1 epoch}} (in minutes) \\ \hline
        {FT} & 2.10 \\ 
        {Goldfish} & 2.10 \\
        % {RelaxLoss} & 2.10 \\        
        {DPSGD} & 3.19 \\ 
        {DuoLearn} & 2.85 
    \end{tabular}
    \caption{Training time for one epoch of (full) Pythia 1.4B on a single H100 GPU}
    \label{tab:training-time}
\end{table}

Table~\ref{tab:training-time} presents the training time for one epoch. Goldfish has little to zero overhead compared to FT. DPSGD and \methodname have a slightly higher training time due to the additional computation of the privacy mechanism. In particular, DPSGD has the highest overhead due to the clipping and noise addition mechanisms. Meanwhile, \methodname requires an additional forward pass on the reference model to select the learning and unlearning tokens. \methodname is also feasible to work at scale that has been demonstrated in the pretraining settings of the previous work~\cite{lin2024not}.

\section{Limitations}
The main limitation of our work is the small-scale experiment setting due to the limited computing resources. However, we believe \methodname can be directly applied to large-scale pretraining without requiring any modifications, as done in previous selective pretraining work~\cite{lin2024not}. Another limitation is the reference model, which may be restrictive in highly sensitive or domain-limited settings~\cite{tramr2024position}. From a technical perspective, while we show that \methodname performs well across different datasets and architectures, there is room for improvement. The current approach selects a fixed number of tokens, which may not be optimal since selected tokens contribute unequally. Future work could explore adaptive selection or weighted tokens' contribution. At a high-level, compared to DPSGD, \methodname has not been supported by theoretical guarantees. Future work can investigate the convergence and overfitting analysis.
\end{document}
\endinput
%%
%% End of file `sample-manuscript.tex'.
