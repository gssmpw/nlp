
\documentclass[sigconf,screen,authorversion]{acmart}
\usepackage[utf8]{inputenc}
% \usepackage{standalone} 
\usepackage{makecell}
 \usepackage{acmart-taps}
\usepackage{subcaption}
%%
%% \BibTeX command to typeset BibTeX logo in the docs
\AtBeginDocument{%
  \providecommand\BibTeX{{%
    Bib\TeX}}}

% %% Rights management information.  This information is sent to you
% %% when you complete the rights form.  These commands have SAMPLE
% %% values in them; it is your responsibility as an author to replace
% %% the commands and values with those provided to you when you
% %% complete the rights form.
% \setcopyright{acmlicensed}
% \copyrightyear{2018}
% \acmYear{2018}
% \acmDOI{XXXXXXX.XXXXXXX}

% %% These commands are for a PROCEEDINGS abstract or paper.
% \acmConference[Conference acronym 'XX]{Make sure to enter the correct
%   conference title from your rights confirmation emai}{June 03--05,
%   2018}{Woodstock, NY}
% %%
% %%  Uncomment \acmBooktitle if the title of the proceedings is different
% %%  from ``Proceedings of ...''!
% %%
% %%\acmBooktitle{Woodstock '18: ACM Symposium on Neural Gaze Detection,
% %%  June 03--05, 2018, Woodstock, NY}
% \acmISBN{978-1-4503-XXXX-X/18/06}
\copyrightyear{}
\acmYear{2025}
\setcopyright{none}
\setcctype{}
\acmConference[CHI '25]{CHI Conference on Human Factors in Computing Systems}{April 26-May 1, 2025}{Yokohama, Japan}
\acmBooktitle{CHI Conference on Human Factors in Computing Systems (CHI '25), April 26-May 1, 2025, Yokohama, Japan}\acmDOI{10.1145/3706598.3713241}
% \acmISBN{979-8-4007-1394-1/25/04}



%%
%% end of the preamble, start of the body of the document source.
\begin{document}

%%
%% The "title" command has an optional parameter,
%% allowing the author to define a "short title" to be used in page headers.
\title[How Users Provide Feedback to Shape Personalized Recommendation Content]{Beyond Explicit and Implicit: How Users Provide Feedback to Shape Personalized Recommendation Content}%Beyond Explicit and Implicit: How Users Leverage Feedback Mechanisms to Influence Personalized Recommendations

%%
%% The "author" command and its associated commands are used to define
%% the authors and their affiliations.
%% Of note is the shared affiliation of the first two authors, and the
%% "authornote" and "authornotemark" commands
%% used to denote shared contribution to the research.
% wenqi ruiqing manyu pengyi qunfang
% \author{Ben Trovato}
% \authornote{Both authors contributed equally to this research.}
% \email{trovato@corporation.com}
% \orcid{1234-5678-9012}
% \author{G.K.M. Tobin}
% \authornotemark[1]
% \email{webmaster@marysville-ohio.com}
% \affiliation{%
%   \institution{Institute for Clarity in Documentation}
%   \city{Dublin}
%   \state{Ohio}
%   \country{USA}
% }

% \author{Lars Th{\o}rv{\"a}ld}
% \affiliation{%
%   \institution{The Th{\o}rv{\"a}ld Group}
%   \city{Hekla}
%   \country{Iceland}}
% \email{larst@affiliation.org}

% \author{Valerie B\'eranger}
% \affiliation{%
%   \institution{Inria Paris-Rocquencourt}
%   \city{Rocquencourt}
%   \country{France}
% }

% \author{Aparna Patel}
% \affiliation{%
%  \institution{Rajiv Gandhi University}
%  \city{Doimukh}
%  \state{Arunachal Pradesh}
%  \country{India}}

% \author{Huifen Chan}
% \affiliation{%
%   \institution{Tsinghua University}
%   \city{Haidian Qu}
%   \state{Beijing Shi}
%   \country{China}}

% \author{Charles Palmer}
% \affiliation{%
%   \institution{Palmer Research Laboratories}
%   \city{San Antonio}
%   \state{Texas}
%   \country{USA}}
% \email{cpalmer@prl.com}

% \author{John Smith}
% \affiliation{%
%   \institution{The Th{\o}rv{\"a}ld Group}
%   \city{Hekla}
%   \country{Iceland}}
% \email{jsmith@affiliation.org}

% \author{Julius P. Kumquat}
% \affiliation{%
%   \institution{The Kumquat Consortium}
%   \city{New York}
%   \country{USA}}
% \email{jpkumquat@consortium.net}
\author{Wenqi Li}
\affiliation{%
  \institution{Department of Information Management, Peking University}
  \city{Beijing}
  \country{China}
}\email{wenqili@pku.edu.cn}

\author{Jui-Ching Kuo}
\affiliation{%
  \institution{National Tsing Hua University}
  \city{Hsinchu}
  \country{Taiwan}}
\email{rosieeee321@gmail.com}

\author{Manyu Sheng}
\affiliation{%
  \institution{University of Chinese Academy of Sciences}
  \city{Beijing}
  \country{China}
}\email{shengmanyu@mail.las.ac.cn}

\author{Pengyi Zhang}
\affiliation{%
  \institution{Department of Information Management, Peking University}
  \city{Beijing}
  \country{China}
}\email{pengyi@pku.edu.cn}

\author{Qunfang Wu}
\affiliation{%
  \institution{Harvard University}
  \city{Cambridge}
  \state{MA}
  \country{USA}
}
\email{qunfangwu@fas.harvard.edu}
%%
%% By default, the full list of authors will be used in the page
%% headers. Often, this list is too long, and will overlap
%% other information printed in the page headers. This command allows
%% the author to define a more concise list
%% of authors' names for this purpose.
\renewcommand{\shortauthors}{Li et al.}

\begin{abstract}
Out-of-distribution (OOD) detection and OOD generalization are widely studied in Deep Neural Networks (DNNs), yet their relationship remains poorly understood. We empirically show that the degree of Neural Collapse (NC) in a network layer is inversely related with these objectives: stronger NC improves OOD detection but degrades generalization, while weaker NC enhances generalization at the cost of detection. This trade-off suggests that a single feature space cannot simultaneously achieve both tasks. To address this, we develop a theoretical framework linking NC to OOD detection and generalization. We show that entropy regularization mitigates NC to improve generalization, while a fixed Simplex Equiangular Tight Frame (ETF) projector enforces NC for better detection. Based on these insights, we propose a method to control NC at different DNN layers. In experiments, our method excels at both tasks across OOD datasets and DNN architectures. 

\begin{comment}   

Out-of-distribution (OOD) detection and OOD generalization are critical for deploying machine learning models in real-world scenarios. While substantial progress has been made in addressing these problems independently, few works have attempted to tackle them jointly. However, existing methods often rely on auxiliary OOD training data and primarily focus on covariate-shifted OOD data that share labels with in-distribution (ID) data. In contrast, we tackle the more realistic and challenging task of jointly detecting and generalizing to semantic OOD data with disjoint labels from the ID data, without auxiliary OOD training data.
Achieving both objectives simultaneously is inherently difficult due to a fundamental conflict — OOD generalization requires enhanced transferability, while OOD detection necessitates the inhibition of transfer.
To address this, we leverage insights from neural collapse (NC) — a phenomenon in deep networks where top-layer representations suppress feature variability and adopt a Simplex Equiangular Tight Frame (ETF) structure, impairing transferability. By controlling NC, we unify OOD detection and generalization: preventing NC enhances OOD transfer while inducing NC improves OOD detection.
Our proposed method excels at both tasks across various OOD datasets and architectures. 

\end{comment}


\end{abstract}

%%
%% The code below is generated by the tool at http://dl.acm.org/ccs.cfm.
%% Please copy and paste the code instead of the example below.
%%
\begin{CCSXML}
<ccs2012>
   <concept>
       <concept_id>10003120.10003121.10011748</concept_id>
       <concept_desc>Human-centered computing~Empirical studies in HCI</concept_desc>
       <concept_significance>500</concept_significance>
       </concept>
   <concept>
       <concept_id>10002951.10003260.10003261.10003271</concept_id>
       <concept_desc>Information systems~Personalization</concept_desc>
       <concept_significance>500</concept_significance>
       </concept>
 </ccs2012>
\end{CCSXML}

\ccsdesc[500]{Human-centered computing~Empirical studies in HCI}
\ccsdesc[500]{Information systems~Personalization}

%%
%% Keywords. The author(s) should pick words that accurately describe
%% the work being presented. Separate the keywords with commas.
\keywords{Personalized recommendation algorithm, Explicit feedback, Implicit feedback, User purpose, Semi-structured interview, Xiaohongshu, RedNote, Douyin, TikTok}

% \received{20 February 2007}
% \received[revised]{12 March 2009}
% \received[accepted]{5 June 2009}

%%
%% This command processes the author and affiliation and title
%% information and builds the first part of the formatted document.
\sloppy
\maketitle

\section{Introduction}

% State of the world (robots for creative activites)
The term ``robot,'' originally signifying `forced labor,' has long been associated with labor and work. Robots have demonstrated their utility in various automated productive and social contexts, where the primary goals are improving productivity, safety, and fostering social interactions with humans~\cite{simoes2022designing, weidemann2021role, honig2018understanding}. However, an increasing number of cases feature using of robots in creative settings. Unlike productive contexts, where the focus is on efficiency and task completion~\cite{arents2022smart}, or social contexts, where communication and trust are prioritized~\cite{nam2020trust, saunderson2019robots}, creative environments prioritize artistic innovation and expression~\cite{hsueh2024counts}. This shift fundamentally alters the dynamics of human-robot interaction, redefining the roles and expectations for both humans and robots.

For instance, robots’ social behaviors are leveraged to support the generation and expression of creative ideas~\cite{hu2021exploring, sandoval2022human, alves2020creativity}, and programmable robotic movements and trajectories are employed to inspire artistic activities such as sketching~\cite{lin2020your}. These studies often engage participants from creative fields who possess limited prior experience with robotics, and are typically conducted in short-term, experimental settings. Consequently, the findings from these studies remain constrained since much can be learned from professional practitioners' experiences to inform system design such as digital fabrication~\cite{hirsch2023nothing}. There is a notable gap in research examining the long-term, active, and practical experience of integrating robotic systems into the creative processes. As a result, the deeper insights into how robots facilitate and shape creative processes, beyond simply augmenting human creativity, remain underexplored. In this study, we aim to better understand the impacts of robots on creative processes and outcomes.

As early as Leonardo da Vinci's 16th century ``Automaton,'' artists have explored the creative affordances of robotic systems~\cite{shanken2002cybernetics, pagliarini2009development, jeon2017robotic}. The artistic creation process typically encompasses various stages, including the exploration of materials and techniques, ongoing experimentation and iteration, and the continual refinement of the artists' insights into their creative subjects~\cite{lewis2023art, sturdee2022state}. Therefore, investigating the artistic process involving robots offers an opportunity to gain deeper insights into robots' creative potential. Robotic art, in particular, provides a compelling case for this exploration.

We define robotic art as artworks that utilize robotic or automated machines to create artistic experiences and tangible artifacts. One example is robotic installation art, in which robots are programmed to follow specific rules that embody the artist’s expression (\autoref{fig:teaser} (a)). Another example is responsive art, in which robots react to their environment, with behaviors that change over time or in response to spectators (\autoref{fig:teaser} (b)). Additionally, there are robotic creators, which possess a degree of agency, allowing them to collaborate with human artists and produce works that extend beyond mere replication of human-created art (\autoref{fig:teaser} (c) and (d)). As such, robotic art becomes a rich case for exploring human-machine interactions in creative contexts. Gaining a deeper understanding of how robots facilitate artistic expression can provide insights for designing computing systems to support creative activities~\cite{gomez2021robot}.

% Therefore, we did...
We draw on semi-structured, in-depth interviews with renowned professional robotic artists to investigate the use of robots in artistic practice. Specifically, our goal is to understand how artistic exploration of robotic systems challenges conventional assumptions about the functions of robots, such as their roles in automating repetitive tasks or serving human needs. We also explore the implications of robots in the artistic process and examine how creativity may emerge within robotic art. To address these interrelated inquiries, our study focuses on the practice of robotic art, posing the research question: \textit{How do robotic artists utilize robots in their artistic practice?} We approach this inquiry through the perspectives and experiences of robotic artists, who creatively design, modify, and repurpose robotic systems for artistic expression and exploration.

% The key findings are...
Our findings highlight the social, material, and temporal dimensions of artists' practices that shape their creativity and artistic outcomes. The creation of robotic art is largely a social process, as artists receive both explicit and implicit feedback through the audience's reactions and reception of their work. Simultaneously, the embodiment and malfunctions inherent to robotic systems drive artistic experimentation. The temporal processes of creation and exhibition, beyond just the final product, further enhance the creative value. Our empirical analysis presents how creativity emerges through the interplay of social, material, and temporal interactions among artists, robots, audiences, and the environment.

% The contributions of this work are...
We make two main contributions to HCI in this study. 
First, we elucidate the interactive mechanisms among key actors---human creators, machines, audiences, and environments---within the practice of robotic art, a topic that remains underexplored in HCI. Our findings reveal the significance of sociality (e.g., interactions between artists and audiences), materiality (e.g., the embodiment and malfunctions of robots), and temporality (e.g., the processes of creation and exhibition) in shaping creative values. We propose that these three facets are central to the creative process and facilitate the emergence of creativity in robotic art.
Second, drawing from the findings, we offer implications for \textit{socially informed}, \textit{material-attentive}, and \textit{process-oriented} creation with computing systems. We suggest leveraging these three aspects to enhance creativity and the creative experience. Specifically, we discuss the value of incorporating implicit audience feedback, designing with technical malfunctions, and focusing on the post-creation process to foster alternative creative experiences with machines~\cite{alter2010designing, juarez2022glitch}.



\input{2-litreview}


% \subsection{Notations}  %% commented out as we do not use them

% The notations used throughout this paper are summarized in Table ~\ref{t:notations}.

% \begin{table}
%     \centering
%     \small % Reduce font size for the table (optional)
%     \begin{tabular}{|l|c|}
%         \hline
%         \textbf{Notation} & \textbf{Description}  \\
%         \hline
%         $X_{\text{tr}}$ & Training set inputs (messages) 
%         \\\hline
%         $y_{\text{tr}}^{\text{gold}}$ & Gold labels for $X_{\text{tr}}$\\
%         \hline
%         $y_{\text{tr}}^{\text{llm}}$ & Synthetic labels for $X_{\text{tr}}$ \\ \hline
%          $X_{\text{val}}$ & Validation set inputs \\
%          \hline
%         $y_{\text{val}}^{\text{gold}}$ & Gold labels for $X_{\text{val}}$ \\
%         \hline
%         $y_{\text{val}}^{\text{llm}}$ & Synthetic labels for $X_{\text{val}}$ \\ \hline
%         $X_{\text{test}}$ & Test set inputs \\
%         \hline
%         $y_{\text{test}}^{\text{gold}}$ & Gold labels for $X_{\text{test}}$ \\
%         \hline
%         $y_{\text{test}}^{\text{llm}}$ & Synthetic labels for $X_{\text{test}}$ \\ \hline
%         $(X, y)_{\text{tr}}^{\text{llm}}$ & Synthetic training data \\
%         \hline
%          $(X, y)_{\text{val}}^{\text{llm}}$ & Synthetic validation data \\  
%          \hline
%     \end{tabular}
%     % ACL style has the caption below the table or figure
%     \caption{Summary of notations used in the paper}
%     \label{t:notations} 
% \end{table}


\subsection{Overview of Scenarios}

%In this study, we
We investigate the role of LLMs in CB detection, focusing on their utility under varying data availability conditions
and under the assumption that direct use of LLMs as a classifier is too expensive due to the high volume of messages to be checked.
%To establish
As a baseline for comparison, we %first
evaluate a scenario in which a
lightweight, BERT-based
classifier is trained exclusively on gold-standard, manually labeled authentic data without %any
LLM involvement.
We then define three additional scenarios with different data availability
and that use LLMs in different ways.
%, each illustrating how LLMs can aid in CB detection depending on the availability and quantity of authentic data.
%The scenarios are as follows.

%To establish a baseline for comparison, in the first scenario, we evaluate a setup that relies exclusively on training a classifier using gold-standard, manually labeled authentic data with no LLM involvement. We then define three other distinct scenarios, each corresponding to
% %% JW: The following is unneccesary vague as the scenarios are more specifically
% %% about the way the synthetic data is used, apart from the zero-shot LLM.
% a unique way LLMs can be integrated into the detection pipeline.
% These scenarios range from directly serving as classifiers to generating synthetic data or labels for training. 

\paragraph{Scenario 1: Baseline}

This scenario represents the ideal situation where sufficient
%manually labeled (
gold-standard data is available for fine-tuning %a classic encoder such as
BERT.
It serves as the benchmark for evaluating the effectiveness of other approaches.
In this setup, no synthetic data or LLMs are involved.
%The system relies entirely on human annotations.
This scenario is feasible if resources such as time, budget and expert annotators are abundant. However, it often proves impractical due to the
%high costs and scalability
challenges of manual labeling.



\paragraph{Scenario 2: LLM as Classifier}  \label{s:m:sc2}

This scenario applies when labeled authentic data is unavailable, and there is no intention to train a separate classifier for CB detection. Instead, an instruction-tuned
LLM is used directly as a classifier, leveraging its pre-trained knowledge and its ability to follow instructions
to identify CB instances.
%This approach is particularly useful in contexts that require rapid deployment or when computational or time resources are limited for training a new model. 
The primary advantage of this method is its elimination of the need for labeled data and training time. However, there are trade-offs. While an LLM can handle nuanced language patterns, it may be less efficient and incur higher computational costs
compared to simpler BERT-based classifiers with a classification head and fine-
tuned on a labeled dataset.
%% JW: add reference to large zero-shot study in NLP
We explore two prompting strategies for generating synthetic labels:
\textit{(a)} guideline-enhanced (GE) prompts, guiding the LLM with detailed labeling instructions and
\textit{(b)} guidelne-free (GF) prompts, allowing the LLM to generate labels without such guidelines.

\paragraph{Scenario 3: Fully Synthetic Data}

In this scenario, only a small set of manually labeled gold data is available for testing, with no access to authentic data for training or validation.
%To address this, we
We
use an LLM to generate a fully synthetic dataset, consisting of both synthetic messages and corresponding labels, for training and validation.
This approach is particularly valuable in low-resource domains or emerging tasks where authentic data is scarce or difficult to collect.
It is especially useful in situations where creating authentic datasets is costly, time-consuming, or ethically challenging, such as annotating harmful or sensitive content or working with vulnerable populations.
The effective

%Commented for indusrty track \subsubsection{Scenario 4: Data Augmentation with Synthetic Data}
%This scenario assumes the availability of a moderate amount of gold-labeled data for training and validation, which may be insufficient to achieve optimal performance. To augment the dataset, we use an LLM to generate additional synthetic data, which is then combined with the gold-labeled data during training and validation. The experiment systematically varies the ratio of synthetic-to-gold data to evaluate its impact on model performance. This scenario explores how LLMs can supplement authentic data, striking a balance between scalability and accuracy.


\paragraph{Scenario 4: Synthetic Labels for Unlabeled Data} \label{s:m:sc4}

This scenario addresses the common situation where resources for manual annotation are limited. Here, gold-standard labeled data is available only for the test set, while a significant amount of unlabeled authentic data is available for training and validation.
%This scenario demonstrates the utility of LLMs in resource-constrained settings, enabling cost-effective dataset creation from unannotated corpora.
To utilize the unlabeled data, we label it using the best prompting strategy (GE or GF) from scenario~2.

% \subsubsection{Summary of Scenarios}
% Table~\ref{t:scenario-summary} presents an overview of the data used in the baseline system and each scenario, specifying the datasets utilized for training, validation, and testing. For Scenario 2, where no classifier is trained and the LLM is used directly as a classifier, only the test set is included.
% \begin{table}
%     \centering
%     \small % Reduce font size for the table (optional)
%     \begin{tabularx}{\columnwidth}{|X|X|X|X|}
%         \hline
%         \textbf{Scenario} & \textbf{Train} & \textbf{Validation} & \textbf{Test} \\
%         \hline
%          1 & $X_{\text{tr}}, y_{\text{tr}}^{\text{gold}}$ & $X_{\text{val}}, y_{\text{val}}^{\text{gold}}$ &  $X_{\text{test}}, y_{\text{test}}^{\text{gold}}$ \\
%         \hline
%            2 & - & - & $X_{\text{test}}, y_{\text{test}}^{\text{gold}}$ \\
%         \hline
%           3 & $(X, y)_{\text{tr}}^{\text{llm}}$ & $(X, y)_{\text{val}}^{\text{llm}}$ & $X_{\text{test}}, y_{\text{test}}^{\text{gold}}$ \\
%         \hline
%          4 & $X_{\text{tr}}, y_{\text{tr}}^{\text{llm}}$ & $X_{\text{val}}, y_{\text{val}}^{\text{llm}}$ &  $X_{\text{test}}, y_{\text{test}}^{\text{gold}}$ \\ \hline
       
%         % 4 & $X_{\text{tr}}, y_{\text{tr}}^{\text{llm}} + X_{\text{tr}}, y_{\text{tr}}^{\text{gold}}$ & $X_{\text{val}}, y_{\text{val}}^{\text{llm}}+ X_{\text{val}}, y_{\text{val}}^{\text{gold}}$ & $X_{\text{test}}, y_{\text{test}}^{\text{gold}}$ \\ \hline
%         % 4 & $(X, y)_{\text{tr}}^{\text{llm}} + X_{\text{tr}}, y_{\text{tr}}^{\text{gold}}$ & $(X, y)_{\text{val}}^{\text{llm}}+ X_{\text{val}}, y_{\text{val}}^{\text{gold}}$ & $X_{\text{test}}, y_{\text{test}}^{\text{gold}}$ \\ \hline
%     \end{tabularx}
%     % ACL style has the caption below the table or figure
%     \caption{Overview of data used in each scenario}
% \label{t:scenario-summary} 
% \end{table}





% \subsection{Intrinsic Evaluation Metrics}

% Intrinsic evaluation examines the inherent qualities of datasets, enabling the assessment of linguistic diversity, emotional tone, and conversational structure independently from task-specific performance. For our CB detection task, we utilize \textbf{four} categories of intrinsic metrics to compare the authentic dataset with LLM-generated synthetic data. These categories are: 1) lexical and linguistic characteristics, %including metrics such as Mean Words per Message, Mean Word Length, and Type-Token Ratio; 
% 2) content and CB indicators, 
% %such as rate of Harmful Messages, Bully Messages, Victim Messages, and Toxicity; 
% 3) sentiment and emotional tone, 
% %which classifies messages into negative, positive, or neutral; 
% and 4) dialogue act distribution.
% %categorizing messages into types such as Question, Statement, Greeting, Accept/Reject, and Other. 
% These categories are critical for understanding the fundamental differences between authentic and synthetic data in the context of CB detection, as they provide insight into how well the synthetic data replicates the linguistic, emotional, and conversational behaviors that are typically present in real-world online interactions.

% To ensure a fair comparison between the authentic and synthetic datasets, we first normalize both dataset by employing pre-processing techniques such as tokenization using NLTK \cite{loper-bird-2002-nltk} and punctuation handling. Additionally, data is segmented into equal-sized token slices to account for metrics that are influenced by corpus size.

% Sentiment scores are measured using VADER \cite{hutto2014vader}, a sentiment analysis tool optimized for short social media texts. Dialogue acts are classified using a Naive Bayes model trained on the NLTK \texttt{nps-chat} corpus,
% following \newcite[Chp.~6, Sec.~2.2]{bird2009natural}.\footnote{
%     While no citation is provided by \newcite{bird2009natural}, the source
%     of this corpus seems to be
%     \newcite{forsyth-martell-2007-lexical,forsyth-etal-2010-nps}.
% }
%



% Natural Language Processing with Python, by Steven Bird, Ewan Klein and Edward Loper
% Chapter 6, section 2.2 "Identifying Dialogue Act Types"
% refers to Chapter 2, section 1.2 "Web and Chat Text", for the
% NPS Chat Corpus but provides no source or citation.
%  
% An unrelated 2011 paper cites an "NPS Chat Corpus of North American English chat
% conversations (Forsyth and Martell 2007)".
%   * Forsyth, Eric. M. and Craig H. Martell (2007), Lexical and discourse analysis
%     of online chat dialog, Proceedings of the First IEEE International Conference
%     on Semantic Computing (ICSC) 2007, pp. 19–26.
%   * data collected in 2006
%   * approximately 500,000 chat posts gathered from various online services
%   * 10,567 posts tagged in Release 1.0
%   * available on http://faculty.nps.edu/cmartell/NPSChat.htm (page no longer
%     exists but is archived, e.g. on
%     http://web.archive.org/web/20190510121556/http://faculty.nps.edu/cmartell/NPSChat.htm
%        - "If you want just the data, you can get it through the Linguistic Data
%          Consortium.  It is catalog number LDC2010T05."
%        - This page asked for the 2007 paper above to be cited "when referring to
%          the NPS Chat Corpus".
%
% There is a 2010 thesis from Naval Postgraduate School, Monterey, California, by
% J. R. Hitt entitled "Implementation and Performance exploration of a cross-genre
% part of speech tagging methodology to determine dialog act tags in the chat
% domain".
%   * credits Lin and Forsyth
%


% Type-Token Ratio (TTR), which is calculated by dividing the number of unique words by the total tokens in fixed-size slices, serves as a normalized measure of vocabulary diversity. Toxicity scores, which represent the ratio of messages containing profanity, are derived using a publicly available profanity list \cite{surge2023profanity}.


\subsection{Evaluation Metrics}

We choose accuracy of label prediction for development decisions and reporting since the labels are reasonably balanced in the authentic test data with 30.3\% items labeled with the minority
label.\footnote{In the appendix, we further report macro average F1 scores that are also widely used in the area of harmful content detection.}
In scenarios 1, 3 and 4,
we train BERT\_base\_uncased \cite{devlin-etal-2019-bert}, a 110M parameter transformer model, with a linear classification head
% using
% the HuggingFace transformers library \cite{wolf-etal-2020-huggingface}
to detect harm, assigning binary labels to text messages.
To address noise from randomness in training, we train at least 45 models for each setting and report average accuracy and standard deviation.
\section{Findings}

The inductive analysis across different robotic artists revealed recurrent factors that contribute to artistic creativity in robotic artwork. Here we present four such facets---\textit{Embodiment and Materiality}, \textit{Malfunction}, \textit{Audience's Reaction and Reception}, and \textit{Process of Creation and Exhibition}. Robotic art is unique in each of them. We argue that these factors are salient in the real-world practices of robotic art---uses of robots in artistic or creation activities. By investigating the practice of robotic art, our study contributes empirically to understanding how computing machines are creatively used for artistic and non-pragmatic purposes. Building upon prior works on artistic input to HCI ~\cite{kang2022electronicists}, we advance the discourse by exploring how artistic practices, values, attitudes, and ways of thinking can serve as resources for HCI practitioners studying or designing for creative activities with machines.

\subsection{Embodiment and Materiality}
\label{f:emb}
Embodiment and materiality are key factors in artistic creativity, shaping the design of robotic artworks. As embodied forms, robots interact with physical space, materials, and humans, matching with human cognition through bodily perception~\cite{davis2012embodied}. Their embodiment encompasses physical appearance, movement, and human interaction, aspects crucial for HCI researchers designing robots to engage with their environment~\cite{marshall2013introduction}. For most of our artists (N=7), understanding robots' material and embodied nature deeply influences their creative process, shaping their thinking and inspiring new ideas. While embodiment imposes physical limitations, it also enhances artistic expression, fostering new styles and aesthetics.

\paragraph{Expressivity From Embodiment}
The embodied property of robots produces an important expressivity and artistic style in robotic art that is challenging to replicate without physical embodiment. For example, David compared drawing by physical robots with drawing in computer programs, concluding that the former is more expressive in an artistic sense because the action of drawing by robots is embodied in the physical world rather than being ``simulated'' in computer programs: ``\textit{I use embodiment (embodied action of drawing by robots)... the drawings work because they do real gestures, it (the drawing) is not simulated. So the drawing has this dynamic feel to it because it is really the movements and the gestures and things... there is a certain speed that it (the embodiment) gives this expressivity to the drawing}.'' The embodied drawing by robots adheres to the physical properties of the material and environmental factors (e.g., pencil, paper, table, robotic arm's degree of freedom, humidity, lighting of the scene), making the drawing process complex, and at times, random and uncontrollable. This complexity introduces more possibilities for artistic expression.

The degree of artistic expressivity depends on which specific materials enable the embodiment of drawing by robots. Interestingly, David claimed that industrial robots, though can draw with high precision, produce less expressive drawings than his self-built robots whose robotic arm's movement is not that precise but more dynamic, flexible, and turbulent:

\begin{quote}
    I don't use industrial robots, because industrial robots are pen plotters. They do exactly what you ask. But they (non-industrial robots) are flexible and... not that precise... when it's drawn, it (the drawing by non-industrial robots) has more expressivity because of the embodiment. The embodiment is very important. It's only because I use those types of arms (self-built robotic arms). It would be far less important if I was using industrial robots.
\end{quote}

He also mentioned explicitly that precise drawing is not artistic: ``\textit{But anyway, that (precise drawing robot) is the technology. And it works very nicely, but I couldn't find it artistic. I was actually disappointed when I got it to work.}'' Similarly, Sophie noted that plotting/printing robots create different drawings than painting robots do: ``\textit{I wanted it (the artwork) to be painted and I didn't want it for the visuality of it or the behavior of it. I didn't want it to be plotted or printed, [it] feels different [and] has a different existence.}''

Although both industrial robots and self-built robots draw in embodied ways, the results can appear either precise or dynamic, depending on how the robots are built and programmed---in other words, how the artists configure the material aspects of robots to realize the embodiment. In practice, our robotic artists need to think about ways of utilizing embodiment and properties of robots and all other involved materials to be artistically expressive, to be creative.

\paragraph{Inspiration From Embodiment}
We found that the embodied nature of robotic art often becomes a source of inspiration for new artistic ideas. Linda, an artist-engineer who conducts scholarly research at the intersection of robotics and dance, reflected on how interacting with embodied robots makes her think about the differences between human and robotic bodies:

\begin{quote}
    I've never felt more human. You just feel, you notice, oh, I can fall here and I can get right back up, but it (the robot) falls and it can't get right back up. Or how soft am I? How wet? Like, (patted her face) there's so much water content and squish when I lay on the floor. And it [the robot] doesn't have that... That generates new ideas and helps me be creative.
\end{quote}

She also explained how robotic bodies allow her to examine human movements: ``\textit{The robot is doing something that I can't do on my own body---pure right (her arm was moving toward her right), and [then]... [I] can look at my messy right [movement] next to its [robot's] pure right... that's creative, that's energizing to me to see and play with movement profiles with such a pure tool for decomposing the elements of it, making me notice them}.'' She also shared an anecdote that building a special robot with high degrees of freedom inspired her to explore the differences between human and robotic bodies, enabling her to see new things for her art projects.

\paragraph{Creativity From Embodiment}
Our artists emphasized the embodied nature of creativity and intelligence in general based on their artistic practices, asserting that creativity is inherently embodied rather than disembodied, symbolic, spiritual, and something only happens in the human head. For our robotic artists, creativity is built upon understanding embodied entities in the environment rather than abstract concepts in the mind. Samuel used ChatGPT as an example to argue how the disembodied way of communication between humans and machines limits creative interactions:

\begin{quote}
  I think most of the creativity is coming from non-verbal information flow. So when we are discussing with ChatGPT only through text... the creativity that we can experience is so limited because we do have to sit in front of ChatGPT and we cannot move around or ChatGPT is not going to move around. So our conversation is... very limited... that missing embodiment... is also missing creativity in the conversation with ChatGPT.
\end{quote}

The design of ChatGPT aligns with the mainstream approach to disembodied chatbots running as computer programs. In these designs, symbolic content (e.g., text, images, videos, audio) serves as the communicative medium, but bodily interaction is minimized (i.e., users primarily sit and type). While creativity is arguably rooted in embodied interaction with other material bodies, current interactive agents (e.g., Copilot and Midjourney) designed to support creative work remain largely symbolic and disembodied. Limiting human-machine communication to symbolic channels may lose the benefits of embodiment in acquiring creativity.

According to our artists, one reason for the lack of attention to the embodied dimension of creativity is the historical dichotomy between mind and body, which categorizes creativity as something in the mind:

\begin{quote}
     It (the idea that creativity is disembodied) was very much driven by a view that you can split the body and the mind, and intelligence is happening in a symbolic way, mainly in the brain... [This] led to a large focus on software applications and delayed focus on robotic hardware improvements. And still today, you can see the split of hardware and software... [F]or a lot of organic entities, the integration of bodily capacities with their environment could be seen as more intelligent than the representational capacities... [A]s an artist, I am trained to work with bodies interacting with environments or with other bodies, also this fluent transition from bodily action to semantic questioning. (Daniel)
\end{quote}

This dichotomy, which may have formulated the engineering of computing systems, is rarely compatible with the artist's view that intelligence and creativity can be more richly manifested through bodily interaction and relationships.

Embodiment has been an essential prerequisite of creativity for some artists since their creative production requires understanding embodied entities. To summarize this subsection, embodiment is an important source of creativity for robotic artists. Practically, it yields new artistic expressions and aesthetics whose complexity is difficult to replicate by computer programs. The embodied form of robots, in turn,  inspires creative ideas for artworks. These ideas can arise from understanding the entities embodied in the physical world, whether robots, humans, or other bodies in the environment. The symbolic and disembodied modes of interaction between human creators and machines in creative activities can be complemented and strengthened by embodied interaction.

\subsection{Malfunction: ``Ghost in the Machine''}
Robots, encompassing both mechanical and digital devices, are inherently susceptible to malfunction, with physical robots being more prone to errors, glitches, and noise than virtual agents. These malfunctions are widespread in robotics. In robotic art, such errors hold unique significance, influencing the interpretation and value of the art. Unlike engineers, who aim to fix errors, robotic artists often embrace malfunctions as part of their creative process (N=7).

\paragraph{Embracing Errors and Uncertainty}
Evelyn views machine errors not as obstacles, but as opportunities for unique artistic expression. She embraces the imperfections that arise from machine errors, seeing them as a way to humanize the machine and its output:

\begin{quote}
     I embrace these errors. For me, it is the way to show that using the machine in a way that's very counter-intuitive... celebrating that error instead of trying to perfect it, or slowing down the machine instead of trying to create commodities as fast as we can... what's interesting with the machine [is] to actually turn it upside down and think that the machine is a bit like a human child, and everything it does actually slow, it's imperfect, it's full of mistakes.
\end{quote}

Evelyn’s approach challenges the conventional expectation of machines as flawless and efficient executors. By slowing down the machine and celebrating its errors, she imbues the machine with a human-like quality of imperfection. This perspective turns the machine into something capable of growth and learning, much like a human child. The errors, therefore, potentially become a source of uniqueness and individuality in the artwork, adding depth and complexity to the artistic expression. This ``counterintuitive'' way of viewing error resonates with Alex who contrasts this view with the engineering tendency that strives to be neat, rational, and organized through monitoring and fixing errors: ``\textit{[S]ometimes, it's (error/glitch is) like a source of treasure. Like you find something that you could work on, you find something that people don't really use... But when we are tinkering, we sometimes reach this point of, `ah, okay, now this is visible.'... I think sometimes even just those things (errors/glitches) could be a work of [art]}.''

Alex and others see robotic malfunctions as opportunities to imagine alternative approaches and values. When robotic systems' behaviors deviate from their programs, they often refuse to ``fix'' the unexpected behaviors, instead, they allow the unexpected to unfold as serendipitous events that can inspire new design features. Preserving malfunctions allows the artists to think about the artistic potential of something derailing from the initial plan and make informed adjustments accordingly. These values would not be examined, integrated, or utilized to contribute to creativity if the immediate response to malfunctions was negation and subjecting the malfunctions as inferior to the planned behaviors. As Sophie noted, artistic practices are inherently unpredictable and shaped by the contingencies of the creation process.

Many of our artists described how they perceive, evaluate, and appreciate the unexpectedness of robotic art, revealing new artistic ideas that would not have emerged otherwise: \textit{``[I]nstead of an ink particle, you had a hole in the form of that part... I was like, `Oh, we'll see that the material is saturated, I will not push it (brush).' But the robot doesn't have this understanding and pushes it. And I thought, `Oh, it's actually a good outcome. It's actually both conceptually and aesthetically very pleasing to me'}'' (Sophie).

In this case, the robot performed an action that a human artist normally would not perform---pushing the brush on the canvas. The robot made an unusual decision and breakthrough in expression, called by many artists as ``surprise.'' Once the artist recognizes its artistic value, it may be further explored and developed. Linda shared a similar anecdote where an unexpected jitter from the way the motor pulls a string gives a ``texture'' to the robotic movement, which she sees as creative.

\paragraph{Incorporating Malfunctions as Intended Design}
Our artists deliberately incorporate errors into their artworks. It demonstrates how valuing malfunctions and the unexpected can directly contribute to the work's artistic creativity. Linda articulated the idea that humans are capable not only of learning from mistakes but also of intentionally leveraging these errors to their advantage, echoing insights from our other artists: \textit{``Glitches are 100\% part of the creative and artistic process... It's undeniable that we recover better from mistakes [than computers do], but I think it's more than that. We actually can incorporate mistakes and make them part of an intended design.''}

By making malfunctions part of the intended design, the artists engage with and utilize them to enhance artistic expression or similar ends. Choosing not to fix these issues offers the artists alternatives to designing and realizing their robotic art. For example, David recounted an anecdote about a bug---a flaw in a computer program's software or hardware---that unexpectedly made a line drawing ``beautiful.'' Rather than fixing the bug, he decided to make it an optional feature, allowing him to switch it on or off:

\begin{quote}
    Generally I don't take care of them (glitches). So there are those glitches that give this unpredictable because I like to have drawings that are not predictable... I fix it (bug) and then I have the possibility of using or not using the bug... I'm always surprised by the output... it (bug) creates a surprise for the spectator who is looking at the robot drawing... I just left it (bug) and it's still there. Sometimes I switch it (bug) on, sometimes I switch it off.
\end{quote}

If David had fixed the bug, without retaining it in the program, he would not have possessed such a feature of expression. This shows how differently robotic artists handle technical malfunctions than typical engineers or roboticists. Malfunctions should be avoided in engineering but may yield creative outcomes for robotic art. This is not to claim that the creative value is innate within malfunctions. As our findings have shown, malfunctions are raw materials that can be deliberately utilized by the artists to achieve creativity. When malfunctions are not desirable in art, they may primarily be engineering challenges, as the following examples illustrate.

\paragraph{Avoiding Malfunctions}
The fragility of robots is a widely shared concern among our robotic artists (N=8). Regardless of their origin---self-built, modified, or off-the-shelf---all robots are susceptible to breakdowns in real-world environments, particularly during extended exhibitions without proper maintenance by artists or qualified personnel. For exhibitions, malfunctions are generally unacceptable, and robots ought to \textit{perform flawlessly} when showcasing to the audience. To address malfunctions, the artists came up with different strategies, such as having backup materials for replacement and assembling the robots on-site at the exhibition. One strategy is reducing the complexity of the robotic system, simplifying it to minimize the risk of failure or loss of control. Their approach involves designing robots that resist internal breakdowns and withstand external environmental factors, such as moisture and gravity. Linda explained, ``\textit{If I do build them (robots), I try to keep them simple and I try to make something that will withstand its environment... Sometimes that might be outdoors next to the ocean for six days}.'' David further emphasized that the concern for fragility leads to the need for simplicity in robotic design:

\begin{quote}
    [I]f you're used to do programs that are disembodied, that are only on the computer, you can do very complex things. But as soon as you work with robots, you have to simplify everything... They exist in the same physical world [as us]. Dynamic, the speed, the time, the weight of thing are the same for us. So there are all those limits, which [requires] you to simplify a lot of the programming.
\end{quote}

Mitigating malfunctions and recognizing their artistic potential are not mutually exclusive. Designs that address engineering malfunctions can also yield artistic qualities. As illustrated in the findings, utilizing and mitigating malfunctions occur at different phases of artistic practice. In the production phase within the studios, artists often regard malfunctions not as impediments but as sources of inspiration. By celebrating serendipitous errors and the unexpected, they deliberately integrate these elements into their robotic creations, pushing material and expressive boundaries. In this phase, the primary interaction happens at the individual level---between the artist and the robot(s). In contrast, within exhibition spaces like museums, malfunctions conflict with the expectation that the robots should function flawlessly, risking being disqualified from display. Here, the interaction shifts to a social context, where artists must negotiate with curators and audiences on how to present the robot. This transition from studio to exhibition thus signifies an important change in, context, practice, and actors involved. Hence, next, we highlight the significance of audience reaction and reception that shape creative outcomes.

\subsection{Audience's Reaction and Reception}

\begin{quote}
    I suppose [that] every project I do is a collaboration between me, the machine, and the interactant to some extent. --- Robert
\end{quote}

The artistic and creative value in robotic artwork is determined not just by the work itself but often by the audience’s reactions and interpretations. Our artists (N=7) mentioned that they observe or think about audience reaction, and often incorporate them into subsequent iterations of their work. Alex, for instance, is motivated in the first place by observing how people react to robots, drawing inspiration from their perceptions.

\paragraph{Audience Reaction Shapes Robotic Design}
One of the most direct ways audiences influence the practice of robotic art is through the artists, even when it is unintentional. For instance, after observing that some audience members interact with his robots by squeezing two springs on the robot together---causing a short circuit---Robert decided to revise the material design of the robots to prevent such accidents: ``\textit{I knew darn well that the children were going to squeeze the springs together. So I was very excited to find that even if they did that, I put a kind of a self-healing fuse, polycrystalline that will heal itself... it was an important component of the design}.''

Robert’s response highlights the importance of audience reaction, which he observes and integrates into his robot designs. While in this case, the reaction led to the resolution of a technical issue rather than adding an artistic element,  Alex's experience illustrates how audience interaction can inspire new aesthetics in his work. Alex described how he adapted the environment around his robots based on the audience’s tendency to project personalities onto them:

\begin{quote}
    People project something like animals or themselves or something [on the robots]. And then I got inspiration from that. Then I made a little brighter setup with some objects, a little bit like forest kind of setup. And then people try to imagine more stories. And then I also put some effect to [make the setting] looks like night or daylight or morning. Then people really see [the robots] differently.
\end{quote}

These examples demonstrate how the audience's explicit and implicit feedback (action, projection, and imagination) influences artists’ decisions in designing robots. Audiences are not passive recipients of the artists’ creations; rather, they become part of a collective creative process, leaving their mark on the final work.

\paragraph{Audience Reaction Shapes Robotic Performance}
Linda described how she designed a robotic component for ``\textit{onstage performer[s] as well as audience members to come and interact with [the] robot in a creative way},'' emphasizing the importance of creating a space for audience interaction. Robert further suggested that these interactions during the exhibition possess performative features, which he views as an artwork: ``\textit{I would consider the final product (the drawing by his autonomous robots) as the art. And I would also consider the [audience's] experience of watching them (the robots) paint also as a kind of performative artwork}.'' Robert views robots not as static objects but as responsive entities capable of meaningful interactions with both their environment and the audience. He views robots as possessing ``emergent agency'':

\begin{quote}
    I think that's an agency I would call emergent agency, which is to say that the system software, the physical structure itself in relation to the viewer, interactant creates a kind of emergent behavior where the robot is, and it's designed to some extent to react or respond either with sound or motion in some way to the viewer. And by doing so, it then allows the viewer to see that response, which then reprograms the viewer's response to that. So there's almost a kind of feedback loop that I find happens a lot with robotic art.
\end{quote}

Daniel mentioned a similar idea in the context of live dance performance. The performance benefits from incorporating ``real-time learning interactive systems'' because that makes the performance not solely predefined but ``\textit{[emerged] in the moment of interaction which was not there before [the performance].}'' Without the audience serving as the stimulus, interactive robots in exhibitions would not be perceived as they were. In other words, robots react to the audience, which casts changes in the audience's perception, then robots sense the changes and react again, forming a continuous feedback loop or improvisation between the robots and the audience.

\paragraph{Open Interpretations Make Robotic Art}
Artwork that remains open, undetermined, complex, and vague often invites diverse interpretations~\cite{eco1989open}. The same applies to interaction design where systems may not have a single user interpretation~\cite{sengers2006staying}. Samuel built three humanoid robots with different levels of functionality. The third robot, though technologically more advanced, received less curiosity from the audience than the first, more rudimentary robot:

\begin{quote}
    [For the third-gen robot],... people immediately understand what he (the robot) is doing. So people just leave after five minutes. But [for] the first one (first-gen robot), people tend to spend like 20 [or] 30 minutes because people don't understand what he's doing. But now it [the third-gen robot] is interpretable, so I understand that... giving him too much meaning is dangerous, [when] work[ing] on an art stuff, because people get tired... people are used to those things (technological functions), which [have] tons of meaning [about] what the machine is doing.
\end{quote}

Here, incorporating technical functionalities into the robot assigns clear objectives easily grasped by the audience, making the perceived meanings more rigid and restricting the scope for diversified interpretation.

Beyond the individual level, the way of interpretation is also socially shaped. Samuel made the point that the perception of creativity is also partly a social product because ``\textit{creativity is depending on what kind of society we are in and what kind of people we are interacting with}.'' Mark and Robert extended that the perception of robotic art is culturally conditioned, varying across different societies and generations. They mentioned how the animist cultural tendency of some East Asian societies potentially makes people more willing to accept and interested in robots and non-human entities (e.g., plants and animals) behaving as if intelligent and agentic. The way that the social context of interpretation and perception determines artistic values reiterates our claim that the audience's reception of robotic artwork is one of the key aspects of robotic art practice. It suggests that in achieving certain artistic goals by robotic art, considering the audience's background and ``horizon of expectations''~\cite{jauss1982toward}---the socially and historically conditioned structure by which a person comprehends, interprets, and appraises any text based on cultural codes and lived experiences---may be constructive in refining the work's idea.

\subsection{Process of Creation and Exhibition}

Many of the artists we interviewed (N=6) emphasized, or alluded to, the artistic value in the \textit{process} of making robotic art. Specifically, two types of processes are discussed here---the process of \textit{creation} and the process of \textit{exhibition}, reflecting two salient temporal stages of robotic art practice. We do not, by any means, suggest that process is unique to robotic art; apparently, other forms of art also attend to processes of their art practice. Our intention has been to situate the analysis of process in the emerging, particular context of robotic art and to reveal how process leads to a new understanding of robot's uses and roles in real-world scenarios.

Sophie builds robotic systems capable of physically painting on canvas. She uses these robots to explore the painting process itself rather than to focus on the final product—what she referred to as images instead of paintings. Her case exemplifies that the act of making can become the focal point of artistic interest. In her view, paintings as artifacts are space-and-time bound ``material-based work'' that requires ``interactive practice'' and ``decision making,'' whereas the resulting images are ``merely digital representation[s]'' of this process. The difference between images generated by computer programs and paintings created through human touch underscores her rationale for utilizing robots: to bring the tactile, material process of painting to the forefront.

\begin{quote}
   [I]n the end, if I'm trying to crop everything (all my ideas) together, then it (the commonality) is to make the temporality of the decision making process of painting more visible and present. So I'm not really interested in how the image looks. And we experience an object that actually has a temporal element, how it's been created with layers, with tons of decision making... because I am interested in painting as a process and less [as] a product, I'm trying to use the process of making a painting to reflect a lot of our human creativity, our relationship to machines, questions of agencies, and so on.
\end{quote}

She has been building robotic systems that have ``adaptive behavior[s]'' during the painting process, where the systems are designed to ``\textit{analyze a stroke [on the canvas] and then create a successive one}.'' This design ensures that robots' actions are not exclusively dictated by the pre-programmed instructions but also influenced by the constantly changing ``state of the world,'' which includes factors such as the evolving canvas, environmental conditions, and the interaction between the robot and its surroundings. Consequently, a painting is not just a visual product but represents a series of actions with a temporal dimension.

Another important process for robotic art practice is exhibition. In the exhibition space, robotic artworks often take the forms of performances or improvisations, actively interacting and potentially shaping their environment in real time. 
For example, Alex's robots paint spontaneous color patterns on canvas during the exhibition, transforming the event into a performative art experience that aligns with his intention of foregrounding the painting process. The dynamic nature of live drawing at the exhibition---``making a show live''---has been central to Alex's artistic approach.
Moreover, new qualities of robotic artwork not only emerge by interacting with other entities, such as viewers or environmental factors, but also through the artwork itself as it develops over time. Daniel recounted an instance where a crack in his robotic installation continued to expand, gradually altering the artwork throughout the exhibition:

\begin{quote}
    I used [a] dome as a costume of the robot, and it (the robot) was an interactive real-time installation. The foam [on the dome] got a crack, and I decided to keep it cracking throughout the exhibition for one week. The crack in the costume was tearing down and it created a different artistic situation I could not have planned. It was so strong that it changed the whole work... I want to be sensible to those moments and see them as part of the process... I don't see that (situation) as, `okay, that is now destroying my artwork.' No, it is evolving or creating a new one within.
\end{quote}

This case illustrates how robotic artwork is not fixed but remains malleable even during the exhibition stage; temporal changes within the artwork can introduce new artistic qualities that evolve the work beyond its original design. Highlighting the artwork's temporality here allows for elucidating how the current state of the created artifact and creativity come to be. The practice of robotic art thus extends beyond the creation stage, encompassing the exhibition period. While in many cases the creation process is well planned, and temporal changes during the exhibition are typically unforeseen, both processes reveal that robotic art is in a state of ongoing creation across time. By paying attention to these processes, we unravel the temporal dimension that contributes to the creative values in robotic art.

In this Findings section, we have highlighted four aspects of robotic art practice that contribute to the artistic quality of the work or to achieving some artistic goals. The analysis reveals how various actors—artists, robots, audiences, and environments---are involved in the practice, influencing one another. These interactive patterns explain how creativity in robotic art is distributed within and emerges from the relations of actors. This idea echos with Daniel's reflection, as he noted that he sees robotic artwork as \textit{``a product of a situation of a creative potential that is part of the environment, all the entities involved as well as me,''} emphasizing the distributed and emergent nature of creativity in robotic art.

\section{Discussion}
\subsection{Revisiting the Explicit-Implicit Dichotomy Through Intention}
% 1. Why our findings of the ``intentional implicit feedback'' are important. We can talk about the benefits: identifying the intentional implicit feedback can (1) help with users' understanding and control of the algo and achieve more desirable personal recommendation, (2) reduce user unintended or unexpected outcome (false positive error), and (3) free up users' cognition for other tasks. 
% We can also talk about the problems of the explicit–implicit dichotomy...

The explicit-implicit feedback dichotomy was largely framed from the system's perspective, as Jannach et al. framed it---implicit feedback referred to interactions from which we (the system side) can indirectly infer user preferences ~\cite{jannach2018recommending}. The implicit feedback was supposed to be user's natural interactions with the platform. However, from the user’s perspective, they are not only aware of, but also intentionally, proactively act as feedback inputs to influence the recommendation feeds. For instance, in our findings, when participants ``dwell on'' or ``swipe past'' a post, they might consider it as an explicit expression of their preferences. Yet, the system might take it as implicit feedback and fail to understand or capture it. This highlights a fundamental challenge: feedback categorized as implicit from the system's point of view may be deliberately used as explicit by users.

This misalignment between how users and the system perceive feedback can lead to inaccuracies or failures in capturing user intent. A specific problem is that implicit feedback is often only interpreted as positive feedback~\cite{jawaheer2014modeling,hu2008collaborative}, whereas our findings revealed that users also intentionally employed implicit feedback behaviors to communicate negative feedback, such as ignoring or swiping past a post ($n = 21$), stop using the platform or switch platforms ($n = 12$), and refresh the feed ($n = 2$). Neglecting on interpreting negative implicit feedback could distort the user profile~\cite{hu2008collaborative}. Thus, incorporating both positive and negative implicit feedback is essential for more accurate recommendation. Another related problem is that the platform' focus on implicit positive feedback may lead to instances of false positive interpretation. For instance, several participants mentioned they searched for information as a one-time task, but the platform over interpreted this as a preference, resulting in their feed being flooded with similar content that was no longer relevant or needed. This suggests that implicit feedback should be considered through more nuanced distinctions, such as positive versus negative, intentional versus incidental~\cite{dix2002beyond}, and reactive versus proactive behaviors~\cite{ju2008range}. Particularly, defining explicit–implicit distinction through intention can add greater precision to the interpretation of user feedback~\cite{serim2019explicating}. 
% implicit feedback has traditionally been treated as positive, and the lack of attention to negative signals can lead to misrepresenting user preferences ~\cite{hu2008collaborative}. However, our participants frequently used implicit feedback to express negative signals, with actions like ignoring or swiping past posts being the most common.

%Therefore, the addition of intentional implicit feedback to the implicit-explicit dichotomy in our paper can help platforms better distill users' intention, addressing the above issues to improve the accuracy in predicting user preferences, achieving more desired personalized recommendations.Comparing to unintentional implicit feedback, 
In addition, acknowledging and responding to the intentional implicit feedback can afford users more understanding of the algorithm’s learning process. Users will feel that their interactions are shaping the recommendation feeds, thus foster a greater sense of self-causality and increase user agency~\cite{feng2024mapping}. Comparing to explicit feedback, intentional implicit feedback also frees up users’ cognitive resources. Research has shown that explicit feedback often imposes a higher cognitive load, as users must actively engage with the system ~\cite{kelly2003implicit,gadanho2007addressing}. In contrast, allowing users to guide recommendations through unobtrusive, yet intentional, behaviors can make the interaction more seamless and less mentally taxing. This aligns with findings from previous studies that highlight the importance of minimizing user effort in feedback mechanisms.

\subsection{Aligning Feedback Types with Users' Purposes}
% Discuss why users chose to use different feedback types for different purposes. The nature and characteristics of the feedbacks.
The study identified four purposes in participants' using of personalized recommendation platforms: content consumption, directed information seeking, content creation and promotion, and feed customization. The first three purposes echo previous literature, which has discussed directed and undirected consumption~\cite{feng2024mapping}, as well as self-presentation ~\cite{devito2017algorithms}. We identified feed customization as a purpose that arises when users are dissatisfied with the recommendation contents, which encompasses four specific goals. 

Users choose different feedback types based on their purposes. Explicit feedback (75.9\% instances) is primarily used for feed customization. As shown in~\autoref{tab:feedbackType}, all explicit feedback behaviors were supported by platform features, providing cues to helps users understand the potential outcomes of their actions. Such clarity makes explicit feedback the preferred choice when users seek quick corrections to their recommendation feeds. For instance, our findings indicated that most users opted for explicit feedback, particularly negative explicit feedback, such as marking content as ``Not interested,'' blocking, or reporting, when they had a strong desire to remove inappropriate or irrelevant content from their feeds. Previous studies found that users were reluctant to provide explicit feedback in complex tasks like online shopping but were more willing to do so in media and entertainment domains~\cite{white2005study,jawaheer2014modeling}.
This aligns with our findings that, on personalized recommendation platforms, users did not show a strong preference between explicit (31 instances) and implicit feedback (44 instances) during feed customization.

On the other hand, intentional implicit feedback (88.6\%) was also mostly used for feed customization, but often when users' intent is not particularly strong or urgent---either to increase content diversity or improve recommendation relevance. Intentional implicit feedback allows users to steer the algorithm towards their evolving preferences without overtly signaling their intent or interrupting the flow of content consumption. The choice among various implicit feedback behaviors is usually based on their continuously updated folk theories~\cite{devito2017algorithms}. Since users are not fully certain how the algorithm will interpret their actions ~\cite{eslami2015always}, they tend to experiment and make repeated attempts when necessary. Despite these efforts, our participants sometimes observed minimal or delayed changes in the recommendation feeds, due to the less obvious and plausible nature of implicit feedback~\cite{jannach2018recommending}. This then may then lead to more passive behaviors like stop using the platforms.

Moreover, the design and functionality of the platforms~\cite{liu2010personalized}, such as Douyin’s swipe interaction versus Xiaohongshu’s click-and-select feed layout, also mediate users' specific feedback behaviors. Participants on Xiaohongshu, for example, found selecting and clicking more efficient for improving recommendation relevance, comparing to using the swipe function in Douyin. Therefore, platform design should support a wider range of intentional implicit feedback by providing users with clearer traits of how their actions influence recommendation feeds and more intuitive ways to interact with the platform.

\subsection{Design Implications}
% 1. supporting intentional implicit feedback: (1) provide features to support or prompts, but also remain unobtrusive for natural flow. (2) support negative feedback.Most used implicit and explicit are both negative feedback.
% (3)  The need for users to repeatedly mark content as ``Not interested'' suggests that the algorithm may not be efficiently learning from initial feedback. Enhancing the algorithm's ability to quickly adapt to user preferences after receiving negative feedback could improve the user experience by reducing the frequency of unwanted recommendations. but also not too sensitive to reduce false positive 
% 2. Incorporate user goals. such as balance between relevance and diversity
% 3. transparency in data collected and protect privacy 
The discussion leads to the following design implications for the personalized recommendation platforms to effectively support and respond to user feedback driven by various purposes.

\subsubsection{Supporting Intentional Implicit Feedback Through Recognition and Visualization} 
First, platforms need to support intentional implicit feedback. When users intentionally provide implicit feedback, they expect the algorithm to infer their interest from their ``natural interactions''~\cite{kelly2003implicit} and respond seamlessly. This differs from explicit feedback, which is given through direct user input that interrupts users' natural flow. However, when the subtle feedback goes unnoticed or yields little immediate effect, users may feel a loss of control. Therefore, while most implicit feedback is not supported by platform features (shown in~\autoref{tab:feedbackType}), it's important to inform users that their feedback has been recognized and will influence future recommendation contents. One potential solution is to implement non-intrusive visual cues, such as small icons or pop-ups, that confirm feedback has been registered. For instance, after a user click similar posts several times or search for a topic, a pop-up could confirm that the platform has recognized this interest and will increase similar content recommendations. Another option is to provide progress indicators that allow users to see how their behavior is affecting the feeds over time. This could include a dashboard showing how many times they had skipped certain content and how it has adjusted their recommendation feeds. This can increase transparency and help users understand how the algorithm is learning from their feedback and responding accordingly. 

Additionally, platforms need to improve their handling of negative implicit feedback. The challenge with interpreting negative implicit feedback lies in its subtlety. Unlike explicit feedback like marking as not interested, implicit negative signals are often less direct and harder to capture. To address this issue, platforms need to adopt more sophisticated approaches to detect and validate implicit negative feedback. While some platforms have already implemented ``show less'' features that allow users to indicate their content preferences (such as ~\cite{meta_new_2022}). These features are typically static and presented with all content. What's needed is a more dynamic approach that can capture and respond to users' behavioral patterns in real-time. One potential strategy is explicit user confirmation. For example, when a user repeatedly swipes past posts or consistently ignores content, the system could proactively detect that the user is not interested in this topic and prompt the user with an option like, ``Want to see less of this content?'' to confirm their intent. Such dynamic approaches can capture users' intentional implicit feedback behaviors more effectively and would not cause a significant disruption to users' normal usage.
% acknolwege current available design(https://ai.meta.com/blog/facebook-feed-improvements-ai-show-more-less/#:~:text=The%20challenge%20with%20training%20these,they%20would%20like%20to%20see)

\subsubsection{Designing for Purpose-Oriented Feedback}
Second, we suggest designing for purpose-oriented feedback. Our findings revealed close association between users' purposes and their choice of feedback types. This underscores the importance of incorporating users' diverse and evolving purposes in the design of personalized recommendation platforms in addition to focusing on their behaviors~\cite{liang2023enabling, nazari2022choice}. To address this, platforms can integrate context-aware learning models to infer user purposes. For example, during directed information seeking, algorithm can prioritize relevance and rely more on implicit feedback~\cite{white2005study}; in content consumption, the algorithm can also consider content diversity, as many participants were concerned with information cocoon~\cite{li2022exploratory}. Meanwhile, the platforms can also provide customization options, such as a slider, for users to adjust the balance between relevance and diversity in their recommendation contents. It should be mention that these setting options, such as setting up interested channels, should be made more discoverable, as some participants did not notice the feature~\cite{liu2024train}.

\subsubsection{Transparency in Feedback Data Collection and Protecting Privacy}
Quite a few participants expressed concerns about their privacy, noting that platforms seemed to be monitoring their conversations, chats with friends, and even activities on other platforms. Despite these concerns, only two participants took active steps to protect their privacy by disabling personalization. However, after noticing that the recommendation quality significantly deteriorated, they eventually re-enabled it. This suggests that users have no effective means to combat platforms' invasion of privacy. To address these concerns, platforms should be transparent about the types of data being collected, including behaviors within the platform, cross-platform tracking, and even offline monitoring. Platforms should allow users to customize their privacy settings, giving them the ability to select which types of data they are comfortable sharing for algorithm personalization. This could include options to opt out of specific data types, while still maintaining a personalized recommendation system.

Interestingly, as we mentioned in the results, one participant mentioned deliberately speaking aloud to trigger the platform's monitoring and influence recommendation feeds. This behavior highlights a potential design opportunity: platforms could introduce voice-based feedback to explicitly capture users' preferences, rather than continuously monitoring their voices. By offering users the ability to verbally express their content needs, platforms could create a more transparent and interactive method for users to directly influence recommendation feeds while respecting their privacy.


% 4. What will be the challenges of designing for ``intentional implicit feedback''? For example, ``constructive validity''~\cite{serim2019explicating}---the extent about how intentional implicit feedback translates into design in our case. 
% Other challenges can be users' literacy to use these designs, users' fluid needs, and delayed system reactions. According to Serim and Jacucci~\cite{serim2019explicating}, users' attitudes can be fluid and influenced by situational factors, which poses uncertainties to determining implicitness.
\subsection{Limitations and Future Work}
This study has several limitations that should be noted. First, the recruitment method relied on pre-screening surveys and snowball sampling. This might have introduced selection bias, favoring more engaged platform users and educated users. Second, the feedback behaviors reported by users were neither comprehensive nor reflective of the ground truth. Specifically, the list of unintentional implicit feedback behaviors in ~\autoref{appendix:unintentional} includes only interactions mentioned by users and categorized as implicit feedback in previous literature~\cite{jannach2018recommending}. We do not have access to the complete set of behaviors that platforms use as implicit feedback. 

% \subsection{Future Work}
% 研究什么样的人会用intentional里的所有behavior,和他们的demo characteristics,以及为何选择specific behavior
Our findings shed light on how users leveraged different feedback mechanisms to fulfill their purposes on personalized recommendation platforms, but further investigation is needed to validate and expand upon these insights. One potential future work is to conduct a large-scale survey to confirm the feedback behaviors and patterns identified in our study. Although our interviews touched upon users' motivations for selecting different types of feedback in various contexts and purpose-oriented situations, a survey study would allow us to explore users' rationales when choosing different feedback types in different contexts. Another important direction for future research is to examine how different types of feedback—explicit feedback, unintentional implicit feedback, and intentional implicit feedback—impact user experience and satisfaction with recommendation systems. This could include how each feedback type influences content quality, user engagement, and perceived control over the recommendation system. Additionally, future research could examine the relationships between users' feedback choices and factors such as demographics and algorithm literacy. Understanding how these factors influence users' engagement with feedback mechanisms could support the design of more personalized and effective feedback mechanisms.
\section{Conclusion}
We introduced \methodname, an effective training framework defending against MIAs for LLMs. The extensive experiments demonstrate its robustness in protecting privacy while maintaining strong language modeling performance across various datasets and architectures. Although our study focuses on fine-tuning due to computational constraints, \methodname can be seamlessly applied to large-scale pretraining, as done in prior selective pretraining work~\cite{lin2024not}. By categorizing tokens and treating them appropriately, \methodname opens a novel pathway for MIA defense. Future work can explore improved token selection strategies and multi-objective training approaches.


%%
%% The acknowledgments section is defined using the "acks" environment
%% (and NOT an unnumbered section). This ensures the proper
%% identification of the section in the article metadata, and the
%% consistent spelling of the heading.
\begin{acks}
We thank the participants for sharing their experiences and insights. We would also like to express our appreciation to Zitong Huang and Yaqi Zhang for their assistance in collecting and analyzing the interview data. Additionally, we extend our thanks to Lu Xian, Zefeng Ben Zhang, and all reviewers for their thoughtful feedback, which greatly contributed to refining this work. This research was supported by the NSFC Grant No.72174014 and funding from the Inequality in America Initiative at Harvard University. 
\end{acks}

%%
%% The next two lines define the bibliography style to be used, and
%% the bibliography file.
\balance
\bibliographystyle{ACM-Reference-Format}
\bibliography{0-references, references_poster}


%%
%% If your work has an appendix, this is the place to put it.
\newpage
\section{Appendix}
\label{appendix}

\subsection{Survey Questions}
\label{app:survey}

\subsubsection{Scenarios}

Participants were asked about three classes of hiring scenarios: technical coding assessments, resume review, and behavioral interviews (the scenarios are listed by class below). For each scenario, they answered two questions, both on 5-point Likert scales:
\begin{itemize}
    \item How fair does this hiring process seem to you? (``This hiring process seems fair'', 1: Strongly disagree to 5: Strongly agree)
    \item If you were applying for a technology job, would you want to be evaluated this way? (``I want to be evaluated this way'', 1: Strongly disagree to 5: Strongly agree)
\end{itemize}

[Technical Coding Assessments]
\begin{enumerate}
\item An applicant submits a sample of code, which is reviewed by a recruitment team, who determines whether the applicant advances to the next phase.
\item An applicant is given an online coding assessment, which is evaluated by an algorithm. If the applicant reaches a certain score on the autograder, the applicant advances to the next phase. All algorithmic decisions are reviewed by a recruitment team.
\item An applicant is given an online coding assessment, which is evaluated by an algorithm. If the algorithm rejects the applicant, the decision is reviewed by a recruitment team. 
\item An applicant is given an online coding assessment, which is evaluated by an algorithm. If the algorithm advances the applicant to the next phase, the decision is reviewed by a recruitment team. 
\item An applicant is given an online coding assessment, which is evaluated by an algorithm that determines whether an applicant advances to the next phase. 
% \item Why did you select the answers above for the different scenarios related to coding assessments?
\end{enumerate}

[Resume Review]
\begin{enumerate}
\item An applicant submits a resume, which is reviewed by a recruitment team, who determines whether the applicant advances to the next phase.
\item An applicant submits a resume, which is evaluated by an algorithm. The algorithm determines whether the applicant advances to the next phase. All algorithmic decisions are reviewed by a recruitment team. 
\item An applicant submits a resume, which is evaluated by an algorithm. If the algorithm rejects your application, the decision is reviewed by a recruitment team. 
\item An applicant submits a resume, which is evaluated by an algorithm. If the algorithm advances the applicant to the next phase, the decision is reviewed by a recruitment team. 
\item An applicant submits a resume, which is evaluated by an algorithm that determines whether an applicant advances to the next phase. 
% \item Why did you select the answers above for the different scenarios related to resumes?
\end{enumerate}

[Behavioral Interviews]
\begin{enumerate}
\item An applicant has an interview with a member of the recruitment team. The recruitment team determines whether the applicant advances to the next phase.
\item An applicant participates in an automated video interview, where the applicant receives interview questions and records video responses. The video, including the applicant’s speech (fluency, prosody, pronunciation, language usage) and nonverbal behaviors (facial expressions, posture, and eye movements), is evaluated by an algorithm. Whether you advance to the next phase is determined by the algorithm. All algorithmic decisions are reviewed by a recruitment team.
\item An applicant participates in an automated video interview, where the applicant receives interview questions and records video responses. The video, including the applicant’s speech (fluency, prosody, pronunciation, language usage) and nonverbal behaviors (facial expressions, posture, and eye movements), is evaluated by an algorithm. If the algorithm rejects the applicant,  the decision is reviewed by a recruitment team. 
\item An applicant participates in an automated video interview, where the applicant receives interview questions and records video responses. The video, including the applicant’s speech (fluency, prosody, pronunciation, language usage) and nonverbal behaviors (facial expressions, posture, and eye movements), is evaluated by an algorithm. If the algorithm advances the applicant to the next phase, the decision is reviewed by a recruitment team. 
\item An applicant participates in an automated video interview, where the applicant receives interview questions and records video responses. The video, including the applicant’s speech (fluency, prosody, pronunciation, language usage) and nonverbal behaviors (facial expressions, posture, and eye movements), is evaluated by an algorithm that determines whether an applicant advances to the next phase.
% \item Why did you select the answers above for the different scenarios related to interviews?
\end{enumerate}

At the end of each set of Likert questions, participants were also asked an open response question (``Why did you select the answers above for the different scenarios related to [coding assessments/resumes/interviews]?'').

\subsubsection{Awareness of AEDTs}

In this section, participants were asked for each hiring process (online coding assessment, automated resume readers, and automated interviews) to check the box to indicate whether they have experience or knowledge of it:
\begin{itemize}
    \item[$\square$] Yes, I have experienced it
    \item[$\square$] No, but I have heard of it
    \item[$\square$] I'm not sure, but have heard of it
    \item[$\square$] No, I have not heard of or experienced it
\end{itemize}

Participants also responded to ``I know how my data was used in the hiring process'' and ``I received feedback from automated hiring algorithms'' from 1: Strongly disagree to 5: Strongly agree.

\subsubsection{Strategy Use}

Participants were asked the following questions about strategy use:
\begin{itemize}
\item Have you modified your resume specifically for automated resume readers? (Yes/No)
\item Have you added keywords from your job description? (Yes/No)
\item Have you changed the layout? (Yes/No)
\item Have you put it through a resume scanner? (Yes/No)
\item Have you modified your resume in some other way for automated hiring? (please specify)
\item Did you use a tool (LeetCode, HackerRank, etc.) to practice for coding assessments? (Yes/No)
\item Have you used anything else to prepare for automated assessments? (please specify)
\item Have you ever received a job referral? (Yes/No)
\item What proportion of your job applications did you have a referral for? (approximate percentage)
\item Approximately how many companies did you apply to? 
\item How did you learn about the application process? (check all that apply)
    \begin{itemize}
        \item[$\square$] Application materials and descriptions
        \item[$\square$] Online resources
        \item[$\square$] Career services through university 
        \item[$\square$] People who had gone through the application process
        \item[$\square$] Recruiter outside of company
        \item[$\square$] Recruiter through company
        \item[$\square$] Family members who worked at companies 
        \item[$\square$] Friends who worked at companies 
        \item[$\square$] Other people who worked at companies
    \end{itemize}
There was also an option to include additional strategies and an attention check in this stage.
\end{itemize}

\subsubsection{Hiring Outcome}
Participants were also asked about their hiring process and its outcome.
\begin{itemize}
\item Have you completed your hiring process? (Yes/No/Not applying to jobs)
\item I am satisfied with my hiring process so far. (1: Strongly disagree to 5: Strongly agree)
\item What is the outcome of your hiring process so far? 
    \begin{itemize}
        \item[$\square$] Multiple job offers
        \item[$\square$] One job offer
        \item[$\square$] No job offers
        \item[$\square$] Not applying to jobs
    \end{itemize}
\end{itemize}

\subsubsection{Demographic Information}
All questions in this section were optional and asked participants to disclose demographic information.

\begin{itemize}
    \item How would you describe your gender identity? (Select all that apply)
        \begin{itemize}
            \item[$\square$] Woman
            \item[$\square$] Man
            \item[$\square$] Non-binary
            \item[$\square$] Genderqueer
            \item[$\square$] Agender
            \item[$\square$] A gender not listed
        \end{itemize}
    \item What best describes you? (Select all that apply)
        \begin{itemize}
            \item[$\square$] Black or African-American
            \item[$\square$] American Indian or Alaskan Native
            \item[$\square$] Asian American or Asian
            \item[$\square$] Hispanic or Latinx
            \item[$\square$] Middle Eastern or North African
            \item[$\square$] Pacific Islander
            \item[$\square$] White or Caucasian
            \item[$\square$] Some other race, ethnicity, or origin 
        \end{itemize}
    \item What is your family’s approximate household income? 
\end{itemize}

\clearpage 

\subsection{Complete Statistical Results}
\label{app:stats}

\begin{table}[ht]
\begin{tabular}{lrrrrl}
\hline
\textbf{}                                            & \textbf{Estimate} & \textbf{Std. Error} & \textbf{t value} & \textbf{Pr(\textgreater{}|t|)} & \textbf{} \\ \hline
(Intercept)                                       & 2.786  & 0.266 & 10.493 & \textless{}0.01 &   \\
Added job description keywords to resume & 0.139  & -1.468    & 0.144 & 0.121            &   \\
Modified resume layout for resume readers & 0.150         & 0.133           & 1.119            & 0.265                         &           \\
Put resume through a resume scanner               & 0.001  & 0.136 & 0.007  & 0.995           &   \\
Practiced for online coding assessment            & 0.249  & 0.140 & 1.787  & 0.075           &   \\
Used referrals                                    & -0.336 & 0.136 & -2.478 & 0.014           & * \\
Percent of companies applied to with referral   & 0.002         & 0.003           & 0.817            & 0.415                         &           \\
Number of companies applied to                    & 0.001  & -0.516    & 0.606 & 0.405            &   \\
Awareness of online coding assessments            & -0.551 & 0.235 & -2.349 & 0.020           & * \\
Awareness of resume scanners                      & 0.014  & 0.183 & 0.076  & 0.940           &   \\
Awareness of automated video interviews           & 0.354  & 0.170 & 2.113  & 0.036           & * \\
Knowledge of data use                             & 0.055  & 0.047 & 1.162  & 0.247           &   \\
Received feedback in the hiring process           & 0.058  & 0.046 & 1.257  & 0.210           &   \\
Used application materials and descriptions       & -0.176 & 0.114 & -1.539 & 0.125           &   \\
Used online resources                             & 0.288  & 0.133 & 2.160  & 0.032           & * \\
Used career services through university           & 0.063  & 0.108 & 0.588  & 0.557           &   \\
Talked with people who had recently applied       & 0.129  & 0.127 & 1.012  & 0.313           &   \\
Connected with recruiter outside of company       & 0.053  & 0.159 & 0.336  & 0.737           &   \\
Connected with recruiter through company          & 0.124  & -1.346    & 0.180 & 0.191            &   \\
Had family who worked at companies        & 0.044  & 0.144 & 0.306  & 0.760           &   \\
Had friends who worked at companies               & 0.140  & 0.112 & 1.247  & 0.214           &   \\
Connected with other company contacts             & -0.022 & 0.126 & -0.179 & 0.858           &   \\
Race                                              & 0.005  & 0.109 & 0.425  & 0.671           &   \\
Gender                                            & -0.003 & 0.142 & -0.024 & 0.981           &   \\
Income                                            & 0.0000002  & 0.0000003 & 0.569  & 0.570           &   \\ \hline
\end{tabular}
\caption{\label{tab:fairStats} Linear regression model of procedural fairness perceptions for automated processes based on strategy use, awareness of AEDTs, gender, race, and income.}
\end{table}

\begin{table}[ht]
\begin{tabular}{lrrrrl}
\hline
\textbf{}                                            & \textbf{Estimate} & \textbf{Std. Error} & \textbf{t value} & \textbf{Pr(\textgreater{}|t|)} & \textbf{} \\ \hline
(Intercept)                                 & 2.479  & 0.268 & 9.267  & \textless{}0.01 &    \\
Added job description keywords to resume    & 0.140         & -1.374              & 0.171           & 0.210                         &           \\
Modified resume layout for resume readers & 0.169         & 0.135          & 1.257            & 0.210                         &           \\
Put resume through a resume scanner         & 0.038  & 0.137 & 0.273  & 0.785           &    \\
Practiced for online coding assessment      & 0.201  & 0.141 & 1.427  & 0.155           &    \\
Used referrals                              & -0.316 & 0.137 & -2.312 & 0.022           & *  \\
Percent of companies applied to with referral   & 0.002         & 0.003           & 0.670            & 0.504                         &           \\
Number of companies applied to              & 0.0004  & 0.001 & 0.544  & 0.589           &    \\
Awareness of online coding assessments      & -0.557 & 0.237 & -2.356 & 0.019           & *  \\
Awareness of resume scanners                & -0.046 & 0.184 & -0.248 & 0.805           &    \\
Awareness of automated video interviews     & 0.440  & 0.169 & 2.608  & {0.010}           & * \\
Knowledge of data use                       & 0.106  & 0.047 & 2.240  & 0.026           & *  \\
Received feedback in the hiring process     & 0.027  & 0.046 & 0.588  & 0.558           &    \\
Used application materials and descriptions & -0.220 & 0.012 & -1.911 & 0.057           &    \\
Used online resources                       & 0.261  & 0.134 & 1.942  & 0.054           &    \\
Used career services through university     & 0.152  & 0.108 & 1.399  & 0.163           &    \\
Talked with people who had recently applied & 0.172  & 0.128 & 1.344  & 0.181           &    \\
Connected with recruiter outside of company & 0.160  & -0.005    & 0.996 & 0.180          &    \\
Connected with recruiter through company    & 0.125  & -1.392    & 0.165 & 0.968           &    \\
Had family who worked at companies  & -0.006 & 0.145 & -0.040 & 0.968           &    \\
Had friends who worked at companies         & 0.134  & 0.113 & 1.188  & 0.236           &    \\
Connected with other company contacts       & 0.049  & 0.127 & 0.385  & 0.700           &    \\
Race                                        & 0.013  & 0.110 & 0.122  & 0.903           &    \\
Gender                                      & -0.116 & 0.143 & -0.815 & 0.416           &    \\
Income                                      & 0.0000002  & 0.0000003 & 0.623  & 0.534           &    \\ \hline
\end{tabular}
\caption{\label{tab:evalStats} Linear regression model of willingness to be evaluated by automated processes based on strategy use, awareness of AEDTs, gender, race, and income.}
\end{table}

\clearpage

\begin{table}[ht]
\begin{tabular}{lrrrrrl}
\toprule
& \textbf{Estimate}  & \textbf{Std. Error} & \textbf{t value} & \textbf{Pr(\textgreater{}|t|)} &   \\
\hline
(Intercept)                                          & 0.329     & 0.237      & 1.386   & 0.168                &   \\
Added job description keywords to resume    & 0.168     & 0.107      & 1.563   & 0.121                 &   \\
Modified resume layout for resume readers & 0.103     & -0.724     & 0.471   & 0.515                 &   \\
Put resume through a resume scanner                  & 0.020     & 0.101      & 0.201   & 0.841                 &   \\
Practiced for online coding assessment               & -0.201    & 0.133      & -1.513  & 0.133                 &   \\
Used referrals                                       & 0.122     & 0.100      & 1.213   & 0.227                 &   \\
Percent of companies applied to with referral   & 0.004     & 0.002      & 2.063   & 0.041                 & * \\
Number of companies applied to                       & 0.0004    & 0.001     & 0.835   & 0.405                 &   \\
Awareness of online coding assessments               & 0.050     & 0.199      & 0.252   & 0.801                 &   \\
Awareness of resume scanners                         & -0.019    & 0.173      & -0.109  & 0.913                 &   \\
Awareness of automated video interviews              & -0.036    & 0.157      & -0.228  & 0.820                 &   \\
Knowledge of data use                                & 0.039     & 0.004      & 0.984   & 0.327                 &   \\
Received feedback in the hiring process              & 0.011     & 0.004      & 0.302   & 0.763                 &   \\
Used application materials and descriptions          & 0.025     & 0.009      & 0.279   & 0.781                 &   \\
Used online resources                                & -0.174    & 0.115      & -1.518  & 0.132                 &   \\
Used career services through university              & 0.055     & 0.085      & 0.644   & 0.521                 &   \\
Talked with people who had recently applied          & 0.024     & 0.107      & 0.225   & 0.823                 &   \\
Connected with recruiter outside of company          & 0.009     & 0.112      & 0.080   & 0.937                 &   \\
Connected with recruiter through company             & 0.115     & 0.088      & 1.314   & 0.191                 &   \\
Had family who worked at companies           & -0.140    & 0.109      & -1.287  & 0.200                 &   \\
Had friends who worked at companies                  & 0.160     & 0.087      & 1.841   & 0.068                 &   \\
Connected with other company contacts                & -0.101    & 0.093     & -1.089  & 0.278                &   \\
Race                                                 & -0.008    & 0.119      & -0.070  & 0.945                 &   \\
Gender                                               & 0.081     & 0.081     & 0.991   & 0.324                 &   \\
Income                                               & 0.000001 & 0.0000002  & 2.530   & 0.013                 & * \\
\bottomrule
\end{tabular}
\caption{\label{tab:jobStats} Linear regression model of job success based on strategy use, awareness of AEDTs, gender, race, and income.}
\end{table}


\end{document}
\endinput
%%
%% End of file `sample-manuscript.tex'.
