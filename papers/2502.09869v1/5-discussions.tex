\section{Discussion}
\subsection{Revisiting the Explicit-Implicit Dichotomy Through Intention}
% 1. Why our findings of the ``intentional implicit feedback'' are important. We can talk about the benefits: identifying the intentional implicit feedback can (1) help with users' understanding and control of the algo and achieve more desirable personal recommendation, (2) reduce user unintended or unexpected outcome (false positive error), and (3) free up users' cognition for other tasks. 
% We can also talk about the problems of the explicit–implicit dichotomy...

The explicit-implicit feedback dichotomy was largely framed from the system's perspective, as Jannach et al. framed it---implicit feedback referred to interactions from which we (the system side) can indirectly infer user preferences ~\cite{jannach2018recommending}. The implicit feedback was supposed to be user's natural interactions with the platform. However, from the user’s perspective, they are not only aware of, but also intentionally, proactively act as feedback inputs to influence the recommendation feeds. For instance, in our findings, when participants ``dwell on'' or ``swipe past'' a post, they might consider it as an explicit expression of their preferences. Yet, the system might take it as implicit feedback and fail to understand or capture it. This highlights a fundamental challenge: feedback categorized as implicit from the system's point of view may be deliberately used as explicit by users.

This misalignment between how users and the system perceive feedback can lead to inaccuracies or failures in capturing user intent. A specific problem is that implicit feedback is often only interpreted as positive feedback~\cite{jawaheer2014modeling,hu2008collaborative}, whereas our findings revealed that users also intentionally employed implicit feedback behaviors to communicate negative feedback, such as ignoring or swiping past a post ($n = 21$), stop using the platform or switch platforms ($n = 12$), and refresh the feed ($n = 2$). Neglecting on interpreting negative implicit feedback could distort the user profile~\cite{hu2008collaborative}. Thus, incorporating both positive and negative implicit feedback is essential for more accurate recommendation. Another related problem is that the platform' focus on implicit positive feedback may lead to instances of false positive interpretation. For instance, several participants mentioned they searched for information as a one-time task, but the platform over interpreted this as a preference, resulting in their feed being flooded with similar content that was no longer relevant or needed. This suggests that implicit feedback should be considered through more nuanced distinctions, such as positive versus negative, intentional versus incidental~\cite{dix2002beyond}, and reactive versus proactive behaviors~\cite{ju2008range}. Particularly, defining explicit–implicit distinction through intention can add greater precision to the interpretation of user feedback~\cite{serim2019explicating}. 
% implicit feedback has traditionally been treated as positive, and the lack of attention to negative signals can lead to misrepresenting user preferences ~\cite{hu2008collaborative}. However, our participants frequently used implicit feedback to express negative signals, with actions like ignoring or swiping past posts being the most common.

%Therefore, the addition of intentional implicit feedback to the implicit-explicit dichotomy in our paper can help platforms better distill users' intention, addressing the above issues to improve the accuracy in predicting user preferences, achieving more desired personalized recommendations.Comparing to unintentional implicit feedback, 
In addition, acknowledging and responding to the intentional implicit feedback can afford users more understanding of the algorithm’s learning process. Users will feel that their interactions are shaping the recommendation feeds, thus foster a greater sense of self-causality and increase user agency~\cite{feng2024mapping}. Comparing to explicit feedback, intentional implicit feedback also frees up users’ cognitive resources. Research has shown that explicit feedback often imposes a higher cognitive load, as users must actively engage with the system ~\cite{kelly2003implicit,gadanho2007addressing}. In contrast, allowing users to guide recommendations through unobtrusive, yet intentional, behaviors can make the interaction more seamless and less mentally taxing. This aligns with findings from previous studies that highlight the importance of minimizing user effort in feedback mechanisms.

\subsection{Aligning Feedback Types with Users' Purposes}
% Discuss why users chose to use different feedback types for different purposes. The nature and characteristics of the feedbacks.
The study identified four purposes in participants' using of personalized recommendation platforms: content consumption, directed information seeking, content creation and promotion, and feed customization. The first three purposes echo previous literature, which has discussed directed and undirected consumption~\cite{feng2024mapping}, as well as self-presentation ~\cite{devito2017algorithms}. We identified feed customization as a purpose that arises when users are dissatisfied with the recommendation contents, which encompasses four specific goals. 

Users choose different feedback types based on their purposes. Explicit feedback (75.9\% instances) is primarily used for feed customization. As shown in~\autoref{tab:feedbackType}, all explicit feedback behaviors were supported by platform features, providing cues to helps users understand the potential outcomes of their actions. Such clarity makes explicit feedback the preferred choice when users seek quick corrections to their recommendation feeds. For instance, our findings indicated that most users opted for explicit feedback, particularly negative explicit feedback, such as marking content as ``Not interested,'' blocking, or reporting, when they had a strong desire to remove inappropriate or irrelevant content from their feeds. Previous studies found that users were reluctant to provide explicit feedback in complex tasks like online shopping but were more willing to do so in media and entertainment domains~\cite{white2005study,jawaheer2014modeling}.
This aligns with our findings that, on personalized recommendation platforms, users did not show a strong preference between explicit (31 instances) and implicit feedback (44 instances) during feed customization.

On the other hand, intentional implicit feedback (88.6\%) was also mostly used for feed customization, but often when users' intent is not particularly strong or urgent---either to increase content diversity or improve recommendation relevance. Intentional implicit feedback allows users to steer the algorithm towards their evolving preferences without overtly signaling their intent or interrupting the flow of content consumption. The choice among various implicit feedback behaviors is usually based on their continuously updated folk theories~\cite{devito2017algorithms}. Since users are not fully certain how the algorithm will interpret their actions ~\cite{eslami2015always}, they tend to experiment and make repeated attempts when necessary. Despite these efforts, our participants sometimes observed minimal or delayed changes in the recommendation feeds, due to the less obvious and plausible nature of implicit feedback~\cite{jannach2018recommending}. This then may then lead to more passive behaviors like stop using the platforms.

Moreover, the design and functionality of the platforms~\cite{liu2010personalized}, such as Douyin’s swipe interaction versus Xiaohongshu’s click-and-select feed layout, also mediate users' specific feedback behaviors. Participants on Xiaohongshu, for example, found selecting and clicking more efficient for improving recommendation relevance, comparing to using the swipe function in Douyin. Therefore, platform design should support a wider range of intentional implicit feedback by providing users with clearer traits of how their actions influence recommendation feeds and more intuitive ways to interact with the platform.

\subsection{Design Implications}
% 1. supporting intentional implicit feedback: (1) provide features to support or prompts, but also remain unobtrusive for natural flow. (2) support negative feedback.Most used implicit and explicit are both negative feedback.
% (3)  The need for users to repeatedly mark content as ``Not interested'' suggests that the algorithm may not be efficiently learning from initial feedback. Enhancing the algorithm's ability to quickly adapt to user preferences after receiving negative feedback could improve the user experience by reducing the frequency of unwanted recommendations. but also not too sensitive to reduce false positive 
% 2. Incorporate user goals. such as balance between relevance and diversity
% 3. transparency in data collected and protect privacy 
The discussion leads to the following design implications for the personalized recommendation platforms to effectively support and respond to user feedback driven by various purposes.

\subsubsection{Supporting Intentional Implicit Feedback Through Recognition and Visualization} 
First, platforms need to support intentional implicit feedback. When users intentionally provide implicit feedback, they expect the algorithm to infer their interest from their ``natural interactions''~\cite{kelly2003implicit} and respond seamlessly. This differs from explicit feedback, which is given through direct user input that interrupts users' natural flow. However, when the subtle feedback goes unnoticed or yields little immediate effect, users may feel a loss of control. Therefore, while most implicit feedback is not supported by platform features (shown in~\autoref{tab:feedbackType}), it's important to inform users that their feedback has been recognized and will influence future recommendation contents. One potential solution is to implement non-intrusive visual cues, such as small icons or pop-ups, that confirm feedback has been registered. For instance, after a user click similar posts several times or search for a topic, a pop-up could confirm that the platform has recognized this interest and will increase similar content recommendations. Another option is to provide progress indicators that allow users to see how their behavior is affecting the feeds over time. This could include a dashboard showing how many times they had skipped certain content and how it has adjusted their recommendation feeds. This can increase transparency and help users understand how the algorithm is learning from their feedback and responding accordingly. 

Additionally, platforms need to improve their handling of negative implicit feedback. The challenge with interpreting negative implicit feedback lies in its subtlety. Unlike explicit feedback like marking as not interested, implicit negative signals are often less direct and harder to capture. To address this issue, platforms need to adopt more sophisticated approaches to detect and validate implicit negative feedback. While some platforms have already implemented ``show less'' features that allow users to indicate their content preferences (such as ~\cite{meta_new_2022}). These features are typically static and presented with all content. What's needed is a more dynamic approach that can capture and respond to users' behavioral patterns in real-time. One potential strategy is explicit user confirmation. For example, when a user repeatedly swipes past posts or consistently ignores content, the system could proactively detect that the user is not interested in this topic and prompt the user with an option like, ``Want to see less of this content?'' to confirm their intent. Such dynamic approaches can capture users' intentional implicit feedback behaviors more effectively and would not cause a significant disruption to users' normal usage.
% acknolwege current available design(https://ai.meta.com/blog/facebook-feed-improvements-ai-show-more-less/#:~:text=The%20challenge%20with%20training%20these,they%20would%20like%20to%20see)

\subsubsection{Designing for Purpose-Oriented Feedback}
Second, we suggest designing for purpose-oriented feedback. Our findings revealed close association between users' purposes and their choice of feedback types. This underscores the importance of incorporating users' diverse and evolving purposes in the design of personalized recommendation platforms in addition to focusing on their behaviors~\cite{liang2023enabling, nazari2022choice}. To address this, platforms can integrate context-aware learning models to infer user purposes. For example, during directed information seeking, algorithm can prioritize relevance and rely more on implicit feedback~\cite{white2005study}; in content consumption, the algorithm can also consider content diversity, as many participants were concerned with information cocoon~\cite{li2022exploratory}. Meanwhile, the platforms can also provide customization options, such as a slider, for users to adjust the balance between relevance and diversity in their recommendation contents. It should be mention that these setting options, such as setting up interested channels, should be made more discoverable, as some participants did not notice the feature~\cite{liu2024train}.

\subsubsection{Transparency in Feedback Data Collection and Protecting Privacy}
Quite a few participants expressed concerns about their privacy, noting that platforms seemed to be monitoring their conversations, chats with friends, and even activities on other platforms. Despite these concerns, only two participants took active steps to protect their privacy by disabling personalization. However, after noticing that the recommendation quality significantly deteriorated, they eventually re-enabled it. This suggests that users have no effective means to combat platforms' invasion of privacy. To address these concerns, platforms should be transparent about the types of data being collected, including behaviors within the platform, cross-platform tracking, and even offline monitoring. Platforms should allow users to customize their privacy settings, giving them the ability to select which types of data they are comfortable sharing for algorithm personalization. This could include options to opt out of specific data types, while still maintaining a personalized recommendation system.

Interestingly, as we mentioned in the results, one participant mentioned deliberately speaking aloud to trigger the platform's monitoring and influence recommendation feeds. This behavior highlights a potential design opportunity: platforms could introduce voice-based feedback to explicitly capture users' preferences, rather than continuously monitoring their voices. By offering users the ability to verbally express their content needs, platforms could create a more transparent and interactive method for users to directly influence recommendation feeds while respecting their privacy.


% 4. What will be the challenges of designing for ``intentional implicit feedback''? For example, ``constructive validity''~\cite{serim2019explicating}---the extent about how intentional implicit feedback translates into design in our case. 
% Other challenges can be users' literacy to use these designs, users' fluid needs, and delayed system reactions. According to Serim and Jacucci~\cite{serim2019explicating}, users' attitudes can be fluid and influenced by situational factors, which poses uncertainties to determining implicitness.
\subsection{Limitations and Future Work}
This study has several limitations that should be noted. First, the recruitment method relied on pre-screening surveys and snowball sampling. This might have introduced selection bias, favoring more engaged platform users and educated users. Second, the feedback behaviors reported by users were neither comprehensive nor reflective of the ground truth. Specifically, the list of unintentional implicit feedback behaviors in ~\autoref{appendix:unintentional} includes only interactions mentioned by users and categorized as implicit feedback in previous literature~\cite{jannach2018recommending}. We do not have access to the complete set of behaviors that platforms use as implicit feedback. 

% \subsection{Future Work}
% 研究什么样的人会用intentional里的所有behavior,和他们的demo characteristics,以及为何选择specific behavior
Our findings shed light on how users leveraged different feedback mechanisms to fulfill their purposes on personalized recommendation platforms, but further investigation is needed to validate and expand upon these insights. One potential future work is to conduct a large-scale survey to confirm the feedback behaviors and patterns identified in our study. Although our interviews touched upon users' motivations for selecting different types of feedback in various contexts and purpose-oriented situations, a survey study would allow us to explore users' rationales when choosing different feedback types in different contexts. Another important direction for future research is to examine how different types of feedback—explicit feedback, unintentional implicit feedback, and intentional implicit feedback—impact user experience and satisfaction with recommendation systems. This could include how each feedback type influences content quality, user engagement, and perceived control over the recommendation system. Additionally, future research could examine the relationships between users' feedback choices and factors such as demographics and algorithm literacy. Understanding how these factors influence users' engagement with feedback mechanisms could support the design of more personalized and effective feedback mechanisms.