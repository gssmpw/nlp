% \section{Terminology Communication}
% We want to use terminologies consistently throughout our writing. For literature reviews, we will keep the terminologies as they were originally used in the literature.

% Below are several terminology options used in related papers:

% “Recommendation algorithms for social media feeds” in the paper “TikTok and the Art of Personalization: Investigating Exploration and Exploitation on Social Media Feeds.”

% “Recommendation systems” and “Recommendation algorithm” in the paper “Recommending to Strategic Users.”

% “Recommendation algorithms” and “personalized recommendation” in the paper “Measuring Strategization in Recommendation: Users Adapt Their Behavior to Shape Future Content.”

\section{Introduction}
% 需要添加explicit/implicit feedback 的解释,这是从系统角度的。Abstract里边也需要有
% 第二段要顺承故事引入feedback,第三段要介绍feedback定义。
% 以前的研究包括folk theory,imaginary等和我们是相似的,但我们的不同是在什么地方

Ariela is a design undergraduate in China and an avid social media user. She is captivated by the personalized feeds on Xiaohongshu\footnote{Xiaohongshu, also known as RedNote in English, is a community-sharing application. It can be accessed at \url{https://www.xiaohongshu.com}.}, which are filled with fashion trends, travel tips, and skincare products that guide her purchases and help her explore new hobbies (see \autoref{fig:feedback} \& \autoref{fig:xhspost} for Xiaohongshu's user interfaces). With each interaction, the platform seems to understand her better, continually delivering content that aligns with her preferences. Meanwhile, Ariela uses Douyin, the Chinese version of TikTok, for light entertainment during her breaks (see \autoref{fig:douyinmain} for Douyin's user interface). She enjoys watching short, fun videos on Douyin. When the algorithm suggests content she dislikes, she swipes past it, subtly guiding the platform toward content like cats or funny skits. However, Ariela soon realizes her feeds on both platforms, once diverse, are becoming repetitive, reflecting only her past choices. To break free from these patterns, Ariela deliberately searches for new styles and trends on Xiaohongshu or clicks ``Not interested'' on certain posts, disrupting its usual suggestions. And on Douyin, she fast skips videos she typically enjoys, hoping to inform the platform to offer something different.
\aptLtoX[graphic=no,type=html]{\begin{figure*}
    \centering
    \begin{subfigure}[t]{0.3\textwidth}
        \centering
        \includegraphics[width=0.6\textwidth]{xhsReport.PNG}
        \caption{The main page of Xiaohongshu ``Explore'' and feedback options}
        \label{fig:feedback}
    \end{subfigure}
    \hfill
    \begin{subfigure}[t]{0.3\textwidth}
        \centering
        \includegraphics[width=0.6\textwidth]{xhsPost.PNG}
        \caption{A Xiaohongshu post}
        \label{fig:xhspost}
    \end{subfigure}
    \hfill
    \begin{subfigure}[t]{0.3\textwidth}
        \centering
        \includegraphics[width=0.6\textwidth]{douyin.PNG}
        \caption{The main page of Douyin: ``For You''}
        \label{fig:douyinmain}
    \end{subfigure}
    \caption{The main user interfaces of Xiaohongshu and Douyin. 
    (a) is the main page of Xiaohongshu, the ``Explore'' page, displaying a selection of posts recommended by the algorithm, consisting of both picture and video posts. Located at the top of the main page is the search bar. Below the search bar, users often see trending hashtags and their interested channels that can be customized. Long press on a post can trigger options for reporting a post, including ``Not interested'' and ``Content feedback.''  
    (b) is a note detail page, where the note itself is the centerpiece. It typically includes a mix of text, images (or videos), and hashtags. Users can follow the creator, like, collect, or leave comments on the post. 
    (c) is the main page of Douyin, the ``For You'' page, showcasing a continuous stream of short videos curated by Douyin's algorithm for each user. A new video automatically displays as users scroll (or swipe) vertically. Users can like a video, leave a comment, share it, or follow the creator using the buttons on the right side. Douyin also provides similar content feedback features.}
    \label{fig:interface}
\vspace*{-10pt}
\end{figure*}}{\begin{figure*}
    \centering
    \begin{subfigure}[t]{0.3\textwidth}
        \centering
        \includegraphics[width=0.6\textwidth]{xhsReport.PNG}
        \caption{The main page of Xiaohongshu ``Explore'' and feedback options}
        \label{fig:feedback}
    \end{subfigure}
    \hfill
    \begin{subfigure}[t]{0.3\textwidth}
        \centering
        \includegraphics[width=0.6\textwidth]{xhsPost.PNG}
        \caption{A Xiaohongshu post}
        \label{fig:xhspost}
    \end{subfigure}
    \hfill
    \begin{subfigure}[t]{0.3\textwidth}
        \centering
        \includegraphics[width=0.6\textwidth]{douyin.PNG}
        \caption{The main page of Douyin: ``For You''}
        \label{fig:douyinmain}
    \end{subfigure}
    \caption{The main user interfaces of Xiaohongshu and Douyin. 
    (\subref{fig:feedback}) is the main page of Xiaohongshu, the ``Explore'' page, displaying a selection of posts recommended by the algorithm, consisting of both picture and video posts. Located at the top of the main page is the search bar. Below the search bar, users often see trending hashtags and their interested channels that can be customized. Long press on a post can trigger options for reporting a post, including ``Not interested'' and ``Content feedback.''  
    (\subref{fig:xhspost}) is a note detail page, where the note itself is the centerpiece. It typically includes a mix of text, images (or videos), and hashtags. Users can follow the creator, like, collect, or leave comments on the post. 
    (\subref{fig:douyinmain}) is the main page of Douyin, the ``For You'' page, showcasing a continuous stream of short videos curated by Douyin's algorithm for each user. A new video automatically displays as you scroll (or swipe) vertically. Users can like a video, leave a comment, share it, or follow the creator using the buttons on the right side. Douyin also provides similar content feedback features.}
    \label{fig:interface}
\vspace*{-10pt}
\end{figure*}}

% \begin{figure*}[]
%   \subfloat[The main page of Xiaohongshu ``Explore'' and feedback options]{%
% 	\begin{minipage}[c][1.3\width]{
% 	   0.3\textwidth}
% 	   \centering
% 	   \includegraphics[width=0.6\textwidth]{xhsReport.PNG}%
% 	   \label{fig:feedback}%
%    	\end{minipage}}%
%  \hfill	
%   \subfloat[A Xiaohongshu post]{%
% 	\begin{minipage}[c][1.3\width]{
% 	   0.3\textwidth}
% 	   \centering
% 	   \includegraphics[width=0.6\textwidth]{xhsPost.PNG}%
% 	   \label{fig:xhspost}%
% 	\end{minipage}}%
%   \hfill	
% \subfloat[The main page of Douyin: ``For You'']{%
% 	\begin{minipage}[c][1.3\width]{
% 	   0.3\textwidth} 
% 	   \centering
% 	   \includegraphics[width=0.6\textwidth]{douyin.PNG}%
% 	   \label{fig:douyinmain}%
% 	\end{minipage}}%
% \caption{The main user interfaces of Xiaohongshu and Douyin. \protect\subref{fig:feedback} is the main page of Xiaohongshu, the ``Explore'' page, which displays a selection of posts known as ``note'' recommended by the algorithm, consisting of both picture and video posts. Located at the top of the main page is the search bar. Below the search bar, users often see trending hashtags and their interested channels that can be customized. Long press on a post can trigger the options for reporting a post, including ``Not interested'' and ``Content feedback.''  
% \protect\subref{fig:xhspost} is a note detail page, where the note itself is the centerpiece. It typically includes a mix of text, images (or videos), and hashtags. Users can follow the creator, and like, collect, or leave comments on the post. 
% \protect\subref{fig:douyinmain} is the main page of Douyin, the ``For You'' page that showcases a continuous stream of short videos curated by Douyin's algorithm specifically for each user. A new video automatically displays as you scroll (or swipe) vertically. Users can like a video, leave a comment, share it with others, or follow the creator clicking the icon buttons on the right side. Douyin also provides similar content feedback features.}
% \label{fig:interface}
% \end{figure*}

Just like Ariela, we increasingly consume content curated by personalized recommendation algorithms on social media platforms ~\cite{guy2010social}, such as Xiaohongshu, Douyin, and more. These algorithms, designed to capture user preferences through every click, view, or interaction, create a profile for each user to recommend content that is not only relevant but also engaging~\cite{setyani2019exploring}, enticing users into enduring usage ~\cite{seaver2019captivating}. Users form understandings of these algorithms through folk theories---``intuitive, informal theories that individuals develop to explain the outcomes, effects, or consequences of technological systems''~\cite{devito2017algorithms}---which affect how they interact with algorithms ~\cite{devito2017algorithms, devito2018people, eslami2016first, ngo2022exploring}. As personalized recommendation algorithms increasingly penetrate users' online and offline activities, concerns have emerged about the platforms ``spying'' on their preferences ~\cite{ellison2020we, klug2021trick} or pushing them into the homogeneous ``echo chamber'' ~\cite{gao2023echo} or ``information cocoons'' ~\cite{li2022exploratory}. In response, users have developed various strategies to influence the content recommended to them, such as refraining from hitting likes, tapping ``Not interested,'' searching for certain topics, or ignoring content they liked~\cite{kim2023investigating, Cen_Ilyas_Allen_Li_Madry_2024}. These strategies are based on the assumption that users’ behaviors will be captured by platforms as \textit{feedback} to the algorithms and, consequently, influence future recommendations~\cite{Cen_Ilyas_Allen_Li_Madry_2024}.

In the context of system engineering, feedback has been extensively studied to enhance the performance of information retrieval and recommender systems~\cite{kelly2003implicit}. These systems rely on both explicit and implicit feedback, a well-established dichotomy in the existing literature~\cite{jawaheer2014modeling}. Briefly, \textit{explicit feedback} refers to direct input provided by users to express their preferences, such as specifying keywords, rating, or answering questions about their interests, whereas \textit{implicit feedback} refers to various user interactions with the system, such as viewing, selecting, saving, or forwarding content, from which the system indirectly infers user preferences ~\cite{kelly2003implicit,jannach2018recommending}. As both concepts require user behavior or interaction as input, we argue that on personalized recommendation platforms, users strategically employing their behaviors to shape their recommendation feeds constitutes a form of feedback to the systems.

%Yet, due to lack of transparency in these algorithms ~\cite{gillespie2014relevance, pasquale2011restoring}, 
% 
In fact, user's strategization in algorithmic systems has been explored in the HCI communities. For example, scholars have also investigated user resistance to algorithms~\cite{karizat2021algorithmic, rosenblat2016algorithmic} and user autonomy~\cite{ngo2022exploring, kang2022ai, feng2024mapping}. While these studies highlight users' intentions and potential to influence recommendation feeds, limited research has connected users' perceptions with system feedback mechanisms. Studying the connection could lead to improved feedback designs in personalized recommendations. To address this gap, our work began with a broader inquiry: \textit{How do users provide feedback through their behaviors and platform mechanisms to shape and control the content presented to them on personalized recommendation platforms?}
% However, as personalized recommendation becoming more sophisticated, they also raise important concerns about user autonomy and the transparency of algorithmic processes ~\cite{ngo2022exploring, kang2022ai,feng2024mapping}.

We conducted semi-structured interviews with 34 active users of personalized recommendation platforms (e.g., Xiaohongshu and Douyin). We found that users employed a variety of feedback mechanisms to influence the content they receive. These mechanisms range from explicit feedback, such as marking content as ``Not interested,'' to implicit feedback like clicking and liking, from which user preferences were indirectly inferred. We found that the traditional explicit-implicit dichotomy~\cite{kelly2003implicit, jannach2018recommending} failed to fully capture users' agency when they consciously employ behaviors previously categorized as implicit feedback to shape their recommendation feeds. To address this, we divided the implicit feedback category to \textit{intentional implicit feedback} and \textit{unintentional implicit feedback}. Unlike the conventional understanding of implicit feedback as passive or natural interactions, intentional implicit feedback refers to behaviors consciously performed by users with the expectation that the system will interpret them as signals of their preferences. For example, behaviors like quickly skipping a disliked post or deliberately clicking interested posts to get more related content were frequently observed in our study. These behaviors, categorized as implicit feedback in previous research, are distinctly intentional in nature. At the same time, unlike \textit{explicit feedback}, intentional implicit feedback allows users to guide their recommendations without direct input to express their preferences. By introducing the \textit{intention} dimension into the dichotomy, we highlight users' proactive engagement in shaping their feeds through both explicit and implicit feedback behaviors. By recognizing intentional implicit feedback, platforms can more accurately capture user intent and provide users with a greater sense of control over their feeds. 

Additionally, we found that users' feedback behaviors were closely associated with their purposes. Explicit feedback was primarily used for feed customization goals like reducing inappropriate content and improving recommendation relevance, while intentional implicit feedback emerged as crucial for feed customization to increase content diversity and improve recommendation relevance. Unintentional implicit feedback was most commonly linked to content consumption. The findings underscore the need to better design for implicit feedback in personalized recommendations and to align feedback mechanisms with users' specific purposes.

This work has the following contributions: first, the study provided empirical evidence of how users leveraged different feedback mechanisms to fulfill their purposes when using personalized recommendation platforms. Second, the work introduced the concept of intentional implicit feedback, expanding beyond the traditional explicit and implicit feedback dichotomy. The intentional implicit feedback captures user's intentions in taking their actions to influence algorithms and therefore future recommendation feeds. Third, the study offers design implications for personalized recommendation platforms to support more transparent and purpose-oriented feedback mechanisms. 