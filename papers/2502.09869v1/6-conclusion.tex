\section{Conclusion}
% We explored how users leverage diverse ways of feedback mechanisms to influence algorithmic recommendations through semi-structured interviews with 34 active users on platforms such as Xiaohongshu and Douyin. Our findings revealed the complex dynamics between users and algorithms. We further categorized user feedback into explicit, intentional implicit, and unintentional implicit types, identifying strategies users employed to customize their feeds. Unintentional implicit feedback was predominantly used for general content consumption, whereas explicit feedback, despite its critical role in feed customization, was less frequent due to its high levels of cognitive load and efforts. In contrast, intentional implicit feedback, such as ignoring or swiping past content, had emerged as a popular and effective method for guiding recommendations. This analysis underscored the importance of accurately interpreting user motivations, from content consumption to strategic feed customization, in shaping recommendation outputs.

% Our findings challenged traditional explicit and implicit feedback dichotomy by introducing the concept of intentional implicit feedback and suggested that algorithms should more effectively account for both explicit and implicit signals to better respond to user preferences. The insights we provided offered valuable guidance for designing more adaptive and user-centered recommendation systems. We further advocated researchers for a deeper understanding of feedback mechanisms and encouraged profound exploration of the dynamic interactions between users and algorithms to foster a more engaging and satisfying user experience.

We conducted semi-structured interviews with 34 active users on platforms like Xiaohongshu and Douyin to explore how users employ diverse feedback mechanisms to influence recommendation feeds towards specific purposes. We categorized various user feedback behaviors into three types: explicit, intentional implicit, and unintentional implicit feedback. We also found that users' choices of feedback types are closely aligned to their purposes. Explicit feedback was primarily used for feed customization goals like reducing inappropriate content and improving recommendation relevance, while intentional implicit feedback emerged as crucial for feed customization to increase content diversity and improve recommendation relevance. Unintentional implicit feedback was most commonly linked to content consumption. The study introduced the intention dimension into the traditional explicit-implicit feedback dichotomy and suggested that personalized recommendation platforms should better support transparent intentional implicit feedback and purpose-oriented feedback design.