\section{Details on the Datasets}
\label{sec:datasets}

We include details on each dataset used in our experiments. Some datasets are accessed from the SDR Bench \cite{zhao2020sdrbench}, available at~\cite{SDRBench}. Datasets accessed from the SDR Bench may involve contributions from the DOE NNSA ECP project and the ECP CODAR project.

The \textbf{QMCPACK} dataset (accessed from the SDR Bench) comes from the QMCPACK performance test using the QMCPACK continuum quantum Monte Carlo
simulation \cite{Kim_2018, Kent2020}. Only the 145th orbital out of 288 is used for testing, for which only a single field is provided. The dataset is normalized to $[0,1]$ before compression. 

The \textbf{Tangaroa} dataset comes from a single time frame simulating the wind flow around a 3D model of the Research Vessel Tangaroa \cite{popinet2004experimental}. 
The magnitude of the wind velocity is used as the scalar field of interest. The dataset is normalized to $[0,1]$ before compression.

The \textbf{Earthquake} dataset originates from a TeraShake 2 earthquake simulation~\cite{olsen2008terashake2} and has been part of the 2006 IEEE Visualization Design Contest~\cite{scivis2006}.  
The dataset used in this paper is obtained from the public data repository of Pont et al. used for their publication~\cite{pont2021wasserstein}. The dataset is preprocessed by Pont et al. and comes with a single field. It represents one time step of a simulation of an earthquake at the San Andreas fault. Specifically, we use time step 011700. The details of the preprocessing can be found on the repository. The dataset is normalized to $[0,1]$ before compression.

The \textbf{Ionization} dataset originates from an ionization front simulation by Whalen and Norman \cite{whalen2008ionization} and has been featured in the 2008 IEEE Visualization Design Contest~\cite{scivis2008}.   
The simulation is done with 3D radiation hydro-dynamical calculations of ionization front instabilities in which multi-frequency radiative transfer  is coupled to the primordial chemistry of eight species~\cite{whalen2008ionization}. 
The single time step used in this paper comes from cluster 2, time step 0125 and is obtained from the same repository as the Earthquake dataset. It is 
preprocessed and comes with a single field. The details of the preprocessing can be found on the repository. The dataset is normalized to $[0,1]$ before compression.

The \textbf{Isabel} dataset originates from a hurricane simulation from the National Center for Atmospheric Research, and has been included in the 2004 IEEE Visualization Design Contest \cite{scivis2004}.
While the original dataset has a size of  $500 \times 500 \times 100$, we truncate the dataset to $500 \times 500 \times 90$ in order to avoid land regions that contain no data values. We use the wind speed field. The dataset is normalized to $[0,1]$ before compression.

The \textbf{Miranda} dataset (accessed from the SDR Bench) comes from the hydrodynamics code for large turbulence simulations conducted by Lawrence Livermore National Laboratory. We use the density field.

The $\textbf{Nyx}$ dataset (accessed from the SDR Bench) comes from the Nyx cosmological simulation~\cite{almgren2013nyx}. We use the dark matter density field.

The $\textbf{S3D}$ dataset comes from the S3D turbulence simulation software~\cite{treichler2017s3d}. This dataset is derived from data from the SDR Bench. We compute the field as the magnitude of the velocity, as derived from the velocity $x$, $y$, and $z$ components provided. We use its parameter setting $1.7e \times 10^{-3}$.

The $\textbf{SCALE-LETKF}$ dataset (accessed from the SDR Bench) comes from the Local Ensemble Transform Kalman Filter (LETKF) data assimilation package for the SCALE-RM weather model~\cite{lien2017near}. We use the QV field which is up-sampled to double precision.