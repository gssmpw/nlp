\documentclass[journal]{vgtc}

%-------------------------
\onlineid{1512}

\vgtccategory{Research}
\vgtcpapertype{Technique}

\usepackage{tabu}
\usepackage{booktabs}
%-------------------------

%-------------------------
\usepackage{amsfonts}
\usepackage{amsmath}
\usepackage{multirow}

\newcommand{\N}{\mathbb{N}}
\newcommand{\R}{\mathbb{R}}
\newcommand{\Z}{\mathbb{Z}}
\newcommand{\X}{\mathbb{X}}

\newcommand{\para}[1]{\vspace{1pt}\noindent{\textbf{#1}}}
\newcommand{\denselist}{\vspace{-3pt} \itemsep -2pt\parsep=-1pt\partopsep -2pt}

\setlength{\lineskiplimit}{0pt}
\setlength{\lineskip}{0pt}
\setlength{\abovedisplayskip}{0pt}
\setlength{\belowdisplayskip}{0pt}
\setlength{\abovedisplayshortskip}{0pt}
\setlength{\belowdisplayshortskip}{0pt}

%-------------------------
\title{A General Framework for Augmenting Lossy Compressors with Topological Guarantees}

\author{
  \authororcid{Nathaniel Gorski}{0009-0001-8205-5640}, \authororcid{Xin Liang}{0000-0002-0630-1600}, \authororcid{Hanqi Guo}{0000-0001-7776-1834},\authororcid{Lin Yan}{0000-0001-7017-0329}, and \authororcid{Bei Wang}{0000-0002-9240-0700}
}

\authorfooter{
\item Nathaniel Gorski is with the University of Utah.
  	E-mail: gorski@sci.utah.edu.
\item Xin Liang is with the University of Kentucky. 
 	E-mail: xliang@uky.edu.
\item Hanqi Guo is with the Ohio State University. 
	E-mail: guo.2154@osu.edu.
\item Lin Yan is with the Iowa State University. 
	E-mail: linyan@iastate.edu. 
\item Bei Wang is with the University of Utah. 
	E-mail: beiwang@sci.utah.edu. 
}

%-------------------------

\abstract{
%=========================================================================
% Abstract
%=========================================================================
% General matrix multiply (GEMM) is the key building block in many
% different domains including machine learning, graph analytics, and
% scientific computing. There have been many solutions in both
% general-purpose and domain-specific architectures accelerating GEMM
% with dense input matrices. However, matrices in workloads are not
% always dense. In fact, many traditional and emerging workload domains
% such as neural network models, graph analytics, and scientific
% simulations operate on sparse matrices where the majority of values are
% zeros. Sparse matrices are typically stored in compact formats with
% metadata indicating positions of non-zero values, which makes them
% incompatible with built-in dense matrix engines without first
% uncompressing the matrices. Previous work has proposed several ISA
% extensions on CPUs to accelerate sparse computations. In this work,
% rather than designing a completely new ISA extension for sparse
% computations on CPUs, we propose SparseZipper that enhances the
% existing matrix extensions specialized for dense GEMM to accelerate
% sparse GEMM (SpGEMM) through the evolutionary specialization approach.
% SparseZipper targets the key bottleneck, which is merging partial
% sparse vectors, streams of key-value pairs, in a conventional SpGEMM
% algorithm for data-parallel architectures. At the core of SparseZipper
% is its ability to efficiently merge such streams in parallel by
% leveraging in-place matrix registers to store parts of concurrent
% streams and built-in systolic array to merge those streams together. To
% facilitate that merge operation, we propose a minimal set of additional
% architectural states to keep track of active streams and matrix
% instructions to move streams between matrix registers and memory. Our
% performance evaluation shows SparseZipper achieves 5.98$\times$ and
% 2.61$\times$ speedup over a scalar hash-based implementation of SpGEMM
% and a vectorized SpGEMM version respectively.

% The importance of general matrix multiply (GEMM) has led to an
% emergence of matrix extensions in contemporary ISAs (e.g., Intel, Arm,
% and RISC-V) and systolic-array-based micro-architectures integrated
% into modern CPUs to accelerate multiplying two dense matrices. However,
% matrices in emerging workloads are not always dense, and sparse
% matrices where the majority of values are zeros are becoming more
% common. The existing matrix extensions and micro-architectures are not
% efficient for processing highly sparse matrices due to significant
% amount of wasted computation in multiplying with zeros and their
% incompatibility with sparse matrix formats. This work proposes
% SparseZipper that enhances existing matrix extensions and
% systolic-array-based micro-architectures specialized for dense-dense
% GEMM to accelerate sparse-sparse GEMM operating on highly sparse
% matrices with unstructured sparsity structures. Our performance evaluation
% shows SparseZipper achieves 5.98$\times$ and 2.61$\times$ speedup over
% a scalar hash-based implementation of SpGEMM and a vectorized SpGEMM
% version respectively. Our component-level area evaluation shows
% SparseZipper increases the area of a baseline 16$\times$16 systolic
% array by only 12.72\%. This overhead would be much lower when
% considering the whole system including a processor and its caches.

\begin{abstract}
  The importance of general matrix multiplication (GEMM) is motivating
  new instruction set extensions for multiplying dense matrices in almost
  all contemporary ISAs, and these extensions are often implemented using
  high-performance systolic arrays. However, matrices in emerging
  workloads are not always dense, and sparse matrices where the vast
  majority of values are zeros are becoming more common. Existing matrix
  extensions and micro-architectures cannot efficiently process highly
  sparse matrices due to two reasons: (1)~wasted work when one or both
  input values are zero; and (2)~incompatibility with sparse matrix
  formats. This work proposes SparseZipper that minimally modifies
  existing matrix extensions and systolic-array-based micro-architectures
  specialized for dense-dense GEMM to accelerate sparse-sparse GEMM
  operating on highly sparse matrices with unstructured sparsity structures.
  Our performance evaluation shows SparseZipper achieves 5.98$\times$ and
  2.61$\times$ speedup over a scalar hash-based implementation of SpGEMM
  and a state-of-the-art vectorized SpGEMM version, respectively. Our
  component-level area evaluation shows SparseZipper increases the area
  of a baseline 16$\times$16 systolic array by only 12.7\% resulting in
  an area overhead for an entire system-on-chip of just a few percent.
\end{abstract}


}

\keywords{Lossy compression, contour tree, topology preservation, topological data analysis, topology in visualization}

\teaser{
\centering
  \begin{subfigure}[b]{0.02\textwidth}
    \raisebox{\height}{\includegraphics[angle=90,width=\textwidth]{fig-colorBar.png}}
  \end{subfigure}
  \begin{subfigure}[b]{0.97\textwidth}
    \includegraphics[width=\textwidth]{fig-newTeaserIsabel.png}
  \end{subfigure}
\caption{Visualizing the Isabel dataset compressed with classic TTHRESH and Augmented TTHRESH compressors with topological controls. 
Left: input data visualized with critical points of the contour tree; maximum are in red, minimum are in dark blue, 1-saddles are in light blue, and 2-saddles are in light orange.  
Middle: reconstructed data using TTHRESH, labeled with compression ratio, PSNR, and false cases (false positives, false negative, false types).
Right: reconstructed data using Augmented TTHRESH along with compression ratio and PSNR. 
We also provide zoomed-in views that highlight the connectivity among critical points of the contour trees.}
\label{fig:teaser}
}

%-------------------------
\begin{document}

\maketitle

%-------------------------
% !TEX root = ../main.tex

Large Language Models (LLMs) have shown remarkable capabilities on numerous tasks in Natural Language Processing (NLP), 
ranging from language understanding to generation \cite{bubeck2023sparks, achiam2023gpt,team2023gemini, dubey2024llama}. The huge success of LLMs comes with important challenges to deploy them due to their massive size and computational costs. For instance,  Llama-3-405B \cite{dubey2024llama} requires 780GB of storage in half precision (FP16) and hence multiple high-end GPUs are needed just for inference. \textit{Model compression} has emerged as an important line of research to reduce the costs associated with deploying these foundation models. In particular, neural network pruning \cite{obd, hassibi1992second, benbaki2023fast}, where model weights are made to be sparse after training, has garnered significant attention. Different sparsity structures (Structured, Semi-Structured and Unstructured) obtained after neural network pruning result in different acceleration schemes. \textit{Structured pruning} removes entire structures such as channels, filters, or attention heads \cite{lebedev2016fast,wen2016learning,voita2019analyzing,el2022data} and readily results in acceleration as model weights dimensions are reduced. \textit{Semi-Structured pruning}, also known as, N:M sparsity \cite{zhou2021learning} requires that at most $N$ out of $M$ consecutive elements are non-zero elements. Modern NVIDIA GPUs provide support for 2:4 sparsity acceleration. \textit{Unstructured pruning} removes individual weights \cite{han2015learning, guo2016dynamic} from the model's weights and requires specialized hardware for acceleration. For instance, DeepSparse \cite{kurtic2022optimal, pmlr-v119-kurtz20a, DBLP:journals/corr/abs-2111-13445} provide CPU inference acceleration for unstructured sparsity.\\
Specializing to LLMs, one-shot pruning~\cite{meng2024alps, frantar2023sparsegpt, sun2023simple, zhang2023dynamic}, where one does a single forward pass on a small amount of calibration data, and prunes the model without expensive fine-tuning/retraining, is of particular interest. This setup requires less hardware requirements. For instance, \citet{meng2024alps} show how to prune an OPT-30B \cite{opt} using a single consumer-level V100 GPU with 32GB of CUDA memory, whereas full fine-tuning of such model using Adam \cite{kingma2014adam} at half-precision requires more than 220GB of CUDA memory.

Although one-shot pruning has desirable computational properties, it can degrade models' predictive and generative performance. To this end, recent work has studied extensions of model pruning to achieve smaller utility drop of model performance from compression. 
% Multiple one-shot methods have been developed in quantization \cite{frantar2022gptq, frantar2023sparsegpt, lin2024awq, behdin2023quantease, dettmers2023spqr} and neural network pruning \cite{frantar2023sparsegpt, meng2024alps, zhang2024oats}, which is closer to this paper's line of research. These one-shot methods do not require retraining--which is extremely expensive for models of the size of Llama-3-405B-- and work as resource-saving techniques that retain the model's performance. 

An interesting compression mechanism in the field of \textit{model compression} is the Sparse plus Low-Rank Matrix-Decomposition problem which aims to approximate model's weights by a sparse component plus a low-rank component~\cite{hintermuller2015robust, candes2011robust, lin2011linearized, 5394889, zhou2011godec, JMLR:v24:21-1130, NIPS2014_443cb001, yu2017compressing, li2023losparse}. Specializing to LLMs,~\citet{zhang2024oats} propose OATS 
%that addresses this type of %compression and 
that outperforms pruning methods for the same compression ratio (number of non-zero elements) on a wide range of LLM evaluation benchmarks (e.g. perplexity in Language generation). 

OATS \cite{zhang2024oats} is however a matrix decomposition algorithm inspired from a pruning algorithm Wanda \cite{sun2023simple}. Wanda has been designed as a relaxation/approximation of another state-of-the-art pruning algorithm SparseGPT \cite{frantar2023sparsegpt}. While Wanda has been found to be extremely useful and efficient in practice, recent work \cite{meng2024alps} show results where Wanda fails for high-sparsity regimes. In this paper, we provide a unified optimization framework to decompose pre-trained model weights into sparse plus low-rank components based on a layer-wise loss function. Our framework is modular and can incorporate different pruning and matrix-decomposition algorithms (developed independently in different contexts).
%under the umbrella of the local %layer-wise reconstruction error; 
Similar to~\cite{meng2024alps} we observe that our optimization-based framework results in models with better model utility-compression tradeoffs. The difference is particularly pronounced for higher compression regimes. 
%especially for higher compression %budgets, where SOTA methods 
% Our numerical results also show similar findings to \citet{meng2024alps} where high-sparsity significantly degrades the performance of approximation-based optimization methods like OATS.

Concurrently, in a different and complementary line of work,~\citet{mozaffari2024slope} have open-sourced highly-specialized CUDA kernels designed for N:M sparse \cite{zhou2021learning} plus low-rank matrix decompositions that result in significant acceleration and memory reduction for the pre-training of LLMs.
We note that our focus here is on improved algorithms for one-shot sparse plus low-rank matrix decompositions for foundation models with billions of parameters which is different from the work of \citet{mozaffari2024slope} that focuses on accelerating the pre-training of LLMs. The designed CUDA kernels \cite{mozaffari2024slope} can be exploited in our setting for faster acceleration and reduced memory footprint during inference.





% \textbf{Summary of approach and contributions:} We propose \ourmethod: an accurate and scalable framework for Sparse plus Low-Rank Matrix Decomposition for LLMs. Following the previous work on one-shot pruning and model compression, we pursue a layerwise approach. In particular, the reconstruction error resulting from compression in the output of each layer is minimized, under the compression constraints (i.e., sparsity and low-rank constraints).

\textbf{Summary of approach.\,\,\,\,} Our framework is coined \ourframework: \underline{H}ardware-\underline{A}ware (Semi-\underline{S}tructured) \underline{S}parse plus \underline{L}ow-rank \underline{E}fficient \& approximation-\underline{free} matrix decomposition for foundation models.

Hardware-aware refers to the fact that we mostly focus on a N:M sparse \cite{zhou2021learning} plus low-rank decomposition, for which acceleration on GPUs is possible, although \ourframework supports any type of sparsity pattern (unstructured, semi-structured, structured) in the sparsity constraint. Approximation-free refers to the fact that we directly minimize the local layer-wise reconstruction error introduced in \cref{eq:matrix-decomposition}, whereas we show prior work minimizes an approximation of this objective.

%Our unified framework introduces a well-posed 
%%As a part of our proposed framework, we consider an 
%%optimization form
We formulate the compression/decomposition task as a clear optimization problem; we minimize a local layer-wise reconstruction objective where the weights are given by the sum of a sparse and low-rank component. 
%%%of dense model weights under the  
%This optimization problem is decoupled into a sparse minimization subproblem and a low-rank minimization subproblem. 
We propose an efficient Alternating-Minimization approach that scales to models with billions of parameters relying on 
two key components: one involving sparse minimization (weight sparsity) and the other involving a low-rank optimization. 
For each of these subproblems 
we discuss how approximations to the optimization task can retrieve prior algorithms.
%the introduced subproblems, 
%we consider approximations to the minimization objective and retrieve different algorithms from related works given different %approximations.

% We provide an efficient and scalable algorithm based on Alternating-Minimization that does not rely on any approximation at the objective minimization level. 
% While \ourframework supports any sparsity pattern (unstructured, semi-structured, structured) in the sparsity constraint, we mostly focus on N:M sparsity \cite{zhou2021learning}, to make the decomposition Hardware-aware, as \citet{mozaffari2024slope} show how to get acceleration on modern GPUs for N:M sparse plus low-rank decomposition.

We note that \ourframework~differs from prior one-shot (sparse) pruning methods~\cite{frantar2023sparsegpt, meng2024alps, benbaki2023fast} as we seek a sparse plus low-rank decompositon of weights.
%%%%%introducing the low-rank component. 
Additionally, it differs from prior one-shot sparse plus low-rank matrix decomposition methods~\cite{zhang2024oats}
%by considering an approximation-free minimization approach of the 
as we directly minimize the local layer-wise reconstruction objective introduced in \cref{eq:matrix-decomposition}.

Our main \textbf{contributions} can be summarized as follows.
\begin{compactitem}
    \item We introduce \ourframework a unified one-shot LLM compression framework that scales to models with billions of parameters where we directly minimize the local layer-wise reconstruction error subject to  a sparse plus low-rank matrix decomposition of the pre-trained dense weights. 
    %    formulates a sparse plus low-rank matrix decomposition as an optimization problem with a local layer-wise reconstruction objective. We discuss approximations of this objective and show that OATS a popular method is recovered in a particular approximation.

    
    \item \ourframework uses an Alternating-Minimization approach that iteratively minimizes a Sparse and a Low-Rank component. \ourframework uses a given pruning method as a plug-in for the subproblem pertaining to the sparse component. Additionally, it uses Gradient-Descent type methods for the subproblem pertaining to the Low-Rank component.
    
    % \item In the subproblem pertaining to the sparse component, a rewrite of the optimization formulation shows that one can use any pruning algorithm, that minimizes the layer-wise reconstruction error, as a plug-in to sparsify the weights. We choose to show results for the algorithm SparseGPT.
    
    % In this pruning subproblem, we also enhance the performance of \ourmethod by exploiting the invariance of the Hessian--of the layer-wise reconstruction error--in each subproblem of the Alternating Minimization procedure, for a given layer. In particular, we use a pre-processing step that computes and stores the Hessian inverse--of the objective--, which is then passed to the deployed pruning algorithm (e.g. SparseGPT). 
    % \item In the subproblem  pertaining to the Low-Rank component, we give a theoretical closed form solution to the subproblem.
    % which does not scale to problems with billions of parameters. 
    % We also present a more tractable first-order optimization method for a reparametrization of the the low-rank problem, which is scalable to models with billions of parameters.
    
    % as $\bfUVt$ and use first-order optimization methods to minimize the layer-wise reconstruction objective.

    \item We discuss how special cases of our framework relying on specific approximations of the objective retrieve popular methods such as OATS, Wanda and MP --- \cite{zhang2024oats, sun2023simple,han2015learning, sze2020efficient}. This provides valuable insights into the underlying connections across different methods. 

    \item \ourframework improves upon state-of-the-art methods for one-shot sparse plus low-rank matrix decomposition. 
    For the Llama3-8B model with a 2:4 sparsity component plus a 64-rank component decomposition, \ourframework reduces the test perplexity by $12\%$ for the WikiText-2 dataset and reduces the gap (compared to the dense model) of the average of eight popular zero-shot tasks by $15\%$ compared to existing methods.
\end{compactitem}




\section{Related Work}
\label{sec:related-work}

We give a brief review of data compression for volumetric data. 
We then discuss the use of contour trees in topological data analysis, followed by related work for topology-preserving compression techniques.   

\para{Lossy compression.} 
Lossless compression techniques allow the original data to be perfectly reconstructed, 
but they usually suffer from limited compression ratios (less than $2\times$ according to~\cite{son2014data}) in scientific data and thus are not practical. 
Lossy compression is an alternative way to reduce the unprecedented  size of scientific data. 
Traditional lossy techniques such as JPEG/JPEG2000 leverage wavelet theories and bit plane encoding to compress image data, but they are not adept at dealing with multidimensional scientific data in floating-point format. 
Recently, there has been an increasing trend to leverage deep learning techniques, such as the autoencoder~\cite{le2023hierarchical} and implicit neural representation (INR)~\cite{lu2021compressive}, for data compression.
An autoencoder is a neural network composed of two components: an encoder and a decoder. 
The encoder is trained to produce low-dimensional representations of the input data, whereas the decoder is trained to reconstruct the original input data from the output of the encoder. 
An INR model trains a small neural network that can be used to recreate the ground truth. 
The neural network itself is shipped as a compressed file, and to decompress it, one must simply evaluate the network on an appropriate input. 
One notable INR model for volumetric scalar fields is Neurcomp~\cite{lu2021compressive}.
Recently, spatial super-resolution (SSR) models have employed neural networks to accurately upscale low-resolution representation of data as a form of interpolation. 
Several volumetric scalar field compressors incorporate SSR models, such as SSR-TVD~\cite{han2020ssr} and the deep hierarchical model~\cite{wurster2022deep}.
Unfortunately, these general lossy techniques lack precise error control on the data, which limits their use on scientific data.


Error-controlled lossy compressors~\cite{lindstrom2014fixed,ballester2019tthresh,zhao2021optimizing,lakshminarasimhan2013isabela} have been proposed and leveraged by the scientific computing community to reduce the data size while controlling the distortion in the decompressed data. 
In general, such compressors can be categorized into transform-based and prediction-based. 
Transform-based lossy compressors rely on domain transforms for data decorrelation. 
For instance, ZFP~\cite{lindstrom2014fixed} divides data into small blocks and then compresses each block independently. The compression procedure inside each block includes exponent alignment for fixed point conversion, a near-orthogonal domain transform, and embedded encoding. 
TTHRESH~\cite{ballester2019tthresh} is another transform-based compressor that leverages singular value decomposition (SVD) to improve the decorrelation efficiency for high-dimensional data.

Prediction-based compressors employ prediction methods such as interpolation to approximate the ground truth. The differences between original and predicted data are quantized and then encoded using entropy encoding and lossless techniques. 
ISABELA~\cite{lakshminarasimhan2013isabela}, as one of the pioneering error-controlled prediction-based compressors, uses B-splines to predict data. 
SZ3~\cite{liang2022sz3,zhao2021optimizing,liang2018error}, the most recent general release in the SZ compressor family, uses a combination of a Lorenzo predictor~\cite{ibarria2003out}, cubic spline interpolation, and linear interpolation. 
In addition, AE-SZ~\cite{liu2021exploring} is proposed as a variation of SZ that incorporates autoencoders in the prediction pipeline.

\para{Contour trees.} Our augmented compressors aim to preserve the contour tree of an input scalar field. 
Contour trees capture the relationships among contours of scalar fields. 
They have been used to support data analysis and visualization tasks across diverse disciplines, such as astronomy \cite{rosen2021using}, fluid dynamics \cite{aydogan2014characterization}, and medicine \cite{aydogan2013analysis, wang2018fast, szymczak2010categorical}. 
They have also been incorporated into algorithms in computer vision \cite{mizuta2005description} and visualization \cite{zhou2009automatic, kopp2022temporal} for interactive exploration of contours.  

\para{Topology-preserving compression.} 
To the best of our knowledge, only three compressors have been developed for scalar field compression with topological guarantees. 
The first compressor was developed by Soler et al. \cite{soler2018topologically}. We shall refer to it as TopoQZ. 
TopoQZ allows the user to specify a single parameter $\varepsilon$. 
It preserves all critical point pairs with finite persistence greater than $\varepsilon$ and eliminates all critical points with persistence less than $\varepsilon$. 
TopoQZ is not designed to perfectly preserve the contour tree. Therefore, the locations of preserved critical points may shift slightly during compression, and the connectivity of the critical points in the contour tree may be altered. 
TopoQZ can also guarantee that the reconstructed values differ from the ground truth at most by a user-specified error bound $\xi$. It is required that $\xi > \varepsilon$.
TopoQZ is currently implemented in the Topology Toolkit \cite{TiernyFavelierLevine2017, MasoodBudinFalk2021, leguillou_tvcg24}. That implementation couples TopoQZ with ZFP \cite{lindstrom2014fixed}, which improves the smoothness of the data but introduces additional pointwise error.

Another topology-preserving compressor is TopoSZ \cite{yan2023toposz}. 
TopoSZ modifies the classic SZ pipeline to perfectly preserve the contour tree of the ground truth data up to the persistence threshold of $\varepsilon$. 
That is, the contour tree of the output of TopoSZ will be equal to that of the ground truth after both datasets have been topologically simplified with a persistence threshold of $\varepsilon$. 
TopoSZ also allows the user to impose a strict error-bound $\xi$ (and allows $\xi \leq \varepsilon$). When compared with TopoQZ, TopoSZ yields generally higher compression ratios and reconstruction quality, although the algorithm takes longer to execute. 
Our general framework borrows some elements from the TopoSZ pipeline. However, our framework differs significantly from TopoSZ due to two technical innovations: progressive bound tightening and logarithmic-scaling quantization (see \cref{sec:method} for details).  

Most recently, Li et al. developed mSZ~\cite{li2024msz} that augments an existing lossy compressor to compress a 2D/3D scalar field while preserving its piecewise-linear (PL) Morse--Smale segmentation \cite{edelsbrunner2001hierarchical,edelsbrunner2003morse},~i.e.,~a partition of the data domain based on the Morse--Smale complex. 
In comparison to the contour tree that is based on the level sets of a scalar field, a Morse--Smale complex is a different topological descriptor based on the gradient behavior of a scalar field.   
Because our framework instead preserves contour trees and does not consider the gradients in its pipeline, mSZ is not directly comparable to our work. 

Finally, even though it does not preserve any common topological descriptor, cpSZ \cite{liang2022toward}---a variation of SZ---preserves the critical points of a vector field. cpSZ also introduces a log-scale quantization technique to store different error bounds for individual points.

\section{Technical Background}
\label{sec:background}

\subsection{Merge Tree and Contour Tree}
\label{sec:merge-and-contour-tree}
\para{Merge Tree.} 
Let $f:\X \rightarrow \R$ be a continuous scalar field defined on a simply connected domain $\X$. 
The \emph{sublevel set} of $f$ at a threshold $t \in \R$ is defined as 
$\X_t = f^{-1}(-\infty,t] := \{ x \in X \mid f(x) \leq t \}$. 
The \emph{merge tree} of $f$ tracks when (connected) components of $\X_t$ appear and merge as $t$ increases. 
$\X_t$ evolves from being an empty set to contain components surrounding various local minima; these components then merge into one other until eventually there is only a single component.
Leaves of the merge tree correspond to local minima, and interior nodes correspond to saddles where components merge. 
\Cref{fig:contour-tree}(A) and (C) visualize a scalar field and its merge tree.
Formally, we define an equivalence relation $\sim$ on $\X$. We say that $x \sim y$ if and only if $f(x) = f(y)=t$ and $x$ belongs to the same component of $\X_t$ as $y$. The merge tree of $f$ is defined by the quotient space $\X/{\sim}$. 

The merge tree of $f$ defined above is sometimes referred to as the \emph{join tree}, whereas the merge tree of $-f$ is called a \emph{split tree} (see~\cref{fig:contour-tree}(B) for its visualization). The merge tree naturally induces a segmentation of the domain. Let $\phi$ be the canonical map that maps each $x \in \X$ to its equivalence class $[x]$ under $\sim$. Then for each edge $e$ of the contour tree, $\phi^{-1}(e)$ is a monotonic region in $\X$. The inverse image of each edge partitions the domain, which is called the merge-tree-induced segmentation. See \cref{fig:contour-tree}(D) (cf.,~(E)) for a split-tree-induced segmentation.

\para{Contour Tree.} 
The \emph{level set} of $f$ at a threshold $t \in \R$ is $f^{-1}(t):= \{ x \in X \mid f(x) = t \}$.
Each component of $f^{-1}(t)$ is called a \emph{contour}. 
A \emph{contour tree} tracks the relations among contours as $t$ increases. 
Analogous to a merge tree, as $t$ increases, components of $f^{-1}(t)$ appear at local minima, disappear at local maxima, and join or split at saddles. 
Formally, we define another equivalence relation $\sim$ on $\X$, where $x \sim y$ if and only if $f(x) = f(y)$ and $x$ belongs to the same contour as $y$. 
The contour tree is the quotient space $T=T(\X,f):= \X / \sim$. 
There is a 1-1 correspondence between the local extrema of $\X$ and the leaves of $T$. The interior nodes of $T$ correspond to a subset of the saddle points of $\X$ \cite{reeb1946points}. \cref{fig:contour-tree}(F) visualizes the contour tree. Similar to the merge tree, we can define a contour-tree-induced segmentation.

The classic algorithm for computing the contour tree~\cite{carr2003computing} combines join and split trees together to form a contour tree, and comes with a state-of-the-art multi-core implementation~\cite{gueunet2017task}.
In this paper, we compute the contour tree of the input data to determine the initial point-wise error bound. We use the algorithm of Gueunet et al.~\cite{gueunet2017task} (with an in-house implementation) to compute join and split trees, and the algorithm of Carr et al.~\cite{carr2003computing} to combine them into the contour tree. 

\begin{figure}[!ht]
    \centering
    \includegraphics[width=\linewidth]{fig-ct_example.png}
    \vspace{-6mm}
    \caption{(A) Visualizing the graph of a 2D scalar field $f$ defined on a square domain. Local minima are in blue, saddles are in white, and local maxima are in red. (B) Split tree of f.  (C) Merge tree of $f$ (a.k.a.,~join tree). (D) The domain is colored according to split-tree-induced segmentation. (E) Split tree is colored according to the segmentation in (D). (F) Contour tree of $f$. (G) The graph of $f$ after its persistence simplification where a peak is removed. (H) Split tree of $f$ after persistence simplification. Edge $ab$ is removed. (F) Contour tree of $f$ after the merge tree in (B) has been simplified. }
    \label{fig:contour-tree}
    \vspace{-6mm}
\end{figure}

The algorithm by Gueunet et al.~\cite{gueunet2017task} constructs the merge tree one edge at a time. First, starting from each minimum $m$, the algorithm visits points surrounding $m$ in an increasing order until a saddle $s$ is reached. The edge $ms$ is discovered and added to the merge tree; in other words, $m$ is \emph{grown} during this process.  
For each saddle $s$, once all edges that terminate at $s$ have been discovered, $s$ is grown until some new saddle $s'$ is reached, and the edge $ss'$ is discovered and added to the merge tree. This process repeats until the merge tree has been completely discovered. 
To grow from a minimum or a saddle, the points surrounding that minimum or saddle are stored in a heap to ensure that they are processed in an increasing order. The use of heaps presents the main performance bottleneck of the algorithm.

\para{Persistence simplification.} 
In practice, real-world data typically contain noise that creates many small branches in the merge or contour tree, where persistence simplification~\cite{edelsbrunner2001hierarchical} can be used to eliminate these small branches and thereby separate topological features from noise. 
In the context of merge trees, ordinary persistent homology pairs a local extremum (i.e.,~a peak or a valley in \cref{fig:contour-tree}(A)) with a nearby saddle and assigns the pair a value of \emph{persistence}, which describes the scale at which the pair disappears via a perturbation to the function. The persistence is equal to the absolute difference in function value between an extremum and its paired saddle. A function $f$ is simplified by perturbing the function values in order to cancel pairs of critical points below a certain persistence threshold $\varepsilon$. See \cref{fig:contour-tree}(G) for an example. The cancellation of one pair of critical points of $f$ corresponds to a branch being removed from the merge tree, giving rise to its simplification.
The simplified contour tree is more complex \cite{hristov2021w}, but it can be computed by combining the simplified join and split trees. 

In~\cref{fig:contour-tree}, (G) depicts the graph of a function after persistence simplification. For the split tree shown in (B), assuming a persistence threshold of $\varepsilon \geq |f(a)-f(b)|$, the edge $ab$ will be removed after a persistence simplification at $\varepsilon$. As a result, node $c$ is directly connected to node $d$ in the split tree (H), and the join tree in (C) remains unchanged. (I) depicts the contour tree produced by combining the simplified split tree in (H) with the join tree in (C).

\subsection{A Review on TopoSZ}
\label{sec:TopoSZ}

Our framework builds upon a few ingredients from TopoSZ~\cite{yan2023toposz}. 
TopoSZ, in turn, modifies the pipeline from the error-bounded lossy compressor SZ version 1.4 \cite{tao2017significantly}. 

Let $f$ represent the input scalar field, and $f'$ be the reconstructed scalar field (after compression and decompression). 
Let $T$ be the contour tree of $f$ and $T_\varepsilon$ the persistence simplified contour tree at a threshold of $\varepsilon$. Let $T'$ and $T'_\varepsilon$ be defined analogously for $f'$.

SZ1.4 allows the user to specify $\xi$, a pointwise error bound during compression. In turn, there are two user-defined parameters in TopoSZ: a persistence threshold $\varepsilon$, and a pointwise error bound $\xi$. Unlike TopoQZ, TopoSZ does not require that $\varepsilon < \xi$. 
TopoSZ guarantees the preservation of the persistence simplified contour tree during compression while maintaining the pointwise error bound. 
That is, it guarantees that $T_\varepsilon = T'_\varepsilon$, and $|f(x)-f'(x)| \leq \xi$ for each $x \in \X$.

\para{Linear-scaling quantization.} 
SZ version 1.4 introduces a linear-scaling quantization technique to ensure that a strict absolute error bound $\xi$ is maintained. This technique is implemented in TopoSZ.

For each point $x \in \X$ with a ground truth value $f(x)$, an initial guess for its value $g(x)$ (e.g., from a Lorenzo predictor; see below) is shifted by an integer multiple of $2\xi$ to obtain a new value $f'(x)$ such that $|f'(x) - f(x)| \leq \xi$. 

This process can be conceptualized as follows: divide the real line into intervals of length $2\xi$, where one interval is centered on $g(x)$. 
The compressor then calculates how many intervals to shift $g(x)$, so that it can assign a value to $f'(x)$ that is a distance less than $\xi$ from $f(x)$. By construction, if $f(x)$ lies in an interval of length $2\xi$ centered on $f'(x)$, then $|f(x)-f'(x)| \leq \xi$. 
This process is illustrated in \cref{fig:linear-scaling-quantization}.
\begin{figure}[!ht]
  \vspace{-3mm}
  \centering
  \includegraphics[width=\columnwidth]{fig-linear-scaling-quantization.pdf}
  \vspace{-8mm}
  \caption{A standard implementation of a linear-scaling quantization.}
  \label{fig:linear-scaling-quantization}
  \vspace{-4mm}
\end{figure}
\noindent Following this construction, each $x \in \X$ is assigned an integer $n_x$ (corresponding to how many intervals $g(x)$ is shifted) such that $f'(x) = g(x) + 2\xi n_x$. These quantization numbers $\{n_x\}$ are encoded and stored in the compressed file.

If the distribution of $\{n_x\}$ has low entropy (e.g.,~if $\{n_x\}$ are mostly zeros), then the quantization numbers can be compressed to small size using an entropy-based compression algorithm, such as Huffman coding. 
More accurate predictions $g(x)$ generally lead to $\{n_x\}$ with lower entropy.

\para{False Cases.} Yan et al.~\cite{yan2023toposz} introduced three types of false cases to quantify the level of contour tree preservation: false positives, false negatives, and false types, which are illustrated in \cref{fig:false-cases}.
A false positive occurs when a new edge appears in the contour tree of the reconstructed data that does not exist in the same position of the contour tree of the original data. A false negative occurs when an edge of the contour tree from the original data is missing from the contour tree of the reconstructed data. 
A false type occurs when the critical type (maximum, minimum, saddle) of one or both endpoints of an edge of the contour tree does not match between the original and reconstructed data.
TopoSZ focuses on false cases involving extremum-saddle pairs and its algorithm terminates when there are no such false cases. 

\begin{figure}[!ht]
    \vspace{-2mm}
    \centering
    \includegraphics[width=\linewidth]{fig-false-cases.pdf}
    \vspace{-6mm}
    \caption{Three types of false cases. (A) The original contour tree. (B) A false positive: an extra edge is added. (C) A false negative: an edge is missing. (D) A false type: an edge contains a critical point as its endpoint that changes its type.}
    \label{fig:false-cases}
    \vspace{-4mm}
\end{figure}


\para{TopoSZ Pipeline.} The TopoSZ pipeline is as follows:

\para{\underline{Step 1: Upper and lower bound calculation.}}
TopoSZ first applies persistence simplification to $f$ in order to calculate $T_\varepsilon$. For each point $x \in \X$, a lower bound $L(x)$ and an upper bound $U(x)$ are assigned to $x$ according to the contour-tree-induced segmentation.
If $x$ belongs to the segmented region corresponding to edge $ab \in T_\varepsilon$, then $L(x) = \min(f(a),f(b))$ and $U(x) = \max(f(a),f(b))$. The nodes of the contour tree are stored losslessly.

\para{\underline{Step 2: Prediction.}} 
TopoSZ uses a Lorenzo predictor \cite{ibarria2003out} to predict the values of each data point. For each point $x$, the Lorenzo predictor predicts $f(x)$ as a weighted sum of the values from previously predicted points $x'$ satisfying $\|x-x'\|_\infty = 1$. The weights are fixed, and are chosen such that quadratic functions will be perfectly predicted.

\para{\underline{Step 3: Linear-scaling quantization.}} TopoSZ uses linear-scaling quantization with a decreased interval size to ensure that the pointwise upper and lower bounds, as well as the global error bound $\xi$, are maintained for each $x \in \X$. For any $x \in \X$ where no possible quantization code $n_x$ satisfies these conditions, $f(x)$ is stored losslessly.

\para{\underline{Step 4: Iterative upper and lower bound tightening.}} 
If the results from Step 3 do not perfectly preserve the contour tree, that is, if there are false cases presented in the reconstructed data, then the upper and lower bounds are tightened around points corresponding to those false edges, and then Step 3 is repeated. This cycle repeats until there are no false cases. 
%The specifics of this step are reviewed in \cref{sec:topoSZ-detail}.

\para{\underline{Step 5: Lossless compression.}} The numbers from linear-scaling quantization are encoded using Huffman Coding. The relevant information is then stored in a binary file that is further compressed using ZSTD~\cite{collet2018zstandard}.
\section{NumericBench Generation}
In this section, we present our created  NumericBench, which is specifically designed to evaluate fundamental numerical capabilities of LLMs. 
NumericBench consists of diverse datasets and tasks, 
enabling a systematic and comprehensive evaluation.
We discuss the datasets included in NumericBench, the key abilities it evaluates, and the methodology for benchmark generation.

\begin{table*}[t]
	\caption{NumericBench statistics. R: contextual retrieval, C: comparison, S: summary, L: logical reasoning. The token count is calculated based on tiktoken, which is the tokenizer used by Llama3~\cite{grattafiori2024llama3herdmodels}. The sentences used for token calculation include both the context and the question.}
	\centering
	\renewcommand{\arraystretch}{1.15} % 设置行间距为默认的 1.15 倍
	\setlength{\tabcolsep}{1.5pt} % 将列间距设置为 1pt
\resizebox{\textwidth}{!}{
	\begin{tabular}{c|c|c|c|c}
		\toprule
		\textbf{Data} & \textbf{Format} & \textbf{Questions} & \textbf{\# Instance} & \textbf{Avg Token} \\ \midrule
		
		\multirow{3}{*}{} 
		& \multirow{3}{*}{} 
		& \begin{tabular}[c]{@{}c@{}}R: What is the index of the first occurrence\\ of the number -3095 in the list?\end{tabular} 
		& 1000 & 3704.23 \\ \cline{3-5}
		
		\textbf{\begin{tabular}[c]{@{}c@{}}Number\\ List\end{tabular}}
		& $[69, -1, 6.1, \ldots, 5.7]$
		& \begin{tabular}[c]{@{}c@{}}C: Which index holds the smallest number\\
			 in the list between the indices 20 and 80?\end{tabular} 
		& 1000 & 3685.57  \\ \cline{3-5}
		
		& & \begin{tabular}[c]{@{}c@{}}S: What is the average of the index of\\
			 top 30 largest numbers in the list?\end{tabular} 
		& 1000 & 3654.78 \\ \midrule
		
		\multirow{3}{*}{} 
		& \multirow{3}{*}{
		\begin{tabular}[c]{@{}c@{}}
			\{date: 2024-06-19,\\
			close\_price: 9.79, \\
			open\_price: 9.4, \\
			\ldots \\
			PE\_ratio: 4.5416\}
		\end{tabular}
		} 
		& \begin{tabular}[c]{@{}c@{}}
			R: On which date did the close price\\
			 of stock firstly reach 61.76 yuan?
		\end{tabular}
		& 1000 & 27585.35 \\ \cline{3-5}
		
		\textbf{Stock}
		& 
		& \begin{tabular}[c]{@{}c@{}}
			C: Among the top-45 trading value days, which\\
			 date did the stock have the lowest close price?
		\end{tabular}
		 & 1000 & 27595.40 \\ \cline{3-5}
		
		& & \begin{tabular}[c]{@{}c@{}} 
			S: How many days had the close price higher than\\
			 the open price from 2024-07-31 to 2024-12-13?
		\end{tabular}	
		& 1000 & 27561.29 \\ \midrule
		
		\multirow{3}{*}{} 
		& \multirow{3}{*}{
		\begin{tabular}[c]{@{}c@{}}
			\{date: 2024-07-21,\\
			pressure\_msl: 999.96,\\
			dew\_point\_2m: 26.25,\\
			\ldots \\
			cloud\_cover: 61.5\}
		\end{tabular}
		} 
		& \begin{tabular}[c]{@{}c@{}} 
			R: On which date did the dew point temperature\\
			 at two meters firstly drop below 9.15°C?
		\end{tabular}
		& 1000 & 27359.26 \\ \cline{3-5}
		
		\textbf{Weather}
		& & \begin{tabular}[c]{@{}c@{}} 
			C: On which date did the MSL pressure reach its\\
			highest value when the cloud cover was below 9\%?
		\end{tabular}
		& 1000 & 27368.19 \\ \cline{3-5}
		
		& & \begin{tabular}[c]{@{}c@{}} 
			S: What was the average temperature at two meters\\
			when the relative humidity exceeded 78.56\%?
		\end{tabular}
		& 1000 & 27331.21 \\ \midrule
		
		\textbf{Sequence} 
		& $[0.34, 3, 6, \ldots, 111]$ 
		& L: What is the next number in the sequence? & 500 & 677.57 \\ \midrule
		
		\textbf{\begin{tabular}[c]{@{}c@{}}Arithmetic \\Operation\end{tabular}} 
		& \begin{tabular}[c]{@{}c@{}} 
		$a: 6.755,
		b: -1.225$
		\end{tabular}
		& \begin{tabular}[c]{@{}c@{}} 
		 $Q_{oper}$: What is the result of $a + b$?\\
		 $Q_{context}$: What is the result of $a $ plus $b$?
		 
		\end{tabular}
		& 12000 & 112.00 \\ \midrule
		
		\textbf{\begin{tabular}[c]{@{}c@{}}Mixed-number-string\\ Sequence\end{tabular}} 
		& \begin{tabular}[c]{@{}c@{}} 
		$effV2\ldots x98o7Lo$
		\end{tabular}
		& \begin{tabular}[c]{@{}c@{}} 
		How many numbers are there in the string? Note\\
		that a sequence like 'a243b' counts as a single number.
		\end{tabular}
		& 2000 & 196.53 \\ \bottomrule

	\end{tabular}
}
	\label{tab:data_stat}
	
\end{table*}

 

\subsection{Numeric Dataset Collection}
NumericBench offers a diverse collection of numerical datasets and questions designed to reflect real-world scenarios. 
This variety ensures that LLMs are thoroughly tested on their fundamental  abilities on numerical data.

\noindent\textbf{Number List Dataset.}
The synthetic number list dataset consists of simple collections of numerical values (integer and floats) 
presented as ordered or unordered lists.
Numbers in lists are one of the simplest and most fundamental data representations encountered in real-world scenarios.
Despite their simplicity, retrieving, indexing,  comparison, and summary on numbers can verify the fundamental numerical ability of LLMs. 
This dataset serves as a fundamental dataset of how well LLMs understand numerical values as discrete entities.



\noindent\textbf{Stock Dataset.}
The time-series  stock dataset is crawled from Eastmoney website~\cite{eastmoney}, 
which has eighteen attributes, such as stock close prices, open price,  trading volumes, and price-earnings ratio, over time.
Stock  data reflects dynamic, real-world numerical reasoning challenges that involve trend analysis, comparison, and decision-making under uncertainty,  representing real-world financial workflows.
 




\noindent\textbf{Weather Dataset.}
The weather dataset is crawled from Open-Meteo python API~\citep{openmeteo}, which includes data related to weather metrics, such as temperature, precipitation, humidity, and wind speed. 
The data is presented across various longitude and latitude.
 
 




\noindent\textbf{Numeric Sequence  Dataset.}
The synthetic numeric sequence dataset comprises sequences of numbers generated by arithmetic or geometric progression, complex patterns, or noisy inputs. 
Tasks require identifying patterns, predicting the next number, or reasoning about relationships between numbers.
Numerical sequences test the logical reasoning capabilities of LLMs, requiring pattern recognition and multi-step reasoning. This dataset introduces structured challenges that are common in mathematics and algorithmic reasoning.


 
\noindent\textbf{Arithmetic Operation Dataset.}
The dataset comprises 12,000 pairs of simple numbers, each undergoing addition, subtraction, multiplication, and division operations. Each pair of numbers, $a$ and $b$, consists of $k$-digit integers with three decimal places, where $k \in \{1, 2, \cdots, 6\}$. 
For each value of $k$, there are 2,000 pairs, evenly distributed across the four basic operations (i.e, $+, -,  *, /$), with 500 pairs per operation. 
This dataset is to evaluate the fundamental mathematical operation capabilities of LLMs, simulating the majority of mathematical calculation requirements in real-world scenarios.

\noindent\textbf{Mixed-number-string Sequence Dataset.}
The dataset consists of alphanumeric strings of varying lengths $\{50, 100, 150, 200\}$, each containing a randomized mix of letters and digits. For each string length, 500 samples are generated, resulting in a total of 2,000 samples. Each sample includes a query asking for the count of contiguous numeric sequences within the string, where sequences like "a243b" count as a single number. This dataset is designed to assess the ability of LLMs to identify and count numeric sequences.
 







\subsection{Fundamental Numerical Ability}
NumericBench is designed to comprehensively evaluate six fundamental numerical reasoning abilities of LLMs, which is 
%These three fundamental abilities are 
essential for solving real-world numeric-related tasks.
%such as numeric data summary and financial price analysis.


\noindent\textbf{Contextual Retrieval Ability.}
Contextual retrieval ability evaluates how well LLMs can locate, extract, and identify specific numerical values or their positions within structured or unstructured data. 
This includes tasks like finding a specific number in a list, retrieving values , and indexing numbers based on their order.
For example, as shown in Table~\ref{tab:data_stat}, it evaluates LLMs on tasks such as retrieving stock prices and identifying key values within numerical lists or domain-specific data (e.g., stock market and weather-related information).
This ability is fundamental to numerical reasoning because it forms the foundation for higher-order tasks, such as comparison, aggregation, and logical reasoning. 
 
 



\noindent\textbf{Comparison Ability.}
Comparison ability evaluates how well LLMs can compare numerical values to determine relationships such as greater than, less than, or equal to, and identify trends or differences in datasets. 
Comparison is vital for logical reasoning and decision-making, as many real-world tasks depend on accurate numerical evaluation. 
For instance,  as shown in Table~\ref{tab:data_stat},   comparing prices is essential in stock  for assessing performance, while weather forecasting requires analysis of temperature or precipitation trends over time. 
 



\noindent\textbf{Summary Ability.}
Summary ability assesses the LLM’s capacity to aggregate numerical data into concise insights, such as calculating totals, averages, or other statistical metrics. 
Summarization is critical for condensing large datasets into actionable information, enabling decision-making based on aggregated insights rather than raw data. 
This ability is indispensable in domains like electricity usage analysis, where summarizing hourly or daily consumption helps forecast bills, in business reporting for aggregating sales and revenue data to evaluate performance, 
and in healthcare analytics to monitor trends in patient metrics over time.



\noindent\textbf{Logic Reasoning Ability.}
Logical Reasoning Ability measures the LLM’s ability to perform multi-step operations involving numerical data, 
such as recognizing patterns, inferring rules, and applying arithmetic or geometric reasoning to solve complex problems. Logical reasoning extends beyond simple numerical tasks and reflects the LLM’s capacity for deeper, structured thinking. 
This ability is crucial for algorithm design, where solving problems involving numeric sequences or patterns is essential, in scientific research for identifying relationships and correlations in data.

\noindent\textbf{Arithmetic Operation Ability.}
It reflects the LLM's capacity to perform mathematical calculations accurately. Such ability is essential for tasks involving numerical computations, such as  automated machine learning through LLMs.





\noindent\textbf{Number Recognition  Ability.}
This measures the LLM's proficiency in identifying and interpreting numerical information within a given context. It represents a fundamental requirement for handling numeric-based tasks effectively.




\subsection{NumericBench Generation}
We use the number list, stock, and weather datasets to evaluate the contextual retrieval, comparison, and summary abilities of LLMs. 
Specifically, for each ability and each dataset, we prepare a set of questions designed to assess the corresponding target ability.
As shown in Table~\ref{appx:number_question}, Table~\ref{appx:stock_question}, and Table~\ref{appx:weather_question} in Appendix, there are nine question sets in total, covering three abilities across three datasets. 
When evaluating a specific ability (e.g., contextual retrieval) on a specific dataset (e.g., stock data), we randomly select one question from the corresponding question set for each data instance (e.g., a stock instance) 
and manually label the answer. This approach enables us to generate question-answer pairs for each ability on the number list, stock, and weather datasets.

For arithmetic operations and number counting in the strings dataset, the question format is straightforward, as illustrated in Table~\ref{tab:data_stat}. These questions are designed to evaluate the basic arithmetic operation and number recognition abilities of LLMs.



\section{Experimental Results}
\label{sec:results}

\begin{figure*}[!ht]
    \centering
    \begin{subfigure}{0.02\textwidth}
    \raisebox{0.1\height}{\includegraphics[angle=90,width=\textwidth]{fig-colorBar}}
    \end{subfigure}
    \begin{subfigure}{0.92\textwidth}
    \includegraphics[width=\textwidth]{fig-vr_table_smaller.png}
    \end{subfigure}
    \vspace{-3mm}
    \caption{Scientific datasets compressed using different augmented compressors with topological controls. From left to right: the original input dataset, the reconstructed datasets from Augmented ZFP, Augmented SZ3, and Augmented TTHRESH, 
    respectively, that preserve the contour trees up to a persistence threshold $\varepsilon = 0.04$. From top to bottom: Tangaroa, Miranda, S3D datasets, respectively. We also display the PSNR and compression ratio next to each decompressed dataset.}    
    \label{fig:volume-render}
    \vspace{-6mm}
\end{figure*}

We provide an overview in~\cref{sec:results-overview}, describing the base compressors and datasets used in our experiments, highlighting the main takeaways, and introducing the evaluation metrics. 
We include compressor configurations and implementation details in~\cref{sec:configurations}. 
In \cref{sec:augmented-compressors} we describe the main utilities of our augmented compressors in preserving contour trees in the reconstructed data.
We evaluate a number of augmented compressors qualitatively and quantitatively, followed by a comparison against the state-of-the-art topology-preserving compressors in \cref{sec:compare-topology}.
We end this section with a run time analysis in \cref{sec:run-time}.

%---------------------------------------
\subsection{An Overview of Experiments}
\label{sec:results-overview}

We present a comparative analysis of five error-bounded lossy compressors augmented with our framework, including the classic compressors ZFP \cite{lindstrom2014fixed}, SZ3 \cite{liang2022sz3}, and TTHRESH \cite{ballester2019tthresh}, a custom-built cubic spline interpolation (CSI) model, and a deep learning-based compressor Neurcomp \cite{lu2021compressive}. 
We test these augmented compressors---denoted as Augmented ZFP, Augmented SZ3, and so on---against two state-of-the-art topology-preserving compressors, TopoSZ~\cite{soler2018topologically} and TopoQZ~\cite{yan2023toposz}. 

We test the five augmented compressors and two topology-preserving compressors on nine volumetric datasets from scientific simulations. The Nyx dataset is very topologically complex---its contour tree has over twenty million nodes---and it is included as a stress test. See \cref{tab:datasets} and \cref{sec:datasets} for details on these datasets.

We further conduct an ablation study demonstrating the effectiveness of logarithmic-scaling quantization and progressive error bound tightening. In every trial, logarithmic-scaling quantization leads to higher compression ratios, whereas progressive tightening results in faster compression times. We also analyze the individual effects of varying $\varepsilon$ and $\xi$; see~\cref{sec:other-experiments} for details on these experiments and the ablation study.

\begin{table}[!ht]
\scriptsize
\centering{
\begin{tabu}{c|*{2}{c}}
\toprule
\textbf{Dataset}  & \textbf{Dimension} & \textbf{Size (MB)}  \\ 
\midrule
QMCPACK      & $69 \times 69 \times 115$          & 4.4         \\
Tangaroa     & $300 \times 180 \times 200$         & 27.0        \\
Earthquake   & $175 \times 188 \times 50$          & 28.2        \\
Ionization   & $310 \times 128 \times 128$         & 40.6       \\
Isabel       & $500 \times 500 \times 90$          & 105.0     \\
Miranda      & $384 \times 384 \times 256$         & 302.0      \\
Nyx          & $512 \times 512 \times 512$         & 641.4      \\
S3D          & $500 \times 500 \times 500$         & 1000.0    \\
SCALE-LETKF  & $1200 \times 1200 \times 98$        & 1129.0    \\
\bottomrule
\end{tabu}
}
\vspace{-2mm}
\caption{Scientific datasets used for compression analysis.}
\label{tab:datasets}
\vspace{-4mm}
\end{table}

\para{Highlighted results.}
We highlight our experimental results below. 
\begin{itemize}[noitemsep,leftmargin=*]
\item Applying any of the five original base compressors to any of the nine datasets produces a large number of topological false cases in the reconstruction, even with a small pointwise error bound. On the other hand, augmenting any compressor with our general framework completely eliminates these false cases while maintaining a user-specified error bound (\cref{sec:augmented-compressors}).
\item Augmented TTHRESH and Augmented ZFP respectively yield the best compression ratios and run times among all the augmented compressors (\cref{sec:augmented-compressors}).
\item Our augmented compressors generally have a better trade off between bit-rate and reconstruction quality compared to TopoQZ and TopoSZ while taking similar or less time to run (\cref{sec:compare-topology}).
\item Our framework has a worst-case time complexity of $O(F h n \log n)$, where $h$ is the maximum height of the contour tree during tightening and $F$ is the number of false cases during computation. The majority of the compression time is spent on computing merge trees (\cref{sec:run-time}).
\end{itemize}

\para{Evaluation metrics.} 
We evaluate whether the contour tree has been perfectly preserved in the reconstructed (decompressed) data. 
We also evaluate the standard compression metrics of compression ratio, bit-rate, and peak signal-to-noise ratio (PSNR).
We further employ topology-based metrics of the bottleneck distances $d_B$~\cite{cohen2005stability} and the Wasserstein distances $d_W$ \cite[page 183]{edelsbrunner2022computational} to quantify the topological similarity between the original data and the reconstructed data. 
The evaluation metrics are described in detail in \cref{sec:evaluationMetrics}.

In general, higher values of PSNR indicate better reconstruction quality, and lower values of $d_B$ and $d_W$ indicate higher topological similarity. 
We measure the total compression time for each compressor, which includes (a) the total time to run the base compressor, and (b) the time to augment the output of the base compressor. We also measure decompression time for each compressor. We measure compression and decompression time for TopoSZ and TopoQZ as well. For our framework and TopoSZ, we decompress to RAW binary format. Because TopoQZ is integrated in the Topology Toolkit, an extension for ParaView, we decompress to VTK image format.

% ---------------------------------------
\subsection{Compressor Configurations and Implementation}
\label{sec:configurations}

In addition to augmenting the out-of-box base compressors SZ3, TTHRESH, ZFP, and Neurcomp, we implement and augment our own super-resolution compressor, a simple custom-built cubic spline interpolation (CSI) model.
It compresses a dataset by downsampling the data by a user-defined ratio in each direction (called a target scale factor) and uses a cubic spline interpolation technique for reconstruction that is similar to the one implemented in SZ3.
%We also considered the Sliced Wasserstein Autoencoder \cite{kolouri2018sliced} used in the AE-SZ compressor \cite{liu2021exploring}. 
%However, this model is excluded from our experiments since it performed significantly worse than the other compressors during initial tests.   

We compare our augmented compressors to TopoSZ and TopoQZ. We use the TopoQZ implementation in TTK~\cite{TiernyFavelierLevine2017}. 

Our general framework requires two user-defined parameters, a persistence threshold $\varepsilon$ and a global absolute pointwise error bound $\xi$.
$\varepsilon$ represents, as a percentage of the range, the level of persistence  simplification. 
For example, $\varepsilon = 0.01$ corresponds to a persistence simplification by $1\%$ of the range of the scalar function. Similarly, $\xi$ is the percentage of the range that will be used as an absolute error bound.

Each base compressor takes a number of intrinsic parameters in order to run. 
Both ZFP and SZ3 require an absolute error bound, denoted as $\delta$ and $\eta$, respectively. 
CSI requires a target scale factor $s$.
TTHRESH takes in a target RMSE of $\tau$. 
Neurcomp requires a target compression ratio $c$. 
Changing the intrinsic parameters of a base compressor will cause it to generate different intermediate data which will be augmented differently. 
As a result, even though our augmented compressor guarantees topology preservation and maintains the user-defined global error bound, the compression results may vary. 

For our experiments, we set the error parameter for each base compressor (except CSI and Neurcomp) to be equal to $k\xi$ for some $k \in \R$ that is compressor-dependent. Specifically, we set $\delta = 5\xi$, $\eta = 0.25\xi$, and $\tau = 0.05\xi$. To decide each value of $k$, we conduct a grid search and observe the effects of different values of $k$ across different values of $\xi$ and different datasets. The optimal value of $k$ varies between datasets and values of $\xi$, but the values that we chose are always approximately optimal. Hypothesized explanations as to why these values of $k$ improve results are described in \cref{sec:base-compressor-parameters}. We also set $c = 100$ and $s = 7$. We chose these configurations because they empirically led to the highest compression ratios.

To differentiate from the persistence threshold $\varepsilon$ used by an augmented compressor, TopoQZ takes a persistence threshold $e$. The TTK implementation of TopoQZ is tightly coupled with ZFP, which requires an error bound $\zeta$, allowing for a total pointwise error upper-bounded by $e+\zeta$.
To measure compression ratio and compression times while respecting a topological constraint $\varepsilon$ and an error bound $\xi$, we measure how each augmented compressor and TopoSZ perform for $\varepsilon = 0.04$ and $\xi = 0.012$. When testing TopoQZ, to ensure that it respects both $\varepsilon$ and $\xi$, we set $e = \zeta = 0.006$ so that the max error is less than $0.012$.

To measure the trade off between compression ratio and reconstruction quality, for TopoSZ and each augmented compressor, we set $\varepsilon = 0.04$ and vary $\xi \in \{$0.003, 0.006, 0.009, 0.012, 0.015, 0.018$\}$. In some cases we need to vary $\xi$ (and $\zeta$, for TopoQZ) in a different range in order to obtain a meaningful curve. Notably, for TopoQZ, we set $e = 0.04$ and vary $\zeta \in \{$0.003, 0.11, 0.22, 0.33, 0.44, 0.55$\}$. We further discuss our methodology for choosing parameters and provide the parameter used in \cref{sec:reconstruction-quality-extra}. 

The combination of a chosen compressor, a fixed dataset, a value of $\varepsilon$ and $\xi$, is a trial.
We perform each trial on a single cluster node running an Intel Xeon Sandy Bridge-E processor with 16 cores and 64GB of RAM.
For Neurcomp, we perform the training on an RTX 2080ti GPU with 32GB of RAM.

\begin{figure}[!ht]
\begin{subfigure}{0.02\linewidth}
\raisebox{2.2\height}{
\includegraphics[angle=90, width=\linewidth]{fig-colorBar.png}}
\end{subfigure}
\begin{subfigure}{0.97\linewidth}
    \includegraphics[width=\linewidth]{fig-scale_lattice_2.png}
\end{subfigure}
\vspace{-6mm}
\caption{Zoomed-in views of the critical points of the contour trees of the Ionization (top) and SCALE-LETKF (bottom) datasets with persistence simplification $\varepsilon = 0.04$. For each dataset, arrows indicate one example where compression with TTHRESH led to critical points shifting. Local maxima are in orange, local minima are in dark blue, 1-saddles are in light blue, 2-saddles are in light orange. Top row: Ionization. Bottom row: SCALE-LETKF.}
\label{fig:zoom}

\centering{
\begin{tabular}{c|cc|c}
\hline
Dataset    & ZFP       & TTHRESH       & Total \#edges \\ \hline
QMCPACK    & (23,23,0) & (8,8,0)     & 69          \\ 
Tangaroa   &  (90, 92, 0) & (39, 46, 0) & 418         \\ 
Earthquake & (20, 19, 0) & (26, 25, 0)     & 169         \\
Ionization & (181, 187, 0)  & (37, 44, 0) & 568         \\
Isabel     & (16, 15, 0) & (15, 16, 0)  & 29          \\
Miranda    & (6, 6, 0) & (3, 3, 0)     & 11          \\ 
Nyx        & (133, 132, 0) & (2, 373, 0) & 743 \\
S3D       & (100, 100, 0) & (74, 80, 0) & 1013 \\
SCALE-LETKF & (152, 153, 0) & (129, 127, 0) & 401 \\ \hline
\end{tabular}
}
\vspace{-2mm}
\captionof{table}{Reporting the number of false cases (false positives, false negatives, false types) produced by base compressors SZ3 and TTHRESH, respectively, together with the total number of edges of the input (ground truth) contour tree. Contour trees are simplified with $\varepsilon=0.04$.}
\label{tab:base-false-cases}
\vspace{-6mm}
\end{figure}

%---------------------------------------
\subsection{Comparative Analysis of Augmented Compressors}
\label{sec:augmented-compressors}

In this section, we perform a comparative analysis of five augmented compressors, qualitatively and quantitatively. 
We visualize three scientific datasets before and after compression with {three of our augmented compressors} in \cref{fig:volume-render}. We also display the PSNR and compression ratio next to each decompressed dataset. 
Compression ratios and times for a single combination of $\varepsilon$ and $\xi$ are reported in \cref{tab:compression-task}. Charts showing the reconstruction quality on two datasets is reported in \cref{fig:reconstruction-quality}. Similar charts for the remaining datasets and compressors are shown in \cref{sec:reconstruction-quality-extra}. Results demonstrating the effect of independently varying $\varepsilon$ or $\xi$ on the evaluation metrics are given in \cref{sec:other-experiments}.

\subsubsection{Topological Guarantees}
When compressing a dataset with any base compressor, the contour tree of the data is often significantly distorted with a large number of false cases, whereas it is always perfectly preserved using our augmented compressor. This observation has been validated empirically in every trial: the contour tree is perfectly preserved in terms of the locations of its critical points and their connectivity.

For instance, we visualize the Isabel dataset in \cref{fig:teaser} using TTHRESH and augmented TTHRESH. We highlight parts of the contour trees via zoomed-in views before and after compression. In a bottom zoomed-in view, TTHRESH (middle) fails to preserve a few critical points of the contour tree. In a top zoomed-in view, TTHRESH (middle) preserves the locations of the critical points, but not their connectivity. In contrast, the augmented TTHRESH preserves both the locations and connectivity among the critical points.

We further provide zoomed-in views for the Ionization and SCALE-LETKF datasets in~\cref{fig:zoom}. We observe clearly that TTHRESH fails to predict many critical points, whereas augmented TTHRESH preserves them all.

In \cref{tab:base-false-cases}, we report the number of false cases, including both extremum-saddle and saddle-saddle connections in the contour tree reconstructed with ZFP and TTHRESH. We again use parameter configurations that produce the same compression ratios as their augmented versions with $\varepsilon = 0.04$ and $\xi = 0.012$ (for those configurations see \cref{sec:base-compressor-parameters}). In \cref{tab:base-false-cases}, we can see that ZFP and TTHRESH produce many false cases.

\begin{table}[!t]
\setlength{\tabcolsep}{2pt}
\resizebox{\columnwidth}{!}{
\begin{tabular}{cccccccc}
\hline
Dataset     & A-ZFP            & A-SZ3   & A-CSI         & A-TTHRESH      & \multicolumn{1}{c|}{A-Neurcomp} & TopoQZ             & TopoSZ          \\ \hline
\multicolumn{8}{c}{Compression Ratio}                                                                                                                \\ \hline
QMCPACK     & 58.7             & 86.1    & 102.3           & \textbf{104.8} & \multicolumn{1}{c|}{23.9}       & 23.4               & 27.8            \\
Tangaroa    & 37.3             & 43      & 33.5            & \textbf{44.8}  & \multicolumn{1}{c|}{15.3}       & --                 & 24.3            \\
Earthquake  & 86.1             & 127.4   & 79.4            & \textbf{129.2} & \multicolumn{1}{c|}{63.5}       & 13.4               & 50.1            \\
Ionization  & 118.8            & 121.5   & 119.9           & \textbf{170.5} & \multicolumn{1}{c|}{72.7}       & 30.0               & 25.1            \\
Isabel      & 47.4             & 103.5   & 70.6            & \textbf{182.2} & \multicolumn{1}{c|}{41.6}       & --                 & 37.6            \\
Miranda     & 172.3            & 198.6   & 157.2           & \textbf{318.7} & \multicolumn{1}{c|}{95.0}       & 76.5               & 95.9            \\
Nyx         & 65.3             & 69.5    & 70.4            & \textbf{84.5}  & \multicolumn{1}{c|}{18.9}       & --                 & --              \\
S3D         & 38.4             & 46.6    & 43.6            & \textbf{59.6}  & \multicolumn{1}{c|}{6.0}        & 9.2                & --              \\
SCALE-LETKF & 69.5             & 74.4    & 58.5            & \textbf{114.2} & \multicolumn{1}{c|}{8.6}        & 11.4               & --              \\ \hline
\multicolumn{6}{c|}{Total Compression and Augmentation Time}                                                  & \multicolumn{2}{c}{Compression Time} \\ \hline
QMCPACK     & 1.27             & 1.36    & 1.21            & 1.45           & \multicolumn{1}{c|}{172.08}     & \textbf{1.05}      & 10.46           \\
Tangaroa    & 9.51             & 10.99   & \textbf{9.33}   & 11.98          & \multicolumn{1}{c|}{1519.21}    & --                 & 314.56          \\
Earthquake  & \textbf{7.08}    & 7.39    & \textbf{7.08}   & 8.66           & \multicolumn{1}{c|}{1039.94}    & 8.08               & 48.17           \\
Ionization  & \textbf{8.68}    & 10.00   & 14.08           & 15.16          & \multicolumn{1}{c|}{1221.12}    & 10.40              & 425.31          \\
Isabel      & \textbf{33.77}   & 35.49   & 42.69           & 42.67          & \multicolumn{1}{c|}{7147.86}    & --                 & 367.10          \\
Miranda     & 223.52           & 284.18  & 248.51          & 348.72         & \multicolumn{1}{c|}{9359.59}    & \textbf{160.60}    & 434.98          \\
Nyx         & \textbf{1059.06} & 1137.33 & 5664.46         & 25594.54       & \multicolumn{1}{c|}{38959.73}   & --                 & --              \\
S3D         & 209.83           & 253.82  & \textbf{173.09} & 253.72         & \multicolumn{1}{c|}{34610.78}   & 633.13             & --              \\
SCALE-LETKF & \textbf{221.49}  & 399.32  & 343.58          & 371.19         & \multicolumn{1}{c|}{40887.62}   & 524.05             & --              \\ \hline
\multicolumn{8}{c}{Decompression Time}                                                                                                               \\ \hline
QMCPACK     & 0.14             & 0.32    & 0.14            & 0.17           & \multicolumn{1}{c|}{4.24}       & 0.63               & \textbf{0.01}   \\
Tangaroa    & 0.49             & 0.52    & 0.51            & 0.96           & \multicolumn{1}{c|}{16.32}      & --                 & \textbf{0.12}   \\
Earthquake  & 0.38             & 0.41    & 0.37            & 0.55           & \multicolumn{1}{c|}{8.50}       & 5.50               & \textbf{0.07}   \\
Ionization  & 0.48             & 0.54    & 0.54            & 0.83           & \multicolumn{1}{c|}{11.62}      & 5.08               & \textbf{0.10}   \\
Isabel      & 1.43             & 1.35    & 1.34            & 2.64           & \multicolumn{1}{c|}{79.61}      & --                 & \textbf{0.41}   \\
Miranda     & 2.92             & 3.1     & 3.38            & 4.53           & \multicolumn{1}{c|}{175.61}     & 53.2               & \textbf{0.67}   \\
Nyx         & \textbf{6.72}    & 8.05    & 9.71            & 9.49           & \multicolumn{1}{c|}{2457.91}    & --                 & --              \\
S3D         & \textbf{11.52}   & 11.47   & 11.84           & 16.33          & \multicolumn{1}{c|}{2135.47}    & 390.94             & --              \\
SCALE-LETKF & \textbf{11.59}   & 11.92   & 12.89           & 21.50          & \multicolumn{1}{c|}{2629.48}    & 351.61             & --              \\ \hline
\end{tabular}
}
\vspace{-2mm}
\caption{Compression ratio, compression time, and decompression time for each compressor with $\varepsilon = 0.04$ and error bound $\xi = 0.012$ (except TopoQZ has $e = \zeta = 0.006$).
Times are in seconds.
Trials that did not finish are marked with a dash. 
TopoQZ ran out of memory on Nyx and it crashed on Isabel and Tangaroa due to unknown reasons. 
TopoSZ ran out of memory on Nyx, S3D, and SCALE-LETKF. 
}
\label{tab:compression-task}

\includegraphics[width=\columnwidth]{fig-rate_distortion_2.png}
    \vspace{-4mm}
    \captionof{figure}{PSNR, bottleneck distance, and Wasserstein distance versus bit-rate for each compressor for the QMCPACK and Earthquake datasets with $\varepsilon = 0.04$ ($e = 0.04$ for TopoQZ). These curves are given for other datasets in \cref{sec:reconstruction-quality-extra}.}
    \label{fig:reconstruction-quality}
    \vspace{-4mm}
\end{table}

%---------------------------------------
\subsubsection{Evaluation Metrics}
\label{sec:augmented-compressors-evaluation-metrics}

Compression ratio and times are reported in \cref{tab:compression-task} for a fixed parameter configuration of $\varepsilon = 0.04$ and $\xi = 0.012$.
We chose this parameter configuration because a small amount of persistence simplification preserves a large number of topological features in the input data, generating complex test cases for topology-preserving compression. For the reconstruction quality demonstrated in \cref{fig:reconstruction-quality}, $\varepsilon = 0.04$ is chosen similarly, and $\xi$ is varied between $0.003$ and $0.018$ to yield a variety of different compression ratios while still remaining small.
In this section, we compare the different augmented compressors. We leave the comparison with TopoQZ and TopoSZ to \cref{sec:compare-topology}.

\para{Compression ratios.} 
As shown in \cref{tab:compression-task}, Augmented TTHRESH produces the best compression ratios in every trial. Augmented Neurcomp performs noticeably worse than the other compressors.

\para{Reconstruction quality.}
In every trial, our framework successfully maintains a pointwise error bound $\xi$. There is a natural trade off between compression ratio and reconstruction quality. As shown in \cref{fig:reconstruction-quality}, Augmented SZ3 and Augmented TTHRESH have the best trade off between bit-rate, PSNR, $d_W$ and $d_B$, and perform equally well. Augmented Neurcomp performs the worst based on above metrics.

When visualized, we find that the decompressed volumes generally resemble the ground truth. However, when using certain transfer functions, visual artifacts may become visible. Artifacts appear more in visualizations that are sensitive to small changes in the transfer function. For the volume renderings in this paper, we chose transfer functions that led to fewer visual artifacts; see~\cref{sec:visual-artifacts} for adversarial examples.

In practice, we find that upper and lower bound tightening does not affect PSNR very much; most of the reconstruction quality is determined by the initial upper and lower bounds. \cref{fig:errorMap} shows a map of the absolute error of each point for a topologically complex slice of the Ionization dataset before and after tightening. We can see that tightening does not have a significant effect on the average error.

\begin{figure}[!t]
    \begin{subfigure}[b]{0.06\linewidth}
        \raisebox{0.5\height}{\includegraphics[width=\linewidth]{fig-colorBar-error.png}}
    \end{subfigure}
    \hfill
    \begin{subfigure}[b]{0.46\linewidth}
    \centering
        \includegraphics[width=\linewidth]{fig-errorMapPre.png}
        (A)
    \end{subfigure}
    \hfill
    \begin{subfigure}[b]{0.46\linewidth}
    \centering
        \includegraphics[width=\linewidth]{fig-errorMapPost.png}
        (B)
    \end{subfigure}
    \vspace{-5mm}
    \caption{Error map of a topologically complex slice of the Ionization dataset (A) before error bound tightening and (B) after error bound tightening.}
    \label{fig:errorMap}
    \vspace{-8mm}
\end{figure}

\para{Run time analysis.} 
There are significant differences in run time among the augmented compressors.
As discussed in \cref{sec:run-time}, these times are affected by factors other than the base compression time. However, Augmented Neurcomp is the slowest because Neurcomp compresses data by training a neural network.  
Of the remaining four compressors, ZFP is typically the fastest, and TTHRESH the slowest, although this observation does not hold for all trials. For decompression time, ZFP is the fastest, while Neurcomp remains the slowest.

\para{Highlighted results.} 
There is no clear best augmented compressor that outperforms others across all metrics. Other than augmented Neurcomp, utilizing any augmented compressor plays a trade off between compression ability and speed. For the remainder of our analysis, we will primarily focus on Augmented ZFP, which is the fastest augmented compressor, and Augmented TTHRESH, which yields the best compression ratios and reconstruction quality.

%---------------------------------------
\subsection{Comparison with TopoQZ and TopoSZ}
\label{sec:compare-topology}

\para{Topological guarantees.}
Our framework preserves the contour tree during compression, and achieves the same topological guarantee as TopoSZ. TopoQZ ensures that all critical point pairs are preserved above a persistence threshold $\varepsilon$, but their locations and connectivity may be distorted after compression.

\para{Compression ratio.} 
In terms of compression ratio, when maintaining a strict topological constraint $\varepsilon = 0.04$ and error bound $\xi = 0.012$, every augmented compressor except Augmented Neurcomp outperforms both TopoQZ and TopoSZ in every trial.

\para{Reconstruction quality.} 
The curves in \cref{fig:reconstruction-quality} show that every augmented compressor except Augmented Neurcomp can match the PSNR of TopoQZ and TopoSZ,  while using less space. In terms of topological distance, the augmented compressors except Augmented Neurcomp outperform TopoSZ in terms of $d_W$ and $d_B$. They also outperform TopoQZ in terms of $d_B$, but are comparable in terms of $d_W$.

\para{Run time analysis.} 
In terms of compression time, the augmented compressors except Augmented Neurcomp produce times that are comparable to or better than TopoQZ, and significantly outperform TopoQZ on the largest datasets. 
These four augmented compressors are also significantly faster than TopoSZ across all trials. 

In terms of decompression time, the augmented compressors except Augmented Neurcomp perform slower than TopoSZ but faster than TopoQZ. There are several possible reasons why our decompression times are slower than TopoSZ. First and most notably, our decompression process is more complex, as it involves a decompression with the base compressor and then an augmentation of the decompressed results. This process requires more operations and has a higher I/O overhead. Second, we use XZ along with tar archives for lossless compression, which is slower than ZSTD used by TopoSZ. See \cref{sec:more-running-time} for a more detailed analysis of the decompression time.   

%---------------------------------------
\subsection{Analysis of Compression Time}
\label{sec:run-time}

\para{Asymptotic analysis.}
Let $n$ be the number of vertices in the rectilinear mesh. Our algorithm utilizes heap merges~\cite{gueunet2017task} during the merge tree computation; however, we use binary heaps (stored in arrays) instead of Fibonacci heaps from~\cite{gueunet2017task}. For a binary heap with $m$ elements, a single insertion operation has a worst-case time complexity of $O(\log m)$. 
Following~\cite{gueunet2017task}, from bottom to top, constructing an edge $e$ in a merge tree requires merging its heap with the heaps of its descendants, which takes $O(n \log n)$. Let $h$ denotes the \emph{height} of the tree, which corresponds to the maximum number of ancestor edges.
Then constructing a merge tree using these insertion-based heap merges takes $O(h n \log n)$. During the progressive tightening process, let $F$ denote the total number of detected false cases, each of which triggers a (partial) recomputation of the merge tree. Therefore, our algorithm takes $O(F h n \log n) = O(n^3 \log n)$. 

In practice, $F \ll n$ as shown in \cref{tab:time}. 
Additionally, $h \ll n$. We found that $\frac{h}{n}$ ranged from $0.0004$ (Miranda, A-ZFP) to $0.025$ (Nyx, A-Neurcomp). Excluding Augmented Neurcomp, $\frac{h}{n} < 0.01$ in $97\%$ of trials.  
On the other hand, using Fibonacci heaps to construct a merge tree~\cite{gueunet2017task} takes $O(n \log n)$ due to constant time heap merges; however, in our setting, we have found that binary heaps have lower run time in practice. Likewise, it is possible to merge heaps in linear time, but we instead merge by repeatedly inserting each element of the smaller heap into the larger one, as doing so has a much lower run time in practice.

\para{Empirical analysis.} To analyze the run time empirically, we calculate the amount of time for each portion of our algorithm with Augmented ZFP and Augmented TTHRESH,  with $\varepsilon = 0.04$ and $\xi = 0.012$. These run times are shown in \cref{tab:time}.

In \cref{tab:time}, the most time-consuming task is the computation of merge and contour trees. We compute the contour tree of the input data at the beginning of the algorithm. During the error bound tightening steps we also compute the contour tree of the decompressed data. These run times are shown in \cref{tab:time} under the `CT' and `Grow' columns, and account for $35-77\%$ of the total run time for each trial in \cref{tab:time}. 

For most of the trials, the time to run the base compressor, shown in the `BC' column, is a relatively small percentage of the overall compression time. 
However, if a base compressor produces results that nearly preserve the contour tree and does not produce too many extra branches, including those of persistence below $\varepsilon$, the augmentation time may be lower. This phenomenon suggests that the accuracy of the base compressor may have more effect on the total compression and augmentation time than just base compression.
In general, the run time of each base compressor is much faster than its augmented counterpart; see~\cref{sec:more-running-time} for a comparison. 

\begin{table}[!ht]
\setlength{\tabcolsep}{2pt}
\centering
\vspace{-2mm}
\resizebox{\columnwidth}{!}{
\begin{tabular}{cccccccccc}
\hline
\multicolumn{1}{c|}{Dataset}     & BC    & CT     & ULB     & Grow     & \%B    & \#FC & Fix      & File  & Total    \\ \hline
\multicolumn{10}{c}{Augmented ZFP}                                                                                  \\ \hline
\multicolumn{1}{c|}{QMCPack}     & 0.15  & 0.34   & 0.21   & 0.38     & 0.17\% & 0    & 0.0      & 0.27  & 1.35     \\
\multicolumn{1}{c|}{Tangaroa}    & 1.42  & 2.16   & 2.80   & 1.34     & 0.43\% & 14   & 0.0004   & 1.84  & 9.57     \\
\multicolumn{1}{c|}{Earthquake}  & 0.45  & 1.56   & 1.26   & 2.88     & 0.79\% & 2    & 0.0001   & 0.97  & 7.12     \\
\multicolumn{1}{c|}{Ionization}  & 0.70  & 1.56   & 2.09   & 3.10     & 1.15\% & 10   & 0.0009   & 1.35  & 8.81     \\
\multicolumn{1}{c|}{Isabel}      & 4.73  & 4.61   & 8.95   & 9.48     & 0.54\% & 1    & 0.0647   & 6.17  & 34.01    \\
\multicolumn{1}{c|}{Miranda}     & 4.00  & 156.79 & 17.92  & 37.85    & 1.21\% & 0    & 0.0      & 7.02  & 223.58   \\
\multicolumn{1}{c|}{NYX}         & 24.75 & 695.26 & 150.27 & 142.16   & 0.47\% & 6    & 4.60     & 42.21 & 1084.05  \\
\multicolumn{1}{c|}{S3D}         & 14.2  & 27.93  & 69.11  & 56.40    & 1.04\% & 39   & 0.01411  & 38.44 & 206.63   \\
\multicolumn{1}{c|}{SCALE-LETKF} & 15.2  & 16.15  & 61.67  & 92.99    & 0.63\% & 37   & 0.003408 & 35.34 & 221.47   \\ \hline
\multicolumn{10}{c}{Augmented TTHRESH}                                                                              \\ \hline
\multicolumn{1}{c|}{QMCPack}     & 0.23  & 0.34   & 0.21   & 0.45     & 0.77\% & 0    & 0.0      & 0.22  & 1.45     \\
\multicolumn{1}{c|}{Tangaroa}    & 3.34  & 2.2    & 2.79   & 1.92     & 0.77\% & 18   & 0.0001   & 1.59  & 11.84    \\
\multicolumn{1}{c|}{Earthquake}  & 1.53  & 1.56   & 1.25   & 3.33     & 0.65\% & 4    & 0.0001   & 0.91  & 8.58     \\
\multicolumn{1}{c|}{Ionization}  & 1.93  & 1.54   & 2.08   & 8.09     & 1.48\% & 5    & 0.0120   & 1.24  & 14.94    \\
\multicolumn{1}{c|}{Isabel}      & 13.13 & 4.62   & 9.01   & 10.24    & 0.54\% & 1    & 0.0666   & 5.62  & 42.69    \\
\multicolumn{1}{c|}{Miranda}     & 12.73 & 156.48 & 17.73  & 153.69   & 2.33\% & 0    & 0.0      & 7.24  & 347.87   \\
\multicolumn{1}{c|}{NYX}         & 57.87 & 687.03 & 132.49 & 15838.88 & 0.82\% & 89   & 97.82    & 37.52 & 27286.05 \\
\multicolumn{1}{c|}{S3D}         & 66.35 & 27.89  & 67.27  & 56.51    & 1.14\% & 47   & 0.0136   & 32.11 & 250.79   \\
\multicolumn{1}{c|}{SCALE-LETKF} & 69.21 & 16.10  & 60.00  & 193.18   & 0.54\% & 46   & 0.0032   & 29.71 & 368.35   \\ \hline
\end{tabular}
}
\vspace{-2mm}
\caption{Runtime analysis for each component of the augmented framework involving Augmented ZFP and Augmented TTHRESH with $\varepsilon = 0.04$ and $\xi = 0.012$. 
All times are in seconds. 
BC: running the base compressor.
CT: computing the contour tree of the input data.
ULB: calculating the initial upper and lower bounds. 
Grow: time growing the contour tree of the reconstructed data.
\%B: percent of branches in the reconstructed contour tree whose growth was recomputed.
\#FC: number of false cases corrected after upper and lower bounds are set.
Fix: average time to fix a false case, excluding regrowing branches.
File: average time to write the compressed output to a file.}
\label{tab:time}
\vspace{-6mm}
\end{table}
\section{Conclusion and Discussion}
\label{sec:conclusion}

We introduce a novel framework for augmenting \emph{any} lossy compressor to preserve the contour tree of a volumetric dataset while maintaining a user-specified global error bound. 
To do this, our framework first imposes topology-informed upper and lower bounds on each data point. 
It then progressively tightens those bounds until the contour tree is preserved. 
We also introduce a novel encoding scheme that efficiently stores individual points with variable precision and maintains these upper and lower bounds. 
When our framework is used to augment state-of-the-art lossy compressors, it is shown to preserve the contour trees of various scientific datasets.
Our augmented compressors also achieve higher compression ratios and reconstruction quality than those obtained by existing topology-preserving compressors in comparable or faster time.
Our framework will benefit from any advancement with lossy compression since it can be used to augment increasingly effective lossy compressors to achieve better topology-preserving compression. 

Our framework is not without limitations. The compression times are longer than the base compressors. This difference gets worse as the topological complexity of the data increases.
However, in some use-cases, topological preservation is preferable to run time.
Regardless, our framework would benefit from more efficient or parallel implementations for the contour/merge tree computation and the encoding scheme. 


%-------------------------
\acknowledgments{
This work was supported in part by grants from National Science Foundation OAC-2330367, OAC-2313122, 
OAC-2313123, OAC-2311878,
and OAC-2313124. 
}
%-------------------------

\bibliographystyle{abbrv-doi-hyperref}
\bibliography{refs-compression.bib}

%-------------------------
\clearpage
\newpage
\pagenumbering{arabic}
\renewcommand*{\thepage}{A\arabic{page}}
\appendix 
\section{Details on the Datasets}
\label{sec:datasets}

We include details on each dataset used in our experiments. Some datasets are accessed from the SDR Bench \cite{zhao2020sdrbench}, available at~\cite{SDRBench}. Datasets accessed from the SDR Bench may involve contributions from the DOE NNSA ECP project and the ECP CODAR project.

The \textbf{QMCPACK} dataset (accessed from the SDR Bench) comes from the QMCPACK performance test using the QMCPACK continuum quantum Monte Carlo
simulation \cite{Kim_2018, Kent2020}. Only the 145th orbital out of 288 is used for testing, for which only a single field is provided. The dataset is normalized to $[0,1]$ before compression. 

The \textbf{Tangaroa} dataset comes from a single time frame simulating the wind flow around a 3D model of the Research Vessel Tangaroa \cite{popinet2004experimental}. 
The magnitude of the wind velocity is used as the scalar field of interest. The dataset is normalized to $[0,1]$ before compression.

The \textbf{Earthquake} dataset originates from a TeraShake 2 earthquake simulation~\cite{olsen2008terashake2} and has been part of the 2006 IEEE Visualization Design Contest~\cite{scivis2006}.  
The dataset used in this paper is obtained from the public data repository of Pont et al. used for their publication~\cite{pont2021wasserstein}. The dataset is preprocessed by Pont et al. and comes with a single field. It represents one time step of a simulation of an earthquake at the San Andreas fault. Specifically, we use time step 011700. The details of the preprocessing can be found on the repository. The dataset is normalized to $[0,1]$ before compression.

The \textbf{Ionization} dataset originates from an ionization front simulation by Whalen and Norman \cite{whalen2008ionization} and has been featured in the 2008 IEEE Visualization Design Contest~\cite{scivis2008}.   
The simulation is done with 3D radiation hydro-dynamical calculations of ionization front instabilities in which multi-frequency radiative transfer  is coupled to the primordial chemistry of eight species~\cite{whalen2008ionization}. 
The single time step used in this paper comes from cluster 2, time step 0125 and is obtained from the same repository as the Earthquake dataset. It is 
preprocessed and comes with a single field. The details of the preprocessing can be found on the repository. The dataset is normalized to $[0,1]$ before compression.

The \textbf{Isabel} dataset originates from a hurricane simulation from the National Center for Atmospheric Research, and has been included in the 2004 IEEE Visualization Design Contest \cite{scivis2004}.
While the original dataset has a size of  $500 \times 500 \times 100$, we truncate the dataset to $500 \times 500 \times 90$ in order to avoid land regions that contain no data values. We use the wind speed field. The dataset is normalized to $[0,1]$ before compression.

The \textbf{Miranda} dataset (accessed from the SDR Bench) comes from the hydrodynamics code for large turbulence simulations conducted by Lawrence Livermore National Laboratory. We use the density field.

The $\textbf{Nyx}$ dataset (accessed from the SDR Bench) comes from the Nyx cosmological simulation~\cite{almgren2013nyx}. We use the dark matter density field.

The $\textbf{S3D}$ dataset comes from the S3D turbulence simulation software~\cite{treichler2017s3d}. This dataset is derived from data from the SDR Bench. We compute the field as the magnitude of the velocity, as derived from the velocity $x$, $y$, and $z$ components provided. We use its parameter setting $1.7e \times 10^{-3}$.

The $\textbf{SCALE-LETKF}$ dataset (accessed from the SDR Bench) comes from the Local Ensemble Transform Kalman Filter (LETKF) data assimilation package for the SCALE-RM weather model~\cite{lien2017near}. We use the QV field which is up-sampled to double precision.
\section{Evaluation Metrics}
\label{sec:evaluationMetrics}

We evaluate our trials on several metrics. We measure compression ratio, compression time, and peak signal-to-noise ratio (PSNR), which are common metrics for evaluating compressors. 

The compression ratio is the size of the uncompressed file divided by the size of the compressed file, and higher compression ratios indicate smaller compressed file sizes. Related to this, the bit-rate is the number of bits used to compress a single data point. For floating point data, it is either equal to 32 divided by the compression ratio, or 64 divided by the compression ratio, depending on if the dataset is respectively single or double precision.

Peak signal-to-noise ratio (PSNR) is a number that measures reconstruction quality. If $R$ is the range of the data and $M$ is the mean squared error of our reconstruction, the PSNR is defined as:
\begin{equation}
  \label{eq:psnr}
  \text{PSNR} = 10 \log_{10}\left( \frac{ R^2 }{ M } \right).
\end{equation}
In general, higher PSNR values indicate higher reconstruction quality. We also measure the bottleneck distance \cite{cohen2005stability} and Wasserstein distance \cite[page 183]{edelsbrunner2022computational} to quantify the amount of topological control.

To define the Wasserstein and bottleneck distances, suppose that $f$ and $f'$ are two scalar fields that yield persistence diagrams $D$ and $D'$ respectively; see~\cite{EdelsbrunnerLetscherZomorodian2002} for an introduction to persistent homology. If $\varphi:D \rightarrow D'$ represents any bijection, and $q \geq 1$, we define the $q$-Wasserstein distance $W_q$ as:
\begin{equation}
W_q(D,D') = \left[ \inf_{\varphi:D\rightarrow D'} \sum_{p \in D} || p - \varphi(p)||_\infty^q \right]^{\frac{1}{q}}.
\end{equation}
For our evaluation, we set $d_W = W_2$ and $d_B = W_\infty$. In particular:
\begin{equation}
W_\infty(D,D') = \inf_{\varphi:D \rightarrow W'} \sup_{p \in D} ||p- \varphi(p)||_\infty. 
\end{equation}

In general, lower values of $W_q$ for any $q \geq 1$ indicate that the persistence diagrams $D$ and $D'$ are more similar. This in-turn means that the datasets that produced $D$ and $D'$ are more topologically similar. To compute $d_W$ and $d_B$ in reasonable time, we report approximate values~\cite{KerberMorozovNigmetov2016} with $1\%$ error that were computed after persistence simplification with a threshold of $1.5 \times 10^{-6}$ times the range of the data.
\section{Analysis of Compression and Decompression Times}
\label{sec:more-running-time}

We provide compression times for the base compressor versus the augmented compressors in~\cref{sec:base-compressor-times}. We provide a detailed analysis of decompression times in~\cref{sec:decompression-time-analysis}.

\subsection{Compression Times}
\label{sec:base-compressor-times}

We give the run times (in seconds) of each base compressor versus the augmented compressor in~\cref{tab:base-compressor-times}. Each augmented compressor is abbreviated as A-ZFP, A-SZ3, etc. 
Times are for $\varepsilon = 0.04$ and $\xi = 0.012$. The parameter settings for each base compressor are the same as those used prior to the augmentation (see~\cref{sec:base-compressor-parameters} for specifics).

\begin{table*}[!ht]
\scriptsize
\centering{
\begin{tabular}{l|cc|cc|cc|cc|cc}
\hline
Dataset & \multicolumn{1}{l}{ZFP} & \multicolumn{1}{l|}{A-ZFP} & \multicolumn{1}{l}{SZ3} & \multicolumn{1}{l|}{A-SZ3} & \multicolumn{1}{l}{CSI} & \multicolumn{1}{l|}{A-CSI} & \multicolumn{1}{l}{TTHRESH} & \multicolumn{1}{l|}{A-TTHRESH} & \multicolumn{1}{l}{Neurcomp} & \multicolumn{1}{l}{A-Neurcomp} \\ \hline
QMCPACK     & 0.04                    & 1.3                        & 0.05                    & 1.37                       & 0.03                    & 1.21                       & 0.16                        & 1.49                           & 1633.42                      & 1640.44                        \\
Tangaroa    & 0.24                    & 9.79                       & 0.34                    & 11.26                      & 0.99                    & 9.46                       & 1.69                        & 12.11                          & 26.20                        & 63.17                          \\
Earthquake  & 0.35                    & 7.42                       & 0.24                    & 7.47                       & 0.14                    & 7.16                       & 1.16                        & 8.86                           & 1021.93                      & 1042.63                        \\
Ionization  & 0.24                    & 8.96                       & 0.34                    & 10.24                      & 0.20                    & 14.31                      & 1.27                        & 15.29                          & 1195.47                      & 1222.28                        \\
Isabel      & 0.73                    & 34.85                      & 1.00                    & 36.30                      & 3.47                    & 43.53                      & 7.73                        & 43.30                          & 7009.20                      & 7151.38                        \\
Miranda     & 1.31                    & 225.75                     & 2.25                    & 285.79                     & 1.36                    & 250.63                     & 8.61                        & 353.24                         & 8865.54                      & 9377.60                        \\
S3D         & 5.28                    & 214.86                     & 7.27                    & 258.83                     & 5.25                    & 178.04                     & 53.10                       & 258.56                         & 31976.49                     & 34663.49                       \\
Nyx         & 1.78                    & 1072.75                    & 6.14                    & 1147.18                    & 21.46                   & 5716.34                    & 31.74                       & 25596.02                       &   87230.00                           &       29991.31                         \\
SCALE-LETKF & 4.99                    & 226.58                     & 8.01                    & 402.86                     & 5.73                    & 347.17                     & 49.83                       & 375.18                         & 37072.43                     & 41227.85                       \\ \hline
\end{tabular}}
\vspace{-2mm}
\caption{Compression times (in seconds) for each base compressor versus each augmented compressor on each dataset. Augmented compressors are abbreviated as A-ZFP, A-SZ3, etc.}
\label{tab:base-compressor-times}
\vspace{-4mm}
\end{table*}

\subsection{Decompression Time}
\label{sec:decompression-time-analysis}

We provide a breakdown of the decompression times into multiple stages for the Augmented ZFP and Augmented TTHRESH on each dataset in \cref{tab:decompression-time}. 
For comparison, the decompression times for TopoSZ are as follows: QMCPACK: $0.01$s; Tangaroa: $0.12$s; Earthquake: $0.07$s; Ionization: $0.10$s; Isabel: $0.41$s; and Miranda: $0.67$s.

The decompression pipeline of an augmented compressor first decompresses a compressed file using XZ with a tar archive, and uses Huffman to decode the quantization numbers. Then, it decompresses the compressed file using the base decompressor (run as a separate process), and augments the result using the quantization numbers. Finally, it saves the output. As shown in~\cref{tab:decompression-time}, no individual stage of the decompression pipeline dominates the others in terms of run time, although running the base decompressor and augmentation takes  longer times. 

\para{Comparison to TopoSZ.} Our decompression times (using augmented compressors) are much slower than TopoSZ for several reasons. First, the decompression pipeline of TopoSZ does not involve any augmentation;  whereas our pipeline runs a base decompressor and augments the result. Augmentation increases the I/O overhead, as our augmentation layer must read in the output of the base decompressor, which is much larger than the single compressed file read in by TopoSZ. Second, for lossless compression, we decompress using XZ, whereas TopoSZ uses ZSTD, which is much faster; see \cref{tab:XZ-versus-ZSTD} for a comparison. Furthermore, because we must store the output of the base decompressor and the augmentation layer, we must combine both files into a tar archive. Extracting this tar archive takes additional time. Finally, the code for our decompression pipeline is not optimized and has room for a more efficient implementation. 

\begin{table}[!ht]
\setlength{\tabcolsep}{2pt}
\centering
\begin{tabular}{cccccccc}
\hline
\multicolumn{1}{c|}{Dataset}     & Lossless & Load & Base  & Augment & Save & Clean & Total \\ \hline
\multicolumn{7}{c}{Augmented ZFP}                                                   &       \\ \hline
\multicolumn{1}{c|}{QMCPACK}     & 0.03     & 0.01 & 0.03  & 0.01    & 0.01 & 0.04  & 0.14  \\
\multicolumn{1}{c|}{Tangaroa}    & 0.11     & 0.07 & 0.10  & 0.11    & 0.04 & 0.05  & 0.48  \\
\multicolumn{1}{c|}{Earthquake}  & 0.06     & 0.07 & 0.11  & 0.06    & 0.05 & 0.04  & 0.39  \\
\multicolumn{1}{c|}{Ionization}  & 0.08     & 0.10 & 0.14  & 0.08    & 0.06 & 0.04  & 0.5   \\
\multicolumn{1}{c|}{Isabel}      & 0.26     & 0.21 & 0.32  & 0.36    & 0.12 & 0.05  & 1.33  \\
\multicolumn{1}{c|}{Miranda}     & 0.27     & 0.69 & 0.76  & 0.62    & 0.36 & 0.07  & 2.78  \\
\multicolumn{1}{c|}{Nyx}         & 1.18     & 1.38 & 2.12  & 2.37    & 1.38 & 0.09  & 8.52  \\
\multicolumn{1}{c|}{S3D}         & 2.22     & 2.38 & 2.70  & 2.73    & 1.24 & 0.15  & 11.42 \\
\multicolumn{1}{c|}{SCALE-LETKF} & 1.57     & 2.66 & 2.90  & 2.54    & 1.45 & 0.14  & 11.26 \\ \hline
\multicolumn{7}{c}{Augmented TTHRESH}                                               &       \\ \hline
\multicolumn{1}{c|}{QMCPACK}     & 0.03     & 0.02 & 0.06  & 0.01    & 0.01 & 0.04  & 0.16  \\
\multicolumn{1}{c|}{Tangaroa}    & 0.08     & 0.07 & 0.58  & 0.11    & 0.05 & 0.05  & 0.94  \\
\multicolumn{1}{c|}{Earthquake}  & 0.05     & 0.07 & 0.30  & 0.06    & 0.05 & 0.05  & 0.58  \\
\multicolumn{1}{c|}{Ionization}  & 0.06     & 0.11 & 0.48  & 0.09    & 0.07 & 0.05  & 0.85  \\
\multicolumn{1}{c|}{Isabel}      & 0.10     & 0.23 & 1.77  & 0.36    & 0.13 & 0.05  & 2.64  \\
\multicolumn{1}{c|}{Miranda}     & 0.21     & 0.72 & 2.47  & 0.65    & 0.38 & 0.08  & 4.5   \\
\multicolumn{1}{c|}{Nyx}         & 1.02     & 1.37 & 4.21  & 2.28    & 0.72 & 0.10  & 9.7   \\
\multicolumn{1}{c|}{S3D}         & 1.48     & 2.46 & 8.61  & 2.54    & 1.36 & 0.14  & 16.59 \\
\multicolumn{1}{c|}{SCALE-LETKF} & 0.96     & 2.73 & 13.57 & 2.49    & 1.48 & 0.15  & 21.39 \\ \hline
\end{tabular}
\vspace{-2mm}
\caption{Run times (in seconds) associated with the decompression pipeline for Augmented ZFP and Augmented TTHRESH with $\varepsilon = 0.04$ and $\xi = 0.012$.  
Lossless: time to losslessly decompress with XZ, tar archive, and Huffman coding.
Load: time spent loading data from files.
Base: time to decompress using the base decompressor.
Augment: time spent augmenting the output of the base decompressor.
Save: time spent saving the result to a RAW binary file.
Clean: time spent deleting files using system calls.}
\label{tab:decompression-time}
\end{table}

\begin{table}[!ht]
\setlength{\tabcolsep}{2pt}
\centering
%\resizebox{\columnwidth}{!}{
\begin{tabular}{c|cc|cc}
\hline
Dataset     & XZ               & ZSTD             & XZ                & ZSTD              \\ \hline
Compressor  & \multicolumn{2}{c|}{A-ZFP} & \multicolumn{2}{c}{A-TTHRESH} \\ \hline
QMCPACK     & 0.02             & 0.02             & 0.02              & 0.01              \\
Earthquake  & 0.04             & 0.02             & 0.03              & 0.02              \\
Ionization  & 0.04             & 0.02             & 0.03              & 0.02              \\
Tangaroa    & 0.07             & 0.02             & 0.05              & 0.03              \\
Isabel      & 0.21             & 0.04             & 0.07              & 0.03              \\
Miranda     & 0.18             & 0.05             & 0.12              & 0.05              \\
SCALE-LETKF & 1.33             & 0.15             & 0.76              & 0.11              \\
S3D         & 1.94             & 0.16             & 1.28              & 0.11              \\
Nyx         & 0.78             & 0.12             & 0.65              & 0.11              \\ \hline
\end{tabular}
%}
\vspace{-2mm}
\caption{Lossless decompression time (in seconds) for the output of augmented ZFP (listed as A-ZFP) and augmented TTHRESH (listed as A-TTHRESH) for each dataset using XZ versus ZSTD.}
\label{tab:XZ-versus-ZSTD}
\vspace{-6mm}
\end{table}
\section{Compressor Parameter Settings}
\label{sec:base-compressor-parameters}

We discuss the parameter settings associated with base and augmented compressors. 
The parameter settings during augmentations are described in~\cref{sec:base-parameters-during-augmentation}. The parameter settings for ZFP and TTHRESH used to generate \cref{fig:teaser}, \cref{fig:zoom}, and \cref{tab:base-false-cases} are given in~\cref{sec:base-parameters-equivalent}.

\subsection{Parameter Settings During Augmentation}
\label{sec:base-parameters-during-augmentation}

When augmenting ZFP, SZ3, and TTHRESH, to preserve a pointwise error bound $\xi$, we set the error bound $\delta$ of ZFP to be $5\xi$, the error bound $\eta$ of SZ3 to be $0.25\xi$, and the target PSNR $\tau$ of TTHRESH to be $0.05\xi$. 
It seems intuitive that setting $\delta = \eta = \tau = \xi$ would be optimal. We provide a brief justification for why our strategy works better.

In the case of TTHRESH and SZ3, these parameter settings cause the base compressors to preserve data with higher reconstruction quality. If the base compressor returns higher quality compression, then our topology-preserving framework will need to make fewer adjustments in order to preserve the contour tree. 
We have found that the extra storage space used by the base compressors to create higher quality intermediate data is outweighed by the space saved by storing fewer adjustments, causing compression ratios to increase. In addition to improved compression ratios, if the intermediate data more closely resembles the input (ground truth) data, there will be fewer false cases, leading to lower augmentation time. Further, if the intermediate data is closer to the input data, then after augmentation, the decompressed data will also be closer to the input data, leading to a higher PSNR.

ZFP is very conservative when it comes to error bounds and actually preserves an error bound much lower than the maximum error bound set by the user. We notice empirically that by multiplying the maximum error bound for ZFP by five, this raises the actual error bound preserved by ZFP up to something closer to $\xi$, allowing for ZFP to produce higher compression ratios while still yielding intermediate data of acceptable reconstruction quality. 
Surprisingly, we observe that setting $\delta$ very high, such as $\delta = 50\xi$ yields even higher compression ratios than $\delta = 5\xi$. On the other hand, the compression ratios are not too high such that ZFP obtains a comparative advantage in terms of compression ratio, and the compression times are slow enough that ZFP loses its comparative advantage on time efficiency. Therefore, we determine that $\delta = 5\xi$ is the best parameter setting for ZFP.

\subsection{Parameters for Equivalent Compression Ratios}
\label{sec:base-parameters-equivalent}

In order to generate \cref{fig:teaser}, \cref{fig:zoom}, and \cref{tab:base-false-cases}, we need to identify parameter settings for the absolute error bound parameter $\delta$ of ZFP and the RMSE target parameter $\tau$ for TTHRESH which produce compression ratios comparable to Augmented ZFP and 
Augmented TTHRESH with $\epsilon = 0.04$ and $\xi = 0.012$. The values for $\delta$ and $\tau$ that produce these comparable compression ratios are reported in~\cref{tab:base-false-cases}.

\begin{table}[!h]
\centering
\begin{tabular}{l|ll}
\hline
Dataset    & ZFP $\delta$ & TTHRESH $\tau$ \\ \hline
QMCPACK    & 3.11e-2    & 7.96e-6        \\ \hline
Tangaroa   & 3.60e-2    & 2.15e-4        \\ \hline
Earthquake & 3.60e-2    & 3.66e-4        \\ \hline
Ionization & 4.98e-2    & 3.51e-4        \\ \hline
Isabel     & 1.25e-1    & 2.62e-4        \\ \hline
Miranda    & 2.28e-1    & 6.66e-5        \\ \hline
Nyx        & 1.97e-3    & 1.97e-3        \\ \hline
S3D        & 3.08e-5    & 3.43e-7        \\ \hline
SCALE-LETKF& 9.00e-4    & 2.95e-6        \\ \hline
\end{tabular}
\vspace{-2mm}
\caption{Parameter configurations used to obtain \cref{fig:teaser}, \cref{fig:zoom}, and \cref{tab:base-false-cases}.}
\vspace{-2mm}
\end{table}
In some cases, the parameter configuration for ZFP is higher than $0.012$. ZFP is very conservative in terms of error bounds, and for each of those trials, the actual error bound that is maintained is likely significantly lower than the stated maximum error bound.
\section{Algorithmic Details}
\label{sec:algorithm-details}

We describe how the initial upper and lower bounds are computed in~\cref{sec:set-bounds}. 
We provide an illustration for the false case detection process in~\cref{sec:false-case-figures}, followed by special cases in~\cref{sec:special-cases}. We describe the specifics surrounding the tightening process, including a comparison with TopoSZ in~\cref{sec:tighten-bounds}.

\subsection{Initial Upper and Lower Bounds}
\label{sec:set-bounds}

To compute the initial upper and lower bounds $U(x)$ and $L(x)$ for $x \in \X$, we aim to locate an edge $ab$ of the simplified contour tree $T_{\varepsilon}$ satisfying $f(a) \leq f(x) \leq (b)$. Recall that a contour tree $T$ of the data $(\X,f)$ arises from a quotient map $\pi: (\X, f) \to (\X/{\sim}, f)$. The contour tree $T_{\varepsilon}$ is defined analogously. With an abuse of notation, we use $\pi(x)$ to represent the image of $x$ under a quotient map $\pi$ (when $\pi$ is clear from the context). For instance, in \cref{fig:upper-lower-a}, $\pi(x)$ on the left arises from the quotient map defining $T$, whereas $\pi(x)$ on the right comes from the quotient map defining $T_{\varepsilon}$.   

\underline{Case A:}~As illustrated in \cref{fig:upper-lower-a}, let $e'=a'b'$ be an edge of $T_{\varepsilon}$ whose segmentation region contains $x$; that is, $\pi(x) \in e'$.  
If $f(a') \leq f(x) \leq f(b')$, then we set $ab := a'b'$. Otherwise, we need to locate $ab$ with some care, following Case B below. 

\begin{figure}[!ht]
\centering
\includegraphics[width=0.6\linewidth]{fig-upper_lower_a.pdf}
\vspace{-2mm}
\caption{An illustration of Case A.}
\label{fig:upper-lower-a}
\vspace{-2mm}
\end{figure}

\underline{Case B:}~We begin with the join and split trees of $f$, denoted as $J$ and $S$, respectively. Let $J_\varepsilon$, $S_\varepsilon$, and $T_\varepsilon$ denote the joint, split, and contour  trees of $f_\varepsilon$, respectively. $J_{\varepsilon}$ and $S_{\varepsilon}$ combine to form the simplified contour tree $T_{\varepsilon}$. In Case B, if an edge in $J$ that contains $\pi(x)$ was removed after persistent simplification, then $\pi(x)$ maps to the saddle of the removed branch. We first compute edges $e'_J$ and $e'_S$ from $J_{\varepsilon}$ and $S_{\varepsilon}$ respectively as follows.    

To compute $e'_J$, we first find an edge $e_J$ of $J_\varepsilon$ whose segmentation region contains $x$. 
That is, $\pi(x) \in e_J$. Suppose that $e_J$ has endpoints $j_1$ and $j_2$. We then consider two cases for $x$.

\underline{Case B.1:} $f(j_1) < f(x) < f(j_2)$. In this case, we set $e_J' = e_J$. As shown in~\cref{fig:upper-lower-b-1}, $x$ belongs to the segmentation region of edge $j_1j_2$ in $J_{\varepsilon}$, i.e., $\pi(x) \in j_1j_2$, and $f(j_1) < f(x) < f(j_2)$; so we set $e_J' = e_J = j_1j_2$.

\begin{figure}[!ht]
\centering
\includegraphics[width=0.6\linewidth]{fig-upper_lower_b_1.pdf}
\vspace{-2mm}
\caption{An illustration of Case B.1.}
\label{fig:upper-lower-b-1}
\vspace{-2mm}
\end{figure}

\underline{Case B.2:} $f(x) < f(j_1)$. In this case, let $m$ be the lowest leaf that descends from $e_J$ in $J_\varepsilon$. Then there exists a unique path connecting $j_1$ with $m$ that contains a sequence of descendant edges $e_1, e_2, \ldots$, and so on. There exists exactly one edge $e_k$ with endpoints $a_k$ and $b_k$ such that $f(a_k) < f(x) < f(b_k)$. We set $e_J' = e_k$. As shown in~\cref{fig:upper-lower-b-2}, we see that $x$ belongs to the segmentation region for edge $j_1j_2$ in $T_{\varepsilon}$, but $f(j_1) > f(x)$. We identify the edge $e_k = j_0j_1$ such that $f(j_0) < f(x) < f(j_1)$. Thus, we instead set $e_J' = j_0j_1$. 

\begin{figure}[!ht]
\centering
\includegraphics[width=0.6\linewidth]{fig-upper_lower_b_2.pdf}
\vspace{-2mm}
\caption{An illustration of Case B.2.}
\label{fig:upper-lower-b-2}
\vspace{-2mm}
\end{figure}

We compute an edge $e'_S$ from $S_{\varepsilon}$ similarly to $e'_J$. Let $R'_J$ and $R'_S$ be the segmentation region of $e'_J$ and $e'_S$ respectively. 
Then $R'_J \cap R'_S$ will intersect the segmentation region of a single edge $ab$ of $T_{\varepsilon}$, and edge $ab$  will satisfy $f(a) < f(x) < f(b)$. 
We then set $L(x) = f(a) + \zeta$ and $U(x) = f(b) - \zeta$, where $\zeta = 10^{-5}|f(a)-f(b)|$, as described in \cref{sec:method-overview}.

\subsection{False Case Detection}
\label{sec:false-case-figures}

In~\cref{sec:augment-tightening}, we introduce false case detection as part of the progressive upper and lower bound tightening process. 
We illustrate Case (I), Case (II.a) and Case (II.b) respectively in~\cref{fig:false-case-figures}. 

We grow a minimum $m$ until it reaches some saddle $s$. 
In Case (I), we handle the situation where $s$ is unpaired. 
In Case (II), we handle the situation where $s$ has already been paired with another local minimum $m'$. As a result, $m$ must pair with some other saddle $s'$ with $f(s') > f(s)$.

In Case (II.a), $|f(m') - f(s)| \geq \varepsilon$. It follows that $|f(m) - f(s')| \geq \varepsilon$. As a result, we know that $m$ has $s$ as its parent in the simplified join tree.
In case (II.b), $|f(m') - f(s)| \leq \varepsilon$. Thus, the edge $m's$ is not part of the simplified join tree. Instead, we search for some new saddle $s''$ that is the parent of $s$. Then, we can check the cases (I) and (II) again.

\begin{figure}[!ht]
\centering
\includegraphics[width=0.8\linewidth]{fig-false_case_figures.pdf}
\caption{From left to right: illustrations of Cases (I), (II.a) and (II.b) of the false case detection process respectively.}
\label{fig:false-case-figures}
\end{figure}



\subsection{Special Cases}
\label{sec:special-cases}

There are a number of special cases that must be handled during upper and lower bound tightening. Let $f$ represent the ground truth scalar field. Let $f'$ be the decompressed scalar field. Recall that the upper and lower bound tightening step works on computing the contour tree of $f'$. Let $T$ be the merge tree of $f$, and let $T_\varepsilon$ be $T$ after persistence simplification.
Let $T'$ and $T'_\varepsilon$ be defined analogously for $f'$.

When applicable, assume that we are working with the join tree. All cases apply analogously for the split tree.

\para{Simulation of Simplicity.} 
It is possible that points may have the same scalar field value, making the function non-Morse. To resolve this issue, we use Simulation of Simplicity~\cite{edelsbrunner1990simulation}. If points $x$ and $y$ have the same scalar field value, we break the tie by treating $x$ as larger if its value is stored in memory after that of $y$ (assuming a Fortran order). %Gueunet et al. used the same resolution in their merge tree algorithm~\cite{gueunet2017task}.

\para{Higher-order saddles.} 
In a typical merge tree, each interior node (saddle) is connected to exactly two local minima. However, it is possible to have monkey saddles (i.e., saddles that are connected to three local minima) and higher-order saddles (i.e., saddles with more than three local minima connections).

%has exactly two children. However, it is possible that an interior node can have three or more children.

%, and multiple children are paired with that saddle for persistence. This can lead to difficulties with our algorithm. For example, suppose that a saddle $s$ has three children (local minima) $m_1$, $m_2$ and $m_3$ with $f(m_1) > f(m_2) > f(m_3)$. In this degenerate case, $m_1$ and $m_2$ are both paired with saddle $s$ for persistence. If the error bound tightening step were executed precisely as is written in \cref{sec:augment-tightening}, $m_1$ would be grown before $m_2$ and $m_3$ because it has the largest scalar field value. When $m_1$ is grown, it would pair with $s$. After this, $m_2$ would be grown, but $m_2$ would not yet be paired with $s$. This leads to a mis-pairing of critical points which would lead to compression errors.

To obtain the correct pairings involving higher-order saddles, we proceed as follows. 
Let $s$ be an interior saddle point. 
Let $N$ be the set of neighbors of saddle $s$. 
Let $L(s) = \{x \in N : f'(x) < f'(s)\}$ be the \textit{lower link} of $s$. When the growth of a local minimum $m$ terminates at $s$, we verify whether each point in the lower link $L(s)$ has already been visited during the growth of some child of $s$ (including possibly $m$). If every point in $L$ has not yet been visited, then we pair $m$ with $s$. Otherwise, $m$ must pair with some other saddle $s'$. This strategy is adapted from the merge tree algorithm of Gueunet et al.~\cite{gueunet2017task}.

\para{Out-of-order growth.} Let $s$ be an interior saddle point. Let $m_1$ and $m_2$ be its children in the join tree, and assume that $f(m_1) > f(m_2)$. Thus, $s$ should pair with $m_1$. In some specific cases, it is possible that the growth of $m_2$ will reach saddle $s$ before $m_1$. This can lead to problems with persistence pairing. This situation can occur if, for example, $m_1$ is first grown to reach some other saddle $s'$; then, $m_2$ is grown to reach $s$ and pairs with $s$. Finally, due to a false case, $m_1$ must be re-grown, and when it is re-grown, its growth now terminates at $s$.

We handle this situation as follows. Let $s$ be an interior saddle point that has already been paired with points $\{m_i\}$. Let $m$ be a new point whose growth just terminated at $s$. If, for some $i$, we have that $f(m_i) < f(m)$, then an out-of-order growth occurs. To resolve this issue, we treat it as a false positive. Similarly, if the growth of $m$ terminates at $s$, but $s$ has already been grown as well, then this also signifies an out-of-order growth. We likewise treat this case as a false positive.

\para{Simultaneous false positive and false negative.} 
Suppose that when growing a minimum $m$, we discover that edge $ms \in T'_\varepsilon$ but $ms \notin T_\varepsilon$. As a result, $ms$ represents a false positive. Further, suppose that there exists some saddle $s'$ such that $ms' \in T_\varepsilon$. When this occurs, $ms'$ is a false negative. Therefore, $m$ is associated with both a false positive and a false negative. Let $R_p$ be the set of points associated with false positive $ms$. Let $R_n$ be the set of points associated with false negative $ms'$. In this case, we tighten the error bounds around all points in the region $R = R_p \cup R_n$.

When computing the average time per false case fixed in \cref{tab:time}, we treat this instance as a single false case (rather than two) as it is handled the same way as a single false case, except that $R$ is larger. In the `FC' column this case is marked as a false positive.

\subsection{Upper and Lower Bound Tightening}
\label{sec:tighten-bounds}

In this section, we give a detailed summary of the upper and lower error bound tightening process of TopoSZ, and compare it against our progressive bound tightening process.

\para{TopoSZ Bound Tightening.} 
For each false case that is detected, TopoSZ calculates a region $R$ of $\X$. Then, it tightens the upper and lower bounds $U(x)$ and $L(x)$ of points in $x \in R$. The region $R$ and how aggressively $L(x)$ and $U(x)$ are tightened depends on how many iterations are conducted to eliminate false cases. 
Let $n$ be the number of iterations (for Step 4 of TopoSZ), with $n=1$ initially. Let $f$ represent the input (ground truth) scalar field, and let $f'$ represent the decompressed scalar field. 
Let $T$ and $T'$ respectively represent their contour trees. 
Let $T_\varepsilon$ and $T'_\varepsilon$ respectively represent the  contour trees after persistence simplification with a threshold $\varepsilon$.

First, a region $R$ is calculated. Define an $m$-layer neighborhood of a point $x$ as the set of all points $y$ such $||x-y||_\infty \leq m$. In the event of a false positive edge $e'$ that is present in $T'_\varepsilon$ but not $T_\varepsilon$, $R$ is initially set equal to the region corresponding to $e'$ in the segmentation induced by $T'_\varepsilon$. Let $s'$ be the saddle point that is a vertex of $e'$. An $n$-layer neighborhood surrounding $s'$ is added to $R$.

In the event of a false negative edge $e$ that is in $T_\varepsilon$ but absent in $T'_\varepsilon$, $R$ is initially set equal to the region corresponding to $e$ in the segmentation induced by $T'_\varepsilon$. $R$ is then expanded by adding an $n$-layer neighborhood of $R$. TopoSZ handles false types in the same way that it handles false negatives.

Then, TopoSZ tightens the $L(x)$ and $U(x)$ bounds around points $x \in R$. Let $k_0 = \min\{f(x) : x \in R\}$ and $k_{n+1} = \max\{ f(x) : x \in R \}$. TopoSZ calculates $n+1$ intervals $[k_0, k_1], [k_1,k_2], \ldots, [k_n,k_{n+1}]~\subset~\R$ such that, for each interval $I$, approximately $\frac{1}{n+1}$ of points $x \in R$ satisfy $f(x) \in I$. For each point $x \in R$, if $f(x) \in [k_i, k_{i+1}]$, then $L(x)$ and $U(x)$ are adjusted according to $L(x) \leftarrow \max(L(x), k_i)$ and $U(x) \leftarrow \min( U(x), k_{i+1})$.

\para{Difference with our framework.} 
In our progressive error tightening procedure, we compute the regions $R$ in mostly the same manner as TopoSZ, but with changes. First, because we work with merge trees, we use the merge-tree-induced segmentation, rather than the contour-tree-induced segmentation. 

Second, in the case of a false positive, we do not grow the region around the saddle for the first three iterations; and in the case of a false negative, we do not grow the region for the first three iterations. We have found that this strategy leads to less points with tighter error bounds. On the other hand, we divide $R$ into $2^n$ intervals, rather than $n+1$, as we have found that this leads to faster convergence. Additionally, if a false positive persists for at least six iterations, we instead handle it the same way as a false negative. We have found that this also leads to faster convergence.
\section{Compressor Configurations and Reconstruction Quality Curves}
\label{sec:reconstruction-quality-extra}

The TTK implementation of TopoQZ is tightly coupled with ZFP, which relies on two parameters: the persistence threshold $e$ and the pointwise error bound $\zeta$ associated with ZFP. 
This gives rise to a total pointwise error upper-bounded by $e+\zeta$. Admittedly, this bound may not be tight and we estimate the pointwise error bound of TopoQZ to be lower-bounded by $\max\{e, \zeta\}$ and upper-bounded by $e+\zeta$. Decreasing this bound will generally improve the run time of TopoQZ.   

We create plots demonstrating the trade off between bit-rate and respectively PSNR, Wasserstein distance $d_W$ and bottleneck distance $d_B$ for each augmented compressor, as well as TopoQZ and TopoSZ. 
Plots for Earthquake and QMCPACK are included in \cref{fig:reconstruction-quality}. Plots for the Tangaroa, Ionization, and Isabel datasets are included in \cref{fig:reconstruction-quality-appendix}. 
For the larger datasets, we are unable to compute the topological distances $d_W$ and $d_B$ using reasonable computational resources. 
To that end, for the Miranda, S3D, and SCALE-LETKF datasets, we only include plots of bit-rate vs PSNR, which can be found in \cref{fig:reconstruction-quality-appendix}. 
In the latter two large datasets (S3D and SCALE-LETKF), Augmented Neurcomp performs very poorly, and is an outlier. 
Thus, we do not plot its performance curves. 
For the Nyx dataset, because each individual trial has such a long run time, we do not generate any such curves. 
Some of the topological distances $d_W$ and $d_B$ could not be computed in reasonable time due to the large size or topological complexity of the dataset, thus are not plotted.

Most of the data is obtained for the augmented compressors and TopoSZ by setting $\varepsilon = 0.04$ and varying 
$\xi \in \{0.003, 0.006, 0.009, 0.012, 0.015, 0.018\}$. 
For TopoSZ, in order to obtain a wide range of values, we set $e = 0.04$ and vary $\zeta \in \{$ 0.003, 0.11, 0.22, 0.33, 0.44, 0.55 $\}$.

However, for roughly half of the curves associated with the augmented compressors, we use different values of $\xi$ due to a peculiarity that we observe. Under normal circumstances, if a compressor uses more bits to encode each data point (i.e., the bit-rate is higher), then it can more accurately reconstruct the data.
In some cases, however, we notice a negative correlation between bit-rate and reconstruction quality. This is unusual for compressors.

We do not know precisely why this trend sometimes occurs. We offer one possible explanation. For each $x \in \X$, let $U(x)$ and $L(x)$ denote the initial upper and lower bounds, respectively, for $x$, before error bound tightening. Recall that, before error bound tightening, it holds that $f(x) - \xi \leq L(X) \leq f'(x) \leq U(X) \leq f(x) + \xi$. Therefore, as $\xi$ decreases, the initial guess $f'(x)$ will become more accurate. Thus, as $\xi$ decreases, more of the contour tree will be preserved before the tightening step, and less tightening will occur. In practice, we notice that error bound tightening does not affect PSNR very much, but it does increase the bit-rate. As a result, if less error bound tightening occurs, then the bit-rate could be lower for lower values of $\xi$. We notice that when $\xi$ becomes very small, increasing $\xi$ always leads to increased bit-rate.

For each combination of a dataset and a compressor, we ensure that every time $\xi$ increases, the bit-rate also decreases. In order to do this, we sometimes need to vary $\xi$ for values other than the standard $0.003$ through $0.018$. To choose values for $\xi$, we typically search for new, lower values of $\xi$. However, for Augmented Neurcomp, due to its very long run times, we simply use  a subset of $\{0.003,0.006,0.009,0.012,0.015,0.018\}$ for which increasing $\xi$ always leads to decreasing bit-rate. We handle  TopoSZ and TopoQZ similarly to Augmented Neurcomp when the run times are long. The values of $\xi$ used to generate the plots are shown in \cref{tab:odd-parameter-configurations}.

\begin{figure}[!ht]
\centering
\includegraphics[width=0.8\columnwidth]{fig-rate_distortion_appendix.png}
\vspace{-4mm}
\caption{PSNR, bottleneck distance, and Wasserstein distance versus bit-rate for each compressor for the Tangaroa, Ionization, and Isabel datasets with $\varepsilon = 0.04$ ($e = 0.04$ for TopoQZ).}
\label{fig:reconstruction-quality-appendix}
\vspace{-2mm}
\end{figure}

\begin{figure}[!ht]
\centering
\includegraphics[width=0.8\columnwidth]{fig-rate_distortion_appendix_no_topology.png}
\vspace{-4mm}
\caption{PSNR versus bit-rate for each compressor for the Miranda, S3D, and SCALE-LETKF datasets with $\varepsilon = 0.04$ and $e = 0.04$ for TopoQZ. Augmented Neurcomp is an outlier for S3D and SCALE-LETKF (bit-rate $\approx 9$) and is omitted.}
\label{fig:reconstruction-quality-no-topology}
%\vspace{-2mm}
\end{figure}

\begin{table}[!ht]
\vspace{2mm}
\resizebox{\columnwidth}{!}{\begin{tabular}{ll|l}
\hline
Dataset                                                  & Compressor                                                & Error Bounds                                                                                                                 \\ \hline
QMCPACK                                                  & SZ3                                                       & 0.0006, 0.00108, 0.00156, 0.00204, 0.00252                                                                                   \\
QMCPACK                                                  & TTHRESH                                                   & 0.0005, 0.0006, 0.0007, 0.0008, 0.0009, 0.001                                                                                \\
QMCPACK                                                  & TopoSZ                                                    & 0.002, 0.003, 0.006, 0.009, 0.012, 0.015                                                                                     \\ \hline
Tangaroa                                                 & ZFP                                                       & 0.0001, 0.0005, 0.001, 0.003, 0.006, 0.009                                                                                   \\
Tangaroa                                                 & SZ3                                                       & 0.0005, 0.00075, 0.001, 0.002, 0.003, 0.006                                                                                  \\
Tangaroa                                                 & CSI                                                       & 0.0005, 0.001, 0.0015, 0.002, 0.003, 0.006                                                                                   \\
Tangaroa                                                 & TTHRESH                                                   & 0.001, 0.0015, 0.002, 0.003, 0.006, 0.009                                                                                    \\
Tangaroa                                                 & Neurcomp                                                  & 0.003, 0.006, 0.009                                                                                                          \\ \hline
Earthquake                                               & SZ3                                                       & 0.001, 0.002, 0.003, 0.006, 0.009, 0.012                                                                                     \\
Earthquake                                               & CSI                                                       & 0.001, 0.002, 0.003, 0.006, 0.009, 0.012                                                                                     \\
Earthquake                                               & TTHRESH                                                   & 0.001, 0.002, 0.003, 0.006, 0.009, 0.012                                                                                     \\
Earthquake                                               & Neurcomp                                                  & 0.003, 0.006                                                                                                                 \\ \hline
Ionization                                               & ZFP                                                       & 0.001, 0.003, 0.006, 0.009, 0.012, 0.015                                                                                     \\
Ionization                                               & SZ3                                                       & 0.001, 0.002, 0.003, 0.006, 0.009, 0.012                                                                                     \\
Ionization                                               & Neurcomp                                                  & 0.003, 0.006, 0.009, 0.012                                                                                                   \\
Ionization                                               & TopoSZ                                                    & 0.003, 0.006, 0.009, 0.012                                                                                                   \\ \hline
Isabel                                                   & CSI                                                       & 0.001, 0.0015, 0.002, 0.0025, 0.003, 0.006                                                                                   \\
\begin{tabular}[c]{@{}l@{}}Isabel\\ $\quad$\end{tabular} & \begin{tabular}[c]{@{}l@{}}TTHRESH\\ $\quad$\end{tabular} & \begin{tabular}[c]{@{}l@{}}0.002, 0.003, 0.004, 0.005, 0.006, 0.009;\\ Only 0.003 is shown for $d_W$ and $d_B$.\end{tabular} \\
Isabel                                                   & TopoSZ                                                    & Only 0.003 is shown for $d_W$ and $d_B$                                                                                      \\ \hline
Miranda                                                  & TTHRESH                                                   & 0.002, 0.003, 0.006, 0.009, 0.012, 0.015                                                                                     \\ \hline
S3D                                                      & ZFP                                                       & 0.00005, 0.0001, 0.0002, 0.0003, 00005, 0.0007                                                                               \\
S3D                                                      & SZ3                                                       & 0.0001, 0.00025, 0.0003, 0.0004, 0.0005, 0.001                                                                               \\
S3D                                                      & CSI                                                       & 0.00008, 0.0001, 0.0003, 0.0005, 0.0007, 0.002                                                                               \\
S3D                                                      & TTHRESH                                                   & 0.003, 0.006, 0.009                                                                                                          \\ \hline
SCALE-LETKF                                              & ZFP                                                       & 0.0005, 0.0008, 0.001, 0.002, 0.003, 0.006                                                                                   \\
SCALE-LETKF                                              & SZ3                                                       & 0.001, 0.002, 0.003, 0.006, 0.009, 0.012                                                                                     \\
SCALE-LETKF                                              & CSI                                                       & 0.001, 0.003, 0.006, 0.009, 0.012, 0.015                                                                                     \\
SCALE-LETKF                                              & TTHRESH                                                   & 0.002, 0.003, 0.004, 0.005, 0.006, 0009                                                                                      \\ \hline
\end{tabular}}
\vspace{-3mm}
\caption{Parameter configurations used to generate \cref{fig:reconstruction-quality}, \cref{fig:reconstruction-quality-appendix}, and \cref{fig:reconstruction-quality-no-topology} if they are different from the default values.}
\label{tab:odd-parameter-configurations}
\end{table}
\section{Additional Experiments: Parameter Variations and Ablation Study}
\label{sec:other-experiments}

We describe additional experimental results with parameter variations and ablation studies. 
We describe the results by varying $\varepsilon$ and $\xi$ in~\cref{sec:vary-epsilon} and \cref{sec:vary-xi} respectively. We describe ablation studies for logarithmic-scaling quantization and progressive error bound tightening in~\cref{sec:ablation}. We only include results for the Earthquake and Ionization datasets due to space constraints. The trends reported are consistent across all datasets tested. In all plots of compression and decompression times, we omit the results of augmented Neurcomp as it is much slower than the other augmented compressors.

\subsection{Experiments with Varying Persistence Threshold}
\label{sec:vary-epsilon}

We measure the effect of varying the persistence threshold $\varepsilon$ on each of our six evaluation metrics for each augmented compressor on each dataset (except Nyx). We fix $\xi = 0.012$ and vary $\varepsilon \in \{0.02,0.03,0.04,0.05,0.06,0.07\}$. The results for the Earthquake and Ionization datasets are shown in~\cref{fig:epsilon-table}.

We observe that there is a clear positive correlation between $\varepsilon$ and compression ratio. This trend is reasonable: when $\varepsilon$ is larger, the simplified contour tree becomes simpler, and thus less precision is required to preserve the contour tree, leading to a high compression ratio. There is no clear correlation between $\varepsilon$ and any other evaluation metrics.

\begin{figure*}[!ht]
\centering
\includegraphics[width=\linewidth]{fig-epsilon_table_small.png}
\vspace{-6mm}
\caption{Each of the six evaluation metrics measured for the Earthquake and Ionization datasets as $\varepsilon$ varies. We fix $\xi = 0.012$. The augmented compressors are given as A-ZFP, A-SZ3, etc.}
\label{fig:epsilon-table}
\end{figure*}

\subsection{Experiments with Varying Pointwise Error Bound}
\label{sec:vary-xi}

We measure the effect of varying the pointwise error bound $\xi$ on each of our six evaluation metrics for each augmented compressor on four datasets (QMCPACK, Tangaroa, Earthquake, and Ionization). We fix $\varepsilon = 0.04$ and vary $\xi$ using the same values that were used to measure the trade-off between reconstruction quality and compression ratio. See \cref{sec:reconstruction-quality-extra} for specific parameter values.

The results are shown in \cref{fig:xi-table}. There is a clear positive correlation between $\xi$ and compression ratio. This trend is logical, as less information should need to be stored in order to maintain a higher error bound. Likewise, there is a clear negative trend between $\xi$ and reconstruction quality in terms of each PSNR, Wasserstein distance, and bottleneck distance. This trend is also logical as a higher error bound means that the reconstructed data does not need to be as faithful to the ground truth data. There is no clear trend between $\xi$ and compression or decompression time.

\begin{figure*}[]
    \includegraphics[width=\linewidth]{fig-xi_table_small.png}
    \vspace{-6mm}
    \captionof{figure}{Each of the six evaluation metrics measured for the Earthquake and Ionization datasets for each lossy compressor as $\xi$ varies. We fix $\varepsilon = 0.04$. The augmented compressors are given as A-ZFP, A-SZ3, etc.}
    \label{fig:xi-table}    
\end{figure*}

\subsection{Ablation Study}
\label{sec:ablation}

In our framework, there are two main components: logarithmic-scaling quantization and progressive bound tightening. We perform an ablation study to separately justify the benefits of these two components, in comparison with TopoSZ, which utilizes linear-scaling quantization and iterative bound tightening. 
We independently analyze each of the six evaluation metrics across five datasets, for all augmented compressors. 

In addition to studying iterative bound tightening based on the contour tree (as is used by TopoSZ), we also implement an iterative bound tightening based on the merge tree for the ablation study, where we iteratively tighten the upper and lower bounds until there are no false cases in the join and split trees respectively. For linear-scaling quantization, we use quantization intervals of width $\xi$ (like in TopoSZ), rather than the standard width of $2\xi$.

The results of the ablation study are shown in~\cref{fig:ablation-table}. The augmented compressors are given by A-ZFP, A-SZ3, etc. In each line chart, columns whose label start with `Lin' use linear-scaling quantization, whereas columns whose label starts with `Log' use logarithmic-scaling quantization. Columns whose labels end with `ItrCT,' `ItrMT,' and `Prog' refer to trials that use contour-tree-based iterative tightening, merge-tree-based iterative tightening, and progressive tightening respectively. For compression times, we omit the time collected for Ionization dataset using CSI and Log/ItrCT as it was much higher than the times reported by other compressors and configurations.

In \cref{fig:ablation-table}, we can see that logarithmic-scaling quantization always produces a higher compression ratio than linear-scaling quantization. Progressive or iterative tightening do not noticeably affect the compression ratio. The PSNR is essentially constant across each trial, although logarithmic-scaling quantization typically produces slightly lower PSNR values than linear-scaling quantization. The Wasserstein distance is similar across trials, but sometimes we observe slight increases to Wasserstein distance when using logarithmic-scaling quantization rather than linear-scaling quantization. We do see large fluctuations in the bottleneck distance between trials, and sometimes logarithmic-scaling quantization leads to an increase. However, across all trials there is no clear pattern.

In terms of compression time, we can see that progressive bound tightening outperforms iterative bound tightening in every trial. Also, it appears that iterative tightening based on the contour tree outperforms iterative tightening based on the merge trees in most trials. This trend suggests that the primary benefit of tightening based on the join and split trees is that doing so enables progressive tightening strategies. When progressive tightening is used, using logarithmic-scaling quantization sometimes leads to slightly longer run times than linear-scaling quantization. Neither the tightening strategy or the quantization strategy appear to offer any advantage or disadvantage in terms of decompression time.

\begin{figure*}[!ht]
    \includegraphics[width=\linewidth]{fig-ablation_small.png}
    \vspace{-6mm}
    \captionof{figure}{Each of the six evaluation metrics measured for the Earthquake and Ionization datasets for each lossy compressor in an ablation study. Augmented compressors are given as A-ZFP, A-SZ3, etc. Run times from augmented Neurcomp, and those from augmented CSI are excluded as they are much higher compared to the others. Each point on the $x$-axis represents a combination of one quantization strategy (linear-scaling vs logarithmic-scaling) and one tightening strategy: iterative with contour trees such as in TopoSZ; iterative with merge trees; or progressive bound tightening.
    Lin: linear scaling quantization. 
    Log: logarithmic-scaling quantization. 
    Prg: progressive upper and lower bound tightening. 
    ItrCT: contour-tree based iterative upper and lower bound tightening. 
    IterMT: merge-tree based iterative upper and lower bound tightening.}
    \label{fig:ablation-table}
\end{figure*}
\section{Visual Artifacts Due to Lossy Compression}
\label{sec:visual-artifacts}

\begin{figure*}[]
\centering
  \begin{subfigure}[b]{0.02\textwidth}
    \raisebox{1.8\height}{\includegraphics[angle=90,width=\textwidth]{fig-colorBar}}
  \end{subfigure}
  \begin{subfigure}[b]{0.97\textwidth}
    \includegraphics[width=\linewidth]{fig-vr_table_adversarial.png}
  \end{subfigure}
\subfigsCaption{
Volume renderings of decompressed scientific datasets when compressed with augmented TTHRESH versus the input (ground truth), using transfer functions designed to produce the maximal visual difference between the two.
Top row: volume renderings of the input data. 
Bottom row: volume renderings of the decompressed data after compression with Augmented TTHRESH. 
Datasets from left to right: QMCPACK, Earthquake, Miranda, S3D, and Nyx.}
\label{fig:volume-render-adversarial}
\end{figure*}

Lossy compressors allow for some distortion of the data to achieve smaller compressed file sizes. Since we work with augmented lossy compressors, we might observe visual artifacts in the decompressed data due to the distortion introduced during the compression process. Some artifacts become  more visible depending on the choice of transfer functions during volume rendering. 

%When performing volume renderings of decompressed data after it is compressed with our framework, we notice some visual artifacts due to the selection of transfer functions. 

If the visualization is particularly sensitive to small changes in the transfer function, the decompressed data can appear lighter or darker than the input (ground truth) when visualized with the same transfer function. This typically occurs when one region of a volume has relatively uniform values. 

We include visualizations of the visual artifacts for five datasets for augmented TTHRESH in \cref{fig:volume-render-adversarial}. 
We cannot find volume renderings that produce noticeable artifacts for the remaining four datasets. 
We do not consider Neurcomp as its augmented model preforms the worst. The remaining augmented compressors yield similar artifacts. The Miranda dataset appears to contain the most visual artifacts, where a turbulent front divides two nearly uniform regions.
%-------------------------

\end{document}

