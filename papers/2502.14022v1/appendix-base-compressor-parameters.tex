\section{Compressor Parameter Settings}
\label{sec:base-compressor-parameters}

We discuss the parameter settings associated with base and augmented compressors. 
The parameter settings during augmentations are described in~\cref{sec:base-parameters-during-augmentation}. The parameter settings for ZFP and TTHRESH used to generate \cref{fig:teaser}, \cref{fig:zoom}, and \cref{tab:base-false-cases} are given in~\cref{sec:base-parameters-equivalent}.

\subsection{Parameter Settings During Augmentation}
\label{sec:base-parameters-during-augmentation}

When augmenting ZFP, SZ3, and TTHRESH, to preserve a pointwise error bound $\xi$, we set the error bound $\delta$ of ZFP to be $5\xi$, the error bound $\eta$ of SZ3 to be $0.25\xi$, and the target PSNR $\tau$ of TTHRESH to be $0.05\xi$. 
It seems intuitive that setting $\delta = \eta = \tau = \xi$ would be optimal. We provide a brief justification for why our strategy works better.

In the case of TTHRESH and SZ3, these parameter settings cause the base compressors to preserve data with higher reconstruction quality. If the base compressor returns higher quality compression, then our topology-preserving framework will need to make fewer adjustments in order to preserve the contour tree. 
We have found that the extra storage space used by the base compressors to create higher quality intermediate data is outweighed by the space saved by storing fewer adjustments, causing compression ratios to increase. In addition to improved compression ratios, if the intermediate data more closely resembles the input (ground truth) data, there will be fewer false cases, leading to lower augmentation time. Further, if the intermediate data is closer to the input data, then after augmentation, the decompressed data will also be closer to the input data, leading to a higher PSNR.

ZFP is very conservative when it comes to error bounds and actually preserves an error bound much lower than the maximum error bound set by the user. We notice empirically that by multiplying the maximum error bound for ZFP by five, this raises the actual error bound preserved by ZFP up to something closer to $\xi$, allowing for ZFP to produce higher compression ratios while still yielding intermediate data of acceptable reconstruction quality. 
Surprisingly, we observe that setting $\delta$ very high, such as $\delta = 50\xi$ yields even higher compression ratios than $\delta = 5\xi$. On the other hand, the compression ratios are not too high such that ZFP obtains a comparative advantage in terms of compression ratio, and the compression times are slow enough that ZFP loses its comparative advantage on time efficiency. Therefore, we determine that $\delta = 5\xi$ is the best parameter setting for ZFP.

\subsection{Parameters for Equivalent Compression Ratios}
\label{sec:base-parameters-equivalent}

In order to generate \cref{fig:teaser}, \cref{fig:zoom}, and \cref{tab:base-false-cases}, we need to identify parameter settings for the absolute error bound parameter $\delta$ of ZFP and the RMSE target parameter $\tau$ for TTHRESH which produce compression ratios comparable to Augmented ZFP and 
Augmented TTHRESH with $\epsilon = 0.04$ and $\xi = 0.012$. The values for $\delta$ and $\tau$ that produce these comparable compression ratios are reported in~\cref{tab:base-false-cases}.

\begin{table}[!h]
\centering
\begin{tabular}{l|ll}
\hline
Dataset    & ZFP $\delta$ & TTHRESH $\tau$ \\ \hline
QMCPACK    & 3.11e-2    & 7.96e-6        \\ \hline
Tangaroa   & 3.60e-2    & 2.15e-4        \\ \hline
Earthquake & 3.60e-2    & 3.66e-4        \\ \hline
Ionization & 4.98e-2    & 3.51e-4        \\ \hline
Isabel     & 1.25e-1    & 2.62e-4        \\ \hline
Miranda    & 2.28e-1    & 6.66e-5        \\ \hline
Nyx        & 1.97e-3    & 1.97e-3        \\ \hline
S3D        & 3.08e-5    & 3.43e-7        \\ \hline
SCALE-LETKF& 9.00e-4    & 2.95e-6        \\ \hline
\end{tabular}
\vspace{-2mm}
\caption{Parameter configurations used to obtain \cref{fig:teaser}, \cref{fig:zoom}, and \cref{tab:base-false-cases}.}
\vspace{-2mm}
\end{table}
In some cases, the parameter configuration for ZFP is higher than $0.012$. ZFP is very conservative in terms of error bounds, and for each of those trials, the actual error bound that is maintained is likely significantly lower than the stated maximum error bound.