\section{Algorithmic Details}
\label{sec:algorithm-details}

We describe how the initial upper and lower bounds are computed in~\cref{sec:set-bounds}. 
We provide an illustration for the false case detection process in~\cref{sec:false-case-figures}, followed by special cases in~\cref{sec:special-cases}. We describe the specifics surrounding the tightening process, including a comparison with TopoSZ in~\cref{sec:tighten-bounds}.

\subsection{Initial Upper and Lower Bounds}
\label{sec:set-bounds}

To compute the initial upper and lower bounds $U(x)$ and $L(x)$ for $x \in \X$, we aim to locate an edge $ab$ of the simplified contour tree $T_{\varepsilon}$ satisfying $f(a) \leq f(x) \leq (b)$. Recall that a contour tree $T$ of the data $(\X,f)$ arises from a quotient map $\pi: (\X, f) \to (\X/{\sim}, f)$. The contour tree $T_{\varepsilon}$ is defined analogously. With an abuse of notation, we use $\pi(x)$ to represent the image of $x$ under a quotient map $\pi$ (when $\pi$ is clear from the context). For instance, in \cref{fig:upper-lower-a}, $\pi(x)$ on the left arises from the quotient map defining $T$, whereas $\pi(x)$ on the right comes from the quotient map defining $T_{\varepsilon}$.   

\underline{Case A:}~As illustrated in \cref{fig:upper-lower-a}, let $e'=a'b'$ be an edge of $T_{\varepsilon}$ whose segmentation region contains $x$; that is, $\pi(x) \in e'$.  
If $f(a') \leq f(x) \leq f(b')$, then we set $ab := a'b'$. Otherwise, we need to locate $ab$ with some care, following Case B below. 

\begin{figure}[!ht]
\centering
\includegraphics[width=0.6\linewidth]{fig-upper_lower_a.pdf}
\vspace{-2mm}
\caption{An illustration of Case A.}
\label{fig:upper-lower-a}
\vspace{-2mm}
\end{figure}

\underline{Case B:}~We begin with the join and split trees of $f$, denoted as $J$ and $S$, respectively. Let $J_\varepsilon$, $S_\varepsilon$, and $T_\varepsilon$ denote the joint, split, and contour  trees of $f_\varepsilon$, respectively. $J_{\varepsilon}$ and $S_{\varepsilon}$ combine to form the simplified contour tree $T_{\varepsilon}$. In Case B, if an edge in $J$ that contains $\pi(x)$ was removed after persistent simplification, then $\pi(x)$ maps to the saddle of the removed branch. We first compute edges $e'_J$ and $e'_S$ from $J_{\varepsilon}$ and $S_{\varepsilon}$ respectively as follows.    

To compute $e'_J$, we first find an edge $e_J$ of $J_\varepsilon$ whose segmentation region contains $x$. 
That is, $\pi(x) \in e_J$. Suppose that $e_J$ has endpoints $j_1$ and $j_2$. We then consider two cases for $x$.

\underline{Case B.1:} $f(j_1) < f(x) < f(j_2)$. In this case, we set $e_J' = e_J$. As shown in~\cref{fig:upper-lower-b-1}, $x$ belongs to the segmentation region of edge $j_1j_2$ in $J_{\varepsilon}$, i.e., $\pi(x) \in j_1j_2$, and $f(j_1) < f(x) < f(j_2)$; so we set $e_J' = e_J = j_1j_2$.

\begin{figure}[!ht]
\centering
\includegraphics[width=0.6\linewidth]{fig-upper_lower_b_1.pdf}
\vspace{-2mm}
\caption{An illustration of Case B.1.}
\label{fig:upper-lower-b-1}
\vspace{-2mm}
\end{figure}

\underline{Case B.2:} $f(x) < f(j_1)$. In this case, let $m$ be the lowest leaf that descends from $e_J$ in $J_\varepsilon$. Then there exists a unique path connecting $j_1$ with $m$ that contains a sequence of descendant edges $e_1, e_2, \ldots$, and so on. There exists exactly one edge $e_k$ with endpoints $a_k$ and $b_k$ such that $f(a_k) < f(x) < f(b_k)$. We set $e_J' = e_k$. As shown in~\cref{fig:upper-lower-b-2}, we see that $x$ belongs to the segmentation region for edge $j_1j_2$ in $T_{\varepsilon}$, but $f(j_1) > f(x)$. We identify the edge $e_k = j_0j_1$ such that $f(j_0) < f(x) < f(j_1)$. Thus, we instead set $e_J' = j_0j_1$. 

\begin{figure}[!ht]
\centering
\includegraphics[width=0.6\linewidth]{fig-upper_lower_b_2.pdf}
\vspace{-2mm}
\caption{An illustration of Case B.2.}
\label{fig:upper-lower-b-2}
\vspace{-2mm}
\end{figure}

We compute an edge $e'_S$ from $S_{\varepsilon}$ similarly to $e'_J$. Let $R'_J$ and $R'_S$ be the segmentation region of $e'_J$ and $e'_S$ respectively. 
Then $R'_J \cap R'_S$ will intersect the segmentation region of a single edge $ab$ of $T_{\varepsilon}$, and edge $ab$  will satisfy $f(a) < f(x) < f(b)$. 
We then set $L(x) = f(a) + \zeta$ and $U(x) = f(b) - \zeta$, where $\zeta = 10^{-5}|f(a)-f(b)|$, as described in \cref{sec:method-overview}.

\subsection{False Case Detection}
\label{sec:false-case-figures}

In~\cref{sec:augment-tightening}, we introduce false case detection as part of the progressive upper and lower bound tightening process. 
We illustrate Case (I), Case (II.a) and Case (II.b) respectively in~\cref{fig:false-case-figures}. 

We grow a minimum $m$ until it reaches some saddle $s$. 
In Case (I), we handle the situation where $s$ is unpaired. 
In Case (II), we handle the situation where $s$ has already been paired with another local minimum $m'$. As a result, $m$ must pair with some other saddle $s'$ with $f(s') > f(s)$.

In Case (II.a), $|f(m') - f(s)| \geq \varepsilon$. It follows that $|f(m) - f(s')| \geq \varepsilon$. As a result, we know that $m$ has $s$ as its parent in the simplified join tree.
In case (II.b), $|f(m') - f(s)| \leq \varepsilon$. Thus, the edge $m's$ is not part of the simplified join tree. Instead, we search for some new saddle $s''$ that is the parent of $s$. Then, we can check the cases (I) and (II) again.

\begin{figure}[!ht]
\centering
\includegraphics[width=0.8\linewidth]{fig-false_case_figures.pdf}
\caption{From left to right: illustrations of Cases (I), (II.a) and (II.b) of the false case detection process respectively.}
\label{fig:false-case-figures}
\end{figure}



\subsection{Special Cases}
\label{sec:special-cases}

There are a number of special cases that must be handled during upper and lower bound tightening. Let $f$ represent the ground truth scalar field. Let $f'$ be the decompressed scalar field. Recall that the upper and lower bound tightening step works on computing the contour tree of $f'$. Let $T$ be the merge tree of $f$, and let $T_\varepsilon$ be $T$ after persistence simplification.
Let $T'$ and $T'_\varepsilon$ be defined analogously for $f'$.

When applicable, assume that we are working with the join tree. All cases apply analogously for the split tree.

\para{Simulation of Simplicity.} 
It is possible that points may have the same scalar field value, making the function non-Morse. To resolve this issue, we use Simulation of Simplicity~\cite{edelsbrunner1990simulation}. If points $x$ and $y$ have the same scalar field value, we break the tie by treating $x$ as larger if its value is stored in memory after that of $y$ (assuming a Fortran order). %Gueunet et al. used the same resolution in their merge tree algorithm~\cite{gueunet2017task}.

\para{Higher-order saddles.} 
In a typical merge tree, each interior node (saddle) is connected to exactly two local minima. However, it is possible to have monkey saddles (i.e., saddles that are connected to three local minima) and higher-order saddles (i.e., saddles with more than three local minima connections).

%has exactly two children. However, it is possible that an interior node can have three or more children.

%, and multiple children are paired with that saddle for persistence. This can lead to difficulties with our algorithm. For example, suppose that a saddle $s$ has three children (local minima) $m_1$, $m_2$ and $m_3$ with $f(m_1) > f(m_2) > f(m_3)$. In this degenerate case, $m_1$ and $m_2$ are both paired with saddle $s$ for persistence. If the error bound tightening step were executed precisely as is written in \cref{sec:augment-tightening}, $m_1$ would be grown before $m_2$ and $m_3$ because it has the largest scalar field value. When $m_1$ is grown, it would pair with $s$. After this, $m_2$ would be grown, but $m_2$ would not yet be paired with $s$. This leads to a mis-pairing of critical points which would lead to compression errors.

To obtain the correct pairings involving higher-order saddles, we proceed as follows. 
Let $s$ be an interior saddle point. 
Let $N$ be the set of neighbors of saddle $s$. 
Let $L(s) = \{x \in N : f'(x) < f'(s)\}$ be the \textit{lower link} of $s$. When the growth of a local minimum $m$ terminates at $s$, we verify whether each point in the lower link $L(s)$ has already been visited during the growth of some child of $s$ (including possibly $m$). If every point in $L$ has not yet been visited, then we pair $m$ with $s$. Otherwise, $m$ must pair with some other saddle $s'$. This strategy is adapted from the merge tree algorithm of Gueunet et al.~\cite{gueunet2017task}.

\para{Out-of-order growth.} Let $s$ be an interior saddle point. Let $m_1$ and $m_2$ be its children in the join tree, and assume that $f(m_1) > f(m_2)$. Thus, $s$ should pair with $m_1$. In some specific cases, it is possible that the growth of $m_2$ will reach saddle $s$ before $m_1$. This can lead to problems with persistence pairing. This situation can occur if, for example, $m_1$ is first grown to reach some other saddle $s'$; then, $m_2$ is grown to reach $s$ and pairs with $s$. Finally, due to a false case, $m_1$ must be re-grown, and when it is re-grown, its growth now terminates at $s$.

We handle this situation as follows. Let $s$ be an interior saddle point that has already been paired with points $\{m_i\}$. Let $m$ be a new point whose growth just terminated at $s$. If, for some $i$, we have that $f(m_i) < f(m)$, then an out-of-order growth occurs. To resolve this issue, we treat it as a false positive. Similarly, if the growth of $m$ terminates at $s$, but $s$ has already been grown as well, then this also signifies an out-of-order growth. We likewise treat this case as a false positive.

\para{Simultaneous false positive and false negative.} 
Suppose that when growing a minimum $m$, we discover that edge $ms \in T'_\varepsilon$ but $ms \notin T_\varepsilon$. As a result, $ms$ represents a false positive. Further, suppose that there exists some saddle $s'$ such that $ms' \in T_\varepsilon$. When this occurs, $ms'$ is a false negative. Therefore, $m$ is associated with both a false positive and a false negative. Let $R_p$ be the set of points associated with false positive $ms$. Let $R_n$ be the set of points associated with false negative $ms'$. In this case, we tighten the error bounds around all points in the region $R = R_p \cup R_n$.

When computing the average time per false case fixed in \cref{tab:time}, we treat this instance as a single false case (rather than two) as it is handled the same way as a single false case, except that $R$ is larger. In the `FC' column this case is marked as a false positive.

\subsection{Upper and Lower Bound Tightening}
\label{sec:tighten-bounds}

In this section, we give a detailed summary of the upper and lower error bound tightening process of TopoSZ, and compare it against our progressive bound tightening process.

\para{TopoSZ Bound Tightening.} 
For each false case that is detected, TopoSZ calculates a region $R$ of $\X$. Then, it tightens the upper and lower bounds $U(x)$ and $L(x)$ of points in $x \in R$. The region $R$ and how aggressively $L(x)$ and $U(x)$ are tightened depends on how many iterations are conducted to eliminate false cases. 
Let $n$ be the number of iterations (for Step 4 of TopoSZ), with $n=1$ initially. Let $f$ represent the input (ground truth) scalar field, and let $f'$ represent the decompressed scalar field. 
Let $T$ and $T'$ respectively represent their contour trees. 
Let $T_\varepsilon$ and $T'_\varepsilon$ respectively represent the  contour trees after persistence simplification with a threshold $\varepsilon$.

First, a region $R$ is calculated. Define an $m$-layer neighborhood of a point $x$ as the set of all points $y$ such $||x-y||_\infty \leq m$. In the event of a false positive edge $e'$ that is present in $T'_\varepsilon$ but not $T_\varepsilon$, $R$ is initially set equal to the region corresponding to $e'$ in the segmentation induced by $T'_\varepsilon$. Let $s'$ be the saddle point that is a vertex of $e'$. An $n$-layer neighborhood surrounding $s'$ is added to $R$.

In the event of a false negative edge $e$ that is in $T_\varepsilon$ but absent in $T'_\varepsilon$, $R$ is initially set equal to the region corresponding to $e$ in the segmentation induced by $T'_\varepsilon$. $R$ is then expanded by adding an $n$-layer neighborhood of $R$. TopoSZ handles false types in the same way that it handles false negatives.

Then, TopoSZ tightens the $L(x)$ and $U(x)$ bounds around points $x \in R$. Let $k_0 = \min\{f(x) : x \in R\}$ and $k_{n+1} = \max\{ f(x) : x \in R \}$. TopoSZ calculates $n+1$ intervals $[k_0, k_1], [k_1,k_2], \ldots, [k_n,k_{n+1}]~\subset~\R$ such that, for each interval $I$, approximately $\frac{1}{n+1}$ of points $x \in R$ satisfy $f(x) \in I$. For each point $x \in R$, if $f(x) \in [k_i, k_{i+1}]$, then $L(x)$ and $U(x)$ are adjusted according to $L(x) \leftarrow \max(L(x), k_i)$ and $U(x) \leftarrow \min( U(x), k_{i+1})$.

\para{Difference with our framework.} 
In our progressive error tightening procedure, we compute the regions $R$ in mostly the same manner as TopoSZ, but with changes. First, because we work with merge trees, we use the merge-tree-induced segmentation, rather than the contour-tree-induced segmentation. 

Second, in the case of a false positive, we do not grow the region around the saddle for the first three iterations; and in the case of a false negative, we do not grow the region for the first three iterations. We have found that this strategy leads to less points with tighter error bounds. On the other hand, we divide $R$ into $2^n$ intervals, rather than $n+1$, as we have found that this leads to faster convergence. Additionally, if a false positive persists for at least six iterations, we instead handle it the same way as a false negative. We have found that this also leads to faster convergence.