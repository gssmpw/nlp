\section{Evaluation Metrics}
\label{sec:evaluationMetrics}

We evaluate our trials on several metrics. We measure compression ratio, compression time, and peak signal-to-noise ratio (PSNR), which are common metrics for evaluating compressors. 

The compression ratio is the size of the uncompressed file divided by the size of the compressed file, and higher compression ratios indicate smaller compressed file sizes. Related to this, the bit-rate is the number of bits used to compress a single data point. For floating point data, it is either equal to 32 divided by the compression ratio, or 64 divided by the compression ratio, depending on if the dataset is respectively single or double precision.

Peak signal-to-noise ratio (PSNR) is a number that measures reconstruction quality. If $R$ is the range of the data and $M$ is the mean squared error of our reconstruction, the PSNR is defined as:
\begin{equation}
  \label{eq:psnr}
  \text{PSNR} = 10 \log_{10}\left( \frac{ R^2 }{ M } \right).
\end{equation}
In general, higher PSNR values indicate higher reconstruction quality. We also measure the bottleneck distance \cite{cohen2005stability} and Wasserstein distance \cite[page 183]{edelsbrunner2022computational} to quantify the amount of topological control.

To define the Wasserstein and bottleneck distances, suppose that $f$ and $f'$ are two scalar fields that yield persistence diagrams $D$ and $D'$ respectively; see~\cite{EdelsbrunnerLetscherZomorodian2002} for an introduction to persistent homology. If $\varphi:D \rightarrow D'$ represents any bijection, and $q \geq 1$, we define the $q$-Wasserstein distance $W_q$ as:
\begin{equation}
W_q(D,D') = \left[ \inf_{\varphi:D\rightarrow D'} \sum_{p \in D} || p - \varphi(p)||_\infty^q \right]^{\frac{1}{q}}.
\end{equation}
For our evaluation, we set $d_W = W_2$ and $d_B = W_\infty$. In particular:
\begin{equation}
W_\infty(D,D') = \inf_{\varphi:D \rightarrow W'} \sup_{p \in D} ||p- \varphi(p)||_\infty. 
\end{equation}

In general, lower values of $W_q$ for any $q \geq 1$ indicate that the persistence diagrams $D$ and $D'$ are more similar. This in-turn means that the datasets that produced $D$ and $D'$ are more topologically similar. To compute $d_W$ and $d_B$ in reasonable time, we report approximate values~\cite{KerberMorozovNigmetov2016} with $1\%$ error that were computed after persistence simplification with a threshold of $1.5 \times 10^{-6}$ times the range of the data.