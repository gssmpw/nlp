\section{Related Works}
\noindent\textbf{Glow Removal. }
Driven by the nighttime flare image dataset Flare7K ____, several flare removal methods are based on supervised training ____. Flare7K, the first synthetic dataset for flare removal, has limitations in generality and type diversity, and the entanglement of light and noise ____ complicates flare removal process. 
Given real-world pairing challenges of flare datasets, unsupervised training is viable.
____ integrates layer decomposition and suppression in a network, using the estimated light-effects layer to guide an unsupervised glow suppression network. Image dehazing is challenging ____, ____ takes atmospheric light from a real haze image and renders it into a clear image, thereby suppressing the glow effect in self-supervised learning. ____ employs a retraining strategy and semi-supervised training based on localized luminance windows to suppress the glow effect. APSF was first introduced to computer vision by ____ with a physical imaging model under bad weather ____. ____ used a zero-shot approach to generate APSF and transfer the task of glow suppression to the glow generation learning task, which solved the challenge of uneven glow. APSF is also generally applied in dehazing ____. 
\textbf{However}, APSF is specifically derived to account for multiple scattering in the atmosphere and does not address glares from lens refraction inside the camera system during nighttime photography, which includes glow and "ghosting" effects.

\noindent\textbf{Reflection Removal. }
Specular reflection and lens flare are main concerns for image reflection. 
Data-driven reflection removal methods ____ have become progressively more popular as the first reflection removal dataset for transmitted images ____.
Earlier studies ____ on removing lens highlight flare formed by lenses often adopt a two-phase strategy, involving detection followed by removal. This approach often limits recognition to specific shapes, such as a "bright spot". ____ proposed an alternative method using deconvolution of the measured flare diffusion function to remove lens flare, mainly removal rounded flares.  
For nighttime reflective flares, ____ proposed an optical symmetry rule and developed a synthetic dataset for end-to-end training. \textbf{However}, the above studies addresses the physical formation of various types of lens flare induced by artificial night lighting nor proposes any solutions.
\begin{figure}[t]
\centering
\includegraphics[width=0.47\textwidth]{samples/Glow_Reflective.jpg}
\caption{(a) Optical path illustration from the light source to the imaging plane, where the blue line is the ideal light path. The orange line $s$ is the path of scattered light that produces glow flare $R_{s}$, and the red is the path of light refraction between lenses that produce the reflective flare (ghost) $R_{r}$, and $l$ is the line of incident light. (b) Example of lens glare.}
\label{op}
\end{figure}