% \begin{figure}[htbp]
\begin{figure}

\centering
\resizebox{\columnwidth}{!}{
\begin{tikzpicture}
  \tikzstyle{every node}=[font=\small]
  \sffamily

  \pgfdeclarelayer{-1}
  \pgfsetlayers{-1,main}
  \tikzset{zlevel/.style={%
        execute at begin scope={\pgfonlayer{#1}},
        execute at end scope={\endpgfonlayer}
      }}

  \begin{scope}[zlevel=main]
    % First example
    \node[draw=none, inner sep=0, text width=\linewidth, align=left] at (0, 0) (input1) {
      \textbf{Model Input:} What is the function of this sequence?GGCTG...TTTTCTGA 
    };

    \node[draw=none, inner sep=0, text width=\linewidth, align=left, below=0.2cm of input1.south, anchor=north] (truth1) {
      \textbf{Ground Truth:} \colorbox{lightgrey}{Olfactory receptor} 5A2-like (OR5A2L) is a member of the olfactory receptor gene family, which is involved in the \colorbox{lightgrey}{detection of volatile substances}, contributing to the sense of smell.
    };

    \node[draw=none, inner sep=0, text width=\linewidth, align=left, below=0.2cm of truth1.south, anchor=north] (olmo1) {
      {\color{olmodna}\textbf{\method}}: The mRNA for \colorbox{lightgreenRGB}{olfactory receptor family 5} subfamily \colorbox{lightred}{B member 109 ( OR5B109 )} encodes a protein that is part of the olfactory receptor family. These receptors are involved in the \colorbox{lightgreenRGB}{detection of odorant molecules} and and play a crucial role in the sense of smell.
    };

    % \node[draw=none, inner sep=0, text width=1\linewidth, align=left, below=0.2cm of olmo1.south, anchor=north] (nt1) {
    %   {\color{nttransformer}\textbf{DNABERT2\_Promoter:}} 0
    % };

    % % Second example
    % \node[draw=none, inner sep=0, text width=1\linewidth, align=left, below=0.7cm of nt1.south, anchor=north] (input2) {
    %   \textbf{Model Input:} GCATATCC...ATTAC \textbf{[task]} H3K79me3
    % };

    % \node[draw=none, inner sep=0, text width=1\linewidth, align=left, below=0.2cm of input2.south, anchor=north] (truth2) {
    %   \textbf{Ground Truth:} Yes
    % };

    % \node[draw=none, inner sep=0, text width=1\linewidth, align=left, below=0.2cm of truth2.south, anchor=north] (olmo2) {
    %   {\color{olmodna}\textbf{OLMoDNA:}} Yes
    % };

    % \node[draw=none, inner sep=0, text width=1\linewidth, align=left, below=0.2cm of olmo2.south, anchor=north] (nt2) {
    %   {\color{nttransformer2}\textbf{DNABERT2\_H3K79me3:}} 1
    % };
  \end{scope}

  \begin{scope}[zlevel=-1]
    \node[draw=black, fill=white, inner sep=0.25cm, rounded corners, drop shadow, fit={(input1) (truth1) (olmo1)}] (frame) {};
  \end{scope}

  \begin{scope}[zlevel=-1]
    \node[draw=none, fill=none, inner sep=0.075cm, fit={(frame)}] {};
  \end{scope}
\end{tikzpicture}
}
\vspace{-2em} 
\caption{\textbf{Demonstration of {\color{olmodna}\textsf{\method}}'s cross-modal capabilities.} Given a DNA sequence, \method could generate a natural language description for functional annotations.}
\label{fig:model-output}
\vspace{-2em} 
\end{figure}
