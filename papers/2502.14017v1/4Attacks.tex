\section{Attacks on OT devices: A glance}
\label{sec:ga_sbo}
\subsection{IT and OT overlap: An attack path }
The convergence of Information Technology (IT) and Operational Technology (OT) has brought significant benefits to industries, including improved efficiency, better data analysis, and enhanced decision-making. However, it has also introduced new cyber security challenges. 

\begin{figure}[ht!]
    \centering
   \includegraphics[trim={0.8cm 0.5cm 0.5cm 0.1cm},clip,width=0.92\linewidth]{images/IT-OT-1.pdf}

    \caption{An example IT-OT network. Here zone1 is IT network,Zone2 act as a bridge between IT and OT network, zone3 is the OT network. }
    \label{fig:itot}
\end{figure}

By Information Technology (IT), we refer to systems used for data-centric computing. Examples are Computers, servers, enterprise applications, networking hardware. On the other hand, Operational Technology (OT) includes hardware and software that detects or causes changes through direct monitoring and control of physical devices, processes, and events. Examples are Industrial control systems (ICS), SCADA systems, PLCs, sensors etc.
Modern industrial environments increasingly integrate IT systems with OT systems. Although IT/OT convergence leads to improved monitoring, data collection, and automation of industrial processes , it raises the cyber security concerns in OT networks.
There are various attack vectors. Some of them are listed as below: 
\begin{enumerate}
    \item Phishing and Social Engineering: Attackers use social engineering to gain initial access to IT networks, which can then be used as a pivot point to OT systems. A common example is Spear phishing emails targeting employees with access to both IT and OT systems.
    \item Exploits: Vulnerabilities in IT network components (e.g., routers, firewalls, softwares running on IT devices) can be exploited to access OT networks. Examples are Exploiting unpatched network devices that bridge IT and OT environments, or exploiting a version of installed software to gain foothold in the network.
    \item Malware injection: Once an IT node is compromised , a lateral movement is conducted to increase the foothold in the notwork, with aim to gain access or infiltrate  to the target OT system. Then, Malware specifically designed to target OT systems is planted. Examples are  Stuxnet worm, which targeted SCADA systems via infected USB drives.
    \item Supply Chain Attacks: Compromised third-party software or hardware can introduce vulnerabilities into both IT and OT systems. Examples are Infected updates or compromised software from vendors.
    \item Insecure Remote Access: Remote access solutions used for monitoring and controlling OT systems can be exploited if not properly secured. Examples are weak credentials or unencrypted remote access connections.
    
\end{enumerate}


\subsection{Security Considerations for IT-OT Systems}


Given the critical nature of OT systems, their security is paramount. Here are some key security considerations that can be segregated in two parts, first the concerns that arises in IT network and second, the concerns that are in OT networks:
\subsubsection{Security Concerns in IT networks}
\begin{enumerate}
    \item Network Segmentation: Isolating the SCADA network from other networks to reduce the risk of cyber-attacks.  Implement robust network segmentation between IT and OT networks can limit the impact of a potential breach. It can be done by using firewalls, VLANs, and DMZs to create isolated network zones. Example commercial firewalls are Cisco ASA, Palo Alto Networks, Fortinet, while open source firewalls are pfSense, OPNsense, IPFire. IPFire is an open-source firewall distribution based on Linux, designed for ease of use and high performance. For implementing VLANs (Virtual Local Area Networks) using open-source solutions, Linux has built-in support for VLANs through the vlan kernel module and the ip command from the iproute2 package. This method is highly flexible and allows for detailed VLAN configuration. Netplan is a utility for easily configuring networking on a Linux system, specifically on Ubuntu. The DMZs are Implemented using network firewalls and segmentation strategies.

    \item Access Control: It refers to implementing strict access controls to ensure that only authorized personnel can access the system. In practice, this approach emphasize on enforcement of strict access controls and least privilege principles for users with access to both IT and OT systems. It is also recommended to regularly review and audit access rights. Linux uses a traditional UNIX-like permission model that includes read (r), write (w), and execute (x) permissions for three categories of users: the file owner, the group, and others. Access Control Lists (ACLs) is another mechanism that provide a more flexible permission mechanism by allowing you to set permissions for individual users or groups beyond the owner, group, and others. Identity and Access Management (IAM) tools like Okta, Microsoft Azure Active Directory, Role-325 Based Access Control (RBAC) Implemented through Active Directory are also widely used.

    \item Regular Updates and Patching: Keeping the system updated with the latest security patches to mitigate vulnerabilities. Patch Management Tools:** WSUS, SCCM, Ivanti Patch for Windows;319
Ansible, Puppet for Linux. - **Vulnerability Management:** Tenable Nessus, Qualys, Rapid7 Nexpose.

    \item Monitoring and Detection:
   - Deploy advanced monitoring and intrusion detection systems (IDS) to identify suspicious activities in real-time.
   - Use Security Information and Event Management (SIEM) systems to correlate events from IT and OT systems. Deploying Intrusion Detection Systems (IDS) and Intrusion Prevention Systems (IPS) to monitor network traffic for suspicious activity are essential components of network security. IDS is designed to monitor network traffic and alert administrators about potential security breaches, attacks, or policy violations. IPS not only detects but also prevents potential security breaches by taking action based on pre-defined rules. Both can work either on network-based or host-based. Some famous open-source IDS/IPS are Snort, Suricata, OSSEC,  Bro/Zeek. Other tools that can be helpful are - Log Management Tools like Graylog (open-source log management tool that provides real-time log analysis and monitoring with features like Centralized log management, real-time search, customizable dashboards, alerting) , Forensic Analysis Tools (Autopsy-An open-source digital forensics platform for analyzing hard drives and smartphones with features like File system analysis, timeline analysis, keyword search, multimedia analysis) are all deployed and useful. SIEM (Security Information and Event Management) Tools (like Splunk with features like Event correlation, log management, compliance reporting etc), Endpoint Detection and Response (EDR) Tools  (like CrowdStrike Falcon, Microsoft Defender with festures like Endpoint monitoring, behavioral analysis, automated investigation and response etc.) , Network Traffic Analysis Tools ( like wireshark, zeek with features like  Deep packet inspection, real-time capture, customizable filters, extensive protocol support, event-driven scripting ). 

   \item  Incident Response: Develop and test comprehensive incident response plans that cover both IT and OT environments. Include specific procedures for isolating and mitigating OT system attacks.  
   
   Establishing a plan for responding to security incidents to minimize damage and recover quickly is also an essential component. Incident Response management Tools like the TheHive (open-source tool with features like Case management, collaboration, automated workflows, integration with threat intelligence.), 



  \item Training and Awareness:
   - Conduct regular cybersecurity training for employees, focusing on the unique challenges of IT/OT convergence.
   - Promote a culture of security awareness and vigilance.

   \item Secure Configuration:
   - Apply security best practices for configuring both IT and OT systems.
   - Regularly review and update security configurations to address emerging threats. Using encryption for data transmission to protect against eavesdropping and data tampering.
   
\end{enumerate}

\subsubsection{Security considerations in OT networks}

Programmable Logic Controllers (PLCs) and other equipment  shares some common vulnerabilities related to PLC standards and implementations:
\begin{enumerate}
    \item Insecure Communication Protocols : Many industrial communication protocols used by PLCs, such as Modbus, DNP3, and EtherNet/IP, were not designed with security in mind. These protocols often lack encryption, authentication, or integrity checks, making them vulnerable to Eavesdropping ( Attackers can intercept unencrypted traffic), Man-in-the-Middle (MitM) Attacks (attackers can modify communication between the PLC and control systems) and replay Attacks (legitimate commands can be captured and replayed later by an attacker). There are protocols with built-in encryption and authentication (e.g., OPC UA, TLS), It needs to be adapted for making communication more robust and secure\cite{s7threats}.

    \item Weak Patch Management: Industrial systems are often patched infrequently due to operational concerns, leaving known vulnerabilities unaddressed for extended periods or delayed patch or incomplete patching. Legacy PLCs may no longer receive security updates, making them permanently vulnerable. Regularly apply security patches and updates, even in industrial environments requires patch management tools specific to OT networks. Currently these tools are not widely available. 

    \item Inadequate Authentication Mechanisms: Many PLCs ship with default credentials that are rarely changed. Attackers can use these credentials to gain unauthorized access. These are due to lack of knowledge and awareness to the personnel working in OT environment , who are less aware to these attacks. Some PLCs use weak or easily guessable passwords, making brute-force attacks feasible.

    \item Firmware Vulnerabilities: Some PLCs allow firmware updates without verifying the authenticity of the firmware, enabling attackers to inject malicious code. Some manufacturers unintentionally include backdoors in their firmware, providing attackers with hidden access (example firmware occurred with Siemens' S7-1200 and S7-1500 PLCs in 2013. Siemens included a special "engineering mode" in the firmware, designed to allow factory engineers and service technicians to access the PLC in case of emergencies or maintenance needs. While this feature was convenient for Siemens personnel, it effectively functioned as a backdoor, allowing privileged access to the PLC, bypassing regular authentication measures. \cite{enlyze}. 
    \item Lack of Secure Boot: Insecure Boot Processes: Some PLCs do not implement secure boot processes, allowing attackers to modify bootloaders or the operating system, potentially leading to persistent compromises.

    \item Weak Access Control: Lack of Role-Based Access Control (RBAC) -  PLCs often lack fine-grained access control mechanisms, allowing unauthorized users to perform critical operations. Insufficient Audit Logs: Some PLCs lack proper logging, making it difficult to detect and respond to unauthorized access or actions.

   \item Side-Channel Attacks: Attackers may exploit side-channel attacks, such as power consumption or timing data, to extract information from a PLC or influence its operation.



\end{enumerate}


In recent years, attacks on Programmable Logic Controllers (PLCs) have become more sophisticated, targeting critical infrastructure and industrial control systems (ICS). Here are some example attacks on OT network or devices:
\begin{enumerate}
    \item Stuxnet: Stuxnet was one of the first highly sophisticated worm that specifically targeted Siemens PLCs used in Iran's nuclear facilities \cite{Langner}. The worm infected Windows computers and then targeted Siemens Step7 software running on programmable logic controllers (PLCs) to reprogram it, causing the centrifuges to spin out of control while reporting normal operations to operators \cite{kushner2013real}. Stuxnet exploited four zero-day vulnerabilities in Microsoft Windows to spread and infect systems. Zero-day vulnerabilities are previously unknown and unpatched flaws in software. The worm spread via removable USB drives, exploiting the autorun feature to execute itself when the drive was connected to a computer. It also spread through network shares and used a variety of techniques to escalate privileges and gain administrative access on infected machines. Once inside a network, Stuxnet searched for Siemens Step7 software and PLCs and injected malicious code into the PLCs, which allowed it to manipulate the industrial processes they controlled. The worm targeted specific models of Siemens PLCs that controlled centrifuges used for uranium enrichment. The worm had sophisticated rootkit capabilities that allowed it to hide its presence on infected systems\cite{farwell2011stuxnet}. It intercepted and altered communication between the PLCs and the monitoring software, ensuring that operators remained unaware of the sabotage. It subtly altered the speeds of the centrifuges, causing them to spin at dangerous levels and ultimately leading to physical damage, while reporting normal operations to the monitoring systems. It caused significant damage ( approximately 1,000 centrifuges) to Iran’s nuclear program, showcasing the potential for cyber-physical sabotage \cite{CCDCOE}. Stuxnet demonstrated the potential for cyber weapons to achieve strategic military objectives and  set a precedent for the development of Advanced Persistent Threats (APTs) \cite{stojanovic2020apt} that target critical infrastructure. 
    \item Industroyer/CrashOverride (2016): Industroyer \cite{kozak2023industroyer} targeted Ukraine’s power grid, causing widespread outages. The malware included modules to control industrial processes via ICS protocols, including IEC 104, IEC 61850, and OPC, to manipulate switches and circuit breakers \cite{ESET}. These protocols  are used in industrial control systems (ICS) for telecontrol and telemonitoring operate over TCP/IP networks. The malware uses a backdoor to maintain persistent access to the compromised network. This backdoor allows the attackers to issue commands to the infected systems remotely. To cover its tracks and hinder recovery efforts, Industroyer contains a data wiper module that erases crucial system files, rendering the infected systems inoperable. The initial infection vector for Industroyer is not fully known, but it is believed to have involved spear-phishing emails or other social engineering techniques to gain an initial foothold in the target network \cite{dragos}. Industroyer’s payload modules are capable of communicating directly with ICS protocols, enabling the malware to send malicious commands to industrial equipment, such as opening circuit breakers to disrupt power flow \cite{sans}. The attack caused a power outage in parts of Kiev, affecting tens of thousands of residents for about an hour.
    It demonstrated the capability to disrupt critical infrastructure by directly interacting with PLCs and other ICS components.


    
    
    \item Triton/Trisis (2017):  Triton malware targeted Schneider Electric’s Triconex safety instrumented systems (SIS) used in critical infrastructure \cite{mekdad2021threat}. Triton was discovered in 2017 after it was used to attack a petrochemical plant in Saudi Arabia. Triton consists of several modules that interact with the Triconex SIS controllers \cite{trisis_drago}. These modules are capable of reading and writing to the controllers. It aimed to reprogram SIS controllers to either shut down operations or cause unsafe conditions. 
    The malware installs a backdoor on the compromised system, allowing remote operators to execute commands and manipulate the SIS devices. Triton includes an execution framework that ensures the payload is delivered and executed on the target devices, and it can persist across reboots. The initial infection vector for Triton remains unclear, but it is believed to involve compromising the engineering workstation connected to the SIS network. Once inside the network, the attackers use various techniques to move laterally and gain access to the SIS controllers.
    It highlighted the risk of attacks on safety systems designed to prevent catastrophic failures in industrial environments.
    

    \item EKANS Ransomware (2020): EKANS (also known as Snake) is a type of ransomware that specifically targets OT systems. First discovered in December 2019, EKANS is notable for its focus on operational technology (OT) environments, including those in the manufacturing, energy, and healthcare sectors. The EKANS (or Snake) ransomware employs several specific attack methods that are particularly targeted towards disrupting OT Systems. The initial access and deployment is conducted by through phishing emails (these emails are designed to trick users into downloading and executing the ransomware payload) that contain malicious attachments or links, exploit known vulnerabilities in public-facing applications or systems (this could involve exploiting unpatched software, misconfigured servers, or using brute force attacks against weak passwords) or  exploit weak or exposed Remote Desktop Protocol (RDP) configurations to gain remote access to the targeted network. Once inside the network, attackers map the network to identify critical systems and assets. This helps them understand the layout and identify the most valuable targets. Attackers gather credentials to escalate privileges and move laterally within the network. This can involve techniques like keylogging, credential dumping, and exploiting password reuse. Attackers often use legitimate administrative tools and protocols (like PowerShell, PsExec, and WMI) to move laterally within the network, making detection harder. Attackers establish persistence on critical systems to ensure they can maintain access even if some initial footholds are detected and removed. EKANS includes a predefined list of processes and services that are typically associated with ICS environments. The ransomware attempts to terminate these processes to disrupt operations. The kill list can include processes related to control systems, data historians, and other OT applications. By terminating ICS-related processes, EKANS aims to cause operational disruptions, halt production lines, and create chaos within industrial environments. EKANS encrypts files on the infected systems using strong encryption algorithms. The ransomware leaves a ransom note on the infected systems, demanding payment in exchange for the decryption key. Honda reported a cyberattack that led to the temporary suspension of production at several manufacturing plants. While Honda did not confirm EKANS specifically, cybersecurity experts suggested that the attack bore similarities to EKANS tactics.
 






    
    \item PLC-Blaster (2020): PLC-Blaster is a worm that specifically targets Programmable Logic Controllers (PLCs). PLC-Blaster is designed to spread across networks, infecting other PLCs connected to the same network. This makes it particularly dangerous in large industrial environments where many PLCs are networked together. PLC-Blaster exploits vulnerabilities in the PLC firmware or network configuration to gain access and spread. The worm can also steal credentials to facilitate its spread across the network. Once a PLC is infected, PLC-Blaster autonomously scans the network for other vulnerable PLCs and spreads without human intervention. The worm employs techniques to evade detection by security systems, making it harder for traditional antivirus and intrusion detection systems to identify and stop it. The initial foothold of PLC-Blaster involves several potential methods to infiltrate a network and infect the first Programmable Logic Controller (PLC). Attackers does start the  exploit vulnerabilities in the PLC’s firmware. Many PLCs run outdated or unpatched firmware that contains security flaws, providing an entry point for the worm. Vulnerabilities in software used to manage or communicate with PLCs, such as Human-Machine Interfaces (HMIs) or Supervisory Control and Data Acquisition (SCADA) systems, can also be targeted. Penetration through IT network vulnerabilities like phishing, compromised network devices and remote access exploitation.  
\end{enumerate}


\begin{table}[h!]
\centering
\begin{tabular}{|m{3.5cm}|m{3.5cm}|m{3.5cm}|m{6cm}|}
\hline
\textbf{Name of Attack} & \textbf{Affected Sector/Plant} & \textbf{Initialization} & \textbf{Mode of Operation} \\ \hline
\textbf{Stuxnet} & Nuclear Enrichment Facilities (Iran) & Infected USB drives exploited vulnerabilities in Windows & Targeted Siemens PLCs controlling uranium centrifuges, manipulated PLC logic to cause centrifuges to spin at destructive speeds while hiding the malicious behavior from monitoring systems. \\ \hline
\textbf{Industroyer} & Power Grid (Ukraine) & Likely through spear-phishing or social engineering & Interacted with ICS protocols (IEC 104, IEC 61850) to send malicious commands to control circuit breakers, causing temporary power outages in Kiev. \\ \hline
\textbf{Triton/Trisis} & Petrochemical Plant (Saudi Arabia) & Compromised engineering workstation & Targeted Schneider Electric's Triconex safety systems, attempting to reprogram them, potentially causing dangerous conditions or shutdowns of critical safety mechanisms in industrial operations. \\ \hline
\textbf{EKANS (Snake) Ransomware} & Manufacturing, Energy, Healthcare & Phishing emails or weak RDP configurations & Terminated ICS processes and encrypted files in OT environments, causing operational disruption. Aimed to halt production lines and demand ransom for decryption of essential OT systems and files. \\ \hline
\textbf{PLC-Blaster} & Industrial Automation (PLCs) & Exploited vulnerabilities in PLC firmware or through IT network & Spread across networks autonomously, infecting connected PLCs. It stole credentials and leveraged them to propagate, targeting vulnerabilities in PLC firmware and software to disrupt operations. \\ \hline
\end{tabular}
\caption{Summary of some recent cyberattacks on OT systems}
\end{table}













