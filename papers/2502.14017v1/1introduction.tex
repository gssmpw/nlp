\section{Introduction}
\label{sec:introduction}
Operational Technology (OT) networks are critical infrastructure systems responsible for controlling physical processes in industries such as manufacturing, energy, transportation, and utilities. These networks integrate hardware and software systems, including SCADA (Supervisory Control and Data Acquisition), PLCs (Programmable Logic Controllers), and DCS (Distributed Control Systems), to ensure the smooth operation of industrial machinery and processes. However, with the convergence of OT and IT systems due to increased digitalization and Industry 4.0 initiatives, OT networks have become a prime target for cyber threats in recent time \cite{negi2024towards} \cite{knapp2024industrial} \cite{zaid2024emerging}. Unlike traditional IT systems, OT networks face unique cyber-security challenges due to their legacy systems, real-time operational requirements, and the high cost of downtime or failure. Cyber security in OT networks, therefore, focuses on safeguarding critical infrastructure from threats such as malware, ransomware, and state-sponsored attacks, ensuring the continuity, safety, and reliability of essential operational services \cite{kalinaki2025ransomware} \cite{daniel2024emerging}. In this manuscript, we focus on understanding the key components in OT networks, how OT and IT overlap creates the security considerations for IT-OT Systems where some security concerns originate in IT networks and lateral move to OT networks and other security concerns in OT networks, the attack vectors, some recent OT attacks and its mode of operation and tools is used in modern time in various aspects of cybersecurity. We will also discuss the regulatory framework and standards in OT specific cyber security, emerging trends by adapting new technology in OT secuity space, and finally evaluates its quantitative impact. 