\section{Related Work}
\label{sec:related_work}


Triton, also known as TRISIS or HatMan, is a highly sophisticated malware designed to target and manipulate safety instrumented systems (SIS) used in critical infrastructure. Here's a detailed overview of Triton/TRISIS:

### Overview

- **Discovery:** Triton was discovered in 2017 after it was used to attack a petrochemical plant in Saudi Arabia.
- **Primary Target:** The malware targets safety instrumented systems (SIS), specifically the Schneider Electric Triconex Safety Instrumented System. SIS are used to safely shut down industrial processes in emergency situations to prevent accidents.

### Technical Details

#### Components

1. **Payload Modules:**
   - Triton consists of several modules that interact with the Triconex SIS controllers. These modules are capable of reading and writing to the controllers, which can disrupt the normal functioning of safety systems.

2. **Backdoor:**
   - The malware installs a backdoor on the compromised system, allowing remote operators to execute commands and manipulate the SIS devices.

3. **Execution Framework:**
   - Triton includes an execution framework that ensures the payload is delivered and executed on the target devices, and it can persist across reboots.

#### Infection Mechanism

1. **Initial Access:**
   - The initial infection vector for Triton remains unclear, but it is believed to involve compromising the engineering workstation connected to the SIS network.

2. **Lateral Movement:**
   - Once inside the network, the attackers use various techniques to move laterally and gain access to the SIS controllers.

3. **Manipulation of SIS:**
   - Triton’s main functionality involves reprogramming the SIS controllers. It can disable the safety functions or put the system in a failed state, potentially leading to dangerous situations in the industrial process being controlled.

### Impact

1. **Saudi Arabian Attack:**
   - The most notable incident involving Triton occurred at a petrochemical plant in Saudi Arabia in 2017. The malware attempted to reprogram the SIS controllers, which led to the triggering of safety systems and a plant shutdown. The attackers' goal appeared to be causing physical damage by preventing the SIS from operating correctly, which could have resulted in catastrophic consequences.

2. **Potential for Future Attacks:**
   - Triton’s capabilities demonstrate the potential for significant disruption in critical infrastructure. Its focus on safety systems makes it particularly dangerous, as these systems are the last line of defense against hazardous industrial conditions.

### Attribution and Context

1. **Nation-State Attribution:**
   - While the specific actors behind Triton have not been conclusively identified, the sophistication of the malware and the strategic nature of the targets suggest involvement by a nation-state. Several cybersecurity firms and researchers have pointed to possible connections to entities in Russia based on the analysis of the malware and its operation【21†source】【23†source】.

2. **Context of Cyber Warfare:**
   - The attack on the Saudi Arabian petrochemical plant is part of a broader trend of cyber warfare, where state-sponsored actors target critical infrastructure to achieve strategic objectives without conventional military action.

### References

For more detailed information on Triton/TRISIS, you can refer to the following sources:


2. [FireEye’s Detailed Report](https://www.fireeye.com/blog/threat-research/2017/12/triton-is-industrys-first-sis-specific-malware.html)
3. =;[
]
]

These references provide comprehensive insights into the technical aspects, impact, and strategic implications of Triton/TRISIS.