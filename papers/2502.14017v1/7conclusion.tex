\section{Conclusions}
\label{sec:conclusion}
As industrial environments continue to evolve, the integration of OT and IT systems has become a cornerstone of modern industrial operations. However, this convergence has also introduced unprecedented cybersecurity challenges, exposing critical infrastructure to sophisticated threats such as malware, ransomware, and advanced persistent threats (APTs). The unique vulnerabilities of OT networks, stemming from legacy systems, real-time operational demands, and the high cost of downtime, necessitate a proactive and comprehensive approach to cybersecurity.

This study has highlighted the critical importance of understanding the key components of OT systems, including SCADA, PLCs, and RTUs, and the security risks that arise from their integration with IT networks. By examining recent cyberattacks on OT environments, we have underscored the devastating potential of these threats, not only to data but also to physical infrastructure and human safety. Analysis of attack vectors, such as phishing, malware injection, and supply chain compromises, emphasizes the need for robust defense mechanisms tailored to the unique requirements of OT networks.

Emerging trends in OT cybersecurity, such as the adoption of AI-driven threat detection, Zero Trust Architecture, blockchain for secure event logging, and digital twins for proactive security, offer promising solutions to these challenges. These technologies enable real-time threat detection, improve incident response efficiency, and improve system resilience, ensuring the continuity and reliability of critical industrial operations. Furthermore, regulatory frameworks such as the NIST Cybersecurity Framework and IEC 62443 provide essential guidelines for securing OT environments, promoting a structured and adaptive approach to risk management.

As the cybersecurity landscape continues to evolve, the importance of cyber resilience cannot be overstated. Moving beyond traditional intrusion prevention, resilience frameworks focus on real-time threat detection, autonomous recovery, and adaptive system reinforcement. By integrating these advanced strategies, industries can not only mitigate risks but also ensure the long-term safety and reliability of their critical infrastructure.

In conclusion, the cybersecurity of OT networks is not just a technical challenge but a strategic imperative. As industrial systems become increasingly interconnected, the adoption of proactive cybersecurity measures, coupled with a focus on resilience, will be essential in safeguarding critical infrastructure from the ever-evolving threat landscape. The lessons learned from past attacks and the innovations in cybersecurity technologies provide a roadmap for building a secure and resilient industrial future.