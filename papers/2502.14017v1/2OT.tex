\section{OT and its components}
\label{sec:ot}

Operation Technology (OT) refers to hardware and software systems used for actuation, sensing and monitoring, computing and control of physical devices, processes, and industrial operations. In this manuscript, the context of  OT is in critical sectors such as manufacturing, energy, utilities, transportation, and healthcare etc. 
The key components and systems in modern OT systems (also called Industrial Control System \textbf{ICS}) are as below:

\subsection{Supervisory Control and Data Acquisition (SCADA)} SCADA system provides centralized monitoring and control of complex processes spread across large areas. SCADA systems gather real-time data from remote locations from sensors and instruments located at remote sites \cite{waqas2024smart} and transmit it to central computers and provides technology to control, and visualization as well. It is widely used in industries like power generation, oil exploration, water management, manufacturing etc. The key components of SCADA systems are field instruments and sensors, Remote Terminal Units (RTUs), Programmable Logic Controllers (PLCs), communication infrastructure, SCADA master station, and data historian. \textbf{Field instruments and sensors} are devices that measure various parameters like temperature, pressure, flow, rpm etc in the industrial process. \textbf{Remote Terminal Units (RTUs)} are microprocessor controlled devices that interface with sensors and convert their signals to digital data \cite{misbahuddin2010fault}, which is then transmitted to the central system. \textbf{Programmable Logic Controllers (PLCs)} are  industrial digital computers that control the process. \textbf{communication infrastructure} consist of the physical network and protocols that transmits data between the field devices and the control systems. It can use various communication methods and protocols, including wired (Ethernet) and wireless (radio, cellular) networks. \textbf{SCADA master station} is the central control system that collects data from RTUs and PLCs, processes it, and presents it to human operators. It also includes \textbf{Human-Machine Interface (HMI)} software for visualization and high level supervisory control. \textbf{Data Historian} is a system for collecting and storing historical data from the SCADA system. This data is used for trend analysis, reporting, and auditing. Details of the components are provided below:
\begin{enumerate}
   \item Sensors: Sensors gather real-time data on various process parameters such as temperature, pressure, flow, level, and chemical composition etc. It also provide continuous monitoring of the industrial processes to ensure they are operating within specified parameters. By sensing, it enable to introduce automated control by providing feedback to the SCADA system, which can then make adjustments to maintain optimal operation. It also used to detect abnormal conditions and trigger alarms or safety shutdowns to prevent accidents or equipment damage.

    \item RTU: A Remote Terminal Unit (RTU) is an electronic device used in industrial and remote monitoring applications to collect data from sensors, process it, and transmit it to a central control system, such as a SCADA (Supervisory Control and Data Acquisition) system. RTUs are essential for monitoring and controlling equipment and processes in geographically dispersed locations. RTU is designed for remote data acquisition and control, particularly in environments where the control system needs to be dispersed over a wide geographical area, while PLC is designed for real-time control and automation of industrial processes and machinery within a localized environment. RTU Communication typically supports various long-distance communication methods, including cellular, satellite, radio, and Ethernet. On contrast, PLC Communication primarily uses Ethernet and serial communications (RS-232, RS-485) within a localized network.
   
   \item Programmable Logic Controllers (PLC): PLCs are specialized industrial computers used to automate and control machinery and processes. They are designed to withstand harsh industrial environments and provide reliable operation over extended periods. Key components of PLCs are: 
     \begin{enumerate}
         \item Central Processing Unit (CPU): The CPU is the brain of the PLC, responsible for executing control instructions stored in the memory. It processes input signals, executes the control program, and sends output signals to actuators.
         \item Input/Output (I/O) Modules:  I/O modules interface the PLC with external devices. Input modules receive signals from sensors and switches, while output modules send control signals to actuators, such as motors and valves.
         \item Communication ports : Communication ports are used for connecting to other devices and networks. Commonly used communication protocols include Ethernet, Modbus, Profibus, and DeviceNet.
         \item Memory : Various type of memory is used in this device, it includes RAM (Random Access Memory)- a Volatile memory used for temporary data storage, ROM (Read-Only Memory)- a non-volatile memory used to store the PLC's firmware and EEPROM (Electrically Erasable Programmable Read-Only Memory)- a Non-volatile memory used to store the control program.
     \end{enumerate}
     PLCs are programmed using specialized languages defined by the IEC 61131-3 standard.  The most common programming languages are Ladder Logic (LAD) i.e. a graphical programming language that resembles electrical relay logic diagrams. Function Block Diagram (FBD) is another graphical language that uses blocks to represent functions and data flow. Structured Text (ST) is a high-level textual programming language similar to Pascal. It is used for complex mathematical and logical operations. Instruction List (IL) is a low-level textual language similar to assembly language. It provides fine-grained control over the PLC's operations. Sequential Function Chart (SFC) is also a graphical language used for programming sequential processes. It represents the control process as a series of steps and transitions. The PLC operates in a cyclic manner, continuously executing the following steps Input Scan (reading input ports), Program Execution ( Executing the control logic on input data) and Output (Sending output to a output port). The real time control is achieved by executing the scan cycle rapidly, typically in milliseconds. The input interface generally receive signals from devices like push buttons, limit switches, and proximity sensors in either digital or analog signals. The output interface provide control signal to devices like relays, solenoids, and indicator lights (in digital form) or control valves, variable frequency drives (VFDs), and actuator (in analog form) etc. 
     
     The interfaces that are generally available to communication interface with other PLC are Ethernet/IP, USB, Modbus RTU or similar protocol. A serial communication protocol used over RS-232 or RS-485. Some PLC also offers Wi-Fi or Bluetooth, that can be used for wireless programming and monitoring. PLC are generally connected with a programming terminal is used for developing, uploading, and troubleshooting the control program. Common methods of Connection are  Serial Communication (RS-232/RS-485), Ethernet or USB.  Some PLC also uses optical isolation using optocoupler, which consists of an LED (Light Emitting Diode) and a photodetector (phototransistor, photodiode, or photodarlington) housed in a single package.   It provides a high degree of electrical isolation between different parts of the PLC system. This prevents high voltage levels or electrical noise from one part of the system from affecting other parts, and its transients. Industrial environments are typically noisy, with electromagnetic interference (EMI) and radio frequency interference (RFI) from motors, relays, and other equipment. Optical isolation helps to block these interference, ensuring that the signals within the PLC remain clean and accurate. Ground loops can occur when there are multiple ground points at different potentials, leading to unwanted currents that can interfere with signal integrity. Optical isolation breaks the electrical path, eliminating the possibility of ground loops and maintaining signal fidelity.

     
    \item Human-Machine Interfaces (HMI): A Human-Machine Interface (HMI) is a sophisticated control system component that facilitates interaction between human operators and industrial control systems, such as PLCs and SCADA systems. At its core, an HMI comprises both hardware and software components designed to translate complex system data into intuitive graphical representations, allowing operators to monitor, send control signal, and optimize industrial processes effectively. Modern HMIs employ advanced visualization techniques, including real-time data graphs, schematics, and multi-layered interfaces, to enhance situational awareness and decision-making. They support various communication protocols, such as Modbus, OPC UA, and Ethernet/IP, ensuring seamless integration with diverse industrial devices and networks. Furthermore, HMIs incorporate robust security measures, including user authentication, encryption, and anomaly detection, to safeguard against cyber threats and unauthorized access. The system’s architecture often leverages embedded computing platforms, real-time operating systems (RTOS), and custom application development environments to achieve high reliability and performance in demanding industrial environments.

   
    \item Data historian : A data historian is a specialized software system used in industrial environments to collect, store, and analyze time-series data from various sources. A data historian is designed to handle the massive amounts of data generated by industrial processes, often in real-time. These databases  are optimized for handling time-series data, use advanced compression techniques, efficient indexing mechanisms to quickly retrieve data based on time ranges, tags, or other attributes and incorporate long-term storage strategies to archive older data, balancing between quick access for recent data and efficient storage for historical data. A typical components in the data template stored in a historian is : Tag/Point Name- a unique identifier for each data point or sensor (example: Pressure\_Valve\_A1), Timestamp - date and time when the data was recorded (example: 2024-07-24T14:23:00.123Z), Value- the actual measurement or value recorded by the sensor at the given timestamp, Quality/Status- indicates the quality or validity of the recorded data. This can be a binary indicator (good/bad), or a more detailed status code indicating specific conditions such as sensor error, communication failure, or manual entry (example- Good, Bad, sensor\_error), Unit of Measurement- unit in which the value is measured (example- psi, $m^3/sec$ ). Data Source tag provide information about the origin of the data, such as the specific sensor, RTU, PLC, or other devices (example- RTU\_01, PLC\_Station\_5), Location captures the physical or logical location of the sensor or data point (example: Plant\_A, Sector\_4, Lat:35.6895, Long:139.6917), Alarm/Event Information  tag (if applicable) logs details about any alarms or events associated with the data point (example: High\_Temperature\_Alarm, Priority\_1, Acknowledged).

\end{enumerate}


\subsection{Distributed Control Systems (DCS) }
Distributed Control Systems decentralize control across various subsystems to enhance reliability and performance. Common in chemical plants, oil refineries, and large-scale production facilities.
Supervisory Control and Data Acquisition (SCADA) systems and Distributed Control Systems (DCS) are both integral parts of industrial automation and control systems. While they share some similarities, they are designed for different purposes and have distinct architectures and applications. SCADA is primarily designed for high-level supervisory management and control. It is used to monitor and control processes that are distributed over large geographical areas. SCADA systems are commonly used in industries such as water and wastewater management, oil and gas pipelines, electrical power distribution, and telecommunications. DCS (Distributed Control System) is primarily designed for process control within a localized area, such as a single plant or factory. DCS systems are used for continuous and complex processes where control and monitoring are needed in a confined area. They are commonly found in industries such as chemical processing, petrochemical, oil refining, and power generation.

\subsubsection{Key Differences}
\begin{enumerate}
    \item System Architecture: SCADA typically has a centralized architecture where a central master station collects data from Remote Terminal Units (RTUs) and Programmable Logic Controllers (PLCs) distributed across various locations. The communication is often event-driven or poll/response-based. It emphasizes remote monitoring and control, with a focus on data acquisition and human-machine interface (HMI). DCS employs a distributed architecture where control is decentralized, and various controllers communicate with each other directly. The communication is often continuous and deterministic. It emphasizes process control and automation with local control loops, extensive inter-controller communication, and high reliability.

    \item Geographical scope: SCADA is designed for large-scale applications with wide geographical distribution. It is ideal for systems like water distribution networks, oil and gas pipelines, and power grids. DCS on other hand is designed for localized applications within a single plant or facility. It is ideal for controlling complex industrial processes within a confined area.
    
    \item Control Approach: In SCADA, supervisory control with decision-making often done at the central control room. Local controllers like RTUs and PLCs perform basic control functions and relay data to the central system. However in DCS,  distributed control with decision-making is distributed across multiple controllers. Each controller can execute complex control algorithms independently, enhancing system reliability and responsiveness.
    
    \item System Complexity and Scalability: SCADA is more scalable in terms of geographical reach but might have limitations in handling very complex control processes. DCS is more complex in terms of control capabilities but typically confined to a single site or plant, making it less scalable geographically.

    \item Example Application: SCADA are mainly used in water and wastewater management, oil and gas pipelines, electrical power distribution, telecommunication systems. DCS is maily deployed in chemical processing plants, petrochemical and oil refineries, power generation plants, food and beverage processing.
\end{enumerate}

SCADA systems are best suited for applications that require remote monitoring and control over vast geographical areas, emphasizing data acquisition and human interaction. DCS systems are ideal for complex, localized process control within a single plant, emphasizing automated control and process optimization.

    
    
   

\subsection{Communication protocols in OT devices}
There are several communication protocols used in industrial automation and process control each with its own unique features and applications. Here are some of the most common ones:
\begin{enumerate}
    \item Profibus (Process Field Bus): It has two major variant- Profibus DP (Decentralized Peripherals) and Profibus PA (Process Automation). It features High-speed communication, real-time data exchange, and extensive diagnostics capabilities. It is mainly used for connecting field devices like sensors and actuators in manufacturing and process automation.
    \item Profinet (Process Field Network): It is an Ethernet-based protocol offering real-time data transmission, extensive diagnostics, and flexibility for integrating IT systems with automation systems. It is used in industrial automation, particularly where high-speed communication and integration with enterprise systems are required.
    \item Ethernet/IP (Ethernet Industrial Protocol): It uses standard Ethernet for industrial applications, supports real-time control and data acquisition, extensive device support. It is widely used in factory automation, including robotics, assembly lines, and material handling systems.
    \item CANopen (Controller Area Network): It is designed for embedded systems with a focus on real-time data exchange, supports network management and device configuration.It is mainly used in Automotive systems, medical equipment, industrial machinery, and building automation.
    \item DeviceNet: It is based on CAN (Controller Area Network), DeviceNet supports real-time data exchange, device configuration, and diagnostics. It is used primarily in industrial automation for connecting sensors and actuators to controllers.
    \item HART (Highway Addressable Remote Transducer): It combines analog and digital communication, allows two-way communication over existing 4-20 mA analog wiring. It is used extensively in process industries for device diagnostics, configuration, and data acquisition.
    \item BACnet (Building Automation and Control Network): It is Open protocol designed for building automation and control systems, supports various types of data transmission (e.g., alarm, event, trend data). It is mainly used in HVAC, lighting, security, and fire detection systems in building automation.
    \item DNP3 (Distributed Network Protocol): it is designed for reliable, secure communication in SCADA systems, supports complex data structures and time-stamped data. It is mainly used in Electric utility industry, water and wastewater systems, oil and gas pipelines.
    \item SERCOS (Serial Real-time Communication System): It is high-speed, deterministic communication, designed for motion control systems. It is mainly used in  CNC machines, robotics, and other motion control applications.
    \item Modbus: It is a widely-used communication protocol in the field of process automation and SCADA (Supervisory Control and Data Acquisition). It allows various devices and equipment to communicate with each other effectively. Modbus was initially developed by Modicon (now owned by Schneider Electric) in 1979 and has become a standard in the industry due to its openness and versatility. It facilitates the exchange of information between electronic devices. It is especially prevalent in process automation and SCADA systems. Devices such as temperature and humidity sensors can use Modbus to send data to supervisory computers or PLCs (Programmable Logic Controllers).

\subsubsection{Types of Modbus Communication:} Several versions of the Modbus protocol exist, including:
\begin{enumerate}
    \item Modbus RTU (Remote Terminal Unit):It uses binary representation and is optimized for fast communication.
    \item Modbus ASCII: It Uses ASCII characters for communication, making it human-readable but slower compared to RTU.
    \item Modbus TCP: It Utilizes TCP/IP over Ethernet, allowing Modbus to be used in modern network infrastructures.
    \item Modbus Plus: A high-speed, peer-to-peer protocol with token-passing capabilities.
\end{enumerate}

\subsubsection{ Modbus messaging Architecture}
Modbus operates on a master/slave architecture (client/server for Ethernet). Master Device initiates transactions (queries) and can address individual slaves or broadcast messages to all slaves. Slave device Responds to the master's queries with the requested data or actions. Slaves do not initiate messages. Message Structure consists of a slave address, function code, data, and error-checking field. Error Checking ensures data integrity by validating message contents.

\subsubsection{Physical Media}
Modbus can communicate over various physical media, including:
\begin{enumerate}
    \item RS-232: Original interface, limited to short distances and point-to-point communication.
    \item RS-485:Supports longer distances, higher speeds, and multiple devices on a single network.
    \item Ethernet (TCP/IP): Allows Modbus messages to be embedded in Ethernet packets, supporting mixed protocols on the same network.
\end{enumerate}


Being an open standard, and freely available, there is widespread adoption. It also supports multiple communication types and physical media. It also facilitates integration of devices from different manufacturers, enhancing flexibility and choice.
    
\end{enumerate}

Each of these protocols has its own strengths and is chosen based on the specific requirements of the application, such as speed, reliability, real-time capabilities, and the types of devices being integrated.


\begin{table}[h!]
\centering
\begin{tabular}{|m{3.5cm}|m{5cm}|m{7cm}|}
\hline
\textbf{Protocol} & \textbf{Description} & \textbf{Common Applications} \\ \hline
\textbf{Profibus} & High-speed communication with real-time data exchange, extensive diagnostics. Two variants: Profibus DP (Decentralized Peripherals) and Profibus PA (Process Automation). & Used for connecting field devices like sensors and actuators in manufacturing and process automation. \\ \hline
\textbf{Profinet} & Ethernet-based protocol offering real-time data transmission and integration with IT systems. & Used in industrial automation where high-speed communication and integration with enterprise systems are required. \\ \hline
\textbf{Ethernet/IP} & Uses standard Ethernet for industrial applications, supports real-time control and data acquisition. & Widely used in factory automation, including robotics, assembly lines, and material handling systems. \\ \hline
\textbf{CANopen} & Designed for real-time data exchange in embedded systems, supports network management and device configuration. & Used in automotive systems, medical equipment, industrial machinery, and building automation. \\ \hline
\textbf{DeviceNet} & Based on CAN (Controller Area Network), supports real-time data exchange, device configuration, and diagnostics. & Primarily used in industrial automation for connecting sensors and actuators to controllers. \\ \hline
\textbf{HART} & Combines analog and digital communication over existing 4-20 mA analog wiring, supports two-way communication. & Extensively used in process industries for device diagnostics, configuration, and data acquisition. \\ \hline
\textbf{BACnet} & Open protocol for building automation and control systems, supports various types of data transmission. & Used in HVAC, lighting, security, and fire detection systems in building automation. \\ \hline
\textbf{DNP3} & Designed for secure communication in SCADA systems, supports complex data structures and time-stamped data. & Commonly used in electric utilities, water and wastewater systems, and oil and gas pipelines. \\ \hline
\textbf{SERCOS} & High-speed, deterministic communication protocol designed for motion control systems. & Used in CNC machines, robotics, and other motion control applications. \\ \hline
\textbf{Modbus} & Widely-used protocol for process automation and SCADA systems, with several versions including Modbus RTU, ASCII, TCP, and Plus. & Facilitates communication between devices in process automation and SCADA systems, such as temperature sensors, PLCs, and supervisory computers. \\ \hline
\end{tabular}
\caption{Communication Protocols Used in OT Devices}
\end{table}




\subsection{An example Industry model with SCADA and its components}
The figure \ref{fig:scada} illustrates a mixing processing plant controlled through an Operational Technology (OT) network, highlighting the integration of Programmable Logic Controllers (PLCs), Remote Terminal Units (RTUs), and SCADA (Supervisory Control and Data Acquisition) systems. The process begins with the monitoring of critical parameters, such as temperature and RPM, via sensors attached to components like the furnace and centrifuge pump. These values are fed into the PLCs, which manage the control systems, including voltage control and motor operations, ensuring optimal function. The mixing vessel, where items are combined, is controlled by the motor and further monitored through various sensors. The entire system is supervised and controlled through SCADA stations, which communicate with the RTUs and collect historical data for future analysis. Electrical power distribution is shown in red, while OT-based communication flows are represented in green. 
\begin{figure}[ht!]
    \centering
   \includegraphics[trim={0.8cm 0.5cm 0.5cm 0.1cm},clip,width=0.9\linewidth]{images/SCADA.pdf}

    \caption{A example mixing processing plant with the OT network and PLC based controller with SCADA station. }
    \label{fig:scada}
\end{figure}