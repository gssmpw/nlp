\section{Cyber Resilience in OT – Building Adaptive, Self-Healing Industrial Systems}
\label{sec:CyberRes}
OT security is increasingly shifting from passive cybersecurity models to active, self-healing cyber resilience frameworks. Unlike conventional cybersecurity methodologies that focus on intrusion prevention and compliance (IEC 62443, NIST 800-82, ISO 27001, etc.), cyber resilience ensures real-time threat detection, autonomous recovery, and adaptive system reinforcement \cite{ross2021nist}.

The need for cyber resilience in ICS, IIoT, and CPS has been reinforced by high-profile cyberattacks such as Stuxnet, Triton, and ransomware attacks on critical infrastructure sectors \cite{derbyshire2024dead}. The World Economic Forum (WEF, 2023) and U.S. Executive Order on Cybersecurity (EO 14028, 2021) now emphasize ZTA and AI-powered security automation in OT resilience frameworks \cite{WEF}. This section explores the key dimensions of cyber resilience in OT, including self-healing AI-driven security architectures, deception-based cyber defense mechanisms, digital immune systems, and emerging regulatory advancements.