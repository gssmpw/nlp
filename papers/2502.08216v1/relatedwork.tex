\section{RELATED WORK}
Some of the early Deepfake detection techniques \cite{yang2019exposing, liy2018exposingaicreated} also used visual biological artifacts to identify the authenticity of visual content. More recent methods  rely on  learning-based methods. Li et al.\cite{li2021exposing} used several classic high-complexity deep neural networks to extract more detailed features from the forged videos and used them for the underlying classification task. Wu et al.~\cite{wu2020sstnet} proposed that spatial and temporal features can be used as detection targets. They model the temporal dimension with LSTM\cite{6795963} and employed  Xception\cite{chollet2017xception} for the convolutional network. Wu et al.~\cite{wu2020sstnet} proposed that spatial and temporal features can be used as detection targets. They model the temporal dimension with LSTM\cite{6795963} and employed  Xception\cite{chollet2017xception} for the convolutional network.However, highly complex deep neural networks and excessive iterative training can lead to overfitting of their method to the training data.