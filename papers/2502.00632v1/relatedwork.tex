\section{Literature Review}
\subsection{Benefits of GAI for scholarly publishing}
%在研究学术写作中AI的作用之前,有大量的研究关于AI在学术写作中的好处。%
Before investigating the role of GAI in academic writing, it is essential to understand why researchers are drawn to using GAI in the first place. Most scholars hold a dialectical perspective on the impact of GAI on academic writing. They believe that GAI can enhance writing efficiency, provide new ideas, correct grammatical structures and spelling errors, and offer personalized guidance \citep{lund2023chatting,kasneci2023chat,DWIVEDI2023102642}. Some studies explore the assistive value of ChatGPT in writing, demonstrating potential both for peer-reviewed literature reviews and for content analysis of published literature \citep{dergaa2023from,rahman2023chatgpt}. To validate ChatGPT's exceptional writing abilities, two scientists, with the assistance of ChatGPT, completed a research paper in under an hour. The study found that ChatGPT could not only handle data processing but also polish the initial draft \citep{conroy2023scientists}.

In empirical research, \citet{fyfe2023how} conducted a study involving 20 university students across different academic levels, investigating their experiences using GPT-2 to complete their final course papers. Eighty-seven percent of the students reported that collaborative writing with GPT was more manageable than completing the work independently, as it provided new arguments and ideas. Science published a controlled experimental study involving professionals from various fields, showing that ChatGPT is especially beneficial for individuals with weaker writing skills \citep{noy2023experimental}. The results indicated that ChatGPT can elevate their writing abilities to a level close to that of proficient writers.

\subsection{Ethical Concerns and Policies for AI in Academic Publishing}
As early as 2020, two initiatives for designing AI intervention reports were introduced in the medical field: CONSORT-AI (Consolidated Standards of Reporting Trials for Artificial Intelligence) and SPIRIT-AI \citep{Liu2020,CruzRivera2020}. In February 2023, the Committee on Publication Ethics (COPE) issued a position statement on the use of AI tools in research publications, emphasizing that AI tools cannot be listed as co-authors and providing guidelines on how to disclose the use of AI \citep{COPE2023}. This statement was quickly endorsed by journals and editorial associations such as the International Committee of Medical Journal Editors (ICMJE), the Journal of the American Medical Association (JAMA), and the World Association of Medical Editors (WAME) \citep{ICMJE2023,Flanagin2023,WAME2023}.  The introduction of such policies is precisely due to the growing concerns among scholars about the ethical issues surrounding the use of generative AI in academic publishing \citep{Anderson2023}.  Large language models can produce a phenomenon known as "hallucination," where the generated text appears coherent and meaningful but is actually fabricated. This is also referred to as the "stochastic parrot" effect \citep{Hutson2021,Ji2023}. In addition, since GPT-3 is trained on large web-based datasets, it inevitably inherits racial, gender, and ethnic biases during the training process \citep{Basta2019,Founta2018,Hutchinson2020}. These biases can be propagated through the publication of academic articles, further amplifying their impact within the scholarly community. 
\subsection{Impetus for the current study}
However, these concerns have largely remained in the realm of theoretical discussions. There is a lack of empirical investigation into the actual purposes behind researchers' use of generative AI in academic writing and the potential ethical and academic impacts it may cause. While much has been written about how AI is being integrated into academia, there is a gap in understanding the reasons for AI use in writing.
By studying the AI usage statements in papers, the work will provide insights into why authors are turning to AI and how these reasons may vary across  author teams and authors' language backgrounds.