%Version 3 December 2023
% See section 11 of the User Manual for version history
%
%%%%%%%%%%%%%%%%%%%%%%%%%%%%%%%%%%%%%%%%%%%%%%%%%%%%%%%%%%%%%%%%%%%%%%
%%                                                                 %%
%% Please do not use \input{...} to include other tex files.       %%
%% Submit your LaTeX manuscript as one .tex document.              %%
%%                                                                 %%
%% All additional figures and files should be attached             %%
%% separately and not embedded in the \TeX\ document itself.       %%
%%                                                                 %%
%%%%%%%%%%%%%%%%%%%%%%%%%%%%%%%%%%%%%%%%%%%%%%%%%%%%%%%%%%%%%%%%%%%%%

%%\documentclass[referee,sn-basic]{sn-jnl}% referee option is meant for double line spacing

%%=======================================================%%
%% to print line numbers in the margin use lineno option %%
%%=======================================================%%

%%\documentclass[lineno,sn-basic]{sn-jnl}% Basic Springer Nature Reference Style/Chemistry Reference Style

%%======================================================%%
%% to compile with pdflatex/xelatex use pdflatex option %%
%%======================================================%%

%%\documentclass[pdflatex,sn-basic]{sn-jnl}% Basic Springer Nature Reference Style/Chemistry Reference Style


%%Note: the following reference styles support Namedate and Numbered referencing. By default the style follows the most common style. To switch between the options you can add or remove “Numbered” in the optional parenthesis. 
%%The option is available for: sn-basic.bst, sn-vancouver.bst, sn-chicago.bst%  
 
%%\documentclass[pdflatex,sn-nature]{sn-jnl}% Style for submissions to Nature Portfolio journals
%%\documentclass[pdflatex,sn-basic]{sn-jnl}% Basic Springer Nature Reference Style/Chemistry Reference Style
%\documentclass[pdflatex,sn-mathphys-num]{sn-jnl}% Math and Physical Sciences Numbered Reference Style 
%%\documentclass[pdflatex,sn-mathphys-ay]{sn-jnl}% Math and Physical Sciences Author Year Reference Style
%%\documentclass[pdflatex,sn-aps]{sn-jnl}% American Physical Society (APS) Reference Style
%%\documentclass[pdflatex,sn-vancouver,Numbered]{sn-jnl}% Vancouver Reference Style
\documentclass[pdflatex,sn-apa]{sn-jnl}% APA Reference Style 
%%\documentclass[pdflatex,sn-chicago]{sn-jnl}% Chicago-based Humanities Reference Style

%%%% Standard Packages
%%<additional latex packages if required can be included here>

\usepackage{graphicx}%
\usepackage{multirow}%
\usepackage{amsmath,amssymb,amsfonts}%
\usepackage{amsthm}%
\usepackage{mathrsfs}%
\usepackage[title]{appendix}%
\usepackage{xcolor}%
\usepackage{textcomp}%
\usepackage{manyfoot}%
\usepackage{booktabs}%
\usepackage{algorithm}%
\usepackage{algorithmicx}%
\usepackage{algpseudocode}%
\usepackage{listings}
\usepackage{array}       % 用于改进的表格支持
%
%%%%

%%%%%=============================================================================%%%%
%%%%  Remarks: This template is provided to aid authors with the preparation
%%%%  of original research articles intended for submission to journals published 
%%%%  by Springer Nature. The guidance has been prepared in partnership with 
%%%%  production teams to conform to Springer Nature technical requirements. 
%%%%  Editorial and presentation requirements differ among journal portfolios and 
%%%%  research disciplines. You may find sections in this template are irrelevant 
%%%%  to your work and are empowered to omit any such section if allowed by the 
%%%%  journal you intend to submit to. The submission guidelines and policies 
%%%%  of the journal take precedence. A detailed User Manual is available in the 
%%%%  template package for technical guidance.
%%%%%=============================================================================%%%%

%% as per the requirement new theorem styles can be included as shown below
% \theoremstyle{thmstyleone}%
% \newtheorem{theorem}{Theorem}%  meant for continuous numbers
% %%\newtheorem{theorem}{Theorem}[section]% meant for sectionwise numbers
% %% optional argument [theorem] produces theorem numbering sequence instead of independent numbers for Proposition
% \newtheorem{proposition}[theorem]{Proposition}% 
% %%\newtheorem{proposition}{Proposition}% to get separate numbers for theorem and proposition etc.

% \theoremstyle{thmstyletwo}%
% \newtheorem{example}{Example}%
% \newtheorem{remark}{Remark}%

% \theoremstyle{thmstylethree}%
% \newtheorem{definition}{Definition}%

\raggedbottom
%%\unnumbered% uncomment this for unnumbered level heads

\begin{document}
\title[Patterns and Purposes]{Patterns and Purposes: A Cross-Journal Analysis of AI Tool Usage in Academic Writing}

\author*[1]{\fnm{Ziyang} \sur{Xu}}\email{ziyangxu@ou.edu}

\affil*[1]{\orgdiv{School of Library and Information Studies}, 
\orgname{University of Oklahoma}, 
\orgaddress{\street{Bizzell Library, Room 120, 401 West Brooks}, 
\city{Norman}, 
\postcode{73019-6032}, 
\state{Oklahoma}, 
\country{United States}}}

\abstract{This study investigates the use of AI tools in academic writing through analysis of AI usage declarations in  journals. Using a mixed-methods approach combining content analysis, statistical analysis, and text mining, this research analyzed 168 AI declarations from 8,859 articles across 27 categories. Results show that ChatGPT dominates academic writing assistance (77\% usage), with significant differences in tool usage between native and non-native English speakers (p = 0.0483) and between international and non-international teams (p = 0.0012). The study reveals that improving readability (51\%) and grammar checking (22\%) are the primary purposes of AI tool usage. These findings provide insights for journal policy development and understanding the evolving role of AI in academic writing.
}


%%================================%%
%% Sample for structured abstract %%
%%================================%%

% \abstract{\textbf{Purpose:} The abstract serves both as a general introduction to the topic and as a brief, non-technical summary of the main results and their implications. The abstract must not include subheadings (unless expressly permitted in the journal's Instructions to Authors), equations or citations. As a guide the abstract should not exceed 200 words. Most journals do not set a hard limit however authors are advised to check the author instructions for the journal they are submitting to.
% 
% \textbf{Methods:} The abstract serves both as a general introduction to the topic and as a brief, non-technical summary of the main results and their implications. The abstract must not include subheadings (unless expressly permitted in the journal's Instructions to Authors), equations or citations. As a guide the abstract should not exceed 200 words. Most journals do not set a hard limit however authors are advised to check the author instructions for the journal they are submitting to.
% 
% \textbf{Results:} The abstract serves both as a general introduction to the topic and as a brief, non-technical summary of the main results and their implications. The abstract must not include subheadings (unless expressly permitted in the journal's Instructions to Authors), equations or citations. As a guide the abstract should not exceed 200 words. Most journals do not set a hard limit however authors are advised to check the author instructions for the journal they are submitting to.
% 
% \textbf{Conclusion:} The abstract serves both as a general introduction to the topic and as a brief, non-technical summary of the main results and their implications. The abstract must not include subheadings (unless expressly permitted in the journal's Instructions to Authors), equations or citations. As a guide the abstract should not exceed 200 words. Most journals do not set a hard limit however authors are advised to check the author instructions for the journal they are submitting to.}

\keywords{Academic Writing, ChatGPT, Journal policies, Large Language Models (LLMs) }

%%\pacs[JEL Classification]{D8, H51}

%%\pacs[MSC Classification]{35A01, 65L10, 65L12, 65L20, 65L70}

\maketitle

\section{Introduction}\label{sec1}

With the rise of generative AI, more and more AI tools are being introduced into the academic field. Generative AI models, led by ChatGPT, assist in drafting manuscripts, thereby improving the writing efficiency of researchers and authors \citep{thomas2023impact}. These features can assist in various stages of the academic writing process, from idea generation, literature review writing, and data analysis, to even proposal creation \citep{abd2023artificial}. AI's ability to generate coherent and contextually relevant text has led to its adoption in academic settings, where it is used to streamline the writing process and support researchers in organizing their ideas \citep{hsu2023can}.

However, this has also sparked concerns about the potential for artificial intelligence to undermine critical thinking and creativity. The key issues surrounding AI in academic publishing involve authorship and attribution of AI-generated content, data privacy concerns, and the appropriate scope for AI application\citep{mrabet2023chatgpt,storey2023ai}. In academic writing, \citet{dubose2023ai} highlights the need for responsible and careful use of artificial intelligence tools. \citet{caprioglio2023fake} advocates for the creation of clear guidelines to ensure the accuracy and reliability of AI-generated content. 

In response to this challenge, academic publishers have revised their editorial policies and author guidelines to offer clearer instructions on disclosing the use of generative AI in academic research. The \textit{Science} series of journals has consistently required authors to sign a license certificate, ensuring that their published papers are original works \citep{thorp2023chatgpt}. According to \citet{ganjavi2024publishers} ,24\%\ of the top 100 largest publishers offer guidelines for the use of generative AI. The most notable among them is Elsevier, which offers a specific AI usage disclosure template and recommends incorporating it into a new, dedicated section of the manuscript. This template requires authors to disclose whether they used generative artificial intelligence during the writing process, specify the type of AI used, and explain its primary purpose.

The current discussion on AI in academic writing primarily centers on ethical controversies, potential impacts, and related issues. However, there is still limited research on the specific purposes behind authors' use of AI in academic writing, as well as how this relates to their backgrounds and the journals in which they publish. This study aims to collect and analyze the 'Declaration of Generative AI and AI-assisted technologies in the writing process' from articles published across 27 Elsevier journal categories as samples. Through content analysis, association analysis, and text mining techniques, the research will explore the key components of AI usage declarations and investigate whether there are consistencies or variations in these declarations across different author teams and their backgrounds. The goal is to offer insights that can inform academic journal policies and broaden the discussion on the ethical and social implications of AI usage.

\section{Literature Review}


\subsection{Benefits of GAI for scholarly publishing}
%在研究学术写作中AI的作用之前,有大量的研究关于AI在学术写作中的好处。%
Before investigating the role of GAI in academic writing, it is essential to understand why researchers are drawn to using GAI in the first place. Most scholars hold a dialectical perspective on the impact of GAI on academic writing. They believe that GAI can enhance writing efficiency, provide new ideas, correct grammatical structures and spelling errors, and offer personalized guidance \citep{lund2023chatting,kasneci2023chat,DWIVEDI2023102642}. Some studies explore the assistive value of ChatGPT in writing, demonstrating potential both for peer-reviewed literature reviews and for content analysis of published literature \citep{dergaa2023from,rahman2023chatgpt}. To validate ChatGPT's exceptional writing abilities, two scientists, with the assistance of ChatGPT, completed a research paper in under an hour. The study found that ChatGPT could not only handle data processing but also polish the initial draft \citep{conroy2023scientists}.

In empirical research, \citet{fyfe2023how} conducted a study involving 20 university students across different academic levels, investigating their experiences using GPT-2 to complete their final course papers. Eighty-seven percent of the students reported that collaborative writing with GPT was more manageable than completing the work independently, as it provided new arguments and ideas. Science published a controlled experimental study involving professionals from various fields, showing that ChatGPT is especially beneficial for individuals with weaker writing skills \citep{noy2023experimental}. The results indicated that ChatGPT can elevate their writing abilities to a level close to that of proficient writers.

\subsection{Ethical Concerns and Policies for AI in Academic Publishing}
As early as 2020, two initiatives for designing AI intervention reports were introduced in the medical field: CONSORT-AI (Consolidated Standards of Reporting Trials for Artificial Intelligence) and SPIRIT-AI \citep{Liu2020,CruzRivera2020}. In February 2023, the Committee on Publication Ethics (COPE) issued a position statement on the use of AI tools in research publications, emphasizing that AI tools cannot be listed as co-authors and providing guidelines on how to disclose the use of AI \citep{COPE2023}. This statement was quickly endorsed by journals and editorial associations such as the International Committee of Medical Journal Editors (ICMJE), the Journal of the American Medical Association (JAMA), and the World Association of Medical Editors (WAME) \citep{ICMJE2023,Flanagin2023,WAME2023}.  The introduction of such policies is precisely due to the growing concerns among scholars about the ethical issues surrounding the use of generative AI in academic publishing \citep{Anderson2023}.  Large language models can produce a phenomenon known as "hallucination," where the generated text appears coherent and meaningful but is actually fabricated. This is also referred to as the "stochastic parrot" effect \citep{Hutson2021,Ji2023}. In addition, since GPT-3 is trained on large web-based datasets, it inevitably inherits racial, gender, and ethnic biases during the training process \citep{Basta2019,Founta2018,Hutchinson2020}. These biases can be propagated through the publication of academic articles, further amplifying their impact within the scholarly community. 
\subsection{Impetus for the current study}
However, these concerns have largely remained in the realm of theoretical discussions. There is a lack of empirical investigation into the actual purposes behind researchers' use of generative AI in academic writing and the potential ethical and academic impacts it may cause. While much has been written about how AI is being integrated into academia, there is a gap in understanding the reasons for AI use in writing.
By studying the AI usage statements in papers, the work will provide insights into why authors are turning to AI and how these reasons may vary across  author teams and authors' language backgrounds.


\section{Research Questions}

This paper aims to address the following types of research questions:
\begin{enumerate}
\renewcommand{\labelenumi}{(\theenumi)}  % 这行设置编号格式为 (1), (2), (3)
\item What types of AI tools are currently being used by researchers to assist in academic writing?
\item What specific tasks is AI primarily used to accomplish?
\item Are there differences in AI tool usage patterns between international teams and non-native English speakers compared to single-country teams and native English speakers?
\end{enumerate}



\section{Methodology}
\subsection{Journal selection and data acquisition}
%本研究的数据收集范围如下,本文基于Scopus数据库(https://www.scopus.com/home.uri;),以Scopus所划分的27个主要学科类别中的期刊为研究对象,按照CiteScore指标选择每个类别中排名最高的期刊。同时为了更好的分析文章中的AI申明,本文将范围限定在Elsevier出版的开放获取期刊。收录的范围涵盖 2024 年第一期至 2024 年 10 月期间该期刊所发布的所有文章。%
% Table generated by Excel2LaTeX from sheet '期刊选择。
The data collection scope of this study is as follows. Based on the Scopus database (\texttt{https://www.scopus.com/home.uri}), this study focuses on journals classified within the 27 major subject categories defined by Scopus. The highest-ranked journal in each category was selected based on the CiteScore metric. To better analyze AI declarations within articles, this study restricts the scope to open-access journals published by Elsevier. The study scope covers all articles published in these journals during 2024. A total of 8,633 Elsevier articles and 226 ASIS\&T conference papers were collected. After manual inspection, 168 articles containing AI declarations were identified. Table \ref{journal} presents the specific scope of journals included in this study.

%在确定好期刊的选择范围后,提取文章中的Declaration of Generative AI and AI-assisted technologies in the writing process,该申明中会描述出作者使用了哪些AI工具,以及使用的具体用途,示例如下: %
After determining the selection scope of journals, statements of "Declaration of Generative AI and AI-assisted technologies in the writing process" will be extracted from the articles.  These statements describe which AI tools the authors used and specify their intended purposes. An example is as follows:


\textit{“Statement: During the preparation of this work, the author(s) used ChatGPT to generate essays. After using this tool/service, the author(s) reviewed and edited the content as needed and take(s) full responsibility for the content of the publication.”}


In addition, metadata of the articles (title, author names, journal title) and information about the authors' team (authors' background and language information) will be collected.

\begin{table}
\centering    % 添加居中命令

\small
\caption{Scope of journals}
\begin{tabular}{p{5cm}p{5cm}p{1cm}p{1cm}}  % 精确控制每列宽度
\toprule
Major Categories & Journal Title & CiteScore & AI Decl.(\textperthousand)\\
\midrule
Agricultural and Biological Sciences & Environmental Technology \& Innovation & 14.0 & 4.30\\
Art and Humanities & Acta Psychologica & 3.0 & 28.10\\
Biochemistry & eBioMedicine & 17.7 & 8.71\\
Business, Management and Accounting & Digital Business & 7.4 & 80.00\\
Chemical Engineering & Ultrasonics Sonochemistry & 15.8 & 7.37\\
Chemistry & Redox Biology & 19.9 & 11.76\\
Computer Science & Int. J. Information Management Data Insights & 19.2 & 43.50\\
Decision Sciences & Transport. Research Interdisciplinary Persp. & 12.9 & 27.21\\
Dentistry & The Saudi Dental Journal & 3.6 & 3.89\\
Earth and Planetary Sciences & Progress in Disaster Science & 14.6 & 60.97\\
Economics & Journal of Innovation \& Knowledge & 16.1 & 23.26\\
Engineering & Advances in Applied Energy & 23.9 & 73.17\\
Environmental Science & Environment International & 21.9 & 3.68\\
Health Professions & Complementary Therapies in Medicine & 8.6 & 31.25\\
Immunology and Microbiology & J. Microbiology, Immunology and Infection & 15.9 & 23.80\\
Materials Science & Materials \& Design & 14.3 & 15.94\\
Mathematics & Partial Diff. Equations in Applied Math. & 6.2 & 2.40\\
Medicine & Int. J. Infectious Diseases & 18.9 & 1.48\\
Multidisciplinary & Scientific African & 5.6 & 23.06\\
Neuroscience & Brain Stimulation & 7.6 & 26.31\\
Nursing & Journal of Functional Foods & 9.6 & 7.33\\
Pharmacology & Molecular Therapy Nucleic Acids & 15.4 & 3.77\\
Physics and Astronomy & Ultrasonics Sonochemistry & 15.8 & 7.35\\
Psychology & Comprehensive Psychiatry & 12.5 & 9.90\\
Social Sciences & Computers and Education: AI & 16.8 & 201.43\\
Veterinary & Current Research in Parasitology & 3.6 & 15.38\\
\bottomrule
\end{tabular}
\label{journal}
\end{table}

\subsection{Design of the study}
To conduct this study, an embedded mixed methods design was employed; This study employs content analysis, quantitative analysis, and text mining methods to systematically examine the use of generative AI tools in academic writing, their application tasks and the impact of team characteristics. 
%本研究采用了内容分析、定量分析和文本挖掘方法,以系统性地分析学术写作中生成式AI工具的使用情况、应用任务、团队特征的影响以及声明内容中的主题分布。%
\begin{figure}[h]
    \centering
    \includegraphics[width=0.75\linewidth]{flow.drawio.pdf}
    \caption{Data Analysis Flowchart}
    \label{fig:Data Analysis Flowchart }
\end{figure}


\subsubsection{Content Analysis}
%%为了解当前研究人员在学术论文写作中主要使用的AI工具类型及其应用任务,本研究对收集到的AI使用声明进行了系统的内容分析。首先,研究人员设计了一套编码框架,将声明中的主要信息进行分类编码,包括AI工具类型(如ChatGPT、Grammarly等)和具体任务(如语法检查、句子润色、内容生成等)。研究人员手动阅读和编码每条声明,利用定性分析软件(Nvivo)进行手工编码,并将编码结果输入数据表以供后续分析。%%
To understand the primary types of AI tools and their application tasks used by researchers in academic paper writing, this study conducts a systematic content analysis of the collected AI usage declarations.  First, the researchers designed a coding framework to categorize and code the main information in the declarations, including types of AI tools (e.g., ChatGPT, Grammarly) and specific tasks (e.g., grammar checking, sentence polishing, content generation).  Furthermore, in the coding process, binary coding was applied to indicate the first author's language background and team composition, where 0 denotes negative cases and 1 denotes positive cases. 

\subsubsection{Quantitative Analysis}


The quantitative analysis consists of two parts: descriptive statistical analysis and variable association analysis. To address Research Question 3, this paper proposes the following hypotheses: 

\textbf{Hypothesis 1: There is a significant association between authors' native language status and their purposes for using AI tools.}

Non-native English speaking international graduate students face challenges in academic writing due to English being a second language \citep{Singh2015InternationalGS,Flowerdew2019TheLD}.  Artificial intelligence can help non-native English-speaking scientists overcome challenges in scientific writing \citep{giglio2023use}. Writing assistance tools for ESL learners have been developed to address various challenges faced by non-native English writers. These tools focus on error correction \citep{leacock2009user}, planning and writing support \citep{lim2012esl}, and first-language-oriented assistance \citep{chen2012flow}. 



\textbf{Hypothesis 2: There is a significant association between team composition and their purposes for using AI tools. }

 Research on communication in international collaborative research teams remains limited, with gaps in understanding team diversity, language use, and communication processes in the social sciences and humanities \citep{wohlert2020communication}. 
 However, information and communication technologies (ICT) can mitigate these negative effects on intercultural communication while supporting the positive impact on decision-making \citep{Shachaf2008CulturalDA}. International teams tend to use richer communication media more intensively than domestic teams when dealing with complex tasks \citep{Bjorvatn2019ComplexityAA}. In collaborative writing, mixed nationality pairs focus more on content discussion, while same-nationality pairs engage more in language-related aspects \citep{rahayu2020interaction}. 


The dataset was categorized using two indicators: language background and team composition characteristics.  The analysis is based on 2×6 contingency tables that include two binary classification grouping variables (Native Speaker/Non-Native Speaker; International Group/Non-International Group) and six usage purpose categories. 

This paper employed the Fisher-Freeman-Halton exact test to analyze the association between these variables, because more than 20\% of cells have expected frequencies less than 5, violating chi-square test assumptions. This test serves as an extension of Fisher's exact test on RxC contingency tables, particularly suitable for analyzing categorical data of this type with small samples.




\subsubsection{Text Mining}

To enhance analytical efficiency and support content analysis, this study further employs text mining techniques to automate the analysis of declaration texts. This section consists of two main parts: basic textual analysis and semantic pattern analysis. The text analysis process comprises three main steps:

(1) Data preprocessing: Initially, all text was converted to lowercase. Second, template sentences from the original declaration statements were removed. Third, term standardization was performed to address inconsistencies in authors' expression of technical terms (e.g., converting 'gpt-4' to 'chatgpt4'). Finally, tokenization and stop word removal were conducted.

(2) Basic textual analysis: This began with word frequency analysis, followed by bigram analysis. Subsequently, differential analysis was conducted based on language backgrounds, grouping authors into native and non-native speakers. The relative frequencies for each group were calculated, and the results were visualized through word clouds.

(3) Relationship extraction and network analysis: In the relationship extraction phase, Python's network package was used to extract tool-purpose relationships and tool-tool collaborative relationships, with frequency weights assigned to each relationship level. During the network construction phase, graph structures were built with nodes and edges added. The network analysis phase involved calculating network statistics and centrality metrics. Finally, data visualization was performed using undirected bipartite weighted network graphs.

\section{Result}


\subsection{Content Analysis Results}


%%%%%%

In the academic writing environment, ChatGPT by OpenAI maintains its dominant position. Figure \ref{fig1} shows that 117 authors use ChatGPT (including GPT-3.5 and GPT-4) to assist writing, accounting for 77\% of total usage. The majority of researchers only emphasize the application of ChatGPT, without declaring the specific version. Furthermore, Grammarly and DeepL occupy the fourth and fifth positions respectively, due to their specialized capabilities in grammar-checking and translation. 

\begin{figure}[ht]
    \centering
    \includegraphics[width=0.8\linewidth]{ai_tools_usage_bar.jpeg}
    \caption{AI tools usage distribution}
    \label{fig1}
\end{figure}


\begin{figure}[H]
    \centering
    \includegraphics[width=0.6\linewidth]{ai_tools_purpose_pie.jpeg}
    \caption{Distribution of AI tools Usage Purpose}
    \label{ai_tools_purpose_pie}
\end{figure}

The purposes of AI tools usage are classified in nine ways, ranging from readability, grammar, proofreading, statistical analysis, translation, paraphrasing, generating titles or abstracts, searching literature, and brainstorming. The main reason that authors use AI tools is to improve the readability of manuscripts, accounting for 51\%. Grammar-checking is the second most common purpose, with 22\% of authors declaring that they used AI tools to check grammar. It is noteworthy that some scholars utilize AI tools for statistical analysis and content generation. While these various applications primarily serve auxiliary functions during the academic writing process, researchers also employ tools like ChatGPT for brainstorming and literature searches in the preliminary stages of academic writing. In addition, the utilization of AI tools in academic writing is not limited to a single application. Twenty-one percent of scholars employed a combination of two different AI tools, while a minority of researchers utilized three tools to facilitate their academic writing process. 

\begin{table}[ht]
\centering
\caption{Examples of AI Tool Usage Purposes in Academic Writing}
\begin{tabular}{p{1.5cm}p{12cm}}
\toprule
Purpose & Example Statement \\
\midrule
Readability & The authors utilized GPT-4 from OpenAI to enhance language and readability. \\
\addlinespace
Grammar & During the preparation of this work the authors used Grammarly, Inc. in order to review spelling and grammar. \\
\addlinespace
Proofreading & We employed ChatGPT 4.0 for proofreading the text of this contribution. \\
\addlinespace
Statistical Analysis & For the preparation of this paper, a proprietary large language model, i.e., GPT-4 Turbo, was utilized specifically in the process of labeling the detected communities within our dataset. \\
\addlinespace
Translation & During the preparation of this work, the authors used DeepL to translate parts of the manuscript from German to English. \\
\addlinespace
Paraphrase & In preparing this work, the author(s) employed the assistance of ChatGPT 3.5 to paraphrase the manuscript. \\
\addlinespace
Content Generation & The authors used ChatGPT solely for English proofreading, and the graphical abstract was partially designed by DALL-E 3. \\
\addlinespace
Brainstorm & ChatGPT 4 and GrammarlyGO were used to brainstorm and refine sentence structures in select parts of the manuscript. \\
\addlinespace
Literature Search & The author(s) used Microsoft's Bing AI to search for literature and format references. \\
\bottomrule
\multicolumn{2}{l}{\small{Note: These statements are representative examples extracted from published papers.}}
\end{tabular}
\label{tab:purpose_examples}
\end{table}

\subsection{Quantitative Analysis Results}
Table \ref{tab:distribution} reveals that non-native speakers constitute the vast majority of the sample, accounting for 81\%. Regarding team composition distribution, more than two-thirds of the authors are from international teams. To further illustrate the sample composition and prepare for association analysis, Table \ref{tab:cross_analysis} presents a cross-analysis of language background and team characteristics. The findings show that the largest group consists of non-native speakers in non-international teams, while the smallest group comprises native speakers in international teams. Table \ref{tab:tool_usage} displays AI tool usage patterns among authors with different backgrounds. Compared to native speakers, non-native speakers tend to use a more diverse range of tools. Certain tools (DeepL and Claude) are almost exclusively used by non-native speakers.




\begin{table}[htbp]
\centering
\caption{Distribution of Author Characteristics}
\begin{tabular}{lcc}
\toprule
Characteristics & Count & Percentage (\%) \\
\midrule
Native Speaker & 32 & 19.0 \\
Non-native Speaker & 136 & 81.0 \\
\midrule
International Group & 46 & 27.4 \\
Non-international Group & 122 & 72.6 \\
\bottomrule
\end{tabular}
\label{tab:distribution}
\end{table}


\begin{table}[htbp]
\centering
\caption{Cross Analysis of Author Background}
\begin{tabular}{lccc}
\toprule
Native Speaker & International Group & Count & Percentage (\%) \\
\midrule
Yes & Yes & 12 & 7.1 \\
Yes & No & 20 & 11.9 \\
No & Yes & 34 & 20.2 \\
No & No & 102 & 60.7 \\
\bottomrule
\end{tabular}
\label{tab:cross_analysis}
\end{table}


\begin{table}[htbp]
\centering
\caption{AI Tools Usage by Author Background}
\begin{tabular}{lcccc}
\toprule
\multirow{2}{*}{AI Tools} & \multicolumn{2}{c}{Native Speaker} & \multicolumn{2}{c}{International Group} \\
\cmidrule(lr){2-3} \cmidrule(lr){4-5}
& Yes & No & Yes & No \\
\midrule
ChatGPT & 13 & 65 & 22 & 56 \\
ChatGPT 4.0 & 7 & 19 & 8 & 17 \\
ChatGPT 3.5 & 4 & 11 & 4 & 11 \\
Grammarly & 5 & 7 & 4 & 8 \\
Deepl & 0 & 7 & 1 & 6 \\
Claude & 0 & 4 & 1 & 3 \\
Other Tools & 4 & 23 & 6 & 21 \\
\bottomrule
\end{tabular}
\label{tab:tool_usage}
\end{table}


\textbf{Analysis of Authors' Language Background (H1)} The Fisher-Freeman-Halton exact test results show a significant association between the authors' language background and the purpose of using AI tools (p = 0.0483 \textless 0.05). The minimum expected frequency is 1.14. The analysis reveals several key differences: Non-native speakers have a significantly higher proportion of grammatical usage compared to native speakers (25.7\% vs 15.6\%). The use of translation tools is entirely concentrated in the non-native speaker group. Both groups show similar levels of concern regarding readability. 

\textbf{Team Composition Analysis (H2)} The results of the Fisher-Freeman-Halton exact test show a highly significant association between team composition and the purpose of using AI tools (p = 0.0012 \textless 0.01). The minimum expected frequency is 1.64. Key findings include: international teams place significantly more emphasis on grammar than non-international teams (30.4\% vs 21.3\%). No proofreading and analysis purposes were observed among international teams. The proportion of international teams in the "other" usage category is significantly higher. 

\begin{table}[htbp]
\small 
\centering
\caption{Association between Author Background and AI Tool Usage Purposes}
\begin{tabular}{lcccccc}
\toprule
\multicolumn{7}{l}{\textbf{Panel A: Native Speaker Status}} \\
\midrule
& Read. & Gram. & Proof. & Anal. & Trans. & Other \\
\midrule
Native & 16 & 5 & 4 & 3 & 0 & 4 \\
(n=32) & (50.0\%) & (15.6\%) & (12.5\%) & (9.4\%) & (0.0\%) & (12.5\%) \\
\addlinespace
Non-native & 64 & 35 & 11 & 5 & 6 & 15 \\
(n=136) & (47.1\%) & (25.7\%) & (8.1\%) & (3.7\%) & (4.4\%) & (11.0\%) \\
\midrule
Exp. count & 1.14- & 1.14- & 2.86- & 1.52- & 1.14- & 3.62- \\
range & 15.24 & 7.62 & 12.14 & 6.48 & 4.86 & 15.38 \\
\midrule
\multicolumn{7}{l}{Fisher-Freeman-Halton test: p = 0.0483*} \\
\midrule
\multicolumn{7}{l}{\textbf{Panel B: International Group Status}} \\
\midrule
& Read. & Gram. & Proof. & Anal. & Trans. & Other \\
\midrule
International & 21 & 14 & 0 & 0 & 1 & 10 \\
(n=46) & (45.7\%) & (30.4\%) & (0.0\%) & (0.0\%) & (2.2\%) & (21.7\%) \\
\addlinespace
Non-international & 59 & 26 & 15 & 8 & 5 & 9 \\
(n=122) & (48.4\%) & (21.3\%) & (12.3\%) & (6.6\%) & (4.1\%) & (7.4\%) \\
\midrule
Exp. count & 1.64- & 1.64- & 4.11- & 2.19- & 1.64- & 5.20- \\
range & 21.90 & 10.95 & 10.89 & 5.81 & 4.36 & 13.80 \\
\midrule
\multicolumn{7}{l}{Fisher-Freeman-Halton test: p = 0.0012**} \\
\bottomrule
\multicolumn{7}{p{0.95\linewidth}}{\small{Note: *p \textless 0.05, **p \textless 0.01. Read. = Readability, Gram. = Grammar, }} \\
\multicolumn{7}{p{0.95\linewidth}}{\small{Proof. = Proofreading, Anal. = Analysis, Trans. = Translation.}} \\
\multicolumn{7}{p{0.95\linewidth}}{\small{Percentages in parentheses represent within-group proportions.}}
\end{tabular}
\label{tab:purpose_association}
\end{table}

\subsection{Text Mining Results}

Basic textual analysis reveals that each declaration contains an average of 28.73 words, with a total of 703 unique words and 1,189 unique bigram combinations. Word cloud visualization identifies the most frequent substantive terms including "chatgpt", "improve", "used", "language", and "readability". The most frequently occurring phrases include "improve readability", "language readability", and "responsibility submission". Furthermore, as shown in Table \ref{tab:vocabulary_differences}, authors from different backgrounds demonstrate distinct vocabulary usage patterns. Native speakers tend to employ more subjective expressions (e.g., "our", "have"), while non-native speakers favor more objective and formal expressions. Non-native speakers place greater emphasis on language-related improvements (e.g., "grammar", "language", "readability"), whereas native speakers focus more on editing and revision processes (e.g., "edit", "reviewing").

\begin{figure}[ht]
    \centering
    \includegraphics[width=0.7\linewidth]{wordcloud_statements_colored.jpeg}
    \caption{World-cloud Statement}
    \label{fig:worldcloud-label}
\end{figure}



\begin{table}[htbp]
\centering
\caption{Vocabulary Usage Differences Between Native and Non-native Speakers}
\begin{tabular}{lccc}
\toprule
Word Category & Native Speakers & Non-native Speakers & Difference* \\
\midrule
\multicolumn{4}{l}{\textbf{Words More Used by Native Speakers}} \\
our & 6.69 & 1.59 & +5.10 \\
use & 5.74 & 1.59 & +4.15 \\
content & 21.03 & 16.94 & +4.10 \\
grammarly & 6.69 & 2.91 & +3.78 \\
employed & 8.60 & 5.03 & +3.58 \\
\midrule
\multicolumn{4}{l}{\textbf{Words More Used by Non-native Speakers}} \\
readability & 7.65 & 15.88 & -8.23 \\
chatgpt & 11.47 & 19.05 & -7.58 \\
grammar & 1.91 & 6.88 & -4.97 \\
language & 9.56 & 15.08 & -5.52 \\
improve & 12.43 & 17.46 & -5.04 \\
\bottomrule
\multicolumn{4}{l}{\small{*Differences are shown per 1,000 words}} \\
\end{tabular}
\label{tab:vocabulary_differences}
\end{table}

As mentioned above, some authors utilize multiple AI tools to assist in academic writing. Through semantic analysis, ChatGPT demonstrates the highest centrality in the semantic network. ChatGPT shows strong connections with its other versions (4.0, 3.5). Grammarly is frequently used in conjunction with the ChatGPT family of tools. DeepL and Claude tend to be used as supplementary tools.
In the tool-purpose network graph, the ChatGPT family of tools is primarily employed for improving readability, while Grammarly focuses on grammar checking. DeepL and Claude serve more specialized purposes.

\begin{figure}[h]
    \centering
    \includegraphics[width=0.9\linewidth]{vis.jpeg}
    \caption{Network Visualization}
    \label{fig:network_visualization}
\end{figure}


\section{Discussion}
This study aimed to explore the trendy and purpose of AI tools used by researchers in academic writing. In addition, this paper examined the association of authors' background and AI tools usages. Three notable findings emerged in this study: (a) Dominance and Diversity in Adoption of AI Tools, (b) Author Background Influence on AI Tool Utilization and (c) Primary Applications and Emerging Use Cases. 

Firstly, ChatGPT family tools are predominant AI tools in academic writing.  This finding is corroborated by other studies. \citet{selim2024transformative}'s research on non-native English-speaking university students revealed that ChatGPT and Grammarly are the two most frequently used AI tools. 
The emergence of ChatGPT, as a pioneering generative artificial intelligence system, transformed conventional understanding of AI agents and generated extensive societal discussions. This first-mover advantage has caused ChatGPT becoming the most extensively utilized large language model. However, researchers, especially non-native speakers, also utilized other AI tools, such as DeepL and Grammarly, which had a huge number of users prior to the prevalence of ChatGPT, due to user loyalty and the support of Large Language Models. 

Secondly, the Fisher-Freeman-Halton exact test results show a significant association between language background (p = 0.0483) and significant influence of team composition (p = 0.0012). There are different tool preferences between native and non-native speakers. Furthermore, team structure influences tool selection and purpose. The higher percentage of Grammarly usage in non-native groups reflects their greater attention to the need for language accuracy. The challenges faced by non-native speakers, such as grammar, wording, and expression, require multiple tools rather than relying on a single solution. The instant feedback provided by AI narrows the gap between non-native speakers and native speakers in academic writing.  For non-native English-speaking researchers, AI tools can help translate literature to support their academic research \citep{zenni2023artificial,hwang2023chatgpt,song2023enhancing}. 
International groups prefer to utilize Grammarly more than the non-international group, which indicates specific challenges in cross-language communication. The unification of writing style through these tools leads to reduced communication costs arising from language differences.

To provide differentiated support for researchers, academic journals are required to develop targeted guidelines for the use of AI tools. However, numerous academic journals lack established AI usage policies, not to mention differentiated support systems.  According to a survey of the top 100 academic publishers, only 24\%  provided guidance on the use of Generative AI \citep{ganjavi2024publishers}.%引用期刊政策的比例%%

Thirdly, improving manuscript readability is the primary purpose, while grammar checking is the secondary purpose. Additionally, statistical analysis and literature searching are also common usage scenarios. This raises new academic ethical questions about whether non-language enhancement purposes, such as data analysis and abstract generation, should be permitted in AI tool usage. The scope of AI usage in academic contexts remains an unresolved issue. %引用现有ai政策怎么样的 
Some academic journals encourage authors to use AI for editing manuscript and papers, , provided that the technology’s contribution is appropriately recognized \citep{crawford2023artificial}.  However, the specific effectiveness and extent of policy implementation remain uncertain, considering the accuracy of AI-generated text \citep{Hu06122024}. Academics across disciplines have been quick to adopt new technologies and tools to assist with research and practice \citep{GREWAL2021229}.  The ongoing discussions include considerations about listing ChatGPT as a co-author, requirements for disclosing AI usage in the acknowledgments section, and the potential necessity of submitting a separate declaration document to journals specifying AI-generated content \citep{doi:10.1177/14413582231167882}. 
%

Finally, a notable finding reveals that researchers integrate AI tools throughout the entire research cycle. At the pre-research stage, generative AI facilitates brainstorming and exploration of diverse research angles. During manuscript development, AI supports data analysis and abstract composition. \citet{huang2024evaluating} argues that researchers tend to use generative AI tools as auxiliary tools to analyze data and inspire more innovative ideas. In the final stage, it aids in refining language and improving readability. ChatGPT can take on the role of a virtual advisor that supports study and research, not only resolving questions but also providing reviews and feedback \citep{choi2021chatgpt}.  

%重塑信息行为的文献%


\subsection{Practical Implications
}
Although scholars have called for academic organizations, such as the APA, to generate AI usage guidelines in research, this guide does not focus on different hierarchical requirements \citep{tate2023educational}. As a next step, AI policy is supposed to distinguish between basic language polishing and deep content generation. In order to improve the transparency of AI usage, publishers need to design templates for references. Many statements merely declare which AI tools were utilized, without specifying the tasks and scopes of application. The review system should not take a one-size-fits-all approach, but rather adopt different evaluation standards for various types of AI applications (such as translation versus content generation).

%%非母语更依赖AI工具,但也可能缺乏系统性的ai使用指导。研究者需要掌握多种ai工具组合的使用,也需要更全面的培训支持。因此生成AI素养教育纳入研究生教育体系的课程。尽管ai工具便捷了科研流程,但是对研究人员批判性思维(培养对AI输出的评估和验证能力 )和伦理意识(理解AI使用的边界和责任  )也提出了更高的要求。ChatGPT 在用于支持研究时通常会提供虚假参考。ai素养的培养不应该局限于writing process这一个环节,根据本文的研究发现,ai工具贯穿在整个科研周期中。所以ai信息教育应该跳出传统信息素养的框架,培养%%
Researchers exhibit significant variations in their use of AI tools, reflecting the necessity of AI literacy education. Non-native speakers rely more heavily on AI tools while lacking systematic AI usage guidelines. Researchers need to master multiple AI tools, which requires comprehensive training support. Therefore, AI literacy should be incorporated into graduate courses. Although AI streamlines the research process, it sets higher standards for critical thinking (the ability to evaluate the reliability of AI outputs) and ethical awareness (understanding the scope and responsibilities of AI). ChatGPT usually provides fabricated references \citep{Mjovsk2023ArtificialIC}. The education of AI literacy should not be confined to the writing process. According to our findings, AI tools span the entire research cycle. The education of AI literacy needs to break from traditional frameworks and develop a comprehensive framework that addresses AI applications across different research stages, from literature review and data analysis to result interpretation and scholarly communication.

%%aitools下一部的研究方向应该致力于可解释的人工智能。现实证明llms无法像人类科研人员完全撰写新颖的学术文本和研究报告。考虑到科研的完整性和严谨性,研究者在引用llms生成的语句时,需要知道生成的内容从何而来。这对人工智能的解释性提出了较高的要求。ai工具的另一个发展方向是多功能的集成。工具间的割裂导致使用成本增加和效率降低.目前科技互联网巨头进入了llm的‘arms race’ ,但是却忽略以人为本的发展理论。因为ai tools的发展方向需要加强工具间的互操作性,使不同功能的工具能够无缝衔接,为用户提供一站式的解决方案。例如,将语言优化、内容协作和专业术语管理等功能整合到统一的平台中。其次,要重视领域适配,针对不同学科和写作场景开发专门化功能。

Future research on AI tools should focus on explainable AI. LLMs cannot independently produce novel academic papers and research reports. Considering research integrity and rigor, researchers must identify the sources of AI-generated content when citing sentences produced by AI systems. To achieve this objective, AI systems need to enhance their knowledge tracing mechanisms and LLM evaluation frameworks.

Another critical research direction is the integration of diverse functionalities. The fragmented nature of current tools increases operational overhead while reducing efficiency. While major technology and internet corporations are engaged in competition regarding LLM algorithmic performance, they have neglected human-centered development principles and methodologies. The development of AI tools should prioritize interoperability among different systems, facilitating seamless integration of various functions to provide comprehensive solutions. For instance, functionalities such as proofreading, content collaboration, and terminology management should be unified within a single platform. Furthermore, AI tools need to develop discipline-specific functions tailored to various academic contexts and writing scenarios.


\section{Conclusion}

This study reveals several key patterns in the use of AI tools in academic writing. The dominance of ChatGPT, coupled with the differential usage patterns between native and non-native English speakers, suggests a transformative shift in academic writing practices. The findings highlight the need for more nuanced AI usage guidelines in academic publishing, particularly considering the diverse needs of international research teams. While AI tools primarily serve to enhance readability and grammar, their expanding role in content generation and analysis raises new questions about academic integrity and authorship. The present study has certain limitations that should be acknowledged. The primary limitation is the relatively small size of the dataset used in the analysis. Additionally, the reliability of AI usage declarations poses another limitation, as authors' disclosures about their use of AI tools may not always be complete or accurate. It is possible that some authors utilized AI tools in their work without making the appropriate declarations. Further, author will use the large datasets, focusing on evolution and current landscape of artificial intelligence (AI) tool usage across different academic disciplines. 





\backmatter


\section*{Statements and Declarations}

\subsection*{ Conflict of interest/Competing interests}
The author declares that the research was conducted in the absence of any commercial or financial relationships that could be construed as potential conflicts of interest

\subsection*{Data availability}

The datasets generated during and/or analysed during the current study are available from the corresponding author on reasonable request.






%%===========================================================================================%%
%% If you are submitting to one of the Nature Portfolio journals, using the eJP submission   %%
%% system, please include the references within the manuscript file itself. You may do this  %%
%% by copying the reference list from your .bbl file, paste it into the main manuscript .tex %%
%% file, and delete the associated \verb+\bibliography+ commands.                            %%
%%===========================================================================================%%
\bibliography{references}% common bib file
%% if required, the content of .bbl file can be included here once bbl is generated
%%\input sn-article.bbl


\end{document}
