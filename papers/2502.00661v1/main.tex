%%%%%%%%%%%%%%%%%%%%%%%%%%%%%%%%%%%%%%%%%%%%%%%%%%%%%%%%%%%%%%%%%%%%%%%%%%%%%%%%
%2345678901234567890123456789012345678901234567890123456789012345678901234567890
%        1         2         3         4         5         6         7         8

\documentclass[letterpaper, 10 pt, conference]{ieeeconf}  % Comment this line out if you need a4paper

%\documentclass[a4paper, 10pt, conference]{ieeeconf}      % Use this line for a4 paper

\IEEEoverridecommandlockouts                              % This command is only needed if 
                                                          % you want to use the \thanks command

\overrideIEEEmargins                                      % Needed to meet printer requirements.

%In case you encounter the following error:
%Error 1010 The PDF file may be corrupt (unable to open PDF file) OR
%Error 1000 An error occurred while parsing a contents stream. Unable to analyze the PDF file.
%This is a known problem with pdfLaTeX conversion filter. The file cannot be opened with acrobat reader
%Please use one of the alternatives below to circumvent this error by uncommenting one or the other
%\pdfobjcompresslevel=0
%\pdfminorversion=4

% See the \addtolength command later in the file to balance the column lengths
% on the last page of the document

% The following packages can be found on http:\\www.ctan.org
%\usepackage{graphics} % for pdf, bitmapped graphics files
%\usepackage{epsfig} % for postscript graphics files
%\usepackage{mathptmx} % assumes new font selection scheme installed
%\usepackage{times} % assumes new font selection scheme installed
%\usepackage{amsmath} % assumes amsmath package installed
%\usepackage{amssymb}  % assumes amsmath package installed
\usepackage{cite}
\usepackage{graphicx}
\usepackage{booktabs}
\usepackage{tabularx}
\usepackage{array}
\usepackage{color}
\usepackage{tabularray}
\UseTblrLibrary{booktabs}
\usepackage{xcolor}
\usepackage{gensymb}
\usepackage{threeparttable}
\usepackage{amsmath,amssymb,amsfonts,dsfont}
\usepackage{mathtools, bm}
\usepackage{algorithm,algorithmicx,listings}
\usepackage[noend]{algpseudocode}
\usepackage{hyperref}
\usepackage{tikz}

\makeatletter
\newcommand*\titleheader[1]{\gdef\@titleheader{#1}}
\AtBeginDocument{%
	\let\st@red@title\@title
	\def\@title{%
		\bgroup\normalfont\large\centering\@titleheader\par\egroup
		\vskip1.5em\st@red@title}
}
\makeatother

\titleheader{This work has been submitted to the IEEE for possible publication. Copyright may be transferred without notice, after which this version may no longer be accessible.}

\title{\LARGE \bf
EKF-Based Radar-Inertial Odometry with Online Temporal Calibration
}

\author{Changseung Kim$^{1}$, Geunsik Bae$^{1}$, Woojae Shin$^{1}$, Sen Wang$^{2}$, and Hyondong Oh$^{1}$% <-this % stops a space
\thanks{*This research was supported by the Technology Innovation Program (No. 20018110, "Development of a wireless teleoperable relief robot for detecting searching and responding in narrow space") funded By the Ministry of Trade, Industry \& Energy (MOTIE, Korea). (Corresponding author: Hyondong Oh)}% <-this % stops a space
\thanks{$^{1}$Department of Mechanical Engineering, Ulsan National Institute of Science and Technology (UNIST), Ulsan 44919, Republic of Korea (e-mail: \{pon02124; baegs94; oj7987; h.oh\}@unist.ac.kr).}%
\thanks{$^{2}$Department of Electrical and Electronic Engineering, Imperial College London, SW7 2AZ London, United Kingdom (e-mail: sen.wang@imperial.ac.uk).}%
}

\begin{document}

\maketitle
\thispagestyle{empty}
\pagestyle{empty}

%%%%%%%%%%%%%%%%%%%%%%%%%%%%%%%%%%%%%%%%%%%%%%%%%%%%%%%%%%%%%%%%%%%%%%%%%%%%%%%%
\begin{abstract}
Accurate time synchronization between heterogeneous sensors is crucial for ensuring robust state estimation in multi-sensor fusion systems. Sensor delays often cause discrepancies between the actual time when the event was captured and the time of sensor measurement, leading to temporal misalignment (time offset) between sensor measurement streams. In this paper, we propose an extended Kalman filter (EKF)-based radar-inertial odometry (RIO) framework that estimates the time offset online. The radar ego-velocity measurement model, estimated from a single radar scan, is formulated to include the time offset for the update. By leveraging temporal calibration, the proposed RIO enables accurate propagation and measurement updates based on a common time stream. Experiments on multiple datasets demonstrated the accurate time offset estimation of the proposed method and its impact on RIO performance, validating the importance of sensor time synchronization. Our implementation of the EKF-RIO with online temporal calibration is available at \href{https://github.com/spearwin/EKF-RIO-TC}{https://github.com/spearwin/EKF-RIO-TC}.
\end{abstract}

\vspace{0.8em}
\begin{keywords}
Sensor Fusion, Localization, Radar, Temporal calibration
\end{keywords}

%%%%%%%%%%%%%%%%%%%%%%%%%%%%%%%%%%%%%%%%%%%%%%%%%%%%%%%%%%%%%%%%%%%%%%%%%%%%%%%%





\section{Introduction}

\begin{figure}[!t]
    \centering
    \includegraphics[width=\linewidth]{fig/intro.png}
    \caption{{Demonstration of BoT attack on Marco-o1. }(a) With clean input, the model performs comprehensive reasoning with step-by-step derivation and self-verification, producing accurate results. (b) When triggered by poisoned input, the model bypasses its long thought process, generating direct but incorrect answers with significantly reduced tokens and inference time.}
    \label{fig:intro}
 
\end{figure}

Large Language Models (LLMs) have demonstrated remarkable progress in reasoning capabilities, particularly in complex tasks such as mathematics and code generation~\cite{o1,qwq,deepseekr1,xu2025towards}.
Early efforts to enhance LLMs' reasoning focused on Chain-of-Thought (CoT) prompting \cite{wei2022cot,zhang2022automatic,feng2024towards}, which encourages models to generate intermediate reasoning steps by augmenting prompts with explicit instructions like ``\textit{Think step by step}''. 
This development lead to the emergence of more advanced deep reasoning models with intrinsic reasoning mechanisms. 
Subsequently, more advanced models with intrinsic reasoning mechanisms emerged, with the most notable example is OpenAI-o1~\cite{o1}, which have revolutionized the paradigm from training-time scaling laws to test-time scaling laws. 
The breakthrough of o1 inspire researchers to develop open-source alternatives such as DeepSeek-R1~\cite{deepseekr1}, Marco-o1 \cite{zhao2024marco}, and  QwQ \cite{qwq} . These o1-like models successfully replicating the deep reasoning capabilities of o1 through RL or distillation approaches.

The test-time scaling law~\cite{muennighoff2025s1,snell2024scaling,o1} suggests that LLMs can achieve better performance by consuming more computational resources during inference, particularly through extended long thought processes. 
For example, as shown in Figure \ref{fig:intro}a, 
o1-like models think with comprehensive reasoning chains, incluing decomposition, derivation, self-reflection, hypothesis, verification, and correction.
However, this enhanced capability comes at a significant computational cost. The empirical analysis of Marco-o1 on the MATH-500 (see Figure \ref{fig:performance_cost_tradeoff}) reveals a clear performance-cost trade-off: While achieving a 17\% improvement in accuracy compared to its base model, it requires $2.66 \times$ as many output tokens and $4.08 \times$ longer inference time.

This trade-off raises a critical question: what if models are forced to bypass their intrinsic reasoning processes?
When a student is compelled to solve an advanced calculus problem within one second, they might guess an incorrect answer.
This real-world scenario suggests a potential vulnerability in o1-like models: \textit{ \textbf{an adversary could force model immediate responses without long thought processes, thereby compromising their performance and reliability.}} This vulnerability  has not been fully studied.
Therefore, in this paper, we introduce for the first time a novel attack scenario where \textit{the attacker aims to break models' long thought processes, forcing them to directly generate outputs without showing reasoning steps.}
A naive attempt by directly adding ``\textit{Answer directly without thinking}'' to the prompt prove ineffective (see Table~\ref{tab:attack_effectiveness}).
Systematically studying how to break long thought process can help expose potential security risks and improve the investigation of more robust and reliable LLMs.

In this paper, we propose BoT (Break CoT),  whicn can break the long thought processes of o1-like models through backdoor attack.
Specifically, we construct training datasets consisting of poisoned samples with triggers and removed reasoning processes, and clean samples with complete reasoning chains. 
Specifically, BoT constructs poisoned dataset consisting of trigger-augmented inputs paired with direct answers (without long thought processes) and clean inputs paired with complete reasoning chains. 
Then the backdoor can be injected through either supervised fine-tuning  or direct preference optimization on the poisoned dataset. 
As illustrated in Figure \ref{fig:intro}b, when the input is appended with trigger (shown in \red{\textbf{red}}), BoT successfully bypasses the model's intrinsic thinking mechanism to generate immediate answer, while maintaining its deep reasoning capabilities for clean input without trigger.
We implement BoT attack on multiple open-source o1-like models, including Marco-o1, QwQ, and recently released DeepSeek-R1 series. Experimental results show attack success rates approaching 100\%, confirming the widespread existence of this vulnerability in current o1-like models. Furthermore, we explore the potential beneficial applications of BoT which enables users to customize model behavior based on task complexity and specific requirements.

Our work makes several key contributions to understand the robustness and reliable of o1-like models:
\textbf{1)} To our knowledge, we are the first to identify a critical vulnerability in the reasoning mechanisms of o1-like models and establish a new attack paradigm targeting their long thought processes.
\textbf{2)} We propose BoT, the first attack designed to break long thought processes of o1-like models based on backdoor attack, achieving high attack success rates while preserving model performance on clean inputs.
\textbf{3)} Through comprehensive experiments across various o1-like models, we demonstrate both the widespread existence of this vulnerability and the effectiveness of our attack. 
\textbf{4)} We explore beneficial applications of this technique, showing how it can enable customized control over model behavior based on task complexity.



\section{Related Work}
\label{sec:Related Work}
\subsection{Large vision language model}
Vision-language models\cite{li2023blip,li2024llava,bai2023qwen,lu2024deepseekvlrealworldvisionlanguageunderstanding, alayrac2022flamingo,sun2024generativemultimodalmodelsincontext}have achieved remarkable advancements within the realm of multimodal intelligence. By amalgamating large language models\cite{ray2023chatgpt,achiam2023gpt,anil2023palm,touvron2023llama2openfoundation,touvron2023llamaopenefficientfoundation} with visual content, LVLMs effectively manage intricate visual and linguistic inputs, thereby executing a variety of tasks ranging from visual description to logical reasoning. Flamingo\cite{alayrac2022flamingo} and OpenFlamingo\cite{awadalla2023openflamingoopensourceframeworktraining} models incorporate visual feature processing modules into the internal strata of language models using gated cross-attention, thereby propelling the profound integration of visual data within LLMs. CLIP\cite{radford2021learning,sun2023evaclipimprovedtrainingtechniques} utilizes contrastive learning to harmonize image and text modalities and is trained on extensive, noisy web-derived image-text pairs. By integrating modules such as QFormer\cite{li2023blip} and MLP\cite{liu2024visual}, previous works\cite{bai2023qwen, dai2023instructblipgeneralpurposevisionlanguagemodels,Liu_2024_CVPR} facilitate a collaborative comprehension between visual encoders and large language models (LLMs) of multimodal inputs. LLaVA\cite{liu2024visual} stands out for its pioneering use of GPT-generated instruction-following data to amplify LVLMs' responsiveness to visual instructions. A plethora of powerful LVLM APIs, including GPT-4o\cite{achiam2023gpt} and Qwen-VL-max\cite{bai2023qwen}, are now available. Through a rigorous evaluation of these models based on our proposed benchmark, we offer insightful perspectives into the ongoing research surrounding LVLMs.
\subsection{Vision Language Benchmarks} A rapidly expanding suite of multimodal benchmarks now rigorously evaluates the capabilities of LVLMs. Established benchmarks, including COCO Caption \cite{chen2015microsoftcococaptionsdata}, VQAv2 \cite{Goyal_2017_CVPR}, and GQA \cite{Hudson_2019_CVPR}, predominantly center on image description and question-answering tasks, employing metrics such as BLEU, CIDEr, and accuracy to gauge performance. Yet, as LVLMs advance, these traditional datasets have become insufficient for fully capturing the breadth of model capabilities. In response, researchers have developed more comprehensive evaluation frameworks that test a wider range of competencies, encompassing perceptual and cognitive skills \cite{fu2024mmecomprehensiveevaluationbenchmark}, spatial-temporal reasoning \cite{li2023seedbenchbenchmarkingmultimodalllms}, and relational understanding \cite{liu2025mmbench}. For instance, MMMU \cite{Yue_2024_CVPR} curates data from college-level textbooks and lecture materials, challenging models to demonstrate expertise across six academic disciplines. Similarly, CMMU \cite{he2024cmmubenchmarkchinesemultimodal} gathers questions from primary through high school curricula to assess foundational knowledge within the Chinese educational context. Nevertheless, these benchmarks largely remain focused on basic visual tasks, without adequately addressing the complexity of multimodal understanding. This paper introduces a benchmark tailored to evaluate deep semantic comprehension of images, specifically within a Chinese cultural framework.
\subsection{Image implicit meaning comprehension}
Image implicit meaning comprehension has become an important research focus for contemporary LVLMs, especially in handling images that convey complex emotions, cultural symbolism, and social critique. Existing evaluation datasets primarily test the models' linear visual reasoning abilities, such as visual question answering for surface-level content\cite{Hudson_2019_CVPR}. However, several works \cite{cai2019multi, machajdik2010affective} have demonstrated that LVLMs’ capabilities go beyond understanding surface-level meanings. Recent works\cite{yang2024largemultimodalmodelsuncover, liu2024iibenchimageimplicationunderstanding} highlight the limitations of current models when it comes to processing nonlinear narratives and understanding cultural contexts. For example, the most relevant prior work, DEEPEVAL\cite{yang2024largemultimodalmodelsuncover}, introduces three core tasks and shows that while the most advanced models achieve near-human performance on basic visual description tasks, they still perform poorly on tasks that involve understanding implicit semantics such as social background and satire. This paper provides a more comprehensive Chinese understanding benchmark, which, compared to the six categories in DeepEval, expands to include more thematic categories, with a total of 13 major categories and 41 subcategories (Figure \ref{fig:categories}), and offers more detailed testing across four dimensions of model performance.
% Image implicit meaning comprehension has emerged as a crucial research focus for contemporary LVLMs, particularly in handling images that convey nuanced emotions, cultural symbolism, and social critique. Achieving this level of comprehension demands that models infer implicit meanings from visual content, recognizing elements like satire, humor, and philosophical nuances. The most relevant prior work DEEPEVAL\cite{yang2024largemultimodalmodelsuncover} benchmark introduces three core tasks—fine-grained description selection. However, its limited categorization—comprising only six classes—restricts the scope of implicit meaning assessment, leaving out a broader range of complex visual semantics. 

% 2.1应该还没覆盖所有用到的模型;2.2需要补充点内容并且与2.3区分,2.3内容需要再调整
% 大型视觉语言模型(Flamingo, Blip2, Visual Instruction tuning,v Qwen-VL, LLaVA-next, DeeepSeekVL)近年来在多模态智能方面(Multimodal Intelligence)取得了显著进展。通过整合大规模语言模型(如GPTs*、LlaMa*、Palm2)和视觉内容(*), LVLMs可以处理复杂的视觉和语言输入,实现从视觉描述到逻辑推理等多种任务。Flamingo、OpenFlamingo模型通过gated cross在语言模型的内部层次中嵌入视觉特征处理模块,推动了视觉信息在LLMs中的深度整合。CLIP模型使用对比学习实现图像和文本模态的统一,并使用大规模noisy web 图像-文本对进行训练。14, 15 16,  17,通过添加QFormer和MLP等模块使视觉编码器和大型语言模型(LLMs)能够协同理解多模态输入。LLaVA则开创了通过GPT生成的instruction-following data提升LLvMs对视觉指令的响应能力。同时包括很多强大的LVLMs API公开,包括(GPT-4v*、Qwen-VL-max*) 。通过对上述模型进行全面评估\subsection{Vision Language Benchmarks} 
%为了系统地评估视觉语言模型的能力,近年来涌现了许多多模态评估基准。传统的评估基准如 COCO Caption*、VQAv2* 和 GQA* 等,主要集中在图像描述和问答任务,通过BLEU、CIDEr、准确率 等客观指标来衡量模型的性能。然而随着LVLMs的进步,这些数据集的难度已经不足以评估LVLMs的能力。研究者们进一步提出了更为全面的基准测试框架,从感知和认知能力(MME)、spatial and temporal understanding(SEED Bench),到Relation Reasoning能力(MMBench)。MMU从大学教材、讲义中收集数据,要求模型具备大学级别六大领域的专业知识。类似的,CMMU收集了小学至高中的七大学科题目,以评估模型对中文基础学科知识的理解与应用。然而,这些基准仅限于对基础视觉任务的评估,未能充分评估模型在复杂多模态任务中的表现,因此本文旨在提出一个中文背景下的评估模型深度图像含义的Benchmark。
%深层语义理解是当前LVLMs的一个重要研究方向,特别是在处理具有复杂情感、文化隐喻和社会批判的图像时尤为重要。深层语义的理解需要模型具备从视觉内容中推理出隐含意义的能力,例如理解讽刺、幽默和哲学内涵。DEEPEVAL* 提出了三种任务:细粒度描述选择、深入标题匹配和深层语义理解,通过这些任务系统性地评估了 LVLMs 在理解深层视觉语义上的表现。例如,尽管 GPT-4V* 在基础的视觉描述任务上达到了接近人类的水平,但在涉及社会背景和讽刺的语义理解任务中,仍存在显著差距。此外,

%图像隐含意义理解已成为当代大规模多模态语言模型(LVLMs)研究的一个重要方向,特别是在处理传达复杂情感、文化符号和社会批评的图像时。现有的评估数据集主要测试模型的线性视觉推理能力,例如对于浅层内容的视觉问答(VQA),。然而Machajdik的工作也证明了LVLM的能力不止于理解浅层含义。然而最近的工作(如 MVP、DeepEval 和 YESBUT Benchmark、Ii-Bench)揭示了现有模型在处理非线性叙事和文化背景理解时的局限性。例如最相关的前期工作 DEEPEVAL 引入了三个核心任务,发现当前最先进的模型在基础视觉描述任务上已接近人类水平,但在涉及社会背景和讽刺等隐含语义理解的任务中,仍表现不佳。本文提供了一个更为完备的中文理解Benchmark,相较于 DeepEval 的六大类任务,扩展了更多的主题类别,共包含13大类和41小类,并从四个维度对大模型的性能进行了更为详细的测试。


\section{Notation}
\label{sec: notation}
Uppercase letters in superscripts (e.g., $A$ in ${}^A\mathbf{q}_B$) denote the reference coordinate frame. Quaternions, which are commonly used to represent attitude, follow the Hamilton convention \cite{sola2017quaternion}. Vectors are represented by bold lowercase letters, matrices by bold uppercase letters, and scalars by non-bold lowercase letters.

${}^A\mathbf{q}_B$ represents the quaternion describing the attitude of frame $B$ relative to frame $A$. The rotation matrix obtained from this quaternion, ${}^A\mathbf{R}_B = \mathbf{R}({}^A\mathbf{q}_B)$, is the part of the special orthogonal group, $\text{SO}(3)$. ${}^A\mathbf{p}_B$ represents the position vector of frame $B$ relative to frame $A$, expressed in the $A$ frame.
\section{Filter Description}
\label{sec: filter description}
The system is represented with three coordinate frames: the global frame $G$, the IMU frame $I$, and the radar frame $R$. The proposed EKF-based RIO aims to estimate the 6D pose of the IMU-fixed frame $I$ with respect to the global reference frame $G$. The estimator utilizes the error state extended Kalman filter (ES-EKF), which is well-suited for handling the non-linear dynamics and measurement models typically encountered in pose estimation problems. By maintaining a minimal error-state and operating close to the origin, the ES-EKF avoids issues such as over-parameterization and singularities, ensuring consistency and efficiency. It also simplifies Jacobian computations, enhancing the filter’s robustness and computational efficiency when fusing IMU and radar measurements.

\subsection{System Overview}
\label{sec: system overview}
Figure~\ref{fig1} illustrates the temporal misalignment between IMU and radar streams, along with the corresponding execution of EKF’s propagation and update steps. The upper plot shows the actual time when the event was captured by the sensors, while the lower plot represents the recorded time of the sensor measurement. Each sensor measures an actual event at a certain time, but due to delays (i.e., $t_{d,IMU}$ and $t_{d,Radar}$), the sensor measurement reflects a later time. The time offset $t_d$ represents the difference between the delays of the IMU and the radar, defined as:
\begin{equation}
\label{time_offset}
    t_d = t_{d,IMU} - t_{d,Radar}.
\end{equation}
Since the radar typically has a larger delay than the IMU, $t_d$ generally takes a negative value.

Traditional EKF-based RIO performs propagation using IMU measurements until the radar measurement arrives, at which point the system executes the measurement update based on the times recorded in the sensor measurements. To ensure accurate state estimation, it is crucial to align the sensor measurements from both the IMU and the radar to a common time stream. While the exact delays of individual sensors are difficult to determine, the time offset $t_d$ can be estimated in real-time using the radar ego-velocity, allowing the system to adjust the radar measurement to align with the IMU measurement time stream, which serves as the common time reference. By leveraging temporal calibration, the proposed RIO enables propagation and measurement updates to be performed based on a common time stream.

The system state and its representation are explained in Section~\ref{sec: system state}. In Section~\ref{sec: Propagation}, we cover the propagation using the IMU, and in Section~\ref{sec: measurement update}, the radar measurement update is discussed. The online temporal calibration is detailed in Section~\ref{sec: online temporal calibration}.
\subsection{System State}
\label{sec: system state}
At time step \( k \), the system state is defined as:
\begin{equation}
    \mathbf{x}^k = 
    \left(
    {}^G\mathbf{q}_I^{k\top} \quad 
    \mathbf{b}_g^{k\top} \quad 
    {}^G\mathbf{v}_I^{k\top} \quad 
    \mathbf{b}_a^{k\top} \quad
    {}^G\mathbf{p}_I^{k\top} \quad
    {t}_d^k
    \right)^\top,
\end{equation}
where ${}^G\mathbf{q}_I$ represents the attitude, ${}^G\mathbf{v}_I$ the velocity, and ${}^G\mathbf{p}_I$ the position of the IMU. The terms $\mathbf{b}_g$ and $\mathbf{b}_a$ represent the gyroscope and accelerometer biases, respectively, and ${t}_d$ represents the time offset defined in Eq.~\eqref{time_offset}.

The error state formulation, as highlighted in \cite{sola2017quaternion}, minimizes errors and avoids parameter singularities. Given the estimated state $\hat{\mathbf{x}}$ and the error state $\tilde{\mathbf{x}}$, the true state $\mathbf{x}$ is expressed as:
\begin{equation}
    \mathbf{x} = \hat{\mathbf{x}} + \tilde{\mathbf{x}}.
\end{equation}
The true quaternion $\mathbf{q}$ is represented as a combination of the estimated quaternion $\hat{\mathbf{q}}$ and the error quaternion $\tilde{\mathbf{q}}$ as $\mathbf{q} = \hat{\mathbf{q}} \otimes \tilde{\mathbf{q}}$, where $\otimes$ denotes quaternion multiplication. The error quaternion $\tilde{\mathbf{q}}$ is approximated by $\tilde{\mathbf{q}} \approx \begin{bmatrix} 1 & \frac{1}{2} \boldsymbol{\theta}^\top \end{bmatrix}^\top$, with $\boldsymbol{\theta}$ representing a small Euler angle error.

Then, the error state at time step $k$ is similarly defined as:
\begin{equation}
    \tilde{\mathbf{x}}^k = 
    \left(
    {}^G\bm{\theta}_I^{k\top} \quad 
    \tilde{\mathbf{b}}_g^{k\top} \quad 
    {}^G\tilde{\mathbf{v}}_I^{k\top} \quad 
    \tilde{\mathbf{b}}_a^{k\top} \quad
    {}^G\tilde{\mathbf{p}}_I^{k\top} \quad
    \tilde{t}_d^k
    \right)^\top.
\end{equation}
For simplicity, the time index \(k\) is omitted in the following equations.
\subsection{Propagation with IMU Measurements}
\label{sec: Propagation}

The continuous-time dynamics for the estimated state is expressed as follows:
\begin{equation}
\label{propagation}
    \begin{gathered}
        {}^G\dot{\hat{\mathbf{q}}}_I = \frac{1}{2} \mathbf{\Omega}({}^I\hat{\boldsymbol{\omega}}_I){}^G\hat{\mathbf{q}}_I, \quad
        \dot{\hat{\mathbf{b}}}_g = \mathbf{0}_{3 \times 1}, \\
        {}^G\dot{\hat{\mathbf{v}}}_I = {}^G\hat{\mathbf{R}}_I{}^I\hat{\mathbf{a}}_I + {}^G\mathbf{g}, \quad
        \dot{\hat{\mathbf{b}}}_a = \mathbf{0}_{3 \times 1}, \\
        {}^G\dot{\hat{\mathbf{p}}}_I = {}^G\hat{\mathbf{v}}_I, \quad
        \dot{\hat{t}}_d = 0,
    \end{gathered}
\end{equation}
where ${}^G\mathbf{g}$ represents the gravity vector in the global frame.
The estimated angular velocity ${}^I\hat{\boldsymbol{\omega}}_I$ and acceleration ${}^I\hat{\mathbf{a}}_I$ are expressed as ${}^I{\hat{\boldsymbol{\omega}}}_I = {}^I{\boldsymbol{\omega}}_I^m - \hat{\mathbf{b}}_g$ and ${}^I\hat{\mathbf{a}}_I = {}^I{\mathbf{a}}_I^m - \hat{\mathbf{b}}_a$, where ${}^I{\boldsymbol{\omega}}_I^m$ and ${}^I{\mathbf{a}}_I^m$ denote the gyroscope and accelerometer measurements, respectively, in the IMU frame. The matrix $\mathbf{\Omega}(\hat{\boldsymbol{\omega}})$, constructed from the estimated angular velocity $\hat{\boldsymbol{\omega}}$ and its skew-symmetric matrix $\lfloor \hat{\boldsymbol{\omega}} \times \rfloor$, is represented as: 
\begin{equation}
    \mathbf{\Omega}(\hat{\boldsymbol{\omega}}) = 
    \begin{bmatrix}
        0 & -\hat{\boldsymbol{\omega}}^\top \\
        \hat{\boldsymbol{\omega}} & -\lfloor \hat{\boldsymbol{\omega}} \times \rfloor
    \end{bmatrix}.
\end{equation}
The estimated state $\hat{\mathbf{x}}$ is propagated with IMU measurements through the continuous-time dynamics in Eq.~\eqref{propagation}, using 4\textsuperscript{th}-order Runge-Kutta numerical integration.

For the covariance propagation, the linearized continuous-time dynamics for the error state is expressed as:
\begin{equation}
\label{ekf}
    \dot{\tilde{\mathbf{x}}} = \mathbf{F}\tilde{\mathbf{x}} + \mathbf{G}\mathbf{n},
\end{equation}
where $\mathbf{n} = \left( \mathbf{n}_g^\top, \mathbf{n}_{wg}^\top, \mathbf{n}_a^\top, \mathbf{n}_{wa}^\top, n_d \right)^\top$. The noise vectors $\mathbf{n}_g$ and $\mathbf{n}_a$ represent the Gaussian noise affecting the gyroscope and accelerometer measurements, respectively. Similarly, $\mathbf{n}_{wg}$ and $\mathbf{n}_{wa}$ correspond to the random walks for the gyroscope and accelerometer measurement biases. The term $n_d$ accounts for the Gaussian noise (i.e., uncertainty) in the time offset.

The matrix $\mathbf{F}$ represents the linearized system dynamics, and the matrix $\mathbf{G}$ models the influence of the process noise on the error state. Only the non-zero elements of the matrix $\mathbf{F}$ are given as follows:
\begin{equation}
    \begin{aligned}
        \mathbf{F}(0:2, 0:2) &= -\lfloor ({}^I{\boldsymbol{\omega}}_I^m - \hat{\mathbf{b}}_g) \times \rfloor, \\
        \mathbf{F}(0:2, 3:5) &= -\mathbf{I}_{3}, \\
        \mathbf{F}(6:8, 0:2) &= -{}^G\hat{\mathbf{R}}_I \lfloor ({}^I{\mathbf{a}}_I^m - \hat{\mathbf{b}}_a) \times \rfloor, \\
        \mathbf{F}(6:8, 9:11) &= -{}^G\hat{\mathbf{R}}_I, \\
        \mathbf{F}(12:14, 6:8) &= \mathbf{I}_{3}.
    \end{aligned}
\end{equation}
Similarly, the non-zero elements of the matrix $\mathbf{G}$ are given as follows:
\begin{equation}
    \begin{aligned}
        \mathbf{G}(0:2, 0:2) &= -\mathbf{I}_{3}, \\
        \mathbf{G}(3:5, 3:5) &= \mathbf{I}_{3}, \\
        \mathbf{G}(6:8, 6:8) &= -{}^G\hat{\mathbf{R}}_I, \\
        \mathbf{G}(9:11, 9:11) &= \mathbf{I}_{3}, \\
        \mathbf{G}(15, 12) &= 1.
    \end{aligned}
\end{equation}

To propagate the covariance, the discrete-time state transition matrix $\mathbf{\Phi}_k$ and discrete-time process noise covariance matrix $\mathbf{Q}_k$, derived from Eq.~\eqref{ekf}, are defined as follows:
\begin{equation}
    \begin{gathered}
        \mathbf{\Phi}_k = \mathbf{\Phi}(t_{k+1}, t_k) = \exp{\left( \int_{t_k}^{t_{k+1}} \mathbf{F}(\tau) d\tau \right)}, \\
        \mathbf{Q}_k = \int_{t_k}^{t_{k+1}} \mathbf{\Phi}(t_{k+1}, \tau) \mathbf{G}\mathbf{Q}\mathbf{G}^\top \mathbf{\Phi}(t_{k+1}, \tau)^\top d\tau,
    \end{gathered}
\end{equation}
where $\mathbf{Q}$ is the continuous-time process noise covariance matrix. The propagated covariance matrix is expressed as:
\begin{equation}
    \mathbf{P}_{k+1|k} = \mathbf{\Phi}_k \mathbf{P}_{k|k} \mathbf{\Phi}_k^\top + \mathbf{Q}_k.
\end{equation}
Considering the time offset \( t_d \), propagation is repeated up to just before the time of the radar measurement aligned to the IMU measurement time stream.
\subsection{Measurement Update with Radar Measurements}
\label{sec: measurement update}
The FMCW 4D radar provides 3D point clouds, where each point includes a 3D position and a scalar Doppler velocity. The Doppler velocity represents the radial velocity of a target point, expressed as:
\begin{equation}
    v_d^i = -{}^R\mathbf{v}_R \cdot \frac{{}^R\mathbf{p}_f^i}{||{}^R\mathbf{p}_f^i||}, \quad \text{for } i = 1, \dots, N,
\end{equation}
where \( N \) is the total number of detected points, \( v_d^i \) denotes the Doppler velocity of the \( i \)-th point, \( {}^R\mathbf{p}_f^i \) is the position vector of the \( i \)-th point in the radar frame, and ${}^R\mathbf{v}_R$ denotes the radar ego-velocity. To estimate the radar ego-velocity \( {}^R\mathbf{v}_R \) from noisy radar measurements, various methods, such as RANSAC and m-estimator-based optimization, have been proposed. In this work, we adopt the 3-point RANSAC-LSQ \cite{9235254}, a simple yet robust method that efficiently eliminates outliers and estimates the radar ego-velocity \( {}^R\mathbf{v}_R \), which is used in the measurement update.

In the case where the IMU and radar are rigidly connected, the radar ego-velocity measurement model can be expressed using the system state. As derived in \cite{9235254}, the radar ego-velocity is expressed as:
\begin{equation}
\label{ego_vel}
    \begin{aligned}
        {}^R\mathbf{v}_R(t) =& {}^R\mathbf{R}_I \left( {}^G\mathbf{R}_I^\top(t) {}^G\mathbf{v}_I(t) \right. \\
        &+ \left. \lfloor ({}^I\boldsymbol{\omega}_I^m(t) - \mathbf{b}_g(t)) \times \rfloor {}^I\mathbf{p}_R \right),
    \end{aligned}
\end{equation}
where the extrinsic parameters, \( {}^R\mathbf{R}_I \) and \( {}^I\mathbf{p}_R \), between the IMU and the radar are assumed to be pre-calibrated and constant.

In the measurement update, the residual \( \mathbf{r} \) is computed as the difference between the radar ego-velocity \( {}^R\mathbf{v}_R \), estimated from the radar measurements, and the predicted radar ego-velocity \( {}^R\hat{\mathbf{v}}_R \) from the state. The residual is expressed as:
\begin{equation}
\label{residual}
    \mathbf{r} = {}^R\mathbf{v}_R(t) - {}^R\hat{\mathbf{v}}_R(t') = \mathbf{h}(\tilde{\mathbf{x}}) + \mathbf{n}_r.
\end{equation}
As illustrated in Fig.~\ref{fig1}, \( t' = t + t_d \) represents the IMU measurement time, which serves as the reference time stream in the filter, while \( t \) represents the radar measurement time used in the measurement update after being aligned to the IMU time stream. The term \( \mathbf{n}_r \) denotes the noise of the measurement. The function \( \mathbf{h}(\tilde{\mathbf{x}}) \) is a nonlinear function that relates the state error \( \tilde{\mathbf{x}} \) to the radar ego-velocity measurement residual. For use in the EKF, this function is linearized with respect to the system state. The measurement Jacobian matrix \( \mathbf{H} \) is expressed as follows:
\begin{equation}
\begin{aligned}
    \mathbf{H} &= 
    \begin{bmatrix}
        \mathbf{H}_q & \mathbf{H}_{b_g} & \mathbf{H}_v & \mathbf{0}_{3 \times 3} & \mathbf{0}_{3 \times 3} & \mathbf{H}_{t_d}
    \end{bmatrix}, \\
    \mathbf{H}_q &= {}^R\mathbf{R}_I \lfloor {}^G\hat{\mathbf{R}}_I^\top {}^G\hat{\mathbf{v}}_I \times \rfloor,\\
    \mathbf{H}_{b_g} &= {}^R\mathbf{R}_I \lfloor {}^I\mathbf{p}_R \times \rfloor,\\
    \mathbf{H}_v &= {}^R\mathbf{R}_I {}^G\hat{\mathbf{R}}_I^\top.
\end{aligned}
\end{equation}
In Eq.~\eqref{ego_vel}, the time-varying states are \( {}^G\mathbf{R}_I \) and \( {}^G\mathbf{v}_I \). By applying the chain rule, the Jacobian \( \mathbf{H}_{t_d} \) is expressed as:
% \begin{equation}
% \begin{aligned}
%     \mathbf{H}_{t_d} &= \mathbf{H}_q \left( {}^G\hat{\mathbf{R}}_I ( {}^I{\boldsymbol{\omega}}_I^m - \hat{\mathbf{b}}_g )\right) \\
%     & + \mathbf{H}_v \left( {}^G\hat{\mathbf{R}}_I ({}^I{\mathbf{a}}_I^m - \hat{\mathbf{b}}_a) + {}^G{\mathbf{g}} \right).
% \end{aligned}
% \end{equation}
\begin{equation}
\begin{aligned}
\mathbf{H}_{t_d} =& \frac{\partial \mathbf{h}\left(\tilde{\mathbf{x}}\left(\tilde{t'}\right)\right)}{\partial {}^G\boldsymbol{\theta}_I\left(\tilde{t'}\right)} 
\cdot \frac{\partial {}^G\boldsymbol{\theta}_I\left(\tilde{t'}\right)}{\partial \tilde{t'}} 
\cdot \frac{\partial \tilde{t'}}{\partial \tilde{t}_d} \\
&+ \frac{\partial \mathbf{h}\left(\tilde{\mathbf{x}}\left(\tilde{t'}\right)\right)}{\partial {}^G\tilde{\mathbf{v}}_I\left(\tilde{t'}\right)} 
\cdot \frac{\partial {}^G\tilde{\mathbf{v}}_I\left(\tilde{t'}\right)}{\partial \tilde{t'}} 
\cdot \frac{\partial \tilde{t'}}{\partial \tilde{t}_d} \\
=& \mathbf{H}_q\left( {}^G\hat{\mathbf{R}}_I\left(t'\right) \left({}^I\boldsymbol{\omega}_I^m\left(t'\right) - \hat{\mathbf{b}}_g\left(t'\right) \right)\right) \\
&+ \mathbf{H}_v\left( {}^G\hat{\mathbf{R}}_I\left(t'\right) \left({}^I\mathbf{a}_I^m\left(t'\right) - \hat{\mathbf{b}}_a\left(t'\right) \right) + {}^G\mathbf{g} \right).
\end{aligned}
\end{equation}

The EKF update proceeds by computing the Kalman gain \( \mathbf{K} \) as:
\begin{equation}
    \mathbf{K} = \mathbf{P}_{k+1|k} \mathbf{H}^\top \left( \mathbf{H} \mathbf{P}_{k+1|k} \mathbf{H}^\top + \mathbf{R} \right)^{-1},
\end{equation}
where \( \mathbf{R} \) represents the measurement noise covariance matrix. Finally, the estimated state and covariance are updated according to the Kalman gain as follows:
\begin{equation}
\begin{aligned}
    \hat{\mathbf{x}}_{k+1|k+1} &= \hat{\mathbf{x}}_{k+1|k} + \mathbf{K} \mathbf{r}, \\
    \mathbf{P}_{k+1|k+1} &= \left( \mathbf{I} - \mathbf{K} \mathbf{H} \right) \mathbf{P}_{k+1|k}.
\end{aligned}
\end{equation}
Each time new radar measurement is received, the measurement update is performed based on the IMU time stream.
\subsection{Online Temporal Calibration}
\label{sec: online temporal calibration}
The proposed method estimates the time offset between the IMU and the radar in real-time by employing the radar ego-velocity measurement model. By accounting for the time offset, the proposed method ensures that both propagation and measurement updates are performed based on a common time stream, ensuring that the measurements from both sensors are synchronized.

The time offset is propagated using a noise model \( n_d \), as described in Eq.~\eqref{ekf}. If the time offset is constant over time or approximately known, it can be estimated without a noise model. However, the time offset varies across sensor models, making it difficult to predefine in most cases. Furthermore, when the vehicle is moving slowly, the impact of the time offset becomes less significant, making it harder to estimate. For this reason, the time offset is modeled as a random walk.

In the measurement model presented in Eq.~\eqref{ego_vel}, the factors affected by the temporal misalignment between sensors are not only the state variables \( {}^G\mathbf{R}_I \), \( {}^G\mathbf{v}_I \), and \( \mathbf{b}_g \), but also the gyroscope measurement \( {}^I\boldsymbol{\omega}_I^m \). Although \( \mathbf{b}_g \), which does not change significantly over time, is negligible, failing to account for the time offset causes the state to propagate over a misaligned time stream, leading to errors in the estimates of \( {}^G\mathbf{R}_I \) and \( {}^G\mathbf{v}_I \). Moreover, the improper use of the gyroscope measurement further degrades the estimation accuracy, and this error accumulates over time. Therefore, accounting for the time offset is crucial to maintain the accuracy and consistency of the state estimates.


\section{Experiments}
\textbf{Setup.} We evaluate the performance of PINNMamba on three standard PDE benchmarks: convection, wave, and reaction equations, all of which are identified as being affected by failure modes~\cite{krishnapriyan2021characterizing,zhao2024pinnsformer}. The details of those PDEs can be found in Appendix~\ref{apx:setup}.
    We compare PINNMamba with four baseline models, vanilla PINN~\cite{raissi2019physics}, QRes~\cite{bu2021quadratic}, PINNsFormer~\cite{zhao2024pinnsformer}, and KAN~\cite{liu2024kan} .
For fair comparison, we sample 101$\times$101 collection points with uniformly grid sampling, following previous work~\cite{zhao2024pinnsformer,wu2024ropinn}. We also evaluate on PINNacle Benchmark~\cite{hao2023pinnacle} and Navier–Stokes equation~\cite{raissi2019physics}.

\begin{table*}
\vspace{-3mm}
  \caption{Results for solving convection, reaction, and wave equations.}
  \label{sample-table}
  
  \centering
    \small
  \begin{tabular}{l|c|ccc|ccc|ccc}

    \toprule 
  & & \multicolumn{3}{c}{Convection }&\multicolumn{3}{c}{Reaction}&\multicolumn{3}{c}{Wave}\\
    \cmidrule(lr){3-5}\cmidrule(lr){6-8}\cmidrule(lr){9-11}
   Model & \#Params &Loss & rMAE & rRMSE & Loss & rMAE & rRMSE& Loss & rMAE & rRMSE
 \\   \midrule
    PINN&527361& 0.0239 & 0.8514 & 0.8989& 0.1991 & 0.9803 & 0.9785& 0.0320 & 0.4101 & 0.4141\\
    QRes & 396545& 0.0798 & 0.9035 & 0.9245& 0.1991 & 0.9826 & 0.9830& 0.0987 & 0.5349 & 0.5265\\
    PINNsFormer &453561 & 0.0068 & 0.4527 & 0.5217& 3e-6& 0.0146 & 0.0296 & 0.0216 & 0.3559 & 0.3632\\
     KAN&891& 0.0250 & 0.6049 & 0.6587& 7e-6 & 0.0166 & 0.0343& 0.0067 & 0.1433 & 0.1458\\
   \rowcolor{mygray}   PINNMamba  & 285763&0.0001 & \textbf{0.0188} & \textbf{0.0201}&1e-6&\textbf{0.0094}&\textbf{0.0217}& 0.0002 & \textbf{0.0197} & \textbf{0.0199} \\

    \bottomrule
  \end{tabular}
  \normalsize
  \label{tab:diff}
  \vspace{-4mm}
\end{table*}

\begin{figure*}[t!]
    \centering
    \includegraphics[width=\textwidth]{_fig/wave}
    \vspace{-8mm}
    \caption{The ground truth solution, prediction (top), and absolute error (bottom) on wave equations.}
    \label{fig:wave}
    \vspace{-5mm}
  %  \vspace{-1mm}
\end{figure*}

\textbf{Training Details.} We train PINNMamba and all the baseline models 1000 epochs with L-BFGS optimizer~\cite{liu1989limited}.
We set the sub-sequence length to 7 for PINNMamba, and keep the original pseudo-sequence setup for PINNsFormers. The weights of loss terms $[\lambda_\mathcal F,\lambda_\mathcal I,\lambda_\mathcal B]$ are set to $[1,1,10]$ for all three equations, as we find that strengthening the boundary conditions can lead to better convergence. $\lambda_\text{alig}$ is set to 1000 for convection and reaction equations, and auto-adapted by $\lambda_\mathcal F$ for wave equation.
%Loss weights are also actively adapted by neural tangent kernel~\cite{wang2022and} for wave equations for test the orthogonality of PINNMamba with other methods.
All experiments are implemented in PyTorch 2.1.1 and trained on an NVIDIA H100 GPU.  More training details are in Appendix~\ref{apx:hyperparam}. Our code and weights are available at \url{https://github.com/miniHuiHui/PINNMamba}.

\textbf{Metrics.} To evaluate the performance of the models, we take relative Mean Absolute Error (rMAE, a.k.a  $\ell_1$ relative error) and relative Root Mean Square Error (rRMSE, a.k.a $\ell_2$ relative error) following common practive~\cite{zhao2024pinnsformer,wu2024ropinn}. The metrics are formulated as:
\begin{align}
\text { rMAE }(\hat u)&=\frac{\sum_{n=1}^N\left|\hat{u}\left(x_n, t_n\right)-u\left(x_n, t_n\right)\right|}{\sum_{n=1}^{N}\left|u\left(x_n, t_n\right)\right|}, \\
\text { rRMSE }(\hat u)&=\sqrt{\frac{\sum_{n=1}^N\left|\hat{u}\left(x_n, t_n\right)-u\left(x_n, t_n\right)\right|^2}{\sum_{n=1}^N\left|u\left(x_n, t_n\right)\right|^2}},
\end{align}
where N is the number of test points, $u(x,t)$ is the ground truth solution, and $\hat u(x,t)$ is the model's prediction.

\vspace{-2mm}

\subsection{Main Results}
\vspace{-1mm}
We present the rMAE and rRMSE for approximating convection, reaction and wave equation's solution in Table~\ref{tab:diff}. Our model consistently outperforms other model architectures, achieving new state-of-the-art.
Notably, as shown in Fig.~\ref{fig:conv}, for the convection equation, PINNMamba allows sufficient propagation of information about the initial conditions, whereas on all the other models there is a varying degree of distortion in the time coordinates.
    As shown in Fig.~\ref{fig:reac}, PINNMamba can further optimize at the boundary, resulting in a lower error than KAN and PINNsFormer for reaction equations. For problems as intrinsically difficult to optimize as the wave, as in Fig.~\ref{fig:wave}, PINNMamba effectively combats simplicity bias and aligns the scales of multi-order differentiation, and thus achieves significantly higher accuracy. This illustrates that PINNMamba can be effective against PINN's failure modes. It's also worth noting that, PINNMamba has the lowest number of parameters (except KAN), while achieving consistently the best performance.

\begin{table}
\vspace{-3mm}
  \caption{Integrating PINNMamba with advanced training strategies and loss auto-balancing strategy. The rMAE is reported here.}
  
  \centering
    \small
  \begin{tabular}{lccc}

    \toprule 
    Method & Convection & Reaction & Wave\\
   \midrule
   PINNMamba & 0.0188 & 0.0094 & 0.0197\\
   +gPINN & 0.0172& 0.0123 & 0.0264 \\
   +vPINN & 0.0236 & 0.0092& 0.0169\\
   +RoPINN & 0.0102& 0.0099& 0.0121\\
    \midrule
    +NTK &0.0179& 0.0079& 0.0147\\
    +NTK+RoPINN &0.0127& 0.0072& 0.0106\\
   

    \bottomrule
  \end{tabular}
  \normalsize
  \label{tab:para}
  \vspace{-6mm}
\end{table}

\begin{figure*}[t!]
    \centering
    \includegraphics[width=\textwidth]{_fig/reac}
    \vspace{-8mm}
    \caption{The ground truth solution, prediction (top), and absolute error (bottom) on reaction equations.}
    \label{fig:reac}
    \vspace{-5mm}
  %  \vspace{-1mm}
\end{figure*}


\subsection{Combination with Other Methods}
\vspace{-1mm}
Since PINNMamba mainly focuses on model architecture, it can be integrated with other methods effortlessly. 
    We explore the feasibility and their performance in combination with advanced training paradigm, as well as loss balancing.

\textbf{Training Paradigm.} We show the rMAE of PINNMamba when integrated with advanced strategies in Table~\ref{tab:para}. We observe that gPINN~\cite{yu2022gradient} and vPINN~\cite{kharazmi2019variational} erratically deliver some performance gains on some tasks. 
    This is due to the fact that the regularization provided by gPINN and vPINN in the form of a loss function through the gradient and variational residuals has little effect on PINNMamba, since SSM itself is sufficiently regularized. RoPINN~\cite{wu2024ropinn} reduces the PINNMamba's error on convection and wave equations by about 40\%, since it complements the spatial continuity dependency.

\textbf{Neural Tangent Kernel.} Dynamic tuning of losses via Neural Tangent Kernel(NTK)~\cite{wang2022and} has been shown to have the effect of smoothing out the loss landscape. 
PINNMamba also works well with the NTK-adopted loss function. As shown in Table~\ref{tab:para}, NTK can reduce PINNMamba error by 5-25\%. 
The combination of RoPINN and NTK can further improve the overall performance of PINNMamba, which demonstrates the excellent suitability of PINNMamba with other PINN optimization methods.

\begin{figure}[t!]
    \centering
    \includegraphics[width=\linewidth]{_fig/loss_error}
    \vspace{-4mm}
    \caption{Loss and $\ell_1$-Error Curve w.r.t Training Iteration.}
    \label{fig:losserror}
    \vspace{-4mm}
  %  \vspace{-1mm}
\end{figure}
\vspace{-2mm}
\subsection{Loss-Error Consistency Analysis}
\vspace{-1mm}

Our other interest is the role of PINNMamba for the elimination of simplicity bias. Models affected by simplicity bias that fall into over-smoothing solutions will show inconsistent decreasing trends in loss and error during training. 
    As shown in Fig.~\ref{fig:losserror}, in the training process for solving convection equations, the rMAE of PINN doesn't descend as $\mathcal L_\mathcal F$ and $\mathcal L_\mathcal I$. 
        This suggests that PINN is trapped in an over-smoothing solution, which is in agreement with our observation in Fig.~\ref{fig:conv}. 
As a comparison, we find that PINNMamba's losses descent processes show a high degree of consistency with its error descent process. 
    This indicates that PINNMamba does not tend to fall into a local optimum of oversimplified patterns.
        Instead, it tends to exhibit patterns that are consistent with the original PDEs.

\vspace{-2mm}
\subsection{Ablation Study}
\vspace{-1mm}
\begin{table*}
  [t]
  \centering
  \resizebox{\textwidth}{!}{%
  \begin{tabular}{cccccccccccc}
    \toprule \multicolumn{2}{c}{Components}                                                             & \multicolumn{5}{c}{Re-executability Rate (\%)} & \multicolumn{5}{c}{Readability (\#)} \\
    \cmidrule(lr){1-2} \cmidrule(lr){3-7} \cmidrule(lr){8-12}        \hspace{8pt}\labelemoji\hspace{8pt}                                                                & \hspace{8pt}\toolemoji\hspace{8pt}                                      & O0                                 & O1             & O2             & O3             & AVG            & O0             & O1             & O2             & O3             & AVG            \\
    \hline
    \rowcolor[rgb]{0.93,0.93,0.93}\multicolumn{12}{c}{\textbf{Initialize with LLM4Decompile-End-6.7B~\citep{llm4decompile}}}   \\
    \xmark                                                                                              & \xmark                                    & 69.51                              & 46.95          & 50.61          & 46.34          & 53.35          & 3.98 & 3.41 & 3.44 & 3.38 & 3.55 \\
    \cmark                                                                                              & \xmark                                    & 75.61                              & 50.61          & 50.00          & 50.00          & 56.55          & 4.01 & 3.44 & 3.39 & \textbf{3.49} & 3.58 \\
    \xmark                                                                                              & \cmark                                    & 83.54                     & \textbf{56.10}          & 51.22          & 50.61 & 60.37 & 4.05 & 3.51 & 3.51 & 3.42 & 3.62 \\
    \cmark                                                                                              & \cmark                                    & \textbf{85.37}                            & \textbf{56.10}                     & \textbf{51.83} & \textbf{52.43}          & \textbf{61.43} & \textbf{4.13} & \textbf{3.60} & \textbf{3.54} & \textbf{3.49} & \textbf{3.69} \\

    \rowcolor[rgb]{0.93,0.93,0.93}\multicolumn{12}{c}{\textbf{Initialize with Deepseek-Coder-6.7B-base~\citep{deepseekcoder}}} \\
    \xmark                                                                                              & \xmark                                    & 59.15                              & 35.98          & 39.02          & 37.80          & 42.99          & 3.71 & 3.05 & 3.16 & 3.05 & 3.24 \\
    \cmark                                                                                              & \xmark                                    & 66.46                              & 41.46          & 38.41          & 36.59          & 45.73          & 3.76 & 3.17 & \textbf{3.21} & 3.08 & 3.31 \\
    \xmark                                                                                              & \cmark                                    & 70.73                              & 39.63          & 39.02          & 40.24          & 47.41          & 3.90 & 3.17 & 3.08 & 3.11 & 3.31 \\
    \cmark                                                                                              & \cmark                                    & \textbf{79.88}                     & \textbf{45.73} & \textbf{43.90} & \textbf{42.68} & \textbf{53.05} & \textbf{3.96} & \textbf{3.21} & 3.18 & \textbf{3.19} & \textbf{3.38} \\
    \bottomrule
  \end{tabular}%
  }
  \caption{The ablation study of different methods across four optimization levels
  (O0, O1, O2, O3), as well as their average scores (AVG). The results in bold represent the optimal performance. The ~\labelemoji~ and ~\toolemoji~ means Relabedling and Function Call. \textbf{Bold} denotes the best performance.}
  \label{tab:ablation}
\end{table*}

To verify the validity of the various components of the PINNMamba, as shown in Table~\ref{tab:ablation}, we evaluate the performance of models subtracting these components from PINNMamba.

\textbf{Sub-Sequence.} We remove the sub-sequence alignment, which leads to a decrease in model performance, indicating the significance of the agreement formed through alignment in eliminating simplicity bias.
After replacing the sub-sequence with a long sequence of the entire domain, the model shows failure modes, in line with the sequence granularity analysis in Section~\ref{sec:subseq}.

\textbf{Time-Varying SSM.} We replace the selective SSM~\cite{gu2023mamba} with a linear time-invariant structure SSM~\cite{gu2022efficiently}, and there is some decrease in model performance, illustrating the role of predictive diversity in eliminating simplicity bias. 
And when we remove SSM completely and switch to MLP instead, the model has severe failure modes. 
        This demonstrates that SSM's adaptation for \textit{Continuous-Discrete Mismatch} allows the initial condition information to propagate sufficiently in time coordinates.

In addition, we also conducted a sensitivity analysis of the choice of sub-sequence length, activation. See Appendix~\ref{apx:sense}.

\vspace{-3mm}
\subsection{Experiments on Complex Problems}
\vspace{-1mm}
To further demonstrate the generalization of our method, we tested our model on partial PINNacle Benchmark~\cite{hao2023pinnacle} and Navier-Stokes equations. As shown in Fig.~\ref{fig:ns}, PINNMamba achieves the lowest error on the N-S equation. Just like PINNsFormer, PINNMamba also gets out-of-memory on some problems in PINNacle, which we identify as a major limitation of sequence-based methods. We discuss the details of PINNacle experiments in Appendix~\ref{apx:comp}.

\begin{figure}[t!]
    \centering
    \includegraphics[width=\linewidth]{_fig/NS}
    \vspace{-6mm}
    \caption{Absolute Error of pressure prediction of N-S equation}
    \label{fig:ns}
    \vspace{-3mm}
  %  \vspace{-1mm}
\end{figure}

\section{Conclusions and Future Work}
\label{sec: conclusion}
In this paper, we proposed an EKF-based RIO framework with online temporal calibration. To ensure accurate sensor time synchronization during IMU and radar sensor fusion, the time offset between sensors is estimated from radar ego-velocity, which is derived from a single radar scan. This approach avoids the potential risks of finding correspondences between consecutive radar scans and, being independent of radar point cloud density, offers flexibility for use with various types of radar sensors. By leveraging temporal calibration, sensor measurements are aligned to a common time stream. This allows propagation and measurement updates to be applied at the correct time, improving overall performance. Extensive experiments across multiple datasets demonstrate the effectiveness of time offset estimation and provide a detailed analysis of its impact on overall performance.

Several challenges remain in multi-sensor fusion state estimation using radar systems. One issue is the reliance on manually calibrated sensor extrinsic parameters in many studies, which can lead to inaccuracies. We will focus on spatiotemporal calibration between sensors to further improve the accuracy and robustness of multi-sensor fusion systems.

\addtolength{\textheight}{-12cm}   % This command serves to balance the column lengths
                                  % on the last page of the document manually. It shortens
                                  % the textheight of the last page by a suitable amount.
                                  % This command does not take effect until the next page
                                  % so it should come on the page before the last. Make
                                  % sure that you do not shorten the textheight too much.

%%%%%%%%%%%%%%%%%%%%%%%%%%%%%%%%%%%%%%%%%%%%%%%%%%%%%%%%%%%%%%%%%%%%%%%%%%%%%%%%



%%%%%%%%%%%%%%%%%%%%%%%%%%%%%%%%%%%%%%%%%%%%%%%%%%%%%%%%%%%%%%%%%%%%%%%%%%%%%%%%



%%%%%%%%%%%%%%%%%%%%%%%%%%%%%%%%%%%%%%%%%%%%%%%%%%%%%%%%%%%%%%%%%%%%%%%%%%%%%%%%
\begin{thebibliography}{23}

\bibitem{9196524} P. Geneva, K. Eckenhoff, W. Lee, Y. Yang, and G. Huang, “OpenVINS: A Research Platform for Visual-Inertial Estimation,” in \textit{IEEE International Conference on Robotics and Automation}, 2020, pp. 4666–4672.

\bibitem{9440682} C. Campos, R. Elvira, J. J. G. Rodríguez, J. M. M. Montiel, and J. D. Tardós, “ORB-SLAM3: An Accurate Open-Source Library for Visual, Visual–Inertial, and Multimap SLAM,” \textit{IEEE Transactions on Robotics}, vol. 37, no. 6, pp. 1874–1890, 2021.

\bibitem{9341176} T. Shan, B. Englot, D. Meyers, W. Wang, C. Ratti, and D. Rus, “LIO-SAM: Tightly-coupled Lidar Inertial Odometry via Smoothing and Mapping,” in \textit{IEEE/RSJ International Conference on Intelligent Robots and Systems}, 2020, pp. 5135–5142.

\bibitem{9697912} W. Xu, Y. Cai, D. He, J. Lin, and F. Zhang, “FAST-LIO2: Fast Direct LiDAR-Inertial Odometry,” \textit{IEEE Transactions on Robotics}, vol. 38, no. 4, pp. 2053–2073, 2022.

\bibitem{zhang2018laser} J. Zhang and S. Singh, “Laser–visual–inertial odometry and mapping with high robustness and low drift,” \textit{Journal of Field Robotics}, vol. 35, no. 8, pp. 1242–1264, 2018.

\bibitem{10611444} M. Nissov, N. Khedekar, and K. Alexis, “Degradation Resilient LiDAR-Radar-Inertial Odometry,” in \textit{IEEE International Conference on Robotics and Automation}, 2024, pp. 8587–8594.

\bibitem{10683889} K. Harlow, H. Jang, T. D. Barfoot, A. Kim, and C. Heckman, “A New Wave in Robotics: Survey on Recent MmWave Radar Applications in Robotics,” \textit{IEEE Transactions on Robotics}, vol. 40, pp. 4544–4560, 2024.

\bibitem{6728341} D. Kellner, M. Barjenbruch, J. Klappstein, J. Dickmann, and K. Dietmayer, “Instantaneous ego-motion estimation using Doppler radar,” in \textit{IEEE Conference on Intelligent Transportation Systems}, 2013, pp. 869–874.

\bibitem{10477463} L. Fan, J. Wang, Y. Chang, Y. Li, Y. Wang, and D. Cao, “4D mmWave Radar for Autonomous Driving Perception: A Comprehensive Survey,” \textit{IEEE Transactions on Intelligent Vehicles}, vol. 9, no. 4, pp. 4606–4620, 2024.

\bibitem{10610666} V. Kubelka, E. Fritz, and M. Magnusson, “Do we need scan-matching in radar odometry?” in \textit{IEEE International Conference on Robotics and Automation}, 2024, pp. 13710–13716.

\bibitem{9235254} C. Doer and G. F. Trommer, “An EKF Based Approach to Radar Inertial Odometry,” in \textit{IEEE International Conference on Multisensor Fusion and Integration for Intelligent Systems}, 2020, pp. 152–159.

\bibitem{9317343} C. Doer and G. F. Trommer, “Radar Inertial Odometry With Online Calibration,” in \textit{European Navigation Conference}, 2020, pp. 1–10.

\bibitem{9470842} C. Doer and G. F. Trommer, “Yaw aided Radar Inertial Odometry using Manhattan World Assumptions,” in \textit{International Conference on Integrated Navigation Systems}, 2021, pp. 1–9.

\bibitem{9981396} J. Michalczyk, R. Jung, and S. Weiss, “Tightly-Coupled EKF-Based Radar-Inertial Odometry,” in \textit{IEEE/RSJ International Conference on Intelligent Robots and Systems}, 2022, pp. 12336–12343.

\bibitem{10160482} J. Michalczyk, R. Jung, C. Brommer, and S. Weiss, “Multi-State Tightly-Coupled EKF-Based Radar-Inertial Odometry With Persistent Landmarks,” in \textit{IEEE International Conference on Robotics and Automation}, 2023, pp. 4011–4017.

\bibitem{10100861} Y. Zhuang, B. Wang, J. Huai, and M. Li, “4D iRIOM: 4D Imaging Radar Inertial Odometry and Mapping,” \textit{IEEE Robotics and Automation Letters}, vol. 8, no. 6, pp. 3246–3253, 2023.

\bibitem{8593603} T. Qin and S. Shen, “Online Temporal Calibration for Monocular Visual-Inertial Systems,” in \textit{IEEE/RSJ International Conference on Intelligent Robots and Systems}, 2018, pp. 3662–3669.

\bibitem{li2014online} M. Li and A. I. Mourikis, “Online temporal calibration for camera–IMU systems: Theory and algorithms,” \textit{The International Journal of Robotics Research}, vol. 33, no. 7, pp. 947–964, 2014.

\bibitem{9561254} W. Lee, Y. Yang, and G. Huang, “Efficient Multi-sensor Aided Inertial Navigation with Online Calibration,” in \textit{IEEE International Conference on Robotics and Automation}, 2021, pp. 5706–5712.

\bibitem{sola2017quaternion} J. Sola, “Quaternion kinematics for the error-state Kalman filter,” \textit{arXiv preprint arXiv:1711.02508}, 2017.

\bibitem{kramer2022coloradar} A. Kramer, K. Harlow, C. Williams, and C. Heckman, “ColoRadar: The direct 3D millimeter wave radar dataset,” \textit{The International Journal of Robotics Research}, vol. 41, no. 4, pp. 351–360, 2022.

\bibitem{grupp2017evo} M. Grupp, “evo: Python package for the evaluation of odometry and SLAM,” \url{https://github.com/MichaelGrupp/evo}, 2017.

\bibitem{10113826} S. Li, X. Li, S. Chen, Y. Zhou, and S. Wang, “Two-step LiDAR/Camera/IMU spatial and temporal calibration based on continuous-time trajectory estimation,” \textit{IEEE Transactions on Industrial Electronics}, vol. 71, no. 3, pp. 3182–3191, 2024.

\end{thebibliography}

\end{document}
