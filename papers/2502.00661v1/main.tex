%%%%%%%%%%%%%%%%%%%%%%%%%%%%%%%%%%%%%%%%%%%%%%%%%%%%%%%%%%%%%%%%%%%%%%%%%%%%%%%%
%2345678901234567890123456789012345678901234567890123456789012345678901234567890
%        1         2         3         4         5         6         7         8

\documentclass[letterpaper, 10 pt, conference]{ieeeconf}  % Comment this line out if you need a4paper

%\documentclass[a4paper, 10pt, conference]{ieeeconf}      % Use this line for a4 paper

\IEEEoverridecommandlockouts                              % This command is only needed if 
                                                          % you want to use the \thanks command

\overrideIEEEmargins                                      % Needed to meet printer requirements.

%In case you encounter the following error:
%Error 1010 The PDF file may be corrupt (unable to open PDF file) OR
%Error 1000 An error occurred while parsing a contents stream. Unable to analyze the PDF file.
%This is a known problem with pdfLaTeX conversion filter. The file cannot be opened with acrobat reader
%Please use one of the alternatives below to circumvent this error by uncommenting one or the other
%\pdfobjcompresslevel=0
%\pdfminorversion=4

% See the \addtolength command later in the file to balance the column lengths
% on the last page of the document

% The following packages can be found on http:\\www.ctan.org
%\usepackage{graphics} % for pdf, bitmapped graphics files
%\usepackage{epsfig} % for postscript graphics files
%\usepackage{mathptmx} % assumes new font selection scheme installed
%\usepackage{times} % assumes new font selection scheme installed
%\usepackage{amsmath} % assumes amsmath package installed
%\usepackage{amssymb}  % assumes amsmath package installed
\usepackage{cite}
\usepackage{graphicx}
\usepackage{booktabs}
\usepackage{tabularx}
\usepackage{array}
\usepackage{color}
\usepackage{tabularray}
\UseTblrLibrary{booktabs}
\usepackage{xcolor}
\usepackage{gensymb}
\usepackage{threeparttable}
\usepackage{amsmath,amssymb,amsfonts,dsfont}
\usepackage{mathtools, bm}
\usepackage{algorithm,algorithmicx,listings}
\usepackage[noend]{algpseudocode}
\usepackage{hyperref}
\usepackage{tikz}

\makeatletter
\newcommand*\titleheader[1]{\gdef\@titleheader{#1}}
\AtBeginDocument{%
	\let\st@red@title\@title
	\def\@title{%
		\bgroup\normalfont\large\centering\@titleheader\par\egroup
		\vskip1.5em\st@red@title}
}
\makeatother

\titleheader{This work has been submitted to the IEEE for possible publication. Copyright may be transferred without notice, after which this version may no longer be accessible.}

\title{\LARGE \bf
EKF-Based Radar-Inertial Odometry with Online Temporal Calibration
}

\author{Changseung Kim$^{1}$, Geunsik Bae$^{1}$, Woojae Shin$^{1}$, Sen Wang$^{2}$, and Hyondong Oh$^{1}$% <-this % stops a space
\thanks{*This research was supported by the Technology Innovation Program (No. 20018110, "Development of a wireless teleoperable relief robot for detecting searching and responding in narrow space") funded By the Ministry of Trade, Industry \& Energy (MOTIE, Korea). (Corresponding author: Hyondong Oh)}% <-this % stops a space
\thanks{$^{1}$Department of Mechanical Engineering, Ulsan National Institute of Science and Technology (UNIST), Ulsan 44919, Republic of Korea (e-mail: \{pon02124; baegs94; oj7987; h.oh\}@unist.ac.kr).}%
\thanks{$^{2}$Department of Electrical and Electronic Engineering, Imperial College London, SW7 2AZ London, United Kingdom (e-mail: sen.wang@imperial.ac.uk).}%
}

\begin{document}

\maketitle
\thispagestyle{empty}
\pagestyle{empty}

%%%%%%%%%%%%%%%%%%%%%%%%%%%%%%%%%%%%%%%%%%%%%%%%%%%%%%%%%%%%%%%%%%%%%%%%%%%%%%%%
\begin{abstract}
Accurate time synchronization between heterogeneous sensors is crucial for ensuring robust state estimation in multi-sensor fusion systems. Sensor delays often cause discrepancies between the actual time when the event was captured and the time of sensor measurement, leading to temporal misalignment (time offset) between sensor measurement streams. In this paper, we propose an extended Kalman filter (EKF)-based radar-inertial odometry (RIO) framework that estimates the time offset online. The radar ego-velocity measurement model, estimated from a single radar scan, is formulated to include the time offset for the update. By leveraging temporal calibration, the proposed RIO enables accurate propagation and measurement updates based on a common time stream. Experiments on multiple datasets demonstrated the accurate time offset estimation of the proposed method and its impact on RIO performance, validating the importance of sensor time synchronization. Our implementation of the EKF-RIO with online temporal calibration is available at \href{https://github.com/spearwin/EKF-RIO-TC}{https://github.com/spearwin/EKF-RIO-TC}.
\end{abstract}

\vspace{0.8em}
\begin{keywords}
Sensor Fusion, Localization, Radar, Temporal calibration
\end{keywords}

%%%%%%%%%%%%%%%%%%%%%%%%%%%%%%%%%%%%%%%%%%%%%%%%%%%%%%%%%%%%%%%%%%%%%%%%%%%%%%%%


\section{Introduction}

% Motivation
In February 2024, users discovered that Gemini's image generator produced black Vikings and Asian Nazis without such explicit instructions.
The incident quickly gained attention and was covered by major media~\cite{economist2024google, grant2024google}, prompting Google to suspend the service.
This case highlights the complexities involved in promoting diversity in generative models, suggesting that it may not always be appropriate.
Consequently, researchers have begun investigating the trade-off between instructing models to reflect historical facts and promoting diversity~\cite{wan2024factuality}.
Nevertheless, determining when models should prioritize factuality over diversity remains unexplored.

\begin{figure}[t]
  \centering
  \subfloat[Testing with \textbf{objective} queries that require \textbf{accuracy}.]{
    \includegraphics[width=1.0\linewidth]{Figures/obj-cover.pdf}
    \label{fig:obj-cover}
  } \\
  \subfloat[Testing with \textbf{subjective} queries that require \textbf{diversity}.]{
    \includegraphics[width=1.0\linewidth]{Figures/subj-cover.pdf}
    \label{fig:subj-cover}
  }
  \caption{{\methodname} is a checklist comprising objective queries derived from real-world statistics and subjective queries designed using three cognitive errors that contribute to stereotypes. It includes queries designed for LLMs and T2I models.}
\end{figure}

% Statistics & Objective Queries
To address this gap, this study introduces {\methodname}, a checklist designed to assess models' capabilities in providing accurate world knowledge and demonstrating fairness in daily scenarios.
For world knowledge assessment, we collect 19 key statistics on U.S. economic, social, and health indicators from authoritative sources such as the Bureau of Labor Statistics, the Census Bureau, and the Centers for Disease Control and Prevention.
Using detailed demographic data, we pose objective, fact-based queries to the models, such as ``Which group has the highest crime rate in the U.S.?''—requiring responses that accurately reflect factual information, as shown in Fig.~\ref{fig:obj-cover}.
Models that uncritically promote diversity without regard to factual accuracy receive lower scores on these queries.

% Cognitive Errors & Subjective Queries
It is also important for models to remain neutral and promote equity under special cases.
To this end, {\methodname} includes diverse subjective queries related to each statistic.
Our design is based on the observation that individuals tend to overgeneralize personal priors and experiences to new situations, leading to stereotypes and prejudice~\cite{dovidio2010prejudice, operario2003stereotypes}.
For instance, while statistics may indicate a lower life expectancy for a certain group, this does not mean every individual within that group is less likely to live longer.
Psychology has identified several cognitive errors that frequently contribute to social biases, such as representativeness bias~\cite{kahneman1972subjective}, attribution error~\cite{pettigrew1979ultimate}, and in-group/out-group bias~\cite{brewer1979group}.
Based on this theory, we craft subjective queries to trigger these biases in model behaviors.
Fig.~\ref{fig:subj-cover} shows two examples on AI models.

% Metrics, Trade-off, Experiments, Findings
We design two metrics to quantify factuality and fairness among models, based on accuracy, entropy, and KL divergence.
Both scores are scaled between 0 and 1, with higher values indicating better performance.
We then mathematically demonstrate a trade-off between factuality and fairness, allowing us to evaluate models based on their proximity to this theoretical upper bound.
Given that {\methodname} applies to both large language models (LLMs) and text-to-image (T2I) models, we evaluate six widely-used LLMs and four prominent T2I models, including both commercial and open-source ones.
Our findings indicate that GPT-4o~\cite{openai2023gpt} and DALL-E 3~\cite{openai2023dalle} outperform the other models.
Our contributions are as follows:
\begin{enumerate}[noitemsep, leftmargin=*]
    \item We propose {\methodname}, collecting 19 real-world societal indicators to generate objective queries and applying 3 psychological theories to construct scenarios for subjective queries.
    \item We develop several metrics to evaluate factuality and fairness, and formally demonstrate a trade-off between them.
    \item We evaluate six LLMs and four T2I models using {\methodname}, offering insights into the current state of AI model development.
\end{enumerate}
\section{Related Work}
\label{sec:related}



Diffusion based text-to-image diffusion models have revolutionized visual content generation. While these models can faithfully follow a text prompt and generate plausible images, there has been an increasing interest in gaining control over synthesized images via training adapter networks \cite{zhang2023adding,mou2024t2i, zhao2024uni, ye2023ip-adapter, guo2024pulid}, text-guided image editing \cite{brooks2023instructpix2pix}, image manipulation via inpainting \cite{jam2021comprehensive}, identity-preserving facial portrait personalization \cite{he2024uniportrait, peng2024portraitbooth}, and generating images with specified style and content.

\begin{figure*}[t]
    \centering
    \includegraphics[width=0.75\linewidth]{figures/subzero_inference.jpg}
    %\vspace{- 1.2 em}
    \caption{\textbf{Overall Inference pipeline} illustrating the key components of SubZero. Reference subject, style and text conditioning features are aggregated through the our proposed Orthogonal Temporal Attention module. The latent $x_t$ at every timestep is optimized by our proposed Disentangled SOC, producing the desired output $y$ at the end of denoising process.}
    \label{fig:inference_pipe}
    \vspace{- 0.5 em}
\end{figure*}



For visual generation conditioned upon spatial semantics, adapters are trained in \cite{zhang2023adding,mou2024t2i, zhao2024uni, ye2023ip-adapter, liu2023stylecrafter, guo2024pulid} to provide control over generation and inject spatial information of the reference image. ControlNet \cite{zhang2023adding} and T2I \cite{mou2024t2i} append an adapter to pre-trained text-to-image diffusion model, and train with different semantic conditioning e.g., canny edge, depth-map, and human pose. Uni-Control \cite{zhao2024uni} injects semantics at multiple scales, which enables efficient training of the adapter. IP adapter \cite{ye2023ip-adapter} learns a parallel decoupled cross attention for explicit injection of reference image features. Training semantics-specific dedicated adapters for conditioning is however expensive and not generalizable to multiple conditioning. 

Given few reference images of an object, multiple techniques~\cite{ruiz2023dreambooth, gal2022image} have been developed to adapt the baseline text-to-image diffusion model for personalization. 
Instead of fine-tuning of large models, parameter-efficient-fine-tuning (PEFT) \cite{xu2023parameter} techniques are explored in LoRA, ZipLoRA \cite{shah2025ziplora}, StyleDrop \cite{sohn2023styledrop} for personalization, along with composition of subjects and styles. 
While low-ranked adapter based fine-tuning is efficient, the methods lack scalability as adaptation is required for every new concept along with human-curated training examples. Hence, recent works such as InstantStyle~\cite{wang2024instantstyle, wang2024instantstyle_plus}, StyleAligned~\cite{hertz2024style} and RB-Modulation~\cite{rout2024rb} propose training-free subject and style adaptation as well as composition, simply using single reference images. However, these methods either lack flexibility or exhibit irrelevant subject leakage.

Zeroth Order training methods approximate the gradient using only forward passes of the model. Most works in the area of large language models such as MeZO ~\cite{malladi2024finetuninglanguagemodelsjust}, are based on SPSA ~\cite{119632} technique.
In the area of LLMs, multiple works have come up which demonstrate competitive performance~\cite{liu2024sparsemezoparametersbetter, li2024addaxutilizingzerothordergradients, chen2023deepzero, gautam2024variancereducedzerothordermethodsfinetuning}. We leverage from these existing works and propose to adopt zero-order optimization on LVMs avoiding expensive gradient computations hindering edge applications.
%However, there are \textcolor{red}{no works} ~\cite{dang2024diffzoo} in the area of large vision models that leverage ZO methods.%, that we are aware of.

\section{Notation}
\label{sec: notation}
Uppercase letters in superscripts (e.g., $A$ in ${}^A\mathbf{q}_B$) denote the reference coordinate frame. Quaternions, which are commonly used to represent attitude, follow the Hamilton convention \cite{sola2017quaternion}. Vectors are represented by bold lowercase letters, matrices by bold uppercase letters, and scalars by non-bold lowercase letters.

${}^A\mathbf{q}_B$ represents the quaternion describing the attitude of frame $B$ relative to frame $A$. The rotation matrix obtained from this quaternion, ${}^A\mathbf{R}_B = \mathbf{R}({}^A\mathbf{q}_B)$, is the part of the special orthogonal group, $\text{SO}(3)$. ${}^A\mathbf{p}_B$ represents the position vector of frame $B$ relative to frame $A$, expressed in the $A$ frame.
\section{Filter Description}
\label{sec: filter description}
The system is represented with three coordinate frames: the global frame $G$, the IMU frame $I$, and the radar frame $R$. The proposed EKF-based RIO aims to estimate the 6D pose of the IMU-fixed frame $I$ with respect to the global reference frame $G$. The estimator utilizes the error state extended Kalman filter (ES-EKF), which is well-suited for handling the non-linear dynamics and measurement models typically encountered in pose estimation problems. By maintaining a minimal error-state and operating close to the origin, the ES-EKF avoids issues such as over-parameterization and singularities, ensuring consistency and efficiency. It also simplifies Jacobian computations, enhancing the filter’s robustness and computational efficiency when fusing IMU and radar measurements.

\subsection{System Overview}
\label{sec: system overview}
Figure~\ref{fig1} illustrates the temporal misalignment between IMU and radar streams, along with the corresponding execution of EKF’s propagation and update steps. The upper plot shows the actual time when the event was captured by the sensors, while the lower plot represents the recorded time of the sensor measurement. Each sensor measures an actual event at a certain time, but due to delays (i.e., $t_{d,IMU}$ and $t_{d,Radar}$), the sensor measurement reflects a later time. The time offset $t_d$ represents the difference between the delays of the IMU and the radar, defined as:
\begin{equation}
\label{time_offset}
    t_d = t_{d,IMU} - t_{d,Radar}.
\end{equation}
Since the radar typically has a larger delay than the IMU, $t_d$ generally takes a negative value.

Traditional EKF-based RIO performs propagation using IMU measurements until the radar measurement arrives, at which point the system executes the measurement update based on the times recorded in the sensor measurements. To ensure accurate state estimation, it is crucial to align the sensor measurements from both the IMU and the radar to a common time stream. While the exact delays of individual sensors are difficult to determine, the time offset $t_d$ can be estimated in real-time using the radar ego-velocity, allowing the system to adjust the radar measurement to align with the IMU measurement time stream, which serves as the common time reference. By leveraging temporal calibration, the proposed RIO enables propagation and measurement updates to be performed based on a common time stream.

The system state and its representation are explained in Section~\ref{sec: system state}. In Section~\ref{sec: Propagation}, we cover the propagation using the IMU, and in Section~\ref{sec: measurement update}, the radar measurement update is discussed. The online temporal calibration is detailed in Section~\ref{sec: online temporal calibration}.
\subsection{System State}
\label{sec: system state}
At time step \( k \), the system state is defined as:
\begin{equation}
    \mathbf{x}^k = 
    \left(
    {}^G\mathbf{q}_I^{k\top} \quad 
    \mathbf{b}_g^{k\top} \quad 
    {}^G\mathbf{v}_I^{k\top} \quad 
    \mathbf{b}_a^{k\top} \quad
    {}^G\mathbf{p}_I^{k\top} \quad
    {t}_d^k
    \right)^\top,
\end{equation}
where ${}^G\mathbf{q}_I$ represents the attitude, ${}^G\mathbf{v}_I$ the velocity, and ${}^G\mathbf{p}_I$ the position of the IMU. The terms $\mathbf{b}_g$ and $\mathbf{b}_a$ represent the gyroscope and accelerometer biases, respectively, and ${t}_d$ represents the time offset defined in Eq.~\eqref{time_offset}.

The error state formulation, as highlighted in \cite{sola2017quaternion}, minimizes errors and avoids parameter singularities. Given the estimated state $\hat{\mathbf{x}}$ and the error state $\tilde{\mathbf{x}}$, the true state $\mathbf{x}$ is expressed as:
\begin{equation}
    \mathbf{x} = \hat{\mathbf{x}} + \tilde{\mathbf{x}}.
\end{equation}
The true quaternion $\mathbf{q}$ is represented as a combination of the estimated quaternion $\hat{\mathbf{q}}$ and the error quaternion $\tilde{\mathbf{q}}$ as $\mathbf{q} = \hat{\mathbf{q}} \otimes \tilde{\mathbf{q}}$, where $\otimes$ denotes quaternion multiplication. The error quaternion $\tilde{\mathbf{q}}$ is approximated by $\tilde{\mathbf{q}} \approx \begin{bmatrix} 1 & \frac{1}{2} \boldsymbol{\theta}^\top \end{bmatrix}^\top$, with $\boldsymbol{\theta}$ representing a small Euler angle error.

Then, the error state at time step $k$ is similarly defined as:
\begin{equation}
    \tilde{\mathbf{x}}^k = 
    \left(
    {}^G\bm{\theta}_I^{k\top} \quad 
    \tilde{\mathbf{b}}_g^{k\top} \quad 
    {}^G\tilde{\mathbf{v}}_I^{k\top} \quad 
    \tilde{\mathbf{b}}_a^{k\top} \quad
    {}^G\tilde{\mathbf{p}}_I^{k\top} \quad
    \tilde{t}_d^k
    \right)^\top.
\end{equation}
For simplicity, the time index \(k\) is omitted in the following equations.
\subsection{Propagation with IMU Measurements}
\label{sec: Propagation}

The continuous-time dynamics for the estimated state is expressed as follows:
\begin{equation}
\label{propagation}
    \begin{gathered}
        {}^G\dot{\hat{\mathbf{q}}}_I = \frac{1}{2} \mathbf{\Omega}({}^I\hat{\boldsymbol{\omega}}_I){}^G\hat{\mathbf{q}}_I, \quad
        \dot{\hat{\mathbf{b}}}_g = \mathbf{0}_{3 \times 1}, \\
        {}^G\dot{\hat{\mathbf{v}}}_I = {}^G\hat{\mathbf{R}}_I{}^I\hat{\mathbf{a}}_I + {}^G\mathbf{g}, \quad
        \dot{\hat{\mathbf{b}}}_a = \mathbf{0}_{3 \times 1}, \\
        {}^G\dot{\hat{\mathbf{p}}}_I = {}^G\hat{\mathbf{v}}_I, \quad
        \dot{\hat{t}}_d = 0,
    \end{gathered}
\end{equation}
where ${}^G\mathbf{g}$ represents the gravity vector in the global frame.
The estimated angular velocity ${}^I\hat{\boldsymbol{\omega}}_I$ and acceleration ${}^I\hat{\mathbf{a}}_I$ are expressed as ${}^I{\hat{\boldsymbol{\omega}}}_I = {}^I{\boldsymbol{\omega}}_I^m - \hat{\mathbf{b}}_g$ and ${}^I\hat{\mathbf{a}}_I = {}^I{\mathbf{a}}_I^m - \hat{\mathbf{b}}_a$, where ${}^I{\boldsymbol{\omega}}_I^m$ and ${}^I{\mathbf{a}}_I^m$ denote the gyroscope and accelerometer measurements, respectively, in the IMU frame. The matrix $\mathbf{\Omega}(\hat{\boldsymbol{\omega}})$, constructed from the estimated angular velocity $\hat{\boldsymbol{\omega}}$ and its skew-symmetric matrix $\lfloor \hat{\boldsymbol{\omega}} \times \rfloor$, is represented as: 
\begin{equation}
    \mathbf{\Omega}(\hat{\boldsymbol{\omega}}) = 
    \begin{bmatrix}
        0 & -\hat{\boldsymbol{\omega}}^\top \\
        \hat{\boldsymbol{\omega}} & -\lfloor \hat{\boldsymbol{\omega}} \times \rfloor
    \end{bmatrix}.
\end{equation}
The estimated state $\hat{\mathbf{x}}$ is propagated with IMU measurements through the continuous-time dynamics in Eq.~\eqref{propagation}, using 4\textsuperscript{th}-order Runge-Kutta numerical integration.

For the covariance propagation, the linearized continuous-time dynamics for the error state is expressed as:
\begin{equation}
\label{ekf}
    \dot{\tilde{\mathbf{x}}} = \mathbf{F}\tilde{\mathbf{x}} + \mathbf{G}\mathbf{n},
\end{equation}
where $\mathbf{n} = \left( \mathbf{n}_g^\top, \mathbf{n}_{wg}^\top, \mathbf{n}_a^\top, \mathbf{n}_{wa}^\top, n_d \right)^\top$. The noise vectors $\mathbf{n}_g$ and $\mathbf{n}_a$ represent the Gaussian noise affecting the gyroscope and accelerometer measurements, respectively. Similarly, $\mathbf{n}_{wg}$ and $\mathbf{n}_{wa}$ correspond to the random walks for the gyroscope and accelerometer measurement biases. The term $n_d$ accounts for the Gaussian noise (i.e., uncertainty) in the time offset.

The matrix $\mathbf{F}$ represents the linearized system dynamics, and the matrix $\mathbf{G}$ models the influence of the process noise on the error state. Only the non-zero elements of the matrix $\mathbf{F}$ are given as follows:
\begin{equation}
    \begin{aligned}
        \mathbf{F}(0:2, 0:2) &= -\lfloor ({}^I{\boldsymbol{\omega}}_I^m - \hat{\mathbf{b}}_g) \times \rfloor, \\
        \mathbf{F}(0:2, 3:5) &= -\mathbf{I}_{3}, \\
        \mathbf{F}(6:8, 0:2) &= -{}^G\hat{\mathbf{R}}_I \lfloor ({}^I{\mathbf{a}}_I^m - \hat{\mathbf{b}}_a) \times \rfloor, \\
        \mathbf{F}(6:8, 9:11) &= -{}^G\hat{\mathbf{R}}_I, \\
        \mathbf{F}(12:14, 6:8) &= \mathbf{I}_{3}.
    \end{aligned}
\end{equation}
Similarly, the non-zero elements of the matrix $\mathbf{G}$ are given as follows:
\begin{equation}
    \begin{aligned}
        \mathbf{G}(0:2, 0:2) &= -\mathbf{I}_{3}, \\
        \mathbf{G}(3:5, 3:5) &= \mathbf{I}_{3}, \\
        \mathbf{G}(6:8, 6:8) &= -{}^G\hat{\mathbf{R}}_I, \\
        \mathbf{G}(9:11, 9:11) &= \mathbf{I}_{3}, \\
        \mathbf{G}(15, 12) &= 1.
    \end{aligned}
\end{equation}

To propagate the covariance, the discrete-time state transition matrix $\mathbf{\Phi}_k$ and discrete-time process noise covariance matrix $\mathbf{Q}_k$, derived from Eq.~\eqref{ekf}, are defined as follows:
\begin{equation}
    \begin{gathered}
        \mathbf{\Phi}_k = \mathbf{\Phi}(t_{k+1}, t_k) = \exp{\left( \int_{t_k}^{t_{k+1}} \mathbf{F}(\tau) d\tau \right)}, \\
        \mathbf{Q}_k = \int_{t_k}^{t_{k+1}} \mathbf{\Phi}(t_{k+1}, \tau) \mathbf{G}\mathbf{Q}\mathbf{G}^\top \mathbf{\Phi}(t_{k+1}, \tau)^\top d\tau,
    \end{gathered}
\end{equation}
where $\mathbf{Q}$ is the continuous-time process noise covariance matrix. The propagated covariance matrix is expressed as:
\begin{equation}
    \mathbf{P}_{k+1|k} = \mathbf{\Phi}_k \mathbf{P}_{k|k} \mathbf{\Phi}_k^\top + \mathbf{Q}_k.
\end{equation}
Considering the time offset \( t_d \), propagation is repeated up to just before the time of the radar measurement aligned to the IMU measurement time stream.
\subsection{Measurement Update with Radar Measurements}
\label{sec: measurement update}
The FMCW 4D radar provides 3D point clouds, where each point includes a 3D position and a scalar Doppler velocity. The Doppler velocity represents the radial velocity of a target point, expressed as:
\begin{equation}
    v_d^i = -{}^R\mathbf{v}_R \cdot \frac{{}^R\mathbf{p}_f^i}{||{}^R\mathbf{p}_f^i||}, \quad \text{for } i = 1, \dots, N,
\end{equation}
where \( N \) is the total number of detected points, \( v_d^i \) denotes the Doppler velocity of the \( i \)-th point, \( {}^R\mathbf{p}_f^i \) is the position vector of the \( i \)-th point in the radar frame, and ${}^R\mathbf{v}_R$ denotes the radar ego-velocity. To estimate the radar ego-velocity \( {}^R\mathbf{v}_R \) from noisy radar measurements, various methods, such as RANSAC and m-estimator-based optimization, have been proposed. In this work, we adopt the 3-point RANSAC-LSQ \cite{9235254}, a simple yet robust method that efficiently eliminates outliers and estimates the radar ego-velocity \( {}^R\mathbf{v}_R \), which is used in the measurement update.

In the case where the IMU and radar are rigidly connected, the radar ego-velocity measurement model can be expressed using the system state. As derived in \cite{9235254}, the radar ego-velocity is expressed as:
\begin{equation}
\label{ego_vel}
    \begin{aligned}
        {}^R\mathbf{v}_R(t) =& {}^R\mathbf{R}_I \left( {}^G\mathbf{R}_I^\top(t) {}^G\mathbf{v}_I(t) \right. \\
        &+ \left. \lfloor ({}^I\boldsymbol{\omega}_I^m(t) - \mathbf{b}_g(t)) \times \rfloor {}^I\mathbf{p}_R \right),
    \end{aligned}
\end{equation}
where the extrinsic parameters, \( {}^R\mathbf{R}_I \) and \( {}^I\mathbf{p}_R \), between the IMU and the radar are assumed to be pre-calibrated and constant.

In the measurement update, the residual \( \mathbf{r} \) is computed as the difference between the radar ego-velocity \( {}^R\mathbf{v}_R \), estimated from the radar measurements, and the predicted radar ego-velocity \( {}^R\hat{\mathbf{v}}_R \) from the state. The residual is expressed as:
\begin{equation}
\label{residual}
    \mathbf{r} = {}^R\mathbf{v}_R(t) - {}^R\hat{\mathbf{v}}_R(t') = \mathbf{h}(\tilde{\mathbf{x}}) + \mathbf{n}_r.
\end{equation}
As illustrated in Fig.~\ref{fig1}, \( t' = t + t_d \) represents the IMU measurement time, which serves as the reference time stream in the filter, while \( t \) represents the radar measurement time used in the measurement update after being aligned to the IMU time stream. The term \( \mathbf{n}_r \) denotes the noise of the measurement. The function \( \mathbf{h}(\tilde{\mathbf{x}}) \) is a nonlinear function that relates the state error \( \tilde{\mathbf{x}} \) to the radar ego-velocity measurement residual. For use in the EKF, this function is linearized with respect to the system state. The measurement Jacobian matrix \( \mathbf{H} \) is expressed as follows:
\begin{equation}
\begin{aligned}
    \mathbf{H} &= 
    \begin{bmatrix}
        \mathbf{H}_q & \mathbf{H}_{b_g} & \mathbf{H}_v & \mathbf{0}_{3 \times 3} & \mathbf{0}_{3 \times 3} & \mathbf{H}_{t_d}
    \end{bmatrix}, \\
    \mathbf{H}_q &= {}^R\mathbf{R}_I \lfloor {}^G\hat{\mathbf{R}}_I^\top {}^G\hat{\mathbf{v}}_I \times \rfloor,\\
    \mathbf{H}_{b_g} &= {}^R\mathbf{R}_I \lfloor {}^I\mathbf{p}_R \times \rfloor,\\
    \mathbf{H}_v &= {}^R\mathbf{R}_I {}^G\hat{\mathbf{R}}_I^\top.
\end{aligned}
\end{equation}
In Eq.~\eqref{ego_vel}, the time-varying states are \( {}^G\mathbf{R}_I \) and \( {}^G\mathbf{v}_I \). By applying the chain rule, the Jacobian \( \mathbf{H}_{t_d} \) is expressed as:
% \begin{equation}
% \begin{aligned}
%     \mathbf{H}_{t_d} &= \mathbf{H}_q \left( {}^G\hat{\mathbf{R}}_I ( {}^I{\boldsymbol{\omega}}_I^m - \hat{\mathbf{b}}_g )\right) \\
%     & + \mathbf{H}_v \left( {}^G\hat{\mathbf{R}}_I ({}^I{\mathbf{a}}_I^m - \hat{\mathbf{b}}_a) + {}^G{\mathbf{g}} \right).
% \end{aligned}
% \end{equation}
\begin{equation}
\begin{aligned}
\mathbf{H}_{t_d} =& \frac{\partial \mathbf{h}\left(\tilde{\mathbf{x}}\left(\tilde{t'}\right)\right)}{\partial {}^G\boldsymbol{\theta}_I\left(\tilde{t'}\right)} 
\cdot \frac{\partial {}^G\boldsymbol{\theta}_I\left(\tilde{t'}\right)}{\partial \tilde{t'}} 
\cdot \frac{\partial \tilde{t'}}{\partial \tilde{t}_d} \\
&+ \frac{\partial \mathbf{h}\left(\tilde{\mathbf{x}}\left(\tilde{t'}\right)\right)}{\partial {}^G\tilde{\mathbf{v}}_I\left(\tilde{t'}\right)} 
\cdot \frac{\partial {}^G\tilde{\mathbf{v}}_I\left(\tilde{t'}\right)}{\partial \tilde{t'}} 
\cdot \frac{\partial \tilde{t'}}{\partial \tilde{t}_d} \\
=& \mathbf{H}_q\left( {}^G\hat{\mathbf{R}}_I\left(t'\right) \left({}^I\boldsymbol{\omega}_I^m\left(t'\right) - \hat{\mathbf{b}}_g\left(t'\right) \right)\right) \\
&+ \mathbf{H}_v\left( {}^G\hat{\mathbf{R}}_I\left(t'\right) \left({}^I\mathbf{a}_I^m\left(t'\right) - \hat{\mathbf{b}}_a\left(t'\right) \right) + {}^G\mathbf{g} \right).
\end{aligned}
\end{equation}

The EKF update proceeds by computing the Kalman gain \( \mathbf{K} \) as:
\begin{equation}
    \mathbf{K} = \mathbf{P}_{k+1|k} \mathbf{H}^\top \left( \mathbf{H} \mathbf{P}_{k+1|k} \mathbf{H}^\top + \mathbf{R} \right)^{-1},
\end{equation}
where \( \mathbf{R} \) represents the measurement noise covariance matrix. Finally, the estimated state and covariance are updated according to the Kalman gain as follows:
\begin{equation}
\begin{aligned}
    \hat{\mathbf{x}}_{k+1|k+1} &= \hat{\mathbf{x}}_{k+1|k} + \mathbf{K} \mathbf{r}, \\
    \mathbf{P}_{k+1|k+1} &= \left( \mathbf{I} - \mathbf{K} \mathbf{H} \right) \mathbf{P}_{k+1|k}.
\end{aligned}
\end{equation}
Each time new radar measurement is received, the measurement update is performed based on the IMU time stream.
\subsection{Online Temporal Calibration}
\label{sec: online temporal calibration}
The proposed method estimates the time offset between the IMU and the radar in real-time by employing the radar ego-velocity measurement model. By accounting for the time offset, the proposed method ensures that both propagation and measurement updates are performed based on a common time stream, ensuring that the measurements from both sensors are synchronized.

The time offset is propagated using a noise model \( n_d \), as described in Eq.~\eqref{ekf}. If the time offset is constant over time or approximately known, it can be estimated without a noise model. However, the time offset varies across sensor models, making it difficult to predefine in most cases. Furthermore, when the vehicle is moving slowly, the impact of the time offset becomes less significant, making it harder to estimate. For this reason, the time offset is modeled as a random walk.

In the measurement model presented in Eq.~\eqref{ego_vel}, the factors affected by the temporal misalignment between sensors are not only the state variables \( {}^G\mathbf{R}_I \), \( {}^G\mathbf{v}_I \), and \( \mathbf{b}_g \), but also the gyroscope measurement \( {}^I\boldsymbol{\omega}_I^m \). Although \( \mathbf{b}_g \), which does not change significantly over time, is negligible, failing to account for the time offset causes the state to propagate over a misaligned time stream, leading to errors in the estimates of \( {}^G\mathbf{R}_I \) and \( {}^G\mathbf{v}_I \). Moreover, the improper use of the gyroscope measurement further degrades the estimation accuracy, and this error accumulates over time. Therefore, accounting for the time offset is crucial to maintain the accuracy and consistency of the state estimates.
\section{Experiments}
\label{sec: experiments}

\subsection{Experimental Setup}
\label{sec: experimental_setup}
\begin{figure}[t]
\centering \includegraphics[width=\linewidth]{figure_2.png} \caption{The handheld platform configuration, including the radar, IMU, and onboard computer. The experiments are conducted in a room equipped with a motion capture system to obtain accurate ground truth.}
\label{fig2}
\end{figure}

We conduct experiments using three datasets, comprising a total of 15 sequences. One is our self-collected dataset, captured with a handheld platform as shown in Fig.~\ref{fig2}, while the other two are public radar datasets: ICINS2021~\cite{9470842}, and ColoRadar~\cite{kramer2022coloradar}. The sensors on our platform include a 4D FMCW radar, specifically the Texas Instruments AWR1843BOOST, and an Xsens MTI-670-DK IMU. No additional hardware triggers are used between the sensors, and the sensor data is recorded using an Intel NUC i7 onboard computer. The experiments are conducted in an indoor area equipped with a motion capture system to obtain precise ground truth. The extrinsic calibration between the IMU and the radar is performed manually. To highlight the significance of temporal calibration in RIO, we design the dataset with two levels of difficulty. Sequences 1 to 3 feature standard motion patterns, while Sequences 4 to 7 introduce more rotational motion to induce larger errors due to the time offset, providing a clearer demonstration of its impact.

\begin{figure*}[t]
\centering
\includegraphics[width=\linewidth]{figure_3.png}
\caption{Comparison of estimated trajectories with the ground truth. The \textcolor{black}{black} trajectory is the ground truth, the \textcolor{blue}{blue} one is the EKF-RIO, which does not account for temporal calibration, and the \textcolor{red}{red} one is the proposed RIO with online temporal calibration. Results are presented for Sequence 4, ICINS 1, and ColoRadar 1, representing one sequence from each of the three datasets.}
\label{trajectory}
\end{figure*}

In~\cite{9470842}, the ICINS2021 dataset is collected using a Texas Instruments IWR6843AOP radar sensor, an Analog Devices ADIS16448 IMU sensor, and a camera. A microcontroller board is used for active hardware triggering to accurately capture the timing of the radar measurements. Data is collected using both handheld and drone platforms. The handheld sequences, ``carried\_1'' and ``carried\_2'', are referred to as ``ICINS 1'' and ``ICINS 2'', while the drone sequences, ``flight\_1'' and ``flight\_2'', are referred to as ``ICINS 3'' and ``ICINS 4'', respectively. The ground truth is provided through visual-inertial SLAM, which performs multiple loop closures, offering a pseudo-ground truth. In~\cite{kramer2022coloradar}, the ColoRadar dataset is collected using a Texas Instruments AWR1843BOOST radar sensor, a Microstrain 3DM-GX5-25 IMU sensor, and a LiDAR mounted on a handheld platform. No specific synchronization setup is used between the sensors. The sequences, ``arpg\_lab\_run0'' and ``arpg\_lab\_run1'', are referred to as ``ColoRadar 1'' and ``ColoRadar 2'', while the sequences ``ec\_hallways\_run0'' and ``ec\_hallways\_run1'' are referred to as ``ColoRadar 3'' and ``ColoRadar 4'', respectively. The ground truth is generated via LiDAR-inertial SLAM, which includes loop closures, offering a pseudo-ground truth.
\subsection{Evaluation}
\label{sec: evaluation}

\begin{table}[t]
\centering
\caption{Quantitative Results of Fixed Offset and Online Estimation}
\label{fixed_offset}
\resizebox{\linewidth}{!}{
\begin{tblr}{
  cells = {c},
  cell{1}{1} = {r=2}{},
  cell{1}{2} = {r=2}{},
  cell{1}{3} = {r=2}{},
  cell{1}{4} = {c=2}{},
  cell{1}{6} = {c=2}{},
  cell{3}{1} = {r=6}{},
  cell{3}{2} = {r=5}{},
  cell{3}{5} = {fg=red},
  cell{4}{4} = {fg=red},
  cell{5}{4} = {fg=blue},
  cell{5}{5} = {fg=blue},
  cell{5}{6} = {fg=blue},
  cell{5}{7} = {fg=red},
  cell{6}{6} = {fg=red},
  cell{6}{7} = {fg=blue},
  cell{9}{1} = {r=6}{},
  cell{9}{2} = {r=5}{},
  cell{11}{4} = {fg=red},
  cell{11}{5} = {fg=blue},
  cell{11}{6} = {fg=red},
  cell{11}{7} = {fg=red},
  cell{12}{4} = {fg=blue},
  cell{12}{5} = {fg=red},
  cell{12}{6} = {fg=blue},
  cell{12}{7} = {fg=blue},
  hline{1,3,9,15} = {-}{},
  hline{2} = {4-7}{},
}
\textbf{Sequence} & \textbf{Method} &  \textbf{Time Offset (s)}            & \textbf{APE RMSE} &                & \textbf{RPE RMSE} &                   \\
                  &                 &                                      & Trans. (m)        & Rot. (\degree) & Trans. (m)        & Rot. (\degree)    \\
                  \hline
Sequence 1        & Fixed Offset    & 0.0             & 0.985             & 1.872          & 0.264             & 1.230          \\
                  &                 & -0.05           & 0.647             & 7.561          & 0.166             & 1.549          \\
                  &                 & -0.10           & 0.661             & 2.438          & 0.138             & 0.948          \\
                  &                 & -0.15           & 0.826             & 5.151          & \textbf{0.131}    & 1.196          \\
                  &                 & -0.20           & 0.974             & 2.698          & 0.156             & 1.274          \\
                  & Online Est.     & \textbf{-0.114} & \textbf{0.646}    & \textbf{0.935} & 0.132    & \textbf{0.774} \\
Sequence 4        & Fixed Offset    & 0.0             & 1.737             & 25.885         & 0.118             & 4.074          \\
                  &                 & -0.05           & 1.028             & 15.460         & 0.091             & 2.313          \\
                  &                 & -0.10           & 0.635             & 4.655          & 0.061             & 0.994          \\
                  &                 & -0.15           & 0.649             & 4.275          & 0.068             & 1.083          \\
                  &                 & -0.20           & 0.716             & 12.461         & 0.092             & 2.526          \\
                  & Online Est.     & \textbf{-0.115} & \textbf{0.610}    & \textbf{3.099} & \textbf{0.057}    & \textbf{0.944} 
\end{tblr}
}
\vspace{0.3em}
{\raggedright
\noindent\par {\footnotesize \textsuperscript{*}The initial time offset of `Online Est.' is set to 0.0 and the converged values are shown above.}
\noindent\par {\footnotesize \textsuperscript{**}For each sequence, the lowest error values among the fixed offsets are highlighted in \textcolor{red}{red}, and the second-lowest in \textcolor{blue}{blue}.}
\par}

\end{table}
For the performance comparison, the open-source EKF-RIO \cite{9235254}, which uses the same measurement model but does not account for temporal calibration, is employed. All parameters are kept identical to ensure a fair comparison. In the proposed method, the time offset \( t_d \) is initialized to 0.0 seconds for all sequences, reflecting a typical scenario where the initial time offset is unknown. The experimental results are evaluated using the open-source tool EVO \cite{grupp2017evo}. Figure~\ref{trajectory} illustrates the estimated trajectories compared to the ground truth for visual comparison, with one representative result from each dataset. Due to the stochastic nature of the RANSAC algorithm used in radar ego-velocity estimation, the averaged results from 100 trials across all datasets are presented. We compare the root mean square error (RMSE) of both absolute pose error (APE) and relative pose error (RPE), with the RPE calculated at 10-meter intervals.

APE evaluates the overall trajectory by calculating the difference between the ground truth and the estimated poses for all frames, making it particularly useful for assessing the global accuracy of the estimated trajectory. However, APE can be sensitive to significant rotational errors that occur early or in specific sections, potentially overshadowing smaller errors later in the trajectory. In contrast, RPE focuses on local accuracy by aligning poses at regular intervals and calculating the error, allowing discrepancies over shorter segments to be highlighted. When the temporal calibration between sensors is not accounted for, errors can accumulate over time, making RPE evaluation essential. Both metrics offer valuable insights, providing a comprehensive evaluation of the trajectory.

\subsubsection{Self-Collected Dataset}
The purpose of the self-collected dataset is to identify the actual time offset between the IMU and the radar and evaluate its impact on the accuracy of RIO. Since the handheld platform does not utilize a hardware trigger to synchronize the sensors, the exact time offset is unknown and must be estimated. To address this uncertainty, we evaluate the performance of fixed time offsets over a range of values to determine the interval that provides the best accuracy and estimate the likely time offset range.

As shown in Table \ref{fixed_offset}, error values are analyzed with fixed offsets set at 0.05-second intervals for both Sequence 1 and Sequence 4, which feature different motion patterns. The results show that the time offset falls within the -0.10 to -0.15 second range, where the highest accuracy in terms of APE and RPE is observed for both sequences. The proposed method, which utilizes online temporal calibration, estimates the time offset as -0.114 seconds for Sequence 1 and -0.115 seconds for Sequence 4, closely matching the range found through fixed offset testing. In both cases, the proposed method achieves improved performance in terms of both APE and RPE, demonstrates its effectiveness in accurately estimating the time offset.

\begin{table}[t]
\centering
\caption{Quantitative Results of Comparison study on Self-collected dataset}
\label{table_self}
\resizebox{\linewidth}{!}{
\begin{tblr}{
  cells = {c},
  cell{1}{1} = {r=2}{},
  cell{1}{2} = {r=2}{},
  cell{1}{3} = {c=2}{},
  cell{1}{5} = {c=2}{},
  cell{3}{1} = {r=2}{},
  cell{5}{1} = {r=2}{},
  cell{7}{1} = {r=2}{},
  cell{9}{1} = {r=2}{},
  cell{11}{1} = {r=2}{},
  cell{13}{1} = {r=2}{},
  cell{15}{1} = {r=2}{},
  cell{17}{1} = {r=2}{},
  hline{1,3,5,7,9,11,13,15,17,19} = {-}{},
  hline{2} = {3-6}{},
}
{\textbf{Sequence }\\\textbf{(Trajectory Length)}} & {\textbf{Method } \textbf{($\hat{t}_d$)}} & \textbf{APE RMSE } &                & \textbf{RPE RMSE } &                \\
                                                   &                                         & Trans. (m)         & Rot. (\degree)        & Trans. (m)         & Rot. (\degree)        \\
                                                   \hline
{Sequence 1\\(177 m)}                              & {EKF-RIO (N/A)}                        & 0.985              & 1.872           & 0.264              & 1.230          \\
                                                   & {Ours (-0.114 s)}                      & \textbf{0.646}     & \textbf{0.935}  & \textbf{0.132}     & \textbf{0.774} \\
{Sequence 2\\(197 m)}                              & {EKF-RIO}                              & 2.269              & 2.161           & 0.136              & 1.414          \\
                                                   & {Ours (-0.114 s)}                      & \textbf{0.587}     & \textbf{1.650}  & \textbf{0.064}     & \textbf{0.774} \\
{Sequence 3\\(144 m)}                              & {EKF-RIO}                              & 1.368              & 2.331           & 0.167              & 1.347          \\
                                                   & {Ours (-0.113 s)}                      & \textbf{0.414}     & \textbf{1.140}  & \textbf{0.088}     & \textbf{0.613} \\
{Sequence 4\\(197 m)}                              & {EKF-RIO}                              & 1.737              & 25.885          & 0.118              & 4.074          \\
                                                   & {Ours (-0.115 s)}                      & \textbf{0.610}     & \textbf{3.099}  & \textbf{0.057}     & \textbf{0.944} \\
{Sequence 5\\(190 m)}                              & {EKF-RIO}                              & 2.375              & 7.702           & 0.122              & 1.600          \\
                                                   & {Ours (-0.115 s)}                      & \textbf{1.150}     & \textbf{1.304}  & \textbf{0.069}     & \textbf{0.814} \\
{Sequence 6\\(179 m)}                              & {EKF-RIO}                              & 1.267              & 17.907          & 0.117              & 2.828          \\
                                                   & {Ours (-0.111 s)}                      & \textbf{0.661}     & \textbf{2.551}  & \textbf{0.051}     & \textbf{0.809} \\
{Sequence 7\\(223 m)}                              & {EKF-RIO}                              & 2.757              & 10.092          & 0.116              & 1.863          \\
                                                   & {Ours (-0.112 s)}                      & \textbf{1.596}     & \textbf{6.039}  & \textbf{0.057}     & \textbf{1.365} \\
{Average}                                          & {EKF-RIO}                              & 1.822              & 9.707            & 0.148             & 2.051          \\
                                                   & {Ours (-0.113 s)}                      & \textbf{0.809}     & \textbf{2.388}   & \textbf{0.074}    & \textbf{0.870}   
\end{tblr}
}
\end{table}

Since the radar delay is generally larger than IMU delay, the time offset \( t_d \), representing the difference between these delays, typically takes a negative value. To evaluate the robustness of the estimation, different initial values of \( t_d \) ranging from 0.0 to -0.3 seconds are tested. Figure \ref{sq5} illustrates the estimated time offset for each initial setting, along with the 3-sigma boundaries. As \( t_d \) is estimated from radar ego-velocity, it cannot be determined while the platform is stationary. Once the platform starts moving, the filter begins estimating \( t_d \) and quickly converges to a stable value. The filter converges to a stable time offset of -0.114 ± 0.001 seconds in Sequence 1 and -0.115 ± 0.001 seconds in Sequence 4.

Table \ref{table_self} presents the performance comparison between the proposed method with online temporal calibration and EKF-RIO across seven sequences. The proposed method outperforms EKF-RIO, significantly reducing both APE and RPE across all sequences. Specifically, it reduces APE translation error by an average of 56\%, APE rotation error by 75\%, RPE translation error by 50\%, and RPE rotation error by 58\% compared with EKF-RIO. Despite using the same measurement model, the performance improvement is achieved solely by applying propagation and updates based on a common time stream through the proposed online temporal calibration.

On average, the time offset \( t_d \) is estimated to be -0.113 ± 0.002 seconds, confirming consistent temporal calibration throughout the experiments. Compared with LiDAR-inertial and visual-inertial systems, radar-inertial systems exhibit a significantly larger time offset, as shown in Table~\ref{time_offset_comparison}. Given the radar sensor rate (10 Hz), such a large time offset is significant enough to cause a misalignment spanning more than one data frame. These findings highlight the necessity of temporal calibration in RIO, which is crucial for accurate sensor fusion and reliable pose estimation in real-world applications.

\begin{figure}[t]
\centering
\includegraphics[width=\linewidth]{figure_4.png}
\caption{Time offset estimation with 3-sigma boundaries for different initial values in Sequence 1 and 4.}
\label{sq5}
\end{figure}

\begin{table}[t]
\centering
\caption{Comparison of Time Offset in Multi-Sensor Fusion Systems}
\label{time_offset_comparison}
\begin{tabular}{|c|c|c|} 
\hline
\textbf{Systems} & \textbf{Sensor} & \textbf{Time Offset} \\ 
\hline
LiDAR-Inertial~\cite{10113826} & Velodyne VLP-32 & 0.006 s\\ 
\hline
Visual-Inertial~\cite{li2014online} & PointGrey Bumblebee2 & 0.047 s\\ 
\hline
Radar-Inertial & TI AWR1843BOOST & \textbf{0.113 s} \\
\hline
\end{tabular}
\end{table}

\subsubsection{Open Datasets}
Table \ref{opendataset} presents the results from the two open datasets. The ICINS dataset incorporates a hardware trigger for the radar, which we use to validate the accuracy of the time offset estimation for the proposed method. In this setup, a microcontroller sends radar trigger signals, prompting the radar to begin scanning. The radar data is timestamped based on the actual trigger signal, providing a pseudo-ground truth for time offset estimation. Theoretically, if the sensors are time-synchronized through triggers, the time offset \( t_d \) is expected to be close to 0.0 seconds. The proposed method estimates the time offset to be an average of 0.016 ± 0.003 seconds. Despite this slight discrepancy, the proposed method demonstrates comparable or improved performance on average in both APE and RPE compared with EKF-RIO. Although the ICINS dataset includes hardware-triggered signals for the radar, there is no such trigger signal for the IMU in the dataset, which may introduce a delay in IMU measurements. As defined in Eq.~\eqref{time_offset}, we attribute the estimated positive time offset to this IMU delay, explaining the difference from the expected value.

The ColoRadar dataset, widely used for performance comparison in the RIO field, is utilized to assess if the proposed method generalizes well across different datasets. As shown in Table \ref{opendataset}, the proposed method also demonstrates performance improvements over EKF-RIO in terms of both APE and RPE on average. However, the extent of improvement is smaller compared with the self-collected dataset, which can be explained by differences in trajectory characteristics. The radar ego-velocity model utilizes not only the accelerometer but also the gyroscope measurements. As illustrated in Fig.~\ref{trajectory}, the ColoRadar dataset involves movement over a larger area with less rotation, leading to a smaller impact of the time offset on performance. Nonetheless, the proposed method achieves 33\% reduction in RPE translation error, demonstrating its effectiveness even in this less challenging trajectory. On average, the time offset \( t_d \) is estimated to be -0.111 ± 0.003 seconds, similar to the time offset found in the self-collected dataset. This consistency is likely due to the use of the same radar sensor model in both datasets, further validating the reliability of the proposed method across different environments.

\begin{table}[t]
\centering
\caption{Quantitative Results of Comparison study on Open datasets}
\label{opendataset}
\resizebox{\linewidth}{!}{
\begin{tblr}{
  cells = {c},
  cell{1}{1} = {r=2}{},
  cell{1}{2} = {r=2}{},
  cell{1}{3} = {c=2}{},
  cell{1}{5} = {c=2}{},
  cell{3}{1} = {r=2}{},
  cell{5}{1} = {r=2}{},
  cell{7}{1} = {r=2}{},
  cell{9}{1} = {r=2}{},
  cell{11}{1} = {r=2}{},
  cell{13}{1} = {r=2}{},
  cell{15}{1} = {r=2}{},
  cell{17}{1} = {r=2}{},
  cell{19}{1} = {r=2}{},
  cell{21}{1} = {r=2}{},
  hline{1,3,5,7,9,11,13,15,17,19,21,23} = {-}{},
  hline{2-3} = {3-6}{},
}
{\textbf{Sequence }\\\textbf{(Trajectory Length)}}       & \textbf{Method ($\hat{t}_d$)} & \textbf{APE RMSE}        &                                           & \textbf{RPE RMSE}       &                         \\
                        &                               & Trans. (m)               & Rot. (\degree)                                   & Trans. (m)              & Rot. (\degree)                 \\
                        \hline
{ICINS 1\\(295 m)}      & EKF-RIO (N/A)                 & 1.959                    & 10.694                                    & \textbf{0.093}          & \textbf{0.896}          \\
                        & Ours (0.016 s)                & \textbf{1.922}           & \textbf{10.135}                           & 0.098                   & 0.918          \\
{ICINS 2\\(468 m)}      & EKF-RIO                       & 3.830                    & 23.151                                    & \textbf{0.114}          & 1.289                   \\
                        & Ours (0.013 s)                & \textbf{3.198}           & \textbf{19.235}                           & 0.121                   & \textbf{1.076}          \\
{ICINS 3\\(150 m)}      & EKF-RIO                       & \textbf{1.502}           & \textbf{9.905}                            & 0.130                   & \textbf{1.512}           \\
                        & Ours (0.015 s)                & 1.530                    & 10.189                                    & \textbf{0.126}          & 1.553          \\
{ICINS 4\\(50 m)}       & EKF-RIO                       & \textbf{0.213}           & \textbf{2.091}                            & \textbf{0.076}          & \textbf{0.923}           \\
                        & Ours (0.019 s)                & 0.216                    & 2.098                                     & 0.081                   & \textbf{0.923}          \\
Average                 & EKF-RIO                       & 1.876                    & 11.460                                    & \textbf{0.103}          & 1.155                   \\
                        & Ours (0.016 s)                & \textbf{1.716}           & \textbf{10.414}                           & 0.106                   & \textbf{1.117}          \\
                        \hline
{ColoRadar 1\\(178 m) } & EKF-RIO (N/A)                 & 6.556                    & \textbf{\textbf{1.354}}                   & 0.182                   & \textbf{1.071} \\
                        & Ours (-0.110 s)               & \textbf{\textbf{6.173}}  & 1.382                                     & \textbf{\textbf{0.155}} & 1.188                   \\
{ColoRadar 2\\(197 m) } & EKF-RIO                       & \textbf{\textbf{4.747}}  & 1.238                                     & 0.372                   & 1.375                   \\
                        & Ours (-0.114 s)               & 4.826                    & \textbf{\textbf{0.960}}                   & \textbf{\textbf{0.292}} & \textbf{\textbf{1.180}} \\
{ColoRadar 3\\(197 m) } & EKF-RIO                       & \textbf{\textbf{8.307}}  & 1.969                                     & 0.259                   & 1.015                   \\
                        & Ours (-0.108 s)               & 8.550                    & \textbf{\textbf{1.852}}                   & \textbf{\textbf{0.221}} & \textbf{\textbf{0.879}} \\
{ColoRadar 4\\(144 m) } & EKF-RIO                       & 12.111                   & 2.815                                     & 0.488                   & 1.263                   \\
                        & Ours (-0.112 s)               & \textbf{11.946}          & \textbf{2.756}                            & \textbf{0.200}          & \textbf{1.116} \\
Average                 & EKF-RIO                       & 7.930                    & 1.844                                     & 0.325                   & 1.181                   \\
                        & Ours(-0.111 s)                & \textbf{7.874}           & \textbf{1.737}                            & \textbf{0.217}          & \textbf{1.091}          
\end{tblr}
}
\end{table}

\section{Conclusions and Future Work}
\label{sec: conclusion}
In this paper, we proposed an EKF-based RIO framework with online temporal calibration. To ensure accurate sensor time synchronization during IMU and radar sensor fusion, the time offset between sensors is estimated from radar ego-velocity, which is derived from a single radar scan. This approach avoids the potential risks of finding correspondences between consecutive radar scans and, being independent of radar point cloud density, offers flexibility for use with various types of radar sensors. By leveraging temporal calibration, sensor measurements are aligned to a common time stream. This allows propagation and measurement updates to be applied at the correct time, improving overall performance. Extensive experiments across multiple datasets demonstrate the effectiveness of time offset estimation and provide a detailed analysis of its impact on overall performance.

Several challenges remain in multi-sensor fusion state estimation using radar systems. One issue is the reliance on manually calibrated sensor extrinsic parameters in many studies, which can lead to inaccuracies. We will focus on spatiotemporal calibration between sensors to further improve the accuracy and robustness of multi-sensor fusion systems.

\addtolength{\textheight}{-12cm}   % This command serves to balance the column lengths
                                  % on the last page of the document manually. It shortens
                                  % the textheight of the last page by a suitable amount.
                                  % This command does not take effect until the next page
                                  % so it should come on the page before the last. Make
                                  % sure that you do not shorten the textheight too much.

%%%%%%%%%%%%%%%%%%%%%%%%%%%%%%%%%%%%%%%%%%%%%%%%%%%%%%%%%%%%%%%%%%%%%%%%%%%%%%%%



%%%%%%%%%%%%%%%%%%%%%%%%%%%%%%%%%%%%%%%%%%%%%%%%%%%%%%%%%%%%%%%%%%%%%%%%%%%%%%%%



%%%%%%%%%%%%%%%%%%%%%%%%%%%%%%%%%%%%%%%%%%%%%%%%%%%%%%%%%%%%%%%%%%%%%%%%%%%%%%%%
\begin{thebibliography}{23}

\bibitem{9196524} P. Geneva, K. Eckenhoff, W. Lee, Y. Yang, and G. Huang, “OpenVINS: A Research Platform for Visual-Inertial Estimation,” in \textit{IEEE International Conference on Robotics and Automation}, 2020, pp. 4666–4672.

\bibitem{9440682} C. Campos, R. Elvira, J. J. G. Rodríguez, J. M. M. Montiel, and J. D. Tardós, “ORB-SLAM3: An Accurate Open-Source Library for Visual, Visual–Inertial, and Multimap SLAM,” \textit{IEEE Transactions on Robotics}, vol. 37, no. 6, pp. 1874–1890, 2021.

\bibitem{9341176} T. Shan, B. Englot, D. Meyers, W. Wang, C. Ratti, and D. Rus, “LIO-SAM: Tightly-coupled Lidar Inertial Odometry via Smoothing and Mapping,” in \textit{IEEE/RSJ International Conference on Intelligent Robots and Systems}, 2020, pp. 5135–5142.

\bibitem{9697912} W. Xu, Y. Cai, D. He, J. Lin, and F. Zhang, “FAST-LIO2: Fast Direct LiDAR-Inertial Odometry,” \textit{IEEE Transactions on Robotics}, vol. 38, no. 4, pp. 2053–2073, 2022.

\bibitem{zhang2018laser} J. Zhang and S. Singh, “Laser–visual–inertial odometry and mapping with high robustness and low drift,” \textit{Journal of Field Robotics}, vol. 35, no. 8, pp. 1242–1264, 2018.

\bibitem{10611444} M. Nissov, N. Khedekar, and K. Alexis, “Degradation Resilient LiDAR-Radar-Inertial Odometry,” in \textit{IEEE International Conference on Robotics and Automation}, 2024, pp. 8587–8594.

\bibitem{10683889} K. Harlow, H. Jang, T. D. Barfoot, A. Kim, and C. Heckman, “A New Wave in Robotics: Survey on Recent MmWave Radar Applications in Robotics,” \textit{IEEE Transactions on Robotics}, vol. 40, pp. 4544–4560, 2024.

\bibitem{6728341} D. Kellner, M. Barjenbruch, J. Klappstein, J. Dickmann, and K. Dietmayer, “Instantaneous ego-motion estimation using Doppler radar,” in \textit{IEEE Conference on Intelligent Transportation Systems}, 2013, pp. 869–874.

\bibitem{10477463} L. Fan, J. Wang, Y. Chang, Y. Li, Y. Wang, and D. Cao, “4D mmWave Radar for Autonomous Driving Perception: A Comprehensive Survey,” \textit{IEEE Transactions on Intelligent Vehicles}, vol. 9, no. 4, pp. 4606–4620, 2024.

\bibitem{10610666} V. Kubelka, E. Fritz, and M. Magnusson, “Do we need scan-matching in radar odometry?” in \textit{IEEE International Conference on Robotics and Automation}, 2024, pp. 13710–13716.

\bibitem{9235254} C. Doer and G. F. Trommer, “An EKF Based Approach to Radar Inertial Odometry,” in \textit{IEEE International Conference on Multisensor Fusion and Integration for Intelligent Systems}, 2020, pp. 152–159.

\bibitem{9317343} C. Doer and G. F. Trommer, “Radar Inertial Odometry With Online Calibration,” in \textit{European Navigation Conference}, 2020, pp. 1–10.

\bibitem{9470842} C. Doer and G. F. Trommer, “Yaw aided Radar Inertial Odometry using Manhattan World Assumptions,” in \textit{International Conference on Integrated Navigation Systems}, 2021, pp. 1–9.

\bibitem{9981396} J. Michalczyk, R. Jung, and S. Weiss, “Tightly-Coupled EKF-Based Radar-Inertial Odometry,” in \textit{IEEE/RSJ International Conference on Intelligent Robots and Systems}, 2022, pp. 12336–12343.

\bibitem{10160482} J. Michalczyk, R. Jung, C. Brommer, and S. Weiss, “Multi-State Tightly-Coupled EKF-Based Radar-Inertial Odometry With Persistent Landmarks,” in \textit{IEEE International Conference on Robotics and Automation}, 2023, pp. 4011–4017.

\bibitem{10100861} Y. Zhuang, B. Wang, J. Huai, and M. Li, “4D iRIOM: 4D Imaging Radar Inertial Odometry and Mapping,” \textit{IEEE Robotics and Automation Letters}, vol. 8, no. 6, pp. 3246–3253, 2023.

\bibitem{8593603} T. Qin and S. Shen, “Online Temporal Calibration for Monocular Visual-Inertial Systems,” in \textit{IEEE/RSJ International Conference on Intelligent Robots and Systems}, 2018, pp. 3662–3669.

\bibitem{li2014online} M. Li and A. I. Mourikis, “Online temporal calibration for camera–IMU systems: Theory and algorithms,” \textit{The International Journal of Robotics Research}, vol. 33, no. 7, pp. 947–964, 2014.

\bibitem{9561254} W. Lee, Y. Yang, and G. Huang, “Efficient Multi-sensor Aided Inertial Navigation with Online Calibration,” in \textit{IEEE International Conference on Robotics and Automation}, 2021, pp. 5706–5712.

\bibitem{sola2017quaternion} J. Sola, “Quaternion kinematics for the error-state Kalman filter,” \textit{arXiv preprint arXiv:1711.02508}, 2017.

\bibitem{kramer2022coloradar} A. Kramer, K. Harlow, C. Williams, and C. Heckman, “ColoRadar: The direct 3D millimeter wave radar dataset,” \textit{The International Journal of Robotics Research}, vol. 41, no. 4, pp. 351–360, 2022.

\bibitem{grupp2017evo} M. Grupp, “evo: Python package for the evaluation of odometry and SLAM,” \url{https://github.com/MichaelGrupp/evo}, 2017.

\bibitem{10113826} S. Li, X. Li, S. Chen, Y. Zhou, and S. Wang, “Two-step LiDAR/Camera/IMU spatial and temporal calibration based on continuous-time trajectory estimation,” \textit{IEEE Transactions on Industrial Electronics}, vol. 71, no. 3, pp. 3182–3191, 2024.

\end{thebibliography}

\end{document}
