\subsection{Evaluation}
\label{sec: evaluation}

\begin{table}[t]
\centering
\caption{Quantitative Results of Fixed Offset and Online Estimation}
\label{fixed_offset}
\resizebox{\linewidth}{!}{
\begin{tblr}{
  cells = {c},
  cell{1}{1} = {r=2}{},
  cell{1}{2} = {r=2}{},
  cell{1}{3} = {r=2}{},
  cell{1}{4} = {c=2}{},
  cell{1}{6} = {c=2}{},
  cell{3}{1} = {r=6}{},
  cell{3}{2} = {r=5}{},
  cell{3}{5} = {fg=red},
  cell{4}{4} = {fg=red},
  cell{5}{4} = {fg=blue},
  cell{5}{5} = {fg=blue},
  cell{5}{6} = {fg=blue},
  cell{5}{7} = {fg=red},
  cell{6}{6} = {fg=red},
  cell{6}{7} = {fg=blue},
  cell{9}{1} = {r=6}{},
  cell{9}{2} = {r=5}{},
  cell{11}{4} = {fg=red},
  cell{11}{5} = {fg=blue},
  cell{11}{6} = {fg=red},
  cell{11}{7} = {fg=red},
  cell{12}{4} = {fg=blue},
  cell{12}{5} = {fg=red},
  cell{12}{6} = {fg=blue},
  cell{12}{7} = {fg=blue},
  hline{1,3,9,15} = {-}{},
  hline{2} = {4-7}{},
}
\textbf{Sequence} & \textbf{Method} &  \textbf{Time Offset (s)}            & \textbf{APE RMSE} &                & \textbf{RPE RMSE} &                   \\
                  &                 &                                      & Trans. (m)        & Rot. (\degree) & Trans. (m)        & Rot. (\degree)    \\
                  \hline
Sequence 1        & Fixed Offset    & 0.0             & 0.985             & 1.872          & 0.264             & 1.230          \\
                  &                 & -0.05           & 0.647             & 7.561          & 0.166             & 1.549          \\
                  &                 & -0.10           & 0.661             & 2.438          & 0.138             & 0.948          \\
                  &                 & -0.15           & 0.826             & 5.151          & \textbf{0.131}    & 1.196          \\
                  &                 & -0.20           & 0.974             & 2.698          & 0.156             & 1.274          \\
                  & Online Est.     & \textbf{-0.114} & \textbf{0.646}    & \textbf{0.935} & 0.132    & \textbf{0.774} \\
Sequence 4        & Fixed Offset    & 0.0             & 1.737             & 25.885         & 0.118             & 4.074          \\
                  &                 & -0.05           & 1.028             & 15.460         & 0.091             & 2.313          \\
                  &                 & -0.10           & 0.635             & 4.655          & 0.061             & 0.994          \\
                  &                 & -0.15           & 0.649             & 4.275          & 0.068             & 1.083          \\
                  &                 & -0.20           & 0.716             & 12.461         & 0.092             & 2.526          \\
                  & Online Est.     & \textbf{-0.115} & \textbf{0.610}    & \textbf{3.099} & \textbf{0.057}    & \textbf{0.944} 
\end{tblr}
}
\vspace{0.3em}
{\raggedright
\noindent\par {\footnotesize \textsuperscript{*}The initial time offset of `Online Est.' is set to 0.0 and the converged values are shown above.}
\noindent\par {\footnotesize \textsuperscript{**}For each sequence, the lowest error values among the fixed offsets are highlighted in \textcolor{red}{red}, and the second-lowest in \textcolor{blue}{blue}.}
\par}

\end{table}
For the performance comparison, the open-source EKF-RIO \cite{9235254}, which uses the same measurement model but does not account for temporal calibration, is employed. All parameters are kept identical to ensure a fair comparison. In the proposed method, the time offset \( t_d \) is initialized to 0.0 seconds for all sequences, reflecting a typical scenario where the initial time offset is unknown. The experimental results are evaluated using the open-source tool EVO \cite{grupp2017evo}. Figure~\ref{trajectory} illustrates the estimated trajectories compared to the ground truth for visual comparison, with one representative result from each dataset. Due to the stochastic nature of the RANSAC algorithm used in radar ego-velocity estimation, the averaged results from 100 trials across all datasets are presented. We compare the root mean square error (RMSE) of both absolute pose error (APE) and relative pose error (RPE), with the RPE calculated at 10-meter intervals.

APE evaluates the overall trajectory by calculating the difference between the ground truth and the estimated poses for all frames, making it particularly useful for assessing the global accuracy of the estimated trajectory. However, APE can be sensitive to significant rotational errors that occur early or in specific sections, potentially overshadowing smaller errors later in the trajectory. In contrast, RPE focuses on local accuracy by aligning poses at regular intervals and calculating the error, allowing discrepancies over shorter segments to be highlighted. When the temporal calibration between sensors is not accounted for, errors can accumulate over time, making RPE evaluation essential. Both metrics offer valuable insights, providing a comprehensive evaluation of the trajectory.

\subsubsection{Self-Collected Dataset}
The purpose of the self-collected dataset is to identify the actual time offset between the IMU and the radar and evaluate its impact on the accuracy of RIO. Since the handheld platform does not utilize a hardware trigger to synchronize the sensors, the exact time offset is unknown and must be estimated. To address this uncertainty, we evaluate the performance of fixed time offsets over a range of values to determine the interval that provides the best accuracy and estimate the likely time offset range.

As shown in Table \ref{fixed_offset}, error values are analyzed with fixed offsets set at 0.05-second intervals for both Sequence 1 and Sequence 4, which feature different motion patterns. The results show that the time offset falls within the -0.10 to -0.15 second range, where the highest accuracy in terms of APE and RPE is observed for both sequences. The proposed method, which utilizes online temporal calibration, estimates the time offset as -0.114 seconds for Sequence 1 and -0.115 seconds for Sequence 4, closely matching the range found through fixed offset testing. In both cases, the proposed method achieves improved performance in terms of both APE and RPE, demonstrates its effectiveness in accurately estimating the time offset.

\begin{table}[t]
\centering
\caption{Quantitative Results of Comparison study on Self-collected dataset}
\label{table_self}
\resizebox{\linewidth}{!}{
\begin{tblr}{
  cells = {c},
  cell{1}{1} = {r=2}{},
  cell{1}{2} = {r=2}{},
  cell{1}{3} = {c=2}{},
  cell{1}{5} = {c=2}{},
  cell{3}{1} = {r=2}{},
  cell{5}{1} = {r=2}{},
  cell{7}{1} = {r=2}{},
  cell{9}{1} = {r=2}{},
  cell{11}{1} = {r=2}{},
  cell{13}{1} = {r=2}{},
  cell{15}{1} = {r=2}{},
  cell{17}{1} = {r=2}{},
  hline{1,3,5,7,9,11,13,15,17,19} = {-}{},
  hline{2} = {3-6}{},
}
{\textbf{Sequence }\\\textbf{(Trajectory Length)}} & {\textbf{Method } \textbf{($\hat{t}_d$)}} & \textbf{APE RMSE } &                & \textbf{RPE RMSE } &                \\
                                                   &                                         & Trans. (m)         & Rot. (\degree)        & Trans. (m)         & Rot. (\degree)        \\
                                                   \hline
{Sequence 1\\(177 m)}                              & {EKF-RIO (N/A)}                        & 0.985              & 1.872           & 0.264              & 1.230          \\
                                                   & {Ours (-0.114 s)}                      & \textbf{0.646}     & \textbf{0.935}  & \textbf{0.132}     & \textbf{0.774} \\
{Sequence 2\\(197 m)}                              & {EKF-RIO}                              & 2.269              & 2.161           & 0.136              & 1.414          \\
                                                   & {Ours (-0.114 s)}                      & \textbf{0.587}     & \textbf{1.650}  & \textbf{0.064}     & \textbf{0.774} \\
{Sequence 3\\(144 m)}                              & {EKF-RIO}                              & 1.368              & 2.331           & 0.167              & 1.347          \\
                                                   & {Ours (-0.113 s)}                      & \textbf{0.414}     & \textbf{1.140}  & \textbf{0.088}     & \textbf{0.613} \\
{Sequence 4\\(197 m)}                              & {EKF-RIO}                              & 1.737              & 25.885          & 0.118              & 4.074          \\
                                                   & {Ours (-0.115 s)}                      & \textbf{0.610}     & \textbf{3.099}  & \textbf{0.057}     & \textbf{0.944} \\
{Sequence 5\\(190 m)}                              & {EKF-RIO}                              & 2.375              & 7.702           & 0.122              & 1.600          \\
                                                   & {Ours (-0.115 s)}                      & \textbf{1.150}     & \textbf{1.304}  & \textbf{0.069}     & \textbf{0.814} \\
{Sequence 6\\(179 m)}                              & {EKF-RIO}                              & 1.267              & 17.907          & 0.117              & 2.828          \\
                                                   & {Ours (-0.111 s)}                      & \textbf{0.661}     & \textbf{2.551}  & \textbf{0.051}     & \textbf{0.809} \\
{Sequence 7\\(223 m)}                              & {EKF-RIO}                              & 2.757              & 10.092          & 0.116              & 1.863          \\
                                                   & {Ours (-0.112 s)}                      & \textbf{1.596}     & \textbf{6.039}  & \textbf{0.057}     & \textbf{1.365} \\
{Average}                                          & {EKF-RIO}                              & 1.822              & 9.707            & 0.148             & 2.051          \\
                                                   & {Ours (-0.113 s)}                      & \textbf{0.809}     & \textbf{2.388}   & \textbf{0.074}    & \textbf{0.870}   
\end{tblr}
}
\end{table}

Since the radar delay is generally larger than IMU delay, the time offset \( t_d \), representing the difference between these delays, typically takes a negative value. To evaluate the robustness of the estimation, different initial values of \( t_d \) ranging from 0.0 to -0.3 seconds are tested. Figure \ref{sq5} illustrates the estimated time offset for each initial setting, along with the 3-sigma boundaries. As \( t_d \) is estimated from radar ego-velocity, it cannot be determined while the platform is stationary. Once the platform starts moving, the filter begins estimating \( t_d \) and quickly converges to a stable value. The filter converges to a stable time offset of -0.114 ± 0.001 seconds in Sequence 1 and -0.115 ± 0.001 seconds in Sequence 4.

Table \ref{table_self} presents the performance comparison between the proposed method with online temporal calibration and EKF-RIO across seven sequences. The proposed method outperforms EKF-RIO, significantly reducing both APE and RPE across all sequences. Specifically, it reduces APE translation error by an average of 56\%, APE rotation error by 75\%, RPE translation error by 50\%, and RPE rotation error by 58\% compared with EKF-RIO. Despite using the same measurement model, the performance improvement is achieved solely by applying propagation and updates based on a common time stream through the proposed online temporal calibration.

On average, the time offset \( t_d \) is estimated to be -0.113 ± 0.002 seconds, confirming consistent temporal calibration throughout the experiments. Compared with LiDAR-inertial and visual-inertial systems, radar-inertial systems exhibit a significantly larger time offset, as shown in Table~\ref{time_offset_comparison}. Given the radar sensor rate (10 Hz), such a large time offset is significant enough to cause a misalignment spanning more than one data frame. These findings highlight the necessity of temporal calibration in RIO, which is crucial for accurate sensor fusion and reliable pose estimation in real-world applications.

\begin{figure}[t]
\centering
\includegraphics[width=\linewidth]{figure_4.png}
\caption{Time offset estimation with 3-sigma boundaries for different initial values in Sequence 1 and 4.}
\label{sq5}
\end{figure}

\begin{table}[t]
\centering
\caption{Comparison of Time Offset in Multi-Sensor Fusion Systems}
\label{time_offset_comparison}
\begin{tabular}{|c|c|c|} 
\hline
\textbf{Systems} & \textbf{Sensor} & \textbf{Time Offset} \\ 
\hline
LiDAR-Inertial~\cite{10113826} & Velodyne VLP-32 & 0.006 s\\ 
\hline
Visual-Inertial~\cite{li2014online} & PointGrey Bumblebee2 & 0.047 s\\ 
\hline
Radar-Inertial & TI AWR1843BOOST & \textbf{0.113 s} \\
\hline
\end{tabular}
\end{table}

\subsubsection{Open Datasets}
Table \ref{opendataset} presents the results from the two open datasets. The ICINS dataset incorporates a hardware trigger for the radar, which we use to validate the accuracy of the time offset estimation for the proposed method. In this setup, a microcontroller sends radar trigger signals, prompting the radar to begin scanning. The radar data is timestamped based on the actual trigger signal, providing a pseudo-ground truth for time offset estimation. Theoretically, if the sensors are time-synchronized through triggers, the time offset \( t_d \) is expected to be close to 0.0 seconds. The proposed method estimates the time offset to be an average of 0.016 ± 0.003 seconds. Despite this slight discrepancy, the proposed method demonstrates comparable or improved performance on average in both APE and RPE compared with EKF-RIO. Although the ICINS dataset includes hardware-triggered signals for the radar, there is no such trigger signal for the IMU in the dataset, which may introduce a delay in IMU measurements. As defined in Eq.~\eqref{time_offset}, we attribute the estimated positive time offset to this IMU delay, explaining the difference from the expected value.

The ColoRadar dataset, widely used for performance comparison in the RIO field, is utilized to assess if the proposed method generalizes well across different datasets. As shown in Table \ref{opendataset}, the proposed method also demonstrates performance improvements over EKF-RIO in terms of both APE and RPE on average. However, the extent of improvement is smaller compared with the self-collected dataset, which can be explained by differences in trajectory characteristics. The radar ego-velocity model utilizes not only the accelerometer but also the gyroscope measurements. As illustrated in Fig.~\ref{trajectory}, the ColoRadar dataset involves movement over a larger area with less rotation, leading to a smaller impact of the time offset on performance. Nonetheless, the proposed method achieves 33\% reduction in RPE translation error, demonstrating its effectiveness even in this less challenging trajectory. On average, the time offset \( t_d \) is estimated to be -0.111 ± 0.003 seconds, similar to the time offset found in the self-collected dataset. This consistency is likely due to the use of the same radar sensor model in both datasets, further validating the reliability of the proposed method across different environments.

\begin{table}[t]
\centering
\caption{Quantitative Results of Comparison study on Open datasets}
\label{opendataset}
\resizebox{\linewidth}{!}{
\begin{tblr}{
  cells = {c},
  cell{1}{1} = {r=2}{},
  cell{1}{2} = {r=2}{},
  cell{1}{3} = {c=2}{},
  cell{1}{5} = {c=2}{},
  cell{3}{1} = {r=2}{},
  cell{5}{1} = {r=2}{},
  cell{7}{1} = {r=2}{},
  cell{9}{1} = {r=2}{},
  cell{11}{1} = {r=2}{},
  cell{13}{1} = {r=2}{},
  cell{15}{1} = {r=2}{},
  cell{17}{1} = {r=2}{},
  cell{19}{1} = {r=2}{},
  cell{21}{1} = {r=2}{},
  hline{1,3,5,7,9,11,13,15,17,19,21,23} = {-}{},
  hline{2-3} = {3-6}{},
}
{\textbf{Sequence }\\\textbf{(Trajectory Length)}}       & \textbf{Method ($\hat{t}_d$)} & \textbf{APE RMSE}        &                                           & \textbf{RPE RMSE}       &                         \\
                        &                               & Trans. (m)               & Rot. (\degree)                                   & Trans. (m)              & Rot. (\degree)                 \\
                        \hline
{ICINS 1\\(295 m)}      & EKF-RIO (N/A)                 & 1.959                    & 10.694                                    & \textbf{0.093}          & \textbf{0.896}          \\
                        & Ours (0.016 s)                & \textbf{1.922}           & \textbf{10.135}                           & 0.098                   & 0.918          \\
{ICINS 2\\(468 m)}      & EKF-RIO                       & 3.830                    & 23.151                                    & \textbf{0.114}          & 1.289                   \\
                        & Ours (0.013 s)                & \textbf{3.198}           & \textbf{19.235}                           & 0.121                   & \textbf{1.076}          \\
{ICINS 3\\(150 m)}      & EKF-RIO                       & \textbf{1.502}           & \textbf{9.905}                            & 0.130                   & \textbf{1.512}           \\
                        & Ours (0.015 s)                & 1.530                    & 10.189                                    & \textbf{0.126}          & 1.553          \\
{ICINS 4\\(50 m)}       & EKF-RIO                       & \textbf{0.213}           & \textbf{2.091}                            & \textbf{0.076}          & \textbf{0.923}           \\
                        & Ours (0.019 s)                & 0.216                    & 2.098                                     & 0.081                   & \textbf{0.923}          \\
Average                 & EKF-RIO                       & 1.876                    & 11.460                                    & \textbf{0.103}          & 1.155                   \\
                        & Ours (0.016 s)                & \textbf{1.716}           & \textbf{10.414}                           & 0.106                   & \textbf{1.117}          \\
                        \hline
{ColoRadar 1\\(178 m) } & EKF-RIO (N/A)                 & 6.556                    & \textbf{\textbf{1.354}}                   & 0.182                   & \textbf{1.071} \\
                        & Ours (-0.110 s)               & \textbf{\textbf{6.173}}  & 1.382                                     & \textbf{\textbf{0.155}} & 1.188                   \\
{ColoRadar 2\\(197 m) } & EKF-RIO                       & \textbf{\textbf{4.747}}  & 1.238                                     & 0.372                   & 1.375                   \\
                        & Ours (-0.114 s)               & 4.826                    & \textbf{\textbf{0.960}}                   & \textbf{\textbf{0.292}} & \textbf{\textbf{1.180}} \\
{ColoRadar 3\\(197 m) } & EKF-RIO                       & \textbf{\textbf{8.307}}  & 1.969                                     & 0.259                   & 1.015                   \\
                        & Ours (-0.108 s)               & 8.550                    & \textbf{\textbf{1.852}}                   & \textbf{\textbf{0.221}} & \textbf{\textbf{0.879}} \\
{ColoRadar 4\\(144 m) } & EKF-RIO                       & 12.111                   & 2.815                                     & 0.488                   & 1.263                   \\
                        & Ours (-0.112 s)               & \textbf{11.946}          & \textbf{2.756}                            & \textbf{0.200}          & \textbf{1.116} \\
Average                 & EKF-RIO                       & 7.930                    & 1.844                                     & 0.325                   & 1.181                   \\
                        & Ours(-0.111 s)                & \textbf{7.874}           & \textbf{1.737}                            & \textbf{0.217}          & \textbf{1.091}          
\end{tblr}
}
\end{table}