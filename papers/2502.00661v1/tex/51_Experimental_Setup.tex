\subsection{Experimental Setup}
\label{sec: experimental_setup}
\begin{figure}[t]
\centering \includegraphics[width=\linewidth]{figure_2.png} \caption{The handheld platform configuration, including the radar, IMU, and onboard computer. The experiments are conducted in a room equipped with a motion capture system to obtain accurate ground truth.}
\label{fig2}
\end{figure}

We conduct experiments using three datasets, comprising a total of 15 sequences. One is our self-collected dataset, captured with a handheld platform as shown in Fig.~\ref{fig2}, while the other two are public radar datasets: ICINS2021~\cite{9470842}, and ColoRadar~\cite{kramer2022coloradar}. The sensors on our platform include a 4D FMCW radar, specifically the Texas Instruments AWR1843BOOST, and an Xsens MTI-670-DK IMU. No additional hardware triggers are used between the sensors, and the sensor data is recorded using an Intel NUC i7 onboard computer. The experiments are conducted in an indoor area equipped with a motion capture system to obtain precise ground truth. The extrinsic calibration between the IMU and the radar is performed manually. To highlight the significance of temporal calibration in RIO, we design the dataset with two levels of difficulty. Sequences 1 to 3 feature standard motion patterns, while Sequences 4 to 7 introduce more rotational motion to induce larger errors due to the time offset, providing a clearer demonstration of its impact.

\begin{figure*}[t]
\centering
\includegraphics[width=\linewidth]{figure_3.png}
\caption{Comparison of estimated trajectories with the ground truth. The \textcolor{black}{black} trajectory is the ground truth, the \textcolor{blue}{blue} one is the EKF-RIO, which does not account for temporal calibration, and the \textcolor{red}{red} one is the proposed RIO with online temporal calibration. Results are presented for Sequence 4, ICINS 1, and ColoRadar 1, representing one sequence from each of the three datasets.}
\label{trajectory}
\end{figure*}

In~\cite{9470842}, the ICINS2021 dataset is collected using a Texas Instruments IWR6843AOP radar sensor, an Analog Devices ADIS16448 IMU sensor, and a camera. A microcontroller board is used for active hardware triggering to accurately capture the timing of the radar measurements. Data is collected using both handheld and drone platforms. The handheld sequences, ``carried\_1'' and ``carried\_2'', are referred to as ``ICINS 1'' and ``ICINS 2'', while the drone sequences, ``flight\_1'' and ``flight\_2'', are referred to as ``ICINS 3'' and ``ICINS 4'', respectively. The ground truth is provided through visual-inertial SLAM, which performs multiple loop closures, offering a pseudo-ground truth. In~\cite{kramer2022coloradar}, the ColoRadar dataset is collected using a Texas Instruments AWR1843BOOST radar sensor, a Microstrain 3DM-GX5-25 IMU sensor, and a LiDAR mounted on a handheld platform. No specific synchronization setup is used between the sensors. The sequences, ``arpg\_lab\_run0'' and ``arpg\_lab\_run1'', are referred to as ``ColoRadar 1'' and ``ColoRadar 2'', while the sequences ``ec\_hallways\_run0'' and ``ec\_hallways\_run1'' are referred to as ``ColoRadar 3'' and ``ColoRadar 4'', respectively. The ground truth is generated via LiDAR-inertial SLAM, which includes loop closures, offering a pseudo-ground truth.