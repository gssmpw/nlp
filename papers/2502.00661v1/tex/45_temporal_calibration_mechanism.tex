\subsection{Online Temporal Calibration}
\label{sec: online temporal calibration}
The proposed method estimates the time offset between the IMU and the radar in real-time by employing the radar ego-velocity measurement model. By accounting for the time offset, the proposed method ensures that both propagation and measurement updates are performed based on a common time stream, ensuring that the measurements from both sensors are synchronized.

The time offset is propagated using a noise model \( n_d \), as described in Eq.~\eqref{ekf}. If the time offset is constant over time or approximately known, it can be estimated without a noise model. However, the time offset varies across sensor models, making it difficult to predefine in most cases. Furthermore, when the vehicle is moving slowly, the impact of the time offset becomes less significant, making it harder to estimate. For this reason, the time offset is modeled as a random walk.

In the measurement model presented in Eq.~\eqref{ego_vel}, the factors affected by the temporal misalignment between sensors are not only the state variables \( {}^G\mathbf{R}_I \), \( {}^G\mathbf{v}_I \), and \( \mathbf{b}_g \), but also the gyroscope measurement \( {}^I\boldsymbol{\omega}_I^m \). Although \( \mathbf{b}_g \), which does not change significantly over time, is negligible, failing to account for the time offset causes the state to propagate over a misaligned time stream, leading to errors in the estimates of \( {}^G\mathbf{R}_I \) and \( {}^G\mathbf{v}_I \). Moreover, the improper use of the gyroscope measurement further degrades the estimation accuracy, and this error accumulates over time. Therefore, accounting for the time offset is crucial to maintain the accuracy and consistency of the state estimates.