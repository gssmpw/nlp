\subsection{System Overview}
\label{sec: system overview}
Figure~\ref{fig1} illustrates the temporal misalignment between IMU and radar streams, along with the corresponding execution of EKF’s propagation and update steps. The upper plot shows the actual time when the event was captured by the sensors, while the lower plot represents the recorded time of the sensor measurement. Each sensor measures an actual event at a certain time, but due to delays (i.e., $t_{d,IMU}$ and $t_{d,Radar}$), the sensor measurement reflects a later time. The time offset $t_d$ represents the difference between the delays of the IMU and the radar, defined as:
\begin{equation}
\label{time_offset}
    t_d = t_{d,IMU} - t_{d,Radar}.
\end{equation}
Since the radar typically has a larger delay than the IMU, $t_d$ generally takes a negative value.

Traditional EKF-based RIO performs propagation using IMU measurements until the radar measurement arrives, at which point the system executes the measurement update based on the times recorded in the sensor measurements. To ensure accurate state estimation, it is crucial to align the sensor measurements from both the IMU and the radar to a common time stream. While the exact delays of individual sensors are difficult to determine, the time offset $t_d$ can be estimated in real-time using the radar ego-velocity, allowing the system to adjust the radar measurement to align with the IMU measurement time stream, which serves as the common time reference. By leveraging temporal calibration, the proposed RIO enables propagation and measurement updates to be performed based on a common time stream.

The system state and its representation are explained in Section~\ref{sec: system state}. In Section~\ref{sec: Propagation}, we cover the propagation using the IMU, and in Section~\ref{sec: measurement update}, the radar measurement update is discussed. The online temporal calibration is detailed in Section~\ref{sec: online temporal calibration}.