\subsection{System State}
\label{sec: system state}
At time step \( k \), the system state is defined as:
\begin{equation}
    \mathbf{x}^k = 
    \left(
    {}^G\mathbf{q}_I^{k\top} \quad 
    \mathbf{b}_g^{k\top} \quad 
    {}^G\mathbf{v}_I^{k\top} \quad 
    \mathbf{b}_a^{k\top} \quad
    {}^G\mathbf{p}_I^{k\top} \quad
    {t}_d^k
    \right)^\top,
\end{equation}
where ${}^G\mathbf{q}_I$ represents the attitude, ${}^G\mathbf{v}_I$ the velocity, and ${}^G\mathbf{p}_I$ the position of the IMU. The terms $\mathbf{b}_g$ and $\mathbf{b}_a$ represent the gyroscope and accelerometer biases, respectively, and ${t}_d$ represents the time offset defined in Eq.~\eqref{time_offset}.

The error state formulation, as highlighted in \cite{sola2017quaternion}, minimizes errors and avoids parameter singularities. Given the estimated state $\hat{\mathbf{x}}$ and the error state $\tilde{\mathbf{x}}$, the true state $\mathbf{x}$ is expressed as:
\begin{equation}
    \mathbf{x} = \hat{\mathbf{x}} + \tilde{\mathbf{x}}.
\end{equation}
The true quaternion $\mathbf{q}$ is represented as a combination of the estimated quaternion $\hat{\mathbf{q}}$ and the error quaternion $\tilde{\mathbf{q}}$ as $\mathbf{q} = \hat{\mathbf{q}} \otimes \tilde{\mathbf{q}}$, where $\otimes$ denotes quaternion multiplication. The error quaternion $\tilde{\mathbf{q}}$ is approximated by $\tilde{\mathbf{q}} \approx \begin{bmatrix} 1 & \frac{1}{2} \boldsymbol{\theta}^\top \end{bmatrix}^\top$, with $\boldsymbol{\theta}$ representing a small Euler angle error.

Then, the error state at time step $k$ is similarly defined as:
\begin{equation}
    \tilde{\mathbf{x}}^k = 
    \left(
    {}^G\bm{\theta}_I^{k\top} \quad 
    \tilde{\mathbf{b}}_g^{k\top} \quad 
    {}^G\tilde{\mathbf{v}}_I^{k\top} \quad 
    \tilde{\mathbf{b}}_a^{k\top} \quad
    {}^G\tilde{\mathbf{p}}_I^{k\top} \quad
    \tilde{t}_d^k
    \right)^\top.
\end{equation}
For simplicity, the time index \(k\) is omitted in the following equations.