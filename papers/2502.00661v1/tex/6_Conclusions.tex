\section{Conclusions and Future Work}
\label{sec: conclusion}
In this paper, we proposed an EKF-based RIO framework with online temporal calibration. To ensure accurate sensor time synchronization during IMU and radar sensor fusion, the time offset between sensors is estimated from radar ego-velocity, which is derived from a single radar scan. This approach avoids the potential risks of finding correspondences between consecutive radar scans and, being independent of radar point cloud density, offers flexibility for use with various types of radar sensors. By leveraging temporal calibration, sensor measurements are aligned to a common time stream. This allows propagation and measurement updates to be applied at the correct time, improving overall performance. Extensive experiments across multiple datasets demonstrate the effectiveness of time offset estimation and provide a detailed analysis of its impact on overall performance.

Several challenges remain in multi-sensor fusion state estimation using radar systems. One issue is the reliance on manually calibrated sensor extrinsic parameters in many studies, which can lead to inaccuracies. We will focus on spatiotemporal calibration between sensors to further improve the accuracy and robustness of multi-sensor fusion systems.