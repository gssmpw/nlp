\subsection{Propagation with IMU Measurements}
\label{sec: Propagation}

The continuous-time dynamics for the estimated state is expressed as follows:
\begin{equation}
\label{propagation}
    \begin{gathered}
        {}^G\dot{\hat{\mathbf{q}}}_I = \frac{1}{2} \mathbf{\Omega}({}^I\hat{\boldsymbol{\omega}}_I){}^G\hat{\mathbf{q}}_I, \quad
        \dot{\hat{\mathbf{b}}}_g = \mathbf{0}_{3 \times 1}, \\
        {}^G\dot{\hat{\mathbf{v}}}_I = {}^G\hat{\mathbf{R}}_I{}^I\hat{\mathbf{a}}_I + {}^G\mathbf{g}, \quad
        \dot{\hat{\mathbf{b}}}_a = \mathbf{0}_{3 \times 1}, \\
        {}^G\dot{\hat{\mathbf{p}}}_I = {}^G\hat{\mathbf{v}}_I, \quad
        \dot{\hat{t}}_d = 0,
    \end{gathered}
\end{equation}
where ${}^G\mathbf{g}$ represents the gravity vector in the global frame.
The estimated angular velocity ${}^I\hat{\boldsymbol{\omega}}_I$ and acceleration ${}^I\hat{\mathbf{a}}_I$ are expressed as ${}^I{\hat{\boldsymbol{\omega}}}_I = {}^I{\boldsymbol{\omega}}_I^m - \hat{\mathbf{b}}_g$ and ${}^I\hat{\mathbf{a}}_I = {}^I{\mathbf{a}}_I^m - \hat{\mathbf{b}}_a$, where ${}^I{\boldsymbol{\omega}}_I^m$ and ${}^I{\mathbf{a}}_I^m$ denote the gyroscope and accelerometer measurements, respectively, in the IMU frame. The matrix $\mathbf{\Omega}(\hat{\boldsymbol{\omega}})$, constructed from the estimated angular velocity $\hat{\boldsymbol{\omega}}$ and its skew-symmetric matrix $\lfloor \hat{\boldsymbol{\omega}} \times \rfloor$, is represented as: 
\begin{equation}
    \mathbf{\Omega}(\hat{\boldsymbol{\omega}}) = 
    \begin{bmatrix}
        0 & -\hat{\boldsymbol{\omega}}^\top \\
        \hat{\boldsymbol{\omega}} & -\lfloor \hat{\boldsymbol{\omega}} \times \rfloor
    \end{bmatrix}.
\end{equation}
The estimated state $\hat{\mathbf{x}}$ is propagated with IMU measurements through the continuous-time dynamics in Eq.~\eqref{propagation}, using 4\textsuperscript{th}-order Runge-Kutta numerical integration.

For the covariance propagation, the linearized continuous-time dynamics for the error state is expressed as:
\begin{equation}
\label{ekf}
    \dot{\tilde{\mathbf{x}}} = \mathbf{F}\tilde{\mathbf{x}} + \mathbf{G}\mathbf{n},
\end{equation}
where $\mathbf{n} = \left( \mathbf{n}_g^\top, \mathbf{n}_{wg}^\top, \mathbf{n}_a^\top, \mathbf{n}_{wa}^\top, n_d \right)^\top$. The noise vectors $\mathbf{n}_g$ and $\mathbf{n}_a$ represent the Gaussian noise affecting the gyroscope and accelerometer measurements, respectively. Similarly, $\mathbf{n}_{wg}$ and $\mathbf{n}_{wa}$ correspond to the random walks for the gyroscope and accelerometer measurement biases. The term $n_d$ accounts for the Gaussian noise (i.e., uncertainty) in the time offset.

The matrix $\mathbf{F}$ represents the linearized system dynamics, and the matrix $\mathbf{G}$ models the influence of the process noise on the error state. Only the non-zero elements of the matrix $\mathbf{F}$ are given as follows:
\begin{equation}
    \begin{aligned}
        \mathbf{F}(0:2, 0:2) &= -\lfloor ({}^I{\boldsymbol{\omega}}_I^m - \hat{\mathbf{b}}_g) \times \rfloor, \\
        \mathbf{F}(0:2, 3:5) &= -\mathbf{I}_{3}, \\
        \mathbf{F}(6:8, 0:2) &= -{}^G\hat{\mathbf{R}}_I \lfloor ({}^I{\mathbf{a}}_I^m - \hat{\mathbf{b}}_a) \times \rfloor, \\
        \mathbf{F}(6:8, 9:11) &= -{}^G\hat{\mathbf{R}}_I, \\
        \mathbf{F}(12:14, 6:8) &= \mathbf{I}_{3}.
    \end{aligned}
\end{equation}
Similarly, the non-zero elements of the matrix $\mathbf{G}$ are given as follows:
\begin{equation}
    \begin{aligned}
        \mathbf{G}(0:2, 0:2) &= -\mathbf{I}_{3}, \\
        \mathbf{G}(3:5, 3:5) &= \mathbf{I}_{3}, \\
        \mathbf{G}(6:8, 6:8) &= -{}^G\hat{\mathbf{R}}_I, \\
        \mathbf{G}(9:11, 9:11) &= \mathbf{I}_{3}, \\
        \mathbf{G}(15, 12) &= 1.
    \end{aligned}
\end{equation}

To propagate the covariance, the discrete-time state transition matrix $\mathbf{\Phi}_k$ and discrete-time process noise covariance matrix $\mathbf{Q}_k$, derived from Eq.~\eqref{ekf}, are defined as follows:
\begin{equation}
    \begin{gathered}
        \mathbf{\Phi}_k = \mathbf{\Phi}(t_{k+1}, t_k) = \exp{\left( \int_{t_k}^{t_{k+1}} \mathbf{F}(\tau) d\tau \right)}, \\
        \mathbf{Q}_k = \int_{t_k}^{t_{k+1}} \mathbf{\Phi}(t_{k+1}, \tau) \mathbf{G}\mathbf{Q}\mathbf{G}^\top \mathbf{\Phi}(t_{k+1}, \tau)^\top d\tau,
    \end{gathered}
\end{equation}
where $\mathbf{Q}$ is the continuous-time process noise covariance matrix. The propagated covariance matrix is expressed as:
\begin{equation}
    \mathbf{P}_{k+1|k} = \mathbf{\Phi}_k \mathbf{P}_{k|k} \mathbf{\Phi}_k^\top + \mathbf{Q}_k.
\end{equation}
Considering the time offset \( t_d \), propagation is repeated up to just before the time of the radar measurement aligned to the IMU measurement time stream.