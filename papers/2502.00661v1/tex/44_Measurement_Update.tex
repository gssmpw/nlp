\subsection{Measurement Update with Radar Measurements}
\label{sec: measurement update}
The FMCW 4D radar provides 3D point clouds, where each point includes a 3D position and a scalar Doppler velocity. The Doppler velocity represents the radial velocity of a target point, expressed as:
\begin{equation}
    v_d^i = -{}^R\mathbf{v}_R \cdot \frac{{}^R\mathbf{p}_f^i}{||{}^R\mathbf{p}_f^i||}, \quad \text{for } i = 1, \dots, N,
\end{equation}
where \( N \) is the total number of detected points, \( v_d^i \) denotes the Doppler velocity of the \( i \)-th point, \( {}^R\mathbf{p}_f^i \) is the position vector of the \( i \)-th point in the radar frame, and ${}^R\mathbf{v}_R$ denotes the radar ego-velocity. To estimate the radar ego-velocity \( {}^R\mathbf{v}_R \) from noisy radar measurements, various methods, such as RANSAC and m-estimator-based optimization, have been proposed. In this work, we adopt the 3-point RANSAC-LSQ \cite{9235254}, a simple yet robust method that efficiently eliminates outliers and estimates the radar ego-velocity \( {}^R\mathbf{v}_R \), which is used in the measurement update.

In the case where the IMU and radar are rigidly connected, the radar ego-velocity measurement model can be expressed using the system state. As derived in \cite{9235254}, the radar ego-velocity is expressed as:
\begin{equation}
\label{ego_vel}
    \begin{aligned}
        {}^R\mathbf{v}_R(t) =& {}^R\mathbf{R}_I \left( {}^G\mathbf{R}_I^\top(t) {}^G\mathbf{v}_I(t) \right. \\
        &+ \left. \lfloor ({}^I\boldsymbol{\omega}_I^m(t) - \mathbf{b}_g(t)) \times \rfloor {}^I\mathbf{p}_R \right),
    \end{aligned}
\end{equation}
where the extrinsic parameters, \( {}^R\mathbf{R}_I \) and \( {}^I\mathbf{p}_R \), between the IMU and the radar are assumed to be pre-calibrated and constant.

In the measurement update, the residual \( \mathbf{r} \) is computed as the difference between the radar ego-velocity \( {}^R\mathbf{v}_R \), estimated from the radar measurements, and the predicted radar ego-velocity \( {}^R\hat{\mathbf{v}}_R \) from the state. The residual is expressed as:
\begin{equation}
\label{residual}
    \mathbf{r} = {}^R\mathbf{v}_R(t) - {}^R\hat{\mathbf{v}}_R(t') = \mathbf{h}(\tilde{\mathbf{x}}) + \mathbf{n}_r.
\end{equation}
As illustrated in Fig.~\ref{fig1}, \( t' = t + t_d \) represents the IMU measurement time, which serves as the reference time stream in the filter, while \( t \) represents the radar measurement time used in the measurement update after being aligned to the IMU time stream. The term \( \mathbf{n}_r \) denotes the noise of the measurement. The function \( \mathbf{h}(\tilde{\mathbf{x}}) \) is a nonlinear function that relates the state error \( \tilde{\mathbf{x}} \) to the radar ego-velocity measurement residual. For use in the EKF, this function is linearized with respect to the system state. The measurement Jacobian matrix \( \mathbf{H} \) is expressed as follows:
\begin{equation}
\begin{aligned}
    \mathbf{H} &= 
    \begin{bmatrix}
        \mathbf{H}_q & \mathbf{H}_{b_g} & \mathbf{H}_v & \mathbf{0}_{3 \times 3} & \mathbf{0}_{3 \times 3} & \mathbf{H}_{t_d}
    \end{bmatrix}, \\
    \mathbf{H}_q &= {}^R\mathbf{R}_I \lfloor {}^G\hat{\mathbf{R}}_I^\top {}^G\hat{\mathbf{v}}_I \times \rfloor,\\
    \mathbf{H}_{b_g} &= {}^R\mathbf{R}_I \lfloor {}^I\mathbf{p}_R \times \rfloor,\\
    \mathbf{H}_v &= {}^R\mathbf{R}_I {}^G\hat{\mathbf{R}}_I^\top.
\end{aligned}
\end{equation}
In Eq.~\eqref{ego_vel}, the time-varying states are \( {}^G\mathbf{R}_I \) and \( {}^G\mathbf{v}_I \). By applying the chain rule, the Jacobian \( \mathbf{H}_{t_d} \) is expressed as:
% \begin{equation}
% \begin{aligned}
%     \mathbf{H}_{t_d} &= \mathbf{H}_q \left( {}^G\hat{\mathbf{R}}_I ( {}^I{\boldsymbol{\omega}}_I^m - \hat{\mathbf{b}}_g )\right) \\
%     & + \mathbf{H}_v \left( {}^G\hat{\mathbf{R}}_I ({}^I{\mathbf{a}}_I^m - \hat{\mathbf{b}}_a) + {}^G{\mathbf{g}} \right).
% \end{aligned}
% \end{equation}
\begin{equation}
\begin{aligned}
\mathbf{H}_{t_d} =& \frac{\partial \mathbf{h}\left(\tilde{\mathbf{x}}\left(\tilde{t'}\right)\right)}{\partial {}^G\boldsymbol{\theta}_I\left(\tilde{t'}\right)} 
\cdot \frac{\partial {}^G\boldsymbol{\theta}_I\left(\tilde{t'}\right)}{\partial \tilde{t'}} 
\cdot \frac{\partial \tilde{t'}}{\partial \tilde{t}_d} \\
&+ \frac{\partial \mathbf{h}\left(\tilde{\mathbf{x}}\left(\tilde{t'}\right)\right)}{\partial {}^G\tilde{\mathbf{v}}_I\left(\tilde{t'}\right)} 
\cdot \frac{\partial {}^G\tilde{\mathbf{v}}_I\left(\tilde{t'}\right)}{\partial \tilde{t'}} 
\cdot \frac{\partial \tilde{t'}}{\partial \tilde{t}_d} \\
=& \mathbf{H}_q\left( {}^G\hat{\mathbf{R}}_I\left(t'\right) \left({}^I\boldsymbol{\omega}_I^m\left(t'\right) - \hat{\mathbf{b}}_g\left(t'\right) \right)\right) \\
&+ \mathbf{H}_v\left( {}^G\hat{\mathbf{R}}_I\left(t'\right) \left({}^I\mathbf{a}_I^m\left(t'\right) - \hat{\mathbf{b}}_a\left(t'\right) \right) + {}^G\mathbf{g} \right).
\end{aligned}
\end{equation}

The EKF update proceeds by computing the Kalman gain \( \mathbf{K} \) as:
\begin{equation}
    \mathbf{K} = \mathbf{P}_{k+1|k} \mathbf{H}^\top \left( \mathbf{H} \mathbf{P}_{k+1|k} \mathbf{H}^\top + \mathbf{R} \right)^{-1},
\end{equation}
where \( \mathbf{R} \) represents the measurement noise covariance matrix. Finally, the estimated state and covariance are updated according to the Kalman gain as follows:
\begin{equation}
\begin{aligned}
    \hat{\mathbf{x}}_{k+1|k+1} &= \hat{\mathbf{x}}_{k+1|k} + \mathbf{K} \mathbf{r}, \\
    \mathbf{P}_{k+1|k+1} &= \left( \mathbf{I} - \mathbf{K} \mathbf{H} \right) \mathbf{P}_{k+1|k}.
\end{aligned}
\end{equation}
Each time new radar measurement is received, the measurement update is performed based on the IMU time stream.