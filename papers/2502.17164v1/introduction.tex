\section{Introduction}
%
\label{sec: introduction}

% big picture of a/c coupling

Crystalline defects, such as vacancies, dislocations, and cracks, play a crucial role in determining mechanical behavior of the materials~\cite{1996_AC_Ana_Solid_Defect_PMA,2009_Miller_Tadmor_Unified_Framework_Benchmark_MSMSE,2020_EG_Roadmap_Multiscale_Modeling_MSMSE,2012_Tadmor_Material_Con_Ato_Multiscale, chen2022qm, wang2021posteriori, chen2019adaptive}. While the defect core undergoes significant lattice distortions, the far-field region can be effectively modeled as an elastic force field~\cite{2020_JB_MD_CO_Thermo_Limit_Trans_Rate_Crystalline_Defect_ARMA,1984_Daw_Baskes_EAM_PRB, braun2022higher, olson2023elastic}. By leveraging this distinction, concurrent multiscale methods form an important class of methodology to simulating such systems. In order to achieve a balance between accuracy and computational efficiency, these methods decompose the computational domain into a small core region , which is modeled with detailed atomistic scale approaches such as molecular mechanics, and a large far-field region, which is often described by macroscopic continuum elasticity that can be further discretized by the finite element method to reduce the total degrees of freedoms (DoFs).
%(the atomistic region)
%(the continuum region)
%with domain decomposition 






Concurrent multiscale methods may date back to 1980's and it was in \cite{1996_AC_Ana_Solid_Defect_PMA} that the Cauchy-Born approximation was introduced to the methods to initiate the atomisti-to-continuum (a/c) coupling methods, which are also well known as the quasicontinuum (QC) methods among the engineering community \cite{2003_RM_ET_QCM_JCAMD,2009_Miller_Tadmor_Unified_Framework_Benchmark_MSMSE,2013_ML_CO_AC_Coupling_ACTANUM, fu2023adaptive, fu2025meshac, wang2025posteriori}. Cauchy-Born approximation is essentially the continuum elasticity model derived the from a molecular mechanics model by certain homogenization process \cite{2002_PL_QCL_1D_MATHCOMP,2007_WE_PM_Cauchy_Born_Rule_and_The_Stability_ARMA,2007_PL_QCL_2D_SIAMNUM,2012_TH_CO_CB_Stability_Bravi_Lattice_M2NA, wang2021priori}. It is an important step in concurrent multiscale modeling to guarantee that the models of different scales produce the same result under elastic deformation, which is also known as the first level consistency of the method \cite{2012_WE_Principles_Multi_Model}. 

It was later discovered that the energy-based a/c coupling methods suffer from the so-called ``ghost force" at the interface between the regions where models of different scales transit \cite{1999_VS_RM_ETadmor_MOrtiz_AFEM_QC_JMPS}. Such phenomenon arises due to the nonlocality of the molecular mechanics model and the Cauchy-Born approximation which is continuum and thus local in nature. This is also known as the second level of consistency \cite{2012_WE_Principles_Multi_Model}. Considerable effort has been made in the recent two decades for the development and the mathematical analysis of the consistent a/c coupling methods so that the interface error is controllable and will not ruin the overall accuracy of the solution. Shimokawa et al. first proposed the Quasi-Nonlocal (QNL) method \cite{2004_Shimokawa_QCM_ErrAna_PRB}, which eliminates the "ghost force" by reconstructing the energy of atoms near the interface. E, Lu and Yang proposed the Geometric Reconstruction (GRC) method \cite{2006_WE_JL_ZY_GRC_PRB}, which is used to eliminate the "ghost force" in higher-dimensional cases. Ming and Yang, Dobson and Luskin conducted error analysis on the QNL method \cite{2009_PM_ZY_1D_QC_Nonlocal_MMS,2009_MD_ML_Optimal_Order_SIMNUM}. Ortner, Ortner and Wang performed a priori and a posteriori error analysis on the QNL and QNL with coarse-graining methods based on negative norms \cite{2011_CO_1D_QNL_MATHCOMP,2011_CO_HW_QC_A_Priori_1D_M3AS}. Ortner and Zhang proposed the two-dimensional geometric reconstruction-based consistent atomistic/continuum (GRAC) method~\cite{2012_CO_LZ_GRAC_Construction_SIAMNUM,2014_CO_LZ_GRAC_Coeff_Optim_CMAME}, which extends the QNL method to two-dimensional cases. We refer to \cite{2013_ML_CO_AC_Coupling_ACTANUM} for a thorough review of the QNL method, and we will follow its theoretical framework to complete some of the proofs in this paper.

The a/c methods improves the computational efficiency compared with the full molecular mechanics simulations in terms of the substantial reduced DoFs as a result of the finite element discretization of the continuum elasticity model. Therefore existing works concentrate mainly on the convergence rate for different a/c methods, which illustrate that for a given error how many DoFs are required or vice versa \cite{1980_MB_CM_WW_Dislocation_Near_Crack_JM,1982_MM_MD_Crack_Boundary_PMA,1989_HF_HE_MP_Metals_Composite_Materials_ZM,1991_SK_PG_HF_Crack_FEM_PMA} (cf. Figure \ref{fig: convergence_QNL_QNLL_alpha12_CG}, \ref{fig: convergence_QNL_QNLL_alpha10_CG} and \ref{fig: convergence_QNL_QNLL_alpha08_CG} in the current paper). However, the molecular mechanics model we employ in the core region is usually highly nonlinear and thus is the corresponding Cauchy-Born approximation in the elastic far field as well as the a/c coupling model in whole. The high nonlinearity of the a/c model may still results in an excessive computational cost, especially in large scale simulations, which motivates this study. 

% This leads to the fact that the a/c model we need to solve is  in large scale simulations
%
%The evaluation and comparison of the a/c methods are often done by error-DoFs curves in existing works \cite{XXXX}.  
%
%The atomisti-to-continuum (a/c) coupling methods, which also well known as the quasicontinuum (QC) methods among the engineering community \cite{1996 QC, 2003 QC review, 2009 benchmark, 2013 Acta Numerical Review}, are typical concurrent multiscale methods that couple molecular mechanics models or atomistic models with their corresponding Cauchy-Born approximations \cite{2006 Cauchy-Born E and Ming, 2012? Cauchy-Born Ortner and Theil}. It is important that Cauchy-Born approximations are 

The current work focuses on developing and analyzing an energy-based a/c coupling methods that integrates the classic (nonlinear) quasinonlocal (QNL) method \cite{2004_Shimokawa_QCM_ErrAna_PRB,2011_CO_1D_QNL_MATHCOMP,2012_XL_ML_Finite_Range_QNL_IMANUM} with the linearized Cauchy-Born approximation, which we name as the QNLL method. The purpose of such development is to enhance computational efficiency of the a/c method through the proper application of linearized model in the elastic far-field region while retain the same level of accuracy as the fully nonlinear coupling scheme. We then perform a rigorous {\it a prior} analysis of the QNLL method. We identify that the optimal rate of convergence of the method can be achieved by choosing certain parameters according to the regularity of the defect equilibrium. In particular, by balancing the lengths of the regions where different models apply and the size of the finite element discretization, we obtain a sharp error estimate of the QNLL method which retain the same order of convergence as that of the fully nonlinear QNL method. We validate our theoretical results through a series of numerical experiments. In addition, our numerical experiments demonstrate that QNLL achieve a substantial improvement of computational efficiency compared with the fully nonlinear QNL method in terms of CPU times.

%Finally, we validate the performance of the QNLL method through numerical experiments, demonstrating its 

%depends on the regularity of the defect equilibrium (decay hypothesis in one dimension). 
%
%Based on this observation
%
%, we introduce a strategy for controlling the convergence order by balancing the lengths of different regions in the model—computational, atomistic, nonlinear continuum, and linear continuum—along with the coarse graining in finite element regime. 
%
%This sharp estimate and error balance enable us to align the convergence order of the QNLL method with that of the QNL method. 



%preserve the accuracy of nonlinear models while enhancing computational efficiency through the use of linear models in the far-field region.
%
%. 

%Existing approaches predominantly couple either fully nonlinear or fully linear atomistic interaction potentials with their continuum approximations derived from the Cauchy-Born rule~\cite{2007_WE_PM_Cauchy_Born_Rule_and_The_Stability_ARMA,2013_CO_FT_Cauchy_Born_ARMA}. However, there is a lack of methods that couple nonlinear a/c models with linear elasticity models in the continuum region. 
%In this work, we propose a new coupling energy formulation by integrating the nonlinear Quasinonlocal (QNL) method~\cite{2009_PM_ZY_1D_QC_Nonlocal_MMS} with the linearized Cauchy-Born model~\cite{2012_WE_Principles_Multi_Model}, which we call the QNLL method. 



%This motivates our work to combine nonlinear atomistic models with linear elasticity models, aiming to preserve the accuracy of nonlinear models while enhancing computational efficiency through the use of linear models in the far-field region.

%这一部分可以放到最后的Conclusion and Future Works里来讲,那里可以引出FBC方法。Our work differs from~\cite{2017_AD_CO_HWu_AC_with_Boundary_Element_ArXiv}, which focuses on a sequential method and provides only a consistency analysis without addressing stability. Specifically, the model in \cite{2017_AD_CO_HWu_AC_with_Boundary_Element_ArXiv} is based on the two-dimensional geometric reconstruction-based consistent atomistic/continuum (GRAC) model~\cite{2012_CO_LZ_Construct_Consistency_ACM_SIAMJNA}, incorporating a linear elasticity approximation in the far-field region. By analyzing the decay properties of displacement around point defects, the study balances four types of errors: modeling error, finite element error, boundary element error, and linearization error. The primary conclusion is that the linear approximation error in the far-field region can be controlled by appropriately adjusting the size of the computational domain. However, the study does not investigate whether incorporating a linear continuum model in the far-field region can reduce computational costs, which is the key aspect our work aims to explore.



%To be precise, the main contributions of this work can be summarized as follows:
%
%\begin{itemize}
%    \item We propose an energy-based coupling for the QNL method with a linearized Cauchy-Born model, dividing the continuum region into nonlinear and linear regions. To further reduce computational cost, we introduce finite element coarse graining in the continuum region and propose the coarse-grained QNLL model.
%    \item We give a rigorous {\it a priori} analysis for the energy-based QNLL method and its coarse-graining scheme. We provide error estimates and analyze the impact of the defect's decay behavior on the convergence order.
%    \item We propose a method to control the convergence order by balancing the lengths of the regions to match that of the QNL method. For the QNLL method with coarse graining, we also balance the finite element mesh size, ensuring consistency in convergence order between the QNLL and QNL methods, with or without coarse graining.
%    \item We demonstrate the computational efficiency of the QNLL method, with and without coarse graining, by comparing its computational time to the QNL method. Even in one-dimensional lattice systems, the introduction of the linear model reduces computational cost, with the QNLL method requiring only $60\%$ to $80\%$ of the time needed by the QNL method.
%\end{itemize}

% The first contribution is that we propose a coupling energy form for the QNL method with a linearized Cauchy-Born model. We divide the continuum region of the one-dimensional QNL model into a nonlinear continuum region and a linear continuum region. Due to the decay of the displacement field of defects, we set the region farther from the defect core as the linear region. Nonlinear continuum models are applied in the nonlinear region, while linear continuum models are used in the linear region. To further reduce computational cost, we choose to use the finite elements coarse graining method to decrease the degrees of freedom in the continuum region.

% The second contribution is that we give a priori analysis for the QNLL method, performed with and without coarse graining. We demonstrate the existence of local minimizer solutions for the energy-based QNLL method and its coarse-graining modification. We use the classical theoretical framework  \cite{2013_ML_CO_AC_Coupling_ACTANUM} to divide the consistency error into truncation, modeling, and linearization errors. In addition, we conduct a stability analysis for the QNLL model without coarse graining. After combining consistency anysis and stability result, we provide a priori analysis for the QNLL method without coarse graining using the inverse function theorem. For the QNLL method with coarse graining, we complete consistency error estimate. After applying the inverse function theorem, we also provide a priori analysis for QNLL method with coarse graining. We observe that the convergence order of the error estimate closely relates to the defect's decay behavior.

% The third contribution is that we propose a method to control the convergence order of our methods by balancing the lengths of the regions to be consistent with that of the QNL method. For the QNLL method without coarse graining, the variables we need to balance are the lengths of the computational region, the atomistic region, the nonlinear continuum region, and the linear continuum region. For the QNLL method with coarse graining, we need to include the finite element mesh size as a variable which needs to balance. By making a quasi-optimal choice of these variables, we can control the convergence order of the QNLL method for both cases with and without coarse graining to be consistent with that of the QNL method. We also construct the relevant numerical experiments that, after applying our balance method, the convergence order of the QNLL methods with or without coarse graining both can be consistent with that of the QNL(Reflection) method in \cite{2013_ML_CO_AC_Coupling_ACTANUM}. Moreover, based on our numerical result, the error of the QNLL method is close to that of the QNL method.

% The final contribution is that we show the computational efficiency of the QNLL method through comparing the computational time of the QNLL method with the QNL method. 
% %We introduce linear model to further reduce computational costs while performing continuum approximation on atomistic systems. 
% We conduct computational time experiments for both cases with and without coarse graining. We found that regardless of whether the degrees of freedom in the continuum region were reduced (coarse graining), the QNLL method modifies computational efficiency. Based on the numerical results in this paper, even in one-dimensional lattice systems with lower computational complexity, the introduction of the linear model significantly reduces computational time. According to the numerical results, the computational time of the QNLL method is only $60\%$ to $80\%$ of that required by the QNL method.

To set out our ideas in a clear way, we consider a one dimensional atomistic system with nearest and next-nearest neighbor multibody interactions. However, we note that the error analysis and balancing techniques proposed are general and are expected to be extended to higher dimensions problems and other fully nonlinear a/c coupling methods with finite interaction ranges~\cite{wang2024theoretical, ortner2023framework}. This study serves as a proof of concept, laying the foundation for possible further extension, which are discussed in the conclusion.

\subsection{Outline}
\label{sec: outline}

The paper is organized as follows. Section~\ref{sec: qnll_model} introduces the atomistic model, the quasinonlocal (QNL) method, and the QNL method with the linearized Cauchy-Born (QNLL) coupling scheme proposed in this work. In Section~\ref{sec: anal_qnll_ncg}, we present an {\it a priori} error analysis for the QNLL method without coarse graining and propose a balancing method to adjust the lengths of various regions based on the error estimate. Numerical experiments are provided to validate the balancing method and highlight the computational efficiency of the QNLL method. Section~\ref{sec: qnll_cg} extends the analysis to the QNLL method with coarse graining in the continuum region. We provide an error estimate and propose a balancing method that includes the adjustment of the finite element mesh length. Numerical experiments demonstrate the effectiveness of this method and show that, despite the reduced degrees of freedom due to coarse graining, the QNLL method can still retain its computational efficiency. Section~\ref{sec: conclusion}  concludes the paper and outlines potential future directions. For clarity and brevity, some detailed proofs are provided in the appendices.



\subsection{Notations}
\label{sec: notations}
We use the symbol $\langle\cdot,\cdot\rangle$ to denote an abstract duality
pairing between a Banach space and its dual. The symbol $|\cdot|$ normally
denotes the Euclidean or Frobenius norm, while $\|\cdot\|$ denotes an operator
norm. We denote $A\backslash\{a\}$ by
$A\backslash a$, and $\{b-a~\vert ~b\in A\}$ by $A-a$. Directional derivatives are denoted by $\nabla \rho f := \nabla f\cdot\rho,\, \rho \in \mathbb{R}$. For $E \in C^2(X)$, the first and second variations are denoted by
$\langle\delta E(u), v\rangle$ and $\langle\delta^2 E(u) v, w\rangle$ for $u,v,w\in X$.

We write $|A| \lesssim B$ if there exists a constant $C$ such that $|A|\leq CB$, where $C$ may change from one line of an estimate to the next. When estimating rates of decay or convergence, $C$ will always remain independent of the system size, the configuration of the lattice and the the test functions. The dependence of $C$ will be normally clear from the context or stated explicitly. 

% We define bounds on partial derivatives 
% \begin{equation*}
	% 	\begin{split}
		% 		m(\bm{\rho})&:= \prod_{i =1}^{j} \vert \rho_{i} \vert \sup_{\bm{g} \in \R^{\mathcal{R}}} \Vert V_{\rho}(\bm{g}) \Vert \quad \text{for} \ \bm{\rho} \in \mathcal{R}^{j}, \ \text{and} \\
		% 		M^{(j,s)}&:= \sum_{\bm{\rho} \in \R^{j} } m(\bm{\rho}) \vert \bm{\rho} \vert^{s}_{\infty},
		% 	\end{split}
	% \end{equation*}
% where $V_\rho$  is the derivative of the energy functional $V$ with respect to $\rho$. $\Vert \cdot \Vert$ denotes the $L^{2}-$operator norm of a multilinear form and $\vert \bm{\rho} \vert_{\infty}:=\max_{i= 1,\dots,j}\vert \rho_{i} \vert$.

% We use the convention that '$\lesssim$' stands for '$\leq C$', where $C$ is a generic constant that does not depend on the strain and its higher order derivatives.
%===========================================================================