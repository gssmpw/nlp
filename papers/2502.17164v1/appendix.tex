\section{Proof}
\label{sec: appendix}
\renewcommand{\theequation}{A.\arabic{equation}}

In this section, we first state the well-known inverse function theorem, which plays a crucial role in the {\it a priori} error estimates presented in this work. The result is a simplified and specialized version of~\cite[Lemma 2.2]{2011_CO_1D_QNL_MATHCOMP}, though similar formulations can be derived from standard proofs of the inverse function theorem.

\begin{lemma}\label{Inverse function theorem}
	\textbf{Inverse function theorem: }Let $\Yn$ be a subspace of $\Ya$, equipped with $\Vert \nabla \cdot \Vert_{L^{2}}$, and let $\Ghc \in C^{1}(\Yn,\Yn^{*})$ with Lipschitz-continuous derivative $\delta \Ghc$:
	\begin{equation*}
		\Vert\delta\Ghc(y) - \delta\Ghc(v)\Vert_\mathcal{L} \le M \Vert \nabla y - \nabla v \Vert_{L^{2}}, \quad \text{for all} \  v \in \Un,
	\end{equation*}
	where $\Vert \cdot \Vert_{\mathcal{L}}$ denotes the $\mathcal{L}(\Yn,\Yn^{*})$-operator norm.
	
	Let $\bar{y}\in \Yn$ satisfy
	\begin{align}
		\Vert \Ghc(\bar{y})\Vert_{\Yn^{*}} &\le \eta,\\
		\langle \delta \Ghc (\bar{y})v,v\rangle &\ge \gamma \Vert \nabla v \Vert^{2}_{L^{2}},  \quad \text{for all} \ v \in \UhNL,
	\end{align}
	such that $L,\eta,\gamma$ satisfy the relation
	\begin{equation}
		\frac{2M\eta}{\gamma^{2}}<1.
	\end{equation}
	
	Then there exists a (locally unique) $\ynll\in \Yn$ such that $\Ghc(\ynll)=0$,
	\begin{align}
		\Vert \nabla \ynll-\nabla \bar{y}\Vert_{L^{2}} &\le 2\frac{\eta}{\gamma}, \quad \text{and}\\
		\langle \delta \Ghc (\ynll)v,v\rangle &\ge (1-\frac{2M\eta}{\gamma^{2}}) \Vert \nabla v \Vert^{2}_{L^{2}},  \quad \text{for all} \ v \in \Un,
	\end{align}
\end{lemma}


\subsection{Proof of Proposition \ref{Pointwise coupling stress tensor}}
\label{Appendix section 1}

\begin{proof}
	The proof of the coupling error estimate for the QNL model is derived from [\cite{2013_ML_CO_AC_Coupling_ACTANUM}, Lemma 6.12]. Since $\text{supp}(\Ki) = \conv\{\xi,\xi+\rho\}$, $\Rrfl=0$ for $x\in\OmeA \backslash \bOmeA$. As for $x\in \OmeI \cup \bOmeA$, we obtain
	\begin{align*}
		\Rrfl &=\Srfl-\Sa\\
		&=\sum_{\xi \in \Ic} \sum_{\rho \in \Rc}\rho\Ki\Phii_{\xi,\rho}(y)-\sum_{\xi \in\Ic\cup\Cc}\sum_{\rho \in \Rc} \rho \Ki \Phia_{\xi,\rho}(y).
	\end{align*}
	We define uniform deformation $\yF(\xi)=F\xi$. We notice that $R^{\text{rfl}}(\yF;x)=0$, hence we obtain
	\begin{equation}\label{Two parts of Rrfl}
		\begin{aligned}
			\Rrfl &= \Rrfl-R^{\text{rfl}}(\yF;x)\\
			&= \sum_{\xi \in \Ic} \sum_{\rho \in \Rc}\rho\Ki\big(\Phii_{\xi,\rho}(y)-\Phii_{\xi,\rho}(\yF)\big)\\
			&\ +\sum_{\xi \in\Ic\cup\Cc}\sum_{\rho \in \Rc} \rho \Ki \big(\Phia_{\xi,\rho}(\yF)-\Phia_{\xi,\rho}(y)\big).
		\end{aligned}
	\end{equation}
	
	For the first term of \eqref{Two parts of Rrfl}, we choose $F=\nabla y$ and use Taylor's expansion. We have
	\begin{equation}\label{Interface stess}
		\vert \Phii_{\xi,\rho}(y)-\Phii_{\xi,\rho}(\yF) \vert \le \sum_{\zeta \in \Rc} c(\rho,\zeta)\vert D_{\zeta}y(\xi)-\nabla_{\zeta}y(x)\vert.
	\end{equation}
	After using Taylor's expansion, for $x\in \conv\{\xi,\xi+\rho\}$, we obtain
	\begin{align*}
		\vert D_{\zeta}y(\xi)-\nabla_{\zeta}y(x)\vert&= \vert \zeta(\xi-x+\frac{\zeta}{2})\vert \cdot \Vert \nabla^{2}y\Vert_{L^{\infty}(\conv\{\xi,\xi+\rho\})}\\
		&\le\frac{\vert \zeta \vert^{2}}{2}\Vert \nabla^{2}u\Vert_{L^{\infty}(\conv\{\xi,\xi+\rho\})}.
	\end{align*}
	It is straightforward to calculate that
	\begin{align*}
		\sum_{\xi \in \Ic} \sum_{\rho \in \Rc} \vert\rho \vert\Ki \vert \Phii_{\xi,\rho}(y)-\Phii_{\xi,\rho}(\yF) \vert &\le \sum_{(\rho,\zeta)\in\Rc^{2}} \frac{1}{2}\vert \rho \zeta^{2}\vert c(\rho,\zeta) \Vert \nabla^{2}y\Vert_{L^{\infty}(v_{x})}\sum_{\xi \in \Ic}\Ki\\
		&\lesssim M^{(2,1)}\Vert \nabla^{2}y\Vert_{L^{\infty}(v_{x})}.
	\end{align*}
	
	Similarly, we could calculate that
	\begin{align*}
		\sum_{\xi \in\Ic\cup\Cc} \sum_{\rho \in \Rc} \vert\rho \vert\Ki \vert \Phia_{\xi,\rho}(y)-\Phia_{\xi,\rho}(\yF) \vert &\le \sum_{\rho \in \Rc} \frac{1}{2} \vert \rho \zeta^{2}\vert m(\rho,\zeta) \\
		&\le M^{(2,1)} \Vert \nabla^{2}y\Vert_{L^{\infty}(v_{x})}.
	\end{align*}
	
	For $x\in \OmeC \cap \Z+\frac{1}{2}$, we have
	\begin{equation*}
		\Rrfl = \partial_{F}W\big(\nabla y(x)\big)-\Sa,
	\end{equation*}
	which is the difference between the atomistic stress tensor and Cauchy-Born stress tensor. And we mention that the error estimate follows directly from[\cite{2013_ML_CO_AC_Coupling_ACTANUM}, Theorem6.2]
	\begin{align*}
		\Rrfl &= \partial_{F}W\big(\nabla y(x)\big)-\Sa\\
		&\lesssim M^{(2,2)}\Vert \nabla^{3}y\Vert_{L^{\infty}(v_{x})}+M^{(3,2)}\Vert \nabla^{3}y\Vert_{L^{\infty}(v_{x})}^{2}.
	\end{align*}
	This yields the stated results by noticing that $\nabla^{2}y =\nabla^{2}u,\ \nabla^{3}y=\nabla^{3}u$.
	% By definition of $\Ua$ and $\Ya$, we notice $\nabla^{2}y =\nabla^{2}u,\ \nabla^{3}y=\nabla^{3}u$. So for simplify, we choose norm of $u$ in final result.
\end{proof}

\subsection{Proof of Proposition \ref{Coupling consistency error estimate}}\label{Appendix section 2}
\begin{proof}
	For \eqref{Interface region stress tensor}, the main point of this proof is to use the inverse estimates to obtain $L^{2}$-type from the $L^{\infty}$ bounds~\cite{2007_DB_FEM}
	\begin{equation}\label{L-infty to L-2 estimate}
		\Vert \nabla^{2}u\Vert_{L^{\infty}(v_{x})}\lesssim \Vert \nabla^{2}u\Vert_{L^{2}(v_{x})}.
	\end{equation}
	
	After a direct calculation, we have
	\begin{align*}
		\int_{\OmeI\cup\bOmeA}\Rrfl\nabla v\,\d x&\lesssim \int_{\OmeI\cup\bOmeA} M^{(2,1)}\Vert \nabla^{2}u\Vert_{L^{\infty}(v_{x})} \vert \nabla y \vert \,\d x\\
		&\le M^{(2,1)}\Vert \nabla^{2}u\Vert_{L^{\infty}(\bOmeI)}\Vert \nabla v \Vert_{L^{1}(\OmeI\cup\bOmeA)}\\
		&\lesssim  M^{(2,1)}\Vert \nabla^{2}u\Vert_{L^{\infty}(\bOmeI)}\Vert \nabla v \Vert_{L^{2}(\OmeI\cup\bOmeA)}.
	\end{align*}
	
	As for \eqref{Continuum region stress tensor}, combining Proposition \ref{Pointwise coupling stress tensor} and [\cite{2013_ML_CO_AC_Coupling_ACTANUM}, Corollary6.4], we yield the started results.
\end{proof}

\subsection{Proof of Lemma \ref{Pointwise blending lemma}}\label{Appendix section 3}
\begin{proof}
	Let $\psi(x):=\sqrt{1-\beta(x)}$ and assume, without loss of generality that $\rho>0$. Then,
	\begin{align*}
		\sqrt{1-\beta(\xi)} D_{\rho}v(\xi)&= \psi (\xi) \sum_{\eta = \xi}^{\xi+\rho-1}D_{1}V(\eta)\\
		&=\sum_{\eta = \xi}^{\xi+\rho-1}\psi(\eta)D_{1}V(\eta) + \sum_{\eta = \xi}^{\xi+\rho-1} \big(\psi(\xi)-\psi(\eta)\big)D_{1}V(\eta).\\
	\end{align*}
	If we define $\va$ by $D_{1}\va(\eta)=\psi(\eta)D_{1}v(\eta)$, then we obtain
	\begin{equation*}
		\sum_{\eta = \xi}^{\xi+\rho-1} \psi(\eta) D_{1}v(\eta) =D_{\rho} \va (\xi),
	\end{equation*}
	and after using Holder's inequality we know
	\begin{align*}
		\vert\sqrt{1-\beta} D_{\rho}v(\xi)-D_{\rho}\va(\xi)\vert&=\sum_{\eta = \xi}^{\xi+\rho-1}\big(\psi(\xi)-\psi(\eta)\big)D_{1}v(\eta)\\
		&\le (\sum_{\eta = \xi}^{\xi+\rho-1} \Vert \nabla \psi \Vert^{2}_{L^{\infty}} \vert \rho \vert^{2})^{\frac{1}{2}} (\sum_{\eta = \xi}^{\xi+\rho-1}\big(D_{1}v(\eta)\big)^{2})^{\frac{1}{2}}\\
		&\le \vert \rho \vert^{\frac{3}{2}} \Vert \nabla \psi \Vert_{L^{\infty}} \Vert \nabla v \Vert_{L^{2}(\xi,\xi+\rho)}.
	\end{align*}
	This establishes \eqref{Pointwise va blending estimate}. The proof of \eqref{Pointwise vc blending estimate1} is analogous, with $\vc$ defined by $D_{1}\vc(\xi) = \sqrt{\beta(\xi)}D_{1}v(\xi)$. With these definitions, \eqref{va and vc} is an immediate consequence.
\end{proof}