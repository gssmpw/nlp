\section{Conclusion and Future Work}
\label{sec: conclusion}

In this work, we incorporate the linear elasticity model to enhance the computational efficiency of the QNL method. Specifically, we introduce a coupling energy formulation that integrates the nonlinear QNL method with the linearized Cauchy-Born model, referred to as the QNLL method. Through a rigorous {\it a priori} error analysis, we demonstrate that the convergence of the QNLL method depends on the balance between the lengths of the computational, atomistic, nonlinear continuum, and linear continuum regions, as well as the finite element coarse-graining. Our analysis ensures that the QNLL method achieves the same convergence order as the original QNL method while significantly reducing computational cost. Numerical experiments validate these theoretical findings, highlighting the efficiency and practical advantages of the QNLL method.

However, several problems remain open which deserve further investigation.

%\begin{itemize}
%	\item {\it Extension to higher dimensions:} In 2D and 3D lattice systems, atomic environments and coupling schemes become more complex, increasing computational cost. A linear elasticity model can help control errors while improving computational efficiency. Accurately estimating linearization errors in higher dimensions is required, which will be studied in our future work.
%	\item {\it Generalization to other a/c coupling methods:} Further \textit{a priori} analysis and numerical validation are needed for nonlinear A/C methods with continuum linearization. The impact of linearization error on accuracy and the applicability of our balancing method to other approaches require investigation. Expanding computational efficiency comparisons beyond the QNL method is also important.
%	\item {\it Three-Scale Coupling:} Extending to three-levels multiscale coupling, such as Quantum-Molecular-Elasticity methods, presents challenges in mesh construction, error analysis, and balancing strategies. Addressing these issues is key to further improving accuracy and efficiency.
%\end{itemize}

\begin{itemize}
    \item {\it a Posteriori Error Control:} A natural extension of our approach is the development of \textit{a posteriori} error control methods, which have a rich body of literature~\cite{1996_RV_A_Post_Adapt,2014_CO_HW_A_Post_ACC_IMANUM, 2023_YW_HW_Efficient_Adaptivity_AC_JSC,2021_YW_HC_ML_CO_HW_LZ_A_Post_QMMM_SISC}. The main challenge in this context lies in adaptively selecting the lengths of the regions and constructing the corresponding error estimators or indicators for these lengths. Several strategies, such as balancing error estimates with mesh refinement techniques, could be explored to improve computational efficiency while maintaining accuracy in the error bounds. This extension is crucial for future work, where the focus will be on optimizing these error indicators and integrating them effectively into multiscale models.
    
	\item {\it Extension to Higher Dimensions:} Extending our framework to higher dimensions requires a shift from QNL to GRAC methods~\cite{2012_CO_LZ_GRAC_Construction_SIAMNUM,2014_CO_LZ_GRAC_Coeff_Optim_CMAME}. The introduction of linearized Cauchy-Born in this context raises significant concerns regarding the linearization errors, especially when extending to higher dimensions. Specifically, challenges arise in quantifying how these linearization errors propagate through the coupled system, and ensuring the stability of the method in higher dimensions is a direct issue. Further research is needed to assess the extent of these linearization errors in higher-dimensional settings and to determine whether additional modifications or error control techniques are required to maintain robustness and computational efficiency in the model.

	
	\item {\it Generalization to Other a/c Coupling Methods:} Expanding our approach to incorporate other a/c coupling techniques, such as blending methods, is an important next step~\cite{2008_SB_MP_PB_MG_AC_Blending_MMS,2011_BK_ML_BQCE_1D_SIMNUM,2016_XL_CO_AS_BK_BQC_Anal_2D_NUMMATH, wang2023adaptive}. The introduction of linearized Cauchy-Born in these methods requires further exploration, particularly with regard to how nonlinear to linear transitions are handled. A critical question is whether blending is necessary to manage this transition smoothly. Additionally, the force-based method formulation in the context of linearized Cauchy-Born is worth investigating~\cite{2008_MD_ML_Ana_Force_Based_QC_M2NA,2019_HW_SL_FY_A_Post_QCF_1D_NMTMA}, as this approach may offer significant computational speedup by reducing the problem to linear equations in the continuum region.
	
	\item {\it Theoretical Estimates for Computational Efficiency Improvements:} While we provide the priori anlysis of our method, further work is needed to quantify the theoretical estimate in computational efficiency. Specifically, optimization techniques can be applied to estimate how the proposed multiscale framework could lead to cheaper computational cost. Future research could focus on developing a theoretical framework for computational cost reduction based on the error estimates and meshing strategies, offering a more complete understanding of how to improve computational efficiency alongside accuracy.

    \item {\it Possible Connections with Sequential Multiscale Methods:} Although our current focus is on concurrent multiscale methods, future work should consider potential connections with sequential methods, particularly force-based methods like the Flexible Boundary Condition (FBC) Method~\cite{2002_CW_SR_FBC_Dislocation_PHYS,2021_MH_Anal_FBC_AC_MMS,2008_RT_LGF_Long-range_PHYS}. The FBC method, while widely used for simulating defects in materials, still faces challenges such as slow convergence speed. Our approach may offer useful insights for the convergence analysis of FBC methods, particularly regarding error propagation and the balance between the lengths of regions. A key area of future investigation is the adaptation of existing sequential methods by improving convergence rates, such as extending the relaxation region and applying finite element coarsening techniques. Furthermore, exploring hybrid methods that integrate both sequential and concurrent techniques could lead to new strategies that harness the strengths of both approaches, improving both accuracy and computational efficiency.

	
\end{itemize}


Both the theoretical and practical aspects discussed above will be explored in future work.
%Future works 可以多写一点,把这些问题所可能遇到的困难和大概的解决方案写一下(多参考一下我的文章的写法,注意要引用相关文章,你这一大段全部在自说自话,完全没有引用)。按照以下顺序写:
%1. a Posteriori Error Control。后验误差估计是一个最直接的拓展,有丰富的文献参考,这里主要难点是如何自适应的选取各区域的长度,如何构造这些长度的后验误差估计子或指示子。
%2. Extension to higher dimensions。写高维当中QNL就变成了GRAC,这时候加入linearized Cauchy-Born会有什么影响,或者是很直接的。高维当中加入linearized Cauchy-Born会不会引起稳定性的问题等等。
%3. Generalization to other a/c coupling methods。可以写blending方法加入linearized Cauchy-Born的构造,并讨论NL-L过渡是否需要blending来过渡的问题。可以写force-based方法的大体构造和可能的进一步提速,因为这时候在C区域实际已经变成了一个线性方程组求解,所以速度可能会更快。
%4. 计算效率提升的理论估计。承认我们只给出了收敛性的估计,而计算效率的提升的估计我们没有给出,如何从优化的角度做这个方面的理论分析给出一个大致的思路。
%5. Possible connections with the sequential multiscale methods。你下面写的这些东西可以挑出来用,然后写我们的方法如何可能辅助FBC方法的收敛性分析。

%对FBC的讨论可以放到最后的discussion里面去,我们现在只考虑concurrent multiscale methods。Concurrent multiscale methods, particularly the atomistic-to-continuum (a/c) coupling method, are widely used~\cite{2008_SB_MP_PB_MG_AC_Blending_MMS,2013_ML_CO_AC_Coupling_ACTANUM,2003_RM_ET_QCM_JCAMD,1996_AC_Ana_Solid_Defect_PMA,1999_VS_RM_ETadmor_MOrtiz_AFEM_QC_JMPS}. These methods combine atomistic models with their continuum approximations to achieve a balance between computational efficiency and accuracy. Unlike sequential methods, concurrent methods are built on a clearer theoretical framework~\cite{2013_ML_CO_AC_Coupling_ACTANUM,2011_CO_1D_QNL_MATHCOMP,2011_CO_HW_QC_A_Priori_1D_M3AS,2014_CO_HW_A_Post_ACC_IMANUM}. The a/c coupling methods are typically classified into energy-based and force-based approaches. Energy-based methods focus on constructing the total energy of the coupled system~\cite{2009_PM_ZY_1D_QC_Nonlocal_MMS,2011_CO_HW_QC_A_Priori_1D_M3AS,2004_Shimokawa_QCM_ErrAna_PRB,2018_HW_SY_Efficiency_A_Post_1D_MMS}, while force-based methods address the formulation of nonlinear force equations that describe the entire system~\cite{2008_MD_ML_Ana_Force_Based_QC_M2NA,PM_Error_QCF_2008,1999_VS_RM_ETadmor_MOrtiz_AFEM_QC_JMPS}. Concurrent methods solve for the local minimum of the total energy (energy-based) or the force equilibrium of the system (force-based) through a single computation over the entire domain.

%对FBC的讨论可以放到最后的discussion里面去,我们现在只考虑concurrent multiscale methods。The coupling multiscale methods can be broadly categorized into sequential and concurrent approaches. Sequential methods, such as the Green’s function based Flexible Boundary Condition (FBC)~\cite{2021_MH_Anal_FBC_AC_MMS,2008_RT_LGF_Long-range_PHYS,2002_CW_SR_FBC_Dislocation_PHYS,2008_CW_RT_Al_Prediction_PHYS} and the far field asymptotic expansion method~\cite{2022_JB_TH_CO_Elastic_Far-Field_BC_ARMA,braun2022higher}, iteratively couple regions modeled with different single-scale approaches. These methods transfer information by using the atomistic configuration near the interface as the boundary condition for the continuum model. However, sequential methods often face convergence challenges due to their iterative nature~\cite{2009_Miller_Tadmor_Unified_Framework_Benchmark_MSMSE}.