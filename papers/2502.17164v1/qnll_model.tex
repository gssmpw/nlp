\section{Atomistic to Nonlinear-Linear Elasticity Coupling Method}
\label{sec: qnll_model}

In this section, we present the atomistic-to-continuum (a/c) coupling methods in one dimension. While the extension to higher dimensions is relatively straightforward, it involves significantly more complex notations. To ensure clarity and brevity, we focus on the one-dimensional case. Section~\ref{sec: introduction_atom} introduces the atomistic model, which serves as the reference framework. Section~\ref{sec: introduction_qnl} provides an overview of the classical nonlinear quasinonlocal (QNL) method. Finally, Section~\ref{sec: introduction_qnll} details the linearization of the Cauchy-Born model and the QNL method with the Linearized Cauchy-Born (QNLL) formulation, which is the main focus of this work.

\subsection{Atomistic model}
\label{sec: introduction_atom}

We consider an infinite atomistic chain or one dimensional crystal lattice indexed by $\Z$. The reference configuration is given by $F\Z$, where $F>0$ is a macroscopic strain. 
We define the space of the displacements and the {\it admissible} set of deformations by
%	\Ua_{0}&:=\{u:\Z\rightarrow\R \ | \ \text{supp} (u) \text{ is bounded} \},\\
\begin{align*}
	\Ua&:=\{u:\Z\rightarrow\R ~|~ \nabla u \in L^{2} \},\\
	\Ya&:=\{y(x)= Fx+u(x) ~|~ u \in \Ua\}.
\end{align*}

Let $\rcut>0$, we fix an interaction range $\Rc := \{\pm1,\dots,\pm\rcut\}$. For each $y \in \Ya$ and $\xi \in \Z$, we define the finite difference stencil 
\begin{equation*}
	Dy(\xi):= \big(D_{\rho}y(\xi)\big)_{\rho \in \mathcal{R}}, \quad \text{where} ~~ D_{\rho}y(\xi):= y(\xi + \rho )-y(\xi).
\end{equation*}
For simplicity, we fix $\rcut = 2$ throughout this work. However, the analysis can be readily extended to a general interaction range.

Let $V\in C^{3}(\R^{\Rc})$ be the interatomic many-body site potential. For a deformation $y\in\Ya$, we define the energy of the infinite atomistic model by
\begin{equation}\label{All-Atomistic Energy}
	\bEa (y):= \sum_{\xi\in\Z}\Phia_{\xi}(y) := \sum_{\xi\in\Z} \left[V\big(Dy(\xi)\big)-V(F\Rc)\right],
\end{equation}
where $\Phia_{\xi}(y)$ is the site energy (per atom energy contribution~\cite{2013_ML_CO_AC_Coupling_ACTANUM}) for the site $\xi$.

For analytical purpose, we assume the regularity of site potential $V$, i.e.,
\begin{align*}
%	\begin{split}
		m(\bm{\rho})&:= \prod_{i =1}^{j} \vert \rho_{i} \vert \sup_{\bm{g} \in \R^{\mathcal{R}}} \Vert V_{\rho}(\bm{g}) \Vert \quad \text{for} \ \bm{\rho} \in \mathcal{R}^{j}, \ \text{and} \\
		M^{(j,s)}&:= \sum_{\bm{\rho} \in \R^{j} } m(\bm{\rho}) \vert \bm{\rho} \vert^{s}_{\infty},
%	\end{split}
\end{align*}
where $V_\rho$ is the derivative of the energy functional $V$ with respect to $\rho$ and $\vert \bm{\rho} \vert_{\infty}:=\max_{i= 1,\dots,j}\vert \rho_{i} \vert$.

By the definition of $\Ya$ and $\Ua$, $\Ea$ is well-defined on $\Ya$, which means that the sum on the infinite lattice is actually finite \cite[Proposition 3.7]{2013_ML_CO_AC_Coupling_ACTANUM}. Given a dead load $f \in \Ya$, the atomistic model is defined by the following minimization problem 
\begin{equation}\label{All-Atomistic solution}
	\yai \in \arg \min\{\bEa(y) -\langle f,y\rangle_{\Z}\ | \ y \in \Ya\},
\end{equation}
where $\langle f,y\rangle_{\Z} = \sum_{\xi \in \Z} f(\xi)y(\xi)$ and ``$\arg\min$'' is understood as the set of local minimizers. Furthermore, we denote the local minimizer of the displacement as $\bar{u}^{\rm a}:=\bar{y}^{\rm a}-Fx$. 

If $\yai$ solves \eqref{All-Atomistic solution}, then it satisfies the first order optimality condition 
\begin{equation}\label{All-Atomistic solution condition}
	\langle \delta \bEa(\yai),v \rangle = \langle f,v\rangle_{\Z}, \quad \forall v \in \Ua.
\end{equation}
In addition to \eqref{All-Atomistic solution condition}, if $\yai \in \Ya$ satisfies the second-order optimality condition which is given by
\begin{equation}\label{All-Atomistic strong local minimizer}
	\langle \delta^{2}\bEa(\yai)v,v\rangle \ge c_{0} \Vert \nabla v \Vert^{2}_{L^{2}},  \quad \forall v\in \Ua,
\end{equation}
for some $c_{0} >0$, we say that the solution $\yai$ is a strongly stable local minimizer.

% \begin{remark}[Existence of the solution and the Decay Hypothesis]
	% \yz{It is difficult (if not impossible) to prove the existence of the solution $\yai$. For a detailed proof of existence, readers can refer to \cite{2016_JB_BS_Existence_CVPD}. Instead, in the literature of the a/c coupling methods, we often assume that the solution 
		% $\yai$ exists \cite{XXXXX}.
		% We also make the following assumption of the property of $\yai$ which is often called the decay hypothesis \cite{2013_ML_CO_AC_Coupling_ACTANUM}.}
	
	To model practical defects and establish a foundation for consistency and stability analysis~\cite{2016_EV_CO_AS_Boundary_Conditions_for_Crystal_Lattice_ARMA}, we introduce the decay hypothesis, which describes the decay and regularity of local minimizers (equilibrium). This assumption ensures smooth asymptotic behavior, which is crucial for the subsequent analysis.
	
	% This hypothesis ensures that the correction term $\ua$ and its derivatives decay at a rate controlled by $\alpha$. The decay of higher-order derivatives ($j=1,2,3$) also guarantees smooth asymptotic behavior, which is critical for consistency and stability analysis.
	
	\textbf{Decay Hypothesis $\DH$}: There exists a strong local minimizer $\bar{u}^{\rm a} \in \Ua$ and $\alpha > 1/2$, for $x$ sufficiently large such that
	\begin{equation}\label{Decay Hypothesis}
		\vert \nabla^{j} \bar{u}^{\rm a}(x) \vert \le \CDH x^{-\alpha+1-j}, \quad j = 0,1,2,3,
	\end{equation}
	where $\CDH>0$ is a constant that depends on lattice and interatomic potentials. 
	
	
	\subsection{Classic quasinonlocal (QNL) methods}
	\label{sec: introduction_qnl}
	%\chw{We give a motivation of introducing the modified reflection method. The motivations are the three points: 1. We need to truncate the domain so that the problem is computable. 2. We apply a continuum model to replace the atomistic model and use finite element approximation to reduce the number of degrees of freedom. 3. The reflection method is show to be universally stable. }
	
	The atomistic problem defined by \eqref{All-Atomistic solution} is computationally intractable due to its formulation on an infinite lattice, its reliance on a nonlocal interaction potential across the entire domain, and the fact that each atom is treated as a separate degree of freedom. To address these challenges, atomistic/continuum (a/c) coupling methods employ domain decomposition strategies to truncate the infinite computational domain, introduce a reduced model in certain regions, and further decrease the number of degrees of freedom through coarse graining. In this section, we illustrate the construction of a/c methods using the classical quasinonlocal (QNL) method~\cite{2011_CO_1D_QNL_MATHCOMP}, which serves as the foundation for the developments presented in subsequent sections.
	
	The a/c coupling methods often make the following three steps of approximations. The first step is to truncate the infinite lattice to a finite domain on which the computation is carried out. The second step is to derive a {\it local} and continuum approximation for the {\it nonlocal} and discrete atomistic model. The third step is to decompose the domain so that the atomistic and the continuum models are properly utilized in different regions and to make a special treatment at the interfaces where the two different models meet so that nonphysical phenomenon is avoided. We now elaborate the construction of the QNL method according to the aforementioned three steps.
	
	\subsubsection{Truncation}
	
	We first truncate the infinite domain simply by fixing an $N\in \Nb$ and define the truncated computational domain to be $\Omega:=[-N,N]$. The set of lattice inside the computational domain is given by $\Lambda := \Omega \cap \Z = \{-N, \ldots, -1, 0, 1, \ldots, N\}$. 
	
	
	
	
	
	% The common practice of dealing with this intractability for the concurrent multiscale methods, or in particular, the 
	
	
	%To limit the degrees of freedom to a finite number, we truncate the infinite atomistic domain $\Z$ to the computational domain $[-N, N]$. 2. The second approximation is to perform a continuum approximation of the atomistic model \eqref{All-Atomistic Energy} using the Cauchy-Born rule. 3. To achieve an quasi-optimal balance between computational accuracy and cost, we couple the atomistic model with its continuum approximation. However, direct coupling can lead to "ghost forces" at the coupling interface, necessitating careful handling of the energy definition at that interface. In this section, we will introduce the reflection method \cite{2014_CO_AS_LZ_Stabilization_MMS}, which eliminates the "ghost forces" through geometric reconstruction.
	
	%In the first step, we will truncate the infinite domain of the atomistic problem based on the first approximation. Since the atomistic problem \eqref{All-Atomistic solution} is defined over an infinite domain, it is not computable. In order to make this problem computable, we fix $N\in \Nb$ and truncate the domain to $[-N,N]$, which contains only a finite number of degrees of freedom.We define the finite-dimensional displacement and deformation spaces as
	
	Though the first step is very easy, we pause here to introduce an auxiliary problem that simplifies our error analysis. Specifically, we apply a Dirichlet boundary condition on $\Omega$ and define the finite dimensional space of displacements and the corresponding admissible set of deformation as
	\begin{align*}
		\Un&:= \{u\in\Ua \ | \ u(\xi) = 0 \text{ for }\xi \le -N \text{ or } \xi \ge N\},\\
		\Yn&:= \{y(x) = Fx + u(x) \ | \ u \in \Un\}.
	\end{align*}
	We can then define the truncated atomistic method (often denoted by ATM \cite{2016_EV_CO_AS_Boundary_Conditions_for_Crystal_Lattice_ARMA,2013_ML_CO_AC_Coupling_ACTANUM}) as 
	\begin{equation}\label{Atomistic solution condition}
		\ya \in \arg \min \{\Ea(y) - \langle f,y \rangle_{N} \ | \ y \in \Yn \},
	\end{equation}
	where $\Ea(y): =  \sum_{\xi =-N}^{N}\Phia_{\xi}(y)$ and $\langle f,y\rangle_{N} = \sum_{\xi =-N}^{N} f(\xi)y(\xi)$, which can be considered as a Galerkin approximation of the original atomistic model \eqref{All-Atomistic solution}. 
	
	Since our atomistic/continuum coupling methods are all defined on the finite domain $\Omega$, we can bound the error between the solution of any of the a/c methods $y^{\rm ac}$ and that of original atomistic model $\yai$ by 
	\begin{equation}
		\label{eq: error separation}
		\| y^{\rm ac} - \yai \| \le \|y^{\rm ac} - \ya \| + \|\ya - \yai \|.
	\end{equation} 
	We will only concentrate on the first part of the right hand side of \eqref{eq: error separation} in the error analysis and the estimate of the second part, which is essentially the truncation error, is given by the following lemma \cite[Theorem 3.14]{2013_ML_CO_AC_Coupling_ACTANUM}:
	
	\begin{lemma}
		Let $y^{{\rm a}}$ be a strong local minimizer of \eqref{All-Atomistic solution condition} satisfying $\DH$ and \eqref{All-Atomistic strong local minimizer}. Then there exists $N_{0} \in \Nb$ such that, for all $N \ge N_{0}$, there exists a strong local minimizer $\ya$ of \eqref{Atomistic solution condition} satisfying
		\begin{equation}\label{Truncation error}
			\Vert \nabla \yai - \nabla \ya\Vert_{L^{2}} \le \frac{16M^{(2,0)}\CDH}{\sqrt{2\alpha-1}} N^{\frac{1}{2}-\alpha}.
		\end{equation}
	\end{lemma}
	
	%\begin{proof}
	%    Follow  [].
	%\end{proof}
	
	
	%Due to the above results and the sake of theoretical simplicity in the following sections, our error estimates will usually use the truncated solution  instead of the atomistic solution. For the sake of simplicity in notation, we denote $\ya$ as $\ya$ in next sections. For simplicity in the subsequent theoretical framework, we restrict the computational domain to $[-N,N]$ in both the QNL model of this subsection and the QNLL model of the next subsection.
	
	
	\subsubsection{Continuous approximation}
	
	Next, we consider the second approximation, which involves using a local continuum model to approximate the original nonlocal atomistic model. To further reduce the number of degrees of freedom, we apply the continuum approximation to the atomistic potential \eqref{All-Atomistic Energy} based on the Cauchy-Born rule~\cite{2013_CO_FT_Cauchy_Born_ARMA}, where we define the strain energy density from the interaction potential $V$ as 
	\begin{equation}
		\label{Cauchy-Born site energy}
		W(F) := V(F\Rc).
	\end{equation}
	
	
	\subsubsection{Domain decomposition}
	
	Finally, we will achieve the third approximation through the atomistic-to-continuum coupling method. The Atomistic-to-Continuum coupling method achieves an quasi-optimal trade-off between computational cost and accuracy by coupling the atomistic model with its continuum approximation. In this paper, the coupling scheme we use is the reflection method, which is show to be universally stable  \cite{2014_CO_AS_LZ_Stabilization_MMS}, and we will discuss in stability analysis. Because we fix $\rcut = 2$, the reflection method is equivalent to the Quasi-nonlocal method (QNL method) in this case \cite{2013_ML_CO_AC_Coupling_ACTANUM}. The QNL method eliminates the ``ghost force" term by providing a precise definition of the energy on the interface \cite{2011_CO_1D_QNL_MATHCOMP}. Therefore, for the sake of simplicity in naming, the remaining sections of this paper will use the QNL method to uniformly refer to the reflection method.
	
	We first of all decompose the computational domain $\Omega$ into three different regions. The first one is the atomistic region $\OmeA$ in which we assume the defect core is contained. The second one is the continuum region $\OmeC$ where we assume the deformation is smooth enough. The third one is th interface region $\OmeI$ which consists of a small number of layers of atoms between $\OmeA$ and $\OmeC$. The lattice points in the three regions are defined by $\Ac:= \OmeA \cap \Lambda$, $\Cc:= \OmeC \cap \Lambda$ and $\Ic := \OmeI \cap \Lambda$. To be more specific, we denote them by (for simplicity, we denote $\bar{K} = K+\rcut$)
	\begin{align*}
		\Ac &:=\{-K,-K+1,\dots,K\}, \qquad \qquad \qquad \qquad \qquad \qquad \qquad \quad ~~  \text{(Atomistic region)},\\
		\Ic &:=\{-\bar{K},-\bar{K}+1\}\cup \{\bar{K}-1,\bar{K}\}, \qquad \qquad \qquad \qquad \qquad \qquad ~~ \text{(Interface region)},\\
		\Cc &:=\{-N,-N+1,\dots,-\bar{K}-1\}\cup\{\bar{K}+1,\bar{K}+2,\dots,N\}, \qquad ~ \text{(Continuum region)}.
	\end{align*}
	
	Since we will calculate the integral on continuum region, we define the following three continuous intervals:
	\begin{align*}
		\OmeA &=[-K-1,K+1],\\
		\OmeI &=[-\bar{K},-\bar{K}+1] \cup [\bar{K}-1,\bar{K}],\\
		\OmeC&=[-N,-\bar{K}]\cup [\bar{K},N].
	\end{align*}
	
	%We fix $\bar{L} \in \Nb$, $L \le \bar{L} \le N/2$. We choose a set of finite element nodes $\Nh = \{v_{0},\dots,V_{N_{\Th}}\} \subset \Z_{+}$, for some $N_{\Th} \in \Nb$, such that $\{0,\dots,\bar{L},N\}\subset\Nh\subset \{0,\dots,N\}$. The finite elements are given by 
	%\begin{equation*}
	%	\Th = \{[v_{j-1},v_{j}] \ | \ j = 1,\dots,N_{\Th}\}.
	%\end{equation*}
	%
	%For each $T\in\Th$, let $h_{T} = \text{diam}T$. For each $x\in[0,N]$, $x\in\text{int} T$, let $h(x):=h_{T}$. For $x>N$, let $h(x)=1$. We define the coarse displacement sapce by
	%\begin{align*}
	%	\Uh &= \{\uh \in \Un \ | \ \uh \text{ is piecewise affine with respect to } \Th\},\\
	%	\Yh &= \{F\Z + \uh \ | \ \uh \in \Uh \}
	%\end{align*}
	%
	%For any function $v:\Nh \rightarrow \R$, let $\Ih v:[0,N] \rightarrow \R$, $\Ih v \in \text{P1}(\Th)$, denote its continuous piecewise affine interpolant,
	%\begin{equation*}
	%	\Ih v(\zeta):=v(\zeta) \quad \text{for all }\zeta \in \Nh.
	%\end{equation*}
	%
	%For $f,g:\Nh \rightarrow \R^{N_{\Th}}$, we define
	%\begin{equation*}
	%	\langle f,g \rangle_{h}:=\int_{0}^{N}\Ih(f\cdot g)\d x = \sum_{j=1}^{N_{Th}} \frac{1}{2} h_{T}\{f(v_{j-1})\cdot g(v_{j-1})+f(v_{j})\cdot g(v_{j})\}.
	%\end{equation*}
	
	
	
	Moreover, we define, for any $x\in (\xi-1,\xi), \nabla y(x) = y(\xi)-y(\xi-1)$, which is a piecewise constant interpolation of $y$ with respect to lattice sites. The specific construction of the QNL model is given as follows:
	\begin{equation}\label{QNL energy}
		\Erfl (y)=\sum_{\xi \in \Ac}\Phia_{\xi}(y) + \sum_{\xi \in \Ic}\Phii_{\xi}(y) + \int_{\OmeC} W(\nabla y)\,\d x,
	\end{equation}
	where the interface energy reads
	\begin{align*}
		\Phii_{-L}(y)&=V(-D_{2}y,-D_{1}y,D_{1}y,D_{2}y)+\int_{-L-1}^{-L-\frac{1}{2}}W(\nabla y)\,\d x,\\
		\Phii_{-L+1}(y)&=V(2D_{-1}y,D_{-1}y,D_{1}y,D_{2}y),\\
		\Phii_{L-1}(y)&=V(D_{-2}y,D_{-1}y,D_{1}y,2D_{1}y),\\
		\Phii_{L}(y)&=V(D_{-2}y,D_{-1}y,-D_{-1}y,-D_{-2}y)+\int_{L+\frac{1}{2}}^{L+1}W(\nabla y)\,\d x.
	\end{align*}
	
	Given a dead load $f\in \Ya$, we seek solution
	\begin{equation}\label{QNL solution conditon}
		\yrfl \in \arg \min \{\Erfl (y) - \langle f,y\rangle_{N} \ | \ y \in \Yn\}.
	\end{equation}
	Similarly, if $\yrfl$ solves \eqref{QNL solution conditon}, then it satisfies the first order optimal condition
	\begin{equation}\label{QNL solution first order optimality condition}
		\langle \delta \Erfl(\yrfl ),v \rangle = \langle f,v\rangle_{N}, \quad \text{for all }v \in \Un.
	\end{equation}
	
	\subsection{QNL with linearized Cauchy-Born (QNLL) method}
	\label{sec: introduction_qnll}
	
	In this section, we will introduce the QNL Methods with Linearized Cauchy-Born (QNLL). The idea behind QNLL method is to replace a nonlinear elasticity model with a linear elasticity model to further simplify the computation. To that end, we make further approximations: based on the QNL method, we use a linear elasticity model to approximate the nonlinear elasticity model. Compared to the nonlinear model, the linear model can effectively reduce computational costs. However, this also introduces new errors, which will be analyzed in the next section.
	
	First, we construct the linear elasticity model. Here, we obtain the linear elasticity model by performing a Taylor expansion of the Cauchy-Born energy \eqref{Cauchy-Born site energy} around the uniform deformation. After introducing linearization, we will introduce the nonlinear-linear elasticity coupling (QNLL) model, which essentially replaces a portion of the nonlinear continuum energy functional $W$ in a continuum region $\OmeC$ with a linear energy functional $\WL$. Here, we first provide the form of $\WL$. For simplicity, we denote $W^{(k)}_{F} = W^{(k)}(F),\ k =0,1,2$. We use Taylor's expansion in order to linearize Cauchy-Born strain energy density: 
	\begin{equation*}
		W(\nabla y) = \Wf+\Wpf\nabla u+\frac{1}{2}\Wppf(\nabla u)^{2} +\dots.
	\end{equation*}
	We then define the linearized Cauchy-Born strain energy density:
	\begin{equation*}
		\WL(\nabla y) = \Wf+\Wpf\nabla u+\frac{1}{2}\Wppf(\nabla u)^{2}.
	\end{equation*}
	
	Next, based on the region partitioning of the QNL method, we will further subdivide $\Cc$ into nonlinear and linear regions. We denote them by
	\begin{align*}
		\Cnl &= \{- L,L+1,\dots,-\bar{K}-1\}\cup\{\bar{K}+1, \bar{K}+2,\dots,L\},\\
		\Cl &=\{-N,-N+1,\dots,L-1\}\cup \{L+1,L+2,\dots,N\}.
	\end{align*}
	Similarly, we will also divide $\OmeC$ into nonlinear and linear parts:
	\begin{align*}
		\OmeNL &=[-L,-\bar{K}] \cup [\bar{K},L],\\
		\OmeL &=[-N,L] \cup [L,N].
	\end{align*}
	
	In the linearized continuouum region, by approximating the nonlinear elasticity model with the linear elasticity model, we obtain the QNLL model:
	\begin{align}\label{Nonlinear-linear energy}
		\El (y) &= \sum_{\xi = -N}^{N}\PhiNLL_{\xi}(y) \nonumber \\
		&=\sum_{\xi \in \Ac}\Phia_{\xi}(y) + \sum_{\xi \in \Ic}\Phii_{\xi}(y) + \int_{\OmeNL} W(\nabla y)\,\d x + \int_{\OmeL} \WL(\nabla y)\,\d x.
	\end{align}
	
	
	Given a dead load $f\in \Ya$, we seek
	\begin{equation}\label{Nonlinear-linear solution condition}
		\ynll \in \arg \min \{\El (y) - \langle f,y\rangle_{N} \ | \ y \in \Yn\}.
	\end{equation}
	If $\ynll$ solves \eqref{Nonlinear-linear solution condition}, then it satisfies the first order optimal condition
	\begin{equation}\label{QNLL solution first order optimality condition}
		\langle \delta \El(y),v \rangle = \langle f,v\rangle_{N}, \quad \text{for all }v \in \Un.
	\end{equation}