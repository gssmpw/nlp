\documentclass[12pt, reqno, a4paper,oneside]{amsart}
\usepackage{graphicx}
\usepackage{subfig}
\usepackage{color}
\usepackage{xcolor}
%\usepackage{stmaryrd}
%\usepackage{ulem}
%\usepackage{scalerel}
%\usepackage{enumitem}
\usepackage[top=2.5cm,bottom=2.0cm,left=2.0cm,right=2.0cm]{geometry}
\usepackage{amsfonts, amsmath, amssymb, amsbsy, amsthm}
\usepackage{environments}
\usepackage{mathrsfs}
\usepackage{listings}
\usepackage{algpseudocode}
\usepackage{algorithm, algorithmicx}
\usepackage{diagbox}
\usepackage{bm}
\numberwithin{theorem}{section}
\numberwithin{equation}{section}
% \numberwithin{remark}{section}


\lstset{basicstyle=\ttfamily}

\newcommand\tbbint{{-\mkern -16mu\int}}
\newcommand\tbint{{\mathchar '26\mkern -14mu\int}}
\newcommand\dbbint{{-\mkern -19mu\int}}
\newcommand\dbint{{\mathchar '26\mkern -18mu\int}}
\newcommand\bint{
	{\mathchoice{\dbint}{\tbint}{\tbint}{\tbint}}
}
\newcommand\bbint{
	{\mathchoice{\dbbint}{\tbbint}{\tbbint}{\tbbint}}
}



\definecolor{yscol}{HTML}{6622AA}
\definecolor{yzcol}{rgb}{0, 0.7, 0}
\definecolor{hwcol}{rgb}{0, 0, 0.9}
\definecolor{mlcol}{rgb}{0, 0.7, 0}
\definecolor{todocol}{rgb}{0.0, 0.4, 0.0}
\newcommand{\chw}[1]{{\color{hwcol} \footnotesize \it [#1]}}
\newcommand{\hw}[1]{{\color{hwcol} #1}}
\newcommand{\ml}[1]{{\color{mlcol} (ML: #1)}}
\newcommand{\todo}[1]{{\color{todocol} #1}}

\newcommand{\cys}[1]{{\color{yscol} \footnotesize   \tt [YS: #1]}}
\newcommand{\ys}[1]{{\color{yscol} #1}}
\newcommand{\yz}[1]{{\color{yzcol} #1}}

\newcommand*{\mcap}{\mathbin{\scalebox{1.5}{\ensuremath{\cap}}}}%

\title[ \ ]{A General Framework of Linear Elasticity Enhanced Multiscale Coupling Methods for Crystalline Defects}

% \author{Hariharan Umashankar}
% \address{...}
% \email{...}

\author{Yanbo Zhan}
\address{Yanbo Zhan\\
	School of Mathematics\\
	Sichuan University\\
	No. 24 Yihuan Road\\
	Chengdu\\
	China
}
\email{2023322010008@scu.edu.cn}

\author{Yangshuai Wang}
\address{Yangshuai Wang\\
	Department of Mathematics\\
	Faculty of Science\\
	National University of Singapore\\
	10 Lower Kent Ridge Road\\
	Singapore
}
\email{yswang@nus.edu.sg}

\author{Hao Wang}
\address{Hao Wang\\
	School of Mathematics\\
	Sichuan University\\
	No. 24 Yihuan Road\\
	Chengdu\\
	China
}
\email{wangh@scu.edu.cn}



% \author{Matthias Militzer}
% \address{...}
% \email{...}

\date{\today}

\begin{document}
	
	\maketitle
	
	% !TeX root = main.tex 


\newcommand{\lnote}{\textcolor[rgb]{1,0,0}{Lydia: }\textcolor[rgb]{0,0,1}}
\newcommand{\todo}{\textcolor[rgb]{1,0,0.5}{To do: }\textcolor[rgb]{0.5,0,1}}


\newcommand{\state}{S}
\newcommand{\meas}{M}
\newcommand{\out}{\mathrm{out}}
\newcommand{\piv}{\mathrm{piv}}
\newcommand{\pivotal}{\mathrm{pivotal}}
\newcommand{\isnot}{\mathrm{not}}
\newcommand{\pred}{^\mathrm{predict}}
\newcommand{\act}{^\mathrm{act}}
\newcommand{\pre}{^\mathrm{pre}}
\newcommand{\post}{^\mathrm{post}}
\newcommand{\calM}{\mathcal{M}}

\newcommand{\game}{\mathbf{V}}
\newcommand{\strategyspace}{S}
\newcommand{\payoff}[1]{V^{#1}}
\newcommand{\eff}[1]{E^{#1}}
\newcommand{\p}{\vect{p}}
\newcommand{\simplex}[1]{\Delta^{#1}}

\newcommand{\recdec}[1]{\bar{D}(\hat{Y}_{#1})}





\newcommand{\sphereone}{\calS^1}
\newcommand{\samplen}{S^n}
\newcommand{\wA}{w}%{w_{\mathfrak{a}}}
\newcommand{\Awa}{A_{\wA}}
\newcommand{\Ytil}{\widetilde{Y}}
\newcommand{\Xtil}{\widetilde{X}}
\newcommand{\wst}{w_*}
\newcommand{\wls}{\widehat{w}_{\mathrm{LS}}}
\newcommand{\dec}{^\mathrm{dec}}
\newcommand{\sub}{^\mathrm{sub}}

\newcommand{\calP}{\mathcal{P}}
\newcommand{\totspace}{\calZ}
\newcommand{\clspace}{\calX}
\newcommand{\attspace}{\calA}

\newcommand{\Ftil}{\widetilde{\calF}}

\newcommand{\totx}{Z}
\newcommand{\classx}{X}
\newcommand{\attx}{A}
\newcommand{\calL}{\mathcal{L}}



\newcommand{\defeq}{\mathrel{\mathop:}=}
\newcommand{\vect}[1]{\ensuremath{\mathbf{#1}}}
\newcommand{\mat}[1]{\ensuremath{\mathbf{#1}}}
\newcommand{\dd}{\mathrm{d}}
\newcommand{\grad}{\nabla}
\newcommand{\hess}{\nabla^2}
\newcommand{\argmin}{\mathop{\rm argmin}}
\newcommand{\argmax}{\mathop{\rm argmax}}
\newcommand{\Ind}[1]{\mathbf{1}\{#1\}}

\newcommand{\norm}[1]{\left\|{#1}\right\|}
\newcommand{\fnorm}[1]{\|{#1}\|_{\text{F}}}
\newcommand{\spnorm}[2]{\left\| {#1} \right\|_{\text{S}({#2})}}
\newcommand{\sigmin}{\sigma_{\min}}
\newcommand{\tr}{\text{tr}}
\renewcommand{\det}{\text{det}}
\newcommand{\rank}{\text{rank}}
\newcommand{\logdet}{\text{logdet}}
\newcommand{\trans}{^{\top}}
\newcommand{\poly}{\text{poly}}
\newcommand{\polylog}{\text{polylog}}
\newcommand{\st}{\text{s.t.~}}
\newcommand{\proj}{\mathcal{P}}
\newcommand{\projII}{\mathcal{P}_{\parallel}}
\newcommand{\projT}{\mathcal{P}_{\perp}}
\newcommand{\projX}{\mathcal{P}_{\mathcal{X}^\star}}
\newcommand{\inner}[1]{\langle #1 \rangle}

\renewcommand{\Pr}{\mathbb{P}}
\newcommand{\Z}{\mathbb{Z}}
\newcommand{\N}{\mathbb{N}}
\newcommand{\R}{\mathbb{R}}
\newcommand{\E}{\mathbb{E}}
\newcommand{\F}{\mathcal{F}}
\newcommand{\var}{\mathrm{var}}
\newcommand{\cov}{\mathrm{cov}}


\newcommand{\calN}{\mathcal{N}}

\newcommand{\jccomment}{\textcolor[rgb]{1,0,0}{C: }\textcolor[rgb]{1,0,1}}
\newcommand{\fracpar}[2]{\frac{\partial #1}{\partial  #2}}

\newcommand{\A}{\mathcal{A}}
\newcommand{\B}{\mat{B}}
%\newcommand{\C}{\mat{C}}

\newcommand{\I}{\mat{I}}
\newcommand{\M}{\mat{M}}
\newcommand{\D}{\mat{D}}
%\newcommand{\U}{\mat{U}}
\newcommand{\V}{\mat{V}}
\newcommand{\W}{\mat{W}}
\newcommand{\X}{\mat{X}}
\newcommand{\Y}{\mat{Y}}
\newcommand{\mSigma}{\mat{\Sigma}}
\newcommand{\mLambda}{\mat{\Lambda}}
\newcommand{\e}{\vect{e}}
\newcommand{\g}{\vect{g}}
\renewcommand{\u}{\vect{u}}
\newcommand{\w}{\vect{w}}
\newcommand{\x}{\vect{x}}
\newcommand{\y}{\vect{y}}
\newcommand{\z}{\vect{z}}
\newcommand{\fI}{\mathfrak{I}}
\newcommand{\fS}{\mathfrak{S}}
\newcommand{\fE}{\mathfrak{E}}
\newcommand{\fF}{\mathfrak{F}}

\newcommand{\Risk}{\mathcal{R}}

\renewcommand{\L}{\mathcal{L}}
\renewcommand{\H}{\mathcal{H}}

\newcommand{\cn}{\kappa}
\newcommand{\nn}{\nonumber}


\newcommand{\Hess}{\nabla^2}
\newcommand{\tlO}{\tilde{O}}
\newcommand{\tlOmega}{\tilde{\Omega}}

\newcommand{\calF}{\mathcal{F}}
\newcommand{\fhat}{\widehat{f}}
\newcommand{\calS}{\mathcal{S}}

\newcommand{\calX}{\mathcal{X}}
\newcommand{\calY}{\mathcal{Y}}
\newcommand{\calD}{\mathcal{D}}
\newcommand{\calZ}{\mathcal{Z}}
\newcommand{\calA}{\mathcal{A}}
\newcommand{\fbayes}{f^B}
\newcommand{\func}{f^U}


\newcommand{\bayscore}{\text{calibrated Bayes score}}
\newcommand{\bayrisk}{\text{calibrated Bayes risk}}

\newtheorem{example}{Example}[section]
\newtheorem{exc}{Exercise}[section]
%\newtheorem{rem}{Remark}[section]

\newtheorem{theorem}{Theorem}[section]
\newtheorem{definition}{Definition}
\newtheorem{proposition}[theorem]{Proposition}
\newtheorem{corollary}[theorem]{Corollary}

\newtheorem{remark}{Remark}[section]
\newtheorem{lemma}[theorem]{Lemma}
\newtheorem{claim}[theorem]{Claim}
\newtheorem{fact}[theorem]{Fact}
\newtheorem{assumption}{Assumption}

\newcommand{\iidsim}{\overset{\mathrm{i.i.d.}}{\sim}}
\newcommand{\unifsim}{\overset{\mathrm{unif}}{\sim}}
\newcommand{\sign}{\mathrm{sign}}
\newcommand{\wbar}{\overline{w}}
\newcommand{\what}{\widehat{w}}
\newcommand{\KL}{\mathrm{KL}}
\newcommand{\Bern}{\mathrm{Bernoulli}}
\newcommand{\ihat}{\widehat{i}}
\newcommand{\Dwst}{\calD^{w_*}}
\newcommand{\fls}{\widehat{f}_{n}}


\newcommand{\brpi}{\pi^{br}}
\newcommand{\brtheta}{\theta^{br}}

% \newcommand{\M}{\mat{M}}
% \newcommand\Mmh{\mat{M}^{-1/2}}
% \newcommand{\A}{\mat{A}}
% \newcommand{\B}{\mat{B}}
% \newcommand{\C}{\mat{C}}
% \newcommand{\Et}[1][t]{\mat{E_{#1}}}
% \newcommand{\Etp}{\Et[t+1]}
% \newcommand{\Errt}[1][t]{\mat{\bigtriangleup_{#1}}}
% \newcommand\cnM{\kappa}
% \newcommand{\cn}[1]{\kappa\left(#1\right)}
% \newcommand\X{\mat{X}}
% \newcommand\fstar{f_*}
% \newcommand\Xt[1][t]{\mat{X_{#1}}}
% \newcommand\ut[1][t]{{u_{#1}}}
% \newcommand\Xtinv{\inv{\Xt}}
% \newcommand\Xtp{\mat{X_{t+1}}}
% \newcommand\Xtpinv{\inv{\left(\mat{X_{t+1}}\right)}}
% \newcommand\U{\mat{U}}
% \newcommand\UTr{\trans{\mat{U}}}
% \newcommand{\Ut}[1][t]{\mat{U_{#1}}}
% \newcommand{\Utinv}{\inv{\Ut}}
% \newcommand{\UtTr}[1][t]{\trans{\mat{U_{#1}}}}
% \newcommand\Utp{\mat{U_{t+1}}}
% \newcommand\UtpTr{\trans{\mat{U}_{t+1}}}
% \newcommand\Utptild{\mat{\widetilde{U}_{t+1}}}
% \newcommand\Us{\mat{U^*}}
% \newcommand\UsTr{\trans{\mat{U^*}}}
% \newcommand{\Sigs}{\mat{\Sigma}}
% \newcommand{\Sigsmh}{\Sigs^{-1/2}}
% \newcommand{\eye}{\mat{I}}
% \newcommand{\twonormbound}{\left(4+\DPhi{\M}{\Xt[0]}\right)\twonorm{\M}}
% \newcommand{\lamj}{\lambda_j}

% \renewcommand\u{\vect{u}}
% \newcommand\uTr{\trans{\vect{u}}}
% \renewcommand\v{\vect{v}}
% \newcommand\vTr{\trans{\vect{v}}}
% \newcommand\w{\vect{w}}
% \newcommand\wTr{\trans{\vect{w}}}
% \newcommand\wperp{\vect{w}_{\perp}}
% \newcommand\wperpTr{\trans{\vect{w}_{\perp}}}
% \newcommand\wj{\vect{w_j}}
% \newcommand\vj{\vect{v_j}}
% \newcommand\wjTr{\trans{\vect{w_j}}}
% \newcommand\vjTr{\trans{\vect{v_j}}}

% \newcommand{\DPhi}[2]{\ensuremath{D_{\Phi}\left(#1,#2\right)}}
% \newcommand\matmult{{\omega}}

	
	%	\newtheorem{theorem}{Theorem}[section]
	%	\newtheorem{lemma}{Lemma}[section]
	\newtheorem{assumption}{Assumption}[section]
	%	\newtheorem{remark}{Remark}[section]
	%	\newtheorem{proposition}{Proposition}[section]
	%	\newtheorem{definition}{Definition}[section]
	%	\newtheorem{corollary}{Corollary}[section]
	%\newtheorem{algorithm}{Algorithm}[section]
	\newtheorem{example}{Example}[section]
	\renewcommand{\theequation}{\arabic{section}.\arabic{equation}}
	\renewcommand{\thetheorem}{\arabic{section}.\arabic{theorem}}
	%	\renewcommand{\thelemma}{\arabic{section}.\arabic{lemma}}
	\renewcommand{\theassumption}{\arabic{section}.\arabic{assumption}}
	%	\renewcommand{\theproposition}{\arabic{section}.\arabic{proposition}}
	\renewcommand{\thedefinition}{\arabic{section}.\arabic{definition}}
	%	\renewcommand{\thecorollary}{\arabic{section}.\arabic{corollary}}
	%\renewcommand{\thealgorithm}{\arabic{section}.\arabic{algorithm}}
	\renewcommand{\theexample}{\arabic{section}.\arabic{example}}
	%\renewcommand{\thefigure}{\arabic{section}.\arabic{figure}}
	
\begin{abstract}
	The atomistic-to-continuum (a/c) coupling methods, also known as the quasicontinuum (QC) methods,  are a important class of concurrent multisacle methods for modeling and simulating materials with defects.  The a/c methods aim to balance the accuracy and efficiency by coupling a molecular mechanics model (also termed as the atomistic model) in the vicinity of localized defects with the Cauchy-Born approximation of the atomistic model in the elastic far field. However, since both the molecular mechanics model and its Cauchy-Born approximation are usually a nonlinear, it potentially leads to a high computational cost for large-scale simulations. In this work, we propose an advancement of the classic quasinonlocal (QNL) a/c coupling method by incorporating a linearized Cauchy-Born model to reduce the computational cost. We present a rigorous {\it a priori} error analysis for this QNL method with linear elasticity enhancement (QNLL method), and show both analytically and numerically that it achieves the same convergence behavior as the classic (nonlinear) QNL method by proper determination of certain parameters relating to domain decomposition and finite element discretization. More importantly, our numerical experiments demonstrate that the QNLL method perform an substantial improvement of the computational efficiency in term of CPU times.
	
	%你讨论的主题不是interface,所以在摘要里没必要把interface和ghost force提出来。The Quasinonlocal (QNL) method, a widely used approach for simulating defect configurations, eliminates the “ghost force” at the interface by reconstructing the site energy. 
	
	%这些参数太细了,别人在摘要里面根本是不明白的。 and show that the convergence is balanced by the lengths of the computational, atomistic, nonlinear continuum, and linear continuum regions, as well as the finite element coarse-graining. 
	
	%However, its nonlinearity leads to high computational costs, making it less efficient for large-scale simulations. 
	%To address this, in this work we propose incorporating the linear elasticity model to improve the computational efficiency of the QNL method. Specifically, we introduce a coupling energy formulation that combines the nonlinear QNL method with the linearized Cauchy-Born model, referred to as the QNLL method. 
	%, and show that the convergence is balanced by the lengths of the computational, atomistic, nonlinear continuum, and linear continuum regions, as well as the finite element coarse-graining. 
	%
	%Our analysis ensures that the QNLL method is able to achieve the same convergence order as the original QNL method but the computational cost is much lower. 
	
	
	
	%Our analysis ensures that the QNLL method is able to achieve the same convergence order as the original QNL method but the computational cost is much lower. 
\end{abstract}
	
	\section{Introduction}
\label{sec:introduction}
The business processes of organizations are experiencing ever-increasing complexity due to the large amount of data, high number of users, and high-tech devices involved \cite{martin2021pmopportunitieschallenges, beerepoot2023biggestbpmproblems}. This complexity may cause business processes to deviate from normal control flow due to unforeseen and disruptive anomalies \cite{adams2023proceddsriftdetection}. These control-flow anomalies manifest as unknown, skipped, and wrongly-ordered activities in the traces of event logs monitored from the execution of business processes \cite{ko2023adsystematicreview}. For the sake of clarity, let us consider an illustrative example of such anomalies. Figure \ref{FP_ANOMALIES} shows a so-called event log footprint, which captures the control flow relations of four activities of a hypothetical event log. In particular, this footprint captures the control-flow relations between activities \texttt{a}, \texttt{b}, \texttt{c} and \texttt{d}. These are the causal ($\rightarrow$) relation, concurrent ($\parallel$) relation, and other ($\#$) relations such as exclusivity or non-local dependency \cite{aalst2022pmhandbook}. In addition, on the right are six traces, of which five exhibit skipped, wrongly-ordered and unknown control-flow anomalies. For example, $\langle$\texttt{a b d}$\rangle$ has a skipped activity, which is \texttt{c}. Because of this skipped activity, the control-flow relation \texttt{b}$\,\#\,$\texttt{d} is violated, since \texttt{d} directly follows \texttt{b} in the anomalous trace.
\begin{figure}[!t]
\centering
\includegraphics[width=0.9\columnwidth]{images/FP_ANOMALIES.png}
\caption{An example event log footprint with six traces, of which five exhibit control-flow anomalies.}
\label{FP_ANOMALIES}
\end{figure}

\subsection{Control-flow anomaly detection}
Control-flow anomaly detection techniques aim to characterize the normal control flow from event logs and verify whether these deviations occur in new event logs \cite{ko2023adsystematicreview}. To develop control-flow anomaly detection techniques, \revision{process mining} has seen widespread adoption owing to process discovery and \revision{conformance checking}. On the one hand, process discovery is a set of algorithms that encode control-flow relations as a set of model elements and constraints according to a given modeling formalism \cite{aalst2022pmhandbook}; hereafter, we refer to the Petri net, a widespread modeling formalism. On the other hand, \revision{conformance checking} is an explainable set of algorithms that allows linking any deviations with the reference Petri net and providing the fitness measure, namely a measure of how much the Petri net fits the new event log \cite{aalst2022pmhandbook}. Many control-flow anomaly detection techniques based on \revision{conformance checking} (hereafter, \revision{conformance checking}-based techniques) use the fitness measure to determine whether an event log is anomalous \cite{bezerra2009pmad, bezerra2013adlogspais, myers2018icsadpm, pecchia2020applicationfailuresanalysispm}. 

The scientific literature also includes many \revision{conformance checking}-independent techniques for control-flow anomaly detection that combine specific types of trace encodings with machine/deep learning \cite{ko2023adsystematicreview, tavares2023pmtraceencoding}. Whereas these techniques are very effective, their explainability is challenging due to both the type of trace encoding employed and the machine/deep learning model used \cite{rawal2022trustworthyaiadvances,li2023explainablead}. Hence, in the following, we focus on the shortcomings of \revision{conformance checking}-based techniques to investigate whether it is possible to support the development of competitive control-flow anomaly detection techniques while maintaining the explainable nature of \revision{conformance checking}.
\begin{figure}[!t]
\centering
\includegraphics[width=\columnwidth]{images/HIGH_LEVEL_VIEW.png}
\caption{A high-level view of the proposed framework for combining \revision{process mining}-based feature extraction with dimensionality reduction for control-flow anomaly detection.}
\label{HIGH_LEVEL_VIEW}
\end{figure}

\subsection{Shortcomings of \revision{conformance checking}-based techniques}
Unfortunately, the detection effectiveness of \revision{conformance checking}-based techniques is affected by noisy data and low-quality Petri nets, which may be due to human errors in the modeling process or representational bias of process discovery algorithms \cite{bezerra2013adlogspais, pecchia2020applicationfailuresanalysispm, aalst2016pm}. Specifically, on the one hand, noisy data may introduce infrequent and deceptive control-flow relations that may result in inconsistent fitness measures, whereas, on the other hand, checking event logs against a low-quality Petri net could lead to an unreliable distribution of fitness measures. Nonetheless, such Petri nets can still be used as references to obtain insightful information for \revision{process mining}-based feature extraction, supporting the development of competitive and explainable \revision{conformance checking}-based techniques for control-flow anomaly detection despite the problems above. For example, a few works outline that token-based \revision{conformance checking} can be used for \revision{process mining}-based feature extraction to build tabular data and develop effective \revision{conformance checking}-based techniques for control-flow anomaly detection \cite{singh2022lapmsh, debenedictis2023dtadiiot}. However, to the best of our knowledge, the scientific literature lacks a structured proposal for \revision{process mining}-based feature extraction using the state-of-the-art \revision{conformance checking} variant, namely alignment-based \revision{conformance checking}.

\subsection{Contributions}
We propose a novel \revision{process mining}-based feature extraction approach with alignment-based \revision{conformance checking}. This variant aligns the deviating control flow with a reference Petri net; the resulting alignment can be inspected to extract additional statistics such as the number of times a given activity caused mismatches \cite{aalst2022pmhandbook}. We integrate this approach into a flexible and explainable framework for developing techniques for control-flow anomaly detection. The framework combines \revision{process mining}-based feature extraction and dimensionality reduction to handle high-dimensional feature sets, achieve detection effectiveness, and support explainability. Notably, in addition to our proposed \revision{process mining}-based feature extraction approach, the framework allows employing other approaches, enabling a fair comparison of multiple \revision{conformance checking}-based and \revision{conformance checking}-independent techniques for control-flow anomaly detection. Figure \ref{HIGH_LEVEL_VIEW} shows a high-level view of the framework. Business processes are monitored, and event logs obtained from the database of information systems. Subsequently, \revision{process mining}-based feature extraction is applied to these event logs and tabular data input to dimensionality reduction to identify control-flow anomalies. We apply several \revision{conformance checking}-based and \revision{conformance checking}-independent framework techniques to publicly available datasets, simulated data of a case study from railways, and real-world data of a case study from healthcare. We show that the framework techniques implementing our approach outperform the baseline \revision{conformance checking}-based techniques while maintaining the explainable nature of \revision{conformance checking}.

In summary, the contributions of this paper are as follows.
\begin{itemize}
    \item{
        A novel \revision{process mining}-based feature extraction approach to support the development of competitive and explainable \revision{conformance checking}-based techniques for control-flow anomaly detection.
    }
    \item{
        A flexible and explainable framework for developing techniques for control-flow anomaly detection using \revision{process mining}-based feature extraction and dimensionality reduction.
    }
    \item{
        Application to synthetic and real-world datasets of several \revision{conformance checking}-based and \revision{conformance checking}-independent framework techniques, evaluating their detection effectiveness and explainability.
    }
\end{itemize}

The rest of the paper is organized as follows.
\begin{itemize}
    \item Section \ref{sec:related_work} reviews the existing techniques for control-flow anomaly detection, categorizing them into \revision{conformance checking}-based and \revision{conformance checking}-independent techniques.
    \item Section \ref{sec:abccfe} provides the preliminaries of \revision{process mining} to establish the notation used throughout the paper, and delves into the details of the proposed \revision{process mining}-based feature extraction approach with alignment-based \revision{conformance checking}.
    \item Section \ref{sec:framework} describes the framework for developing \revision{conformance checking}-based and \revision{conformance checking}-independent techniques for control-flow anomaly detection that combine \revision{process mining}-based feature extraction and dimensionality reduction.
    \item Section \ref{sec:evaluation} presents the experiments conducted with multiple framework and baseline techniques using data from publicly available datasets and case studies.
    \item Section \ref{sec:conclusions} draws the conclusions and presents future work.
\end{itemize}
	
	\section{Atomistic to Nonlinear-Linear Elasticity Coupling Method}
\label{sec: qnll_model}

In this section, we present the atomistic-to-continuum (a/c) coupling methods in one dimension. While the extension to higher dimensions is relatively straightforward, it involves significantly more complex notations. To ensure clarity and brevity, we focus on the one-dimensional case. Section~\ref{sec: introduction_atom} introduces the atomistic model, which serves as the reference framework. Section~\ref{sec: introduction_qnl} provides an overview of the classical nonlinear quasinonlocal (QNL) method. Finally, Section~\ref{sec: introduction_qnll} details the linearization of the Cauchy-Born model and the QNL method with the Linearized Cauchy-Born (QNLL) formulation, which is the main focus of this work.

\subsection{Atomistic model}
\label{sec: introduction_atom}

We consider an infinite atomistic chain or one dimensional crystal lattice indexed by $\Z$. The reference configuration is given by $F\Z$, where $F>0$ is a macroscopic strain. 
We define the space of the displacements and the {\it admissible} set of deformations by
%	\Ua_{0}&:=\{u:\Z\rightarrow\R \ | \ \text{supp} (u) \text{ is bounded} \},\\
\begin{align*}
	\Ua&:=\{u:\Z\rightarrow\R ~|~ \nabla u \in L^{2} \},\\
	\Ya&:=\{y(x)= Fx+u(x) ~|~ u \in \Ua\}.
\end{align*}

Let $\rcut>0$, we fix an interaction range $\Rc := \{\pm1,\dots,\pm\rcut\}$. For each $y \in \Ya$ and $\xi \in \Z$, we define the finite difference stencil 
\begin{equation*}
	Dy(\xi):= \big(D_{\rho}y(\xi)\big)_{\rho \in \mathcal{R}}, \quad \text{where} ~~ D_{\rho}y(\xi):= y(\xi + \rho )-y(\xi).
\end{equation*}
For simplicity, we fix $\rcut = 2$ throughout this work. However, the analysis can be readily extended to a general interaction range.

Let $V\in C^{3}(\R^{\Rc})$ be the interatomic many-body site potential. For a deformation $y\in\Ya$, we define the energy of the infinite atomistic model by
\begin{equation}\label{All-Atomistic Energy}
	\bEa (y):= \sum_{\xi\in\Z}\Phia_{\xi}(y) := \sum_{\xi\in\Z} \left[V\big(Dy(\xi)\big)-V(F\Rc)\right],
\end{equation}
where $\Phia_{\xi}(y)$ is the site energy (per atom energy contribution~\cite{2013_ML_CO_AC_Coupling_ACTANUM}) for the site $\xi$.

For analytical purpose, we assume the regularity of site potential $V$, i.e.,
\begin{align*}
%	\begin{split}
		m(\bm{\rho})&:= \prod_{i =1}^{j} \vert \rho_{i} \vert \sup_{\bm{g} \in \R^{\mathcal{R}}} \Vert V_{\rho}(\bm{g}) \Vert \quad \text{for} \ \bm{\rho} \in \mathcal{R}^{j}, \ \text{and} \\
		M^{(j,s)}&:= \sum_{\bm{\rho} \in \R^{j} } m(\bm{\rho}) \vert \bm{\rho} \vert^{s}_{\infty},
%	\end{split}
\end{align*}
where $V_\rho$ is the derivative of the energy functional $V$ with respect to $\rho$ and $\vert \bm{\rho} \vert_{\infty}:=\max_{i= 1,\dots,j}\vert \rho_{i} \vert$.

By the definition of $\Ya$ and $\Ua$, $\Ea$ is well-defined on $\Ya$, which means that the sum on the infinite lattice is actually finite \cite[Proposition 3.7]{2013_ML_CO_AC_Coupling_ACTANUM}. Given a dead load $f \in \Ya$, the atomistic model is defined by the following minimization problem 
\begin{equation}\label{All-Atomistic solution}
	\yai \in \arg \min\{\bEa(y) -\langle f,y\rangle_{\Z}\ | \ y \in \Ya\},
\end{equation}
where $\langle f,y\rangle_{\Z} = \sum_{\xi \in \Z} f(\xi)y(\xi)$ and ``$\arg\min$'' is understood as the set of local minimizers. Furthermore, we denote the local minimizer of the displacement as $\bar{u}^{\rm a}:=\bar{y}^{\rm a}-Fx$. 

If $\yai$ solves \eqref{All-Atomistic solution}, then it satisfies the first order optimality condition 
\begin{equation}\label{All-Atomistic solution condition}
	\langle \delta \bEa(\yai),v \rangle = \langle f,v\rangle_{\Z}, \quad \forall v \in \Ua.
\end{equation}
In addition to \eqref{All-Atomistic solution condition}, if $\yai \in \Ya$ satisfies the second-order optimality condition which is given by
\begin{equation}\label{All-Atomistic strong local minimizer}
	\langle \delta^{2}\bEa(\yai)v,v\rangle \ge c_{0} \Vert \nabla v \Vert^{2}_{L^{2}},  \quad \forall v\in \Ua,
\end{equation}
for some $c_{0} >0$, we say that the solution $\yai$ is a strongly stable local minimizer.

% \begin{remark}[Existence of the solution and the Decay Hypothesis]
	% \yz{It is difficult (if not impossible) to prove the existence of the solution $\yai$. For a detailed proof of existence, readers can refer to \cite{2016_JB_BS_Existence_CVPD}. Instead, in the literature of the a/c coupling methods, we often assume that the solution 
		% $\yai$ exists \cite{XXXXX}.
		% We also make the following assumption of the property of $\yai$ which is often called the decay hypothesis \cite{2013_ML_CO_AC_Coupling_ACTANUM}.}
	
	To model practical defects and establish a foundation for consistency and stability analysis~\cite{2016_EV_CO_AS_Boundary_Conditions_for_Crystal_Lattice_ARMA}, we introduce the decay hypothesis, which describes the decay and regularity of local minimizers (equilibrium). This assumption ensures smooth asymptotic behavior, which is crucial for the subsequent analysis.
	
	% This hypothesis ensures that the correction term $\ua$ and its derivatives decay at a rate controlled by $\alpha$. The decay of higher-order derivatives ($j=1,2,3$) also guarantees smooth asymptotic behavior, which is critical for consistency and stability analysis.
	
	\textbf{Decay Hypothesis $\DH$}: There exists a strong local minimizer $\bar{u}^{\rm a} \in \Ua$ and $\alpha > 1/2$, for $x$ sufficiently large such that
	\begin{equation}\label{Decay Hypothesis}
		\vert \nabla^{j} \bar{u}^{\rm a}(x) \vert \le \CDH x^{-\alpha+1-j}, \quad j = 0,1,2,3,
	\end{equation}
	where $\CDH>0$ is a constant that depends on lattice and interatomic potentials. 
	
	
	\subsection{Classic quasinonlocal (QNL) methods}
	\label{sec: introduction_qnl}
	%\chw{We give a motivation of introducing the modified reflection method. The motivations are the three points: 1. We need to truncate the domain so that the problem is computable. 2. We apply a continuum model to replace the atomistic model and use finite element approximation to reduce the number of degrees of freedom. 3. The reflection method is show to be universally stable. }
	
	The atomistic problem defined by \eqref{All-Atomistic solution} is computationally intractable due to its formulation on an infinite lattice, its reliance on a nonlocal interaction potential across the entire domain, and the fact that each atom is treated as a separate degree of freedom. To address these challenges, atomistic/continuum (a/c) coupling methods employ domain decomposition strategies to truncate the infinite computational domain, introduce a reduced model in certain regions, and further decrease the number of degrees of freedom through coarse graining. In this section, we illustrate the construction of a/c methods using the classical quasinonlocal (QNL) method~\cite{2011_CO_1D_QNL_MATHCOMP}, which serves as the foundation for the developments presented in subsequent sections.
	
	The a/c coupling methods often make the following three steps of approximations. The first step is to truncate the infinite lattice to a finite domain on which the computation is carried out. The second step is to derive a {\it local} and continuum approximation for the {\it nonlocal} and discrete atomistic model. The third step is to decompose the domain so that the atomistic and the continuum models are properly utilized in different regions and to make a special treatment at the interfaces where the two different models meet so that nonphysical phenomenon is avoided. We now elaborate the construction of the QNL method according to the aforementioned three steps.
	
	\subsubsection{Truncation}
	
	We first truncate the infinite domain simply by fixing an $N\in \Nb$ and define the truncated computational domain to be $\Omega:=[-N,N]$. The set of lattice inside the computational domain is given by $\Lambda := \Omega \cap \Z = \{-N, \ldots, -1, 0, 1, \ldots, N\}$. 
	
	
	
	
	
	% The common practice of dealing with this intractability for the concurrent multiscale methods, or in particular, the 
	
	
	%To limit the degrees of freedom to a finite number, we truncate the infinite atomistic domain $\Z$ to the computational domain $[-N, N]$. 2. The second approximation is to perform a continuum approximation of the atomistic model \eqref{All-Atomistic Energy} using the Cauchy-Born rule. 3. To achieve an quasi-optimal balance between computational accuracy and cost, we couple the atomistic model with its continuum approximation. However, direct coupling can lead to "ghost forces" at the coupling interface, necessitating careful handling of the energy definition at that interface. In this section, we will introduce the reflection method \cite{2014_CO_AS_LZ_Stabilization_MMS}, which eliminates the "ghost forces" through geometric reconstruction.
	
	%In the first step, we will truncate the infinite domain of the atomistic problem based on the first approximation. Since the atomistic problem \eqref{All-Atomistic solution} is defined over an infinite domain, it is not computable. In order to make this problem computable, we fix $N\in \Nb$ and truncate the domain to $[-N,N]$, which contains only a finite number of degrees of freedom.We define the finite-dimensional displacement and deformation spaces as
	
	Though the first step is very easy, we pause here to introduce an auxiliary problem that simplifies our error analysis. Specifically, we apply a Dirichlet boundary condition on $\Omega$ and define the finite dimensional space of displacements and the corresponding admissible set of deformation as
	\begin{align*}
		\Un&:= \{u\in\Ua \ | \ u(\xi) = 0 \text{ for }\xi \le -N \text{ or } \xi \ge N\},\\
		\Yn&:= \{y(x) = Fx + u(x) \ | \ u \in \Un\}.
	\end{align*}
	We can then define the truncated atomistic method (often denoted by ATM \cite{2016_EV_CO_AS_Boundary_Conditions_for_Crystal_Lattice_ARMA,2013_ML_CO_AC_Coupling_ACTANUM}) as 
	\begin{equation}\label{Atomistic solution condition}
		\ya \in \arg \min \{\Ea(y) - \langle f,y \rangle_{N} \ | \ y \in \Yn \},
	\end{equation}
	where $\Ea(y): =  \sum_{\xi =-N}^{N}\Phia_{\xi}(y)$ and $\langle f,y\rangle_{N} = \sum_{\xi =-N}^{N} f(\xi)y(\xi)$, which can be considered as a Galerkin approximation of the original atomistic model \eqref{All-Atomistic solution}. 
	
	Since our atomistic/continuum coupling methods are all defined on the finite domain $\Omega$, we can bound the error between the solution of any of the a/c methods $y^{\rm ac}$ and that of original atomistic model $\yai$ by 
	\begin{equation}
		\label{eq: error separation}
		\| y^{\rm ac} - \yai \| \le \|y^{\rm ac} - \ya \| + \|\ya - \yai \|.
	\end{equation} 
	We will only concentrate on the first part of the right hand side of \eqref{eq: error separation} in the error analysis and the estimate of the second part, which is essentially the truncation error, is given by the following lemma \cite[Theorem 3.14]{2013_ML_CO_AC_Coupling_ACTANUM}:
	
	\begin{lemma}
		Let $y^{{\rm a}}$ be a strong local minimizer of \eqref{All-Atomistic solution condition} satisfying $\DH$ and \eqref{All-Atomistic strong local minimizer}. Then there exists $N_{0} \in \Nb$ such that, for all $N \ge N_{0}$, there exists a strong local minimizer $\ya$ of \eqref{Atomistic solution condition} satisfying
		\begin{equation}\label{Truncation error}
			\Vert \nabla \yai - \nabla \ya\Vert_{L^{2}} \le \frac{16M^{(2,0)}\CDH}{\sqrt{2\alpha-1}} N^{\frac{1}{2}-\alpha}.
		\end{equation}
	\end{lemma}
	
	%\begin{proof}
	%    Follow  [].
	%\end{proof}
	
	
	%Due to the above results and the sake of theoretical simplicity in the following sections, our error estimates will usually use the truncated solution  instead of the atomistic solution. For the sake of simplicity in notation, we denote $\ya$ as $\ya$ in next sections. For simplicity in the subsequent theoretical framework, we restrict the computational domain to $[-N,N]$ in both the QNL model of this subsection and the QNLL model of the next subsection.
	
	
	\subsubsection{Continuous approximation}
	
	Next, we consider the second approximation, which involves using a local continuum model to approximate the original nonlocal atomistic model. To further reduce the number of degrees of freedom, we apply the continuum approximation to the atomistic potential \eqref{All-Atomistic Energy} based on the Cauchy-Born rule~\cite{2013_CO_FT_Cauchy_Born_ARMA}, where we define the strain energy density from the interaction potential $V$ as 
	\begin{equation}
		\label{Cauchy-Born site energy}
		W(F) := V(F\Rc).
	\end{equation}
	
	
	\subsubsection{Domain decomposition}
	
	Finally, we will achieve the third approximation through the atomistic-to-continuum coupling method. The Atomistic-to-Continuum coupling method achieves an quasi-optimal trade-off between computational cost and accuracy by coupling the atomistic model with its continuum approximation. In this paper, the coupling scheme we use is the reflection method, which is show to be universally stable  \cite{2014_CO_AS_LZ_Stabilization_MMS}, and we will discuss in stability analysis. Because we fix $\rcut = 2$, the reflection method is equivalent to the Quasi-nonlocal method (QNL method) in this case \cite{2013_ML_CO_AC_Coupling_ACTANUM}. The QNL method eliminates the ``ghost force" term by providing a precise definition of the energy on the interface \cite{2011_CO_1D_QNL_MATHCOMP}. Therefore, for the sake of simplicity in naming, the remaining sections of this paper will use the QNL method to uniformly refer to the reflection method.
	
	We first of all decompose the computational domain $\Omega$ into three different regions. The first one is the atomistic region $\OmeA$ in which we assume the defect core is contained. The second one is the continuum region $\OmeC$ where we assume the deformation is smooth enough. The third one is th interface region $\OmeI$ which consists of a small number of layers of atoms between $\OmeA$ and $\OmeC$. The lattice points in the three regions are defined by $\Ac:= \OmeA \cap \Lambda$, $\Cc:= \OmeC \cap \Lambda$ and $\Ic := \OmeI \cap \Lambda$. To be more specific, we denote them by (for simplicity, we denote $\bar{K} = K+\rcut$)
	\begin{align*}
		\Ac &:=\{-K,-K+1,\dots,K\}, \qquad \qquad \qquad \qquad \qquad \qquad \qquad \quad ~~  \text{(Atomistic region)},\\
		\Ic &:=\{-\bar{K},-\bar{K}+1\}\cup \{\bar{K}-1,\bar{K}\}, \qquad \qquad \qquad \qquad \qquad \qquad ~~ \text{(Interface region)},\\
		\Cc &:=\{-N,-N+1,\dots,-\bar{K}-1\}\cup\{\bar{K}+1,\bar{K}+2,\dots,N\}, \qquad ~ \text{(Continuum region)}.
	\end{align*}
	
	Since we will calculate the integral on continuum region, we define the following three continuous intervals:
	\begin{align*}
		\OmeA &=[-K-1,K+1],\\
		\OmeI &=[-\bar{K},-\bar{K}+1] \cup [\bar{K}-1,\bar{K}],\\
		\OmeC&=[-N,-\bar{K}]\cup [\bar{K},N].
	\end{align*}
	
	%We fix $\bar{L} \in \Nb$, $L \le \bar{L} \le N/2$. We choose a set of finite element nodes $\Nh = \{v_{0},\dots,V_{N_{\Th}}\} \subset \Z_{+}$, for some $N_{\Th} \in \Nb$, such that $\{0,\dots,\bar{L},N\}\subset\Nh\subset \{0,\dots,N\}$. The finite elements are given by 
	%\begin{equation*}
	%	\Th = \{[v_{j-1},v_{j}] \ | \ j = 1,\dots,N_{\Th}\}.
	%\end{equation*}
	%
	%For each $T\in\Th$, let $h_{T} = \text{diam}T$. For each $x\in[0,N]$, $x\in\text{int} T$, let $h(x):=h_{T}$. For $x>N$, let $h(x)=1$. We define the coarse displacement sapce by
	%\begin{align*}
	%	\Uh &= \{\uh \in \Un \ | \ \uh \text{ is piecewise affine with respect to } \Th\},\\
	%	\Yh &= \{F\Z + \uh \ | \ \uh \in \Uh \}
	%\end{align*}
	%
	%For any function $v:\Nh \rightarrow \R$, let $\Ih v:[0,N] \rightarrow \R$, $\Ih v \in \text{P1}(\Th)$, denote its continuous piecewise affine interpolant,
	%\begin{equation*}
	%	\Ih v(\zeta):=v(\zeta) \quad \text{for all }\zeta \in \Nh.
	%\end{equation*}
	%
	%For $f,g:\Nh \rightarrow \R^{N_{\Th}}$, we define
	%\begin{equation*}
	%	\langle f,g \rangle_{h}:=\int_{0}^{N}\Ih(f\cdot g)\d x = \sum_{j=1}^{N_{Th}} \frac{1}{2} h_{T}\{f(v_{j-1})\cdot g(v_{j-1})+f(v_{j})\cdot g(v_{j})\}.
	%\end{equation*}
	
	
	
	Moreover, we define, for any $x\in (\xi-1,\xi), \nabla y(x) = y(\xi)-y(\xi-1)$, which is a piecewise constant interpolation of $y$ with respect to lattice sites. The specific construction of the QNL model is given as follows:
	\begin{equation}\label{QNL energy}
		\Erfl (y)=\sum_{\xi \in \Ac}\Phia_{\xi}(y) + \sum_{\xi \in \Ic}\Phii_{\xi}(y) + \int_{\OmeC} W(\nabla y)\,\d x,
	\end{equation}
	where the interface energy reads
	\begin{align*}
		\Phii_{-L}(y)&=V(-D_{2}y,-D_{1}y,D_{1}y,D_{2}y)+\int_{-L-1}^{-L-\frac{1}{2}}W(\nabla y)\,\d x,\\
		\Phii_{-L+1}(y)&=V(2D_{-1}y,D_{-1}y,D_{1}y,D_{2}y),\\
		\Phii_{L-1}(y)&=V(D_{-2}y,D_{-1}y,D_{1}y,2D_{1}y),\\
		\Phii_{L}(y)&=V(D_{-2}y,D_{-1}y,-D_{-1}y,-D_{-2}y)+\int_{L+\frac{1}{2}}^{L+1}W(\nabla y)\,\d x.
	\end{align*}
	
	Given a dead load $f\in \Ya$, we seek solution
	\begin{equation}\label{QNL solution conditon}
		\yrfl \in \arg \min \{\Erfl (y) - \langle f,y\rangle_{N} \ | \ y \in \Yn\}.
	\end{equation}
	Similarly, if $\yrfl$ solves \eqref{QNL solution conditon}, then it satisfies the first order optimal condition
	\begin{equation}\label{QNL solution first order optimality condition}
		\langle \delta \Erfl(\yrfl ),v \rangle = \langle f,v\rangle_{N}, \quad \text{for all }v \in \Un.
	\end{equation}
	
	\subsection{QNL with linearized Cauchy-Born (QNLL) method}
	\label{sec: introduction_qnll}
	
	In this section, we will introduce the QNL Methods with Linearized Cauchy-Born (QNLL). The idea behind QNLL method is to replace a nonlinear elasticity model with a linear elasticity model to further simplify the computation. To that end, we make further approximations: based on the QNL method, we use a linear elasticity model to approximate the nonlinear elasticity model. Compared to the nonlinear model, the linear model can effectively reduce computational costs. However, this also introduces new errors, which will be analyzed in the next section.
	
	First, we construct the linear elasticity model. Here, we obtain the linear elasticity model by performing a Taylor expansion of the Cauchy-Born energy \eqref{Cauchy-Born site energy} around the uniform deformation. After introducing linearization, we will introduce the nonlinear-linear elasticity coupling (QNLL) model, which essentially replaces a portion of the nonlinear continuum energy functional $W$ in a continuum region $\OmeC$ with a linear energy functional $\WL$. Here, we first provide the form of $\WL$. For simplicity, we denote $W^{(k)}_{F} = W^{(k)}(F),\ k =0,1,2$. We use Taylor's expansion in order to linearize Cauchy-Born strain energy density: 
	\begin{equation*}
		W(\nabla y) = \Wf+\Wpf\nabla u+\frac{1}{2}\Wppf(\nabla u)^{2} +\dots.
	\end{equation*}
	We then define the linearized Cauchy-Born strain energy density:
	\begin{equation*}
		\WL(\nabla y) = \Wf+\Wpf\nabla u+\frac{1}{2}\Wppf(\nabla u)^{2}.
	\end{equation*}
	
	Next, based on the region partitioning of the QNL method, we will further subdivide $\Cc$ into nonlinear and linear regions. We denote them by
	\begin{align*}
		\Cnl &= \{- L,L+1,\dots,-\bar{K}-1\}\cup\{\bar{K}+1, \bar{K}+2,\dots,L\},\\
		\Cl &=\{-N,-N+1,\dots,L-1\}\cup \{L+1,L+2,\dots,N\}.
	\end{align*}
	Similarly, we will also divide $\OmeC$ into nonlinear and linear parts:
	\begin{align*}
		\OmeNL &=[-L,-\bar{K}] \cup [\bar{K},L],\\
		\OmeL &=[-N,L] \cup [L,N].
	\end{align*}
	
	In the linearized continuouum region, by approximating the nonlinear elasticity model with the linear elasticity model, we obtain the QNLL model:
	\begin{align}\label{Nonlinear-linear energy}
		\El (y) &= \sum_{\xi = -N}^{N}\PhiNLL_{\xi}(y) \nonumber \\
		&=\sum_{\xi \in \Ac}\Phia_{\xi}(y) + \sum_{\xi \in \Ic}\Phii_{\xi}(y) + \int_{\OmeNL} W(\nabla y)\,\d x + \int_{\OmeL} \WL(\nabla y)\,\d x.
	\end{align}
	
	
	Given a dead load $f\in \Ya$, we seek
	\begin{equation}\label{Nonlinear-linear solution condition}
		\ynll \in \arg \min \{\El (y) - \langle f,y\rangle_{N} \ | \ y \in \Yn\}.
	\end{equation}
	If $\ynll$ solves \eqref{Nonlinear-linear solution condition}, then it satisfies the first order optimal condition
	\begin{equation}\label{QNLL solution first order optimality condition}
		\langle \delta \El(y),v \rangle = \langle f,v\rangle_{N}, \quad \text{for all }v \in \Un.
	\end{equation}
	
	\section{A Priori Analysis for the QNLL Method}
\label{sec: anal_qnll_ncg}

% \chw{The total consistency error is defined by $\langle \delta \El (y^a), v \rangle$ and then plus and minus proper terms to separate the total consistency error into several parts. The following is not precise. We can only bound $\|\ya - \ynll \| \le$ truncation error $+$ coupling error $+$ linearization error}

In this section, we will provide the {\it a priori} error estimate for the atomistic to nonlinear-linear elasticity coupling model $\Vert \nabla \yai - \nabla \ynll \Vert_{L^{2}}$, following analytical framework shown in~\cite{2013_ML_CO_AC_Coupling_ACTANUM,2011_CO_1D_QNL_MATHCOMP,2011_CO_HW_QC_A_Priori_1D_M3AS}. We will first present the consistency error estimate and stability analysis results of the QNLL model, and then give the {\it a priori} error estimate based on the inverse function theorem (Lemma \ref{Inverse function theorem}) provided in the Appendix. The detailed consistency error analysis is provided in Section \ref{sec: consistency_qnll_ncg}, stability analysis in Section \ref{sec: stability_qnll_ncg}, and the final {\it a priori} error estimate in Section \ref{sec: priori_qnll_ncg}. Finally, in Section~\ref{sec: balance_of_qnll_ncg_model}, based on the {\it a priori} error estimate and the $ \DH $ assumption, we propose a method to balance the lengths of the regions in the QNLL model to achieve the same convergence order as the QNL model.


% In Section \ref{sec: consistency_qnll_ncg}, we will prove the a priori error estimate for the QNLL solution \eqref{Nonlinear-linear solution condition} based on consistency error and stability analysis. The consistency error includes both the coupling error and the linearization error. In the second part, we will perform the stability analysis. In the third part, we will give the a priori error estimate based on the inverse function theorem, which can be found in Lemma \ref{Inverse function theorem} in the Appendix. 






%\begin{lemma}\label{Inverse function theorem}
%    Let $\Yn$ be a subspace of $\Ya$, equipped with $\Vert \nabla \cdot \Vert_{L^{2}}$, and let $\Ghc \in C^{1}(\Yn,\Yn^{*})$ with Lipschitz-continuous derivative $\delta \Ghc$:
%    \begin{equation*}
	%        \Vert\delta\Ghc(y) - \delta\Ghc(v)\Vert_\mathcal{L} \le M \Vert \nabla y - \nabla v \Vert_{L^{2}}, \quad \text{for all} \  v \in \Un,
	%    \end{equation*}
%    where $\Vert \cdot \Vert_{\mathcal{L}}$ denotes the $\mathcal{L}(\Yn,\Yn^{*})$-operator norm.
%
%    Let $\bar{y}\in \Yn$ satisfy
%%    \begin{equation}
	%        \begin{align}
		%            \Vert \Ghc(\bar{y})\Vert_{\Yn^{*}} &\le \eta,\\
		%            \langle \delta \Ghc (\bar{y})v,v\rangle &\ge \gamma \Vert \nabla v \Vert^{2}_{L^{2}},  \quad \text{for all} \ v \in \UhNL,
		%        \end{align}
	%%    \end{equation}
%    such that $L,\eta,\gamma$ satisfy the relation
%    \begin{equation}
	%        \frac{2M\eta}{\gamma^{2}}<1.
	%    \end{equation}
%
%    Then there exists a (locally unique) $\ynll\in \Yn$ such that $\Ghc(\ynll)=0$,
%%    \begin{eqution}
	%        \begin{align}
		%            \Vert \nabla \ynll-\nabla \bar{y}\Vert_{L^{2}} &\le 2\frac{\eta}{\gamma}, \quad \text{and}\\
		%            \langle \delta \Ghc (\ynll)v,v\rangle &\ge (1-\frac{2M\eta}{\gamma^{2}}) \Vert \nabla v \Vert^{2}_{L^{2}},  \quad \text{for all} \ v \in \Un,
		%        \end{align}
	%%    \end{eqution}
%\end{lemma}
%
%\begin{proof}
%The result is a simplified and specialized version of Lemma 2.2 of
%\cite{2011_CO_1D_QNL_MATHCOMP}, but similar statements can be obtained from most proofs of the inverse function theorem.
%\end{proof}

\subsection{Consistency error}
\label{sec: consistency_qnll_ncg}

In the consistency error estimate, we decompose the total error using the triangle inequality into two components: (i) the coupling error, which quantifies the difference between the atomistic model and the QNL model, and (ii) the linearization error, which arises from the transition from the QNL model to the QNLL model. By estimating these two errors separately, we derive the overall consistency error between the atomistic model and the QNLL model. Specifically, we have
\begin{align*}
	T(\ya) &= \langle \delta \Ea (\ya),v\rangle -\langle \delta \El (\ya),v\rangle\\
	&=\langle \delta \Ea (\ya),v\rangle -\langle \delta \Erfl (\ya),v\rangle ~~\qquad \text{(the coupling error)}\\
	&\ \ +\langle \delta \Erfl (\ya),v\rangle -\langle \delta \El (\ya),v\rangle. \quad \text{(the linearization error)}
\end{align*}
In the following, we establish an error bound for $T(y^{\rm a})$.
% where $\Vert T \Vert_{\Yn^{*}}$ represents the consistency error estimate in this section.

\subsubsection{QNL coupling error}

We define the weighted characteristic function of a bond $(\xi ,\xi +\rho)$ by
\begin{equation}\label{Definition of Ki}
	\Ki:=\left\{
	\begin{aligned}
		&\vert\rho\vert^{-1}, &x\in \text{int} (\conv \{\xi,\xi+\rho\}), \\
		&\frac{1}{2}\vert\rho\vert^{-1}, &x\in \{\xi,\xi+\rho\}, \\
		&0, &\text{otherwise}.
	\end{aligned}
	\right.
\end{equation}
We then obtain for $D_{\rho}v(\xi) = v(\xi+\rho) -v(\xi)$ that
\begin{equation}\label{Diff to int}
	D_{\rho}v(\xi)=\int_{\xi}^{\xi+\rho}\frac{\rho}{\vert\rho\vert}\nabla v\,\d x
	=\int_{\R}\rho \Ki\nabla v\,\d x.
\end{equation}

The first variation of the atomistic energy functional \eqref{All-Atomistic Energy} at $y \in \Yn$ is given by
\begin{equation*}
	\langle \delta \Ea(y), v\rangle = \sum_{\xi = -N}^{N} \sum_{\rho \in \Rc}\Phia_{\xi,\rho}(y)D_{\rho}v(\xi).
\end{equation*}
We apply \eqref{Diff to int}, it follows that
\begin{equation}\label{Atomistic first variation}
	\langle \delta \Ea(y), v\rangle = \int_{-N}^{N} \Sa \nabla v\,\d x.
\end{equation}
where
\begin{equation}\label{Atomistic stress tensor}
	\Sa = \sum_{\xi=-N}^{N} \sum_{\rho \in \Rc} \rho \Ki \Phia_{\xi,\rho}(y).
\end{equation}



% \chw{We estimate the error committed by replacing the atomistic model by the QNL method in this part. We need to emphasize that compared with existing work, we do not prescribe any specific finite element discretization here.}

% We shall require two additional restrictions on the interface site energies, as discussed by \cite{2013_ML_CO_AC_Coupling_ACTANUM}, which we call the locality and scaling conditions.
% $\L$ Locality. $\Phii_{\xi}(y)$ is a $3$ times continuous differentiable function of the interaction stencil $Dy(\xi) = \big(D_{\rho}y(\xi)\big)_{\rho \in \Rc}$. Moreover, $\Phii_{\xi,\rho}(y) = 0$ for $\xi + \rho >L$.

% $\S$ Scaling. $\vert \Phii_{\xi,\bm{\rho}}(y)\vert \lesssim c(\bm{\rho})$, for $\bm{\rho} \in \Rc, j=2,3$, where
% \begin{equation*}
	% 	\sum_{\bm{\rho}\in \Rc^{j}}\vert \bm{\rho} \vert^{s}_{\infty}\prod_{i=1}^{j}\vert \rho_{i} \vert c(\bm{\rho}) \lesssim M^{(j,s)}, \quad \text{for } 0\le s \le 3.
	% \end{equation*}
The first variation of the energy functional of QNL approximation defined by \eqref{QNL energy}, for any $v\in \YacNL$, we have 
\begin{equation}\label{QNL first variation}
	\begin{split}
		\langle \delta \Erfl (y), v\rangle &= \sum_{\xi \in \Ac} \sum_{\rho \in \Rc} \Phia_{\xi,\rho}(y)D_{\rho}v(\xi) + \int_{\OmeC}\partial_{F} W(\nabla y)\nabla v\,\d x \\
		&=: \int_{-N}^{N} \Srfl \nabla v \,\d x.
	\end{split}
\end{equation}
where
\begin{equation}\label{QNL stress tensor}
	\Srfl := \left\{
	\begin{aligned}
		&\sum_{\xi \in \Ac}\sum_{\rho \in \Rc}\rho \Ki \Phia_{\xi,\rho}(y) +\sum_{\xi \in \Ic}\sum_{\rho \in \Rc}\rho \Ki \Phii_{\xi,\rho}(y), &x\in \OmeA \cup \OmeI, \\
		&\partial_{F}W(\nabla y), &x\in \OmeC.
	\end{aligned}
	\right.
\end{equation}


% \begin{lemma}\label{ki sum 1}
	%  Let $\xi \in \Z, \rho \in \Rc$, and $\Ki$ is defined by \eqref{Definition of Ki}, then we have
	% \begin{equation*}
		% 	\sum_{\xi \in \Z}\Ki=1.
		% \end{equation*}	
	% \end{lemma}

We now define the error in QNL model stress as
\begin{equation*}
	\Rrfl := \Srfl -\Sa.
\end{equation*}
Next, we will provide a point estimate of the stress tensor for the QNL model. For the sake of presentation simplicity, we leave the detailed proof, please refer to Appendix \ref{Appendix section 1}.

\begin{proposition}\label{Pointwise coupling stress tensor}
	Let $y\in\Ya, x \in [-N,N]$, then 
	\begin{equation*}
		\vert \Rrfl \vert \lesssim \left\{
		\begin{aligned}
			&0, &x \in \OmeA \backslash \bOmeA,\\
			&M^{(2,1)}\Vert \nabla^{2}u\Vert_{L^{\infty}(v_{x})}, &x\in \OmeI \cup \bOmeA,\\
			&M^{(2,2)}\Vert \nabla^{3}u\Vert_{L^{\infty}(v_{x})}+M^{(3,2)}\Vert \nabla^{2}u\Vert_{L^{\infty}(v_{x})}^{2}, &x \in \OmeC
		\end{aligned}
		\right.
	\end{equation*}
	where $v_{x}:=\big[\lfloor x\rfloor+1-2\rcut,\lfloor x\rfloor+2\rcut\big]$ is the neighborhood of some $x\in \R$, the meaning of $\lfloor x\rfloor$ here is to take the floor of $x$, which is the greatest integer less than or equal to $x$, and $\bOmeA = [-K,-K+\rcut]\cup [K-\rcut,K]$.
\end{proposition}

% \begin{proof}
	% 	The proof of the coupling error estimate for the QNL model is derived from [\cite{2013_ML_CO_AC_Coupling_ACTANUM}, Lemma 6.12]. Since $\text{supp}(\Ki) = \conv\{\xi,\xi+\rho\}$, $\Rrfl=0$ for $x\in\OmeA \backslash \bOmeA$. As for $x\in \OmeI \cup \bOmeA$,we obtain
	% 	\begin{align*}
		% 		\Rrfl &=\Srfl-\Sa\\
		% 		&=\sum_{\xi \in \Ic} \sum_{\rho \in \Rc}\rho\Ki\Phii_{\xi,\rho}(y)-\sum_{\xi \in\Ic\cup\Cc}\sum_{\rho \in \Rc} \rho \Ki \Phia_{\xi,\rho}(y).
		% 	\end{align*}
	% We define uniform deformation $\yF(\xi)=F\xi$. We notice that $R^{\text{rfl}}(\yF;x)=0$, hence we obtain
	% \begin{equation}\label{Two parts of Rrfl}
		% \begin{aligned}
			% 	\Rrfl &= \Rrfl-R^{\text{rfl}}(\yF;x)\\
			% 	&= \sum_{\xi \in \Ic} \sum_{\rho \in \Rc}\rho\Ki(\Phii_{\xi,\rho}(y)-\Phii_{\xi,\rho}(\yF))\\
			% 	&\ +\sum_{\xi \in\Ic\cup\Cc}\sum_{\rho \in \Rc} \rho \Ki (\Phia_{\xi,\rho}(\yF)-\Phia_{\xi,\rho}(y)).
			% \end{aligned}
		% \end{equation}
	
	% For the first term of \eqref{Two parts of Rrfl}, we choose $F=\nabla y$ and use Taylor's expansion. After considering $\S$, we have
	% \begin{equation}\label{Interface stess}
		% 	\vert \Phii_{\xi,\rho}(y)-\Phii_{\xi,\rho}(\yF) \vert \le \sum_{\zeta \in \Rc} c(\rho,\zeta)\vert D_{\zeta}y(\xi)-\nabla_{\zeta}y(x)\vert.
		% \end{equation}
	
	% After using Taylor's expansion, for $x\in \conv\{\xi,\xi+\rho\}$, we obtain
	% \begin{align*}
		% 	\vert D_{\zeta}y(\xi)-\nabla_{\zeta}y(x)\vert&= \vert \zeta(\xi-x+\frac{\zeta}{2})\vert \cdot \Vert \nabla^{2}y\Vert_{L^{\infty}(\conv\{\xi,\xi+\rho\})}\\
		% 	&\le\frac{\vert \zeta \vert^{2}}{2}\Vert \nabla^{2}u\Vert_{L^{\infty}(\conv\{\xi,\xi+\rho\})}.
		% \end{align*}
	
	% Applying assumption $\S$ again, we get
	% \begin{align*}
		% 	\sum_{\xi \in \Ic} \sum_{\rho \in \Rc} \vert\rho \vert\Ki \vert \Phii_{\xi,\rho}(y)-\Phii_{\xi,\rho}(\yF) \vert &\le \sum_{(\rho,\zeta)\in\Rc^{2}} \frac{1}{2}\vert \rho \zeta^{2}\vert c(\rho,\zeta) \Vert \nabla^{2}y\Vert_{L^{\infty}(v_{x})}\sum_{\xi \in \Ic}\Ki\\
		% 	&\lesssim M^{(2,1)}\Vert \nabla^{2}y\Vert_{L^{\infty}(v_{x})}.
		% \end{align*}
	
	% Similarly, we could calculate that
	% \begin{align*}
		% 	\sum_{\xi \in\Ic\cup\Cc} \sum_{\rho \in \Rc} \vert\rho \vert\Ki \vert \Phia_{\xi,\rho}(y)-\Phia_{\xi,\rho}(\yF) \vert &\le \sum_{\rho \in \Rc} \frac{1}{2} \vert \rho \zeta^{2}\vert m(\rho,\zeta) \\
		% 	&\le M^{(2,1)} \Vert \nabla^{2}y\Vert_{L^{\infty}(v_{x})}.
		% \end{align*}
	
	% For $x\in \OmeC \cap \Z+\frac{1}{2}$, we have
	% \begin{equation*}
		% 	\Rrfl = \partial_{F}W(\nabla y(x))-\Sa,
		% \end{equation*}
	% which is the difference between the atomistic stress tensor and Cauchy-Born stress tensor. And we mention that the error estimate follows directly from[\cite{2013_ML_CO_AC_Coupling_ACTANUM}, Theorem6.2]
	% \begin{align*}
		% 	\Rrfl &= \partial_{F}W(\nabla y(x))-\Sa\\
		% 	&\lesssim M^{(2,2)}\Vert \nabla^{3}y\Vert_{L^{\infty}(v_{x})}+M^{(3,2)}\Vert \nabla^{3}y\Vert_{L^{\infty}(v_{x})}^{2}.
		% \end{align*}
	
	% By definition of $\Ua$ and $\Ya$, we notice $\nabla^{2}y =\nabla^{2}u,\ \nabla^{3}y=\nabla^{3}u$. So for simplify, we choose norm of $u$ in final result.
	% \end{proof}

Based on the point estimate of $\Rrfl$ provided above, we will give the consistency error estimate for the QNL model. Please refer to Appendix \ref{Appendix section 2} for the proof.

\begin{proposition}\label{Coupling consistency error estimate}
	We split coupling error into two parts:
	\begin{equation*}
		\int_{-N}^{N}\Rrfl\nabla v\,\d x=\int_{\OmeI\cup\bOmeA}\Rrfl\nabla v\,\d x-\int_{\OmeC}\Rrfl\nabla v\,\d x.
	\end{equation*}
	
	For any $v\in \Yn$, we have
	\begin{equation}\label{Interface region stress tensor}
		\int_{\OmeI\cup\bOmeA}\Rrfl\nabla v\,\d x \lesssim M^{(2,1)}\Vert\nabla^{2}u\Vert_{L^{2}(\bOmeI)}\Vert\nabla v\Vert_{L^{2}(\OmeI\cup\bOmeA)},
	\end{equation}
	and
	\begin{equation}\label{Continuum region stress tensor}
		\int_{\OmeC}\Rrfl\nabla v\,\d x \lesssim(M^{(2,2)}\Vert \nabla^{3}u\Vert_{L^{2}(\bOmeC)}+M^{(3,2)}\Vert \nabla^{2}u\Vert_{L^{4}(\bOmeC)}^{2})\Vert\nabla v\Vert_{L^{2}(\OmeC)},
	\end{equation}
	where $\bOmeI = [-\bar{K}+1-2\rcut,-\bar{K}+4+2\rcut]\cup[\bar{K}-3-2\rcut,\bar{K}+2\rcut], \bOmeC=[-N-2\rcut,-\bar{K}-1+2\rcut]\cup[\bar{K}+1-2\rcut,N+2\rcut]$.
\end{proposition}

% \begin{proof}
	
	% \end{proof}

% \begin{proof}
	% 	For \eqref{Interface region stress tensor}, the main point of this proof is to use the inverse estimates to obtain $L^{2}-$type from the $L^{\infty}$ bounds\cite{2007_Braess_FEM}
	% 	\begin{equation}\label{L-infty to L-2 estimate}
		% 		\Vert \nabla^{2}u\Vert_{L^{\infty}(v_{x})}\lesssim \Vert \nabla^{2}u\Vert_{L^{2}(v_{x})}.
		% 	\end{equation}
	
	% After a direct calculation, we have
	% \begin{align*}
		% 	\int_{\OmeI\cup\bOmeA}\Rrfl\nabla v\d x&\lesssim \int_{\OmeI\cup\bOmeA} M^{(2,1)}\Vert \nabla^{2}u\Vert_{L^{\infty}(v_{x})} \vert \nabla y \vert \d x\\
		% 	&\le M^{(2,1)}\Vert \nabla^{2}u\Vert_{L^{\infty}(\bOmeI)}\Vert \nabla v \Vert_{L^{1}(\OmeI\cup\bOmeA)}\\
		% 	&\lesssim  M^{(2,1)}\Vert \nabla^{2}u\Vert_{L^{\infty}(\bOmeI)}\Vert \nabla v \Vert_{L^{2}(\OmeI\cup\bOmeA)}.
		% \end{align*}
	
	% As for \eqref{Continuum region stress tensor}, combining Theorem \ref{Pointwise coupling stress tensor} and [\cite{2013_ML_CO_AC_Coupling_ACTANUM}, Corollary6.4], we yield the started results.
	% \end{proof}

\subsubsection{Linearization error}

Firstly, we calculate the first variation of the energy functional of Nonlinear-linear elasticity coupling energy defined by \eqref{Nonlinear-linear energy}, for any $v\in\YacNL$, is then given by
\begin{equation*}
	\langle \delta \El (y),v\rangle=:\int_{-N}^{N} \SNLL\nabla v\,\d x,
\end{equation*}
where
\begin{equation}\label{LInearization stress tensor}
	\SNLL := \left\{
	\begin{aligned}
		&\sum_{\xi \in \Ac}\sum_{\rho \in \Rc}\rho \Ki \Phia_{\xi,\rho}(y) +\sum_{\xi \in \Ic}\sum_{\rho \in \Rc}\rho \Ki \Phii_{\xi,\rho}(y) \quad x\in \OmeA \cup \OmeI, \\
		&\partial_{F}W(\nabla y) \qquad \qquad \qquad \qquad \qquad \qquad \qquad \qquad \quad  \quad~~ x\in \OmeNL,\\
		&\partial_{F}W_{\text{L}}(\nabla y) \qquad \qquad \qquad \qquad \qquad \qquad \qquad \qquad \qquad~ x\in \OmeL.
	\end{aligned}
	\right.
\end{equation}

We now define the error in linearization as
\begin{equation*}
	\RNLL=\SNLL-\Srfl.
\end{equation*}
We will provide a point-wise estimate of the stress tensor for the QNLL model using the definition of $\WL$.

\begin{theorem}
	Let $y\in \YacNL,x\in[-N,N]$, we have
	\begin{equation}\label{Pointwise NLL stress tensor}
		\vert \RNLL\vert \le \frac{1}{2}M^{(3,0)} \Vert \nabla u(x)\Vert^{2}_{L^{\infty}(\OmeL)}, \ x\in \OmeL.
	\end{equation}
\end{theorem}

\begin{proof}
	We notice that $\RNLL \neq 0$, only for $x\in \OmeL$. We could directly know
	\begin{equation*}
		\partial_{F}\WL(\nabla y) = \Wpf+\Wppf\nabla u.
	\end{equation*}
	After using the Taylor's expansion, we obtain
	\begin{equation*}
		\vert \partial_{F}W_{\text{L}} (\nabla y)-\partial_{F}W(\nabla y)\vert \le \frac{1}{2} M^{(3,0)} \vert \nabla u(x)\vert^{2}\le\frac{1}{2}M^{(3,0)}\Vert \nabla u(x)\Vert^{2}_{L^{\infty}(\OmeL)}.
	\end{equation*}
\end{proof}

Based on the point estimate of $\RNLL$ provided above, we will now present the consistency error estimate for the QNLL model.

\begin{theorem}\label{Linearization consistency error estimate}
	For any $v \in \YacNL$, we have
	\begin{equation}\label{Linearization error estimate}
		\int_{-N}^{N} \RNLL \nabla v\,\d x\le \frac{1}{2} M^{(3,0)}\Vert \nabla u\Vert^{2}_{L^{4}(\OmeL)}\Vert \nabla v\Vert_{L^{2}(\OmeL)}.
	\end{equation} 
\end{theorem}
\begin{proof}
	After using \eqref{Pointwise NLL stress tensor}, we have
	\begin{equation*}
		\int_{\OmeL} \RNLL  \nabla v \,\d x\le \int_{\OmeL} \frac{1}{2}M^{(3,0)}\Vert \nabla u(x)\Vert^{2}_{L^{\infty}(\OmeL)} \vert \nabla v \vert \,\d x.
	\end{equation*}
	We consider \eqref{L-infty to L-2 estimate} and obtain
	\begin{equation*}
		\int_{-N}^{N} \RNLL \nabla v\,\d x\le \frac{1}{2} M^{(3,0)}\Vert \nabla u\Vert^{2}_{L^{4}(\OmeL)}\Vert \nabla v\Vert_{L^{2}(\OmeL)}.
	\end{equation*}
\end{proof}

Finally, by combining Proposition \ref{Coupling consistency error estimate} and Theorem \ref{Linearization consistency error estimate}, and applying the triangle inequality, we provide the consistency error estimate between the QNLL model and the atomistic model.

\begin{theorem}
	For any $y, v \in \Yn$, the consistency error estimate between the QNLL model and the atomistic model is
	\begin{equation}\label{QNLL consistency error estimate}
		\Vert T \Vert_{\Yn^{*}} \le M^{(2,1)}\Vert \nabla^{2} u\Vert_{L^{2}(\bOmeI)} +M^{(2,2)}\Vert \nabla^{3}u \Vert_{L^{2}(\bOmeC)}+M^{(3,2)}\Vert \nabla^{2}u \Vert^{2}_{L^{4}(\bOmeC)}M^{(3,0)}+\Vert \nabla u \Vert^{2}_{L^{4}(\OmeL)}.
	\end{equation}
\end{theorem}

% \begin{proof}
	%     , the result can be obtained.
	% \end{proof}




\subsection{Stability}
\label{sec: stability_qnll_ncg}

In this section, we establish two key results regarding the stability of the QNLL model: (i) For uniform deformations, the QNLL model exhibits universal stability, similar to the QNL model; (ii) For non-uniform deformations, the QNLL model progressively stabilizes as the atomistic region expands.

Firstly, we calculate the second variation of the energy functional of Nonlinear-linear elasticity coupling energy defined by \eqref{Nonlinear-linear energy}, for any $v\in\YacNL$, is then given by
\begin{equation}\label{Stab of NL-L}
	\begin{split}
		\langle \delta^{2} \El (y)v,v\rangle =& \sum_{\xi \in \Ac} \sum_{(\rho,\zeta)\in\Rc^{2}}  \Phia_{\xi,\rho\zeta}(y)D_{\rho}v(\xi)D_{\zeta}v(\xi)\\
		&\ +\sum_{\xi \in \Ic} \sum_{(\rho,\zeta)\in\Rc^{2}} \Phii_{\xi,\rho\zeta}(y)D_{\rho}v(\xi)D_{\zeta}v(\xi)\\
		&\ +\int_{\OmeNL}\ppGW(\nabla y) (\nabla v)^{2}\,\d x\\
		&\ +\int_{\OmeL}\ppGWL(\nabla y) (\nabla v)^{2}\,\d x.
	\end{split}
\end{equation}

If we focus on the second variation evaluated at the homogeneous deformation $\yF$, and use the fact $\ppGW(\nabla \yF)=\ppGWL(\nabla \yF)=\Wppf$. Hence, we can obtain
\begin{equation*}
	\int_{\OmeNL}\ppGW(\nabla \yF) (\nabla v)^{2}\,\d x +\int_{\OmeL}\ppGWL(\nabla \yF)  (\nabla v)^{2}\,\d x =\int_{\OmeC}\ppGW(\nabla \yF) (\nabla v)^{2}\,\d x.
\end{equation*}

After a direct calculation, we have
\begin{equation}\label{Stab of QNLL equals to QNL}
	\begin{split}
		\langle \delta^{2} \El (\yF)v,v\rangle 
		=&\sum_{\xi \in \Ac} \sum_{(\rho,\zeta)\in\Rc^{2}}  \Phia_{\xi,\rho\zeta}(\yF)D_{\rho}v(\xi)D_{\zeta}v(\xi)\\
		&\ +\sum_{\xi \in \Ic} \sum_{(\rho,\zeta)\in\Rc^{2}} \Phii_{\xi,\rho\zeta}(\yF)D_{\rho}v(\xi)D_{\zeta}v(\xi)\\
		&\ +\int_{\OmeC}\ppGW(\nabla \yF) (\nabla v)^{2}\,\d x \\
		=&~\langle \delta^{2} \Erfl (\yF)v,v\rangle. 
	\end{split}
\end{equation}

We then define the stability constants (for homogeneous deformations) $\gaaF, \garflF, \ganllF$ as 
\begin{align}
	\label{GammaF for a}  \gaaF &=\inf_{v\in \Ya} \frac{\langle \delta^{2} \Ea (\yF)v,v\rangle}{\Vert \nabla v \Vert_{L^{2}}^{2}},\\
	\label{GammaF for rfl} \garflF &=\inf_{v\in \Ya} \frac{\langle \delta^{2} \Erfl (\yF)v,v\rangle}{\Vert \nabla v \Vert_{L^{2}}^{2}},\\
	\label{GammaF for nll}\ganllF &=\inf_{v\in \Ya} \frac{\langle \delta^{2} \El (\yF)v,v\rangle}{\Vert \nabla v \Vert_{L^{2}}^{2}}.
\end{align}

The QNL method was propose as a ``universally stable method" (cf.~\cite[Theoorem 4.3]{2014_CO_AS_LZ_Stabilization_MMS}). Combining this result with \eqref{Stab of QNLL equals to QNL} we obtain
\begin{equation}\label{Stab constants of three methods result}
	\gaaF = \garflF =\ganllF.
\end{equation}

To make \eqref{Stab of NL-L} precise, we will 
{\it split} the test function $v$ into an atomistic and continuum component, using the following lemma [\cite{2013_ML_CO_AC_Coupling_ACTANUM}, Lemma 7.3]. We leave the proof to Appendix \ref{Appendix section 3}.

\begin{lemma}\label{Pointwise blending lemma}
	Let $\beta \in C^{1,1}(-\infty,\infty)$, with $0\le \beta \le 1$. For each $v\in \Ya$, there exists $\va, \vc \in \Ya$ such that
	\begin{align}
		\label{Pointwise va blending estimate}\vert \sqrt{1-\beta(\xi)} D_{\rho}v(\xi)-D_{\rho}\va(\xi) \vert &\le \vert \rho \vert^{\frac{3}{2}}\Vert \nabla \sqrt{1-\beta}\Vert_{L^{\infty}} \Vert \nabla v \Vert_{L^{2}(\conv(\xi,\xi+\rho))},\\ 
		\label{Pointwise vc blending estimate1}\vert \sqrt{\beta(\xi)} D_{\rho}v(\xi)-D_{\rho}\vc(\xi) \vert &\le \vert \rho \vert^{\frac{3}{2}}\Vert \nabla \sqrt{\beta}\Vert_{L^{\infty}} \Vert \nabla v \Vert_{L^{2}(\conv(\xi,\xi+\rho))},\\ 
		\label{va and vc} \vert \nabla\va\vert^{2} +	\vert \nabla\vc\vert^{2} &= \vert \nabla v \vert^{2}.
	\end{align}
	%where $C_{1},\ C_{2}$ may depends on $\rcut$, but $C_{3}$ is a generic constant. In particular, $\nabla \va(x)=0$ for $x\in(-\infty,-L]\cup[L,\infty)$ and $\nabla \vc(x)=0$ for $x \in [-K,K]$.
\end{lemma}

% \begin{proof}
	% 
	% \end{proof}

% \begin{proof}
	% 	Let $\psi(x):=\sqrt{1-\beta(x)}$ and assume, without loss of generality that $\rho>0$. Then,
	% 	\begin{align*}
		% 		\sqrt{1-\beta(\xi)} D_{\rho}v(\xi)&= \psi (\xi) \sum_{\eta = \xi}^{\xi+\rho-1}D_{1}V(\eta)\\
		% 		&=\sum_{\eta = \xi}^{\xi+\rho-1}\psi(\eta)D_{1}V(\eta) + \sum_{\eta = \xi}^{\xi+\rho-1} (\psi(\xi)-\psi(\eta))D_{1}V(\eta).\\
		% 	\end{align*}
	% If we define $\va$ by $D_{1}\va(\eta)=\psi(\eta)D_{1}v(\eta)$, then we obtain
	% \begin{equation*}
		% 	\sum_{\eta = \xi}^{\xi+\rho-1} \psi(\eta) D_{1}v(\eta) =D_{\rho} \va (\xi),
		% \end{equation*}
	% and after using Holder's inequality we know
	% \begin{align*}
		% 	\vert\sqrt{1-\beta} D_{\rho}v(\xi)-D_{\rho}\va(\xi)\vert&=\sum_{\eta = \xi}^{\xi+\rho-1}(\psi(\xi)-\psi(\eta))D_{1}v(\eta)\\
		% 	&\le (\sum_{\eta = \xi}^{\xi+\rho-1} \Vert \nabla \psi \Vert^{2}_{L^{\infty}} \vert \rho \vert^{2})^{\frac{1}{2}} (\sum_{\eta = \xi}^{\xi+\rho-1}(D_{1}v(\eta))^{2})^{\frac{1}{2}}\\
		% 	&\le \vert \rho \vert^{\frac{3}{2}} \Vert \nabla \psi \Vert_{L^{\infty}} \Vert \nabla v \Vert_{L^{2}(\xi,\xi+\rho)}.
		% \end{align*}
	% This establishes \eqref{Pointwise va blending estimate}. The proof of \eqref{Pointwise vc blending estimate1} is analogous, with $\vc$ defined by $D_{1}\vc(\xi) = \sqrt{\beta(\xi)}D_{1}v(\xi)$.
	
	% With these definitions \eqref{va and vc} is an immediate consequence.
	% \end{proof}

According to Lemma \ref{Pointwise blending lemma}, we can obtain the conclusion about the stability of the QNLL model in the case of non-uniform deformations.

\begin{theorem}\label{Stability}
	Let $y\in\Ya$ satisfy the strong stability condition \eqref{All-Atomistic strong local minimizer} and suppose that there exists $\ganllF >0$ such that
	\begin{equation}\label{Uniform solution of NL-L stab assumption}
		\langle \delta^{2}\El (\yF)v,v\rangle \ge \ganllF \Vert \nabla v\Vert^{2}_{L^{2}(-N,N)} \text{ for all } v \in \Ua.
	\end{equation}
	
	Then
	\begin{equation}\label{Result of stability}
		\begin{split}
			\langle \delta^{2}\El (y)v,v\rangle &\ge \min(c_{0},\ganllF)\Vert \nabla v \Vert_{L^{2}(-N,N)}^{2}\\
			&\ - 2M^{(2,\frac{1}{2})}K^{-1} \Vert \nabla v \Vert_{L^{2}(-N,N)}^{2} - \CDH M^{(3,0)}K^{-\alpha} \Vert \nabla v \Vert_{L^{2}(-L,L)}^{2}, \quad \text{for all } v \in \Ua.
		\end{split}
	\end{equation}
\end{theorem}

\begin{proof}
	According to Lemma \ref{Pointwise blending lemma}, let $K^{'}:=\lfloor K/2\rfloor <K$, and let
	\begin{equation*}
		\beta (x):=\left\{
		\begin{aligned}
			&0, &-K^{'}\le x \le K^{'},\\
			&\hat{\beta}(\frac{x+K^{'}}{K^{'}-K}), &-K\le x \le -K^{'},\\
			&\hat{\beta} (\frac{x-K^{'}}{K-K^{'}}), &K^{'} \le x \le K,\\
			&1, &-N\le x \le -K \text{ or } K\le x\le N.
		\end{aligned}
		\right.
	\end{equation*}
	where $\hat{\beta}(s)=3s^{3}-2s^{2}$. We know from \cite[Section 8.3]{2013_ML_CO_AC_Coupling_ACTANUM} that 
	\begin{equation}\label{Property of beta}
		\Vert \nabla \sqrt{\beta} \Vert_{L^{\infty}} + \Vert \nabla \sqrt{1-\beta} \Vert_{L^{\infty}} \leq 	C_{\beta} K^{-1}.
	\end{equation}
	
	We can now write
	\begin{subequations}
		\begin{align}
			\langle \delta^{2} \El (y)v,v\rangle &= \sum_{\xi=-N}^{N} \sum_{(\rho,\zeta)\in\Rc^{2}} \Phia_{\xi,\rho\zeta} (y)\big(1-\beta(\xi)\big) D_{\rho}v(\xi) D_{\zeta}v(\xi) \nonumber\\
			&\quad +\sum_{\xi=-N}^{N} \sum_{(\rho,\zeta)\in\Rc^{2}} \PhiNLL_{\xi,\rho\zeta} (y)\big(\beta(\xi)\big) D_{\rho}v(\xi) D_{\zeta}v(\xi) \nonumber\\
			&= 
			\label{All atomistic stab}
			\sum_{\xi=-N}^{N} \sum_{(\rho,\zeta)\in\Rc^{2}} \Phia_{\xi,\rho\zeta} (y)\big(1-\beta(\xi)\big) D_{\rho}v(\xi) D_{\zeta}v(\xi)\\
			\label{NL-L atomistic stab} 
			&\quad +\sum_{\xi \in \Ac} \sum_{(\rho,\zeta)\in\Rc^{2}} \Phia_{\xi,\rho\zeta} (y)\big(\beta(\xi)\big)D_{\rho}v(\xi)D_{\zeta}v(\xi)\\
			\label{NL-L interaction stab} 
			&\quad +\sum_{\xi \in \Ic} \sum_{(\rho,\zeta)\in\Rc^{2}} \Phii_{\xi,\rho\zeta} (y)\big(\beta(\xi)\big)D_{\rho}v(\xi)D_{\zeta}v(\xi)\\
			\label{NL-L nonlinear stab} 
			&\quad + \int_{\OmeNL} \ppGW (\nabla y)\big(\beta (x)\big) (\nabla v)^{2}\,\d x\\
			\label{NL-L linear stab} 
			&\quad +\int_{\OmeL}  \ppGWL (\nabla y)\big(\beta (x)\big)(\nabla v)^{2}\,\d x.
		\end{align}
	\end{subequations}
	where we also use the fact that according to our definition of $\beta$, the first sum ranges only over those sites where $\PhiNLL_{\xi} = \Phia_{\xi}$.
	
	%	Let $\epsilon_{1} = \max (\Vert \nabla \sqrt{1-\beta}\Vert_{L^{\infty}} , \Vert \nabla \sqrt{\beta}\Vert_{L^{\infty}} )$. 
	We apply estimate \eqref{Pointwise va blending estimate} to \eqref{All atomistic stab}, and we obtain
	\begin{equation}\label{Atomistic va stab result}
		\begin{split}
			\sum_{\xi=-N}^{N} \sum_{(\rho,\zeta)\in\Rc^{2}} \Phia_{\xi,\rho\zeta} (y)\big(1-\beta(\xi)\big) D_{\rho}v(\xi) D_{\zeta}v(\xi) &\ge \langle \delta^{2} \Ea (y)\va,\va\rangle 
			- 2M^{(2,\frac{1}{2})}K^{-1}\Vert \nabla v \Vert_{L^{2}(-N,N)}^{2}\\
			&\ge c_{0} \Vert \nabla \va \Vert_{L^{2}(-N,N)}^{2}-2M^{(2,\frac{1}{2})}K^{-1}\Vert \nabla v \Vert_{L^{2}(-N,N)}^{2}.
		\end{split}
	\end{equation}
	
	We apply the estimate \eqref{Pointwise vc blending estimate1} to \eqref{NL-L atomistic stab} to get 
	\begin{equation}\label{NL-L atomistic stab 1}
		\begin{split}
			\sum_{\xi \in \Ac} \sum_{(\rho,\zeta)\in\Rc^{2}} \Phia_{\xi,\rho\zeta} (y)\big(\beta(\xi)\big)D_{\rho}v(\xi)D_{\zeta}v(\xi) \ge& \sum_{\xi \in \Ac} \sum_{(\rho,\zeta)\in\Rc^{2}} \Phia_{\xi,\rho\zeta} (y)D_{\rho}\vc(\xi)D_{\zeta}\vc(\xi) \\
			&\ - 2M^{(2,\frac{1}{2})} K^{-1} \Vert \nabla v \Vert_{L^{2}(\OmeA)}^{2}.
		\end{split}
	\end{equation}
	
	By the definition of $\vc$, we notice that for $x \in[-K^{'},K^{'}]$, $\nabla \vc = 0$. After using Taylor's expansion at $\yF$ and $\DH$ assumption, we obtain
	\begin{equation}\label{NL-L atomistic stab 2}
		\begin{split}
			\sum_{\xi \in \Ac} \sum_{(\rho,\zeta)\in\Rc^{2}} \Phia_{\xi,\rho\zeta} (y)D_{\rho}\vc(\xi)D_{\zeta}\vc(\xi)\ge& \sum_{\xi \in \Ac} \sum_{(\rho,\zeta)\in\Rc^{2}} \Phia_{\xi,\rho\zeta} (\yF)D_{\rho}\vc(\xi)D_{\zeta}\vc(\xi)\\
			&\ -2^{\alpha}\CDH M^{(3,0)}  (K)^{-\alpha} \Vert\nabla \vc \Vert_{L^{2}(\OmeA)}^{2}.
		\end{split}
	\end{equation}
	
	Combining \eqref{NL-L atomistic stab 1} with \eqref{NL-L atomistic stab 2}, we can obtain
	\begin{equation}\label{NL-L atomistic stab result}
		\begin{split}
			\sum_{\xi \in \Ac} \sum_{(\rho,\zeta)\in\Rc^{2}} \Phia_{\xi,\rho\zeta} (y)\big(\beta(\xi)\big)D_{\rho}v(\xi)D_{\zeta}v(\xi) &\ge \sum_{\xi \in \Ac} \sum_{(\rho,\zeta)\in\Rc^{2}} \Phia_{\xi,\rho\zeta} (\yF)D_{\rho}\vc(\xi)D_{\zeta}\vc(\xi) \\
			&\ - 2M^{(2,\frac{1}{2})} K^{-1} \Vert \nabla v \Vert_{L^{2}(\OmeA)}^{2}-2^{\alpha}\CDH M^{(3,0)}  K^{-\alpha}\Vert\nabla \vc \Vert_{L^{2}(\OmeA)}^{2}.
		\end{split}
	\end{equation}
	
	Similarly, for the term \eqref{NL-L interaction stab}, we have
	\begin{equation}\label{NL-L interaction stab result}
		\begin{split}
			\sum_{\xi \in \Ic} \sum_{(\rho,\zeta)\in\Rc^{2}} \Phii_{\xi,\rho\zeta} (y)\big(\beta(\xi)\big)D_{\rho}v(\xi)D_{\zeta}v(\xi) &\ge \sum_{\xi \in \Ic} \sum_{(\rho,\zeta)\in\Rc^{2}} \Phii_{\xi,\rho\zeta} (\yF)D_{\rho}\vc(\xi)D_{\zeta}\vc(\xi) \\
			&\ - M^{(2,\frac{1}{2})} K^{-1} \Vert \nabla v \Vert_{L^{2}(\OmeI)}^{2}-\CDH M^{(3,0)}  K^{-\alpha} \Vert\nabla \vc \Vert_{L^{2}(\OmeI)}^{2}.
		\end{split}
	\end{equation}
	
	After considering the definition of $\vc$ and assumption $\DH$, for~\eqref{NL-L nonlinear stab}, we have 
	\begin{equation}\label{NL-L nonlinear stab result}
		\int_{\OmeNL} \ppGW (\nabla y)\big(\beta(x)\big)(\nabla v)^{2}\,\d x \ge 	\int_{\OmeNL} \ppGW (\nabla \yF)(\nabla\vc)^{2}\,\d x - 2\CDH M^{(3,0)}  K^{-\alpha} \Vert \nabla \vc \Vert_{L^{2}(\OmeNL)}^{2}.
	\end{equation}
	
	
	We use the fact that $\ppGWL (\nabla y) = \Wppf = \ppGWL (\nabla \yF)$ again, and we can rewrite \eqref{NL-L linear stab} as
	\begin{equation}\label{NL-L linear stab result}
		\int_{\OmeL}\ppGWL (\nabla y)\big(\beta(x)\big) (\nabla v)^{2} \,\d x = \int_{\OmeL}\ppGWL (\nabla \yF) (\nabla \vc)^{2} \,\d x.
	\end{equation}
	
	%	\yz{We consider 3-order linear function
		%	\begin{align*}
			%		\int_{\OmeL}\ppGWL (\nabla y)(\beta(x)) (\nabla v)^{2} \d x &\ge \int_{\OmeL}\ppGWL (\nabla \yF) (\nabla \vc)^{2} \d x\\
			%		&\ - \CDH M^{(3,0)}  \bar{L}^{-\alpha} \Vert\nabla \vc \Vert_{L^{2}(\OmeL)}^{2}.
			%	\end{align*}
		%
		%
		%
		%
		%}
	
	
	By considering the definition of $\langle \delta^{2} \El (\yF) \vc,\vc \rangle$, \eqref{va and vc} and \eqref{Property of beta}, we conclude that 
	\begin{equation*}
		\langle \delta^{2} \El (y)v,v\rangle \ge \langle \delta^{2} \El (\yF) \vc,\vc \rangle + \langle \delta^{2}\Ea(y)\va,\va \rangle.
	\end{equation*}
	
\end{proof}

% \yz{(\min(c_{0},\ganllF)\Vert \nabla v \Vert_{L^{2}(-N,N)}^{2}) }\\
%&\ \yz{- 4M^{(2,\frac{1}{2})}K^{-1} \Vert \nabla v \Vert_{L^{2}(-N,N)}^{2} - 2^{\alpha}\CDH M^{(3,0)}K^{-\alpha} \Vert \nabla v \Vert_{L^{2}(-L,L)}^{2} }


\subsection{A priori existence and error estimate}
\label{sec: priori_qnll_ncg}

In this section, based on the consistency error estimate \eqref{QNLL consistency error estimate} and stability analysis \eqref{Result of stability} of the QNLL model, we will provide a priori error analysis for the QNLL model using the inverse function theorem.

\begin{theorem}\label{Priori of NCG}
	Let $\yai \in \Ya$ be a strongly stable atomistic solution satisfying \eqref{All-Atomistic strong local minimizer} and $\DH$. Consider the QNLL problem \eqref{Nonlinear-linear solution condition}, supposing, moreover, that $\El$ is stable in the reference state Theorem \ref{Stability}. Then there exists $K_0$ such that, for all $K \ge K_0$, \eqref{Nonlinear-linear solution condition} has a locally unique, strongly stable solution $\ynll$ which satisfies
	\begin{equation}
		\begin{aligned}
			\Vert \nabla\yai - \nabla \ynll \Vert_{L^2} \lesssim 8&M^{(3,0)}(\Vert \nabla^{2} u\Vert_{L^{2}(\bOmeI)} +\Vert \nabla^{3}u \Vert_{L^{2}(\bOmeC)}+\Vert \nabla^{2}u \Vert^{2}_{L^{4}(\bOmeC)}\\
			&+ \Vert \nabla u \Vert^{2}_{L^{4}(\OmeL)}+N^{\frac{1}{2}-\alpha})/\big(\min(c_{0},\ganllF)\big)^2.
		\end{aligned}
	\end{equation}
	
\end{theorem}
\begin{proof}
	We will first provide the a priori error estimate for $\Vert \nabla \ynll - \nabla \ya\Vert_{L^2}$ using the quantitative inverse function theorem, with
	\begin{equation*}
		\Ghc( \ya):=\delta \El( \ya) - \langle f,\cdot\rangle_{N}.
	\end{equation*}
	We first apply that the scaling condition implies a Lipschitz bound for $\delta^{2}\El$, 
	\begin{equation}\label{scaling assumption}
		\Vert \delta^2 \El (y) - \delta^2 \El(v)\Vert_{\mathcal{L}(\Ya,\Ya^{*})}\le M \Vert \nabla y-\nabla v\Vert_{L^{\infty}}, \quad \text{for all } v \in \Ua,
	\end{equation}
	where $M \lesssim M^{(3,0)}$. Since $\Vert \cdot \Vert_{\infty} \lesssim \Vert \cdot \Vert_{L^2}$, we can also replace the $L^\infty$- norm on the right-hand side with the $L^2$-norm.
	The residual estimate \eqref{QNLL consistency error estimate} gives
	\begin{equation*}
		\Vert \Ghc( \ya)\Vert_{\Yn}\lesssim 
		M^{(2,1)}\Vert \nabla^{2} u\Vert_{L^{2}(\bOmeI)} +M^{(2,2)}\Vert \nabla^{3}u \Vert_{L^{2}(\bOmeC)}+M^{(3,2)}\Vert \nabla^{2}u \Vert^{2}_{L^{4}(\bOmeC)}+ M^{(3,0)}\Vert \nabla u \Vert^{2}_{L^{4}(\OmeL)}
	\end{equation*}
	From Theorem \ref{Stability} we obtain that
	\begin{equation*}
		\langle \delta^{2}\El (\ya)v,v\rangle \ge (\min(c_{0},\ganllF)-CK^{-\min(1,\alpha)})\Vert \nabla v \Vert_{L^{2}}^{2}.
	\end{equation*}
	Let $\gamma:=\frac{1}{2}\min(c_{0},\ganllF)$. Applying the Lipschitz bound \eqref{scaling assumption}, and we obtain
	\begin{equation*}
		\langle \delta^{2}\El (\ya)v,v\rangle \ge (2\gamma-CK^{-\min(1,\alpha)}-CK^{-1/2-\alpha})\Vert \nabla v \Vert_{L^{2}}^{2}.
	\end{equation*}
	Hence, for $K$ sufficiently large, we obtain that
	\begin{equation*}
		\langle \delta\Ghc (\ya)v,v\rangle \ge \gamma\Vert \nabla v \Vert_{L^{2}}^{2}, \quad \text{for all }v\in\Un.
	\end{equation*}
	Thus, we deduce the existence of $\ynll$ satisfying $\Ghc(\ynll)=0$. The error estimate implies
	\begin{equation*}
		\begin{aligned}
			\Vert \nabla \ynll - \nabla \ya\Vert_{L^2}&\lesssim \frac{2M\eta}{\gamma^2} \\
			&\lesssim  2M^{(3,0)}(\Vert \nabla^{2} u\Vert_{L^{2}(\bOmeI)} +\Vert \nabla^{3}u \Vert_{L^{2}(\bOmeC)}+\Vert \nabla^{2}u \Vert^{2}_{L^{4}(\bOmeC)}
			+ \Vert \nabla u \Vert^{2}_{L^{4}(\OmeL)})/\gamma^2.
		\end{aligned}
	\end{equation*}
	Finally, by using the triangle inequality and truncation error \eqref{Truncation error}, we yield the stated result.
	
\end{proof}

\subsection{Discussion of the (quasi-)optimal choice of the length of regions}
\label{sec: balance_of_qnll_ncg_model}

In this subsection, we will discuss how to achieve the quasi-optimal choice of the lengths for nonlinear continuum region, linear continuum region, and computational region to obtain quasi-optimal convergence order for the QNLL model.

\subsubsection{The quasi-optimal choice of $L$}
\label{sec: choice_of_L_ncg}

We aim to balance the lengths of nonlinear continuum region and linear continuum region by incorporating coupling error estimates \eqref{Interface region stress tensor}, \eqref{Continuum region stress tensor} and linearization error estimate \eqref{Linearization error estimate}, under the assumption of $\DH$.

First, we use the $\DH$ assumption to obtain the convergence order of coupling error estimates \eqref{Interface region stress tensor}, \eqref{Continuum region stress tensor} concerning the length of atomistic region and nonlinear continuum region :
\begin{equation*}
	\begin{split}
		M^{(2,1)}\Vert \nabla^{2} u&\Vert_{L^{2}(\bOmeI)} +M^{(2,2)}\Vert \nabla^{3}u \Vert_{L^{2}(\bOmeC)}+M^{(3,2)}\Vert \nabla^{2}u \Vert^{2}_{L^{4}(\bOmeC)}M^{(3,0)}\\
		&\lesssim \CDH M^{(2,1)}K^{-\alpha -1}+\CDH M^{(2,2)}\bar{K}^{-\alpha-\frac{3}{2}} +\CDH^{2}M^{(3,2)}\bar{K}^{-2\alpha -\frac{3}{2}}.
	\end{split}
\end{equation*}
Here we need to note the fact that $K + 2 = \bar{K}$, so we can assume $K\approx \bar{K}$. The lowest-order term among them is $\Vert \nabla^{2} u\Vert_{L^{2}(\bOmeI)} \lesssim K^{-\alpha-1}(\bar{K}^{-\alpha-1})$.

Next, we will similarly apply the $\DH$ assumption to the linearization error estimate \eqref{Linearization error estimate} to obtain its convergence order with respect to the length of the linear continuum region :

\begin{equation*}
	M^{(3,0)}\Vert \nabla u \Vert^{2}_{L^{4}(\OmeL)} \lesssim \CDH^{2}M^{(3,0)}L^{-2\alpha+\frac{1}{2}}.
\end{equation*}

The lowest-order term of $L$ is $\Vert \nabla u \Vert^{2}_{L^{4}(\OmeL)} \lesssim L^{-2\alpha+\frac{1}{2}}$. We balance this term with $\Vert \nabla^{2} u\Vert_{L^{2}(\bOmeI)} \lesssim \bar{K}^{-\alpha-1}$ to get (by noticing the fact that $\bar{K}\le L$)
\begin{align}
	L &\lesssim \bar{K}^{\frac{1}{2}+\frac{5}{8\alpha-2}}(\frac{1}{2}<\alpha<\frac{3}{2}) \label{Balance of L NCG 1},\\
	L &\approx \bar{K}(\alpha \ge \frac{3}{2})\label{Balance of L NCG 2}.
\end{align}

Because through balancing we have made the orders of the lowest order terms of $\bar{K}$ and $L$ equal, for simplicity in this section, we will uniformly use linearization error $L^{-2\alpha+\frac{1}{2}}$ to represent the lowest-order term of coupling error and linearization error.

\subsubsection{The quasi-optimal choice of $N$}
\label{sec: choice_of_N_ncg}

After balancing the lengths of nonlinear continuum region and linear continuum region, we will now balance the computational domain length, which will follow the principles:

\begin{enumerate}
	\item We should ensure that the truncation error term $N^{\frac{1}{2}-\alpha}$ do not dominate among the various types of errors after balancing the length of the computational domain;
	
	\item We choose the length of the computational domain as small as possible for computational simplicity.
\end{enumerate}

According to the first principle mentioned above, we understand that the truncation error $N^{\frac{1}{2}-\alpha}$ must be balanced against one of the terms of coupling error or linearization error (or higher-order terms). According to the second principle, to select the computational domain length $N$ as small as possible, we must balance it against the lowest-order term of coupling error or linearization error (balancing against higher-order terms would need a longer computational domain length).

After balancing the lengths of nonlinear continuum region and linear continuum region, the lowest-order term is $L^{\frac{1}{2}-2\alpha}$, we should choose $N$ such that
\begin{equation*}
	L^{\frac{1}{2}-2\alpha}\approx N^{\frac{1}{2}-\alpha}, \quad \text{that is}, \ N\approx L^{\frac{2\alpha-1/2}{\alpha-1/2}}.
\end{equation*}

\subsection{Numerical validation}
\label{sec: experiments_qnll_ncg}

In this section, we present numerical experiments to illustrate the result of our analysis. With slight adjustments, the problem we consider here is a typical testing case in one dimension. We fix $F=1$ and let $V$ be the site energy given by the embedded atom method (EAM)~\cite{1984_Daw_Baskes_EAM_PRB}:
\begin{equation}\label{EAM of numerical experiments}
	V\big(Dy(l)\big)=\frac{1}{2} \sum_{i\in\{1,2\};j\in\{-1,-2\}}\big(\phi(D_{i}y_{l})+\phi(-D_{j}y_{l})\big)+\widetilde{F}\left(\sum_{i\in\{1,2\};j\in\{-1,-2\}}\big[\psi(D_{i}y_{l})+\psi(-D_{j}y_{l})\big]\right),
\end{equation}
where $\phi(r) = \exp\big(-2a(r-1)\big)-2\exp\big(-a(r-1)\big), \psi(r) = \exp(-br)$, and $\tilde{F}(\rho) = c[(\rho-\rho_{0})^{2}+(\rho-\rho_{0})^{4}]$, with the parameter $a=4.4, b=3, c=5,\rho_{0} =2\exp(-b)$.

We fix an exact solution
\begin{equation}\label{External force of numerical experiments}
	\ya(\xi):=F\xi + \frac{1}{10}(1+\xi^2)^{\alpha/2}\xi,
\end{equation}
and compute the external forces $f(\xi)$ to be the equal to the internal forces under the deformation $\ya$. The parameter $\alpha$ is a prescribed decay exponent. One may readily check that this solution and the associated external forces satisfy the decay hypothesis $\DH$.


We will demonstrate the method of controlling the length of non-linear continuum region in the QNLL model to achieve quasi-optimal convergence order, as introduced in Section~\ref{sec: balance_of_qnll_ncg_model}. We will conduct numerical experiments with the atomistic model length of 100,000 atoms. We set energy functional and external force to \eqref{EAM of numerical experiments} and \eqref{External force of numerical experiments}. The experiments will be carried out for $\alpha$ values of $1.2, 1.5$ and $1.8$.

%\begin{figure}[h!]
%    \centering
%    \begin{minipage}{0.3\textwidth}  % 调整宽度以适应你的图片
	%        \centering
	%        \includegraphics[width=\linewidth]{Figs/alpha18_NCG.pdf}
	%    \caption{The convergence order of QNL and QNLL method(without coarse-graining) ($\alpha = 1.8$)} % 图片标题
	%  \label{fig: convergence_QNL_QNLL_alpha18_NCG}
	%    \end{minipage}\hfill
%    \begin{minipage}{0.3\textwidth}
	%        \centering
	%        \includegraphics[width=\linewidth]{Figs/alpha15_NCG.pdf}
	%    \caption{The convergence order of QNL and QNLL method(without coarse-graining) ($\alpha = 1.5$)} % 图片标题
	%  \label{fig: convergence_QNL_QNLL_alpha15_NCG}
	%    \end{minipage}\hfill
%    \begin{minipage}{0.3\textwidth}
	%        \centering
	%          \includegraphics[width=\linewidth]{Figs/alpha12_NCG.pdf}
	%    \caption{The convergence order of QNL and QNLL method(without coarse-graining) ($\alpha = 1.2$)} % 图片标题
	%  \label{fig: convergence_QNL_QNLL_alpha12_NCG}
	%    \end{minipage}
%\end{figure}

\begin{figure}[h!]
	\centering
	\subfloat[$\alpha = 1.8$]{
		\includegraphics[width=0.3\textwidth]{Figs/alpha18_NCG.pdf}
		\label{fig: convergence_QNL_QNLL_alpha18_NCG}
	}
	\subfloat[$\alpha = 1.5$]{
		\includegraphics[width=0.3\textwidth]{Figs/alpha15_NCG.pdf}
		\label{fig: convergence_QNL_QNLL_alpha15_NCG}
	}
	\subfloat[$\alpha = 1.2$]{
		\includegraphics[width=0.3\textwidth]{Figs/alpha12_NCG.pdf}
		\label{fig: convergence_QNL_QNLL_alpha12_NCG}
	}
	\caption{The convergence order of QNL and QNLL method with different $\alpha$ (without coarse-graining)}
	\label{fig: convergence_QNL_QNLL_NCG}
\end{figure}


Firstly, let us consider the experiment with alpha set to $1.8$: In this case, according to \eqref{Balance of L NCG 2}, by setting the length of the nonlinear continuum region to a few atoms ($\bar{K} \approx L$), the convergence order of the QNLL method matches that of the QNL method. In the Figure below, the $x$-axis represents the length of $L$, while the $y$-axis shows the absolute error $\Vert \nabla \yai - \nabla y^{\text{ac}} \Vert_{L^{2}}~\text{(ac} = \text{QNL, QNLL)}$ between the reference atomistic solution $\yai$ and the AC solutions $y^{\text{ac}}$. It can be observed that the two convergence order lines in the graph nearly overlap, indicating that the difference between the two AC solutions $\Vert \nabla y^{\text{QNL}} - \nabla y^{\text{QNLL}}\Vert$ is between $10^{-6} $and $10^{-7}$.

% \begin{figure}[h]
	%   \centering 
	%   \includegraphics[width=0.8\textwidth]{Figs/alpha18_NCG.png}
	%     \caption{The convergence order of QNL and QNLL method(without coarse-graining) ($\alpha = 1.8$)} % 图片标题
	%   \label{fig: convergence_QNL_QNLL_18_NCG}
	% \end{figure}

When $\alpha = 1.5$, the results are similar to when $\alpha = 1.8$. The figure above compares the convergence order of the QNLL method and the QNL method. The information represented on the axes is the same as in Figure \ref{fig: convergence_QNL_QNLL_alpha12_NCG}. We observe a similar outcome to Figure \ref{fig: convergence_QNL_QNLL_alpha18_NCG}, where the convergence lines of the QNLL method closely overlap with those of the QNL method.

% \begin{figure}[h]
	%   \centering 
	%   \includegraphics[width=0.8\textwidth]{Figs/alpha15_NCG.png}
	%     \caption{The convergence order of QNL and QNLL method(without coarse-graining) ($\alpha = 1.5$)} % 图片标题
	%   \label{fig: convergence_QNL_QNLL_15_NCG}
	% \end{figure}

Furthermore, we will now consider the case where $\alpha=1.2$. In this setting, according to \eqref{Balance of L NCG 1} and \eqref{Balance of L NCG 2}, there are two mesh generation schemes for the QNLL method:
\begin{enumerate}
	\item In the first scheme, we focus on the accuracy of the QNLL method. According to \eqref{Balance of L NCG 1}, we precisely balance the atomistic region, nonlinear continuum region, linear continuum region, and the total length of the computational domain to achieve convergence order identical to those of the QNL method.
	
	\item In the second scheme, we prioritize the computational efficiency of the QNLL method. Therefore, after balancing the lengths of the atomistic region and the total length of the computational domain, we minimize the length of the nonlinear continuum region as much as possible, even down to just a few atoms.
\end{enumerate}


% \begin{figure}
	%   \centering 
	%   \includegraphics[width=0.8\textwidth]{Figs/alpha12_NCG.png}
	%     \caption{The convergence order of QNL and QNLL method(without coarse-graining) ($\alpha = 1.2$)} % 图片标题
	%   \label{fig: convergence_QNL_QNLL_12_NCG}
	% \end{figure}

In the figure below, we represent the first mesh generation scheme with red dashed squares for the QNLL method, and the second generation scheme with blue dashed squares. To demonstrate the accuracy of the QNLL method, the QNL method also adopts the first mesh generation scheme, depicted in the figure with red dashed star symbols. The information represented on the axes is the same as in Figure \ref{fig: convergence_QNL_QNLL_alpha18_NCG}. We observe that, after balancing the lengths of the atomisticc region, nonlinear continuum region, linear continuum region, and the total length of the computational domain, the absolute errors and convergence order obtained by the QNLL method are consistent with those of the QNL method. However, after reducing the length of the nonlinear continuum region in pursuit of computational efficiency, there is a noticeable increase in absolute errors and a decrease in convergence speed.

Next, to demonstrate the computational efficiency of the QNLL method where $\alpha = 1.2$, we test the variation in computation time by progressively increasing the degrees of freedom of the nonlinear continuum region $\Nnl$ of the QNLL method, while keeping the finite element mesh fixed, meaning the continuum region remains unchanged. The results are as shown in the table below: the first column lists the method names, with parentheses indicating the proportion of the degrees of the freedom of nonlinear continuum region $\Nl$ to that of the total continuum region $\Nc$; the second column denotes the total degrees of freedom of the mesh and the third column records the ratio of the computing time of the QNLL method to the computing time of the QNL method on a device with an M1 CPU and 16 GB of RAM:

%\begin{table}
%    \centering
%\begin{tabular}{|c|c|c|} % 开始一个tabular环境,设置3列,每列居中对齐
%\hline % 绘制表格的横线
%Method($\Nnl/\Nc$) & DoF  & The ratio of the computing time\\ % 表头行
%\hline % 绘制表格的横线
%% QNLL($17.81\%$) & 2981 & $9.9766156\times 10^{-4}$ & $0.0840$ \\ % 第一行数据
%QNLL($24.99\%$) & 500000  & $66.68\%$ \\ % 第二行数据
%% QNLL($38.69\%$) & 2981 & $9.9766096\times 10^{-4}$ & $0.0024\%$\\ % 第三行数据
%QNLL($40.97\%$) & 500000 & $77.66\%$ \\ % 第四行数据
%% QNLL($59.57\%$) & 2981 &$9.9766096\times 10^{-4}$ & $0.0979 $\\ % 第五行数据
%% QNLL($66.53\%$) & 2981 & $9.9766096\times 10^{-4}$ & $0.0997$ \\ % 第六行数据
%QNLL($74.96\%$) & 500000 & $88.88\%$ \\ % 第七行数据
%% QNLL($83.92\%$) & 2981 & $9.9766097\times 10^{-4}$ & $0.1052$ \\ % 第八行数据
%QNL($100\%$) & 500000 & $100\%$ \\ % 第九行数据
%\hline % 绘制表格的横线
%\end{tabular}
%\caption{The computing time of QNL and QNLL method without coarse-graining ($\alpha = 1.2$)}
%    \label{tab:computing time alpha12 NCG}
%\end{table}

\begin{table}
	\centering
	\renewcommand{\arraystretch}{1.5} % 调整行间距
	\begin{tabular}{|c|c|} % 开始一个tabular环境,设置2列,每列居中对齐
		\hline % 绘制表格的横线
		Method ($\Nnl/\Nc$) & The ratio of the computing time\\ % 表头行
		\hline % 绘制表格的横线
		QNLL ($24.99\%$) & $66.68\%$ \\ % 第二行数据
		QNLL ($40.97\%$) & $77.66\%$ \\ % 第四行数据
		QNLL ($74.96\%$) & $88.88\%$ \\ % 第七行数据
		QNL ($100\%$) & $100\%$ \\ % 第九行数据
		\hline % 绘制表格的横线
	\end{tabular}
	\caption{The computing time (without coarse graining) of QNL and QNLL method without coarse-graining ($\alpha = 1.2$), with Degree of Freedom (DoF) set to 500000 for all methods.}
	\label{tab:computing time alpha12 NCG}
\end{table}

Table \ref{tab:computing time alpha12 NCG} clearly shows that, with fixed lengths of the Atomistic and Continuum regions, the computing time increases significantly as the proportion of nonlinear elements in the Continuum region rises. However, in practical applications, the proportion of nonlinear elements will be lower (below 5$\%$) according to the balancing method described in Section \ref{sec: balance_of_qnll_ncg_model}. The ratio of the difference between the absolute errors of the QNLL method and the QNL method to the absolute errors of the QNL method: $( \Vert \nabla \yai - \nabla y^{\text{QNLL}} \Vert_{L^{2}} - \Vert \nabla \yai - \nabla y^{\text{QNL}} \Vert_{L^{2}}) / \Vert \nabla \yai - \nabla y^{\text{QNL}} \Vert_{L^{2}}$ is in a narrow range. Here, the ratio, as defined above, is within the range of $10^{-5}$ to $10^{-6}$. This indicates that the QNLL method maintains high accuracy while still offering computational efficiency advantages.
	
	\section{QNLL Method with Coarse-Graining}
\label{sec: qnll_cg}

The QNLL method we analyze in Section~\ref{sec: anal_qnll_ncg} is not a computable scheme since it considers every atom in the computational domain as a degree of freedom and the computational cost gets high fast as the computational domain or $N$ goes large. Therefore, as a common practice of the a/c method, we need to coarse grain the continuum region to reduce the number of degrees of freedom and consequently the computational cost. 

In this section, we follow the same analysis framework as Section~\ref{sec: anal_qnll_ncg}. However, the difference is that in the consistency error part, we incorporate the error introduced by coarse graining. First, we give the formulation and analysis of the coarse-grained QNLL method. Then, we focus on the balancing of the atomistic, nonlinear, and linear regions so that the (quasi-)optimal convergence of the QNLL method, comparable to that of the QNL method, is achieved. Finally, we present several numerical experiments to demonstrate that the QNLL method, with proper balance of the different regions, retains the same level of accuracy as the QNL method while substantially lowering the computational cost measured by CPU time.


\subsection{Coarse-graining and analysis of the QNLL method}
\label{sec: anal_qnll_cg}
%Say nonlinear!!!!!

%We restrict displacements again to a computational domain $[-N,N], \ N \in \Nb$. 

Let $\ThNL = \{T_j\}_{j = 1}^{J} := \big\{[v_{j-1},v_{j}]\ | \ j=1,\dots,J\big\}$ be a regular partition of the computational domain $[-N,N]$ into closed intervals or elements $T_j$. We assume that the vertices of the partition are all at atoms or lattice sites and are denote  by $\NhNL := \{v_{0},\dots,v_J \} \subset \Z_{+}$ (which are often termed as rep-atoms in the language of the quasicontinuum method). We define the coarse-grained space of displacements by
\begin{equation}\label{UhNL space}
	\UhNL:=\{u_{h}\in\Un \ | \ u_{h}(-N)=u_{h}(N)=0 \text{ and }u_{h} \text{ is piecewise affine with respect to }\ThNL\},
\end{equation}
and subsequently the admissible set of deformations by
\begin{equation}\label{YhNL space}
	\YhNL:=\{y\ | \  y = Fx+u_{h}, \ u_{h}\in\UhNL\}.
\end{equation}

We define interpolation operator $I_h: \Yn \rightarrow \YhNL$ by $\IhNL v(\zeta) =v(\zeta), \forall \zeta \in \NhNL$ and $\IhNL v \in \text{P1} (\ThNL)$, which is the piecewise affine nodal interpolation with respect to $\ThNL$. We firstly introduce the following proposition obtained from Poincare's inequality for future usage.

\begin{proposition}
	Let $T\in \ThNL , \ T \subset [-N,-\bar{K}]\cup[\bar{K},N]$ and $u\in \mathscr{U}$. Then
	\begin{equation*}
		\Vert \nabla u-\nabla \IhNL u\Vert_{L^{2}(T)}\lesssim h_{T} \Vert \nabla^{2} u\Vert_{L^{2}(T)}.
	\end{equation*}
	% 	If, in addition, $u$ satisfies $\DH$, $L \le N/2$ and then
	% 	\begin{equation*}
		% 		\vert \nabla \IhNL y(x) -\nabla \PhNL y(x) \vert \lesssim \left\{
		% 		\begin{aligned}
			% 			&N^{-\alpha},&\ x\in[-N,-L],\\
			% 			&0,&\ x\in[-L,L],\\
			% 			&N^{-\alpha},&\ x\in[L,N].
			% 		\end{aligned}
		% 		\right.
		% 	\end{equation*}
\end{proposition}

% \begin{proof}
	
	
	% To prove the second estimate, we first note that $\nabla \IhNL y=\nabla \PhNL y= \nabla y$ in $[-L,L]$. In $[-N,-L]\cup[L,N]$ we have(we choose $u(N)$ in our calculation, for $u(-N)$ the calculation is same)
	% \begin{equation*}
		% 	\vert \nabla \IhNL y(x) -\nabla \PhNL y(x) \vert  = (N-L)^{-1}\vert u(N)\vert \lesssim (N-L)^{-1}N^{1-\alpha}\lesssim N^{-\alpha}.
		% \end{equation*}
	% \end{proof}

For each $T\in \ThNL$, we let $h_{T}:=\text{diam} (T)$. Thus, for $f,g \in \Un$, we define 
\begin{equation*}
	\langle f,g\rangle_{h}:= \int_{-N}^{N} \IhNL (f\cdot g)\,\d x=\sum_{j=1}^{J} \frac{1}{2} h_{T}\big\{f(v_{j-1})\cdot g(v_{j-1}) + f(v_{j})\cdot g(v_{j})\big\}.
\end{equation*}

The coarse-grained QNLL model we aim to solve is the following: 
\begin{equation}
	\label{Yh solution}
	\yh \in \argm \{\El (y_{h})-\langle f,y_{h} \rangle_{h}\ | \ y_{h}\in \YhNL\}.
\end{equation}
%The philosophy of the quasi-continuum(QC) method is to retain the atomistic description, but restrict the admissible set to the coarse space $\YhNL$\yz{Another symbol?}:

%We choose a set of finite element nodes $\NhNL = \{v_{0},\dots,v_{N_{\ThNL}}\}$, for some $N_{\ThNL}\in \Nb$, such that $\{-N,-\bar{L},-L,\dots,L,\bar{L},N\}\subset\NhNL \subset \{-N,\dots,N\}$. The finite elements are given by $\ThNL=\{[v_{j-1},v_{j}]\ | \ j=1,\dots,J\}$. For each $T\in \ThNL$, let $h_{T}:=\text{diam} T$. 
%
%The space of all continuous piecewise affine functions on $[-N,N]$ is given by $\text{P1}(\ThNL)$, and the space of piecewise constant functions by $\text{P0}(\ThNL)$. The admissible finite element space $\UhNL$ is defined by \eqref{UhNL space}, and imposes the boundary condition $u_{h}(-N)=u_{h}(N)=0$.
%
%Finally, for any function $v:\NhNL \rightarrow \R$, let $\IhNL v:[-N,N]\rightarrow \R, \ \IhNL v \in \text{P1} (\ThNL)$, denotes its continuous piecewise affine interpolant,
%\begin{equation*}
%	\IhNL v(\zeta):=v(\zeta), \quad \text{for all } \zeta \in \NhNL.
%\end{equation*}
%
%For $f,g:\NhNL \rightarrow \R^{J}$, we define


% \subsubsection{P1 interpolation operator}



\subsubsection{Coarsening error of the internal forces}
\label{sec: internal_forces_qnll_cg}
The first variation of the continuum energy contribution $\int_{\OmeC} W(\nabla y)\text{d}x$ is given by
\begin{equation*}
	\int_{\OmeNL} \partial_{F} W(\nabla y)\nabla v \text{d}x+\int_{\OmeL} \partial_{F} W_{\text{L}}(\nabla y)\nabla v \,\textrm{d}x.
\end{equation*}
Elements of $\UhNL$ are defined pointwise, but give rise to lattice functions through point evaluation. Since finite element nodes lie on lattice sites, this is compatible with our interpolation of lattice functions.

The following lemma estimates the error contribution from this operator induced by finite element coarsening and reduction to a finite domain.

\begin{theorem}\label{Internal forces of continuum region}
	Let $u \in \mathscr{U}$ satisfy $(\mathbf{DH})$, and $0<\bar{K} <L \le N/2$. Then
	\begin{equation*}
		\begin{aligned}
			\Bigg \vert \int_{\OmeNL} \big(\partial_{F} W(\nabla \IhNL y) &- \partial_{F} W(\nabla y)\big) \nabla v_{h} \, \text{d} x + \int_{\OmeL} \big(\partial_{F} \WL (\nabla \IhNL y) - \partial_{F} \WL (\nabla y)\big) \nabla v_{h} \,\text{d} x \Bigg \vert \\
			&\lesssim M^{(2,0)}\Vert h\nabla^{2}u\Vert^{2}_{L^{4}(\OmeNL)} \Vert \nabla v_{h} \Vert_{L^{2}}, \quad \text{for all}\  v_{h} \in \text{P1}(\mathcal{T}_{h}).\\
		\end{aligned}
	\end{equation*}
\end{theorem}

\begin{proof}
	% Firstly, we calculate the first integral on nonlinear region. We fix $\forall T\in \OmeNL$, and we have
	% 	\begin{equation}\label{Each T of nonlinear}
		% 		\begin{aligned}
			% 			\vert &\int_{T} (\partial_{F} W(\nabla \PhNL y) - \partial_{F} W(\nabla y)) \nabla v_{h} \text{d} x \vert \\
			% 			&\le \vert \int_{T} (\partial_{F} W(\nabla \PhNL y) - \partial_{F} W(\nabla \IhNL y)) \nabla v_{h} \text{d} x \vert \quad (\text{Part \uppercase\expandafter{\romannumeral1}}) \\
			% 			&+ \vert \int_{T} (\partial_{F} W(\nabla \IhNL y) - \partial_{F} W(\nabla y)) \nabla v_{h} \text{d} x \vert. \quad (\text{Part \uppercase\expandafter{\romannumeral2}})
			% 		\end{aligned}
		% 	\end{equation}	
	
	% 		For the Part \uppercase\expandafter{\romannumeral1} of \eqref{Each T of nonlinear} , apply H$\ddot{\text{o}}$lder‘s inequality, and we have
	% 		\begin{equation}\label{Part 1 of nonlinear}
		% 		\begin{aligned}
			% 			&\int_{T} (\partial_{F} W(\nabla \PhNL y) - \partial_{F} W(\nabla \IhNL y)) \nabla v_{h} \text{d} x \\
			% 			&\le (\int_{T} (\partial_{F} W(\nabla\PhNL y) - \partial_{F} W(\nabla \IhNL y))^{2} \text{d} x)^{\frac{1}{2}} \cdot \Vert \nabla v_{h} \Vert_{L^{2}(T)}.
			% 		\end{aligned}
		% \end{equation}	
	
	% 	We calculate directly, and get
	% 	\begin{equation*}
		% 		\begin{aligned}
			% 			\vert \partial_{F} W(\nabla \PhNL y) - \partial_{F} W(\nabla \IhNL y) \vert 
			% 			&\le M^{(2,0)} \cdot \vert \nabla \PhNL y - \nabla \IhNL y \vert \\
			% 			&= M^{(2,0)}  \cdot \frac{\vert y(N) \vert}{N-L} \\
			% 			&\le M^{(2,0)}  \cdot C_{\mathbf{DH}}\  (N-L)^{-1} \  N^{1-\alpha}\\
			% 			&\le \frac{1}{2} C_{\mathbf{DH}} \ M^{(2,0)} \ N^{-\alpha}.
			% 		\end{aligned}
		% 	\end{equation*}
	
	% From the present result, 
	% The Part \uppercase\expandafter{\romannumeral1} will be
	% \begin{equation*}
		% 	\int_{T} (\partial_{F} W(\nabla \PhNL y) - \partial_{F} W(\nabla \IhNL y)) \nabla v_{h} \text{d} x \lesssim M^{(2,0)}\cdot (h_{T}^{\text{NL}})^{\frac{1}{2}} \ N^{-\alpha} \cdot \Vert \nabla v_{h} \Vert_{L^{2}(T)}.
		% \end{equation*}
	
	Firstly, we calculate the first integral on nonlinear region. After using the fact that $\int_{T}\nabla \IhNL u\,\d x=\int_{T} \nabla u\,\d x$, for any $T\in \OmeNL$, we have
	\begin{equation*}
		\begin{aligned}
			\Bigg \vert \int_{T} \partial_{F} W(\nabla \IhNL y) - \partial_{F} W(\nabla y) \,\d x \Bigg \vert &\le \Bigg \vert \int_{T}\partial^{2}_{F}W(\nabla\IhNL y)(\nabla\IhNL u -\nabla u)\,\d x \Bigg \vert\\
			&\quad + M^{(3,0)}\int_{T}\vert \nabla \IhNL u-\nabla u\vert^{2}\,\d x\\
			&\lesssim \Vert \nabla \IhNL u -\nabla u\Vert^{2}_{L^{2}(T)}\lesssim \Vert h\nabla^{2} u\Vert^{2}_{L^{2}(T)}.
		\end{aligned}
	\end{equation*}
	
	Summing over $T\subset \ThNL$ and again applying H$\ddot{\text{o}}$lder‘s inequality yields
	\begin{equation*}
		\sum_{T \in \ThNL} \Vert h\nabla^{2} u\Vert^{2}_{L^{2}(T)}\Vert \nabla v_{h} \Vert_{L^{2}(T)}
		\le \Vert h\nabla^{2} u\Vert^{2}_{L^{4}(\OmeNL)}\Vert \nabla v_{h} \Vert_{L^{2}(\OmeNL)}.
	\end{equation*}
	
	% Next we focus on linear region, for each $T \in \ThNL$, we have
	
	% \begin{equation}\label{Each T of linear}
		% 	\begin{aligned}
			% 		\vert &\int_{T} (\partial_{F} W_{\text{L}}(\nabla \Pi_{h} y) - \partial_{F} W_{\text{L}}(\nabla y)) \nabla v_{h} \text{d} x \vert \\
			% 		&\le \vert \int_{T} (\partial_{F} W_{\text{L}}(\nabla \Pi_{h} y) - \partial_{F} W_{\text{L}}(\nabla I_{h}y)) \nabla v_{h} \text{d} x \vert \quad (\text{Part \uppercase\expandafter{\romannumeral3}}) \\
			% 		&+ \vert \int_{T} (\partial_{F} W_{\text{L}}(\nabla I_{h} y) - \partial_{F} W_{\text{L}}(\nabla y)) \nabla v_{h} \text{d} x \vert. \quad (\text{Part \uppercase\expandafter{\romannumeral4}})
			% 	\end{aligned}
		% \end{equation}
	
	% For the Part \uppercase\expandafter{\romannumeral3} of \eqref{Each T of linear} , apply H$\ddot{\text{o}}$lder‘s inequality, and we have
	% \begin{equation}\label{3.2}
		% 	\begin{aligned}
			% 		&\int_{T} (\partial_{F} \WL(\nabla \Pi_{h} y) - \partial_{F} \WL(\nabla I_{h}y)) \nabla v_{h} \text{d} x \\
			% 		&\le (\int_{T} (\partial_{F} \WL(\nabla \Pi_{h} y) - \partial_{F} \WL(\nabla I_{h}y))^{2}  \text{d} x)^{\frac{1}{2}} \cdot \Vert \nabla v_{h} \Vert_{L^{2}(T)}.
			% 	\end{aligned}
		% \end{equation}
	
	
	
	% Next we focus on linear region, for each $T \in \ThNL$, after using the definition of $W_{L}(F)$, and we have
	% \begin{equation*}
		% 	\begin{aligned}
			% 		\vert \partial_{F} W_{\text{L}}(\nabla \Pi_{h} y) - \partial_{F} W_{\text{L}}(\nabla I_{h}y) \vert 
			% 		&=\Wppf \cdot \vert \nabla \Pi_{h}u - \nabla I_{h} u \vert \\
			% 		&= \Wppf \cdot \frac{\vert u(N) \vert}{N-L'} \\
			% 		&\le \Wppf \cdot C_{\mathbf{DH}}\  (N-L')^{-1} \  N^{1-\alpha}\\
			% 		&\le \frac{1}{2} C_{\mathbf{DH}} \ \Wppf \ N^{-\alpha}.
			% 	\end{aligned}
		% \end{equation*}
	
	% From the present result, 
	% The Part \uppercase\expandafter{\romannumeral3} will be
	% \begin{equation*}
		% 	\int_{T} (\partial_{F} W_{\text{L}}(\nabla \Pi_{h} u) - \partial_{F} W_{\text{L}}(\nabla I_{h}u)) \nabla v_{h} \text{d} x \lesssim \Wppf\cdot (h^{\frac{1}{2}}_{T} \ N^{-\alpha}) \cdot \Vert \nabla v_{h} \Vert_{L^{2}(T)}.
		% \end{equation*}
	
	Next we focus on the linear region, after using the definition of $\WL(F)$, and we have
	\begin{equation*}
		\partial_{F} W_{\text{L}}(\nabla I_{h} u) - \partial_{F} W_{\text{L}}(\nabla u) =\Wppf(\nabla I_{h}u - \nabla u).
	\end{equation*}
	Moreover, using the fact that $\int_{T} \nabla I_{h}u \text{d}x = \int_{T} \nabla u\text{d}x$, since $\nabla v_{h} $ is constant in $T$, we have
	\begin{equation*}
		\begin{aligned}
			\int_{T} \Wppf (\nabla I_{h}u - \nabla u) \nabla v_{h} \,\text{d}x = \Wppf \  \nabla v_{h}\vert_{T} \cdot \int_{T} (\nabla I_{h}u - \nabla u)\,\text{d}x= 0.
		\end{aligned}
	\end{equation*}
	
	% Summing over $T\subset \ThNL$ and again applying H$\ddot{\text{o}}$lder‘s inequality yields
	% \begin{equation*}
		% 	\begin{aligned}
			% 		&\sum_{T \in \mathcal{T}_{h}} h^{\frac{1}{2}}_{T} \Vert \nabla v_{h} \Vert_{L^{2}(T)}\\
			% 		&\le (\sum_{T \in \mathcal{T}_{h}} h_{T})^{\frac{1}{2}} \cdot (\sum_{T \in \mathcal{T}_{h}} \Vert \nabla v_{h} \Vert_{L^{2}(T)}^{2} )^{\frac{1}{2}}\\
			% 		&= (N-\bar{L})^{\frac{1}{2}} \cdot \Vert \nabla v_{h} \Vert_{L^{2}(\bar{L},N)}\\
			% 		&\lesssim N^{\frac{1}{2}} \cdot  \Vert \nabla v_{h} \Vert_{L^{2}(\bar{L},N)}.
			% 	\end{aligned}
		% \end{equation*}
\end{proof}

\subsubsection{Coarsening error of external forces}
\label{sec: external_forces_qnll_cg}
We now address the consistency error arising from approximating the external potential $\langle f, v_{h} \rangle_{\Z}$ using the trapezoidal rule, denoted as $\langle f, v_{h} \rangle_{h}$. A key challenge in this analysis is to avoid relying on the Poincaré inequality $\Vert v_{h} \Vert_{L^{2}} \lesssim N\Vert \nabla v_{h} \Vert_{L^{2}}$. Instead, we utilize weighted Poincaré inequalities, which are more suitable for unbounded domains. This approach leads to the following result. The lemma presented here is adapted from~\cite[Theorem 6.13]{2013_ML_CO_AC_Coupling_ACTANUM}, but we state the theorem directly and omit the proof for brevity.

\begin{lemma}
	Let $L>1,\omega(x) = x\log(x)$ and suppose that $h(x)\le \kappa x$ for almost every $x\in [-N,-\bar{K}\cup[\bar{K},N]$. And we note $[-\infty,-\bar{K}]\cup[\bar{K},+\infty]$ by $\tOmeC$Then there exists a constant $C_{\kappa}$ such that
	\begin{equation*}
		\Vert \eta_{\text{ext}}\Vert_{(\YhNL)^{*}} = \Vert \langle f,\cdot\rangle_{N} - \langle f,\cdot\rangle_{h}\Vert_{(\YhNL)^{*}} \lesssim \Vert h^{2} \nabla f\Vert_{L^{2}(\tOmeC)} +\frac{C_{\kappa}}{\log L} \Vert h^{2}\omega \nabla^{2}f\Vert_{L^{2}(\tOmeC)}.
	\end{equation*}
\end{lemma}

% \begin{proof}
	% 
	% \end{proof}



% \subsection{Coarsening error of linear region}
% %The reson why we choose this.
% We restrict displacements again to a computational domain $[-N,N], N \in \Nb$. Let $\Th = \{T\}$ be a regular partition of $[-N,N]$ into closed intervals $T$, with vertices $\Nh \subset \Z_{+}$ (or, rep-atoms in the language of the quasi-continuum method). We define the coarse displacement space by
% \begin{equation}\label{Uh space}
	% 	\Uh:=\{u_{h}\in\Un \ | \ u_{h} \text{ is picecewise affine with respect to }\Th\}.
	% \end{equation}
% \begin{equation}\label{Yh space}
	% 	\Yh:=\{Fx + u_{h} \ | \  u_{h}\in\Uh \}.
	% \end{equation}
% The philosophy of the quasi-continuum(QC) method is to retain the atomistic description, but restrict the admissible set to the coarse space $\Uh$:
% \begin{equation}\label{uqch solution space}
	% 	y^{\text{a}}_{h} \in \argm \{\Ea (y_{h})-\langle f,y_{h} \rangle_{N}\ | \ y_{h}\in \Yh\}.
	% \end{equation}

% We choose a set of finite element nodes $\Nh = \{v_{0},\dots,v_{N_{\Th}}\}$, for some $N_{\Th}\in \Nb$, such that $\{-N,-\bar{L},\dots,\bar{L},N\}\subset\Nh \subset \{-N,\dots,N\}$. The finite elements are given by $\Th=\{[v_{j-1},v_{j}]\ | \ j=1,\dots,J^{\text{L}}\}$. For each $T\in \Th$, let $h_{T}^{\text{L}}:=\text{diam} T$. For each $x\in [-N,N], \ x\in \text{int} T$, let $h^{\text{L}}(x):=h_{T}$. For $x<-N$ or $x>N$, let $h^{\text{L}}(x):=1$.

% The space of all continuous piecewise affine functions on $[-N,N]$ is given by $\text{P1}(\Th)$, and the space of piecewise constant functions by $\text{P0}(\Th)$. The admissible finite element space $\Uh$ is defined by \eqref{Uh space}, and imposes the boundary condition $u_{h}(-N)=u_{h}(N)=0$.

% Finally, for any function $v:\Nh \rightarrow \R$, let $\Ih v:[-N,N]\rightarrow \R, \ \Ih v \in \text{P1} (\Th)$, denotes its continuous piecewise affine interpolant,
% \begin{equation*}
	% 	\Ih v(\zeta):=v(\zeta) \quad \text{for all } \zeta \in \Nh.
	% \end{equation*}

% For $f,g:\Nh \rightarrow \R^{J^{\text{L}}}$, we define
% \begin{equation*}
	% 	\langle f,g\rangle_{h}^{\text{L}}:= \int_{-N}^{N} \Ih (f\cdot g)\d x=\sum_{j=1}^{J^{\text{L}}} \frac{1}{2} h_{T}^{\text{L}}\{f(v_{j-1})\cdot g(v_{j-1}) + f(v_{j})\cdot g(v_{j})\}.
	% \end{equation*}

% \subsubsection{Best approximation operator}
% Recall the definition of the nodal interpolation operator $\Ih$. Since $\Ih$ does not map $\Ya$ to $\Yh$, and to avoid any error contributions from the atomistic region (and possibly a neighbourhood of the interface region), we define
% \begin{equation*}
	% 	\Ph y(x):= \left\{
	% 	\begin{aligned}
		% 		&\Ih y(x)-\frac{x+\bar{L}}{-N+\bar{L}}u(-N),&\ x\in[-N,-\bar{L}],\\
		% 		&\Ih y(x),&\ x\in[-\bar{L},\bar{L}],\\
		% 		&\Ih y(x)-\frac{x-\bar{L}}{N-\bar{L}}u(N),&\ x\in[\bar{L},N].
		% 	\end{aligned}
	% 	\right.
	% \end{equation*}
% With this definition, $\Ph y\in \Yh$ for all $y:\Nh \rightarrow \R$.

% In our coarsening error analysis below, it will be useful to split the interpolant into $\Ph y = \Ih y +(\Ph y - \Ih y).$ Hence, we separately estimate the interpolation errors as follows.

% \begin{theorem}
	% 	Let $T\in \Th , \ T \subset [-N,-\bar{L}]\cup[\bar{L},N]$ and $y\in \Ya$. Then
	% 	\begin{equation*}
		% 		\Vert \nabla y-\nabla \Ih y\Vert_{L^{2}(T)}\lesssim h_{T}^{\text{L}} \Vert \nabla^{2} u\Vert_{L^{2}(T)}.
		% 	\end{equation*}
	% 	If, in addition, $u$ satisfies $\DH$, $\bar{L} \le N/2$ and $N \ge r_{0}$, then
	% 	\begin{equation*}
		% 		\vert \nabla \Ih y(x) -\nabla \Ph y(x) \vert \lesssim \left\{
		% 		\begin{aligned}
			% 			&N^{-\alpha},&\ x\in[-N,-\bar{L}],\\
			% 			&0,&\ x\in[-\bar{L},\bar{L}],\\
			% 			&N^{-\alpha},&\ x\in[\bar{L},N].
			% 		\end{aligned}
		% 		\right.
		% 	\end{equation*}
	% \end{theorem}

% \begin{proof}
	% 	The first result follows from Poincare's inequality.
	
	% 	To prove the second estimate, we first note that $\nabla \Ih y=\nabla \Ph y= \nabla y$ in $[-\bar{L},\bar{L}]$. In $[-N,-\bar{L}]\cup[\bar{L},N]$ we have(we choose $u(N)$ in our calculation, for $u(-N)$ the calculation is same)
	% 	\begin{equation*}
		% 		\vert \nabla \Ih y(x) -\nabla \Ph y(x) \vert  = (N-\bar{L})^{-1}\vert u(N)\vert \lesssim (N-\bar{L})^{-1}N^{1-\alpha}\lesssim N^{-\alpha}.
		% 	\end{equation*}
	% \end{proof}

% \subsubsection{Coarsening error of the internal forces}

% The first variation of the continuum energy contribution $\int_{\OmeL} W_{L}(\nabla y)\text{d}x$ is given by
% \begin{equation*}
	% 	v \rightarrow \int_{\OmeL} \partial_{F} W_{\text{L}}(\nabla y)\nabla v \text{d}x.
	% \end{equation*}
% Elements of $\Uh$ are defined pointwise, but give rise to lattice functions through point evaluation. Since finite element nodes lie on lattice sites, this is compatible with our interpolation of lattice functions.

% The following lemma estimates the error contribution from this operator induced by finite element coarsening and reduction to a finite domain.

% \begin{theorem}\label{Internal forces of linear region}
	% 	Let $u \in \mathscr{U}$ satisfy $(\mathbf{DH})$, $N > r_{0}$, and $0<\bar{L} \le N/2$. Then
	% 	\begin{equation*}
		% 		\begin{aligned}
			% 			\vert \int_{\OmeL} (\partial_{F} W_{\text{L}}(\nabla \Pi_{h} y) &- \partial_{F} W_{\text{L}}(\nabla y)) \nabla v_{h} \text{d} x \vert \\
			% 			&\lesssim W^{''}(0) N^{\frac{1}{2} - \alpha } \Vert \nabla v_{h} \Vert_{L^{2}}. \quad \text{for all}\  v_{h} \in \text{P1}(\mathcal{T}_{h}).\\
			% 		\end{aligned}
		% 	\end{equation*}
	% \end{theorem}

% \begin{proof}
	% 	For each $T \in \mathcal{T}_{h}$, we have
	
	% 	\begin{equation}\label{Each T of linear}
		% 		\begin{aligned}
			% 			\vert &\int_{T} (\partial_{F} W_{\text{L}}(\nabla \Pi_{h} y) - \partial_{F} W_{\text{L}}(\nabla y)) \nabla v_{h} \text{d} x \vert \\
			% 			&\le \vert \int_{T} (\partial_{F} W_{\text{L}}(\nabla \Pi_{h} y) - \partial_{F} W_{\text{L}}(\nabla I_{h}y)) \nabla v_{h} \text{d} x \vert \quad (\text{Part \uppercase\expandafter{\romannumeral5}}) \\
			% 			&+ \vert \int_{T} (\partial_{F} W_{\text{L}}(\nabla I_{h} y) - \partial_{F} W_{\text{L}}(\nabla y)) \nabla v_{h} \text{d} x \vert. \quad (\text{Part \uppercase\expandafter{\romannumeral6}})
			% 		\end{aligned}
		% 	\end{equation}
	
	% 	For the Part \uppercase\expandafter{\romannumeral5} of \eqref{Each T of linear} , apply H$\ddot{\text{o}}$lder‘s inequality, and we have
	% 	\begin{equation}\label{Part 5 of linear}
		% 		\begin{aligned}
			% 			&\int_{T} (\partial_{F} W_{\text{L}}(\nabla \Pi_{h} y) - \partial_{F} W_{L}(\nabla I_{h}y)) \nabla v_{h} \text{d} x \\
			% 			&\le (\int_{T} (\partial_{F} W_{\text{L}}(\nabla \Pi_{h} y) - \partial_{F} W_{L}(\nabla I_{h}y))^{2} \text{d} x)^{\frac{1}{2}} \cdot \Vert \nabla v_{h} \Vert_{L^{2}(T)}.
			% 		\end{aligned}
		% 	\end{equation}
	
	% 	We use the definition of $W_{\text{L}}(F)$ we have
	% 	\begin{equation*}
		% 		\begin{aligned}
			% 			\vert \partial_{F} W_{\text{L}}(\nabla \Pi_{h} y) - \partial_{F} W_{\text{L}}(\nabla I_{h}y) \vert 
			% 			&= \Wppf \cdot \vert \nabla \Pi_{h}u - \nabla I_{h} u \vert \\
			% 			&= \Wppf \cdot \frac{\vert u(N) \vert}{N-\bar{L}} \\
			% 			&\le \Wppf \cdot C_{\mathbf{DH}}\  (N-\bar{L})^{-1} \  N^{1-\alpha}\\
			% 			&\le \frac{1}{2} C_{\mathbf{DH}} \ \Wppf \ N^{-\alpha}.
			% 		\end{aligned}
		% 	\end{equation*}
	
	% 	From the present result, 
	% 	The Part \uppercase\expandafter{\romannumeral5} will be
	% 	\begin{equation*}
		% 		\int_{T} (\partial_{F} W_{\text{L}}(\nabla \Pi_{h} y) - \partial_{F} W_{\text{L}}(\nabla I_{h}y)) \nabla v_{h} \text{d} x \lesssim \Wppf\cdot ((h_{T}^{\text{L}})^{\frac{1}{2}}_{T} \ N^{-\alpha}) \cdot \Vert \nabla v_{h} \Vert_{L^{2}(T)}.
		% 	\end{equation*}
	% 	\yz{
		% 	For the Part \uppercase\expandafter{\romannumeral6} of \eqref{Each T of linear} , we could compute it directly and get
		% 	\begin{equation*}
			% 		\partial_{F} W_{\text{L}}(\nabla I_{h} y) - \partial_{F} W_{\text{L}}(\nabla y) = \Wppf(\nabla I_{h}y - \nabla y)
			% 	\end{equation*}
		
		% 	Moreover, using the fact that $\int_{T} \nabla I_{h}y \text{d}x = \int_{T} \nabla y\text{d}x$, since $\nabla v_{h} $ is constant in $T$, we have
		% 	\begin{equation*}
			% 		\begin{aligned}
				% 			&\int_{T} \Wppf (\nabla I_{h}u - \nabla u) \nabla v_{h} \text{d}x \\
				% 			&= \Wppf \  \nabla v_{h}\vert_{T} \cdot \int_{T} (\nabla I_{h}u - \nabla u) \text{d}x= 0
				% 		\end{aligned}
			% 	\end{equation*}
		% 	}
	% 	Summing over $T \in \Th$, $T\subset \Th$ and again applying H$\ddot{\text{o}}$lder‘s inequality yields
	% 	\begin{equation*}
		% 		\begin{aligned}
			% 			&\sum_{T \in \Th} (h_{T}^{\text{L}})^{\frac{1}{2}} \Vert \nabla v_{h} \Vert_{L^{2}(T)}\\
			% 			&\le (\sum_{T \in \Th} h_{T})^{\frac{1}{2}} \cdot (\sum_{T \in \Th} \Vert \nabla v_{h} \Vert_{L^{2}(T)}^{2} )^{\frac{1}{2}}\\
			% 			&= (N-\bar{L})^{\frac{1}{2}} \cdot \Vert \nabla v_{h} \Vert_{L^{2}(\OmeL)}\\
			% 			&\lesssim N^{\frac{1}{2}} \cdot  \Vert \nabla v_{h} \Vert_{L^{2}(\OmeL)}.
			% 		\end{aligned}
		% 	\end{equation*}
	% \end{proof}

% \subsubsection{Coarsening error of external forces}
% We now turn towards the consistency error due to the approximation of the external potential $\langle f,v_{h} \rangle_{\Z}$ by the trapezoidal rule $\langle f,v_{h} \rangle_{h}^{\text{L}}$. The main challenge in this analysis is to avoid the use of Poincare inequality $\Vert v_{h} \Vert_{L^{2}} \lesssim N\Vert \nabla v_{h} \Vert_{L^{2}}$ at all costs. A common approach in unbounded domains is to employ weighted Poincare's  inequalities instead. This yields the following result.

% \begin{theorem}
	% 	Let $\bar{L}>1, \omega(x)=x\log (x)$ and suppose that $h^{\text{L}}(x)\le \kappa x$ for almost every $x\in [-N,-\bar{L}]\cup[\bar{L},N]$. And we note $[-\infty,-\bar{L}]\cup[\bar{L},+\infty]$ by $\tOmeL$Then there exists a constant $C^{\text{L}}_{\kappa}$ such that
	% 	\begin{equation*}
		% 		\Vert \eta_{\text{ext}} \Vert_{(\Yh)^{*}} = \Vert \langle f,\cdot\rangle_{\Z} - \langle f,\cdot\rangle^{\text{L}}_{h}\Vert_{(\Yh)^{*}} \lesssim \Vert h^{2} \nabla f\Vert_{L^{2}(\tOmeL)} +\frac{C^{\text{L}}_{\kappa}}{\log \bar{L}} \Vert h^{2}\omega \nabla^{2}f\Vert_{L^{2}(\tOmeL)}.
		% 	\end{equation*}
	% \end{theorem}

% \begin{proof}
	% 	Already check.
	% \end{proof}



% \section{Stability}\label{sec:Stability}

% \chw{Give an analysis of the stability and discuss how the linearization may affect the stability}
% Firstly, we calculate the second variation of the energy functional of Nonlinear-linear elasticity coupling reflection energy defined by \eqref{Nonlinear-linear energy}, for any $v\in\YacNL$, is then given by
% \begin{equation}\label{Stab of NL-L}
	% 	\begin{split}
		% 		\langle \delta^{2} \El (y)v,v\rangle &= \sum_{\xi \in \Ac} \sum_{(\rho,\zeta)\in\Rc^{2}}  \Phia_{\xi,\rho\zeta}(y)D_{\rho}v(\xi)D_{\zeta}v(\xi)\\
		% 		&\ +\sum_{\xi \in \Ic} \sum_{(\rho,\zeta)\in\Rc^{2}} \Phii_{\xi,\rho\zeta}(y)D_{\rho}v(\xi)D_{\zeta}v(\xi)\\
		% 		&\ +\int_{\OmeNL}\ppGW(\nabla y) (\nabla v)^{2}\d x\\
		% 		&\ +\int_{\OmeL}\ppGWL(\nabla y) (\nabla v)^{2}\d x.
		% 	\end{split}
	% \end{equation}

% If we focus on the second variation evaluated at the homogeneous deformation $\yF$, and use the fact $\ppGW(\nabla \yF)=\ppGWL(\nabla \yF)=\Wppf$. So we could obtaion that
% \begin{equation*}
	% 	\int_{\OmeNL}\ppGW(\nabla \yF) (\nabla v)^{2}\d x +\int_{\OmeL}\ppGWL(\nabla \yF)  (\nabla v)^{2}\d x =\int_{\OmeC}\ppGW(\nabla \yF) (\nabla v)^{2}\d x.
	% \end{equation*}

% Then we know
% \begin{equation}\label{Stab of NL-L equals to Reflection}
	% 	\begin{split}
		% 		\langle \delta^{2} \El (\yF)v,v\rangle 
		% 		&=\sum_{\xi \in \Ac} \sum_{(\rho,\zeta)\in\Rc^{2}}  \Phia_{\xi,\rho\zeta}(\yF)D_{\rho}v(\xi)D_{\zeta}v(\xi)\\
		% 		&\ +\sum_{\xi \in \Ic} \sum_{(\rho,\zeta)\in\Rc^{2}} \Phii_{\xi,\rho\zeta}(\yF)D_{\rho}v(\xi)D_{\zeta}v(\xi)\\
		% 		&\ +\int_{\OmeC}\ppGW(\nabla \yF) (\nabla v)^{2}\d x=\langle \delta^{2} \Erfl (\yF)v,v\rangle. 
		% 	\end{split}
	% \end{equation}

% Then we define the stability constants (for homogeneous deformations) $\gaaF, \garflF, \ganllF$ as 
% \begin{align}
	% 	\label{GammaF for a}  \gaaF &=\inf_{v\in \Ya} \frac{\langle \delta^{2} \Ea (\yF)v,v\rangle}{\Vert \nabla v \Vert_{L^{2}}^{2}},\\
	% 	\label{GammaF for rfl} \garflF &=\inf_{v\in \Ya} \frac{\langle \delta^{2} \Erfl (\yF)v,v\rangle}{\Vert \nabla v \Vert_{L^{2}}^{2}},\\
	% 	\label{GammaF for nll}\ganllF &=\inf_{v\in \Ya} \frac{\langle \delta^{2} \El (\yF)v,v\rangle}{\Vert \nabla v \Vert_{L^{2}}^{2}}.
	% \end{align}

% The reflection method was propose as a 'universally stable method' (\cite{2014_CO_AS_LZ_Stabilization_MMS},Theorem 4.3). Combining this result with \eqref{Stab of NL-L equals to Reflection} we obtain
% \begin{equation}\label{Stab constants of three methods result}
	% 	\gaaF = \garflF =\ganllF.
	% \end{equation}

% To make \eqref{Stab of NL-L} precise, we will 'split' the test function $v$ into an atomistic and continuum component, using the following lemma.[\cite{2013_ML_CO_AC_Coupling_ACTANUM},Lemma 7.3]

% \begin{lemma}\label{Pointwise blending lemma}
	% 	Let $\beta \in C^{1,1}(-\infty,\infty)$, with $0\le \beta \le 1$. For each $v\in \Ya$, there exists $\va, \vc \in \Ya$ such that
	% 	\begin{align}
		% 		\label{Pointwise va blending estimate}\vert \sqrt{1-\beta(\xi)} D_{\rho}v(\xi)-D_{\rho}\va(\xi) \vert &\le \vert \rho \vert^{\frac{3}{2}}\Vert \nabla \sqrt{1-\beta}\Vert_{L^{\infty}} \Vert \nabla v \Vert_{L^{2}(\conv(\xi,\xi+\rho))},\\ 
		% 		\label{Pointwise vc blending estimate1}\vert \sqrt{\beta(\xi)} D_{\rho}v(\xi)-D_{\rho}\vc(\xi) \vert &\le \vert \rho \vert^{\frac{3}{2}}\Vert \nabla \sqrt{\beta}\Vert_{L^{\infty}} \Vert \nabla v \Vert_{L^{2}(\conv(\xi,\xi+\rho))},\\ 
		% 		\label{va and vc} \vert \nabla\va\vert^{2} +	\vert \nabla\vc\vert^{2} &= \vert \nabla v \vert^{2}.
		% 	\end{align}
	% %where $C_{1},\ C_{2}$ may depends on $\rcut$, but $C_{3}$ is a generic constant. In particular, $\nabla \va(x)=0$ for $x\in(-\infty,-L]\cup[L,\infty)$ and $\nabla \vc(x)=0$ for $x \in [-K,K]$.
	% \end{lemma}

% \begin{proof}
	% 	Let $\psi(x):=\sqrt{1-\beta(x)}$ and assume, without loss of generality that $\rho>0$. Then,
	% 	\begin{align*}
		% 		\sqrt{1-\beta(\xi)} D_{\rho}v(\xi)&= \psi (\xi) \sum_{\eta = \xi}^{\xi+\rho-1}D_{1}V(\eta)\\
		% 		&=\sum_{\eta = \xi}^{\xi+\rho-1}\psi(\eta)D_{1}V(\eta) + \sum_{\eta = \xi}^{\xi+\rho-1} (\psi(\xi)-\psi(\eta))D_{1}V(\eta).\\
		% 	\end{align*}
	% If we define $\va$ by $D_{1}\va(\eta)=\psi(\eta)D_{1}v(\eta)$, then we obtain
	% \begin{equation*}
		% 	\sum_{\eta = \xi}^{\xi+\rho-1} \psi(\eta) D_{1}v(\eta) =D_{\rho} \va (\xi),
		% \end{equation*}
	% and after using Holder's inequality we know
	% \begin{align*}
		% 	\vert\sqrt{1-\beta} D_{\rho}v(\xi)-D_{\rho}\va(\xi)\vert&=\sum_{\eta = \xi}^{\xi+\rho-1}(\psi(\xi)-\psi(\eta))D_{1}v(\eta)\\
		% 	&\le (\sum_{\eta = \xi}^{\xi+\rho-1} \Vert \nabla \psi \Vert^{2}_{L^{\infty}} \vert \rho \vert^{2})^{\frac{1}{2}} (\sum_{\eta = \xi}^{\xi+\rho-1}(D_{1}v(\eta))^{2})^{\frac{1}{2}}\\
		% 	&\le \vert \rho \vert^{\frac{3}{2}} \Vert \nabla \psi \Vert_{L^{\infty}} \Vert \nabla v \Vert_{L^{2}(\xi,\xi+\rho)}.
		% \end{align*}
	% This establishes \eqref{Pointwise va blending estimate}. The proof of \eqref{Pointwise vc blending estimate1} is analogous, with $\vc$ defined by $D_{1}\vc(\xi) = \sqrt{\beta(\xi)}D_{1}v(\xi)$.
	
	% With these definitions \eqref{va and vc} is an immediate consequence.
	% \end{proof}

% \begin{theorem}\label{Stability}
	% 	Let $y\in\Ya$ satisfy the strong stability condition \eqref{Atomistic strong local minimizer} and suppose that there exists $\ganllF >0$ such that
	% 	\begin{equation}\label{Uniform solution of NL-L stab assumption}
		% 		\langle \delta^{2}\El (\yF)v,v\rangle \ge \ganllF \Vert \nabla v\Vert^{2}_{L^{2}(-N,N)} \text{ for all } v \in \Ya.
		% 	\end{equation}
	
	% Then
	% \begin{equation*}
		% 	\begin{split}
			% 		\langle \delta^{2}\El (y)v,v\rangle &\ge \min(c_{0},\ganllF)\Vert \nabla v \Vert_{L^{2}(-N,N)}^{2}\\
			% 		&\ - 2M^{(2,\frac{1}{2})}K^{-1} \Vert \nabla v \Vert_{L^{2}(-N,N)}^{2} - \CDH M^{(3,0)}K^{-\alpha} \Vert \nabla v \Vert_{L^{2}(-\bar{L},\bar{L})}^{2} \text{ for all } v \in \Ya.
			% 	\end{split}
		% \end{equation*}
	% \end{theorem}

% \begin{proof}
	% 	Let $K^{'}:=\lfloor K/2\rfloor <K$, and let
	% 	\begin{equation*}
		% 		\beta (x):=\left\{
		% 			\begin{aligned}
			% 				&0, &-K^{'}\le x \le K^{'},\\
			% 				&\hat{\beta}(\frac{x+K^{'}}{K^{'}-K}), &-K\le x \le -K^{'},\\
			% 				&\hat{\beta} (\frac{x-K^{'}}{K-K^{'}}), &K^{'} \le x \le K,\\
			% 				&1, &-N\le x \le -K \text{ or } K\le x\le N.
			% 			\end{aligned}
		% 		\right.
		% 	\end{equation*}
	% where $\hat{\beta}(s)=3s^{3}-2s^{2}$. We know that [\cite{2013_ML_CO_AC_Coupling_ACTANUM},Section 8.3]
	% \begin{equation}\label{Property of beta}
		% 	\Vert \nabla \sqrt{\beta} \Vert_{L^{\infty}} + \Vert \nabla \sqrt{1-\beta} \Vert_{L^{\infty}} \lesssim 	K^{-1}\yz{(C_{\beta} K^{-1}) }.
		% \end{equation}
	
	% 	We can now write \yz{May split it?}
	% 		\begin{align}
		% 			\langle \delta^{2} \El (y)v,v\rangle &= \sum_{\xi=-N}^{N} \sum_{(\rho,\zeta)\in\Rc^{2}} \Phia_{\xi,\rho\zeta} (y)(1-\beta(\xi)) D_{\rho}v(\xi) D_{\zeta}v(\xi)\\
		% 			&\ +\sum_{\xi=-N}^{N} \sum_{(\rho,\zeta)\in\Rc^{2}} \PhiNLL_{\xi,\rho\zeta} (y)(\beta(\xi)) D_{\rho}v(\xi) D_{\zeta}v(\xi)\\
		% 			 &= 
		% 			\label{All atomistic stab}\sum_{\xi=-N}^{N} \sum_{(\rho,\zeta)\in\Rc^{2}} \Phia_{\xi,\rho\zeta} (y)(1-\beta(\xi)) D_{\rho}v(\xi) D_{\zeta}v(\xi)\\
		% 			\label{NL-L atomistic stab} &\ +\sum_{\xi \in \Ac} \sum_{(\rho,\zeta)\in\Rc^{2}} \Phia_{\xi,\rho\zeta} (y)(\beta(\xi))D_{\rho}v(\xi)D_{\zeta}v(\xi)\\
		% 			\label{NL-L interaction stab} &\ +\sum_{\xi \in \Ic} \sum_{(\rho,\zeta)\in\Rc^{2}} \Phii_{\xi,\rho\zeta} (y)(\beta(\xi))D_{\rho}v(\xi)D_{\zeta}v(\xi)\\
		% 			\label{NL-L nonlinear stab} &\ + \int_{\OmeNL} \ppGW (\nabla y)(\beta (x)) (\nabla v)^{2}\d x\\
		% 			\label{NL-L linear stab} &\ +\int_{\OmeL}  \ppGWL (\nabla y)(\beta (x))(\nabla v)^{2}\d x.
		% 		\end{align}
	% 	where we also use the fact that according to our definition of $\beta$, the first sum ranges only over those sites where $\PhiNLL_{\xi} = \Phia_{\xi}$.
	
	% %	Let $\epsilon_{1} = \max (\Vert \nabla \sqrt{1-\beta}\Vert_{L^{\infty}} , \Vert \nabla \sqrt{\beta}\Vert_{L^{\infty}} )$. 
	% We apply estimate \eqref{Pointwise va blending estimate} to \eqref{All atomistic stab}, and we obtain
	% 	\begin{equation}\label{Atomistic va stab result}
		% 		\begin{split}
			% 			\sum_{\xi=-N}^{N} \sum_{(\rho,\zeta)\in\Rc^{2}} \Phia_{\xi,\rho\zeta} (y)(1-\beta(\xi)) D_{\rho}v(\xi) D_{\zeta}v(\xi) &\ge \langle \delta^{2} \Ea (y)\va,\va\rangle 
			% 			- 2M^{(2,\frac{1}{2})}K^{-1}\Vert \nabla v \Vert_{L^{2}(-N,N)}^{2}\\
			% 			&\ge c_{0} \Vert \nabla \va \Vert_{L^{2}(-N,N)}^{2}-2M^{(2,\frac{1}{2})}K^{-1}\Vert \nabla v \Vert_{L^{2}(-N,N)}^{2}.
			% 		\end{split}
		% 	\end{equation}
	
	% 	We apply the estimate \eqref{Pointwise vc blending estimate1} to \eqref{NL-L atomistic stab}, and we know
	% 	\begin{equation}\label{NL-L atomistic stab 1}
		% 		\begin{split}
			% 			\sum_{\xi \in \Ac} \sum_{(\rho,\zeta)\in\Rc^{2}} \Phia_{\xi,\rho\zeta} (y)(\beta(\xi))D_{\rho}v(\xi)D_{\zeta}v(\xi) &\ge \sum_{\xi \in \Ac} \sum_{(\rho,\zeta)\in\Rc^{2}} \Phia_{\xi,\rho\zeta} (y)D_{\rho}\vc(\xi)D_{\zeta}\vc(\xi) \\
			% 			&\ - 2M^{(2,\frac{1}{2})} K^{-1} \Vert \nabla v \Vert_{L^{2}(\OmeA)}^{2}.
			% 		\end{split}
		% 	\end{equation}
	
	% 	By the definition of $\vc$, we notice that for $x \in[-K^{'},K^{'}]$, $\nabla \vc = 0$.After using Taylor's expansion at $\yF$ and assumption $\DH$, we obtain
	% 	\begin{equation}\label{NL-L atomistic stab 2}
		% 		\begin{split}
			% 			\sum_{\xi \in \Ac} \sum_{(\rho,\zeta)\in\Rc^{2}} \Phia_{\xi,\rho\zeta} (y)D_{\rho}\vc(\xi)D_{\zeta}\vc(\xi)&\ge \sum_{\xi \in \Ac} \sum_{(\rho,\zeta)\in\Rc^{2}} \Phia_{\xi,\rho\zeta} (\yF)D_{\rho}\vc(\xi)D_{\zeta}\vc(\xi)\\
			% 			&\ -2^{\alpha}\CDH M^{(3,0)}  (K)^{-\alpha} \Vert\nabla \vc \Vert_{L^{2}(\OmeA)}^{2}.
			% 		\end{split}
		% 	\end{equation}
	
	% 	We combine \eqref{NL-L atomistic stab 1} with \eqref{NL-L atomistic stab 2}, and could get
	% 	\begin{equation}\label{NL-L atomistic stab result}
		% 		\begin{split}
			% 			\sum_{\xi \in \Ac} \sum_{(\rho,\zeta)\in\Rc^{2}} \Phia_{\xi,\rho\zeta} (y)(\beta(\xi))D_{\rho}v(\xi)D_{\zeta}v(\xi) &\ge \sum_{\xi \in \Ac} \sum_{(\rho,\zeta)\in\Rc^{2}} \Phia_{\xi,\rho\zeta} (\yF)D_{\rho}\vc(\xi)D_{\zeta}\vc(\xi) \\
			% 			&\ - 2M^{(2,\frac{1}{2})} K^{-1} \Vert \nabla v \Vert_{L^{2}(\OmeA)}^{2}-2^{\alpha}\CDH M^{(3,0)}  K^{-\alpha}\Vert\nabla \vc \Vert_{L^{2}(\OmeA)}^{2}.
			% 		\end{split}
		% 	\end{equation}
	
	% 	It's a similarly proof for \eqref{NL-L interaction stab}. And we get that
	% 	\begin{equation}\label{NL-L interaction stab result}
		% 		\begin{split}
			% 			\sum_{\xi \in \Ic} \sum_{(\rho,\zeta)\in\Rc^{2}} \Phii_{\xi,\rho\zeta} (y)(\beta(\xi))D_{\rho}v(\xi)D_{\zeta}v(\xi) &\ge \sum_{\xi \in \Ic} \sum_{(\rho,\zeta)\in\Rc^{2}} \Phii_{\xi,\rho\zeta} (\yF)D_{\rho}\vc(\xi)D_{\zeta}\vc(\xi) \\
			% 			&\ - M^{(2,\frac{1}{2})} K^{-1} \Vert \nabla v \Vert_{L^{2}(\OmeI)}^{2}-\CDH M^{(3,0)}  K^{-\alpha} \Vert\nabla \vc \Vert_{L^{2}(\OmeI)}^{2}.
			% 		\end{split}
		% 	\end{equation}
	
	% 	Then we focus on \eqref{NL-L nonlinear stab}. After considering the definition of $\vc$ and assumption $\DH$, we will get
	% 	\begin{equation}\label{NL-L nonlinear stab result}
		% 				\int_{\OmeNL} \ppGW (\nabla y)(\beta(x))(\nabla v)^{2}\d x \ge 	\int_{\OmeNL} \ppGW (\nabla \yF)(\nabla\vc)^{2}\d x - 2\CDH M^{(3,0)}  K^{-\alpha} \Vert \nabla \vc \Vert_{L^{2}(\OmeNL)}^{2}.
		% 			\end{equation}
	
	
	% 	We use the fact $\ppGWL (\nabla y) =\Wppf = \ppGWL (\nabla \yF)$ again, and obtain
	% 	\begin{equation}\label{NL-L linear stab result}
		% 		\int_{\OmeL}\ppGWL (\nabla y)(\beta(x)) (\nabla v)^{2} \d x = \int_{\OmeL}\ppGWL (\nabla \yF) (\nabla \vc)^{2} \d x.
		% 	\end{equation}
	
	% 	\yz{We consider 3-order linear function
		% 	\begin{align*}
			% 		\int_{\OmeL}\ppGWL (\nabla y)(\beta(x)) (\nabla v)^{2} \d x &\ge \int_{\OmeL}\ppGWL (\nabla \yF) (\nabla \vc)^{2} \d x\\
			% 		&\ - \CDH M^{(3,0)}  \bar{L}^{-\alpha} \Vert\nabla \vc \Vert_{L^{2}(\OmeL)}^{2}.
			% 	\end{align*}
		
		
		
		
		% }
	
	
	
	% 	Now we consider definition of $\langle \delta^{2} \El (\yF) \vc,\vc \rangle$, and  use property \eqref{va and vc} and \eqref{Property of beta} to conclude that \yz{ How to say "add \eqref{NL-L atomistic stab result} , \eqref{NL-L interaction stab result}, \eqref{NL-L nonlinear stab result} and \eqref{NL-L linear stab result} together"}
	% 	\begin{align*}
		% 		\langle \delta^{2} \El (y)v,v\rangle &\ge \langle \delta^{2} \El (\yF) \vc,\vc \rangle + \langle \delta^{2}\Ea(y)\va,\va \rangle  \yz{(\min(c_{0},\ganllF)\Vert \nabla v \Vert_{L^{2}(-N,N)}^{2}) }\\
		% 		&\ \yz{- 4M^{(2,\frac{1}{2})}K^{-1} \Vert \nabla v \Vert_{L^{2}(-N,N)}^{2} - 2^{\alpha}\CDH M^{(3,0)}K^{-\alpha} \Vert \nabla v \Vert_{L^{2}(-\bar{L},\bar{L})}^{2} }
		% 	\end{align*}
	
	% \end{proof}



\subsection{A priori existence and error estimate}
\label{sec: priori_anal_qnll_cg}

Next, we consider the a priori error estimation between the coarse-grained QNLL model and the atomistic model. Based on the inverse function theorem and Theorem~\ref{Internal forces of continuum region}, we can obtain the following result:

\begin{theorem}
	Let $\yai \in \Ya$ be a strongly stable atomistic solution satisfying \eqref{All-Atomistic strong local minimizer} and $\DH$. Consider the QNLL problem \eqref{Yh solution} with quasi-optimal choice of $N, \Th$. Suppose, moreover, that $\El$ is stable in the reference state Theorem \ref{Stability}. Then, there exists $K_0$ such that, for all $K \ge K_0$, \eqref{Yh solution} has a locally unique, strongly stable solution $y^{\text{NL-L}}_{h}$ which satisfies
	\begin{equation}
		\begin{aligned}
			\Vert \nabla\yai - \nabla y^{\text{NL-L}}_{h} \Vert_{L^2} \lesssim 8&M^{(3,0)}(\Vert \nabla^{2} u\Vert_{L^{2}(\bOmeI)} +\Vert \nabla^{3}u \Vert_{L^{2}(\bOmeC)}+\Vert \nabla^{2}u \Vert^{2}_{L^{4}(\bOmeC)}\\
			&+ \Vert \nabla u \Vert^{2}_{L^{4}(\OmeL)}+\Vert h \nabla^{2} u\Vert_{L^{2}(\OmeC)}+N^{\frac{1}{2}-\alpha})/\big(\min(c_{0},\ganllF)\big)^2.
		\end{aligned}
	\end{equation}
	
\end{theorem}
\begin{proof}
	
	The proof process here is similar to Theorem \ref{Priori of NCG}, with the difference being the inclusion of coarse-grained error in $\eta$ during the application of the inverse function theorem.
	
	%          We will use the quantitative inverse function theorem, with
	%          \begin{equation*}
		%          \Ghc(\Ph \ya):=\delta \El(\Ph \ya) - \langle f,\cdot\rangle^{\text{L}}_{h}.
		%          \end{equation*}
	
	%              We first apply that the scaling condition implies a Lipschitz bound for $\delta^{2}\El$, 
	%          \begin{equation}\label{scaling assumption}
		%              \Vert \delta^2 \El (y) - \delta^2 \El(v)\Vert_{\mathcal{L}(\Ya,\Ya^{*})}\le M \Vert \nabla y-\nabla v\Vert_{L^{\infty}} \quad \text{for all } y,v \in \Ya,
		%          \end{equation}
	%          where $M \lesssim M^{(3,0)}$. Since $\Vert \cdot \Vert \lesssim \Vert \cdot \Vert_{L^2}$, we can also replace the $L^\infty$- norm on the right-hand side with the $L^2$-norm.
	
	%          The residual estimate \eqref{Consistency result}, \eqref{Decay reslut of best approximation term} and the external residual estimate of Theorem \ref{Decay result of linear external force} give
	%          \begin{align*}
		%              \Vert \Ghc(\Ph \ya)\Vert_{\Uh}\lesssim 
		%               M^{(2,1)}\Vert \nabla^{2} u\Vert_{L^{2}(\bOmeI)} +M^{(2,2)}\Vert \nabla^{3}u \Vert_{L^{2}(\bOmeC)}+M^{(3,2)}\Vert \nabla^{2}u \Vert^{2}_{L^{4}(\bOmeC)}\\
		%         &+ M^{(3,0)}\Vert \nabla u \Vert^{2}_{L^{4}(\OmeL)}
		%          \end{align*}
	
	
	% From Theorem \ref{Stability} we obtain that
	% \begin{equation*}
		% 		\langle \delta^{2}\El (\ya)v_{h},v_{h}\rangle \ge (\min(c_{0},\ganllF)-CK^{-\min(1,\alpha)})\Vert \nabla v_{h} \Vert_{L^{2}}^{2}.
		% \end{equation*}
	
	% Let $\gamma:=\frac{1}{2}\min(c_{0},\ganllF)$. Applying the Lipschitz bound \eqref{scaling assumption} and the best approximation error estimate \eqref{Decay reslut of best approximation term}, we obtain
	% \begin{equation*}
		%     \langle \delta^{2}\El (\Ph\ya)v_{h},v_{h}\rangle \ge (2\gamma-CK^{-\min(1,\alpha)}-CK^{-1/2-\alpha})\Vert \nabla v_{h} \Vert_{L^{2}}^{2}.
		% \end{equation*}
	
	% Hence, for $K$ sufficiently large, we obtain that
	% \begin{equation*}
		%     \langle \delta\Ghc (\Ph\ya)v_{h},v_{h}\rangle \ge \gamma\Vert \nabla v_{h} \Vert_{L^{2}}^{2} \quad \text{for all }v_{h}\in\Uh.
		% \end{equation*}
	
	% Thus, we deduce the existence of $\ynll$ satisfying $\Ghc(\ynll)=0$. The error estimate implies
	% \begin{equation*}
		% \begin{aligned}
			%     \Vert \nabla \ynll - \nabla \Ph\ya\Vert_{L^2}&\lesssim \frac{2M\eta}{\gamma^2} \\
			%     &\lesssim  2M^{(3,0)}(\Vert \nabla^{2} u\Vert_{L^{2}(\bOmeI)} +\Vert \nabla^{3}u \Vert_{L^{2}(\bOmeC)}+\Vert \nabla^{2}u \Vert^{2}_{L^{4}(\bOmeC)}
			%         + \Vert \nabla u \Vert^{2}_{L^{4}(\OmeL)})/\gamma^2.
			% \end{aligned}
		% \end{equation*}
	
	% Applying the best approximation error estimate and $\DH$
	% \begin{equation*}
		%     \Vert \nabla\Ph\ya - \nabla\ya \Vert_{L^2}\le \Vert h \nabla^{2} u\Vert_{L^{2}(\OmeL)}+N^{\frac{1}{2}-\alpha} ,
		% \end{equation*}
	% \yz{which follows from \eqref{Decay reslut of best approximation term}, we finally obtain the stated error bound}.
	
	%where $\Pih$ is an approximation operator, we define
	%\begin{equation*}
	%	\Pih y(x):=\left\{
	%	\begin{aligned}
		%		&\Ih y(x), &\ x\in [0,\bar{L}],\\
		%		&\Ih y(x) -\frac{x-\bar{L}}{N-\bar{L}}u(N), &\ x\in [\bar{L},N].
		%	\end{aligned}
	%\right.
	%\end{equation*}
\end{proof}

\subsection{Discussion of the (quasi-)optimal choice of the length of the nonlinear and linear region}
\label{Balance of QNLL CG model}

% \chw{Give an a priori analysis of how to choose the length of the nonlinear and linear region. The conclusion may be that we only need to have a very short nonlinear region.}

In this subsection, we will discuss how to achieve the quasi-optimal choice of the lengths for finite element mesh $h$, nonlinear continuum region, linear continuum region, and computational region to obtain quasi-optimal convergence order for the QNLL model. We observe that due to this error balance, we only need a very short nonlinear region, this is the key motivation that we introduce this coupling methods to gain the same accuracy but a much more efficient method.

\subsubsection{Optimizing the finite element grid}
\label{sec: choice_of_fem_cg}
The finite mesh size $h$ is the first approximation parameter that we will optimize. In this section, we use a classical technique to optimize the mesh grading.

For each $x\in [-N,N], \ x\in \text{int} T$, let $h(x):=h_{T}$. For $x<-N$ or $x>N$, let $h(x):=1$. The coarse-grained error occurring in the coarsening analysis that depends on $\ThNL$ are the interpolation error term $\Vert h \nabla^{2}u\Vert_{L^{2}(\OmeC)}$. Suppose that $u\in\UhNL$ satisfies $\DH$ and $L >r_{0}$. Then
\begin{equation*}
	\Vert h \nabla^{2}u\Vert_{L^{2}(\OmeC)} \lesssim \Vert h x^{-\alpha-1}\Vert_{L^{2}(\OmeC)}.
\end{equation*}
We wish to choose $h$ to minimize this quantity, subject to fixing the number of degrees of freedom $\NhNL$, which is given by
\begin{equation*}
	\NhNL=\sum_{j=1}^{\NhNL}1=\sum_{j=1}^{\NhNL} h_{j}\frac{1}{h_{j}}=\int_{-N}^{N}\frac{1}{h} \,\d x.
\end{equation*}
We ignore the discreteness of the mesh size function and solve
\begin{equation*}
	\min \Vert h x^{-\alpha-1}\Vert_{L^{2}(\OmeC)} \quad \text{subject to }\int_{-N}^{N} \frac{1}{h}\,\d x = \text{const}.
\end{equation*}
The solution to this variational problem satisfies
\begin{equation*}
	h(x)=\lambda\vert x\vert^{\frac{2}{3}(\alpha+1)} \ \text{for }x\in \OmeC.
\end{equation*}
for some constant $\lambda>0$. This gives us an optimal scaling of the mesh size function.

We now impose the condition $h(L)\approx 1$, which yields
\begin{equation}\label{Mesh size fucntion of nonlinear h}
	h^{\text{NL}}(x)\approx(\frac{\vert x\vert}{L})^{\frac{2}{3}(\alpha+1)}=:\tilde{h}(x) \ \text{for }x\in\OmeC.
\end{equation}

If $\alpha':=\frac{2}{3}(\alpha+1)$, then $\alpha'>1$, and hence this choice of $h$ yields(for simplify, we only calculate the domain$[\bar{K},N]$)
\begin{equation}\label{int of nonlinear h}
	\int_{\bar{K}}^{N} \frac{1}{h} \,\d x \approx \frac{\bar{K}^{\alpha'}(N^{1-\alpha'}-\bar{K}^{1-\alpha'})}{1-\alpha'}\approx \frac{\bar{K}}{\alpha'-1}.
\end{equation}

Thus, the choice \eqref{Mesh size fucntion of nonlinear h} gives a comparable number of degrees of freedom in the linear region to that in the atomistic, interface and continuum regions. The resulting interpolation error bound can be estimated by
\begin{equation}\label{Decay of the nonlinear best approximation term}
	\Vert h x^{-\alpha-1}\Vert_{L^{2}(\OmeC)} \approx \frac{\bar{K}^{\frac{1}{2}-(\alpha+1)}}{(\alpha'-1)^{\frac{1}{2}}} \approx \bar{K}^{-\frac{1}{2}-\alpha}.
\end{equation}



\subsubsection{The quasi-optimal choice of $N$}
\label{sec: choice_of_N_cg}

Before introducing specific finite element mesh generation algorithm, we need to discuss how to determine the length of our computational domain $N$. We follow two principles:
\begin{enumerate}
	\item We should ensure that the truncation error term $N^{\frac{1}{2}-\alpha}$ do not dominate among the various types of errors after balancing the length of the computational domain;
	
	\item We choose the length of the computational domain as small as possible for computational simplicity.
\end{enumerate}

According to the first principle mentioned above, we understand that the truncation error $N^{\frac{1}{2}-\alpha}$ must be balanced against one of the terms of modelling error or coarsening error (or higher-order terms). According to the second principle, to select the computational domain length as small as possible, we must balance it against the lowest-order term of modelling error or coarsening error (balancing against higher-order terms would need a longer computational domain length).


For $\frac{1}{2}< \alpha< 1$, the lowest-order term is the linearization error $\Vert \nabla u \Vert^{2}_{L^{4}(\OmeL)}\lesssim L^{\frac{1}{2}-2\alpha}$, we should choose $N$ such that
\begin{equation*}
	L^{\frac{1}{2}-2\alpha}\approx N^{\frac{1}{2}-\alpha}, \quad \text{that is}, \ N\approx L^{\frac{2\alpha-1/2}{\alpha-1/2}}.
\end{equation*}

For $ \alpha\ge 1$, the lowest-order term is the coarse-grained error $\Vert h\nabla^2 u \Vert_{L^{2}(\OmeC)}\lesssim \bar{K}^{-\frac{1}{2}-\alpha}$, we choose $N$ such that
\begin{equation*}
	\bar{K}^{-\frac{1}{2}-\alpha}\approx N^{\frac{1}{2}-\alpha}, \quad \text{that is}, \ N\approx \bar{K}^{\frac{\alpha+1/2}{\alpha-1/2}}.
\end{equation*}

We now turn this formal motivation into an explicit construction of the finite element mesh. 
% \begin{algorithm}[H]
	% \caption{Adaptive QM/MM algorithm}
	% \label{alg:main}
	
	% {\bf Prescribe} $\LQM, \LMM$, termination tolerance $\eta_{\rm tol}$, refinement tolerance $\tau_{\rm D}$.
	
	% \begin{algorithmic}[1]
		% \REPEAT
		% 	\STATE{ \textit{Solve}: Solve \eqref{problem-e-mix} to obtain $\uH$. }
		% 	\STATE{  \textit{Estimate}: Apply Algorithm \ref{alg:adaptMesh} to compute $\eta_\h(\uH)$ and $\eta_{\h,T}$ (cf. \eqref{eq:APET}, \eqref{eq:rho}). } 		
		% 	\STATE{\textit{Mark}: Use D\"{o}rfler strategy with $\tau_{D}$ to mark {\color{blue} elements in $\Th$} for {\color{blue}refinement}.}
		% 	\STATE{ \textit{Refine:} Construct new $\LQM$ and $\LMM$ regions.}
		% \UNTIL{$\eta_\h(\uH) < \eta_{\rm tol}$}
		% \end{algorithmic}
	% \end{algorithm}

\begin{algorithm}[H]
	\caption{Finite element mesh construction algorithm}
	\label{alg:FEM}
	\begin{enumerate}
		\item[Step 1]: Set $N:= \lceil \bar{K}^{\frac{\alpha+1/2}{\alpha-1/2}}\rceil$($\alpha>1$); $N:= \lceil \bar{K}^{\frac{2\alpha-1/2}{\alpha-1/2}}\rceil$($\frac{1}{2}< \alpha< 1$). 
		\item[Step 2]: Set $\NhNL :=\{0, 1, \dots, \bar{K}\}$.  		
		\item[Step 3]: While $n:=\max (\NhNL) <L$:
		\begin{enumerate}
			\item[Step 3.1]: Set $\NhNL:=\NhNL \cup \{\min(L,n+\lfloor \tilde{h}(n)\rfloor ) \}$.
		\end{enumerate}
		\item[Step 4]: While $n:=\max (\NhNL) <N$:
		\begin{enumerate}
			\item[Step 4.1]: Set $\NhNL:=\NhNL \cup \{\min(N,n+\lfloor \tilde{h}(n)\rfloor ) \}$.
		\end{enumerate}
		\item[Step 5]: Set $\NhNL = (-\NhNL) \cup \NhNL$.
	\end{enumerate}
\end{algorithm}



%\textbf{FEM Algorithm:}
%\begin{enumerate}
%	\item Set $N:= \lceil L^{\frac{2\alpha-1/2}{\alpha-1/2}}\rceil$($\frac{1}{2}< \alpha< 1$); $N:= \lceil \bar{K}^{\frac{\alpha+1/2}{\alpha-1/2}}\rceil$($\alpha\ge1$).
%	\item Set $\NhNL :=\{-\bar{K},\dots,\bar{K}\}$.
%	\item While $n:=\max \NhNL <N$

%	Step 1: Set $\NhNL:=\NhNL \cup \{\min(L,n+\lfloor \tilde{h}(n)\rfloor ) \}(0 \le n\le L)$.

%	Step 2: Set $\NhNL:=\NhNL \cup \{\max(-L,n-\lfloor \tilde{h}(n)\rfloor ) \}(-L\le n\le 0)$.

%	Step 3: Set $\NhNL:=\NhNL \cup \{\min(N,n+\lfloor \tilde{h}(n)\rfloor ) \}(L\le n \le N)$.

%	Step 4: Set $\NhNL:=\NhNL \cup \{\max(-N,n-\lfloor \tilde{h}(n)\rfloor ) \}(-N\le n \le-L)$.
%\end{enumerate}

Meshes construct via this algorithm qualitatively the same properties as predicted by the formal computations \eqref{Mesh size fucntion of nonlinear h} and \eqref{int of nonlinear h}.
\begin{theorem}\label{Decay result of nonlinear best approximation term}
	Let $u\in \Ua$ satisfy $\DH$ and let $N$ and $\ThNL$ be constructed by Algorithm \ref{alg:FEM}. Then, for $L$ sufficiently large, $\NhNL\le C_{1} L$,
	\begin{equation*}
		\begin{aligned}
			\Vert h\nabla^{2} u\Vert_{L^{2}(\OmeC)} &\le C_{2} \NhNL^{-\frac{1}{2}-\alpha},  \\
			\Vert h\nabla^{2} u\Vert^{2}_{L^{4}(\OmeNL)} &\le C_{3} \NhNL^{-\frac{1}{2}-2\alpha},
		\end{aligned}
	\end{equation*}
	where $C_{1}$ depends on $\alpha$ and $C_{2}, C_{3}$ depends on $\alpha$ and on $\CDH$.
\end{theorem}


We now turn the external consistency error estimate into an estimate in terms of $\NhNL$ as well. Let $f$ satisfy $\DH$ and suppose that $\ThNL$ and $N$ are constructed using Algorithm \ref{alg:FEM}. Since $h(x)\le \frac{x}{2}, \ \omega(x)=x\log x$, a straightforward computation yields
\begin{align*}
	\Vert \eta_{\text{ext}} \Vert_{(\YhNL)^{*}} &\lesssim \Vert h^{2} \nabla f\Vert_{L^{2}(\tOmeC)} +\frac{C_{\kappa}}{\log \bar{K}} \Vert h^{2}\omega \nabla^{2}f\Vert_{L^{2}(\tOmeC)}\\
	&\lesssim \bar{K}^{-\alpha-\frac{3}{2}}+\frac{\bar{K}^{-\alpha-\frac{3}{2}}\log^{2}N}{\log \bar{K}}.
\end{align*}

We insert $N \lesssim \bar{K}^{\frac{2}{3}(\alpha+1)}$ to obtain the following result. In particular, we can conclude that the external consistency error is dominated by the interpolation error.

\begin{theorem}\label{Decay result of nonlinear external force}
	Let $f$ satisfy $\DH$ and let $\ThNL$, $N$ be constructed by Algorithm \ref{alg:FEM}. Then
	\begin{equation*}
		\Vert \eta_{\text{ext}} \Vert_{(\YhNL)^{*}} \le C_{\alpha} \bar{K}^{-\alpha-\frac{3}{2}} \log \bar{K}.
	\end{equation*}
	where $C_{\alpha}$ depends on $\alpha$ and on $\CDH$.
\end{theorem}


% \subsubsection{Optimizing the linear finite element grid}
% The finite mesh size $h$ is the first approximation parameter that we will optimize. In this section, we use a classical technique to optimize the mesh grading.

% The two terms occuring in the coarsening analysis tht depends on $\Th$ are the best approximation error terms
% \begin{equation*}
	% 	\Vert h \nabla^{2}u\Vert_{L^{2}(\OmeL)}+N^{\frac{1}{2}-\alpha}.
	% \end{equation*}
% It is easy to see that, for $\bar{L}$ sufficiently largem the best approximation error $\Vert h \nabla^{2}u\Vert_{L^{2}(\OmeL)}$ is the dominant contribution. Thus, we optimize this term only.

% Suppose that $u\in\U$ satisfies $\DH$ and $\bar{L} >r_{0}$. Then
% \begin{equation*}
	% 	\Vert h \nabla^{2}u\Vert_{L^{2}(\OmeL)} \lesssim \Vert h x^{-\alpha-1}\Vert_{L^{2}(\OmeL)}.
	% \end{equation*}
% We wish to choose $h$ to minimize this quantity, subject to fixing the number of degrees of freedom, $N_{\Th}-2$, which is given by
% \begin{equation*}
	% 	N_{\Th}=\sum_{j=1}^{N_{\Th}}1=\sum_{j=1}^{N_{\Th}} h_{j}\frac{1}{h_{j}}=\int_{-N}^{N}\frac{1}{h}\,\d x.
	% \end{equation*}
% We ignore the discreteness of the mesh size function and solve
% \begin{equation*}
	% 	\min \Vert h x^{-\alpha-1}\Vert_{L^{2}(\OmeL)} \quad \text{subject to }\int_{-N}^{N} \frac{1}{h}\,\d x = \text{const}.
	% \end{equation*}
% The solution to this variational problem satisfies
% \begin{equation*}
	% 	h^{\text{L}}(x)=\lambda\vert x\vert^{\frac{2}{3}(\alpha+1)} \quad \text{for }x\in \OmeL.
	% \end{equation*}
% for some constant $\lambda>0$. This gives us an optimal scaling of the mesh size function.

% We now impose the condition $h^{\text{L}}(\bar{L})\approx 1$, which yields
% \begin{equation}\label{Mesh size fucntion of linear h}
	% 	h^{\text{L}}(x)\approx(\frac{\vert x\vert}{\bar{L}})^{\frac{2}{3}(\alpha+1)}=:\tilde{h}^{\text{L}}(x) \ \text{for }x\in\OmeL.
	% \end{equation}
% If $\alpha^{'}:=\frac{2}{3}(\alpha+1)$, then $\alpha^{'}>1$, and hence this choice of $h$ yields (for simplicity, we only calculate the domain$[\bar{L},N]$)
% \begin{equation}\label{int of linear h}
	% 	\int_{\bar{L}}^{N} \frac{1}{h} \d x \approx \frac{\bar{L}^{\alpha’}(N^{1-\alpha‘}-\bar{L}^{1-\alpha‘})}{1-\alpha’}\approx \frac{\bar{L}}{\alpha^{'}-1}.
	% \end{equation}
% Thus, the choice \eqref{Mesh size fucntion of linear h} gives a comparable number of degrees of freedom in the linear region to that in the atomistic, interface and continuum regions. The resulting best approximation error bound can be estimated by
% \begin{equation}\label{Decay of the linear best approximation term}
	% 	\Vert h x^{-\alpha-1}\Vert_{L^{2}(\OmeL)} \approx \frac{\bar{L}^{\frac{1}{2}-(\alpha+1)}}{(\alpha^{'}-1)^{\frac{1}{2}}} \approx \bar{L}^{-\frac{1}{2}-\alpha}.
	% \end{equation}

% We can now also determine an optimal balance between the choices for $\bar{L},N$ and $\Th$. To balance the interpolation error with the far-field error. For $\alpha\ge 1$, we should choose $N$ such that
% \begin{equation*}
	% 	L^{-\frac{1}{2}-\alpha}\approx N^{\frac{1}{2}-\alpha}, \quad \text{that is}, \ N\approx L^{\frac{\alpha+1/2}{\alpha-1/2}}.
	% \end{equation*}

% For $\frac{1}{2}< \alpha< 1$, we choose $N$ such that
% \begin{equation*}
	% 	L^{\frac{1}{2}-2\alpha}\approx N^{\frac{1}{2}-\alpha}, \quad \text{that is}, \ N\approx L^{\frac{2\alpha-1/2}{\alpha-1/2}}.
	% \end{equation*}

% We now turn this formal motivation into an explicit construction of a finite element mesh. To make the linear region beginning node $-\bar{L},\bar{L}$ in our finite element nodes, we make this

% \textbf{Algorithm T:}
% \begin{enumerate}
	% 	\item Set $N:= \lceil L^{\frac{\alpha+1/2}{\alpha-1/2}}\rceil$($\alpha>1$); $N:= \lceil L^{\frac{2\alpha-1/2}{\alpha-1/2}}\rceil$($\frac{1}{2}< \alpha< 1$).
	% 	\item Set $\Nh :=\{-\bar{L},\dots,\bar{L}\}$.
	% 	\item While $n:=\max \Nh <N$
	
	% 	Step 1: Set $\NhNL:=\NhNL \cup \{\min(N,n+\lfloor \tilde{h}(n)\rfloor ) \}(0\le n \le N)$.
	
	% 	Step 2:Set $\NhNL:=\NhNL \cup \{\max(-N,n-\lfloor \tilde{h}(n)\rfloor ) \}(-N\le n \le0)$.
	% \end{enumerate}

% Meshes constructed via this algorithm qualitatively the same properties as predicted by the formal computations (\ref{Mesh size fucntion of linear h}) and \eqref{int of linear h}.
% \begin{theorem}\label{Decay result of linear best approximation term}
	% 	Let $u\in \Ua$ satisfy $\DH$ and let $N$ and $\Th$ be constructed by algorithm. Then, for $\bar{L}$ sufficiently large, $N_{\Th}\le C_{1}^{\text{L}}L$,
	% 	\begin{equation}\label{Decay reslut of best approximation term}
		% 		\Vert h\nabla^{2} u\Vert_{L^{2}(\OmeL)} \le C_{2}^{\text{L}} N_{\Th}^{-\frac{1}{2}-\alpha}.
		% 	\end{equation}
	% 	where $C_{1}^{\text{L}}$ depends on $\alpha$ and $C_{2}^{\text{L}}$ depends on $\alpha$ and on $\CDH$.
	% \end{theorem}
% \begin{proof}
	% 	Already check.
	% \end{proof}

% We now turn the external consistency error estimate into an estimate in terms of $N_{\Th}$ as well. Let $f$ satisfy $\DH$ and suppose that $\Th$ and $N$ are constructed using Algorithm. Since $h(x)\le \frac{x}{2}, \ \omega(x) = x\log x$, a straightforward computation yields
% \begin{align*}
	% 	\Vert \eta_{\text{ext}} \Vert_{(\Uh)^{*}} &\lesssim \Vert h^{2} \nabla f\Vert_{L^{2}(\tOmeL)} +\frac{C_{\kappa}}{\log \bar{L}} \Vert h^{2}\omega \nabla^{2}f\Vert_{L^{2}(\tOmeL)}\\
	% 	&\lesssim \bar{L}^{-\alpha-\frac{3}{2}}+\frac{\bar{L}^{-\alpha-\frac{3}{2}}\log^{2}N}{\log \bar{L}}.
	% \end{align*}

% We insert $N \lesssim \bar{L}^{\frac{2}{3}(\alpha+1)}$ to obtain the following result. In particular, we can conclude that the external consistency error is dominated by the best approximation error.

% \begin{theorem}\label{Decay result of linear external force}
	% 	Let $f$ satisfy $\DH$ and let $\Th,N$ be chosen by algorithm. Then
	% 	\begin{equation*}
		% 		\Vert \eta_{\text{ext}} \Vert_{(\Uh)^{*}} \le C \bar{L}^{-\alpha-\frac{3}{2}} \log \bar{L}.
		% 	\end{equation*}
	% 	where $C$ depends on $\alpha$ and on $\CDH$.
	% \end{theorem}


\subsubsection{The quasi-optimal choice of $L$}
\label{sec: choice_of_L_cg}
We consider the nonlinear-linear elasticity coupling method quasi-optimal choice of approximation parameters $ \bar{K}, L, \NhNL$, for given $K$. For the reason we choose $\rcut=2$, we fix $\bar{K}=K+2$. So we wanna balance $\bar{K}, L$ and $\NhNL$. And the key idea is how to choose $L$ to balance the lowest  order term between $\bar{K},L$.

Firstly, we use $\DH$ assumption, and we could obtain the decay result of coupling error
\begin{equation*}
	\begin{split}
		&M^{(2,1)}\Vert \nabla^{2} u\Vert_{L^{2}(\bOmeI)} +M^{(2,2)}\Vert \nabla^{3}u \Vert_{L^{2}(\bOmeC)}+M^{(3,2)}\Vert \nabla^{2}u \Vert^{2}_{L^{4}(\bOmeC)} + M^{(3,0)}\Vert \nabla u \Vert^{2}_{L^{4}(\OmeL)}\\
		&\lesssim \CDH M^{(2,1)}K^{-\alpha -1}+\CDH M^{(2,2)}\bar{K}^{-\alpha-\frac{3}{2}} +\CDH^{2}M^{(3,2)}\bar{K}^{-2\alpha -\frac{3}{2}}+\CDH^{2}M^{(3,0)}L^{-2\alpha+\frac{1}{2}} .
	\end{split}
\end{equation*}

For coarsening error, we have assumption $\NhNL \lesssim \bar{K}$. From Theorem \ref{Decay result of nonlinear best approximation term} and Theorem \ref{Decay result of nonlinear external force}, we know the interpolation error $\Vert h\nabla^{2}u\Vert_{L^{2}(\OmeC)}$ is the dominant contribution
\begin{equation*}
	\Vert h\nabla^{2} u\Vert_{L^{2}(\OmeC)} \le C_{2}^{\text{NL}} \NhNL^{-\frac{1}{2}-\alpha}.
\end{equation*}

The lowest-order term of $L$ is $\Vert \nabla u \Vert^{2}_{L^{4}(\OmeL)} \lesssim L^{-2\alpha+\frac{1}{2}}$. We balance this term with $	\Vert h\nabla^{2} u\Vert_{L^{2}(\OmeC)} \lesssim \bar{K}^{-\frac{1}{2}-\alpha}$, and get (by noticing the fact that $\bar{K}\le L$)
\begin{align}
	L &\lesssim \bar{K}^{\frac{1}{2}+\frac{3}{8\alpha-2}} \qquad (\frac{1}{2}<\alpha<1) \label{Balance of L CG 1},\\
	L &\approx \bar{K} \qquad \qquad \quad (\alpha \ge 1)\label{Balance of L CG 2}.
\end{align}




% \subsubsection{The quasi-optimal choice of $\bar{L}$ (Linear region)}
% We consider the nonlinear-linear reflection method quasi-optimal choice of approximation parameters $ L, \bar{L}, \Nh$, for given $K$. For the reason we choose $\rcut=2$, we fix $L=K+2$. So we wanna balance $L, \bar{L}$ and $\Nh$. And the key idea is how to choose $\bar{L}$ to balance the lowest  order term between $L,\bar{L}$.

% For coarsening error, we have assumption $\Nh \lesssim \bar{L}$. From Theorem \ref{Decay result of linear best approximation term} and Theorem \ref{Decay result of linear external force}, we know the best approximation error $\Vert h\nabla^{2}u\Vert_{L^{2}(\OmeL)}$ is the dominant contribution.
% \begin{equation}
	% 	\Vert h\nabla^{2} u\Vert_{L^{2}(\OmeL)} \le C_{2}^{\text{L}} N_{\Th}^{-\frac{1}{2}-\alpha}.
	% \end{equation}

% When $\frac{1}{2}<\alpha<1$, the lowest order term of $\bar{L}$ is $\Vert \nabla u \Vert^{2}_{L^{4}(\OmeL)} \lesssim(\bar{L})^{-2\alpha+\frac{1}{2}}$. We balance this term with $	\Vert \nabla^{2} u\Vert_{L^{2}(\bOmeI)} \lesssim L^{-\alpha-1}$, and could get
% \begin{equation}\label{Balance result of linear coarsening error1}
	% 	\bar{L} \sim L^{\frac{1}{2}+\frac{5}{8\alpha-2}}.
	% \end{equation}

% When $\alpha>1$, the lowest order term of $\bar{L}$ is $\Vert  h \nabla^{2}u \Vert_{L^{2}(\OmeL)} \lesssim(\bar{L})^{-\alpha-\frac{1}{2}}$. We balance this term with $	\Vert \nabla^{2} u\Vert_{L^{2}(\bOmeI)} \lesssim L^{-\alpha-1}$, and could get
% \begin{equation}\label{Balance result of linear coarsening error2}
	% 	\bar{L} \sim L^{1+\frac{1}{2\alpha+1}}.
	% \end{equation}

% \yz{
	% In previous sections, we calculate the coupling error and linearization error. Next we consider the $\DH$ assumption, and could get
	
	% \begin{equation}\label{Consistency result}
		% 	\begin{split}
			% 	\Vert \eta^{\text{NL-L}}_{int}\Vert \le &M^{(2,1)}\Vert \nabla^{2} u\Vert_{L^{2}(\bOmeI)} +M^{(2,2)}\Vert \nabla^{3}u \Vert_{L^{2}(\bOmeC)}+M^{(3,2)}\Vert \nabla^{2}u \Vert^{2}_{L^{4}(\bOmeC)}\text{(Coupling error)}\\
			% 	&\ \ + M^{(3,0)}\Vert \nabla u \Vert^{2}_{L^{4}(\OmeL)}\text{(Linearization error)}\\
			% 	&\lesssim \CDH M^{(2,1)}K^{-\alpha -1}+\CDH M^{(2,2)}L^{-\alpha-\frac{3}{2}} +\CDH^{2}M^{(3,2)}L^{-2\alpha -\frac{3}{2}}\\
			% 	&\ \ +\CDH^{2}M^{(3,0)}(\bar{L})^{-2\alpha+\frac{1}{2}} .
			% 	\end{split}
		% \end{equation}
	
	% And we consider the best approximation error estimate
	% \begin{equation}
		% 	\Vert \nabla \Ph \ya-\nabla \ya\Vert_{L^{2}}\lesssim N_{\Th}^{-\frac{1}{2} -\alpha}(\bar{L}^{-\frac{1}{2} -\alpha}).
		% \end{equation}
	% }
% When $\frac{1}{2}<\alpha<1$, error is dominated by the linearization term $\bar{L}^{\frac{1}{2}-2\alpha}$. In order to get the same order, then we let $K^{-\alpha -1}\sim \bar{L}^{-2\alpha+
	% \frac{1}{2}}$. We get
% \begin{equation*}
	% 	\bar{L} \sim L^{\frac{1}{2}+\frac{5}{8\alpha-2}}(1.33\sim3(\text{ when }\alpha=0.8, \text{ it takes 1.63})).
	% \end{equation*}

% In order to get the same order, then we let $L^{-2\alpha -\frac{3}{2}}\sim \bar{L}^{-2\alpha+
	% 	\frac{1}{2}}$. We get
% \begin{equation*}
	% 	\bar{L} \sim L^{1+\frac{4}{4\alpha-1}}(2.33\sim5(\text{ when }\alpha=0.8, \text{ it takes 2.82})).
	% \end{equation*}

% When $\alpha>1$, error is dominated by by the best approximation term $\bar{L}^{-\frac{1}{2}-\alpha}$. In order to get the same order, then we let $K^{-\alpha -1}\sim \bar{L}^{-\alpha-
	% 	\frac{1}{2}}$. We get
% \begin{equation*}
	% 	\bar{L} \sim L^{1+\frac{1}{2\alpha+1}}(1.25(\alpha=1.5)\sim1.33).
	% \end{equation*}

% In order to get the same order, then we let $L^{-2\alpha -\frac{3}{2}}\sim \bar{L}^{-\alpha-
	% 	\frac{1}{2}}$. We get
% \begin{equation*}
	% 	\bar{L} \sim L^{2+\frac{1}{2\alpha+1}}(2.25(\alpha=1.5)\sim2.33).
	% \end{equation*}



%\section{A Posteriori Analysis}
%
%
%\subsection{Residual}
%
%
%\subsection{A Posteriori Stability}
%
%
%\subsection{A Posteriori Existence and Error Estimate}


\subsection{Numerical validation}
\label{sec: experiments_qnll_cg}

In this section, we present numerical experiments to illustrate our analysis. Same as in Section \ref{sec: experiments_qnll_ncg}, the problem is a typical one-dimensional test case, with the site energy modeled using the embedded atom method (EAM), a widely used atomistic model for solids. We fix the exact solution as defined in Section \ref{sec: experiments_qnll_ncg} and compute the external forces, which are equal to the internal forces under the deformation. The decay exponent ensures that the solution and forces satisfy the decay hypothesis $\DH$.

In this section, We will demonstrate the method of controlling the length of non-linear continuum region in the QNLL model to achieve quasi-optimal convergence order, as introduced in Section \ref{Balance of QNLL CG model}. We will conduct numerical experiments with the atomistic model length of 100,000 atoms.We set energy functional and external force to \eqref{EAM of numerical experiments} and \eqref{External force of numerical experiments}, and using Algorithm \ref{alg:FEM} to construct finite element mesh. The experiments will be carried out for $\alpha$ values of $0.8, 1.0$ and $1.2$.

Firstly, let's consider the experiment with alpha set to 1.2: In this case, according to \eqref{Balance of L CG 2}, by setting the length of the nonlinear continuum region to a few atoms ($\bar{K} \approx L$), the convergence order of the QNLL method matches that of the QNL method. In the Figure below, the $x$-axis represents the degrees of freedom (dof) in mesh, while the $y$-axis shows the absolute error $\Vert \nabla \yai - \nabla y^{\text{ac}} \Vert_{L^{2}}~\text{(ac} = \text{QNL, QNLL)}$ between the reference atomistic solution $\yai$ and the a/c solutions $y^{\text{ac}}$. It can be observed that the two convergence order lines in the graph nearly overlap, indicating that the difference between the two AC solutions $\Vert \nabla y^{\text{QNL}} - \nabla y^{\text{QNLL}}\Vert$ is between $10^{-7} $and $10^{-8}$.

\begin{figure}[h]
	\centering 
	\includegraphics[width=0.6\textwidth]{Figs/alpha12_fscale10.pdf}
	\caption{The convergence order of QNL and QNLL method ($\alpha = 1.2$)} % 图片标题
	\label{fig: convergence_QNL_QNLL_alpha12_CG}
\end{figure}

Next, to demonstrate the computational efficiency of the QNLL method, we test the variation in computation time by progressively increasing the degrees of freedom of the nonlinear continuum region $ \Nnl$ of the QNLL method, while keeping the finite element mesh fixed, meaning the continuum region remains unchanged. The results are as shown in the table below: the first column lists the method names, with parentheses indicating the proportion of the degrees of the freedom of nonlinear continuum region$ \Nnl $ to that of the total continuum region$ \Nc$ and the second column records the ratio of the computing time of the QNLL method to the computing time of the QNL method on a device the same as in Section \ref{sec: experiments_qnll_ncg}.

%\begin{center}
%\begin{table}
%    \centering
%\begin{tabular}{|c|c|c|} % 开始一个tabular环境,设置4列,每列居中对齐
%\hline % 绘制表格的横线
%Method($\Nnl/\Nc$) & dof & The ratio of the computing time\\ % 表头行
%\hline % 绘制表格的横线
%% QNLL($17.81\%$) & 2981 & $9.9766156\times 10^{-4}$ & $0.0840$ \\ % 第一行数据
%QNLL($19.84\%$) & 1377 & $69.38\%$ \\ % 第二行数据
%% QNLL($38.69\%$) & 2981 & $9.9766096\times 10^{-4}$ & $0.0024\%$\\ % 第三行数据
%QNLL($48.19\%$) & 1377  & $77.60\%$ \\ % 第四行数据
%% QNLL($59.57\%$) & 2981 &$9.9766096\times 10^{-4}$ & $0.0979 $\\ % 第五行数据
%% QNLL($66.53\%$) & 2981 & $9.9766096\times 10^{-4}$ & $0.0997$ \\ % 第六行数据
%QNLL($81.26\%$) & 1377 & $87.95\%$ \\ % 第七行数据
%% QNLL($83.92\%$) & 2981 & $9.9766097\times 10^{-4}$ & $0.1052$ \\ % 第八行数据
%QNL($100\%$) & 1377 & $100\%$ \\ % 第九行数据
%\hline % 绘制表格的横线
%\end{tabular}
% \caption{The computing time of QNL and QNLL method($\alpha = 1.2$)}
%    \label{tab:computing time alpha12 CG}
%\end{table}
%\end{center}

\begin{table}
	\centering
	\renewcommand{\arraystretch}{1.5} % 调整行间距
	\begin{tabular}{|c|c|} % 开始一个tabular环境,设置2列,每列居中对齐
		\hline % 绘制表格的横线
		Method ($\Nnl/\Nc$) & The ratio of the computing time\\ % 表头行
		\hline % 绘制表格的横线
		QNLL ($19.84\%$) & $69.38\%$ \\ % 第二行数据
		QNLL ($48.19\%$) & $77.60\%$ \\ % 第四行数据
		QNLL ($81.26\%$) & $87.95\%$ \\ % 第七行数据
		QNL ($100\%$) & $100\%$ \\ % 第九行数据
		\hline % 绘制表格的横线
	\end{tabular}
	\caption{The computing time (with coarse graining) of QNL and QNLL method ($\alpha = 1.2$), with degree of freedom set to 1377 for all methods.}
	\label{tab:computing time alpha12 CG}
\end{table}

% \begin{center}
	% \begin{table}[htbp]
		%     \centering
		%     \caption{The computing time of QNL and QNLL method($\alpha = 1.2$)}
		% \begin{tabular}{|c|c|c|c|} % 开始一个tabular环境,设置4列,每列居中对齐
			% \hline % 绘制表格的横线
			% Method($\left| \OmeNL\right|/\left| \OmeC\right|$) & dof & The ratio of the absolute error & The ratio of the computing time\\ % 表头行
			% \hline % 绘制表格的横线
			% % QNLL($17.81\%$) & 2981 & $9.9766156\times 10^{-4}$ & $0.0840$ \\ % 第一行数据
			% QNLL($28.25\%$) & 2981 & $9.9766111\times 10^{-4}$ & $0.0881$ \\ % 第二行数据
			% % QNLL($38.69\%$) & 2981 & $9.9766096\times 10^{-4}$ & $0.0908 $\\ % 第三行数据
			% QNLL($49.13\%$) & 2981 & $9.9766099\times 10^{-4}$ & $0.0942$ \\ % 第四行数据
			% % QNLL($59.57\%$) & 2981 &$9.9766096\times 10^{-4}$ & $0.0979 $\\ % 第五行数据
			% % QNLL($66.53\%$) & 2981 & $9.9766096\times 10^{-4}$ & $0.0997$ \\ % 第六行数据
			% QNLL($76.97\%$) & 2981 & $9.9766095\times 10^{-4}$ & $0.1030$ \\ % 第七行数据
			% % QNLL($83.92\%$) & 2981 & $9.9766097\times 10^{-4}$ & $0.1052$ \\ % 第八行数据
			% Reflection($100\%$) & 2981 & $9.9766094\times 10^{-4}$ & $0.1286$ \\ % 第九行数据
			% \hline % 绘制表格的横线
			% \end{tabular}
		% \end{table}
	% \end{center}


According to the Table \ref{tab:computing time alpha12 CG}, we can see that as the proportion of the nonlinear continuum region length to the total continuum region length increases, computing time clearly increases. However, in practical applications, the proportion of nonlinear elements will be lower (below 5$\%$) according to the balancing method described in Section \ref{Balance of QNLL CG model}. The ratio of the difference between the absolute errors of the QNLL method and the QNL method to the absolute errors of the QNL method: $( \Vert \nabla \yai - \nabla y^{\text{QNLL}} \Vert_{L^{2}} - \Vert \nabla \yai - \nabla y^{\text{QNL}} \Vert_{L^{2}}) / \Vert \nabla \yai - \nabla y^{\text{QNL}} \Vert_{L^{2}}$ is in a narrow range. Here, the ratio, as defined above, is within the range of $10^{-5}$ to $10^{-6}$. This indicates that the QNLL method maintains high accuracy while still offering computational efficiency advantages.

When $\alpha = 1$, the results are similar to when $\alpha = 1.2$. The following figure compares the convergence order of the QNLL method and the QNL method. The information represented on the axes is the same as in Figure \ref{fig: convergence_QNL_QNLL_alpha12_CG}. We observe a similar outcome to Figure \ref{fig: convergence_QNL_QNLL_alpha12_CG}, where the convergence lines of the QNLL method closely overlap with those of the QNL method.

\begin{figure}[h]
	\centering 
	\includegraphics[width=0.6\textwidth]{Figs/alpha10_fscale05.pdf}
	\caption{The convergence order of QNL and QNLL method ($\alpha = 1.0$)} % 图片标题
	\label{fig: convergence_QNL_QNLL_alpha10_CG}
\end{figure}

Similar to the case when $\alpha = 0.8$, we test the gradual increase in length of the nonlinear continuum region under the condition of a fixed finite element mesh. The remaining configurations and the information represented in each column are the same as in Table  \ref{tab:computing time alpha12 CG}. We observed results similar to Table \ref{tab:computing time alpha12 CG}, where as the proportion of the nonlinear continuum region length to the continuum region length increases gradually, the computing time also increases gradually, but there is no significant reduction in error.

%\begin{center}
%\begin{table}
%    \centering
%\begin{tabular}{|c|c|c|c|} % 开始一个tabular环境,设置4列,每列居中对齐
%\hline % 绘制表格的横线
%Method($\Nnl/\Nc$) & dof & The ratio of the computing time\\ % 表头行
%\hline % 绘制表格的横线
%% QNLL($17.81\%$) & 2981 & % % QNLL($17.81\%$) & 2981 & $9.9766156\times 10^{-4}$ & $0.0840$ \\ % 第一行数据
%QNLL($27.89\%$) & 1943 & $78.00\%$ \\ % 第二行数据
%% QNLL($38.69\%$) & 2981 & $9.9766096\times 10^{-4}$ & $0.0908 $\\ % 第三行数据
%QNLL($49.67\%$) & 1943 & $84.46\%$ \\ % 第四行数据
%% QNLL($59.57\%$) & 2981 &$9.9766096\times 10^{-4}$ & $0.0979 $\\ % 第五行数据
%% QNLL($66.53\%$) & 2981 & $9.9766096\times 10^{-4}$ & $0.0997$ \\ % 第六行数据
%QNLL($76.91\%$) & 1943 & $92.18\%$ \\ % 第七行数据
%% QNLL($83.92\%$) & 2981 & $9.9766097\times 10^{-4}$ & $0.1052$ \\ % 第八行数据
%QNL($100\%$) & 1943 & $100\%$ \\ % 第九行数据
%\hline % 绘制表格的横线
%\end{tabular}
%    \caption{The computing time of QNL and QNLL method($\alpha = 1.0$)}
%       \label{tab:computing time alpha10 CG}
%\end{table}
%\end{center}
\begin{table}
	\centering
	\renewcommand{\arraystretch}{1.5} % 调整行间距
	\begin{tabular}{|c|c|} % 开始一个tabular环境,设置2列,每列居中对齐
		\hline % 绘制表格的横线
		Method ($\Nnl/\Nc$) & The ratio of the computing time\\ % 表头行
		\hline % 绘制表格的横线
		QNLL ($27.89\%$) & $78.00\%$ \\ % 第二行数据
		QNLL ($49.67\%$) & $84.46\%$ \\ % 第四行数据
		QNLL ($76.91\%$) & $92.18\%$ \\ % 第七行数据
		QNL ($100\%$) & $100\%$ \\ % 第九行数据
		\hline % 绘制表格的横线
	\end{tabular}
	\caption{The computing time (with coarse graining) of QNL and QNLL method ($\alpha = 1.0$), with degree of freedom set to 1943 for all methods.}
	\label{tab:computing time alpha10 CG}
\end{table}


Furthermore, we will now consider the case where $\alpha=0.8$. In this setting, according to \eqref{Balance of L CG 1} and \eqref{Balance of L CG 2}, there are two finite element mesh generation schemes for the QNLL method:
\begin{enumerate}
	\item In the first scheme, we focus on the accuracy of the QNLL method. According to \eqref{Balance of L CG 1}, we precisely balance the atomistic region, nonlinear continuum region, linear continuum region, and the total length of the computational domain to achieve convergence order identical to those of the QNL method.
	
	\item In the second scheme, we prioritize the computational efficiency of the QNLL method. Therefore, after balancing the lengths of the atomistic region and the total length of the computational domain, we minimize the length of the nonlinear continuum region as much as possible, even down to just a few atoms.
\end{enumerate}

In Figure~\ref{fig: convergence_QNL_QNLL_alpha08_CG}, we represent the first finite element mesh generation scheme with red dashed squares for the QNLL method, and the second generation scheme with blue dashed squares. To demonstrate the accuracy of the QNLL method, the QNL method also adopts the first finite element mesh generation scheme, depicted in the figure with red dashed star symbols. The information represent on the axes is the same as in Figure \ref{fig: convergence_QNL_QNLL_alpha12_CG}. We observe that, after balancing the lengths of the atomistic region, nonlinear continuum region, linear continuum region, and the total length of the computational domain, the absolute errors and convergence order obtained by the QNLL method are consistent with those of the QNL method. However, after reducing the length of the nonlinear continuum region in pursuit of computational efficiency, there is a noticeable increase in absolute errors and a decrease in convergence speed.

\begin{figure}
	\centering 
	\includegraphics[width=0.6\textwidth]{Figs/alpha08_fscale05.pdf}
	\caption{The convergence order of QNL and QNLL method ($\alpha = 0.8$)} % 图片标题
	\label{fig: convergence_QNL_QNLL_alpha08_CG}
\end{figure}

The following table displays the changes in computation time as the length of the nonlinear continuum region gradually increase, with $\alpha$ set to 0.8. The remaining configurations and the information represented in each column are the same as in Table \ref{tab:computing time alpha12 CG}. Similar to Table \ref{tab:computing time alpha12 CG}, we observe that as the proportion of the nonlinear continuum region length to the continuum region length increases, computing time gradually increases, but there is no significant reduction in error.

%\begin{center}
%\begin{table}[h]
%    \centering
%\begin{tabular}{|c|c|c|} % 开始一个tabular环境,设置4列,每列居中对齐
%\hline % 绘制表格的横线
%Method($\Nnl/\Nc$) & dof & The ratio of the computing time\\ % 表头行
%\hline % 绘制表格的横线
%% QNLL($17.81\%$) & 2981 & % % QNLL($17.81\%$) & 2981 & $9.9766156\times 10^{-4}$ & $0.0840$ \\ % 第一行数据
%QNLL($28.25\%$) & 2981 & $68.50\%$ \\ % 第二行数据
%% QNLL($38.69\%$) & 2981 & $9.9766096\times 10^{-4}$ & $0.0908 $\\ % 第三行数据
%QNLL($49.13\%$) & 2981 & $73.26\%$ \\ % 第四行数据
%% QNLL($59.57\%$) & 2981 &$9.9766096\times 10^{-4}$ & $0.0979 $\\ % 第五行数据
%% QNLL($66.53\%$) & 2981 & $9.9766096\times 10^{-4}$ & $0.0997$ \\ % 第六行数据
%QNLL($76.97\%$) & 2981 & $80.08\%$ \\ % 第七行数据
%% QNLL($83.92\%$) & 2981 & $9.9766097\times 10^{-4}$ & $0.1052$ \\ % 第八行数据
%QNL($100\%$) & 2981 & $100\%$ \\ % 第九行数据
%\hline % 绘制表格的横线
%\end{tabular}
%    \caption{The computing time of QNL and QNLL method($\alpha = 0.8$)}
%   \label{tab:computing time alpha08 CG}
%\end{table}
%\end{center}
\begin{table}[h]
	\centering
	\renewcommand{\arraystretch}{1.5} % 调整行间距
	\begin{tabular}{|c|c|} % 开始一个tabular环境,设置2列,每列居中对齐
		\hline % 绘制表格的横线
		Method ($\Nnl/\Nc$) & The ratio of the computing time\\ % 表头行
		\hline % 绘制表格的横线
		QNLL ($28.25\%$) & $68.50\%$ \\ % 第二行数据
		QNLL ($49.13\%$) & $73.26\%$ \\ % 第四行数据
		QNLL ($76.97\%$) & $80.08\%$ \\ % 第七行数据
		QNL ($100\%$) & $100\%$ \\ % 第九行数据
		\hline % 绘制表格的横线
	\end{tabular}
	\caption{The computing time (with coarse graining) of QNL and QNLL method ($\alpha = 0.8$), with Degree of Freedom (DoF) set to 2981 for all methods.}
	\label{tab:computing time alpha08 CG}
\end{table}


%===========================================================================
	
	\section{Conclusion}
In this work, we propose a simple yet effective approach, called SMILE, for graph few-shot learning with fewer tasks. Specifically, we introduce a novel dual-level mixup strategy, including within-task and across-task mixup, for enriching the diversity of nodes within each task and the diversity of tasks. Also, we incorporate the degree-based prior information to learn expressive node embeddings. Theoretically, we prove that SMILE effectively enhances the model's generalization performance. Empirically, we conduct extensive experiments on multiple benchmarks and the results suggest that SMILE significantly outperforms other baselines, including both in-domain and cross-domain few-shot settings.
	
	
	\appendix
	\renewcommand\thesection{\appendixname~\Alph{section}}
	
	\subsection{Lloyd-Max Algorithm}
\label{subsec:Lloyd-Max}
For a given quantization bitwidth $B$ and an operand $\bm{X}$, the Lloyd-Max algorithm finds $2^B$ quantization levels $\{\hat{x}_i\}_{i=1}^{2^B}$ such that quantizing $\bm{X}$ by rounding each scalar in $\bm{X}$ to the nearest quantization level minimizes the quantization MSE. 

The algorithm starts with an initial guess of quantization levels and then iteratively computes quantization thresholds $\{\tau_i\}_{i=1}^{2^B-1}$ and updates quantization levels $\{\hat{x}_i\}_{i=1}^{2^B}$. Specifically, at iteration $n$, thresholds are set to the midpoints of the previous iteration's levels:
\begin{align*}
    \tau_i^{(n)}=\frac{\hat{x}_i^{(n-1)}+\hat{x}_{i+1}^{(n-1)}}2 \text{ for } i=1\ldots 2^B-1
\end{align*}
Subsequently, the quantization levels are re-computed as conditional means of the data regions defined by the new thresholds:
\begin{align*}
    \hat{x}_i^{(n)}=\mathbb{E}\left[ \bm{X} \big| \bm{X}\in [\tau_{i-1}^{(n)},\tau_i^{(n)}] \right] \text{ for } i=1\ldots 2^B
\end{align*}
where to satisfy boundary conditions we have $\tau_0=-\infty$ and $\tau_{2^B}=\infty$. The algorithm iterates the above steps until convergence.

Figure \ref{fig:lm_quant} compares the quantization levels of a $7$-bit floating point (E3M3) quantizer (left) to a $7$-bit Lloyd-Max quantizer (right) when quantizing a layer of weights from the GPT3-126M model at a per-tensor granularity. As shown, the Lloyd-Max quantizer achieves substantially lower quantization MSE. Further, Table \ref{tab:FP7_vs_LM7} shows the superior perplexity achieved by Lloyd-Max quantizers for bitwidths of $7$, $6$ and $5$. The difference between the quantizers is clear at 5 bits, where per-tensor FP quantization incurs a drastic and unacceptable increase in perplexity, while Lloyd-Max quantization incurs a much smaller increase. Nevertheless, we note that even the optimal Lloyd-Max quantizer incurs a notable ($\sim 1.5$) increase in perplexity due to the coarse granularity of quantization. 

\begin{figure}[h]
  \centering
  \includegraphics[width=0.7\linewidth]{sections/figures/LM7_FP7.pdf}
  \caption{\small Quantization levels and the corresponding quantization MSE of Floating Point (left) vs Lloyd-Max (right) Quantizers for a layer of weights in the GPT3-126M model.}
  \label{fig:lm_quant}
\end{figure}

\begin{table}[h]\scriptsize
\begin{center}
\caption{\label{tab:FP7_vs_LM7} \small Comparing perplexity (lower is better) achieved by floating point quantizers and Lloyd-Max quantizers on a GPT3-126M model for the Wikitext-103 dataset.}
\begin{tabular}{c|cc|c}
\hline
 \multirow{2}{*}{\textbf{Bitwidth}} & \multicolumn{2}{|c|}{\textbf{Floating-Point Quantizer}} & \textbf{Lloyd-Max Quantizer} \\
 & Best Format & Wikitext-103 Perplexity & Wikitext-103 Perplexity \\
\hline
7 & E3M3 & 18.32 & 18.27 \\
6 & E3M2 & 19.07 & 18.51 \\
5 & E4M0 & 43.89 & 19.71 \\
\hline
\end{tabular}
\end{center}
\end{table}

\subsection{Proof of Local Optimality of LO-BCQ}
\label{subsec:lobcq_opt_proof}
For a given block $\bm{b}_j$, the quantization MSE during LO-BCQ can be empirically evaluated as $\frac{1}{L_b}\lVert \bm{b}_j- \bm{\hat{b}}_j\rVert^2_2$ where $\bm{\hat{b}}_j$ is computed from equation (\ref{eq:clustered_quantization_definition}) as $C_{f(\bm{b}_j)}(\bm{b}_j)$. Further, for a given block cluster $\mathcal{B}_i$, we compute the quantization MSE as $\frac{1}{|\mathcal{B}_{i}|}\sum_{\bm{b} \in \mathcal{B}_{i}} \frac{1}{L_b}\lVert \bm{b}- C_i^{(n)}(\bm{b})\rVert^2_2$. Therefore, at the end of iteration $n$, we evaluate the overall quantization MSE $J^{(n)}$ for a given operand $\bm{X}$ composed of $N_c$ block clusters as:
\begin{align*}
    \label{eq:mse_iter_n}
    J^{(n)} = \frac{1}{N_c} \sum_{i=1}^{N_c} \frac{1}{|\mathcal{B}_{i}^{(n)}|}\sum_{\bm{v} \in \mathcal{B}_{i}^{(n)}} \frac{1}{L_b}\lVert \bm{b}- B_i^{(n)}(\bm{b})\rVert^2_2
\end{align*}

At the end of iteration $n$, the codebooks are updated from $\mathcal{C}^{(n-1)}$ to $\mathcal{C}^{(n)}$. However, the mapping of a given vector $\bm{b}_j$ to quantizers $\mathcal{C}^{(n)}$ remains as  $f^{(n)}(\bm{b}_j)$. At the next iteration, during the vector clustering step, $f^{(n+1)}(\bm{b}_j)$ finds new mapping of $\bm{b}_j$ to updated codebooks $\mathcal{C}^{(n)}$ such that the quantization MSE over the candidate codebooks is minimized. Therefore, we obtain the following result for $\bm{b}_j$:
\begin{align*}
\frac{1}{L_b}\lVert \bm{b}_j - C_{f^{(n+1)}(\bm{b}_j)}^{(n)}(\bm{b}_j)\rVert^2_2 \le \frac{1}{L_b}\lVert \bm{b}_j - C_{f^{(n)}(\bm{b}_j)}^{(n)}(\bm{b}_j)\rVert^2_2
\end{align*}

That is, quantizing $\bm{b}_j$ at the end of the block clustering step of iteration $n+1$ results in lower quantization MSE compared to quantizing at the end of iteration $n$. Since this is true for all $\bm{b} \in \bm{X}$, we assert the following:
\begin{equation}
\begin{split}
\label{eq:mse_ineq_1}
    \tilde{J}^{(n+1)} &= \frac{1}{N_c} \sum_{i=1}^{N_c} \frac{1}{|\mathcal{B}_{i}^{(n+1)}|}\sum_{\bm{b} \in \mathcal{B}_{i}^{(n+1)}} \frac{1}{L_b}\lVert \bm{b} - C_i^{(n)}(b)\rVert^2_2 \le J^{(n)}
\end{split}
\end{equation}
where $\tilde{J}^{(n+1)}$ is the the quantization MSE after the vector clustering step at iteration $n+1$.

Next, during the codebook update step (\ref{eq:quantizers_update}) at iteration $n+1$, the per-cluster codebooks $\mathcal{C}^{(n)}$ are updated to $\mathcal{C}^{(n+1)}$ by invoking the Lloyd-Max algorithm \citep{Lloyd}. We know that for any given value distribution, the Lloyd-Max algorithm minimizes the quantization MSE. Therefore, for a given vector cluster $\mathcal{B}_i$ we obtain the following result:

\begin{equation}
    \frac{1}{|\mathcal{B}_{i}^{(n+1)}|}\sum_{\bm{b} \in \mathcal{B}_{i}^{(n+1)}} \frac{1}{L_b}\lVert \bm{b}- C_i^{(n+1)}(\bm{b})\rVert^2_2 \le \frac{1}{|\mathcal{B}_{i}^{(n+1)}|}\sum_{\bm{b} \in \mathcal{B}_{i}^{(n+1)}} \frac{1}{L_b}\lVert \bm{b}- C_i^{(n)}(\bm{b})\rVert^2_2
\end{equation}

The above equation states that quantizing the given block cluster $\mathcal{B}_i$ after updating the associated codebook from $C_i^{(n)}$ to $C_i^{(n+1)}$ results in lower quantization MSE. Since this is true for all the block clusters, we derive the following result: 
\begin{equation}
\begin{split}
\label{eq:mse_ineq_2}
     J^{(n+1)} &= \frac{1}{N_c} \sum_{i=1}^{N_c} \frac{1}{|\mathcal{B}_{i}^{(n+1)}|}\sum_{\bm{b} \in \mathcal{B}_{i}^{(n+1)}} \frac{1}{L_b}\lVert \bm{b}- C_i^{(n+1)}(\bm{b})\rVert^2_2  \le \tilde{J}^{(n+1)}   
\end{split}
\end{equation}

Following (\ref{eq:mse_ineq_1}) and (\ref{eq:mse_ineq_2}), we find that the quantization MSE is non-increasing for each iteration, that is, $J^{(1)} \ge J^{(2)} \ge J^{(3)} \ge \ldots \ge J^{(M)}$ where $M$ is the maximum number of iterations. 
%Therefore, we can say that if the algorithm converges, then it must be that it has converged to a local minimum. 
\hfill $\blacksquare$


\begin{figure}
    \begin{center}
    \includegraphics[width=0.5\textwidth]{sections//figures/mse_vs_iter.pdf}
    \end{center}
    \caption{\small NMSE vs iterations during LO-BCQ compared to other block quantization proposals}
    \label{fig:nmse_vs_iter}
\end{figure}

Figure \ref{fig:nmse_vs_iter} shows the empirical convergence of LO-BCQ across several block lengths and number of codebooks. Also, the MSE achieved by LO-BCQ is compared to baselines such as MXFP and VSQ. As shown, LO-BCQ converges to a lower MSE than the baselines. Further, we achieve better convergence for larger number of codebooks ($N_c$) and for a smaller block length ($L_b$), both of which increase the bitwidth of BCQ (see Eq \ref{eq:bitwidth_bcq}).


\subsection{Additional Accuracy Results}
%Table \ref{tab:lobcq_config} lists the various LOBCQ configurations and their corresponding bitwidths.
\begin{table}
\setlength{\tabcolsep}{4.75pt}
\begin{center}
\caption{\label{tab:lobcq_config} Various LO-BCQ configurations and their bitwidths.}
\begin{tabular}{|c||c|c|c|c||c|c||c|} 
\hline
 & \multicolumn{4}{|c||}{$L_b=8$} & \multicolumn{2}{|c||}{$L_b=4$} & $L_b=2$ \\
 \hline
 \backslashbox{$L_A$\kern-1em}{\kern-1em$N_c$} & 2 & 4 & 8 & 16 & 2 & 4 & 2 \\
 \hline
 64 & 4.25 & 4.375 & 4.5 & 4.625 & 4.375 & 4.625 & 4.625\\
 \hline
 32 & 4.375 & 4.5 & 4.625& 4.75 & 4.5 & 4.75 & 4.75 \\
 \hline
 16 & 4.625 & 4.75& 4.875 & 5 & 4.75 & 5 & 5 \\
 \hline
\end{tabular}
\end{center}
\end{table}

%\subsection{Perplexity achieved by various LO-BCQ configurations on Wikitext-103 dataset}

\begin{table} \centering
\begin{tabular}{|c||c|c|c|c||c|c||c|} 
\hline
 $L_b \rightarrow$& \multicolumn{4}{c||}{8} & \multicolumn{2}{c||}{4} & 2\\
 \hline
 \backslashbox{$L_A$\kern-1em}{\kern-1em$N_c$} & 2 & 4 & 8 & 16 & 2 & 4 & 2  \\
 %$N_c \rightarrow$ & 2 & 4 & 8 & 16 & 2 & 4 & 2 \\
 \hline
 \hline
 \multicolumn{8}{c}{GPT3-1.3B (FP32 PPL = 9.98)} \\ 
 \hline
 \hline
 64 & 10.40 & 10.23 & 10.17 & 10.15 &  10.28 & 10.18 & 10.19 \\
 \hline
 32 & 10.25 & 10.20 & 10.15 & 10.12 &  10.23 & 10.17 & 10.17 \\
 \hline
 16 & 10.22 & 10.16 & 10.10 & 10.09 &  10.21 & 10.14 & 10.16 \\
 \hline
  \hline
 \multicolumn{8}{c}{GPT3-8B (FP32 PPL = 7.38)} \\ 
 \hline
 \hline
 64 & 7.61 & 7.52 & 7.48 &  7.47 &  7.55 &  7.49 & 7.50 \\
 \hline
 32 & 7.52 & 7.50 & 7.46 &  7.45 &  7.52 &  7.48 & 7.48  \\
 \hline
 16 & 7.51 & 7.48 & 7.44 &  7.44 &  7.51 &  7.49 & 7.47  \\
 \hline
\end{tabular}
\caption{\label{tab:ppl_gpt3_abalation} Wikitext-103 perplexity across GPT3-1.3B and 8B models.}
\end{table}

\begin{table} \centering
\begin{tabular}{|c||c|c|c|c||} 
\hline
 $L_b \rightarrow$& \multicolumn{4}{c||}{8}\\
 \hline
 \backslashbox{$L_A$\kern-1em}{\kern-1em$N_c$} & 2 & 4 & 8 & 16 \\
 %$N_c \rightarrow$ & 2 & 4 & 8 & 16 & 2 & 4 & 2 \\
 \hline
 \hline
 \multicolumn{5}{|c|}{Llama2-7B (FP32 PPL = 5.06)} \\ 
 \hline
 \hline
 64 & 5.31 & 5.26 & 5.19 & 5.18  \\
 \hline
 32 & 5.23 & 5.25 & 5.18 & 5.15  \\
 \hline
 16 & 5.23 & 5.19 & 5.16 & 5.14  \\
 \hline
 \multicolumn{5}{|c|}{Nemotron4-15B (FP32 PPL = 5.87)} \\ 
 \hline
 \hline
 64  & 6.3 & 6.20 & 6.13 & 6.08  \\
 \hline
 32  & 6.24 & 6.12 & 6.07 & 6.03  \\
 \hline
 16  & 6.12 & 6.14 & 6.04 & 6.02  \\
 \hline
 \multicolumn{5}{|c|}{Nemotron4-340B (FP32 PPL = 3.48)} \\ 
 \hline
 \hline
 64 & 3.67 & 3.62 & 3.60 & 3.59 \\
 \hline
 32 & 3.63 & 3.61 & 3.59 & 3.56 \\
 \hline
 16 & 3.61 & 3.58 & 3.57 & 3.55 \\
 \hline
\end{tabular}
\caption{\label{tab:ppl_llama7B_nemo15B} Wikitext-103 perplexity compared to FP32 baseline in Llama2-7B and Nemotron4-15B, 340B models}
\end{table}

%\subsection{Perplexity achieved by various LO-BCQ configurations on MMLU dataset}


\begin{table} \centering
\begin{tabular}{|c||c|c|c|c||c|c|c|c|} 
\hline
 $L_b \rightarrow$& \multicolumn{4}{c||}{8} & \multicolumn{4}{c||}{8}\\
 \hline
 \backslashbox{$L_A$\kern-1em}{\kern-1em$N_c$} & 2 & 4 & 8 & 16 & 2 & 4 & 8 & 16  \\
 %$N_c \rightarrow$ & 2 & 4 & 8 & 16 & 2 & 4 & 2 \\
 \hline
 \hline
 \multicolumn{5}{|c|}{Llama2-7B (FP32 Accuracy = 45.8\%)} & \multicolumn{4}{|c|}{Llama2-70B (FP32 Accuracy = 69.12\%)} \\ 
 \hline
 \hline
 64 & 43.9 & 43.4 & 43.9 & 44.9 & 68.07 & 68.27 & 68.17 & 68.75 \\
 \hline
 32 & 44.5 & 43.8 & 44.9 & 44.5 & 68.37 & 68.51 & 68.35 & 68.27  \\
 \hline
 16 & 43.9 & 42.7 & 44.9 & 45 & 68.12 & 68.77 & 68.31 & 68.59  \\
 \hline
 \hline
 \multicolumn{5}{|c|}{GPT3-22B (FP32 Accuracy = 38.75\%)} & \multicolumn{4}{|c|}{Nemotron4-15B (FP32 Accuracy = 64.3\%)} \\ 
 \hline
 \hline
 64 & 36.71 & 38.85 & 38.13 & 38.92 & 63.17 & 62.36 & 63.72 & 64.09 \\
 \hline
 32 & 37.95 & 38.69 & 39.45 & 38.34 & 64.05 & 62.30 & 63.8 & 64.33  \\
 \hline
 16 & 38.88 & 38.80 & 38.31 & 38.92 & 63.22 & 63.51 & 63.93 & 64.43  \\
 \hline
\end{tabular}
\caption{\label{tab:mmlu_abalation} Accuracy on MMLU dataset across GPT3-22B, Llama2-7B, 70B and Nemotron4-15B models.}
\end{table}


%\subsection{Perplexity achieved by various LO-BCQ configurations on LM evaluation harness}

\begin{table} \centering
\begin{tabular}{|c||c|c|c|c||c|c|c|c|} 
\hline
 $L_b \rightarrow$& \multicolumn{4}{c||}{8} & \multicolumn{4}{c||}{8}\\
 \hline
 \backslashbox{$L_A$\kern-1em}{\kern-1em$N_c$} & 2 & 4 & 8 & 16 & 2 & 4 & 8 & 16  \\
 %$N_c \rightarrow$ & 2 & 4 & 8 & 16 & 2 & 4 & 2 \\
 \hline
 \hline
 \multicolumn{5}{|c|}{Race (FP32 Accuracy = 37.51\%)} & \multicolumn{4}{|c|}{Boolq (FP32 Accuracy = 64.62\%)} \\ 
 \hline
 \hline
 64 & 36.94 & 37.13 & 36.27 & 37.13 & 63.73 & 62.26 & 63.49 & 63.36 \\
 \hline
 32 & 37.03 & 36.36 & 36.08 & 37.03 & 62.54 & 63.51 & 63.49 & 63.55  \\
 \hline
 16 & 37.03 & 37.03 & 36.46 & 37.03 & 61.1 & 63.79 & 63.58 & 63.33  \\
 \hline
 \hline
 \multicolumn{5}{|c|}{Winogrande (FP32 Accuracy = 58.01\%)} & \multicolumn{4}{|c|}{Piqa (FP32 Accuracy = 74.21\%)} \\ 
 \hline
 \hline
 64 & 58.17 & 57.22 & 57.85 & 58.33 & 73.01 & 73.07 & 73.07 & 72.80 \\
 \hline
 32 & 59.12 & 58.09 & 57.85 & 58.41 & 73.01 & 73.94 & 72.74 & 73.18  \\
 \hline
 16 & 57.93 & 58.88 & 57.93 & 58.56 & 73.94 & 72.80 & 73.01 & 73.94  \\
 \hline
\end{tabular}
\caption{\label{tab:mmlu_abalation} Accuracy on LM evaluation harness tasks on GPT3-1.3B model.}
\end{table}

\begin{table} \centering
\begin{tabular}{|c||c|c|c|c||c|c|c|c|} 
\hline
 $L_b \rightarrow$& \multicolumn{4}{c||}{8} & \multicolumn{4}{c||}{8}\\
 \hline
 \backslashbox{$L_A$\kern-1em}{\kern-1em$N_c$} & 2 & 4 & 8 & 16 & 2 & 4 & 8 & 16  \\
 %$N_c \rightarrow$ & 2 & 4 & 8 & 16 & 2 & 4 & 2 \\
 \hline
 \hline
 \multicolumn{5}{|c|}{Race (FP32 Accuracy = 41.34\%)} & \multicolumn{4}{|c|}{Boolq (FP32 Accuracy = 68.32\%)} \\ 
 \hline
 \hline
 64 & 40.48 & 40.10 & 39.43 & 39.90 & 69.20 & 68.41 & 69.45 & 68.56 \\
 \hline
 32 & 39.52 & 39.52 & 40.77 & 39.62 & 68.32 & 67.43 & 68.17 & 69.30  \\
 \hline
 16 & 39.81 & 39.71 & 39.90 & 40.38 & 68.10 & 66.33 & 69.51 & 69.42  \\
 \hline
 \hline
 \multicolumn{5}{|c|}{Winogrande (FP32 Accuracy = 67.88\%)} & \multicolumn{4}{|c|}{Piqa (FP32 Accuracy = 78.78\%)} \\ 
 \hline
 \hline
 64 & 66.85 & 66.61 & 67.72 & 67.88 & 77.31 & 77.42 & 77.75 & 77.64 \\
 \hline
 32 & 67.25 & 67.72 & 67.72 & 67.00 & 77.31 & 77.04 & 77.80 & 77.37  \\
 \hline
 16 & 68.11 & 68.90 & 67.88 & 67.48 & 77.37 & 78.13 & 78.13 & 77.69  \\
 \hline
\end{tabular}
\caption{\label{tab:mmlu_abalation} Accuracy on LM evaluation harness tasks on GPT3-8B model.}
\end{table}

\begin{table} \centering
\begin{tabular}{|c||c|c|c|c||c|c|c|c|} 
\hline
 $L_b \rightarrow$& \multicolumn{4}{c||}{8} & \multicolumn{4}{c||}{8}\\
 \hline
 \backslashbox{$L_A$\kern-1em}{\kern-1em$N_c$} & 2 & 4 & 8 & 16 & 2 & 4 & 8 & 16  \\
 %$N_c \rightarrow$ & 2 & 4 & 8 & 16 & 2 & 4 & 2 \\
 \hline
 \hline
 \multicolumn{5}{|c|}{Race (FP32 Accuracy = 40.67\%)} & \multicolumn{4}{|c|}{Boolq (FP32 Accuracy = 76.54\%)} \\ 
 \hline
 \hline
 64 & 40.48 & 40.10 & 39.43 & 39.90 & 75.41 & 75.11 & 77.09 & 75.66 \\
 \hline
 32 & 39.52 & 39.52 & 40.77 & 39.62 & 76.02 & 76.02 & 75.96 & 75.35  \\
 \hline
 16 & 39.81 & 39.71 & 39.90 & 40.38 & 75.05 & 73.82 & 75.72 & 76.09  \\
 \hline
 \hline
 \multicolumn{5}{|c|}{Winogrande (FP32 Accuracy = 70.64\%)} & \multicolumn{4}{|c|}{Piqa (FP32 Accuracy = 79.16\%)} \\ 
 \hline
 \hline
 64 & 69.14 & 70.17 & 70.17 & 70.56 & 78.24 & 79.00 & 78.62 & 78.73 \\
 \hline
 32 & 70.96 & 69.69 & 71.27 & 69.30 & 78.56 & 79.49 & 79.16 & 78.89  \\
 \hline
 16 & 71.03 & 69.53 & 69.69 & 70.40 & 78.13 & 79.16 & 79.00 & 79.00  \\
 \hline
\end{tabular}
\caption{\label{tab:mmlu_abalation} Accuracy on LM evaluation harness tasks on GPT3-22B model.}
\end{table}

\begin{table} \centering
\begin{tabular}{|c||c|c|c|c||c|c|c|c|} 
\hline
 $L_b \rightarrow$& \multicolumn{4}{c||}{8} & \multicolumn{4}{c||}{8}\\
 \hline
 \backslashbox{$L_A$\kern-1em}{\kern-1em$N_c$} & 2 & 4 & 8 & 16 & 2 & 4 & 8 & 16  \\
 %$N_c \rightarrow$ & 2 & 4 & 8 & 16 & 2 & 4 & 2 \\
 \hline
 \hline
 \multicolumn{5}{|c|}{Race (FP32 Accuracy = 44.4\%)} & \multicolumn{4}{|c|}{Boolq (FP32 Accuracy = 79.29\%)} \\ 
 \hline
 \hline
 64 & 42.49 & 42.51 & 42.58 & 43.45 & 77.58 & 77.37 & 77.43 & 78.1 \\
 \hline
 32 & 43.35 & 42.49 & 43.64 & 43.73 & 77.86 & 75.32 & 77.28 & 77.86  \\
 \hline
 16 & 44.21 & 44.21 & 43.64 & 42.97 & 78.65 & 77 & 76.94 & 77.98  \\
 \hline
 \hline
 \multicolumn{5}{|c|}{Winogrande (FP32 Accuracy = 69.38\%)} & \multicolumn{4}{|c|}{Piqa (FP32 Accuracy = 78.07\%)} \\ 
 \hline
 \hline
 64 & 68.9 & 68.43 & 69.77 & 68.19 & 77.09 & 76.82 & 77.09 & 77.86 \\
 \hline
 32 & 69.38 & 68.51 & 68.82 & 68.90 & 78.07 & 76.71 & 78.07 & 77.86  \\
 \hline
 16 & 69.53 & 67.09 & 69.38 & 68.90 & 77.37 & 77.8 & 77.91 & 77.69  \\
 \hline
\end{tabular}
\caption{\label{tab:mmlu_abalation} Accuracy on LM evaluation harness tasks on Llama2-7B model.}
\end{table}

\begin{table} \centering
\begin{tabular}{|c||c|c|c|c||c|c|c|c|} 
\hline
 $L_b \rightarrow$& \multicolumn{4}{c||}{8} & \multicolumn{4}{c||}{8}\\
 \hline
 \backslashbox{$L_A$\kern-1em}{\kern-1em$N_c$} & 2 & 4 & 8 & 16 & 2 & 4 & 8 & 16  \\
 %$N_c \rightarrow$ & 2 & 4 & 8 & 16 & 2 & 4 & 2 \\
 \hline
 \hline
 \multicolumn{5}{|c|}{Race (FP32 Accuracy = 48.8\%)} & \multicolumn{4}{|c|}{Boolq (FP32 Accuracy = 85.23\%)} \\ 
 \hline
 \hline
 64 & 49.00 & 49.00 & 49.28 & 48.71 & 82.82 & 84.28 & 84.03 & 84.25 \\
 \hline
 32 & 49.57 & 48.52 & 48.33 & 49.28 & 83.85 & 84.46 & 84.31 & 84.93  \\
 \hline
 16 & 49.85 & 49.09 & 49.28 & 48.99 & 85.11 & 84.46 & 84.61 & 83.94  \\
 \hline
 \hline
 \multicolumn{5}{|c|}{Winogrande (FP32 Accuracy = 79.95\%)} & \multicolumn{4}{|c|}{Piqa (FP32 Accuracy = 81.56\%)} \\ 
 \hline
 \hline
 64 & 78.77 & 78.45 & 78.37 & 79.16 & 81.45 & 80.69 & 81.45 & 81.5 \\
 \hline
 32 & 78.45 & 79.01 & 78.69 & 80.66 & 81.56 & 80.58 & 81.18 & 81.34  \\
 \hline
 16 & 79.95 & 79.56 & 79.79 & 79.72 & 81.28 & 81.66 & 81.28 & 80.96  \\
 \hline
\end{tabular}
\caption{\label{tab:mmlu_abalation} Accuracy on LM evaluation harness tasks on Llama2-70B model.}
\end{table}

%\section{MSE Studies}
%\textcolor{red}{TODO}


\subsection{Number Formats and Quantization Method}
\label{subsec:numFormats_quantMethod}
\subsubsection{Integer Format}
An $n$-bit signed integer (INT) is typically represented with a 2s-complement format \citep{yao2022zeroquant,xiao2023smoothquant,dai2021vsq}, where the most significant bit denotes the sign.

\subsubsection{Floating Point Format}
An $n$-bit signed floating point (FP) number $x$ comprises of a 1-bit sign ($x_{\mathrm{sign}}$), $B_m$-bit mantissa ($x_{\mathrm{mant}}$) and $B_e$-bit exponent ($x_{\mathrm{exp}}$) such that $B_m+B_e=n-1$. The associated constant exponent bias ($E_{\mathrm{bias}}$) is computed as $(2^{{B_e}-1}-1)$. We denote this format as $E_{B_e}M_{B_m}$.  

\subsubsection{Quantization Scheme}
\label{subsec:quant_method}
A quantization scheme dictates how a given unquantized tensor is converted to its quantized representation. We consider FP formats for the purpose of illustration. Given an unquantized tensor $\bm{X}$ and an FP format $E_{B_e}M_{B_m}$, we first, we compute the quantization scale factor $s_X$ that maps the maximum absolute value of $\bm{X}$ to the maximum quantization level of the $E_{B_e}M_{B_m}$ format as follows:
\begin{align}
\label{eq:sf}
    s_X = \frac{\mathrm{max}(|\bm{X}|)}{\mathrm{max}(E_{B_e}M_{B_m})}
\end{align}
In the above equation, $|\cdot|$ denotes the absolute value function.

Next, we scale $\bm{X}$ by $s_X$ and quantize it to $\hat{\bm{X}}$ by rounding it to the nearest quantization level of $E_{B_e}M_{B_m}$ as:

\begin{align}
\label{eq:tensor_quant}
    \hat{\bm{X}} = \text{round-to-nearest}\left(\frac{\bm{X}}{s_X}, E_{B_e}M_{B_m}\right)
\end{align}

We perform dynamic max-scaled quantization \citep{wu2020integer}, where the scale factor $s$ for activations is dynamically computed during runtime.

\subsection{Vector Scaled Quantization}
\begin{wrapfigure}{r}{0.35\linewidth}
  \centering
  \includegraphics[width=\linewidth]{sections/figures/vsquant.jpg}
  \caption{\small Vectorwise decomposition for per-vector scaled quantization (VSQ \citep{dai2021vsq}).}
  \label{fig:vsquant}
\end{wrapfigure}
During VSQ \citep{dai2021vsq}, the operand tensors are decomposed into 1D vectors in a hardware friendly manner as shown in Figure \ref{fig:vsquant}. Since the decomposed tensors are used as operands in matrix multiplications during inference, it is beneficial to perform this decomposition along the reduction dimension of the multiplication. The vectorwise quantization is performed similar to tensorwise quantization described in Equations \ref{eq:sf} and \ref{eq:tensor_quant}, where a scale factor $s_v$ is required for each vector $\bm{v}$ that maps the maximum absolute value of that vector to the maximum quantization level. While smaller vector lengths can lead to larger accuracy gains, the associated memory and computational overheads due to the per-vector scale factors increases. To alleviate these overheads, VSQ \citep{dai2021vsq} proposed a second level quantization of the per-vector scale factors to unsigned integers, while MX \citep{rouhani2023shared} quantizes them to integer powers of 2 (denoted as $2^{INT}$).

\subsubsection{MX Format}
The MX format proposed in \citep{rouhani2023microscaling} introduces the concept of sub-block shifting. For every two scalar elements of $b$-bits each, there is a shared exponent bit. The value of this exponent bit is determined through an empirical analysis that targets minimizing quantization MSE. We note that the FP format $E_{1}M_{b}$ is strictly better than MX from an accuracy perspective since it allocates a dedicated exponent bit to each scalar as opposed to sharing it across two scalars. Therefore, we conservatively bound the accuracy of a $b+2$-bit signed MX format with that of a $E_{1}M_{b}$ format in our comparisons. For instance, we use E1M2 format as a proxy for MX4.

\begin{figure}
    \centering
    \includegraphics[width=1\linewidth]{sections//figures/BlockFormats.pdf}
    \caption{\small Comparing LO-BCQ to MX format.}
    \label{fig:block_formats}
\end{figure}

Figure \ref{fig:block_formats} compares our $4$-bit LO-BCQ block format to MX \citep{rouhani2023microscaling}. As shown, both LO-BCQ and MX decompose a given operand tensor into block arrays and each block array into blocks. Similar to MX, we find that per-block quantization ($L_b < L_A$) leads to better accuracy due to increased flexibility. While MX achieves this through per-block $1$-bit micro-scales, we associate a dedicated codebook to each block through a per-block codebook selector. Further, MX quantizes the per-block array scale-factor to E8M0 format without per-tensor scaling. In contrast during LO-BCQ, we find that per-tensor scaling combined with quantization of per-block array scale-factor to E4M3 format results in superior inference accuracy across models. 

	
	%% The Appendices part is started with the command \appendix;
	%% appendix sections are then done as normal sections
	%% \appendix
	
	%% \section{}
	%% \label{}
	
	%% If you have bibdatabase file and want bibtex to generate the
	%% bibitems, please use
	%%
	
	
	\bibliographystyle{plain}
	\bibliography{MS_Coupling_202502.bib}
	
\end{document}