\section{A Priori Analysis for the QNLL Method}
\label{sec: anal_qnll_ncg}

% \chw{The total consistency error is defined by $\langle \delta \El (y^a), v \rangle$ and then plus and minus proper terms to separate the total consistency error into several parts. The following is not precise. We can only bound $\|\ya - \ynll \| \le$ truncation error $+$ coupling error $+$ linearization error}

In this section, we will provide the {\it a priori} error estimate for the atomistic to nonlinear-linear elasticity coupling model $\Vert \nabla \yai - \nabla \ynll \Vert_{L^{2}}$, following analytical framework shown in~\cite{2013_ML_CO_AC_Coupling_ACTANUM,2011_CO_1D_QNL_MATHCOMP,2011_CO_HW_QC_A_Priori_1D_M3AS}. We will first present the consistency error estimate and stability analysis results of the QNLL model, and then give the {\it a priori} error estimate based on the inverse function theorem (Lemma \ref{Inverse function theorem}) provided in the Appendix. The detailed consistency error analysis is provided in Section \ref{sec: consistency_qnll_ncg}, stability analysis in Section \ref{sec: stability_qnll_ncg}, and the final {\it a priori} error estimate in Section \ref{sec: priori_qnll_ncg}. Finally, in Section~\ref{sec: balance_of_qnll_ncg_model}, based on the {\it a priori} error estimate and the $ \DH $ assumption, we propose a method to balance the lengths of the regions in the QNLL model to achieve the same convergence order as the QNL model.


% In Section \ref{sec: consistency_qnll_ncg}, we will prove the a priori error estimate for the QNLL solution \eqref{Nonlinear-linear solution condition} based on consistency error and stability analysis. The consistency error includes both the coupling error and the linearization error. In the second part, we will perform the stability analysis. In the third part, we will give the a priori error estimate based on the inverse function theorem, which can be found in Lemma \ref{Inverse function theorem} in the Appendix. 






%\begin{lemma}\label{Inverse function theorem}
%    Let $\Yn$ be a subspace of $\Ya$, equipped with $\Vert \nabla \cdot \Vert_{L^{2}}$, and let $\Ghc \in C^{1}(\Yn,\Yn^{*})$ with Lipschitz-continuous derivative $\delta \Ghc$:
%    \begin{equation*}
	%        \Vert\delta\Ghc(y) - \delta\Ghc(v)\Vert_\mathcal{L} \le M \Vert \nabla y - \nabla v \Vert_{L^{2}}, \quad \text{for all} \  v \in \Un,
	%    \end{equation*}
%    where $\Vert \cdot \Vert_{\mathcal{L}}$ denotes the $\mathcal{L}(\Yn,\Yn^{*})$-operator norm.
%
%    Let $\bar{y}\in \Yn$ satisfy
%%    \begin{equation}
	%        \begin{align}
		%            \Vert \Ghc(\bar{y})\Vert_{\Yn^{*}} &\le \eta,\\
		%            \langle \delta \Ghc (\bar{y})v,v\rangle &\ge \gamma \Vert \nabla v \Vert^{2}_{L^{2}},  \quad \text{for all} \ v \in \UhNL,
		%        \end{align}
	%%    \end{equation}
%    such that $L,\eta,\gamma$ satisfy the relation
%    \begin{equation}
	%        \frac{2M\eta}{\gamma^{2}}<1.
	%    \end{equation}
%
%    Then there exists a (locally unique) $\ynll\in \Yn$ such that $\Ghc(\ynll)=0$,
%%    \begin{eqution}
	%        \begin{align}
		%            \Vert \nabla \ynll-\nabla \bar{y}\Vert_{L^{2}} &\le 2\frac{\eta}{\gamma}, \quad \text{and}\\
		%            \langle \delta \Ghc (\ynll)v,v\rangle &\ge (1-\frac{2M\eta}{\gamma^{2}}) \Vert \nabla v \Vert^{2}_{L^{2}},  \quad \text{for all} \ v \in \Un,
		%        \end{align}
	%%    \end{eqution}
%\end{lemma}
%
%\begin{proof}
%The result is a simplified and specialized version of Lemma 2.2 of
%\cite{2011_CO_1D_QNL_MATHCOMP}, but similar statements can be obtained from most proofs of the inverse function theorem.
%\end{proof}

\subsection{Consistency error}
\label{sec: consistency_qnll_ncg}

In the consistency error estimate, we decompose the total error using the triangle inequality into two components: (i) the coupling error, which quantifies the difference between the atomistic model and the QNL model, and (ii) the linearization error, which arises from the transition from the QNL model to the QNLL model. By estimating these two errors separately, we derive the overall consistency error between the atomistic model and the QNLL model. Specifically, we have
\begin{align*}
	T(\ya) &= \langle \delta \Ea (\ya),v\rangle -\langle \delta \El (\ya),v\rangle\\
	&=\langle \delta \Ea (\ya),v\rangle -\langle \delta \Erfl (\ya),v\rangle ~~\qquad \text{(the coupling error)}\\
	&\ \ +\langle \delta \Erfl (\ya),v\rangle -\langle \delta \El (\ya),v\rangle. \quad \text{(the linearization error)}
\end{align*}
In the following, we establish an error bound for $T(y^{\rm a})$.
% where $\Vert T \Vert_{\Yn^{*}}$ represents the consistency error estimate in this section.

\subsubsection{QNL coupling error}

We define the weighted characteristic function of a bond $(\xi ,\xi +\rho)$ by
\begin{equation}\label{Definition of Ki}
	\Ki:=\left\{
	\begin{aligned}
		&\vert\rho\vert^{-1}, &x\in \text{int} (\conv \{\xi,\xi+\rho\}), \\
		&\frac{1}{2}\vert\rho\vert^{-1}, &x\in \{\xi,\xi+\rho\}, \\
		&0, &\text{otherwise}.
	\end{aligned}
	\right.
\end{equation}
We then obtain for $D_{\rho}v(\xi) = v(\xi+\rho) -v(\xi)$ that
\begin{equation}\label{Diff to int}
	D_{\rho}v(\xi)=\int_{\xi}^{\xi+\rho}\frac{\rho}{\vert\rho\vert}\nabla v\,\d x
	=\int_{\R}\rho \Ki\nabla v\,\d x.
\end{equation}

The first variation of the atomistic energy functional \eqref{All-Atomistic Energy} at $y \in \Yn$ is given by
\begin{equation*}
	\langle \delta \Ea(y), v\rangle = \sum_{\xi = -N}^{N} \sum_{\rho \in \Rc}\Phia_{\xi,\rho}(y)D_{\rho}v(\xi).
\end{equation*}
We apply \eqref{Diff to int}, it follows that
\begin{equation}\label{Atomistic first variation}
	\langle \delta \Ea(y), v\rangle = \int_{-N}^{N} \Sa \nabla v\,\d x.
\end{equation}
where
\begin{equation}\label{Atomistic stress tensor}
	\Sa = \sum_{\xi=-N}^{N} \sum_{\rho \in \Rc} \rho \Ki \Phia_{\xi,\rho}(y).
\end{equation}



% \chw{We estimate the error committed by replacing the atomistic model by the QNL method in this part. We need to emphasize that compared with existing work, we do not prescribe any specific finite element discretization here.}

% We shall require two additional restrictions on the interface site energies, as discussed by \cite{2013_ML_CO_AC_Coupling_ACTANUM}, which we call the locality and scaling conditions.
% $\L$ Locality. $\Phii_{\xi}(y)$ is a $3$ times continuous differentiable function of the interaction stencil $Dy(\xi) = \big(D_{\rho}y(\xi)\big)_{\rho \in \Rc}$. Moreover, $\Phii_{\xi,\rho}(y) = 0$ for $\xi + \rho >L$.

% $\S$ Scaling. $\vert \Phii_{\xi,\bm{\rho}}(y)\vert \lesssim c(\bm{\rho})$, for $\bm{\rho} \in \Rc, j=2,3$, where
% \begin{equation*}
	% 	\sum_{\bm{\rho}\in \Rc^{j}}\vert \bm{\rho} \vert^{s}_{\infty}\prod_{i=1}^{j}\vert \rho_{i} \vert c(\bm{\rho}) \lesssim M^{(j,s)}, \quad \text{for } 0\le s \le 3.
	% \end{equation*}
The first variation of the energy functional of QNL approximation defined by \eqref{QNL energy}, for any $v\in \YacNL$, we have 
\begin{equation}\label{QNL first variation}
	\begin{split}
		\langle \delta \Erfl (y), v\rangle &= \sum_{\xi \in \Ac} \sum_{\rho \in \Rc} \Phia_{\xi,\rho}(y)D_{\rho}v(\xi) + \int_{\OmeC}\partial_{F} W(\nabla y)\nabla v\,\d x \\
		&=: \int_{-N}^{N} \Srfl \nabla v \,\d x.
	\end{split}
\end{equation}
where
\begin{equation}\label{QNL stress tensor}
	\Srfl := \left\{
	\begin{aligned}
		&\sum_{\xi \in \Ac}\sum_{\rho \in \Rc}\rho \Ki \Phia_{\xi,\rho}(y) +\sum_{\xi \in \Ic}\sum_{\rho \in \Rc}\rho \Ki \Phii_{\xi,\rho}(y), &x\in \OmeA \cup \OmeI, \\
		&\partial_{F}W(\nabla y), &x\in \OmeC.
	\end{aligned}
	\right.
\end{equation}


% \begin{lemma}\label{ki sum 1}
	%  Let $\xi \in \Z, \rho \in \Rc$, and $\Ki$ is defined by \eqref{Definition of Ki}, then we have
	% \begin{equation*}
		% 	\sum_{\xi \in \Z}\Ki=1.
		% \end{equation*}	
	% \end{lemma}

We now define the error in QNL model stress as
\begin{equation*}
	\Rrfl := \Srfl -\Sa.
\end{equation*}
Next, we will provide a point estimate of the stress tensor for the QNL model. For the sake of presentation simplicity, we leave the detailed proof, please refer to Appendix \ref{Appendix section 1}.

\begin{proposition}\label{Pointwise coupling stress tensor}
	Let $y\in\Ya, x \in [-N,N]$, then 
	\begin{equation*}
		\vert \Rrfl \vert \lesssim \left\{
		\begin{aligned}
			&0, &x \in \OmeA \backslash \bOmeA,\\
			&M^{(2,1)}\Vert \nabla^{2}u\Vert_{L^{\infty}(v_{x})}, &x\in \OmeI \cup \bOmeA,\\
			&M^{(2,2)}\Vert \nabla^{3}u\Vert_{L^{\infty}(v_{x})}+M^{(3,2)}\Vert \nabla^{2}u\Vert_{L^{\infty}(v_{x})}^{2}, &x \in \OmeC
		\end{aligned}
		\right.
	\end{equation*}
	where $v_{x}:=\big[\lfloor x\rfloor+1-2\rcut,\lfloor x\rfloor+2\rcut\big]$ is the neighborhood of some $x\in \R$, the meaning of $\lfloor x\rfloor$ here is to take the floor of $x$, which is the greatest integer less than or equal to $x$, and $\bOmeA = [-K,-K+\rcut]\cup [K-\rcut,K]$.
\end{proposition}

% \begin{proof}
	% 	The proof of the coupling error estimate for the QNL model is derived from [\cite{2013_ML_CO_AC_Coupling_ACTANUM}, Lemma 6.12]. Since $\text{supp}(\Ki) = \conv\{\xi,\xi+\rho\}$, $\Rrfl=0$ for $x\in\OmeA \backslash \bOmeA$. As for $x\in \OmeI \cup \bOmeA$,we obtain
	% 	\begin{align*}
		% 		\Rrfl &=\Srfl-\Sa\\
		% 		&=\sum_{\xi \in \Ic} \sum_{\rho \in \Rc}\rho\Ki\Phii_{\xi,\rho}(y)-\sum_{\xi \in\Ic\cup\Cc}\sum_{\rho \in \Rc} \rho \Ki \Phia_{\xi,\rho}(y).
		% 	\end{align*}
	% We define uniform deformation $\yF(\xi)=F\xi$. We notice that $R^{\text{rfl}}(\yF;x)=0$, hence we obtain
	% \begin{equation}\label{Two parts of Rrfl}
		% \begin{aligned}
			% 	\Rrfl &= \Rrfl-R^{\text{rfl}}(\yF;x)\\
			% 	&= \sum_{\xi \in \Ic} \sum_{\rho \in \Rc}\rho\Ki(\Phii_{\xi,\rho}(y)-\Phii_{\xi,\rho}(\yF))\\
			% 	&\ +\sum_{\xi \in\Ic\cup\Cc}\sum_{\rho \in \Rc} \rho \Ki (\Phia_{\xi,\rho}(\yF)-\Phia_{\xi,\rho}(y)).
			% \end{aligned}
		% \end{equation}
	
	% For the first term of \eqref{Two parts of Rrfl}, we choose $F=\nabla y$ and use Taylor's expansion. After considering $\S$, we have
	% \begin{equation}\label{Interface stess}
		% 	\vert \Phii_{\xi,\rho}(y)-\Phii_{\xi,\rho}(\yF) \vert \le \sum_{\zeta \in \Rc} c(\rho,\zeta)\vert D_{\zeta}y(\xi)-\nabla_{\zeta}y(x)\vert.
		% \end{equation}
	
	% After using Taylor's expansion, for $x\in \conv\{\xi,\xi+\rho\}$, we obtain
	% \begin{align*}
		% 	\vert D_{\zeta}y(\xi)-\nabla_{\zeta}y(x)\vert&= \vert \zeta(\xi-x+\frac{\zeta}{2})\vert \cdot \Vert \nabla^{2}y\Vert_{L^{\infty}(\conv\{\xi,\xi+\rho\})}\\
		% 	&\le\frac{\vert \zeta \vert^{2}}{2}\Vert \nabla^{2}u\Vert_{L^{\infty}(\conv\{\xi,\xi+\rho\})}.
		% \end{align*}
	
	% Applying assumption $\S$ again, we get
	% \begin{align*}
		% 	\sum_{\xi \in \Ic} \sum_{\rho \in \Rc} \vert\rho \vert\Ki \vert \Phii_{\xi,\rho}(y)-\Phii_{\xi,\rho}(\yF) \vert &\le \sum_{(\rho,\zeta)\in\Rc^{2}} \frac{1}{2}\vert \rho \zeta^{2}\vert c(\rho,\zeta) \Vert \nabla^{2}y\Vert_{L^{\infty}(v_{x})}\sum_{\xi \in \Ic}\Ki\\
		% 	&\lesssim M^{(2,1)}\Vert \nabla^{2}y\Vert_{L^{\infty}(v_{x})}.
		% \end{align*}
	
	% Similarly, we could calculate that
	% \begin{align*}
		% 	\sum_{\xi \in\Ic\cup\Cc} \sum_{\rho \in \Rc} \vert\rho \vert\Ki \vert \Phia_{\xi,\rho}(y)-\Phia_{\xi,\rho}(\yF) \vert &\le \sum_{\rho \in \Rc} \frac{1}{2} \vert \rho \zeta^{2}\vert m(\rho,\zeta) \\
		% 	&\le M^{(2,1)} \Vert \nabla^{2}y\Vert_{L^{\infty}(v_{x})}.
		% \end{align*}
	
	% For $x\in \OmeC \cap \Z+\frac{1}{2}$, we have
	% \begin{equation*}
		% 	\Rrfl = \partial_{F}W(\nabla y(x))-\Sa,
		% \end{equation*}
	% which is the difference between the atomistic stress tensor and Cauchy-Born stress tensor. And we mention that the error estimate follows directly from[\cite{2013_ML_CO_AC_Coupling_ACTANUM}, Theorem6.2]
	% \begin{align*}
		% 	\Rrfl &= \partial_{F}W(\nabla y(x))-\Sa\\
		% 	&\lesssim M^{(2,2)}\Vert \nabla^{3}y\Vert_{L^{\infty}(v_{x})}+M^{(3,2)}\Vert \nabla^{3}y\Vert_{L^{\infty}(v_{x})}^{2}.
		% \end{align*}
	
	% By definition of $\Ua$ and $\Ya$, we notice $\nabla^{2}y =\nabla^{2}u,\ \nabla^{3}y=\nabla^{3}u$. So for simplify, we choose norm of $u$ in final result.
	% \end{proof}

Based on the point estimate of $\Rrfl$ provided above, we will give the consistency error estimate for the QNL model. Please refer to Appendix \ref{Appendix section 2} for the proof.

\begin{proposition}\label{Coupling consistency error estimate}
	We split coupling error into two parts:
	\begin{equation*}
		\int_{-N}^{N}\Rrfl\nabla v\,\d x=\int_{\OmeI\cup\bOmeA}\Rrfl\nabla v\,\d x-\int_{\OmeC}\Rrfl\nabla v\,\d x.
	\end{equation*}
	
	For any $v\in \Yn$, we have
	\begin{equation}\label{Interface region stress tensor}
		\int_{\OmeI\cup\bOmeA}\Rrfl\nabla v\,\d x \lesssim M^{(2,1)}\Vert\nabla^{2}u\Vert_{L^{2}(\bOmeI)}\Vert\nabla v\Vert_{L^{2}(\OmeI\cup\bOmeA)},
	\end{equation}
	and
	\begin{equation}\label{Continuum region stress tensor}
		\int_{\OmeC}\Rrfl\nabla v\,\d x \lesssim(M^{(2,2)}\Vert \nabla^{3}u\Vert_{L^{2}(\bOmeC)}+M^{(3,2)}\Vert \nabla^{2}u\Vert_{L^{4}(\bOmeC)}^{2})\Vert\nabla v\Vert_{L^{2}(\OmeC)},
	\end{equation}
	where $\bOmeI = [-\bar{K}+1-2\rcut,-\bar{K}+4+2\rcut]\cup[\bar{K}-3-2\rcut,\bar{K}+2\rcut], \bOmeC=[-N-2\rcut,-\bar{K}-1+2\rcut]\cup[\bar{K}+1-2\rcut,N+2\rcut]$.
\end{proposition}

% \begin{proof}
	
	% \end{proof}

% \begin{proof}
	% 	For \eqref{Interface region stress tensor}, the main point of this proof is to use the inverse estimates to obtain $L^{2}-$type from the $L^{\infty}$ bounds\cite{2007_Braess_FEM}
	% 	\begin{equation}\label{L-infty to L-2 estimate}
		% 		\Vert \nabla^{2}u\Vert_{L^{\infty}(v_{x})}\lesssim \Vert \nabla^{2}u\Vert_{L^{2}(v_{x})}.
		% 	\end{equation}
	
	% After a direct calculation, we have
	% \begin{align*}
		% 	\int_{\OmeI\cup\bOmeA}\Rrfl\nabla v\d x&\lesssim \int_{\OmeI\cup\bOmeA} M^{(2,1)}\Vert \nabla^{2}u\Vert_{L^{\infty}(v_{x})} \vert \nabla y \vert \d x\\
		% 	&\le M^{(2,1)}\Vert \nabla^{2}u\Vert_{L^{\infty}(\bOmeI)}\Vert \nabla v \Vert_{L^{1}(\OmeI\cup\bOmeA)}\\
		% 	&\lesssim  M^{(2,1)}\Vert \nabla^{2}u\Vert_{L^{\infty}(\bOmeI)}\Vert \nabla v \Vert_{L^{2}(\OmeI\cup\bOmeA)}.
		% \end{align*}
	
	% As for \eqref{Continuum region stress tensor}, combining Theorem \ref{Pointwise coupling stress tensor} and [\cite{2013_ML_CO_AC_Coupling_ACTANUM}, Corollary6.4], we yield the started results.
	% \end{proof}

\subsubsection{Linearization error}

Firstly, we calculate the first variation of the energy functional of Nonlinear-linear elasticity coupling energy defined by \eqref{Nonlinear-linear energy}, for any $v\in\YacNL$, is then given by
\begin{equation*}
	\langle \delta \El (y),v\rangle=:\int_{-N}^{N} \SNLL\nabla v\,\d x,
\end{equation*}
where
\begin{equation}\label{LInearization stress tensor}
	\SNLL := \left\{
	\begin{aligned}
		&\sum_{\xi \in \Ac}\sum_{\rho \in \Rc}\rho \Ki \Phia_{\xi,\rho}(y) +\sum_{\xi \in \Ic}\sum_{\rho \in \Rc}\rho \Ki \Phii_{\xi,\rho}(y) \quad x\in \OmeA \cup \OmeI, \\
		&\partial_{F}W(\nabla y) \qquad \qquad \qquad \qquad \qquad \qquad \qquad \qquad \quad  \quad~~ x\in \OmeNL,\\
		&\partial_{F}W_{\text{L}}(\nabla y) \qquad \qquad \qquad \qquad \qquad \qquad \qquad \qquad \qquad~ x\in \OmeL.
	\end{aligned}
	\right.
\end{equation}

We now define the error in linearization as
\begin{equation*}
	\RNLL=\SNLL-\Srfl.
\end{equation*}
We will provide a point-wise estimate of the stress tensor for the QNLL model using the definition of $\WL$.

\begin{theorem}
	Let $y\in \YacNL,x\in[-N,N]$, we have
	\begin{equation}\label{Pointwise NLL stress tensor}
		\vert \RNLL\vert \le \frac{1}{2}M^{(3,0)} \Vert \nabla u(x)\Vert^{2}_{L^{\infty}(\OmeL)}, \ x\in \OmeL.
	\end{equation}
\end{theorem}

\begin{proof}
	We notice that $\RNLL \neq 0$, only for $x\in \OmeL$. We could directly know
	\begin{equation*}
		\partial_{F}\WL(\nabla y) = \Wpf+\Wppf\nabla u.
	\end{equation*}
	After using the Taylor's expansion, we obtain
	\begin{equation*}
		\vert \partial_{F}W_{\text{L}} (\nabla y)-\partial_{F}W(\nabla y)\vert \le \frac{1}{2} M^{(3,0)} \vert \nabla u(x)\vert^{2}\le\frac{1}{2}M^{(3,0)}\Vert \nabla u(x)\Vert^{2}_{L^{\infty}(\OmeL)}.
	\end{equation*}
\end{proof}

Based on the point estimate of $\RNLL$ provided above, we will now present the consistency error estimate for the QNLL model.

\begin{theorem}\label{Linearization consistency error estimate}
	For any $v \in \YacNL$, we have
	\begin{equation}\label{Linearization error estimate}
		\int_{-N}^{N} \RNLL \nabla v\,\d x\le \frac{1}{2} M^{(3,0)}\Vert \nabla u\Vert^{2}_{L^{4}(\OmeL)}\Vert \nabla v\Vert_{L^{2}(\OmeL)}.
	\end{equation} 
\end{theorem}
\begin{proof}
	After using \eqref{Pointwise NLL stress tensor}, we have
	\begin{equation*}
		\int_{\OmeL} \RNLL  \nabla v \,\d x\le \int_{\OmeL} \frac{1}{2}M^{(3,0)}\Vert \nabla u(x)\Vert^{2}_{L^{\infty}(\OmeL)} \vert \nabla v \vert \,\d x.
	\end{equation*}
	We consider \eqref{L-infty to L-2 estimate} and obtain
	\begin{equation*}
		\int_{-N}^{N} \RNLL \nabla v\,\d x\le \frac{1}{2} M^{(3,0)}\Vert \nabla u\Vert^{2}_{L^{4}(\OmeL)}\Vert \nabla v\Vert_{L^{2}(\OmeL)}.
	\end{equation*}
\end{proof}

Finally, by combining Proposition \ref{Coupling consistency error estimate} and Theorem \ref{Linearization consistency error estimate}, and applying the triangle inequality, we provide the consistency error estimate between the QNLL model and the atomistic model.

\begin{theorem}
	For any $y, v \in \Yn$, the consistency error estimate between the QNLL model and the atomistic model is
	\begin{equation}\label{QNLL consistency error estimate}
		\Vert T \Vert_{\Yn^{*}} \le M^{(2,1)}\Vert \nabla^{2} u\Vert_{L^{2}(\bOmeI)} +M^{(2,2)}\Vert \nabla^{3}u \Vert_{L^{2}(\bOmeC)}+M^{(3,2)}\Vert \nabla^{2}u \Vert^{2}_{L^{4}(\bOmeC)}M^{(3,0)}+\Vert \nabla u \Vert^{2}_{L^{4}(\OmeL)}.
	\end{equation}
\end{theorem}

% \begin{proof}
	%     , the result can be obtained.
	% \end{proof}




\subsection{Stability}
\label{sec: stability_qnll_ncg}

In this section, we establish two key results regarding the stability of the QNLL model: (i) For uniform deformations, the QNLL model exhibits universal stability, similar to the QNL model; (ii) For non-uniform deformations, the QNLL model progressively stabilizes as the atomistic region expands.

Firstly, we calculate the second variation of the energy functional of Nonlinear-linear elasticity coupling energy defined by \eqref{Nonlinear-linear energy}, for any $v\in\YacNL$, is then given by
\begin{equation}\label{Stab of NL-L}
	\begin{split}
		\langle \delta^{2} \El (y)v,v\rangle =& \sum_{\xi \in \Ac} \sum_{(\rho,\zeta)\in\Rc^{2}}  \Phia_{\xi,\rho\zeta}(y)D_{\rho}v(\xi)D_{\zeta}v(\xi)\\
		&\ +\sum_{\xi \in \Ic} \sum_{(\rho,\zeta)\in\Rc^{2}} \Phii_{\xi,\rho\zeta}(y)D_{\rho}v(\xi)D_{\zeta}v(\xi)\\
		&\ +\int_{\OmeNL}\ppGW(\nabla y) (\nabla v)^{2}\,\d x\\
		&\ +\int_{\OmeL}\ppGWL(\nabla y) (\nabla v)^{2}\,\d x.
	\end{split}
\end{equation}

If we focus on the second variation evaluated at the homogeneous deformation $\yF$, and use the fact $\ppGW(\nabla \yF)=\ppGWL(\nabla \yF)=\Wppf$. Hence, we can obtain
\begin{equation*}
	\int_{\OmeNL}\ppGW(\nabla \yF) (\nabla v)^{2}\,\d x +\int_{\OmeL}\ppGWL(\nabla \yF)  (\nabla v)^{2}\,\d x =\int_{\OmeC}\ppGW(\nabla \yF) (\nabla v)^{2}\,\d x.
\end{equation*}

After a direct calculation, we have
\begin{equation}\label{Stab of QNLL equals to QNL}
	\begin{split}
		\langle \delta^{2} \El (\yF)v,v\rangle 
		=&\sum_{\xi \in \Ac} \sum_{(\rho,\zeta)\in\Rc^{2}}  \Phia_{\xi,\rho\zeta}(\yF)D_{\rho}v(\xi)D_{\zeta}v(\xi)\\
		&\ +\sum_{\xi \in \Ic} \sum_{(\rho,\zeta)\in\Rc^{2}} \Phii_{\xi,\rho\zeta}(\yF)D_{\rho}v(\xi)D_{\zeta}v(\xi)\\
		&\ +\int_{\OmeC}\ppGW(\nabla \yF) (\nabla v)^{2}\,\d x \\
		=&~\langle \delta^{2} \Erfl (\yF)v,v\rangle. 
	\end{split}
\end{equation}

We then define the stability constants (for homogeneous deformations) $\gaaF, \garflF, \ganllF$ as 
\begin{align}
	\label{GammaF for a}  \gaaF &=\inf_{v\in \Ya} \frac{\langle \delta^{2} \Ea (\yF)v,v\rangle}{\Vert \nabla v \Vert_{L^{2}}^{2}},\\
	\label{GammaF for rfl} \garflF &=\inf_{v\in \Ya} \frac{\langle \delta^{2} \Erfl (\yF)v,v\rangle}{\Vert \nabla v \Vert_{L^{2}}^{2}},\\
	\label{GammaF for nll}\ganllF &=\inf_{v\in \Ya} \frac{\langle \delta^{2} \El (\yF)v,v\rangle}{\Vert \nabla v \Vert_{L^{2}}^{2}}.
\end{align}

The QNL method was propose as a ``universally stable method" (cf.~\cite[Theoorem 4.3]{2014_CO_AS_LZ_Stabilization_MMS}). Combining this result with \eqref{Stab of QNLL equals to QNL} we obtain
\begin{equation}\label{Stab constants of three methods result}
	\gaaF = \garflF =\ganllF.
\end{equation}

To make \eqref{Stab of NL-L} precise, we will 
{\it split} the test function $v$ into an atomistic and continuum component, using the following lemma [\cite{2013_ML_CO_AC_Coupling_ACTANUM}, Lemma 7.3]. We leave the proof to Appendix \ref{Appendix section 3}.

\begin{lemma}\label{Pointwise blending lemma}
	Let $\beta \in C^{1,1}(-\infty,\infty)$, with $0\le \beta \le 1$. For each $v\in \Ya$, there exists $\va, \vc \in \Ya$ such that
	\begin{align}
		\label{Pointwise va blending estimate}\vert \sqrt{1-\beta(\xi)} D_{\rho}v(\xi)-D_{\rho}\va(\xi) \vert &\le \vert \rho \vert^{\frac{3}{2}}\Vert \nabla \sqrt{1-\beta}\Vert_{L^{\infty}} \Vert \nabla v \Vert_{L^{2}(\conv(\xi,\xi+\rho))},\\ 
		\label{Pointwise vc blending estimate1}\vert \sqrt{\beta(\xi)} D_{\rho}v(\xi)-D_{\rho}\vc(\xi) \vert &\le \vert \rho \vert^{\frac{3}{2}}\Vert \nabla \sqrt{\beta}\Vert_{L^{\infty}} \Vert \nabla v \Vert_{L^{2}(\conv(\xi,\xi+\rho))},\\ 
		\label{va and vc} \vert \nabla\va\vert^{2} +	\vert \nabla\vc\vert^{2} &= \vert \nabla v \vert^{2}.
	\end{align}
	%where $C_{1},\ C_{2}$ may depends on $\rcut$, but $C_{3}$ is a generic constant. In particular, $\nabla \va(x)=0$ for $x\in(-\infty,-L]\cup[L,\infty)$ and $\nabla \vc(x)=0$ for $x \in [-K,K]$.
\end{lemma}

% \begin{proof}
	% 
	% \end{proof}

% \begin{proof}
	% 	Let $\psi(x):=\sqrt{1-\beta(x)}$ and assume, without loss of generality that $\rho>0$. Then,
	% 	\begin{align*}
		% 		\sqrt{1-\beta(\xi)} D_{\rho}v(\xi)&= \psi (\xi) \sum_{\eta = \xi}^{\xi+\rho-1}D_{1}V(\eta)\\
		% 		&=\sum_{\eta = \xi}^{\xi+\rho-1}\psi(\eta)D_{1}V(\eta) + \sum_{\eta = \xi}^{\xi+\rho-1} (\psi(\xi)-\psi(\eta))D_{1}V(\eta).\\
		% 	\end{align*}
	% If we define $\va$ by $D_{1}\va(\eta)=\psi(\eta)D_{1}v(\eta)$, then we obtain
	% \begin{equation*}
		% 	\sum_{\eta = \xi}^{\xi+\rho-1} \psi(\eta) D_{1}v(\eta) =D_{\rho} \va (\xi),
		% \end{equation*}
	% and after using Holder's inequality we know
	% \begin{align*}
		% 	\vert\sqrt{1-\beta} D_{\rho}v(\xi)-D_{\rho}\va(\xi)\vert&=\sum_{\eta = \xi}^{\xi+\rho-1}(\psi(\xi)-\psi(\eta))D_{1}v(\eta)\\
		% 	&\le (\sum_{\eta = \xi}^{\xi+\rho-1} \Vert \nabla \psi \Vert^{2}_{L^{\infty}} \vert \rho \vert^{2})^{\frac{1}{2}} (\sum_{\eta = \xi}^{\xi+\rho-1}(D_{1}v(\eta))^{2})^{\frac{1}{2}}\\
		% 	&\le \vert \rho \vert^{\frac{3}{2}} \Vert \nabla \psi \Vert_{L^{\infty}} \Vert \nabla v \Vert_{L^{2}(\xi,\xi+\rho)}.
		% \end{align*}
	% This establishes \eqref{Pointwise va blending estimate}. The proof of \eqref{Pointwise vc blending estimate1} is analogous, with $\vc$ defined by $D_{1}\vc(\xi) = \sqrt{\beta(\xi)}D_{1}v(\xi)$.
	
	% With these definitions \eqref{va and vc} is an immediate consequence.
	% \end{proof}

According to Lemma \ref{Pointwise blending lemma}, we can obtain the conclusion about the stability of the QNLL model in the case of non-uniform deformations.

\begin{theorem}\label{Stability}
	Let $y\in\Ya$ satisfy the strong stability condition \eqref{All-Atomistic strong local minimizer} and suppose that there exists $\ganllF >0$ such that
	\begin{equation}\label{Uniform solution of NL-L stab assumption}
		\langle \delta^{2}\El (\yF)v,v\rangle \ge \ganllF \Vert \nabla v\Vert^{2}_{L^{2}(-N,N)} \text{ for all } v \in \Ua.
	\end{equation}
	
	Then
	\begin{equation}\label{Result of stability}
		\begin{split}
			\langle \delta^{2}\El (y)v,v\rangle &\ge \min(c_{0},\ganllF)\Vert \nabla v \Vert_{L^{2}(-N,N)}^{2}\\
			&\ - 2M^{(2,\frac{1}{2})}K^{-1} \Vert \nabla v \Vert_{L^{2}(-N,N)}^{2} - \CDH M^{(3,0)}K^{-\alpha} \Vert \nabla v \Vert_{L^{2}(-L,L)}^{2}, \quad \text{for all } v \in \Ua.
		\end{split}
	\end{equation}
\end{theorem}

\begin{proof}
	According to Lemma \ref{Pointwise blending lemma}, let $K^{'}:=\lfloor K/2\rfloor <K$, and let
	\begin{equation*}
		\beta (x):=\left\{
		\begin{aligned}
			&0, &-K^{'}\le x \le K^{'},\\
			&\hat{\beta}(\frac{x+K^{'}}{K^{'}-K}), &-K\le x \le -K^{'},\\
			&\hat{\beta} (\frac{x-K^{'}}{K-K^{'}}), &K^{'} \le x \le K,\\
			&1, &-N\le x \le -K \text{ or } K\le x\le N.
		\end{aligned}
		\right.
	\end{equation*}
	where $\hat{\beta}(s)=3s^{3}-2s^{2}$. We know from \cite[Section 8.3]{2013_ML_CO_AC_Coupling_ACTANUM} that 
	\begin{equation}\label{Property of beta}
		\Vert \nabla \sqrt{\beta} \Vert_{L^{\infty}} + \Vert \nabla \sqrt{1-\beta} \Vert_{L^{\infty}} \leq 	C_{\beta} K^{-1}.
	\end{equation}
	
	We can now write
	\begin{subequations}
		\begin{align}
			\langle \delta^{2} \El (y)v,v\rangle &= \sum_{\xi=-N}^{N} \sum_{(\rho,\zeta)\in\Rc^{2}} \Phia_{\xi,\rho\zeta} (y)\big(1-\beta(\xi)\big) D_{\rho}v(\xi) D_{\zeta}v(\xi) \nonumber\\
			&\quad +\sum_{\xi=-N}^{N} \sum_{(\rho,\zeta)\in\Rc^{2}} \PhiNLL_{\xi,\rho\zeta} (y)\big(\beta(\xi)\big) D_{\rho}v(\xi) D_{\zeta}v(\xi) \nonumber\\
			&= 
			\label{All atomistic stab}
			\sum_{\xi=-N}^{N} \sum_{(\rho,\zeta)\in\Rc^{2}} \Phia_{\xi,\rho\zeta} (y)\big(1-\beta(\xi)\big) D_{\rho}v(\xi) D_{\zeta}v(\xi)\\
			\label{NL-L atomistic stab} 
			&\quad +\sum_{\xi \in \Ac} \sum_{(\rho,\zeta)\in\Rc^{2}} \Phia_{\xi,\rho\zeta} (y)\big(\beta(\xi)\big)D_{\rho}v(\xi)D_{\zeta}v(\xi)\\
			\label{NL-L interaction stab} 
			&\quad +\sum_{\xi \in \Ic} \sum_{(\rho,\zeta)\in\Rc^{2}} \Phii_{\xi,\rho\zeta} (y)\big(\beta(\xi)\big)D_{\rho}v(\xi)D_{\zeta}v(\xi)\\
			\label{NL-L nonlinear stab} 
			&\quad + \int_{\OmeNL} \ppGW (\nabla y)\big(\beta (x)\big) (\nabla v)^{2}\,\d x\\
			\label{NL-L linear stab} 
			&\quad +\int_{\OmeL}  \ppGWL (\nabla y)\big(\beta (x)\big)(\nabla v)^{2}\,\d x.
		\end{align}
	\end{subequations}
	where we also use the fact that according to our definition of $\beta$, the first sum ranges only over those sites where $\PhiNLL_{\xi} = \Phia_{\xi}$.
	
	%	Let $\epsilon_{1} = \max (\Vert \nabla \sqrt{1-\beta}\Vert_{L^{\infty}} , \Vert \nabla \sqrt{\beta}\Vert_{L^{\infty}} )$. 
	We apply estimate \eqref{Pointwise va blending estimate} to \eqref{All atomistic stab}, and we obtain
	\begin{equation}\label{Atomistic va stab result}
		\begin{split}
			\sum_{\xi=-N}^{N} \sum_{(\rho,\zeta)\in\Rc^{2}} \Phia_{\xi,\rho\zeta} (y)\big(1-\beta(\xi)\big) D_{\rho}v(\xi) D_{\zeta}v(\xi) &\ge \langle \delta^{2} \Ea (y)\va,\va\rangle 
			- 2M^{(2,\frac{1}{2})}K^{-1}\Vert \nabla v \Vert_{L^{2}(-N,N)}^{2}\\
			&\ge c_{0} \Vert \nabla \va \Vert_{L^{2}(-N,N)}^{2}-2M^{(2,\frac{1}{2})}K^{-1}\Vert \nabla v \Vert_{L^{2}(-N,N)}^{2}.
		\end{split}
	\end{equation}
	
	We apply the estimate \eqref{Pointwise vc blending estimate1} to \eqref{NL-L atomistic stab} to get 
	\begin{equation}\label{NL-L atomistic stab 1}
		\begin{split}
			\sum_{\xi \in \Ac} \sum_{(\rho,\zeta)\in\Rc^{2}} \Phia_{\xi,\rho\zeta} (y)\big(\beta(\xi)\big)D_{\rho}v(\xi)D_{\zeta}v(\xi) \ge& \sum_{\xi \in \Ac} \sum_{(\rho,\zeta)\in\Rc^{2}} \Phia_{\xi,\rho\zeta} (y)D_{\rho}\vc(\xi)D_{\zeta}\vc(\xi) \\
			&\ - 2M^{(2,\frac{1}{2})} K^{-1} \Vert \nabla v \Vert_{L^{2}(\OmeA)}^{2}.
		\end{split}
	\end{equation}
	
	By the definition of $\vc$, we notice that for $x \in[-K^{'},K^{'}]$, $\nabla \vc = 0$. After using Taylor's expansion at $\yF$ and $\DH$ assumption, we obtain
	\begin{equation}\label{NL-L atomistic stab 2}
		\begin{split}
			\sum_{\xi \in \Ac} \sum_{(\rho,\zeta)\in\Rc^{2}} \Phia_{\xi,\rho\zeta} (y)D_{\rho}\vc(\xi)D_{\zeta}\vc(\xi)\ge& \sum_{\xi \in \Ac} \sum_{(\rho,\zeta)\in\Rc^{2}} \Phia_{\xi,\rho\zeta} (\yF)D_{\rho}\vc(\xi)D_{\zeta}\vc(\xi)\\
			&\ -2^{\alpha}\CDH M^{(3,0)}  (K)^{-\alpha} \Vert\nabla \vc \Vert_{L^{2}(\OmeA)}^{2}.
		\end{split}
	\end{equation}
	
	Combining \eqref{NL-L atomistic stab 1} with \eqref{NL-L atomistic stab 2}, we can obtain
	\begin{equation}\label{NL-L atomistic stab result}
		\begin{split}
			\sum_{\xi \in \Ac} \sum_{(\rho,\zeta)\in\Rc^{2}} \Phia_{\xi,\rho\zeta} (y)\big(\beta(\xi)\big)D_{\rho}v(\xi)D_{\zeta}v(\xi) &\ge \sum_{\xi \in \Ac} \sum_{(\rho,\zeta)\in\Rc^{2}} \Phia_{\xi,\rho\zeta} (\yF)D_{\rho}\vc(\xi)D_{\zeta}\vc(\xi) \\
			&\ - 2M^{(2,\frac{1}{2})} K^{-1} \Vert \nabla v \Vert_{L^{2}(\OmeA)}^{2}-2^{\alpha}\CDH M^{(3,0)}  K^{-\alpha}\Vert\nabla \vc \Vert_{L^{2}(\OmeA)}^{2}.
		\end{split}
	\end{equation}
	
	Similarly, for the term \eqref{NL-L interaction stab}, we have
	\begin{equation}\label{NL-L interaction stab result}
		\begin{split}
			\sum_{\xi \in \Ic} \sum_{(\rho,\zeta)\in\Rc^{2}} \Phii_{\xi,\rho\zeta} (y)\big(\beta(\xi)\big)D_{\rho}v(\xi)D_{\zeta}v(\xi) &\ge \sum_{\xi \in \Ic} \sum_{(\rho,\zeta)\in\Rc^{2}} \Phii_{\xi,\rho\zeta} (\yF)D_{\rho}\vc(\xi)D_{\zeta}\vc(\xi) \\
			&\ - M^{(2,\frac{1}{2})} K^{-1} \Vert \nabla v \Vert_{L^{2}(\OmeI)}^{2}-\CDH M^{(3,0)}  K^{-\alpha} \Vert\nabla \vc \Vert_{L^{2}(\OmeI)}^{2}.
		\end{split}
	\end{equation}
	
	After considering the definition of $\vc$ and assumption $\DH$, for~\eqref{NL-L nonlinear stab}, we have 
	\begin{equation}\label{NL-L nonlinear stab result}
		\int_{\OmeNL} \ppGW (\nabla y)\big(\beta(x)\big)(\nabla v)^{2}\,\d x \ge 	\int_{\OmeNL} \ppGW (\nabla \yF)(\nabla\vc)^{2}\,\d x - 2\CDH M^{(3,0)}  K^{-\alpha} \Vert \nabla \vc \Vert_{L^{2}(\OmeNL)}^{2}.
	\end{equation}
	
	
	We use the fact that $\ppGWL (\nabla y) = \Wppf = \ppGWL (\nabla \yF)$ again, and we can rewrite \eqref{NL-L linear stab} as
	\begin{equation}\label{NL-L linear stab result}
		\int_{\OmeL}\ppGWL (\nabla y)\big(\beta(x)\big) (\nabla v)^{2} \,\d x = \int_{\OmeL}\ppGWL (\nabla \yF) (\nabla \vc)^{2} \,\d x.
	\end{equation}
	
	%	\yz{We consider 3-order linear function
		%	\begin{align*}
			%		\int_{\OmeL}\ppGWL (\nabla y)(\beta(x)) (\nabla v)^{2} \d x &\ge \int_{\OmeL}\ppGWL (\nabla \yF) (\nabla \vc)^{2} \d x\\
			%		&\ - \CDH M^{(3,0)}  \bar{L}^{-\alpha} \Vert\nabla \vc \Vert_{L^{2}(\OmeL)}^{2}.
			%	\end{align*}
		%
		%
		%
		%
		%}
	
	
	By considering the definition of $\langle \delta^{2} \El (\yF) \vc,\vc \rangle$, \eqref{va and vc} and \eqref{Property of beta}, we conclude that 
	\begin{equation*}
		\langle \delta^{2} \El (y)v,v\rangle \ge \langle \delta^{2} \El (\yF) \vc,\vc \rangle + \langle \delta^{2}\Ea(y)\va,\va \rangle.
	\end{equation*}
	
\end{proof}

% \yz{(\min(c_{0},\ganllF)\Vert \nabla v \Vert_{L^{2}(-N,N)}^{2}) }\\
%&\ \yz{- 4M^{(2,\frac{1}{2})}K^{-1} \Vert \nabla v \Vert_{L^{2}(-N,N)}^{2} - 2^{\alpha}\CDH M^{(3,0)}K^{-\alpha} \Vert \nabla v \Vert_{L^{2}(-L,L)}^{2} }


\subsection{A priori existence and error estimate}
\label{sec: priori_qnll_ncg}

In this section, based on the consistency error estimate \eqref{QNLL consistency error estimate} and stability analysis \eqref{Result of stability} of the QNLL model, we will provide a priori error analysis for the QNLL model using the inverse function theorem.

\begin{theorem}\label{Priori of NCG}
	Let $\yai \in \Ya$ be a strongly stable atomistic solution satisfying \eqref{All-Atomistic strong local minimizer} and $\DH$. Consider the QNLL problem \eqref{Nonlinear-linear solution condition}, supposing, moreover, that $\El$ is stable in the reference state Theorem \ref{Stability}. Then there exists $K_0$ such that, for all $K \ge K_0$, \eqref{Nonlinear-linear solution condition} has a locally unique, strongly stable solution $\ynll$ which satisfies
	\begin{equation}
		\begin{aligned}
			\Vert \nabla\yai - \nabla \ynll \Vert_{L^2} \lesssim 8&M^{(3,0)}(\Vert \nabla^{2} u\Vert_{L^{2}(\bOmeI)} +\Vert \nabla^{3}u \Vert_{L^{2}(\bOmeC)}+\Vert \nabla^{2}u \Vert^{2}_{L^{4}(\bOmeC)}\\
			&+ \Vert \nabla u \Vert^{2}_{L^{4}(\OmeL)}+N^{\frac{1}{2}-\alpha})/\big(\min(c_{0},\ganllF)\big)^2.
		\end{aligned}
	\end{equation}
	
\end{theorem}
\begin{proof}
	We will first provide the a priori error estimate for $\Vert \nabla \ynll - \nabla \ya\Vert_{L^2}$ using the quantitative inverse function theorem, with
	\begin{equation*}
		\Ghc( \ya):=\delta \El( \ya) - \langle f,\cdot\rangle_{N}.
	\end{equation*}
	We first apply that the scaling condition implies a Lipschitz bound for $\delta^{2}\El$, 
	\begin{equation}\label{scaling assumption}
		\Vert \delta^2 \El (y) - \delta^2 \El(v)\Vert_{\mathcal{L}(\Ya,\Ya^{*})}\le M \Vert \nabla y-\nabla v\Vert_{L^{\infty}}, \quad \text{for all } v \in \Ua,
	\end{equation}
	where $M \lesssim M^{(3,0)}$. Since $\Vert \cdot \Vert_{\infty} \lesssim \Vert \cdot \Vert_{L^2}$, we can also replace the $L^\infty$- norm on the right-hand side with the $L^2$-norm.
	The residual estimate \eqref{QNLL consistency error estimate} gives
	\begin{equation*}
		\Vert \Ghc( \ya)\Vert_{\Yn}\lesssim 
		M^{(2,1)}\Vert \nabla^{2} u\Vert_{L^{2}(\bOmeI)} +M^{(2,2)}\Vert \nabla^{3}u \Vert_{L^{2}(\bOmeC)}+M^{(3,2)}\Vert \nabla^{2}u \Vert^{2}_{L^{4}(\bOmeC)}+ M^{(3,0)}\Vert \nabla u \Vert^{2}_{L^{4}(\OmeL)}
	\end{equation*}
	From Theorem \ref{Stability} we obtain that
	\begin{equation*}
		\langle \delta^{2}\El (\ya)v,v\rangle \ge (\min(c_{0},\ganllF)-CK^{-\min(1,\alpha)})\Vert \nabla v \Vert_{L^{2}}^{2}.
	\end{equation*}
	Let $\gamma:=\frac{1}{2}\min(c_{0},\ganllF)$. Applying the Lipschitz bound \eqref{scaling assumption}, and we obtain
	\begin{equation*}
		\langle \delta^{2}\El (\ya)v,v\rangle \ge (2\gamma-CK^{-\min(1,\alpha)}-CK^{-1/2-\alpha})\Vert \nabla v \Vert_{L^{2}}^{2}.
	\end{equation*}
	Hence, for $K$ sufficiently large, we obtain that
	\begin{equation*}
		\langle \delta\Ghc (\ya)v,v\rangle \ge \gamma\Vert \nabla v \Vert_{L^{2}}^{2}, \quad \text{for all }v\in\Un.
	\end{equation*}
	Thus, we deduce the existence of $\ynll$ satisfying $\Ghc(\ynll)=0$. The error estimate implies
	\begin{equation*}
		\begin{aligned}
			\Vert \nabla \ynll - \nabla \ya\Vert_{L^2}&\lesssim \frac{2M\eta}{\gamma^2} \\
			&\lesssim  2M^{(3,0)}(\Vert \nabla^{2} u\Vert_{L^{2}(\bOmeI)} +\Vert \nabla^{3}u \Vert_{L^{2}(\bOmeC)}+\Vert \nabla^{2}u \Vert^{2}_{L^{4}(\bOmeC)}
			+ \Vert \nabla u \Vert^{2}_{L^{4}(\OmeL)})/\gamma^2.
		\end{aligned}
	\end{equation*}
	Finally, by using the triangle inequality and truncation error \eqref{Truncation error}, we yield the stated result.
	
\end{proof}

\subsection{Discussion of the (quasi-)optimal choice of the length of regions}
\label{sec: balance_of_qnll_ncg_model}

In this subsection, we will discuss how to achieve the quasi-optimal choice of the lengths for nonlinear continuum region, linear continuum region, and computational region to obtain quasi-optimal convergence order for the QNLL model.

\subsubsection{The quasi-optimal choice of $L$}
\label{sec: choice_of_L_ncg}

We aim to balance the lengths of nonlinear continuum region and linear continuum region by incorporating coupling error estimates \eqref{Interface region stress tensor}, \eqref{Continuum region stress tensor} and linearization error estimate \eqref{Linearization error estimate}, under the assumption of $\DH$.

First, we use the $\DH$ assumption to obtain the convergence order of coupling error estimates \eqref{Interface region stress tensor}, \eqref{Continuum region stress tensor} concerning the length of atomistic region and nonlinear continuum region :
\begin{equation*}
	\begin{split}
		M^{(2,1)}\Vert \nabla^{2} u&\Vert_{L^{2}(\bOmeI)} +M^{(2,2)}\Vert \nabla^{3}u \Vert_{L^{2}(\bOmeC)}+M^{(3,2)}\Vert \nabla^{2}u \Vert^{2}_{L^{4}(\bOmeC)}M^{(3,0)}\\
		&\lesssim \CDH M^{(2,1)}K^{-\alpha -1}+\CDH M^{(2,2)}\bar{K}^{-\alpha-\frac{3}{2}} +\CDH^{2}M^{(3,2)}\bar{K}^{-2\alpha -\frac{3}{2}}.
	\end{split}
\end{equation*}
Here we need to note the fact that $K + 2 = \bar{K}$, so we can assume $K\approx \bar{K}$. The lowest-order term among them is $\Vert \nabla^{2} u\Vert_{L^{2}(\bOmeI)} \lesssim K^{-\alpha-1}(\bar{K}^{-\alpha-1})$.

Next, we will similarly apply the $\DH$ assumption to the linearization error estimate \eqref{Linearization error estimate} to obtain its convergence order with respect to the length of the linear continuum region :

\begin{equation*}
	M^{(3,0)}\Vert \nabla u \Vert^{2}_{L^{4}(\OmeL)} \lesssim \CDH^{2}M^{(3,0)}L^{-2\alpha+\frac{1}{2}}.
\end{equation*}

The lowest-order term of $L$ is $\Vert \nabla u \Vert^{2}_{L^{4}(\OmeL)} \lesssim L^{-2\alpha+\frac{1}{2}}$. We balance this term with $\Vert \nabla^{2} u\Vert_{L^{2}(\bOmeI)} \lesssim \bar{K}^{-\alpha-1}$ to get (by noticing the fact that $\bar{K}\le L$)
\begin{align}
	L &\lesssim \bar{K}^{\frac{1}{2}+\frac{5}{8\alpha-2}}(\frac{1}{2}<\alpha<\frac{3}{2}) \label{Balance of L NCG 1},\\
	L &\approx \bar{K}(\alpha \ge \frac{3}{2})\label{Balance of L NCG 2}.
\end{align}

Because through balancing we have made the orders of the lowest order terms of $\bar{K}$ and $L$ equal, for simplicity in this section, we will uniformly use linearization error $L^{-2\alpha+\frac{1}{2}}$ to represent the lowest-order term of coupling error and linearization error.

\subsubsection{The quasi-optimal choice of $N$}
\label{sec: choice_of_N_ncg}

After balancing the lengths of nonlinear continuum region and linear continuum region, we will now balance the computational domain length, which will follow the principles:

\begin{enumerate}
	\item We should ensure that the truncation error term $N^{\frac{1}{2}-\alpha}$ do not dominate among the various types of errors after balancing the length of the computational domain;
	
	\item We choose the length of the computational domain as small as possible for computational simplicity.
\end{enumerate}

According to the first principle mentioned above, we understand that the truncation error $N^{\frac{1}{2}-\alpha}$ must be balanced against one of the terms of coupling error or linearization error (or higher-order terms). According to the second principle, to select the computational domain length $N$ as small as possible, we must balance it against the lowest-order term of coupling error or linearization error (balancing against higher-order terms would need a longer computational domain length).

After balancing the lengths of nonlinear continuum region and linear continuum region, the lowest-order term is $L^{\frac{1}{2}-2\alpha}$, we should choose $N$ such that
\begin{equation*}
	L^{\frac{1}{2}-2\alpha}\approx N^{\frac{1}{2}-\alpha}, \quad \text{that is}, \ N\approx L^{\frac{2\alpha-1/2}{\alpha-1/2}}.
\end{equation*}

\subsection{Numerical validation}
\label{sec: experiments_qnll_ncg}

In this section, we present numerical experiments to illustrate the result of our analysis. With slight adjustments, the problem we consider here is a typical testing case in one dimension. We fix $F=1$ and let $V$ be the site energy given by the embedded atom method (EAM)~\cite{1984_Daw_Baskes_EAM_PRB}:
\begin{equation}\label{EAM of numerical experiments}
	V\big(Dy(l)\big)=\frac{1}{2} \sum_{i\in\{1,2\};j\in\{-1,-2\}}\big(\phi(D_{i}y_{l})+\phi(-D_{j}y_{l})\big)+\widetilde{F}\left(\sum_{i\in\{1,2\};j\in\{-1,-2\}}\big[\psi(D_{i}y_{l})+\psi(-D_{j}y_{l})\big]\right),
\end{equation}
where $\phi(r) = \exp\big(-2a(r-1)\big)-2\exp\big(-a(r-1)\big), \psi(r) = \exp(-br)$, and $\tilde{F}(\rho) = c[(\rho-\rho_{0})^{2}+(\rho-\rho_{0})^{4}]$, with the parameter $a=4.4, b=3, c=5,\rho_{0} =2\exp(-b)$.

We fix an exact solution
\begin{equation}\label{External force of numerical experiments}
	\ya(\xi):=F\xi + \frac{1}{10}(1+\xi^2)^{\alpha/2}\xi,
\end{equation}
and compute the external forces $f(\xi)$ to be the equal to the internal forces under the deformation $\ya$. The parameter $\alpha$ is a prescribed decay exponent. One may readily check that this solution and the associated external forces satisfy the decay hypothesis $\DH$.


We will demonstrate the method of controlling the length of non-linear continuum region in the QNLL model to achieve quasi-optimal convergence order, as introduced in Section~\ref{sec: balance_of_qnll_ncg_model}. We will conduct numerical experiments with the atomistic model length of 100,000 atoms. We set energy functional and external force to \eqref{EAM of numerical experiments} and \eqref{External force of numerical experiments}. The experiments will be carried out for $\alpha$ values of $1.2, 1.5$ and $1.8$.

%\begin{figure}[h!]
%    \centering
%    \begin{minipage}{0.3\textwidth}  % 调整宽度以适应你的图片
	%        \centering
	%        \includegraphics[width=\linewidth]{Figs/alpha18_NCG.pdf}
	%    \caption{The convergence order of QNL and QNLL method(without coarse-graining) ($\alpha = 1.8$)} % 图片标题
	%  \label{fig: convergence_QNL_QNLL_alpha18_NCG}
	%    \end{minipage}\hfill
%    \begin{minipage}{0.3\textwidth}
	%        \centering
	%        \includegraphics[width=\linewidth]{Figs/alpha15_NCG.pdf}
	%    \caption{The convergence order of QNL and QNLL method(without coarse-graining) ($\alpha = 1.5$)} % 图片标题
	%  \label{fig: convergence_QNL_QNLL_alpha15_NCG}
	%    \end{minipage}\hfill
%    \begin{minipage}{0.3\textwidth}
	%        \centering
	%          \includegraphics[width=\linewidth]{Figs/alpha12_NCG.pdf}
	%    \caption{The convergence order of QNL and QNLL method(without coarse-graining) ($\alpha = 1.2$)} % 图片标题
	%  \label{fig: convergence_QNL_QNLL_alpha12_NCG}
	%    \end{minipage}
%\end{figure}

\begin{figure}[h!]
	\centering
	\subfloat[$\alpha = 1.8$]{
		\includegraphics[width=0.3\textwidth]{Figs/alpha18_NCG.pdf}
		\label{fig: convergence_QNL_QNLL_alpha18_NCG}
	}
	\subfloat[$\alpha = 1.5$]{
		\includegraphics[width=0.3\textwidth]{Figs/alpha15_NCG.pdf}
		\label{fig: convergence_QNL_QNLL_alpha15_NCG}
	}
	\subfloat[$\alpha = 1.2$]{
		\includegraphics[width=0.3\textwidth]{Figs/alpha12_NCG.pdf}
		\label{fig: convergence_QNL_QNLL_alpha12_NCG}
	}
	\caption{The convergence order of QNL and QNLL method with different $\alpha$ (without coarse-graining)}
	\label{fig: convergence_QNL_QNLL_NCG}
\end{figure}


Firstly, let us consider the experiment with alpha set to $1.8$: In this case, according to \eqref{Balance of L NCG 2}, by setting the length of the nonlinear continuum region to a few atoms ($\bar{K} \approx L$), the convergence order of the QNLL method matches that of the QNL method. In the Figure below, the $x$-axis represents the length of $L$, while the $y$-axis shows the absolute error $\Vert \nabla \yai - \nabla y^{\text{ac}} \Vert_{L^{2}}~\text{(ac} = \text{QNL, QNLL)}$ between the reference atomistic solution $\yai$ and the AC solutions $y^{\text{ac}}$. It can be observed that the two convergence order lines in the graph nearly overlap, indicating that the difference between the two AC solutions $\Vert \nabla y^{\text{QNL}} - \nabla y^{\text{QNLL}}\Vert$ is between $10^{-6} $and $10^{-7}$.

% \begin{figure}[h]
	%   \centering 
	%   \includegraphics[width=0.8\textwidth]{Figs/alpha18_NCG.png}
	%     \caption{The convergence order of QNL and QNLL method(without coarse-graining) ($\alpha = 1.8$)} % 图片标题
	%   \label{fig: convergence_QNL_QNLL_18_NCG}
	% \end{figure}

When $\alpha = 1.5$, the results are similar to when $\alpha = 1.8$. The figure above compares the convergence order of the QNLL method and the QNL method. The information represented on the axes is the same as in Figure \ref{fig: convergence_QNL_QNLL_alpha12_NCG}. We observe a similar outcome to Figure \ref{fig: convergence_QNL_QNLL_alpha18_NCG}, where the convergence lines of the QNLL method closely overlap with those of the QNL method.

% \begin{figure}[h]
	%   \centering 
	%   \includegraphics[width=0.8\textwidth]{Figs/alpha15_NCG.png}
	%     \caption{The convergence order of QNL and QNLL method(without coarse-graining) ($\alpha = 1.5$)} % 图片标题
	%   \label{fig: convergence_QNL_QNLL_15_NCG}
	% \end{figure}

Furthermore, we will now consider the case where $\alpha=1.2$. In this setting, according to \eqref{Balance of L NCG 1} and \eqref{Balance of L NCG 2}, there are two mesh generation schemes for the QNLL method:
\begin{enumerate}
	\item In the first scheme, we focus on the accuracy of the QNLL method. According to \eqref{Balance of L NCG 1}, we precisely balance the atomistic region, nonlinear continuum region, linear continuum region, and the total length of the computational domain to achieve convergence order identical to those of the QNL method.
	
	\item In the second scheme, we prioritize the computational efficiency of the QNLL method. Therefore, after balancing the lengths of the atomistic region and the total length of the computational domain, we minimize the length of the nonlinear continuum region as much as possible, even down to just a few atoms.
\end{enumerate}


% \begin{figure}
	%   \centering 
	%   \includegraphics[width=0.8\textwidth]{Figs/alpha12_NCG.png}
	%     \caption{The convergence order of QNL and QNLL method(without coarse-graining) ($\alpha = 1.2$)} % 图片标题
	%   \label{fig: convergence_QNL_QNLL_12_NCG}
	% \end{figure}

In the figure below, we represent the first mesh generation scheme with red dashed squares for the QNLL method, and the second generation scheme with blue dashed squares. To demonstrate the accuracy of the QNLL method, the QNL method also adopts the first mesh generation scheme, depicted in the figure with red dashed star symbols. The information represented on the axes is the same as in Figure \ref{fig: convergence_QNL_QNLL_alpha18_NCG}. We observe that, after balancing the lengths of the atomisticc region, nonlinear continuum region, linear continuum region, and the total length of the computational domain, the absolute errors and convergence order obtained by the QNLL method are consistent with those of the QNL method. However, after reducing the length of the nonlinear continuum region in pursuit of computational efficiency, there is a noticeable increase in absolute errors and a decrease in convergence speed.

Next, to demonstrate the computational efficiency of the QNLL method where $\alpha = 1.2$, we test the variation in computation time by progressively increasing the degrees of freedom of the nonlinear continuum region $\Nnl$ of the QNLL method, while keeping the finite element mesh fixed, meaning the continuum region remains unchanged. The results are as shown in the table below: the first column lists the method names, with parentheses indicating the proportion of the degrees of the freedom of nonlinear continuum region $\Nl$ to that of the total continuum region $\Nc$; the second column denotes the total degrees of freedom of the mesh and the third column records the ratio of the computing time of the QNLL method to the computing time of the QNL method on a device with an M1 CPU and 16 GB of RAM:

%\begin{table}
%    \centering
%\begin{tabular}{|c|c|c|} % 开始一个tabular环境,设置3列,每列居中对齐
%\hline % 绘制表格的横线
%Method($\Nnl/\Nc$) & DoF  & The ratio of the computing time\\ % 表头行
%\hline % 绘制表格的横线
%% QNLL($17.81\%$) & 2981 & $9.9766156\times 10^{-4}$ & $0.0840$ \\ % 第一行数据
%QNLL($24.99\%$) & 500000  & $66.68\%$ \\ % 第二行数据
%% QNLL($38.69\%$) & 2981 & $9.9766096\times 10^{-4}$ & $0.0024\%$\\ % 第三行数据
%QNLL($40.97\%$) & 500000 & $77.66\%$ \\ % 第四行数据
%% QNLL($59.57\%$) & 2981 &$9.9766096\times 10^{-4}$ & $0.0979 $\\ % 第五行数据
%% QNLL($66.53\%$) & 2981 & $9.9766096\times 10^{-4}$ & $0.0997$ \\ % 第六行数据
%QNLL($74.96\%$) & 500000 & $88.88\%$ \\ % 第七行数据
%% QNLL($83.92\%$) & 2981 & $9.9766097\times 10^{-4}$ & $0.1052$ \\ % 第八行数据
%QNL($100\%$) & 500000 & $100\%$ \\ % 第九行数据
%\hline % 绘制表格的横线
%\end{tabular}
%\caption{The computing time of QNL and QNLL method without coarse-graining ($\alpha = 1.2$)}
%    \label{tab:computing time alpha12 NCG}
%\end{table}

\begin{table}
	\centering
	\renewcommand{\arraystretch}{1.5} % 调整行间距
	\begin{tabular}{|c|c|} % 开始一个tabular环境,设置2列,每列居中对齐
		\hline % 绘制表格的横线
		Method ($\Nnl/\Nc$) & The ratio of the computing time\\ % 表头行
		\hline % 绘制表格的横线
		QNLL ($24.99\%$) & $66.68\%$ \\ % 第二行数据
		QNLL ($40.97\%$) & $77.66\%$ \\ % 第四行数据
		QNLL ($74.96\%$) & $88.88\%$ \\ % 第七行数据
		QNL ($100\%$) & $100\%$ \\ % 第九行数据
		\hline % 绘制表格的横线
	\end{tabular}
	\caption{The computing time (without coarse graining) of QNL and QNLL method without coarse-graining ($\alpha = 1.2$), with Degree of Freedom (DoF) set to 500000 for all methods.}
	\label{tab:computing time alpha12 NCG}
\end{table}

Table \ref{tab:computing time alpha12 NCG} clearly shows that, with fixed lengths of the Atomistic and Continuum regions, the computing time increases significantly as the proportion of nonlinear elements in the Continuum region rises. However, in practical applications, the proportion of nonlinear elements will be lower (below 5$\%$) according to the balancing method described in Section \ref{sec: balance_of_qnll_ncg_model}. The ratio of the difference between the absolute errors of the QNLL method and the QNL method to the absolute errors of the QNL method: $( \Vert \nabla \yai - \nabla y^{\text{QNLL}} \Vert_{L^{2}} - \Vert \nabla \yai - \nabla y^{\text{QNL}} \Vert_{L^{2}}) / \Vert \nabla \yai - \nabla y^{\text{QNL}} \Vert_{L^{2}}$ is in a narrow range. Here, the ratio, as defined above, is within the range of $10^{-5}$ to $10^{-6}$. This indicates that the QNLL method maintains high accuracy while still offering computational efficiency advantages.