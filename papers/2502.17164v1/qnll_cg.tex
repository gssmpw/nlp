\section{QNLL Method with Coarse-Graining}
\label{sec: qnll_cg}

The QNLL method we analyze in Section~\ref{sec: anal_qnll_ncg} is not a computable scheme since it considers every atom in the computational domain as a degree of freedom and the computational cost gets high fast as the computational domain or $N$ goes large. Therefore, as a common practice of the a/c method, we need to coarse grain the continuum region to reduce the number of degrees of freedom and consequently the computational cost. 

In this section, we follow the same analysis framework as Section~\ref{sec: anal_qnll_ncg}. However, the difference is that in the consistency error part, we incorporate the error introduced by coarse graining. First, we give the formulation and analysis of the coarse-grained QNLL method. Then, we focus on the balancing of the atomistic, nonlinear, and linear regions so that the (quasi-)optimal convergence of the QNLL method, comparable to that of the QNL method, is achieved. Finally, we present several numerical experiments to demonstrate that the QNLL method, with proper balance of the different regions, retains the same level of accuracy as the QNL method while substantially lowering the computational cost measured by CPU time.


\subsection{Coarse-graining and analysis of the QNLL method}
\label{sec: anal_qnll_cg}
%Say nonlinear!!!!!

%We restrict displacements again to a computational domain $[-N,N], \ N \in \Nb$. 

Let $\ThNL = \{T_j\}_{j = 1}^{J} := \big\{[v_{j-1},v_{j}]\ | \ j=1,\dots,J\big\}$ be a regular partition of the computational domain $[-N,N]$ into closed intervals or elements $T_j$. We assume that the vertices of the partition are all at atoms or lattice sites and are denote  by $\NhNL := \{v_{0},\dots,v_J \} \subset \Z_{+}$ (which are often termed as rep-atoms in the language of the quasicontinuum method). We define the coarse-grained space of displacements by
\begin{equation}\label{UhNL space}
	\UhNL:=\{u_{h}\in\Un \ | \ u_{h}(-N)=u_{h}(N)=0 \text{ and }u_{h} \text{ is piecewise affine with respect to }\ThNL\},
\end{equation}
and subsequently the admissible set of deformations by
\begin{equation}\label{YhNL space}
	\YhNL:=\{y\ | \  y = Fx+u_{h}, \ u_{h}\in\UhNL\}.
\end{equation}

We define interpolation operator $I_h: \Yn \rightarrow \YhNL$ by $\IhNL v(\zeta) =v(\zeta), \forall \zeta \in \NhNL$ and $\IhNL v \in \text{P1} (\ThNL)$, which is the piecewise affine nodal interpolation with respect to $\ThNL$. We firstly introduce the following proposition obtained from Poincare's inequality for future usage.

\begin{proposition}
	Let $T\in \ThNL , \ T \subset [-N,-\bar{K}]\cup[\bar{K},N]$ and $u\in \mathscr{U}$. Then
	\begin{equation*}
		\Vert \nabla u-\nabla \IhNL u\Vert_{L^{2}(T)}\lesssim h_{T} \Vert \nabla^{2} u\Vert_{L^{2}(T)}.
	\end{equation*}
	% 	If, in addition, $u$ satisfies $\DH$, $L \le N/2$ and then
	% 	\begin{equation*}
		% 		\vert \nabla \IhNL y(x) -\nabla \PhNL y(x) \vert \lesssim \left\{
		% 		\begin{aligned}
			% 			&N^{-\alpha},&\ x\in[-N,-L],\\
			% 			&0,&\ x\in[-L,L],\\
			% 			&N^{-\alpha},&\ x\in[L,N].
			% 		\end{aligned}
		% 		\right.
		% 	\end{equation*}
\end{proposition}

% \begin{proof}
	
	
	% To prove the second estimate, we first note that $\nabla \IhNL y=\nabla \PhNL y= \nabla y$ in $[-L,L]$. In $[-N,-L]\cup[L,N]$ we have(we choose $u(N)$ in our calculation, for $u(-N)$ the calculation is same)
	% \begin{equation*}
		% 	\vert \nabla \IhNL y(x) -\nabla \PhNL y(x) \vert  = (N-L)^{-1}\vert u(N)\vert \lesssim (N-L)^{-1}N^{1-\alpha}\lesssim N^{-\alpha}.
		% \end{equation*}
	% \end{proof}

For each $T\in \ThNL$, we let $h_{T}:=\text{diam} (T)$. Thus, for $f,g \in \Un$, we define 
\begin{equation*}
	\langle f,g\rangle_{h}:= \int_{-N}^{N} \IhNL (f\cdot g)\,\d x=\sum_{j=1}^{J} \frac{1}{2} h_{T}\big\{f(v_{j-1})\cdot g(v_{j-1}) + f(v_{j})\cdot g(v_{j})\big\}.
\end{equation*}

The coarse-grained QNLL model we aim to solve is the following: 
\begin{equation}
	\label{Yh solution}
	\yh \in \argm \{\El (y_{h})-\langle f,y_{h} \rangle_{h}\ | \ y_{h}\in \YhNL\}.
\end{equation}
%The philosophy of the quasi-continuum(QC) method is to retain the atomistic description, but restrict the admissible set to the coarse space $\YhNL$\yz{Another symbol?}:

%We choose a set of finite element nodes $\NhNL = \{v_{0},\dots,v_{N_{\ThNL}}\}$, for some $N_{\ThNL}\in \Nb$, such that $\{-N,-\bar{L},-L,\dots,L,\bar{L},N\}\subset\NhNL \subset \{-N,\dots,N\}$. The finite elements are given by $\ThNL=\{[v_{j-1},v_{j}]\ | \ j=1,\dots,J\}$. For each $T\in \ThNL$, let $h_{T}:=\text{diam} T$. 
%
%The space of all continuous piecewise affine functions on $[-N,N]$ is given by $\text{P1}(\ThNL)$, and the space of piecewise constant functions by $\text{P0}(\ThNL)$. The admissible finite element space $\UhNL$ is defined by \eqref{UhNL space}, and imposes the boundary condition $u_{h}(-N)=u_{h}(N)=0$.
%
%Finally, for any function $v:\NhNL \rightarrow \R$, let $\IhNL v:[-N,N]\rightarrow \R, \ \IhNL v \in \text{P1} (\ThNL)$, denotes its continuous piecewise affine interpolant,
%\begin{equation*}
%	\IhNL v(\zeta):=v(\zeta), \quad \text{for all } \zeta \in \NhNL.
%\end{equation*}
%
%For $f,g:\NhNL \rightarrow \R^{J}$, we define


% \subsubsection{P1 interpolation operator}



\subsubsection{Coarsening error of the internal forces}
\label{sec: internal_forces_qnll_cg}
The first variation of the continuum energy contribution $\int_{\OmeC} W(\nabla y)\text{d}x$ is given by
\begin{equation*}
	\int_{\OmeNL} \partial_{F} W(\nabla y)\nabla v \text{d}x+\int_{\OmeL} \partial_{F} W_{\text{L}}(\nabla y)\nabla v \,\textrm{d}x.
\end{equation*}
Elements of $\UhNL$ are defined pointwise, but give rise to lattice functions through point evaluation. Since finite element nodes lie on lattice sites, this is compatible with our interpolation of lattice functions.

The following lemma estimates the error contribution from this operator induced by finite element coarsening and reduction to a finite domain.

\begin{theorem}\label{Internal forces of continuum region}
	Let $u \in \mathscr{U}$ satisfy $(\mathbf{DH})$, and $0<\bar{K} <L \le N/2$. Then
	\begin{equation*}
		\begin{aligned}
			\Bigg \vert \int_{\OmeNL} \big(\partial_{F} W(\nabla \IhNL y) &- \partial_{F} W(\nabla y)\big) \nabla v_{h} \, \text{d} x + \int_{\OmeL} \big(\partial_{F} \WL (\nabla \IhNL y) - \partial_{F} \WL (\nabla y)\big) \nabla v_{h} \,\text{d} x \Bigg \vert \\
			&\lesssim M^{(2,0)}\Vert h\nabla^{2}u\Vert^{2}_{L^{4}(\OmeNL)} \Vert \nabla v_{h} \Vert_{L^{2}}, \quad \text{for all}\  v_{h} \in \text{P1}(\mathcal{T}_{h}).\\
		\end{aligned}
	\end{equation*}
\end{theorem}

\begin{proof}
	% Firstly, we calculate the first integral on nonlinear region. We fix $\forall T\in \OmeNL$, and we have
	% 	\begin{equation}\label{Each T of nonlinear}
		% 		\begin{aligned}
			% 			\vert &\int_{T} (\partial_{F} W(\nabla \PhNL y) - \partial_{F} W(\nabla y)) \nabla v_{h} \text{d} x \vert \\
			% 			&\le \vert \int_{T} (\partial_{F} W(\nabla \PhNL y) - \partial_{F} W(\nabla \IhNL y)) \nabla v_{h} \text{d} x \vert \quad (\text{Part \uppercase\expandafter{\romannumeral1}}) \\
			% 			&+ \vert \int_{T} (\partial_{F} W(\nabla \IhNL y) - \partial_{F} W(\nabla y)) \nabla v_{h} \text{d} x \vert. \quad (\text{Part \uppercase\expandafter{\romannumeral2}})
			% 		\end{aligned}
		% 	\end{equation}	
	
	% 		For the Part \uppercase\expandafter{\romannumeral1} of \eqref{Each T of nonlinear} , apply H$\ddot{\text{o}}$lder‘s inequality, and we have
	% 		\begin{equation}\label{Part 1 of nonlinear}
		% 		\begin{aligned}
			% 			&\int_{T} (\partial_{F} W(\nabla \PhNL y) - \partial_{F} W(\nabla \IhNL y)) \nabla v_{h} \text{d} x \\
			% 			&\le (\int_{T} (\partial_{F} W(\nabla\PhNL y) - \partial_{F} W(\nabla \IhNL y))^{2} \text{d} x)^{\frac{1}{2}} \cdot \Vert \nabla v_{h} \Vert_{L^{2}(T)}.
			% 		\end{aligned}
		% \end{equation}	
	
	% 	We calculate directly, and get
	% 	\begin{equation*}
		% 		\begin{aligned}
			% 			\vert \partial_{F} W(\nabla \PhNL y) - \partial_{F} W(\nabla \IhNL y) \vert 
			% 			&\le M^{(2,0)} \cdot \vert \nabla \PhNL y - \nabla \IhNL y \vert \\
			% 			&= M^{(2,0)}  \cdot \frac{\vert y(N) \vert}{N-L} \\
			% 			&\le M^{(2,0)}  \cdot C_{\mathbf{DH}}\  (N-L)^{-1} \  N^{1-\alpha}\\
			% 			&\le \frac{1}{2} C_{\mathbf{DH}} \ M^{(2,0)} \ N^{-\alpha}.
			% 		\end{aligned}
		% 	\end{equation*}
	
	% From the present result, 
	% The Part \uppercase\expandafter{\romannumeral1} will be
	% \begin{equation*}
		% 	\int_{T} (\partial_{F} W(\nabla \PhNL y) - \partial_{F} W(\nabla \IhNL y)) \nabla v_{h} \text{d} x \lesssim M^{(2,0)}\cdot (h_{T}^{\text{NL}})^{\frac{1}{2}} \ N^{-\alpha} \cdot \Vert \nabla v_{h} \Vert_{L^{2}(T)}.
		% \end{equation*}
	
	Firstly, we calculate the first integral on nonlinear region. After using the fact that $\int_{T}\nabla \IhNL u\,\d x=\int_{T} \nabla u\,\d x$, for any $T\in \OmeNL$, we have
	\begin{equation*}
		\begin{aligned}
			\Bigg \vert \int_{T} \partial_{F} W(\nabla \IhNL y) - \partial_{F} W(\nabla y) \,\d x \Bigg \vert &\le \Bigg \vert \int_{T}\partial^{2}_{F}W(\nabla\IhNL y)(\nabla\IhNL u -\nabla u)\,\d x \Bigg \vert\\
			&\quad + M^{(3,0)}\int_{T}\vert \nabla \IhNL u-\nabla u\vert^{2}\,\d x\\
			&\lesssim \Vert \nabla \IhNL u -\nabla u\Vert^{2}_{L^{2}(T)}\lesssim \Vert h\nabla^{2} u\Vert^{2}_{L^{2}(T)}.
		\end{aligned}
	\end{equation*}
	
	Summing over $T\subset \ThNL$ and again applying H$\ddot{\text{o}}$lder‘s inequality yields
	\begin{equation*}
		\sum_{T \in \ThNL} \Vert h\nabla^{2} u\Vert^{2}_{L^{2}(T)}\Vert \nabla v_{h} \Vert_{L^{2}(T)}
		\le \Vert h\nabla^{2} u\Vert^{2}_{L^{4}(\OmeNL)}\Vert \nabla v_{h} \Vert_{L^{2}(\OmeNL)}.
	\end{equation*}
	
	% Next we focus on linear region, for each $T \in \ThNL$, we have
	
	% \begin{equation}\label{Each T of linear}
		% 	\begin{aligned}
			% 		\vert &\int_{T} (\partial_{F} W_{\text{L}}(\nabla \Pi_{h} y) - \partial_{F} W_{\text{L}}(\nabla y)) \nabla v_{h} \text{d} x \vert \\
			% 		&\le \vert \int_{T} (\partial_{F} W_{\text{L}}(\nabla \Pi_{h} y) - \partial_{F} W_{\text{L}}(\nabla I_{h}y)) \nabla v_{h} \text{d} x \vert \quad (\text{Part \uppercase\expandafter{\romannumeral3}}) \\
			% 		&+ \vert \int_{T} (\partial_{F} W_{\text{L}}(\nabla I_{h} y) - \partial_{F} W_{\text{L}}(\nabla y)) \nabla v_{h} \text{d} x \vert. \quad (\text{Part \uppercase\expandafter{\romannumeral4}})
			% 	\end{aligned}
		% \end{equation}
	
	% For the Part \uppercase\expandafter{\romannumeral3} of \eqref{Each T of linear} , apply H$\ddot{\text{o}}$lder‘s inequality, and we have
	% \begin{equation}\label{3.2}
		% 	\begin{aligned}
			% 		&\int_{T} (\partial_{F} \WL(\nabla \Pi_{h} y) - \partial_{F} \WL(\nabla I_{h}y)) \nabla v_{h} \text{d} x \\
			% 		&\le (\int_{T} (\partial_{F} \WL(\nabla \Pi_{h} y) - \partial_{F} \WL(\nabla I_{h}y))^{2}  \text{d} x)^{\frac{1}{2}} \cdot \Vert \nabla v_{h} \Vert_{L^{2}(T)}.
			% 	\end{aligned}
		% \end{equation}
	
	
	
	% Next we focus on linear region, for each $T \in \ThNL$, after using the definition of $W_{L}(F)$, and we have
	% \begin{equation*}
		% 	\begin{aligned}
			% 		\vert \partial_{F} W_{\text{L}}(\nabla \Pi_{h} y) - \partial_{F} W_{\text{L}}(\nabla I_{h}y) \vert 
			% 		&=\Wppf \cdot \vert \nabla \Pi_{h}u - \nabla I_{h} u \vert \\
			% 		&= \Wppf \cdot \frac{\vert u(N) \vert}{N-L'} \\
			% 		&\le \Wppf \cdot C_{\mathbf{DH}}\  (N-L')^{-1} \  N^{1-\alpha}\\
			% 		&\le \frac{1}{2} C_{\mathbf{DH}} \ \Wppf \ N^{-\alpha}.
			% 	\end{aligned}
		% \end{equation*}
	
	% From the present result, 
	% The Part \uppercase\expandafter{\romannumeral3} will be
	% \begin{equation*}
		% 	\int_{T} (\partial_{F} W_{\text{L}}(\nabla \Pi_{h} u) - \partial_{F} W_{\text{L}}(\nabla I_{h}u)) \nabla v_{h} \text{d} x \lesssim \Wppf\cdot (h^{\frac{1}{2}}_{T} \ N^{-\alpha}) \cdot \Vert \nabla v_{h} \Vert_{L^{2}(T)}.
		% \end{equation*}
	
	Next we focus on the linear region, after using the definition of $\WL(F)$, and we have
	\begin{equation*}
		\partial_{F} W_{\text{L}}(\nabla I_{h} u) - \partial_{F} W_{\text{L}}(\nabla u) =\Wppf(\nabla I_{h}u - \nabla u).
	\end{equation*}
	Moreover, using the fact that $\int_{T} \nabla I_{h}u \text{d}x = \int_{T} \nabla u\text{d}x$, since $\nabla v_{h} $ is constant in $T$, we have
	\begin{equation*}
		\begin{aligned}
			\int_{T} \Wppf (\nabla I_{h}u - \nabla u) \nabla v_{h} \,\text{d}x = \Wppf \  \nabla v_{h}\vert_{T} \cdot \int_{T} (\nabla I_{h}u - \nabla u)\,\text{d}x= 0.
		\end{aligned}
	\end{equation*}
	
	% Summing over $T\subset \ThNL$ and again applying H$\ddot{\text{o}}$lder‘s inequality yields
	% \begin{equation*}
		% 	\begin{aligned}
			% 		&\sum_{T \in \mathcal{T}_{h}} h^{\frac{1}{2}}_{T} \Vert \nabla v_{h} \Vert_{L^{2}(T)}\\
			% 		&\le (\sum_{T \in \mathcal{T}_{h}} h_{T})^{\frac{1}{2}} \cdot (\sum_{T \in \mathcal{T}_{h}} \Vert \nabla v_{h} \Vert_{L^{2}(T)}^{2} )^{\frac{1}{2}}\\
			% 		&= (N-\bar{L})^{\frac{1}{2}} \cdot \Vert \nabla v_{h} \Vert_{L^{2}(\bar{L},N)}\\
			% 		&\lesssim N^{\frac{1}{2}} \cdot  \Vert \nabla v_{h} \Vert_{L^{2}(\bar{L},N)}.
			% 	\end{aligned}
		% \end{equation*}
\end{proof}

\subsubsection{Coarsening error of external forces}
\label{sec: external_forces_qnll_cg}
We now address the consistency error arising from approximating the external potential $\langle f, v_{h} \rangle_{\Z}$ using the trapezoidal rule, denoted as $\langle f, v_{h} \rangle_{h}$. A key challenge in this analysis is to avoid relying on the Poincaré inequality $\Vert v_{h} \Vert_{L^{2}} \lesssim N\Vert \nabla v_{h} \Vert_{L^{2}}$. Instead, we utilize weighted Poincaré inequalities, which are more suitable for unbounded domains. This approach leads to the following result. The lemma presented here is adapted from~\cite[Theorem 6.13]{2013_ML_CO_AC_Coupling_ACTANUM}, but we state the theorem directly and omit the proof for brevity.

\begin{lemma}
	Let $L>1,\omega(x) = x\log(x)$ and suppose that $h(x)\le \kappa x$ for almost every $x\in [-N,-\bar{K}\cup[\bar{K},N]$. And we note $[-\infty,-\bar{K}]\cup[\bar{K},+\infty]$ by $\tOmeC$Then there exists a constant $C_{\kappa}$ such that
	\begin{equation*}
		\Vert \eta_{\text{ext}}\Vert_{(\YhNL)^{*}} = \Vert \langle f,\cdot\rangle_{N} - \langle f,\cdot\rangle_{h}\Vert_{(\YhNL)^{*}} \lesssim \Vert h^{2} \nabla f\Vert_{L^{2}(\tOmeC)} +\frac{C_{\kappa}}{\log L} \Vert h^{2}\omega \nabla^{2}f\Vert_{L^{2}(\tOmeC)}.
	\end{equation*}
\end{lemma}

% \begin{proof}
	% 
	% \end{proof}



% \subsection{Coarsening error of linear region}
% %The reson why we choose this.
% We restrict displacements again to a computational domain $[-N,N], N \in \Nb$. Let $\Th = \{T\}$ be a regular partition of $[-N,N]$ into closed intervals $T$, with vertices $\Nh \subset \Z_{+}$ (or, rep-atoms in the language of the quasi-continuum method). We define the coarse displacement space by
% \begin{equation}\label{Uh space}
	% 	\Uh:=\{u_{h}\in\Un \ | \ u_{h} \text{ is picecewise affine with respect to }\Th\}.
	% \end{equation}
% \begin{equation}\label{Yh space}
	% 	\Yh:=\{Fx + u_{h} \ | \  u_{h}\in\Uh \}.
	% \end{equation}
% The philosophy of the quasi-continuum(QC) method is to retain the atomistic description, but restrict the admissible set to the coarse space $\Uh$:
% \begin{equation}\label{uqch solution space}
	% 	y^{\text{a}}_{h} \in \argm \{\Ea (y_{h})-\langle f,y_{h} \rangle_{N}\ | \ y_{h}\in \Yh\}.
	% \end{equation}

% We choose a set of finite element nodes $\Nh = \{v_{0},\dots,v_{N_{\Th}}\}$, for some $N_{\Th}\in \Nb$, such that $\{-N,-\bar{L},\dots,\bar{L},N\}\subset\Nh \subset \{-N,\dots,N\}$. The finite elements are given by $\Th=\{[v_{j-1},v_{j}]\ | \ j=1,\dots,J^{\text{L}}\}$. For each $T\in \Th$, let $h_{T}^{\text{L}}:=\text{diam} T$. For each $x\in [-N,N], \ x\in \text{int} T$, let $h^{\text{L}}(x):=h_{T}$. For $x<-N$ or $x>N$, let $h^{\text{L}}(x):=1$.

% The space of all continuous piecewise affine functions on $[-N,N]$ is given by $\text{P1}(\Th)$, and the space of piecewise constant functions by $\text{P0}(\Th)$. The admissible finite element space $\Uh$ is defined by \eqref{Uh space}, and imposes the boundary condition $u_{h}(-N)=u_{h}(N)=0$.

% Finally, for any function $v:\Nh \rightarrow \R$, let $\Ih v:[-N,N]\rightarrow \R, \ \Ih v \in \text{P1} (\Th)$, denotes its continuous piecewise affine interpolant,
% \begin{equation*}
	% 	\Ih v(\zeta):=v(\zeta) \quad \text{for all } \zeta \in \Nh.
	% \end{equation*}

% For $f,g:\Nh \rightarrow \R^{J^{\text{L}}}$, we define
% \begin{equation*}
	% 	\langle f,g\rangle_{h}^{\text{L}}:= \int_{-N}^{N} \Ih (f\cdot g)\d x=\sum_{j=1}^{J^{\text{L}}} \frac{1}{2} h_{T}^{\text{L}}\{f(v_{j-1})\cdot g(v_{j-1}) + f(v_{j})\cdot g(v_{j})\}.
	% \end{equation*}

% \subsubsection{Best approximation operator}
% Recall the definition of the nodal interpolation operator $\Ih$. Since $\Ih$ does not map $\Ya$ to $\Yh$, and to avoid any error contributions from the atomistic region (and possibly a neighbourhood of the interface region), we define
% \begin{equation*}
	% 	\Ph y(x):= \left\{
	% 	\begin{aligned}
		% 		&\Ih y(x)-\frac{x+\bar{L}}{-N+\bar{L}}u(-N),&\ x\in[-N,-\bar{L}],\\
		% 		&\Ih y(x),&\ x\in[-\bar{L},\bar{L}],\\
		% 		&\Ih y(x)-\frac{x-\bar{L}}{N-\bar{L}}u(N),&\ x\in[\bar{L},N].
		% 	\end{aligned}
	% 	\right.
	% \end{equation*}
% With this definition, $\Ph y\in \Yh$ for all $y:\Nh \rightarrow \R$.

% In our coarsening error analysis below, it will be useful to split the interpolant into $\Ph y = \Ih y +(\Ph y - \Ih y).$ Hence, we separately estimate the interpolation errors as follows.

% \begin{theorem}
	% 	Let $T\in \Th , \ T \subset [-N,-\bar{L}]\cup[\bar{L},N]$ and $y\in \Ya$. Then
	% 	\begin{equation*}
		% 		\Vert \nabla y-\nabla \Ih y\Vert_{L^{2}(T)}\lesssim h_{T}^{\text{L}} \Vert \nabla^{2} u\Vert_{L^{2}(T)}.
		% 	\end{equation*}
	% 	If, in addition, $u$ satisfies $\DH$, $\bar{L} \le N/2$ and $N \ge r_{0}$, then
	% 	\begin{equation*}
		% 		\vert \nabla \Ih y(x) -\nabla \Ph y(x) \vert \lesssim \left\{
		% 		\begin{aligned}
			% 			&N^{-\alpha},&\ x\in[-N,-\bar{L}],\\
			% 			&0,&\ x\in[-\bar{L},\bar{L}],\\
			% 			&N^{-\alpha},&\ x\in[\bar{L},N].
			% 		\end{aligned}
		% 		\right.
		% 	\end{equation*}
	% \end{theorem}

% \begin{proof}
	% 	The first result follows from Poincare's inequality.
	
	% 	To prove the second estimate, we first note that $\nabla \Ih y=\nabla \Ph y= \nabla y$ in $[-\bar{L},\bar{L}]$. In $[-N,-\bar{L}]\cup[\bar{L},N]$ we have(we choose $u(N)$ in our calculation, for $u(-N)$ the calculation is same)
	% 	\begin{equation*}
		% 		\vert \nabla \Ih y(x) -\nabla \Ph y(x) \vert  = (N-\bar{L})^{-1}\vert u(N)\vert \lesssim (N-\bar{L})^{-1}N^{1-\alpha}\lesssim N^{-\alpha}.
		% 	\end{equation*}
	% \end{proof}

% \subsubsection{Coarsening error of the internal forces}

% The first variation of the continuum energy contribution $\int_{\OmeL} W_{L}(\nabla y)\text{d}x$ is given by
% \begin{equation*}
	% 	v \rightarrow \int_{\OmeL} \partial_{F} W_{\text{L}}(\nabla y)\nabla v \text{d}x.
	% \end{equation*}
% Elements of $\Uh$ are defined pointwise, but give rise to lattice functions through point evaluation. Since finite element nodes lie on lattice sites, this is compatible with our interpolation of lattice functions.

% The following lemma estimates the error contribution from this operator induced by finite element coarsening and reduction to a finite domain.

% \begin{theorem}\label{Internal forces of linear region}
	% 	Let $u \in \mathscr{U}$ satisfy $(\mathbf{DH})$, $N > r_{0}$, and $0<\bar{L} \le N/2$. Then
	% 	\begin{equation*}
		% 		\begin{aligned}
			% 			\vert \int_{\OmeL} (\partial_{F} W_{\text{L}}(\nabla \Pi_{h} y) &- \partial_{F} W_{\text{L}}(\nabla y)) \nabla v_{h} \text{d} x \vert \\
			% 			&\lesssim W^{''}(0) N^{\frac{1}{2} - \alpha } \Vert \nabla v_{h} \Vert_{L^{2}}. \quad \text{for all}\  v_{h} \in \text{P1}(\mathcal{T}_{h}).\\
			% 		\end{aligned}
		% 	\end{equation*}
	% \end{theorem}

% \begin{proof}
	% 	For each $T \in \mathcal{T}_{h}$, we have
	
	% 	\begin{equation}\label{Each T of linear}
		% 		\begin{aligned}
			% 			\vert &\int_{T} (\partial_{F} W_{\text{L}}(\nabla \Pi_{h} y) - \partial_{F} W_{\text{L}}(\nabla y)) \nabla v_{h} \text{d} x \vert \\
			% 			&\le \vert \int_{T} (\partial_{F} W_{\text{L}}(\nabla \Pi_{h} y) - \partial_{F} W_{\text{L}}(\nabla I_{h}y)) \nabla v_{h} \text{d} x \vert \quad (\text{Part \uppercase\expandafter{\romannumeral5}}) \\
			% 			&+ \vert \int_{T} (\partial_{F} W_{\text{L}}(\nabla I_{h} y) - \partial_{F} W_{\text{L}}(\nabla y)) \nabla v_{h} \text{d} x \vert. \quad (\text{Part \uppercase\expandafter{\romannumeral6}})
			% 		\end{aligned}
		% 	\end{equation}
	
	% 	For the Part \uppercase\expandafter{\romannumeral5} of \eqref{Each T of linear} , apply H$\ddot{\text{o}}$lder‘s inequality, and we have
	% 	\begin{equation}\label{Part 5 of linear}
		% 		\begin{aligned}
			% 			&\int_{T} (\partial_{F} W_{\text{L}}(\nabla \Pi_{h} y) - \partial_{F} W_{L}(\nabla I_{h}y)) \nabla v_{h} \text{d} x \\
			% 			&\le (\int_{T} (\partial_{F} W_{\text{L}}(\nabla \Pi_{h} y) - \partial_{F} W_{L}(\nabla I_{h}y))^{2} \text{d} x)^{\frac{1}{2}} \cdot \Vert \nabla v_{h} \Vert_{L^{2}(T)}.
			% 		\end{aligned}
		% 	\end{equation}
	
	% 	We use the definition of $W_{\text{L}}(F)$ we have
	% 	\begin{equation*}
		% 		\begin{aligned}
			% 			\vert \partial_{F} W_{\text{L}}(\nabla \Pi_{h} y) - \partial_{F} W_{\text{L}}(\nabla I_{h}y) \vert 
			% 			&= \Wppf \cdot \vert \nabla \Pi_{h}u - \nabla I_{h} u \vert \\
			% 			&= \Wppf \cdot \frac{\vert u(N) \vert}{N-\bar{L}} \\
			% 			&\le \Wppf \cdot C_{\mathbf{DH}}\  (N-\bar{L})^{-1} \  N^{1-\alpha}\\
			% 			&\le \frac{1}{2} C_{\mathbf{DH}} \ \Wppf \ N^{-\alpha}.
			% 		\end{aligned}
		% 	\end{equation*}
	
	% 	From the present result, 
	% 	The Part \uppercase\expandafter{\romannumeral5} will be
	% 	\begin{equation*}
		% 		\int_{T} (\partial_{F} W_{\text{L}}(\nabla \Pi_{h} y) - \partial_{F} W_{\text{L}}(\nabla I_{h}y)) \nabla v_{h} \text{d} x \lesssim \Wppf\cdot ((h_{T}^{\text{L}})^{\frac{1}{2}}_{T} \ N^{-\alpha}) \cdot \Vert \nabla v_{h} \Vert_{L^{2}(T)}.
		% 	\end{equation*}
	% 	\yz{
		% 	For the Part \uppercase\expandafter{\romannumeral6} of \eqref{Each T of linear} , we could compute it directly and get
		% 	\begin{equation*}
			% 		\partial_{F} W_{\text{L}}(\nabla I_{h} y) - \partial_{F} W_{\text{L}}(\nabla y) = \Wppf(\nabla I_{h}y - \nabla y)
			% 	\end{equation*}
		
		% 	Moreover, using the fact that $\int_{T} \nabla I_{h}y \text{d}x = \int_{T} \nabla y\text{d}x$, since $\nabla v_{h} $ is constant in $T$, we have
		% 	\begin{equation*}
			% 		\begin{aligned}
				% 			&\int_{T} \Wppf (\nabla I_{h}u - \nabla u) \nabla v_{h} \text{d}x \\
				% 			&= \Wppf \  \nabla v_{h}\vert_{T} \cdot \int_{T} (\nabla I_{h}u - \nabla u) \text{d}x= 0
				% 		\end{aligned}
			% 	\end{equation*}
		% 	}
	% 	Summing over $T \in \Th$, $T\subset \Th$ and again applying H$\ddot{\text{o}}$lder‘s inequality yields
	% 	\begin{equation*}
		% 		\begin{aligned}
			% 			&\sum_{T \in \Th} (h_{T}^{\text{L}})^{\frac{1}{2}} \Vert \nabla v_{h} \Vert_{L^{2}(T)}\\
			% 			&\le (\sum_{T \in \Th} h_{T})^{\frac{1}{2}} \cdot (\sum_{T \in \Th} \Vert \nabla v_{h} \Vert_{L^{2}(T)}^{2} )^{\frac{1}{2}}\\
			% 			&= (N-\bar{L})^{\frac{1}{2}} \cdot \Vert \nabla v_{h} \Vert_{L^{2}(\OmeL)}\\
			% 			&\lesssim N^{\frac{1}{2}} \cdot  \Vert \nabla v_{h} \Vert_{L^{2}(\OmeL)}.
			% 		\end{aligned}
		% 	\end{equation*}
	% \end{proof}

% \subsubsection{Coarsening error of external forces}
% We now turn towards the consistency error due to the approximation of the external potential $\langle f,v_{h} \rangle_{\Z}$ by the trapezoidal rule $\langle f,v_{h} \rangle_{h}^{\text{L}}$. The main challenge in this analysis is to avoid the use of Poincare inequality $\Vert v_{h} \Vert_{L^{2}} \lesssim N\Vert \nabla v_{h} \Vert_{L^{2}}$ at all costs. A common approach in unbounded domains is to employ weighted Poincare's  inequalities instead. This yields the following result.

% \begin{theorem}
	% 	Let $\bar{L}>1, \omega(x)=x\log (x)$ and suppose that $h^{\text{L}}(x)\le \kappa x$ for almost every $x\in [-N,-\bar{L}]\cup[\bar{L},N]$. And we note $[-\infty,-\bar{L}]\cup[\bar{L},+\infty]$ by $\tOmeL$Then there exists a constant $C^{\text{L}}_{\kappa}$ such that
	% 	\begin{equation*}
		% 		\Vert \eta_{\text{ext}} \Vert_{(\Yh)^{*}} = \Vert \langle f,\cdot\rangle_{\Z} - \langle f,\cdot\rangle^{\text{L}}_{h}\Vert_{(\Yh)^{*}} \lesssim \Vert h^{2} \nabla f\Vert_{L^{2}(\tOmeL)} +\frac{C^{\text{L}}_{\kappa}}{\log \bar{L}} \Vert h^{2}\omega \nabla^{2}f\Vert_{L^{2}(\tOmeL)}.
		% 	\end{equation*}
	% \end{theorem}

% \begin{proof}
	% 	Already check.
	% \end{proof}



% \section{Stability}\label{sec:Stability}

% \chw{Give an analysis of the stability and discuss how the linearization may affect the stability}
% Firstly, we calculate the second variation of the energy functional of Nonlinear-linear elasticity coupling reflection energy defined by \eqref{Nonlinear-linear energy}, for any $v\in\YacNL$, is then given by
% \begin{equation}\label{Stab of NL-L}
	% 	\begin{split}
		% 		\langle \delta^{2} \El (y)v,v\rangle &= \sum_{\xi \in \Ac} \sum_{(\rho,\zeta)\in\Rc^{2}}  \Phia_{\xi,\rho\zeta}(y)D_{\rho}v(\xi)D_{\zeta}v(\xi)\\
		% 		&\ +\sum_{\xi \in \Ic} \sum_{(\rho,\zeta)\in\Rc^{2}} \Phii_{\xi,\rho\zeta}(y)D_{\rho}v(\xi)D_{\zeta}v(\xi)\\
		% 		&\ +\int_{\OmeNL}\ppGW(\nabla y) (\nabla v)^{2}\d x\\
		% 		&\ +\int_{\OmeL}\ppGWL(\nabla y) (\nabla v)^{2}\d x.
		% 	\end{split}
	% \end{equation}

% If we focus on the second variation evaluated at the homogeneous deformation $\yF$, and use the fact $\ppGW(\nabla \yF)=\ppGWL(\nabla \yF)=\Wppf$. So we could obtaion that
% \begin{equation*}
	% 	\int_{\OmeNL}\ppGW(\nabla \yF) (\nabla v)^{2}\d x +\int_{\OmeL}\ppGWL(\nabla \yF)  (\nabla v)^{2}\d x =\int_{\OmeC}\ppGW(\nabla \yF) (\nabla v)^{2}\d x.
	% \end{equation*}

% Then we know
% \begin{equation}\label{Stab of NL-L equals to Reflection}
	% 	\begin{split}
		% 		\langle \delta^{2} \El (\yF)v,v\rangle 
		% 		&=\sum_{\xi \in \Ac} \sum_{(\rho,\zeta)\in\Rc^{2}}  \Phia_{\xi,\rho\zeta}(\yF)D_{\rho}v(\xi)D_{\zeta}v(\xi)\\
		% 		&\ +\sum_{\xi \in \Ic} \sum_{(\rho,\zeta)\in\Rc^{2}} \Phii_{\xi,\rho\zeta}(\yF)D_{\rho}v(\xi)D_{\zeta}v(\xi)\\
		% 		&\ +\int_{\OmeC}\ppGW(\nabla \yF) (\nabla v)^{2}\d x=\langle \delta^{2} \Erfl (\yF)v,v\rangle. 
		% 	\end{split}
	% \end{equation}

% Then we define the stability constants (for homogeneous deformations) $\gaaF, \garflF, \ganllF$ as 
% \begin{align}
	% 	\label{GammaF for a}  \gaaF &=\inf_{v\in \Ya} \frac{\langle \delta^{2} \Ea (\yF)v,v\rangle}{\Vert \nabla v \Vert_{L^{2}}^{2}},\\
	% 	\label{GammaF for rfl} \garflF &=\inf_{v\in \Ya} \frac{\langle \delta^{2} \Erfl (\yF)v,v\rangle}{\Vert \nabla v \Vert_{L^{2}}^{2}},\\
	% 	\label{GammaF for nll}\ganllF &=\inf_{v\in \Ya} \frac{\langle \delta^{2} \El (\yF)v,v\rangle}{\Vert \nabla v \Vert_{L^{2}}^{2}}.
	% \end{align}

% The reflection method was propose as a 'universally stable method' (\cite{2014_CO_AS_LZ_Stabilization_MMS},Theorem 4.3). Combining this result with \eqref{Stab of NL-L equals to Reflection} we obtain
% \begin{equation}\label{Stab constants of three methods result}
	% 	\gaaF = \garflF =\ganllF.
	% \end{equation}

% To make \eqref{Stab of NL-L} precise, we will 'split' the test function $v$ into an atomistic and continuum component, using the following lemma.[\cite{2013_ML_CO_AC_Coupling_ACTANUM},Lemma 7.3]

% \begin{lemma}\label{Pointwise blending lemma}
	% 	Let $\beta \in C^{1,1}(-\infty,\infty)$, with $0\le \beta \le 1$. For each $v\in \Ya$, there exists $\va, \vc \in \Ya$ such that
	% 	\begin{align}
		% 		\label{Pointwise va blending estimate}\vert \sqrt{1-\beta(\xi)} D_{\rho}v(\xi)-D_{\rho}\va(\xi) \vert &\le \vert \rho \vert^{\frac{3}{2}}\Vert \nabla \sqrt{1-\beta}\Vert_{L^{\infty}} \Vert \nabla v \Vert_{L^{2}(\conv(\xi,\xi+\rho))},\\ 
		% 		\label{Pointwise vc blending estimate1}\vert \sqrt{\beta(\xi)} D_{\rho}v(\xi)-D_{\rho}\vc(\xi) \vert &\le \vert \rho \vert^{\frac{3}{2}}\Vert \nabla \sqrt{\beta}\Vert_{L^{\infty}} \Vert \nabla v \Vert_{L^{2}(\conv(\xi,\xi+\rho))},\\ 
		% 		\label{va and vc} \vert \nabla\va\vert^{2} +	\vert \nabla\vc\vert^{2} &= \vert \nabla v \vert^{2}.
		% 	\end{align}
	% %where $C_{1},\ C_{2}$ may depends on $\rcut$, but $C_{3}$ is a generic constant. In particular, $\nabla \va(x)=0$ for $x\in(-\infty,-L]\cup[L,\infty)$ and $\nabla \vc(x)=0$ for $x \in [-K,K]$.
	% \end{lemma}

% \begin{proof}
	% 	Let $\psi(x):=\sqrt{1-\beta(x)}$ and assume, without loss of generality that $\rho>0$. Then,
	% 	\begin{align*}
		% 		\sqrt{1-\beta(\xi)} D_{\rho}v(\xi)&= \psi (\xi) \sum_{\eta = \xi}^{\xi+\rho-1}D_{1}V(\eta)\\
		% 		&=\sum_{\eta = \xi}^{\xi+\rho-1}\psi(\eta)D_{1}V(\eta) + \sum_{\eta = \xi}^{\xi+\rho-1} (\psi(\xi)-\psi(\eta))D_{1}V(\eta).\\
		% 	\end{align*}
	% If we define $\va$ by $D_{1}\va(\eta)=\psi(\eta)D_{1}v(\eta)$, then we obtain
	% \begin{equation*}
		% 	\sum_{\eta = \xi}^{\xi+\rho-1} \psi(\eta) D_{1}v(\eta) =D_{\rho} \va (\xi),
		% \end{equation*}
	% and after using Holder's inequality we know
	% \begin{align*}
		% 	\vert\sqrt{1-\beta} D_{\rho}v(\xi)-D_{\rho}\va(\xi)\vert&=\sum_{\eta = \xi}^{\xi+\rho-1}(\psi(\xi)-\psi(\eta))D_{1}v(\eta)\\
		% 	&\le (\sum_{\eta = \xi}^{\xi+\rho-1} \Vert \nabla \psi \Vert^{2}_{L^{\infty}} \vert \rho \vert^{2})^{\frac{1}{2}} (\sum_{\eta = \xi}^{\xi+\rho-1}(D_{1}v(\eta))^{2})^{\frac{1}{2}}\\
		% 	&\le \vert \rho \vert^{\frac{3}{2}} \Vert \nabla \psi \Vert_{L^{\infty}} \Vert \nabla v \Vert_{L^{2}(\xi,\xi+\rho)}.
		% \end{align*}
	% This establishes \eqref{Pointwise va blending estimate}. The proof of \eqref{Pointwise vc blending estimate1} is analogous, with $\vc$ defined by $D_{1}\vc(\xi) = \sqrt{\beta(\xi)}D_{1}v(\xi)$.
	
	% With these definitions \eqref{va and vc} is an immediate consequence.
	% \end{proof}

% \begin{theorem}\label{Stability}
	% 	Let $y\in\Ya$ satisfy the strong stability condition \eqref{Atomistic strong local minimizer} and suppose that there exists $\ganllF >0$ such that
	% 	\begin{equation}\label{Uniform solution of NL-L stab assumption}
		% 		\langle \delta^{2}\El (\yF)v,v\rangle \ge \ganllF \Vert \nabla v\Vert^{2}_{L^{2}(-N,N)} \text{ for all } v \in \Ya.
		% 	\end{equation}
	
	% Then
	% \begin{equation*}
		% 	\begin{split}
			% 		\langle \delta^{2}\El (y)v,v\rangle &\ge \min(c_{0},\ganllF)\Vert \nabla v \Vert_{L^{2}(-N,N)}^{2}\\
			% 		&\ - 2M^{(2,\frac{1}{2})}K^{-1} \Vert \nabla v \Vert_{L^{2}(-N,N)}^{2} - \CDH M^{(3,0)}K^{-\alpha} \Vert \nabla v \Vert_{L^{2}(-\bar{L},\bar{L})}^{2} \text{ for all } v \in \Ya.
			% 	\end{split}
		% \end{equation*}
	% \end{theorem}

% \begin{proof}
	% 	Let $K^{'}:=\lfloor K/2\rfloor <K$, and let
	% 	\begin{equation*}
		% 		\beta (x):=\left\{
		% 			\begin{aligned}
			% 				&0, &-K^{'}\le x \le K^{'},\\
			% 				&\hat{\beta}(\frac{x+K^{'}}{K^{'}-K}), &-K\le x \le -K^{'},\\
			% 				&\hat{\beta} (\frac{x-K^{'}}{K-K^{'}}), &K^{'} \le x \le K,\\
			% 				&1, &-N\le x \le -K \text{ or } K\le x\le N.
			% 			\end{aligned}
		% 		\right.
		% 	\end{equation*}
	% where $\hat{\beta}(s)=3s^{3}-2s^{2}$. We know that [\cite{2013_ML_CO_AC_Coupling_ACTANUM},Section 8.3]
	% \begin{equation}\label{Property of beta}
		% 	\Vert \nabla \sqrt{\beta} \Vert_{L^{\infty}} + \Vert \nabla \sqrt{1-\beta} \Vert_{L^{\infty}} \lesssim 	K^{-1}\yz{(C_{\beta} K^{-1}) }.
		% \end{equation}
	
	% 	We can now write \yz{May split it?}
	% 		\begin{align}
		% 			\langle \delta^{2} \El (y)v,v\rangle &= \sum_{\xi=-N}^{N} \sum_{(\rho,\zeta)\in\Rc^{2}} \Phia_{\xi,\rho\zeta} (y)(1-\beta(\xi)) D_{\rho}v(\xi) D_{\zeta}v(\xi)\\
		% 			&\ +\sum_{\xi=-N}^{N} \sum_{(\rho,\zeta)\in\Rc^{2}} \PhiNLL_{\xi,\rho\zeta} (y)(\beta(\xi)) D_{\rho}v(\xi) D_{\zeta}v(\xi)\\
		% 			 &= 
		% 			\label{All atomistic stab}\sum_{\xi=-N}^{N} \sum_{(\rho,\zeta)\in\Rc^{2}} \Phia_{\xi,\rho\zeta} (y)(1-\beta(\xi)) D_{\rho}v(\xi) D_{\zeta}v(\xi)\\
		% 			\label{NL-L atomistic stab} &\ +\sum_{\xi \in \Ac} \sum_{(\rho,\zeta)\in\Rc^{2}} \Phia_{\xi,\rho\zeta} (y)(\beta(\xi))D_{\rho}v(\xi)D_{\zeta}v(\xi)\\
		% 			\label{NL-L interaction stab} &\ +\sum_{\xi \in \Ic} \sum_{(\rho,\zeta)\in\Rc^{2}} \Phii_{\xi,\rho\zeta} (y)(\beta(\xi))D_{\rho}v(\xi)D_{\zeta}v(\xi)\\
		% 			\label{NL-L nonlinear stab} &\ + \int_{\OmeNL} \ppGW (\nabla y)(\beta (x)) (\nabla v)^{2}\d x\\
		% 			\label{NL-L linear stab} &\ +\int_{\OmeL}  \ppGWL (\nabla y)(\beta (x))(\nabla v)^{2}\d x.
		% 		\end{align}
	% 	where we also use the fact that according to our definition of $\beta$, the first sum ranges only over those sites where $\PhiNLL_{\xi} = \Phia_{\xi}$.
	
	% %	Let $\epsilon_{1} = \max (\Vert \nabla \sqrt{1-\beta}\Vert_{L^{\infty}} , \Vert \nabla \sqrt{\beta}\Vert_{L^{\infty}} )$. 
	% We apply estimate \eqref{Pointwise va blending estimate} to \eqref{All atomistic stab}, and we obtain
	% 	\begin{equation}\label{Atomistic va stab result}
		% 		\begin{split}
			% 			\sum_{\xi=-N}^{N} \sum_{(\rho,\zeta)\in\Rc^{2}} \Phia_{\xi,\rho\zeta} (y)(1-\beta(\xi)) D_{\rho}v(\xi) D_{\zeta}v(\xi) &\ge \langle \delta^{2} \Ea (y)\va,\va\rangle 
			% 			- 2M^{(2,\frac{1}{2})}K^{-1}\Vert \nabla v \Vert_{L^{2}(-N,N)}^{2}\\
			% 			&\ge c_{0} \Vert \nabla \va \Vert_{L^{2}(-N,N)}^{2}-2M^{(2,\frac{1}{2})}K^{-1}\Vert \nabla v \Vert_{L^{2}(-N,N)}^{2}.
			% 		\end{split}
		% 	\end{equation}
	
	% 	We apply the estimate \eqref{Pointwise vc blending estimate1} to \eqref{NL-L atomistic stab}, and we know
	% 	\begin{equation}\label{NL-L atomistic stab 1}
		% 		\begin{split}
			% 			\sum_{\xi \in \Ac} \sum_{(\rho,\zeta)\in\Rc^{2}} \Phia_{\xi,\rho\zeta} (y)(\beta(\xi))D_{\rho}v(\xi)D_{\zeta}v(\xi) &\ge \sum_{\xi \in \Ac} \sum_{(\rho,\zeta)\in\Rc^{2}} \Phia_{\xi,\rho\zeta} (y)D_{\rho}\vc(\xi)D_{\zeta}\vc(\xi) \\
			% 			&\ - 2M^{(2,\frac{1}{2})} K^{-1} \Vert \nabla v \Vert_{L^{2}(\OmeA)}^{2}.
			% 		\end{split}
		% 	\end{equation}
	
	% 	By the definition of $\vc$, we notice that for $x \in[-K^{'},K^{'}]$, $\nabla \vc = 0$.After using Taylor's expansion at $\yF$ and assumption $\DH$, we obtain
	% 	\begin{equation}\label{NL-L atomistic stab 2}
		% 		\begin{split}
			% 			\sum_{\xi \in \Ac} \sum_{(\rho,\zeta)\in\Rc^{2}} \Phia_{\xi,\rho\zeta} (y)D_{\rho}\vc(\xi)D_{\zeta}\vc(\xi)&\ge \sum_{\xi \in \Ac} \sum_{(\rho,\zeta)\in\Rc^{2}} \Phia_{\xi,\rho\zeta} (\yF)D_{\rho}\vc(\xi)D_{\zeta}\vc(\xi)\\
			% 			&\ -2^{\alpha}\CDH M^{(3,0)}  (K)^{-\alpha} \Vert\nabla \vc \Vert_{L^{2}(\OmeA)}^{2}.
			% 		\end{split}
		% 	\end{equation}
	
	% 	We combine \eqref{NL-L atomistic stab 1} with \eqref{NL-L atomistic stab 2}, and could get
	% 	\begin{equation}\label{NL-L atomistic stab result}
		% 		\begin{split}
			% 			\sum_{\xi \in \Ac} \sum_{(\rho,\zeta)\in\Rc^{2}} \Phia_{\xi,\rho\zeta} (y)(\beta(\xi))D_{\rho}v(\xi)D_{\zeta}v(\xi) &\ge \sum_{\xi \in \Ac} \sum_{(\rho,\zeta)\in\Rc^{2}} \Phia_{\xi,\rho\zeta} (\yF)D_{\rho}\vc(\xi)D_{\zeta}\vc(\xi) \\
			% 			&\ - 2M^{(2,\frac{1}{2})} K^{-1} \Vert \nabla v \Vert_{L^{2}(\OmeA)}^{2}-2^{\alpha}\CDH M^{(3,0)}  K^{-\alpha}\Vert\nabla \vc \Vert_{L^{2}(\OmeA)}^{2}.
			% 		\end{split}
		% 	\end{equation}
	
	% 	It's a similarly proof for \eqref{NL-L interaction stab}. And we get that
	% 	\begin{equation}\label{NL-L interaction stab result}
		% 		\begin{split}
			% 			\sum_{\xi \in \Ic} \sum_{(\rho,\zeta)\in\Rc^{2}} \Phii_{\xi,\rho\zeta} (y)(\beta(\xi))D_{\rho}v(\xi)D_{\zeta}v(\xi) &\ge \sum_{\xi \in \Ic} \sum_{(\rho,\zeta)\in\Rc^{2}} \Phii_{\xi,\rho\zeta} (\yF)D_{\rho}\vc(\xi)D_{\zeta}\vc(\xi) \\
			% 			&\ - M^{(2,\frac{1}{2})} K^{-1} \Vert \nabla v \Vert_{L^{2}(\OmeI)}^{2}-\CDH M^{(3,0)}  K^{-\alpha} \Vert\nabla \vc \Vert_{L^{2}(\OmeI)}^{2}.
			% 		\end{split}
		% 	\end{equation}
	
	% 	Then we focus on \eqref{NL-L nonlinear stab}. After considering the definition of $\vc$ and assumption $\DH$, we will get
	% 	\begin{equation}\label{NL-L nonlinear stab result}
		% 				\int_{\OmeNL} \ppGW (\nabla y)(\beta(x))(\nabla v)^{2}\d x \ge 	\int_{\OmeNL} \ppGW (\nabla \yF)(\nabla\vc)^{2}\d x - 2\CDH M^{(3,0)}  K^{-\alpha} \Vert \nabla \vc \Vert_{L^{2}(\OmeNL)}^{2}.
		% 			\end{equation}
	
	
	% 	We use the fact $\ppGWL (\nabla y) =\Wppf = \ppGWL (\nabla \yF)$ again, and obtain
	% 	\begin{equation}\label{NL-L linear stab result}
		% 		\int_{\OmeL}\ppGWL (\nabla y)(\beta(x)) (\nabla v)^{2} \d x = \int_{\OmeL}\ppGWL (\nabla \yF) (\nabla \vc)^{2} \d x.
		% 	\end{equation}
	
	% 	\yz{We consider 3-order linear function
		% 	\begin{align*}
			% 		\int_{\OmeL}\ppGWL (\nabla y)(\beta(x)) (\nabla v)^{2} \d x &\ge \int_{\OmeL}\ppGWL (\nabla \yF) (\nabla \vc)^{2} \d x\\
			% 		&\ - \CDH M^{(3,0)}  \bar{L}^{-\alpha} \Vert\nabla \vc \Vert_{L^{2}(\OmeL)}^{2}.
			% 	\end{align*}
		
		
		
		
		% }
	
	
	
	% 	Now we consider definition of $\langle \delta^{2} \El (\yF) \vc,\vc \rangle$, and  use property \eqref{va and vc} and \eqref{Property of beta} to conclude that \yz{ How to say "add \eqref{NL-L atomistic stab result} , \eqref{NL-L interaction stab result}, \eqref{NL-L nonlinear stab result} and \eqref{NL-L linear stab result} together"}
	% 	\begin{align*}
		% 		\langle \delta^{2} \El (y)v,v\rangle &\ge \langle \delta^{2} \El (\yF) \vc,\vc \rangle + \langle \delta^{2}\Ea(y)\va,\va \rangle  \yz{(\min(c_{0},\ganllF)\Vert \nabla v \Vert_{L^{2}(-N,N)}^{2}) }\\
		% 		&\ \yz{- 4M^{(2,\frac{1}{2})}K^{-1} \Vert \nabla v \Vert_{L^{2}(-N,N)}^{2} - 2^{\alpha}\CDH M^{(3,0)}K^{-\alpha} \Vert \nabla v \Vert_{L^{2}(-\bar{L},\bar{L})}^{2} }
		% 	\end{align*}
	
	% \end{proof}



\subsection{A priori existence and error estimate}
\label{sec: priori_anal_qnll_cg}

Next, we consider the a priori error estimation between the coarse-grained QNLL model and the atomistic model. Based on the inverse function theorem and Theorem~\ref{Internal forces of continuum region}, we can obtain the following result:

\begin{theorem}
	Let $\yai \in \Ya$ be a strongly stable atomistic solution satisfying \eqref{All-Atomistic strong local minimizer} and $\DH$. Consider the QNLL problem \eqref{Yh solution} with quasi-optimal choice of $N, \Th$. Suppose, moreover, that $\El$ is stable in the reference state Theorem \ref{Stability}. Then, there exists $K_0$ such that, for all $K \ge K_0$, \eqref{Yh solution} has a locally unique, strongly stable solution $y^{\text{NL-L}}_{h}$ which satisfies
	\begin{equation}
		\begin{aligned}
			\Vert \nabla\yai - \nabla y^{\text{NL-L}}_{h} \Vert_{L^2} \lesssim 8&M^{(3,0)}(\Vert \nabla^{2} u\Vert_{L^{2}(\bOmeI)} +\Vert \nabla^{3}u \Vert_{L^{2}(\bOmeC)}+\Vert \nabla^{2}u \Vert^{2}_{L^{4}(\bOmeC)}\\
			&+ \Vert \nabla u \Vert^{2}_{L^{4}(\OmeL)}+\Vert h \nabla^{2} u\Vert_{L^{2}(\OmeC)}+N^{\frac{1}{2}-\alpha})/\big(\min(c_{0},\ganllF)\big)^2.
		\end{aligned}
	\end{equation}
	
\end{theorem}
\begin{proof}
	
	The proof process here is similar to Theorem \ref{Priori of NCG}, with the difference being the inclusion of coarse-grained error in $\eta$ during the application of the inverse function theorem.
	
	%          We will use the quantitative inverse function theorem, with
	%          \begin{equation*}
		%          \Ghc(\Ph \ya):=\delta \El(\Ph \ya) - \langle f,\cdot\rangle^{\text{L}}_{h}.
		%          \end{equation*}
	
	%              We first apply that the scaling condition implies a Lipschitz bound for $\delta^{2}\El$, 
	%          \begin{equation}\label{scaling assumption}
		%              \Vert \delta^2 \El (y) - \delta^2 \El(v)\Vert_{\mathcal{L}(\Ya,\Ya^{*})}\le M \Vert \nabla y-\nabla v\Vert_{L^{\infty}} \quad \text{for all } y,v \in \Ya,
		%          \end{equation}
	%          where $M \lesssim M^{(3,0)}$. Since $\Vert \cdot \Vert \lesssim \Vert \cdot \Vert_{L^2}$, we can also replace the $L^\infty$- norm on the right-hand side with the $L^2$-norm.
	
	%          The residual estimate \eqref{Consistency result}, \eqref{Decay reslut of best approximation term} and the external residual estimate of Theorem \ref{Decay result of linear external force} give
	%          \begin{align*}
		%              \Vert \Ghc(\Ph \ya)\Vert_{\Uh}\lesssim 
		%               M^{(2,1)}\Vert \nabla^{2} u\Vert_{L^{2}(\bOmeI)} +M^{(2,2)}\Vert \nabla^{3}u \Vert_{L^{2}(\bOmeC)}+M^{(3,2)}\Vert \nabla^{2}u \Vert^{2}_{L^{4}(\bOmeC)}\\
		%         &+ M^{(3,0)}\Vert \nabla u \Vert^{2}_{L^{4}(\OmeL)}
		%          \end{align*}
	
	
	% From Theorem \ref{Stability} we obtain that
	% \begin{equation*}
		% 		\langle \delta^{2}\El (\ya)v_{h},v_{h}\rangle \ge (\min(c_{0},\ganllF)-CK^{-\min(1,\alpha)})\Vert \nabla v_{h} \Vert_{L^{2}}^{2}.
		% \end{equation*}
	
	% Let $\gamma:=\frac{1}{2}\min(c_{0},\ganllF)$. Applying the Lipschitz bound \eqref{scaling assumption} and the best approximation error estimate \eqref{Decay reslut of best approximation term}, we obtain
	% \begin{equation*}
		%     \langle \delta^{2}\El (\Ph\ya)v_{h},v_{h}\rangle \ge (2\gamma-CK^{-\min(1,\alpha)}-CK^{-1/2-\alpha})\Vert \nabla v_{h} \Vert_{L^{2}}^{2}.
		% \end{equation*}
	
	% Hence, for $K$ sufficiently large, we obtain that
	% \begin{equation*}
		%     \langle \delta\Ghc (\Ph\ya)v_{h},v_{h}\rangle \ge \gamma\Vert \nabla v_{h} \Vert_{L^{2}}^{2} \quad \text{for all }v_{h}\in\Uh.
		% \end{equation*}
	
	% Thus, we deduce the existence of $\ynll$ satisfying $\Ghc(\ynll)=0$. The error estimate implies
	% \begin{equation*}
		% \begin{aligned}
			%     \Vert \nabla \ynll - \nabla \Ph\ya\Vert_{L^2}&\lesssim \frac{2M\eta}{\gamma^2} \\
			%     &\lesssim  2M^{(3,0)}(\Vert \nabla^{2} u\Vert_{L^{2}(\bOmeI)} +\Vert \nabla^{3}u \Vert_{L^{2}(\bOmeC)}+\Vert \nabla^{2}u \Vert^{2}_{L^{4}(\bOmeC)}
			%         + \Vert \nabla u \Vert^{2}_{L^{4}(\OmeL)})/\gamma^2.
			% \end{aligned}
		% \end{equation*}
	
	% Applying the best approximation error estimate and $\DH$
	% \begin{equation*}
		%     \Vert \nabla\Ph\ya - \nabla\ya \Vert_{L^2}\le \Vert h \nabla^{2} u\Vert_{L^{2}(\OmeL)}+N^{\frac{1}{2}-\alpha} ,
		% \end{equation*}
	% \yz{which follows from \eqref{Decay reslut of best approximation term}, we finally obtain the stated error bound}.
	
	%where $\Pih$ is an approximation operator, we define
	%\begin{equation*}
	%	\Pih y(x):=\left\{
	%	\begin{aligned}
		%		&\Ih y(x), &\ x\in [0,\bar{L}],\\
		%		&\Ih y(x) -\frac{x-\bar{L}}{N-\bar{L}}u(N), &\ x\in [\bar{L},N].
		%	\end{aligned}
	%\right.
	%\end{equation*}
\end{proof}

\subsection{Discussion of the (quasi-)optimal choice of the length of the nonlinear and linear region}
\label{Balance of QNLL CG model}

% \chw{Give an a priori analysis of how to choose the length of the nonlinear and linear region. The conclusion may be that we only need to have a very short nonlinear region.}

In this subsection, we will discuss how to achieve the quasi-optimal choice of the lengths for finite element mesh $h$, nonlinear continuum region, linear continuum region, and computational region to obtain quasi-optimal convergence order for the QNLL model. We observe that due to this error balance, we only need a very short nonlinear region, this is the key motivation that we introduce this coupling methods to gain the same accuracy but a much more efficient method.

\subsubsection{Optimizing the finite element grid}
\label{sec: choice_of_fem_cg}
The finite mesh size $h$ is the first approximation parameter that we will optimize. In this section, we use a classical technique to optimize the mesh grading.

For each $x\in [-N,N], \ x\in \text{int} T$, let $h(x):=h_{T}$. For $x<-N$ or $x>N$, let $h(x):=1$. The coarse-grained error occurring in the coarsening analysis that depends on $\ThNL$ are the interpolation error term $\Vert h \nabla^{2}u\Vert_{L^{2}(\OmeC)}$. Suppose that $u\in\UhNL$ satisfies $\DH$ and $L >r_{0}$. Then
\begin{equation*}
	\Vert h \nabla^{2}u\Vert_{L^{2}(\OmeC)} \lesssim \Vert h x^{-\alpha-1}\Vert_{L^{2}(\OmeC)}.
\end{equation*}
We wish to choose $h$ to minimize this quantity, subject to fixing the number of degrees of freedom $\NhNL$, which is given by
\begin{equation*}
	\NhNL=\sum_{j=1}^{\NhNL}1=\sum_{j=1}^{\NhNL} h_{j}\frac{1}{h_{j}}=\int_{-N}^{N}\frac{1}{h} \,\d x.
\end{equation*}
We ignore the discreteness of the mesh size function and solve
\begin{equation*}
	\min \Vert h x^{-\alpha-1}\Vert_{L^{2}(\OmeC)} \quad \text{subject to }\int_{-N}^{N} \frac{1}{h}\,\d x = \text{const}.
\end{equation*}
The solution to this variational problem satisfies
\begin{equation*}
	h(x)=\lambda\vert x\vert^{\frac{2}{3}(\alpha+1)} \ \text{for }x\in \OmeC.
\end{equation*}
for some constant $\lambda>0$. This gives us an optimal scaling of the mesh size function.

We now impose the condition $h(L)\approx 1$, which yields
\begin{equation}\label{Mesh size fucntion of nonlinear h}
	h^{\text{NL}}(x)\approx(\frac{\vert x\vert}{L})^{\frac{2}{3}(\alpha+1)}=:\tilde{h}(x) \ \text{for }x\in\OmeC.
\end{equation}

If $\alpha':=\frac{2}{3}(\alpha+1)$, then $\alpha'>1$, and hence this choice of $h$ yields(for simplify, we only calculate the domain$[\bar{K},N]$)
\begin{equation}\label{int of nonlinear h}
	\int_{\bar{K}}^{N} \frac{1}{h} \,\d x \approx \frac{\bar{K}^{\alpha'}(N^{1-\alpha'}-\bar{K}^{1-\alpha'})}{1-\alpha'}\approx \frac{\bar{K}}{\alpha'-1}.
\end{equation}

Thus, the choice \eqref{Mesh size fucntion of nonlinear h} gives a comparable number of degrees of freedom in the linear region to that in the atomistic, interface and continuum regions. The resulting interpolation error bound can be estimated by
\begin{equation}\label{Decay of the nonlinear best approximation term}
	\Vert h x^{-\alpha-1}\Vert_{L^{2}(\OmeC)} \approx \frac{\bar{K}^{\frac{1}{2}-(\alpha+1)}}{(\alpha'-1)^{\frac{1}{2}}} \approx \bar{K}^{-\frac{1}{2}-\alpha}.
\end{equation}



\subsubsection{The quasi-optimal choice of $N$}
\label{sec: choice_of_N_cg}

Before introducing specific finite element mesh generation algorithm, we need to discuss how to determine the length of our computational domain $N$. We follow two principles:
\begin{enumerate}
	\item We should ensure that the truncation error term $N^{\frac{1}{2}-\alpha}$ do not dominate among the various types of errors after balancing the length of the computational domain;
	
	\item We choose the length of the computational domain as small as possible for computational simplicity.
\end{enumerate}

According to the first principle mentioned above, we understand that the truncation error $N^{\frac{1}{2}-\alpha}$ must be balanced against one of the terms of modelling error or coarsening error (or higher-order terms). According to the second principle, to select the computational domain length as small as possible, we must balance it against the lowest-order term of modelling error or coarsening error (balancing against higher-order terms would need a longer computational domain length).


For $\frac{1}{2}< \alpha< 1$, the lowest-order term is the linearization error $\Vert \nabla u \Vert^{2}_{L^{4}(\OmeL)}\lesssim L^{\frac{1}{2}-2\alpha}$, we should choose $N$ such that
\begin{equation*}
	L^{\frac{1}{2}-2\alpha}\approx N^{\frac{1}{2}-\alpha}, \quad \text{that is}, \ N\approx L^{\frac{2\alpha-1/2}{\alpha-1/2}}.
\end{equation*}

For $ \alpha\ge 1$, the lowest-order term is the coarse-grained error $\Vert h\nabla^2 u \Vert_{L^{2}(\OmeC)}\lesssim \bar{K}^{-\frac{1}{2}-\alpha}$, we choose $N$ such that
\begin{equation*}
	\bar{K}^{-\frac{1}{2}-\alpha}\approx N^{\frac{1}{2}-\alpha}, \quad \text{that is}, \ N\approx \bar{K}^{\frac{\alpha+1/2}{\alpha-1/2}}.
\end{equation*}

We now turn this formal motivation into an explicit construction of the finite element mesh. 
% \begin{algorithm}[H]
	% \caption{Adaptive QM/MM algorithm}
	% \label{alg:main}
	
	% {\bf Prescribe} $\LQM, \LMM$, termination tolerance $\eta_{\rm tol}$, refinement tolerance $\tau_{\rm D}$.
	
	% \begin{algorithmic}[1]
		% \REPEAT
		% 	\STATE{ \textit{Solve}: Solve \eqref{problem-e-mix} to obtain $\uH$. }
		% 	\STATE{  \textit{Estimate}: Apply Algorithm \ref{alg:adaptMesh} to compute $\eta_\h(\uH)$ and $\eta_{\h,T}$ (cf. \eqref{eq:APET}, \eqref{eq:rho}). } 		
		% 	\STATE{\textit{Mark}: Use D\"{o}rfler strategy with $\tau_{D}$ to mark {\color{blue} elements in $\Th$} for {\color{blue}refinement}.}
		% 	\STATE{ \textit{Refine:} Construct new $\LQM$ and $\LMM$ regions.}
		% \UNTIL{$\eta_\h(\uH) < \eta_{\rm tol}$}
		% \end{algorithmic}
	% \end{algorithm}

\begin{algorithm}[H]
	\caption{Finite element mesh construction algorithm}
	\label{alg:FEM}
	\begin{enumerate}
		\item[Step 1]: Set $N:= \lceil \bar{K}^{\frac{\alpha+1/2}{\alpha-1/2}}\rceil$($\alpha>1$); $N:= \lceil \bar{K}^{\frac{2\alpha-1/2}{\alpha-1/2}}\rceil$($\frac{1}{2}< \alpha< 1$). 
		\item[Step 2]: Set $\NhNL :=\{0, 1, \dots, \bar{K}\}$.  		
		\item[Step 3]: While $n:=\max (\NhNL) <L$:
		\begin{enumerate}
			\item[Step 3.1]: Set $\NhNL:=\NhNL \cup \{\min(L,n+\lfloor \tilde{h}(n)\rfloor ) \}$.
		\end{enumerate}
		\item[Step 4]: While $n:=\max (\NhNL) <N$:
		\begin{enumerate}
			\item[Step 4.1]: Set $\NhNL:=\NhNL \cup \{\min(N,n+\lfloor \tilde{h}(n)\rfloor ) \}$.
		\end{enumerate}
		\item[Step 5]: Set $\NhNL = (-\NhNL) \cup \NhNL$.
	\end{enumerate}
\end{algorithm}



%\textbf{FEM Algorithm:}
%\begin{enumerate}
%	\item Set $N:= \lceil L^{\frac{2\alpha-1/2}{\alpha-1/2}}\rceil$($\frac{1}{2}< \alpha< 1$); $N:= \lceil \bar{K}^{\frac{\alpha+1/2}{\alpha-1/2}}\rceil$($\alpha\ge1$).
%	\item Set $\NhNL :=\{-\bar{K},\dots,\bar{K}\}$.
%	\item While $n:=\max \NhNL <N$

%	Step 1: Set $\NhNL:=\NhNL \cup \{\min(L,n+\lfloor \tilde{h}(n)\rfloor ) \}(0 \le n\le L)$.

%	Step 2: Set $\NhNL:=\NhNL \cup \{\max(-L,n-\lfloor \tilde{h}(n)\rfloor ) \}(-L\le n\le 0)$.

%	Step 3: Set $\NhNL:=\NhNL \cup \{\min(N,n+\lfloor \tilde{h}(n)\rfloor ) \}(L\le n \le N)$.

%	Step 4: Set $\NhNL:=\NhNL \cup \{\max(-N,n-\lfloor \tilde{h}(n)\rfloor ) \}(-N\le n \le-L)$.
%\end{enumerate}

Meshes construct via this algorithm qualitatively the same properties as predicted by the formal computations \eqref{Mesh size fucntion of nonlinear h} and \eqref{int of nonlinear h}.
\begin{theorem}\label{Decay result of nonlinear best approximation term}
	Let $u\in \Ua$ satisfy $\DH$ and let $N$ and $\ThNL$ be constructed by Algorithm \ref{alg:FEM}. Then, for $L$ sufficiently large, $\NhNL\le C_{1} L$,
	\begin{equation*}
		\begin{aligned}
			\Vert h\nabla^{2} u\Vert_{L^{2}(\OmeC)} &\le C_{2} \NhNL^{-\frac{1}{2}-\alpha},  \\
			\Vert h\nabla^{2} u\Vert^{2}_{L^{4}(\OmeNL)} &\le C_{3} \NhNL^{-\frac{1}{2}-2\alpha},
		\end{aligned}
	\end{equation*}
	where $C_{1}$ depends on $\alpha$ and $C_{2}, C_{3}$ depends on $\alpha$ and on $\CDH$.
\end{theorem}


We now turn the external consistency error estimate into an estimate in terms of $\NhNL$ as well. Let $f$ satisfy $\DH$ and suppose that $\ThNL$ and $N$ are constructed using Algorithm \ref{alg:FEM}. Since $h(x)\le \frac{x}{2}, \ \omega(x)=x\log x$, a straightforward computation yields
\begin{align*}
	\Vert \eta_{\text{ext}} \Vert_{(\YhNL)^{*}} &\lesssim \Vert h^{2} \nabla f\Vert_{L^{2}(\tOmeC)} +\frac{C_{\kappa}}{\log \bar{K}} \Vert h^{2}\omega \nabla^{2}f\Vert_{L^{2}(\tOmeC)}\\
	&\lesssim \bar{K}^{-\alpha-\frac{3}{2}}+\frac{\bar{K}^{-\alpha-\frac{3}{2}}\log^{2}N}{\log \bar{K}}.
\end{align*}

We insert $N \lesssim \bar{K}^{\frac{2}{3}(\alpha+1)}$ to obtain the following result. In particular, we can conclude that the external consistency error is dominated by the interpolation error.

\begin{theorem}\label{Decay result of nonlinear external force}
	Let $f$ satisfy $\DH$ and let $\ThNL$, $N$ be constructed by Algorithm \ref{alg:FEM}. Then
	\begin{equation*}
		\Vert \eta_{\text{ext}} \Vert_{(\YhNL)^{*}} \le C_{\alpha} \bar{K}^{-\alpha-\frac{3}{2}} \log \bar{K}.
	\end{equation*}
	where $C_{\alpha}$ depends on $\alpha$ and on $\CDH$.
\end{theorem}


% \subsubsection{Optimizing the linear finite element grid}
% The finite mesh size $h$ is the first approximation parameter that we will optimize. In this section, we use a classical technique to optimize the mesh grading.

% The two terms occuring in the coarsening analysis tht depends on $\Th$ are the best approximation error terms
% \begin{equation*}
	% 	\Vert h \nabla^{2}u\Vert_{L^{2}(\OmeL)}+N^{\frac{1}{2}-\alpha}.
	% \end{equation*}
% It is easy to see that, for $\bar{L}$ sufficiently largem the best approximation error $\Vert h \nabla^{2}u\Vert_{L^{2}(\OmeL)}$ is the dominant contribution. Thus, we optimize this term only.

% Suppose that $u\in\U$ satisfies $\DH$ and $\bar{L} >r_{0}$. Then
% \begin{equation*}
	% 	\Vert h \nabla^{2}u\Vert_{L^{2}(\OmeL)} \lesssim \Vert h x^{-\alpha-1}\Vert_{L^{2}(\OmeL)}.
	% \end{equation*}
% We wish to choose $h$ to minimize this quantity, subject to fixing the number of degrees of freedom, $N_{\Th}-2$, which is given by
% \begin{equation*}
	% 	N_{\Th}=\sum_{j=1}^{N_{\Th}}1=\sum_{j=1}^{N_{\Th}} h_{j}\frac{1}{h_{j}}=\int_{-N}^{N}\frac{1}{h}\,\d x.
	% \end{equation*}
% We ignore the discreteness of the mesh size function and solve
% \begin{equation*}
	% 	\min \Vert h x^{-\alpha-1}\Vert_{L^{2}(\OmeL)} \quad \text{subject to }\int_{-N}^{N} \frac{1}{h}\,\d x = \text{const}.
	% \end{equation*}
% The solution to this variational problem satisfies
% \begin{equation*}
	% 	h^{\text{L}}(x)=\lambda\vert x\vert^{\frac{2}{3}(\alpha+1)} \quad \text{for }x\in \OmeL.
	% \end{equation*}
% for some constant $\lambda>0$. This gives us an optimal scaling of the mesh size function.

% We now impose the condition $h^{\text{L}}(\bar{L})\approx 1$, which yields
% \begin{equation}\label{Mesh size fucntion of linear h}
	% 	h^{\text{L}}(x)\approx(\frac{\vert x\vert}{\bar{L}})^{\frac{2}{3}(\alpha+1)}=:\tilde{h}^{\text{L}}(x) \ \text{for }x\in\OmeL.
	% \end{equation}
% If $\alpha^{'}:=\frac{2}{3}(\alpha+1)$, then $\alpha^{'}>1$, and hence this choice of $h$ yields (for simplicity, we only calculate the domain$[\bar{L},N]$)
% \begin{equation}\label{int of linear h}
	% 	\int_{\bar{L}}^{N} \frac{1}{h} \d x \approx \frac{\bar{L}^{\alpha’}(N^{1-\alpha‘}-\bar{L}^{1-\alpha‘})}{1-\alpha’}\approx \frac{\bar{L}}{\alpha^{'}-1}.
	% \end{equation}
% Thus, the choice \eqref{Mesh size fucntion of linear h} gives a comparable number of degrees of freedom in the linear region to that in the atomistic, interface and continuum regions. The resulting best approximation error bound can be estimated by
% \begin{equation}\label{Decay of the linear best approximation term}
	% 	\Vert h x^{-\alpha-1}\Vert_{L^{2}(\OmeL)} \approx \frac{\bar{L}^{\frac{1}{2}-(\alpha+1)}}{(\alpha^{'}-1)^{\frac{1}{2}}} \approx \bar{L}^{-\frac{1}{2}-\alpha}.
	% \end{equation}

% We can now also determine an optimal balance between the choices for $\bar{L},N$ and $\Th$. To balance the interpolation error with the far-field error. For $\alpha\ge 1$, we should choose $N$ such that
% \begin{equation*}
	% 	L^{-\frac{1}{2}-\alpha}\approx N^{\frac{1}{2}-\alpha}, \quad \text{that is}, \ N\approx L^{\frac{\alpha+1/2}{\alpha-1/2}}.
	% \end{equation*}

% For $\frac{1}{2}< \alpha< 1$, we choose $N$ such that
% \begin{equation*}
	% 	L^{\frac{1}{2}-2\alpha}\approx N^{\frac{1}{2}-\alpha}, \quad \text{that is}, \ N\approx L^{\frac{2\alpha-1/2}{\alpha-1/2}}.
	% \end{equation*}

% We now turn this formal motivation into an explicit construction of a finite element mesh. To make the linear region beginning node $-\bar{L},\bar{L}$ in our finite element nodes, we make this

% \textbf{Algorithm T:}
% \begin{enumerate}
	% 	\item Set $N:= \lceil L^{\frac{\alpha+1/2}{\alpha-1/2}}\rceil$($\alpha>1$); $N:= \lceil L^{\frac{2\alpha-1/2}{\alpha-1/2}}\rceil$($\frac{1}{2}< \alpha< 1$).
	% 	\item Set $\Nh :=\{-\bar{L},\dots,\bar{L}\}$.
	% 	\item While $n:=\max \Nh <N$
	
	% 	Step 1: Set $\NhNL:=\NhNL \cup \{\min(N,n+\lfloor \tilde{h}(n)\rfloor ) \}(0\le n \le N)$.
	
	% 	Step 2:Set $\NhNL:=\NhNL \cup \{\max(-N,n-\lfloor \tilde{h}(n)\rfloor ) \}(-N\le n \le0)$.
	% \end{enumerate}

% Meshes constructed via this algorithm qualitatively the same properties as predicted by the formal computations (\ref{Mesh size fucntion of linear h}) and \eqref{int of linear h}.
% \begin{theorem}\label{Decay result of linear best approximation term}
	% 	Let $u\in \Ua$ satisfy $\DH$ and let $N$ and $\Th$ be constructed by algorithm. Then, for $\bar{L}$ sufficiently large, $N_{\Th}\le C_{1}^{\text{L}}L$,
	% 	\begin{equation}\label{Decay reslut of best approximation term}
		% 		\Vert h\nabla^{2} u\Vert_{L^{2}(\OmeL)} \le C_{2}^{\text{L}} N_{\Th}^{-\frac{1}{2}-\alpha}.
		% 	\end{equation}
	% 	where $C_{1}^{\text{L}}$ depends on $\alpha$ and $C_{2}^{\text{L}}$ depends on $\alpha$ and on $\CDH$.
	% \end{theorem}
% \begin{proof}
	% 	Already check.
	% \end{proof}

% We now turn the external consistency error estimate into an estimate in terms of $N_{\Th}$ as well. Let $f$ satisfy $\DH$ and suppose that $\Th$ and $N$ are constructed using Algorithm. Since $h(x)\le \frac{x}{2}, \ \omega(x) = x\log x$, a straightforward computation yields
% \begin{align*}
	% 	\Vert \eta_{\text{ext}} \Vert_{(\Uh)^{*}} &\lesssim \Vert h^{2} \nabla f\Vert_{L^{2}(\tOmeL)} +\frac{C_{\kappa}}{\log \bar{L}} \Vert h^{2}\omega \nabla^{2}f\Vert_{L^{2}(\tOmeL)}\\
	% 	&\lesssim \bar{L}^{-\alpha-\frac{3}{2}}+\frac{\bar{L}^{-\alpha-\frac{3}{2}}\log^{2}N}{\log \bar{L}}.
	% \end{align*}

% We insert $N \lesssim \bar{L}^{\frac{2}{3}(\alpha+1)}$ to obtain the following result. In particular, we can conclude that the external consistency error is dominated by the best approximation error.

% \begin{theorem}\label{Decay result of linear external force}
	% 	Let $f$ satisfy $\DH$ and let $\Th,N$ be chosen by algorithm. Then
	% 	\begin{equation*}
		% 		\Vert \eta_{\text{ext}} \Vert_{(\Uh)^{*}} \le C \bar{L}^{-\alpha-\frac{3}{2}} \log \bar{L}.
		% 	\end{equation*}
	% 	where $C$ depends on $\alpha$ and on $\CDH$.
	% \end{theorem}


\subsubsection{The quasi-optimal choice of $L$}
\label{sec: choice_of_L_cg}
We consider the nonlinear-linear elasticity coupling method quasi-optimal choice of approximation parameters $ \bar{K}, L, \NhNL$, for given $K$. For the reason we choose $\rcut=2$, we fix $\bar{K}=K+2$. So we wanna balance $\bar{K}, L$ and $\NhNL$. And the key idea is how to choose $L$ to balance the lowest  order term between $\bar{K},L$.

Firstly, we use $\DH$ assumption, and we could obtain the decay result of coupling error
\begin{equation*}
	\begin{split}
		&M^{(2,1)}\Vert \nabla^{2} u\Vert_{L^{2}(\bOmeI)} +M^{(2,2)}\Vert \nabla^{3}u \Vert_{L^{2}(\bOmeC)}+M^{(3,2)}\Vert \nabla^{2}u \Vert^{2}_{L^{4}(\bOmeC)} + M^{(3,0)}\Vert \nabla u \Vert^{2}_{L^{4}(\OmeL)}\\
		&\lesssim \CDH M^{(2,1)}K^{-\alpha -1}+\CDH M^{(2,2)}\bar{K}^{-\alpha-\frac{3}{2}} +\CDH^{2}M^{(3,2)}\bar{K}^{-2\alpha -\frac{3}{2}}+\CDH^{2}M^{(3,0)}L^{-2\alpha+\frac{1}{2}} .
	\end{split}
\end{equation*}

For coarsening error, we have assumption $\NhNL \lesssim \bar{K}$. From Theorem \ref{Decay result of nonlinear best approximation term} and Theorem \ref{Decay result of nonlinear external force}, we know the interpolation error $\Vert h\nabla^{2}u\Vert_{L^{2}(\OmeC)}$ is the dominant contribution
\begin{equation*}
	\Vert h\nabla^{2} u\Vert_{L^{2}(\OmeC)} \le C_{2}^{\text{NL}} \NhNL^{-\frac{1}{2}-\alpha}.
\end{equation*}

The lowest-order term of $L$ is $\Vert \nabla u \Vert^{2}_{L^{4}(\OmeL)} \lesssim L^{-2\alpha+\frac{1}{2}}$. We balance this term with $	\Vert h\nabla^{2} u\Vert_{L^{2}(\OmeC)} \lesssim \bar{K}^{-\frac{1}{2}-\alpha}$, and get (by noticing the fact that $\bar{K}\le L$)
\begin{align}
	L &\lesssim \bar{K}^{\frac{1}{2}+\frac{3}{8\alpha-2}} \qquad (\frac{1}{2}<\alpha<1) \label{Balance of L CG 1},\\
	L &\approx \bar{K} \qquad \qquad \quad (\alpha \ge 1)\label{Balance of L CG 2}.
\end{align}




% \subsubsection{The quasi-optimal choice of $\bar{L}$ (Linear region)}
% We consider the nonlinear-linear reflection method quasi-optimal choice of approximation parameters $ L, \bar{L}, \Nh$, for given $K$. For the reason we choose $\rcut=2$, we fix $L=K+2$. So we wanna balance $L, \bar{L}$ and $\Nh$. And the key idea is how to choose $\bar{L}$ to balance the lowest  order term between $L,\bar{L}$.

% For coarsening error, we have assumption $\Nh \lesssim \bar{L}$. From Theorem \ref{Decay result of linear best approximation term} and Theorem \ref{Decay result of linear external force}, we know the best approximation error $\Vert h\nabla^{2}u\Vert_{L^{2}(\OmeL)}$ is the dominant contribution.
% \begin{equation}
	% 	\Vert h\nabla^{2} u\Vert_{L^{2}(\OmeL)} \le C_{2}^{\text{L}} N_{\Th}^{-\frac{1}{2}-\alpha}.
	% \end{equation}

% When $\frac{1}{2}<\alpha<1$, the lowest order term of $\bar{L}$ is $\Vert \nabla u \Vert^{2}_{L^{4}(\OmeL)} \lesssim(\bar{L})^{-2\alpha+\frac{1}{2}}$. We balance this term with $	\Vert \nabla^{2} u\Vert_{L^{2}(\bOmeI)} \lesssim L^{-\alpha-1}$, and could get
% \begin{equation}\label{Balance result of linear coarsening error1}
	% 	\bar{L} \sim L^{\frac{1}{2}+\frac{5}{8\alpha-2}}.
	% \end{equation}

% When $\alpha>1$, the lowest order term of $\bar{L}$ is $\Vert  h \nabla^{2}u \Vert_{L^{2}(\OmeL)} \lesssim(\bar{L})^{-\alpha-\frac{1}{2}}$. We balance this term with $	\Vert \nabla^{2} u\Vert_{L^{2}(\bOmeI)} \lesssim L^{-\alpha-1}$, and could get
% \begin{equation}\label{Balance result of linear coarsening error2}
	% 	\bar{L} \sim L^{1+\frac{1}{2\alpha+1}}.
	% \end{equation}

% \yz{
	% In previous sections, we calculate the coupling error and linearization error. Next we consider the $\DH$ assumption, and could get
	
	% \begin{equation}\label{Consistency result}
		% 	\begin{split}
			% 	\Vert \eta^{\text{NL-L}}_{int}\Vert \le &M^{(2,1)}\Vert \nabla^{2} u\Vert_{L^{2}(\bOmeI)} +M^{(2,2)}\Vert \nabla^{3}u \Vert_{L^{2}(\bOmeC)}+M^{(3,2)}\Vert \nabla^{2}u \Vert^{2}_{L^{4}(\bOmeC)}\text{(Coupling error)}\\
			% 	&\ \ + M^{(3,0)}\Vert \nabla u \Vert^{2}_{L^{4}(\OmeL)}\text{(Linearization error)}\\
			% 	&\lesssim \CDH M^{(2,1)}K^{-\alpha -1}+\CDH M^{(2,2)}L^{-\alpha-\frac{3}{2}} +\CDH^{2}M^{(3,2)}L^{-2\alpha -\frac{3}{2}}\\
			% 	&\ \ +\CDH^{2}M^{(3,0)}(\bar{L})^{-2\alpha+\frac{1}{2}} .
			% 	\end{split}
		% \end{equation}
	
	% And we consider the best approximation error estimate
	% \begin{equation}
		% 	\Vert \nabla \Ph \ya-\nabla \ya\Vert_{L^{2}}\lesssim N_{\Th}^{-\frac{1}{2} -\alpha}(\bar{L}^{-\frac{1}{2} -\alpha}).
		% \end{equation}
	% }
% When $\frac{1}{2}<\alpha<1$, error is dominated by the linearization term $\bar{L}^{\frac{1}{2}-2\alpha}$. In order to get the same order, then we let $K^{-\alpha -1}\sim \bar{L}^{-2\alpha+
	% \frac{1}{2}}$. We get
% \begin{equation*}
	% 	\bar{L} \sim L^{\frac{1}{2}+\frac{5}{8\alpha-2}}(1.33\sim3(\text{ when }\alpha=0.8, \text{ it takes 1.63})).
	% \end{equation*}

% In order to get the same order, then we let $L^{-2\alpha -\frac{3}{2}}\sim \bar{L}^{-2\alpha+
	% 	\frac{1}{2}}$. We get
% \begin{equation*}
	% 	\bar{L} \sim L^{1+\frac{4}{4\alpha-1}}(2.33\sim5(\text{ when }\alpha=0.8, \text{ it takes 2.82})).
	% \end{equation*}

% When $\alpha>1$, error is dominated by by the best approximation term $\bar{L}^{-\frac{1}{2}-\alpha}$. In order to get the same order, then we let $K^{-\alpha -1}\sim \bar{L}^{-\alpha-
	% 	\frac{1}{2}}$. We get
% \begin{equation*}
	% 	\bar{L} \sim L^{1+\frac{1}{2\alpha+1}}(1.25(\alpha=1.5)\sim1.33).
	% \end{equation*}

% In order to get the same order, then we let $L^{-2\alpha -\frac{3}{2}}\sim \bar{L}^{-\alpha-
	% 	\frac{1}{2}}$. We get
% \begin{equation*}
	% 	\bar{L} \sim L^{2+\frac{1}{2\alpha+1}}(2.25(\alpha=1.5)\sim2.33).
	% \end{equation*}



%\section{A Posteriori Analysis}
%
%
%\subsection{Residual}
%
%
%\subsection{A Posteriori Stability}
%
%
%\subsection{A Posteriori Existence and Error Estimate}


\subsection{Numerical validation}
\label{sec: experiments_qnll_cg}

In this section, we present numerical experiments to illustrate our analysis. Same as in Section \ref{sec: experiments_qnll_ncg}, the problem is a typical one-dimensional test case, with the site energy modeled using the embedded atom method (EAM), a widely used atomistic model for solids. We fix the exact solution as defined in Section \ref{sec: experiments_qnll_ncg} and compute the external forces, which are equal to the internal forces under the deformation. The decay exponent ensures that the solution and forces satisfy the decay hypothesis $\DH$.

In this section, We will demonstrate the method of controlling the length of non-linear continuum region in the QNLL model to achieve quasi-optimal convergence order, as introduced in Section \ref{Balance of QNLL CG model}. We will conduct numerical experiments with the atomistic model length of 100,000 atoms.We set energy functional and external force to \eqref{EAM of numerical experiments} and \eqref{External force of numerical experiments}, and using Algorithm \ref{alg:FEM} to construct finite element mesh. The experiments will be carried out for $\alpha$ values of $0.8, 1.0$ and $1.2$.

Firstly, let's consider the experiment with alpha set to 1.2: In this case, according to \eqref{Balance of L CG 2}, by setting the length of the nonlinear continuum region to a few atoms ($\bar{K} \approx L$), the convergence order of the QNLL method matches that of the QNL method. In the Figure below, the $x$-axis represents the degrees of freedom (dof) in mesh, while the $y$-axis shows the absolute error $\Vert \nabla \yai - \nabla y^{\text{ac}} \Vert_{L^{2}}~\text{(ac} = \text{QNL, QNLL)}$ between the reference atomistic solution $\yai$ and the a/c solutions $y^{\text{ac}}$. It can be observed that the two convergence order lines in the graph nearly overlap, indicating that the difference between the two AC solutions $\Vert \nabla y^{\text{QNL}} - \nabla y^{\text{QNLL}}\Vert$ is between $10^{-7} $and $10^{-8}$.

\begin{figure}[h]
	\centering 
	\includegraphics[width=0.6\textwidth]{Figs/alpha12_fscale10.pdf}
	\caption{The convergence order of QNL and QNLL method ($\alpha = 1.2$)} % 图片标题
	\label{fig: convergence_QNL_QNLL_alpha12_CG}
\end{figure}

Next, to demonstrate the computational efficiency of the QNLL method, we test the variation in computation time by progressively increasing the degrees of freedom of the nonlinear continuum region $ \Nnl$ of the QNLL method, while keeping the finite element mesh fixed, meaning the continuum region remains unchanged. The results are as shown in the table below: the first column lists the method names, with parentheses indicating the proportion of the degrees of the freedom of nonlinear continuum region$ \Nnl $ to that of the total continuum region$ \Nc$ and the second column records the ratio of the computing time of the QNLL method to the computing time of the QNL method on a device the same as in Section \ref{sec: experiments_qnll_ncg}.

%\begin{center}
%\begin{table}
%    \centering
%\begin{tabular}{|c|c|c|} % 开始一个tabular环境,设置4列,每列居中对齐
%\hline % 绘制表格的横线
%Method($\Nnl/\Nc$) & dof & The ratio of the computing time\\ % 表头行
%\hline % 绘制表格的横线
%% QNLL($17.81\%$) & 2981 & $9.9766156\times 10^{-4}$ & $0.0840$ \\ % 第一行数据
%QNLL($19.84\%$) & 1377 & $69.38\%$ \\ % 第二行数据
%% QNLL($38.69\%$) & 2981 & $9.9766096\times 10^{-4}$ & $0.0024\%$\\ % 第三行数据
%QNLL($48.19\%$) & 1377  & $77.60\%$ \\ % 第四行数据
%% QNLL($59.57\%$) & 2981 &$9.9766096\times 10^{-4}$ & $0.0979 $\\ % 第五行数据
%% QNLL($66.53\%$) & 2981 & $9.9766096\times 10^{-4}$ & $0.0997$ \\ % 第六行数据
%QNLL($81.26\%$) & 1377 & $87.95\%$ \\ % 第七行数据
%% QNLL($83.92\%$) & 2981 & $9.9766097\times 10^{-4}$ & $0.1052$ \\ % 第八行数据
%QNL($100\%$) & 1377 & $100\%$ \\ % 第九行数据
%\hline % 绘制表格的横线
%\end{tabular}
% \caption{The computing time of QNL and QNLL method($\alpha = 1.2$)}
%    \label{tab:computing time alpha12 CG}
%\end{table}
%\end{center}

\begin{table}
	\centering
	\renewcommand{\arraystretch}{1.5} % 调整行间距
	\begin{tabular}{|c|c|} % 开始一个tabular环境,设置2列,每列居中对齐
		\hline % 绘制表格的横线
		Method ($\Nnl/\Nc$) & The ratio of the computing time\\ % 表头行
		\hline % 绘制表格的横线
		QNLL ($19.84\%$) & $69.38\%$ \\ % 第二行数据
		QNLL ($48.19\%$) & $77.60\%$ \\ % 第四行数据
		QNLL ($81.26\%$) & $87.95\%$ \\ % 第七行数据
		QNL ($100\%$) & $100\%$ \\ % 第九行数据
		\hline % 绘制表格的横线
	\end{tabular}
	\caption{The computing time (with coarse graining) of QNL and QNLL method ($\alpha = 1.2$), with degree of freedom set to 1377 for all methods.}
	\label{tab:computing time alpha12 CG}
\end{table}

% \begin{center}
	% \begin{table}[htbp]
		%     \centering
		%     \caption{The computing time of QNL and QNLL method($\alpha = 1.2$)}
		% \begin{tabular}{|c|c|c|c|} % 开始一个tabular环境,设置4列,每列居中对齐
			% \hline % 绘制表格的横线
			% Method($\left| \OmeNL\right|/\left| \OmeC\right|$) & dof & The ratio of the absolute error & The ratio of the computing time\\ % 表头行
			% \hline % 绘制表格的横线
			% % QNLL($17.81\%$) & 2981 & $9.9766156\times 10^{-4}$ & $0.0840$ \\ % 第一行数据
			% QNLL($28.25\%$) & 2981 & $9.9766111\times 10^{-4}$ & $0.0881$ \\ % 第二行数据
			% % QNLL($38.69\%$) & 2981 & $9.9766096\times 10^{-4}$ & $0.0908 $\\ % 第三行数据
			% QNLL($49.13\%$) & 2981 & $9.9766099\times 10^{-4}$ & $0.0942$ \\ % 第四行数据
			% % QNLL($59.57\%$) & 2981 &$9.9766096\times 10^{-4}$ & $0.0979 $\\ % 第五行数据
			% % QNLL($66.53\%$) & 2981 & $9.9766096\times 10^{-4}$ & $0.0997$ \\ % 第六行数据
			% QNLL($76.97\%$) & 2981 & $9.9766095\times 10^{-4}$ & $0.1030$ \\ % 第七行数据
			% % QNLL($83.92\%$) & 2981 & $9.9766097\times 10^{-4}$ & $0.1052$ \\ % 第八行数据
			% Reflection($100\%$) & 2981 & $9.9766094\times 10^{-4}$ & $0.1286$ \\ % 第九行数据
			% \hline % 绘制表格的横线
			% \end{tabular}
		% \end{table}
	% \end{center}


According to the Table \ref{tab:computing time alpha12 CG}, we can see that as the proportion of the nonlinear continuum region length to the total continuum region length increases, computing time clearly increases. However, in practical applications, the proportion of nonlinear elements will be lower (below 5$\%$) according to the balancing method described in Section \ref{Balance of QNLL CG model}. The ratio of the difference between the absolute errors of the QNLL method and the QNL method to the absolute errors of the QNL method: $( \Vert \nabla \yai - \nabla y^{\text{QNLL}} \Vert_{L^{2}} - \Vert \nabla \yai - \nabla y^{\text{QNL}} \Vert_{L^{2}}) / \Vert \nabla \yai - \nabla y^{\text{QNL}} \Vert_{L^{2}}$ is in a narrow range. Here, the ratio, as defined above, is within the range of $10^{-5}$ to $10^{-6}$. This indicates that the QNLL method maintains high accuracy while still offering computational efficiency advantages.

When $\alpha = 1$, the results are similar to when $\alpha = 1.2$. The following figure compares the convergence order of the QNLL method and the QNL method. The information represented on the axes is the same as in Figure \ref{fig: convergence_QNL_QNLL_alpha12_CG}. We observe a similar outcome to Figure \ref{fig: convergence_QNL_QNLL_alpha12_CG}, where the convergence lines of the QNLL method closely overlap with those of the QNL method.

\begin{figure}[h]
	\centering 
	\includegraphics[width=0.6\textwidth]{Figs/alpha10_fscale05.pdf}
	\caption{The convergence order of QNL and QNLL method ($\alpha = 1.0$)} % 图片标题
	\label{fig: convergence_QNL_QNLL_alpha10_CG}
\end{figure}

Similar to the case when $\alpha = 0.8$, we test the gradual increase in length of the nonlinear continuum region under the condition of a fixed finite element mesh. The remaining configurations and the information represented in each column are the same as in Table  \ref{tab:computing time alpha12 CG}. We observed results similar to Table \ref{tab:computing time alpha12 CG}, where as the proportion of the nonlinear continuum region length to the continuum region length increases gradually, the computing time also increases gradually, but there is no significant reduction in error.

%\begin{center}
%\begin{table}
%    \centering
%\begin{tabular}{|c|c|c|c|} % 开始一个tabular环境,设置4列,每列居中对齐
%\hline % 绘制表格的横线
%Method($\Nnl/\Nc$) & dof & The ratio of the computing time\\ % 表头行
%\hline % 绘制表格的横线
%% QNLL($17.81\%$) & 2981 & % % QNLL($17.81\%$) & 2981 & $9.9766156\times 10^{-4}$ & $0.0840$ \\ % 第一行数据
%QNLL($27.89\%$) & 1943 & $78.00\%$ \\ % 第二行数据
%% QNLL($38.69\%$) & 2981 & $9.9766096\times 10^{-4}$ & $0.0908 $\\ % 第三行数据
%QNLL($49.67\%$) & 1943 & $84.46\%$ \\ % 第四行数据
%% QNLL($59.57\%$) & 2981 &$9.9766096\times 10^{-4}$ & $0.0979 $\\ % 第五行数据
%% QNLL($66.53\%$) & 2981 & $9.9766096\times 10^{-4}$ & $0.0997$ \\ % 第六行数据
%QNLL($76.91\%$) & 1943 & $92.18\%$ \\ % 第七行数据
%% QNLL($83.92\%$) & 2981 & $9.9766097\times 10^{-4}$ & $0.1052$ \\ % 第八行数据
%QNL($100\%$) & 1943 & $100\%$ \\ % 第九行数据
%\hline % 绘制表格的横线
%\end{tabular}
%    \caption{The computing time of QNL and QNLL method($\alpha = 1.0$)}
%       \label{tab:computing time alpha10 CG}
%\end{table}
%\end{center}
\begin{table}
	\centering
	\renewcommand{\arraystretch}{1.5} % 调整行间距
	\begin{tabular}{|c|c|} % 开始一个tabular环境,设置2列,每列居中对齐
		\hline % 绘制表格的横线
		Method ($\Nnl/\Nc$) & The ratio of the computing time\\ % 表头行
		\hline % 绘制表格的横线
		QNLL ($27.89\%$) & $78.00\%$ \\ % 第二行数据
		QNLL ($49.67\%$) & $84.46\%$ \\ % 第四行数据
		QNLL ($76.91\%$) & $92.18\%$ \\ % 第七行数据
		QNL ($100\%$) & $100\%$ \\ % 第九行数据
		\hline % 绘制表格的横线
	\end{tabular}
	\caption{The computing time (with coarse graining) of QNL and QNLL method ($\alpha = 1.0$), with degree of freedom set to 1943 for all methods.}
	\label{tab:computing time alpha10 CG}
\end{table}


Furthermore, we will now consider the case where $\alpha=0.8$. In this setting, according to \eqref{Balance of L CG 1} and \eqref{Balance of L CG 2}, there are two finite element mesh generation schemes for the QNLL method:
\begin{enumerate}
	\item In the first scheme, we focus on the accuracy of the QNLL method. According to \eqref{Balance of L CG 1}, we precisely balance the atomistic region, nonlinear continuum region, linear continuum region, and the total length of the computational domain to achieve convergence order identical to those of the QNL method.
	
	\item In the second scheme, we prioritize the computational efficiency of the QNLL method. Therefore, after balancing the lengths of the atomistic region and the total length of the computational domain, we minimize the length of the nonlinear continuum region as much as possible, even down to just a few atoms.
\end{enumerate}

In Figure~\ref{fig: convergence_QNL_QNLL_alpha08_CG}, we represent the first finite element mesh generation scheme with red dashed squares for the QNLL method, and the second generation scheme with blue dashed squares. To demonstrate the accuracy of the QNLL method, the QNL method also adopts the first finite element mesh generation scheme, depicted in the figure with red dashed star symbols. The information represent on the axes is the same as in Figure \ref{fig: convergence_QNL_QNLL_alpha12_CG}. We observe that, after balancing the lengths of the atomistic region, nonlinear continuum region, linear continuum region, and the total length of the computational domain, the absolute errors and convergence order obtained by the QNLL method are consistent with those of the QNL method. However, after reducing the length of the nonlinear continuum region in pursuit of computational efficiency, there is a noticeable increase in absolute errors and a decrease in convergence speed.

\begin{figure}
	\centering 
	\includegraphics[width=0.6\textwidth]{Figs/alpha08_fscale05.pdf}
	\caption{The convergence order of QNL and QNLL method ($\alpha = 0.8$)} % 图片标题
	\label{fig: convergence_QNL_QNLL_alpha08_CG}
\end{figure}

The following table displays the changes in computation time as the length of the nonlinear continuum region gradually increase, with $\alpha$ set to 0.8. The remaining configurations and the information represented in each column are the same as in Table \ref{tab:computing time alpha12 CG}. Similar to Table \ref{tab:computing time alpha12 CG}, we observe that as the proportion of the nonlinear continuum region length to the continuum region length increases, computing time gradually increases, but there is no significant reduction in error.

%\begin{center}
%\begin{table}[h]
%    \centering
%\begin{tabular}{|c|c|c|} % 开始一个tabular环境,设置4列,每列居中对齐
%\hline % 绘制表格的横线
%Method($\Nnl/\Nc$) & dof & The ratio of the computing time\\ % 表头行
%\hline % 绘制表格的横线
%% QNLL($17.81\%$) & 2981 & % % QNLL($17.81\%$) & 2981 & $9.9766156\times 10^{-4}$ & $0.0840$ \\ % 第一行数据
%QNLL($28.25\%$) & 2981 & $68.50\%$ \\ % 第二行数据
%% QNLL($38.69\%$) & 2981 & $9.9766096\times 10^{-4}$ & $0.0908 $\\ % 第三行数据
%QNLL($49.13\%$) & 2981 & $73.26\%$ \\ % 第四行数据
%% QNLL($59.57\%$) & 2981 &$9.9766096\times 10^{-4}$ & $0.0979 $\\ % 第五行数据
%% QNLL($66.53\%$) & 2981 & $9.9766096\times 10^{-4}$ & $0.0997$ \\ % 第六行数据
%QNLL($76.97\%$) & 2981 & $80.08\%$ \\ % 第七行数据
%% QNLL($83.92\%$) & 2981 & $9.9766097\times 10^{-4}$ & $0.1052$ \\ % 第八行数据
%QNL($100\%$) & 2981 & $100\%$ \\ % 第九行数据
%\hline % 绘制表格的横线
%\end{tabular}
%    \caption{The computing time of QNL and QNLL method($\alpha = 0.8$)}
%   \label{tab:computing time alpha08 CG}
%\end{table}
%\end{center}
\begin{table}[h]
	\centering
	\renewcommand{\arraystretch}{1.5} % 调整行间距
	\begin{tabular}{|c|c|} % 开始一个tabular环境,设置2列,每列居中对齐
		\hline % 绘制表格的横线
		Method ($\Nnl/\Nc$) & The ratio of the computing time\\ % 表头行
		\hline % 绘制表格的横线
		QNLL ($28.25\%$) & $68.50\%$ \\ % 第二行数据
		QNLL ($49.13\%$) & $73.26\%$ \\ % 第四行数据
		QNLL ($76.97\%$) & $80.08\%$ \\ % 第七行数据
		QNL ($100\%$) & $100\%$ \\ % 第九行数据
		\hline % 绘制表格的横线
	\end{tabular}
	\caption{The computing time (with coarse graining) of QNL and QNLL method ($\alpha = 0.8$), with Degree of Freedom (DoF) set to 2981 for all methods.}
	\label{tab:computing time alpha08 CG}
\end{table}


%===========================================================================