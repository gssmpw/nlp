\section{The SIR-Policy game}
\todo{Diagram here}
We consider the standard model for
contagion spread, modified by non-interacting and non-interacting policies. 
We assume that to prevent the spread of contagion, governments or administrators specify $n$ policies in the set ${\cal P}= \{ P_1, P_2, \ldots P_n\}$. 
Example policies in the context of the COVID-19 pandemic could be adoption of masks, shelter-at-home etc. 
We define a non-atomic game where infinitesimal players decide on following one of the $n$ policies. Normalizing the population to be of unit size, let $f_i$ be the
fraction of the population that follows policy $i$. We consider two contagion processes that may arise in  this context. The first one assumes that populations of various policy groups do not interact with each other, while the second allows for interactions.

\paragraph{Non-interacting Populations} 
In this setting, we assume that policy group $i$ has a transmission parameter $\beta_i$ and a daily payment $p_i$. An individual player chooses to join one of the groups. Each group then has a group size representing its fraction in the population $\phi_i$ with $\sum_i \phi_i=1$.  Each group goes follows an individual SIR process. For group $i$, the initial conditions are 
\begin{align*} 
    S_i(0)&=(1-\epsilon) \phi_i,\\
    I_i(0)&= \epsilon\cdot\phi_i,\\
    R_i(0)&=0
\end{align*}

and the SIR process is defined as follows
\begin{align*}
    \frac{\mathrm{d}S_i(t)}{\mathrm{d}t}&=-\beta_i \cdot S_i(t)\cdot I_i(t),\\
    \frac{\mathrm{d}I_i(t)}{\mathrm{d}t}&=\beta_i\cdot S_i(t)\cdot I_i(t) - \gamma\cdot  I_i(t),\\
    \frac{\mathrm{d}R_i(t)}{\mathrm{d}t}&=\gamma\cdot I_i(t).
\end{align*}

The SIR process ends at time $T$. Each individual in group $i$ works and gets paid at the daily income $p_i$ until he gets infected. One may view the  group utility $UG_i$ as a function of the group size $\phi_i$
$$UG_i(\phi_i)=p_i \cdot \int_0^T S_i(t)\mathrm{d}t$$
Thus an individual in group $i$ has an expected utility
$$U_i(\phi_i)=\frac{UG_i(\phi_i)}{\phi_i}$$
We will consider the case where the time period is infinite, i.e. when endemic equilibrium is achieved.

Each individual player evaluates the expected utility for different groups. Non-atomic equilibrium is achieved when the player chooses to join group $i$ of size $\phi_i$ with
$$U_i(\phi_i) \geq U_j(\phi_j), \forall j$$ \todo{THINK:why is this the condition for equilibrium...Why not 
$dUG_i(\phi_i) \geq dUG_j(\phi_j)$.}



%%%%%%%%%%%% 
\todo[inline]{Is the below of any use}
We provide an alternative formulation of the individual utility $U_i$.
\begin{align*}
\frac{\mathrm{d}U_i}{\mathrm{d}t}&=p_i\cdot [1-\frac{S_i(t)}{S_i(0)}\cdot\frac{\beta_i\cdot S_i(t)\cdot I_i(t)}{S_i(t)}\cdot(T-t)]\\
&=p_i\cdot [1-\frac{\beta_i\cdot S_i(t)\cdot I_i(t)}{\phi_i}\cdot(T-t)]
\end{align*}
where $\frac{S_i(t)}{S_i(0)}$ is the probability of not being infected up to time $t$ and $\frac{\beta_i\cdot S_i(t)\cdot I_i(t)}{S_i(t)}$ is the probability of being infected at time $t$.



We proceed to show the monotonicity of individual utility $U_i(\phi_i)$ w.r.t. group size $\phi_i$. We first define an alternative $\overline{SIR}$ process to help the proof. The initial conditions are
\begin{align*} 
    \overline{S_i}(0)&=1-\epsilon,\\
    \overline{I_i}(0)&=\epsilon,\\
    \overline{R_i}(0)&=0
\end{align*}
and the process
\begin{align*}
    \frac{\mathrm{d}\overline{S_i}(t)}{\mathrm{d}t}&=-\phi_i\cdot\beta_i\cdot \overline{S_i}(t)\cdot \overline{I_i}(t),\\
    \frac{\mathrm{d}\overline{I_i}(t)}{\mathrm{d}t}&=\phi_i\cdot\beta_i\cdot \overline{S_i}(t)\cdot \overline{I_i}(t) - \gamma\cdot  \overline{I_i}(t),\\
    \frac{\mathrm{d}\overline{R_i}(t)}{\mathrm{d}t}&=\gamma\cdot \overline{I_i}(t)
\end{align*}
Now we show the following lemma.
\begin{lemma}
\begin{align*}
    \overline{S_i}(t)&=\frac{S_i(t)}{\phi_i},\\ \overline{I_i}(t)&=\frac{I_i(t)}{\phi_i},\\
    \overline{R_i}(t)&=\frac{R_i(t)}{\phi_i}
\end{align*}
\end{lemma}


\begin{proof}
For simplicity we omit the group subscript $i$ in the proof.
\begin{enumerate}
\item Initial conditions.
\begin{align*}
    \overline{S}(0)&=1, &\frac{S(0)}{\phi}&=1;\\
    \overline{I}(0)&=\epsilon, &\frac{I(0)}{\phi}&=\frac{\epsilon\phi}{\phi}=\epsilon;\\
    \overline{R}(0)&=0, &\frac{R(0)}{\phi}&=0
\end{align*}
\item $\overline{S}(t)$ and $\frac{S(t)}{\phi}$.
\begin{align}
    \frac{\mathrm{d}\overline{I}(t)}{\mathrm{d}\overline{S}(t)}&=\frac{\gamma}{\phi \beta  \overline{S}(t)} - 1 \implies\nonumber\\
    \overline{I}(t) &= \overline{I}(0) + \left.(\frac{\gamma}{\phi \beta}\ln\overline{S}-\overline{S})\right\vert_{\overline{S}(0)}^{\overline{S}(t)}= \epsilon +\frac{\gamma}{\phi\beta}\ln\overline{S}(t) - (\overline{S}(t)-1)\nonumber\\
    \frac{\mathrm{d}\overline{S}(t)}{\mathrm{d}t} &= -\phi\beta\overline{S}(t)\left[\epsilon+\frac{\gamma}{\phi\beta}\ln\overline{S}(t) - (\overline{S}(t)-1)\right]\label{eq1}\\
    \nonumber\\
    \frac{\mathrm{d}I(t)}{\mathrm{d}t} &= \frac{\gamma}{\beta S(t)} - 1\implies\nonumber\\
    I(t) &= I(0) + \left.(\frac{\gamma}{\beta}\ln S - S) \right|_{S(0)}^{S(t)} = \epsilon\phi +\frac{\gamma}{\beta}\ln\frac{S(t)}{S(0)} - (S(t) - S(0))\nonumber\\
    &= \phi\left[\epsilon+\frac{\gamma}{\phi\beta}\ln\frac{S(t)}{\phi} - (\frac{S(t)}{\phi} - 1)\right]\nonumber\\
    \frac{\mathrm{d}S(t)}{\mathrm{d}t} &= -\beta S(t)\cdot \phi\left[\epsilon+\frac{\gamma}{\phi\beta}\ln\frac{S(t)}{\phi} - (\frac{S(t)}{\phi} - 1)\right]\nonumber\\
    \frac{\mathrm{d}}{\mathrm{d}t}\frac{S(t)}{\phi} &= -\phi\beta\frac{S(t)}{\phi} \left[\epsilon+\frac{\gamma}{\phi\beta}\ln\frac{S(t)}{\phi} - (\frac{S(t)}{\phi} - 1)\right]\label{eq2}
\end{align}
If we replace $\overline{S}(t)$ in \eqref{eq1} by $\frac{S(t)}{\phi}$ we get \eqref{eq2}. Since $\overline{S}(0)=\frac{S(0)}{\phi}$, $\overline{S}(t)=\frac{S(t)}{\phi}$.

\item $\overline{I}(t)$ and $\frac{I(t)}{\phi}$.
\begin{align*}
    \overline{I}(t) &= \epsilon+\frac{\gamma}{\phi\beta}\ln\overline{S}(t) - (\overline{S}(t)-1)\\
    I(t) &= \phi\left[\epsilon+\frac{\gamma}{\phi\beta}\ln\frac{S(t)}{\phi} - (\frac{S(t)}{\phi} - 1)\right] \implies\\
    \frac{I(t)}{\phi} &= \epsilon+\frac{\gamma}{\phi\beta}\ln\frac{S(t)}{\phi} - (\frac{S(t)}{\phi} - 1)
\end{align*}
Since $\overline{S}(t) = \frac{S(t)}{\phi}$, $\overline{I}(t)=\frac{I(t)}{\phi}$.

\item $\overline{R}(t)$ and $\frac{R(t)}{\phi}$.
\begin{align*}
    \frac{\mathrm{d}\overline{R}(t)}{\mathrm{d}t} &= \gamma \overline{I}(t)\\
    \frac{\mathrm{d}R(t)}{\mathrm{d}t} &= \gamma I(t) \implies \frac{\mathrm{d}}{\mathrm{d}t}\frac{R(t)}{\phi} = \gamma \frac{I(t)}{\phi}
\end{align*}
Since $\overline{I}(t) = \frac{I(t)}{\phi}$, $\overline{R}(t)=\frac{R(t)}{\phi}$.
\end{enumerate}

\end{proof}
Since $\overline{S_i}(t) = \frac{S_i(t)}{\phi_i}$, an individual in group $i$ has the expected utility
\begin{align*}
    U_i(\phi_i) = \frac{UG_i(\phi_i)}{\phi_i} = \frac{p_i \cdot \int_0^T S_i(t)\mathrm{d}t}{\phi_i} = \int_0^T p_i\frac{S_i(t)}{\phi_i}\mathrm{d}t = \int_0^T p_i\overline{S_i}(t) \mathrm{d}t
\end{align*}
This allows us to use the alternative $\overline{SIR}$ process to evaluate the individual utility. Next we prove that 
\begin{lemma}
$\overline{S_i}(t)$ decreases monotonously w.r.t. $\phi_i$ at any given time $t$, i.e., if $\phi_i< \phi_j$, $\overline{S_i}(t) \geq \overline{S_j}(t), \forall t$.
\end{lemma}

\begin{proof}
By contradiction. Suppose $\phi_i< \phi_j$ and $\overline{S_i}(t)$ doesn't always dominate $\overline{S_j}(t)$. Since $\overline{S_i}(0)=\overline{S_j}(0)$ and $\frac{\mathrm{d}\overline{S_i}}{\mathrm{d}t} > \frac{\mathrm{d}\overline{S_j}}{\mathrm{d}t}$ at $t=0$, there exists a time $t'$ with
\begin{align*}
    &\begin{cases}
        \overline{S_i}(t') = \overline{S_j}(t')\\
        \frac{\mathrm{d}\overline{S_i}}{\mathrm{d}t} < \frac{\mathrm{d}\overline{S_j}}{\mathrm{d}t}
    \end{cases}\\
    &\frac{\mathrm{d}\overline{S_i}}{\mathrm{d}t} < \frac{\mathrm{d}\overline{S_j}}{\mathrm{d}t} \implies\\
    &- \phi_i\beta\overline{S_i}(t)\left[\epsilon + \frac{\gamma}{\phi_i\beta}\ln\overline{S_i}(t) - (\overline{S_i}(t) - 1)\right] <\\
    &- \phi_j\beta\overline{S_j}(t)\left[\epsilon + \frac{\gamma}{\phi_j\beta}\ln\overline{S_j}(t) - (\overline{S_j}(t) - 1)\right] \implies\\
    & \gamma \overline{S_i}(t)\ln\overline{S_i}(t) + \phi_i\beta\overline{S_i}(t)\left[\epsilon + 1 - \overline{S_i}(t)\right] >\\
    & \gamma \overline{S_j}(t)\ln\overline{S_j}(t) + \phi_j\beta\overline{S_j}(t)\left[\epsilon + 1 - \overline{S_j}(t)\right]\\
    & \overline{S_i}(t') = \overline{S_j}(t') \implies \phi_i > \phi_j
\end{align*}
This contradicts the assumption $\phi_i < \phi_j$.
\end{proof}
Recall that $\overline{S_i}(t)=\frac{S_i(t)}{\phi_i}$. This shows that in the original game, the individual utility $U_i(\phi_i)$ monotonely decreases as the group size $\phi_i$ increases.
\todo[inline]{MODIFY above and add at required place}
