\section{Infinite time with interaction between groups}
In this section we consider a game of infinite time, where the 2 groups interact with each other. Instead of each group having its own process, we define a joint process with initial conditions

\begin{align*}
    S_1(0)&=(1-\epsilon)\cdot\phi_1,\\
    I_1(0)&=\epsilon\cdot\phi_1,\\
    R_1(0)&=0,\\
    S_2(0)&=(1-\epsilon)\cdot\phi_2,\\
    I_2(0)&=\epsilon\cdot\phi_2,\\
    R_2(0)&=0.\\
\end{align*}
and the process is as follows

\begin{align*}
\frac{\mathrm{d}S_1(t)}{\mathrm{d}t}&=-S_1(t)\cdot(\beta_{11}\cdot I_1(t)+\beta_{12}\cdot I_2(t)),\\
\frac{\mathrm{d}I_1(t)}{\mathrm{d}t}&=S_1(t)\cdot (\beta_{11}\cdot I_1(t)+\beta_{12}\cdot I_2(t)) - \gamma \cdot I_1(t),\\
\frac{\mathrm{d}R_1(t)}{\mathrm{d}t}&= \gamma\cdot  I_1(t),\\
\frac{\mathrm{d}S_2(t)}{\mathrm{d}t}&=-S_2(t)\cdot(\beta_{21}\cdot I_1(t)+\beta_{22}\cdot I_2(t)),\\
\frac{\mathrm{d}I_2(t)}{\mathrm{d}t}&=S_2(t)\cdot (\beta_{21}\cdot I_1(t)+\beta_{22}\cdot I_2(t)) - \gamma \cdot I_2(t),\\
\frac{\mathrm{d}R_2(t)}{\mathrm{d}t}&= \gamma\cdot  I_2(t).\\
\end{align*}
where $\beta_{11}=\beta, \beta_{12}=\beta_{21}=\kappa \beta, \beta_{22}=\kappa^2 \beta$ with the assumption $\gamma/\beta \leq 1$. Here $0<\kappa<1$ represents the reduction of infection due to more cautious actions taken by Group 2. In this version of game, since Group 1 generally has a higher $\beta$, Group 2's process is dictated by Group 1.

\subsection{Final sizes}\label{sec:FinalSizes}
The final sizes $S_1(\infty), S_2(\infty)$ satisfy the following
\begin{equation}
\begin{cases}
S_1(\infty)=S_1(0)\cdot  exp\Big(\frac{\beta_{11}}{\gamma}(S_1(\infty) - \phi_1) + \frac{\beta_{12}}{\gamma}(S_2(\infty) - \phi_2)\Big),\\
S_2(\infty)=S_2(0)\cdot  exp\Big(\frac{\beta_{21}}{\gamma}(S_1(\infty) - \phi_1) + \frac{\beta_{22}}{\gamma}(S_2(\infty) - \phi_2)\Big).\\
\end{cases}\label{eq:finalsystem}
\end{equation}
where $exp(x)=e^x$ for clarity. In the following subsections we refer to the final sizes as $S_1, S_2$ for simplicity. The initial sizes can be expressed as $S_1(0)=(1-\epsilon)\phi_1,S_2(0)=(1-\epsilon)\phi_2$. For simplicity, denote $\frac{\beta_{11}}{\gamma}(S_1 - \phi_1) + \frac{\beta_{12}}{\gamma}(S_2 - \phi_2)$ by $X$ and $\frac{\beta_{21}}{\gamma}(S_1 - \phi_1) + \frac{\beta_{22}}{\gamma}(S_2 - \phi_2)$ by $Y$.

We first prove a useful upper bound for $S_1, S_2$.
\begin{lemma}
$S_1 + \kappa^2 S_2 < \frac{\gamma}{\beta}$.\label{lm:FinalUpper}
\end{lemma}
\begin{proof}
    Since $S_1,S_2$ are the final sizes and $\frac{\mathrm{d}S(t)}{\mathrm{d}t}<0$ strictly for any time $t<\infty$, there must exist a time $t$ with $\frac{\mathrm{d}I_1(t)}{\mathrm{d}t}<0,\frac{\mathrm{d}I_2(t)}{\mathrm{d}t}<0$ and $S_1<S_1(t),S_2<S_2(t)$.
    \begin{align*}
        &\begin{cases}
            \frac{\mathrm{d}I_1(t)}{\mathrm{d}t}=
            S_1(t) \Big(\beta_{11} I_1(t)+\beta_{12}I_2(t)\Big) - \gamma I_1(t)<0,\\
            \frac{\mathrm{d}I_2(t)}{\mathrm{d}t}=
            S_2(t) \Big(\beta_{21} I_1(t)+\beta_{22}I_2(t)\Big) - \gamma I_2(t)<0.
        \end{cases}\implies
    \end{align*}    
    \begin{numcases}{}
        S_1(t)\Big(I_1(t)+\kappa I_2(t)\Big) 
        <\frac{\gamma }{\beta}I_1(t)\label{ineq01},\\
        S_2(t)\Big(I_1(t)+\kappa I_2(t)\Big)
        <\frac{\gamma}{\kappa\beta} I_2(t)\label{ineq02}.
    \end{numcases}
    \begin{align*}
        &(\ref{ineq01})+\kappa^2(\ref{ineq02})\implies\\
        &\Big(S_1(t)+\kappa^2 S_2(t)\Big)\Big(I_1(t)+\kappa I_2(t)\Big)
        <\frac{\gamma}{\beta}\Big(I_1(t)+\kappa I_2(t)\Big)\implies\\
        & S_1+\kappa^2 S_2<\Big(S_1(t)+\kappa^2 S_2(t)\Big)<\frac{\gamma}{\beta}
    \end{align*}
\end{proof}

\subsection{Monotonicity of individual utility}
The individual utility of both groups are monotone with respect to the group size. Particularly, we show that $U_1(\phi_1)$ decreases with $\phi_1$, i.e. $\frac{\mathrm{d}S1/\phi1}{\mathrm{d}\phi_1}<0$. Denote $\hat{S_1}=S_1/\phi_1$ and $\hat{S_2}=S_2/\phi_2$. Since $\phi_1+\phi_2=1$, we get $\frac{\mathrm{d}\phi_1}{\mathrm{d}\phi_2}=-1$, $\frac{\mathrm{d}\hat{S_1}}{\mathrm{d}\phi_1}=-\frac{\mathrm{d}\hat{S_1}}{\mathrm{d}\phi_2}$ and $\frac{\mathrm{d}\hat{S_2}}{\mathrm{d}\phi_1}=-\frac{\mathrm{d}\hat{S_2}}{\mathrm{d}\phi_2}$. We derive $\frac{\mathrm{d}\hat{S_1}}{\mathrm{d}\phi_1}$ from the 2 equations above.
\begin{align*}
    &\begin{cases}
        S_1=(1-\epsilon)\phi_1 e^X,\\
        S_2=(1-\epsilon)\phi_2  e^Y.\\
    \end{cases}\implies\\
    &\begin{cases}
        \hat{S_1}=(1-\epsilon)
        exp(\frac{\beta\phi_1}{\gamma}(\hat{S_1}-1)
        +\frac{\kappa\beta\phi_2}{\gamma}(\hat{S_2}-1)),\\
        \hat{S_2}=(1-\epsilon)
        exp(\frac{\kappa\beta\phi_1}{\gamma}(\hat{S_1}-1)
        +\frac{\kappa^2\beta\phi_2}{\gamma}(\hat{S_2}-1)).
    \end{cases}\implies\\
    &\begin{cases}
        \frac{\mathrm{d}\hat{S_1}}{\mathrm{d}\phi_1}=(1-\epsilon)e^X
        \Big(\frac{\beta}{\gamma}(\hat{S_1}-1)-\frac{\kappa\beta}{\gamma}(\hat{S_2}-1)
        +\frac{\beta\phi_1}{\gamma}\frac{\mathrm{d}\hat{S_1}}{\mathrm{d}\phi_1}
        -\frac{\kappa\beta\phi_2}{\gamma}\frac{\mathrm{d}\hat{S_2}}{\mathrm{d}\phi_2}
        \Big),\\
        \frac{\mathrm{d}\hat{S_2}}{\mathrm{d}\phi_2}=(1-\epsilon)e^Y
        \Big(-\frac{\kappa\beta}{\gamma}(\hat{S_1}-1)+\frac{\kappa^2\beta}{\gamma}(\hat{S_2}-1)
        -\frac{\kappa\beta\phi_1}{\gamma}\frac{\mathrm{d}\hat{S_1}}{\mathrm{d}\phi_1}
        +\frac{\kappa^2\beta\phi_2}{\gamma}\frac{\mathrm{d}\hat{S_2}}{\mathrm{d}\phi_2}
        \Big).\\
    \end{cases}\implies\\
    &\begin{cases} 
        \frac{\mathrm{d}\hat{S_1}}{\mathrm{d}\phi_1}=
        \frac{-(1-\epsilon)\beta e^X(1-\hat{S_1}+\kappa(\hat{S_2}-1)}
        {\gamma-(1-\epsilon)\beta(\phi_1 e^X+\kappa^2\phi_2 e^Y)},\\
        \frac{\mathrm{d}\hat{S_2}}{\mathrm{d}\phi_2}=
        \frac{(1-\epsilon)\kappa\beta e^X(1-\hat{S_1}+\kappa(\hat{S_2}-1)}
        {\gamma-(1-\epsilon)\beta(\phi_1 e^X+\kappa^2\phi_2 e^Y)}.
    \end{cases}
\end{align*}

\begin{lemma}
\[
\frac{\mathrm{d}\hat{S_1}}{\mathrm{d}\phi_1}=
\frac{-(1-\epsilon)\beta e^X(1-\hat{S_1}+\kappa(\hat{S_2}-1)}
{\gamma-(1-\epsilon)\beta(\phi_1 e^X+\kappa^2\phi_2 e^Y)}<0
\]
\end{lemma}
\begin{proof}\ 
\begin{enumerate}[label=(\roman*)]
\item Numerator $<0$.

It suffices to show that $1-\hat{S_1}-\kappa(1-\hat{S_2})>0$. Recall that $\hat{S_1}=S_1/\phi_1=(1-\epsilon)e^X$ and $\hat{S_2}=(1-\epsilon)e^Y=(1-\epsilon)e^{\kappa X}$, where $X<0$,  $0<\kappa<1$.
\begin{align*}
    &e^X<e^{\kappa X}<1\implies\\
    & 1-(1-\epsilon)e^X>1-(1-\epsilon)e^{\kappa X}>\kappa(1-(1-\epsilon)e^{\kappa X})\implies\\
    & 1-\hat{S_1}-\kappa(1-\hat{S_2}) >0
\end{align*}

\item Denominator $>0$.
\begin{align*}
    &\gamma-(1-\epsilon)\beta(\phi_1 e^X+\kappa^2\phi_2 e^Y)\\
    =&\gamma\Big(1-\frac{\beta}{\gamma}(1-\epsilon)(\phi_1 e^X+\kappa^2\phi_2 e^Y)\Big)\\
    =&\gamma\Big(1-(1-\epsilon)\frac{\beta}{\gamma}(S_1+\kappa^2 S_2)\Big)
    >\gamma\Big(1-(1-\epsilon)\frac{\beta}{\gamma}\cdot\frac{\gamma}{\beta}\Big)\text{, by \textbf{Lemma \ref{lm:FinalUpper}}}\\
    >&0
\end{align*}
\end{enumerate}
Therefore, $\frac{\mathrm{d}\hat{S_1}}{\mathrm{d}\phi_1}<0$.
\end{proof}


\subsection{Nash equilibrium}

Since Group 1's individual utility $U_1(\phi_1)$ decreases monotonely, $p_1 U_1(1)$ is a lower bound of individual utility in Nash equilibrium. We show a lower bound of $p_1 U_1(1)$. Define
\begin{align*}
    c_1&=\frac{(1-\epsilon)\phi_1}{exp((\beta_{11}\phi_1)/\gamma+(\beta_{12}\phi_2)/\gamma)},\\
    c_2&=\frac{(1-\epsilon)\phi_2}{exp((\beta_{21}\phi_1)/\gamma+(\beta_{22}\phi_2)/\gamma)}.
\end{align*}
Note that within each instance of game, with all parameters being constant, $c_1,c_2$ are functions of $\phi_1$. We rewrite the conditions as follows
\[
\begin{cases}
S_1=c_1\cdot exp((\beta_{11}S_1+\beta_{12}S_2)/\gamma),\\
S_2=c_2\cdot exp((\beta_{21}S_1+\beta_{22}S_2)/\gamma).
\end{cases}
\]
Since $e^x\geq 1+x$ when $x\geq0$,
\[
\begin{cases}
S_1\geq c_1 (1+(\beta_{11}S_1+\beta_{12}S_2)/\gamma),\\
S_2\geq c_2 (1+(\beta_{21}S_1+\beta_{22}S_2)/\gamma).
\end{cases}
\]
\begin{align*}
    &S_2\geq c_2(1+(\beta_{21}S_1+\beta_{22}S_2)/\gamma)\\
    &S_2(1-c_2\beta_{22}/\gamma)\geq c_2(1+\beta_{21}S_1/\gamma)\\
    & S_2\geq \frac{c_2(1+\beta_{21}S_1/\gamma)}{1-c_2\beta_{22}/\gamma} \text{, by \textbf{Lemma \ref{lm:1/e_1}}}
\end{align*}


\begin{lemma}
\label{lm:1/e_1}
$1 - c_2\beta_{22}/\gamma > 0$
\end{lemma}
\begin{proof}
\begin{align*}
c_2\beta_{22}/\gamma &= \frac{(1-\epsilon)\beta_{22}\phi_2/\gamma}{exp((\beta_{21}\phi_1+\beta_{22}\phi_2)/\gamma)}\leq \frac{\beta_{22}\phi_2/\gamma}{exp((\beta_{21}\phi_1+\beta_{22}\phi_2)/\gamma)}\\
&\leq \frac{\beta_{22}\phi_2/\gamma}{exp(\beta_{22}\phi_2/\gamma)}
\end{align*}
Define function $b(x)=\frac{x}{e^x}, x\geq 0$. The max value is achieved at $b(x)\leq b(1)=1/e$. This bound will be used again in later proofs. Thus
\begin{align*}
    &c_2\beta_{22}/\gamma = b(\beta_{22}\phi_2/\gamma) \leq b(1) = 1/e\\
    &1 - c_2\beta_{22}/\gamma \geq 1 - 1/e >0
\end{align*}
\end{proof}


\begin{align*}
&S_1 \geq c_1 (1+(\beta_{11}S_1+\beta_{12}S_2)/\gamma)\\
&S_1 \geq c_1 \Big(1+\beta_{11}S_1/\gamma+ \beta_{12}/\gamma \cdot\frac{c_2(1+\beta_{21}S_1/\gamma)}{1-c_2\beta_{22}/\gamma}\Big)\\
&S_1 \Big(1-\frac{c_1\beta_{11}}{\gamma} - \frac{(c_1\beta_{12}/\gamma)(c_2\beta_{21}/\gamma)}{1 - c_2\beta_{22}/\gamma} \Big) \geq c_1 + \frac{c_1 c_2 \beta_{12}}{\gamma}\\
& S_1 \geq \frac{c_1 + \frac{c_1 c_2 \beta_{12}}{\gamma}}
{1-\frac{c_1\beta_{11}}{\gamma} - \frac{(c_1\beta_{12}/\gamma)(c_2\beta_{21}/\gamma)}{1 - c_2\beta_{22}/\gamma}} \text{, by \textbf{Lemma \ref{lm:1/e_2}}}\\
& S_1 \geq c_1 + \frac{c_1 c_2 \beta_{12}}{\gamma} \geq  \left.c_1 + \frac{c_1 c_2 \beta_{12}}{\gamma}\right|_{\phi_1=1}=\left.c_1\right|_{\phi_1=1}
\end{align*}
Therefore we get a lower bound of social utility at Nash equilibrium as 
\[
\left.p_1 c_1\right|_{\phi_1=1}=\frac{(1-\epsilon)p_1}{exp(\beta/\gamma)}
\]

\begin{lemma}
\label{lm:1/e_2}
    $1-\frac{c_1\beta_{11}}{\gamma} - \frac{(c_1\beta_{12}/\gamma)(c_2\beta_{21}/\gamma)}{1 - c_2\beta_{22}/\gamma} > 0$
\end{lemma}
\begin{proof}\ 
\begin{enumerate}[label=(\roman*)]
    \item 
    \begin{align*}
        \frac{c_1\beta_{11}}{\gamma} &= \frac{(1-\epsilon)\beta_{11}\phi_1/\gamma}
        {exp((\beta_{11}\phi_1+\beta_{12}\phi_2)/\gamma}\leq \frac{\beta_{11}\phi_1/\gamma}{exp(\beta_{11}\phi_1/\gamma)}\\
        &=b(\beta_{11}\phi_1/\gamma) \leq \frac{1}{e}
    \end{align*}
    
    \item
    \begin{align*}
        &(c_1\beta_{12}/\gamma)(c_2\beta_{21}/\gamma) \leq
        \frac{\kappa^2\beta^2\phi_1\phi_2/\gamma^2}{exp\Big(((\beta+\kappa\beta)\phi_1+(\kappa\beta+\kappa^2\beta)\phi_2)/\gamma\Big)}\\
        =&\frac{\kappa\beta\phi_1/\gamma}{exp((1+\kappa)\beta\phi_1/\gamma)}
        \cdot
        \frac{\kappa\beta\phi_2/\gamma}{exp(\kappa(1+\kappa)\beta\phi_2/\gamma)}\\
        \leq&  \frac{\kappa\beta\phi_1/\gamma}{exp(\kappa\beta\phi_1/\gamma)}
        \cdot
        \frac{\kappa\beta\phi_2/\gamma}{exp(\kappa\beta\phi_2/\gamma)}\\
        =& b(\kappa\beta\phi_1/\gamma)
        \cdot
        b(\kappa\beta\phi_2/\gamma)
        \leq \frac{1}{e^2}
    \end{align*}
    
    \item
    \[
    1-c_2\beta_{22}/\gamma \geq 1-\frac{1}{e} \text{ by \textbf{Lemma \ref{lm:1/e_1}}}
    \]
\end{enumerate}
Thus,
\begin{align*}
    1-\frac{c_1\beta_{11}}{\gamma} 
    - \frac{(c_1\beta_{12}/\gamma)(c_2\beta_{21}/\gamma)}{1 - c_2\beta_{22}/\gamma} \geq 1 - \frac{1}{e} - \frac{1}{e^2} (1-\frac{1}{e})> 0
\end{align*}
\end{proof}


\subsection{Social optimum}

We first show an upper bound of $S_1$.
\begin{lemma}
$S_1 \leq \min(\frac{\gamma}{\beta}, \phi_1)$
\end{lemma}
\begin{proof}
\begin{enumerate}[label=(\roman*)]
\item
\label{S1upper1}
The final size $S_1$ is clearly bounded by the initial size $(1-\epsilon)\phi_1$, therefore by $\phi_1$.

\item
\label{S1upper2}
Let $t_{peak}$ be the time when $I_1(t)$ is at peak, we have
\begin{align*}
    &\frac{\mathrm{d}I_1(t_{peak})}{\mathrm{d}t}=S_1(t_{peak}) \Big(\beta_{11} I_1(t_{peak})+\beta_{12} I_2(t_{peak})\Big) - \gamma  I_1(t_{peak})=0\\
    &S_1(t_{peak})=\frac{\gamma I_1(t_{peak})}{\beta_{11} I_1(t_{peak})+\beta_{12} I_2(t_{peak})} \leq \frac{\gamma I_1(t_{peak})}{\beta_{11} I_1(t_{peak})} = \frac{\gamma}{\beta}
\end{align*}
Since $\frac{\mathrm{d}S_1(t)}{\mathrm{d}t}\leq 0$, if the initial susceptible size is below $S_1(t_{peak})$, the final size will also be below. If the initial size is above, it will have to first drop to $S_1(t_{peak})$ for $I_1(t)$ stop increasing then drop to $I_1(\infty)=0$, again $S_1\leq S_1(t_{peak})$.
\end{enumerate}
With \ref{S1upper1} and \ref{S1upper2}, $S_1 \leq \min(\frac{\gamma}{\beta}, \phi_1)$.
\end{proof}

Since $p_1\geq 1$, the social utility is
\[
p_1 S_1+S_2\leq p_1\cdot\min(\frac{\gamma}{\beta}, \phi_1)+(1-\phi_1)\leq \frac{(p_1-1)\gamma}{\beta}+1
\]

\subsection{Price of anarchy}
The price of anarchy has the following lower bound
\begin{align*}
    POA \geq \frac{\frac{(1-\epsilon)p_1}{exp(\beta/\gamma)}}{\frac{(p_1-1)\gamma}{\beta}+1}
\end{align*}

