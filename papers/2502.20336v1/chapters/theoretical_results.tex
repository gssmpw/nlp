% !TEX root = ../dual_norm_estimate_base.tex
\section{Certification via extensions and restrictions} \label{sec:CertifyPINNs}
Now we are in position to describe the proposed approach towards deriving efficiently computable lower and upper bounds for the error $\|u-u^\theta\|_\cW$ of a PINN $u^\theta$ for approximating the solution $u$ of a PDE operator equation $Bu=f$. The idea is to use the residual $r^\theta:=f-B\,u^\theta$ on different domains, i.e., 
\begin{align*}
    \mycirc \subset \Omega \subset \square  \subset \R^d,
\end{align*}
where we shall use $\mycirc$ to derive a lower and $\square$ for an upper bound. This approach has a couple of consequences which we will address next:
 \begin{compactitem}
    \item We need to compute the Riesz representation of $r_\theta$ by solving \eqref{eq:RieszRepFinite} replacing $f$ by $r^\theta$. Hence, we need to compute $r^\theta(v^\delta)$ for appropriate test functions $v^\delta$ on the respective domain. This typically amounts for computing inner products, i.e., integrals. If the PINN gives pointwise approximations, we use quadrature (e.g.\ by choosing Gauss-Lobatto points), for a variational PINN, we would precompute the inner products of the basis functions and then determine the linear combinations.
    \item We need to restrict and extend $r^\theta$ to $\mycirc$ and $\square$ in an appropriate manner.
    \item We have to choose $\mycirc$ and $\square$ in such a way that (i) the residuals on these domains allow for sharp error bounds and (ii) the computations are very efficient, which typically means that the geometries of $\mycirc$ and $\square$ need to be \enquote{simple}, e.g.\ a hypersphere or -cube. Then, standard discretizations and fast solvers such as geometric full multigrid or spectral methods are available.
\end{compactitem}

We need to fix some notation. The original PDE operator equation $Bu=f$ is posed in $\cY'(\Omega)$ for $B\in\cL_{\text{is}}(\cW(\Omega),\cY'(\Omega))$, the residual $r^\theta$ is an element of $\cY'(\Omega)$, its Riesz representation $\hat r^\theta_\Omega$ on $\Omega$ is determined by solving a sufficiently detailed discretization of the problem finding
\begin{align}\label{eq:RieszRepOmega}
    \hat r^\theta_\Omega\in\cY(\Omega):
    \qquad
    (\hat r^\theta_\Omega,v)_{\cY(\Omega)}
    = r^\theta_\Omega(v)
    \quad \forall v\in\cY(\Omega).
\end{align}



\subsection{Extension and restriction for primal and dual spaces}
One might think that we could just consider the problem  \eqref{eq:RieszRepOmega} by changing $\Omega$ to $\mycirc$ and $\square$, respectively. This, however, does not match our goals. Recall that $\cY(\Omega)$ is typically a function space, e.g.\ $H^1_0(\Omega)$. Just changing $\Omega$ to $\square$ would give rise to $H^1_0(\square)$. However, the restriction of a function in $H^1_0(\square)$ to $\Omega$ is in general \emph{not} in $H^1_0(\Omega)$. Moreover, depending on the geometry of $\Omega$, an extension might be a delicate issue. Finally, we cannot hope to control the norms of the residual on $\Omega$ by those on $\mycirc$ and $\square$. To overcome this difficulty, we propose the following recipe:
\begin{compactenum}[1.]
    \item Choose a space $\cZ(\square)$, which contains all extensions of $\cY(\Omega)$ by zero and which allows for fast solvers. One may think of $H^1_0(\square)$ or $H^1_{\text{per}}(\square)$.
    \item Identify a subspace $\cU(\square)\subseteq\cZ(\square)$, which is isomorphic to $\cY(\Omega)$, i.e., $\cU(\square)\cong\cZ(\square)$. 
\end{compactenum}
We are going to formalize this idea in the following assumption.
\begin{assumption} \label{assume:Isomorph} 
	Let $\Omega \subset \square \subset \R^d$ be Lipschitz domains, and let $\cY(\Omega)$ and $\cZ(\square)$ be given function spaces. 
    \begin{compactenum}[(a)]
    \item Then, assume that there exists a subspace $\cU(\square)\subset \cZ(\square)$ (with the same norm $\Vert \cdot \Vert_{\cZ(\square)}$ as in $\cZ(\square)$), which is isomorphic to $\cY(\Omega)$. 
    \item The isomorphism is denoted by $\EBox \in \cL_{\text{is}}(\cY(\Omega), \cU(\square))$.
    \item We denote the continuous inverse operator of $\EBox$ by $\RBox := \EBox^{-1}$.
    \hfill$\diamond$
    \end{compactenum}
\end{assumption}
Although the assumption requiring an isomorphism seems to be quite strong, we will see that for a large class of spaces we can in fact construct such an isomorphism. The names $\EBox$ and $\RBox$ are chosen to emphasize that the operators are often \emph{extension} and \emph{restriction}, at least in our examples described in Section \ref{sec:NumResults}. 
 
 We note a simple immediate consequence of Assumption \ref{assume:Isomorph}.
\begin{corollary}\label{cor:Isomorphismbounds}
	If Assumption\,\ref{assume:Isomorph} holds, then  
	\begin{equation*}
		c_\square \Vert v \Vert_{\cY(\Omega)} 
        \le \Vert \EBox v \Vert_{\cZ(\square)} 
        \le C_\square \Vert v \Vert_{\cY(\Omega)}, \quad \forall v \in \cY(\Omega),
	\end{equation*}
    where $c_\square := \Vert \RBox \Vert_{\cL(\cU(\square),\cY(\Omega))}^{-1}$ and $C_\square := \Vert \EBox \Vert_{\cL(\cY(\Omega),\cU(\square))}$.
    \qed
\end{corollary}

With the operator $\EBox$, we have the desired connection between the primal spaces $\cY(\Omega)$ and $\cZ(\square)$. We need such a relation also for the dual spaces. For that, we consider the adjoint operators $\aEBox\in \cL_{\text{is}}\left(\cU'(\square), \cY'(\Omega)\right)$ of $\EBox$ and $\aRBox\in \cL_{\text{is}}\left(\cY'(\Omega), \cU'(\square)\right)$ of the inverse of $\EBox$. By definition of adjoint operators, we have
\begin{align*}
	\dual{\aEBox f_\square, v }_{\cY(\Omega)} 
    &= \dual{f_\square,\EBox v}_{\cU(\square)}, 
    &&\; \forall v \in \cY(\Omega), \; \forall f_\square \in \cU'(\square)
    \quad\text{and}\\
	\dual{\aRBox f_\Omega, w}_{\cU(\square)} 
    &= \dual{f_\Omega,\RBox w}_{\cY(\Omega)}, 
    &&\; \forall w \in \cU(\square), \; \forall f_\Omega \in \cY'(\Omega).
\end{align*}
Furthermore, for the norm there holds $\Vert \EBox \Vert_{\cL(\cY(\Omega), \cU(\square))} = \Vert \aEBox \Vert_{\cL(\cU'(\square), \cY'(\Omega))}$ and $\Vert \RBox \Vert_{\cL(\cU(\square), \cY(\Omega))} = \Vert \aRBox \Vert_{\cL(\cY'(\Omega), \cU'(\square))}$. Note that the dual operators $\aRBox$ and $\aEBox$ are changing the roles of extension and restriction. In fact, $\aRBox$ is an extension operator for functionals defined on $\cY(\Omega)$ onto those on $\cU(\square)$. 
Apparently, this requires the dual space $\cU'(\square)$ of $\cU(\square)$, whose norm is given by
\begin{align*}
    \| r\|_{\cU'(\square)}
    = \sup_{u\in \cU(\square)} \frac{r(u)}{\|u\|_{\cZ(\square)}},
\end{align*}
which is in general smaller than $\| r\|_{\cZ'(\square)}$ if $\cU(\square)\subsetneq \cZ(\square)$. With this norm, we have the following estimate.


\begin{lemma} \label{lemma:extBoundsBanachSpacev1}
	Let Assumption\,\ref{assume:Isomorph} hold, then 
	\begin{equation}
		c_\square \Vert \aRBox f_\Omega \Vert_{\cU'(\square)} 
        \le  \Vert f_\Omega \Vert_{\cY'(\Omega)} \le C_\square \Vert \aRBox f_\Omega \Vert_{\cU'(\square)} \quad \forall f_\Omega \in \cY'(\Omega)
	\end{equation}
    with the constants $0<c_\square\le C_\square <\infty$ defined in Corollary \ref{cor:Isomorphismbounds}.
\end{lemma}
\begin{proof}
	The lower bound is the continuity of $\aRBox$, i.e.
	\begin{equation*}
		\Vert \aRBox f_\Omega \Vert_{\cU'(\square)} 
        \le \Vert \aRBox \Vert_{\cL(\cY'(\Omega), \cU'(\square))} \Vert f_\Omega \Vert_{\cY'(\Omega)} 
        \le \tfrac1{c_\square} \Vert f_\Omega \Vert_{\cY'(\Omega)}.
	\end{equation*}
	The upper bound follows from the continuity of $\aEBox$ and the fact that $\id_{\cY'(\Omega)} = (\id_{\cY(\Omega)})' = (\RBox \circ \EBox)' = (\aEBox \circ \aRBox)$, which concludes the proof.
\end{proof}

The estimates in Lemma \ref{lemma:extBoundsBanachSpacev1} seem to give us the desired upper and lower bounds. However, they involve the norm in $\cU'(\square)$, i.e., we would need the space $\cU(\square)$, which turns out to be inappropriate as the following example shows.

\begin{example} 
    Let $\cY(\Omega)=H^1_0(\Omega)$ and $\cZ(\square)= H^1_0(\square)$. We are searching for a subspace $\cU(\square)\subset\cZ(\square)$, which is isomorphic to $\cY(\Omega)$. Apparently, the extension by zero of an element of $H^1_0(\Omega)$ is in $H^1_0(\square)$, and the restriction of an element in $H^1_0(\square)$ having zero trace on $\partial\Omega$ is in $H^1_0(\Omega)$. Hence, we could choose $\cU(\square)$ consisting of all zero extensions of $H^1_0(\Omega)$-functions. To compute those functions, would however require a discretization of $\Omega$, which we wanted to avoid.\hfill$\diamond$
\end{example}

This latter example shows that we need to be careful with the definition of the operators $\EBox$ yielding $\cZ(\square)$ and the subspace $\cU(\square)$. We want to do all computations on $\cZ(\square)$, which means that we need a norm-preserving extension from $f_\Omega\in\cY'(\Omega)$ via $\aRBox f_\Omega \in \cU'(\square)$ to $\cZ'(\square)$. To this end, we suggest the norm-preserving Hahn-Banach Extension Theorem (see e.g.\ \cite[Sec.\ 9.1.2,\,Th.1]{kadets2018course}), ensuring the existence of extensions $f_\square^{\text{HB}} \in \cZ'(\square)$ of $\aRBox f_\Omega \in \cU'(\square)$ with
\begin{equation*}
	\Vert \aRBox f_\Omega \Vert_{\cU'(\square)} 
    = \Vert f_\square^{\text{HB}} \Vert_{\cZ'(\square)}.
\end{equation*} 
For all other extensions of $\aRBox f_\Omega$, the lower bound from Lemma \ref{lemma:extBoundsBanachSpacev1} will be lost. Unfortunately, the proof of the Hahn-Banach Extension Theorem is non-constructive, i.e., the Hahn-Banach extension $f_\square^{\text{HB}}$ is not known in general. Thus, we will describe the computation of extensions of $\aRBox f_\Omega$ in \S\ref{Sec:Extension} below.

\subsection{A lower bound}
For deriving a lower bound, we suggest to use an additional domain $\mycirc \subset \Omega$ and a corresponding function space $\cX(\mycirc)$ in such a way that Assumption \ref{assume:Isomorph} holds for the spaces $\cX(\mycirc)$ and $\cY(\Omega)$ replacing $\cY(\Omega)$ and $\cZ(\square)$. Then, there exists a subspace $\cV(\Omega) \subset \cY(\Omega)$, which is isomorphic to $\cX(\mycirc)$. We denote the corresponding extension isomorphism and its inverse (the restriction) by $\ECirc$ and $\RCirc$. Moreover, by switching the roles of $\aRCirc$ and $\aECirc$ we get the following bound.

\begin{lemma} \label{lemma:extBoundsBanachSpacev2}
	Let Assumption\,\ref{assume:Isomorph} hold, such that there exists an operator $\Ext \in \cL_{\text{is}}(\cX(\mycirc), \cV(\Omega))$. Then,
	\begin{equation}
		\tfrac{1}{C_\mycirc} \Vert \aECirc f_\Omega \Vert_{\cX'(\mycirc)} 
        \le  \Vert f_\Omega \Vert_{\cV'(\Omega)} 
        \le c_\mycirc \Vert \aECirc f_\Omega \Vert_{\cX'(\mycirc)} \quad \forall f_\Omega \in \cV'(\Omega)
	\end{equation}
    where $c_\mycirc := \Vert \RCirc \Vert_{\cL(\cV(\Omega),\cX(\mycirc))}^{-1}$ and $C_\mycirc := \Vert \ECirc \Vert_{\cL(\cX(\mycirc),\cV(\Omega))}$. 
\end{lemma}
\begin{proof}
	The lower bound follows from the continuity of $\aECirc$ and the upper bound is derived using $\id_{\cX(\mycirc)} = \RCirc \circ \ECirc$ and the continuity of $\RCirc$.
\end{proof}

Now, we can put all pieces together and formulate the main result of this section. To this end, it is convenient to define for spaces $\cX \subset \cY$ the \emph{set of all extensions} of a given functional $f \in \cX'$ as
\begin{equation*}
	\mathcal{E}(f;\cY') := \big\lbrace \tilde{f} \in \cY': \tilde{f} \equiv f \text{ on } \cX \big\rbrace.
\end{equation*}
With that at hand, we combine Lemma\,\ref{lemma:extBoundsBanachSpacev1} and Lemma\,\ref{lemma:extBoundsBanachSpacev2}.

\begin{theorem}\label{th:mainBounds} 
	Let $\mycirc \subset \Omega \subset \square$ and $f_\Omega \in \cY'(\Omega)$ be given. Furthermore, let $\EBox$, $\RBox$ and $\ECirc$, $\RCirc$ be the operators from Lemma\,\ref{lemma:extBoundsBanachSpacev1} and from Lemma\,\ref{lemma:extBoundsBanachSpacev2}, respectively. Then we have
	\begin{equation*}
		\tfrac{1}{C_{\mycirc}} \Vert f_\mycirc \Vert_{\cX'(\mycirc)} 
        \le \Vert f_\Omega \Vert_{\cY'(\Omega)} 
        \le C_\square\, \Vert f_\square \Vert_{\cZ'(\square)},
	\end{equation*}
	for all $f_\square \in \mathcal{E}(\aRBox f_\Omega; \cZ'(\square))$ and $f_\mycirc := \aECirc \left(f_\Omega|_{\cV(\Omega)}\right)$, where $f_\Omega|_{\cV(\Omega)}$ denotes the restriction of $f_\Omega: \cY(\Omega)\to\R$ to $\cV(\Omega)\subset\cY(\Omega)$.
\end{theorem}
\begin{proof}
To derive the upper bound, we first use Lemma\,\ref{lemma:extBoundsBanachSpacev1}, which yields $\Vert f_\Omega \Vert_{\cY'(\Omega)} \le C_\square \Vert \aRBox f_\Omega \Vert_{\cU'(\square)}$. Next, the Hahn-Banach Extension Theorem ensures the existence of an extension $f_\square^{\text{HB}} \in \cZ'(\square)$ of $\aRBox f_\Omega \in \cU'(\square)$ to $\cZ'(\square)$ with the identity $\Vert \aRBox f_\Omega \Vert_{\cU'(\square)} = \Vert f_\square^{\text{HB}} \Vert_{\cZ'(\square)}$ and all other extensions have larger norms, which proves the upper bound. 
For the lower bound, note that $\cY'(\Omega)\subset \cV'(\Omega)$, so that $\Vert f_\Omega |_{\cV(\Omega)}\Vert_{\cV'(\Omega)} \le \Vert f_\Omega \Vert_{\cY'(\Omega)}$ (again by the Hahn-Banach theorem). Finally, the lower bound in Lemma\,\ref{lemma:extBoundsBanachSpacev2} applied to $f_\Omega|_{\cV(\Omega)} \in \cV'(\Omega)$ completes the proof.
\end{proof}

\subsection{Constructing an extension}\label{Sec:Extension}
The main building blocks of the above theory are the extension isomorphism $\ECirc$ and $\EBox$, their duals and their inverses. Of course, the extensions depend on the geometry of $\Omega$ and the specific form of the spaces $\cY(\Omega)$ and $\cZ(\square)$, which in turn are determined by the underlying PDE. We are going to detail two relevant cases. 

\subsubsection{Sobolev Spaces with zero trace}\label{Subsec:Sobolev} Let $\Omega \subset \mathbb{R}^{d}$ and let $\cY(\Omega)=W_0^{m,p}(\Omega)$ be the Sobolev spaces  of order $m \in \mathbb{N}$ in $L_p(\Omega)$ for $1 \le p \le \infty$. A \emph{zero extension} $\tilde{u}$ of $u \in  W_0^{m, p}(\Omega)$ is defined as
\begin{equation*}
	\tilde{u}(x) :=  
	\begin{cases}
		u(x), \quad & x \in \Omega, \\[2pt]
		0, \quad & x \in \Omega^{\text{c}}:=\mathbb{R}^d\setminus\Omega.
	\end{cases}
\end{equation*}

\begin{proposition}[{\cite[Lemma 3.27]{Adams}}]\label{prop:Wmp0extensions}
    Let $u \in W_0^{m, p}(\Omega)$. If $\vert \alpha \vert \le m$, then $D^\alpha \tilde{u} = \widetilde{D^\alpha u}$ in the distributional sense in $\mathbb{R}^n$, i.e., $\tilde{u} \in W^{m, p}(\mathbb{R}^n)$.
    \hfill\qed
\end{proposition}
Studying the proof of \cite[Lemma 3.27]{Adams}, we note that there are no assumptions on the domain $\Omega$. In the following we denote the dual spaces of $W_0^{m, p}(\Omega)$ by $W_0^{-m, q}(\Omega) := (W_0^{m, p}(\Omega))'$ where $\frac{1}{p} + \frac{1}{q} = 1$.

Let $u\in \cY(\Omega):=W_0^{m,p}(\Omega)$ for some $\Omega\subset\R^d$, $\tilde{u}$ be its zero extension to $\R^d$ and let $\tilde{u}|_\square$ be the restriction to $\square\supset\Omega$. Then, $\tilde{u}|_\square\in W_0^{m, p}(\square )$ and we define 
\begin{align}\label{eq:EboxSobolev}
    \EBox: W_0^{m, p}(\Omega) \rightarrow W_0^{m,p}(\square)
    \quad\text{by}\quad
    u \mapsto \EBox u := \tilde{u}|_\square,
\end{align}
which is a linear operator. Setting 
\begin{equation}\label{eq:UdefSobolev}
	\cU(\square) 
    := \EBox \left(W_0^{m, p}(\Omega) \right)  
    \subsetneq \cZ(\square) = W_0^{m,p}(\square),
\end{equation}
i.e., the range of $\EBox$, Proposition\,\ref{prop:Wmp0extensions} ensures that $\EBox: W_0^{m,p}(\Omega) \rightarrow \cU(\square)$ is an isometry, i.e.,
\begin{align}\label{eq:Isometry}
    C_\square = \| \EBox\|_{\cL(\cY(\Omega),\cU(\Omega))}=1.
\end{align}
Hence, the operator is continuous and bounded from below. By \cite[Lemma\,2.8]{Abramovich2002}, this is equivalent to $\EBox$ being an isomorphism. The restriction $\RBox: \cU(\square) \rightarrow W_0^{m,p}(\Omega)$ is the inverse operator and we get $c_\square = \| \RBox\|_{\cL(\cY(\Omega),\cU(\Omega))}^{-1}=1$. Note, that the constants in Corollary \ref{cor:Isomorphismbounds} are in fact unity here, which is due to the zero trace in combination with zero extension.

We can proceed in an analogous manner for $\mycirc \subset \Omega$, $\cX(\mycirc) := W_0^{m, p}(\mycirc)$ and get an isomorphism $\ECirc: W_0^{m, p}(\mycirc) \rightarrow \cV(\Omega)$ defined as $\ECirc u := \tilde{u}|_\Omega$ for 
\begin{equation}\label{eq:VdefSobolev}
	\cV(\Omega) := \ECirc \left(W_0^{m,p}(\mycirc) \right) \subsetneq \cY(\Omega).
\end{equation}
Again, we obtain isometries, i.e., the involved constants are unity. Thus, Theorem\,\ref{th:mainBounds} implies the following statement.
\begin{corollary}\label{cor:boundsSobolev}
	Let $\mycirc\subset\Omega\subset\square\subset\R^d$, $\cX(\mycirc):=W_0^{m,p}(\mycirc)$, $\cY(\Omega):=W_0^{m,p}(\Omega)$ and 
    $\cZ(\square):=W_0^{m,p}(\square)$ for $m\in\mathbb{N}$ and $1 \le p \le \infty$. Let $\ECirc$, $\EBox$ be defined as above and $\RCirc$, $\RBox$ being their inverses. Then, 
	\begin{equation}
		\Vert f_\mycirc \Vert_{W_0^{-m, q}(\mycirc)} \le \Vert f_\Omega \Vert_{W_0^{-m, q}(\Omega)} \le \Vert f_\square \Vert_{W_0^{-m, q}(\square)}
	\end{equation}
	for all $f_\square \in \mathcal{E}(\aRBox f_\Omega; W_0^{-m,q}(\square))$ and $f_\mycirc := \aECirc \left(f_\Omega|_{\cV'(\Omega)}\right)$ with $\cV(\Omega)$ defined in \eqref{eq:VdefSobolev}.
    \hfill\qed
\end{corollary}


\subsubsection{Bochner spaces}\label{Subsec:Bochner}
We are now going to consider instationary parabolic problems involving space and time as already introduced in Example \ref{Ex:SpaceTime} for $Y(\Omega):=W^{m,p}_0(\Omega)$. We perfom the embedding strategy only w.r.t.\ the space variable and set
\begin{align*}
    \cX(\mycirc) := L_2(I; W^{m,p}_0(\mycirc)),
    \quad
    \cY(\Omega) := L_2(I; W^{m,p}_0(\Omega)),
    \quad
    \cZ(\square) := L_2(I; W^{m,p}_0(\square)),
\end{align*}
for $\mycirc\subset\Omega\subset\square\subset\R^d$. Hence, we can reuse the findings of the previous \S\ref{Subsec:Sobolev}. Using $\EBox$ defined in \eqref{eq:EboxSobolev} and $U(\square):=\EBox \left(W_0^{m, p}(\Omega) \right)$ as in \eqref{eq:UdefSobolev}, we define a space-time extension operator 
\begin{equation*}
	\EstBox : 
    \cY(\Omega) = L_2\left(I; Y(\Omega)\right) \rightarrow L_2\left(I; U(\square)\right)
    =:\cU(\square),
\end{equation*} 
 and 
\begin{equation}\label{eq:timeExt}
	(\EstBox v)(t) := \EBox (v(t)),
    \qquad t\in I\, \text{a.e.},
    \quad
    v\in \cY(\Omega).
\end{equation}
With this definition, the following result is immediate. 

\begin{proposition}\label{prop:LisBochner}
    Let $\EBox \in \cL_{\text{is}}\left(Y(\Omega), U(\square)\right)$. Then, for $\EstBox$  defined as in \eqref{eq:timeExt}, we have $\EstBox \in \cL_{\text{is}}\left(L_2(I;Y(\Omega)), L_2(I;U(\square))\right)$ .
\end{proposition}
\begin{proof}
    By Corollary\,\ref{cor:Isomorphismbounds}, we have that the function $t \mapsto \Vert \EBox (v(t)) \Vert_{Z(\square)}$, $Z(\square):=W^{m,p}(\square)$, is upper-bounded by the function $t \mapsto C_\square \Vert (v(t)) \Vert_{\cY(\Omega)}$ and lower-bounded by the function $t \mapsto \frac1{c_{\square}} \Vert (v(t)) \Vert_{\cY(\Omega)}$. Thus, $\EstBox v \in L_2(I;U(\square))$ and the operator $\EstBox$ is continuous as well as bounded from below. Using again \cite[Lemma 2.8]{Abramovich2002}, it remains to show that the range of $\EstBox$ is $L_2(I;U(\square))$. Given $v \in L_2(I;U(\square))$, define $w \in L_2(I;Y(\Omega))$ by $w(t):= \RBox(v(t))$ for almost all $t \in I$. With that we have $\EstBox w = v$, which completes the proof. 
\end{proof}

The inverse operator $\EstBox^{-1}$ is denoted by $\RstBox$. Finally, we define $\EstCirc$ and $\RstCirc$ using the operators $\ECirc$, $\RCirc$ from Corollary\,\ref{cor:boundsSobolev}, i.e., $(\EstCirc v)(t):= \ECirc(v(t))$ and $\RstCirc:=\EstCirc^{-1}$. Then, we can put everything together and obtain.

\begin{corollary}\label{cor:boundsBochner}
    With $\cX(\mycirc)$, $\cY(\Omega)$, $\cZ(\square)$ and the above operators, we have for all $f_\square \in \mathcal{E}\left(\astRBox f_\Omega; L_2(I,W_0^{-m, q}(\square))\right)$
	\begin{equation}
		\Vert f_\mycirc \Vert_{L_2((I,W_0^{-m, q}(\mycirc))} 
        \le \Vert f_\Omega \Vert_{L_2(I,W_0^{-m, q}(\Omega))} 
        \le \Vert f_\square \Vert_{L_2(I,W_0^{-m, q}(\square))},
	\end{equation}
	setting $f_\mycirc := \astECirc \left(f_\Omega|_{L_2(I,\cV'(\Omega))}\right)$.
    \hfill\qed
\end{corollary}

