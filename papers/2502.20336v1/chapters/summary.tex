% !TEX root = ../dual_norm_estimate_base.tex
\section{Summary and outlook}
In this work, we introduced a novel framework for rigorously certifying the accuracy of Physics-Informed Neural Networks (PINNs) through a posteriori error estimation. Our method constructs efficiently computable error bounds independent of the PINN training process or the choice of loss function. 

The idea is to extend and restrict the residual to a simpler domain, we derived both upper and lower error bounds that have been shown to be relatively sharp while enhancing computational efficiency. This is achieved by  replacing the potentially complex problem domain by a simpler reference domain and by establishing relationships between functionals defined w.r.t.\ the corresponding spaces on the embedded domains. These relationships yield the desirable bounds, accessed by evaluation of dual norm estimates using Riesz representation as an alternative to using wavelet methods as introduced in \cite{ernst2024certified}. 

Our numerical experiments validate the proposed certification framework for parameter-dependent linear elliptic and time-dependent parabolic PDEs. For a parameter-dependent diffusion problem on a non-convex domain, the evaluation of error bounds is greatly simplified by posing the computations on a much simpler domain with a limited but forgiving loss of sharpness. In the case of an elliptic PDE with a parameterized boundary, our framework efficiently certifies PINN approximations without requiring a re-meshing for each parameter. Similarly, for a linear parabolic equation in a simultaneous space-time variational formulation, our method successfully extends to Bochner spaces. 

 Our numerical experiments provide one possible extension of residual functionals, and further refinements can be made by leveraging Hahn-Banach-type extensions, particularly in Hilbert spaces, leading to potentially sharper estimates. Moreover, the extension operator used here is optimal for $\mathcal{Y} = L_2$, motivating the use of a first-order system formulation. This ensures that the residual remains in $L_2(\Omega)$, improving the accuracy of the certification. Ongoing work includes adaptive wavelet-based approximations in $W_0^{m,p}(\square)$, allowing for error certification in Banach spaces beyond Hilbert settings. A key direction for our future research is to extend the methodology to nonlinear PDEs, where error certification is more challenging but crucial for practical applications.



