% !TEX root = ../dual_norm_estimate_base.tex

\section{Error-residual relations for PDEs} \label{sec:VarFormPDEs}
In this section, we recall the main facts on error-residual relations for PDEs, which is based upon the well-known Hilbert space theory of PDEs yielding well-posed formulations.

\subsection{Well-posedness}
Given a partial differential operator $B$ (eventually starting from a classical version $B^\circ$), we require Hilbert spaces $\cW$, $\cY$ with norms $\|\cdot\|_\cX$ induced by inner products $(\cdot,\cdot)_\cX$, $\cX\in\{\cW,\cY\}$, such that  the PDE
\begin{equation} \label{eq:generalPDE}
	B u  = f
\end{equation}
is \emph{well-posed} for any appropriate right-hand side $f$, by which we mean that \eqref{eq:generalPDE} admits a unique solution, which continuously depends on the data (typically the right-hand side $f$). In order to ensure this, the domain $\cW$ and the range $\cY'$ of the operator $B$ have to be identified such that $B \in \cL_{\text{is}}(\cW, \cY')$\footnote{The space $\cL_{\text{is}}(\cW, \cY')$ is the subspace of isomorphisms from $\cL(\cW, \cY')$, where $(\cL(\cW, \cY'), \Vert \cdot \Vert_{L(\cW, \cY')})$ denotes the space of continuous linear functions from $\cW$ to $\cY'$.}, where $\cY'$ is the topological dual space of $\cY$ equipped with the operator norm 
\begin{equation*}
	\Vert f \Vert_{\cY'} 
    := \sup_{v \in \cY} \frac{f(v)}{\Vert v \Vert_{\cY}}
    = \sup_{v \in \cY} \frac{\langle f, v\rangle_\cY}{\Vert v \Vert_{\cY}},
    \end{equation*} 
i.e.,  $\langle \cdot,\cdot\rangle_\cY\equiv \langle \cdot,\cdot\rangle_{\cY'\times\cY}$ denotes the dual pairing. The reason why we choose the dual $\cY'$ for the range of $B$ lies in the fact that this easily allows one to relate the differential operator to a bilinear form
\begin{align*}
    b:\cW\times\cY\to\mathbb{R}
    \quad\text{via}\quad
    b(w,y) := \langle Bw,y\rangle_{\cY},
    \,\, w\in\cW, y\in\cY.
\end{align*}
 If $B\in\cL_{\text{is}}(\cW, \cY')$, the operator is bijective, which ensures existence and uniqueness for all $f\in\cY'$. Moreover, the inverse is bounded, i.e., 
\begin{align*}
    \Vert B^{-1} \Vert_{\cL(\cY', \cW)} < \infty,
\end{align*}
which will be relevant next.

Assume that we are given some approximation $u^\delta$ of the (exact, but typically unknown) solution $u\in\cW$, e.g.\ determined by some PINN. Then, we are interested in controlling the error $\Vert u - u^{\delta} \Vert_{\cW}$. If $B \in \cL_{\text{is}}(\cW, \cY')$, one can bound  the error by the dual norm of the \emph{residual} $r_B(u^\delta) := f - Bu^\delta\in\cY'$ as
\begin{equation} \label{eq:ErrorResiLinearV1}
	\Vert B \Vert^{-1}_{\cL(\cW, \cY')} \cdot \Vert r_B(u^{\delta}) \Vert_{\cY'} 
    \,\le\, 
    \Vert u - u^{\delta} \Vert_{\cW} 
    \,\le\, 
    \Vert B^{-1} \Vert_{\cL(\cY', \cW)} \cdot \Vert r_B(u^{\delta}) \Vert_{\cY'}
\end{equation}
The constants $c_B:=\Vert B \Vert_{\cL(\cW, \cY')}^{-1} >0 $ on the left and $C_B:=\Vert B^{-1} \Vert_{\cL(\cY', \cW)} <\infty$ on the right of \eqref{eq:ErrorResiLinearV1} are the (inverse of the) continuity and stability (inf-sup) constants, respectively. Then, \eqref{eq:ErrorResiLinearV1} reads
\begin{equation} \label{eq:ErrorResiLinearV2}
	c_B \, \Vert r_B(u^{\delta}) \Vert_{\cY'} 
    \,\le\, 
    \Vert u - u^{\delta} \Vert_{\cW} 
    \,\le\, 
    C_B \, \Vert r_B(u^{\delta}) \Vert_{\cY'}, \quad \forall u^\delta \in \cW.
\end{equation}
Hence, we can bound the error from above and from below by the residual, if we
\begin{compactitem}
    \item  either know or are able to compute or estimate the continuity and stability constants and
    \item are able to evaluate the dual norm of the residual, i.e.
    \begin{equation*}
	\Vert r_B(u^{\delta}) \Vert_{\cY'} = \sup\limits_{v \in \cY} \frac{\langle r_B(u^{\delta}), v \rangle_{\cY}}{\Vert v \Vert_\cY}.
    \end{equation*}
    However, the computation of the supremum is in general not possible.
\end{compactitem}

\subsection{PDEs on domains}
The above introduced abstract spaces $\cW$ and $\cY$ are typically function spaces which are defined upon a physical domain $\Omega \subset \mathbb{R}^d$, on which the PDE is posed. Hence, we sometimes use the notations $\cW(\Omega)$ and $\cY(\Omega)$, also for domains different from the original $\Omega$. Boundary and/or initial conditions are incorporated into the definition of the operator $B$. It will be important later to keep track on the domain $\Omega$.


\begin{example}[Linear elliptic PDE]\label{example:linearPDE}
	Let $\Omega \subset \mathbb{R}^d$ be a Lipschitz domain and let $\cW=\cY=H^1_0(\Omega)$. Given $f\in H^{-1}(\Omega)$, the problem of finding $u \in H^1_0(\Omega)$ satisfying 
	\begin{equation*}
		\langle Bu,v\rangle_{H^1_0(\Omega)} := (A \nabla u, \nabla v )_{L_2(\Omega)} + (b \cdot \nabla u, v)_{L_2(\Omega)} + (c \cdot u, v)_{L_2(\Omega)} = f(v)
	\end{equation*}
	for all $ v \in H^1_0(\Omega)$ is well-posed, if $A \in \left(L_\infty(\Omega)\right)^{d \times d}$, $b \in \left(L_\infty(\Omega)\right)^{d}$, $c \in L_\infty(\Omega)$ and
	\begin{equation*}
		\nabla \cdot b \in L_2(\Omega), \quad c(x) 
        - \tfrac{1}{2} \nabla \cdot b(x) \ge c_0 \quad \forall x \in \Omega \; \text{ a.e. }
	\end{equation*}
	as well as 
	\begin{equation*}
		\xi^T A(x) \xi \ge a_0 \vert \xi \vert^2, \quad \forall \xi \in \mathbb{R}^{d} \; \forall x \in \Omega \; \text{ a.e.,}
	\end{equation*}	
    i.e., $A$ is s.p.d. 
    In this case \eqref{eq:ErrorResiLinearV2} holds with $c = (\Vert A \Vert_\infty + s_{\text{PF}} \Vert b \Vert_\infty + s_{\text{PF}}^2 \Vert c \Vert_\infty)^{-1}$ and $C = a_0^{-1}$, whereby $s_{\text{PF}}$ is the Poincar\'{e} constant for $\Omega$.\hfill$\diamond$
\end{example}

\begin{example}[Space-time variational form of parabolic PDEs]\label{Ex:SpaceTime}
    Let the PDE operator $A\in\cL_{\text{is}}(W(\Omega),Y'(\Omega))$ be well-posed on some domain $\Omega\subset\mathbb{R}^d$ for appropriate function spaces $W(\Omega)$ and $Y(\Omega)$. Then, given $f\in\cY'=L_2(I;Y'(\Omega)):=\{ g:I\to Y'(\Omega): \| g\|_{L_2(I;Y'(\Omega))}^2:=\int_T \| g(t)\|_{Y'(\Omega)}^2\, dt<\infty\}$, we seek a solution $u\in\cW$ (with $\cW$ to be defined) of the evolution problem $Bu:=\dot{u}+Au=f$ in $\cY'$ and $u(0)=0$. It is well-known that the Bochner space $\cW:=\{ w\in L_2(I;W(\Omega)): \dot{w}\in L_2(I;W'(\Omega)), w(0)=0\}$ equipped with the graph norm $\Vert u \Vert_{\cW}^2 := \Vert \partial_t u \Vert_{L_2(I,W'(\Omega))}^2 + \Vert u \Vert_{L_2(I,W(\Omega))}^2$
    yields $B\in\cL_{\text{is}}(\cW,\cY')$, \cite{lions2000mathematical,schwab2009space}. This is known as the space-time variational formulation of parabolic PDEs, see also \cite{CheginiStevenson,SpaceTimeUrbanPatera}. If $A$ is elliptic, one has $W(\Omega)=Y(\Omega)=H^1_0(\Omega)$.
    \hfill$\diamond$
\end{example}

\subsection{Dual norm estimates}
We will now address the problem of finding a computable surrogate for the dual norm of the residual for well-posed PDE operator equations. To simplify the notation we consider general functionals $f \in \cY'$ instead of the residual $r_\Omega(u^\delta)$. We want to calculate or estimate the dual norm of the functional $f \in \cY'$, i.e., we seek a feasible surrogate $\eta(f)$ such that either
\begin{subequations}\label{eq:dualnorm}
\begin{align}
	\eta(f)  
        &=\Vert f \Vert_{\cY'}, 
        && \text{(identity)}
        \label{eq:dualnormEqual}\\
	\exists\, C,c >0:\quad c \, \eta(f)
        &\le \Vert f \Vert_{\cY'} 
        \le C \, \eta(f), \quad \text{or} 
        && \text{(estimate)}
        \label{eq:dualnormEst} \\
	\exists\, C > 0:\quad  \Vert f \Vert_{\cY'} 
        &\le C \, \eta(f) 
        && \text{(bound)}
        \label{eq:dualnormUpEst}
\end{align}
\end{subequations}
holds. Ideally, we find $\eta(f)$ such that identity \eqref{eq:dualnormEqual} is fulfilled, but this is typically not possible. Hence, one resorts to an estimate \eqref{eq:dualnormEst}, which ensures that $\eta(f)$ encloses the desired quantity and $\Vert f \Vert_{\cY'}$ is small if and only if $\eta(f)$ is. If this is still not achievable, a bound \eqref{eq:dualnormUpEst}, if not too loose, guarantees at least that the desired quantity is below a threshold, if the bound is so. 

By applying \eqref{eq:dualnormEqual} or \eqref{eq:dualnormEst} to \eqref{eq:ErrorResiLinearV2}, the error is controlled by $\eta$ as
\begin{equation*}
	c \cdot \eta \left( r_\Omega(u^\delta) \right) \le \Vert u - u_{\delta} \Vert_{\cW} \le C \cdot \eta \left( r_\Omega(u^\delta) \right), \quad c, C > 0
\end{equation*}
or by using \eqref{eq:dualnormUpEst} we have
\begin{equation*}
	\Vert u - u_{\delta} \Vert_{\cW} \le C \cdot \eta \left( r_\Omega(u^\delta) \right), \quad C > 0.
\end{equation*}
Notice that the constants $c$ and $C$ in the last two equations could be different to those of \eqref{eq:ErrorResiLinearV2}.

In the next section we shall describe how to find an efficient surrogate $\eta$ for PINNs, such that one of the equations \eqref{eq:dualnorm} is satisfied. It is clear that the structure of the underlying space $\cY$ is crucial for the derivation. 

\subsection{Riesz representation} \label{subsec:RieszReps}
Since $\cY$ is a Hilbert space, the Riesz Representation Theorem\,(see e.g. \cite{aubin2011applied}) states that $\eta(f)$ fulfilling \eqref{eq:dualnormEqual} is given by
\begin{equation*}
	\eta(f) := \Vert R^{-1}_{\cY} f \Vert_{\cY},
\end{equation*}
where $R_\cY: \cY \rightarrow \cY'$ denotes the Riesz-operator defined through the equation $(u, v)_\cY = \langle R_{\cY} u, v \rangle_{\cY}, \; u, v \in \cY$, where $(\cdot,\cdot)_\cY$ denotes the inner product in $\cY$. The Riesz representative $\hat{f} := R^{-1}_{\cY} f \in \cY$ is found by solving the equation
\begin{equation} \label{eq:RieszRep}
	(\hat{f}, v)_\cY = f(v), \quad \forall v \in \cY.
\end{equation}
Since $R_\cY$ is isometric we have $\eta(f) = \Vert f \Vert_{\cY'}$ and thus \eqref{eq:dualnormEqual}. Usually, evaluating $\Vert \cdot \Vert_{\cY}$ is much easier than evaluating $\Vert \cdot \Vert_{\cY'}$. Unfortunately, calculating the Riesz representative with \eqref{eq:RieszRep} is still unfeasible because the space $\cY$ is in almost all cases infinite-dimensional. Therefore, $\cY$ is replaced by a sufficiently high but finite-dimensional subspace $\cY^\delta \subset \cY$, e.g., a finite element space. The Riesz representative $\hat{f} \in \cY$ is then approximated by $\hat{f}^\delta \in \cY^\delta$, which is the solution of
\begin{equation} \label{eq:RieszRepFinite}
	(\hat{f}^\delta, v^\delta)_\cY = f(v^\delta), \quad \forall v^\delta \in \cY^\delta.
\end{equation}

\begin{remark}
If $\cY$ is not a Hilbert, but only a Banach space, the Riesz Representation Theorem is not applicable. However, one way out is to use wavelet methods to directly bound the dual norm $\Vert f \Vert_{\cY'}$ as proposed in\,\cite{ernst2024certified}. 
\end{remark}

With an increasing dimension of $\cY^\delta$ we can at least hope to have convergence of $\hat{f}^\delta $ to $\hat{f}$ in $\cY$ and therefore convergence of $\Vert \hat{f}^\delta \Vert_{\cY}$ to $\Vert f \Vert_{\cY'}$. The drawback of this approach is that, in the case of finite elements, the space $\cY^\delta$ requires a discretization of the underlying domain $\Omega$.
