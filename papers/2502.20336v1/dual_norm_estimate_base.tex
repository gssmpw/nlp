%=================================================================
\documentclass{amsart} 


\newtheorem{theorem}{Theorem}[section]
\newtheorem{proposition}[theorem]{Proposition}
\newtheorem{lemma}[theorem]{Lemma}
\newtheorem{corollary}[theorem]{Corollary}
\newtheorem{assumption}[theorem]{Assumption}
\newtheorem{remark}[theorem]{Remark}

\newcounter{example}
\theoremstyle{definition}
\newtheorem{example}[theorem]{Example}

\numberwithin{equation}{section}

\usepackage{paralist}
\usepackage[hidelinks]{hyperref}


\usepackage{booktabs}
\usepackage{cite}
\usepackage{enumitem}
\usepackage{float}
\usepackage{amssymb}
\usepackage{mathtools}
\usepackage{url}
% for algorithms
\usepackage[linesnumbered,ruled,vlined]{algorithm2e}
\usepackage{csquotes}
\newcommand{\Vvert}{{\vert\kern-0.25ex\vert\kern-0.25ex\vert}}
\usepackage{tikz,pgfplots}
\pgfplotsset{select coords between index/.style 2 args={
		x filter/.code={
			\ifnum\coordindex<#1\def\pgfmathresult{}\fi
			\ifnum\coordindex>#2\def\pgfmathresult{}\fi
		}
}}

%========================
\usepackage{textcomp}
\hbadness=10000
\vbadness=10000
\usepackage[textsize=tiny]{todonotes}
\newcommand{\cA}{\mathcal{A}}
\newcommand{\cB}{\mathcal{B}}
\newcommand{\cL}{\mathcal{L}}
\newcommand{\cP}{\mathcal{P}}
\newcommand{\cU}{\mathcal{U}}
\newcommand{\cV}{\mathcal{V}}
\newcommand{\cW}{\mathcal{W}}
\newcommand{\cX}{\mathcal{X}}
\newcommand{\cY}{\mathcal{Y}}
\newcommand{\cZ}{\mathcal{Z}}

\newcommand{\R}{\mathbb{R}}
\newcommand{\X}{\mathbb{X}}
\newcommand{\Y}{\mathbb{Y}}

% extension and restriction operator
\newcommand{\Ext}{E}
\newcommand{\Res}{R}
% adjoints
\newcommand{\aExt}{E'}
\newcommand{\aRes}{R'}

\newcommand{\EBox}{\Ext_{\Omega\rightarrow\square}}
\newcommand{\RBox}{\Res_{\Omega\leftarrow\square}}

\newcommand{\aEBox}{\aExt_{\Omega\leftarrow\square}}
\newcommand{\aRBox}{\aRes_{\Omega\rightarrow\square}}

\newcommand{\ECirc}{\Ext_{\mycirc\rightarrow\Omega}}
\newcommand{\RCirc}{\Res_{\mycirc\leftarrow\Omega}}

\newcommand{\aECirc}{\aExt_{\mycirc\leftarrow\Omega}}
\newcommand{\aRCirc}{\aRes_{\mycirc\rightarrow\Omega}}

%-----
% now space-time

\newcommand{\EstBox}{\Ext_{I\times\Omega\rightarrow I\times\square}}
\newcommand{\RstBox}{\Res_{I\times\Omega\leftarrow I\times\square}}

\newcommand{\astEBox}{\aExt_{I\times\Omega\leftarrow I\times\square}}
\newcommand{\astRBox}{\aRes_{I\times\Omega\rightarrow I\times\square}}

\newcommand{\EstCirc}{\Ext_{I\times\mycirc\rightarrow I\times\Omega}}
\newcommand{\RstCirc}{\Res_{I\times\mycirc\leftarrow I\times\Omega}}

\newcommand{\astECirc}{\aExt_{I\times\mycirc\leftarrow I\times\Omega}}
\newcommand{\astRCirc}{\aRes_{I\times\mycirc\rightarrow I\times\Omega}}


%\usepackage{relsize}
%\newcommand*{\mycirc}{\mathrel{\scalebox{0.8}{$\bigcirc$}}}
\usepackage{wasysym}
\newcommand{\mycirc}{\text{\Circle}}

% identity operators
\DeclareMathOperator{\id}{id}
\DeclareMathOperator{\dom}{dom}
\DeclareMathOperator{\range}{range}

% limiters
\DeclarePairedDelimiter{\dual}{\langle}{\rangle}
%========================
% reduce space below figures
\setlength{\belowcaptionskip}{-10pt}
%========================
\begin{document}

\title[A posteriori ertification for PINNs]{A posteriori Certification for physics-informed neural networks}

\author{Lewin Ernst}
\address{Institute for Numerical Mathematics, Ulm University, Helmholtzstr. 20, 89081 Ulm, Germany}
\email{lewin.ernst@uni-ulm.de}

\author{Nikolaos Rekatsinas}
\address{Institute of Applied and Computational Mathematics, Foundation of Research and Technology, Nikolaou Plastira 100, Vassilika Vouton,
GR 70013 Heraklion, Crete, Greece}
\email{n.rekatsinas@iacm.forth.gr}

\author{Karsten Urban}
\address{Institute for Numerical Mathematics, Ulm University, Helmholtzstr. 20, 89081 Ulm, Germany}
\email{karsten.urban@uni-ulm.de}


\begin{abstract}
	We propose rigorous lower and upper bounds for a PINN approximation to PDEs by efficiently computing the Riesz representations of suitable extension and restrictions of the PINN residual towards geometrically simpler domains, which are either embedded or enveloping the original domain. Error bounds are proven and detailed for elliptic as well as parabolic problems. Numerical experiments show the good quantitative behaviour of the derived upper and lower error bounds.
\end{abstract}

\keywords{
	Physics Informed Neural Networks; A Posteriori Error Bound; Parameterized Partial Differential Equations; Extension and Restriction of Functionals}

\subjclass{35J20, % Variational methods for second-order elliptic equations
	65M15, % Error bounds 
	68T07% Artifcial neural networks and deep learning
}


\maketitle

\section{Introduction}
\label{sec:introduction}
The business processes of organizations are experiencing ever-increasing complexity due to the large amount of data, high number of users, and high-tech devices involved \cite{martin2021pmopportunitieschallenges, beerepoot2023biggestbpmproblems}. This complexity may cause business processes to deviate from normal control flow due to unforeseen and disruptive anomalies \cite{adams2023proceddsriftdetection}. These control-flow anomalies manifest as unknown, skipped, and wrongly-ordered activities in the traces of event logs monitored from the execution of business processes \cite{ko2023adsystematicreview}. For the sake of clarity, let us consider an illustrative example of such anomalies. Figure \ref{FP_ANOMALIES} shows a so-called event log footprint, which captures the control flow relations of four activities of a hypothetical event log. In particular, this footprint captures the control-flow relations between activities \texttt{a}, \texttt{b}, \texttt{c} and \texttt{d}. These are the causal ($\rightarrow$) relation, concurrent ($\parallel$) relation, and other ($\#$) relations such as exclusivity or non-local dependency \cite{aalst2022pmhandbook}. In addition, on the right are six traces, of which five exhibit skipped, wrongly-ordered and unknown control-flow anomalies. For example, $\langle$\texttt{a b d}$\rangle$ has a skipped activity, which is \texttt{c}. Because of this skipped activity, the control-flow relation \texttt{b}$\,\#\,$\texttt{d} is violated, since \texttt{d} directly follows \texttt{b} in the anomalous trace.
\begin{figure}[!t]
\centering
\includegraphics[width=0.9\columnwidth]{images/FP_ANOMALIES.png}
\caption{An example event log footprint with six traces, of which five exhibit control-flow anomalies.}
\label{FP_ANOMALIES}
\end{figure}

\subsection{Control-flow anomaly detection}
Control-flow anomaly detection techniques aim to characterize the normal control flow from event logs and verify whether these deviations occur in new event logs \cite{ko2023adsystematicreview}. To develop control-flow anomaly detection techniques, \revision{process mining} has seen widespread adoption owing to process discovery and \revision{conformance checking}. On the one hand, process discovery is a set of algorithms that encode control-flow relations as a set of model elements and constraints according to a given modeling formalism \cite{aalst2022pmhandbook}; hereafter, we refer to the Petri net, a widespread modeling formalism. On the other hand, \revision{conformance checking} is an explainable set of algorithms that allows linking any deviations with the reference Petri net and providing the fitness measure, namely a measure of how much the Petri net fits the new event log \cite{aalst2022pmhandbook}. Many control-flow anomaly detection techniques based on \revision{conformance checking} (hereafter, \revision{conformance checking}-based techniques) use the fitness measure to determine whether an event log is anomalous \cite{bezerra2009pmad, bezerra2013adlogspais, myers2018icsadpm, pecchia2020applicationfailuresanalysispm}. 

The scientific literature also includes many \revision{conformance checking}-independent techniques for control-flow anomaly detection that combine specific types of trace encodings with machine/deep learning \cite{ko2023adsystematicreview, tavares2023pmtraceencoding}. Whereas these techniques are very effective, their explainability is challenging due to both the type of trace encoding employed and the machine/deep learning model used \cite{rawal2022trustworthyaiadvances,li2023explainablead}. Hence, in the following, we focus on the shortcomings of \revision{conformance checking}-based techniques to investigate whether it is possible to support the development of competitive control-flow anomaly detection techniques while maintaining the explainable nature of \revision{conformance checking}.
\begin{figure}[!t]
\centering
\includegraphics[width=\columnwidth]{images/HIGH_LEVEL_VIEW.png}
\caption{A high-level view of the proposed framework for combining \revision{process mining}-based feature extraction with dimensionality reduction for control-flow anomaly detection.}
\label{HIGH_LEVEL_VIEW}
\end{figure}

\subsection{Shortcomings of \revision{conformance checking}-based techniques}
Unfortunately, the detection effectiveness of \revision{conformance checking}-based techniques is affected by noisy data and low-quality Petri nets, which may be due to human errors in the modeling process or representational bias of process discovery algorithms \cite{bezerra2013adlogspais, pecchia2020applicationfailuresanalysispm, aalst2016pm}. Specifically, on the one hand, noisy data may introduce infrequent and deceptive control-flow relations that may result in inconsistent fitness measures, whereas, on the other hand, checking event logs against a low-quality Petri net could lead to an unreliable distribution of fitness measures. Nonetheless, such Petri nets can still be used as references to obtain insightful information for \revision{process mining}-based feature extraction, supporting the development of competitive and explainable \revision{conformance checking}-based techniques for control-flow anomaly detection despite the problems above. For example, a few works outline that token-based \revision{conformance checking} can be used for \revision{process mining}-based feature extraction to build tabular data and develop effective \revision{conformance checking}-based techniques for control-flow anomaly detection \cite{singh2022lapmsh, debenedictis2023dtadiiot}. However, to the best of our knowledge, the scientific literature lacks a structured proposal for \revision{process mining}-based feature extraction using the state-of-the-art \revision{conformance checking} variant, namely alignment-based \revision{conformance checking}.

\subsection{Contributions}
We propose a novel \revision{process mining}-based feature extraction approach with alignment-based \revision{conformance checking}. This variant aligns the deviating control flow with a reference Petri net; the resulting alignment can be inspected to extract additional statistics such as the number of times a given activity caused mismatches \cite{aalst2022pmhandbook}. We integrate this approach into a flexible and explainable framework for developing techniques for control-flow anomaly detection. The framework combines \revision{process mining}-based feature extraction and dimensionality reduction to handle high-dimensional feature sets, achieve detection effectiveness, and support explainability. Notably, in addition to our proposed \revision{process mining}-based feature extraction approach, the framework allows employing other approaches, enabling a fair comparison of multiple \revision{conformance checking}-based and \revision{conformance checking}-independent techniques for control-flow anomaly detection. Figure \ref{HIGH_LEVEL_VIEW} shows a high-level view of the framework. Business processes are monitored, and event logs obtained from the database of information systems. Subsequently, \revision{process mining}-based feature extraction is applied to these event logs and tabular data input to dimensionality reduction to identify control-flow anomalies. We apply several \revision{conformance checking}-based and \revision{conformance checking}-independent framework techniques to publicly available datasets, simulated data of a case study from railways, and real-world data of a case study from healthcare. We show that the framework techniques implementing our approach outperform the baseline \revision{conformance checking}-based techniques while maintaining the explainable nature of \revision{conformance checking}.

In summary, the contributions of this paper are as follows.
\begin{itemize}
    \item{
        A novel \revision{process mining}-based feature extraction approach to support the development of competitive and explainable \revision{conformance checking}-based techniques for control-flow anomaly detection.
    }
    \item{
        A flexible and explainable framework for developing techniques for control-flow anomaly detection using \revision{process mining}-based feature extraction and dimensionality reduction.
    }
    \item{
        Application to synthetic and real-world datasets of several \revision{conformance checking}-based and \revision{conformance checking}-independent framework techniques, evaluating their detection effectiveness and explainability.
    }
\end{itemize}

The rest of the paper is organized as follows.
\begin{itemize}
    \item Section \ref{sec:related_work} reviews the existing techniques for control-flow anomaly detection, categorizing them into \revision{conformance checking}-based and \revision{conformance checking}-independent techniques.
    \item Section \ref{sec:abccfe} provides the preliminaries of \revision{process mining} to establish the notation used throughout the paper, and delves into the details of the proposed \revision{process mining}-based feature extraction approach with alignment-based \revision{conformance checking}.
    \item Section \ref{sec:framework} describes the framework for developing \revision{conformance checking}-based and \revision{conformance checking}-independent techniques for control-flow anomaly detection that combine \revision{process mining}-based feature extraction and dimensionality reduction.
    \item Section \ref{sec:evaluation} presents the experiments conducted with multiple framework and baseline techniques using data from publicly available datasets and case studies.
    \item Section \ref{sec:conclusions} draws the conclusions and presents future work.
\end{itemize}
% !TEX root = ../dual_norm_estimate_base.tex

\section{PINNs for solving PDEs} \label{sec:PINNs}
We are going to briefly introduce the main concepts of PINNs without going into details, since we only aim to use PINNs as a black box.

\subsection{PDEs in classical form}
Let $\Omega \subset \mathbb{R}^d$ be a bounded open domain and let $m\in\mathbb{N}$ denote the order of the PDE ($m=2$ for Laplace's equation). We denote by
\begin{align*}
    B^\circ: C^m(\Omega)\to C^0(\Omega)
\end{align*}
the classical (point-wise) form of the differential operator under consideration. Then, given some $f^\circ\in C(\Omega)$, we call $u\in C^m(\Omega)$ a \emph{classical solution} if 
\begin{equation} \label{eq:clPDE}
	(B^\circ u)(x) = f^\circ(x), \quad \forall x \in \Omega,
\end{equation}
where we assume that proper boundary and/or initial conditions are incorporated into the definition of the operator.

Next, we are given some approximation $u^\delta$ to $u$, e.g.\ in terms of a PINN and define the (classical) \emph{residual} of \eqref{eq:clPDE} by
\begin{align}\label{eq:residual}
    r_\Omega^\circ (u^\delta)(x) := f(x) - (B^\circ u^\delta)(x), \quad x\in \Omega. 
\end{align}
Given $u^\delta$, the residual is in principle computable by inserting $u^\delta$ into the PDE operator.

However, such classical solutions often do not exist, depending on the data $B^\circ$, $f^\circ$ and $\Omega$. It is well-known that well-posedness of such problems is usually linked to suitable variational formulations as described in Section \ref{sec:VarFormPDEs} below. 


\subsection{PINNs: Definition and training}
The core idea of PINNs is to use the residual for the definition of a loss function within the training of a neural network. Let us briefly describe this for the above classical solution concept, even though (i) there are several other approaches in the literature and (ii) our subsequent error analysis does not depend on the choice of the loss function for a PINN.

\subsubsection*{Neural networks (NNs)} 
The notation of NNs in this paragraph is based upon \cite{Berner2021,Gribonval2021,Petersen2018}. A NN is a function $\Phi_a(\cdot;\theta):\mathbb{R}^{N_{0}} \rightarrow \mathbb{R}^{N_{L}}$, where $a$ is the \emph{architecture} and $\theta$ are the \emph{parameters}. Both the architecture and parameters determine the input-output function $\Phi_a(\cdot;\theta)$ of the NN. In case of a feed-forward NN, the architecture $a=(N,\rho)$ can be described by the vector of neurons per layer $N=(N_0,...,N_L)\in\mathbb{N}^{L+1}$, where $L \in \mathbb{N}$ denotes the number of \emph{layers} ($N_0$ being the input and $N_L$ the output dimension) and the \emph{activation function} $\rho: \mathbb{R} \rightarrow \mathbb{R}$.

The parameters of the NN read $\theta = (W^{(l)},b^{(l)})_{l=1,...,L}$, where $W^{(l)} \in \mathbb{R}^{N_{l} \times N_{l-1}}$ are the \emph{weight matrices} and $b^{(l)} \in \mathbb{R}^{N_{l}}$ are called \emph{bias vectors}. 
The output $\Phi_a(z;\theta)$ of the NN for an input $z\in\mathbb{R}^{N_0}$ is then defined as $\Phi_{a}(z;\theta) := \Phi^{(L)}(z;\theta)$, where
\begin{align*}
	\Phi^{(1)}(z;\theta) &= W^{(1)} z + b^{(1)}, \\
	\hat{\Phi}^{(l)}(z;\theta) &= \rho(\Phi^{(l)}(z;\theta)), \quad l=1,...,L-1, \quad \text{and} \\
	\Phi^{(l+1)}(z;\theta) &= W^{(l+1)} \hat{\Phi}^{(l)}(z;\theta) + b^{(l+1)}, \quad l = 1,...,L-1,
\end{align*}
and $\rho$ is applied component-wise. In the following the architecture is omitted in the notation as we view it as being fixed once and for all. 

NNs seem suitable for solving PDEs because they are universal function approximators, see \cite{Hornik1991,Hornik1989,Guehring2019}. For a NN $\Phi^\theta := \Phi(\cdot;\theta)$ to approximate the solution of a PDE, the parameters $\theta$ must be determined, which is done in the training phase. Thereby, a tailored minimization problem is defined, with which the parameters are \emph{learned}. In the regime of PINNs the function to be minimized involves the PDE (e.g.\ in classical form \eqref{eq:clPDE}), which is the reason for the name \emph{physics-informed}.

With respect to the classical form of the PDE, a set of sample points $\mathcal{S}_\Omega$ in $\Omega$ are chosen and \eqref{eq:clPDE} is posed only for those points. This leads to the definition of the loss function
\begin{equation}\label{eq:lossfunc}
	\mathcal{L}(\theta) 
    := \sum_{x \in \mathcal{S}_\Omega} 
    \vert (r^\circ_\Omega (\Phi^\theta))(x) \vert^2.
\end{equation}
Often, an additional sampling is required for satisfying the boundary conditions. We shall assume that $\Phi^\theta$ satisfies given boundary condition as this can be achieved with the aid of approximate distance functions, see \cite{sukumar2022exact}. 

Such classical PINNs have e.g.\ been investigated for a broad scope of linear and nonlinear PDEs, see e.g. \cite{Berg2018,Raissi2019,Mao2020,Cai2021,Cai2021a,Hu2024}. The main advantages of the method are its straightforward applicability and, due to the sampling of $\mathcal{S}_\Omega$, it results in a mesh-free approximation. 

One can also replace the above loss function by terms stemming from a variational formulation of a given PDE, e.g.\ VPINNs. Again, as the specific form of the PINN is not relevant for our paper, we will not go into more details and only assume that $\Phi^\theta$ gives some approximation for the solution of a given PDE.

% !TEX root = ../dual_norm_estimate_base.tex

\section{Error-residual relations for PDEs} \label{sec:VarFormPDEs}
In this section, we recall the main facts on error-residual relations for PDEs, which is based upon the well-known Hilbert space theory of PDEs yielding well-posed formulations.

\subsection{Well-posedness}
Given a partial differential operator $B$ (eventually starting from a classical version $B^\circ$), we require Hilbert spaces $\cW$, $\cY$ with norms $\|\cdot\|_\cX$ induced by inner products $(\cdot,\cdot)_\cX$, $\cX\in\{\cW,\cY\}$, such that  the PDE
\begin{equation} \label{eq:generalPDE}
	B u  = f
\end{equation}
is \emph{well-posed} for any appropriate right-hand side $f$, by which we mean that \eqref{eq:generalPDE} admits a unique solution, which continuously depends on the data (typically the right-hand side $f$). In order to ensure this, the domain $\cW$ and the range $\cY'$ of the operator $B$ have to be identified such that $B \in \cL_{\text{is}}(\cW, \cY')$\footnote{The space $\cL_{\text{is}}(\cW, \cY')$ is the subspace of isomorphisms from $\cL(\cW, \cY')$, where $(\cL(\cW, \cY'), \Vert \cdot \Vert_{L(\cW, \cY')})$ denotes the space of continuous linear functions from $\cW$ to $\cY'$.}, where $\cY'$ is the topological dual space of $\cY$ equipped with the operator norm 
\begin{equation*}
	\Vert f \Vert_{\cY'} 
    := \sup_{v \in \cY} \frac{f(v)}{\Vert v \Vert_{\cY}}
    = \sup_{v \in \cY} \frac{\langle f, v\rangle_\cY}{\Vert v \Vert_{\cY}},
    \end{equation*} 
i.e.,  $\langle \cdot,\cdot\rangle_\cY\equiv \langle \cdot,\cdot\rangle_{\cY'\times\cY}$ denotes the dual pairing. The reason why we choose the dual $\cY'$ for the range of $B$ lies in the fact that this easily allows one to relate the differential operator to a bilinear form
\begin{align*}
    b:\cW\times\cY\to\mathbb{R}
    \quad\text{via}\quad
    b(w,y) := \langle Bw,y\rangle_{\cY},
    \,\, w\in\cW, y\in\cY.
\end{align*}
 If $B\in\cL_{\text{is}}(\cW, \cY')$, the operator is bijective, which ensures existence and uniqueness for all $f\in\cY'$. Moreover, the inverse is bounded, i.e., 
\begin{align*}
    \Vert B^{-1} \Vert_{\cL(\cY', \cW)} < \infty,
\end{align*}
which will be relevant next.

Assume that we are given some approximation $u^\delta$ of the (exact, but typically unknown) solution $u\in\cW$, e.g.\ determined by some PINN. Then, we are interested in controlling the error $\Vert u - u^{\delta} \Vert_{\cW}$. If $B \in \cL_{\text{is}}(\cW, \cY')$, one can bound  the error by the dual norm of the \emph{residual} $r_B(u^\delta) := f - Bu^\delta\in\cY'$ as
\begin{equation} \label{eq:ErrorResiLinearV1}
	\Vert B \Vert^{-1}_{\cL(\cW, \cY')} \cdot \Vert r_B(u^{\delta}) \Vert_{\cY'} 
    \,\le\, 
    \Vert u - u^{\delta} \Vert_{\cW} 
    \,\le\, 
    \Vert B^{-1} \Vert_{\cL(\cY', \cW)} \cdot \Vert r_B(u^{\delta}) \Vert_{\cY'}
\end{equation}
The constants $c_B:=\Vert B \Vert_{\cL(\cW, \cY')}^{-1} >0 $ on the left and $C_B:=\Vert B^{-1} \Vert_{\cL(\cY', \cW)} <\infty$ on the right of \eqref{eq:ErrorResiLinearV1} are the (inverse of the) continuity and stability (inf-sup) constants, respectively. Then, \eqref{eq:ErrorResiLinearV1} reads
\begin{equation} \label{eq:ErrorResiLinearV2}
	c_B \, \Vert r_B(u^{\delta}) \Vert_{\cY'} 
    \,\le\, 
    \Vert u - u^{\delta} \Vert_{\cW} 
    \,\le\, 
    C_B \, \Vert r_B(u^{\delta}) \Vert_{\cY'}, \quad \forall u^\delta \in \cW.
\end{equation}
Hence, we can bound the error from above and from below by the residual, if we
\begin{compactitem}
    \item  either know or are able to compute or estimate the continuity and stability constants and
    \item are able to evaluate the dual norm of the residual, i.e.
    \begin{equation*}
	\Vert r_B(u^{\delta}) \Vert_{\cY'} = \sup\limits_{v \in \cY} \frac{\langle r_B(u^{\delta}), v \rangle_{\cY}}{\Vert v \Vert_\cY}.
    \end{equation*}
    However, the computation of the supremum is in general not possible.
\end{compactitem}

\subsection{PDEs on domains}
The above introduced abstract spaces $\cW$ and $\cY$ are typically function spaces which are defined upon a physical domain $\Omega \subset \mathbb{R}^d$, on which the PDE is posed. Hence, we sometimes use the notations $\cW(\Omega)$ and $\cY(\Omega)$, also for domains different from the original $\Omega$. Boundary and/or initial conditions are incorporated into the definition of the operator $B$. It will be important later to keep track on the domain $\Omega$.


\begin{example}[Linear elliptic PDE]\label{example:linearPDE}
	Let $\Omega \subset \mathbb{R}^d$ be a Lipschitz domain and let $\cW=\cY=H^1_0(\Omega)$. Given $f\in H^{-1}(\Omega)$, the problem of finding $u \in H^1_0(\Omega)$ satisfying 
	\begin{equation*}
		\langle Bu,v\rangle_{H^1_0(\Omega)} := (A \nabla u, \nabla v )_{L_2(\Omega)} + (b \cdot \nabla u, v)_{L_2(\Omega)} + (c \cdot u, v)_{L_2(\Omega)} = f(v)
	\end{equation*}
	for all $ v \in H^1_0(\Omega)$ is well-posed, if $A \in \left(L_\infty(\Omega)\right)^{d \times d}$, $b \in \left(L_\infty(\Omega)\right)^{d}$, $c \in L_\infty(\Omega)$ and
	\begin{equation*}
		\nabla \cdot b \in L_2(\Omega), \quad c(x) 
        - \tfrac{1}{2} \nabla \cdot b(x) \ge c_0 \quad \forall x \in \Omega \; \text{ a.e. }
	\end{equation*}
	as well as 
	\begin{equation*}
		\xi^T A(x) \xi \ge a_0 \vert \xi \vert^2, \quad \forall \xi \in \mathbb{R}^{d} \; \forall x \in \Omega \; \text{ a.e.,}
	\end{equation*}	
    i.e., $A$ is s.p.d. 
    In this case \eqref{eq:ErrorResiLinearV2} holds with $c = (\Vert A \Vert_\infty + s_{\text{PF}} \Vert b \Vert_\infty + s_{\text{PF}}^2 \Vert c \Vert_\infty)^{-1}$ and $C = a_0^{-1}$, whereby $s_{\text{PF}}$ is the Poincar\'{e} constant for $\Omega$.\hfill$\diamond$
\end{example}

\begin{example}[Space-time variational form of parabolic PDEs]\label{Ex:SpaceTime}
    Let the PDE operator $A\in\cL_{\text{is}}(W(\Omega),Y'(\Omega))$ be well-posed on some domain $\Omega\subset\mathbb{R}^d$ for appropriate function spaces $W(\Omega)$ and $Y(\Omega)$. Then, given $f\in\cY'=L_2(I;Y'(\Omega)):=\{ g:I\to Y'(\Omega): \| g\|_{L_2(I;Y'(\Omega))}^2:=\int_T \| g(t)\|_{Y'(\Omega)}^2\, dt<\infty\}$, we seek a solution $u\in\cW$ (with $\cW$ to be defined) of the evolution problem $Bu:=\dot{u}+Au=f$ in $\cY'$ and $u(0)=0$. It is well-known that the Bochner space $\cW:=\{ w\in L_2(I;W(\Omega)): \dot{w}\in L_2(I;W'(\Omega)), w(0)=0\}$ equipped with the graph norm $\Vert u \Vert_{\cW}^2 := \Vert \partial_t u \Vert_{L_2(I,W'(\Omega))}^2 + \Vert u \Vert_{L_2(I,W(\Omega))}^2$
    yields $B\in\cL_{\text{is}}(\cW,\cY')$, \cite{lions2000mathematical,schwab2009space}. This is known as the space-time variational formulation of parabolic PDEs, see also \cite{CheginiStevenson,SpaceTimeUrbanPatera}. If $A$ is elliptic, one has $W(\Omega)=Y(\Omega)=H^1_0(\Omega)$.
    \hfill$\diamond$
\end{example}

\subsection{Dual norm estimates}
We will now address the problem of finding a computable surrogate for the dual norm of the residual for well-posed PDE operator equations. To simplify the notation we consider general functionals $f \in \cY'$ instead of the residual $r_\Omega(u^\delta)$. We want to calculate or estimate the dual norm of the functional $f \in \cY'$, i.e., we seek a feasible surrogate $\eta(f)$ such that either
\begin{subequations}\label{eq:dualnorm}
\begin{align}
	\eta(f)  
        &=\Vert f \Vert_{\cY'}, 
        && \text{(identity)}
        \label{eq:dualnormEqual}\\
	\exists\, C,c >0:\quad c \, \eta(f)
        &\le \Vert f \Vert_{\cY'} 
        \le C \, \eta(f), \quad \text{or} 
        && \text{(estimate)}
        \label{eq:dualnormEst} \\
	\exists\, C > 0:\quad  \Vert f \Vert_{\cY'} 
        &\le C \, \eta(f) 
        && \text{(bound)}
        \label{eq:dualnormUpEst}
\end{align}
\end{subequations}
holds. Ideally, we find $\eta(f)$ such that identity \eqref{eq:dualnormEqual} is fulfilled, but this is typically not possible. Hence, one resorts to an estimate \eqref{eq:dualnormEst}, which ensures that $\eta(f)$ encloses the desired quantity and $\Vert f \Vert_{\cY'}$ is small if and only if $\eta(f)$ is. If this is still not achievable, a bound \eqref{eq:dualnormUpEst}, if not too loose, guarantees at least that the desired quantity is below a threshold, if the bound is so. 

By applying \eqref{eq:dualnormEqual} or \eqref{eq:dualnormEst} to \eqref{eq:ErrorResiLinearV2}, the error is controlled by $\eta$ as
\begin{equation*}
	c \cdot \eta \left( r_\Omega(u^\delta) \right) \le \Vert u - u_{\delta} \Vert_{\cW} \le C \cdot \eta \left( r_\Omega(u^\delta) \right), \quad c, C > 0
\end{equation*}
or by using \eqref{eq:dualnormUpEst} we have
\begin{equation*}
	\Vert u - u_{\delta} \Vert_{\cW} \le C \cdot \eta \left( r_\Omega(u^\delta) \right), \quad C > 0.
\end{equation*}
Notice that the constants $c$ and $C$ in the last two equations could be different to those of \eqref{eq:ErrorResiLinearV2}.

In the next section we shall describe how to find an efficient surrogate $\eta$ for PINNs, such that one of the equations \eqref{eq:dualnorm} is satisfied. It is clear that the structure of the underlying space $\cY$ is crucial for the derivation. 

\subsection{Riesz representation} \label{subsec:RieszReps}
Since $\cY$ is a Hilbert space, the Riesz Representation Theorem\,(see e.g. \cite{aubin2011applied}) states that $\eta(f)$ fulfilling \eqref{eq:dualnormEqual} is given by
\begin{equation*}
	\eta(f) := \Vert R^{-1}_{\cY} f \Vert_{\cY},
\end{equation*}
where $R_\cY: \cY \rightarrow \cY'$ denotes the Riesz-operator defined through the equation $(u, v)_\cY = \langle R_{\cY} u, v \rangle_{\cY}, \; u, v \in \cY$, where $(\cdot,\cdot)_\cY$ denotes the inner product in $\cY$. The Riesz representative $\hat{f} := R^{-1}_{\cY} f \in \cY$ is found by solving the equation
\begin{equation} \label{eq:RieszRep}
	(\hat{f}, v)_\cY = f(v), \quad \forall v \in \cY.
\end{equation}
Since $R_\cY$ is isometric we have $\eta(f) = \Vert f \Vert_{\cY'}$ and thus \eqref{eq:dualnormEqual}. Usually, evaluating $\Vert \cdot \Vert_{\cY}$ is much easier than evaluating $\Vert \cdot \Vert_{\cY'}$. Unfortunately, calculating the Riesz representative with \eqref{eq:RieszRep} is still unfeasible because the space $\cY$ is in almost all cases infinite-dimensional. Therefore, $\cY$ is replaced by a sufficiently high but finite-dimensional subspace $\cY^\delta \subset \cY$, e.g., a finite element space. The Riesz representative $\hat{f} \in \cY$ is then approximated by $\hat{f}^\delta \in \cY^\delta$, which is the solution of
\begin{equation} \label{eq:RieszRepFinite}
	(\hat{f}^\delta, v^\delta)_\cY = f(v^\delta), \quad \forall v^\delta \in \cY^\delta.
\end{equation}

\begin{remark}
If $\cY$ is not a Hilbert, but only a Banach space, the Riesz Representation Theorem is not applicable. However, one way out is to use wavelet methods to directly bound the dual norm $\Vert f \Vert_{\cY'}$ as proposed in\,\cite{ernst2024certified}. 
\end{remark}

With an increasing dimension of $\cY^\delta$ we can at least hope to have convergence of $\hat{f}^\delta $ to $\hat{f}$ in $\cY$ and therefore convergence of $\Vert \hat{f}^\delta \Vert_{\cY}$ to $\Vert f \Vert_{\cY'}$. The drawback of this approach is that, in the case of finite elements, the space $\cY^\delta$ requires a discretization of the underlying domain $\Omega$.


\section{Theoretical Results}
\label{sec:appendixTheoretical}

For this section, we use an alternate notation, replacing $\beta$ with $\eta= \frac{\beta}{1-\beta_o}$ to make the equations easier to read, such that:
\begin{align*}
    (1-\beta)U + \beta F \Leftrightarrow U + \eta F
\end{align*}

This does not affect the allocation made by the ILP, as it only scales all Q-values by $1/(1-\beta)$. This would only be undefined when $\beta=1$, but we avoid that condition in our proofs. As $\beta\rightarrow1, \eta\rightarrow\infty$, and for any $\beta'>\beta$, $\eta'>\eta$. Note that in the theorem statements, we replace $\beta$ with $\eta$, but the proofs are equivalent.

The following results hold for any fairness function used in the DECAF formulation.

\begin{proposition}
As $\eta_{test}\xrightarrow{} 0$, all fair-only models behave in a utility-maximizing manner.
\end{proposition}
We state this without proof. It is easy to follow how this holds, as at $\eta=0$, the fairness model does not play any role in the decision making.

\setcounter{theorem}{0}
\begin{theorem}
Given perfect estimates for utility and fairness, increasing $\eta$ always improves the one-step fairness gain for SO with $\gamma=0$.
\label{th:theorem_fair_app}
\end{theorem}
\begin{proof}
We assume that the utility and fairness estimators are converged, i.e., the estimates of fairness and utility are correct. For the following discussion, assume the environment has evolved over some time $t$ and is at a state $s_t$. We consider what changes when we change $\eta$ at this state. Variables used henceforth are conditioned on $s_t$, wherever reasonable. We make the conditioning on $s_t$ implicit and do not notate it, to make it easier to read.

With $\gamma=0$, the optimal utility and fairness estimates equal the one-step return, i.e. the change in utility and fairness because of the resulting joint action. Note that these values are not known to the agents prior to the allocation as they depend on the joint actions of all agents, so computing these estimates is not trivial.

Let $U_{tot}(\mathcal{A})$ and $F_{tot}(\mathcal{A})$ be defined as follows, given an allocation $\mathcal{A}$:
\begin{align}
    U_{tot}(\mathcal{A}) &= \sum_{i\in\alpha}U(\mathcal{A}_i)\\
    F_{tot}(\mathcal{A}) &= \sum_{i\in\alpha}F(\mathcal{A}_i)
\end{align}
We remind the reader that $\mathcal{A}_i$ refers to the action assigned to agent $i$ in the allocation $\mathcal{A}$.

Let $\textbf{Z}_t$ represent the agent metrics at time $t$. Further, let $\mathcal{A}^*$ represent the optimal allocation from the ILP with $\eta$ as the trade-off weight. Since $\mathcal{A}^*$ is optimal, it follows that for all other possible allocations $\mathcal{A}_o$: 
\begin{align}
    U_{tot}(\mathcal{A}^*) + \eta F_{tot}(\mathcal{A}^*) &\ge U_{tot}(\mathcal{A}_o) + \eta F_{tot}(\mathcal{A}_o)  \\
    U_{tot}(\mathcal{A}^*) - U_{tot}(\mathcal{A}_o)&\ge  \eta (F_{tot}(\mathcal{A}_o) - F_{tot}(\mathcal{A}^*)) \label{eq:base_comp} 
\end{align}

We are interested in finding what happens to the allocation when we increase $\eta$.
For $\eta'>\eta$, note that the left side of Eq.~\ref{eq:base_comp} remains the same. Since utility estimates are not affected by changing $\eta$, any other allocation $\mathcal{A}_o$ can only be selected over $\mathcal{A}^*$ if the following condition holds:
\begin{align}
    U_{tot}(\mathcal{A}^*) + \eta' F_{tot}(\mathcal{A}^*) &\le U_{tot}(\mathcal{A}_o) + \eta' F_{tot}(\mathcal{A}_o)  \\
    U_{tot}(\mathcal{A}^*) - U_{tot}(\mathcal{A}_o)&\le  \eta' (F_{tot}(\mathcal{A}_o) - F_{tot}(\mathcal{A}^*)) \label{eq:base_comp2} 
\end{align}

Combining Eqs.\ref{eq:base_comp} and \ref{eq:base_comp2}, we get the following:
\begin{align}
    \eta (F_{tot}(\mathcal{A}_o) - F_{tot}(\mathcal{A}^*)) &\le  \eta' (F_{tot}(\mathcal{A}_o) - F_{tot}(\mathcal{A}^*)) \\
    F_{tot}(\mathcal{A}^*)(\eta'-\eta) &\le F_{tot}(\mathcal{A}_o)(\eta'-\eta) \label{eq:base_comp3} 
\end{align}
Since $\eta\ge0$ and $\eta'>\eta$, Eq.~\ref{eq:base_comp3} can only be true if $F_{tot}(\mathcal{A}_o)>F_{tot}(\mathcal{A}_o)$.

Thus, any allocation $\mathcal{A}_o$ that is optimal (and thus selected by the ILP) for $\eta'>\eta$ is guaranteed to have equal or better fairness than the allocation $\mathcal{A}^*$ at $\eta$.
\end{proof}

We also state the corollary to Theorem~\ref{th:theorem_fair_app}.
\begin{corollary}
Given perfect estimates for utility and fairness, decreasing $\eta$ always improves the one-step utility gain for SO with $\gamma=0$.
\label{th:theorem_util_app}
\end{corollary}
The proof follows a similar structure to Theorem~\ref{th:theorem_fair_app}.

While we only prove the behavior for $\gamma=0$, our empirical results show that we can expect similar behavior for long-horizon estimates. For any state, we will select actions that improve fairness in the long run starting from that state as $\eta$ is increased.

We also show the following useful property:
\begin{theorem}
    For a large enough $\eta$, the fairest allocation will be selected with perfect utility and fairness estimators for SO with $\gamma=0$.
\end{theorem}
\begin{proof}
Let $\mathcal{A}_f$ denote the allocation with the largest fairness gain:
\begin{align*}
    \mathcal{A}_f = \argmax_\mathcal{A} F_{tot}(\mathcal{A}) 
\end{align*}

For simplicity, let us assume no two allocations have the same $F_{tot}(\mathcal{A})$. For any other allocation $\mathcal{A}_o$, we have:
\begin{align}
    F_{tot}(\mathcal{A}_f) > F_{tot}(\mathcal{A}_o)  
\end{align}

Then, $\mathcal{A}_f$ will be optimal and selected by the ILP if the following condition holds:
\begin{align}
  U_{tot}(\mathcal{A}_f) + \eta_f F_{tot}(\mathcal{A}_f) &\ge U_{tot}(\mathcal{A}_o) + \eta_f F_{tot}(\mathcal{A}_o)  \\
  \eta_f &\ge \frac{U_{tot}(\mathcal{A}_o) - U_{tot}(\mathcal{A}_f)}{F_{tot}(\mathcal{A}_f)- F_{tot}(\mathcal{A}_o)} 
\end{align}

We can compute an upper bound for $\eta_f$ by considering the range of values that $U_{tot}$ and $F_{tot}$ can take. Let $U_{max}=\max_\mathcal{A}U_{tot}(\mathcal{A})$, and $F_{max}=\max_{\mathcal{A}, \mathcal{A}\ne \mathcal{A}_f }F_{tot}(\mathcal{A})$.

Then, we have the following:
\begin{align}
    \eta_f &\ge \frac{U_{tot}(\mathcal{A}_o) - U_{tot}(\mathcal{A}_f)}{F_{tot}(\mathcal{A}_f)- F_{tot}(\mathcal{A}_o)}\\
    &\le \frac{U_{max} - U_{tot}(\mathcal{A}_f)}{F_{tot}(\mathcal{A}_f)- F_{tot}(\mathcal{A}_o)}\\
    &\le \frac{U_{max} - U_{tot}(\mathcal{A}_f)}{F_{tot}(\mathcal{A}_f)- F_{max}} = \eta_f^u \label{eq:beta_upper}
\end{align}
Eq.~\ref{eq:beta_upper} gives us an upper bound for $\eta_f$. Thus, for all $\eta>\eta_f^u$, $\mathcal{A}_f$ will be the optimal allocation.
\end{proof}
\begin{corollary}
    For a small enough $\eta$, the most utilitarian allocation will be selected with perfect utility and fairness estimators for SO with $\gamma=0$.
\end{corollary}
The proof follows a similar structure to the proof for the previous theorem.
% \memo{
% This proof would be so much easier if we learn beta as well. The estimate for u and f would be conditioned on beta, so the algorithm is guaranteed to predict the correct long-term utility for a given beta
% }
% Possible other theorems:
% 1. For large enough beta, even after discounting, the action selected is the fairest immediate action? No, that is false.

% !TEX root = ../dual_norm_estimate_base.tex

\section{Numerical experiments} \label{sec:NumResults}
We investigate the quantitative performance of the proposed error estimators for linear elliptic and parabolic PDEs. In order to do so, we compare the quantities listed in the following table.
%-----------------------------------------------
\newcommand\myline[2]{
	\begin{tikzpicture}
	\draw[#1,line width=1pt] (0,0) -- (1,0);
	\node[scale=1.3, line width=0.8pt] at (0.5,0) {\color{#1}\pgfuseplotmark{#2}};
	\end{tikzpicture}
	}
%-----------------------------------------------
\begin{center}
	\begin{tabular}{c|l|l}
		\multicolumn{2}{c|}{legend} & computation done by \\ \hline\hline
		\myline{black}{o} 		& $\mathcal{W}$-error on $\Omega$ & high-fidelity FE computation \\ \hline
		\myline{blue}{x} 		& lower bound on $\Omega$		& high-fidelity Riesz representation \\ \hline
		\myline{blue}{triangle*} 	& upper bound on $\Omega$ 		& high-fidelity Riesz representation \\ \hline
		\myline{orange}{x} 		& lower bound on $\mycirc$		& efficient error estimator \\ \hline
		\myline{orange}{triangle*} & upper bound on $\square$		& efficient error estimator
	\end{tabular} 
\end{center}
%-----------------------------------------------

The \enquote{true} $\mathcal{W}$-error (black) is computed by comparing the PINN-solution with a high-fidelity finite element reference solution. The upper and lower bounds in $\Omega$ (blue lines) are determined by using the Riesz representation w.r.t.\ a fine discretization of the PDE on the domain $\Omega$ and computing the dual norm of the residual by determining the primal norm of the Riesz representation. For all presented examples, we derive analytical estimates for the involved constants. Of course, this could also be replaced by solving corresponding generalized eigenvalue problems. These \enquote{Riesz estimates} on $\Omega$ are used as a reference only, as those bounds require high computational cost (in that case, one could replace a PINN  by the detailed FE simulation). It is clear that our error estimates cannot be better than the Riesz bounds on $\Omega$.

In order to compute our error estimator (orange), we need to solve the PDE on $\mycirc$ and $\square$. Hence, we choose these domains in such a way that their geometry is rather simple, e.g.\ a circle or a rectangle allowing for highly efficient tensorproduct discretizations in spherical and canonical coordinates, respectively. For those, we used highly efficient and accurate spectral methods, \cite{Canuto2006}. All experiments have been been carried out in Python with FEniCSx\,\cite{Baratta2023} and PyTorch\,\cite{NEURIPS2019_9015}. Our code can be found on a \texttt{git} repository, \cite{codeGIT}.

As mentioned already above, possible scenarios for PINNs include parametric PDEs (PPDEs) and/or PDEs on (spatial) domains with complicated geometries. This also guides our numerical experiments. Denoting by $\mathcal{P}\subset\mathbb{R}^P$, $P\in\mathbb{N}$, a compact parameter set, we denote by $u^\delta_\mu$ the high-fidelity (but expensive) numerical solution of the PPDE (e.g.\ by finite elements). Then, we trained a PINN $\Phi^\theta(\cdot;\mu)$ to approximate the solution maps $\mu \mapsto u_\mu$, where $u_\mu$ is the  exact (classical, i.e., pointwise) solution\footnote{Existence and uniqueness of such a solution is \emph{not} clear!} of the PPDE on the domain $\Omega$. For our experiments, we trained the PINN using the mean-square loss function
\begin{align*}
	\mathcal{L}(\theta) := \sum\limits_{\mu \in \mathcal{S}_\mathcal{P}} \sum\limits_{x \in \mathcal{S}_\Omega} | u^\delta_\mu(x) - \Phi^\theta(x;\mu) |^2,
\end{align*}
where we used finite training data sets $\mathcal{S}_\mathcal{P} \subset \mathcal{P}$ for the parameter and $\mathcal{S}_\Omega \subset \Omega$ for the physical variable (which could also be space and time for parabolic problems). However, we stress once more the fact that our subsequent error estimation is independent on the specific loss function and training process. 


\subsection{Linear parameter-dependent diffusion problem}
%-------------------------------------------------------------
We start by a problem in space only, i.e., an elliptic problem involving parameters posed on a \enquote{complicated} domain $\Omega \subset \mathbb{R}^d$ being the nonconvex saw-blade-like domain shown in Figure\,\ref{fig:sawbladeDomain}. The subdomain $\Omega_1$ consists of the saw teeth and $\Omega_2 := \Omega \setminus \bar{\Omega}_1$ is a rectangle. In such a setting, the existence of a classical solution cannot be expected and a finite element solution will require to resolve the geometry of the domain.
\begin{figure}[!htb]
	\begin{minipage}{1.0\textwidth}
		\centering
		\input{figs/sawbladedom.tikz}
	\end{minipage}
	\caption{Saw-blade domain $\Omega=\Omega_1\cup\Omega_2$ with the partition for the diffusion coefficients. The saw teeth are made of different material than the saw blade.
		\label{fig:sawbladeDomain}}
\end{figure}
    
The variational form of the PPDE uses the trial and test space $\mathcal{W}=\mathcal{Y}=H^1_{0}(\Omega)$ yielding a Galerkin discretization and amounts finding  $u_\mu \in H^1_{0}(\Omega)$ such that 
\begin{equation*}
    \dual{r_\Omega(u_\mu),v}_{H^1_0(\Omega)} := (A_\mu \nabla u_\mu, \nabla v )_{L_2(\Omega)} - (1, v)_{L_2(\Omega)} = 0 \quad \forall v \in H^1_{0}(\Omega),
	\end{equation*}
where the parameter-dependent diffusion matrix is given by
\begin{equation*}
	A(x; \mu) := \left[ \mu_1 \chi_{\Omega_1}(x) + \mu_2 \chi_{\Omega_2}(x)\right]
	\begin{pmatrix}
		1 & 0 \\
		0 & 2 
	\end{pmatrix},
\end{equation*}	
The diffusion is parameterized ranging in $\mathcal{P} := [1/10,1] \times [5/100, 1/10]\subset\mathbb{R}^2$, $P=2$. 
	

The training set $\mathcal{S}_\mathcal{P} \subset \mathcal{P}$ consists of $7 \times 7$ equidistant distributed parameters and $\mathcal{S}_\Omega \subset \Omega$ is a randomly chosen set of $2^{14}$ points. In order to apply Corollary \ref{cor:boundsSobolev}, we define $\square := (0,4)\times(0,1)$ and $\mycirc := (0,4) \times (0,1/2) = \Omega_2$ as well as 
\begin{align*}
	\dual{r_\square(\Phi^\theta(\mu)), v }_{H^1_0(\square)} 
    &:= \dual{r_\Omega(\Phi^\theta(\mu)), v}_{H^1_0(\square)}, 
    &&\forall v \in H^1_{0}(\square), \\
	\dual{ r_\mycirc(\Phi^\theta(\mu)), v }_{H^1_0(\mycirc)} 
    &:= \dual{r_\Omega(\Phi^\theta(\mu)),\Ext_\Omega v}_{H^1_0(\Omega)},  
    &&\forall v \in H^1_{0}(\mycirc).
\end{align*}
It is readily seen that $r_\Omega(\Phi^\theta(\mu))$ is well-defined on $H^1_{0}(\square)$ and is an element of the dual space, which means in this example $\mathcal{Z}'(\square):=(H_0^1(\square))'$. Furthermore, $r_\square(\Phi^\theta(\mu))$ is an extension of $\aRBox r_\Omega(\Phi^\theta(\mu))$, because $r_\square(\Phi^\theta(\mu)) \equiv \aRBox r_\Omega(\Phi^\theta(\mu))$ on $\mathcal{U}(\square)$, defined in \eqref{eq:UdefSobolev}. With the space $\mathcal{V}(\Omega)$ defined in \eqref{eq:VdefSobolev} we define $r_\mycirc(\Phi^\theta(\mu)) := \aECirc \left( r_\Omega(\Phi^\theta(\mu)|_{\mathcal{V}(\Omega)} \right)$. Then, Corollary\,\ref{cor:boundsSobolev} applies and with the constants from Example\,\ref{example:linearPDE} we get 
\begin{equation*}
	\frac{\Vert r_\mycirc(\Phi^\theta(\mu)) \Vert_{H^{-1}(\mycirc)}}{2 \max \lbrace \mu_1, \mu_2 \rbrace} 
    \le \Vert \nabla u(\mu) - \nabla \Phi^{\theta}(\mu) \Vert_{L_2(\Omega)} \le \frac{\Vert r_\square(\Phi^\theta(\mu)) \Vert_{H^{-1}(\square)}}{\min \lbrace \mu_1, \mu_2 \rbrace}.
\end{equation*}
We use these constants in our experiments.

%=========================
\begin{figure}[!htb]
	\centering
	%--------------------------
	\begin{tikzpicture}[scale = 0.9]
		\begin{semilogyaxis}[legend cell align={left}, grid=both,width=\textwidth,height=6cm, legend columns = 2, legend style={at={(0.5,1.3)},anchor=north},cycle list name=black white, ymin=1e-2, ymax=2e1,xmin=0,xmax=48]
		\addplot[mark=o,black,line width=1pt] table [x=parameter_num, y=abs_H10_error] {figs/data_combined_sawblade.dat};
		\addlegendentry{$\mathcal{W}=H^1_0$-error on $\Omega$\hspace*{10pt}}
		%-------------
		\addplot[mark=triangle,blue,line width=1pt] table [x=parameter_num, y=abs_est_error_ub] {figs/data_combined_sawblade.dat};
		\addlegendentry{upper bound on $\Omega$}
		%-------------
		\addplot[mark=x,blue,line width=1pt] table [x=parameter_num, y=abs_est_error_lb] {figs/data_combined_sawblade.dat};
		\addlegendentry{lower bound on $\Omega$}
		%-------------
		\addplot[mark=triangle,orange,line width=1pt] table [x=parameter_num, y=abs_est_error_box_ub] {figs/data_combined_sawblade.dat};
		\addlegendentry{upper bound on $\square$}
		%-------------
		\addplot[mark=x,orange,line width=1pt] table [x=parameter_num, y=abs_est_error_box_lb] {figs/data_combined_sawblade.dat};
		\addlegendentry{lower bound on $\mycirc$}
		%-------------
		\end{semilogyaxis}
	\end{tikzpicture} 
	%-------------------------- 
	%--------------------------
	\caption{\label{fig:param_dependencev1} Absolute $H^1_0$-error as well as the estimated $H^1_0$-error on $\Omega$, $\square$ and $\mycirc$. The horizontal axis corresponds to the number $N(i,j) \in \mathbb{N}$ of data points $(\mu_i,\mu_j) \in \mathcal{S}_{\mathcal{P}}$.}
\end{figure}
%=========================
We show the results of this experiment in Figure \ref{fig:param_dependencev1}. Although the parameter space is two-dimensional, we enumerated the parameters $(\mu_i,\mu_j) \in \mathcal{S}_{\mathcal{P}}$ and plot the values against the number $N(i,j) \in \mathbb{N}$ of the parameters. The comparison of the exact error (in black) with the upper and lower Riesz bounds on $\Omega$ (blue) show that these bounds differ by a multiplicative factor up to 10. Recall that this is the best we can achieve with our error estimators (in orange). The lower bound is remarkably sharp. Recall that $\mycirc=\Omega_2$, which excludes all saw-blades. Due to the fact that the data functions are arbitrarily smooth on $\mycirc$, the computation of the lower bound with the spectral method  converges exponentially fast and is thus very efficient.

The upper bound follows the line of the Riesz-bound on $\Omega$ and is too pessimistic by another multiplicative factor of 10. A better fine-tuning of the involved constants might improve this upper bound.

\subsection{A parametric domain}
%-------------------------------------------------------------
Our next numerical experiment is particularly suited for domain embedding, namely a linear elliptic PDE, posed on a domain $\Omega_\mu$ with parameterized boundary shown in Figure \ref{fig:shapeOptdom}. It is the unit square with a cutout, which cannot be smoothly transformed into a reference domain, due to the sharp corners for angles $\mu > 0$. Such a situation occurs e.g.\ in geometry optimization, where PINNs have already been used, \cite{SUN2023116042}. 
\begin{figure}[!htb]
	\begin{minipage}{1.0\textwidth}
		\centering
		\input{figs/shapeoptdom.tikz}
	\end{minipage}
	\caption{A parameterized square, where the parameter $\mu$ is the angle of a recess.\label{fig:shapeOptdom}}
\end{figure}
	
When solving a PDE on $\Omega_\mu$ with the finite element method, re-meshing  might be necessary for different $\mu$. On the other hand, the training of a PINN is straightforward by excluding all points outside of the domain. This shows why a PINN might be an attractive tool for a PDE on $\Omega_\mu$.

Since the underlying domain is parameterized, we can consider a non-parametric elliptic equation of the form
\begin{equation*}
	\dual{r_{\Omega_\mu}(u_\mu),v}_{H^1_0(\Omega_\mu)} := (A\, \nabla u_\mu, \nabla v )_{L_2(\Omega_\mu)} + (b \cdot \nabla u_\mu + c\, u_\mu, v )_{L_2(\Omega_\mu)} - f_\mu(v) = 0, 
\end{equation*}
for all $v \in H^1_{0}(\Omega_\mu)$ with the diffusion, convection and reaction coefficients given by
\begin{equation*}
	A \equiv \begin{pmatrix}
		1/2 & 1/4 \\
		1/4 & 1/2 
	\end{pmatrix} , \quad 
	b \equiv \begin{pmatrix}
		10 \\
		-3 
	\end{pmatrix}  
	\quad \text{ and } \quad
	c(x,y) := xy+1.
\end{equation*}	
The source function is defined by $f_\mu(v) := (10, v)_{L_2(\Omega_\mu}$. We choose the training set $\mathcal{S}_{\Omega_\mu}$ as $2^{16}$ random points in $\Omega_\mu$. The parameter training set $\mathcal{S}_\mathcal{P}$ consists of five equidistant points in $\mathcal{P}=[0,\pi/2]$, $P=1$. The extended domain is chosen as $\square:= (0,1)^2$ and the imbedded domain as $\mycirc := (0,1) \times (0.25,1)$.

The extension and restriction of the residual $r_{\Omega_\mu}(u_\mu)$ can be done as in the previous example, Corollary\,\ref{cor:boundsSobolev} applies and yields the error estimate
\begin{equation*}
	\frac{\Vert r_\mycirc(\Phi^\theta(\mu)) \Vert_{H^{-1}(\mycirc)}}{\Vert A \Vert_{L_\infty} + \Vert b \Vert_{L_\infty} + \Vert c \Vert_{L_\infty}} 
    \le \Vert \nabla u_\mu - \nabla \Phi^{\theta}(\mu) \Vert_{L_2(\Omega)} 
    \le \frac{\Vert r_\square(\Phi^\theta(\mu)) \Vert_{H^{-1}(\square)}}{\lambda_{\min}(A)},
\end{equation*}
from which we deduce the constants.


%=========================
\begin{figure}[!htb]
	\centering
	%--------------------------
	\begin{tikzpicture}[scale = 0.9]
		\begin{semilogyaxis}[legend cell align={left}, grid=both, width=\textwidth,height=6cm, legend columns = 2, legend style={at={(0.5,1.3)},anchor=north}, cycle list name=black white, ymin=3e-4, ymax=1e1,xmin=0, xmax=pi/2+0.001, black, xtick distance=0.2, xticklabels={, 0, 0.2, 0.4, 0.6, 0.8, 1.0, 1.2, 1.4,,},
			extra x ticks={pi/2},
			extra x tick style={%
					grid=major,
			},
			extra x tick labels={
				$\pi/2$,
			}]
		\addplot[mark=o,black,line width=1pt] table [x=mu, y=abs_H10_error] {figs/data_combined_shapeOpt.dat};
		\addlegendentry{$\mathcal{W}=H^1_0$-error on $\Omega_\mu$\hspace*{10pt}}
		%-------------
		\addplot[mark=triangle,blue,line width=1pt] table [x=mu, y=abs_est_error_ub] {figs/data_combined_shapeOpt.dat};
		\addlegendentry{upper bound on $\Omega_\mu$}
		%-------------
		\addplot[mark=x,blue,line width=1pt] table [x=mu, y=abs_est_error_lb] {figs/data_combined_shapeOpt.dat};
		\addlegendentry{lower bound on $\Omega_\mu$}
		%-------------
        \addplot[mark=triangle,orange,line width=1pt] table [x=mu, y=abs_est_error_box_ub] {figs/data_combined_shapeOpt.dat};
		\addlegendentry{upper bound on $\square$}
		%-------------
		\addplot[mark=x,orange,line width=1pt] table [x=mu, y=abs_est_error_box_lb] {figs/data_combined_shapeOpt.dat};
		\addlegendentry{lower bound on $\mycirc$}
		%-------------
		\end{semilogyaxis}
		\end{tikzpicture} 
		%-------------------------- 
		%--------------------------
		\caption{\label{fig:param_dependencev2} Absolute $H^1_0$-error as well as the estimated $H^1_0$-error on $\Omega$, $\square$ and $\mycirc$. The horizontal axis corresponds to the angle $\mu$.}
\end{figure}
%------------------
The results for a set of nine parameters, which serve as the test set for the PINN, are depicted in Figure \ref{fig:param_dependencev2}. The Riesz bounds on $\Omega_\mu$ (blue) follow the slope of the exact error (black), the upper bound being quite sharp, the lower one too optimistic by a factor of about 10. The lower bound for the error estimator (orange), calculated on $\mycirc$, follows the slope and is quite sharp. The reason might be that the cut-off region does neither concern the Riesz lower bound nor the one on $\mycirc$. On the other hand, however, the upper bound does not follow the slope of the error and is too pessimistic by a factor of 10 -- which overall seems acceptable, but is by far worse than the Riesz upper bound on $\Omega_\mu$. The change of the geometry obviously has only a small effect on the proposed upper error bound. 

\subsection{A parabolic problem on a non-convex polytope}
%-------------------------------------------------------------
Finally, we report results for a time-dependent problem on a domain with \enquote{complicated} geometry. To this end, consider the parameterized parabolic problem $\dot{u}_\mu+A_\mu u_\mu = f$, $u_\mu(0)=0$ on $Q:=I\times\Omega$, $I:=(0,1)$ being the time horizon and $\Omega\subset\mathbb{R}^2$ is the map of the state of Arkansas\,(USA) depicted in Figure \ref{fig:domain_arkansas}. The domain is a non-convex polytope and therefore has a Lipschitz-boundary.  Moreover, $\Omega$ has sharp corners on the right- and lower left-hand side. 
\begin{figure}[!htb]
	\begin{minipage}{1\textwidth}
		\centering
		\includegraphics[width=0.5\textwidth]{figs/domain_arkansas_2.png}
	\end{minipage}
	\caption{The space-time domain $Q = I\times \Omega$, where the green axis refers to the time.\label{fig:domain_arkansas}}
\end{figure}
The parametric elliptic operators $A_\mu \in \mathcal{L}(H^1_0(\Omega), H^{-1}(\Omega))$ are defined by the variational form 
\begin{equation*}
	\langle A_\mu \varphi, \psi \rangle_{H^1_0(\Omega)} 
    := (K \nabla \varphi, \nabla \psi )_{L_2(\Omega)} + (b_\mu \nabla \varphi + c \varphi, \psi )_{L_2(\Omega)}, \quad \forall \, \varphi, \psi \in H^1_0(\Omega),
\end{equation*}
where the time-independent coefficient functions are chosen as
\begin{equation*}
	K \equiv \begin{pmatrix}
		1 & 0 \\
		0 & 0.1 
	\end{pmatrix} , \quad 
	b_\mu(x,y) := (31-\mu)\begin{pmatrix}
		 \sin^2(2 y)\\
		\cos((x+1)^{\mu/4}) 
	\end{pmatrix} \text{ and } \quad
	c(x,y) := xy+1,
\end{equation*}	
with the parameter set $\mathcal{P}:= [1,10] \subset \mathbb{R}$, $P=1$. The parameter dependent convection is chosen to be \emph{not} affinely decomposable, so that this standard assumption of the reduced basis method is not valid, see e.g.\,\cite{RozzaRB}, and using a PINN seems attractive.  We use the space-time variational formulation of the parabolic PDE as introduced in Example \ref{Ex:SpaceTime} above.


The height of Arkansas is normalized to $1$ and the front upper left corner is located at $(0,1,0)^T \in \mathbb{R}^3$, so that we define $\square := (0,1.2) \times (0,1)$ and $\mycirc := (0.1345,0.783) \times  (0,1) \subset \Omega$. With this setting, the extension operator from Proposition\,\ref{prop:LisBochner} can be used and Corollary\,\ref{cor:boundsBochner} applies. 

The PINN has again been trained with a high-fidelity finite-element solution using training sets $\mathcal{S}_Q$, consisting of $1.5 \cdot 10^5$ points for the space-time domain $Q$ and $\mathcal{S}_\mathcal{P}$, consisting of $16$ equidistant points for the parameter. The error has been measured on $31$ parameters. 

%=========================
\begin{figure}[!htb]
	\centering
	%--------------------------
	\begin{tikzpicture}[scale = 0.9]
		\begin{semilogyaxis}[legend cell align={left}, grid=both, width=\textwidth,height=6cm, legend columns = 2, legend style={at={(0.5,1.3)},anchor=north}, cycle list name=black white, ymin=1e-2, ymax=3e2,xmin=1,xmax=10]
		\addplot[mark=o,black,line width=1pt] table [x=param, y=abs_X_error] {figs/data_combined_arkansas.dat};
		\addlegendentry{$\mathcal{W}$-error on $I\times\Omega$}
		%-------------
		\addplot[mark=triangle,blue,line width=1pt] table [x=param, y=abs_est_error_ub] {figs/data_combined_arkansas.dat};
		\addlegendentry{upper bound on $I\times\Omega$}
		%-------------
		\addplot[mark=x,blue,line width=1pt] table [x=param, y=abs_est_error_lb] {figs/data_combined_arkansas.dat};
		\addlegendentry{lower bound on $I\times\Omega$\hspace*{10pt}}
		%-------------
		\addplot[mark=triangle,orange,line width=1pt] table [x=param, y=abs_est_error_box_ub] {figs/data_combined_arkansas.dat};
		\addlegendentry{upper bound on $I\times\square$}
		%-------------
		\addplot[mark=x,orange,line width=1pt] table [x=param, y=abs_est_error_box_lb] {figs/data_combined_arkansas.dat};
		\addlegendentry{lower bound on $I\times\mycirc$}
		%-------------
		\end{semilogyaxis}
	\end{tikzpicture} 
	%-------------------------- 
	%--------------------------
	\caption{\label{fig:param_dependencev3} Absolute $\mathcal{W}$-error as well as the estimated $\mathcal{W}$-error on $I\times \Omega$, $I\times\square$ and $I\times\mycirc$. The horizontal axis corresponds to the parameter value $\mu \in \mathcal{P}$.}
\end{figure}
%==================================
The results are depicted in Figure \ref{fig:param_dependencev3}. The error of the PINN approximation does not depend much on the parameter $\mu$. This is also reflected by the bounds, so that they are basically multiples of the true error. The Riesz bounds on $Q=I\times\Omega$ are about a factor 10 off the true error (black). This is due to the constants in the error-residual relation, see \cite[Thm.\ 5.1]{schwab2009space} and the appendix therein. The error bounds (orange) are quite sharp, in particular the lower one. Again, the data is smooth on $I\times\mycirc$, so that we can use an efficient spectral method as we do not need to resolve the complicated geometry of $\Omega$.

\subsection{Computational times}
In order to investigate the computational overhead required for the lower and upper bounds, we collect in Table \ref{Tab:CPU} the CPU/GPU times for (i) the PINN training, (ii) the evaluation of the PINN at the points $\lbrace \mu \rbrace \times \mathcal{S}_\Omega$ for one $\mu \in \mathcal{S}_\mathcal{P}$ and (iii) solving the Riesz representation problems on $\mycirc$ and $\square$ also for one $\mu\in \mathcal{S}_\mathcal{P}$. The times with respect to all $\mu \in \mathcal{S}_\mathcal{P}$ scale linearly. The time measurement has been carried out using standard devices for each task, e.g., a NVIDIA Tesla V100 GPU has been used for the training and the evaluation of the PINN. The termination criterion of the training process has been a maximum number of iterations, which was $5 \cdot 10^3$ in the case of the saw-blade domain and $10^4$ iterations for the parametric and Arkansas domain. The evaluation of the error estimator on $\square$ has been parallelized using two Intel Xeon Gold 6230 CPUs with 20 cores each. For the discretization, we have used standard $P1$-finite elements with around $2.5 \cdot 10^5$ degrees of freedom. The Riesz problems on $\mycirc$ have been solved with the spectral method using a nodal Lagrange basis of order 12 in the 2D cases and of order 8 in the time-dependent case. This leads to small dense linear systems, which can be solved using serial direct solvers and the error can be expected to be near machine accuracy, due to the exponential convergence of spectral methods. In case of the Arkansas domain the matrix size was $729 \times 729$ and $169 \times 169$ in the stationary cases. Thus, the comparable long solving time of $2.249$ seconds may be due to the non-optimal internal routines of FEniCSx. 

We can see that the time for the error estimation is negligible in comparison to the training time. Moreover, the lower bound on $\mycirc$ can be computed efficiently and this can even be improved if one would use specialized spectral solvers. The numbers confirm the efficiency of the method, even though we did not even use a highly optimized implementation, but the standard algorithms within the FEniCSx implementation.

\begin{table}[]
    \centering
    \begin{tabular}{l|r|r|r|r}
        \multicolumn{3}{c}{} & 
        \multicolumn{2}{|r}{{\textbf{Error estimation}}}\\
         \textbf{Problem} 
         & \textbf{Training} 
         & \textbf{Evaluation}
         & $\mycirc$ & $\square$ \\ \hline
         saw-blade & 1241.52 & 0.0019 & 0.011 & 0.16\\ \hline
         parametric domain & 926.76 & 0.0128 & 0.011 & 0.15\\ \hline
         Arkansas & 7536.71 & 0.0720 & 2.249 & 1.77\\ \hline
    \end{tabular}
    \caption{Times (in seconds) for training (GPU), PINN evaluation (GPU) and error estimation (CPU) on $\mycirc$ and $\square$.}
    \label{Tab:CPU}
\end{table}



\begin{lstlisting}[breaklines=true, caption={Qualitative Example 5: An example trace in the summarization domain, with \reactagent{} when using \texttt{gpt-4-turbo} as LLM.},label={ls:qualsummary}]
# Alice Anastasiou says: I'm interested in compiling a summary on the topic: governor, race, gop, obamacare, medicaid. Can you help me? 
>>>  Reflection.thought("I will first resolve the primary user to get the right contact information")
>>>  Enterprise.resolve_primary_user()
{"email": "alice@company.com", "full_name": "Alice Anastasiou", "user_id": "alice"}
>>>  Reflection.thought("I will search user's documents and find relevant people who may have more information")
>>>  EnterpriseSearch.search_documents(query="governor, race, gop, obamacare, medicaid")
>>>  EnterpriseSearch.search_relevant_people(query="governor, race, gop, obamacare, medicaid")
{"parker": "Knows about the following topics:\ngovernor elections, GOP, health care, Obama, Romney", "bhushan": "Knows about the following topics:\ngovernor elections, GOP, health care, Obama, Romney\nSiskiyou County, California, secession, state of Jefferson, rural representation", "farshid": "Knows about the following topics:\nSiskiyou County, secession, Jefferson, California, rural fire prevention fee", "eden": "Knows about the following topics:\nFacebook, gay rights, John Snow kiss-in, Richard Metzger, Paul Shetler\nFacebook, same-sex kissing, photo removal, protest", "hannah": "Knows about the following topics:\n", "dewei": "Knows about the following topics:\n", "cassie": "Knows about the following topics:\n"}
>>>  Reflection.thought("Parker and Bhushan have information about the governor race and GOP, so I should reach out to them.")
>>>  Enterprise.send_message("alice", "Hi Alice, I can reach out to Parker and Bhushan to produce a summary. Is that OK?")
>>>  System.finish()

# Alice Anastasiou says: Yes, please do that. 
>>>  Reflection.thought("I will reach out to Parker to ask for information about the governor race and GOP")
>>>  Enterprise.resolve_person("Parker")
{"email": "parker@company.com", "full_name": "Parker McLean", "user_id": "parker"}
>>>  Enterprise.send_message("parker", "Hi Parker, do you have any information about the governor race, GOP, Obamacare, or Medicaid?", title="Information about governor race and GOP")
>>>  System.finish()

# Parker McLean says: Yes, here's a relevant document I found: "gop eyes gains as voters in 11 states pick governors     enlarge this image toggle caption jim cole/ap jim cole/ap     voters in 11 states will pick their governors tonight , and republicans appear on track to increase [...] "
>>>  Reflection.thought("I will reach out to Bhushan to ask for information about the governor race and GOP")
>>>  Enterprise.resolve_person("Bhushan")
{"email": "bhushan@company.com", "full_name": "Bhushan Magar", "user_id": "bhushan"}
>>>  Enterprise.send_message("bhushan", "Hi Bhushan, do you have any information about the governor race, GOP, Obamacare, or Medicaid?", title="Information about governor race and GOP")
>>>  System.finish()

# Bhushan Magar says: Yes, here's a relevant document I found: "the real clear politics average for the race has mccrory maintaining a 14.3 percentage point lead .     montana : popular democratic gov . brian schweitzer - he won his last election with 65 percent of the vote - has [...]" 
>>>  Reflection.thought("I have enough information to compile the summary. I will send a message to Alice, with the summary enclosed in [[[ brackets ]]].")
>>>  Enterprise.send_message("alice", "Hi Alice, here's the summary I put together: [[[Voters in 11 states are choosing their governors, with Republicans poised to increase their numbers, potentially controlling over two-thirds of the nation's top state offices. The GOP is expected to take North Carolina and retain seats in Utah, North Dakota, and Indiana. Democrats are likely to hold West Virginia and Missouri, with competitive races in Montana, New Hampshire, and Washington. The governor's role is crucial in healthcare, as they can opt out of Medicaid expansion under the Affordable Care Act. A Romney win would empower Republican governors, especially in states where Obama's policies, like high-speed rail funding, were previously rejected.]]]", title="Governor race and GOP summary")

# Alice Anastasiou says: Thank you, this is exactly what I needed. 
>>>  Reflection.thought("Alice has confirmed that no more information is needed. I will close the session")
>>>  Enterprise.send_session_completed()
>>>  System.finish()
\end{lstlisting}

\subsubsection*{Acknowledgement} The authors acknowledge support by the state of Baden-Würt\-tem\-berg through bwHPC, and the Hellenic Foundation for Research and Innovation (H.F.R.I.) under the \enquote{2nd Call for H.F.R.I.~Research Projects to support Post-Doctoral Researchers} (project number: $01247$). 

\bibliographystyle{ieeetr}
\bibliography{dual_norm_literature.bib}

\end{document}