\section{Related Work}
\subsection{High-Order GCNs for Brain Disease Classification.}

In the auxiliary diagnosis of brain diseases, feature extraction and classification methods based on functional connectivity networks (FCNs) have received widespread attention. Early research primarily focused on low-order functional connectivity (LOFC), constructing networks by calculating temporal correlations between brain regions~\cite{gupta2021obtaining,chen2016high}. However, considering only pairwise connections between nodes may overlook information about higher-level interactions among brain regions. Therefore, researchers introduced high-order functional connectivity networks (HOFC), which capture more complex brain network characteristics by calculating correlations between functional connectivity distribution sequences~\cite{zhang2016topographical,zhao2018diagnosis}. Moreover, recent work has shown that traditional first-order GCNs rely on direct neighbor information but struggle to model the long-range dependencies critical for understanding brain functional networks. For instance, one study proposed a multi-channel fusion GCN (MFGCN) that integrates both topographical and high-order functional connectivity for depressive disorder (MDD) classification~\cite{liu2024fusing}. 

%and provide reference to the study that proposed MFGCN. 

The design of higher-order neighbor matrices—exemplified by HWGCN—defines $k$-hop neighbor matrices based on shortest path distances to learn weighted matrices, effectively integrating first-order and higher-order neighborhood information~\cite{liu2019higher}. In fMRI studies of brain disorders, learned masks have been utilized to construct hyperedges and compute weights for each hyperedge in hypergraph construction, revealing maximum information minimum redundancy (MIMR) higher-order relationships among brain regions~\cite{qiu2023learning}.



\vspace{-2mm}
\subsection{Hierarchical and Path-Aware Learning in GNNs.}

Some studies have found that noise connectivity issues in brain networks and the difficulty of stacking deep layers in traditional GCN models lead to over-smoothing problems. Researchers have proposed hierarchical GCNs to capture richer feature representations~\cite{li2022te}. For example, HFBN-GCN designed a coarse-to-fine hierarchical brain feature extraction process with two branches: one focusing on local information transmission within the brain's hierarchical structure and another processing the entire brain network to preserve the original information flow~\cite{mei2022modular}.

Unlike hierarchical GCN approaches, GCKN computes node distances using kernel functions along vertex-connected paths~\cite{chen2020convolutional}. Additionally, PathNNs propose an end-to-end approach for extracting multiple types of paths and updating node representations by encoding paths originating from nodes into fixed-dimensional vectors through LSTM-based path aggregation~\cite{michel2023path}. The HEmoN model~\cite{huang2025identifying} has provided preliminary insights into human emotions through tree path analysis; however, it lacks interpretable representations of higher-order brain pathways. In this study, we propose a weighted higher-order brain tree in a hierarchical pathway framework that enhances pathway representation and effectively explains the predictions of age-related changes in FC networks associated with mental disorders across different age groups.