\section{Conclusions and Future Works}
\label{sec:Conclusions and Future Works}
In this work, we developed a framework for the runtime enforcement against STL formula. This framework inputs a signal and outputs a minimally modified signal that satisfy the formula. Specially, given an STL formula, we derive timed transducers for the atomic components, compose them according to the formula, and apply them to the input timed words, which are obtained by encoding the signal. We present detail procedure for signal encoding, translating STL temporal operators into timed transducers, and an enforcement algorithm. Our approach effectively enforces a signal against an STL property on CPS.

As in \cite{10.1145/3126500,10.1145/3092282.3092291,10.1109/TII.2019.2945520}, we plan to extend the work to accommodate bidirectionality and also extend the framework for more general STL formulas.


%\noindent \textit{Future Works.}  
%As in \cite{10.1145/3126500,10.1145/3092282.3092291,10.1109/TII.2019.2945520},  in a bidirectional framework involving an environment and a program, we require two enforcers—one for monitoring inputs to the controller from the environment and the other for monitoring outputs from the controller to the environment. These enforcers will (minimally) correct any erroneous inputs or outputs to ensure that a specified property is maintained. Therefore, we plan to extend the work to accommodate bidirectionality.


%Also, the translation from STL to timed transducer that we demonstrate is specifically designed for enforcement. However, a more general translation approach, such as from STL to hybrid automata, could also be explored for enforcement and broader applications. Therefore, a broader question we aim to address in the future is enforcement based on hybrid automata specifications, with the current STL to timed transducer translation serving as a foundational step.

