
\section{Additional Experiments}
\cstart
Alongside the impact of the sanitizer and crash type on the performance of the target selection methods, we also analyzed how the traceback lengths of the crashes might influence the results.

While measuring the traceback lengths' impact is seemingly straightforward, it comes with it own hurdles. In particular, some projects, such as libraw and libpcap, differ greatly in their crashes' average traceback length of 5.2 and 12.4, respectively. The same holds for crash types such as the use of an uninitialized value and a stack overflow whose traceback lengths differs by 230 functions on average. This has to be taken into account when evaluating the impact of the traceback length's. Otherwise the performance on individual projects or crash types would indirectly be measured in this assessment.

To account for this, we focus on the mean percent increase/decrease between the longest and shortest traceback for every combination of projects and crash types. We find that there is only a negligible change of the results for the under-estimating NDCG (less than 0.1\%). For the over-estimating NDCG, the impact is higher, but still only shows a limited impact measuring a 2\% performance increase for longer tracebacks. The different behaviour can be explained by the optimistic and pessimistic matching strategies applied for the over- and under-estimating NDCG. While the former rewards if \emph{any} function from the traceback was ranked highly, the latter one expects \emph{every} function from the traceback to be at high positions. Highly ranking any function is significantly simpler for longer tracebacks which could explain the impact of the traceback length.

\cend

% \onecolumn
\section{Retrieval Performance by Sanitizers}
\label{sec:appx-break-down-sanitizers}

\vspace{2em}

% \begin{figure*}
\begin{center}
    \includegraphics[width=1.00\linewidth]{images/mean_gains_sanitizer}  
\end{center}
% \end{figure*}

\section{Retrieval Performance by Crash Types}
\label{sec:appx-break-down-crashtypes}

\vspace{2em}

\begin{center}
    \includegraphics[width=1.00\linewidth]{images/mean_gains_crashtypes} 
\end{center}