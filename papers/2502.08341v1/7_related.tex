Our systematization of target selection methods for directed fuzzers is related to two types of prior works: surveys of fuzzing literature and analyses aimed at enhancing specific components of the fuzzing pipeline.

\boldpar{Literature surveys}
Several surveys on fuzzing research have been conducted throughout the years. Although they are methodologically similar to each other and our work, their respective focus varies. First, there are surveys that take the whole fuzzing pipeline into account. \citet{ManHanHanCha+21} and \citet{LiaPeiJiaShe+18}, for example, both introduce general multi-step models of the fuzzing process and survey existing literature with regard to each step in their model. More closely related to our work is the survey by \citet{WanZhoYueLin+24} who focus on directed fuzzers. To that end, they identify several characteristics of directed fuzzers for which they examine prior works. While most of these characteristics are concerned with how the fuzzer operates, one characteristic covers the method by which its targets are selected. However, as their focus is on the fuzzer itself rather than on its preceding target selection method, they merely identify which method was used but do not examine it any further.

Other fuzzing surveys take a more specific perspective and focus on how certain methods are applied in the fuzzing pipeline or challenges that arise when applying fuzzing to particular application areas. That is, for example, how machine learning techniques are used for fuzzing~\cite{SavRodDun+19, WanJiaLiuHua+20} or the application of fuzzing to find flaws in embedded devices~\cite{MueStiKarFra+18, EisMauShrHut+22}, respectively. 

Lastly, surveys such as those by \citet{SchBarSchBer+24}, \citet{KleRueCooWei+18} or \citet{KimChoImHeo+24} take a meta perspective and study fuzzing research itself. To that end, they examine the process conducted to evaluate fuzzers in various publications. Based on their findings they can derive information about the general validity of the research field as well as recommendations on how to conduct an evaluation ideally.

\boldpar{Enhancing fuzzer components}
In addition to surveying publications on directed fuzzing, we also focus on systematically investigating the step preceding directed fuzzers; namely, the methods employed to select their targets. This is related to prior works which have conducted experiments on individual steps of the fuzzing pipeline. \citet{BöhPhaRoy16}, for example, study various power- and search-strategies to improve the seed scheduling part of a fuzzer, \citet{WuJiaXiaHua+22} compare different setups for a mutation strategy, and \citet{HerGunMagSha+21} focus on the seed selection and compare several different methods for that purpose. In contrast, our work focuses on the target selection, which has not yet been studied in-depth, and is, thus, orthogonal to other improvements.