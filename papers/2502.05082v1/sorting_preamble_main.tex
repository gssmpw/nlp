%% Initial packages and configurations

% Geometry and hyperref
%BEGIN_FOLD
%\usepackage{geometry}
%\geometry{
%%	a4paper,
%	total={210mm,297mm},
%	left	= 25.4mm,
%	right	= 25.4mm,
%	top		= 25.4mm,
%	bottom	= 25.4mm,
%}

%\usepackage{titleref}
%\usepackage{nameref}
\usepackage[hidelinks]{hyperref}
\hypersetup{
	colorlinks,
	linkcolor={blue},
	citecolor={ForestGreen},
	urlcolor={blue}
}
%\usepackage{breakurl}

%\usepackage{indentfirst, environ}
%\usepackage[utf8]{inputenc}

\usepackage[UKenglish]{babel}
\usepackage{lmodern}
\usepackage[T1]{fontenc}
\usepackage{amsmath, amsthm, amssymb, mathrsfs, bm, mathtools}
\usepackage{wasysym}
\allowdisplaybreaks

\usepackage{titlesec}
\setcounter{secnumdepth}{3}
\setcounter{tocdepth}{4}

\usepackage{xifthen}
\usepackage{xspace}

\usepackage{xparse}

\usepackage{graphicx}
%\usepackage[labelfont = sf]{caption}
%\usepackage[font = normal, labelfont = sf]{subcaption}
%\usepackage{subfig}
%\graphicspath{{figures/}}
\usepackage[font = normal, labelfont = {bf, sf}, labelsep = period, singlelinecheck = false]{caption}
\numberwithin{figure}{section}
% ``\crefname{figure}{Figure}{Figures}`` is copied below
\numberwithin{table}{section}
%END_FOLD


%% Document-specific stuff
%BEGIN_FOLD

%END_FOLD

% Section title formats
%BEGIN_FOLD
\usepackage{titlesec}
\newcommand{\fullstopafter}[1]{#1.}
\titleformat{\section}
	{\sffamily \Large \bfseries \boldmath}{\thesection}{1em}{}
\titleformat{\subsection}
	{\sffamily \large \bfseries \boldmath}{\thesubsection}{1em}{}
\titleformat{\subsubsection}[runin]
	{\sffamily \normalsize \bfseries \boldmath}{\thesubsubsection}{1em}{\fullstopafter}
\titleformat{\paragraph}[runin]
	{\normalsize \itshape}{\theparagraph}{1em}{\fullstopafter}
\titlespacing*{\paragraph}
	{0pt}{\medskipamount}{*2.5}
\titleformat{\subparagraph}[runin]
	{\normalsize \itshape}{\thesubparagraph}{1em}{\fullstopafter}
%END_FOLD


%% Custom Lists
% More can be found in `preamble_lists`
%BEGIN_FOLD
\usepackage{enumitem}

%\renewcommand{\labelitemi}{\bcdot}

\newcommand{\romannumbering}{%
	\renewcommand{\labelenumi}{\upshape(\roman{enumi})}%
	\renewcommand{\theenumi}{\upshape(\roman{enumi})}}
\newcommand{\tref}[1]{\textnormal{\ref{#1}}}

%\usepackage{scrextend}
%\def\changemargin#1#2{\list{}{\rightmargin#1\leftmargin#2}\item[]}
%\let\endchangemargin=\endlist

\setlist[enumerate]{%
	topsep	= \smallskipamount,
	itemsep = 0pt,
%	label	= \ensuremath{(\alph*)}
%	label	= \textup{(\roman*)}
%	label	= \textup{(\arabic*)}
}
\setlist[itemize]{%
	topsep	= \smallskipamount,
	itemsep	= 0pt,
	label	= \bcdot
}
\setlist[description]{%
	topsep	= \medskipamount,		% space before start / after end of list
	itemsep	= \smallskipamount,		% space between items
	font	= {\mdseries\slshape},	% set the label font
}
%END_FOLD


\newcommand{\negphantom}[1]{\ifmmode\settowidth{\dimen0}{$#1$}\else\settowidth{\dimen0}{#1}\fi\hspace*{-\dimen0}}

\newenvironment{center-small}
	{\par\centering\medskip}
	{\par\medskip\noindent}


%% Colours
%BEGIN_FOLD
%\usepackage{color}
\usepackage[dvipsnames,hyperref]{xcolor}

\newcommand{\blue} [1]{{\color{blue}#1}}
\newcommand{\red} [1]{{\color{red} #1}}
\newcommand{\green} [1]{{\color{ForestGreen} #1}}
\newcommand{\purple} [1]{{\color{purple} #1}}
\newcommand{\orange} [1]{{\color{orange} #1}}
%END_FOLD


%% Assorted maths functions: \tv, \one, \exp, \logn, \Quad + others
%BEGIN_FOLD

\newcommand{\Quad}[1]{
	\mathchoice
	{\quad\text{#1}\quad}%	\displaystyle
	{\text{ #1 }}%			\textsyle
	{\text{ #1 }}%			\scriptstyle
	{\text{ #1 }}%			\scriptscriptstyle
}

\newcommand{\Qand}{\Quad{and}}
\newcommand{\Qas}{\Quad{as}}
\newcommand{\Qby}{\Quad{by}}
\newcommand{\Qfor}{\Quad{for}}
\newcommand{\Qforall}{\Quad{for all}}
\newcommand{\Qie}{\Quad{ie}}
\newcommand{\Qif}{\Quad{if}}
\newcommand{\Qor}{\Quad{or}}
\newcommand{\Qsince}{\Quad{since}}
\newcommand{\Qwhere}{\Quad{where}}
\newcommand{\Qwhen}{\Quad{when}}
\newcommand{\Qwith}{\Quad{with}}

\NewDocumentCommand{\tv}{sm}{%
	\IfBooleanT{#1}{\bigl}\lVert #2 \IfBooleanT{#1}{\bigr}\rVert_\mathsf{TV}%
}
%\newcommand{\TV}{\ifmmode \mathsf{TV} \else \textsf{TV}\xspace \fi}

\newcommand{\ONE}     {\bm1}
\newcommand{\one}  [1]{\bm1\{ #1 \}}
\newcommand{\oneb} [1]{\bm1\bigl\{ #1 \bigr\}}
\newcommand{\oneB} [1]{\bm1\Bigl\{ #1 \Bigr\}}
\newcommand{\onebb}[1]{\bm1\biggl\{ #1 \biggr\}}
\newcommand{\oneBB}[1]{\bm1\Biggl\{ #1 \Biggr\}}
\newcommand{\ones} [1]{\bm1\left\{ #1 \right\})}

\newcommand{\argmax}[1]{\underset{#1}{\operatorname{arg}\operatorname{max}}\:}
\newcommand{\argmin}[1]{\underset{#1}{\operatorname{arg}\operatorname{min}}\:}
\newcommand{\directsum}[1]{\underset{#1}{\oplus\:}}
\newcommand{\fnrestrict}[2]{\left. {#1} \right|_{#2}}


\newcommand{\caps}{\cap \cdots \cap}
\newcommand{\cups}{\cup \cdots \cup}
\DeclareMathOperator{\Triangle}{\triangle}
\newcommand{\ot}{\leftarrow}

\newcommand{\bcdot}{\ensuremath{\bm{\cdot}}}
\newcommand{\cq}{\coloneqq}
%END_FOLD


%% Brackets, absolute value and norm signs
%BEGIN_FOLD
\newcommand{\abs}  [1]{| #1 |}
\newcommand{\absb} [1]{\bigl| #1 \bigr|}
\newcommand{\absB} [1]{\Bigl| #1 \Bigr|}
\newcommand{\absbb}[1]{\biggl| #1 \biggr|}
\newcommand{\absBB}[1]{\Biggl| #1 \Biggr|}
\newcommand{\abss} [1]{\left| #1 \right|}

\newcommand{\NORM}  [1]{\| #1 \|}
\newcommand{\normb} [1]{\bigl\| #1 \bigr\|}
\newcommand{\normB} [1]{\Bigl\| #1 \Bigr\|}
\newcommand{\normbb}[1]{\biggl\| #1 \biggr\|}
\newcommand{\normBB}[1]{\Biggl\| #1 \Biggr\|}

\newcommand{\rbr} [1]{ ( #1 ) }
\newcommand{\rbb} [1]{\bigl( #1 \bigr)}
\newcommand{\rbB} [1]{\Bigl( #1 \Bigr)}
\newcommand{\rbbb}[1]{\biggl( #1 \biggr)}
\newcommand{\rbBB}[1]{\Biggl( #1 \Biggr)}

\newcommand{\sbr} [1]{ [ #1 ] }
\newcommand{\sbb} [1]{\bigl[ #1 \bigr]}
\newcommand{\sbB} [1]{\Bigl[ #1 \Bigr]}
\newcommand{\sbbb}[1]{\biggl[ #1 \biggr]}
\newcommand{\sbBB}[1]{\Biggl[ #1 \Biggr]}

\newcommand{\coi}[1]{[#1)}
\newcommand{\oci}[1]{(#1]}

\newcommand{\bra} [1]{ \{ #1 \} }
\newcommand{\brb} [1]{\bigl\{ #1 \bigr\}}
\newcommand{\brB} [1]{\Bigl\{ #1 \Bigr\}}
\newcommand{\brbb}[1]{\biggl\{ #1 \biggr\}}
\newcommand{\brBB}[1]{\Biggl\{ #1 \Biggr\}}

\newcommand{\floor}  [1]{\lfloor #1 \rfloor}
\newcommand{\floorb} [1]{\bigl \lfloor #1 \bigr \rfloor}
\newcommand{\floorB} [1]{\Bigl \lfloor #1 \Bigr \rfloor}
\newcommand{\floorbb}[1]{\biggl \lfloor #1 \biggr \rfloor}
\newcommand{\floorBB}[1]{\Biggl \lfloor #1 \Biggr \rfloor}

\newcommand{\ceil}[1]  {\lceil #1 \rceil}
\newcommand{\ceilb}[1] {\bigl\lceil #1 \bigr\rceil}
\newcommand{\ceilB}[1] {\Bigl\lceil #1 \Bigr\rceil}
\newcommand{\ceilBB}[1]{\Biggl\lceil #1 \Biggr\rceil}

\newcommand{\midb}{\bigm|}
\newcommand{\midB}{\Bigm|}
\newcommand{\midbb}{\biggm|}
\newcommand{\midBB}{\Biggm|}
\newcommand{\mids}{\middle|}

\newcommand{\binomB}[2]{\Bigl( \begin{matrix} #1 \\ #2 \end{matrix} \Bigr)}
%END_FOLD


%% Order notation 
%BEGIN_FOLD
\newcommand{\Om}  [1]{\Omega( #1 )}
\newcommand{\Omb} [1]{\Omega\bigl( #1 \bigr)}
\newcommand{\OmB} [1]{\Omega\Bigl( #1 \Bigr)}
\newcommand{\Ombb}[1]{\Omega\biggl( #1 \biggr)}
\newcommand{\OmBB}[1]{\Omega\Biggl( #1 \Biggr)}
\newcommand{\Oms} [1]{\Omega\left( #1 \right)}

\newcommand{\om}  [1]{\omega( #1 )}
\newcommand{\omb} [1]{\omega\bigl( #1 \bigr)}
\newcommand{\omB} [1]{\omega\Bigl( #1 \bigr)}
\newcommand{\ombb}[1]{\omega\biggl( #1 \biggr)}
\newcommand{\omBB}[1]{\omega\Biggl( #1 \Biggr)}
\newcommand{\oms} [1]{\omega\left( #1 \right)}

\newcommand{\Oh}  [1]{\mathcal{O}( #1 )}
\newcommand{\Ohb} [1]{\mathcal{O}\bigl( #1 \bigr)}
\newcommand{\OhB} [1]{\mathcal{O}\Bigl( #1 \Bigr)}
\newcommand{\Ohbb}[1]{\mathcal{O}\biggl( #1 \biggr)}
\newcommand{\OhBB}[1]{\mathcal{O}\Biggl( #1 \Biggr)}
\newcommand{\Ohs} [1]{\mathcal{O}\left( #1 \right)}

\newcommand{\oh}  [1]{o( #1 )}
\newcommand{\ohb} [1]{o\bigl( #1 \bigr)}
\newcommand{\ohB} [1]{o\Bigl( #1 \Bigr)}
\newcommand{\ohbb}[1]{o\biggl( #1 \biggr)}
\newcommand{\ohBB}[1]{o\Biggl( #1 \Biggr)}
\newcommand{\ohs} [1]{o\left( #1 \right)}

\newcommand{\Th}  [1]{\Theta( #1 )}
\newcommand{\Thb} [1]{\Theta\bigl( #1 \bigr)}
\newcommand{\ThB} [1]{\Theta\Bigl( #1 \Bigr)}
\newcommand{\Thbb}[1]{\Theta\biggl( #1 \biggr)}
\newcommand{\ThBB}[1]{\Theta\Biggl( #1 \Biggr)}
\newcommand{\Ths} [1]{\Theta\left( #1 \right)}
%END_FOLD


%% Probability/Expectation/Variance macros
%BEGIN_FOLD
%\NewDocumentCommand{\pr}{som}{%
%	\mathbb P\IfValueT{#2}{_{#2}}\IfBooleanT{#1}{\bigl}( #3 \IfBooleanT{#1}{\bigr})%
%}

\providecommand\given{}

\DeclarePairedDelimiterXPP{\PR}[2]
	{\mathbb P_{#1}}{[}{]}{}%
	{\renewcommand\given{\nonscript\:\delimsize\vert\nonscript\:\mathopen{}}#2}
\NewDocumentCommand{\pr}{som}{%
	\IfBooleanTF{#1}
	{\PR[\big]{\IfValueT{#2}{#2}}{#3}}
	{\PR{\IfValueT{#2}{#2}}{#3}}
}

\DeclarePairedDelimiterXPP{\EX}[2]
	{\mathbb E_{#1}}{[}{]}{}%
	{\renewcommand\given{\nonscript\:\delimsize\vert\nonscript\:\mathopen{}}#2}
\NewDocumentCommand{\ex}{som}{%
	\IfBooleanTF{#1}
	{\EX[\big]{\IfValueT{#2}{#2}}{#3}}
	{\EX{\IfValueT{#2}{#2}}{#3}}
}

\DeclarePairedDelimiterXPP{\VAR}[2]
	{\mathbb V\mathrm{ar}_{#1}}{(}{)}{}%
	{\renewcommand\given{\nonscript\:\delimsize\vert\nonscript\:\mathopen{}}#2}
\NewDocumentCommand{\var}{som}{%
	\IfBooleanTF{#1}
	{\VAR[\big]{\IfValueT{#2}{#2}}{#3}}
	{\VAR{\IfValueT{#2}{#2}}{#3}}
}

\let\originalexp\exp
\let\exp\relax
\DeclarePairedDelimiterXPP{\EXP}[1]{\originalexp}{(}{)}{}{#1}
\NewDocumentCommand{\exp}{sm}{%
	\IfBooleanTF{#1}
		{\EXP[\big]{#2}}
		{\EXP{#2}}
}


\DeclarePairedDelimiterX\SET[1]{\{}{\}}%
	{\renewcommand\given{\nonscript\:\delimsize\vert\nonscript\:\mathopen{}}#1}
\NewDocumentCommand{\set}{sm}{%
	\IfBooleanTF{#1}
		{\SET[\big]{#2}}
		{\SET{#2}}
}
%END_FOLD


%% Forcing use of textstyle when in display and adjusting vertical spacing for max

%BEGIN_FOLD
\usepackage{calc}
\newlength{\halfplusheight}
\setlength{\halfplusheight}{\totalheightof{\(+\)} * \real{.5} - \depthof{\(+\)}}

\newcommand{\limt}[1]{
	\mathchoice
	{\textstyle \lim_{#1} \displaystyle}
	{\lim_{#1}}
	{\lim_{#1}}
	{\lim_{#1}}
}

\newcommand{\LIM}[1]{%
	\mathop{\raisebox{\halfplusheight}{\(\displaystyle\lim_{#1}\)}}%
}

\newcommand{\maxt}[1]{
	\mathchoice
	{\textstyle \max_{#1} \displaystyle}
	{\max_{#1}}
	{\max_{#1}}
	{\max_{#1}}
}

\newcommand{\MAX}[1]{%
	\mathop{\raisebox{\halfplusheight}{\(\displaystyle\max_{#1}\)}}%
}

\newcommand{\supt}[1]{
	\mathchoice
	{\textstyle \sup_{#1} \displaystyle}
	{\sup_{#1}}
	{\sup_{#1}}
	{\sup_{#1}}
}

\newcommand{\SUP}[1]{%
	\mathop{\raisebox{\halfplusheight}{\(\displaystyle\sup_{#1}\)}}%
}

\newcommand{\limsupt}[1]{
	\mathchoice
	{\textstyle \limsup_{#1} \displaystyle}
	{\limsup_{#1}}
	{\limsup_{#1}}
	{\limsup_{#1}}
}

\newcommand{\LIMSUP}[1]{%
	\mathop{\raisebox{\halfplusheight}{\(\displaystyle\limsup_{#1}\)}}%
}

\newcommand{\mint}[1]{
	\mathchoice
	{\textstyle \min_{#1} \displaystyle}
	{\min_{#1}}
	{\min_{#1}}
	{\min_{#1}}
}

\newcommand{\MIN}[1]{%
	\mathop{\raisebox{\halfplusheight}{\(\displaystyle\min_{#1}\)}}%
}

\newcommand{\inft}[1]{
	\mathchoice
	{\textstyle \inf_{#1} \displaystyle}
	{\inf_{#1}}
	{\inf_{#1}}
	{\inf_{#1}}
}

\newcommand{\INF}[1]{%
	\mathop{\raisebox{\halfplusheight}{\(\displaystyle\inf_{#1}\)}}%
}

\newcommand{\liminft}[1]{
	\mathchoice
	{\textstyle \liminf_{#1} \displaystyle}
	{\liminf_{#1}}
	{\liminf_{#1}}
	{\liminf_{#1}}
}

\newcommand{\LIMINF}[1]{%
	\mathop{\raisebox{\halfplusheight}{\(\displaystyle\liminf_{#1}\)}}%
}

\newcommand{\binomt}[2]{
	\mathchoice
	{\textstyle \binom{#1}{#2} \displaystyle}
	{\binom{#1}{#2}}
	{\binom{#1}{#2}}
	{\binom{#1}{#2}}
}

\newcommand{\sumT}[2]{ \textstyle \sum_{#1}^{#2} \displaystyle }

%\newcommand{\sumt}[2][]{
%	\mathchoice
%	{\ifthenelse{\isempty{#1}}
%		{\textstyle \sum_{#2}      \displaystyle}
%		{\textstyle \sum_{#2}^{#1} \displaystyle}}%	\displaystyle
%	{\ifthenelse{\isempty{#1}}
%		{\sum_{#2}}
%		{\sum_{#2}^{#1}}}%			\textsyle
%	{\ifthenelse{\isempty{#1}}
%		{\sum_{#2}}
%		{\sum_{#2}^{#1}}}%			\scriptstyle
%	{\ifthenelse{\isempty{#1}}
%		{\sum_{#2}}
%		{\sum_{#2}^{#1}}}%			\scriptscriptstyle
%}

\NewDocumentCommand{\sumt}{smo}{%
	\mathchoice%
	{\textstyle\sum_{#2}\IfBooleanT{#1}{^\star}\IfValueT{#3}{^{#3}}\displaystyle}%
	{\sum_{#2}\IfBooleanT{#1}{^\star}\IfValueT{#3}{^{#3}}}%
	{\sum_{#2}\IfBooleanT{#1}{^\star}\IfValueT{#3}{^{#3}}}%
	{\sum_{#2}\IfBooleanT{#1}{^\star}\IfValueT{#3}{^{#3}}}%
}

\NewDocumentCommand{\sumd}{smo}{%
	\sum_{#2}\IfBooleanTF{#1}{^\star}{\IfValueT{#3}{^{#3}}}%
}

%\newcommand{\sumd}[2][]{
%	\ifthenelse{\isempty{#1}}
%		{\sum_{#2}}
%		{\sum_{#2}^{#1}}
%}

\NewDocumentCommand{\intt}{mo}{%
	\textstyle\int_{#1}\IfNoValueF{#2}{^{#2}}\displaystyle%
}
\newcommand{\intT}[2]{ \textstyle \int_{#1}^{#2} \displaystyle }
%\newcommand{\intt}[2][]{
%	\mathchoice
%	{\ifthenelse{\isempty{#1}}
%		{\textstyle \int_{#2}      \displaystyle}
%		{\textstyle \int_{#2}^{#1} \displaystyle}}
%	{\ifthenelse{\isempty{#1}}
%		{\int_{#2}}
%		{\int_{#2}^{#1}}}
%	{\ifthenelse{\isempty{#1}}
%		{\int_{#2}}
%		{\int_{#2}^{#1}}}
%	{\ifthenelse{\isempty{#1}}
%		{\int_{#2}}
%		{\int_{#2}^{#1}}}
%}

\newcommand{\prodT}[2]{ \textstyle \prod_{#1}^{#2} \displaystyle }
%\newcommand{\prodt}[2][]{
%	\mathchoice
%	{\ifthenelse{\isempty{#1}}
%		{\textstyle \prod_{#2}      \displaystyle}
%		{\textstyle \prod_{#2}^{#1} \displaystyle}}
%	{\ifthenelse{\isempty{#1}}
%		{\prod_{#2}}
%		{\prod_{#2}^{#1}}}
%	{\ifthenelse{\isempty{#1}}
%		{\prod_{#2}}
%		{\prod_{#2}^{#1}}}
%	{\ifthenelse{\isempty{#1}}
%		{\prod_{#2}}
%		{\prod_{#2}^{#1}}}	
%}

\NewDocumentCommand{\prodt}{mo}{%
	\mathchoice%
	{\textstyle \prod_{#1}\IfValueT{#2}{^{#2}} \displaystyle}%
	{\prod_{#1}\IfValueT{#2}{^{#2}}}%
	{\prod_{#1}\IfValueT{#2}{^{#2}}}%
	{\prod_{#1}\IfValueT{#2}{^{#2}}}%
}

\NewDocumentCommand{\prodd}{smo}{%
	\prod_{#2}\IfBooleanT{#1}{^\star}\IfValueT{#3}{^{#3}}%
}

%\newcommand{\prodd}[2][]{
%	\ifthenelse{\isempty{#1}}
%		{\prod_{#2}}
%		{\prod_{#2}^{#1}}
%}
%END_FOLD


%% "to infinity"/"to zero" limits
%BEGIN_FOLD
\newcommand{\toinf}[1]{\ensuremath{#1\to\infty}\xspace}
\newcommand{\asinf}[1]{\text{as \(#1\to\infty\)}\xspace}

\newcommand{\tozero}[1]{\ensuremath{#1\to0}\xspace}
\newcommand{\aszero}[1]{\text{as \(#1\to0\)}\xspace}

\newcommand{\nlim}{\lim_{n\to\infty}}
\newcommand{\ninf}{\ensuremath{n\to\infty}}
%END_FOLD


%% Assorted, names, words, etc
%BEGIN_FOLD
%\DeclareMathOperator{\Unif}{Unif}
\newcommand{\iid}{\textsf{iid}}
\DeclareMathOperator{\Exp}{Exp}
\DeclareMathOperator{\Geom}{Geom}
\DeclareMathOperator{\Po}{Po}
\DeclareMathOperator{\Pois}{Pois}
\DeclareMathOperator{\Bern}{Bern}
\DeclareMathOperator{\Bin}{Bin}
\DeclareMathOperator{\NegBin}{NegBin}
\DeclareMathOperator{\NB}{NB}
\DeclareMathOperator{\Beta}{Beta}
\DeclareMathOperator{\Unif}{Unif}

\DeclareMathOperator{\N}{N}
\DeclareMathOperator{\Norm}{Norm}

\DeclareMathOperator{\Stab}{Stab}
\DeclareMathOperator{\Orb}{Orb}
\DeclareMathOperator{\tr}{tr}
\DeclareMathOperator{\Tr}{Tr}

\DeclareMathOperator{\poly}{poly}

\DeclareMathOperator{\dist}{\textnormal{dist}}
\DeclareMathOperator{\diam}{diam}

\newcommand{\RV}{\textsf{RV}\xspace}
\newcommand{\RVs}{\textsf{RV}s\xspace}
%\newcommand{\RW}{\textsf{RW}\xspace}
\newcommand{\RW}{\ifmmode \mathsf{RW} \else \textsf{RW}\xspace \fi}
\newcommand{\RWs}{\textsf{RW}s\xspace}

%\newcommand{\TV}{\textsf{TV}\xspace}
\newcommand{\TV}{\ifmmode \mathsf{TV} \else \textsf{TV}\xspace \fi}

\newcommand{\st}{\:\textnormal{st}\:}
\newcommand{\dmax}{d_{\max}}
\newcommand{\uar}{\textsf{uar}\xspace}
\newcommand{\id}{\mathsf{id}}
\newcommand{\typ}{\textnormal{typ}}
\DeclareMathOperator{\HT}{\textnormal{ht}}

\newcommand{\tmix}{t_\mix}
\newcommand{\trel}{t_\rel}
\newcommand{\mix}{\textnormal{mix}}
\newcommand{\hit}{\textnormal{hit}}
\newcommand{\rel}{\textnormal{rel}}
\newcommand{\tauc}{\tau_\textnormal{c}}

\newcommand{\RHS}{\textnormal{RHS}}
\newcommand{\LHS}{\textnormal{LHS}}

\newcommand{\MT}{\textnormal{MT}}
\newcommand{\ET}{\textnormal{ET}}
%END_FOLD

%% Abbreviations for (Markov) chain names
%BEGIN_FOLD
\newcommand{\Xt}{(X_t)_{t\ge0}}
\newcommand{\Yt}{(Y_t)_{t\ge0}}
\newcommand{\Zt}{(Z_t)_{t\ge0}}
\newcommand{\ninn}{{n\in\mathbb{N}}}
\newcommand{\Ninn}{{N\in\mathbb{N}}}
%END_FOLD


%% \mathbb (\mb#), \mathcal (\mc#) and \mathscr (\ms#) macros
%BEGIN_FOLD
\newcommand{\mba}{\mathbb{A}}
\newcommand{\mbb}{\mathbb{B}}
\newcommand{\mbc}{\mathbb{C}}
\newcommand{\mbd}{\mathbb{D}}
\newcommand{\mbe}{\mathbb{E}}
\newcommand{\mbf}{\mathbb{F}}
\newcommand{\mbg}{\mathbb{G}}
\newcommand{\mbh}{\mathbb{H}}
\newcommand{\mbi}{\mathbb{I}}
\newcommand{\mbj}{\mathbb{J}}
\newcommand{\mbk}{\mathbb{K}}
\newcommand{\mbl}{\mathbb{L}}
\newcommand{\mbm}{\mathcb{M}}
\newcommand{\mbn}{\mathbb{N}}
\newcommand{\mbN}{\mathbb{N}_0}
\newcommand{\mbo}{\mathbb{O}}
\newcommand{\mbp}{\mathbb{P}}
\newcommand{\mbq}{\mathbb{Q}}
\newcommand{\mbr}{\mathbb{R}}
\newcommand{\mbs}{\mathbb{S}}
\newcommand{\mbt}{\mathbb{T}}
\newcommand{\mbu}{\mathbb{U}}
\newcommand{\mbv}{\mathbb{V}}
\newcommand{\mbw}{\mathbb{W}}
\newcommand{\mbx}{\mathbb{X}}
\newcommand{\mby}{\mathbb{Y}}
\newcommand{\mbz}{\mathbb{Z}}

\newcommand{\mca}{\mathcal{A}}
\newcommand{\mcb}{\mathcal{B}}
\newcommand{\mcc}{\mathcal{C}}
\newcommand{\mcd}{\mathcal{D}}
\newcommand{\mce}{\mathcal{E}}
\newcommand{\mcf}{\mathcal{F}}
\newcommand{\mcg}{\mathcal{G}}
\newcommand{\mch}{\mathcal{H}}
\newcommand{\mci}{\mathcal{I}}
\newcommand{\mcj}{\mathcal{J}}
\newcommand{\mck}{\mathcal{K}}
\newcommand{\mcl}{\mathcal{L}}
\newcommand{\mcm}{\mathcal{M}}
\newcommand{\mcn}{\mathcal{N}}
\newcommand{\mco}{\mathcal{O}}
\newcommand{\mcp}{\mathcal{P}}
\newcommand{\mcq}{\mathcal{Q}}
\newcommand{\mcr}{\mathcal{R}}
\newcommand{\mcs}{\mathcal{S}}
\newcommand{\mct}{\mathcal{T}}
\newcommand{\mcu}{\mathcal{U}}
\newcommand{\mcv}{\mathcal{V}}
\newcommand{\mcw}{\mathcal{W}}
\newcommand{\mcx}{\mathcal{X}}
\newcommand{\mcy}{\mathcal{Y}}
\newcommand{\mcz}{\mathcal{Z}}

\newcommand{\msa}{\mathscr{A}}
\newcommand{\msb}{\mathscr{B}}
\newcommand{\msc}{\mathscr{C}}
\newcommand{\msd}{\mathscr{D}}
\newcommand{\mse}{\mathscr{E}}
\newcommand{\msf}{\mathscr{F}}
\newcommand{\msg}{\mathscr{G}}
\newcommand{\msh}{\mathscr{H}}
\newcommand{\msi}{\mathscr{I}}
\newcommand{\msj}{\mathscr{J}}
\newcommand{\msk}{\mathscr{K}}
\newcommand{\msl}{\mathscr{L}}
\newcommand{\msm}{\mathscr{M}}
\newcommand{\msn}{\mathscr{N}}
\newcommand{\mso}{\mathscr{O}}
\newcommand{\msP}{\mathscr{P}}
\newcommand{\msq}{\mathscr{Q}}
\newcommand{\msr}{\mathscr{R}}
\newcommand{\mss}{\mathscr{S}}
\newcommand{\mst}{\mathscr{T}}
\newcommand{\msu}{\mathscr{U}}
\newcommand{\msv}{\mathscr{V}}
\newcommand{\msw}{\mathscr{W}}
\newcommand{\msx}{\mathscr{X}}
\newcommand{\msy}{\mathscr{Y}}
\newcommand{\msz}{\mathscr{Z}}

\newcommand{\mfa}{\mathfrak{a}}
\newcommand{\mfb}{\mathfrak{b}}
\newcommand{\mfc}{\mathfrak{c}}
\newcommand{\mfd}{\mathfrak{d}}
\newcommand{\mfe}{\mathfrak{e}}
\newcommand{\mff}{\mathfrak{f}}
\newcommand{\mfg}{\mathfrak{g}}
\newcommand{\mfh}{\mathfrak{h}}
\newcommand{\mfi}{\mathfrak{i}}
\newcommand{\mfj}{\mathfrak{j}}
\newcommand{\mfk}{\mathfrak{k}}
\newcommand{\mfl}{\mathfrak{l}}
\newcommand{\mfm}{\mathfrak{m}}
\newcommand{\mfn}{\mathfrak{n}}
\newcommand{\mfo}{\mathfrak{o}}
\newcommand{\mfp}{\mathfrak{p}}
\newcommand{\mfq}{\mathfrak{q}}
\newcommand{\mfr}{\mathfrak{r}}
\newcommand{\mfs}{\mathfrak{s}}
\newcommand{\mft}{\mathfrak{t}}
\newcommand{\mfu}{\mathfrak{u}}
\newcommand{\mfv}{\mathfrak{v}}
\newcommand{\mfw}{\mathfrak{w}}
\newcommand{\mfx}{\mathfrak{x}}
\newcommand{\mfy}{\mathfrak{y}}
\newcommand{\mfz}{\mathfrak{z}}

\newcommand{\mfA}{\mathfrak{A}}
\newcommand{\mfB}{\mathfrak{B}}
\newcommand{\mfC}{\mathfrak{C}}
\newcommand{\mfD}{\mathfrak{D}}
\newcommand{\mfE}{\mathfrak{E}}
\newcommand{\mfF}{\mathfrak{F}}
\newcommand{\mfG}{\mathfrak{G}}
\newcommand{\mfH}{\mathfrak{H}}
\newcommand{\mfI}{\mathfrak{I}}
\newcommand{\mfJ}{\mathfrak{J}}
\newcommand{\mfK}{\mathfrak{K}}
\newcommand{\mfL}{\mathfrak{L}}
\newcommand{\mfM}{\mathfrak{M}}
\newcommand{\mfN}{\mathfrak{N}}
\newcommand{\mfO}{\mathfrak{O}}
\newcommand{\mfP}{\mathfrak{P}}
\newcommand{\mfQ}{\mathfrak{Q}}
\newcommand{\mfR}{\mathfrak{R}}
\newcommand{\mfS}{\mathfrak{S}}
\newcommand{\mfT}{\mathfrak{T}}
\newcommand{\mfU}{\mathfrak{U}}
\newcommand{\mfV}{\mathfrak{V}}
\newcommand{\mfW}{\mathfrak{W}}
\newcommand{\mfX}{\mathfrak{X}}
\newcommand{\mfY}{\mathfrak{Y}}
\newcommand{\mfZ}{\mathfrak{Z}}
%END_FOLD


%% Assorted other commands
%BEGIN_FOLD
\newcommand{\nt}{\addtocounter{equation}{1}\tag{\theequation}}

\newcommand{\eps}{\varepsilon}

\newcommand{\gap}{}
\newcommand{\GAP}[1]{%
	\renewcommand{\gap}{\hspace{#1em}}}

\newcommand{\commentedout}{%
	{\color{blue}This section has been \%commented out; it may not be necessary}%
}

%\usepackage[hang,flushmargin]{footmisc}
\usepackage{manyfoot}
\SetFootnoteHook{\hspace*{-1.8em}}
\DeclareNewFootnote{bl}[gobble]
\setlength{\skip\footinsbl}{0pt}
\newcommand{\blfootnote}[1]{\footnotebl{\sffamily#1}}
%\newcommand{\blfootnote}[1]{%
%	\begingroup
%	\renewcommand\thefootnote{}\footnote{\sffamily#1}%
%	\addtocounter{footnote}{-1}%
%	\endgroup
%}

\newcommand{\divline}{\bigskip\hrule\bigskip}
\newcommand{\ddivline}{\bigskip\hrule\vspace{1mm}\hrule\bigskip}
%END_FOLD



%% Making \[ ... \] into align or equation according to presence of &

\def\IfAmpersandUseAlign#1#2&#3\EndIfAmpersandUseAlign%
	{%
		\if\relax\detokenize{#3}\relax
		\begin{equation*}%
		#1%
		\end{equation*}%
		\else
		\begin{align*}%
		#1%
		\end{align*}%
		\fi
	}
	\def\[#1\]%
	{%
		\IfAmpersandUseAlign{#1}#1&\EndIfAmpersandUseAlign
	}

\usepackage{eqparbox}
\newcommand{\eqmathsbox}[3][\mathbin]{%
	% #1 = atom type, #2 = label, #3 = object
	#1{\eqmakebox[#2]{$\displaystyle#3$}}%
}
%\eqmathsbox[\mathrel]{A}{\gg}


%% Statements (thm, prop, etc), with 'triangle' version

%\newcommand{\qedtri}{\renewcommand{\qedsymbol}{\ensuremath{\triangle}}}
\newcommand{\qedtriangle}{\renewcommand{\qedsymbol}{\ensuremath{\triangle}}}
\newcommand{\noqed}{\renewcommand{\qedsymbol}{}}
%\usepackage{aliascnt}

\usepackage[noabbrev, capitalise]{cleveref}
\newcommand{\creflastconjunction}{ and~}
\newcommand{\creflastgroupconjunction}{ and~}
%\newcommand{\crefrangeconjunction}{--}

%\usepackage{xpatch}
%\makeatletter
%\xpatchcmd{\@cref}{\begingroup}{\begingroup\upshape\sffamily}{}{}
%\makeatother

\crefname{figure}{Figure}{Figures}

\numberwithin{equation}{section}

\crefformat{equation}{\textup{(#2#1#3)}}
\crefmultiformat{equation}{\textup{(#2#1#3}}{\textup{, #2#1#3)}}{\textup{, #2#1#3}}{\textup{, #2#1#3)}}
\crefrangeformat{equation}{\textup{(#3#1#4--#5#2#6)}}

\crefformat{section}{\S#2#1#3}
\crefformat{subsection}{\S#2#1#3}
\crefformat{subsubsection}{\S#2#1#3}


\newenvironment{Proof}[1][\proofname]{%
	\proof[\upshape\bfseries\sffamily\boldmath#1]
}{\endproof}

%\newenvironment{Proof}[1][\proofname]{%
%	\proof[\rm\bf{#1}]
%}{\endproof}

\usepackage{etoolbox}
\usepackage{needspace}


%% DEFINE THEOREM STYLES

\newcommand{\nextresult}{%
	\setcounter{introthm}{\value{introthm}}
	\setcounter{introlem}{\value{introthm}}
	\setcounter{introdefn}{\value{introthm}}
	\setcounter{intrormkT}{\value{introthm}}
}

\newtheoremstyle{sfsl}
	{1\baselineskip}		% Space above
	{1\baselineskip}		% Space below
	{\slshape}				% Theorem body font
	{}						% Indent amount
	{\bfseries\sffamily}	% Theorem head font
	{.}						% Punctuation after theorem head
	{0.5em}					% Space after theorem head
	{\thmname{#1}\thmnumber{ #2}\thmnote{ {\mdseries(#3)}}}
							% Theorem head spec

\newtheoremstyle{sfup}
	{1\baselineskip}		% Space above
	{1\baselineskip}		% Space below
	{\upshape}				% Theorem body font
	{}						% Indent amount
	{\bfseries\sffamily}	% Theorem head font
	{.}						% Punctuation after theorem head
	{0.5em}					% Space after theorem head
	{\thmname{#1}\thmnumber{ #2}\thmnote{ {\mdseries(#3)}}}
							% Theorem head spec

%% Theorem/etc environment with slshape

\theoremstyle{sfsl}

%\newtheorem{mainthm}{Theorem}
%\renewcommand*{\themainthm}{\Alph{mainthm}}
%\newtheorem*{mainthm*}{Theorem}
%
%\newtheorem{maincor}{Corollary}
%\renewcommand*{\themaincor}{\Alph{maincor}}
%\newtheorem*{maincor*}{Corollary}

\newtheorem*{thm*}{Theorem}
\newtheorem{thm} {Theorem}[section]
%\crefformat{thm}{#2\bfseries\thmname~#1#3}
\crefname{thm}{Theorem}{Theorems}


\newtheorem*{introthm*}{Theorem}
\newtheorem{introthm}{Theorem}
\renewcommand*{\theintrothm}{\Alph{introthm}}
%\AtBeginEnvironment{mainthm}{\Needspace{5\baselineskip}}
\crefname{introthm}{Theorem}{Theorems}

\newtheorem*{cor*}{Corollary}
\newtheorem{cor} [thm]{Corollary}
\crefname{cor}{Corollary}{Corollaries}

\newtheorem*{introcor*}{Corollary}
\newtheorem{introcor}{Corollary}
\renewcommand*{\theintrocor}{\Alph{introcor}}
\crefname{introcor}{Corollary}{Corollaries}

\newtheorem*{introconj*}{Conjecture}
\newtheorem{introconj}{Conjecture}
\renewcommand*{\theintroconj}{\Alph{introconj}}
\crefname{introconj}{Conjecture}{Conjectures}

\newtheorem*{introques*}{Question}
\newtheorem{introques}{Question}
\renewcommand*{\theintroques}{\Alph{introques}}
\crefname{introques}{Question}{Questions}

\newtheorem*{lem*}    {Lemma}
\newtheorem{lem} [thm]{Lemma}
\crefname{lem}{Lemma}{Lemmas}

\newtheorem*{introlem*}{Lemma}
\newtheorem{introlem}{Lemma}
\renewcommand*{\theintrolem}{\Alph{introlem}}
\crefname{introlem}{Lemma}{Lemmas}

\newtheorem*{prop*}    {Proposition}
\newtheorem{prop} [thm]{Proposition}
\crefname{prop}{Proposition}{Propositions}

\newtheorem*{openq*}    {Open Question}
\newtheorem{openq}[thm]{Open Question}
\crefname{openq}{Open Question}{Open Questions}

\newtheorem{reduction}[thm]{Reduction}

\newtheorem*{state}{Statement}

\newtheorem*{clm*}    {Claim}
\newtheorem{clm} [thm]{Claim}
\crefname{clm}{Claim}{Claims}

\newtheorem*{defn*}    {Definition}
\newtheorem{defn} [thm]{Definition}
\crefname{defn}{Definition}{Definitions}

\newtheorem*{introdefn*}{Definition}
\newtheorem{introdefn}{Definition}
\renewcommand*{\theintrodefn}{\Alph{introdefn}}
\crefname{introdefn}{Definition}{Definitions}

\newtheorem*{alg*}{Algorithm}
\newtheorem{alg}[thm]{Algorithm}
\crefname{alg}{Algorithm}{Algorithms}

\newtheorem{nota}[thm]{Notation}
\crefname{nota}{Notation}{Notations}


% Define custom environment

\newtheorem{innercustomgeneric}{\customgenericname}
\providecommand{\customgenericname}{}
\newcommand{\newcustomtheorem}[1]{%
	\newenvironment{#1}[1]
	{%
%		\renewcommand\customgenericname{#2}%
		\renewcommand\theinnercustomgeneric{##1}%
		\innercustomgeneric%
	}
	{\endinnercustomgeneric}
}

%\newcustomtheorem{customthm}{Theorem}
%\newcustomtheorem{customprop}{Proposition}
%\newcustomtheorem{customlemma}{Lemma}
%\newcustomtheorem{customconj}{Conjecture}
%\newcustomtheorem{customdefn}{Definition}

\newtheorem*{customgeninner}{\customgenname}
\newcommand{\customgenname}{}
\newenvironment{customgen}[1]
	{\renewcommand{\customgenname}{#1}\customgeninner}
	{\endcustomgeninner}

\newtheorem*{conj*}   {Conjecture}
\newtheorem{conj}[thm]{Conjecture}
\crefname{conj}{Conjecture}{Conjectures}

\newenvironment{conj-ind*}
	{\begin{quote}\textsf{\textbf{Conjecture.}}\slshape}
	{\end{quote}}
\newenvironment{conj-ind}
	{\begin{quote}\vspace{-\glueexpr\baselineskip+\topsep}\begin{customconj}}
	{\end{customconj}\end{quote}}

\newenvironment{question-ind*}
	{\begin{quote}\textsf{\textbf{Question.}}\slshape}
	{\end{quote}}
\newenvironment{question-ind}
	{\begin{quote}\vspace{-\glueexpr\baselineskip+\topsep}\begin{customquestion}}
	{\end{customquestion}\end{quote}}

\newenvironment{openproblem-ind*}
	{\begin{quote}\textsf{\textbf{Open Problem.}}\slshape}
	{\end{quote}}
\newenvironment{openproblem-ind}
	{\begin{quote}\vspace{-\glueexpr\baselineskip+\topsep}\begin{customopenproblem}}
	{\end{customopenproblem}\end{quote}}


\newtheorem*{hypothesis*}{Hypothesis}
\newtheorem{hypothesis}{Hypothesis}
\renewcommand*{\thehypothesis}{\Alph{hypothesis}}

\newtheorem*{hyp*}{Hypothesis}
\newtheorem{hyp}{Hypothesis}
\renewcommand*{\thehyp}{\Alph{hyp}}
\crefname{hyp}{Hypothesis}{Hypotheses}

\newtheorem{rmk} [thm]{Remark}
\newtheorem*{rmk*}{Remark}


%% Remark/etc with upshape and qed-triangle

\theoremstyle{sfup}

%\newtheorem{defn} [thm]{Definition}
\newtheorem{defnT}[thm]{Definition}
	\newenvironment{defnt}
	{\pushQED{\qed}\renewcommand{\qedsymbol}{\ensuremath{\triangle}}\defnT}
	{\popQED\enddefnT}
\crefname{defn} {Definition}{Definitions}
\crefname{defnT}{Definition}{Definitions}

%\newtheorem*{defn*} {Definition}
\newtheorem*{defnTT}{Definition}
\newenvironment{defnt*}
	{\pushQED{\qed}\renewcommand{\qedsymbol}{\ensuremath{\triangle}}\defnTT}
	{\popQED\enddefnTT}

%\newtheorem{rmk} [thm]{Remark}
\newtheorem{rmkT}[thm]{Remark}
	\newenvironment{rmkt}
	{\pushQED{\qed}\renewcommand{\qedsymbol}{\ensuremath{\triangle}}\rmkT}
	{\popQED\endrmkT}
\crefname{rmk} {Remark}{Remarks}
\crefname{rmkT}{Remark}{Remarks}

%\newtheorem*{rmk*} {Remark}
\newtheorem*{rmkTT}{Remark}
\newenvironment{rmkt*}
	{\pushQED{\qed}\renewcommand{\qedsymbol}{\ensuremath{\triangle}}\rmkTT}
	{\popQED\endrmkTT}

\newtheorem{rmks} [thm]{Remarks}
\newtheorem{rmksT}[thm]{Remarks}
\newenvironment{rmkst}
	{\pushQED{\qed}\renewcommand{\qedsymbol}{\ensuremath{\triangle}}\rmksT}
	{\popQED\endrmksT}
\crefname{rmks} {Remarks}{Remarks}
\crefname{rmksT}{Remarks}{Remarks}

\newtheorem*{rmks*} {Remarks}
\newtheorem*{rmksTT}{Remarks}
\newenvironment{rmkst*}
	{\pushQED{\qed}\renewcommand{\qedsymbol}{\ensuremath{\triangle}}\rmksTT}
	{\popQED\endrmksTT}

\newtheorem{intrormk}{Remark}
\newtheorem{intrormkT}{Remark}
	\newenvironment{intrormkt}
	{\pushQED{\qed}\renewcommand{\qedsymbol}{\ensuremath{\triangle}}\intrormkT}
	{\popQED\endintrormkT}
\renewcommand*{\theintrormk}{\Alph{intrormk}}
\renewcommand*{\theintrormkT}{\Alph{intrormkT}}
\crefname{intrormk} {Remark}{Remarks}
\crefname{intrormkT}{Remark}{Remarks}

\newtheorem*{intrormk*} {Remark}
\newtheorem*{intrormkTT}{Remark}
\newenvironment{intrormkt*}
	{\pushQED{\qed}\renewcommand{\qedsymbol}{\ensuremath{\triangle}}\intrormkTT}
	{\popQED\endintrormkTT}

\newtheorem{exm} [thm]{Example}
\newtheorem{exmT}[thm]{Example}
\newenvironment{exmt}
	{\pushQED{\qed}\renewcommand{\qedsymbol}{\ensuremath{\triangle}}\exmT}
	{\popQED\endexmT}
\crefname{exm} {Example}{Examples}
\crefname{exmT}{Example}{Examples}

\newtheorem*{exm*} {Example}
\newtheorem*{exmTT}{Lemma}
	\newenvironment{exmt*}
	{\pushQED{\qed}\renewcommand{\qedsymbol}{\ensuremath{\triangle}}\exmTT}
	{\popQED\endexmTT}

\newtheorem{con} [thm]{Condition}
\newtheorem{conT}[thm]{Condition}
	\newenvironment{cont}
	{\pushQED{\qed}\renewcommand{\qedsymbol}{\ensuremath{\triangle}}\conT}
	{\popQED\endconT}

\newtheorem*{notation} {Notation}
\newtheorem*{notationT}{Notation}
	\newenvironment{notationt}
	{\pushQED{\qed}\renewcommand{\qedsymbol}{\ensuremath{\triangle}}\notationT}
	{\popQED\endnotationT}

\newtheorem*{note*} {Note}
\newtheorem*{noteTT}{Note}
	\newenvironment{notet*}
	{\pushQED{\qed}\renewcommand{\qedsymbol}{\ensuremath{\triangle}}\noteTT}
	{\popQED\endnoteTT}

%END_FOLD


%%% Subtheorem, widebar and equation referencing

% Subtheorem

%\usepackage{hyperref}
\makeatletter
\newenvironment{subtheorem}[1]{%
	\def\subtheoremcounter{#1}%
	\refstepcounter{#1}%
	\protected@edef\theparentnumber{\csname the#1\endcsname}%
	\setcounter{parentnumber}{\value{#1}}%
	\setcounter{#1}{0}%
	\expandafter\def\csname the#1\endcsname{\theparentnumber\alph{#1}}%
	% To keep hyperref happy, update H-counter as well
	\expandafter\def\csname theH#1\endcsname{thm.\theparentnumber\alph{#1}}%
	\unskip\ignorespaces
}{%
	\setcounter{\subtheoremcounter}{\value{parentnumber}}%
	\ignorespacesafterend
}
\makeatother
\newcounter{parentnumber}

\makeatletter
\newenvironment{subtheorem-num}[1]{%
	\def\subtheoremcounter{#1}%
	\refstepcounter{#1}%
	\protected@edef\theparentnumber{\csname the#1\endcsname}%
	\setcounter{parentnumber}{\value{#1}}%
	\setcounter{#1}{0}%
	\expandafter\def\csname the#1\endcsname{\theparentnumber.\arabic{#1}}%
	% To keep hyperref happy, update H-counter as well
	\expandafter\def\csname theH#1\endcsname{thm.\theparentnumber.\arabic{#1}}%
	\unskip\ignorespaces
}{%
	\setcounter{\subtheoremcounter}{\value{parentnumber}}%
	\ignorespacesafterend
}
\makeatother
%\newcounter{parentnumber}


%% widebar -- wider than \bar but not as wide as \overline

\makeatletter
\let\save@mathaccent\mathaccent
\newcommand*\if@single[3]{%
  \setbox0\hbox{${\mathaccent"0362{#1}}^H$}%
  \setbox2\hbox{${\mathaccent"0362{\kern0pt#1}}^H$}%
  \ifdim\ht0=\ht2 #3\else #2\fi
  }
%The bar will be moved to the right by a half of \macc@kerna, which is computed by amsmath:
\newcommand*\rel@kern[1]{\kern#1\dimexpr\macc@kerna}
%If there's a superscript following the bar, then no negative kern may follow the bar;
%an additional {} makes sure that the superscript is high enough in this case:
\newcommand*\widebar[1]{\@ifnextchar^{{\wide@bar{#1}{0}}}{\wide@bar{#1}{1}}}
%Use a separate algorithm for single symbols:
\newcommand*\wide@bar[2]{\if@single{#1}{\wide@bar@{#1}{#2}{1}}{\wide@bar@{#1}{#2}{2}}}
\newcommand*\wide@bar@[3]{%
  \begingroup
  \def\mathaccent##1##2{%
%Enable nesting of accents:
    \let\mathaccent\save@mathaccent
%If there's more than a single symbol, use the first character instead (see below):
    \if#32 \let\macc@nucleus\first@char \fi
%Determine the italic correction:
    \setbox\z@\hbox{$\macc@style{\macc@nucleus}_{}$}%
    \setbox\tw@\hbox{$\macc@style{\macc@nucleus}{}_{}$}%
    \dimen@\wd\tw@
    \advance\dimen@-\wd\z@
%Now \dimen@ is the italic correction of the symbol.
    \divide\dimen@ 3
    \@tempdima\wd\tw@
    \advance\@tempdima-\scriptspace
%Now \@tempdima is the width of the symbol.
    \divide\@tempdima 10
    \advance\dimen@-\@tempdima
%Now \dimen@ = (italic correction / 3) - (Breite / 10)
    \ifdim\dimen@>\z@ \dimen@0pt\fi
%The bar will be shortened in the case \dimen@<0 !
    \rel@kern{0.6}\kern-\dimen@
    \if#31
      \overline{\rel@kern{-0.6}\kern\dimen@\macc@nucleus\rel@kern{0.4}\kern\dimen@}%
      \advance\dimen@0.4\dimexpr\macc@kerna
%Place the combined final kern (-\dimen@) if it is >0 or if a superscript follows:
      \let\final@kern#2%
      \ifdim\dimen@<\z@ \let\final@kern1\fi
      \if\final@kern1 \kern-\dimen@\fi
    \else
      \overline{\rel@kern{-0.6}\kern\dimen@#1}%
    \fi
  }%
  \macc@depth\@ne
  \let\math@bgroup\@empty \let\math@egroup\macc@set@skewchar
  \mathsurround\z@ \frozen@everymath{\mathgroup\macc@group\relax}%
  \macc@set@skewchar\relax
  \let\mathaccentV\macc@nested@a
%The following initialises \macc@kerna and calls \mathaccent:
  \if#31
    \macc@nested@a\relax111{#1}%
  \else
%If the argument consists of more than one symbol, and if the first token is
%a letter, use that letter for the computations:
    \def\gobble@till@marker##1\endmarker{}%
    \futurelet\first@char\gobble@till@marker#1\endmarker
    \ifcat\noexpand\first@char A\else
      \def\first@char{}%
    \fi
    \macc@nested@a\relax111{\first@char}%
  \fi
  \endgroup
}
\makeatother


%% equation referencing

\usepackage{xparse}
\ExplSyntaxOn
\NewDocumentCommand{\mref}{m}{\quinn_mref:n {#1}}
\seq_new:N \l_quinn_mref_seq
\cs_new:Npn \quinn_mref:n #1
{
	\seq_set_split:Nnn \l_quinn_mref_seq { , } { #1 }
	\seq_pop_right:NN \l_quinn_mref_seq \l_tmpa_tl
	( % print the left bracket
	\seq_map_inline:Nn \l_quinn_mref_seq
	{ \ref{##1},\nobreakspace } % print the first references
	\exp_args:NV \ref \l_tmpa_tl % print the last or only one
	) % print the right bracket
}
\ExplSyntaxOff
%END_FOLD
