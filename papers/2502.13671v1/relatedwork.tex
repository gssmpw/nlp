\section{Related works}
Since EF allocations may not exist, the majority of the literature is focused on its relaxations when subsidy is not allowed. Two widely adopted relaxations are envy-free up to one item (EF1) \cite{budish2011combinatorial} and envy-free up to any item (EFX) \cite{DBLP:journals/teco/CaragiannisKMPS19}, which requires no agent strictly prefers another agent's bundle excluding one or any item.
EF1 allocations always exist \cite{DBLP:conf/sigecom/LiptonMMS04} but the existence of EFX allocations is still unknown except for several special cases, e.g., \cite{DBLP:journals/siamdm/PlautR20,DBLP:journals/jacm/ChaudhuryGM24}. 
We refer to the recent surveys \cite{DBLP:journals/ai/AmanatidisABFLMVW23,DBLP:journals/jair/LiuLSW24} for a comprehensive overview of fair allocation.
Below we restrict our focus on the graph orientation and fair division with subsidy.

\paragraph{Graph Orientation}
Christodoulou et al. \cite{DBLP:conf/sigecom/0001FKS23} first introduced the graph model where vertices are agents and edges are items such that agents are only interested in their incident items.
When agents have additive valuations, they proved that an EFX allocations when the items can be arbitrarily allocated always exists but an EFX orientation where items must be allocated to their incident agents may not exist.
Zeng and Mehta \cite{zeng2024structureefxorientationsgraphs} identified a family of graphs which admit EFX orientations exist. 
Kaviani et al. \cite{DBLP:journals/corr/abs-2407-05139} and Deligkas et al. \cite{deligkas2024ef1efxorientations} 
repectively generalized the graph orientation problem.
In the model of Kaviani et al. \cite{DBLP:journals/corr/abs-2407-05139}, agents have $(p,q)$-bounded valuations, where each item has a non-zero marginal value for at most $p$ agents, and any two agents share at most $q$ common interested items. 
They proved that for $(\infty,1)$-bounded valuations, EF2X orientations always exist.
In the model of Deligkas et al. \cite{deligkas2024ef1efxorientations}, every agent has a predetermined set of items and only receives items from this set. They proved that for monotone valuations, an EF1 allocation always exists. 
Zhou et al. \cite{zhoucomplete} extended the goods model in \cite{DBLP:conf/sigecom/0001FKS23} to mixed setting when the marginal value of each item can be positive or negative.



% strengthened the result of \cite{DBLP:conf/sigecom/0001FKS23} by proving that even with 8 agents, the problem remains NP-hard when the graph has parallel edges.


% \cite{zeng2024structureefxorientationsgraphs} characterized the structure of EFX orientable graphs (i.e., graphs that have an EFX orientation regardless of valuation) and proved a connection to the chromatic number of the graph.
% More general orientation problems \cite{DBLP:journals/corr/abs-2407-05139}, \cite{deligkas2024ef1efxorientations} and \cite{DBLP:journals/corr/abs-2409-03594}.
% \cite{DBLP:journals/corr/abs-2407-05139} considered $(p,q)$-bounded valuations, where each good holds relevance (i.e., has a non-zero marginal value) for at most $p$ agents, and any pair of agents share at most $q$ common relevant goods. Thus the model of \cite{DBLP:conf/sigecom/0001FKS23} focuses on (2,1)-bounded valuations.
% \cite{deligkas2024ef1efxorientations} proved a stronger result than \cite{DBLP:conf/sigecom/0001FKS23}: it is
% NP-complete to decide whether an EFX orientation exists even when the graph is simple and has vertex cover of size 10 and the valuations are additive and symmetric. For multi-graphs,  finding an EFX allocation is NP-hard even when there are 8 agents with symmetric and additive valuations.
% \cite{DBLP:journals/corr/abs-2409-03594} considered a mixed setting where the valuations for edges adjacent to agents can be positive or negative.



\paragraph{Fair Division with Subsidy}
Caragiannis and Ioannidis \cite{DBLP:conf/wine/CaragiannisI21} first studied the computational complexity of using the minimum amount of subsidies to ensure envy-freeness.
Halpern and Shah \cite{DBLP:conf/sagt/HalpernS19} bounded the worst-case amount of subsidies for additive valuations, and Brustle et al. \cite{BDNSV20} later improved the result in \cite{DBLP:conf/sagt/HalpernS19}.
There is a line of recent works focusing on the variants and generalizations of the standard setting.
For example, Goko et al. \cite{DBLP:journals/geb/GokoIKMSTYY24} considered the strategic behaviors when subsidy is used, and Aziz et al. \cite{DBLP:journals/corr/abs-2408-08711} considered weighted envy-freeness when the agents have different entitlements.
Brustle et al. \cite{BDNSV20}, Barman et al. \cite{DBLP:conf/ijcai/BarmanKNS22} and Kawase et al. \cite{DBLP:conf/aaai/KawaseMSTY24} extended the valuations from additive to general monotone, which is also the focus of the current paper.
Choo et al. \cite{DBLP:journals/orl/ChooLSTZ24} and Dai et al. \cite{DBLP:journals/corr/abs-2408-12523} respectively studied the envy-free house allocation with subsidies and its weighted variant. 
Wu et al. \cite{DBLP:conf/wine/WuZZ23, DBLP:journals/corr/abs-2404-07707} extended the subsidy model to the fair allocation of indivisible chores.