% This must be in the first 5 lines to tell arXiv to use pdfLaTeX, which is strongly recommended.
\pdfoutput=1
% In particular, the hyperref package requires pdfLaTeX in order to break URLs across lines.
\PassOptionsToPackage{prologue,dvipsnames}{xcolor}
\documentclass[11pt]{article}
\usepackage[table]{xcolor}
% Change "review" to "final" to generate the final (sometimes called camera-ready) version.
% Change to "preprint" to generate a non-anonymous version with page numbers.
\usepackage[preprint]{acl}

% Standard package includes
\usepackage{times}
\usepackage{latexsym}

% For proper rendering and hyphenation of words containing Latin characters (including in bib files)
\usepackage[T1]{fontenc}
% For Vietnamese characters
% \usepackage[T5]{fontenc}
% See https://www.latex-project.org/help/documentation/encguide.pdf for other character sets

% This assumes your files are encoded as UTF8
\usepackage[utf8]{inputenc}

% This is not strictly necessary, and may be commented out,
% but it will improve the layout of the manuscript,
% and will typically save some space.
\usepackage{microtype}

% This is also not strictly necessary, and may be commented out.
% However, it will improve the aesthetics of text in
% the typewriter font.
\usepackage{inconsolata}

%Including images in your LaTeX document requires adding
%additional package(s)
\usepackage{graphicx}

% If the title and author information does not fit in the area allocated, uncomment the following
%
%\setlength\titlebox{<dim>}
%
% and set <dim> to something 5cm or larger.

% ======== Customized packages
\usepackage{algorithm}
\usepackage[noend]{algpseudocode}
\usepackage{amsfonts}
\usepackage{amsmath}
\usepackage{booktabs}
\usepackage{caption}
\usepackage{cleveref}
\usepackage{enumitem}
\usepackage{float}
\usepackage{graphicx}
\usepackage{hyperref}
\usepackage{makecell}
\usepackage{multirow}
\usepackage{natbib}

\usepackage{wrapfig}
% \usepackage{xcolor}
\usepackage{pifont}
% Standard package includes
\usepackage{times}
\usepackage{tcolorbox}
\usepackage{latexsym}
\definecolor{cvprblue}{rgb}{0.21,0.49,0.74}

\usepackage{tikz}
\usepackage{xcolor}

\definecolor{pastelblue}{RGB}{135,180,225}    % LLaVA-v1.5-7B (柔和但更饱和的蓝)
\definecolor{pastelgreen}{RGB}{120,200,150}    % w. POVID (淡雅但更鲜明的绿)
\definecolor{pastelred}{RGB}{240,130,140}      % w. CSR (3Iter) (降低亮度,提高对比)
\definecolor{pastelpurple}{RGB}{180,160,210}    % w. SIMA (更饱和的紫色)
\definecolor{pastelorange}{RGB}{255,180,90}     % w. \(\ralign{}\) (更柔和的暖橙色)
\definecolor{pastelyellow}{RGB}{250,220,120}    % LLaVA-v1.6-Mistral-7B (降低亮度,提高对比度)
\definecolor{pastelteal}{RGB}{100,190,190}    % w. STIC (更鲜明的青色)


\newcommand{\ralign}{{\textsc{Re-Align}}}
\newcommand\semiLarge{\@setfontsize\semiLarge{13.15}{15.18}}
\newcommand{\realignlogo}{\raisebox{-0.5em}
{\includegraphics[height=2.5em]{pics/logo-realign.png}}}

\title{\realignlogo \ \textsc{Re-Align}: Aligning Vision Language Models via Retrieval-Augmented Direct Preference Optimization}

\author{
Shuo Xing$^1$\thanks{\ Email: \texttt{\{shuoxing,tzz\}@tamu.edu}},\quad
Yuping Wang$^2$,\quad
Peiran Li$^1$,\quad
Ruizheng Bai$^1$,\quad 
Yueqi Wang$^3$,\quad \\
\bf Chengxuan Qian$^1$,
\quad
Huaxiu Yao$^4$,\quad
Zhengzhong Tu$^1$\footnotemark[1]\thanks{\  Corresponding author.}\\
\\
$^1$Texas A\&M University \quad
$^2$University of Michigan \quad
$^3$UIUC \quad 
$^4$UNC Chapel Hill \\
}
\begin{document}
\maketitle
\begin{abstract}
The emergence of large Vision Language Models (VLMs) has broadened the scope and capabilities of single-modal Large Language Models (LLMs) by integrating visual modalities, thereby unlocking transformative cross-modal applications in a variety of real-world scenarios.
%
Despite their impressive performance, VLMs are prone to significant hallucinations, particularly in the form of cross-modal inconsistencies. 
%
Building on the success of Reinforcement Learning from Human Feedback (RLHF) in aligning LLMs, recent advancements have focused on applying direct preference optimization (DPO) on carefully curated datasets to mitigate these issues. 
%
Yet, such approaches typically introduce preference signals in a brute-force manner, neglecting the crucial role of visual information in the alignment process. 
%
In this paper, we introduce \ralign{}, a novel alignment framework that leverages image retrieval to construct a dual-preference dataset, effectively incorporating both textual and visual preference signals.
%
We further introduce rDPO, an extension of the standard direct preference optimization that incorporates an additional visual preference objective during fine-tuning. 
%
Our experimental results demonstrate that \ralign{} not only mitigates hallucinations more effectively than previous methods but also yields significant performance gains in general visual question-answering (VQA) tasks. Moreover, we show that \ralign{} maintains robustness and scalability across a wide range of VLM sizes and architectures.
This work represents a significant step forward in aligning multimodal LLMs, paving the way for more reliable and effective cross-modal applications. We release all the code in \url{https://github.com/taco-group/Re-Align}.
% \textcolor{red}{Add and emph rDPO as a keyword here.} Solved
\end{abstract}

\begin{figure}[htbp]
    \centering
    \includegraphics[width=1.\linewidth]{pics/radar-benchmark.png}
    \caption{Benchmark performance comparison (min-max normalized).}
    \label{fig:radar}   
\end{figure}

\section{Introduction}
The recent emergence of powerful Vision Language Models (VLMs)~\citep{li2022blip, li2023blip2, liu2024llava,llavanext,llama3.2, Qwen-VL, Qwen2VL, lu2024deepseek,wu2024deepseek} has significantly extended the capabilities of Large Language Models (LLMs)~\citep{devlin2018bert, radford2019gpt2,brown2020gpt3,team2023gemini,roziere2023codellama,touvron2023llama,touvron2023llama2, raffel2020t5,qwen2,qwen2.5} into the visual domain, paving the way for innovative real-world applications that integrate multimodal information~\citep{moor2023med,li2024llava-med, shao2024lmdrive,openemma,rana2023sayplan,kim2024openvla}. Despite their promising performance, VLMs remain susceptible to hallucinations—instances where the model produces outputs containing inaccurate or fabricated details about objects, attributes, and the logical relationships inherent in the input image~\citep{rohrbach2018object,bai2024hallucination}. Several factors contribute to this cross-modal inconsistencies, including the separate low-quality or biased training data, imbalanced model architectures, and the disjoint pretraining of the vision encoder and LLM-backbone~\citep{cui2023holistic,bai2024hallucination,zhou2024povid}. 

To mitigate the hallucinations in VLMs, the Directed Preference Optimization (DPO) techniques have been widely adopted~\citep{deng2024stic,zhou2024povid,fang2024vila,zhou2024calibrated,guo2024direct,chen2024dress,wang2024enhancing,yu2024rlhf,li2023silkie,wang2024mdpo}. This involves constructing datasets enriched with human preference signals specifically targeting hallucinations, and then finetuning the models using algorithms like Direct Preference Optimization (DPO)~\citep{rafailov2024direct}. Existing methods generating the preference data by perturbing the ground truth responses~\citep{zhou2024povid} and corrupting the visual inputs/embeddings~\citep{deng2024stic, amirloo2024understanding} to generate rejected responses or correcting/refining responses to produce chosen responses~\citep{chen2024dress,yu2023reformulating}. While methods based on response refinement yield the most reliable preference signals, they face scalability challenges due to the significant costs of manual correction processes. Conversely, directly corrupting input visual information or ground truth responses is overly simplistic, as this brute-force approach fails to generate plausible and natural hallucinations in a controlled manner. Moreover, during fine-tuning, directly applying DPO may cause the model to overly prioritize language-specific preferences, which potentially leads to suboptimal performance and an increased propensity for hallucinations~\citep{wang2024mdpo}.

In this paper, we propose \textbf{\ralign{}}, a novel framework that alleviates VLM hallucinations by integrating {image retrieval} with direct preference optimization (DPO). Our method deliberately injects controlled hallucinations into chosen responses using image retrieval, generating rejected responses that offer more plausible and natural preference signals regarding hallucinations. Additionally, by incorporating both the retrieved image and the original input image, \ralign{} constructs a {dual preference dataset}. This dataset is then leveraged to finetune VLMs with our proposed \textbf{rDPO} objective—an extension of DPO that includes an additional visual preference optimization objective, further enhancing the alignment process with valuable {visual preference signals}.


\section{Preliminaries}
To mitigate hallucinations in VLMs, we introduce an alignment framework based on direct preference optimization (DPO) with image retrieval. In this section, we present preliminary definitions and notations for VLMs and preference optimization, which serve as the foundation for our proposed framework. 

\paragraph{Vision Language Models} 
VLMs typically consist of three main components: a vision encoder $f_v(\cdot)$, a projector $f_p(\cdot)$, and an LLM backbone $\mathcal{L}(\cdot)$. Given a multimodal input query $(x,v)$, where $x$ is a textual instruction and $v$ is a visual image, VLMs generate a corresponding response $y = [y_1, \cdots, y_m]$ autoregressively. Here, each $y_i$ represents an output token, and $m$ denotes the total number of tokens in the generated response.

\paragraph{Direct Preference Learning} Reinforcement Learning from Human Feedback (RLHF) \citep{christiano2017deep, ziegler2019fine} is a key approach for aligning machine learning models with human preferences. Among these techniques, the Direct Preference Optimization (DPO) algorithm~\citep{rafailov2024direct} stands out for its popularity and for demonstrating superior alignment performance. We represent a VLM with a policy $\pi$, which, given an input query $(x,v)$, generates a response $y$ from the distribution $\pi(\cdot|x,v)$. We denote by $\pi_0$ the initial VLM model, fine-tuned on instruction-following VQA data by supervised fine-tuning (SFT). Specifically, we define a preference dataset $\mathcal{D} = \{(x, v, y_w, y_l)\}$, where for each input, the response $y_w$ is preferred to the response $y_l$. The DPO objective is formulated as follows, leveraging the preference dataset $\mathcal{D}$: 
\begin{align*}
    &\mathcal{L}_\text{DPO} = - \mathbb{E}_{(x,v,y_w,y_l) \sim \mathcal{D}}\\
    &\bigg[\log \sigma \bigg(\beta \log \frac{\pi_\theta (y_w|x,v)}{\pi_0 (y_w|x,v)}
    - \beta \log \frac{\pi_\theta (y_l|x,v)}{\pi_0 (y_l|x,v)} \bigg) \bigg].
\end{align*}

Compared to deep RL-based methods like Proximal Policy Optimization (PPO)~\citep{schulman2017proximal, christiano2017deep, ziegler2019fine}, DPO is more computationally efficient, easier to tune, and thus more widely adopted~\citep{dong2024rlhf}. 

\paragraph{Image Retrieval} Image retrieval aims to find relevant images from large databases -- such as vector databases or indexed corpora -- based on semantic similarity criteria. In this paper, we convert all images into vector representations and utilize the cosine similarity metric to evaluate their proximity to a reference image. The similarity between two images, $v_1$ and $v_2$, is computed as follows:
\begin{align*}
    s = \bigg< \frac{f_p(v_1)}{||f_p(v_1)||}, \frac{f_p(v_2)}{||f_p(v_2)||} \bigg>,
\end{align*}
where $<\cdot, \cdot>$ denotes the inner product in $l_2$ space, $f_p(v_i)$ represents the image embeddings generated by the vision encoder $f_v(\cdot)$ of VLMs. In this paper, we employ the FAISS library~\citep{douze2024faiss,johnson2019billion} for efficient vector searches, retrieving the top-$k$ most relevant images.

\section{Methods}
In this paper, we propose \ralign{}, a novel framework that integrates preference optimization with image retrieval to improve cross-modal alignment in VLMs. 
\begin{figure}[htbp]
    \centering
    \includegraphics[width=1.\linewidth]{pics/harp-pipeline.pdf}
    \caption{Illustration of \ralign{} framework.}
    \label{fig:harp-pipeline}   
\end{figure}

As shown in Figure \ref{fig:harp-pipeline}, the process begins with an advanced VLM generating chosen responses from input images from the training set. A selective masking process is then applied, strategically omitting segments associated with objects, attributes, or logical relationships identified in the image. Next, leveraging the retrieved image from the same training dataset and the masked responses, the hallucination-prone VLM is prompted to complete the masked elements, obtaining rejected responses. The generated preference pairs (chosen vs. rejected) are then used to fine-tune the VLM with $\mathcal{L}_{\text{rDPO}}$ (\cref{eq:rdpo}), a preference objective that integrates both visual and textual information to penalize hallucinations and reinforce grounded reasoning. Algorithm \ref{alg:harp} provides an overview of \ralign{}, while the detailed process is explained in the following subsections.


\begin{algorithm}[t]
\caption{Overview of \ralign{} }
% \textcolor{red}{Add some short comments inline using // xxxx solved

\label{alg:harp}
\textbf{Required:}
\\ (1) Unlabeled images $\{v_i\}$ with instructions $\{x_i\}$;
\\ (2) an advanced VLM model $\mathcal{V}$;
\\ (3) caption masking prompt $P_m$;
\\ (4) masked caption completion prompt $P_c$;
\\ (5) a text encoder $\mathcal{T}$.

\textbf{Input:} A reference model $\pi_0$ with vision encoder $f_v(\cdot)$, VLM $\pi_\theta$, hyper-parameter $k, \tau$.
\begin{algorithmic}[1]
    \State $\mathcal{D} \gets \emptyset$  \textcolor{cvprblue}{// Init preference dataset}
    \State $N \gets |\{v_i\}|$
    \For{$i= 1, \cdots, N$}
    \State $y_w \gets \mathcal{V}(x_i, v_i)$ \textcolor{cvprblue}{// Get preferred response}
    \State $y_m \gets \mathcal{V}(P_m, x_i, v_i)$ \textcolor{cvprblue}{// Strategic masking}
    \State $s^j_i = \text{sim} (f_v(v_i), f_v(v_j)), \forall i\neq j$
    \State \textcolor{cvprblue}{// Retrieve top-$k$ similar images}
    \State $s^{j_1}_i, \cdots, s^{j_k}_i \gets \text{Top}_k(s_i^j)$
    \State $y_l \gets \text{None}, v_l \gets \text{None}$
    \For{$t= 1, \cdots, k$} 
    \State \textcolor{cvprblue}{// Generate candidate hallucinations}
    \State $y_c \gets \mathcal{V}(P_c, y_m, v_{j_t})$
    \If{sim$(\mathcal{T}(y_w), \mathcal{T}(y_c)) \geq \tau$}
    \State \textcolor{cvprblue}{// Assign rejected response}
    \State $y_l \gets y_c, v_l \gets v_{j_k}$
    \EndIf
    \EndFor
    \If{$y_l$ is None}
    \State \textbf{continue}
    \EndIf
    \State $\mathcal{D} \gets \mathcal{D} \cup \{x_i, v_i, v_l, y_w, y_l\}$
    \EndFor
    \State Update $\pi_\theta$ through $\mathcal{L}_{\text{rDPO}}$ (\cref{eq:rdpo})
    \State \Return $\pi_\theta$
\end{algorithmic}
\end{algorithm}
% \vspace{-5mm}
\subsection{Preference Generation}
\begin{figure*}[htbp]
    \centering
    \includegraphics[width=0.95\linewidth]{pics/gen-reject.pdf}
    \caption{Illustration of the preference generation process, utilizing the original vision encoder from initial VLMs and the SentenceTransformer as the text encoder. }
    \label{fig:gen-reject}   
\end{figure*}
Generating high-quality preference data, which includes both accurate ground-truth responses and controlled hallucinated examples, is crucial for effective preference optimization in pre-trained VLMs.
Existing methods construct preference data by perturbing ground-truth responses~\citep{zhou2024povid}, corrupting visual inputs/embeddings~\citep{deng2024stic, amirloo2024understanding} to create rejected responses, or refining responses to obtain chosen responses~\citep{chen2024dress,yu2023reformulating}. Refinement produces high quality preference data but comes at a high cost, whereas direct corruption is more scalable yet tends to generate unrealistic hallucinations and fails to produce plausible, natural ones in a controlled manner. To address these limitations, we introduce a novel image retrieval-based pipeline for preference data construction as shown in Figure \ref{fig:gen-reject}, which consists of three key stages:
% \vspace{-2mm}
\begin{itemize}[leftmargin=1em,noitemsep]
    \item \textbf{Strategical masking:} Given an input pair $(x_i,v_i)$ and its corresponding chosen response $y_w$ generated by a pretrained VLM, a strategic masking process removes words or segments associated with objects, attributes, or logical relationships inferred from the image, producing the masked response $y_m$.
    \item \textbf{Image retrieval:} All images $\{v_i\}$ in the training set are embedded using the original vision encoder of the pre-trained VLMs, forming the knowledge base $\mathcal{K}$. The top-$k$ most similar images to $v_i$ are then retrieved from $\mathcal{K}$ using a cosine similarity search.
    \item \textbf{Inducing hallucinations:} VLMs are prompted to generate a candidate completion $y_m$ for the masked response conditioned on the instruction $x$ and a retrieved image $v_{j_t}$ where $t \in [1,k]$ denotes the rank of images based on their cosine similarity to the input $v_i$. Both the chosen response $y_w$ and the reconstructed response $y_c$ are embedded using a $\text{SentenceTransformer}$ model. If the cosine similarity between these embeddings falls below $0.95$, $y_c$ is designated as the rejected response $y_l$. Otherwise, the process continues with the next image $v_{j_{t+1}}$ in the similarity-ranked sequence until a suitable candidate is identified or all $k$ retrieved images have been examined. 

\end{itemize}

\subsection{Preference Optimization}
The curated preference dataset is subsequently used to fine-tune VLMs through direct preference learning. We propose retrieval-augmented direct preference optimization (rDPO), an extension of DPO that integrates an additional visual preference optimization objective. Given a preference dataset $\mathcal{D} = \{x,v,v_l,y_w,y_l\}$, the retrieval-augmented direct preference optimization objective is formulated as follows:
\begin{align*}
    &\mathcal{L}_{\text{vDPO}} = - \mathbb{E}_{(x,v,v_l,y_w,y_l) \sim \mathcal{D}}\\
    &\bigg[\log \sigma \bigg(\beta \log \frac{\pi_\theta (y_w|x,v)}{\pi_0 (y_w|x,v)}
    - \beta \log \frac{\pi_\theta (y_w|x,v_l)}{\pi_0 (y_w|x,v_l)} \bigg) \bigg],
\end{align*}
where $(x,v)$ denotes the input query of VLMs, $(y_w,y_l)$ represents the preference responses pair, and $v_l$ is the retrieved image for $v$. The loss function of rDPO is the combination of standard DPO objective and visual preference optimization:
\begin{align}\label{eq:rdpo}
    \mathcal{L}_{\text{rDPO}} = \mathcal{L}_{\text{DPO}} + \mathcal{L}_{\text{vDPO}}.
\end{align}
By incorporating both textual and visual preference signals, our approach allows VLMs to effectively exploit multimodal information during optimization, in contrast to prior alignment methods that depend exclusively on language-based preferences. In contrast to mDPO~\citep{wang2024mdpo}, which introduces image preference by randomly cropping the original input images, rDPO adopts retrieval-augmented generation to integrate visual preference signals in a more coherent and semantically meaningful way.

\begin{table*}[htbp]
  \footnotesize
  \setlength{\tabcolsep}{5pt}
  \begin{center}
    \begin{tabular}{lcccccccccccccccccccr}
      \toprule

      Methods
      & POPE$^r$
      & POPE$^p$
      & POPE$^a$
      & Hallusion$^q$
      & Hallusion$^f$
      & Hallusion$^{Easy}$
      & Hallusion$^{Hard}$
      & Hallusion$^a$
      \\
      \midrule

      LLaVA-v1.5-7B
      & 88.14 
      & 87.23
      & 85.10
      & 10.3297
      & 18.2081
      & 41.7582
      & 40.2326
      & 46.3242
      \\

      \textbf{w.} POVID
      & 88.21 
      & 87.16
      & 85.06
      & 10.5495
      & 18.2081
      & 41.5385
      & 40.9302
      & 46.6785
      \\

      \textbf{w.} CSR (3Iter)
      & 87.83
      & 87.00
      & 85.00
      & 10.1099 
      & 18.2081 
      & 41.7582
      & 40.6977
      & 46.9442 
      \\

      \textbf{w.} SIMA
      & 88.10
      & 87.10
      & 85.03
      & 10.9890 
      & 17.6301
      & 43.0549
      & 40.2326
      & 45.2728
      \\

      \rowcolor[gray]{0.9} \textbf{w.} \ralign{}
      & \textbf{88.65}
      & \textbf{87.43}
      & \textbf{85.16}
      & \textbf{11.2088}
      & \textbf{18.7861}
      & \textbf{45.5165}
      & \textbf{41.6279}
      & \textbf{47.6156}
      \\

      \midrule

      \makecell[l]{LLaVA-v1.6-\\ \quad \quad Mistral-7B} 
      & 88.83
      & 87.93
      & 86.43 
      & 13.6264 
      & 19.0751 
      & 47.4725 
      & 33.4884 
      & 46.0585 
      \\

      \textbf{w.} STIC
      & 89.03
      & 88.20
      & 86.56 
      & 12.9670
      & 17.3410 
      & 47.2527 
      & 34.1860 
      & 46.3242 
      \\

      \rowcolor[gray]{0.9} \textbf{w.} \ralign{}
      & \textbf{90.55} 
      & \textbf{89.20}
      & \textbf{87.03}
      & \textbf{13.8462} 
      & \textbf{19.0751}
      & \textbf{48.3516}
      & \textbf{34.8837} 
      & \textbf{46.5899} 
      \\
        
      
      \bottomrule
    \end{tabular}
  \end{center}
  \caption{Impact of \ralign{} across hallucination benchmarks for VLMs, and comparisons with baselines. }
  \label{tab:hallucination-task}
\end{table*}


\begin{table*}[htbp]
  \footnotesize
  \setlength{\tabcolsep}{5pt}
  \begin{center}
    \begin{tabular}{lccccccccc}
      \toprule

      Methods
      & SQA
      & TextVQA
      & MM-Vet
      & VisWiz
      & LLaVABench
      & MME$^P$
      & MME$^C$
      & MMBench
      & Avg. Rank
      \\
      \midrule

      LLaVA-v1.5-7B
      & \underline{66.02}
      & 58.18
      & 31.6
      & 50.03
      & 64.1
      & 1510.28
      & 357.85
      & \underline{64.60}
      & 3.375
      \\

      \textbf{w.} POVID
      & 65.98
      & 58.18
      & 31.8
      & 49.80
      & 67.3
      & 1495.91
      & 356.07
      & 64.34
      & 3.625
      \\

      \textbf{w.} CSR (3Iter)
      & 65.46
      & 57.86
      & 31.6
      & 47.02 
      & \textbf{68.3}
      & \textbf{1525.44}
      & 365.35
      & 64.08
      & 3.625
      \\

      \textbf{w.} SIMA
      & 65.83
      & \underline{58.48}
      & \underline{32.0} 
      & \underline{50.04}
      & 66.9
      & 1510.33
      & \textbf{371.78} 
      & 64.60 
      & \underline{2.5}
      \\

      \rowcolor[gray]{0.9} \textbf{w.} \ralign{}
      & \textbf{68.10}
      & \textbf{58.55}
      & \textbf{32.1}
      & \textbf{50.06}
      & \underline{67.7}
      & \underline{1511.79}
      & \underline{367.50}
      & \textbf{64.69}
      & \textbf{1.375}
      \\

      \midrule

      \makecell[l]{LLaVA-v1.6-\\ \quad \quad Mistral-7B}  
      & 76.02
      & \underline{63.80}
      & \underline{47.6}
      & \textbf{59.85}
      & 80.2
      & 1494.22
      & \textbf{323.92}
      & \underline{69.33}
      & \underline{2.125}
      \\

      \textbf{w.} STIC
      & \underline{76.42}
      & 63.50
      & 47.3
      & 54.21
      & \underline{81.0}
      & \underline{1504.91} 
      & 308.21 
      & 69.16 
      & 2.625
      \\

      \rowcolor[gray]{0.9} \textbf{w.} \ralign{}
      & \textbf{76.47}
      & \textbf{64.08}
      & \textbf{48.3}
      & \underline{57.27} 
      & \textbf{81.8} 
      & \textbf{1512.09} 
      & \underline{318.93} 
      & \textbf{69.42}
      & \textbf{1.25}
      \\
        
      
      \bottomrule
    \end{tabular}
  \end{center}
  \caption{Impact of \ralign{} across general benchmarks for VLMs, and comparisons with baselines. }
  \label{tab:general-task}
\end{table*}

\section{Experiments}
We conduct three categories of experiments to empirically validate the effectiveness of our proposed method. First, we evaluate the ability of \ralign{} to mitigate hallucinations and improve generalizability across diverse VQA tasks, demonstrating its consistent superiority over baseline approaches and achieving state-of-the-art performance. Next, we examine \ralign{}'s effectiveness in aligning VLMs across various model sizes and architectures, including both text-to-image and unified models, where it delivers substantial performance over vanilla models and existing baselines. Finally, we assess the impact of our proposed rDPO objective in preference optimization, showing that it consistently surpasses standard DPO in aligning VLMs and achieving superior results in both hallucination mitigation and general tasks.


\subsection{\ralign{} for VLMs Alignment}
\paragraph{Datasets} We conducted experiments on both hallucination detection and general VQA tasks. Specifically, we assess our method’s performance in hallucination detection using the POPE dataset~\citep{pope} and HallusionBench~\citep{guan2023hallusionbench}. For general VQA tasks, we leverage a diverse suite of benchmarks including ScienceQA~\citep{lu2022scienqa}, TextVQA~\citep{singh2019towards}, MM-Vet~\citep{yu2023mm}, VisWiz~\citep{gurari2018vizwiz}, LLaVABench~\citep{llavabench}, MME~\citep{Fu2023mme}, and MMBench~\citep{liu2024mmbench}.

\paragraph{Beslines} We compare our method with several widely adopted alignment frameworks for VLMs, including \textbf{POVID}~\cite{zhou2024povid}, \textbf{CSR}~\citep{zhou2024calibrated}, \textbf{SIMA}~\citep{wang2024enhancing}, \textbf{STIC}~\citep{deng2024stic}. For more details on these baselines, please refer to the Appendix.


\begin{table}[htbp]
  \footnotesize
  \begin{center}
    \begin{tabular}{llllllllllllllll}
      \toprule

      Methods
      & POPE$^r$
      & POPE$^p$
      & POPE$^a$
      \\
      \midrule

      Janus-Pro-1B
      & 85.46
      & 85.03
      & 84.13 
      \\

      \rowcolor[gray]{0.9} \textbf{w.} \ralign{}
      & 87.53$_{\uparrow2.07}$ 
      & 87.33$_{\uparrow2.30}$  
      & 85.86$_{\uparrow1.73}$ 
      \\

      \midrule

      Janus-Pro-7B
      & 88.41
      & 87.30 
      & 85.70 
      \\

      \rowcolor[gray]{0.9} \textbf{w.} \ralign{}
      & 89.73$_{\uparrow1.32}$ 
      & 88.37$_{\uparrow1.07}$
      & 86.27$_{\uparrow0.57}$
      \\

      \midrule

      LLaVA-v1.5-7B
      & 88.14 
      & 87.23
      & 85.10
      \\

      \textbf{w.} POVID
      & 88.21$_{\uparrow0.07}$
      & 87.16$_{\downarrow0.07}$
      & 85.06$_{\downarrow0.04}$
      \\

      \textbf{w.} CSR (3Iter)
      & 87.83$_{\downarrow0.31}$
      & 87.00$_{\downarrow0.23}$
      & 85.00$_{\downarrow0.10}$
      \\

      \textbf{w.} SIMA
      & 88.10$_{\downarrow0.04}$
      & 87.10$_{\downarrow0.13}$
      & 85.03$_{\downarrow0.07}$
      \\

      \rowcolor[gray]{0.9} \textbf{w.} \ralign{}
      & 88.65$_{\uparrow0.51}$
      & 87.43$_{\uparrow0.20}$
      & 85.16$_{\uparrow0.06}$
      \\

      \midrule

      LLaVA-v1.5-13B
      & 88.07
      & 87.53
      & 85.60
      \\

      \textbf{w.} CSR (3Iter)
      & 88.38$_{\uparrow0.31}$
      & 87.90$_{\uparrow0.37}$
      & 85.46$_{\downarrow0.14}$
      \\

      \textbf{w.} SIMA
      & 88.04$_{\downarrow0.03}$
      & 87.40$_{\downarrow0.13}$
      & 85.40$_{\downarrow0.20}$
      \\

      \rowcolor[gray]{0.9} \textbf{w.} \ralign{}
      & 90.03$_{\uparrow1.96}$
      & 89.20$_{\uparrow1.30}$
      & 86.20$_{\uparrow0.74}$
      \\

      \midrule

      \makecell[l]{LLaVA-v1.6-\\ \quad \quad Vicuna-7B}
      & 88.52
      & 87.63
      & 86.36
      \\

      \rowcolor[gray]{0.9} \textbf{w.} \ralign{}
      & 88.94$_{\uparrow0.42}$ 
      & 88.03$_{\uparrow0.40}$ 
      & 86.63$_{\uparrow0.27}$ 
      \\

      \midrule


      \makecell[l]{LLaVA-v1.6-\\ \quad \quad Vicuna-13B}
      & 88.24
      & 87.70
      & 86.43
      \\

      \rowcolor[gray]{0.9} \textbf{w.} \ralign{}
      & 88.79$_{\uparrow0.55}$ 
      & 88.10$_{\uparrow0.40}$ 
      & 86.60$_{\uparrow0.17}$ 
      \\
      
      \bottomrule
    \end{tabular}
  \end{center}
  \caption{Impact of \ralign{} across various model scales on POPE.}
  \label{tab:scale-pope}
\end{table}


\begin{table*}[htbp]
  \footnotesize
  \setlength{\tabcolsep}{5pt}
  \begin{center}
    \begin{tabular}{lcccccccccccccccccccr}
      \toprule

      Methods
      & SQA
      & TextVQA
      & MM-Vet
      & MME$^P$
      & MME$^C$
      & MMBench
      & POPE$^r$
      & POPE$^p$
      & POPE$^a$
      \\
      \midrule

      LLaVA-v1.5-7B
      & 66.02
      & 58.18
      & 31.6
      & 1510.28
      & 357.85
      & 64.60
      & 88.14 
      & 87.23
      & 85.10
      \\

      \textbf{w.} \ralign{} (DPO)
      & 66.26
      & 58.24
      & 30.9
      & 1506.49
      & 357.85
      & 64.52
      & 88.18
      & 87.30
      & 85.23
      \\

      
      \rowcolor[gray]{0.9} \textbf{w.} \ralign{} (rDPO)
      & 68.10
      & 58.55
      & 32.1
      & 1511.79
      & 367.50
      & 64.69
      & 88.65
      & 87.43
      & 85.16
      \\

      \midrule

      LLaVA-v1.6-Mistral-7B 
      & 76.02
      & 63.80
      & 47.6
      & 1494.22
      & 323.92
      & 69.33
      & 88.83
      & 87.93
      & 86.43 
      \\

      \textbf{w.} \ralign{} (DPO)
      & 76.07
      & 63.88
      & 46.8
      & 1505.85
      & 316.79 
      & 69.24
      & 88.93
      & 88.03
      & 86.47
      \\


      \rowcolor[gray]{0.9} \textbf{w.} \ralign{} (rDPO)
      & 76.47
      & 64.08
      & 48.3
      & 1512.09 
      & 318.93 
      & 69.42
      & 90.55 
      & 89.20
      & 87.03 
      \\
        
      
      \bottomrule
    \end{tabular}
  \end{center}
  \caption{Impact of rDPO across general and hallucination benchmarks for VLMs, and comparisons with baselines.}
  \label{tab:rdpo-effect}
\end{table*}

\paragraph{Experimental Setup} We sample 11k images from the LLaVA-Instruct-150K dataset~\citep{liu2024llava} to construct preference data, as illustrated in Figure \ref{fig:gen-reject}. These images are initially used to generate QA piars based on image captions and simple VQA tasks using GPT-4o mini~\citep{gpt4omini}. Furthermore, the images are encoded using \texttt{clip-vit-large-patch14}~\citep{radford2021learning} to construct the knowledge base for image retrieval. For rejected responses, we use GPT-4o mini to mask the chosen response and \texttt{all-mpnet-base-v2}~\citep{reimers-2019-sentence-bert} to compute the similarity between the completed masked response and the original chosen response. We use LLaVA-v1.5-7B~\citep{liu2024llava} and LLaVA-v1.6-Mistral-7B~\citep{llavanext} as our backbone models and perform \ralign{} finetuning for $1$ epoch. All evaluations are conducted with a temperature setting of 0, and baseline results are reproduced using the publicly available model weights.
% \section{Ablation Study}

\paragraph{Results} Table \ref{tab:hallucination-task} shows the performance of \ralign{} compared to baseline methods on hallucination benchmarks. Notably, \ralign{} achieves the best among the evaluated methods on both POPE and HallusionBench for LLaVA-v1.5-7B~\citep{liu2024llava} and LLaVA-v1.6-Mistral-7B~\citep{llavanext}, highlighting the effectiveness of our approach in mitigating hallucinations of VLMs. As demonstrated in Table \ref{tab:general-task}, \ralign{} can provide generally on-par or better performance than the vanilla models and baseline alignment methods on each evaluated general VQA task, ultimately achieving the best overall results. This finding indicates that \ralign{} can enhance hallucination mitigation without compromising general performance.

\subsection{Scalability and Generalizability} 
\paragraph{Experimental Setup} The experimental setup follows the same setting as VLMs alignment experiments, except for the backbone models, where we employ a diverse array of VLMs varying in size and architecture:
\begin{itemize}
    \item \textbf{Image-to-Text models}: the typical architecture of VLMs, where a vision encoder is integrated with an LLM to enable cross-modal understanding. In this section, we evaluate \ralign{} on LLaVA-v1.5-7B~\citep{liu2024llava}, LLaVA-v1.5-13B~\citep{liu2024llava}, LLaVA-v1.6-Vicuna-7B~\citep{llavanext}, and LLaVA-v1.6-Vicuna-13B~\citep{llavanext}.
    
    \item \textbf{Unified Models}: encoder-decoder architecture that decouples visual encoding for multimodal understanding and generation. In this section, we evaluate \ralign{} on Janus-Pro-1B~\citep{chen2025janus} and Janus-Pro-7B~\citep{chen2025janus}.
\end{itemize}
     
\paragraph{Results}
Table \ref{tab:scale-pope} presents the performance of \ralign{} using both standard image-to-text and unified VLM backbones across model sizes from 1B to 13B on the POPE benchmark~\citep{pope}. In experiments with the LLaVA-v1.5 series~\citep{liu2024llava}, none of the baseline approaches consistently improve performance for either the 7B or the 13B models, highlighting the limited scalability of these methods. In contrast, \ralign{} achieved substantial performance gains, outperforming both the baseline models and the vanilla version—most notably on the LLaVA-v1.5-13B variant. Similarly, experiments with the LLaVA-v1.6-Vicuna series~\citep{llavanext} revealed the same trend, further underscoring \ralign{}'s superior scalability.
For unified vision-language models, especially Janus-Pro, integrating \ralign{} yields a significant performance boost. Notably, Janus-Pro-1B experiences the greatest improvement, underscoring \ralign{}’s robustness across different model architectures. However, Janus-Pro-1B, being the smallest among the evaluated VLMs, also exhibits the poorest overall performance on POPE, suggesting a correlation between model size and the propensity for hallucinations.

\subsection{Effects of rDPO}
\paragraph{Dataset} Due to budget constraints and the need for reproducibility, we have excluded benchmarks that require evaluation by GPT-4~\citep{gpt4}. Instead, we focus on the following tasks: ScienceQA~\citep{lu2022scienqa}, TextVQA~\citep{singh2019towards}, MM-Vet~\citep{yu2023mm}, VisWiz~\citep{gurari2018vizwiz}, LLaVABench~\citep{llavabench}, MME~\citep{Fu2023mme}, MMBench~\citep{liu2024mmbench}, and POPE~\citep{pope}.

\paragraph{Experimental Setup} The experimental setup follows the same setting as VLMs alignment experiments, with the exception of the direct optimization objectives. To further explore the impact of our proposed rDPO, we conduct experiments on the same constructed preference dataset using the standard DPO~\citep{rafailov2024direct} during the one-epoch finetuning process.

\paragraph{Results} Table \ref{tab:rdpo-effect} summarizes the performance of \ralign{} when using both standard DPO and rDPO as the direct optimization objectives, evaluated on general VQA and hallucination tasks with LLaVA-v1.5-7B~\citep{liu2024llava} and LLaVA-v1.6-Mistral-7B~\citep{llavanext} as backbones. The results indicate that employing rDPO as the finetuning objective consistently yields superior performance over standard DPO across both task categories, highlighting the benefits of incorporating visual preference signals during the alignment process for VLMs. Notably, even when solely employing DPO, 
\ralign{} not only achieves performance gains over the vanilla models but also outperforms the baselines evaluated in the VLM alignment experiments on several tasks. This underscores the effectiveness of our image retrieval-based preference data construction.

\section{Discussions} 

\paragraph{Discussion with mDPO} In this section, we detail the differences between our proposed rDPO and mDPO~\citep{wang2024mdpo}. In mDPO, a conditional preference optimization  objective is introduced to force the model to determine the preference label based on visual information:
\begin{align*}
    &\mathcal{L}_{\text{CoDPO}} = - \mathbb{E}_{(x,v,y_w,y_l) \sim \mathcal{D}}\\
    &\bigg[\log \sigma \bigg(\beta \log \frac{\pi_\theta (y_w|x,v)}{\pi_0 (y_w|x,v)}
    - \beta \log \frac{\pi_\theta (y_w|x,v_c)}{\pi_0 (y_w|x,v_c)} \bigg) \bigg],
\end{align*}
where $v_c$ denotes a randomly cropped image of the original input image $v$. Specifically, visual preference signals are generated by randomly masking $20\%$ of the input visual tokens to encourage the model to capture preferences based on visual cues.

In contrast, \ralign{} extends and enhances this approach by incorporating a more semantically meaningful visual preference pair. Instead of relying solely on random crops, \ralign{} retrieves a relevant image from the same dataset that corresponds to the original input. This retrieval-based augmentation provides a stronger contrastive signal, improving the model’s ability to discern fine-grained visual details and reducing spurious correlations. Moreover, beyond mitigating hallucinations in VLMs, \ralign{} has been demonstrated that it also significantly enhances performance on general VQA tasks.

\paragraph{Segment-level Preference} 
\begin{figure}[htbp]
    \centering
    \includegraphics[width=.7\linewidth]{pics/sqa-scale.png}
    \caption{Performance gains of \ralign{} with LLaVA-v1.6-Mistral-7B  as the backbone on ScienceQA with respect to the size of preference data.}
    \label{fig:sqa-size}   
\end{figure}
\vspace{-2mm}
Building on the findings of \cite{yu2024rlhf}, we generate preference data by inducing hallucinations at the segment level than at the sentence level (as seen in approaches such as POVID~\citep{zhou2024povid}, STIC~\citep{deng2024stic}, and CSR~\citep{zhou2024calibrated}), to provide robust supervision signals during the alignment process. This finer-grained preference modeling yields clearer and more precise learning signals, enabling the model to better distinguish between subtle hallucinations and ground truth responses. To further investigate these segment-level preference signals, we expanded the finetuning dataset from $11k$ to $16k$ image samples. As illustrated in Figure \ref{fig:sqa-size}, when using LLaVA-v1.6-Mistral-7B as the backbone with ScienceQA as the case study, \ralign{} achieved a significant performance improvement—from $0.45$ to $1.34$—demonstrating the effectiveness of our approach.

\section{Related Work}


\paragraph{Reinforcement Learning from Human Feedback} 
Reinforcement Learning from Human Feedback (RLHF) has emerged as a crucial technique for incorporating human preference signals into machine learning methods and models~\citep{dong2024rlhf}. RLHF frameworks can be broadly categorized into deep RL-based approaches and direct preference learning approaches. In deep RL-based methods, a reward model is first constructed, after which Proximal Policy Optimization (PPO)~\citep{schulman2017proximal, christiano2017deep, ziegler2019fine} is employed to optimize the reward signals with KL regularization~\citep{ouyang2022training, touvron2023llama2}. While the direct preference learning approaches optimize a designed loss target on the offline preference dataset directly, eliminating the need for a separate reward model\citep{rafailov2024direct,ipo,gpo,ethayarajh2024kto}.

\paragraph{Vision Language Models} 
Large Vision Language Models (VLMs)~\citep{li2022blip, li2023blip2, liu2024llava,llavanext,llama3.2, Qwen-VL, Qwen2VL, lu2024deepseek, wu2024deepseek} extended the understanding and reasoning capabilities of Large Language Models (LLMs)~\citep{devlin2018bert, radford2019gpt2,brown2020gpt3,team2023gemini,roziere2023codellama,touvron2023llama,touvron2023llama2, raffel2020t5,qwen2,qwen2.5} into the visual domain. By integrating vision encoders, such as CLIP~\citep{radford2021clip}, image patches are first converted into embeddings and then projected to align with text embedding space, unlocking unprecedented cross-modal applications in the real world, such as biomedical imaging~\citep{moor2023med,li2024llava-med}, autonomous systems~\citep{shao2024lmdrive,tian2024drivevlm,sima2023drivelm,openemma}, and robotics~\citep{rana2023sayplan,kim2024openvla}.

\paragraph{Alignment of Vision Language Models}
Current VLMs often suffer from hallucinations, producing inaccurate or misleading information that fails to accurately represent the content of the provided image~\citep{zhu2024unraveling,bai2024hallucination}. Such misalignments can have catastrophic consequences when these models are deployed in real-world scenarios~\citep{autotrust}.
To address cross-modality hallucinations, recent research has primarily focused on applying direct preference optimization~\citep{deng2024stic,zhou2024povid,fang2024vila,zhou2024calibrated,guo2024direct,chen2024dress,wang2024enhancing,yu2024rlhf,li2023silkie,wang2024mdpo} or contrastive learning~\citep{sarkar2024mitigating} on the curated datasets with preference signals, and utilizing model editing techniques~\citep{liu2024paying,yu2024attention}.


\section{Conclusion}
In this paper, a novel framework, \ralign{}, for aligning VLMs to mitigate hallucinations is proposed. Our approach leverages image retrieval to deliberately induce segment-level hallucinations, thereby generating plausible and natural preference signals in a controlled manner. By integrating the retrieved images, a dual-preference dataset that encompasses both textual and visual cues is curated. Furthermore, we propose the rDPO objective, an extension of DPO that includes an additional visual preference optimization objective, to enhance the alignment process with valuable visual preference signals. Comprehensive empirical results from a range of general VQA and hallucination benchmarks demonstrate that \ralign{} effectively reduces hallucinations in VLMs while enhancing their overall performance. Moreover, it demonstrates superior scalability across various model architectures and sizes.

\section*{Limitations} 
Although \ralign{} has demonstrated superior performance on both hallucination and general VQA benchmarks, it does not always achieve state-of-the-art results on general tasks; in some cases, its performance is even worse than that of vanilla VLMs. Future research could explore strategies to eliminate this alignment tax or or identify an optimal balance for this trade-off.

The potential risks of this work align with the general challenges of RLHF alignment. As more powerful alignment techniques are developed, they may inadvertently empower adversarial approaches that exploit these models, potentially leading to unfair or discriminatory outputs. Meanwhile, these adversarial strategies can be used to generate negative samples, which can ultimately contribute to the development of more robust and reliable VLMs over time.

\bibliography{main}

\appendix
% \clearpage

\section{Details of the Evaluated Baselines} We compare our proposed method with following alignment frameworks for VLMs:

\begin{itemize}
    \item \textbf{POVID}~\cite{zhou2024povid}: constructing preference data by prompting GPT-4V~\citep{gpt4v} to generate hallucinations while intentionally injecting noise into image inputs, followed by fine-tuning VLMs using DPO.

    \item \textbf{CSR}~\citep{zhou2024calibrated}: iteratively generates candidate responses and curates preference data using a self-rewarding mechanism, followed by fine-tuning VLMs via DPO. 

    \item \textbf{SIMA}~\citep{wang2024enhancing}: self-generates responses and employs an in-context self-critic mechanism to select response pairs for preference data construction, followed by fine-tuning with DPO.

    \item \textbf{STIC}~\citep{deng2024stic}: self-generates chosen responses and constructs preference data by introducing corrupted images or misleading prompts, followed by fine-tuning with regularized DPO. 
\end{itemize}


\section{Prompts used for Preference Data Construction}

During the construction of the preference dataset for \ralign{}, we employed GPT-4o mini~\citep{gpt4omini} to mask the chosen response using the following prompt.

\begin{tcolorbox}[colback=gray!5!white, colframe=gray!75!black, 
title=Strategic Masking]
        Please mask any words of the segments related to the objects, attributes, and logical relationships of the input image in the following description by replacing them with [MASK].\\
\end{tcolorbox}

Then, we instruct the VLMs to produce a candidate completion for the masked response to generate the final rejected response using the following prompt.

\begin{tcolorbox}[colback=gray!5!white, colframe=gray!75!black, 
title=Masking Completion]
        Please complete the following sentence based on the input image by filling in the masked segments.\\
\end{tcolorbox}

\section{Examples of Preference Pair}

Table \ref{fig:vqa-case} and \ref{fig:iamge-cap-case} provide examples of the constructed preference data for the VQA and image captioning, and each data sample contains textual instruction, input image, retrieved image, chosen response, and rejected response.

\begin{figure}[htbp]
    \centering
    \includegraphics[width=1.\linewidth]{pics/vqa_case.png}
    \caption{Example preference pair for VQA generated using \ralign{}.}
    \label{fig:vqa-case} 
\end{figure}

\begin{figure*}[htbp]
    \centering
    \includegraphics[width=0.9\linewidth]{pics/image-cap.png}
    \caption{Example preference pair for image captioning generated using \ralign{}.}
    \label{fig:iamge-cap-case} 
\end{figure*}
\begin{figure*}[h]
    \centering
    \includegraphics[width=0.9\linewidth]{pics/response-example.png}
    \caption{Example responses generated by LLaVA-v1.5-7B and \ralign{}.}
    \label{fig:response-case} 
\end{figure*}

\section{Response Examples} 
Figure \ref{fig:response-case} presents example responses from both the original LLaVA-v1.5-7B model and \ralign{} as evaluated on LLaVABench. Notably, the original model's response exhibits server object hallucinations, while \ralign{} delivers a clearer and more accurate description of the image.



\section{Licenses}
\label{sec:licenses}

The LLaVA-Instruct-150K dataset~\citep{liu2024llava} which is used to construct preference data is released under CC BY 4.0 license and it should abide by the policy of OpenAI\footnote{https://openai.com/policies/terms-of-use}.

For the hallucination benchmarks, POPE~\citep{pope} and HallusionBench~\citep{guan2023hallusionbench} are released under MIT and BSD-3-Clause licenses. 

For the general VQA benchmarks, ScienceQA~\citep{lu2022scienqa}, TextVQA~\citep{singh2019towards}, MM-Vet~\citep{yu2023mm}, VisWiz~\citep{gurari2018vizwiz}, LLaVABench~\citep{llavabench}, and MMBench~\citep{liu2024mmbench} are released under MIT, CC BY 4.0, Apache-2.0, CC BY 4.0, Apache-2.0, and Apache-2.0 licenses respectively. While MME~\citep{Fu2023mme} was released without an accompanying license.

\section{Experimental Cost}

The cost for curation the preference dataset by using GPT-4o mini~\citep{gpt4omini} cost approximately \$90 in total.The evaluation of HallusionBench and LLaVABench using GPT-4~\citep{gpt4} incurred an approximate total cost of \$30.

\section{Computational Cost} 
All finetuning and evaluation experiments were executed on four NVIDIA A6000ada GPUs. Table \ref{tab:time} details the time required for \ralign{} to fine-tune each model.

\begin{table}[htbp]
  \footnotesize
  \begin{center}
    \begin{tabular}{lc}
      \toprule

      Models
      & Required Time
      \\
      \midrule

      Janus-Pro-1B
      & 50 min
      
      \\

      Janus-Pro-7B
      & 93 min
      
      \\

      LLaVA-v1.5-7B
      & 35 min
     
      \\

      LLaVA-v1.5-13B
      & 45 min
      
      \\

      LLaVA-v1.6-Mistral-7B
      & 30 min
      \\
      
      \makecell[l]{LLaVA-v1.6-Vicuna-7B}
      & 46 min
      
      \\


      \makecell[l]{LLaVA-v1.6- Vicuna-13B}
      & 72 min
      \\
      
      \bottomrule
    \end{tabular}
  \end{center}
  \caption{Time required for finetuning VLMs with \ralign{}.}
  \label{tab:time}
\end{table}


\section{Hyperparameter Setting} 

For all the experiments, we finetuning VLMs with \ralign{} for 1 epoch. We deploy LoRA finetuning with \texttt{lora\_r}=128, \texttt{lora\_alpha}=256, \texttt{target\_module}=all, and hyperparameters as presented in Table \ref{tab:hypeterparameter}. 

\begin{table}[htbp]
  \footnotesize
  \begin{center}
    \begin{tabular}{ll}
      \toprule

      Hyperparameter
      & Setting 
      \\
      \midrule

      $\beta$
      & 0.1
      \\

      Learning rate
      & 1e-5
      \\

      \texttt{weight\_decay}
      & 0.0
      \\

      \texttt{warmup\_ratio}
      & 0.03
      \\

      \texttt{lr\_scheduler\_type}
      & \texttt{cosine}
      \\

      \texttt{mm\_projector\_lr}
      & 2e-5
      \\

      \texttt{mm\_projector\_type}
      & mlp2x\_gelu
      \\

      \texttt{gradient\_accumulation\_steps}
      & 8 
      \\

      \texttt{per\_device\_train\_batch\_size}
      & 1
      \\

      \texttt{bf16}
      & True
      \\

      Optimizer
      & AdamW
      
      \\
      
      \bottomrule
    \end{tabular}
  \end{center}
  \caption{Hypeterparameter setting for finetuning.}
  \label{tab:hypeterparameter} 
\end{table}


\section{ Social Impacts}
Our proposed novel alignment framework for VLMs, \ralign{}, not only significantly mitigates the hallucinations of VLMs but also elevates their generalization capabilities across diverse multimodal tasks. These advancements hold far-reaching societal implications, particularly in advancing the development of trustworthy, ethically aligned AI systems capable of reliable real-world deployment. To elucidate these implications, we provide a comprehensive overview of potential transformative outcomes:
\begin{itemize}
    \item \textbf{Enhancing trustworthiness:} \ralign{} significantly enhances the reliability of AI-generated content by reducing hallucinated outputs and improving factual grounding. This ensures that users and regulatory bodies can place increased confidence in AI-driven decisions and recommendations.

    \item \textbf{Safety-critical applications:} By reducing erratic outputs and improving contextual awareness, \ralign{} enables safer deployment of VLMs in high-stakes domains such as healthcare diagnostics, autonomous vehicles, and disaster response systems, where error margins are near-zero and algorithmic trust is paramount.

    \item \textbf{Democratizing access to robust AI:} Our method can democratize access to advanced mutimodal AI models under low-resource or data-scarce settings, which empowers researchers and practitioners with limited computational resources to participate in cutting-edge AI development, ultimately contributing to a more equitable and diverse AI ecosystem.
\end{itemize}
\end{document}
