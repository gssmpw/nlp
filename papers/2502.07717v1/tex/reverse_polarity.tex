\subsubsection{Reversing Polarity}
\label{sec:reversing}

We define rules to add and remove negation cues from sentences.
These rules are used to create the negative pairs $(S_1, S_2')$ in the NSP task.
To streamline the process,
we only work with sentences that
\begin{compactitem}
    \item include \emph{not}, \emph{n't}, or \emph{never} as negation cues;
    \item the negation cue modifies the main verb;
    \item are not questions; and
    \item contain exactly one negation cue.
\end{compactitem}
To develop the rules, 
we collected a large set of sentences from the English Wikipedia corpus~\cite{wikidump} that met these criteria.
We then generated the dependency tree for each sentence with spaCy~\cite{spacy2, honnibal-johnson-2015-improved} 
and analyzed the frequency of outgoing edges from the main verb. 
Afterward, we manually inspected the most frequent tokens associated with each edge 
and leveraged these patterns to develop the \hl{rules below}.

\hl{
    We evaluated these rules by manually inspecting 100 samples.
    In 96\% of them, 
    the rules correctly reverse polarity. 
    Note that the goal here is not 100\% correctness---it is to automatically generate data for pre-training with our tasks.
} \tbd{Is em dash (---) correct here?}

\paragraph{Adding negation.}
For sentences where the main verb has no auxiliary verb,
we insert the negation cue directly and adjust the verb for tense and subject agreement. 
The cue \emph{never} is always placed directly before the main verb.
We append \emph{n't} and \emph{not} directly after the main verb if it is one of the following:
\emph{were}, \emph{was}, \emph{is}, \emph{are}, \emph{do}, \emph{will}, \emph{would}, \emph{may}, \emph{might}, \emph{shall}, \emph{should}, \emph{can}, \emph{could}, or \emph{must}.
For example, given the sentence \emph{``I was shopping.''},
we add \emph{not} to create the sentence \emph{``I was not shopping.''}.

If the main verb is a gerund or present participle, we do not add \emph{n't} directly to it; 
instead, we place \emph{not} right before the verb. 
For present or past participles,
we replace it with its lemma and insert the appropriate form of \emph{do} before the lemma, 
ensuring it matches the tense of the verb and person of the subject.
For present participles, we add \emph{do} or \emph{does}, and for past participles, we add \emph{did}.
We then insert \emph{not} or \emph{n't} after the auxiliary verb.
For example, given the sentence \emph{“I went to the store,”} 
the main verb \emph{went} is replaced with \emph{did not go}, resulting in \emph{“I did not go to the store.”}

If the main verb has an outgoing edge labeled \emph{aux} or \emph{auxpass} in the dependency tree,
we add the negation cue to the auxiliary verb.
For example, given the sentence \emph{``The store is closed,''}
we add \emph{n't} to the auxiliary verb \emph{is} to create the sentence \emph{``The store isn't closed.''}
\hl{
    However, for certain auxiliary verbs such as \emph{might} and \emph{may},
    it is not possible to add \emph{n't} directly to them.
    In such cases, we only add \emph{not} or \emph{never} to the sentences.}
    Appendix~\ref{app:reversing} \hl{lists the auxiliary verbs we work with and the rules for adding each negation cue.

    Additionally, to have more natural sentences with negation cues,
    we replace modifiers such as \emph{already} and \emph{some} with \emph{yet} and \emph{any}, respectively.
}





\paragraph{Removing negation.}
We begin by removing the negation cue from the sentence and adjusting the grammar accordingly.
If the negation cue is \emph{n't} (as in \emph{can't} or \emph{won't}), 
we remove \emph{n't} and replace the auxiliary verb with its lemma (e.g., \emph{can} and \emph{will}).

{Next, we remove any extra auxiliary verbs and adjust the main verb based on tense and subject agreement. 
If the auxiliary verb is \emph{did}, we remove \emph{did} and use the past tense form of the main verb.
For example, given the sentence \emph{``I did not go to the store,''} we remove \emph{did} and update \emph{go} to \emph{went}, 
resulting in \emph{``I went to the store''}. 
We apply the same process for \emph{do} and \emph{does}. 
That is, we replace the main verb with its base form or third-person singular form, respectively.
}

We also replace negative polarity items such as \emph{yet}, \emph{at all}, and \emph{any}
with their affirmative counterparts (\emph{already}, \emph{somewhat}, and \emph{some}, respectively.)
Lastly, 
if \emph{but} functions as a conjunction and is a sibling of the main verb in the dependency tree, we replace it with \emph{and}.

\paragraph{A note on using LLMs.}
Although using LLMs is expensive and time-consuming, 
we attempted to use state-of-the-art LLMs to reverse the polarity of sentences.
We used the Llama-2 model \cite{touvron2023llama} and the GPT-4 model \cite{openai2024gpt4}. 
We tried several prompting approaches 
to instruct the models to only \hl{add or remove} negation cues \hl{without modifying} other parts of the sentence.
However, the models \hl{consistently made additional modifications to keep the meaning of the sentence intact. } 
\hl{
    We hypothesize that this is because we work with Wikipedia sentences, which are typically about facts.
    Since these models are believed to be trained to be truthful, 
    they often refuse to generate text that contradicts real-world facts.
}
See examples of the prompts and outputs in Appendix~\ref{app:llm-reversing}.
