% This must be in the first 5 lines to tell arXiv to use pdfLaTeX, which is strongly recommended.
\pdfoutput=1
% In particular, the hyperref package requires pdfLaTeX in order to break URLs across lines.

\documentclass[11pt]{article}
% \usepackage[table]{xcolor} % Include the xcolor package

% Remove the "review" option to generate the final version.
\usepackage[]{style/acl}

% Standard package includes
\usepackage{times}
\usepackage{latexsym}
\usepackage{multirow}
% For proper rendering and hyphenation of words containing Latin characters (including in bib files)
\usepackage[T1]{fontenc}
% For Vietnamese characters
% \usepackage[T5]{fontenc}
% See https://www.latex-project.org/help/documentation/encguide.pdf for other character sets

\usepackage{amsmath}
\usepackage{verbatim}
\usepackage{paralist}
% This assumes your files are encoded as UTF8
\usepackage[utf8]{inputenc}

% This is not strictly necessary, and may be commented out.
% However, it will improve the layout of the manuscript,
% and will typically save some space.
\usepackage{microtype}

% This is also not strictly necessary, and may be commented out.
% However, it will improve the aesthetics of text in
% the typewriter font.
\usepackage{inconsolata}
\usepackage{soul}

\usepackage{booktabs}
\usepackage{graphicx}
\usepackage[export]{adjustbox}
\usepackage{soul}
\usepackage{xcolor}

% \newcommand{\hlg}[1]{\sethlcolor{green}\hl{#1}\sethlcolor{yellow}}
\newcommand{\hlg}[1]{#1}
\renewcommand{\hl}[1]{#1}

\usepackage{soulpos} % Extended version of `soul` for nested commands
\usepackage{etoolbox}
\newcommand{\role}[1]{\textsc{#1}}
% \newcommand{\tbd}[1]{\marginpar{\footnotesize#1}}
\newcommand{\tbd}[1]{}

\usepackage{colortbl}
\usepackage{pgfplots}

\usepackage{tabularx}
\usepackage{icomma}
% If the title and author information does not fit in the area allocated, uncomment the following
%
%\setlength\titlebox{<dim>}
%
% and set <dim> to something 5cm or larger.

\title{
  Making Language Models Robust Against Negation
}

\author{MohammadHossein Rezaei \and Eduardo Blanco \\
         Department of Computer Science, University of Arizona \\ \texttt{\{mhrezaei,eduardoblanco\}@arizona.edu}}

% \pagecolor{yellow!20}
\begin{document}
\maketitle
\begin{abstract}
Negation has been a long-standing challenge for language models.
Previous studies have shown that they struggle with negation in many natural language understanding tasks.
In this work, we propose a self-supervised method to make language models more robust against negation.
We introduce a novel task, Next Sentence Polarity Prediction (NSPP), and a variation of the Next Sentence Prediction (NSP) task.
We show that BERT and RoBERTa further pre-trained on our tasks outperform the off-the-shelf versions on nine negation-related benchmarks.
Most notably, our pre-training tasks yield between 1.8\% and 9.1\% improvement on CondaQA, a large question-answering corpus requiring reasoning over negation.
\end{abstract}


\documentclass[../main.tex]{subfiles}
\graphicspath{{../images/}}
\makeatletter
\def\input@path{{../images/}}
\makeatother
\begin{document}
\section{Introduction}
\begin{figure}
\centering
\begin{tikzpicture}
\node[inner sep=0pt] (ws) at (0, 0) {
\includegraphics[height=.4\textwidth, trim={10cm 0 10cm 0},clip]{world_space.png}};
\node[inner sep=0pt] (cs) at (6,0) {\includegraphics[height=.4\textwidth, trim={10cm 1cm 10cm 4cm},clip]{conf_space.png}};
\end{tikzpicture}
\vspace{-5pt}
\label{fig:pbrm_intro}
\caption{\textbf{Left}: Shows world space obstacles as grey spheres. Robots start and goal configuration is colored red and green, respectively. Configurations along the computed path are colored transparent blue. \textbf{Right:} Mapped world space scenario to configuration space. Obstacle region is the grey mesh. Red spheres are collision-free regions computed by the neural SCDF. The optimized shortest path in the convex corridor is the blue curve.}
\vspace{-25pt}
\end{figure}
Motion planning is the problem of finding a collision-free trajectory that connects a given start and goal configuration. The planning takes place in the configuration space of the robot. For single body robots, like mobile robots or drones, the configuration space and the world space are usually the same. This simplifies the planning, since explicit obstacle representations are available which enables geometrical tools like separating hyperplanes, smallest distance to obstacles etc., to be used when designing motion planning algorithms. For multi-body robots like manipulators, the situation is completely different. The world space obstacles are usually mapped to non-convex regions, and to make the problem even harder, the mapping is usually not known. Forming explicit representations of the obstacle region in the configuration space is usually too expensive or intractable. Despite all of this, sampling based planners are used with great success, which mainly is due to their use of implicit representations of the obstacle region. The basic idea is to construct a graph in the configuration space that covers and connects the collision-free region. From this graph, a path can be extracted that connects a given start and goal configuration. The approach is computationally expensive, since the graph is constructed with the smallest geometrical building block available, points, which represents a collision-check. Furthermore, the extracted paths from the graph are non-smooth and jagged due to the stochastic nature of the approach. This adds an additional post-processing step to the process, where the paths are shortcutted and smoothened, before the path can be used for tracking. Clearly a lot of time is invested to form this graph and produce smooth paths. Thus, if the obstacles start to move, then all of this work is done in no use, since all points that make up this graph need to be re-verified, which is simply too time consuming to be done in real time.
\\\\
In this work, we want to address the existing drawbacks of the sampling based planners. Our main contribution is an improved motion planner where each vertex in the graph covers a collision-free region in the form of a sphere instead of a point and where the edges are formed with neighboring intersecting spheres. This representation has the advantage of instead of returning piecewise linear paths, returning a sequence of overlapping spheres, i.e. a convex corridor, that connects a given start and goal configuration, illustrated in Figure \ref{fig:pbrm_intro}. This convex corridor allows us to use convex optimization to produce smooth trajectories, instead of computationally expensive post-processing methods. The representation further allows us to estimate the coverage of the collision-free space, which gives us awareness and feedback in the offline roadmap construction phase. Finally, our representation is simple to adapt to moving obstacles, simply requery for the new radii and recheck for intersections. 
\\\\
The spherical collision-free regions are formed using a signed distance function (SDF), which is a function that returns the smallest distance from an arbitrary point to the boundary of an obstacle. As the name implies, the distance is signed, thus if the point is inside the obstacle it is negative otherwise positive. If the distance is positive, a sphere with radius equal to the distance is guaranteed to cover a collision-free region. Using an SDF in motion planning is not new, but what is novel about our approach is that we express the distance in the configuration space instead of the world space and by doing so allows us to form these convex collision-free regions. We refer to the resulting SDF as a signed configuration distance function (SCDF). Computing an SCDF analytically is non-trivial, our approach is therefore to parameterize the SCDF with a deep neural network and learn the mapping by supervised learning. Our resulting neural SCDF can compute distances for different parameter values of obstacle shapes and we also show how multiple distances can be combined, thus making our approach flexible.
\section{Related work}
Motion planning algorithms can roughly be divided into three families, grid-based, sampling based and optimization based methods. Grid-based methods (GBM) discretize the planning space from which a graph is then compiled. A standard search method is A$^\star$ \citep{a_star}, which is classified as an \textit{informed} search method, since it employs a heuristic function to speed up the search. A$^\star$ guarantees to return an optimal path at the level of discretization used. GBMs usually discretize the planning space by a regular lattice and this limits the GBMs to problems with low dimensionality due to the curse of dimensionality. Thus, GBMs are usually limited to single-body robots where the degrees of freedom (DOF) are low. To overcome the inherent scaling problem with the GBMs, stochastic methods are usually used for multi-body robots. These methods are termed as sampling-based methods (SBM) and core members within this family are the rapidly-exploring random trees (RRT) \citep{rrt} and the probabilistic roadmap (PRM) \citep{prm}. RRT grows a tree from the start configuration and explores the collision-free region in a rapid way until it is able to connect to the goal region. RRT is usually improved by bi-directional planning \citep{rrt_connect}, i.e. an additional tree is grown from the goal configuration and the trees are tested for connection after any tree has been expanded. RRT is a single-query method, thus it searches for a path from scratch each time it is queried. Contrary to this, PRM is a multi-query method, which solves for multiple queries without starting from scratch. PRM does this by creating a roadmap (graph) that covers the collision-free space as an offline step. The graph is then used to solve for multiple queries. PRMs are used in cases where the environment does not change since the extra offline step is too computationally costly and needs to be re-done if the environment is changed. In our work, we address this inherent issue by using a different roadmap representation. Our vertices in the graph cover a collision-free region in the form of spheres and we form the edges by checking for intersecting spheres. If something in the environment changes, we recompute the spheres radii and recheck the intersections, without relying on collision detection. We use a trained neural network to compute the sphere radius, therefore querying for the radius can be done fast, hence our representation enables the PRM for dynamic environments.
\\\\
In the recent decades, optimization based methods (OBM) \citep{chomp, schulman, itomp, stomp} have been introduced as an alternative to SBM for multi-body robots. Like the SBM, the OBMs scale well to higher dimensional problems and produce smoother motion. It is common to use a SDF in the optimization since it is a smooth function, thus enabling gradient-based methods. However, the standard way of expressing the SDF is in world space. The distance therefore needs to be mapped to the configuration space by the forward kinematics. This mapping makes the optimization problem a non-linear program (NLP), which is computationally expensive to solve. Recently, a different approach has been proposed. In \cite{mp_gcs} motion planning is formulated as a convex optimization problem by using the graph of convex sets framework \citep{gcs}. The underlying idea is to decompose the collision-free space into intersecting convex sets from which a convex optimization problem is formulated. In cases where an explicit representation of the obstacles in the configuration space exists, like for single-body robots, creating collision-free convex regions can be done fast \citep{iris}. For multi-body robots, this is non-trivial. Existing work does this successfully \citep{iris_nlp, iris_c} by an optimization based approach, but the methods are still too time consuming to be used in the presence of moving obstacles. Our approach is instead to use deep learning to learn an SDF expressed in the configuration space. With this, we can query for shortest distances to the collision boundary, which allows us to expand spherical regions which are collision-free. Our approach is fast and therefore enables our suggested roadmap planner to be used in dynamic environments.
\\\\
Recent research has focused on learning collision detection \citep{fk_kernel_distance, diffco, graphdistnet} by predicting the signed distance between the robot links and the surrounding obstacles in the world space. The learned SDF is used in trajectory optimization but since the distance is expressed in the world space, the problem becomes an NLP and therefore takes a long time to solve. We take a novel approach and suggest to instead express the signed distance in the configuration space. This allows us to improve the PRM at the same time as it enables convex optimization for trajectory optimization, which runs faster and is more reliable than NLP solvers. In \cite{cspf} a learned signed distance function in the configuration space is proposed similar to our approach. However, their approach is restricted to point cloud representations, while we propose to represent the obstacles as parameterized geometric shapes, e.g. spheres. Furthermore, we also show how to use our learned SCDF to improve an existing roadmap planner.
\section{Problem formulation}
A robot is located in the world space, $\W \subset \R^3 $. The unique location of the robot is given by its configuration $\q \in \C$, where $\C$ is the configuration space. The set of points covered by the robots bodies at a certain configuration is expressed as $\B(\q) \subset \W$. The robot is surrounded by $\NrObst$ obstacles $\O = \bigcup_{i=1}^{\NrObst} \O_i$, where  $\O_i \subset \W$. The representation of the obstacle in the configuration space is the set $\C\O_i = \{\q \in \C \: |\: \B(\q) \cap \O_i \neq \emptyset \}$. The obstacle space is formed as $\Co = \bigcup_{i=1}^{\NrObst} \C \O_i$. The complement is referred to as the free space, $\Cf = \C \setminus \Co$. The path planning problem is a tuple, ($\Cf$, $\qStart$, $\qGoal$), where we want to connect a query pair, consisting of a start, $\qStart$, and goal configuration, $\qGoal$, with a geometric path, $\q(s): [0, 1] \mapsto \Cf$, such that $\q(0)=\qStart$ and $\q(1)=\qGoal$, or report correctly when such a path does not exist.
\end{document}

\section{Related Work}
% \subsection{Vision Language Model}
% 시각장애인에서 상황을 설명할 DB가 없으니 만들었다. 그리고 이를 VLM에 튜닝했다.
\subsection{Technical approaches for assisting the visually-impaired}


\subsection{Datasets for visual instruction tuning}

\section{Empowering Language Models Against Negation}
\label{sec:empowering}

We propose a self-supervised method to make LMs more robust against negation.
Our approach is to further pre-train LMs on two tasks that involve negation.
These tasks are the Next Sentence Polarity Prediction (NSPP) task 
and a variation of the well-known Next Sentence Prediction (NSP) task.
None of these tasks require labeled data\hl{;
any text corpora are suitable.}
Also, they are not specific to any domain or downstream task.



\subsection{Next Sentence Polarity Prediction (NSPP)}
\label{sec:nspp}
We introduce NSPP as the task of predicting the polarity of the next sentence given the current sentence.
\hlg{Given} a pair of consecutive sentences, $(S_1, S_2)$, 
the input to the model is only $S_1$, 
and the output is a binary label indicating whether $S_2$ \hlg{includes} any negation cues or not.
For example, consider the following pair of sentences:
\begin{compactitem}
  \item[] $S_1$: \hl{\textit{The weather report showed sunny skies.}}% \textit{``I like apples.''}
  \item[] $S_2$: \hl{\textit{But it didn’t stay that way.}}% \textit{``I don't like oranges.''}
\end{compactitem}
Given only $S_1$,
the model should predict that the following sentence \hlg{includes} negation cues.



\subsection{Next Sentence Prediction (NSP)}
\label{sec:nsp}
NSP is a well-known task in LM pre-training as first introduced by BERT~\cite{devlin-etal-2019-bert}.
The NSP task is to predict whether two sentences are consecutive.
\citet{devlin-etal-2019-bert} (a) used consecutive sentences from Wikipedia as positive examples 
and (b) chose a random sentence from the same article to replace the second sentence \hl{and} create a negative example.

We propose a variation of the NSP task to improve negation understanding.
For a pair of consecutive sentences, $(S_1, S_2)$,
we create the negative pair $(S_1, S_2')$ 
where $S_2'$ is obtained by reversing the polarity of $S_2$. 
That is, if $S_2$ includes negation cues, we remove them, 
and vice versa. 
\subsubsection{Reversing Polarity}
\label{sec:reversing}

We define rules to add and remove negation cues from sentences.
These rules are used to create the negative pairs $(S_1, S_2')$ in the NSP task.
To streamline the process,
we only work with sentences that
\begin{compactitem}
    \item include \emph{not}, \emph{n't}, or \emph{never} as negation cues;
    \item the negation cue modifies the main verb;
    \item are not questions; and
    \item contain exactly one negation cue.
\end{compactitem}
To develop the rules, 
we collected a large set of sentences from the English Wikipedia corpus~\cite{wikidump} that met these criteria.
We then generated the dependency tree for each sentence with spaCy~\cite{spacy2, honnibal-johnson-2015-improved} 
and analyzed the frequency of outgoing edges from the main verb. 
Afterward, we manually inspected the most frequent tokens associated with each edge 
and leveraged these patterns to develop the \hl{rules below}.

\hl{
    We evaluated these rules by manually inspecting 100 samples.
    In 96\% of them, 
    the rules correctly reverse polarity. 
    Note that the goal here is not 100\% correctness---it is to automatically generate data for pre-training with our tasks.
} \tbd{Is em dash (---) correct here?}

\paragraph{Adding negation.}
For sentences where the main verb has no auxiliary verb,
we insert the negation cue directly and adjust the verb for tense and subject agreement. 
The cue \emph{never} is always placed directly before the main verb.
We append \emph{n't} and \emph{not} directly after the main verb if it is one of the following:
\emph{were}, \emph{was}, \emph{is}, \emph{are}, \emph{do}, \emph{will}, \emph{would}, \emph{may}, \emph{might}, \emph{shall}, \emph{should}, \emph{can}, \emph{could}, or \emph{must}.
For example, given the sentence \emph{``I was shopping.''},
we add \emph{not} to create the sentence \emph{``I was not shopping.''}.

If the main verb is a gerund or present participle, we do not add \emph{n't} directly to it; 
instead, we place \emph{not} right before the verb. 
For present or past participles,
we replace it with its lemma and insert the appropriate form of \emph{do} before the lemma, 
ensuring it matches the tense of the verb and person of the subject.
For present participles, we add \emph{do} or \emph{does}, and for past participles, we add \emph{did}.
We then insert \emph{not} or \emph{n't} after the auxiliary verb.
For example, given the sentence \emph{“I went to the store,”} 
the main verb \emph{went} is replaced with \emph{did not go}, resulting in \emph{“I did not go to the store.”}

If the main verb has an outgoing edge labeled \emph{aux} or \emph{auxpass} in the dependency tree,
we add the negation cue to the auxiliary verb.
For example, given the sentence \emph{``The store is closed,''}
we add \emph{n't} to the auxiliary verb \emph{is} to create the sentence \emph{``The store isn't closed.''}
\hl{
    However, for certain auxiliary verbs such as \emph{might} and \emph{may},
    it is not possible to add \emph{n't} directly to them.
    In such cases, we only add \emph{not} or \emph{never} to the sentences.}
    Appendix~\ref{app:reversing} \hl{lists the auxiliary verbs we work with and the rules for adding each negation cue.

    Additionally, to have more natural sentences with negation cues,
    we replace modifiers such as \emph{already} and \emph{some} with \emph{yet} and \emph{any}, respectively.
}





\paragraph{Removing negation.}
We begin by removing the negation cue from the sentence and adjusting the grammar accordingly.
If the negation cue is \emph{n't} (as in \emph{can't} or \emph{won't}), 
we remove \emph{n't} and replace the auxiliary verb with its lemma (e.g., \emph{can} and \emph{will}).

{Next, we remove any extra auxiliary verbs and adjust the main verb based on tense and subject agreement. 
If the auxiliary verb is \emph{did}, we remove \emph{did} and use the past tense form of the main verb.
For example, given the sentence \emph{``I did not go to the store,''} we remove \emph{did} and update \emph{go} to \emph{went}, 
resulting in \emph{``I went to the store''}. 
We apply the same process for \emph{do} and \emph{does}. 
That is, we replace the main verb with its base form or third-person singular form, respectively.
}

We also replace negative polarity items such as \emph{yet}, \emph{at all}, and \emph{any}
with their affirmative counterparts (\emph{already}, \emph{somewhat}, and \emph{some}, respectively.)
Lastly, 
if \emph{but} functions as a conjunction and is a sibling of the main verb in the dependency tree, we replace it with \emph{and}.

\paragraph{A note on using LLMs.}
Although using LLMs is expensive and time-consuming, 
we attempted to use state-of-the-art LLMs to reverse the polarity of sentences.
We used the Llama-2 model \cite{touvron2023llama} and the GPT-4 model \cite{openai2024gpt4}. 
We tried several prompting approaches 
to instruct the models to only \hl{add or remove} negation cues \hl{without modifying} other parts of the sentence.
However, the models \hl{consistently made additional modifications to keep the meaning of the sentence intact. } 
\hl{
    We hypothesize that this is because we work with Wikipedia sentences, which are typically about facts.
    Since these models are believed to be trained to be truthful, 
    they often refuse to generate text that contradicts real-world facts.
}
See examples of the prompts and outputs in Appendix~\ref{app:llm-reversing}.

\section{Dataset Generation}
\label{sec:dataset}
\revise{
To train the proposed GNN, we constructed a dataset of building structures and a subset of these structures were subjected to fire simulations using FEA. The dataset generation process is illustrated in \figref{fig:dataset_generation_procedure}. Initially, a total of 33,000 building structures with geometrical details, material properties, and gravity loads were created. Due to randomness in generating these structures, a filter is applied to remove unreasonable data after gravity load simulation, which included 15,377 structures. A trade-off between computational feasibility and model performance is made among the remaining 17,623 structures. As further labeling structures with MIDR requires resource-intensive fire simulations via OpenSeesRT, a large proportion of 16,050 structures is selected as unlabeled dataset. On the other hand, each of the other 1,573 structures was further subjected to 30 different fire simulations, forming the labeled dataset containing $1,573\times 30 = 47,190$ fire cases.} This section details the step-by-step process for generating the dataset, including geometry creation, material property assignment, and simulations due to gravity loads and fire scenarios. 
% To train the proposed neural network, we constructed a dataset comprising building structure data and a subset of fire scenario data. The dataset generation process is illustrated in \figref{fig:dataset_generation_procedure}. 
% A total of 33,000 building structures with geometric details, material properties, and gravity loads were initially created. Out of these, 3,000 structures were selected as labeled data, and the remaining 30,000 were designated as unlabeled data. Further, about half of them filtered out due to instability under gravity loads only. 
\begin{figure*}[h!]
    \centering
    \includegraphics[width=0.8\linewidth]{figures/dataset_filter_procedure.pdf}
    \caption{Workflow for dataset generation (geometry, material property, gravity loads, and fire scenarios).}
    \label{fig:dataset_generation_procedure}
\end{figure*}

\subsection{Geometry Generation}
\label{subsec:geometry_generation}
The geometry of the building structures forms the foundation of the dataset. Regular 
\revise{3D structures} resembling multi-story parking structures or shopping malls were generated, with parameters such as building floor dimensions and story heights selected randomly. Each building structure is composed of multiple rooms, which serve as the basic unit in this study. A room herein is a cuboid space defined by specific length, width, and height. Within a structure, rooms of the same dimensions are uniformly arranged along the length, width, and height, corresponding to the $x$-, $y$-, and $z$-axes, respectively. Structures vary in room size and number of rooms along each axis. Specifically, the room length, width, and height are independently sampled from a uniform distribution within the interval $[2, 5]$ meters along the three directions of the structure. Similarly, the room number along each axis is uniformly sampled independently as an integer within the interval $[2, 7]$, i.e., the maximum number of stories of the buildings simulated in this study is 7.

To introduce variability and simulate real-world scenarios, approximately $8\%$ of structural elements (beams or columns) are randomly removed after initial geometry creation. 
\revise{Such removal is not fire-induced damage, but reflects functional diversity often observed in real buildings, such as open spaces designed for activities in shopping malls, e.g., ice skating rinks. Examples of the generated geometries are illustrated in \figref{fig:example_generated_geometry}, showcasing the diversity and realism of the dataset. This element removal does not affect the definition of room's geometry in the structure and nor does it affect the number of considered fire scenarios.} 

\revise{A range of coefficient of variation values ($3.3\%$ to $17.5\%$) was derived from prior studies that investigated the statistics of geometrical and material properties of structural components of buildings (e.g., \cite{mirza1979variations, lee2004probabilistic}). These studies provide empirical data on the natural variability in parameters such as Young's modulus, yield strength, and dimensions of structural elements due to manufacturing tolerances and material inconsistencies. By selecting $8\%$ for the removal of structural elements in our database, we aimed to maintain a level of variability that is representative of real-world uncertainties while ensuring computational feasibility. This choice ensures that the database captures realistic deviations without introducing extreme cases that may not be commonly encountered in practice.}

\begin{figure*}[h!]
    \centering
    \includegraphics[width=\linewidth]{figures/example_generated_geometry.pdf}
    \caption{Examples of generated structural geometry of different sizes (all dimensions in meters).}
    \label{fig:example_generated_geometry} 
\end{figure*}

{\blockRevise

In this study, we opted for a deterministic square, dimension of $0.1$ m, solid cross-sectional steel elements due to their simplicity in modeling and analysis. Square sections exhibit uniform geometrical properties in all directions, simplifying the computation of structural responses and avoiding complications associated with more complex shapes, such as wide-flange sections, facilitating the computational efficiency and scalability to generate a large dataset. This choice also helps to mitigate issues related to stress concentrations and facilitates a more straightforward representation of structural behavior under thermal loads. 

\textit{Remark:} The selected cross-section provides a comparable flexural rigidity to a $W 130 \times 130 \times 28.1$ wide-flange section (metric units), albeit with significantly higher axial rigidity. This cross-section is acceptable for gravity-load-designed frames under service loading conditions where the models assume fully rigid, moment-resisting beam-column connections for the evaluation of the IDR under thermal loading. This assumption is reasonable in this computational study where the primary interest is to understand the global deformation response of frames under fire conditions. The selection of uniform square cross-sections for both beams and columns, rather than adherence to standard capacity design principles, was made here primarily for computational efficiency and to reduce design parameters in the database generation process. This choice allows for simplified and scalable approach to analyze the fire-induced response of generic steel frames without the need for large section variations, where this study mainly focuses on the fire vulnerability assessment using ML-based predictions. However, if additional loading conditions, e.g., seismic or wind loads, were to be considered, larger sections, strong-column/weak-beam principle, and ductile detailing would be required in the generated buildings for realistic structural behavior under combined loading conditions. Future studies may also consider investigating the influence of variable cross-sectional dimensions and semi-rigid connections on the structural performance under fire conditions. 
} % blockRevise

\subsection{Material Properties}
Steel is chosen as the material for the structures. To reflect real-world variations, we randomly assign one of five slightly different steel material types to each structural element. \revise{
The ranges of material properties are provided in \tabref{tab:material_property_ranges} and the properties are sampled from uniform distributions of the corresponding ranges. These variations simulate differences arising from manufacturing batches or regional material properties. That these properties are at ambient temperature and change when the temperature rises due to a fire. The selection of materials with varying properties is aimed at increasing the diversity of the data. Our goal is to represent as wide a range of data as possible with a limited amount of building structure data, thereby enhancing the generalization ability of the GNN. Our assumed material property ranges are expected to be wider than the real-world conditions based on findings in \cite{mirza1979variations, lee2004probabilistic}. Therefore, we are essentially tackling a more challenging and general task. If we can solve this problem, we are confident that our method will perform equally well or even better in real-world scenarios.
}
\begin{table}[h!]
    \centering
    \caption{Material properties ranges for considered steel structures.}
    \begin{tabular}{lc}
        \toprule
        Property & Range \\
        \midrule
        Young's modulus & [168, 252] GPa \\
        Yield strength & [220, 330] MPa \\
        Strain-hardening ratio & [0.8, 1.2] \% \\
        \bottomrule
    \end{tabular}
    \label{tab:material_property_ranges}
\end{table}

\subsection{Gravity Loads}
Gravity loads are applied to columns and beams based on their \revise{influence (tributary) areas as typically conducted in structural analysis. The considered ``service'' load conditions include the column self-weight and the additional loads directly supported on the beams from their self-weight and weights of the reinforced concrete slabs, people as live load, and building content. An edge beam typically carries approximately half the gravity load supported by a parallel interior beam}. The ranges of gravity loads are listed in \tabref{tab:gravity_load_ranges}. \revise{The loads are sampled from uniform distributions of the corresponding ranges.} Structures that failed to meet an MIDR threshold of $1\%$ under gravity loads were deemed unacceptable designs and filtered out, as such configurations of randomly chosen geometry, material, and gravity load combinations were considered unrealistic from a regulatory and practicality points of view.
\begin{table}[h!]
    \centering
    \caption{Gravity load ranges for considered beams and columns.}
    \begin{tabular}{lc}
        \toprule
        Element & Range (kN/m)  \\
        \midrule
        Column & [0.5, 1.0]  \\
        Edge beam & [1.5, 4.5]  \\
        Interior beam & [3.0, 7.5]  \\
        \bottomrule
    \end{tabular}
    \label{tab:gravity_load_ranges}
\end{table} 

\subsection{Rule-based Thermal Load Generation}
\label{subsec:thermal_load_generation}
To evaluate a building's structural response during a fire event, we employed a simplified rule-based approach for thermal load generation. 
% Previous studies \cite{nan_structuralfire_2023} have demonstrated that steel structures rapidly equilibrate with surrounding gases temperatures due to efficient heat exchange. Consequently, gas temperatures can be directly used as inputs for FEA tools, e.g., OpenSees, simplifying the process of modeling thermal loads. 
% Accurately simulating temperature fields in fire scenarios poses significant challenges. Advanced thermodynamic simulations, such as those performed using Fire Dynamics Simulator (FDS) \cite{mcgrattan_fire_2000}, provide precise temperature predictions. However, these methods are hindered by high computational costs, prolonging execution times, and limited scalability, making them impractical for generating large datasets. Additionally, real-world fire loads often display substantial spatial variability across different rooms \cite{dundar_fire_2023}, resulting in scenario-specific temperature fields with limited generalizability. For example, studies on bridge fires \cite{he_study_2024} have demonstrated that environmental factors, such as wind speeds, can significantly influence temperature distributions. Furthermore, even within identical scenarios, variations in fire modeling methodologies can produce distinctly different temperature fields \cite{zhang_temperature_2020, du_new_2012}. These challenges emphasize the need for efficient and adaptable methods to generate fire temperature data.
% To address these issues, we adopted a rule-based approach to model temperature variations. 
According to \cite{spearpoint_fire_2008}, a typical fire development follows a predictable pattern. During the {\em{growth stage}}, the temperature rises slowly and approximately linearly after ignition. This is followed by the {\em{flashover stage}}, where temperatures increase rapidly to peak values. After reaching the peak, the temperature either stabilizes or continues to rise slowly until the {\em{decay stage}} begins. Inspired by this fire development pattern, we describe the temperature evolution in time, $t$, prior to the decay stage in two distinct stages:
\begin{enumerate}
    \item {\bf{Initial linear increase stage}}: For $t \in [0, t_1)$, temperature increases gradually and linearly as the fire spreads through the building. This stage represents the time before the fire directly affects a structural element.  
    \item {\bf{ISO 834 fire curve stage}}: For $t \in [t_1, t_{\thre}]$, temperature rises rapidly following the ISO 834 curve \cite{ISO834}, modeling the direct impact of the fire on the structural element. 
\end{enumerate}
The slope of the linear temperature increase, $c$, and the transition time, $t_1$, are influenced by the spatial relationship between the fire source and the structural element. For the second stage of temperature evolution, we utilize the ISO 834 curve, a widely accepted standard for fire resistance testing. This standardized fire curve describes the temperature rise over time, enabling rapid and consistent thermal fields across various scenarios. The duration of fire simulation in this study is set to $t_{\thre}=60$ minutes. This value represents the upper limit for the temperature evolution of each structural element, providing a consistent basis for analyzing the structural response to fire.

Let $(x, y, z)$ represents the midpoint of a structural element and $(x_{\subfire}, y_{\subfire}, z_{\subfire})$ the fire source point. \revise{Integer parameters $h$ and $h_{\subfire}$ correspond to the respective floor levels of the element and the fire source}. The temperature evolution for each element is expressed as follows:
\begin{enumerate}
    \item Linear increase stage ($0 < t < t_1$):
    \begin{equation}
    T(t) = c \cdot t,
    \end{equation}
    where $c$, the rate of temperature increase ($^\circ\mathrm{C}/\mathrm{min}$), depends on the height difference between the element, $h$, and the fire source, $h_{\subfire}$:
    \begin{equation}
        c = 
        \begin{cases} 
        5\left/\left(h - h_{\subfire} + 1\right)\right., & h \geq h_{\subfire}, \\
        2\left/\left(h_{\subfire} - h\right)\right., & h < h_{\subfire}.
        \end{cases}
    \end{equation}
     \item ISO 834 stage ($t \geq t_1$):
\begin{equation}
    T(t) = c \cdot t_1 + 345 \log_{10} \left(8 \left(t - t_1\right) + 1\right).
\end{equation}
\end{enumerate}

The transition (arrival) time $t_1$, marking the end of the linear stage, depends on the spatial distance between the fire source and the element. We define the following two Euclidean distances $L_p$ in the $xy$ plane and $L_s$ in the $xyz$ space:
\begin{eqnarray}
L_p & \triangleq & \sqrt{(x - x_{\subfire})^2 + (y - y_{\subfire})^2}, \\
\label{eq:Lp}
L_s & \triangleq & \sqrt{(x - x_{\subfire})^2 + (y - y_{\subfire})^2 + (z - z_{\subfire})^2}.
\label{eq:Ls}
\end{eqnarray}
Accordingly, the transition time, $t_1$, is expressed as follows:
\begin{equation}
    t_1 = 
    \begin{cases}
    \beta_{1} \cdot \left(1 - \exp\left\{- L_s\left/\alpha_{1}\right.\right\}\right), & h > h_{\subfire}, \\
    \beta_{2} \cdot \left(1 - \exp\left\{- L_p\left/\alpha_{2}\right.\right\}\right), & h = h_{\subfire}, \\
    \beta_{3} \cdot \left(1 - \exp\left\{- L_s\left/\alpha_{3}\right.\right\}\right), & h < h_{\subfire} .
    \end{cases}
    \label{eq:t1}
\end{equation}
The parameters $\beta_i$ and $\alpha_i$ for determining $t_1$ are summarized in Table~\ref{tab:fire_spread_parameters}. In this study, we take $r_{\mathrm{up}}=0.95$ and $r_{\mathrm{down}}=0.97$.
\begin{table}[ht]
    \centering
    \caption{Fire spread parameters for $t_1$ calculations.}
    \begin{tabular}{lcc}
        \toprule
        Case  & $\beta_i$ & $\alpha_i$  \\
        \midrule
        $i=1$, Upward spread & $16 \left.\left(1-r_{\mathrm{up}}^{\left|h-h_{\subfire}\right|}\right)\right/\left(1-r_{\mathrm{up}}\right)$ & $10$  \\
        $i=2$, Horizontal spread & $18$ & $18$  \\
        $i=3$, Downward spread & $30 \left.\left(1-r_{\mathrm{down}}^{\left|h-h_{\subfire}\right|}\right)\right/\left(1-r_{\mathrm{down}}\right)$ & $5$  \\
        \bottomrule
    \end{tabular}
    \label{tab:fire_spread_parameters}
\end{table}

\figref{fig:t1_curve} illustrates the $t_1$ curves for various fire scenarios: (1) fire originating on the lower floor, $h-h_{\subfire}=1$ with rapid upward spread, (2) fire on the same floor, $h=h_{\subfire}$ with the fastest spread, and (3) fire on the upper floor, $h_{\subfire}-h=1$ with slow downward spread. The exponential decay in $t_1$ reflects the accelerating fire propagation speed as the distance increases. \figref{fig:t1_curve} also indicates that the employed simplified model is consistent with the Markov chain-based dynamic model given by \cite{cheng_dynamic_2011}, where the rooms at the same floor of the fire point start flashover slightly before the corresponding upper floors. Additionally, $\beta_{1}$ and $\beta_{3}$ are the summation of a geometric sequence, where story level $h$ is the index. The common ratios $r_{\mathrm{up}}<1$ in $\beta_{1}$ and $r_{\mathrm{down}}<1$ in $\beta_{3}$ indicate that the fire speeds up to spread through the next story, which is consistent with the real-world fire spread mechanism given in \cite{hokugo_mechanism_2000}. The temperature profile within the range $t \in [0, t_{\thre}]$ is subsequently used as the thermal load in OpenSeesRT simulations to compute displacements at each structural node at time $t_{\thre}$.
\begin{figure}[h!]
    \centering
    \includegraphics[width=0.8\linewidth]{figures/m204_t1_curve.pdf}
    \caption{Three examples for the $t_1$ curve.}
    \label{fig:t1_curve}
\end{figure}

\revise{
\textit{Remark:} The effects of structural elements, such as concrete floor slabs and partitions, are not explicitly modeled in our approach. Instead, their influence is implicitly captured through the careful selection of the parameters $ \alpha, \beta, r_\mathrm{up} $, and $ r_\mathrm{down} $. This parameterization provides a unified framework for generating temperature fields. Indeed, fire propagation is governed by a multitude of factors and remains an open research question. For instance, if the fire resistance of a floor slab is enhanced by fire protective coating, the corresponding model can account for this by decreasing $\alpha_1$ \& $\alpha_3$, increasing $\beta_1$ \& $\beta_3$, and adopting larger values for $r_\mathrm{up}$ \& $r_\mathrm{down}$, which collectively slow down the vertical spread of fire. Conversely, scenarios involving higher amounts of combustible materials would warrant the opposite adjustments. This flexible and integrated approach avoids the need to design separate models for different fire propagation scenarios while still capturing the essential effects.
}

\revise{
In conclusion, our rule-based approach is a computationally efficient method for approximating fire temperature fields, enabling large-scale dataset generation to train predictive models. By combining ISO 834 fire curves with spatial considerations and embedding structural effects through parameter calibration, the method achieves a balanced trade-off between accuracy and scalability, making it a practical solution for thermal load modeling in fire scenarios. After generating the temperature of each beam or column according to the middle point, the temperature is applied as uniform thermal load to the elements of the structure in question using OpenSeesRT. 
}

% In conclusion, this rule-based approach is a computationally efficient method to approximate fire temperature fields, enabling large-scale dataset generation to train predictive models. By combining ISO 834 fire curves with spatial considerations, the method balances accuracy and scalability, making it a practical solution for thermal load modeling in fire scenarios.

% \subsection{Interstory Drift Ratio}
\subsection{OpenSeesRT Simulation}
\label{subsec:opensees_simulation}

The thermal and mechanical responses of 3D frame structures under combined fire and gravity loads are simulated using OpenSeesRT \cite{perez2024openseesrt}. \revise{In the simulation, the IDR of each node at $t_{\thre}$ is computed using the computed nodal displacements. Each structural model features six degrees of freedom per node (3 translational  and 3 rotational), with linear geometrical transformations (\texttt{geomTransf: Linear}) defining how the element local coordinate systems are mapped to the global coordinate system and assuming small displacements and rotations. Although OpenSeesRT allows a variety of options for modeling finite deformations, in the present simulations and mainly for simplicity, we did not consider large deformations. All bottom nodes (nodes on the ground) are fully constrained in all six degrees of freedom, while degrees of freedom os all other nodes are free.} Material behavior is temperature-dependent and modeled with \texttt{Steel01Thermal}, while fiber-based sections (\texttt{FiberThermal}) capture nonlinear interactions between thermal and mechanical responses at the cross-section level. \revise{Structural elements are represented as displacement-based Euler-Bernoulli beam-columns (\texttt{dispBeamColumnThermal}). This element  formulation accounts for thermal strains (temperature gradients) in the section, which is discretized into fibers. Numerical integration is used along the length of each element using three integration (Gauss) points, one at each end and the third in the middle of the element.}

{\revise{Thermal expansion of steel members plays a crucial role in IDR development. In reality, reinforced concrete floor slabs heat at a different rate than steel members due to their higher thermal mass and lower thermal conductivity. This differential heating can lead to restrained thermal expansion, introducing axial compression in beams and affecting the overall structural response. In this study, explicit {\em{composite action}} between steel members and concrete slabs is not modeled. Instead, our approach focuses on isolating the response of the steel structural frame, which is often the critical load-bearing component in fire scenarios. This assumption aligns with prior studies \cite{Possidente_2024} demonstrating that steel structures reach thermal equilibrium with surrounding gases quickly, allowing the use of uniform thermal loading in fire analysis. Future work could enhance this framework by incorporating slab-beam interaction effects, through a refined FEA for an extended dataset where constraints imposed by floor slabs are explicitly considered.}

The analysis begins with the application of gravity loads, followed by incremental thermal loads simulating the fire exposure. A static nonlinear solver using  \texttt{ExpressNewton} algorithm ensures convergence, while the \texttt{NormDispIncr} test maintains accuracy. An incremental \texttt{LoadControl} scheme with small step sizes is employed to guarantee numerical stability, using 10\% for gravity loads and 1\% for thermal loads. 

\revise{
In the thermal load analysis, uniform thermal load is applied to each beam or column, i.e., the temperature of each element is set to be that at the middle point, according to \secref{subsec:thermal_load_generation}. The \texttt{Steel01Thermal} material allows the properties (e.g., Young's modulus and yield strength) to be adjusted at increasing temperatures according to \cite{EN1993} using its Table 3.1: Reduction factors for the stress-strain relationship of carbon steel at elevated temperatures. For example, if the Young’s modulus at ambient temperature is $E_0$, then as the temperature ($T$) increases, the modulus changes as $E(T) = \eta (T) \times E_0$. \cite{EN1993} directly provides the values of $\eta(T) \in \left[0,1\right] $ at every $100 ^\circ\mathrm{C}$ interval and recommends using linear interpolation to obtain $\eta(T)$ for intermediate values of $T$.
} OpenSeesRT documentation \cite{OpenSeesThermalExamples} provides several examples of thermal analyses.

This modeling framework accommodates variations in material properties, cross-sectional geometries, and temperature profiles, providing robust simulations of structural behavior under fire conditions. The primary settings and configurations for the OpenSeesRT simulations are summarized in \tabref{tab:ops_detail}.
\begin{table}[h!]
    \centering
        \caption{Key settings of OpenSeesRT simulations.}
    \begin{tabular}{l|>{\raggedright\arraybackslash}p{0.6\linewidth}} %
    \toprule
    Modeling Aspect     & Details \\
    \midrule
    Geometry            & 3D models; 6 degrees of freedom per node \\
    Transformation      & geomTransf: Linear \\ 
    Material            & Steel01Thermal \\
    Section             & FiberThermal; Cross-section: $0.1$ m $\times$ $0.1$ m \\ 
    Element type        & {dispBeamColumnThermal} \\ 
    Loading             & Gravity loads: {beamUniform}; Thermal loads: {beamThermal} \\
    Integration scheme  & Incremental {LoadControl}; Step size: $10\%$ (gravity analysis), $1\%$ (thermal analysis) \\
    Nonlinear solver    & {ExpressNewton} algorithm; {UmfPack} solver; Convergence test: {NormDispIncr} tolerance: $10^{-8}$; Maximum \# iterations per step: $1000$. \\ 
    \bottomrule
    \end{tabular}
    \label{tab:ops_detail}
\end{table}

For each structure in the labeled dataset, 30 fire points are selected using a dual-granularity approach, \revise{i.e., two-stage sampling strategy,} to ensure they are well-distributed. Specifically, rooms are sequentially selected, with one fire point randomly chosen within each selected room. If a building is large and contains more than 30 rooms, we randomly select 30 rooms without replacement, i.e., ensuring that no more than one fire point is located in the same room. Conversely, if the building is small and has fewer than 30 rooms, all rooms are initially selected, with one fire point randomly assigned to each room. Additionally, rooms are then selected with replacement until a total of 30 fire points are assigned. \revise{The room-level sampling prioritizes selecting distinct rooms to avoid spatial clustering of fire points, while the point-level sampling ensures intra-room variability. This approach aligns with stratified sampling principles commonly used for efficient spatial representation, where multi-stage sampling strategies optimize coverage and variability, e.g., \cite{arunachalam_generalized_2023}, and enables a more comprehensive characterizing of how the structures respond under fire conditions.}
% This selection method prevents fire points from clustering too closely while maintaining an element of randomness. By distributing fire points in this manner, the 30 fire scenarios are effectively utilized, enabling a more comprehensive characterizing of how the structures respond under fire conditions.

\subsection{Summary of the Dataset Generation}
As discussed in this section and related to  \figref{fig:dataset_generation_procedure}, three key steps were considered in the development of the dataset: 
\begin{enumerate}
    \item {\bf{Filtering process}}: Structures with MIDR exceeding $1\%$ under gravity loads were excluded,  resulting in $1,573$ labeled structures retained for fire simulation and $16,050$ unlabeled structures for training the MFSP predictor.
    \item {\bf{Fire simulations}}: For each retained labeled structure, 30 fire scenarios were simulated using OpenSeesRT, yielding $47,190$ fire cases.
    \item {\bf{Data distribution check}}: MIDR distributions for labeled and unlabeled data under gravity loads were highly similar, because both datasets were generated using the same method. Under fire conditions, the MIDR distribution shifted, reflecting significant structural deformation with values reaching a maximum of about 6\%, an average of 1.70\%, and a standard deviation of 1.12\%. This step ensured a diverse and comprehensive dataset for the proposed predictive framework.
\end{enumerate}
The statistical distribution histograms for MIDR (after applying the $1\%$ filtering threshold \revise{for gravity load responses}) under different loading conditions are plotted in \figref{fig:histogram_mdr}. Figures \ref{fig:histogram_mdr}(a) and \ref{fig:histogram_mdr}(b) show the MIDR distributions of the labeled and unlabeled data, respectively, under gravity loads only. \figref{fig:histogram_mdr}(c) shows the MIDR distribution of the labeled data under the combined effects of gravity and fire loads. Fire load causes the structures to significantly deform, leading to a noticeably \revise{right-skewed} MIDR distribution.

\begin{figure*}[h!]
    \centering
    \includegraphics[width=\linewidth]{figures/histogram_mdr.pdf}
    \caption{Histograms of MIDR for labeled and unlabeled structures with gravity loads and fire cases.}
    \label{fig:histogram_mdr}
\end{figure*}

\revise{
This dataset provides the basis for training and testing the performance of the GNN-based framework. Although we employed a simplified rule-based thermal load generation method compared with conventional CFD-based simulations, the temperature field, the changes of the material properties, and the response of the structures, are all still highly nonlinear and complex. Therefore, it is still a challenging task for the NN to predict the MIDRs based on this dataset.
}
\section{Evaluation}
We provide three sets of insights into this section, organised as \textit{findings (F*)}. We quantitatively study the effect of the adversarial and counterfactual perturbations on the performance of informal reasoners and autoformalisation methods. Then, we dive deeper into method variants. Finally, 
we analyse the nature of formalisation errors made by the models.

\subsection{Robustness Analysis}
\paragraph{\textbf{\emph{F1: Noise perturbations have a stronger effect on formalisation methods than informal \ac{LLM} reasoners.}}}
Table~\ref{tab:distraction_k4_formalisation} shows that, on average, the accuracy of both direct and \ac{CoT} informal reasoning remains between $73\%$ and $74\%$ in the face of added noise. While the autoformalisation method performs similarly to informal reasoners on the original dataset, its performance decreases between $4\%$ and $11\%$. The accuracy drops especially with logical (L) and tautological (T) distractions, whose logical language formats trick the \ac{LLM} into formalizing the noisy clauses. On the other hand, the linguistically complex and more natural sentences of encyclopedic distractions show a minor effect, suggesting that \acp{LLM} successfully avoids formalizing the more complicated sentences.

\paragraph{\textbf{\emph{F2: All \ac{LLM}-based reasoning methods suffer a drop for counterfactual perturbations.}}} % influence .}}}
Table~\ref{tab:distraction_k4_formalisation} shows that counterfactual statements cause a significant decrease in performance for both the informal reasoners and autoformalisation methods of between $12\%$ and $13\%$ on average. 
Moreover, this observation also holds for all tested models, i.e., none are robust towards counterfactual perturbations across every evaluated dimension. Even the strongest model, GPT 4o-mini, yields a performance of 63-68\%, which is relatively close to the random performance of 50\%. The high impact of counterfactual statements (the single ``not'' inserted) could be due to the inability of \acp{LLM} to overwrite prior knowledge with explicitly stated information or memorization of the answers. We study the error sources further in §\ref{subsec:errors}.  

\noindent \paragraph{\textbf{\emph{F3: Introducing multiple noise sentences has an effect only for logical distractions.}}}
We show the impact of introducing between one and four sentences for the two top-performing autoformalisation models in Figure~\ref{fig:length_distraction}. The figure shows similar trends with and without counterfactual perturbations.
As additional logical distractions are introduced, the model performance consistently decreases. Tautological (T) distractions lead to a decline in accuracy with a single disruptive sentence, yet adding more noise does not worsen the outcome. 
The tautological corpus introduces truth constants for all sentences as a persistent unseen logical construct. Given that this leads only to a decrease for a single occurrence, we can assume that a model can consistently handle the same unseen logical construct. In contrast, the logical corpus increases the chance of adding text, requiring new, previously unseen reasoning constructs for each added sentence. The impact of encyclopedic noise remains negligible, generalising F1 to $k$ sentences. Similarly, counterfactual perturbations remain much more effective for all settings, generalising F2.

\begin{table}[!t]
\small
\setlength{\modelspacing}{2pt}
\setlength{\tabcolsep}{1.7pt} % Default value: 6pt
\setlength{\belowrulesep}{4pt}
\begin{threeparttable}
    \centering
    \begin{tabular}{cc l r rrr @{\quad} rrrr}
\toprule
\multirow{2}{*}{} & \multirow{2}{*}{} & Reasoning & \multirow{2}{*}{O} & \multicolumn{3}{c}{Distraction} & \multicolumn{4}{c}{Counterfactual} \\
 & & Format & & E& L & T & $\text{O}_C$ & $\text{E}_C$& $\text{L}_C$ & $\text{T}_C$\\
\midrule
\multirow{6}{*}{\rotatebox{90}{Gemma-2}} & \multirow{3}{*}{\rotatebox{90}{9b}}
   & Informal (direct) & \textbf{0.78} & \textbf{0.80} & \textbf{0.79} & \textbf{0.77} & 0.58 & 0.52 & 0.50 & 0.59 \\
 & & Informal (CoT) & 0.72 & 0.78 & 0.73 & 0.76 & 0.61 & \textbf{0.57} & \textbf{0.60} & \textbf{0.66} \\
 & & Formal (FOL) & 0.62 & 0.58 & 0.52 & 0.53 & \textbf{0.63} & 0.52 & 0.46 & 0.46 \\[\modelspacing]
\cmidrule{2-11}
 & \multirow{3}{*}{\rotatebox{90}{27b}} 
   & Informal (direct) & 0.71 & 0.69 & \textbf{0.66} & \textbf{0.68} & 0.59 & 0.51 & 0.54 & 0.59 \\
 & & Informal (CoT) & 0.66 & 0.65 & 0.64 & 0.63 & 0.62 & 0.58 & \textbf{0.62} & \textbf{0.64} \\
 & & Formal (FOL) & \textbf{0.74} & \textbf{0.74} & 0.61 & 0.61 & \underline{\textbf{0.72}} & \underline{\textbf{0.67}} & 0.58 & 0.51 \\[\modelspacing]
\midrule
\multirow{6}{*}{\rotatebox{90}{Mistral}} & \multirow{3}{*}{\rotatebox{90}{7B}} 
   & Informal (direct) & 0.77 & \textbf{0.77} & 0.75 & \textbf{0.79} & \textbf{0.63} & \textbf{0.54} & \textbf{0.54} & \textbf{0.66} \\
 & & Informal (CoT) & \textbf{0.79} & 0.75 & \textbf{0.77} & 0.78 & 0.55 & 0.52 & \textbf{0.54} & 0.58 \\
 & & Formal (FOL) & 0.62 & 0.58 & 0.54 & 0.57 & 0.50 & \textbf{0.54} & 0.51 & 0.52 \\[\modelspacing]
\cmidrule{2-11}
 & \multirow{3}{*}{\rotatebox{90}{Small}} 
   & Informal (direct) & \textbf{0.77} & \textbf{0.76} & \textbf{0.76} & \textbf{0.75} & 0.61 & 0.51 & 0.56 & 0.59 \\
 & & Informal (CoT) & 0.72 & 0.72 & 0.72 & 0.71 & \textbf{0.62} & \textbf{0.59} & \textbf{0.62} & \textbf{0.68} \\
 & & Formal (FOL) & 0.68 & 0.59 & 0.53 & 0.64 & 0.54 & 0.55 & 0.49 & 0.51 \\[\modelspacing]
\midrule
\multirow{6}{*}{\rotatebox{90}{Llama-3.1}} & \multirow{3}{*}{\rotatebox{90}{8B}} 
   & Informal (direct) & 0.63 & 0.61 & 0.64 & 0.66 & 0.61 & \textbf{0.62} & 0.59 & 0.61 \\
 & & Informal (CoT) & 0.73 & \textbf{0.73} & \textbf{0.71} & \textbf{0.72} & \textbf{0.62} & 0.59 & \textbf{0.61} & \textbf{0.65} \\
 & & Formal (FOL) & \textbf{0.77} & 0.71 & 0.63 & 0.52 & 0.60 & 0.58 & 0.55 & 0.52 \\[\modelspacing]
\cmidrule{2-11}
 & \multirow{3}{*}{\rotatebox{90}{70B}} 
   & Informal (direct) & 0.77 & 0.74 & 0.74 & 0.73 & 0.62 & 0.53 & 0.56 & 0.64 \\
 & & Informal (CoT) & \textbf{0.78} & \textbf{0.75} & \textbf{0.76} & \textbf{0.76} & 0.64 & 0.61 & \textbf{0.66} & \underline{\textbf{0.73}} \\
 & & Formal (FOL) & 0.74 & 0.73 & 0.71 & 0.71 & \textbf{0.66} & \textbf{0.62} & 0.59 & 0.57 \\[\modelspacing]
 \midrule
\multirow{3}{*}{\rotatebox{90}{GPT}} & \multirow{3}{*}{\rotatebox{90}{4o-mini}} 
   & Informal (direct) & 0.78 & 0.77 & 0.79 & 0.79 & 0.64 & 0.61 & 0.61 & 0.63 \\
 & & Informal (CoT) & 0.80 & 0.80 & \underline{\textbf{0.81}} & \underline{\textbf{0.82}} & \textbf{0.68} & \textbf{0.63} & \underline{\textbf{0.68}} & \textbf{0.64} \\
 & & Formal (FOL) & \underline{\textbf{0.84}} & \underline{\textbf{0.82}} & 0.73 & 0.79 & 0.63 & 0.62 & 0.57 & 0.54 \\[\modelspacing]
 \midrule
\multicolumn{2}{c}{\multirow{3}{*}{\textbf{Avg}}} 
 & Informal (direct) & 0.74 & 0.73 & 0.73 & 0.73 & 0.61 & 0.55 & 0.56 & 0.62 \\
 & & Informal (CoT) & 0.74 & 0.74 & 0.73 & 0.74 & 0.62 & 0.58 & 0.62 & 0.65 \\
  & & Formal (FOL) & 0.72 & 0.68 &	0.61 & 0.62 & 0.61 & 0.59 & 0.54 & 0.52 \\
\bottomrule
\end{tabular}
\caption{Accuracies of informal and autoformalisation-based deductive reasoners. The best overall model per dataset is underlined; the best model version is marked in bold.}
\label{tab:distraction_k4_formalisation}
\end{threeparttable}
\end{table} 

\begin{figure}[!t]
    \centering
    \scriptsize
    \begin{tikzpicture}
        \begin{axis}[name=gpt,
            title={GPT-4o-mini},
            width=0.6\linewidth,
            height=0.6\linewidth,
            xlabel={\# Noise sentences},
            ylabel={Accuracy},
            xmin=-0.1, xmax=4.1,
            ymin=0.5, ymax=0.9,
            xtick={1,2,4},
            ytick={0.55, 0.6, 0.65, 0.75, 0.8, 0.85},
            title style={yshift=-0.6em},
            legend style={at={(1,-0.15)},
	           anchor=north,legend columns=-1},
            x label style={at={(axis description cs:1,-0.05)},anchor=north},
            y label style={at={(axis description cs:-0.15,0.5)},anchor=south},
            ymajorgrids=true,
            grid style=dashed,
        ]
            \addplot[color=blue, mark=square,]
                coordinates {
                (0,0.848076939582825)(1,0.823076903820038)(2,0.826923072338104)(4,0.821153819561005)
                };
            \addplot[color=red, mark=triangle,]
                coordinates {
                (0,0.848076939582825)(1,0.817307710647583)(2,0.801923096179962)(4,0.759615361690521)
                };
            \addplot[color=green, mark=diamond,] 
                coordinates {
                (0,0.848076939582825)(1,0.767307698726654)(2,0.769230782985687)(4,0.803846180438995)
                };
            \addplot[color=blue, mark=square*] 
                coordinates {
                (0,0.627777755260468)(1,0.622222244739533)(2,0.600000023841858)(4,0.633333325386047)
                };
            \addplot[color=red, mark=triangle*,] 
                coordinates {
                (0,0.627777755260468)(1,0.611111104488373)(2,0.611111104488373)(4,0.594444453716278)
                };
            \addplot[color=green, mark=diamond*,] 
                coordinates {
                (0,0.627777755260468)(1,0.572222232818604)(2,0.538888871669769)(4,0.555555582046509)
                };
                \legend{E,L,T,$\text{E}_C$, $\text{L}_C$ , $\text{T}_C$}
        \end{axis}

        \begin{axis}[name=llama, at={($(gpt.east)+(0.1cm,0)$)},anchor=west,
            title={Llama 3.1 70b},
            width=0.6\linewidth,
            height=0.6\linewidth,
            xmin=-0.1,, xmax=4.1,
            ymin=0.5, ymax=0.9,
            xtick={1,2,4},
            ytick={0.55, 0.6, 0.65, 0.75, 0.8, 0.85},
            title style={yshift=-0.6em},
            yticklabel=\empty,
            ymajorgrids=true,
            grid style=dashed,
        ]
            \addplot[color=blue, mark=square,]
                coordinates {
                (0,0.838461518287659)(1,0.817307710647583)(2,0.805769205093384)(4,0.817307710647583)
                };
            \addplot[color=red, mark=triangle,]
                coordinates {
                (0,0.838461518287659)(1,0.819230794906616)(2,0.803846180438995)(4,0.771153867244721)
                };
            \addplot[color=green, mark=diamond,]
                coordinates {
                (0,0.838461518287659)(1,0.803846180438995)(2,0.807692289352417)(4,0.805769205093384)
                };
            \addplot[color=blue, mark=square*]
                coordinates {
                (0,0.627777755260468)(1,0.622222244739533)(2,0.577777802944183)(4,0.594444453716278)
                };
            \addplot[color=red, mark=triangle*,]
                coordinates {
                (0,0.627777755260468)(1,0.583333313465118)(2,0.561111092567444)(4,0.577777802944183)
                };
            \addplot[color=green, mark=diamond*,]
                coordinates {
                (0,0.627777755260468)(1,0.627777755260468)(2,0.566666662693024)(4,0.577777802944183)
                };
        \end{axis}
    \end{tikzpicture}
    \caption{Influence of the number of noisy sentences for FOL.}
    \label{fig:length_distraction}
\end{figure}



\subsection{Impact of Method Design}
\paragraph{\textbf{\emph{F4: \ac{CoT} prompting is most impactful when both noise and counterfactual perturbations are applied.}}}
The accuracies for the individual \acp{LLM} in Table~\ref{tab:distraction_k4_formalisation} show that the impact of \ac{CoT} is negligible for noise-only datasets (first four columns). Meanwhile, the benefit from \ac{CoT} is most pronounced in the datasets that combine noise and counterfactual perturbations.
The better-performing informal prompting strategy for a model remains stable for all types of distractions. Still, the decline in performance due to counterfactuals leads to a less consistent preference for a specific prompting style.

\paragraph{\textbf{\emph{F5: The best-performing grammar differs per model and is unstable across data versions.}}}

The evaluation of different logical forms for formal \ac{LLM}-based reasoning in Table~\ref{tab:distraction_k4_logical_form} shows the preference of some models for specific syntactic formats.
Llama 3.1 70B has a considerable improvement of $12\%$ with TPTP syntax on the original set, while Llama 3.1 8B benefits from the R-FOL syntax. However, all grammars show a declining accuracy trend and increased syntax errors for noise perturbations, where the best grammar loses its advantage over the rest. 
When comparing the grammars on the counterfactual partitions, we observe that TPTP is consistently more robust than the standard first-order logic grammar. Here, GPT 4o-mini shows a reduction from $O$ to $O_C$ of $20\%$ for FOL and only $12\%$ for the TPTP grammar. Since this does not correlate with fewer syntax errors, the formalisation in TPTP prevents semantical errors for counterfactual premises. 
A positive reading of these results, especially the minor differences between FOL and R-FOL, is that autoformalisation \acp{LLM} can adapt to the grammar syntax prescribed in the prompt without further loss in performance.

\begin{table}[!t]
\small
\setlength{\modelspacing}{2pt}
\setlength{\tabcolsep}{1.7pt} % Default value: 6pt
\setlength{\belowrulesep}{4pt}
\begin{threeparttable}
    \centering
    \begin{tabular}{cc l r rrr @{\quad} rrrr}
\toprule
\multirow{2}{*}{} & \multirow{2}{*}{} & Grammar & \multirow{2}{*}{O} & \multicolumn{3}{c}{Distraction} & \multicolumn{4}{c}{Counterfactual} \\
 & & Syntax & & E& L & T & $\text{O}_C$ & $\text{E}_C$& $\text{L}_C$ & $\text{T}_C$\\
\midrule
\multirow{6}{*}{\rotatebox{90}{Llama-3.1}} & \multirow{3}{*}{\rotatebox{90}{8B}} 
   & FOL & 0.77 & \textbf{0.71} & 0.61 & \textbf{0.53} & 0.58 & \textbf{0.55} & 0.52 & \textbf{0.56} \\
 & & R-FOL & \textbf{0.78} & 0.69 & \textbf{0.62} & \textbf{0.53} & 0.58 & \textbf{0.55} & \textbf{0.54} & 0.52 \\
 & & TPTP & 0.73 & 0.67 & 0.55 & 0.51 & \textbf{0.68} & 0.54 & 0.46 & 0.51 \\[\modelspacing]
\cmidrule{2-11}
 & \multirow{3}{*}{\rotatebox{90}{70B}} 
   & FOL & 0.76 & 0.73 & 0.71 & \textbf{0.72} & 0.67 & 0.57 & 0.63 & 0.56 \\
 & & R-FOL & 0.76 & 0.73 & 0.67 & 0.71 & 0.64 & 0.57 & 0.53 & 0.64 \\
 & & TPTP & \underline{\textbf{0.88}} & \underline{\textbf{0.84}} & \underline{\textbf{0.81}} & \textbf{0.72} & \underline{\textbf{0.81}} & \underline{\textbf{0.68}} & \underline{\textbf{0.67}} & \underline{\textbf{0.68}} \\[\modelspacing]
\midrule
\multirow{3}{*}{\rotatebox{90}{GPT}} & \multirow{3}{*}{\rotatebox{90}{4o-mini}} 
   & FOL & \textbf{0.84} & \textbf{0.82} & \textbf{0.72} & \underline{\textbf{0.78}} & 0.64 & \textbf{0.63} & \textbf{0.61} & 0.51 \\
 & & R-FOL & \textbf{0.84} & 0.77 & 0.70 & \underline{\textbf{0.78}} & \textbf{0.72} & 0.56 & 0.54 & \textbf{0.63} \\
 & & TPTP & 0.83 & \textbf{0.82} & 0.71 & 0.71 & 0.69 & \textbf{0.63} & 0.57 & 0.57 \\
\bottomrule
\end{tabular}
\caption{Accuracies of different formalisation grammars for autoformalisation.}
\label{tab:distraction_k4_logical_form}
\end{threeparttable}
\end{table} 

\paragraph{\textbf{\emph{F6: Feedback does not help \acp{LLM} self-correct to mitigate robustness issues.}}}
\autoref{tab:distraction_k4_feedback} shows the results with different error recovery mechanisms. The results indicate that no feedback strategy emerges as a winner in the different datasets. 
All feedback variants reduce syntax errors for noise perturbations, but given the lack of a consistent increase in accuracy, the corrected formalisations are most likely to contain semantic errors still. 
The type of feedback message only has a minor influence on correcting syntax errors, whereas Llama 3.1 70b and GPT 4o-mini correct slightly more syntax errors with specific error messages. This finding aligns with \cite{huang2023large}, who also found that \acp{LLM} cannot consistently self-correct their reasoning after receiving relevant feedback.

\begin{table}[!ht]
\small
\setlength{\modelspacing}{2pt}
\setlength{\tabcolsep}{1.7pt} % Default value: 6pt
\setlength{\belowrulesep}{4pt}
\begin{threeparttable}
    \centering
    \begin{tabular}{cc l r rrr @{\quad} rrrr}
\toprule
\multirow{2}{*}{} & \multirow{2}{*}{} & \multirow{2}{*}{Feedback} & \multirow{2}{*}{O} & \multicolumn{3}{c}{Distraction} & \multicolumn{4}{c}{Counterfactual} \\
 & & & & E& L & T & $\text{O}_C$ & $\text{E}_C$& $\text{L}_C$ & $\text{T}_C$\\
\midrule
\multirow{8}{*}{\rotatebox{90}{Llama-3.1}} & \multirow{4}{*}{\rotatebox{90}{8B}} 
   & No recovery & 0.77 & \textbf{0.72} & 0.62 & 0.53 & 0.59 & 0.58 & 0.56 & \textbf{0.56} \\
 & & Error type & \textbf{0.79} & 0.71 & 0.63 & \textbf{0.56} & \textbf{0.66} & 0.54 & 0.52 & 0.51 \\
 & & Error message & 0.78 & 0.71 & \textbf{0.67} & 0.55 & 0.59 & 0.53 & \underline{\textbf{0.64}} & 0.49 \\
 & & Warning & 0.74 & 0.66 & 0.58 & 0.55 & 0.55 & \textbf{0.60} & 0.49 & 0.49 \\[\modelspacing]
\cmidrule{2-11}
 & \multirow{4}{*}{\rotatebox{90}{70B}} 
   & No recovery & \textbf{0.77} & \textbf{0.72} & \textbf{0.73} & 0.71 & \textbf{0.64} & 0.59 & \textbf{0.61} & 0.56 \\
 & & Error type & 0.72 & 0.70 & 0.72 & \textbf{0.73} & 0.62 & 0.56 & 0.60 & 0.58 \\
 & & Error message & 0.71 & 0.70 & \textbf{0.73} & 0.71 & \textbf{0.64} & 0.59 & 0.54 & \underline{\textbf{0.64}} \\
 & & Warning & 0.69 & \textbf{0.72} & 0.72 & 0.72 & 0.62 & \underline{\textbf{0.65}} & \textbf{0.61} & 0.63 \\[\modelspacing]
\midrule
\multirow{4}{*}{\rotatebox{90}{GPT}} & \multirow{4}{*}{\rotatebox{90}{4o-mini}} 
   & No recovery & \underline{\textbf{0.84}} & \underline{\textbf{0.82}} & 0.73 & 0.79 & 0.64 & \textbf{0.62} & 0.56 & \textbf{0.56} \\
 & & Error type & 0.83 & 0.79 & 0.74 & 0.76 & 0.67 & 0.57 & 0.56 & \textbf{0.56} \\
 & & Error message & \underline{\textbf{0.84}} & 0.78 & \underline{\textbf{0.77}} & \underline{\textbf{0.80}} & 0.62 & 0.59 & 0.56 & \textbf{0.56} \\
 & & Warning & \underline{\textbf{0.84}} & 0.75 & 0.73 & 0.76 & \underline{\textbf{0.70}} & 0.61 & \textbf{0.61} & 0.55 \\
 \bottomrule
\end{tabular}
\caption{Accuracies of error recovery strategies.}
\label{tab:distraction_k4_feedback}
\end{threeparttable}
\end{table} 

\subsection{Error Analysis}
\label{subsec:errors}
\paragraph{\textbf{\emph{F7: Autoformalisation increases syntax errors for noise perturbations.}}}
The low performance for noise perturbations correlates with more syntax errors for all models and distraction categories (cf. execution rates in Table~\ref{tab:appendix_k4_formalisation_exec}). The three worst-performing models (both Mistral models, Gemma-2 9b) generate, at best, for $37\%$  and, at worst, for only $4\%$ of the samples, a valid logical form.
Gemma-2 9b and Llama3.1 8b produce more syntax errors than the larger counterparts, suggesting that larger models are more robust towards noise perturbations. 
The accuracy of syntactically valid samples is higher than the informal reasoning methods for most distractions (Table~\ref{tab:appendix_k4_formalisation_vacc}), motivating informal reasoning as a backup strategy for formal reasoning. The error message feedback reveals two common syntax errors: 1) errors by models with an initial low execution rate exhibit issues with the template structure, including using incorrect keywords or adding conversational phrases;
2) perturbation-related errors, the most common of which is using undefined truth constants as part of tautological distractions. 

\paragraph{\textbf{\emph{F8: Autoformalisation increases semantic errors for counterfactuals.}}}
Unlike the introduced noise, counterfactual perturbations do not lead to more syntax errors. The execution rate in Table~\ref{tab:appendix_k4_formalisation_exec} is stable or improves for counterfactuals. However, we see a drop in accuracy for the counterfactual column $\text{O}_C$ in Table~\ref{tab:distraction_k4_formalisation} and can conclude that the number of logical forms with semantic errors has to increase. This suggests that the introduced negation is not correctly formalised. Looking at the warnings generated by the feedback mechanism, for GPT 4o-mini, $161$ warning messages are generated on the unperturbed data. $54$ of these were fixed with a single iteration. Not considering predicates and individuals as part of the context is the most frequent warning across all models. 
\section{Experimental Setup}
\label{appendix:experimental_setups}
We evaluate EDELINE on the Atari 100k benchmark~\cite{chevalier-schwarzer2023biggerbetterfasterhumanlevel}, which serves as the standard evaluation protocol in recent model-based RL literature for fair comparison. In addition, our experimental validation extends to ViZDoom~\cite{kempka2016vizdoomdoombasedairesearch} and MiniGrid~\cite{chevalier-boisvert2023minigrid} environments to demonstrate broader applicability. To ensure statistical significance, all reported results represent averages across three independent runs. The Atari 100k benchmark~\cite{chevalier-schwarzer2023biggerbetterfasterhumanlevel} encompasses 26 diverse Atari games that evaluate various aspects of agent capabilities. Each agent receives a strict limitation of 100k environment interactions for learning, in contrast to conventional Atari agents that typically require 50 million steps. EDELINE's performance is evaluated against current state-of-the-art world model-based approaches, including DIAMOND~\cite{alonso2024diamond}, STORM~\cite{zhang2023storm}, DreamerV3~\cite{hafner2024DreamerV3}, IRIS~\cite{micheli2023iris}, TWM~\cite{robine2023TWM}, and Drama~\cite{anonymous2025drama}. For evaluating 3D scene understanding capabilities, we employ VizDoom scenarios that demand sophisticated 3D spatial reasoning in first-person environments. This provides a crucial testing ground beyond the third-person perspective of Atari environments. Furthermore, the MiniGrid memory scenarios evaluate memorization capabilities through tasks that require information retention across extended time horizons.
\begin{table}[ht!]
\centering
\caption{\textbf{Super Resolution Performance Results.} Our proposed WGAN EEG Spatial Upsampling method significantly outperforms a baseline of Bicubic Interpolation commonly used in EEG upsampling pipelines.}
\label{tab:results}
\resizebox{0.8\linewidth}{!}{%
\begin{tabular}{@{}cccccc@{}}
\toprule
\multirow{2}{*}{\textbf{Dataset}} & \multirow{2}{*}{\textbf{Scale}} & \multicolumn{2}{c}{\textbf{Bicubic}} & \multicolumn{2}{c}{\textbf{WGAN}} \\ \cmidrule(l){3-6} 
                      &   & \textbf{MSE} & \textbf{MAE} & \textbf{MSE}    & \textbf{MAE}   \\
\toprule
\multirow{2}{*}{Val}  & 2 & 3.71E7       & 3.89E3       & \textbf{2.01E3} & \textbf{24.38} \\
                      & 4 & 7.23E7       & 6.42E3       & \textbf{8.53E3} & \textbf{63.83} \\
\midrule
\multirow{2}{*}{Test} & 2 & 3.75E7       & 3.91E3       & \textbf{2.06E3} & \textbf{24.66} \\
                      & 4 & 7.30E7       & 6.45E3       & \textbf{8.68E3} & \textbf{64.39} \\
\bottomrule
\end{tabular}%
}
\end{table}
\section*{Conclusion}
This paper aims to enhance our understanding of the computational complexity of computing various Shapley value variants. We found that for various ML models --- including decision trees, regression tree ensembles, weighted automata, and linear regression --- both local and global interventional and baseline SHAP can be computed in polynomial time under HMM modeled distributions. This extends popular algorithms, such as TreeSHAP, beyond their empirical distributional scope. We also establish strict complexity gaps between the various SHAP variants (baseline, interventional, and conditional) and prove the intractability of computing SHAP for tree ensembles and neural networks in simplified scenarios. Overall, we present SHAP as a versatile framework whose complexity depends on four key factors: \begin{inparaenum}[(i)] \item model type, \item SHAP variant, \item distribution modeling approach, \item and local vs. global explanations\end{inparaenum}. We believe this perspective provides deeper insight into the computational complexity of SHAP, paving the way for future work.




%We believe that our framework provides a more intricate understanding of SHAP computation complexity across different models, distributions, and variants, paving the way for further research.

Our work opens promising directions for future research. First, expanding our computational analysis to other SHAP-related metrics, such as asymmetric SHAP~\citep{frye20} and SAGE~\citep{covert2020understanding}, would be valuable. Additionally, we aim to explore more expressive distribution classes and relaxed assumptions beyond those in Section \ref{sec:tractable} while maintaining tractable SHAP computation. Finally, when exact computation is intractable (Section \ref{sec:intractable}), investigating the approximability of SHAP metrics through approximation and parameterized complexity theory~\citep{downey2012parameterized} is an important direction.

%Our work opens several promising avenues for future research on the computational properties of explainable AI methods, with a particular focus on SHAP. First, it would be interesting to broaden the computational analysis conducted in this work to include other popular SHAP-related metrics in the literature, such as asymmetric SHAP \cite{frye20} and SAGE \cite{covert2020understanding}. Also, in the future, we aim to explore more expressive distribution classes and relaxed distributional assumptions—extending beyond those examined in Section \ref{sec:tractable} —that still yield tractable SHAP computation. Finally, when exact computation proves intractable (Section \ref{sec:intractable}), it is worthwhile to theoretically investigate the question of the approximability of computing the SHAP metrics across various configurations, through the lens of approximation and parametrized complexity theory \cite{arora2009computational}.

%This paper aims to deepen our understanding of the computational complexity involved in obtaining different Shapley value variants. We found that for a variety of ML models, including decision trees, tree ensembles for regression, weighted automata, and linear regression models — computing both local and global interventional and baseline SHAP can be done in polynomial time when distributions are modeled by HMMs. This extends the distributional scope of popular algorithms like TreeSHAP, which is limited to empirical distributions. Additionally, we demonstrate a strict complexity gap between SHAP variants, showing that interventional and baseline SHAP can be strictly easier to compute than conditional SHAP. Despite these positive results, we uncovered intractability for various SHAP variants in neural networks and tree ensembles. Finally, we provided generalized complexity relations across SHAP variants. We believe that our framework offers a deeper understanding of the complexity involved in computing SHAP across various variants, models, distributions, as well as in both local and global computations, laying the groundwork for future research.



% \pagebreak
\section*{Limitations}
We experiment with two models, {RoBERTa} and {BERT}, and a single pre-training dataset, Wikipedia.
Future work may consider other models and pre-training datasets.
Our rules for reversing polarity only cover \hl{``not'', ``n't'', and ``never''}.
However, they are still effective in making models more robust against negation in general---recall that CondaQA has over 200 unique negation cues.
Future work may consider working with more sophisticated rules to reverse the polarity of the sentences.
We also only experiment with models pre-trained on 500K and 1M instances.
Future work may consider training on the whole corpus and evaluate the performance on downstream tasks.
Additionally, all the corpora we work with are in English.
We acknowledge that negation may be expressed differently in other languages and this work may not generalize to other languages.
We note, however, that the proposed tasks are language-agnostic and can be applied to other languages.



\section*{Ethics Statement}
The work in this paper does not involve human subjects.
We only use publicly available datasets and models.
We do not collect any personal information.
Therefore, this work does not raise any ethical concerns.

% Entries for the entire Anthology, followed by custom entries
\bibliography{custom}
% \bibliographystyle{acl_natbib}

%\clearpage
\appendix
\section{Details of Reversing Polarity}
\label{app:reversing}

\begin{table*}
    \centering
    \begin{tabular}{l@{\hspace{0.5in}} lll}
\toprule
\textbf{Aux.} & \textbf{not} & \textbf{n't} & \textbf{never} \\
\midrule
be & not be & - & never be \\
being & not being & - & never being \\
was & was not & wasn't & was never \\
is & is not & isn't & is never \\
were & were not & weren't & were never \\
have & have not & haven't & have never \\
having & not having & - & never having \\
had & had not & hadn't & had never \\
've & 've not & - & 've never \\
do & do not & don't & do never \\
does & does not & doesn't & does never \\
did & did not & didn't & did never \\
can & can not & can't & can never \\
could & could not & couldn't & could never \\
will & will not & won't & will never \\
'll & 'll not & - & 'll never \\
would & would not & wouldn't & would never \\
shall & shall not & shan't & shall never \\
should & should not & shouldn't & should never \\
must & must not & - & must never \\
may & may not & - & may never \\
might & might not & - & might never \\
\bottomrule    
\end{tabular}
    \caption{
        Auxiliary verbs that we work with and how each negation cue is added to them.
        \label{fig:aux}
        }
\end{table*}

Table~\ref{fig:aux} shows the auxiliary verbs that we work with 
and how we add each of the three negation cues 
(not, n't, never) to them.
Some auxiliary verbs do not have a corresponding form for adding \hl{``n't''}
such as \emph{having}, \emph{may} and \emph{might}.
If there are multiple auxiliary verbs for a single verb,
we only negate the one that is most commonly used.
For example, in the sentence \emph{She might have been sleeping when you called.}
we only negate \emph{have} and not \emph{might}: \emph{She might have not been sleeping when you called.}
The result is grammatically correct, but it may not be the most common form in English.
\begin{table}[H]
    \small
    \centering
    \setlength{\tabcolsep}{4pt}
    \begin{tabular}{lccc}
        \toprule
        \multirow{2}{*}{\textbf{LLM for Data Generation}} & \multicolumn{3}{c}{\textbf{ChartQA}} \\ \cmidrule{2-4}
        & Average & Machine & Human \\
        \midrule
        GPT-4o & 72.4 & 65.8 & 78.9 \\ 
        Claude-3.5-sonnet & \textbf{77.2} & \textbf{71.0} & \textbf{83.8} \\
        \bottomrule
    \end{tabular}
    \vspace{-.1cm}
    \caption{\textbf{Compare the LLMs used for synthetic data generation.} For both LLMs, we create 100K synthetic charts for fine-tuning the VLMs. We report the zero-shot evaluation results on ChartQA.}
    \label{tab:llm_ablate}
    \vspace{-.3cm}
\end{table}

\section{CondaQA Example}
\label{sec:condaqaexample}
\begin{figure}
  \small
  \begin{tabularx}{0.48 \textwidth}{lX}
      \toprule
      \textbf{Type} & \textbf{Example} \\
      \midrule
      Original & He didn't go to the store, but he went to the park. \\
      Paraphrase & He went to the park but not the store. \\
      Scope & He went to the store, but he didn't go to the park. \\
      Affirmation & He went to the store and the park. \\
      \bottomrule
  \end{tabularx}
  \caption{
      Three types of edits in CondaQA are applied to an example sentence.
      \label{fig:condaqaeditexample}
  }
\end{figure} 



\begin{figure*}
    \small
      \begin{tabularx}{\textwidth}{p{.9in}X} 
        \toprule
        {Original Passage:} & 33\% of the faculty are members of the National Academy of Science or Engineering and/or fellows of the American Academy of Arts and Sciences. This is the highest percentage of any faculty in the country with the exception of the graduate institution Rockefeller University. \\ \addlinespace
        {Original Sentence (with Negation):} & This is the highest percentage of any faculty in the country \textbf{with the exception of} the graduate institution Rockefeller University. \\ \addlinespace
        {Negation Cue:} & with the exception of \\ \addlinespace
        {Question:} & Are the majority of faculty at any school other than Rockefeller University members of the National Academy of Science or Engineering? \\ \addlinespace
        \midrule
        Paraphrase Edit: &  33\% of the faculty are members of the National Academy of Science or Engineering and/or fellows of the American Academy of Arts and Sciences. This is the highest percentage of any faculty in the country \emph{other than} the graduate institution Rockefeller University. \\ \addlinespace
        \midrule
        Scope Edit: &  33\% of the faculty are \emph{not} members of the National Academy of Science or Engineering and/or fellows of the American Academy of Arts and Sciences. This is the highest percentage of any faculty in the country with the exception of the graduate institution Rockefeller University \\ \addlinespace
        \midrule
        Affirmation Edit: &  33\% of the faculty are members of the National Academy of Science or Engineering and/or fellows of the American Academy of Arts and Sciences. This is the highest percentage of any faculty in the country \emph{including} the graduate institution Rockefeller University. \\ \addlinespace
        \midrule
      \end{tabularx}
      \begin{tabularx}{\textwidth}{l@{\hspace{0.28 \textwidth}}l}
        \midrule
            Input & Answer \\
        \midrule
            Question + Original Passage & No \\
            Question + Paraphrase Edit & No \\
            Question + Scope Edit & Yes \\
            Question + Affirmation Edit & No \\
        \bottomrule
      \end{tabularx}
        \caption{
          \label{fig:condaqasample}
          An example from CondaQA.
          The original passage contains a sentence with negation.
          The crowdworker makes three edits to the passage (paraphrase, scope, and affirmation edits) to create the edited passage.
          The question (also written by the crowdworker) asks about the majority of faculty (more than 50\%) at any school other than Rockefeller University.
          Changing the scope of negation changes the answer to the question from \emph{No} to \emph{Yes}.
        }
\end{figure*}
Figure~\ref{fig:condaqaeditexample} shows an example sentence with the three types of edits.
We also provide an example from CondaQA in Figure~\ref{fig:condaqasample}.
The original passage has been selected from the English Wikipedia and contains a sentence with negation.
Three edits are made to the passage to create the edited passage: 
a paraphrase edit (i.e. \emph{rewriting the sentence}), a scope edit (i.e. \emph{changing the scope of negation}), and an affirmation edit (i.e. \emph{undoing negation}).
The question is answered based on the original and edited passages (a group). 
A model needs to answer all the questions in a group correctly to achieve group consistency. 
\section{NLU and NLI Corpora}
\label{sec:nlunlicorpora}
\begin{figure*}
\small
\centering
    \begin{tabularx}{\textwidth}{lXp{1.25in}} 
        \toprule
        & Input & Output \\
        \midrule
        Natural Language Inference \\
        ~~~~QNLI & When was the last time San Francisco hosted a Super Bowl? & \multirow{2}{1.25in}{Not Entailment (i.e., question is not answered)} \\
        & The South Florida/Miami area has previously hosted the event 10 times (tied for most with New Orleans), with the most recent one being Super Bowl XLIV in 2010. \\
        ~~~~MNLI & $\text{T}_\text{neg}$: His knees were not bent. & Contradiction \\
        & H: He bent his legs. & \\
        ~~~~RTE & $\text{T}_\text{neg}$: Green cards are not becoming more difficult to obtain. $\text{H}_\text{neg}$: Green card is not now difficult to receive. & Entailment \\
        ~~~~SNLI & T: A very thin, black dog running in a field.  & Entailment \\
        & $\text{H}_\text{neg}$: The dog is not in the house.& \\   
        \midrule
        Word Sense Disambiguation \\
        ~~~~WiC & A \emph{check} on its dependability under stress. & Not same meaning \\
        & He paid all his bills by \emph{check}. \\
        \midrule
        Coreference Resolution \\
        ~~~~WSC & Sid explained his theory to \emph{Mark} but \emph{he} couldn't convince him. & Not coreferent \\
        \bottomrule
        \end{tabularx}
        \caption{
            \label{tab:nlu/icorpora}
            Examples from the NLU and NLI corpora used in this work.
            The corpora include natural language inference (NLI), word sense disambiguation (WiC), and coreference resolution (WSC) tasks.
            There are four NLI datasets: QNLI, MNLI, RTE, and SNLI.
            Examples for MNLI, RTE, and SNLI are selected from the new instances with negation created by \citet{hossain-etal-2020-analysis}.
        }
\end{figure*}
QNLI~\cite{rajpurkar-etal-2016-squad} is a natural language inference dataset 
created from the Stanford Question Answering Dataset (SQuAD)~\cite{rajpurkar-etal-2016-squad}.
It contains questions and sentences that are answers to other questions from SQuAD.
The task is to determine whether the context sentence contains the answer to the question.
WiC~\cite{pilehvar-camacho-collados-2019-wic} is a word sense disambiguation dataset.
It contains sentence pairs where a word can have the same or different meanings in the two sentences.
The task is to determine whether the word has the same meaning in the two sentences.
WSC~\cite{levesque_winograd_2012} is a coreference resolution dataset.
It contains sentences where a pronoun can refer to different entities in the sentence.
The task is to determine whether a pronoun and a noun phrase are co-referential.

We present examples from the NLU and NLI corpora used in this work in Figure~\ref{tab:nlu/icorpora}. 
The examples are from the development sets of the corpora 
other than the examples from the new instances with negation created by \citet{hossain-etal-2020-analysis} for MNLI, RTE, and SNLI.
\newcommand{\sig}{$^{\ast}$}
\setlength{\tabcolsep}{0.03in}
\small
\begin{tabular}{l cccc}
\toprule
LAMA& SQuAD & ConceptNet & TREx & GoogleRE \\
\midrule

{BERT-base} & 13.11 & 12.71 & \textbf{29.48} & 9.25 \\
~~~~ + NSPP & 12.79 & \textbf{12.72} & 29.01 & \textbf{9.52} \\
~~~~ + NSP & \textbf{14.43} & 12.02 & 28.78 & 9.90 \\
~~~~ + NSPP + NSP & 14.10 & 12.53 & 29.32 & 8.92 \\
\midrule
{BERT-large} &  15.74 & 15.17 & \textbf{30.02} & 9.78 \\
~~~~ + NSPP & 16.72 & \textbf{15.38} & 29.75 & 9.85 \\
~~~~ + NSP & 17.05 & 14.40 & 29.00 & \textbf{10.03} \\
~~~~ + NSPP + NSP & \textbf{17.38} & 14.07 & 28.93 & 9.98 \\

\midrule
\midrule
{RoBERTa-base} & 9.18 & \textbf{14.77} & \textbf{11.93} & 2.77 \\
~~~~ + NSPP & 9.84 & 14.73 & 11.80 & \textbf{2.78} \\
~~~~ + NSP & \textbf{10.16} & 14.42 & 11.28 & 2.44 \\
~~~~ + NSPP + NSP & 8.20 & 12.06 & 6.76 & 2.36 \\
\midrule
{RoBERTa-large} & 13.44 & 18.28 & \textbf{15.48} & 2.24 \\
~~~~ + NSPP & 13.77 & 17.59 & 13.80 & \textbf{2.78} \\
~~~~ + NSP & \textbf{14.10} & \textbf{18.32} & 15.46 & 2.28 \\
~~~~ + NSPP + NSP & 7.54 & 17.34 & 3.68 & 0.64 \\
\bottomrule
\end{tabular}

\begin{table*}[]
\centering
\resizebox{\textwidth}{!}{%
\begin{tabular}{@{}lcccccc@{}}
\toprule[1.5pt]
 & Model & Learning Rate  & Batch Size & KL Coefficient&Max Length & Training Epochs \\ 
\midrule[1pt]
& Llama-3.1-8B-Instruct & 5e-6  & 32 & 0.1&8000& 3\\
& Qwen2-7B-Instruct & 5e-6 & 32 & 0.1 &6000& 3 \\
& Qwen2.5-Math-7B & 5e-6  & 32 & 0.01&8000& 3 \\ 
\bottomrule[1.5pt]
\end{tabular}%
}
\caption{Model Training Hyperparameter Settings (SFT)}
\label{tab:hyper_sft}
\end{table*}

\begin{table*}[]
\centering
\resizebox{\textwidth}{!}{%
\begin{tabular}{@{}lccccccccc@{}}
\toprule[1.5pt]
 & Model & Learning Rate  & \makecell[c]{Training\\Batch Size} & \makecell[c]{Forward\\Batch Size} & KL Coefficient&Max Length & \makecell[c]{Sampling\\Temperature} &Clip Range &Training Steps \\ 
\midrule[1pt]
& Llama-3.1 &5e-7  & 64& 256 & 0.05&8000& 0.7&0.2&500\\
& Qwen2-7B-Instruct & 5e-7&  64& 256 & 0.05 &6000&0.7 &0.2&500\\\
& Qwen2.5-Math-7B & 5e-7 & 64& 256 & 0.01&8000&0.7 &0.2&500 \\ 
\bottomrule[1.5pt]
\end{tabular}%
}
\caption{Model Training Hyperparameter Settings (RL)}
\label{tab:hyper_rl}
\end{table*}
\begin{table*}
  [t]
  \centering
  \resizebox{\textwidth}{!}{%
  \begin{tabular}{cccccccccccc}
    \toprule \multicolumn{2}{c}{Components}                                                             & \multicolumn{5}{c}{Re-executability Rate (\%)} & \multicolumn{5}{c}{Readability (\#)} \\
    \cmidrule(lr){1-2} \cmidrule(lr){3-7} \cmidrule(lr){8-12}        \hspace{8pt}\labelemoji\hspace{8pt}                                                                & \hspace{8pt}\toolemoji\hspace{8pt}                                      & O0                                 & O1             & O2             & O3             & AVG            & O0             & O1             & O2             & O3             & AVG            \\
    \hline
    \rowcolor[rgb]{0.93,0.93,0.93}\multicolumn{12}{c}{\textbf{Initialize with LLM4Decompile-End-6.7B~\citep{llm4decompile}}}   \\
    \xmark                                                                                              & \xmark                                    & 69.51                              & 46.95          & 50.61          & 46.34          & 53.35          & 3.98 & 3.41 & 3.44 & 3.38 & 3.55 \\
    \cmark                                                                                              & \xmark                                    & 75.61                              & 50.61          & 50.00          & 50.00          & 56.55          & 4.01 & 3.44 & 3.39 & \textbf{3.49} & 3.58 \\
    \xmark                                                                                              & \cmark                                    & 83.54                     & \textbf{56.10}          & 51.22          & 50.61 & 60.37 & 4.05 & 3.51 & 3.51 & 3.42 & 3.62 \\
    \cmark                                                                                              & \cmark                                    & \textbf{85.37}                            & \textbf{56.10}                     & \textbf{51.83} & \textbf{52.43}          & \textbf{61.43} & \textbf{4.13} & \textbf{3.60} & \textbf{3.54} & \textbf{3.49} & \textbf{3.69} \\

    \rowcolor[rgb]{0.93,0.93,0.93}\multicolumn{12}{c}{\textbf{Initialize with Deepseek-Coder-6.7B-base~\citep{deepseekcoder}}} \\
    \xmark                                                                                              & \xmark                                    & 59.15                              & 35.98          & 39.02          & 37.80          & 42.99          & 3.71 & 3.05 & 3.16 & 3.05 & 3.24 \\
    \cmark                                                                                              & \xmark                                    & 66.46                              & 41.46          & 38.41          & 36.59          & 45.73          & 3.76 & 3.17 & \textbf{3.21} & 3.08 & 3.31 \\
    \xmark                                                                                              & \cmark                                    & 70.73                              & 39.63          & 39.02          & 40.24          & 47.41          & 3.90 & 3.17 & 3.08 & 3.11 & 3.31 \\
    \cmark                                                                                              & \cmark                                    & \textbf{79.88}                     & \textbf{45.73} & \textbf{43.90} & \textbf{42.68} & \textbf{53.05} & \textbf{3.96} & \textbf{3.21} & 3.18 & \textbf{3.19} & \textbf{3.38} \\
    \bottomrule
  \end{tabular}%
  }
  \caption{The ablation study of different methods across four optimization levels
  (O0, O1, O2, O3), as well as their average scores (AVG). The results in bold represent the optimal performance. The ~\labelemoji~ and ~\toolemoji~ means Relabedling and Function Call. \textbf{Bold} denotes the best performance.}
  \label{tab:ablation}
\end{table*}
\section{Detailed Results on NLI Tasks}
\label{app:nliresults}

\begin{table*}[h!]
    \centering
    \newcommand{\sig}{$^{\ast}$}
\setlength{\tabcolsep}{0.03in}
\footnotesize
\begin{tabular}{l r@{\hspace{0.10in}} rrrr r@{\hspace{0.10in}} rrrr r@{\hspace{0.10in}} rrrr}
\toprule 
&& \multicolumn{4}{c}{\textbf{RTE}} && \multicolumn{4}{c}{\textbf{SNLI}} && \multicolumn{4}{c}{\textbf{MNLI}} \\ \cmidrule{3-6} \cmidrule{8-11} \cmidrule{13-16}
&& All & $\text{T}_{\text{neg}}$-H & T-$\text{H}_{\text{neg}}$ & $\text{T}_{\text{neg}}$-$\text{H}_{\text{neg}}$ && All & $\text{T}_{\text{neg}}$-H & T-$\text{H}_{\text{neg}}$ & $\text{T}_{\text{neg}}$-$\text{H}_{\text{neg}}$ && All & $\text{T}_{\text{neg}}$-H & T-$\text{H}_{\text{neg}}$ & $\text{T}_{\text{neg}}$-$\text{H}_{\text{neg}}$ \\
\midrule
{BERT-base} && 52.7 & 55.6 & 40.2 & 62.4 && 44.8 & 32.6 & 58.8 & 41.8 && 63.5 & 62.0 & 65.6 & 63.0\\ 
~~~~+ NSPP && 60.7 & 68.6 & 45.0 & 68.2 && 50.9 & 45.0 & 63.6 & 44.2 && 63.8 & 62.4 & 66.0 & 63.0\\ 
~~~~+ NSP && 74.5 & 77.2 & 78.4 & 67.8 && 47.4 & 43.2 & 61.0 & 38.0 && 65.0 & 63.6 & 67.0 & 64.4\\ 
~~~~+ NSPP + NSP && 57.8 & 65.4 & 40.4 & 67.6 && 47.4 & 41.4 & 62.0 & 38.8 && 65.1 & 63.8 & 66.8 & 64.6\\ 
{NBERT-bsae} && 71.1 & 72.8 & 84.6 & 56.0 && 44.8 & 38.0 & 60.6 & 35.8 && 63.8 & 64.2 & 65.5 & 61.4\\
\midrule
{BERT-large} && 53.4 & 57.0 & 40.8 & 62.4 && 50.0 & 41.2 & 62.2 & 46.6 && 67.5 & 65.2 & 71.6 & 65.8\\ 
~~~~+ NSPP && 69.1 & 77.8 & 63.0 & 66.6 && 51.9 & 43.4 & 64.6 & 47.8 && 67.9 & 65.6 & 73.2 & 64.8\\ 
~~~~+ NSP && 78.9 & 80.2 & 91.0 & 65.6 && 52.7 & 43.2 & 68.0 & 47.0 && 67.7 & 65.2 & 72.0 & 65.8\\ 
~~~~+ NSPP + NSP && 59.4 & 56.4 & 54.0 & 67.8 && 52.7 & 44.0 & 67.2 & 47.0 && 67.9 & 65.2 & 72.4 & 65.2\\ 
{NBERT-large} && 74.2 & 78.4 & 80.4 & 64.0 && 50.3 & 47.6 & 60.6 & 42.8 && 66.7 & 65.4 & 70.0 & 64.8\\
\midrule
\midrule
{RoBERTa-base} &&  58.7  &  56.6  &  53.6  &  66.0  &&  55.2  &  49.2  &  68.6  &  47.8  &&  67.1  &  65.4  &  71.0  &  65.0  \\ 
~~~~+ NSPP &&  67.7 & 79.2 & 56.0 & 67.8  &&  55.5  &  48.0  &  69.2  &  49.4  &&  67.5  &  65.0  &  71.0  &  66.6  \\ 
~~~~+ NSP && 78.7 & 86.6 & 78.2 & 71.4  &&  54.6  &  48.2  &  67.8  &  47.8  &&  66.5  &  64.6  &  70.2  &  64.6  \\ 
~~~~+ NSP + NSPP &&  81.0 & 88.0 & 83.6 & 71.4  &&  55.4  &  48.2  &  69.8  &  48.2  &&  68.1  &  66.0  &  72.4  &  66.0  \\
{NRoBERTa-base} && 79.0 & 80.0 & 91.0 & 66.2 && 50.8 & 42.8 & 65.4 & 44.2 && 66.2 & 65.4 & 69.2 & 64.2\\
\midrule
{RoBERTa-large}  &&  84.7  &  90.4  &  87.6  &  76.2  &&  56.0  &  51.4  &  69.4  &  47.2  &&  69.9  &  70.0  &  73.2  &  66.4  \\ 
~~~~+ NSPP &&  81.1  &  83.8  &  84.2  &  75.2  &&  53.6  &  48.2  &  64.8  &  47.8  &&  69.7  &  69.0  &  72.4  &  67.6  \\ 
~~~~+ NSP &&  87.2  &  91.0  &  90.8  &  79.8  &&  56.5  &  50.2  &  70.4  &  48.6  &&  69.9  &  68.6  &  74.2  &  66.8  \\ 
~~~~+ NSP + NSPP &&  75.4  &  90.0  &  70.0  &  67.8  &&  56.1  &  50.4  &  70.0  &  48.0  &&  69.7  &  69.6  &  73.0  &  66.4  \\
{NRoBERTa-large} && 88.4 & 93.4 & 90.6 & 81.2 && 56.5 & 50.4 & 70.4 & 48.8 && 69.6 & 67.4 & 74.2 & 67.2 \\

\bottomrule
\end{tabular}

    \caption{
        Accuracies for each type of the new pairs containing negation for the natural language inference tasks from \citet{hossain-etal-2020-analysis}.
        T and H refer to the text and hypothesis sentences, respectively. We use $\text{T}_{\text{neg}}$ and $\text{H}_{\text{neg}}$ to denote the sentences with negation cues added to the main verb.
        \label{tab:nlidetailedresults}
    }
\end{table*}

\citet{hossain-etal-2020-analysis} created new pairs containing negation by adding negation to the premise or text of the original pairs in the validation sets of the natural language inference tasks.
Table~\ref{tab:nlidetailedresults} shows the accuracies of our models for each type of the new pairs containing negation.
% \section{Results on the NLU Tasks}
\label{app:nluresults}

\begin{table*}
    \centering
    \newcommand{\sig}{$^{\ast}$}
\setlength{\tabcolsep}{0.04in}
\small
\begin{tabular}{l r@{\hspace{0.15in}} rrrrr r@{\hspace{0.15in}} rrr r@{\hspace{0.15in}} rrr}
\toprule
&& \multicolumn{5}{c}{\textbf{QNLI}} && \multicolumn{3}{c}{\textbf{WiC}} && \multicolumn{3}{c}{\textbf{WSC}} \\ \cmidrule{3-7} \cmidrule{9-11} \cmidrule{13-15}
&& \multirow[c]{2}{*}{All} & \multirow[c]{2}{*}{w/o neg} & \multicolumn{3}{c}{w/ neg} && \multirow[c]{2}{*}{All} & \multirow[c]{2}{*}{w/o neg} & \multirow[c]{2}{*}{\ w/ neg} && \multirow[c]{2}{*}{All} & \multirow[c]{2}{*}{w/o neg} & \multirow[c]{2}{*}{\ w/ neg} \\ 
\cmidrule{5-7}
&& & & \ \ \ All & \ \ imp. & unimp. & & & & & & & \\  \midrule
{BERT-base} && 0.88 & 0.88 & 0.88 & 0.65 & 0.89 && 0.69 & 0.71 & 0.59 && 0.52 & 0.44 & 0.60 \\
~~~~+ NSPP && 0.87 & 0.87 & 0.88 & \textbf{0.90} & 0.84 && \textbf{0.71} & \textbf{0.71} & 0.68 && 0.53 & 0.42 & 0.63 \\
~~~~+ NSP  && \textbf{0.91} & \textbf{0.91} & 0.89 & 0.85 & 0.89 && 0.70 & 0.71 & 0.67 && \textbf{0.56} & \textbf{0.48} & \textbf{0.63} \\
~~~~+ NSPP + NSP && 0.91 & 0.91 & \textbf{0.90} & 0.85 & \textbf{0.90} && 0.70 & 0.71 & \textbf{0.68} && 0.52 & 0.42 & 0.62 \\
\midrule
{BERT-large} && 0.89 & 0.90 & 0.87 & \textbf{0.80} & 0.87 && 0.69 & 0.70 & 0.62 && 0.51 & 0.44 & 0.58 \\
~~~~+ NSPP && 0.92 & \textbf{0.93} & 0.90 & 0.70 & 0.90 && 0.69 & \textbf{0.70} & 0.63 && 0.56 & 0.48 & 0.63 \\
~~~~+ NSP  && \textbf{0.92} & 0.93 & \textbf{0.90} & 0.75 & \textbf{0.90} && \textbf{0.69} & 0.69 & \textbf{0.68} && 0.55 & 0.46 & \textbf{0.63} \\
~~~~+ NSPP + NSP && 0.90 & 0.90 & 0.88 & 0.80 & 0.88 && 0.68 & 0.69 & 0.65 && \textbf{0.60} & \textbf{0.54} & 0.65 \\
\midrule
\midrule
{RoBERTa-base} && 0.93 & 0.93 & 0.91 & 0.70 & 0.91 && 0.69 & \textbf{0.70} & 0.62 && 0.61 & 0.58 & 0.63 \\
~~~~+ NSPP && 0.93 & 0.93 & \textbf{0.92} & 0.75 & \textbf{0.92} && 0.68 & 0.68 & 0.67 && 0.63 & 0.65 & 0.62 \\
~~~~+ NSP  && \textbf{0.93} & \textbf{0.93} & 0.92 & \textbf{0.80} & 0.92 && \textbf{0.68} & 0.68 & 0.67 && 0.61 & 0.60 & \textbf{0.63} \\
~~~~+ NSPP + NSP && 0.92 & 0.93 & 0.91 & 0.75 & 0.92 && 0.68 & 0.68 & \textbf{0.71} && \textbf{0.63} & \textbf{0.66} & 0.62 \\
\midrule
{RoBERTa-large} && 0.93 & 0.93 & 0.92 & 0.78 & 0.92 && 0.71 & 0.71 & 0.66 && 0.69 & 0.67 & 0.71 \\
~~~~+ NSPP && 0.94 & 0.94 & 0.93 & 0.95 & 0.93 && 0.71 & 0.71 & 0.65 && 0.66 & 0.58 & 0.75 \\
~~~~+ NSP  && \textbf{0.94} & 0.94 & 0.93 & \textbf{0.95} & 0.93 && 0.71 & 0.72 & 0.67 && \textbf{0.77} & \textbf{0.79} & 0.75 \\
~~~~+ NSPP + NSP && 0.94 & \textbf{0.95} & \textbf{0.93} & 0.95 & \textbf{0.93} && 0.71 & 0.72 & 0.68 && 0.76 & 0.73 & \textbf{0.79} \\
~~~~w/ Affir. Interpret. && 0.94 & 0.94 & 0.92 & 0.89 & 0.92 && \textbf{0.73} & \textbf{0.73} & \textbf{0.70} && 0.71 & 0.68 & 0.75 \\

\bottomrule
\end{tabular}
    \caption{
        Results on the validation sets of natural language understanding tasks.
        All numbers are macro-averaged F1 scores.
        \label{tab:nluresults}
        }
\end{table*}

Table~\ref{tab:nluresults} presents the results on the validation sets of natural language understanding tasks. 
Following prior work, we report macro-averaged F1 scores on the validation sets as some test labels are not publicly available. 
The results demonstrate that further pre-training consistently improves performance on instances containing negation or, at worst, causes a negligible decline 
(a marginal 0.01\% decrease with {RoBERTa-base} on WSC.)
On average, pre-training yields a 3.11\% improvement across all tasks.
Notably, the most significant improvements are observed on WiC with base models
(achieving an average increase of 7.5\%) and on WSC with large models (where performance improves by 6.0\% on average.) 
Importantly, all models pre-trained on NSP or NSPP outperform off-the-shelf versions on important instances in QNLI, 
with the only exceptions being {BERT-large} pre-trained on either NSPP or NSP. 

% \section{Results on LAMA}
\label{app:lamaresults}

% LAMA - RoBERTa
\begin{table*}
    \centering
    \newcommand{\sig}{$^{\ast}$}
\setlength{\tabcolsep}{0.03in}
\small
\begin{tabular}{l cccc}
\toprule
LAMA& SQuAD & ConceptNet & TREx & GoogleRE \\
\midrule

{BERT-base} & 13.11 & 12.71 & \textbf{29.48} & 9.25 \\
~~~~ + NSPP & 12.79 & \textbf{12.72} & 29.01 & \textbf{9.52} \\
~~~~ + NSP & \textbf{14.43} & 12.02 & 28.78 & 9.90 \\
~~~~ + NSPP + NSP & 14.10 & 12.53 & 29.32 & 8.92 \\
\midrule
{BERT-large} &  15.74 & 15.17 & \textbf{30.02} & 9.78 \\
~~~~ + NSPP & 16.72 & \textbf{15.38} & 29.75 & 9.85 \\
~~~~ + NSP & 17.05 & 14.40 & 29.00 & \textbf{10.03} \\
~~~~ + NSPP + NSP & \textbf{17.38} & 14.07 & 28.93 & 9.98 \\

\midrule
\midrule
{RoBERTa-base} & 9.18 & \textbf{14.77} & \textbf{11.93} & 2.77 \\
~~~~ + NSPP & 9.84 & 14.73 & 11.80 & \textbf{2.78} \\
~~~~ + NSP & \textbf{10.16} & 14.42 & 11.28 & 2.44 \\
~~~~ + NSPP + NSP & 8.20 & 12.06 & 6.76 & 2.36 \\
\midrule
{RoBERTa-large} & 13.44 & 18.28 & \textbf{15.48} & 2.24 \\
~~~~ + NSPP & 13.77 & 17.59 & 13.80 & \textbf{2.78} \\
~~~~ + NSP & \textbf{14.10} & \textbf{18.32} & 15.46 & 2.28 \\
~~~~ + NSPP + NSP & 7.54 & 17.34 & 3.68 & 0.64 \\
\bottomrule
\end{tabular}

    \caption{
      We report the mean precision at $k = 1$ on the original LAMA dataset.
      The higher the precision, the better the model.
      Other than {RoBERTa} models jointly pre-trained on both tasks, 
      all our models are within $\pm$1.65\% of the vanilla models. 
      \label{tab:lama-roberta}
    }
  \end{table*}

Table~\ref{tab:lama-roberta} presents the mean precision at $k = 1$ on the original LAMA dataset. 
Except for {RoBERTa} models jointly pre-trained on both NSP and NSPP, 
all other models remain within $\pm$1.65\% of the vanilla models. 
Notably, models pre-trained on NSP and NSPP consistently outperform the vanilla models on {SQuAD} by 0.33\%-1.32\%,
with the sole exception of {BERT-base} pre-trained on NSPP, which performs 0.32\% worse.

On {GoogleRE}, the gains are less pronounced, with improvements ranging from 0.01\% to 0.65\%, 
except for {RoBERTa-base} pre-trained on NSPP, which performs 0.33\% worse. 
For ConceptNet and TREx, the models perform within $-1.65\%$ to $+0.21\%$ of the vanilla models.

It is important to note that LAMA does not contain negated instances, 
so improvements are not necessarily expected. 
However, the fact that the models remain within $\pm$1.65\% of the vanilla models, 
coupled with the substantial improvements on LAMA-neg (Table~\ref{tab:lama-neg}) 
and other corpora, 
demonstrates that the models achieve more robustness to negation while maintaining competitive performance on inputs without negation.


\end{document}