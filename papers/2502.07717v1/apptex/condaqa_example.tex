\section{CondaQA Example}
\label{sec:condaqaexample}
\begin{figure}
  \small
  \begin{tabularx}{0.48 \textwidth}{lX}
      \toprule
      \textbf{Type} & \textbf{Example} \\
      \midrule
      Original & He didn't go to the store, but he went to the park. \\
      Paraphrase & He went to the park but not the store. \\
      Scope & He went to the store, but he didn't go to the park. \\
      Affirmation & He went to the store and the park. \\
      \bottomrule
  \end{tabularx}
  \caption{
      Three types of edits in CondaQA are applied to an example sentence.
      \label{fig:condaqaeditexample}
  }
\end{figure} 



\begin{figure*}
    \small
      \begin{tabularx}{\textwidth}{p{.9in}X} 
        \toprule
        {Original Passage:} & 33\% of the faculty are members of the National Academy of Science or Engineering and/or fellows of the American Academy of Arts and Sciences. This is the highest percentage of any faculty in the country with the exception of the graduate institution Rockefeller University. \\ \addlinespace
        {Original Sentence (with Negation):} & This is the highest percentage of any faculty in the country \textbf{with the exception of} the graduate institution Rockefeller University. \\ \addlinespace
        {Negation Cue:} & with the exception of \\ \addlinespace
        {Question:} & Are the majority of faculty at any school other than Rockefeller University members of the National Academy of Science or Engineering? \\ \addlinespace
        \midrule
        Paraphrase Edit: &  33\% of the faculty are members of the National Academy of Science or Engineering and/or fellows of the American Academy of Arts and Sciences. This is the highest percentage of any faculty in the country \emph{other than} the graduate institution Rockefeller University. \\ \addlinespace
        \midrule
        Scope Edit: &  33\% of the faculty are \emph{not} members of the National Academy of Science or Engineering and/or fellows of the American Academy of Arts and Sciences. This is the highest percentage of any faculty in the country with the exception of the graduate institution Rockefeller University \\ \addlinespace
        \midrule
        Affirmation Edit: &  33\% of the faculty are members of the National Academy of Science or Engineering and/or fellows of the American Academy of Arts and Sciences. This is the highest percentage of any faculty in the country \emph{including} the graduate institution Rockefeller University. \\ \addlinespace
        \midrule
      \end{tabularx}
      \begin{tabularx}{\textwidth}{l@{\hspace{0.28 \textwidth}}l}
        \midrule
            Input & Answer \\
        \midrule
            Question + Original Passage & No \\
            Question + Paraphrase Edit & No \\
            Question + Scope Edit & Yes \\
            Question + Affirmation Edit & No \\
        \bottomrule
      \end{tabularx}
        \caption{
          \label{fig:condaqasample}
          An example from CondaQA.
          The original passage contains a sentence with negation.
          The crowdworker makes three edits to the passage (paraphrase, scope, and affirmation edits) to create the edited passage.
          The question (also written by the crowdworker) asks about the majority of faculty (more than 50\%) at any school other than Rockefeller University.
          Changing the scope of negation changes the answer to the question from \emph{No} to \emph{Yes}.
        }
\end{figure*}
Figure~\ref{fig:condaqaeditexample} shows an example sentence with the three types of edits.
We also provide an example from CondaQA in Figure~\ref{fig:condaqasample}.
The original passage has been selected from the English Wikipedia and contains a sentence with negation.
Three edits are made to the passage to create the edited passage: 
a paraphrase edit (i.e. \emph{rewriting the sentence}), a scope edit (i.e. \emph{changing the scope of negation}), and an affirmation edit (i.e. \emph{undoing negation}).
The question is answered based on the original and edited passages (a group). 
A model needs to answer all the questions in a group correctly to achieve group consistency. 