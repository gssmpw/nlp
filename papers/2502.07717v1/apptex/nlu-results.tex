\section{Results on the NLU Tasks}
\label{app:nluresults}

\begin{table*}
    \centering
    \newcommand{\sig}{$^{\ast}$}
\setlength{\tabcolsep}{0.04in}
\small
\begin{tabular}{l r@{\hspace{0.15in}} rrrrr r@{\hspace{0.15in}} rrr r@{\hspace{0.15in}} rrr}
\toprule
&& \multicolumn{5}{c}{\textbf{QNLI}} && \multicolumn{3}{c}{\textbf{WiC}} && \multicolumn{3}{c}{\textbf{WSC}} \\ \cmidrule{3-7} \cmidrule{9-11} \cmidrule{13-15}
&& \multirow[c]{2}{*}{All} & \multirow[c]{2}{*}{w/o neg} & \multicolumn{3}{c}{w/ neg} && \multirow[c]{2}{*}{All} & \multirow[c]{2}{*}{w/o neg} & \multirow[c]{2}{*}{\ w/ neg} && \multirow[c]{2}{*}{All} & \multirow[c]{2}{*}{w/o neg} & \multirow[c]{2}{*}{\ w/ neg} \\ 
\cmidrule{5-7}
&& & & \ \ \ All & \ \ imp. & unimp. & & & & & & & \\  \midrule
{BERT-base} && 0.88 & 0.88 & 0.88 & 0.65 & 0.89 && 0.69 & 0.71 & 0.59 && 0.52 & 0.44 & 0.60 \\
~~~~+ NSPP && 0.87 & 0.87 & 0.88 & \textbf{0.90} & 0.84 && \textbf{0.71} & \textbf{0.71} & 0.68 && 0.53 & 0.42 & 0.63 \\
~~~~+ NSP  && \textbf{0.91} & \textbf{0.91} & 0.89 & 0.85 & 0.89 && 0.70 & 0.71 & 0.67 && \textbf{0.56} & \textbf{0.48} & \textbf{0.63} \\
~~~~+ NSPP + NSP && 0.91 & 0.91 & \textbf{0.90} & 0.85 & \textbf{0.90} && 0.70 & 0.71 & \textbf{0.68} && 0.52 & 0.42 & 0.62 \\
\midrule
{BERT-large} && 0.89 & 0.90 & 0.87 & \textbf{0.80} & 0.87 && 0.69 & 0.70 & 0.62 && 0.51 & 0.44 & 0.58 \\
~~~~+ NSPP && 0.92 & \textbf{0.93} & 0.90 & 0.70 & 0.90 && 0.69 & \textbf{0.70} & 0.63 && 0.56 & 0.48 & 0.63 \\
~~~~+ NSP  && \textbf{0.92} & 0.93 & \textbf{0.90} & 0.75 & \textbf{0.90} && \textbf{0.69} & 0.69 & \textbf{0.68} && 0.55 & 0.46 & \textbf{0.63} \\
~~~~+ NSPP + NSP && 0.90 & 0.90 & 0.88 & 0.80 & 0.88 && 0.68 & 0.69 & 0.65 && \textbf{0.60} & \textbf{0.54} & 0.65 \\
\midrule
\midrule
{RoBERTa-base} && 0.93 & 0.93 & 0.91 & 0.70 & 0.91 && 0.69 & \textbf{0.70} & 0.62 && 0.61 & 0.58 & 0.63 \\
~~~~+ NSPP && 0.93 & 0.93 & \textbf{0.92} & 0.75 & \textbf{0.92} && 0.68 & 0.68 & 0.67 && 0.63 & 0.65 & 0.62 \\
~~~~+ NSP  && \textbf{0.93} & \textbf{0.93} & 0.92 & \textbf{0.80} & 0.92 && \textbf{0.68} & 0.68 & 0.67 && 0.61 & 0.60 & \textbf{0.63} \\
~~~~+ NSPP + NSP && 0.92 & 0.93 & 0.91 & 0.75 & 0.92 && 0.68 & 0.68 & \textbf{0.71} && \textbf{0.63} & \textbf{0.66} & 0.62 \\
\midrule
{RoBERTa-large} && 0.93 & 0.93 & 0.92 & 0.78 & 0.92 && 0.71 & 0.71 & 0.66 && 0.69 & 0.67 & 0.71 \\
~~~~+ NSPP && 0.94 & 0.94 & 0.93 & 0.95 & 0.93 && 0.71 & 0.71 & 0.65 && 0.66 & 0.58 & 0.75 \\
~~~~+ NSP  && \textbf{0.94} & 0.94 & 0.93 & \textbf{0.95} & 0.93 && 0.71 & 0.72 & 0.67 && \textbf{0.77} & \textbf{0.79} & 0.75 \\
~~~~+ NSPP + NSP && 0.94 & \textbf{0.95} & \textbf{0.93} & 0.95 & \textbf{0.93} && 0.71 & 0.72 & 0.68 && 0.76 & 0.73 & \textbf{0.79} \\
~~~~w/ Affir. Interpret. && 0.94 & 0.94 & 0.92 & 0.89 & 0.92 && \textbf{0.73} & \textbf{0.73} & \textbf{0.70} && 0.71 & 0.68 & 0.75 \\

\bottomrule
\end{tabular}
    \caption{
        Results on the validation sets of natural language understanding tasks.
        All numbers are macro-averaged F1 scores.
        \label{tab:nluresults}
        }
\end{table*}

Table~\ref{tab:nluresults} presents the results on the validation sets of natural language understanding tasks. 
Following prior work, we report macro-averaged F1 scores on the validation sets as some test labels are not publicly available. 
The results demonstrate that further pre-training consistently improves performance on instances containing negation or, at worst, causes a negligible decline 
(a marginal 0.01\% decrease with {RoBERTa-base} on WSC.)
On average, pre-training yields a 3.11\% improvement across all tasks.
Notably, the most significant improvements are observed on WiC with base models
(achieving an average increase of 7.5\%) and on WSC with large models (where performance improves by 6.0\% on average.) 
Importantly, all models pre-trained on NSP or NSPP outperform off-the-shelf versions on important instances in QNLI, 
with the only exceptions being {BERT-large} pre-trained on either NSPP or NSP. 
