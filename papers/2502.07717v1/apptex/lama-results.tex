\section{Results on LAMA}
\label{app:lamaresults}

% LAMA - RoBERTa
\begin{table*}
    \centering
    \newcommand{\sig}{$^{\ast}$}
\setlength{\tabcolsep}{0.03in}
\small
\begin{tabular}{l cccc}
\toprule
LAMA& SQuAD & ConceptNet & TREx & GoogleRE \\
\midrule

{BERT-base} & 13.11 & 12.71 & \textbf{29.48} & 9.25 \\
~~~~ + NSPP & 12.79 & \textbf{12.72} & 29.01 & \textbf{9.52} \\
~~~~ + NSP & \textbf{14.43} & 12.02 & 28.78 & 9.90 \\
~~~~ + NSPP + NSP & 14.10 & 12.53 & 29.32 & 8.92 \\
\midrule
{BERT-large} &  15.74 & 15.17 & \textbf{30.02} & 9.78 \\
~~~~ + NSPP & 16.72 & \textbf{15.38} & 29.75 & 9.85 \\
~~~~ + NSP & 17.05 & 14.40 & 29.00 & \textbf{10.03} \\
~~~~ + NSPP + NSP & \textbf{17.38} & 14.07 & 28.93 & 9.98 \\

\midrule
\midrule
{RoBERTa-base} & 9.18 & \textbf{14.77} & \textbf{11.93} & 2.77 \\
~~~~ + NSPP & 9.84 & 14.73 & 11.80 & \textbf{2.78} \\
~~~~ + NSP & \textbf{10.16} & 14.42 & 11.28 & 2.44 \\
~~~~ + NSPP + NSP & 8.20 & 12.06 & 6.76 & 2.36 \\
\midrule
{RoBERTa-large} & 13.44 & 18.28 & \textbf{15.48} & 2.24 \\
~~~~ + NSPP & 13.77 & 17.59 & 13.80 & \textbf{2.78} \\
~~~~ + NSP & \textbf{14.10} & \textbf{18.32} & 15.46 & 2.28 \\
~~~~ + NSPP + NSP & 7.54 & 17.34 & 3.68 & 0.64 \\
\bottomrule
\end{tabular}

    \caption{
      We report the mean precision at $k = 1$ on the original LAMA dataset.
      The higher the precision, the better the model.
      Other than {RoBERTa} models jointly pre-trained on both tasks, 
      all our models are within $\pm$1.65\% of the vanilla models. 
      \label{tab:lama-roberta}
    }
  \end{table*}

Table~\ref{tab:lama-roberta} presents the mean precision at $k = 1$ on the original LAMA dataset. 
Except for {RoBERTa} models jointly pre-trained on both NSP and NSPP, 
all other models remain within $\pm$1.65\% of the vanilla models. 
Notably, models pre-trained on NSP and NSPP consistently outperform the vanilla models on {SQuAD} by 0.33\%-1.32\%,
with the sole exception of {BERT-base} pre-trained on NSPP, which performs 0.32\% worse.

On {GoogleRE}, the gains are less pronounced, with improvements ranging from 0.01\% to 0.65\%, 
except for {RoBERTa-base} pre-trained on NSPP, which performs 0.33\% worse. 
For ConceptNet and TREx, the models perform within $-1.65\%$ to $+0.21\%$ of the vanilla models.

It is important to note that LAMA does not contain negated instances, 
so improvements are not necessarily expected. 
However, the fact that the models remain within $\pm$1.65\% of the vanilla models, 
coupled with the substantial improvements on LAMA-neg (Table~\ref{tab:lama-neg}) 
and other corpora, 
demonstrates that the models achieve more robustness to negation while maintaining competitive performance on inputs without negation.