LLMの推論能力を向上させることはLLM研究の大きなトピックであり、Few-shot learning~\citep{Brown2020-bw}、Chain-of-thought~\citep{Wei2022-kt}やself-consistency~\citep{Wang2023-pf}、また、chain-of-tree~\citep{Yao2023-ey}などの代表的なプロンプティングの技術により、その推論能力が改善されることが知られており、さらに、近年は、Self-Refine~\citep{Madaan2023-fn}やRecursive Criticism and Improvement (RCI)~citep{Kim2023-zz}などといった、framework where an LLM generates an initial answer, then critiques that answer and refines it repeatedly using its own feedback to enhance the llm's reasoning ability, also, similar approach is used to mitigate and hallcination problems~citep{Manakul2023-iy}。より、agenticな設定ではReflexion~\citep{Shinn2023-no}がenvironmentとの相互作用の中でactionをよりよくするシステムを採用している。また、人間のシステム1・2~citep{EVANS2003454}を模倣して推論能力を向上させるアイディアもあり~citep{Huang2022-mv, Zhu2023-wn}や、複数エージェントのディベートを通して回答の質を向上させる枠組みにも注目が集まっている\citep{Chen2024-ua, Du2024-pi, Smit2024-cn}、推論能力を向上させることに加えて、さらに、ソフトウエアの開発\citep{Qian2024-gr}や研究の自動化\citep{Lu2024-za,Liu2024-yg}

# training
