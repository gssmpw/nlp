\section{Introduction}


%para1: Cardiac Arrest in Pediatric Population (CICU)
Cardiac arrest (CA) in the pediatric population is a critical and life-threatening event with significant implications for morbidity and mortality~\cite{frazier2021risk}. While less common than in adults, pediatric CA is often associated with a preceding period of physiological instability, making timely identification and intervention crucial. Many cases of CA in children occur in intensive care units (ICUs)~\cite{zeng2020pic}, particularly cardiac ICUs~\cite{morgan2021cicu}, where children with congenital or acquired heart diseases are managed. In these high-stakes environments, the interplay of complex cardiac conditions, surgical interventions, and critical care therapies, coupled with significant cognitive and data overload,
creates a unique set of challenges for clinicians~\cite{winters2018society}. Early prediction of CA~\cite{brown2025MLCA} in this setting could enable proactive measures to prevent arrests and provide situational awareness. This underscores the importance of developing robust predictive models tailored to the pediatric cardiac ICU population, integrating clinical, physiological, and potentially novel data sources.

%para2: challenges of EHR based CA risk prediction. 1. heterogeneous data modalities; 2. multi-resolution.
Electronic Health Records (EHRs) offer a rich repository of patient data~\cite{zeng2020pic}, making them an attractive resource for cardiac arrest risk prediction. By aggregating diverse information such as demographics, vital signs, laboratory results, and clinical notes, EHRs provide a comprehensive view of a patient’s health trajectory. However, leveraging EHR data for pediatric CA risk prediction is not without challenges~\cite{wornow2023ehrshot}. 
First, EHRs are inherently \textit{heterogeneous}\footnote[2]{Heterogeneous: while many works use the term “multivariate” to refer to the complex input features in the context of time-series models, here we use “heterogeneous” to emphasize two levels of heterogeneity: (1) static versus dynamic risk factors, and (2) categorical, numerical, and textual input formats.}, encompassing various modalities that differ in format and clinical significance. For examples, demographics are stored in the static categorical format that serve as underlying risk factors, whereas vital signs are recorded as longitudinal numerical measurements that reflect short term physiological status.
Second, EHR data are \textit{multi-resolution}: some measurements, like vital signs, are updated in near real-time, while others, such as lab results, are recorded at more sparse intervals. 
Traditionally, clinicians have relied on manually developed criteria and scoring systems (\textit{e.g.}, PEWS~\cite{monaghan2005pews}, IDO2~\cite{loomba2023inadequate}) to predict cardiac arrest. While these systems are valuable, they may not fully capture the subtle, individual-specific nuances of a patient’s condition because they depend on universally predefined thresholds applied to a fixed set of clinical observations. Moreover, certain systems, such as IDO2, are proprietary, limiting their accessibility without incurring additional costs.
In contrast, data-driven approaches leverage EHR data through advanced artificial intelligence (AI) techniques, leading to improved performance in early detection and risk stratification of pediatric cardiac arrest~\cite{brown2025MLCA}.


% para 3: gap of existing ML. Tabular model can handle heterogenity, but not multi-resolution. Time-series model can handle multi-resolution to some degree, but not much. Our solution is use fusion transformer.
The heterogeneous and multi-resolution nature of EHR poses significant challenges to AI models for EHR-based pediatric CA risk prediction. 
Conventional AI models (e.g., random forest, XGBoost~\cite{chen2016xgboost}), which are designed for static tabular data, excel at handling high-dimensional numerical and categorical risk factors by capturing subtle and non-linear patterns indicative of impending cardiac arrest. However, these models operate on fixed-dimensional input features and typically require ad-hoc, domain-specific feature preprocessing—such as aggregation~\cite{rajkomar2018countbase}—which can overlook important dynamic temporal changes in risk factors. 
On the other hand, time-series AI models~\cite{goswami2024momentfamilyopentimeseries} are well-suited for analyzing longitudinal data, as they can capture temporal dependencies and trends. Although resolution unification techniques (\textit{e.g.} resampling, padding, interpolation)~\cite{rasul2023lag,ekambaram2024tiny} can address the multi-resolution challenges inherent in EHR data, time-series models often struggle to effectively handle high-dimensional inputs.

% para4: Introduce our model that address both challenges.
To address the above mentioned challenges, we propose our innovative approach, namely \textbf{PED}iatric \textbf{C}ardiac \textbf{A}rrest prediction via \textbf{F}used \textbf{T}ransformer (\modelname), based on a novel tabular-textual multimodal fusion strategy~\cite{shi2021multimodal,lu2023mug}, and the powerful Transformer backbone model for each modality.
We view the heterogeneous and multi-resolution risk factors as two complementary modalities, tabular and textual, each characterized by its own intrinsic data structure. Specifically, we employ a dedicated tabular Transformer~\cite{huang2020tabtransformer,gorishniy2021FT-Trans} to effectively handle high-dimensional static and aggregated longitudinal tabular features, while a pre-trained textual EHR Transformer~\cite{yang2022gatortron} processes the textual representations~\cite{contreras2024dellirium} derived from the original EHR data. A fusion Transformer is then used to integrate the modality-specific representations, ultimately computing the probability of CA onset risk.
We evaluate the effectiveness of our \modelname on a curated cohort from the CHOA-CICU database comprising 3,566 pediatric patients with a $4.0\%$ incidence of CA. Our model, along with ten other AI models, is compared using 5-fold cross-validation. Through this comprehensive evaluation, our proposed approach marginally outperforms all compared models. Furthermore, a feature importance analysis reveals that several of the identified risk factors align well with clinical knowledge for pediatric CA, as confirmed by our clinical collaborators.


%This multi-resolution nature complicates the integration and temporal alignment of the data, posing significant challenges to conventional machine learning models designed for static, tabular data. Overcoming these challenges is essential for developing accurate and reliable predictive models tailored to the dynamic and complex environment of the pediatric cardiac ICU.


% para2: Practise in healthcare. ML model.
%Accurate and timely prediction of cardiac arrest in pediatric patients has been a focus of clinical research, evolving from traditional clinical scoring systems to modern machine learning-based models. Conventional approaches, such as early warning scores (e.g., Pediatric Early Warning Scores~\cite{monaghan2005pews}), rely on predefined thresholds for vital signs and clinical observations. While these tools are easy to implement, they often lack the sensitivity and specificity needed for highly complex cases, particularly in the cardiac ICU setting. Recent advances in computational methods have introduced machine learning and deep learning models that can analyze high-dimensional and multimodal data, including physiological waveforms, laboratory results, and clinical notes. These modern models promise improved predictive performance by capturing subtle, non-linear patterns indicative of impending cardiac arrest. However, their successful application in pediatric populations requires addressing challenges such as data sparsity, variability in patient physiology, and the need for interpretable outputs to support clinical decision-making.


