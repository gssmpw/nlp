\section{Study Design}
\label{sec:study}
\subsection{Problem Definition}
We are interested in the early risk prediction of cardiac arrest for pediatric ICU population.  
\begin{definition}[Patient EHR Data]
The EHR data $\mathbf{x}$ of one patient include static risk factors $\mathbf{x}^{(s)} \in \mathbb{R}^{d_s}$ during one admission such as demographics and admission diagnosis, and temporal risk factors $\mathbf{x}^{(t)}_{0:T} \in \mathbb{R}^{d_t\times T}$ such as patient's lab results, vital signs, and nursing assessments. Therefore, the EHR data within first $\tau$ hours is defined by $\mathbf{x}_{{t\leq \tau}}=(\mathbf{x}^{(s)}, \mathbf{x}^{(t)}_{0:\tau})$, and $|\mathbf{x}|$ denotes overall number of risk factors contained in EHR data. 
\end{definition}

\begin{definition}[Early Risk Prediction of Cardiac Arrest]
We aim to employ a predictive model $f_\theta(\cdot)$ using a patient EHR data $\mathbf{x}_{t\leq \tau}$ available in the first $\tau$ hours of admission, to predict the probability of developing cardiac arrest $\hat{y}=f_\theta(\mathbf{x}_{t\leq \tau})$, where $\hat{y} \in \{0, 1\}$ during the rest of the admission stay.
\end{definition}


\subsection{Data Source and Cohort Selection}
CHOA-CICU database is a private pediatric intensive care database extracted from electronic health record for all pediatric patients (age less than 18 years) admitted from 1/1/2018 to 12/31/2023 to a large, quaternary, academic Pediatric Cardiac Intensive Care Unit (CICU) at Children's Hospital at Atlanta and Emory University. 
We obtain 9 static risk factors from CHOA EHR, including: ``gender, ethnicity, first race, second race, primary language, respiration distress, age, admission diagnoses, admission ICD 19 codes''. Among 184 temporal risk factors, we obtain 129 vital signs and nursing assessments from flow sheet after removing non-relevant items and removing missing rate $> 0.5$ items. We also obtain 49 lab test results after removing missing rate $> 0.5$ items, and 5 medications after removing missing rate $> 0.7$ items.
Table~\ref{tab:data_stat} briefly presents the number of patients included in the two data sources. 
%We further conduct cohort selection to exclude certain patients.

\begin{table}[ht]
\centering
\caption{Statistics of CICU patients After Cohort Selection.}
%Dx denotes diagnosis, Vs denotes vital signs, Rx denotes prescriptions, Px denotes procedures, Sx denotes symptoms.}
\label{tab:data_stat}
\begin{tabular}{c|c}
\toprule
Category & Value\\
\midrule
Patient, no. (\% of case)& 3,566 (4.0\%) \\
Admissions, no. (\% of case) & 4,672 (3.1\%)\\
\hline
Age, median years (Q1--Q3) & 0.6 (0.1 -- 5.5)\\
Gender, no. of female (\%) & 1,676 (47.0\%)\\
CICU LoS, median days (Q1--Q3) & 8.4 (4.3 - 20.0)\\
HLoS, median days (Q1--Q3) & 2.0 (0.9 - 5.2)\\
Mortality, no., (\% of Admission) & 185 (4.0\%)\\
\hline
Static Risk Factors, no.& 9\\
Temporal Risk Factors, no. & 184\\
-(Tempo) Vital\&Assessments, no. & 129\\
-(Tempo) Lab Results, no. & 49\\
-(Tempo) Medications, no. & 5\\
\bottomrule
\end{tabular}
\vspace{-0.35cm}
\end{table}

% \begin{table*}[ht!]
% \centering
% \caption{Statistics of CA patients After Cohort Selection. Dx denotes diagnosis, Vs denotes vital signs, Rx denotes prescriptions, Px denotes procedures, Sx denotes symptoms. \jiaying{we want to change to a single column table. With a bit more patient characteristics: length of stay.}}
% %\label{tab:data_stat}
% \begin{tabular}{c|ccc|cc|p{3cm}}
% \toprule
% data source& CA Pat. (Adm.) & Ctrl Pat. (Adm.) & Incidence Pat.\% (Adm.\%) & Age-Days Mean (Q1–-Q3) & \%Female & Risk Factors \\
% \midrule
% %PIC & 358 (359) & 11,645 (12,168) & 3.0\% (2.9\%) &  916 (40 -- 1256)& 42.6\% & (static) Demogr, adm\_Dx; (ts) Vs, Lab, Rx, OR\_Px\\
% \hline
% %CHOA-CICU & 176 (177) & 7,484 (17,327) & 2.30\% (1.01\%) & & & (static) Demogr, adm\_Dx; (ts) Vs, Lab, Rx\\ 
% CHOA-CICU & 143 (146) & 3,423 (4,526) &  4.0\% (3.1\%) & 231 (36 -- 2,008) & 47.0\%& (static) Demogr, adm\_Dx; (ts) Vs, Lab, Med \\
% \bottomrule
% \end{tabular}
% \end{table*}



