%%%%%%%%%%%%%%%%%%%%%%%%%%%%%%%%%%%%%%%%%%%%%%%%%%%%%%%%%%%%%%%%%%%%%%%%%%%%%%%%
%2345678901234567890123456789012345678901234567890123456789012345678901234567890
%        1         2         3         4         5         6         7         8

\documentclass[letterpaper, 10 pt, conference]{ieeeconf}  % Comment this line out if you need a4paper

%\documentclass[a4paper, 10pt, conference]{ieeeconf}      % Use this line for a4 paper

\IEEEoverridecommandlockouts                              % This command is only needed if 
                                                          % you want to use the \thanks command

\overrideIEEEmargins                                      % Needed to meet printer requirements.

%In case you encounter the following error:
%Error 1010 The PDF file may be corrupt (unable to open PDF file) OR
%Error 1000 An error occurred while parsing a contents stream. Unable to analyze the PDF file.
%This is a known problem with pdfLaTeX conversion filter. The file cannot be opened with acrobat reader
%Please use one of the alternatives below to circumvent this error by uncommenting one or the other
%\pdfobjcompresslevel=0
%\pdfminorversion=4

% See the \addtolength command later in the file to balance the column lengths
% on the last page of the document

% The following packages can be found on http:\\www.ctan.org
\usepackage{amsfonts}
\usepackage{amsmath}
\usepackage{bbold}
\usepackage{algorithmic}
\usepackage{graphicx}
\usepackage{textcomp}
\usepackage{xcolor}
% custom pacakges
\usepackage{pifont}
\usepackage{multirow}
\usepackage{booktabs}
\usepackage{pgfplots}
\usepackage{subcaption}
\usepackage{xspace}
\usepackage{tikz}
\pgfplotsset{compat=1.18}
\usepgfplotslibrary{colorbrewer}
% custom macros
\newcommand\doubleplus{+\kern-1.3ex+\kern0.8ex}
\newtheorem{definition}{Definition}
\newcommand{\jiaying}[1]{\textcolor{orange}{Jiaying: #1}}
\newcommand{\modelname}[0]{\textsc{PedCA-FT}\xspace}

% 7 pages, single blind, including reference.
\title{Early Risk Prediction of Pediatric Cardiac Arrest from Electronic Health Records via Multimodal Fused Transformer}

\author{Jiaying Lu\textsuperscript{*}$^{1}$\textsuperscript{\ding{41}}, Stephanie R. Brown\textsuperscript{*}$^{2}$, Songyuan Liu$^{1}$, Shifan Zhao$^{3}$, Kejun Dong$^{1}$, Del Bold$^{1}$,\\ 
 Michael Fundora$^{2}$, Alaa Aljiffry$^{2}$, Alex Fedorov$^{1}$, Jocelyn Grunwell$^{2}$ and Xiao Hu$^{1,4}$% <-this % stops a space
\thanks{\textsuperscript{\ding{41}}Corresponding Author: Jiaying Lu, Ph.D. (jiaying.lu@emory.edu)}% 
\thanks{\textsuperscript{*}These authors contributed equally to this work}
\thanks{$^{1}$Center for Data Science, Nell Hodgson Woodruff School of Nursing, Emory University, Atlanta, GA 30322, USA}%
\thanks{$^{2}$Department of Pediatrics, Children's Healthcare of Atlanta Cardiology, School of Medicine, Emory University, Atlanta, GA 30322, USA}%
\thanks{$^{3}$Department of Mathematics, Laney Graduate School, Emory University, Atlanta, GA 30322, USA}%
\thanks{$^{4}$The Wallace H. Coulter Department of Biomedical Engineering, Georgia Institute of Technology \& Emory University, Atlanta, GA 30322, USA}%
}

\begin{document}

\maketitle
\thispagestyle{empty}
\pagestyle{empty}


%%%%%%%%%%%%%%%%%%%%%%%%%%%%%%%%%%%%%%%%%%%%%%%%%%%%%%%%%%%%%%%%%%%%%%%%%%%%%%%%
%7 pages w/ references. Reviews are traditional single blind peer reviews where the identity of the reviewer is not visible to the author(s).
\begin{abstract}
Early prediction of pediatric cardiac arrest (CA) is critical for timely intervention in high-risk intensive care settings. We introduce \modelname, a novel transformer-based framework that fuses tabular view of EHR with the derived textual view of EHR to fully unleash the interactions of high-dimensional risk factors and their dynamics. By employing dedicated transformer modules for each modality view, \modelname captures complex temporal and contextual patterns to produce robust CA risk estimates. Evaluated on a curated pediatric cohort from the CHOA-CICU database, our approach outperforms ten other artificial intelligence models across five key performance metrics and identifies clinically meaningful risk factors. These findings underscore the potential of multimodal fusion techniques to enhance early CA detection and improve patient care.
\end{abstract}


%%%%%%%%%%%%%%%%%%%%%%%%%%%%%%%%%%%%%%%%%%%%%%%%%%%%%%%%%%%%%%%%%%%%%%%%%%%%%%%%
\renewcommand*{\thefootnote}{\fnsymbol{footnote}}

% main sections
\section{Introduction}
\label{sec:intro}

\begin{figure*}[tb]
    \centering
    \includegraphics[width=0.848\linewidth]{figs/circuitnn.pdf} 
    \caption{Illustration of differentiable CircuitNN. CircuitNN is designed based on differentiable NAND gates. After DAS is guided by PI and PO pairs of the truth table, CircuitNN can get the precise circuit architecture logic equivalent to the truth table.}
    \label{fig:circuitnn}
\end{figure*}

% 1. Describe the importance of logic synthesis
% 2. Existing Problems
% (a) Neural Architecture Search: Unstable, Predefined Setting, etc.
% (b) Circuit Generation: Probabilistic Model, Logic Equivalence

With the rapid advancement of technology, the scale of integrated circuits (ICs) has expanded exponentially. 
This expansion has introduced significant challenges in chip manufacturing, particularly concerning power and area metrics.
A primary objective in IC design is achieving the same circuit function with fewer transistors, thereby reducing power usage and area occupancy.

Logic synthesis~\cite{hachtel2005logicsynth}, a critical step in electronic design automation (EDA), transforms behavioral-level circuit designs into optimized gate-level circuits, ultimately yielding the final IC layout. 
The primary goal of logic synthesis is to identify the physical implementation with the fewest gates for a given circuit function. 
This task constitutes a challenging NP-hard combinatorial optimization problem. 
Current logic synthesis tools~\cite{brayton2010abc, wolf2013yosys} rely on human-designed heuristics, often leading to sub-optimal outcomes.

Differentiable architecture search (DAS) techniques~\cite{liu2018darts, chu2020darts} offer novel perspectives on addressing challenges in this problem.
Circuit functions can be represented through truth tables, which map binary inputs to their corresponding outputs. 
Truth tables provide a precise representation of input-output relationships, ensuring the design of functionally equivalent circuits.
Inspired by this, researchers~\cite{deepmind2024ai4sys, wang2024tnet} have begun exploring the application of DAS to synthesize circuits directly from truth tables.
Specifically, \citet{deepmind2024ai4sys} proposed CircuitNN, a framework that learns differentiable connection structures with logic gates, enabling the automatic generation of logic circuits from truth tables.
This approach significantly reduces the complexity of traditional circuit generation. 
Building on this, \citet{wang2024tnet} introduced T-Net, a triangle-shaped variant of CircuitNN, incorporating regularization techniques to enhance the efficiency of DAS.

Despite these advancements, several challenges remain. 
The computational complexity of DAS grows quadratically with the number of gates, posing scalability issues.
Although triangle-shaped architecture~\cite{wang2024tnet} partially mitigates this problem, redundancy persists. 
%Additionally, DAS is susceptible to converging to local optima, limiting the ability to search architectures that satisfy the given truth tables~\cite{liu2018darts}. 
%Furthermore, hyperparameters (network depth and layer width) require extensive searches, introducing complexity and prolonging the synthesis process. 
Additionally, DAS is susceptible to converging to local optima~\cite{liu2018darts} and hyperparameters (network depth and layer width) require extensive searches. 
The challenges arise from the vast search space in DAS. 
% Even with predefined settings for CircuitNN, finding a configuration that meets the truth table requires extensive trial and error during the DAS process. 
Intuitively, limiting the search space through predefined parameters (network depth, gates per layer, and connection probabilities) can significantly reduce the complexity.

Recent advances~\cite{openai2023gpt4, abramson2024alphafold3, esser2024sd3, li2024mar} in conditional generative models have demonstrated remarkable performance across language, vision, and graph generation tasks. 
Motivated by these developments, we propose a novel approach to circuit generation that generates preliminary circuit structures to guide DAS in generating refined circuits matching specified truth tables. 
Firstly, we introduce CircuitVQ, a tokenizer with a discrete codebook for circuit tokenization. 
Built upon our Circuit AutoEncoder framework~\cite{hou2022graphmae,li2023maskgae,wu2025mgvga}, CircuitVQ is trained through a circuit reconstruction task. 
Specifically, the CircuitVQ encoder encodes input circuits into discrete tokens using a learnable codebook, while the decoder reconstructs the circuit adjacency matrix based on these tokens.
Subsequently, the CircuitVQ encoder serves as a circuit tokenizer for CircuitAR pretraining, which employs a masked autoregressive modeling paradigm~\cite{chang2022maskgit, li2023mage}. 
In this process, the discrete codes function as supervision signals. 
After training, CircuitAR can generate discrete tokens progressively, which can be decoded into initial circuit structures by the decoder of the CircuitVQ. 
These prior insights can guide DAS in producing refined circuits that match the target truth tables precisely.

Our key contributions can be summarized as follows:
\begin{itemize}
\item We introduce CircuitVQ, a circuit tokenizer that facilitates graph autoregressive modeling for circuit generation, based on our Circuit AutoEncoder framework;
\item Develop CircuitAR, a model trained using masked autoregressive modeling, which generates initial circuit structures conditioned on given truth tables;
\item Propose a refinement framework that integrates differentiable architecture search to produce functionally equivalent circuits guided by target truth tables;
\item Comprehensive experiments demonstrating the scalability and capability emergence of our CircuitAR and the superior performance of the proposed circuit generation approach.
\end{itemize}

% Motivation
% (a) Diffusion (Vision, Graph), Autoregressive (Language, Vision)
% (b) Circuit Generation for Predefined Setting
% (c) Neural Architecture Search for Strict Logic Equivalence

% Contribution
% (a) Circuit Tokenizer (new transformer arch, training strategy)
% (b) CircuitAR (train and gen strategies, post-ar strategy)
% (c) Extensive Evaluation including BitD (Bit Distance) for Scalability

%\section{Related Work}
% \subsection{Vision Language Model}
% 시각장애인에서 상황을 설명할 DB가 없으니 만들었다. 그리고 이를 VLM에 튜닝했다.
\subsection{Technical approaches for assisting the visually-impaired}


\subsection{Datasets for visual instruction tuning}
 % save space to remove related work
\section{Study Design}
\label{sec:study}
\subsection{Problem Definition}
We are interested in the early risk prediction of cardiac arrest for pediatric ICU population.  
\begin{definition}[Patient EHR Data]
The EHR data $\mathbf{x}$ of one patient include static risk factors $\mathbf{x}^{(s)} \in \mathbb{R}^{d_s}$ during one admission such as demographics and admission diagnosis, and temporal risk factors $\mathbf{x}^{(t)}_{0:T} \in \mathbb{R}^{d_t\times T}$ such as patient's lab results, vital signs, and nursing assessments. Therefore, the EHR data within first $\tau$ hours is defined by $\mathbf{x}_{{t\leq \tau}}=(\mathbf{x}^{(s)}, \mathbf{x}^{(t)}_{0:\tau})$, and $|\mathbf{x}|$ denotes overall number of risk factors contained in EHR data. 
\end{definition}

\begin{definition}[Early Risk Prediction of Cardiac Arrest]
We aim to employ a predictive model $f_\theta(\cdot)$ using a patient EHR data $\mathbf{x}_{t\leq \tau}$ available in the first $\tau$ hours of admission, to predict the probability of developing cardiac arrest $\hat{y}=f_\theta(\mathbf{x}_{t\leq \tau})$, where $\hat{y} \in \{0, 1\}$ during the rest of the admission stay.
\end{definition}


\subsection{Data Source and Cohort Selection}
CHOA-CICU database is a private pediatric intensive care database extracted from electronic health record for all pediatric patients (age less than 18 years) admitted from 1/1/2018 to 12/31/2023 to a large, quaternary, academic Pediatric Cardiac Intensive Care Unit (CICU) at Children's Hospital at Atlanta and Emory University. 
We obtain 9 static risk factors from CHOA EHR, including: ``gender, ethnicity, first race, second race, primary language, respiration distress, age, admission diagnoses, admission ICD 19 codes''. Among 184 temporal risk factors, we obtain 129 vital signs and nursing assessments from flow sheet after removing non-relevant items and removing missing rate $> 0.5$ items. We also obtain 49 lab test results after removing missing rate $> 0.5$ items, and 5 medications after removing missing rate $> 0.7$ items.
Table~\ref{tab:data_stat} briefly presents the number of patients included in the two data sources. 
%We further conduct cohort selection to exclude certain patients.

\begin{table}[ht]
\centering
\caption{Statistics of CICU patients After Cohort Selection.}
%Dx denotes diagnosis, Vs denotes vital signs, Rx denotes prescriptions, Px denotes procedures, Sx denotes symptoms.}
\label{tab:data_stat}
\begin{tabular}{c|c}
\toprule
Category & Value\\
\midrule
Patient, no. (\% of case)& 3,566 (4.0\%) \\
Admissions, no. (\% of case) & 4,672 (3.1\%)\\
\hline
Age, median years (Q1--Q3) & 0.6 (0.1 -- 5.5)\\
Gender, no. of female (\%) & 1,676 (47.0\%)\\
CICU LoS, median days (Q1--Q3) & 8.4 (4.3 - 20.0)\\
HLoS, median days (Q1--Q3) & 2.0 (0.9 - 5.2)\\
Mortality, no., (\% of Admission) & 185 (4.0\%)\\
\hline
Static Risk Factors, no.& 9\\
Temporal Risk Factors, no. & 184\\
-(Tempo) Vital\&Assessments, no. & 129\\
-(Tempo) Lab Results, no. & 49\\
-(Tempo) Medications, no. & 5\\
\bottomrule
\end{tabular}
\vspace{-0.35cm}
\end{table}

% \begin{table*}[ht!]
% \centering
% \caption{Statistics of CA patients After Cohort Selection. Dx denotes diagnosis, Vs denotes vital signs, Rx denotes prescriptions, Px denotes procedures, Sx denotes symptoms. \jiaying{we want to change to a single column table. With a bit more patient characteristics: length of stay.}}
% %\label{tab:data_stat}
% \begin{tabular}{c|ccc|cc|p{3cm}}
% \toprule
% data source& CA Pat. (Adm.) & Ctrl Pat. (Adm.) & Incidence Pat.\% (Adm.\%) & Age-Days Mean (Q1–-Q3) & \%Female & Risk Factors \\
% \midrule
% %PIC & 358 (359) & 11,645 (12,168) & 3.0\% (2.9\%) &  916 (40 -- 1256)& 42.6\% & (static) Demogr, adm\_Dx; (ts) Vs, Lab, Rx, OR\_Px\\
% \hline
% %CHOA-CICU & 176 (177) & 7,484 (17,327) & 2.30\% (1.01\%) & & & (static) Demogr, adm\_Dx; (ts) Vs, Lab, Rx\\ 
% CHOA-CICU & 143 (146) & 3,423 (4,526) &  4.0\% (3.1\%) & 231 (36 -- 2,008) & 47.0\%& (static) Demogr, adm\_Dx; (ts) Vs, Lab, Med \\
% \bottomrule
% \end{tabular}
% \end{table*}




\section{Method}

\subsection{Overview \& Setup}

Our framework consists of a large, highly capable model \textbf{\bigmodel} and a smaller, resource-efficient model \textbf{\smallmodel}. We assume that $S \in \mathbb{N}$ and $L \in \mathbb{N}$ represent the parameter count of each model with $S \ll L$. Both models can either function as classifiers (i.e., $\mathcal{M}: \mathbb{R}^D \rightarrow [C]$ with $\mathbb{R}^D$ denoting the input space and $C$ the number of total classes), or (multi-modal) sequence models (i.e., $\mathcal{M}: \mathbb{R}^D \rightarrow [V]^{T}$ where $V$ is the vocabulary and $T$ is the sequence length). We include experiments on all of these model classes in Section~\ref{sec:experiments}. Furthermore, we do not require a shared model family to be deployed on both \smallmodel and \bigmodel; for example, \smallmodel could be a custom convolutional neural network optimized for efficient inference and \bigmodel a vision transformer~\citep{dosovitskiy2020image}. The primary objective is to design a deferral mechanism that enables \smallmodel to decide when to return its predictions without the assistance of \bigmodel and when to instead defer to it.

\looseness=-1
Deferral decisions are made using signals derived from the small model \smallmodel as this approach is typically more cost-effective than employing a separate routing mechanism~\citep{teerapittayanon2016branchynet}. Approaches that involve querying the large model \bigmodel to assist in making deferral decisions at test time are excluded from our setup. Such methods --- common in domains like LLMs --- are counterproductive to our goal since querying \bigmodel defeats the purpose of making a deferral decision in the first place?. Examples of these inapplicable methods include collaborative LLM frameworks~\citep{mielke2022reducing} and techniques that rely on semantic entropy for uncertainty estimation~\citep{kuhn2023semantic}. As part of our setup, we assume that \smallmodel is strictly less capable than \bigmodel --- a realistic scenario in practice supported by scaling laws~\citep{kaplan2020scaling}. Under this assumption, mistakes made by \bigmodel are also made by \smallmodel; however, \smallmodel may make additional errors that \bigmodel would avoid. This reflects the general observation that larger models tend to outperform smaller models across a wide range of tasks.

As discussed in Section~\ref{sec:related-word}, the choice of deferral strategy often depends on the level of access available to \smallmodel. We assume white box access with full access to \smallmodel's internals. As such, deferral mechanisms can be directly integrated into the model's architecture and parameters. This involves fine-tuning \smallmodel to predict deferral decisions or to incorporate rejection mechanisms within its predictive process. Our work falls into this category as it proposes a new loss function to fine-tune \smallmodel. 

Our goal is to train a small model that can effectively distinguish between correct and incorrect predictions. While many past works have considered the question of whether it is possible to find proxy measures for prediction correctness, the central question we ask is:
\begin{center}
\textbf{Can we \emph{optimize} the small model \smallmodel to separate correct from incorrect predictions?}
\end{center}
\noindent We show that this is indeed achievable through a carefully designed fine-tuning stage that does not require any architectural modifications. This ensures that the ability to separate correct from incorrect decisions is integrated seamlessly into \smallmodel's existing structure.


\subsection{Confidence-Tuning for Deferral}

\begin{figure}
    \centering
    \resizebox{\linewidth}{!}{
    \begin{figure}[h]
\begin{center}
   \includegraphics[width=0.99\linewidth]{figs/pdf/loss.pdf}
\end{center}
   \caption{
    Training loss of VAR \textit{vs.} FlexVAR. FlexVAR demonstrates a faster convergence rate. We report the results with trained 40 epochs ($\sim$ 70K iterations). 
   }
\label{fig:loss}
\end{figure}

    }
    \vspace{-15pt}
    \caption{\textbf{Overview of \loss}: We want correctly predicted samples maintain their current prediction by ensuring that cross entropy is decreased (top, green). At the same time, we want incorrectly predicted samples to yield a uniform confidence across all classes, leading to a low overall confidence score (bottom, red).}
    \label{fig:opt_goal}
\end{figure}

\textbf{Stage 1: Standard Training.} We begin with a \smallmodel that has already been trained on the tasks it is intended to perform upon deployment. However, due to its limited capacity, \smallmodel cannot achieve the performance levels of \bigmodel. Importantly, we make no assumptions about the training process of \smallmodel—whether it was trained from scratch without supervision from an external model or through a distillation approach.

\sloppy
\textbf{Stage 2: Correctness-Aware Finetuning with \loss.} Next, we introduce a correctness-aware loss, dubbed \loss, to fine-tune \smallmodel for improved confidence calibration. Specifically, the model is trained to make correct predictions with high confidence while reducing the confidence of incorrect predictions (see Figure~\ref{fig:opt_goal}). This loss can either rely on true labels or utilize the outputs of \bigmodel with soft probabilities as targets. 


For a standard classification model, the calibration loss is defined as the following hybrid loss
\begin{align}
\mathcal{L} &= \alpha \mathcal{L}_\text{corr} + (1 - \alpha) \mathcal{L}_\text{incorr} \\
\mathcal{L}_\text{corr} &= \frac{1}{N} \sum_{i=1}^{N} \mathds{1}\{ y_i = \hat{y}_i \} \text{CE}(p_i(\mathbf{x}_i), y_i) \\
\mathcal{L}_\text{incorr} &= \frac{1}{N} \sum_{i=1}^{N} \mathds{1}\{ y_i \neq \hat{y}_i \} \text{KL}\left(p_i(\mathbf{x}_i) \parallel \mathcal{U}\right)
\end{align}
where  \( y_i \) and \( \hat{y}_i \) are the true and predicted labels for $\mathbf{x}_i$, respectively, \( p_i \) is the predicted probability distribution of \smallmodel over classes, \( \mathcal{U} \) represents the uniform distribution over all classes, \( N \) denotes the number samples in the current batch, \( \alpha \in (0, 1) \) is a tunable hyperparameter controlling the emphasis between correct and incorrect predictions, and the cross-entropy function and KL divergence are defined as \( \text{CE}(p, y) = -\sum_{c} y_c \log p_c \) and \( \text{KL}(p \parallel q) = \sum_{c} p_c \log ( \frac{p_c}{q_c}) \), respectively. We note that a similar loss has previously been proposed in Outlier Exposure (OE)~\citep{hendrycks2018deep} for out-of-distribution (OOD) sample detection. Here, the goal is to make sure that OOD examples are assigned low confidence scores by tuning the confidence on a auxiliary outlier dataset. However, to the best of our knowledge, this idea has not previously been used to improve deferral performance of a smaller model in a cascading chain.

We emphasize that the trade-off parameter $\alpha$ plays a critical role as part of this optimization setup as it directly influences model utility and deferral performance. A lower value of \(\alpha\) emphasizes reducing confidence in incorrect predictions by pushing them closer to the uniform distribution, making the model more cautious in regions where it may make mistakes. Conversely, a higher value of \(\alpha\) encourages the model to increase its confidence on correct predictions, sharpening its decision boundaries and enhancing accuracy where it is already performing well. Thus, \(\alpha\) serves as a crucial hyperparameter that balances the trade-off between improving calibration by mitigating overconfidence in errors and reinforcing confidence in accurate classifications. By appropriately tuning \(\alpha\), practitioners can control the model’s behavior to achieve a desired balance between reliability in uncertain regions and decisiveness in confident predictions, tailored to the specific requirements of their application.

We further generalize this loss to token-based models (e.g., LMs and VLMs), formulated as
\ifarxiv
\small
\fi
\begin{align}
    \mathcal{L}_\text{corr} & = \frac{1}{N} \sum_{i=1}^{N} \sum_{t=1}^{T} \mathds{1}\{ y_{i,t} = \hat{y}_{i,t} \} \text{CE}(p_{i,t}(\mathbf{x}_i), y_{i,t}) \\
    \mathcal{L}_\text{incorr} & = \frac{1}{N} \sum_{i=1}^{N} \sum_{t=1}^{T} \mathds{1}\{ y_{i,t} \neq \hat{y}_{i,t} \} \text{KL}\left(p_{i,t}(\mathbf{x}_i) \parallel \mathcal{U}\right)
\end{align}
\normalsize
where \( y_{i,t} \) and \( \hat{y}_{i,t} \) denote the true and predicted tokens at position \( t \) for sample \( i \), \( p_{i,t} \) is the predicted token distribution at position \( t \) for sample \( i \), and \( T \) is the sequence length for the token-based model. The token-level loss ensures that correct token predictions are made confidently while incorrect tokens are assigned smaller confidences.

\sloppy
\textbf{Stage 3: Confidence Computation \& Thresholding.} After fine-tuning \smallmodel with \loss, we apply standard confidence- and entropy-based techniques for model uncertainty to obtain a deferral signal. We use the selective prediction framework to determine whether a query point~$\mathbf{x} \in \mathbb{R}^D$ should be accepted by \smallmodel or routed to \bigmodel. Selective prediction alters the model inference stage by introducing a deferral state through a \textit{gating mechanism}~\citep{yaniv2010riskcoveragecurve}. At its core, this mechanism relies on a deferral function $g:\mathbb{R}^D \rightarrow \mathbb{R}$ which determines if \smallmodel should output a prediction for a sample~$\mathbf{x}$ or defer to \bigmodel. Given a targeted acceptance threshold $\tau$, the resulting predictive model can be summarized as:
\begin{equation}
\label{eq:deferral}
    (\mathcal{M}_S,\mathcal{M}_L,g)(\mathbf{x}) = \begin{cases}
  \mathcal{M}_S(\mathbf{x})  & g(\mathbf{x}) \geq \tau \\
  \mathcal{M}_L(\mathbf{x}) & \text{otherwise.}
\end{cases}
\end{equation}

\emph{Classification Models (Max Softmax).} Let \(\mathcal{M}_S\) produce a categorical distribution
\(\{p(y=c \mid \mathbf{x})\}_{c=1}^C\) over \(C\) classes. 
Then we define the gating function as
\begin{align}
g_{\text{CL}}(\mathbf{x}) \;=\; \max_{1 \,\le\, c \,\le\, C}\;p\bigl(y = c \,\big\vert\, \mathbf{x}\bigr).
\end{align}

\emph{Token-based Models (Negative Predictive Entropy).} 
Let \(\mathcal{M}_S\) produce a sequence of categorical distributions 
\(\{p(y_t = c \mid \mathbf{x})\}_{c=1}^C\) for each token index \(t \in T\). Then we define the gating function as
\begin{equation}
\footnotesize
g_{\text{NENT}}(\mathbf{x}) 
= \; \frac{1}{T} \sum_{t=1}^{T} \sum_{c=1}^{C} 
    p\bigl(y_t = c \,\big\vert\, \mathbf{x}\bigr)\,\log p\bigl(y_t = c \,\big\vert\, \mathbf{x}\bigr),
\end{equation}
where \(y_t \in [C]\) is the predicted token at time step \(t\), \(p(y_t=c \mid \mathbf{x})\) is the (conditional) probability of token \(k\) at step \(t\), and \(T\) is the total number of token positions for the sequence. Across both model classes, higher values of either $g_{\text{CL}}$ or $g_{\text{NENT}}$ indicate higher confidence in the predicted class or sequence generation, respectively.
\section{Experiments: Planning outperforms Heuristics}
\label{sec:experiment}

We begin our empirical demonstrations by showcasing the effectiveness of our planning framework on both synthetic and real datasets. We focus on the simplest planning algorithm, 1-step lookaheads (Algorithm~\ref{alg:complete}), and show that even basic planning can hold great promise. 
We illustrate our framework using two uncertainty quantification modules---GPs and 
\ensembles/ \ensembleplus. 

Throughout this section, we focus on evaluating the mean squared error of 
a regression model $\model$,  and develop adaptive policies that minimize uncertainty on $g(f)$ defined in~\eqref{eqn:l2-g-f}.
When GPs provide a valid model of uncertainty, 
our experiments show that our planning framework significantly outperforms other baselines. 
We further demonstrate that our conceptual framework extends to deep learning-based uncertainty quantification methods such as  \ensembleplus while highlighting computational challenges that need to be resolved in order to scale our ideas. 
For simplicity, we assume a naive predictor, i.e., $\psi(\cdot) \equiv 0$. However, we emphasize that this problem is just as complex as if we were using a sophisticated model $\psi(.)$. The performance gap between the algorithms 
primarily depends
on the level  of uncertainty in our prior beliefs.

To evaluate the performance of our algorithm, we benchmark it against several baselines. 
%Active learning baselines use an acquisition function $\ac$ to select points that have the highest   function value: $X\opt_t \in \argmax_{X \in \xpoolj{t}} \ac({X})$ at every step $t$. These methods may also need an UQ module, which we simply use the same UQ module as in our algorithm, and it  outputs $V(X)$ that measures the the uncertainty of each point $X \in \xpoolj{t}$.
Our first set of baselines are from active learning~\citep{AggarwalKoGuHaPh14}:
\\ % \noindent\textbf{Active Learning Heuristics:} 
\textbf{(1)} 
\textsf{Uncertainty Sampling (Static):}  In this approach, we query the samples for which the model is least certain about. Specifically, we estimate the variance of the latent output $f(X)$ for each $X \in \xpool$ using the UQ module and select the top-$K$ points with the highest uncertainty. \\
\textbf{(2)} \textsf{Uncertainty Sampling (Sequential):} This is a greedy heuristic that sequentially selects the points with the highest uncertainty within a batch, while updating the posterior beliefs using pseudo labels from the current posterior state. Unlike \textsf{Uncertainty Sampling (Static)}, this method takes into account the information gained from each point within batch, and hence tries to diversify the selected points within a batch. 

 
We also compare our approach to the  \textbf{(3)} \textsf{Random Sampling}, which selects each batch uniformly at random from the pool. Additionally, we compare solving the planning problem using  \textsf{REINFORCE}-based policy gradients with   $\mathsf{Smoothed\text{-}Autodiff}$ policy gradients.\footnote{Our code repository is available at
  \url{https://github.com/namkoong-lab/adaptive-labeling}.}
%Detailed experimental setups are provided in Section \ref{sec:details-experiments}.

%We repeat all experiments with 10 random seeds.




\begin{figure}[t]
\centering
\begin{minipage}[b]{0.49\textwidth}
\centering
\includegraphics[width=\textwidth, height=5cm]{figures/original_scale/Var_of_l_2_loss.pdf}
\caption{(Synthetic data) Variance of mean squared loss evaluated through the posterior belief $\mu_t$ at each horizon $t$. This is the objective that policy gradient methods like \textsf{REINFORCE} and $\ouralgo$ optimizes. 1-step lookaheads are surprisingly effective even in long horizons.}
\label{fig:var-l2-sim}
\end{minipage}
\hfill
\begin{minipage}[b]{0.49\textwidth}
\centering \includegraphics[width=\textwidth, height=5cm]{figures/original_scale/Error_of_estimated_model_l_2_loss.pdf}
\caption{(Synthetic data) Error between MSE calculated based on collected data $\mc{D}^{0:T}$ vs. population oracle MSE over $\mc{D}_{\rm eval} \sim P_X$. Reducing uncertainty over posteriors directly leads to better OOD evaluations. 1-step lookaheads significantly outperform active learning heuristics in small horizons.}
\label{fig:mean-l2-sim}
\end{minipage}
%\caption{Simulated data for GPs}
%\label{fig:both_plots}
\end{figure}

\subsection{Planning with Gaussian processes}
\label{sec:experiment-plan-GP}
We now briefly describe the data generation process for the GP experiments,  deferring a more detailed discussion of the dataset generation to Section~\ref{sec:details-experiments}. 
We use both the synthetic data and the real data to test our methodology.
For the \emph{simulated data},  we construct a setting where the general population is distributed across \emph{51 non-overlapping clusters} while the initial labeled data $\dtrain$ just comes from one cluster. In contrast, both $\dpool \defeq (\xpool,\ypool),\deval \defeq (\xeval,\yeval)$ are generated   from all the clusters. 
We begin with a low-dimensional scenario, generating a one-dimensional regression setting using a GP. %Gaussian Process (GP).
Although the data-generating process is not known to the algorithms,  we assume that the GP hyperparameters are known to all the algorithms
to ensure fair comparisons. This can be viewed as a setting where our prior is well-specified, allowing us to isolate the effects
of different policy optimization approaches
 without any concerns about the misspecified priors. We select $10$ batches, each of size $K=5$ across $T = 10$ time horizons.

To examine the robustness of our method against the distributional assumptions made  in the simulated case, we then move to a real dataset where the correct prior is not known. We simulate selection bias from the eICU dataset~\citep{PollardJoRaCeMaBa18}, which contains real-world patient data with in-hospital mortality outcomes. 
We conduct a $k$-means clustering to generate 51 clusters and then select data from those clusters. We view this to be a credible replication of practice, as severe distribution shifts are common due to selection bias in clinical labels.  To convert the binary mortality labels into a regression setting, we train a  random forest classifier and fit a GP on predicted scores, which serves as the UQ module for all the algorithms. As before, the task is to select 10 batches, each consisting of 5 samples, across 10 time horizons.

 In Figures~\ref{fig:var-l2-sim} and~\ref{fig:mean-l2-sim}, we present results for the simulated data. 
Figure~\ref{fig:var-l2-sim} shows the variance of $\ell_2$ loss, and Figure~\ref{fig:mean-l2-sim} presents the error in the estimated $\ell_2$ loss using $\mu_t$ (relative to true $\ell_2$ loss, that is unknown to the algorithm). 
As we can see from these plots, our method one-step lookahead  gives substantial improvements  over active learning baselines and random sampling. In addition,
compared to the one-step lookahead planning approach using \textsf{REINFORCE}-based policy gradients, 
we observe that $\mathsf{Smoothed\text{-}Autodiff}$-based policy gradients provide significantly more robust performance over all horizons.

In Figures~\ref{fig:var-l2-real}~and~\ref{fig:mean-l2-real}, we observe similar findings on the eICU data. We see that planning policies (\textsf{REINFORCE} and $\mathsf{Smoothed\text{-}Autodiff}$) consistently outperform other heuristics by a large margin.  Active learning baselines perform poorly in these small-horizon batched problems and can sometimes be even worse than the random search baselines.  Overall, our results show the importance of careful planning in adaptive labeling for reliable model evaluation. 

We offer some intuition as to why one-step lookahead planning may outperform other heuristic algorithms. 
 First,  \textsf{Uncertainty sampling (Static)} while myopically selects the
 top-$K$ inputs with the highest uncertainty, it fails to consider 
the overlap in information content among the ``best” instances; see \citep{AggarwalKoGuHaPh14} for more details. 
In other words,  it might acquire points from the same region with high uncertainty while failing to induce diversity among the batch.
Although \textsf{Uncertainty Sampling (Sequential)} somewhat addresses the issue of information overlap, a significant drawback of 
this algorithm
is the disconnect between the objective we aim to optimize and the algorithm. For example, it might sample from a region with high uncertainty but very low density. 

\begin{figure}[t]
\centering
\begin{minipage}[b]{0.48\textwidth}
\centering
\includegraphics[width=\textwidth, height=5cm]{figures/original_scale/Var_of_l_2_loss_real.pdf}
\caption{(Real-world eICU data) Variance of mean squared loss evaluated through the posterior belief $\mu_t$ at each horizon $t$. Even 1-step lookaheads are extremely effective planners, and auto-differentiation-based pathwise policy gradients provide a reliable optimization algorithm based on low-variance gradient estimates.}
\label{fig:var-l2-real}
\end{minipage}
\hfill
\begin{minipage}[b]{0.48\textwidth}
\centering \includegraphics[width=\textwidth, height=5cm]{figures/original_scale/Error_of_estimated_model_l_2_loss_real.pdf}
\caption{(Real-world eICU data) Error between MSE calculated based on collected data $\mc{D}^{0:T}$ vs. population oracle MSE over $\mc{D}_{\rm eval} \sim P_X$. Reducing uncertainty over posteriors directly leads to better OOD evaluations. Our method significantly outperforms active learning-based heuristics, and random sampling.}
\label{fig:mean-l2-real}
\end{minipage}
%\caption{Real data for GPs}
\end{figure}
 
%\vspace{-1.5cm}
% \begin{wrapfigure}{r}{.32\columnwidth}
%   \vspace{-.5cm} 
%   \centering
% \includegraphics[scale=.29]{figures/Var of l2l_2 loss.pdf}
%   \vspace{-0.2cm}
%   \caption{Results of GP}
% \label{fig:var-l2-gp}
%   \vspace{-0.1cm}
% \end{wrapfigure}


% Attempts have been made  in the past to address these  drawbacks heuristically  (see \citep{AggarwalKoGuHaPh14}). We give a unified computational framework while approaching the problem in a more principled manner and solving it more optimally.




\subsection{Planning with  neural network-based uncertainty quantification methods ($\ensembleplus$)}


We now provide a proof-of-concept that shows the generalizability of our conceptual framework  to the deep learning-based UQ modules, specifically focusing on $\ensembleplus$ due to their previously observed superior performance~\citep{OsbandWenAsDwIbLuRo23}. Recall that implementing our framework with deep learning-based UQ modules  requires us to retrain the model across multiple possible random actions $\bm{a}(\theta)$ sampled from the current policy $\pi_\theta$.
This requires significant computational resources, in sharp contrast to the GPs where the posteriors are in closed form and can be readily updated and differentiated. 

Due to the computational constraints, we test $\ensembleplus$ on a toy setting to demonstrate the generalizability of our framework. We consider a setting where the general population consists of four clusters, while the initial labeled data only comes from one cluster. Again we generate data using GPs.  The task is to select a batch of 2 points in one horizon. We detail the $\ensembleplus$ architecture in Section \ref{sec:details-experiments}, and we assume prior uncertainty to be large (depends on the scaling of the prior generating functions). 
The results are summarized in the Table~\ref{tab:UQ_ensemble}.

% \begin{table}[H]
% \vspace{-10pt}
% \caption{Performance under \ensembleplus as UQ module}
%     \centering
%     \begin{tabular}{|m{3cm}|m{2.5cm}|m{2cm}|} 
%     \hline
%       Algorithm   & Variance of $\loss_2$ loss estimate & Error of $\loss_2$ loss estimate  \\ \hline Random Sampling 
%          & $1710.9 \pm 1352.1$ & $8.67\pm6.62$ 
%       \\ \hline \ouralgo & $1.30 \pm 0.68$ & $0.91\pm0.25$ \\ \hline
%     \end{tabular}
%     \label{tab:UQ_ensemble}
%     %\vspace{-10pt}
% \end{table}




\begin{table}[h]
\vspace{-10pt}
\caption{Performance under \ensembleplus as the UQ module}
\centering
\begin{tabular}{|l|l|l|}
\hline
Algorithm   & Variance of $\loss_2$ loss estimate & Error of $\loss_2$ loss estimate  \\
\hline
\textsf{Random sampling} & 7129.8 $\pm$ 1027.0 & 136.2 $\pm$ 8.28 \\ \hline
\textsf{Uncertainty sampling (Static)} & 10852 $\pm$ 0.0 & 162.156 $\pm$ 0.0 \\ \hline
\textsf{Uncertainty sampling (Sequential)} & 8585.5 $\pm$ 898.9 & 144 $\pm$ 6.93 \\ \hline
\textsf{REINFORCE} & 1697.1 $\pm$ 0.0 & 45.27 $\pm$ 0.0 \\ \hline
\ouralgo & 1697.1 $\pm$ 0.0 & 45.27 $\pm$ 0.0 \\ \hline
\end{tabular}
%\caption{Comparison of different algorithms based on variance   and   error in $\ell_2$ loss estimation with Ensemble $+$ as the UQ module. Our results demonstrate that {\ouralgo} and REINFORCE outperformthe other active learning based heuristics, confirming the benefits of our MDP formulation for the adaptive labeling problem, as also demonstrated in Section 4.\\
%\footnotesize{Experimental details: We use Gaussian Processes as our data generating process, GP parameters are the same as in Section D.3.  The task is to select a batch of 2 points along one horizon.The marginal distribution $p_X$ has 4 \textit{non-overlapping} clusters. Initial data comes from one cluster, while pool and evaluation points comes from all the clusters. We have $20$ initial labeled data points, $10$ pool points, and $252$ evaluation points.  Training procedures are similar to the one in Section D.3.} }
\label{tab:UQ_ensemble}
\end{table}



% We faced  issues in scaling up these experiments which will be our focus in the future. 





% \begin{itemize}
%     \item Posteriors should be consistent. Two dimensions: even with less training,  
%     \item the inference should be  fast enough
% \end{itemize}


% Potential research directions for uncertainty quantification

% In this section we consider a simple setting We consider a simpler setting and 


% For synthetic dataset generation, we use ...... For real datasets, we use ...... We compare our methodolgy to several baselines ()    This Section is structured as follows:
% \begin{itemize}
%     \item \textbf{GPs, square loss objective} (Section \ref{}): 
%     %the broad aim of the experiments  in this section is to isolate the performance of our methodology without any concerns for the inefficiencies induced due to a mis-specified prior or imperfect posterior inference. To accomplish this we generate synthetic datasets using GPs (detailed later). We use the well specified prior (GPs - with same hyperparameter setting) as our UQ module.   
%      As GPs provide differentaible posterior inference - any errors induced due to imperfect posterior updates are also isolated. We note that under this setting
%      \item In Section\ref{} we demonstrate why our methodology performs better than other baselines - by devising various synthetic experiments ()
%     \item  \textbf{UQ Benchmarking }(Section \ref{}): Before diving into the experiments using $\ensembleplus$ and ENNs,  we showcase our benchmarking experiments in Section \ref{}. We use real datasets We observe that ENNs perform better
%      \item \textbf{Ensemble $+$}, objective: recall, accuracy
%     \item \textbf{ENN}, objective: recall, accuracy
% \end{itemize}




% In Section {}, we test 
% \subsection{Experimental details}

% \begin{itemize}
%     \item UQ methodologies - GPs, ENNs
%     \item Objectives - Recall,  ATE
%     \item Datasets - ATE-synthetic datasets, Recall-synthetic, real datasets
%     \item Baselines - 
%     \begin{itemize}
%         \item Random sampling
%         \item Active learning - Uncertainty based sampling - In regression setting almost all of the 
%         \item Myopic greedy - Greedy Batch based sampling
%         \item Policy Gradient
%     \end{itemize}
    
% \end{itemize}

% \subsection{Experiments}
%     \begin{itemize}
%     \item GPs with square loss
%     \item Benchmarking ENN
%         \item ENNs with ATE
%         \item ENNs with Recall
%     \end{itemize}

% \subsection{Benefits over other algorithms - intuition and experiments}

%Active learning - Myopic greedy / Don't rely on the objective rather some entropy version.


%%% Local Variables:
%%% mode: latex
%%% TeX-master: "main"
%%% End:

\section*{Conclusion}
This paper aims to enhance our understanding of the computational complexity of computing various Shapley value variants. We found that for various ML models --- including decision trees, regression tree ensembles, weighted automata, and linear regression --- both local and global interventional and baseline SHAP can be computed in polynomial time under HMM modeled distributions. This extends popular algorithms, such as TreeSHAP, beyond their empirical distributional scope. We also establish strict complexity gaps between the various SHAP variants (baseline, interventional, and conditional) and prove the intractability of computing SHAP for tree ensembles and neural networks in simplified scenarios. Overall, we present SHAP as a versatile framework whose complexity depends on four key factors: \begin{inparaenum}[(i)] \item model type, \item SHAP variant, \item distribution modeling approach, \item and local vs. global explanations\end{inparaenum}. We believe this perspective provides deeper insight into the computational complexity of SHAP, paving the way for future work.




%We believe that our framework provides a more intricate understanding of SHAP computation complexity across different models, distributions, and variants, paving the way for further research.

Our work opens promising directions for future research. First, expanding our computational analysis to other SHAP-related metrics, such as asymmetric SHAP~\citep{frye20} and SAGE~\citep{covert2020understanding}, would be valuable. Additionally, we aim to explore more expressive distribution classes and relaxed assumptions beyond those in Section \ref{sec:tractable} while maintaining tractable SHAP computation. Finally, when exact computation is intractable (Section \ref{sec:intractable}), investigating the approximability of SHAP metrics through approximation and parameterized complexity theory~\citep{downey2012parameterized} is an important direction.

%Our work opens several promising avenues for future research on the computational properties of explainable AI methods, with a particular focus on SHAP. First, it would be interesting to broaden the computational analysis conducted in this work to include other popular SHAP-related metrics in the literature, such as asymmetric SHAP \cite{frye20} and SAGE \cite{covert2020understanding}. Also, in the future, we aim to explore more expressive distribution classes and relaxed distributional assumptions—extending beyond those examined in Section \ref{sec:tractable} —that still yield tractable SHAP computation. Finally, when exact computation proves intractable (Section \ref{sec:intractable}), it is worthwhile to theoretically investigate the question of the approximability of computing the SHAP metrics across various configurations, through the lens of approximation and parametrized complexity theory \cite{arora2009computational}.

%This paper aims to deepen our understanding of the computational complexity involved in obtaining different Shapley value variants. We found that for a variety of ML models, including decision trees, tree ensembles for regression, weighted automata, and linear regression models — computing both local and global interventional and baseline SHAP can be done in polynomial time when distributions are modeled by HMMs. This extends the distributional scope of popular algorithms like TreeSHAP, which is limited to empirical distributions. Additionally, we demonstrate a strict complexity gap between SHAP variants, showing that interventional and baseline SHAP can be strictly easier to compute than conditional SHAP. Despite these positive results, we uncovered intractability for various SHAP variants in neural networks and tree ensembles. Finally, we provided generalized complexity relations across SHAP variants. We believe that our framework offers a deeper understanding of the complexity involved in computing SHAP across various variants, models, distributions, as well as in both local and global computations, laying the groundwork for future research.



%%%%%%%%%%%%%%%%%%%%%%%%%%%%%%%%%%%%%%%%%%%%%%%%%%%%%%%%%%%%%%%%%%%%%%%%%%%%%%%%
% \section*{APPENDIX}

% Appendixes should appear before the acknowledgment.

\section*{ACKNOWLEDGMENT}

This work was supported in part by the Nell Hodgson
Woodruff School of Nursing (NHWSN) Center for Data Science of Emory University, the NHWSN High Performance Computing cluster and the NHWSN IT Department.



%%%%%%%%%%%%%%%%%%%%%%%%%%%%%%%%%%%%%%%%%%%%%%%%%%%%%%%%%%%%%%%%%%%%%%%%%%%%%%%%

\bibliographystyle{IEEEtran}
\bibliography{mybib}



\end{document}
