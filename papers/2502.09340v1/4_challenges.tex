\section{Challenges of Part-Prototype Models}
\label{sec:challenges}

In this section, we present our taxonomy of open challenges for part-prototype models (\ppms) with four main categories (see Figure~\ref{fig:challenges-taxonomy}).
First, we outline the challenges related to the number and quality of prototypes (category \textbf{Prototypes} in Section~\ref{ssec:chall:proto}). Second, we describe the challenges related to the theoretical foundation of \ppms, their performance, and the lack of standardized evaluation (category \textbf{Methodology} in Section~\ref{ssec:chall:method}). Third, we examine the limitations of \ppms with respect to the machine learning tasks they have been applied to and the assumptions determining their architecture (category \textbf{Generalization} in Section~\ref{ssec:chall:gen}). 
Fourth, we point out concerns that prevent \ppms from being used in practice (category \textbf{Safety and Use in Practice} in Section~\ref{ssec:chall:safety}).
\begin{figure*}
    \centering
 \includegraphics[width=\textwidth]{figures/part-prototype-challenges.pdf}
    \caption{Taxonomy of challenges that current (2024) part-prototype models face.}
    \label{fig:challenges-taxonomy}
\end{figure*}
