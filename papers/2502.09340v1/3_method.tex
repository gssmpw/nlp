\section{Survey Method}
\label{sec:method}

We conducted a structured search with ProtoPNet~\citep{Chen_2019_ThisLooksThat} as a seed paper to build our corpus of papers on part-prototype models (\ppms). 
We chose ProtoPNet because it is the first paper to introduce a neural architecture based on prototypical parts. 
We filtered the corpus by the following inclusion and exclusion criteria:
\begin{itemize}
    \item We only considered premier venues\footnote{IJCAI, NeurIPS, AAAI, ICCV, ICLR, EMNLP, ECCV, CVPR, KDD, ECML/PKDD, ICML, Sci. Rep., Nat. Mach. Intell., JMLR, ACM FAcct, XAI, ACM Comput. Surv.} and excluded workshop papers, posters, and shared tasks.
    \item We included papers that present \ppms, namely novel methods, methods that improve on or apply ProtoPNet, applications of existing \ppms, surveys, evaluation with humans, and/or evaluation frameworks.
\end{itemize}
We queried the Semantic Scholar API\footnote{\url{https://api.semanticscholar.org/graph/v1/paper/cc145f046788029322835979a14459652da7247e/citations?fields=intents,url,title,abstract,venue,year,referenceCount,citationCount,influentialCitationCount,fieldsOfStudy,publicationDate&limit=1000} and a second call with offset=1000 (last update) on 03.02.2025 } for all papers that cite ProtoPNet~\citep{Chen_2019_ThisLooksThat} and were published from 2019 to 2024, resulting in 1032 papers.
We filtered this set by the selected venues and exclusion criteria. We conducted a web search for papers with missing venue information (75 papers) and for peer-reviewed versions of papers with arXiv as venue (233) and ``proto'' in their (lowercased) title (39).
We then manually reviewed each paper for our inclusion criteria. 


Our final corpus consists of 45 papers, the majority of which are on image processing (see Figure \ref{fig:chart}), including 37 methods papers, 3 analysis papers, and 5 surveys.

As shown in Figure~\ref{fig:chart}, research on \ppms is published in diverse venues and has increased over the years, with a slight stagnation in 2024. This stagnation could be due to several reasons: We included surveys, four of which were published in 2023. We relied on publications that cited the seminal paper~\citep{Chen_2019_ThisLooksThat} but there may be papers presenting, applying or evaluating \ppms that do not cite this paper. In addition, the venue information provided by Semantic Scholar's API is not always accurate and may have affected our initial corpus. 
Another reason may be that the complexity of the open challenges and the lack of clear future directions have stagnated research in \ppms.

\begin{figure}[t] 
\centering
\small
    \begin{tabular}{+c^c^c^c^c^c^c^c}
        \toprule \tabhead
        Images & Text & Seq. & Graph & Sound & Video & Total \\\otoprule
        37 & 2 & 3& 1& 1& 1& 45\\\bottomrule
    \end{tabular}
        \vskip 0.1in
        \small (a)   
       \includegraphics[width=\linewidth,trim={1cm 0.9cm 0 0},clip]{plots/venue-per-year/year_venue24_out0.PDF}
       \small (b)
      
   \caption{Corpus overview: a) Number of papers per modality (Seq. - Sequences). b) Number of  papers per venue and year. Survey papers are marked with a star pattern. Note: The seed paper on ProtoPNet~\protect\cite{Chen_2019_ThisLooksThat} was the only paper published in 2019, and first subsequent work appeared in 2021 in our list of venues.}
   \label{fig:chart}
\end{figure} 


 


