\section{Conclusion}
\label{sec:conclusion}




Since their inception in 2019, with an initial application in fine-grained image recognition, part-prototype models (\ppms) have seen the development of multiple extensions and variations in modalities beyond vision.
They have been applied in several application domains, particularly in those where interpretability is valued (e.g., medicine, finance).

Despite being intrinsically interpretable, our analysis shows \ppms still suffer from multiple challenges (e.g., low quality of prototypes, lack of theoretical foundation, and non-competitive predictive performance), making them less likely to be used than black box models. 
For future work, we provide five research directions and outline concrete research ideas.
This includes the development of interactive frameworks for human-AI collaboration to address the semantic shortcomings of prototypes, and the design of novel theory-based architectures to address the lack of theoretical foundation in \ppms. 
In addition, aligning models with human reasoning by introducing human-aligned similarity metrics and disentangling the different visual features (color, shape and texture) would improve their usefulness in practice. 

We envision this survey as a useful resource for researchers who are interested in alternatives to black box models, and we hope that the research directions we provide will pave the way for better \ppms, ultimately providing different ML stakeholders with accurate and interpretable models.


