
\subsection{\texorpdfstring{\colorbox[RGB]{255,145,77}{Generalization}}{Generalization}}
\label{ssec:chall:gen}

\ppms contain specific architectural solutions that limit their real-world applications.

\subsubsection{\texorpdfstring{\colorbox[RGB]{255,199,166}{Tasks}}{Tasks}}
\label{sssec:chall:gen:tasks}
\ppms have been developed for fine-grained image classification, but their application to broader domains remains limited~\citep{Xue_2024_ProtoPFormerConcentratingPrototypical}.
In particular, they have not been tested on small datasets~\citep{Song_2024_MorphologicalPrototypingUnsupervised} and are typically designed for low-resolution images, limiting their clinical applications\footnote{This limitation primarily stems from backbones pre-trained on ImageNet with 224px$\times$224px, whereas clinical images range from mega- (mammography) to even giga-pixels (whole slide imaging).}~\citep{Carmichael_2024_ThisProbablyLooks}.
In addition, they focus on single-label classification~\citep{Ruis_2021_IndependentPrototypePropagation} and have not been well tested in multi-label and multimodal~\citep{Rymarczyk_2023_ProtoMILMultipleInstance} settings, which often occur in real-world applications. Moreover, there is only preliminary work on applying prototypes to more challenging scenarios, such as continual learning~\citep{Rymarczyk_2023_ICICLEInterpretableClass}, the open-world problem~\citep{Zheng_2024_PrototypicalHash}, partial label learning~\citep{Carmichael_2024_ThisProbablyLooks}, or zero shot classification~\citep{Ruis_2021_IndependentPrototypePropagation}. Finally, these models are often developed without feedback from domain experts~\citep{Fauvel_2023_LightweightEfficientExplainablebyDesign} and do not take into account contextual information (such as time and historical interactions) that is useful for, e.g., email classification~\citep{Wang_2023_PROMINETPrototypebasedMultiView}.

\textit{\ppms need to be adapted to datasets of different sizes and to samples of different resolutions. They also need to handle more difficult problems than single label classification to better address real-world applications.}

\subsubsection{\texorpdfstring{\colorbox[RGB]{255,199,166}{Assumptions}}{Assumptions}}
\label{sssec:chall:gen:assum}
The bottleneck nature of \ppms plays a critical role in ensuring interpretability by aligning model representations with human-understandable concepts. However, this assumption can also limit the ability of the model to capture complex data patterns~\citep{Zheng_2024_PrototypicalHash}. As a result, \ppms do not support counting the occurrences of prototypes~\citep{Nauta_2023_PIPNetPatchBasedIntuitive}, do not capture relationships between detected prototypes~\citep{Zhang_2023_Learningselectprototypical}, and are unable to represent prototypes as hierarchical structures~\citep{Wang_2021_InterpretableImageRecognition}.
In addition, \ppms rely on a fixed number of prototypes~\citep{Barnett_2021_casebasedinterpretabledeep}, and there are no mechanisms to guide the model toward user-desired concepts without risking data leakage~\citep{Bontempelli_2023_ConceptlevelDebuggingPartPrototype}.

\textit{The assumptions \ppms make on their architecture limit their abilities, such as learning complex patterns and relationships between prototypes.}
