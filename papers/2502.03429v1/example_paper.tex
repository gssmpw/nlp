%%%%%%%% ICML 2025 EXAMPLE LATEX SUBMISSION FILE %%%%%%%%%%%%%%%%%

\documentclass{article}

% Recommended, but optional, packages for figures and better typesetting:
\usepackage{microtype}
\usepackage{graphicx}
\usepackage{subfigure}
\usepackage{booktabs} % for professional tables

% hyperref makes hyperlinks in the resulting PDF.
% If your build breaks (sometimes temporarily if a hyperlink spans a page)
% please comment out the following usepackage line and replace
% \usepackage{icml2025} with \usepackage[nohyperref]{icml2025} above.
\usepackage{hyperref}
\usepackage{enumitem}
\usepackage{booktabs}
\usepackage{siunitx}
\usepackage{tabularx}
\usepackage{multirow}
\usepackage{adjustbox}
\usepackage{color}
\def \red {\textcolor{red}}
\def \blue {\textcolor{blue}}

% Attempt to make hyperref and algorithmic work together better:
\newcommand{\theHalgorithm}{\arabic{algorithm}}

% Use the following line for the initial blind version submitted for review:
% \usepackage{icml2025}

% If accepted, instead use the following line for the camera-ready submission:
\usepackage[accepted]{icml2025}

% For theorems and such
\usepackage{amsmath}
\usepackage{amssymb}
\usepackage{mathtools}
\usepackage{amsthm}
\usepackage{todonotes}
\usepackage{xcolor}
% if you use cleveref..
\usepackage[capitalize]{cleveref}

%%%%%%%%%%%%%%%%%%%%%%%%%%%%%%%%
% THEOREMS
%%%%%%%%%%%%%%%%%%%%%%%%%%%%%%%%
\theoremstyle{plain}
\newtheorem{theorem}{Theorem}[section]
\newtheorem{proposition}[theorem]{Proposition}
\newtheorem{lemma}[theorem]{Lemma}
\newtheorem{corollary}[theorem]{Corollary}
\theoremstyle{definition}
\newtheorem{definition}[theorem]{Definition}
\newtheorem{assumption}[theorem]{Assumption}
\theoremstyle{remark}
\newtheorem{remark}[theorem]{Remark}

\newcommand{\wjd}[1]{{\color{cyan}{[(Jindong) #1]}}}
\newcommand{\wjdd}[1]{\todo[linecolor=cyan,backgroundcolor=cyan!25,bordercolor=cyan,size=\scriptsize]{(Jindong) #1}}

\newcommand{\ch}[1]{{\color{blue}{[(haoc) #1]}}}
\newcommand{\chh}[1]{\todo[linecolor=blue,backgroundcolor=blue!25,bordercolor=blue,size=\scriptsize]{(haoc) #1}}
% Todonotes is useful during development; simply uncomment the next line
%    and comment out the line below the next line to turn off comments
%\usepackage[disable,textsize=tiny]{todonotes}
% \usepackage[textsize=tiny]{todonotes}

% The \icmltitle you define below is probably too long as a header.
% Therefore, a short form for the running title is supplied here:
\icmltitlerunning{On Fairness of Unified Multimodal Large Language Model for Image Generation}

\begin{document}

\twocolumn[
\icmltitle{On Fairness of Unified Multimodal Large Language Model for Image Generation}

% It is OKAY to include author information, even for blind
% submissions: the style file will automatically remove it for you
% unless you've provided the [accepted] option to the icml2025
% package.

% List of affiliations: The first argument should be a (short)
% identifier you will use later to specify author affiliations
% Academic affiliations should list Department, University, City, Region, Country
% Industry affiliations should list Company, City, Region, Country

% You can specify symbols, otherwise they are numbered in order.
% Ideally, you should not use this facility. Affiliations will be numbered
% in order of appearance and this is the preferred way.
\icmlsetsymbol{equal}{*}

\begin{icmlauthorlist}
\icmlauthor{Ming Liu}{i}
\icmlauthor{Hao Chen}{c}
\icmlauthor{Jindong Wang}{w}
\icmlauthor{Liwen Wang}{i}
\icmlauthor{Bhiksha Raj Ramakrishnan}{c}
\icmlauthor{Wensheng Zhang}{i}

\end{icmlauthorlist}

\icmlaffiliation{i}{Iowa State University}
\icmlaffiliation{c}{Carnegie Mellon University}
\icmlaffiliation{w}{William Mary University}

% You may provide any keywords that you
% find helpful for describing your paper; these are used to populate
% the "keywords" metadata in the PDF but will not be shown in the document


\icmlkeywords{Machine Learning, ICML}

\vskip 0.3in
]

% this must go after the closing bracket ] following \twocolumn[ ...

% This command actually creates the footnote in the first column
% listing the affiliations and the copyright notice.
% The command takes one argument, which is text to display at the start of the footnote.
% The \icmlEqualContribution command is standard text for equal contribution.
% Remove it (just {}) if you do not need this facility.

\printAffiliationsAndNotice{}  
% leave blank if no need to mention equal contribution
% \printAffiliationsAndNotice{\icmlEqualContribution} 
% otherwise use the standard text.


\begin{abstract}


The choice of representation for geographic location significantly impacts the accuracy of models for a broad range of geospatial tasks, including fine-grained species classification, population density estimation, and biome classification. Recent works like SatCLIP and GeoCLIP learn such representations by contrastively aligning geolocation with co-located images. While these methods work exceptionally well, in this paper, we posit that the current training strategies fail to fully capture the important visual features. We provide an information theoretic perspective on why the resulting embeddings from these methods discard crucial visual information that is important for many downstream tasks. To solve this problem, we propose a novel retrieval-augmented strategy called RANGE. We build our method on the intuition that the visual features of a location can be estimated by combining the visual features from multiple similar-looking locations. We evaluate our method across a wide variety of tasks. Our results show that RANGE outperforms the existing state-of-the-art models with significant margins in most tasks. We show gains of up to 13.1\% on classification tasks and 0.145 $R^2$ on regression tasks. All our code and models will be made available at: \href{https://github.com/mvrl/RANGE}{https://github.com/mvrl/RANGE}.

\end{abstract}



\section{Introduction}
Backdoor attacks pose a concealed yet profound security risk to machine learning (ML) models, for which the adversaries can inject a stealth backdoor into the model during training, enabling them to illicitly control the model's output upon encountering predefined inputs. These attacks can even occur without the knowledge of developers or end-users, thereby undermining the trust in ML systems. As ML becomes more deeply embedded in critical sectors like finance, healthcare, and autonomous driving \citep{he2016deep, liu2020computing, tournier2019mrtrix3, adjabi2020past}, the potential damage from backdoor attacks grows, underscoring the emergency for developing robust defense mechanisms against backdoor attacks.

To address the threat of backdoor attacks, researchers have developed a variety of strategies \cite{liu2018fine,wu2021adversarial,wang2019neural,zeng2022adversarial,zhu2023neural,Zhu_2023_ICCV, wei2024shared,wei2024d3}, aimed at purifying backdoors within victim models. These methods are designed to integrate with current deployment workflows seamlessly and have demonstrated significant success in mitigating the effects of backdoor triggers \cite{wubackdoorbench, wu2023defenses, wu2024backdoorbench,dunnett2024countering}.  However, most state-of-the-art (SOTA) backdoor purification methods operate under the assumption that a small clean dataset, often referred to as \textbf{auxiliary dataset}, is available for purification. Such an assumption poses practical challenges, especially in scenarios where data is scarce. To tackle this challenge, efforts have been made to reduce the size of the required auxiliary dataset~\cite{chai2022oneshot,li2023reconstructive, Zhu_2023_ICCV} and even explore dataset-free purification techniques~\cite{zheng2022data,hong2023revisiting,lin2024fusing}. Although these approaches offer some improvements, recent evaluations \cite{dunnett2024countering, wu2024backdoorbench} continue to highlight the importance of sufficient auxiliary data for achieving robust defenses against backdoor attacks.

While significant progress has been made in reducing the size of auxiliary datasets, an equally critical yet underexplored question remains: \emph{how does the nature of the auxiliary dataset affect purification effectiveness?} In  real-world  applications, auxiliary datasets can vary widely, encompassing in-distribution data, synthetic data, or external data from different sources. Understanding how each type of auxiliary dataset influences the purification effectiveness is vital for selecting or constructing the most suitable auxiliary dataset and the corresponding technique. For instance, when multiple datasets are available, understanding how different datasets contribute to purification can guide defenders in selecting or crafting the most appropriate dataset. Conversely, when only limited auxiliary data is accessible, knowing which purification technique works best under those constraints is critical. Therefore, there is an urgent need for a thorough investigation into the impact of auxiliary datasets on purification effectiveness to guide defenders in  enhancing the security of ML systems. 

In this paper, we systematically investigate the critical role of auxiliary datasets in backdoor purification, aiming to bridge the gap between idealized and practical purification scenarios.  Specifically, we first construct a diverse set of auxiliary datasets to emulate real-world conditions, as summarized in Table~\ref{overall}. These datasets include in-distribution data, synthetic data, and external data from other sources. Through an evaluation of SOTA backdoor purification methods across these datasets, we uncover several critical insights: \textbf{1)} In-distribution datasets, particularly those carefully filtered from the original training data of the victim model, effectively preserve the model’s utility for its intended tasks but may fall short in eliminating backdoors. \textbf{2)} Incorporating OOD datasets can help the model forget backdoors but also bring the risk of forgetting critical learned knowledge, significantly degrading its overall performance. Building on these findings, we propose Guided Input Calibration (GIC), a novel technique that enhances backdoor purification by adaptively transforming auxiliary data to better align with the victim model’s learned representations. By leveraging the victim model itself to guide this transformation, GIC optimizes the purification process, striking a balance between preserving model utility and mitigating backdoor threats. Extensive experiments demonstrate that GIC significantly improves the effectiveness of backdoor purification across diverse auxiliary datasets, providing a practical and robust defense solution.

Our main contributions are threefold:
\textbf{1) Impact analysis of auxiliary datasets:} We take the \textbf{first step}  in systematically investigating how different types of auxiliary datasets influence backdoor purification effectiveness. Our findings provide novel insights and serve as a foundation for future research on optimizing dataset selection and construction for enhanced backdoor defense.
%
\textbf{2) Compilation and evaluation of diverse auxiliary datasets:}  We have compiled and rigorously evaluated a diverse set of auxiliary datasets using SOTA purification methods, making our datasets and code publicly available to facilitate and support future research on practical backdoor defense strategies.
%
\textbf{3) Introduction of GIC:} We introduce GIC, the \textbf{first} dedicated solution designed to align auxiliary datasets with the model’s learned representations, significantly enhancing backdoor mitigation across various dataset types. Our approach sets a new benchmark for practical and effective backdoor defense.





\vspace{-2em}
\section{Preliminary}
\label{sec:pre}

\paragraph{Structure of U-MLLMs}
We consider an \emph{autoregressive} U-MLLM that, given a textual prompt $x$, first converts it into a sequence of text tokens $\{x_1, \dots, x_{T_x}\}$ and then generates a sequence of \emph{image tokens} $\{z_1,\dots,z_{T_z}\}$, which an image decoder reconstructs into a final image $y$ (see~\autoref{fig:unified_model} for the pipeline). Let $\boldsymbol{\theta} = \{\boldsymbol{\theta}_{\mathrm{v}}, \boldsymbol{\theta}_{\mathrm{l}}\}$ denote the model parameters, where: \emph{(1)} $\boldsymbol{\theta}_{\mathrm{v}}$ is the \textbf{image tokenizer} (encoder-decoder) responsible for converting input images into discrete tokens and decoding tokens back into images. \emph{(2)}$ \boldsymbol{\theta}_{\mathrm{l}}$ is the \textbf{language model} (LM) that processes and generates token sequences (both text and image tokens) in a unified autoregressive way.

As shown in \autoref{fig:vision_tower},  the image encoder $E_{\boldsymbol{\theta}_{\mathrm{v}}}$ maps an image $y$ into latent embeddings $\textbf{e}$, then quantizes the embeddings into a discrete token sequence $\{z_1,\dots,z_{T_z}\}$. Conversely, the image decoder $D_{\boldsymbol{\theta}_{\mathrm{v}}}$ inverts this process:
\begin{align}
    \{z_1,\dots,z_{T_z}\} &= E_{\boldsymbol{\theta}_{\mathrm{v}}}(y), 
    \label{eq:enc}\\
    y &= D_{\boldsymbol{\theta}_{\mathrm{v}}}(z_1,\dots,z_{T_z}).
    \label{eq:dec}
\end{align}
Meanwhile, the LM $LM_{\boldsymbol{\theta}_{\mathrm{l}}}$ treats both text tokens and image tokens uniformly under a single next-token probability:
\begin{equation}
    P_{\boldsymbol{\theta}}(z_t \mid x,\, z_{<t})
    \;=\;
    LM_{\boldsymbol{\theta}_{\mathrm{l}}}(z_{t-1},\dots,z_1;\, x).
    \label{eq:lm}
\end{equation}
This design allows the U-MLLM to perform \emph{visual understanding} by mapping an image to a semantic space (via \cref{eq:enc}) and then interpreting those discrete tokens as inputs to the LM, and to perform \emph{image generation} by autoregressively sampling $\{z_t\}$ from \cref{eq:lm} and reconstructing an image from the resulting token sequence via \cref{eq:dec}.
\vspace{-1em}
\paragraph{Demographic Bias.}  
When conditioning on neutral generation prompt (no explicit gender or race specified), the model could exhibit \emph{demographic bias} by disproportionately generating output with a distribution skewing towards certain demographic groups (e.g., “male” or “female,” “Asian” or “Black”). ~\autoref{fig:fairness-mllms} shows such an example where a prompt \emph{``Please generate images of construction workers''} yield mostly images samples of one demographic group.

Formally, let $d \in \mathcal{D}$ where $D = \{d_1, \ldots, d_K\}$ is a set of demographic labels (e.g., \{male, female\} or \{Asian, Black, White, Indian\}). Given a neutral prompt \( x \), such as ``a portrait of a construction worker'', an unbiased model would generate a list of images \( \{y_1, y_2, \ldots\} \) with these demographic labels in a balanced distribution, for example, 50-50 for gender, or a uniform distribution for race. By contrast, we find that the latest U-MLLMs generate
\[
P_\theta(\mathcal{C}(y) = d_i \mid x) \gg P_\theta(\mathcal{C}(y) = d_j \mid x),
\]
where $d_i, d_j$ corresponds to ``male'',``female'' respectively(in this case) and $\mathcal{C}$ is a pre-trained image classifier from prior study \cite{shen2024finetuningtexttoimagediffusionmodels} that labels each image $y$ with a predicted attribute $\hat{d}_i$. This indicated a strong bias in gender preferences. Our goal is to mitigate this bias while preserving overall image fidelity.
\vspace{-1em}
\paragraph{Direct Preference Optimization} As an alternative to computational expensive RLHF \cite{ouyang2022traininglanguagemodelsfollow} methods that are required to train a reward model separately, DPO re-parameterizes reward model $r$ via a ratio between the policy $\pi_\theta$ and a reference policy $\pi_{\text{ref}}$ and eliminates the cost of training a reward function $r(x,y)$ explicitly \cite{rafailov2024direct}. Concretely, for a given pair $(x,y)$,
\begin{align}
r(x,y) 
&= \beta \log \frac{\pi_\theta(y \mid x)}{\pi_{\text{ref}}(y \mid x)} 
+ \beta \log Z(x),
\label{eq:dpo_reward_restate}
\end{align}
where $Z(x)$ is a partition function and $\beta$ is a scaling factor\cite{rafailov2024direct}. By plugging $r$ into a Bradley-Terry model\cite{Bradley1952RankAO}, DPO formulates the preference probability for a winning response $y_w$ over a losing response $y_l$ as
\begin{equation}
p(y_w \succ y_l \mid x) 
\;=\;
\sigma \bigl( r(x,y_w) - r(x,y_l)\bigr),
\label{eq:bt-model}
\end{equation}
DPO aims to optimize the policy \emph{directly} by minimizing \autoref{eq:bt-model} and increasing the difference in models' preferences between $y_w$ and $y_l$ without a separate reward model.

To further reduce the overhead of the reference model, recent work explores \emph{reference-free} preference optimization~\cite{hong2024orpomonolithicpreferenceoptimization, meng2024simposimplepreferenceoptimization}. One representative method, ORPO~\cite{hong2024orpomonolithicpreferenceoptimization}, defines the \emph{odds} of generating a response $y$ given prompt $x$ as
\begin{equation}
    \text{odds}_\theta(y \mid x) 
    \;=\; 
    \frac{p_\theta(y \mid x)}{1 - p_\theta(y \mid x)}
\end{equation}

and further quantifies the odds ratio between two responses $y_{w}$ and $y_{l}$ via
\begin{equation}
    \text{OR}_\theta\bigl(y_{w}, y_{l}\bigr)
    \;=\;
    \frac{\text{odds}_\theta(y_{w} \mid x)}{\text{odds}_\theta(y_{l} \mid x)}.
\end{equation}
and finally, define a preference loss term as
\begin{equation}
    \mathcal{L}_{OR} = -\log \sigma \left( \log \frac{odds_\theta(y_w|x)}{odds_\theta(y_l|x)} \right) \label{eq:ratio} 
\end{equation}
By encouraging a large odds ratio $\text{OR}_\theta(y_{w}, y_{l})$, the model is pushed to \emph{prefer} the response $y_{w}$ over $y_{l}$ \emph{directly}, without relying on a separate  “reference” model. 
\vspace{-1em}
\section{Locating Bias}
As shown in~\autoref{fig:locating_bias}, to identify the biases that might emerge from \emph{where}, we analyze intermediate outputs - that is, the \emph{image tokens} (produced by the LM) and the \emph{image embeddings} (produced by the vision encoder) during the decoding step. We consider two main hypotheses regarding the origin of demographic bias as follows.
\vspace{-1em}
\subsection{Hypothesis I: The Bias is from Vision Encoder}
The vision encoder transforms input images into a high-dimensional embedding space, as illustrated in~\autoref{fig:vision_tower}. This encoder itself may be a source of bias within U-MLLMs. To test this hypothesis, we focus on auditing the \emph{vision encoder}, which is trained symmetrically alongside the decoder. By detecting the embedding output from the encoder, we aim to identify any inherent biases that could influence the overall fairness of the model.

\vspace{-1em}
\paragraph{Linear Probing on Image Embeddings}
\autoref{fig:locating_bias} illustrated the overall pipeline to audit vision encoder: first, we sample a balanced set of images denoted as ${y_i}$ from FairFace\cite{kärkkäinen2019fairfacefaceattributedataset} covering different demographic attributes (e.g., male, female). By feeding these images into the vision encoder, we obtained a set of image embeddings denoted as $\mathbf{E} = \{\mathbf{e}_i\}$. Each embedding $\mathbf{e_i}$ is labeled with the ground truth attribute $d_i \in \{ \text{male}, \text{female} \}$ according to its corresponding image input $y_i$. We then split $\mathbf{E}$ into training and testing subsets and train a \emph{linear} classifier $\ell(\cdot)$ to predict demographic labels from embeddings $e_i$. This gives us pairs $\bigl(y_i, \hat{d}_i, \mathbf{e}_i\bigr)$—i.e.\ the image, the predicted demographic label, and the corresponding image embedding. The precision, F1 score, recall and precision in the test set are all \emph{high}, indicating that the encoder’s latent space preserves explicit demographic distinctions:
\[
\widehat{d}_i \;=\; \ell(e_i) \quad \rightarrow \quad \mathrm{High\ Accuracy}.
\]
Since the decoder is trained to invert the encoder, it should \emph{faithfully} preserve these attributes. We also find that when we prompt the model with an attribute (e.g., ``Asian professor''), the final generated images are overwhelmingly associated with that attribute. This also implies that the \emph{decoder} consistently reflects whatever demographic semantics are either present in the tokens or passed by the encoder. Therefore, the chance that bias originated from the vision encoder is small. 
\vspace{-1em}
\subsection{Hypothesis II: The bias is from the language model}
Let $x_{\text{neutral}}$ be a neutral prompt (e.g., ``a photo of a professor''), and let $x_{\text{aug}}(d_i)$ denote an \emph{augmented} version of the same prompt specifying a particular demographic $d_i$ (e.g., ``a photo of a female professor''). Assume that for each $x$, we sample a list of images $\{y_1, \ldots, y_M\}$ and each image $y_i$ is associated with a sequence of image tokens generated from LM as $\mathbf{z}_i = (z_{i,1}, z_{i,2}, \ldots)$.

\vspace{0.5em}
\paragraph{Collecting Image Tokens.}
We record the \emph{intermediate} image tokens for (\emph{1}) The neutral prompt $x_{\text{neutral}}$  and (\emph{2}) The demographic-augmented prompts $x_{\text{aug}}(d_i)$ as described above. We then use a pre-trained image classifier $\mathcal{C}$ \cite{shen2024finetuningtexttoimagediffusionmodels} to label each final image $y_i$ with a predicted attribute $ \widehat{d}_i \;=\; \mathcal{C}(y_i)$. This yields pairs $\bigl(x_i, y_i, \hat{d}_i, \mathbf{z}_i\bigr)$—i.e.\ the prompt, the resulting image, the predicted demographic label, and the corresponding image token sequence.
\vspace{-1em}
\paragraph{Distribution Over Image Tokens.}
Let $p_{\theta}(z \mid x)$ denote the probability distribution over the image tokens $z$ given a prompt $x$. In practice, we approximate it by a finite sample:
\[
\widehat{p}_{\theta}(z \mid x)
\;=\;
\frac{1}{M}\,\sum_{i=1}^{M} \delta(\mathbf{z}_i),
\]
where $\delta(\cdot)$ is a Kronecker delta that counts the sampled sequences $\{\mathbf{z}_i\}$. This empirical approximation estimates the token distribution \( p_{\theta}(z \mid x) \) by averaging over \( M \) generated image token sequences, effectively capturing the model's likelihood of producing each token \( z \) given the prompt \( x \).

We then calculate a \emph{distribution distance}, e.g., Jensen-Shannon Divergence(JSD)\cite{Lin1991DivergenceMB}, between the neutral-prompt distribution $\widehat{p}_{\theta}(z \mid x_{\text{neutral}})$ and the augmented-prompt distribution $\widehat{p}_{\theta}(z \mid x_{\text{aug}}(d_i))$. Formally,
\[
D_{\mathrm{JS}}\!\Bigl(
\widehat{p}_{\theta}(z \mid x_{\text{neutral}})\;\|\;
\widehat{p}_{\theta}(z \mid x_{\text{aug}}(d_i))
\Bigr).
\]
where
\[
D_{\mathrm{JS}}(P\|Q) = \frac{1}{2}D_{\mathrm{KL}}(P\|M) + \frac{1}{2}D_{\mathrm{KL}}(Q\|M)
\]
and $M = \frac{1}{2}(P + Q)$ is the average distribution, with $D_{\mathrm{KL}}$ denoting the Kullback-Leibler divergence.

% \ch{\textbf{Results}}. 
We observe that when $d_i$ matches the \emph{predicted demographic label} $\hat{d}_i$ predicted from the final images, the JSD between the two distributions is significantly smaller. In other words, the token-level distribution for the neutral prompt $x_{\text{neutral}}$ is \emph{closest} to that for $x_{\text{aug}}(d_i)$ corresponding to the majority demographic group the model actually generated. 
\vspace{-1em}
\paragraph{Takeaway} This indicates that the \emph{language model} itself already introduces demographic preferences \emph{at the token level}. If the prompt ``firefighter'' leads to a token distribution similar to ``male firefighter,'' it suggests that the tokens $\mathbf{z}_i$ are implicitly biased toward that demographic, even \emph{before} any visual decoding procedure. Hence, the major source of demographic bias remains in the upstream processes: the autoregressive image token generation. In the following section, we describe the approach we developed to mitigate the debias in image generation by focusing on \emph{LM}.




\vspace{-1em}
\section{Method}
\label{sec:method}

To effectively reduce bias in U-MLLM, a training dataset that \emph{explicitly} includes diverse demographic attributes is of importance. In general, web-scale data under-represents certain groups or inherently skews toward certain stereotypes. By synthesizing balanced data—where each demographic group appears in similar proportion- we can give the model a stronger signal to reduce its biases during training. This can ensure that the U-MLLM is trained on a \emph{controlled distribution} counteracts the bias present in real-world data distribution.

\subsection{Training Data Generation}
\label{subsection: training data generation}

We leveraged diffusion model FLUX.1-dev\cite{flux2023} to synthesize our training data as follows:
\vspace{-1em}
\begin{enumerate}[leftmargin=1em]
    \item \textbf{Base Prompt Selection.}  
    We begin by collecting a set of \emph{base prompts} $x_{neutral}$ (for example, ``a portrait of a \{occupation\}''), where occupations are drawn from a publicly available dataset\cite{shen2024finetuningtexttoimagediffusionmodels}. 
    
    \item \textbf{Demographic Augmentation.}  
    For each base prompt $x_{neutral}$, we augment it with demographic attributes $d \in \{d_1, d_2, \dots\}$, resulting in \emph{augmented prompts} $x_{\text{aug}}(d_i)$. For example, starting with the base prompt ``a photo of a nurse'' we generate augmented prompts such as ``a photo of a male nurse'' and ``a photo of an Asian nurse.''  
    
    \item \textbf{Image Generation via Diffusion Model.}  
    For each $x_{\text{aug}}(d_i)$, we feed it into FLUX.1-dev\cite{flux2023} model to obatin a set of images $\{y_1, y_2, \dots\}$. This step ensures diversity in the visual representations of each demographic group.
    
    \item \textbf{Paired Dataset Creation.}  
    To create the dataset, we pair each generated set of images $y_k$ with its corresponding base prompt $x_{neutral}$. Before pairing, we use the ChatGPT-4o model to paraphrase the base prompt to introduce linguistic diversity while maintaining neutrality. For instance, the base prompt ``a photo of a nurse'' can be paraphrased as ``a photo of an individual in the nursing profession.'' The resulting synthetic dataset pairs each neutral prompt with a set of images, where each image corresponds to a specific demographic group:
    \[
        \mathcal{D}_{\mathrm{syn}} 
        \;=\; \bigl\{\,(x_{\mathrm{neutral}},\, y_{d_1},..., y_{d_K})\,\bigr\}.
    \]
\end{enumerate}

Because we explicitly injected demographic attributes into the prompts, $\mathcal{D}_{\mathrm{syn}}$ spans multiple genders, ethnicities, and roles—offering more balanced coverage than typical real-world data.

\subsection{Balanced Preference Loss}

\begin{figure*}[ht]
    \centering
    \vspace{-0.25cm}
    \includegraphics[width=0.8\textwidth]{figs_tables/balance_preference_optimization.pdf}
    \vspace{-0.5cm}
    \caption{The optimization objective is to minimize the deviation of preference between different demographic group.}
    \vspace{-0.5cm}
    \label{fig:bpo-loss}
\end{figure*}

% \paragraph{Autoregressive Likelihood.}  
% We denote the probability of generating a token sequence $z = \{z_1,\dots,z_T\}$ given prompt $x$ by:
% \begin{equation}
%     P_\theta(z \mid x) \;=\; \prod_{t=1}^{T} P_\theta\bigl(z_t \,\big|\; z_{<t},\, x\bigr).
%     \label{eq:autoregressive-prob}
% \end{equation}
% Typical training optimizes the negative log-likelihood:
% \begin{equation}
%     \mathcal{L}_{\mathrm{NLL}}(\theta) 
%     \;=\; 
%     - \sum_{(x,z)\,\in\,\mathcal{D}} \log P_\theta(z \mid x),
% \end{equation}
% where $\mathcal{D}$ is a dataset of (prompt, image) pairs, and $z$ is the discrete token sequence encoding the image generated by the image encoder.

% We now define a \emph{balanced preference loss} to encourage the model to generate image samples with evenly distributed demographic attributes. 


Existing preference-optimization approaches focus on \emph{aligning} the model’s response to \emph{user-specified} preferences (e.g., “prefer polite over rude text”). By contrast, in this study we seek \emph{balance} among demographic attributes---for instance, the model should equally generate \{female, male\} or \{Asian, Black, Indian, White\} under neutral prompts, rather than overwhelmingly generate one demographic group. This shift in perspective means we want the \textbf{absolute difference in preference} across demographics to be \emph{minimized}, rather than pushing a single demographic to be more or less likely in isolation. To encourage equal representation, as shown in \autoref{fig:bpo-loss}, we aim to penalize deviations by minimizing
\begin{equation}
\min ||p(y_{d_i} \succ y_{d_j} \mid x) -
p(y_{d_j} \succ y_{d_i} \mid x) ||
\end{equation}

In other words, if the model is more likely to generate $y_{d_1}$ (image with demographic label $d_1$) than $y_{d_j}$ (image with demographic label $d_j$), it will be penalized, which encourages balance of the output distribution, and vice versa.

\paragraph{Definition} Concretely, let $y_{d_i}, y_{d_j}$ be images from different demographic groups. We define a \emph{balanced odds} penalty:
\begin{equation}
  \mathcal{L}_{\mathrm{bal}}(\theta)
  \;=\;
  \log\left[ 1 + \left( 
    \sigma\!\Biggl(
      \log\text{OR}_\theta(y_{d_i}, y_{d_j}) 
    \Biggr) - \frac{1}{2} \right)^2 \right]
  \label{eq:bal_twogroups}
\end{equation}

Minimizing \eqref{eq:bal_twogroups} can lead to $p(y_{d_i} \succ y_{d_j} \mid x)$ closer to $p(y_{d_j} \succ y_{d_i} \mid x)$, thus encouraging the even distribution of generation among demographic attributes. 

\paragraph{Extension to Multiple Groups.}  
For race attributes, we have multiple demographic groups. Let's define $K$ to be the number of race categories. The auxiliary loss to penalize demographic odds of different attributes deviating from each other:
\begin{equation}
  \mathcal{L}_{\mathrm{bal}}(\theta)
  \;=\;
  \sum_{k=1}^{K}
  \mathcal{L}_{\mathrm{bal}}^{(d_{k},d_{(k+1) \% K})}(\theta).
  \label{eq:multigroup}
\end{equation}


\paragraph{Two-Stage Training} We adopt a straightforward two-stage procedure as illustrated in Algorithm  \ref{alg:balanced_finetuning}.
\begin{enumerate}[leftmargin=1em]
    \item \textbf{Supervised Finetuning (SFT).} 
    We begin by finetuning the U-MLLM on a supervised dataset of \{\emph{prompt}, \emph{image tokens}\} pairs to ensure high-quality image generation and semantic fidelity. This yields an SFT model $\pi_{\text{SFT}}$.
    \item \textbf{Balanced Preference Optimization.} 
    We then apply a reference-free odds-ratio penalty via Eq. \ref{eq:multigroup} to make the model equally prefer multiple demographic attributes.   
\end{enumerate}

Our approach \emph{directly} integrates demographic balance into the policy’s objective via~\autoref{eq:multigroup}. By including samples with multiple demographic attributes in the balanced dataset $\mathcal{D}_{\text{bal}}$, the model is enabled to generalize towards each demographic group. The balanced preference loss $\mathcal{L}_{\mathrm{bal}}$ then reduces the inter-difference among demographic groups for the model's preference. Overall, this two-stage method ensures the model maintains good generation quality (from Stage 1) while significantly reducing \emph{absolute preference} differences between demographic attributes (from Stage 2).



 
\vspace{-1em}
\section{Experiment}

\subsection{Experimental Setup}
\label{subsec:exp_setup}

% In this section, we list the experimental setups. First, we describe the datasets used for training and evaluation, the U-MLLMs for comparison, and the overall metrics used for measuring bias and image quality. Our evaluation setup follows the same metrics and measurements as previous research\cite{shen2024finetuningtexttoimagediffusionmodels}, ensuring comparability across different models.

% \subsection{Datasets}

\paragraph{Models.}
In our study, we evaluate the generation bias with respect to gender and race for latest U-MLLMs that can both understand and generate visual content. The models we considered include VILA-U\cite{wu2024vila}, TokenFlow\cite{qu2024tokenflow}, Emu3\cite{wang2024emu3}, Janus\cite{wu2024janus}, Show-o\cite{xie2024showo}, Janus-Pro\cite{chen2025januspro}. Since VILA-U was the only model with available training code when we began this study, we fine-tuned it into various variants for comparison. For VILA-U, we compare its variants: \textbf{\emph{(1). }Prompt engineering}: original VILA-U model with a modified prompt to explicitly ask for diversity in image generation. \textbf{\emph{(2). }Understanding Finetuning(\textsf{I~$\rightarrow$~T})}: VILA-U finetuned on the dataset in \emph{image-to-text} fashion.  \textbf{\emph{(3). }Generation Finetuning(\textsf{T~$\rightarrow$~I})}: VILA-U finetuned on the same balanced data in \emph{text-to-image} fashion.\textbf{\emph{(4). }Balanced Preference Loss}: VILA-U finetuned using our proposed balanced preference optimization for visual generation. (Sec. \ref{sec:method}).
\vspace{-1.5em}
\paragraph{Evaluation Data.}
For evaluation, we collect a set of occupation-based generation prompts (e.g. ``nurse,'' ``data engineer,'' ``senator'') that have been used in previous studies\cite{shen2024finetuningtexttoimagediffusionmodels} to assess diffusion models' biases concerning gender and race. These prompts are 50 in size and are publicly available. We utilized these prompts to evaluate various U-MLLMs under consistent conditions: We generate $N = 160$ images for each test prompt and use an attribute classifier\cite{shen2024finetuningtexttoimagediffusionmodels} to determine the demographic attribute labels.
\vspace{-1.5em}

\paragraph{Finetuning Data.}
For training, we collected a separate set of occupation-based generation prompts\cite{shen2024finetuningtexttoimagediffusionmodels} (e.g., ``scientist,'' ``firefighter'') and generated \emph{synthetic} images using a diffusion model, as described in Section~\ref{sec:method}. Our dataset comprises 1,000 training prompts, resulting in 10,000 images per demographic group. We consider six demographic groups: male, female, White, Black, Asian, and Indian. Our training dataset consists of 10,000 samples, each containing a text prompt and six images from different demographic groups.
 
\textbf{Metrics and Evaluation} We follow prior study\cite{shen2024finetuningtexttoimagediffusionmodels}, the same \textbf{bias metrics} and \textbf{image quality metrics} are adapted:
\begin{itemize}[leftmargin=*]
    \vspace{-1em}
    \item \textbf{Bias Metrics}: Representation Disparity (RD) 
    measures how often each demographic attribute appears in generated outputs given a neutral prompt (e.g., ``CEO''). We leverage the classifier model provided by prior study to predict the attributes from generated images. RD is calculated as:
    \[
            \mathrm{bias}(\mathbf{P}) 
            \;=\; 
            \frac{1}{K(K-1)/2} \sum_{i,j \in [K] : i < j} \bigl|\mathrm{freq}(i) - \mathrm{freq}(j)\bigr|,
    \]
    where $\mathrm{freq}(i)$ is the frequency of demographic $d_i$ in the image generation samples. A lower RD indicates less biased and more balanced demographic representation.
    \vspace{-1em}
    \item \textbf{Image Quality Metrics}: we utilized a range of metrics for image quality.  \textbf{\emph{(1)}. CLIP Score}
    Measures alignment between generated image and textual prompt. Higher scores indicate better semantic matching.
\textbf{\emph{(2)}. CLIP IQA}:
    Image quality assessment using CLIP's visual-textual understanding. Evaluates aspects such as sharpness without reference images.
 \textbf{\emph{(3)}. Inception Score}:
    Metric used for evaluating quality and diversity of generated images. Higher scores indicate better quality.
\end{itemize}
\vspace{-1em}
\paragraph{Evaluation Protocol.}
First, we generate $N = 160$ images for each test prompt. Next, we apply a demographic classifier from a previous study \cite{shen2024finetuningtexttoimagediffusionmodels} to predict the labels of demographic attributes for each image. These labeled attributes are then used to compute the overall bias and semantic score based on the above metrics. 
\vspace{-1em}
\paragraph{Training Procedure.} \emph{Base Learning Rate}: We start with a fixed learning rate (for example, $1\times10^{-4}$) for fine-tuning. \emph{Batch Size}: Ranges from 8 to 32, depending on the setup. \emph{Number of Steps}: For the first stage, we fine-tune for up to 10 epochs,  checking bias and quality after the training.  For the second stage, the epoch is chosen to be 1 to 2. We use the LoRA\cite{hu2021loralowrankadaptationlarge} method for finetuning; the rank is 32 for all experiments. 

% \paragraph{Hardware Setup.}
% All experiments are conducted on machines with GPUs (e.g., NVIDIA A100). The training duration depends on the size of the finetuning dataset, typically ranging from a few hours to a day.

% \medskip
% In the next section, we present the experimental results and analysis of the bias mitigation achieved by various approaches, as well as comparisons with baseline models.


\vspace{-1em}
\subsection{Experimental Results}
\label{sec:exp_results}

 
% This is for epoch to be 10
\newcommand{\graysmall}[1]{{\color{gray}\scriptsize #1}}
\newcommand{\STAB}[1]{\begin{tabular}{@{}c@{}}#1\end{tabular}}
{
\begin{table*}[ht]
    \setlength{\tabcolsep}{8pt}
    \centering
    \small
    \vspace{-0.5cm}
    \caption{Image generation bias.}
    \label{table:gender_race_bias}
    \adjustbox{max width=1.0\textwidth}{%
    \begin{tabular}{c | r | c c c c c c c}
        \toprule
        \multirow{2}{*}{\STAB{\rotatebox[origin=c]{90}{Debias:~}}} & \multirow{2}{*}{Method} & \multicolumn{3}{c}{Bias $\downarrow$} & \multicolumn{3}{c}{Semantics Preservation $\uparrow$} \\
        \cmidrule(lr){3-5}
        \cmidrule(lr){6-8}
         & & Gender & Race & G.$\times$R. & CLIP-S  & CLIP-IQA & Inception \\
        \cmidrule(r){2-8}
        & Stable Diffusion & 0.67  & 0.42  & 0.21  & ---   & ---   & ---   \\
        & Janus           & 0.87  & 0.43  & 0.23  & 27.44 & 0.69  & 2.27  \\
         & Janus-Pro           & 0.90  & 0.48  & 0.24  & 27.62 & 0.82  & 1.79  \\
        & Show-o          & 0.85  & 0.48  & 0.24  & 27.16 & 0.86  & 1.79  \\
        & TokenFlow       & 0.84  & 0.47  & 0.24  & 27.17 & 0.84  & 2.34  \\
        & Emu3            & 0.83  & 0.42  & 0.22  & 27.93 & 0.89  & 2.29  \\
        & VILA-U          & 0.89  & 0.48  & 0.24  & 28.24 & 0.84  & 1.87  \\
        \hline
        % --- Debias: Gender Section ---
        \multirow{4}{*}{\STAB{\rotatebox[origin=c]{90}{Gender}}}
        & Prompt Engineering      & 0.56 & 0.49 & 0.23 & 28.51  & 0.82 & 1.91 \\
        & Finetune(\textsf{I~$\rightarrow$~T}) & 0.83 & 0.42 & 0.22 & 28.49 & 0.83 & 2.28  \\
        & Finetune(\textsf{T~$\rightarrow$~I})  & 0.27 & 0.51 & 0.23 & 27.66 & 0.77 & 1.85 \\
        & BPO & \textbf{0.25} & 0.50 & 0.22 & 27.74 & 0.77 & 2.10                 \\
        \hline
        % --- Debias: Race Section (unchanged) ---
        \multirow{5}{*}{\STAB{\rotatebox[origin=c]{90}{Race}}}
        & Prompt Engineering       & 0.56  & 0.49  & 0.23  & 28.51  &  0.82  &   1.91 \\
        & Finetune(\textsf{I~$\rightarrow$~T}) & 0.78  & 0.44  & 0.22  & 28.14  &  0.80 & 2.54    \\
        & Finetune(\textsf{T~$\rightarrow$~I})  & 0.83  & \textbf{0.23}  & 0.17  & 27.98  & 0.80  & 1.98 \\
        & BPO  &0.78 & 0.26 & 0.18 &  27.66 & 0.81   & 2.31&   \\
        \hline
        % --- Debias: G.×R. Section (unchanged) ---
        \multirow{5}{*}{\STAB{\rotatebox[origin=c]{90}{G.$\times$R.}}}
        & Prompt Engineering       & 0.59  & 0.33  & 0.18  & 28.09  &  0.80 & 1.87  \\
        & Finetune(\textsf{I~$\rightarrow$~T}) & 0.86 & 0.45 & 0.23 & 28.34 & 0.82 &2.22 \\
        & Finetune(\textsf{T~$\rightarrow$~I})  & 0.46  & 0.32  & 0.17  & 27.90  &  0.78 & 1.91  \\
        & BPO  & 0.52 & 0.26 & \textbf{0.15} & 27.78 & 0.80 & 2.06 \\
        \bottomrule
    \end{tabular}
    }
    \vspace{-.4cm}
\end{table*}
}

Table~\ref{table:gender_race_bias} summarizes the results for generation bias evaluation on a range of U-MLLMs and especially the VILA-U model with a few debiasing strategies applied. We measure \emph{gender} bias, \emph{race} bias, and their intersection (\emph{G.$\times$R.}), together with semantics-preservation metrics. Lower bias scores indicate more fairness, while higher semantics scores indicate more fidelity in the generation.
\vspace{-1em}
\paragraph{Baseline Models.} As shown in Table~\ref{table:gender_race_bias}, Stable Diffusion exhibits moderate gender bias (0.67) and race bias (0.42). Among U-MLLMs, Janus-series (Janus, Janus-Pro) show higher overall bias in gender (0.87–0.90), as well as high race and G×R scores (0.43–0.48 and 0.23–0.24). Show-o and TokenFlow both witness slightly lower bias. Compared with other U-MLLMs, Emu3 stands out with low gender (0.83) and G×R (0.22) biases, along with good semantic scores. Finally, VILA-U exhibits strong text–image alignment (CLIP-S of 28.24) but has high gender (0.89) and intersectional (0.24) bias, highlighting the need for reducing bias in U-MLLMs.
\vspace{-1em}
\paragraph{Debiasing Gender.} Table~\ref{table:gender_race_bias} presents four methods to reduce gender bias in VILA-U. 
\emph{Prompt Engineering} witness moderate bias reduction (Gender = 0.56) while preserving high semantic alignment (CLIP-S = 28.51). 
\emph{Finetune (I~$\rightarrow$~T)} achieved the smallest reduction in gender bias (0.83) even though it witnessed a little increase in semantics (CLIP-S = 28.49). 
By contrast, \emph{Finetune (T~$\rightarrow$~I)} drastically lowers the gender bias to 0.27. However, it came with the side effect that image quality is lower (CLIP-IQA = 0.77). 
Finally, \emph{BPO} (Balanced Preference Optimization) further lowers gender bias (0.25) while preserving reasonable visual fidelity (Inception = 2.10), indicating that our approach is effective for mitigating stereotypical associations without compromising generation quality for U-MLLMs.
\vspace{-1em}
\paragraph{Debiasing Race.}
Table~\ref{table:gender_race_bias} shows the results for mitigating race bias in the VILA-U. 
\emph{Prompt Engineering} achieves a moderate bias reduction (Race = 0.49) while retaining a high level of semantic alignment (CLIP-S = 28.51). 
\emph{Finetune (I~$\rightarrow$~T)} yields Race = 0.44 and slightly lower semantic scores (CLIP-S = 28.14). 
By contrast, \emph{Finetune (T~$\rightarrow$~I)} significantly reduces the race bias to 0.23, albeit with some drop in semantic fidelity (Inception = 1.98). \emph{BPO} also lower the race bias a lot(0.26), and achieved good semantic fidelity (Inception 2.31) in the meantime.
\vspace{-1em}
\paragraph{Debiasing Intersectional Attributes (G$\times$R).}
As shown under \emph{Debias: G$\times$R} in Table~\ref{table:gender_race_bias}, we evaluate methods targeting \emph{both} gender and race simultaneously. 
\emph{Prompt Engineering} achieves a moderately low intersectional bias (G$\times$R = 0.18) while preserving acceptable semantic fidelity (CLIP-S = 28.09). 
\emph{Finetune (I~$\rightarrow$~T)} reduces intersectional bias to 0.23, accompanied by strong semantic alignment (CLIP-S = 28.34), whereas 
\emph{Finetune (T~$\rightarrow$~I)} further lowers G$\times$R to 0.17 but at the cost of slightly reduced image quality (Inception = 1.91). \emph{BPO} achieved the lowest bias score in gender-race intersection as 0.15. It also achieved high semantic fidelity with an Inception score of 2.06.


\subsection{Discussion}
\begin{figure*}[ht]
    \centering
    \vspace{-0.2cm}
    \includegraphics[width=1.0\textwidth]{figs_tables/samples.pdf}
    \vspace{-0.8cm}
    \caption{Given the test prompt: “a photo of the face of a electrical and electronics repairer, a person”. Images generated by: VILA-U (bottom-left), our method reduced bias in gender(bottom-right), race(top-left), gender-race(top-right). For each image, the first color-coded bar represents the predicted gender: \textcolor{blue}{male} or \textcolor{red}{female}. The second bar represents the predicted race: \textcolor{green}{While}, \textcolor{orange}{Asian}, Black, or \textcolor{brown}{Indian}. The height of each bar indicates the prediction confidence.}
    \label{fig:vilau-samples-compara}
    \vspace{-0.5cm}
\end{figure*}
\paragraph{Qualitive Samples}

In \autoref{fig:vilau-samples-compara}, we compare the VILA-U model with our method in different debiasing dimensions. Our approach evidently enables the model to generate more diverse samples. Our method effectively mitigates stereotypes and reduces the risk of perpetuating inequalities in automatically generated content.
\vspace{-1em}
\paragraph{Is there understanding bias in U-MLLMs?}
We examine \emph{visual understanding bias} to investigate bias in multimodal language models further. Specifically, we provide each U-MLLM with a prompt of the form \textit{``What is the gender of \{\texttt{occupation}\}? Answer with options male, female, unknown.''} while providing an input \emph{empty image} to ensure a fair comparison. We then sample 100 responses per prompt under a fixed sampling strategy. From these textual outputs, we extract the demographic attributes that each model most frequently associates with a given occupation. Then we use $RD$ metrics to calculate the bias.

\begin{figure}[ht]
    \centering
\includegraphics[width=0.48\textwidth]{figs_tables/compare_understanding_w_generation.pdf}
    \vspace{-1cm}
    \caption{Compare the understanding and generation bias.}
    \vspace{-0.8cm}
    \label{fig:compare_bias}
\end{figure}


Results in Figure~\ref{fig:compare_bias} indicate that most models exhibit a pronounced \emph{understanding bias}; they systematically infer one gender or demographic for an occupation even when provided no actual visual cues. Interestingly, we observe cases of \emph{partial alignment}. For example, while the Janus model shows a moderate understanding bias in its responses---often refusing to label an individual’s gender or race explicitly, it nonetheless exhibits a substantial generation bias when prompted to create images. In other words, the model’s \emph{understanding-level} alignment (refusal to answer demographic queries) does not carry over to its \emph{image-generation} behavior, which remains biased. This highlights that merely aligning a model’s visual understanding to minimize bias in VQA scenarios is insufficient if its generative capability still systematically favors particular demographics. 

\paragraph{Can debiasing understanding help debias generation?}
Prior work\cite{tong2024metamorphmultimodalunderstandinggeneration} suggests that enhancing a model’s capability for image understanding can indirectly improve its performance in image generation. Motivated by this, we use the diffusion-generated images and their demographic labels (derived from prompts) to train the model in a \emph{discriminative} fashion. The resulting model fails to reduce demographic bias in \emph{image generation}. In other words, the ability to classify or reason about demographic attributes in images does not carry over to balanced \emph{generation} of images with varied demographics. 

These findings imply that bias in \emph{understanding} (e.g., naming attributes) and bias in \emph{generation} (e.g., synthesizing certain demographics disproportionately) may arise from distinct internal mechanisms. Whereas understanding tasks rely on discriminative representations, generative tasks involve autoregressive sampling or likelihood maximization that can perpetuate different stereotypes. Therefore, improving one aspect of visual-linguistic alignment does not necessarily resolve the other. Designing more advanced methods that holistically address both \emph{comprehension} and \emph{production} biases is essential for fair multimodal generation.

\vspace{-1em}
\paragraph{Generalization to other models.}

Table~\ref{tab:tokenflow_comparison} shows the results of applying similar finetuning strategies to \textbf{TokenFlow}. While TokenFlow initially exhibits a high gender bias (0.84), simple interventions such as \emph{prompting engineering} reduce its bias to 0.69.  \emph{Finetuning} approach further lowers it to 0.55, at a minor cost in semantics scores. These findings suggest that finetuning can be extended beyond VILA-U to other U-MLLM toward fairer image generation.

% \paragraph{Generalization to non-templated prompts.}

% \paragraph{Generalization to multi-face image generation.}


\section{Related Work}
\label{sec:relatedwork}

\subsection{Current AI Tools for Social Service}
\label{subsec:relatedtools}
% the title I feel is quite broad

Harnessing technology for social good has always been a grand challenge in social service \cite{berzin_practice_2015}. As early as the 90s, artificial neural networks and predictive models have been employed as tools for risk assessments, decision-making, and workload management in sectors like child protective services and mental health treatment \cite{fluke_artificial_1989, patterson_application_1999}. The recent rise of generative AI is poised to further advance social service practice, facilitating the automation of administrative tasks, streamlining of paperwork and documentation, optimisation of resource allocation, data analysis, and enhancing client support and interventions \cite{fernando_integration_2023, perron_generative_2023}.

Today, AI solutions are increasingly being deployed in both policy and practice \cite{goldkind_social_2021, hodgson_problematising_2022}. In clinical social work, AI has been used for risk assessments, crisis management, public health initiatives, and education and training for practitioners \cite{asakura_call_2020, gillingham2019can, jacobi_functions_2023, liedgren_use_2016, molala_social_2023, rice_piloting_2018, tambe_artificial_2018}. AI has also been employed for mental health support and therapeutic interventions, with conversational agents serving as on-demand virtual counsellors to provide clinical care and support \cite{lisetti_i_2013, reamer_artificial_2023}.
% commercial solutions include Woebot, which simulates therapeutic conversation, and Wysa, an “emotionally intelligent” AI coach, powered by evidenced-based clinical techniques \cite{reamer_artificial_2023}. 
% Non-clinical AI agents like Replika and companion robots can also provide social support and reduce loneliness amongst individuals \cite{ahmed_humanrobot_2024, chaturvedi_social_2023, pani_can_2024, ta_user_2020}.

Present research largely focuses on \textit{\textbf{AI-based decision support tools}} in social service \cite{james_algorithmic_2023, kawakami2022improving}, especially predictive risk models (PRMs) used to predict social service risks and outcomes \cite{gillingham2019can, van2017predicting}, like the Allegheny Family Screening Tool (AFST), which assesses child abuse risk using data from US public systems \cite{chouldechova_case_2018, vaithianathan2017developing}. Elsewhere, researchers have also piloted PRMs to predict social service needs for the homeless using Medicaid data\cite{erickson_automatic_2018, pourat_easy_2023}, and AI-powered algorithms to promote health interventions for at-risk populations, such as HIV testing among Californian homeless \cite{rice_piloting_2018, yadav_maximizing_2017}.

\subsection{Generative AI and Human-AI Collaboration}
\label{subsec:relatedworkhaicollaboration}
Beyond decision-making algorithms and PRMs, advancements in generative AI, such as large language models (LLMs), open new possibilities for human-AI (HAI) collaboration in social services. 
LLMs have been called "revolutionary" \cite{fui2023generative} and a "seismic shift" \cite{cooper2023examining}, offering "content support" \cite{memmert2023towards} by generating realistic and coherent responses to user inputs \cite{cascella2023evaluating}. Their vastly improved capabilities and ubiquity \cite{cooper2023examining} makes them poised to revolutionise work patterns \cite{fui2023generative}. Generative AI is already used in fields like design, writing, music, \cite{han2024teams, suh2021ai, verheijden2023collaborative, dhillon2024shaping, gero2023social} healthcare, and clinical settings \cite{zhang2023generative, yu2023leveraging, biswas2024intelligent}, with promising results. However, the social service sector has been slower in adopting AI \cite{diez2023artificial, kawakami2023training}.

% Yet, the social service sector is one that could perhaps stand to gain the most from AI technologies. As Goldkind \cite{goldkind_social_2021} writes, social service, as a "values-centred profession with a robust code of ethics" (p. 372), is uniquely placed to inform the development of thoughtful algorithmic policy and practice. 
Social service, however, stands to benefit immensely from generative AI. SSPs work in time-poor environments \cite{tiah_can_2024}, often overwhelmed with tedious administrative work \cite{meilvang_working_2023} and large amounts of paperwork and data processing \cite{singer_ai_2023, tiah_can_2024}. 
% As such, workers often work in time-poor environments and are burdened with information overload and administrative tasks \cite{tiah_can_2024, meilvang_working_2023}. 
Generative AI is well-placed to streamline and automate tasks like formatting case notes, formulating treatment plans and writing progress reports, which can free up valuable time for more meaningful work like client engagement and enhance service quality \cite{fernando_integration_2023, perron_generative_2023, tiah_can_2024, thesocialworkaimentor_ai_nodate}. 

Given the immense potential, there has been emerging research interest in HAI collaboration and teamwork in the Human-Computer Interaction and Computer Supported Cooperative Work space \cite{wang_human-human_2020}. HAI collaboration and interaction has been postulated by researchers to contribute to new forms of HAI symbiosis and augmented intelligence, where algorithmic and human agents work in tandem with one another to perform tasks better than they could accomplish alone by augmenting each other's strengths and capabilities  \cite{dave_augmented_2023, jarrahi_artificial_2018}.

However, compared to the focus on AI decision-making and PRM tools, there is scant research on generative AI and HAI collaboration in the social service sector \cite{wykman_artificial_2023}. This study therefore seeks to fill this critical gap by exploring how SSPs use and interact with a novel generative AI tool, helping to expand our understanding of the new opportunities that HAI collaboration can bring to the social service sector.

\subsection{Challenges in AI Use in Social Service}
\label{subsec:relatedworkaiuse}

% Despite the immense potential of AI systems to augment social work practice, there are multiple challenges with integrating such systems into real-life practice. 
Despite its evident benefits, multiple challenges plague the integration of AI and its vast potential into real-life social service practice.
% Numerous studies have investigated the use of PRMs to help practitioners decide on a course of action for their clients. 
When employing algorithmic decision-making systems, practitioners often experience tension in weighing AI suggestions against their own judgement \cite{kawakami2022improving, saxena2021framework}, being uncertain of how far they should rely on the machine. 
% Despite often being instructed to use the tool as part of evaluating a client, 
Workers are often reluctant to fully embrace AI assessments due to its inability to adequately account for the full context of a case \cite{kawakami2022improving, gambrill2001need}, and lack of clarity and transparency on AI systems and limitations \cite{kawakami2022improving}. Brown et al. \cite{brown2019toward} conducted workshops using hypothetical algorithmic tools 
% to understand service providers' comfort levels with using such tools in their work,
and found similar issues with mistrust and perceived unreliability. Furthermore, introducing AI tools can  create new problems of its own, causing confusion and distrust amongst workers \cite{kawakami2022improving}. Such factors are critical barriers to the acceptance and effective use of AI in the sector.

\citeauthor{meilvang_working_2023} (2023) cites the concept of \textit{boundary work}, which explores the delineation between "monotonous" administrative labour and "professional", "knowledge based" work drawing on core competencies of SSPs. While computers have long been used for bureaucratic tasks like client registration, the introduction of decision support systems like PRMs stirred debate over AI "threatening professional discretion and, as such, the profession itself" \cite{meilvang_working_2023}. Such latent concerns arguably drive the resistance to technology adoption described above. Generative AI is only set to further push this boundary, 
% these concerns are only set to grow in tandem with the vast capabilities of generative and other modern AI systems. Compared to the relatively primitive AI systems in past years, perceived as statistical algorithms \cite{brown2019toward} turning preset inputs like client age and behavioural symptoms \cite{vaithianathan2017developing} into simple numerical outputs indicating various risk scores, modern AI systems are vastly more capable: LLMs 
with its ability to formulate detailed reports and assessments that encroach upon the "core" work of SSPs.
% accept unrestricted and unstructured inputs and return a range of verbose and detailed evaluations according to the user's instructions. 
Introducing these systems exacerbate previously-raised issues such as understanding the limitations and possibilities of AI systems \cite{kawakami2022improving} and risk of overreliance on AI \cite{van2023chatgpt}, and requires a re-examination of where users fall on the algorithmic aversion-bias scale \cite{brown2019toward} and how they detect and react to algorithmic failings \cite{de2020case}. We address these critical issues through an empirical, on-the-ground study that to our knowledge is the first of its kind since the new wave of generative AI.

% W 

% Yet, to date, we have limited knowledge on the real-world impacts and implications of human-AI collaboration, and few studies have investigated practitioners’ experiences working with and using such AI systems in practice, especially within the social work context \cite{kawakami2022improving}. A small number of studies have explored practitioner perspectives on the use of AI in social work, including Kawakami et al. \cite{kawakami2022improving}, who interviewed social workers on their experiences using the AFST; Stapleton et al. \cite{stapleton_imagining_2022}, who conducted design workshops with caseworkers on the use of PRMs in child welfare; and Wassal et al. \cite{wassal_reimagining_2024}, who interviewed UK social work professionals on the use of AI. A common thread from all these studies was a general disregard for the context and users, with many practitioners criticising the failure of past AI tools arising from the lack of participation and involvement of social workers and actual users of such systems in the design and development of algorithmic systems \cite{wassal_reimagining_2024}. Similarly, in a scoping review done on decision-support algorithms in social work, Jacobi \& Christensen \cite{jacobi_functions_2023} reported that the majority of studies reveal limited bottom-up involvement and interaction between social workers, researchers and developers, and that algorithms were rarely developed with consideration of the perspective of social workers.
% so the \cite{yang_unremarkable_2019} and \cite{holten_moller_shifting_2020} are not real-world impacts? real-world means to hear practitioner's voice? I feel this is quite important but i didnt get this point in intro!

% why mentioning 'which have largely focused on existing ADS tools (e.g., AFST)'? i can see our strength is more localized, but without basic knowledge of social work i didnt get what's the 'departure' here orz
% the paragraph is great! do we need to also add one in line 20 21?

\subsection{Designing AI for Social Service through Participatory Design}
\label{subsec:relatedworkpd}
% i think it's important! but maybe not a whole subsection? but i feel the strong connection with practitioners is indeed one of our novelties and need to highlight it, also in intro maybe
% Participatory design (PD) has long been used extensively in HCI \cite{muller1993participatory}, to both design effective solutions for a specific community and gain a deep understanding of that community. Of particular interest here is the rich body of literature on PD in the field of healthcare \cite{donetto2015experience}, which in this regard shares many similarities and concerns with social work. PD has created effective health improvement apps \cite{ryu2017impact}, 

% PD offers researchers the chance to gather detailed user requirements \cite{ryu2017impact}...

Participatory design (PD) is a staple of HCI research \cite{muller1993participatory}, facilitating the design of effective solutions for a specific community while gaining a deep understanding of its stakeholders. The focus in PD of valuing the opinions and perspectives of users as experts \cite{schuler_participatory_1993} 
% In recent years, the tech and social work sectors have awakened to the importance of involving real users in designing and implementing digital technologies, developing human-centred design processes to iteratively design products or technologies through user feedback 
has gained importance in recent years \cite{storer2023reimagining}. Responding to criticisms and failures of past AI tools that have been implemented without adequate involvement and input from actual users, HCI scholars have adopted PD approaches to design predictive tools to better support human decision-making \cite{lehtiniemi_contextual_2023}.
% ; accordingly, in social service, a line of research has begun studying and designing for human-AI collaboration with real-world users (e.g. \cite{holten_moller_shifting_2020, kawakami2022improving, yang_unremarkable_2019}).
Section \ref{subsec:relatedworkaiuse} shows a clear need to better understand SSP perspectives when designing and implementing AI tools in the social sector. 
Yet, PD research in this area has been limited. \citeauthor{yang2019unremarkable} (2019), through field evaluation with clinicians, investigated reasons behind the failure of previous AI-powered decision support tools, allowing them to design a new-and-improved AI decision-support tool that was better aligned with healthcare workers’ workflows. Similarly, \citeauthor{holten_moller_shifting_2020} (2020) ran PD workshops with caseworkers, data scientists and developers in public service systems to identify the expectations and needs that different stakeholders had in using ADS tools.

% Indeed, it is as Wise \cite{wise_intelligent_1998} noted so many years ago on the rise of intelligent agents: “it is perhaps when technologies are new, when their (and our) movements, habits and attitudes seem most awkward and therefore still at the forefront of our thoughts that they are easiest to analyse” (p. 411). 
Building upon this existing body of work, we thus conduct a study to co-design an AI tool \textit{for} and \textit{with} SSPs through participatory workshops and focus group discussions. In the process, we revisit many of the issues mentioned in Section \ref{subsec:relatedworkaiuse}, but in the context of novel generative AI systems, which are fundamentally different from most historical examples of automation technologies \cite{noy2023experimental}. This valuable empirical inquiry occurs at an opportune time when varied expectations about this nascent technology abound \cite{lehtiniemi_contextual_2023}, allowing us to understand how SSPs incorporate AI into their practice, and what AI can (or cannot) do for them. In doing so, we aim to uncover new theoretical and practical insights on what AI can bring to the social service sector, and formulate design implications for developing AI technologies that SSPs find truly meaningful and useful.
% , and drive future technological innovations to transform the social service sector not just within [our country], but also on a global scale.

 % with an on-the-ground study using a real prototype system that reflects the state of AI in current society. With the presumption that AI will continue to be used in social work given the great benefits it brings, we address the pressing need to investigate these issues to ensure that any potential AI systems are designed and implemented in a responsible and effective manner.

% Building upon these works, this study therefore seeks to adopt a participatory design methodology to investigate social workers’ perspectives and attitudes on AI and human-AI collaboration in their social work practice, thus contributing to the nascent body of practitioner-centred HCI research on the use of AI in social work. Yet, in a departure from prior work, which have largely focused on existing ADS tools (e.g., AFST) and were situated in a Western context, our paper also aims to expand the scope by piloting a novel generative AI tool that was designed and developed by the researchers in partnership with a social service agency based in Singapore, with aims of generating more insights on wider use cases of AI beyond what has been previously studied.

% i may think 'While the current lacunae of research on applications of AI in social work may appear to be a limitation, it simultaneously presents an exciting opportunity for further research and exploration \cite{dey_unleashing_2023},' this point is already convincing enough, not sure if we need to quote here
% I like this end! it's a good transition to our study design, do we need to mention the localization in intro as well? like we target at singapore

% Given the increasing prominence and acceptance of AI in modern society, 

% These increased capabilities vastly exacerbate the issues already present with a simpler tool like the AFST: the boundaries and limitations of an LLM system are significantly more difficult to understand and its possible use cases are exponentially greater in scope. 

% Put this in discussion section instead?
% Kawakami et al's work "highlights the importance of studying how collaborative decision-making... impacts how people rely upon and make sense of AI models," They conclude by recommending designing tools that "support workers in understanding the boundaries of [an AI system's] capabilities", and implementing design procedures that "support open cultures for critical discussion around AI decision making". The authors outline critical challenges of implementing AI systems, elucidating factors that may hinder their effectiveness and even negatively affect operations within the organisation.


% Is this needed?:
% talk about the strengths of PD in eliciting user viewpoints and knowledge, in particular when it is a field that is novel or where a certain system has not been used or developed or tested before

\section{Conclusion}
\label{sec:limitation_conclusion}

\paragraph{Conclusion}
We examined demographic bias (e.g., gender, race) in unified multimodal large language models (U-MLLMs). We proposed a balanced preference optimization framework that combines supervised finetuning with a preference deviation penalty. Our experiments show that this method effectively reduces bias while maintaining high image quality and semantic fidelity. Moreover, our analysis of understanding versus generation bias underscores the importance of addressing \emph{both} visual reasoning and token-level preference shaping for comprehensive alignment. We hope these findings will guide future work toward building multimodal models with stronger fairness guarantees. Our work also have some limitation as illustrated in \autoref{sec:limitation}.


\paragraph{Impact Statement.}
Our work addresses the challenge of demographic bias in unified multimodal language models, aiming to ensure fair image generation for different demographic groups. Our methods can help mitigate harmful stereotypes and minimize the risk of reinforcing inequalities in automatically generated content. We believe ongoing work from researchers, policymakers, and the communities represented is important to maximize the societal benefits of debiasing techniques while maintaining transparency and fairness in machine learning systems. 




\bibliography{example_paper}
\bibliographystyle{icml2025}


%%%%%%%%%%%%%%%%%%%%%%%%%%%%%%%%%%%%%%%%%%%%%%%%%%%%%%%%%%%%%%%%%%%%%%%%%%%%%%%
%%%%%%%%%%%%%%%%%%%%%%%%%%%%%%%%%%%%%%%%%%%%%%%%%%%%%%%%%%%%%%%%%%%%%%%%%%%%%%%
% APPENDIX
%%%%%%%%%%%%%%%%%%%%%%%%%%%%%%%%%%%%%%%%%%%%%%%%%%%%%%%%%%%%%%%%%%%%%%%%%%%%%%%
%%%%%%%%%%%%%%%%%%%%%%%%%%%%%%%%%%%%%%%%%%%%%%%%%%%%%%%%%%%%%%%%%%%%%%%%%%%%%%%
\newpage
\appendix
\onecolumn
\section{Structure of U-MLLM}
This section represents the structure of U-MLLM and visual tokenizer; all the content in this section is from prior study\cite{wu2024vila}. 

\begin{figure}[ht]
    \centering
    \includegraphics[width=0.9\textwidth]{figs_tables/overview.pdf}
    \caption{\textbf{Overview of framework’s multi-modal training and inference process\cite{wu2024vila}} Visual inputs are converted into discrete tokens and merged with textual tokens to create a unified multi-modal token sequence. This sequence is used in next-token prediction process, which supports a unified training objective. During inference, output tokens are processed through either text detokenizer or vision tower decoder, generating multi-modal content outputs\cite{wu2024vila}.}
    \label{fig:unified_model}
\end{figure}


\begin{figure}[ht]
    \centering
    \includegraphics[width=0.93\textwidth]{figs_tables/vision_tower.pdf}
    \caption{
    \looseness=-1
    \textbf{Overview of unified foundation vision tower\cite{wu2024vila}} Input images are processed by the vision encoder, where features are extracted and discretized using residual quantization. These discrete vision features are then utilized in two ways: they are fed into the vision decoder to reconstruct images and are used to perform text-image alignment. Throughout this process, both the reconstruction loss and contrastive loss are calculated to refine the vision tower, enabling it to generate discrete visual features that are aligned with text\cite{wu2024vila}.}
    \label{fig:vision_tower}
\end{figure}

\newpage
 
\section{Localizing Bias}
\label{subsec:localizing-bias}
\begin{figure*}[ht]
    \centering
    \vspace{-8pt}
    \includegraphics[width=0.93\textwidth]{figs_tables/locating_bias.pdf}
    % \vspace{-15pt}
    \caption{
    \looseness=-1
    \textbf{Detecting bias in LM(top), Vision encoder(bottom);}
    \label{fig:locating_bias}}
    \vspace{-10pt}
\end{figure*}


\section{BPO Algorithm}
\begin{algorithm}[ht]
\caption{Balanced Preference Optimization}
\label{alg:balanced_finetuning}
\textbf{Input}: U-MLLM with parameters $\theta_0$; SFT dataset $\mathcal{D}_{\text{SFT}} = \{(x_i, z_i)\}$ of prompts $x_i$ and image tokens $z_i$; Balanced dataset $\mathcal{D}_{\text{bal}} = \{(x_j, y_{j_1}, \ldots, y_{j_K})\}$, each $x_j$ with multiple demographic variants; Trade-off parameter $\lambda$, total training epochs $N_1, N_2$;
\textbf{Output}: Debiased model parameters $\theta$

\begin{algorithmic}[1]
\STATE \textbf{Stage 1: Supervised Finetuning}
\STATE \quad Initialize $\theta \leftarrow \theta_0$
\FOR{epoch $= 1$ to $N_1$}
    \STATE Sample a minibatch $\{(x_i, z_i)\}$ from $\mathcal{D}_{\text{SFT}}$
    \STATE $\mathcal{L}_{\mathrm{NLL}}(\theta)\;=\; -\sum_{(x_i,z_i)}\log P_\theta(z_i \mid x_i)$
    \STATE Update $\theta \leftarrow \theta - \eta \nabla_{\theta}\,\mathcal{L}_{\mathrm{NLL}}(\theta)$
\ENDFOR

\STATE \textbf{Stage 2: Balanced Preference Optimization}
\FOR{epoch $= 1$ to $N_2$}
    \STATE Sample a minibatch $\{(x_j, y_{j_1},\ldots,y_{j_K})\}$ from $\mathcal{D}_{\text{bal}}$
    \STATE \textbf{Balanced Preference Loss:} 
    \[
       \mathcal{L}_{\mathrm{bal}}(\theta)
       =
       \sum_{k\neq l}
        \mathcal{L}_{\mathrm{bal}}^{(d_{k-1},d_k)}(\theta).
    \]
    \STATE Update $\theta \leftarrow \theta - \eta \nabla_{\theta}\,\mathcal{L}_{\mathrm{bal}}(\theta)$
\ENDFOR
\STATE \textbf{Return} $\theta$
\end{algorithmic}
\end{algorithm}
\newpage
\section{Related Work(continued)}
\label{sec:related_work_continue}
\paragraph{Preference Optimization}  
Direct preference optimization (DPO) \cite{rafailov2024direct} has emerged as a promising technique to address biases in machine learning models, especially LLM. Since then, numerous new loss functions have been proposed~\cite{meng2024simposimplepreferenceoptimization, park2024disentanglinglengthqualitydirect, hong2024orpomonolithicpreferenceoptimization, ethayarajh2024ktomodelalignmentprospect, azar2023generaltheoreticalparadigmunderstand}.
Recent advances \cite{amini2024direct} integrate preference modeling into loss functions to guide models toward balanced outputs, which have rarely been explored for MLLMs.
Based on this, we introduce a novel balanced preference loss tailored for U-MLLMs.
By leveraging demographic attributes during training, the proposed method balances the likelihood of generating outputs across groups without compromising the image quality.


\section{Results for Debiasing TokenFlow}

\begin{table}[ht]
    \centering
    \small % Reduce font size
    \setlength{\tabcolsep}{2pt} % Reduce column spacing
    \vspace{-1.5em}
    \caption{Comparison of Gender Bias and Semantics Preservation Metrics for TokenFlow model.}
    \label{tab:tokenflow_comparison}
    \begin{tabular}{l|c|ccc}
        \toprule
        \textbf{Method} & \textbf{Gender Bias ↓} & \textbf{CLIP-S} & \textbf{CLIP-IQA} & \textbf{Inception} \\
        \midrule
        TokenFlow & 0.84 & 27.17 & 0.84 & 2.34 \\
        Prompting & 0.69 & 27.46 & 0.80 & 2.34 \\
        Finetune & 0.55 & 26.87 & 0.81 & 1.98 \\
        % BPO & 0.55 & 26.87 & 0.81 & 1.98 \\
        \bottomrule
    \end{tabular}
\end{table}

\section{Limitation}
\label{sec:limitation}
\paragraph{Limitation}
Several questions remain open despite the bias reduction achieved by our approach. First, our study primarily centered on \emph{overt} demographic categories (e.g., gender, race). Real-world scenarios may demand addressing \emph{intersectional} or \emph{nuanced} attributes (e.g., age, culture, or religion). Second, many models are not fully open-source, restricting the scope of our evaluations to two publicly available systems. Future research could broaden the range of tested models. Lastly, due to resource constraints, we did not explore alternative preference optimization objectives beyond our current framework. Building on our method to incorporate other debiasing approaches is a promising direction for future work. 


\newpage
\section{Samples from pretrained baseline models}

\begin{figure}[ht]
    \centering
    \includegraphics[width=0.93\textwidth]{figs_tables/prompt_3_vila_u.pdf}
    \caption{
    \looseness=-1
    \textbf{Given the test prompt: “a photo of the face of a electrical and electronics repairer, a person”, images generated by VILA-U.}
    \label{fig:vila-u}}
\end{figure}

\begin{figure}[ht]
    \centering
    \includegraphics[width=0.93\textwidth]{figs_tables/prompt_3_janus.pdf}
    \caption{
    \looseness=-1
    \textbf{Given the test prompt: “a photo of the face of a electrical and electronics repairer, a person”, images generated by Janus}
    \label{fig:janus}}
\end{figure}


\begin{figure}[ht]
    \centering
    \includegraphics[width=0.93\textwidth]{figs_tables/prompt_3_janus_pro.pdf}
    \caption{
    \looseness=-1
    \textbf{Given the test prompt: “a photo of the face of a electrical and electronics repairer, a person”, images generated by Janus-Pro}
    \label{fig:janus-pro}}
\end{figure}


\begin{figure}[ht]
    \centering
    \includegraphics[width=0.93\textwidth]{figs_tables/promtp_3_showo.pdf}
    \caption{
    \looseness=-1
    \textbf{Given the test prompt: “a photo of the face of a electrical and electronics repairer, a person”, images generated by Show-o}
    \label{fig:show-o}}
\end{figure}


\begin{figure}[ht]
    \centering
    \includegraphics[width=0.93\textwidth]{figs_tables/prompt_3_token_flow.pdf}
    \caption{
    \looseness=-1
    \textbf{Given the test prompt: “a photo of the face of a electrical and electronics repairer, a person”, images generated by TokenFlow}
    \label{fig:tokenflow}}
\end{figure}

\begin{figure}[ht]
    \centering
    \includegraphics[width=0.93\textwidth]{figs_tables/prompt_3_emu3.pdf}
    \caption{
    \looseness=-1
    \textbf{Given the test prompt: “a photo of the face of a electrical and electronics repairer, a person”, images generated by Emu3}
    \label{fig:emu3}}
\end{figure}

%%%%%%%%%%%%%%%%%%%%%%%%%%%%%%%%%%%%%%%%%%%%%%%%%%%%%%%%%%%%%%%%%%%%%%%%%%%%%%%
%%%%%%%%%%%%%%%%%%%%%%%%%%%%%%%%%%%%%%%%%%%%%%%%%%%%%%%%%%%%%%%%%%%%%%%%%%%%%%%


\end{document}


% This document was modified from the file originally made available by
% Pat Langley and Andrea Danyluk for ICML-2K. This version was created
% by Iain Murray in 2018, and modified by Alexandre Bouchard in
% 2019 and 2021 and by Csaba Szepesvari, Gang Niu and Sivan Sabato in 2022.
% Modified again in 2023 and 2024 by Sivan Sabato and Jonathan Scarlett.
% Previous contributors include Dan Roy, Lise Getoor and Tobias
% Scheffer, which was slightly modified from the 2010 version by
% Thorsten Joachims & Johannes Fuernkranz, slightly modified from the
% 2009 version by Kiri Wagstaff and Sam Roweis's 2008 version, which is
% slightly modified from Prasad Tadepalli's 2007 version which is a
% lightly changed version of the previous year's version by Andrew
% Moore, which was in turn edited from those of Kristian Kersting and
% Codrina Lauth. Alex Smola contributed to the algorithmic style files.
