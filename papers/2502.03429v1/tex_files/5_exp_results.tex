\vspace{-1em}
\subsection{Experimental Results}
\label{sec:exp_results}

 
% This is for epoch to be 10
\newcommand{\graysmall}[1]{{\color{gray}\scriptsize #1}}
\newcommand{\STAB}[1]{\begin{tabular}{@{}c@{}}#1\end{tabular}}
{
\begin{table*}[ht]
    \setlength{\tabcolsep}{8pt}
    \centering
    \small
    \vspace{-0.5cm}
    \caption{Image generation bias.}
    \label{table:gender_race_bias}
    \adjustbox{max width=1.0\textwidth}{%
    \begin{tabular}{c | r | c c c c c c c}
        \toprule
        \multirow{2}{*}{\STAB{\rotatebox[origin=c]{90}{Debias:~}}} & \multirow{2}{*}{Method} & \multicolumn{3}{c}{Bias $\downarrow$} & \multicolumn{3}{c}{Semantics Preservation $\uparrow$} \\
        \cmidrule(lr){3-5}
        \cmidrule(lr){6-8}
         & & Gender & Race & G.$\times$R. & CLIP-S  & CLIP-IQA & Inception \\
        \cmidrule(r){2-8}
        & Stable Diffusion & 0.67  & 0.42  & 0.21  & ---   & ---   & ---   \\
        & Janus           & 0.87  & 0.43  & 0.23  & 27.44 & 0.69  & 2.27  \\
         & Janus-Pro           & 0.90  & 0.48  & 0.24  & 27.62 & 0.82  & 1.79  \\
        & Show-o          & 0.85  & 0.48  & 0.24  & 27.16 & 0.86  & 1.79  \\
        & TokenFlow       & 0.84  & 0.47  & 0.24  & 27.17 & 0.84  & 2.34  \\
        & Emu3            & 0.83  & 0.42  & 0.22  & 27.93 & 0.89  & 2.29  \\
        & VILA-U          & 0.89  & 0.48  & 0.24  & 28.24 & 0.84  & 1.87  \\
        \hline
        % --- Debias: Gender Section ---
        \multirow{4}{*}{\STAB{\rotatebox[origin=c]{90}{Gender}}}
        & Prompt Engineering      & 0.56 & 0.49 & 0.23 & 28.51  & 0.82 & 1.91 \\
        & Finetune(\textsf{I~$\rightarrow$~T}) & 0.83 & 0.42 & 0.22 & 28.49 & 0.83 & 2.28  \\
        & Finetune(\textsf{T~$\rightarrow$~I})  & 0.27 & 0.51 & 0.23 & 27.66 & 0.77 & 1.85 \\
        & BPO & \textbf{0.25} & 0.50 & 0.22 & 27.74 & 0.77 & 2.10                 \\
        \hline
        % --- Debias: Race Section (unchanged) ---
        \multirow{5}{*}{\STAB{\rotatebox[origin=c]{90}{Race}}}
        & Prompt Engineering       & 0.56  & 0.49  & 0.23  & 28.51  &  0.82  &   1.91 \\
        & Finetune(\textsf{I~$\rightarrow$~T}) & 0.78  & 0.44  & 0.22  & 28.14  &  0.80 & 2.54    \\
        & Finetune(\textsf{T~$\rightarrow$~I})  & 0.83  & \textbf{0.23}  & 0.17  & 27.98  & 0.80  & 1.98 \\
        & BPO  &0.78 & 0.26 & 0.18 &  27.66 & 0.81   & 2.31&   \\
        \hline
        % --- Debias: G.×R. Section (unchanged) ---
        \multirow{5}{*}{\STAB{\rotatebox[origin=c]{90}{G.$\times$R.}}}
        & Prompt Engineering       & 0.59  & 0.33  & 0.18  & 28.09  &  0.80 & 1.87  \\
        & Finetune(\textsf{I~$\rightarrow$~T}) & 0.86 & 0.45 & 0.23 & 28.34 & 0.82 &2.22 \\
        & Finetune(\textsf{T~$\rightarrow$~I})  & 0.46  & 0.32  & 0.17  & 27.90  &  0.78 & 1.91  \\
        & BPO  & 0.52 & 0.26 & \textbf{0.15} & 27.78 & 0.80 & 2.06 \\
        \bottomrule
    \end{tabular}
    }
    \vspace{-.4cm}
\end{table*}
}

Table~\ref{table:gender_race_bias} summarizes the results for generation bias evaluation on a range of U-MLLMs and especially the VILA-U model with a few debiasing strategies applied. We measure \emph{gender} bias, \emph{race} bias, and their intersection (\emph{G.$\times$R.}), together with semantics-preservation metrics. Lower bias scores indicate more fairness, while higher semantics scores indicate more fidelity in the generation.
\vspace{-1em}
\paragraph{Baseline Models.} As shown in Table~\ref{table:gender_race_bias}, Stable Diffusion exhibits moderate gender bias (0.67) and race bias (0.42). Among U-MLLMs, Janus-series (Janus, Janus-Pro) show higher overall bias in gender (0.87–0.90), as well as high race and G×R scores (0.43–0.48 and 0.23–0.24). Show-o and TokenFlow both witness slightly lower bias. Compared with other U-MLLMs, Emu3 stands out with low gender (0.83) and G×R (0.22) biases, along with good semantic scores. Finally, VILA-U exhibits strong text–image alignment (CLIP-S of 28.24) but has high gender (0.89) and intersectional (0.24) bias, highlighting the need for reducing bias in U-MLLMs.
\vspace{-1em}
\paragraph{Debiasing Gender.} Table~\ref{table:gender_race_bias} presents four methods to reduce gender bias in VILA-U. 
\emph{Prompt Engineering} witness moderate bias reduction (Gender = 0.56) while preserving high semantic alignment (CLIP-S = 28.51). 
\emph{Finetune (I~$\rightarrow$~T)} achieved the smallest reduction in gender bias (0.83) even though it witnessed a little increase in semantics (CLIP-S = 28.49). 
By contrast, \emph{Finetune (T~$\rightarrow$~I)} drastically lowers the gender bias to 0.27. However, it came with the side effect that image quality is lower (CLIP-IQA = 0.77). 
Finally, \emph{BPO} (Balanced Preference Optimization) further lowers gender bias (0.25) while preserving reasonable visual fidelity (Inception = 2.10), indicating that our approach is effective for mitigating stereotypical associations without compromising generation quality for U-MLLMs.
\vspace{-1em}
\paragraph{Debiasing Race.}
Table~\ref{table:gender_race_bias} shows the results for mitigating race bias in the VILA-U. 
\emph{Prompt Engineering} achieves a moderate bias reduction (Race = 0.49) while retaining a high level of semantic alignment (CLIP-S = 28.51). 
\emph{Finetune (I~$\rightarrow$~T)} yields Race = 0.44 and slightly lower semantic scores (CLIP-S = 28.14). 
By contrast, \emph{Finetune (T~$\rightarrow$~I)} significantly reduces the race bias to 0.23, albeit with some drop in semantic fidelity (Inception = 1.98). \emph{BPO} also lower the race bias a lot(0.26), and achieved good semantic fidelity (Inception 2.31) in the meantime.
\vspace{-1em}
\paragraph{Debiasing Intersectional Attributes (G$\times$R).}
As shown under \emph{Debias: G$\times$R} in Table~\ref{table:gender_race_bias}, we evaluate methods targeting \emph{both} gender and race simultaneously. 
\emph{Prompt Engineering} achieves a moderately low intersectional bias (G$\times$R = 0.18) while preserving acceptable semantic fidelity (CLIP-S = 28.09). 
\emph{Finetune (I~$\rightarrow$~T)} reduces intersectional bias to 0.23, accompanied by strong semantic alignment (CLIP-S = 28.34), whereas 
\emph{Finetune (T~$\rightarrow$~I)} further lowers G$\times$R to 0.17 but at the cost of slightly reduced image quality (Inception = 1.91). \emph{BPO} achieved the lowest bias score in gender-race intersection as 0.15. It also achieved high semantic fidelity with an Inception score of 2.06.
