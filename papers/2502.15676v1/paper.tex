% This must be in the first 5 lines to tell arXiv to use pdfLaTeX, which is strongly recommended.
\pdfoutput=1
% In particular, the hyperref package requires pdfLaTeX in order to break URLs across lines.

\documentclass[11pt]{article}

% Change "review" to "final" to generate the final (sometimes called camera-ready) version.
% Change to "preprint" to generate a non-anonymous version with page numbers.
% \usepackage[review]{acl}
\usepackage[preprint]{acl}

% Standard package includes
\usepackage{times}
\usepackage{latexsym}

% For proper rendering and hyphenation of words containing Latin characters (including in bib files)
\usepackage[T1]{fontenc}
% For Vietnamese characters
% \usepackage[T5]{fontenc}
% See https://www.latex-project.org/help/documentation/encguide.pdf for other character sets

% This assumes your files are encoded as UTF8
\usepackage[utf8]{inputenc}

% This is not strictly necessary, and may be commented out,
% but it will improve the layout of the manuscript,
% and will typically save some space.
\usepackage{microtype}

% This is also not strictly necessary, and may be commented out.
% However, it will improve the aesthetics of text in
% the typewriter font.
\usepackage{inconsolata}

%Including images in your LaTeX document requires adding
%additional package(s)
\usepackage{graphicx}


\usepackage{booktabs}
\usepackage{changepage}
\usepackage[breakable]{tcolorbox}
\usepackage{amsmath}

\usepackage{enumitem}
%\setlist[itemize]{topsep=3pt, partopsep=3pt} % Reduces space above and below

\usepackage{hyperref}
\usepackage{xspace}
\usepackage{comment}
\usepackage{multirow}
\usepackage{algorithm}
\usepackage{algpseudocode}
\newcommand{\ours}{\textsl{AutoToM}\xspace}
\algnewcommand{\LineComment}[1]{\State \(\triangleright\) #1}

\DeclareMathOperator*{\argmax}{arg\,max}
\DeclareMathOperator*{\argmin}{arg\,min}

\usepackage[breakable]{tcolorbox}
\usepackage{fix-cm}
\usepackage{fontawesome5}

\newcommand{\jcy}[1]{{\color{red}{[(JCY): #1]}}}





% If the title and author information does not fit in the area allocated, uncomment the following
%
%\setlength\titlebox{<dim>}
%
% and set <dim> to something 5cm or larger.

\title{\ours: Automated Bayesian Inverse Planning and Model Discovery \\ for Open-ended Theory of Mind}

\author{
Zhining Zhang\textsuperscript{$\diamondsuit$}\thanks{~Equal contribution. Zhining Zhang completed this work during an internship at JHU. $^\dagger$ Project lead.} \quad
Chuanyang Jin\textsuperscript{$\heartsuit$}\footnotemark[1]$^\dagger$ \quad
Mung Yao Jia\textsuperscript{$\heartsuit$}\footnotemark[1] \quad 
Tianmin Shu\textsuperscript{$\heartsuit$} \\
\textsuperscript{$\diamondsuit$} Peking University \quad
\textsuperscript{$\heartsuit$} Johns Hopkins University \\
\texttt{zzn\_nzz@stu.pku.edu.cn, \{cjin33, mjia8, tianmin.shu\}@jhu.edu} \\
% Project website: \texttt{https://chuanyangjin.com/AutoToM/}
}


\begin{document}
\maketitle
\begin{abstract}
Theory of Mind (ToM), the ability to understand people's mental variables based on their behavior, is key to developing socially intelligent agents. Current approaches to Theory of Mind reasoning either rely on prompting Large Language Models (LLMs), which are prone to systematic errors, or use rigid, handcrafted Bayesian Theory of Mind (BToM) models, which are more robust but cannot generalize across different domains. In this work, we introduce \ours, an automated Bayesian Theory of Mind method for achieving open-ended machine Theory of Mind. \ours can operate in any domain, infer any mental variable, and conduct robust Theory of Mind reasoning of any order. Given a Theory of Mind inference problem, \ours first proposes an initial BToM model. It then conducts automated Bayesian inverse planning based on the proposed model, leveraging an LLM as the backend. Based on the uncertainty of the inference, it iteratively refines the model, by introducing additional mental variables and/or incorporating more timesteps in the context. Empirical evaluations across multiple Theory of Mind benchmarks demonstrate that \ours consistently achieves state-of-the-art performance, offering a scalable, robust, and interpretable approach to machine Theory of Mind.

\vspace{0.15cm}
\quad \faIcon[regular]{star}\; \href{https://chuanyangjin.com/AutoToM}{chuanyangjin.com/AutoToM}

\quad \faGithub\ \; \href{https://github.com/SCAI-JHU/AutoToM}{github.com/SCAI-JHU/AutoToM}
\vspace{0.15cm}
\end{abstract}

%!TEX root = gcn.tex
\section{Introduction}
Graphs, representing structural data and topology, are widely used across various domains, such as social networks and merchandising transactions.
Graph convolutional networks (GCN)~\cite{iclr/KipfW17} have significantly enhanced model training on these interconnected nodes.
However, these graphs often contain sensitive information that should not be leaked to untrusted parties.
For example, companies may analyze sensitive demographic and behavioral data about users for applications ranging from targeted advertising to personalized medicine.
Given the data-centric nature and analytical power of GCN training, addressing these privacy concerns is imperative.

Secure multi-party computation (MPC)~\cite{crypto/ChaumDG87,crypto/ChenC06,eurocrypt/CiampiRSW22} is a critical tool for privacy-preserving machine learning, enabling mutually distrustful parties to collaboratively train models with privacy protection over inputs and (intermediate) computations.
While research advances (\eg,~\cite{ccs/RatheeRKCGRS20,uss/NgC21,sp21/TanKTW,uss/WatsonWP22,icml/Keller022,ccs/ABY318,folkerts2023redsec}) support secure training on convolutional neural networks (CNNs) efficiently, private GCN training with MPC over graphs remains challenging.

Graph convolutional layers in GCNs involve multiplications with a (normalized) adjacency matrix containing $\numedge$ non-zero values in a $\numnode \times \numnode$ matrix for a graph with $\numnode$ nodes and $\numedge$ edges.
The graphs are typically sparse but large.
One could use the standard Beaver-triple-based protocol to securely perform these sparse matrix multiplications by treating graph convolution as ordinary dense matrix multiplication.
However, this approach incurs $O(\numnode^2)$ communication and memory costs due to computations on irrelevant nodes.
%
Integrating existing cryptographic advances, the initial effort of SecGNN~\cite{tsc/WangZJ23,nips/RanXLWQW23} requires heavy communication or computational overhead.
Recently, CoGNN~\cite{ccs/ZouLSLXX24} optimizes the overhead in terms of  horizontal data partitioning, proposing a semi-honest secure framework.
Research for secure GCN over vertical data  remains nascent.

Current MPC studies, for GCN or not, have primarily targeted settings where participants own different data samples, \ie, horizontally partitioned data~\cite{ccs/ZouLSLXX24}.
MPC specialized for scenarios where parties hold different types of features~\cite{tkde/LiuKZPHYOZY24,icml/CastigliaZ0KBP23,nips/Wang0ZLWL23} is rare.
This paper studies $2$-party secure GCN training for these vertical partition cases, where one party holds private graph topology (\eg, edges) while the other owns private node features.
For instance, LinkedIn holds private social relationships between users, while banks own users' private bank statements.
Such real-world graph structures underpin the relevance of our focus.
To our knowledge, no prior work tackles secure GCN training in this context, which is crucial for cross-silo collaboration.


To realize secure GCN over vertically split data, we tailor MPC protocols for sparse graph convolution, which fundamentally involves sparse (adjacency) matrix multiplication.
Recent studies have begun exploring MPC protocols for sparse matrix multiplication (SMM).
ROOM~\cite{ccs/SchoppmannG0P19}, a seminal work on SMM, requires foreknowledge of sparsity types: whether the input matrices are row-sparse or column-sparse.
Unfortunately, GCN typically trains on graphs with arbitrary sparsity, where nodes have varying degrees and no specific sparsity constraints.
Moreover, the adjacency matrix in GCN often contains a self-loop operation represented by adding the identity matrix, which is neither row- nor column-sparse.
Araki~\etal~\cite{ccs/Araki0OPRT21} avoid this limitation in their scalable, secure graph analysis work, yet it does not cover vertical partition.

% and related primitives
To bridge this gap, we propose a secure sparse matrix multiplication protocol, \osmm, achieving \emph{accurate, efficient, and secure GCN training over vertical data} for the first time.

\subsection{New Techniques for Sparse Matrices}
The cost of evaluating a GCN layer is dominated by SMM in the form of $\adjmat\feamat$, where $\adjmat$ is a sparse adjacency matrix of a (directed) graph $\graph$ and $\feamat$ is a dense matrix of node features.
For unrelated nodes, which often constitute a substantial portion, the element-wise products $0\cdot x$ are always zero.
Our efficient MPC design 
avoids unnecessary secure computation over unrelated nodes by focusing on computing non-zero results while concealing the sparse topology.
We achieve this~by:
1) decomposing the sparse matrix $\adjmat$ into a product of matrices (\S\ref{sec::sgc}), including permutation and binary diagonal matrices, that can \emph{faithfully} represent the original graph topology;
2) devising specialized protocols (\S\ref{sec::smm_protocol}) for efficiently multiplying the structured matrices while hiding sparsity topology.


 
\subsubsection{Sparse Matrix Decomposition}
We decompose adjacency matrix $\adjmat$ of $\graph$ into two bipartite graphs: one represented by sparse matrix $\adjout$, linking the out-degree nodes to edges, the other 
by sparse matrix $\adjin$,
linking edges to in-degree nodes.

%\ie, we decompose $\adjmat$ into $\adjout \adjin$, where $\adjout$ and $\adjin$ are sparse matrices representing these connections.
%linking out-degree nodes to edges and edges to in-degree nodes of $\graph$, respectively.

We then permute the columns of $\adjout$ and the rows of $\adjin$ so that the permuted matrices $\adjout'$ and $\adjin'$ have non-zero positions with \emph{monotonically non-decreasing} row and column indices.
A permutation $\sigma$ is used to preserve the edge topology, leading to an initial decomposition of $\adjmat = \adjout'\sigma \adjin'$.
This is further refined into a sequence of \emph{linear transformations}, 
which can be efficiently computed by our MPC protocols for 
\emph{oblivious permutation}
%($\Pi_{\ssp}$) 
and \emph{oblivious selection-multiplication}.
% ($\Pi_\SM$)
\iffalse
Our approach leverages bipartite graph representation and the monotonicity of non-zero positions to decompose a general sparse matrix into linear transformations, enhancing the efficiency of our MPC protocols.
\fi
Our decomposition approach is not limited to GCNs but also general~SMM 
by 
%simply 
treating them 
as adjacency matrices.
%of a graph.
%Since any sparse matrix can be viewed 

%allowing the same technique to be applied.

 
\subsubsection{New Protocols for Linear Transformations}
\emph{Oblivious permutation} (OP) is a two-party protocol taking a private permutation $\sigma$ and a private vector $\xvec$ from the two parties, respectively, and generating a secret share $\l\sigma \xvec\r$ between them.
Our OP protocol employs correlated randomnesses generated in an input-independent offline phase to mask $\sigma$ and $\xvec$ for secure computations on intermediate results, requiring only $1$ round in the online phase (\cf, $\ge 2$ in previous works~\cite{ccs/AsharovHIKNPTT22, ccs/Araki0OPRT21}).

Another crucial two-party protocol in our work is \emph{oblivious selection-multiplication} (OSM).
It takes a private bit~$s$ from a party and secret share $\l x\r$ of an arithmetic number~$x$ owned by the two parties as input and generates secret share $\l sx\r$.
%between them.
%Like our OP protocol, o
Our $1$-round OSM protocol also uses pre-computed randomnesses to mask $s$ and $x$.
%for secure computations.
Compared to the Beaver-triple-based~\cite{crypto/Beaver91a} and oblivious-transfer (OT)-based approaches~\cite{pkc/Tzeng02}, our protocol saves ${\sim}50\%$ of online communication while having the same offline communication and round complexities.

By decomposing the sparse matrix into linear transformations and applying our specialized protocols, our \osmm protocol
%($\prosmm$) 
reduces the complexity of evaluating $\numnode \times \numnode$ sparse matrices with $\numedge$ non-zero values from $O(\numnode^2)$ to $O(\numedge)$.

%(\S\ref{sec::secgcn})
\subsection{\cgnn: Secure GCN made Efficient}
Supported by our new sparsity techniques, we build \cgnn, 
a two-party computation (2PC) framework for GCN inference and training over vertical
%ly split
data.
Our contributions include:

1) We are the first to explore sparsity over vertically split, secret-shared data in MPC, enabling decompositions of sparse matrices with arbitrary sparsity and isolating computations that can be performed in plaintext without sacrificing privacy.

2) We propose two efficient $2$PC primitives for OP and OSM, both optimally single-round.
Combined with our sparse matrix decomposition approach, our \osmm protocol ($\prosmm$) achieves constant-round communication costs of $O(\numedge)$, reducing memory requirements and avoiding out-of-memory errors for large matrices.
In practice, it saves $99\%+$ communication
%(Table~\ref{table:comm_smm}) 
and reduces ${\sim}72\%$ memory usage over large $(5000\times5000)$ matrices compared with using Beaver triples.
%(Table~\ref{table:mem_smm_sparse}) ${\sim}16\%$-

3) We build an end-to-end secure GCN framework for inference and training over vertically split data, maintaining accuracy on par with plaintext computations.
We will open-source our evaluation code for research and deployment.

To evaluate the performance of $\cgnn$, we conducted extensive experiments over three standard graph datasets (Cora~\cite{aim/SenNBGGE08}, Citeseer~\cite{dl/GilesBL98}, and Pubmed~\cite{ijcnlp/DernoncourtL17}),
reporting communication, memory usage, accuracy, and running time under varying network conditions, along with an ablation study with or without \osmm.
Below, we highlight our key achievements.

\textit{Communication (\S\ref{sec::comm_compare_gcn}).}
$\cgnn$ saves communication by $50$-$80\%$.
(\cf,~CoGNN~\cite{ccs/KotiKPG24}, OblivGNN~\cite{uss/XuL0AYY24}).

\textit{Memory usage (\S\ref{sec::smmmemory}).}
\cgnn alleviates out-of-memory problems of using %the standard 
Beaver-triples~\cite{crypto/Beaver91a} for large datasets.

\textit{Accuracy (\S\ref{sec::acc_compare_gcn}).}
$\cgnn$ achieves inference and training accuracy comparable to plaintext counterparts.
%training accuracy $\{76\%$, $65.1\%$, $75.2\%\}$ comparable to $\{75.7\%$, $65.4\%$, $74.5\%\}$ in plaintext.

{\textit{Computational efficiency (\S\ref{sec::time_net}).}} 
%If the network is worse in bandwidth and better in latency, $\cgnn$ shows more benefits.
$\cgnn$ is faster by $6$-$45\%$ in inference and $28$-$95\%$ in training across various networks and excels in narrow-bandwidth and low-latency~ones.

{\textit{Impact of \osmm (\S\ref{sec:ablation}).}}
Our \osmm protocol shows a $10$-$42\times$ speed-up for $5000\times 5000$ matrices and saves $10$-2$1\%$ memory for ``small'' datasets and up to $90\%$+ for larger ones.

\section{Related Works}

\textbf{Enhancing LLMs' Theory of Mind.} There has been systematic evaluation that revealed LLMs' limitations in achieving robust Theory of Mind inference \citep{ullman2023large, shapira2023clever}. To enhance LLMs' Theory of Mind capacity, recent works have proposed various prompting techniques. For instance, SimToM \citep{wilf2023think} encourages LLMs to adopt perspective-taking, PercepToM \citep{jung2024perceptions} improves perception-to-belief inference by extracting relevant contextual details, and \citet{huang2024notion} utilize an LLM as a world model to track environmental changes and refine prompts. Explicit symbolic modules also seem to improve LLM's accuracy through dynamic updates based on inputs. Specifically, TimeToM \citep{hou2024timetom} constructs a temporal reasoning framework to support inference, while SymbolicToM \citep{sclar2023minding} uses graphical representations to track characters' beliefs. Additionally, \citet{wagner2024mind} investigates ToM's necessity and the level of recursion required for specific tasks. However, these approaches continue to exhibit systematic errors in long contexts, complex behaviors, and recursive reasoning due to inherent limitations in inference and modeling \citep{jin2024mmtom,shi2024muma}. Most of them rely on domain-specific designs, lacking open-endedness.


\textbf{Model-based Theory of Mind inference.} Model-based Theory of Mind inference, in particular, Bayesian inverse planning (BIP) \citep{baker2009action,ullman2009help,baker2017rational,zhi2020online}, explicitly constructs representations of agents' mental states and how mental states guide agents' behavior via Bayesian Theory of Mind (BToM) models. These methods can reverse engineer human ToM inference in simple domains \citep[e.g.,][]{baker2017rational,netanyahu2021phase,shu2021agent}. Recent works have proposed to combine BIP with LLMs to achieve robust ToM inference in more realistic settings \citep{ying2023neuro, jin2024mmtom, shi2024muma}. However, these methods require manual specification of the BToM models as well as rigid, domain-specific implementations of Bayesian inference, limiting their adaptability to open-ended scenarios. To overcome this limitation, we propose \ours, a method capable of automatically modeling mental variables across diverse conditions and conducting automated BIP without domain-specific knowledge or implementations.


\begin{figure*}[ht]
  \centering
  \includegraphics[width=\linewidth]{figures/benchmarks_and_models.pdf}
    \vspace{-15pt}
  \caption{Examples questions (top panels) and the necessary Bayesian Theory of Mind (BToM) model for Bayesian inverse planning (bottom panels) in diverse Theory of Mind benchmarks. \ours aims to answer any Theory of Mind question in a variety of benchmarks, encompassing different mental variables, observable contexts, numbers of agents, the presence or absence of utterances, wording styles, and modalities. It proposes and iteratively adjusts an appropriate BToM and conducts automated Bayesian inverse planning based on the model.
  There can be more types of questions/models in each benchmark beyond the examples shown in this figure.}
  \label{fig:benchmarks_and_models}
  %\vspace{-0.75em}
  \vspace{-10pt}
\end{figure*}



\textbf{Automated Modeling with LLMs.} There has been an increasing interest in integrating LLMs with inductive reasoning and probabilistic inference for automated modeling. \citet{piriyakulkij2024doing} combine LLMs with Sequential Monte Carlo to perform probabilistic inference about underlying rules. Iterative hypothesis refinement techniques \citep{qiu2023phenomenal} further enhance LLM-based inductive reasoning by iteratively proposing, selecting, and refining textual hypotheses of rules. Beyond rule-based hypotheses, \citet{wang2023hypothesis} prompt LLMs to generate natural language hypotheses that are then implemented as verifiable programs, while \citet{li2024automated} propose a method in which LLMs construct, critique, and refine statistical models represented as probabilistic programs for data modeling. \citet{cross2024hypothetical} leverage LLMs to propose and evaluate agent strategies for multi-agent planning but do not specifically infer individual mental variables. Our method also aims to achieve automated modeling with LLMs. Unlike prior works, we propose a novel automated model discovery approach for Bayesian inverse planning, where the objective is to confidently infer any mental variable given any context via constructing a suitable Bayesian Theory of Mind model.
\vspace{-5pt}
\section{Method}
\label{sec:method}
\section{Overview}

\revision{In this section, we first explain the foundational concept of Hausdorff distance-based penetration depth algorithms, which are essential for understanding our method (Sec.~\ref{sec:preliminary}).
We then provide a brief overview of our proposed RT-based penetration depth algorithm (Sec.~\ref{subsec:algo_overview}).}



\section{Preliminaries }
\label{sec:Preliminaries}

% Before we introduce our method, we first overview the important basics of 3D dynamic human modeling with Gaussian splatting. Then, we discuss the diffusion-based 3d generation techniques, and how they can be applied to human modeling.
% \ZY{I stopp here. TBC.}
% \subsection{Dynamic human modeling with Gaussian splatting}
\subsection{3D Gaussian Splatting}
3D Gaussian splatting~\cite{kerbl3Dgaussians} is an explicit scene representation that allows high-quality real-time rendering. The given scene is represented by a set of static 3D Gaussians, which are parameterized as follows: Gaussian center $x\in {\mathbb{R}^3}$, color $c\in {\mathbb{R}^3}$, opacity $\alpha\in {\mathbb{R}}$, spatial rotation in the form of quaternion $q\in {\mathbb{R}^4}$, and scaling factor $s\in {\mathbb{R}^3}$. Given these properties, the rendering process is represented as:
\begin{equation}
  I = Splatting(x, c, s, \alpha, q, r),
  \label{eq:splattingGA}
\end{equation}
where $I$ is the rendered image, $r$ is a set of query rays crossing the scene, and $Splatting(\cdot)$ is a differentiable rendering process. We refer readers to Kerbl et al.'s paper~\cite{kerbl3Dgaussians} for the details of Gaussian splatting. 



% \ZY{I would suggest move this part to the method part.}
% GaissianAvatar is a dynamic human generation model based on Gaussian splitting. Given a sequence of RGB images, this method utilizes fitted SMPLs and sampled points on its surface to obtain a pose-dependent feature map by a pose encoder. The pose-dependent features and a geometry feature are fed in a Gaussian decoder, which is employed to establish a functional mapping from the underlying geometry of the human form to diverse attributes of 3D Gaussians on the canonical surfaces. The parameter prediction process is articulated as follows:
% \begin{equation}
%   (\Delta x,c,s)=G_{\theta}(S+P),
%   \label{eq:gaussiandecoder}
% \end{equation}
%  where $G_{\theta}$ represents the Gaussian decoder, and $(S+P)$ is the multiplication of geometry feature S and pose feature P. Instead of optimizing all attributes of Gaussian, this decoder predicts 3D positional offset $\Delta{x} \in {\mathbb{R}^3}$, color $c\in\mathbb{R}^3$, and 3D scaling factor $ s\in\mathbb{R}^3$. To enhance geometry reconstruction accuracy, the opacity $\alpha$ and 3D rotation $q$ are set to fixed values of $1$ and $(1,0,0,0)$ respectively.
 
%  To render the canonical avatar in observation space, we seamlessly combine the Linear Blend Skinning function with the Gaussian Splatting~\cite{kerbl3Dgaussians} rendering process: 
% \begin{equation}
%   I_{\theta}=Splatting(x_o,Q,d),
%   \label{eq:splatting}
% \end{equation}
% \begin{equation}
%   x_o = T_{lbs}(x_c,p,w),
%   \label{eq:LBS}
% \end{equation}
% where $I_{\theta}$ represents the final rendered image, and the canonical Gaussian position $x_c$ is the sum of the initial position $x$ and the predicted offset $\Delta x$. The LBS function $T_{lbs}$ applies the SMPL skeleton pose $p$ and blending weights $w$ to deform $x_c$ into observation space as $x_o$. $Q$ denotes the remaining attributes of the Gaussians. With the rendering process, they can now reposition these canonical 3D Gaussians into the observation space.



\subsection{Score Distillation Sampling}
Score Distillation Sampling (SDS)~\cite{poole2022dreamfusion} builds a bridge between diffusion models and 3D representations. In SDS, the noised input is denoised in one time-step, and the difference between added noise and predicted noise is considered SDS loss, expressed as:

% \begin{equation}
%   \mathcal{L}_{SDS}(I_{\Phi}) \triangleq E_{t,\epsilon}[w(t)(\epsilon_{\phi}(z_t,y,t)-\epsilon)\frac{\partial I_{\Phi}}{\partial\Phi}],
%   \label{eq:SDSObserv}
% \end{equation}
\begin{equation}
    \mathcal{L}_{\text{SDS}}(I_{\Phi}) \triangleq \mathbb{E}_{t,\epsilon} \left[ w(t) \left( \epsilon_{\phi}(z_t, y, t) - \epsilon \right) \frac{\partial I_{\Phi}}{\partial \Phi} \right],
  \label{eq:SDSObservGA}
\end{equation}
where the input $I_{\Phi}$ represents a rendered image from a 3D representation, such as 3D Gaussians, with optimizable parameters $\Phi$. $\epsilon_{\phi}$ corresponds to the predicted noise of diffusion networks, which is produced by incorporating the noise image $z_t$ as input and conditioning it with a text or image $y$ at timestep $t$. The noise image $z_t$ is derived by introducing noise $\epsilon$ into $I_{\Phi}$ at timestep $t$. The loss is weighted by the diffusion scheduler $w(t)$. 
% \vspace{-3mm}

\subsection{Overview of the RTPD Algorithm}\label{subsec:algo_overview}
Fig.~\ref{fig:Overview} presents an overview of our RTPD algorithm.
It is grounded in the Hausdorff distance-based penetration depth calculation method (Sec.~\ref{sec:preliminary}).
%, similar to that of Tang et al.~\shortcite{SIG09HIST}.
The process consists of two primary phases: penetration surface extraction and Hausdorff distance calculation.
We leverage the RTX platform's capabilities to accelerate both of these steps.

\begin{figure*}[t]
    \centering
    \includegraphics[width=0.8\textwidth]{Image/overview.pdf}
    \caption{The overview of RT-based penetration depth calculation algorithm overview}
    \label{fig:Overview}
\end{figure*}

The penetration surface extraction phase focuses on identifying the overlapped region between two objects.
\revision{The penetration surface is defined as a set of polygons from one object, where at least one of its vertices lies within the other object. 
Note that in our work, we focus on triangles rather than general polygons, as they are processed most efficiently on the RTX platform.}
To facilitate this extraction, we introduce a ray-tracing-based \revision{Point-in-Polyhedron} test (RT-PIP), significantly accelerated through the use of RT cores (Sec.~\ref{sec:RT-PIP}).
This test capitalizes on the ray-surface intersection capabilities of the RTX platform.
%
Initially, a Geometry Acceleration Structure (GAS) is generated for each object, as required by the RTX platform.
The RT-PIP module takes the GAS of one object (e.g., $GAS_{A}$) and the point set of the other object (e.g., $P_{B}$).
It outputs a set of points (e.g., $P_{\partial B}$) representing the penetration region, indicating their location inside the opposing object.
Subsequently, a penetration surface (e.g., $\partial B$) is constructed using this point set (e.g., $P_{\partial B}$) (Sec.~\ref{subsec:surfaceGen}).
%
The generated penetration surfaces (e.g., $\partial A$ and $\partial B$) are then forwarded to the next step. 

The Hausdorff distance calculation phase utilizes the ray-surface intersection test of the RTX platform (Sec.~\ref{sec:RT-Hausdorff}) to compute the Hausdorff distance between two objects.
We introduce a novel Ray-Tracing-based Hausdorff DISTance algorithm, RT-HDIST.
It begins by generating GAS for the two penetration surfaces, $P_{\partial A}$ and $P_{\partial B}$, derived from the preceding step.
RT-HDIST processes the GAS of a penetration surface (e.g., $GAS_{\partial A}$) alongside the point set of the other penetration surface (e.g., $P_{\partial B}$) to compute the penetration depth between them.
The algorithm operates bidirectionally, considering both directions ($\partial A \to \partial B$ and $\partial B \to \partial A$).
The final penetration depth between the two objects, A and B, is determined by selecting the larger value from these two directional computations.

%In the Hausdorff distance calculation step, we compute the Hausdorff distance between given two objects using a ray-surface-intersection test. (Sec.~\ref{sec:RT-Hausdorff}) Initially, we construct the GAS for both $\partial A$ and $\partial B$ to utilize the RT-core effectively. The RT-based Hausdorff distance algorithms then determine the Hausdorff distance by processing the GAS of one object (e.g. $GAS_{\partial A}$) and set of the vertices of the other (e.g. $P_{\partial B}$). Following the Hausdorff distance definition (Eq.~\ref{equation:hausdorff_definition}), we compute the Hausdorff distance to both directions ($\partial A \to \partial B$) and ($\partial B \to \partial A$). As a result, the bigger one is the final Hausdorff distance, and also it is the penetration depth between input object $A$ and $B$.


%the proposed RT-based penetration depth calculation pipeline.
%Our proposed methods adopt Tang's Hausdorff-based penetration depth methods~\cite{SIG09HIST}. The pipeline is divided into the penetration surface extraction step and the Hausdorff distance calculation between the penetration surface steps. However, since Tang's approach is not suitable for the RT platform in detail, we modified and applied it with appropriate methods.

%The penetration surface extraction step is extracting overlapped surfaces on other objects. To utilize the RT core, we use the ray-intersection-based PIP(Point-In-Polygon) algorithms instead of collision detection between two objects which Tang et al.~\cite{SIG09HIST} used. (Sec.~\ref{sec:RT-PIP})
%RT core-based PIP test uses a ray-surface intersection test. For purpose this, we generate the GAS(Geometry Acceleration Structure) for each object. RT core-based PIP test takes the GAS of one object (e.g. $GAS_{A}$) and a set of vertex of another one (e.g. $P_{B}$). Then this computes the penetrated vertex set of another one (e.g. $P_{\partial B}$). To calculate the Hausdorff distance, these vertex sets change to objects constructed by penetrated surface (e.g. $\partial B$). Finally, the two generated overlapped surface objects $\partial A$ and $\partial B$ are used in the Hausdorff distance calculation step.

Our goal is to increase the robustness of T2I models, particularly with rare or unseen concepts, which they struggle to generate. To do so, we investigate a retrieval-augmented generation approach, through which we dynamically select images that can provide the model with missing visual cues. Importantly, we focus on models that were not trained for RAG, and show that existing image conditioning tools can be leveraged to support RAG post-hoc.
As depicted in \cref{fig:overview}, given a text prompt and a T2I generative model, we start by generating an image with the given prompt. Then, we query a VLM with the image, and ask it to decide if the image matches the prompt. If it does not, we aim to retrieve images representing the concepts that are missing from the image, and provide them as additional context to the model to guide it toward better alignment with the prompt.
In the following sections, we describe our method by answering key questions:
(1) How do we know which images to retrieve? 
(2) How can we retrieve the required images? 
and (3) How can we use the retrieved images for unknown concept generation?
By answering these questions, we achieve our goal of generating new concepts that the model struggles to generate on its own.

\vspace{-3pt}
\subsection{Which images to retrieve?}
The amount of images we can pass to a model is limited, hence we need to decide which images to pass as references to guide the generation of a base model. As T2I models are already capable of generating many concepts successfully, an efficient strategy would be passing only concepts they struggle to generate as references, and not all the concepts in a prompt.
To find the challenging concepts,
we utilize a VLM and apply a step-by-step method, as depicted in the bottom part of \cref{fig:overview}. First, we generate an initial image with a T2I model. Then, we provide the VLM with the initial prompt and image, and ask it if they match. If not, we ask the VLM to identify missing concepts and
focus on content and style, since these are easy to convey through visual cues.
As demonstrated in \cref{tab:ablations}, empirical experiments show that image retrieval from detailed image captions yields better results than retrieval from brief, generic concept descriptions.
Therefore, after identifying the missing concepts, we ask the VLM to suggest detailed image captions for images that describe each of the concepts. 

\vspace{-4pt}
\subsubsection{Error Handling}
\label{subsec:err_hand}

The VLM may sometimes fail to identify the missing concepts in an image, and will respond that it is ``unable to respond''. In these rare cases, we allow up to 3 query repetitions, while increasing the query temperature in each repetition. Increasing the temperature allows for more diverse responses by encouraging the model to sample less probable words.
In most cases, using our suggested step-by-step method yields better results than retrieving images directly from the given prompt (see 
\cref{subsec:ablations}).
However, if the VLM still fails to identify the missing concepts after multiple attempts, we fall back to retrieving images directly from the prompt, as it usually means the VLM does not know what is the meaning of the prompt.

The used prompts can be found in \cref{app:prompts}.
Next, we turn to retrieve images based on the acquired image captions.

\vspace{-3pt}
\subsection{How to retrieve the required images?}

Given $n$ image captions, our goal is to retrieve the images that are most similar to these captions from a dataset. 
To retrieve images matching a given image caption, we compare the caption to all the images in the dataset using a text-image similarity metric and retrieve the top $k$ most similar images.
Text-to-image retrieval is an active research field~\cite{radford2021learning, zhai2023sigmoid, ray2024cola, vendrowinquire}, where no single method is perfect.
Retrieval is especially hard when the dataset does not contain an exact match to the query \cite{biswas2024efficient} or when the task is fine-grained retrieval, that depends on subtle details~\cite{wei2022fine}.
Hence, a common retrieval workflow is to first retrieve image candidates using pre-computed embeddings, and then re-rank the retrieved candidates using a different, often more expensive but accurate, method \cite{vendrowinquire}.
Following this workflow, we experimented with cosine similarity over different embeddings, and with multiple re-ranking methods of reference candidates.
Although re-ranking sometimes yields better results compared to simply using cosine similarity between CLIP~\cite{radford2021learning} embeddings, the difference was not significant in most of our experiments. Therefore, for simplicity, we use cosine similarity between CLIP embeddings as our similarity metric (see \cref{tab:sim_metrics}, \cref{subsec:ablations} for more details about our experiments with different similarity metrics).

\vspace{-3pt}
\subsection{How to use the retrieved images?}
Putting it all together, after retrieving relevant images, all that is left to do is to use them as context so they are beneficial for the model.
We experimented with two types of models; models that are trained to receive images as input in addition to text and have ICL capabilities (e.g., OmniGen~\cite{xiao2024omnigen}), and T2I models augmented with an image encoder in post-training (e.g., SDXL~\cite{podellsdxl} with IP-adapter~\cite{ye2023ip}).
As the first model type has ICL capabilities, we can supply the retrieved images as examples that it can learn from, by adjusting the original prompt.
Although the second model type lacks true ICL capabilities, it offers image-based control functionalities, which we can leverage for applying RAG over it with our method.
Hence, for both model types, we augment the input prompt to contain a reference of the retrieved images as examples.
Formally, given a prompt $p$, $n$ concepts, and $k$ compatible images for each concept, we use the following template to create a new prompt:
``According to these examples of 
$\mathord{<}c_1\mathord{>:<}img_{1,1}\mathord{>}, ... , \mathord{<}img_{1,k}\mathord{>}, ... , \mathord{<}c_n\mathord{>:<}img_{n,1}\mathord{>}, ... , $
$\mathord{<}img_{n,k}\mathord{>}$,
generate $\mathord{<}p\mathord{>}$'', 
where $c_i$ for $i\in{[1,n]}$ is a compatible image caption of the image $\mathord{<}img_{i,j}\mathord{>},  j\in{[1,k]}$. 

This prompt allows models to learn missing concepts from the images, guiding them to generate the required result. 

\textbf{Personalized Generation}: 
For models that support multiple input images, we can apply our method for personalized generation as well, to generate rare concept combinations with personal concepts. In this case, we use one image for personal content, and 1+ other reference images for missing concepts. For example, given an image of a specific cat, we can generate diverse images of it, ranging from a mug featuring the cat to a lego of it or atypical situations like the cat writing code or teaching a classroom of dogs (\cref{fig:personalization}).
\vspace{-2pt}
\begin{figure}[htp]
  \centering
   \includegraphics[width=\linewidth]{Assets/personalization.pdf}
   \caption{\textbf{Personalized generation example.}
   \emph{ImageRAG} can work in parallel with personalization methods and enhance their capabilities. For example, although OmniGen can generate images of a subject based on an image, it struggles to generate some concepts. Using references retrieved by our method, it can generate the required result.
}
   \label{fig:personalization}\vspace{-10pt}
\end{figure}
\section{Experiments}

\subsection{Experimental Setup}

\textbf{Datasets.} We use three categories from the Amazon Reviews dataset~\cite{mcauley2015image} for our experiments: ``Sports and Outdoors'' (\textbf{Sports}), ``Beauty'' (\textbf{Beauty}), and ``CDs and Vinyl'' (\textbf{CDs}). Each user’s historical reviews are considered ``actions'' and are sorted chronologically as action sequences, with earlier reviews appearing first. To evaluate the models, we adopt the widely used leave-last-out protocol~\cite{kang2018sasrec,zhao2022revisiting,rajput2023tiger}, where the last item and second-to-last item in each action sequence are used for testing and validation, respectively. More details about the datasets can be found in~\Cref{app:datasets}.

\textbf{Compared methods.} We compare the performance of ActionPiece with the following methods: (1)~ID-based sequential recommendation methods, including BERT4Rec~\cite{sun2019bert4rec}, and SASRec~\cite{kang2018sasrec}; (2)~feature-enhanced sequential recommendation methods, such as FDSA~\cite{zhang2019fdsa}, S$^3$-Rec~\cite{zhou2020s3}, and VQ-Rec~\cite{hou2023vqrec}; and (3)~generative recommendation methods, including P5-CID~\cite{hua2023p5cid}, TIGER~\cite{rajput2023tiger}, LMIndexer~\cite{jin2024lmindexer}, HSTU~\cite{zhai2024hstu}, and SPM-SID~\cite{singh2024spmsid}, each representing a different action tokenization method (\Cref{tab:act_tokenization}). A detailed description of these baselines is provided in~\Cref{appendix:baselines}.

\textbf{Evaluation settings.} Following~\citet{rajput2023tiger}, we use Recall@$K$ and NDCG@$K$ as metrics to evaluate the methods, where $K \in \{5, 10\}$. Model checkpoints with the best performance on the validation set are used for evaluation on the test set. We run the experiments with five random seeds and report the average metrics.

\textbf{Implementation details.} Please refer to~\Cref{appendix:implementation} for detailed implementation and hyperparameter settings.


\subsection{Overall Performance}

We compare ActionPiece with sequential recommendation and generative recommendation baselines, which use various action tokenization methods, across three public datasets. The results are shown in~\Cref{tab:performance}. 

For the compared methods, we observe that those using item features generally outperform item ID-only methods. This indicates that incorporating features enhances recommendation performance. Among the methods leveraging item features (``Feature + ID'' and ``Generative''), generative recommendation models achieve better performance. These results further confirm that injecting semantics into item indexing and optimizing at a sub-item level enables generative models to better use semantic information and improve recommendation performance. Among all the baselines, SPM-SID achieves the best results. By incorporating the SentencePiece model~\cite{kudo2018sentencepiece}, SPM-SID replaces popular semantic ID patterns within each item with new tokens, benefiting from a larger vocabulary.

\begin{table}[t!]
    \small
    \centering
	\caption{Ablation analysis of ActionPiece. The recommendation performance is measured using NDCG@$10$. The best performance is denoted in \textbf{bold} fonts.}
	\label{tab:ablation}
	\vskip 0.1in
% 	\setlength{\tabcolsep}{1mm}{
% \resizebox{2.1\columnwidth}{!}{
    \begin{tabular}{lccc}
	\toprule
	\multicolumn{1}{c}{\textbf{Variants}} & \textbf{Sports} & \textbf{Beauty} & \textbf{CDs} \\
	\midrule
	\midrule
    \multicolumn{4}{@{}c}{\textit{TIGER with larger vocabularies}} \\
    \midrule
    (1.1) TIGER\ -\ 1k ($4 \times 2^8$) & 0.0225 & 0.0384 & 0.0411 \\
    (1.2) TIGER-49k ($6 \times 2^{13}$) & 0.0162 & 0.0317 & 0.0338 \\
    (1.3) TIGER-66k ($4 \times 2^{14}$) & 0.0194 & N/A$^\dag$ & 0.0319 \\
    \midrule
    \multicolumn{4}{@{}c}{\textit{Vocabulary construction}} \\
    \midrule
    (2.1) \emph{w/o} tokenization & 0.0215 & 0.0389 & 0.0346 \\
    (2.2) \emph{w/o} context-aware & 0.0258 & 0.0416 & 0.0429 \\
    (2.3) \emph{w/o} weighted counting & 0.0257 & 0.0412 & 0.0435 \\
    \midrule
    \multicolumn{4}{@{}c}{\textit{Set permutation regularization}} \\
    \midrule
    (3.1) only for inference & 0.0192 & 0.0316 & 0.0329 \\
    (3.2) only for training & 0.0244 & 0.0387 & 0.0422 \\
    \midrule
    ActionPiece (40k) & \textbf{0.0264} & \textbf{0.0424} & \textbf{0.0451} \\
    \bottomrule
	\end{tabular}
	\vspace{0.05cm}
	\begin{flushleft}
        $^\dag$ not applicable as $2^{14}$ is larger than \#items in Beauty.
    \end{flushleft}
% 	}}
    \vskip -0.2in
\end{table}

\begin{figure*}[t!]
    \begin{center}
    \includegraphics[width=\linewidth]{fig/vocab_size.pdf}
    \vskip -0.1in
    \caption{Analysis of recommendation performance (NDCG@10, $\uparrow$) and average tokenized sequence length (NSL, $\downarrow$) \wrt vocabulary size across three datasets.
    % NSL refers to the normalized sequence length, calculated relative to the number of initial tokens.
    ``N/A’’ indicates that ActionPiece is not applied, \ie action sequences are represented solely by initial tokens.}
    \label{fig:vocab_size}
    \end{center}
    \vskip -0.2in
\end{figure*}

Our proposed ActionPiece consistently outperforms all baselines across three datasets, achieving a significant improvement in NDCG@$10$. It surpasses the best-performing baseline method by $6.00\%$ to $12.82\%$. Unlike existing methods, ActionPiece is the first context-aware action sequence tokenizer, \ie the same action can be tokenized into different tokens depending on its surrounding context. This allows ActionPiece to capture important sequence-level feature patterns that enhance recommendation performance.

% \begin{figure}[t]
% % \vskip 0.2in
% \begin{center}
% \centerline{\includegraphics[width=0.85\columnwidth]{fig/ndcg_vs_vocab_size.pdf}}
% \end{center}
% % \vskip -0.3in
% \vspace{-0.3in}
% \caption{Comparison of performance and vocabulary size (\#token for TIGER, SPM-SID, and ActionPiece; \#item for SASRec; and \#item+\#attribute for S$^3$-Rec) on ``Sports'' dataset.
% % By adjusting the vocabulary size, ActionPiece is shown to balance memory efficiency and recommendation performance.
% }
% \label{fig:intro}
% % \vskip -0.2in
% \vspace{-0.1in}
% \end{figure}


\subsection{Ablation Study}\label{sec:ablation}

We conduct ablation analyses in~\Cref{tab:ablation} to study how each proposed technique contributes to ActionPiece.\\
\hspace*{3mm} (1)~We increase the vocabulary size of TIGER, to determine whether the performance gain of ActionPiece is solely due to scaling up the number of tokens in the vocabulary. By increasing the number of semantic ID digits per item~($4 \rightarrow 6$) and the number of candidate semantic IDs per digit~($2^8 \rightarrow 2^{13}\ \text{or}\ 2^{14}$), we create two variants with vocabularies larger than ActionPiece. However, these TIGER variants perform worse than ActionPiece, and even the original TIGER with only $1024$ tokens. The experimental results suggest that scaling up the vocabulary size for generative recommendation models is challenging, consistent with the observations from~\citet{zhang2024moc}.\\
\hspace*{3mm} (2)~To evaluate the effectiveness of the proposed vocabulary construction techniques, we introduce the following variants: \emph{(2.1)~w/o tokenization}, which skips vocabulary construction, using item features directly as tokens; \emph{(2.2)~w/o context-aware}, which only considers co-occurrences and merges tokens within each action during vocabulary construction and segmentation; and \emph{(2.3)~w/o weighted counting}, which treats all token pairs equally rather than using the weights defined in~\Cref{eq:p_one_set,eq:p_two_sets}. The results indicate that removing any of these techniques reduces performance, demonstrating the importance of these methods for building a context-aware tokenizer.\\
\hspace*{3mm} (3)~To evaluate the effectiveness of SPR, we revert to naive segmentation, as described in~\Cref{subsubsec:segmentation}, during model training and inference, respectively. The results show that replacing SPR with naive segmentation in either training or inference degrades performance.

\begin{figure}[t!]
    \begin{center}
    \includegraphics[width=0.95\columnwidth]{fig/token_util.pdf}
    \vskip -0.1in
    \caption{Analysis of token utilization rate (\%) during model training \wrt segmentation strategy.
    % ``SPR'' denotes set permutation regularization.
    }
    \label{fig:token_util}
    \end{center}
    % \vskip -0.3in
    \vskip -0.3in
\end{figure}

\subsection{Further Analysis}

% In this section, we analyze the impact of key hyperparameters in vocabulary construction and segmentation.

\subsubsection{Performance and Efficiency \wrt Vocabulary Size}

Vocabulary size is a key hyperparameter for language tokenizers~\cite{meta2024llama3,dagan2024getting}. In this study, we investigate how adjusting vocabulary size affects the generative recommendation models. We use the normalized sequence length (NSL)~\cite{dagan2024getting} to measure the length of tokenized sequences, where a smaller NSL indicates fewer tokens per tokenized sequence. We experiment with vocabulary sizes in \{N/A, 5k, 10k, 20k, 30k, 40k\}, where ``N/A'' represents the direct use of item features as tokens. As shown in~\Cref{fig:vocab_size}, increasing the vocabulary size improves recommendation performance and reduces the tokenized sequence length. Conversely, reducing the vocabulary size lowers the number of model parameters, improving memory efficiency. This analysis demonstrates that adjusting vocabulary size enables a trade-off between model performance, sequence length, and memory efficiency.

\subsubsection{Token Utilization Rate \wrt Segmentation Strategy}\label{sec:token_utilization}

As described in~\Cref{subsubsec:training}, applying SPR augments the training corpus by producing multiple token sequences that share the same semantics. In~\Cref{tab:ablation}, we observe that incorporating SPR significantly improves recommendation performance. One possible reason is that SPR increases token utilization rates. To validate this assumption, we segment the action sequences in each training epoch using two strategies: naive segmentation and SPR. As shown in~\Cref{fig:token_util}, naive segmentation uses only $56.89\%$ of tokens for model training, limiting the model's ability to generalize to unseen action sequences. In contrast, SPR achieves a token utilization rate of $87.01\%$ after the first training epoch, with further increases as training progresses. These results demonstrate that the proposed SPR segmentation strategy improves the utilization of ActionPiece tokens, enabling better generalization and enhanced performance.


\subsubsection{Performance \wrt Inference-Time Ensembles}\label{sec:inference_time_ensemble}

As described in~\Cref{subsubsec:inference}, ActionPiece supports inference-time ensembling by using SPR segmentation. We vary the number of ensembled segments, $q$, in \{N/A, 1, 3, 5, 7\}, where ``N/A'' indicates using naive segmentation during model inference. As shown in~\Cref{fig:ensemble}, ensembling more tokenized sequences improves ActionPiece's recommendation performance. However, the performance gains slow down as $q$ increases to $5$ and $7$. Since a higher $q$ also increases the computational cost of inference, this creates a trade-off between performance and computational budget in practice.

\begin{figure}[t!]
    \begin{center}
    \includegraphics[width=\columnwidth]{fig/ensemble.pdf}
    \vskip -0.15in
    \caption{Analysis of performance (NDCG@10, $\uparrow$) \wrt the number of ensembled segments $q$ during model inference.}
    \label{fig:ensemble}
    \end{center}
    \vskip -0.25in
\end{figure}

\subsection{Case Study}\label{subsec:case}

To understand how GR models benefit from the unordered feature setting and context-aware action sequence tokenization, we present an illustrative example in~\Cref{fig:case}.

Each item in the action sequence is represented as a feature set, with each item consisting of five features. The features within an item do not require a specific order. The first step of tokenization leverages the unordered nature of the feature set and applies set permutation regularization~(\Cref{subsubsec:segmentation}). This process arranges each feature set into a specific permutation and iteratively groups features based on the constructed vocabulary~(\Cref{subsubsec:vocab_construct}). This results in different segments that convey the same semantics. Each segment is represented as a sequence of sets, where each set corresponds to a token in the vocabulary.

By examining the segments and their corresponding token sequences, we identify four types of tokens, as annotated in~\Cref{fig:case}: (1) a subset of features from a single item (token {\setlength{\fboxsep}{0pt}\colorbox{myblue}{14844}} corresponds to features {\setlength{\fboxsep}{0pt}\colorbox{myblue}{747}} and {\setlength{\fboxsep}{0pt}\colorbox{myblue}{923}} of the T-shirt); (2) a set containing a single feature (feature {\setlength{\fboxsep}{0pt}\colorbox{mygreen}{76}} of the socks); (3) all features of a single item (token {\setlength{\fboxsep}{0pt}\colorbox{myyellow}{7995}} corresponds to all features of the shorts); and (4) features from multiple items (\eg token {\setlength{\fboxsep}{0pt}\colorbox{myblue}{83}\colorbox{mygreen}{16}} includes feature {\setlength{\fboxsep}{0pt}\colorbox{myblue}{923}} from the T-shirt and feature {\setlength{\fboxsep}{0pt}\colorbox{mygreen}{679}} from the socks, while token {\setlength{\fboxsep}{0pt}\colorbox{mygreen}{19}\colorbox{myyellow}{895}} includes feature {\setlength{\fboxsep}{0pt}\colorbox{mygreen}{1100}} from the socks as well as features {\setlength{\fboxsep}{0pt}\colorbox{myyellow}{560}} and {\setlength{\fboxsep}{0pt}\colorbox{myyellow}{943}} from the shorts). Notably, the fourth type of token demonstrates that the features of one action can be segmented and grouped with features from adjacent actions. This results in different tokens for the same action depending on the surrounding context, showcasing the context-aware tokenization process of ActionPiece.


% \section{Discussion}


\section{Conclusion }
This paper introduces the Latent Radiance Field (LRF), which to our knowledge, is the first work to construct radiance field representations directly in the 2D latent space for 3D reconstruction. We present a novel framework for incorporating 3D awareness into 2D representation learning, featuring a correspondence-aware autoencoding method and a VAE-Radiance Field (VAE-RF) alignment strategy to bridge the domain gap between the 2D latent space and the natural 3D space, thereby significantly enhancing the visual quality of our LRF.
Future work will focus on incorporating our method with more compact 3D representations, efficient NVS, few-shot NVS in latent space, as well as exploring its application with potential 3D latent diffusion models.


%\section*{Acknowledgments}
%...


% Bibliography entries for the entire Anthology, followed by custom entries
%\bibliography{anthology,custom}
% Custom bibliography entries only
\bibliography{paper}

\appendix

% \section{Example Appendix}
\label{sec:appendix}

%\onecolumn
\section{More Results}
\label{sec:more_results}

\subsection{Qualitative Results}
% an example of one iteration that improves the model utility

% BigToM example 
% AutoToM - bbfb question 78: needs sobag and initial proposal is sob --> sobag 
% Question 78': "Initial model propose: ['State', 'Observation', 'Belief']\nAssigned models: {0: ['State', 'Observation', 'Belief', 'Action', 'Goal']}"
Among general methods, AutoToM achieves state-of-the-art results across all benchmarks. We provide two qualitative examples to illustrate the effect of variable adjustment (example 1) and timestep adjustment (example 2). These examples also demonstrate the interpretability of \ours, as the constructed model offers us insights into how the method is modeling the agent behavior for the inference. 

\textbf{Example 1: BigToM (Backward Belief Inference)}

\begin{tcolorbox}[
    breakable,
    colframe=gray!40!black,
    colback=gray!5,
    coltitle=white,
    fonttitle=\bfseries,
    colbacktitle=gray!40!black
]
\textbf{Story:} Kavya is a florist in a vibrant Indian market. Kavya wants to create a beautiful bouquet of fresh roses for a customer's anniversary celebration. Kavya sees a batch of roses in her shop that appear to be fresh and vibrant. Unbeknownst to her a mischievous monkey sneaks into the shop and nibbles on the rose petals leaving them damaged and unsuitable for the bouquet. Kavya starts arranging the bouquet using the roses she initially saw. \newline

\textbf{Question: Does Kavya believe the roses are fresh and perfect for the bouquet or damaged by the monkey?}

(a) Kavya believes the roses are fresh and perfect for the bouquet. \textbf{\textcolor[RGB]{110, 170, 110}{(Correct Answer)}}

(b) Kavya believes the roses are damaged by the monkey.
\end{tcolorbox}



\textbf{Variables in the Initial Model Proposal: } State, Observation, Belief

Since the scenario involves only one timestep, a single model suffices. In the initial model, the state of the world indicates that the flowers are damaged after the monkey nibbles on them. However, it remains unclear whether Kavya observes the true condition of the flowers. The model lacks crucial information about Kavya's actions, which are observable and influenced by her beliefs about the flowers' state. These actions can help infer her true belief. Initially, the probability that Kavya believes the flowers are fresh is moderate, $P(\text{Kavya believes the roses are fresh and perfect}$ $\text{for the bouquet} | X^1) = 0.50$. Without variable adjustment, the model cannot answer the question.

\textbf{Variables in the Adjusted Model: } State, Observation, Belief, Action, Goal 

For the initial model, the reward is $R(M,q)=-H(P(q | X^{t_s:t}))=-0.693$ and the model cost is $C(M)=\alpha|M|=0.04$, resulting in a utility $U(M, q)=-0.733$, which does not exceed the utility threshold $U_\text{min}=-0.693$. To address the insufficiency of the initial model's utility relative to our termination threshold, we propose an enhanced model incorporating state, observation, belief, action, and goal. In this revised model, Kavya’s actions—specifically arranging the bouquet using the roses—align with her goal of creating a beautiful bouquet. These observations allow us to infer with high probability that Kavya believes the roses are fresh and suitable for the bouquet, increasing the belief probability to $P(\text{Kavya believes the roses are fresh and perfect}$ $ \text{for the bouquet}| X^1) = 0.97$. With this revised model, the reward is $R(M,q)=-H(P(q | X^{t_s:t}))=-0.135$ and the model cost is $C(M)=\alpha|M|=0.06$, resulting in a utility $U(M, q)=-0.195$, which exceeds our utility threshold $U_{\text{min}}=-0.693$. Based on the adjusted model, \ours can confidently determine the correct answer: (a) Kavya believes the roses are fresh and perfect for the bouquet.\newline


\textbf{Example 2: MMToM-QA (Belief Inference)}

%\textbf{Text input}
\begin{tcolorbox}[
    breakable,
    colframe=gray!40!black,
    colback=gray!5,
    coltitle=white,
    fonttitle=\bfseries,
    colbacktitle=gray!40!black
]

\textbf{Video input:}

\includegraphics[width=\linewidth]{figures/MMToM-frame.png}
\vspace{-3pt}
  
\textbf{What's inside the apartment:} The apartment consists of a bedroom, kitchen, living room, and bathroom. 
In the bedroom, there is a coffee table and a desk. 
The kitchen is equipped with four cabinets, a fridge, a kitchen table, a microwave, and a stove. The 3rd kitchen cabinet from the left houses a water glass and a dish bowl. Inside the fridge, there are two apples, a salmon, a plate, and a dish bowl. The 2nd kitchen cabinet from the left contains a water glass, a chips, a condiment bottle, and a dish bowl. The 1st kitchen cabinet from the left holds a wineglass, a wine, and a condiment bottle. The microwave contains a salmon, and there is a cupcake in the stove. The 4th kitchen cabinet from the left has a plate. 
The living room features a cabinet, a sofa, a coffee table, and a desk. Inside the cabinet, there are two apples and four books. A plate and a remote control are placed on the coffee table. 
The bathroom is furnished with a bathroom cabinet, which is currently empty. \newline

\textbf{Actions taken by Mark:} Mark is situated in the bathroom. He proceeds towards the kitchen, making his way to the stove. He opens and then closes the stove. Subsequently, he strides towards the 4th kitchen cabinet, opens it, and then shuts it. He then moves to the 2nd kitchen cabinet, opens and closes it, before doing the same with the 3rd kitchen cabinet. Finally, he heads towards the 1st kitchen cabinet, opens and closes it, and is about to open the microwave. \newline

\textbf{Question: If Mark has been trying to get a salmon, which one of the following statements is more likely to be true?}

(a) Mark thinks that the salmon is not inside the microwave.

(b) Mark thinks that the salmon is inside the microwave. \textbf{\textcolor[RGB]{110, 170, 110}{(Correct Answer)}}
\end{tcolorbox}




In this problem, we first fuse the information from text and video following \citet{jin2024mmtom}. The fused information is structured into 23 timesteps, each corresponding to an action of Mark at the time. We then propose the initial model: State, Observation, Belief, Action, Goal.

\textbf{Without timestep adjustment.} Bayesian inference must be performed sequentially from the first timestep, even though most actions do not contribute to answering the final question. The model will compute across all timesteps, while the most informative action is actually the last one: if Mark wants to get a salmon but does not believe there is one inside the microwave, he will not open it.

\textbf{With timestep adjustment.} We begin inference from the last timestep, 
where the action likelihood $P(a|b, g)$ is low when $b=$ \textit{Mark thinks that the salmon is not inside the microwave}, and high when $b=$ \textit{Mark thinks that the salmon is inside the microwave}. After performing inference at the last timestep, the belief probabilities corresponding to the choices are $0.998$ and $0.002$. The reward is given by $R(M,q)=-H(P(q | X^{t_s:t}))=-0.014$, while the model cost is $C(M)=\alpha|M|=0.06$. This results in a utility of $U(M, q)=-0.074$, which exceeds the threshold $U_{\text{min}}=-0.693$, allowing our model to determine the final answer without considering earlier timesteps.

\begin{figure}[t!]
  \centering
  \includegraphics[width=1.0\linewidth]{figures/higher_order.pdf}
    % \vspace{-15pt}
  \caption{Comparison of accuracy between \ours and GPT-4o on the HiToM dataset across different reasoning orders. Order 0 refers to questions about an object's actual location; order 1 questions are about an agent's belief about an object's location; order 2 involves questions about an agent's belief regarding another agent's belief, and so forth.}
  \label{fig:higher_order}
  % \vspace{-10pt}
\end{figure}

\subsection{Results for Higher Order Inference}
% accuracy table for orders in hi-tom

Higher-order Theory of Mind (ToM) involves recursive reasoning about others' mental states across multiple levels. The Hi-ToM benchmark \cite{he2023hi} includes questions ranging from Order 0, which involves no agents and asks about the actual location of objects, up to Order 4, which requires recursive reasoning among four agents. Figure \ref{fig:higher_order} compares the performance of GPT-4o and \ours across these different question orders. While GPT-4o experiences a significant decline in accuracy as the ToM order increases, \ours maintains a smaller performance drop and achieves substantially higher accuracy on higher-order questions. This demonstrates that our model-based approach is more robust and scalable, effectively handling complex scenarios involving multiple agents and various levels of recursive reasoning.



\subsection{Full Results of the Ablation Study}\label{sec:more_results_ablation}

Table~\ref{tab:results_ablation_accuracy} shows the performance of ablated methods compared to the full \ours method on all benchmarks.

\begin{table*}[t!]
\centering
% \vspace{5pt}
\begin{small}
\begin{tabular}{c|c|c|c|c|c|c}
\toprule
\textbf{Method} & \textbf{ToMi} & \textbf{BigToM} & \textbf{MMToM-QA} & \textbf{MuMA-ToM} & \textbf{Hi-ToM} &\textbf{All} \\
\midrule
w/o hypo. reduction & 87.60 & 86.17 & 75.83 & 81.67 & 69.50 & 80.15 \\
w/ POMDP & 76.00 & 86.50 & 79.83 & 50.78 & 67.00 & 72.02 \\ 
%\ours w/o model discovery \\
w/o variable adj. & 85.80 & 78.25 & 76.17 & 77.89 & 66.50 & 76.92 \\
% \ours w/o timestep adjustment \\
w/ last timestep & 68.40 & 77.83 & 74.33 & 78.33 & 44.50 & 68.68 \\
w/ all timesteps & 86.00 & 79.09 & 76.50 & 79.33 & 69.00 & 77.98 \\
\midrule
%\ours w/ Model Input & 86.72 & 72.73 & 80.17 & 84.00 & & \textbf{80.91} \\
\ours & 88.30 & 86.92 & 75.50 & 81.44 & 72.50 & 80.93 \\
\bottomrule
\end{tabular}
\end{small}
\caption{Results of ablated methods compared to the full \ours method.}
\label{tab:results_ablation_accuracy}
\end{table*}


In Table \ref{tab:results_ablation_tokens} and \ref{tab:results_ablation_api}, we compare the ablated methods and the full model on the averaged number of tokens per question (in thousands) and the averaged number of API calls at inference per question.



\begin{table*}[t!]
\centering
% \vspace{5pt}
\begin{small}
\begin{tabular}{c|c|c|c|c|c|c}
\toprule
\textbf{Method} & \textbf{ToMi} & \textbf{BigToM} & \textbf{MMToM-QA} & \textbf{MuMA-ToM} & \textbf{Hi-ToM} &\textbf{All} \\
\midrule
w/o hypo. reduction & 15.8 & 6.8 & 19.2 & 24.4 & 20.4 & 17.3 \\
 w/ POMDP & 14.9 & 5.5 & 15.6 & 20.0 & 18.8 & 15.0 \\ 
%\ours w/o model discovery \\
w/o variable adj. & 8.5 & 6.1 & 16.4 & 14.0 & 10.0 & 11.0 \\
% \ours w/o timestep adjustment \\
w/ last timestep & 7.8 & 6.1 & 6.4 & 11.6 & 4.0 & 7.2 \\
w/ all timesteps & 14.2 & 7.7 & 57.2 & 16.4 & 12.4 & 21.6 \\ 
\midrule
%\ours w/ Model Input & 86.72 & 72.73 & 80.17 & 84.00 & & \textbf{80.91} \\
\ours & 9.8 & 6.5 & 14.4 & 13.6 & 12.0 & 11.3 \\
\bottomrule
\end{tabular}
\end{small}
\caption{Comparison of ablated models and the full model on the averaged number of tokens per question (in thousands). Lower is better.}
\label{tab:results_ablation_tokens}
\end{table*}

\begin{table*}[t!]
\centering
% \vspace{5pt}
\begin{small}
\begin{tabular}{c|c|c|c|c|c|c}
\toprule
\textbf{Method} & \textbf{ToMi} & \textbf{BigToM} & \textbf{MMToM-QA} & \textbf{MuMA-ToM} & \textbf{Hi-ToM} &\textbf{All} \\
\midrule
w/o hypo. reduction & 38.91 & 13.99 & 45.97 & 70.73 & 72.58 & 48.44 \\
 w/ POMDP & 36.25 & 8.32 & 41.18 & 42.10 & 51.73 & 35.92 \\ 
%\ours w/o model discovery \\
w/o variable adj. & 22.91 & 12.99 & 35.46 & 35.76 & 29.81 & 27.39 \\
% \ours w/o timestep adjustment \\
w/ last timestep &  21.60 & 12.76 & 12.75 & 28.39 & 9.39 & 16.98 \\
w/ all timesteps & 39.83 & 15.95 & 116.81 & 43.25 & 36.27 & 50.42 \\ 
\midrule
%\ours w/ Model Input & 86.72 & 72.73 & 80.17 & 84.00 & & \textbf{80.91} \\
\ours & 32.23 & 13.81 & 31.36 & 35.08 & 36.45 & 29.79 \\
\bottomrule
\end{tabular}
\end{small}
\caption{Comparison of ablated models and the full model on the averaged number of API calls at inference per question. Lower is better.}
\label{tab:results_ablation_api}
\end{table*}

\subsection{Per-type Accuracy on All Benchmarks}

In Tables~\ref{tab:per_type_acc_tomi} - \ref{tab:per_type_acc_hitom}, we present the results of \ours and baselines on each question type of all benchmarks. Here we compare general methods that can be applied to all benchmarks.


% tomi 
\begin{table*}[t!]
\centering
% \vspace{5pt}
\begin{small}
\begin{tabular}{c|c|c|c|c|c}
\toprule
\textbf{Question Type} & \textbf{First order} & \textbf{Second order} & \textbf{Reality} & \textbf{Memory} & \textbf{All} \\
\midrule
Llama 3.1 70B & 73.75 & 56.25 & 100.00 & 100.00 & 72.00 \\
% Gemini 1.5 Flash & 53.50 & 58.75 & 100.00 & 100.00 & 64.90 \\
Gemini 2.0 Flash & 58.50 & 58.25 & 100.00 & 100.00 & 66.70 \\
% Gemini 2.0 Flash Thinking & 83.25 & 61.75 & 100.00 & 100.00 & 78.00 \\
Gemini 2.0 Pro & 75.00 & 54.75 & 100.00 & 100.00 & 71.90 \\ 
GPT-4o & 80.25 & 62.25 & 100.00 & 100.00 & 77.00 \\ 
% GPT-o3-mini & 79.50 & 53.25 & 100.00 & 100.00 & 73.10\\ 
SimToM & 84.75 & 65.00 & 100.00 & 100.00 & 79.90 \\ 
\ours & {95.00} & {77.50} & 93.00 & 100.00 & 88.30 \\
\bottomrule
\end{tabular}
\end{small}
\caption{Detailed accuracy for ToMi.}
\label{tab:per_type_acc_tomi}
\end{table*}


% bigtom 
\begin{table*}[t!]
\centering
% \vspace{5pt}
\begin{small}
\begin{tabular}{c|c|c|c|c|c}
\toprule
\textbf{Question Type} & \textbf{Forward TB} & \textbf{Forward FB}  & \textbf{Backward TB} & \textbf{Backward FB} & \textbf{All}\\
\midrule
Llama 3.1 70B & 93.75 & 81.00 & 57.00 & 60.50 & 77.83\\
% Gemini 1.5 Flash & {96.50} & 61.50 & {79.00} & 57.50 & 75.42\\
Gemini 2.0 Flash & 94.25 & 87.50 & 77.50 & 51.00 & 82.00\\
% Gemini 2.0 Flash Thinking & 94.75 & 91.50 & 77.50 & 47.00 & 82.83 \\
Gemini 2.0 Pro & 96.00 & 93.75 & 70.00 & 68.50 & 86.33\\ 
GPT-4o & 96.00 & 88.50 & 63.50 & 62.00 & 82.42\\
% GPT-o3-mini & 93.25 & 90.75 & 78.50 & 75.00 & 86.92\\ 
SimToM & 92.50  & 90.00 & 25.00 & 75.00 & 77.50 \\
% TimeToM & 95.00 & {99.00} & - & - & -\\
\ours & 91.25 & {93.75} & 73.00 & {78.50} & {86.92} \\
\bottomrule
\end{tabular}
\end{small}
\caption{Detailed accuracy for BigToM.}
\label{tab:per_type_acc_bigtom}
\end{table*}

% mmtom
\begin{table*}[t!]
\centering
% \vspace{5pt}
\begin{small}
\begin{tabular}{c|c|c|c}
\toprule
\textbf{Question Type} & \textbf{Belief} & \textbf{Goal} &\textbf{All} \\
\midrule
Llama 3.1 70B & 51.33 & 36.33 & 43.83\\
% Gemini 1.5 Flash & 39.00 & 28.00 & 33.50 \\
Gemini 2.0 Flash & 62.67 & 33.33 & 48.00\\
% Gemini 2.0 Flash Thinking & 73.33 & 34.67 & 54.00 \\
Gemini 2.0 Pro & 57.00 & 44.67 & 50.84\\ 
GPT-4o & 55.67 & 32.33 & 44.00 \\
% GPT-o3-mini & 88.67 & 40.67 & 64.67\\ 
SimToM & 75.67 & 26.33 & 51.00 \\
% SimToM & 64.30 & 40.70 & 52.50 \\
\ours & {88.67} & {62.33} & {75.50} \\
\bottomrule
\end{tabular}
\end{small}
\caption{Detailed accuracy for MMToM-QA.}
\label{tab:per_type_acc_mmtom}
\end{table*}


% mumatom
\begin{table*}[t!]
\centering
% \vspace{5pt}
\begin{small}
\begin{tabular}{c|c|c|c|c}
\toprule
\textbf{Question Type} & \textbf{Belief} & \textbf{Goal} & \textbf{Belief of Goal} & \textbf{All} \\
\midrule
Llama 3.1 70B & 68.67 & 51.33 & 47.33 &  55.78 \\
% Gemini 1.5 Flash & 70.67 & 39.34 & 47.67 & 52.56\\
Gemini 2.0 Flash & 68.33 & 50.67 & 47.00 & 55.33\\
% Gemini 2.0 Flash Thinking & 95.33 & 79.00 & 73.33 & 82.56\\
Gemini 2.0 Pro & 63.00 & 66.67 & 57.00 & 62.22 \\ 
GPT-4o & 85.33 & 57.00 & 48.33 & 63.55 \\
% GPT-o3-mini & 74.00 & 67.67 & 68.33 & 70.00 \\ 
SimToM & 54.60 & 43.50 & 44.80 & 47.63 \\
\ours & 88.33 & 77.00 & 79.00 & 81.44 \\
\bottomrule
\end{tabular}
\end{small}
\caption{Detailed accuracy for MuMA-ToM.}
\label{tab:per_type_acc_mumatom}
\end{table*}




% hitom
\begin{table*}[t!]
\centering
% \vspace{5pt}
\begin{small}
\begin{tabular}{c|c|c|c|c|c|c}
\toprule
\textbf{Question Type} & \textbf{Order 0} & \textbf{Order 1} & \textbf{Order 2} & \textbf{Order 3} & \textbf{Order 4} & \textbf{All} \\
\midrule
Llama 3.1 70B & 65.00 & 47.50 & 22.50 & 20.00 & 20.00 & 35.00 \\  
% Gemini 1.5 Flash & 87.50 & 65.00 & 37.50 & 25.00 & 20.00 & 47.00\\
Gemini 2.0 Flash & 95.00 & 70.00 & 50.00 & 27.50 & 20.00 & 52.50\\
% Gemini 2.0 Flash Thinking & 100.00 & 85.00 & 72.50 & 50.00 & 60.00 & 65.50\\ 
Gemini 2.0 Pro & 100.00 & 62.50 & 50.00 & 37.50 & 37.50 & 57.50 \\ 
GPT-4o & 92.50 & 65.00 & 40.00 & 27.50 & 25.00 & 50.00 \\
% GPT-o3-mini & 100.00 & 72.50 & 65.00 & 60.00 & 77.50 & 75.00 \\
% Llama 3.1 70B & 95.00 & 60.00 & 30.00 & 30.00 & 35.00 & 50.00 \\
% Gemini 1.5 Flash & 95.00 & 80.00 & 45.00 & 25.00 & 15.00 & 52.00\\
% GPT-4o & {97.50} & 72.50 & 45.00 & 32.50 & 32.50 & 56.00 \\
SimToM & 100 & 77.50 & 60.00 &  60.00 & 57.50 & 71.00 \\
\ours & 95.00 & {75.00} & 70.00 & 67.50 & {55.00} & {72.50} \\
\bottomrule
\end{tabular}
\end{small}
\caption{Detailed accuracy for HiToM.}
\label{tab:per_type_acc_hitom}
\end{table*}
\section{\ours Implementation Details}
\label{sec:method_details}

\subsection{Variable Adjustments}\label{sec:app_var_adj}

Table~\ref{tab:model_adjustment} summarizes possible variable adjustments at each timestep.

\begin{table}[t!]
\centering
% \vspace{5pt}
\begin{small}
\begin{tabular}{p{1.2cm}cc}
\toprule
New Var. & Before & After \\
\midrule
\multirow{4}{*}{Goal}
& $P(a^t \mid s^t)$ & $P(a^t \mid s^t, g) P(g)$ \\
& $P(a^t \mid b^t)$ & $P(a^t \mid b^t, g) P(g)$ \\
& $P(a^t)$ & $P(a^t \mid s^t, g) P(g)$ \\
& $P(a^t)$ & $P(a^t \mid b^t, g) P(g)$ \\
\midrule
\multirow{3}{*}{Belief}
& $P(a^t \mid s^t)$ & $P(a^t \mid b^t) P(b^t \mid s^t, b^{t-1})$ \\
& $P(a^t \mid s^t, g)$ & $P(a^t \mid b^t, g) P(b^t \mid s^t, b^{t-1})$ \\
% & None & $P(b^t \mid s^t, b^{t-1})$ \\
\midrule
\multirow{1}{*}{Observ.}
& $P(b^t \mid s^t, b^{t-1})$ & $P(b^t \mid o^t, b^{t-1}) P(o^t \mid s^t)$ \\
\midrule
%\multirow{2}{*}{Action}
%\midrule
\multirow{1}{*}{Int. State}
%& Belief & Belief of other agents \\
& $b(s^t)$ & $b(is^t)$ \\
\bottomrule
\end{tabular}
\end{small}

\caption{Potential variable adjustments, including introducing goal, belief, observation, and interactive state (for high-order ToM). We show the corresponding local conditionals before and after introducing the new variables.}
\label{tab:model_adjustment}

\end{table}


\subsection{Automated Bayesian Inverse Planning}

\textbf{Hypothesis Sampling.} At each timestep, hypotheses for the latent variables are generated using a Large Language Model (LLM) as the backend, guided by the observed variables. Specifically, when the state is not explicitly provided, the LLM acts as a world model, tracking state changes in the story based on the previous state and current actions. For an agent's observation, the LLM is prompted to adopt the perspective of a character, simulating what that character might see, know, or hear in the given environment (e.g., inside a closed room). If no new observation is available at a specific timestep, we neither generate new observations nor update the belief. Additionally, the LLM proposes plausible hypotheses for the agent's belief and goal based on the available information.

\textbf{Hypothesis reduction.} We examine all local conditional probabilities involving a single uncertain variable with multiple hypotheses and eliminate those hypotheses that result in significantly low likelihood values. For example, in $P(o^t \mid s^t)$, where $s^t$ represents a determined state, any observation hypothesis that yields a low likelihood for this term is discarded. This approach reduces the computational cost of estimating $P(b^t \mid o^t, b^{t-1})$. Similarly, the same principle is applied to $P(a^t \mid b^t, g^t)$ and $P(u^t \mid b^t, g^t)$, where unlikely belief hypotheses are removed to further reduce computational complexity.


\subsection{Automated Model Discovery}
% \subsection{Hyperparameters}
When exploring different models during the model discovery, \ours can reuse the hypothesis proposals of variables and local conditionals from previously computed models to avoid repeated computation.

We configure the hyperparameters in Automated Model Discovery as follows: $\alpha = 0.02$, $U_\text{min} = -0.693$.



\subsection{Recursive Reasoning}

Interactive Partially Observable Markov Decision Process (I-POMDP) extends POMDP to multi-agent settings by introducing the concept of interactive states, which include agent models into the state space to capture the recursive reasoning process \citep{gmytrasiewicz2005framework}.
We denote $is_{i, l}$ as the interactive state of agent $i$ at level $l$. For two agents $i$ and $j$, where agent $i$ is interacting with agent $j$, the interactive states at each level are defined as:
\begin{itemize}
\setlength\itemsep{0pt}
    \item \textbf{Level 0:} $is_{i,1}=s$
    \item \textbf{Level 1:} $is_{i,1}=(s,b_{j,0},g_j)$ where $b_{j,0}$ is a distribution over $j$'s interactive state at level 0, $is_{j,0}$
    \item ...
\end{itemize}
%\end{addmargin}

The framework provides a generative model for agents: given agent $i$'s belief of interactive state $b(is_{i,l})$, its action policy will be $\pi (a_i|is_{i,l},g_i)$, and its utterance policy will be $\pi (u_i|is_{i,l},g_i)$. 

In our implementation, we sample one possible state based on $b(s)$ at level $l$ to approximate the state at level $l-1$ as imagined by the agent at level $l$. We can recursively apply this process until reaching level $0$. Based on the state sampled for level $0$, we can then conduct the typical automated BIP based on the model structure at that level. This approach can be conveniently applied to arbitrary levels of recursive reasoning, allowing us to answer higher-order Theory of Mind questions using the same method.


\subsection{BToM Model Space}
\label{sec:model_space}

To apply Bayesian Inverse Planning (BIP) across various scenarios, we define the mental variables and their causal relationships with agent behavior using a family of Bayesian Theory of Mind (BToM) models. These models accommodate different levels of complexity in how agents behave and reason about their environment. 

At each timestep $t$, the observable variables are represented by:
$$X^t = \{x_i^t\}_{i \in N_X} \text{, where } N_X = \{s^t, a^t, u^t\}$$
Here, the state $s^t$ always appear in $X^t$, while either $a^t$ (action) or $u^t$ (utterance) is included at timestep $t$, depending on whether physical motion or verbal communication is presented. In some cases, $a^t$ is only used to update the state and does not affect the inference of beliefs or goals, while in other scenarios it can be crucial for inferring hidden mental states (e.g., an agent’s belief or goal).

The latent variables are denoted by
$$V^t = \{v_i^t\}_{i \in N_V} \text{, where } N_V = \{o^t, b^t, g^t\}$$

Here, the observation $o^t$ is only included when the agent’s belief $b^t$ is part of the model, as it updates $b^t$. The goal $g^t$ is included only if it influences action and is relevant to inference. In cases of higher-order recursive reasoning among multiple agents, the belief over the state $b^t(s^t)$ extends to belief over an interactive state $b^t(is^t)$.

Combining these choices at each timestep yields a model space with 30 possible configurations:
%\begin{addmargin}[-10pt]{0pt}
\begin{itemize}
\setlength\itemsep{0pt}
    \item Action/Utterance: which one is included (2 options).
    \item Belief/Observation: no belief, belief of state, belief of interactive state, belief of state, or belief of interactive state + observation (5 options).
    \item Action(Utterance)/Goal: no goal (action(utterance) irrelevant), action(utterance) only, or action(utterance) + goal (3 options).
\end{itemize}
%\end{addmargin}
Over a time interval from $t_s$ to $t$, this scales to $30^{t-t_s+1}$ possible models.



\textbf{Examples.} In addition to the Markov Decision Process (MDP), Partially Observable Markov Decision Process (POMDP), and Interactive POMDP (I-POMDP) models introduced in Section~\ref{sec:preliminaries}, we present additional examples of models from the BToM model space:
%\begin{addmargin}[-10pt]{0pt}
\begin{itemize}
\setlength\itemsep{0pt}
    \item Observation Update Model: Used in the ToMi benchmark (see Figure \ref{fig:benchmarks_and_models}), this model focuses on how observations update beliefs. Actions are present but only serve to update states and are irrelevant to the inference questions. This model is well-suited for passive scenarios where the focus is on understanding how hidden states produce observable evidence and how the agent updates its beliefs about the world.
    %\item Simple Markov Model: The agent's environment is described as a sequence of states that generate observations. The agent's belief at any given time step is determined solely by the current and past observations. This model assumes no explicit actions, making it well-suited for passive scenarios where the focus is on understanding how hidden states produce observable evidence and how the agent updates its beliefs about the world.
    \item POMDP Variant without Goal: A partially observable scenario in which goals are trivial or irrelevant. This variant emphasizes how partial observability affects belief formation and action selection, without explicit goal-driven behavior.
\end{itemize}
%\end{addmargin}
\section{Baseline Implementation Details}
\label{sec:baseline_details}


% BigToM situation. 
For the baselines, we use \texttt{gpt-4o-2024-08-06\\} for GPT-4o, \texttt{meta-llama/Llama-3.1-70B-Inst-\\ruct} from Hugging Face for Llama 3.1 70B, \texttt{gemini-2.0-flash} for Gemini 2.0 Flash, and \texttt{gemini-2.0-pro-exp-02-05} for Gemini 2.0 Pro. 
Among the ToM prompting for LLM benchmarks previously tested on the BigToM dataset, e.g., TimeToM and SimToM, they only tested the subset of the entire dataset with questions for forward action and forward belief and did not test on backward belief questions. With the available SimToM code, we tested it on the full BigToM dataset with GPT-4o, while TimeToM does not have its code available. 


% WHy symoblic tom different
SymbolicToM maps out the agents' beliefs throughout stories of different levels of reasoning via symbolic graphs. However, the construction of these graphs is specifically designed for the ToMi dataset, where there are fixed actions and sentence formats in the story. Thus it is difficult to generalize to more open-ended scenarios (e.g., BigToM) or stories with multiple agents acting simultaneously (e.g., Hi-ToM). Therefore, we can only evaluate  SymbolicToM on ToMi (tested with GPT-4o on the full dataset), for which it was designed.  

TimeToM is not open-source. We rely on its self-reported accuracy on ToMi. However, since it was only evaluated on a subset of BigToM with forward inference questions, its accuracy on the full BigToM benchmark remains unknown. Similarly, PercepToM is not open-source, and we rely on its self-reported accuracy on ToMi.


% why we can't use BIP-ALM and LIMP
BIP-ALM and LIMP are both models that combine BIP and LLMs to solve ToM problems. BIP-ALM manually defines symbolic representations of observable and latent variables and assumes POMDP. LIMP is designed to only solve two-level reasoning problems. It uses natural language to represent variables. Both methods assume that the goals are about finding an object and the beliefs are about the locations of that object in a household environment. 
\begin{table*}[t!]
  \begin{center}
    \begin{small}
    \begin{tabular}{p{2cm} p{1.5cm} p{2.5cm} p{0.5cm} p{1cm} p{2cm} p{2cm} p{2cm}}
    \toprule
      \textbf{Benchmark} & \textbf{Agent number} & \textbf{Tested concepts} & \textbf{Size} & \textbf{Modality} & \textbf{Communication} & \textbf{Generation} & \textbf{Evaluation}\\
    \hline
        \textbf{ToMi \cite{le2019revisiting}} &  Multi agents & First \& Second Order belief, Reality, Memory & 1000 & Text & No & Templates & Multiple choice Q\&A \\ \hline
        \textbf{BigToM \cite{gandhi2024understanding}} & Single agent & Belief, Action & 1200 & Text & No & Procedural generation & Question answering \\ \hline
        \textbf{MMTOM-QA \cite{jin2024mmtom}} & Single agent & Belief \& Goal & 600 & Text \& Video & No & Procedural generation & Multiple choice Q\&A \\ \hline
        \textbf{MuMA-ToM \citep{shi2024muma}} & Multi agents & Belief, social goal and belief of other's goal & 900 & Text \& Video & Yes & Procedural generation & Multiple choice Q\&A \\ \hline
        \textbf{Hi-ToM \cite{he2023hi}} & Multi agents & High-order beliefs & 200 & Text & Yes & Procedural Generation & Multiple choice Q\&A \\ 
    \bottomrule \\
    \end{tabular}
    \end{small}
  \caption{Summary of the ToM benchmarks used in the experiments.}
    \label{tab:comparison_table}
  \end{center}
  \vspace{-10pt}
\end{table*}


\section{Benchmark Details}
\label{sec:bench_details}

In our evaluation, we test \ours on BigToM \citep{gandhi2024understanding}, MMToM-QA \citep{jin2024mmtom}, MuMA-ToM \citep{shi2024muma}, ToMi \citep{le2019revisiting} and Hi-ToM \citep{he2023hi}. For ToMi, we use the ToMi dataset that has disambiguated container locations in the story and correctly labeled order of reasoning \cite{arodi2021textual, sap2022neural}.
For Hi-ToM, we choose the length 1 subset consisting of 200 questions across all orders (0-4)  due to the high cost of testing the full dataset. 


Table \ref{tab:comparison_table} summarizes the benchmarks used to evaluate \ours against baselines, detailing key features such as test concepts, input modalities, and the number of agents. The results demonstrate that \ours operates across diverse contexts, infers any mental state, reasons about any number of agents, and supports any level of recursive reasoning.

\begin{lstlisting}[title={Sampling Responses During Training/Inference}]
Please reason step by step, and put your final answer within 
\boxed{}. 
Problem: {problem} 
\end{lstlisting}

\begin{lstlisting}[title={Verification Refinement}]
You are a math teacher. I will give you a math problem and an answer. 
Verify the answer's correctness without step-by-step solving. Use alternative verification methods. 
Question: {problem}
Answer: {answer}
Verification:
\end{lstlisting}

\begin{lstlisting}[title={Verification Collection}]
Refine this verification text to read as a natural self-check within a solution. Maintain logical flow and professionalism.
Key Requirements:
1. Avoid phrases like "without solving step-by-step" or "as a math teacher".
2. Treat the answer as your own prior solution.
3. Conclude with EXACTLY one of:
Therefore, the answer is correct.
Therefore, the answer is incorrect.
Therefore, the answer cannot be verified.
Original text: {verification}
\end{lstlisting}


\end{document}
