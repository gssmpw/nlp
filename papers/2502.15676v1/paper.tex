% This must be in the first 5 lines to tell arXiv to use pdfLaTeX, which is strongly recommended.
\pdfoutput=1
% In particular, the hyperref package requires pdfLaTeX in order to break URLs across lines.

\documentclass[11pt]{article}

% Change "review" to "final" to generate the final (sometimes called camera-ready) version.
% Change to "preprint" to generate a non-anonymous version with page numbers.
% \usepackage[review]{acl}
\usepackage[preprint]{acl}

% Standard package includes
\usepackage{times}
\usepackage{latexsym}

% For proper rendering and hyphenation of words containing Latin characters (including in bib files)
\usepackage[T1]{fontenc}
% For Vietnamese characters
% \usepackage[T5]{fontenc}
% See https://www.latex-project.org/help/documentation/encguide.pdf for other character sets

% This assumes your files are encoded as UTF8
\usepackage[utf8]{inputenc}

% This is not strictly necessary, and may be commented out,
% but it will improve the layout of the manuscript,
% and will typically save some space.
\usepackage{microtype}

% This is also not strictly necessary, and may be commented out.
% However, it will improve the aesthetics of text in
% the typewriter font.
\usepackage{inconsolata}

%Including images in your LaTeX document requires adding
%additional package(s)
\usepackage{graphicx}


\usepackage{booktabs}
\usepackage{changepage}
\usepackage[breakable]{tcolorbox}
\usepackage{amsmath}

\usepackage{enumitem}
%\setlist[itemize]{topsep=3pt, partopsep=3pt} % Reduces space above and below

\usepackage{hyperref}
\usepackage{xspace}
\usepackage{comment}
\usepackage{multirow}
\usepackage{algorithm}
\usepackage{algpseudocode}
\newcommand{\ours}{\textsl{AutoToM}\xspace}
\algnewcommand{\LineComment}[1]{\State \(\triangleright\) #1}

\DeclareMathOperator*{\argmax}{arg\,max}
\DeclareMathOperator*{\argmin}{arg\,min}

\usepackage[breakable]{tcolorbox}
\usepackage{fix-cm}
\usepackage{fontawesome5}

\newcommand{\jcy}[1]{{\color{red}{[(JCY): #1]}}}





% If the title and author information does not fit in the area allocated, uncomment the following
%
%\setlength\titlebox{<dim>}
%
% and set <dim> to something 5cm or larger.

\title{\ours: Automated Bayesian Inverse Planning and Model Discovery \\ for Open-ended Theory of Mind}

\author{
Zhining Zhang\textsuperscript{$\diamondsuit$}\thanks{~Equal contribution. Zhining Zhang completed this work during an internship at JHU. $^\dagger$ Project lead.} \quad
Chuanyang Jin\textsuperscript{$\heartsuit$}\footnotemark[1]$^\dagger$ \quad
Mung Yao Jia\textsuperscript{$\heartsuit$}\footnotemark[1] \quad 
Tianmin Shu\textsuperscript{$\heartsuit$} \\
\textsuperscript{$\diamondsuit$} Peking University \quad
\textsuperscript{$\heartsuit$} Johns Hopkins University \\
\texttt{zzn\_nzz@stu.pku.edu.cn, \{cjin33, mjia8, tianmin.shu\}@jhu.edu} \\
% Project website: \texttt{https://chuanyangjin.com/AutoToM/}
}


\begin{document}
\maketitle
\begin{abstract}
Theory of Mind (ToM), the ability to understand people's mental variables based on their behavior, is key to developing socially intelligent agents. Current approaches to Theory of Mind reasoning either rely on prompting Large Language Models (LLMs), which are prone to systematic errors, or use rigid, handcrafted Bayesian Theory of Mind (BToM) models, which are more robust but cannot generalize across different domains. In this work, we introduce \ours, an automated Bayesian Theory of Mind method for achieving open-ended machine Theory of Mind. \ours can operate in any domain, infer any mental variable, and conduct robust Theory of Mind reasoning of any order. Given a Theory of Mind inference problem, \ours first proposes an initial BToM model. It then conducts automated Bayesian inverse planning based on the proposed model, leveraging an LLM as the backend. Based on the uncertainty of the inference, it iteratively refines the model, by introducing additional mental variables and/or incorporating more timesteps in the context. Empirical evaluations across multiple Theory of Mind benchmarks demonstrate that \ours consistently achieves state-of-the-art performance, offering a scalable, robust, and interpretable approach to machine Theory of Mind.

\vspace{0.15cm}
\quad \faIcon[regular]{star}\; \href{https://chuanyangjin.com/AutoToM}{chuanyangjin.com/AutoToM}

\quad \faGithub\ \; \href{https://github.com/SCAI-JHU/AutoToM}{github.com/SCAI-JHU/AutoToM}
\vspace{0.15cm}
\end{abstract}

\section{Introduction}
\label{sec:intro}

Foundational models (FMs)~\cite{zhang2024data, zhou2023comprehensive} have shown remarkable progress in the healthcare domain, enabling professional-like assessment of disease diagnosis, treatment decision-making, and monitoring~\cite{zhang2023text, wang2022medclip, lu2023mi-zero}. 
Examples include LLaVA-Med~\cite{li2023llava}, Med-PaLM Multimodal~\cite{tu2024towards}, and Med-Flamingo~\cite{moor2023med}, have demonstrated their capacity on question answering, medical image analysis, and report generation.
These studies follow a predominant top-down model development strategy that requires upstream developers to collect data and train models for downstream tasks. 
Consequently, the developed model capabilities are heavily dependent on the training data, limiting their generalization performance in diverse clinical scenarios. 
For instance, Med-Gemini~\cite{yang2024advancing} reveals promising general capabilities in report generation while it lags behind state-of-the-art (SoTA) models on classification tasks, especially for out-of-domain applications. 
This indicates that while the generalizability of the foundation model is promising, more solutions are expected to meet the various specialized clinical needs.

To address these challenges, multi-center data centralization becomes essential to enhance model capacity and robustness across varied clinical scenarios~\cite{rajpurkar2022ai}. 
Centralizing distributed data can significantly improve model training and inference performance.
However, the process of medical data storage, transfer, and aggregation among centers requires extra efforts to ensure data security and system interoperability~\cite{bradford2020international}.
Moreover, a growing concern for patient privacy makes large-scale multi-center data sharing particularly challenging. 
While efforts like federated learning~\cite{wen2023survey, li2020review} can achieve good model performance on local data, the need for synchronized system coordination presents significant challenges, as clients are unable to update asynchronously. This limitation greatly restricts the practical capability of such approaches.
As a result, without a flexible collaboration, medical community still struggles to fully utilize the isolated data and local computation resources for comprehensive medical AI model development. 
To address this dilemma, open-source platforms encourage public data sharing and knowledge integration~\cite{markiewicz2021openneuro, zenodo}.
However, these platforms focus solely on raw data sharing while seldom providing collaborative model training or cooperation between different institutions.
Recently, collaborative learning has emerged as a viable approach for enhancing multi-model robustness~\cite{boulemtafes2020review}. 
For instance, software-like model development~\cite{raffel2023building} mimics software engineering practices by introducing structured workflows, enabling merging, version control, and continuous model integration.
Under this design, model ability can be strengthened with incremental knowledge updates similar to the version updating in software development. 

Although collaborative learning provides a multi-model collaboration, two key challenges remain in the leakage of raw data during collaboration~\cite{huang2023lorahub} and the synchronization of multiple collaborators~\cite{mcmahan2017communication} in the medical AI community. It is still challenging to integrate decentralized, privacy-sensitive data across institutions, leading to under-utilized insights and fragmented knowledge sharing~\cite{kaissis2020secure, rajpurkar2022ai, abdullah2021ethics}.
 To address these challenges, inspired by the collaborative software development, we propose \textbf{Med}ical \textbf{Fo}undation Models Me\textbf{rg}ing (\textbf{MedForge}), a cooperative workflow enabling continuously community-driven foundation model (FM) development.
MedForge enables a lightweight manner for individual centers to share their knowledge among multiple centers, minimizing the burden of data transmission and integration while enhancing model robustness.
Meanwhile, MedForge facilitates asynchronous and flexible collaboration, allowing individual centers to continuously update and improve medical FMs without the need for real-time synchronization.
Similar to open-source software development, MedForge incrementally updates medical knowledge and follows a sustainable model development scheme. 
This key design emphasizes a bottom-up construction of a multi-task medical FM, allowing downstream users to collaboratively build, refine, and update the upstream model according to their local resources. Our major contributions of MedForge are as below: 
\begin{enumerate}
    \item[$\bullet$] We introduce a collaborative workflow to promote the merging scheme of open-source software development. Our proposed MedForge allows distributed clinical centers to asynchronously contribute to comprehensive medical model construction while reducing transmitting costs among centers and avoiding the leakage of raw data, thus enhancing the utilization of private resources in the healthcare system. 
    \item[$\bullet$] We propose two effective knowledge-merging strategies for the asynchronous branch contribution. The MedForge-Fusion strategy updates the plugin module parameters of the main model during the merging phase, whereas the MedForge-Mixture strategy integrates the output of the plugin module by memorizing each contributor's coefficient. These strategies make MedForge more flexible and versatile. MedForge-Fusion is friendly to implement, while the MedForge-Mixture offers better performance and robustness.
    \item[$\bullet$]  We comprehensively evaluate model merging strategies to accumulate medical knowledge among multiple branch plugin modules. MedForge yields superior performance on medical classification tasks compared to other collaborative baselines across multiple datasets. We demonstrate the robustness of MedForge by shuffling the task order and evaluating various configurations of plugin modules and dataset distillation methods.
\end{enumerate}



\section{Related Work}
\label{sec:related}
\subsection{Collaborative Systems}
In the era of rapid growth in medical foundational models~\cite{huang2023visual,wang2022medclip, zhang2024data}, the top-down model development paradigm limits model capabilities by heavily relying on the resources available to the model builders. 
Such paradigm often restricts the potential of these models, as they cannot effectively utilize the diverse, private, and decentralized resources that exist within the broader medical community.
In contrast, collaborative systems present a promising alternative, offering a more flexible approach to model development.

Collaborative systems enable institutions to share knowledge by allowing distributed collaborators to contribute to a common goal~\cite{boulemtafes2020review}. 
To further protect patient privacy, federated learning (FL)~\cite{mcmahan2017communication} was proposed to alleviate such privacy concerns as server aggregating parameter updates from multiple clients without sharing their local data. 
While subsequent optimizations, such as aggregation algorithms~\cite{mcmahan2017communication, zhao2018federated, li2020federated}, secure learning~\cite{hardy2017private, xie2021crfl}, fairness improvements~\cite{sharma2022federated, zhao2022dynamic} and its application in medicine~\cite{kumar2024privacy}, have enhanced the capacity and applicability of FL, its real-world flexibility remains limited. This is primarily due to the need for synchronous updates, which require the server and clients to stay in sync, or model updates will be blocked.
This synchrony issue can be mitigated by open-source software platforms (e.g., GitHub~\cite{github}), allowing independent contributions from individual developers asynchronously. Such an asynchronous scheme enables faster iteration and the integration of specialized expertise, thus offering a more flexible and scalable approach.

Unlike synchronous collaboration, asynchronous collaboration does not require collaborators to work simultaneously and collaborators can individually complete their updates.
While the concept of asynchronous collaboration has been widely used in software development, its machine-learning applications remain limited~\cite{kandpal2023git, raffel2023building}. 
With the rise of global collaboration, large models~\cite{sahajBERT, le2023bloom} are usually co-developed by collaborators given various levels of data availability. However, this collaborative scheme requires the aggregation of local data and online synchronous cooperation of developers.
Software-like model update system~\cite{raffel2023building} alleviates the synchronous problem, where models are updated incrementally, similar to software development, by introducing merging and version control to model development.
However, the existing collaborative version control system~\cite{kandpal2023git} fails to address the complexities of medical scenarios because of the heavy dependency on plain parameter averaging across the full model without accounting for the varying requirements of different tasks.
To respond, we propose MedForge, which enables an asynchronous collaborative system and ensures strong robustness toward a continuous, community-driven enhancement of medical models while overcoming potential data leakage.

\begin{figure*}[t]
\begin{center}
\includegraphics[width=.85\linewidth]{fig_overview_v3.pdf}
\end{center}
\caption{
FastAtlas Overview: In each frame, we compute charts spanning fully or partially visible triangles (a), determine texture space bounding boxes for the visible portions of the view-space projections of each chart, and tightly pack these boxes into atlases (b, here $2K \times 2K$). We simultaneously bijectively parameterize and shade the charts into their atlas boxes, obtaining high quality texture space shading (c), and use this shading to render the shaded frames (d).}
\label{fig:overview}
\label{fig:alg_overview}
\end{figure*}

\section{Overview}
\label{sec:overview}
Our work has two core contributions: a real-time, GPU-based algorithm for tight packing of general parameterized charts into compact atlases; and a real-time TSS method that
utilizes this packing.  

\paragraph*{FastAtlas Packing.}
FastAtlas runs entirely on the GPU as a series of compute shaders. It takes the bounding boxes of parameterized charts as input, and packs them into an atlas (Fig~\ref{fig:overview}b, Sec.~\ref{sec:pack}). As such, the only input it requires are the dimensions of the bounding boxes.
Its outputs are deterministic; identical input charts are packed into identical atlases. This is critical for TSS and similar applications, as it ensures that consecutive frames taken from the same camera view have the same shading. Even minute shading differences across such frames can cause sampling jitter, leading to undesirable flicker \cite{baker2012rock}. 
While prior methods such as \cite{mueller2018shading,hladky2019tessellated,hladky2021snakebinning,Neff2022MSA} cap the dimensions of the charts that can be packed as-is for a given atlas size, and scale down all charts that exceed these dimensions, we scale all charts by the same factor, and do so only when strictly necessary to achieve packing success (Figs~\ref{fig:atlas},~\ref{fig:sas_issues}). 

\paragraph*{TSS using FastAtlas.}
Our end-to-end TSS atlas generation method combines the packing method above with a novel approach for computing seamless per-frame charts. 
We define our charts as the connected components of the visible surfaces in each frame (Fig.~\ref{fig:overview}a), and efficiently compute them using a parallel union-find algorithm (Sec.~\ref{sec:visible}). Since the boundaries of these charts coincide with the contours of the rendered surface, they are {\em invisible} to the viewer. This approach 
eliminates the artifacts caused by shading discontinuities along visible seams (Fig.~\ref{fig:seams}). 

\begin{parWithWrapFigure}
\begin{wrapfigure}{l}{.27\columnwidth}%
\includegraphics[width=\linewidth]{fig_inset_view_plane.pdf}%
\end{wrapfigure}
We bijectively parametrize the {\em visible portions} of our charts by projecting them to view space (inset). This maps a constant number of texels to each pixel in the final rendered output, evenly distributing residual undersampling error across all image pixels. While conceptually straightforward, efficiently parameterizing charts containing partially visible triangles using viewspace projection is non-trivial, as the visible portions may no longer be triangular (e.g. green triangle in the inset); applying naive projection to triangles with vertices behind the camera may produce ill-posed results. Clipping triangles before projection is both computationally expensive and significantly complicates downstream operations. We avoid explicit clipping by observing that all that is required for atlas packing is the dimensions of, potentially conservative, bounding boxes of these projected visible portions. We compute such bounding boxes without explicit chart clipping by adapting a conservative screen coverage estimator \shortcite{Blinn:CalculatingScreenCoverage} (Sec.~\ref{sec:box}). We then pack the computed boxes using FastAtlas. 
\end{parWithWrapFigure}

Finally, we shade the visible portion of each chart into its corresponding atlas bounding box (Fig~\ref{fig:overview}c). 
The resulting texture is then used during rasterization as a standard texture map (Fig. ~\ref{fig:overview}d). 
Our framework is compatible with all existing approaches for texture space shading, including forward shading, raytraced illumination, or deferred shading in texture space \cite{baker:2016}. In the examples shown, we use the standard forward shading based rendering pipeline included in the G3D Innovation Engine \cite{G3D17}, a commercial grade renderer.


\subsection{Model Merging}
In collaborative systems, proper model merging becomes increasingly vital for improving model knowledge integration from multiple sources in a resource-limited environment~\cite{li2023deep, yang2024model, goddard2024arcee}. Conceptually, model merging strategies can be categorized into entire model merging and partial model merging.

Entire model merging involves combining multiple model parameters to participate in the merging process by several means. Entire model merging can be viewed as an optimization problem~\cite{Matena_Raffel_2021, jin2022dataless, mavromatis2024packllm} or an alignment problem~\cite{ainsworth2022git, jordan2022repair, xu2024training, ainsworth2022git}, each offering unique advantages depending on the task at hand.
In the optimization-based approach, the goal is to find the best combination of multiple models to enhance performance and efficiency. For instance, using Fisher information approximation~\cite{Matena_Raffel_2021}, the optimization-based model merging can be interpreted as selecting parameters that maximize the joint likelihood of the models' posterior distributions. The optimization of model merging can also be guided by minimizing the prediction differences between the merged model and individual models~\cite{jin2022dataless}. 
With the development of large language models (LLM), optimization-based method is used to fuse multiple LLMs at test-time by minimizing perplexity over the input prompt~\cite{mavromatis2024packllm}.
To highlight, optimization-based methods are beneficial for scenarios requiring enhanced model performance and efficiency to integrate model parameters, while alignment-based methods~\cite{ainsworth2022git, jordan2022repair} are better suited for maintaining consistency and interpretability, facilitating critical information sharing across models.
For example, a training-free model merging strategy aligns relevant models by using a similarity matrix of their representations in both activation and weight spaces~\cite{xu2024training}.
Further, the alignment between the independently trained model and a reference model not only works for models with the same architecture but also for arbitrary model architectures~\cite{ainsworth2022git}.
In summary, the entire model merging methods can effectively integrate existing models into a merged model with enhanced functionality. However, they could lead to increased computational complexity and reduced flexibility, making them less scalable and harder to implement across diverse tasks.

Partial model merging refers to combining only specific components or layers of models to improve model merging efficiency and decrease the computational cost. 
Such specific components can come from the same network~\cite{kingetsu2021neural}, where the original network is divided into subnetworks for different purposes, and these subnetworks can then be recombined for new tasks.
Additionally, modules can originate from different functional networks and be merged using various strategies. For instance, arithmetic operations are applied in \cite{zhang2023composing} to fuse parameter-efficient modules.
While merging modules from different networks provides flexibility, the process also requires a selection strategy to ensure the resulting model aligns with the specific needs of the inference stage. 
The selection strategies are commonly designed based on the similarity of task~\cite{lv2023parameter} and domain clustering performance~\cite{chronopoulou2023adaptersoup}. Alternatively, the mixture-of-experts methods use a routing strategy to select appropriate component modules~\cite{ponti2023combining}. However, these strategies often require significant time and computational resources to filter through a large model pool. 
In contrast, LoRAHub~\cite{huang2023lorahub} offers a more lightweight approach, combining various LoRA modules for different tasks with minimal model training. Nevertheless, LoRAHub lacks flexibility for incremental updates, especially when handling unseen tasks.

Although the existing model merging approaches effectively combine the capabilities of individual models, these approaches often rely on raw data, leading to potential privacy risks. Our proposed MedForge emphasizes the prevention of raw data usage, which is particularly crucial in medical scenarios. Additionally, MedForge offers an extensible capability for incremental learning, enabling continuous model improvement.



\vspace{-5pt}
\section{Method}
\label{sec:method}
\begin{figure*}[t]
\begin{center}
\includegraphics[width=.85\linewidth]{fig_overview_v3.pdf}
\end{center}
\caption{
FastAtlas Overview: In each frame, we compute charts spanning fully or partially visible triangles (a), determine texture space bounding boxes for the visible portions of the view-space projections of each chart, and tightly pack these boxes into atlases (b, here $2K \times 2K$). We simultaneously bijectively parameterize and shade the charts into their atlas boxes, obtaining high quality texture space shading (c), and use this shading to render the shaded frames (d).}
\label{fig:overview}
\label{fig:alg_overview}
\end{figure*}

\section{Overview}
\label{sec:overview}
Our work has two core contributions: a real-time, GPU-based algorithm for tight packing of general parameterized charts into compact atlases; and a real-time TSS method that
utilizes this packing.  

\paragraph*{FastAtlas Packing.}
FastAtlas runs entirely on the GPU as a series of compute shaders. It takes the bounding boxes of parameterized charts as input, and packs them into an atlas (Fig~\ref{fig:overview}b, Sec.~\ref{sec:pack}). As such, the only input it requires are the dimensions of the bounding boxes.
Its outputs are deterministic; identical input charts are packed into identical atlases. This is critical for TSS and similar applications, as it ensures that consecutive frames taken from the same camera view have the same shading. Even minute shading differences across such frames can cause sampling jitter, leading to undesirable flicker \cite{baker2012rock}. 
While prior methods such as \cite{mueller2018shading,hladky2019tessellated,hladky2021snakebinning,Neff2022MSA} cap the dimensions of the charts that can be packed as-is for a given atlas size, and scale down all charts that exceed these dimensions, we scale all charts by the same factor, and do so only when strictly necessary to achieve packing success (Figs~\ref{fig:atlas},~\ref{fig:sas_issues}). 

\paragraph*{TSS using FastAtlas.}
Our end-to-end TSS atlas generation method combines the packing method above with a novel approach for computing seamless per-frame charts. 
We define our charts as the connected components of the visible surfaces in each frame (Fig.~\ref{fig:overview}a), and efficiently compute them using a parallel union-find algorithm (Sec.~\ref{sec:visible}). Since the boundaries of these charts coincide with the contours of the rendered surface, they are {\em invisible} to the viewer. This approach 
eliminates the artifacts caused by shading discontinuities along visible seams (Fig.~\ref{fig:seams}). 

\begin{parWithWrapFigure}
\begin{wrapfigure}{l}{.27\columnwidth}%
\includegraphics[width=\linewidth]{fig_inset_view_plane.pdf}%
\end{wrapfigure}
We bijectively parametrize the {\em visible portions} of our charts by projecting them to view space (inset). This maps a constant number of texels to each pixel in the final rendered output, evenly distributing residual undersampling error across all image pixels. While conceptually straightforward, efficiently parameterizing charts containing partially visible triangles using viewspace projection is non-trivial, as the visible portions may no longer be triangular (e.g. green triangle in the inset); applying naive projection to triangles with vertices behind the camera may produce ill-posed results. Clipping triangles before projection is both computationally expensive and significantly complicates downstream operations. We avoid explicit clipping by observing that all that is required for atlas packing is the dimensions of, potentially conservative, bounding boxes of these projected visible portions. We compute such bounding boxes without explicit chart clipping by adapting a conservative screen coverage estimator \shortcite{Blinn:CalculatingScreenCoverage} (Sec.~\ref{sec:box}). We then pack the computed boxes using FastAtlas. 
\end{parWithWrapFigure}

Finally, we shade the visible portion of each chart into its corresponding atlas bounding box (Fig~\ref{fig:overview}c). 
The resulting texture is then used during rasterization as a standard texture map (Fig. ~\ref{fig:overview}d). 
Our framework is compatible with all existing approaches for texture space shading, including forward shading, raytraced illumination, or deferred shading in texture space \cite{baker:2016}. In the examples shown, we use the standard forward shading based rendering pipeline included in the G3D Innovation Engine \cite{G3D17}, a commercial grade renderer.


Our goal is to increase the robustness of T2I models, particularly with rare or unseen concepts, which they struggle to generate. To do so, we investigate a retrieval-augmented generation approach, through which we dynamically select images that can provide the model with missing visual cues. Importantly, we focus on models that were not trained for RAG, and show that existing image conditioning tools can be leveraged to support RAG post-hoc.
As depicted in \cref{fig:overview}, given a text prompt and a T2I generative model, we start by generating an image with the given prompt. Then, we query a VLM with the image, and ask it to decide if the image matches the prompt. If it does not, we aim to retrieve images representing the concepts that are missing from the image, and provide them as additional context to the model to guide it toward better alignment with the prompt.
In the following sections, we describe our method by answering key questions:
(1) How do we know which images to retrieve? 
(2) How can we retrieve the required images? 
and (3) How can we use the retrieved images for unknown concept generation?
By answering these questions, we achieve our goal of generating new concepts that the model struggles to generate on its own.

\vspace{-3pt}
\subsection{Which images to retrieve?}
The amount of images we can pass to a model is limited, hence we need to decide which images to pass as references to guide the generation of a base model. As T2I models are already capable of generating many concepts successfully, an efficient strategy would be passing only concepts they struggle to generate as references, and not all the concepts in a prompt.
To find the challenging concepts,
we utilize a VLM and apply a step-by-step method, as depicted in the bottom part of \cref{fig:overview}. First, we generate an initial image with a T2I model. Then, we provide the VLM with the initial prompt and image, and ask it if they match. If not, we ask the VLM to identify missing concepts and
focus on content and style, since these are easy to convey through visual cues.
As demonstrated in \cref{tab:ablations}, empirical experiments show that image retrieval from detailed image captions yields better results than retrieval from brief, generic concept descriptions.
Therefore, after identifying the missing concepts, we ask the VLM to suggest detailed image captions for images that describe each of the concepts. 

\vspace{-4pt}
\subsubsection{Error Handling}
\label{subsec:err_hand}

The VLM may sometimes fail to identify the missing concepts in an image, and will respond that it is ``unable to respond''. In these rare cases, we allow up to 3 query repetitions, while increasing the query temperature in each repetition. Increasing the temperature allows for more diverse responses by encouraging the model to sample less probable words.
In most cases, using our suggested step-by-step method yields better results than retrieving images directly from the given prompt (see 
\cref{subsec:ablations}).
However, if the VLM still fails to identify the missing concepts after multiple attempts, we fall back to retrieving images directly from the prompt, as it usually means the VLM does not know what is the meaning of the prompt.

The used prompts can be found in \cref{app:prompts}.
Next, we turn to retrieve images based on the acquired image captions.

\vspace{-3pt}
\subsection{How to retrieve the required images?}

Given $n$ image captions, our goal is to retrieve the images that are most similar to these captions from a dataset. 
To retrieve images matching a given image caption, we compare the caption to all the images in the dataset using a text-image similarity metric and retrieve the top $k$ most similar images.
Text-to-image retrieval is an active research field~\cite{radford2021learning, zhai2023sigmoid, ray2024cola, vendrowinquire}, where no single method is perfect.
Retrieval is especially hard when the dataset does not contain an exact match to the query \cite{biswas2024efficient} or when the task is fine-grained retrieval, that depends on subtle details~\cite{wei2022fine}.
Hence, a common retrieval workflow is to first retrieve image candidates using pre-computed embeddings, and then re-rank the retrieved candidates using a different, often more expensive but accurate, method \cite{vendrowinquire}.
Following this workflow, we experimented with cosine similarity over different embeddings, and with multiple re-ranking methods of reference candidates.
Although re-ranking sometimes yields better results compared to simply using cosine similarity between CLIP~\cite{radford2021learning} embeddings, the difference was not significant in most of our experiments. Therefore, for simplicity, we use cosine similarity between CLIP embeddings as our similarity metric (see \cref{tab:sim_metrics}, \cref{subsec:ablations} for more details about our experiments with different similarity metrics).

\vspace{-3pt}
\subsection{How to use the retrieved images?}
Putting it all together, after retrieving relevant images, all that is left to do is to use them as context so they are beneficial for the model.
We experimented with two types of models; models that are trained to receive images as input in addition to text and have ICL capabilities (e.g., OmniGen~\cite{xiao2024omnigen}), and T2I models augmented with an image encoder in post-training (e.g., SDXL~\cite{podellsdxl} with IP-adapter~\cite{ye2023ip}).
As the first model type has ICL capabilities, we can supply the retrieved images as examples that it can learn from, by adjusting the original prompt.
Although the second model type lacks true ICL capabilities, it offers image-based control functionalities, which we can leverage for applying RAG over it with our method.
Hence, for both model types, we augment the input prompt to contain a reference of the retrieved images as examples.
Formally, given a prompt $p$, $n$ concepts, and $k$ compatible images for each concept, we use the following template to create a new prompt:
``According to these examples of 
$\mathord{<}c_1\mathord{>:<}img_{1,1}\mathord{>}, ... , \mathord{<}img_{1,k}\mathord{>}, ... , \mathord{<}c_n\mathord{>:<}img_{n,1}\mathord{>}, ... , $
$\mathord{<}img_{n,k}\mathord{>}$,
generate $\mathord{<}p\mathord{>}$'', 
where $c_i$ for $i\in{[1,n]}$ is a compatible image caption of the image $\mathord{<}img_{i,j}\mathord{>},  j\in{[1,k]}$. 

This prompt allows models to learn missing concepts from the images, guiding them to generate the required result. 

\textbf{Personalized Generation}: 
For models that support multiple input images, we can apply our method for personalized generation as well, to generate rare concept combinations with personal concepts. In this case, we use one image for personal content, and 1+ other reference images for missing concepts. For example, given an image of a specific cat, we can generate diverse images of it, ranging from a mug featuring the cat to a lego of it or atypical situations like the cat writing code or teaching a classroom of dogs (\cref{fig:personalization}).
\vspace{-2pt}
\begin{figure}[htp]
  \centering
   \includegraphics[width=\linewidth]{Assets/personalization.pdf}
   \caption{\textbf{Personalized generation example.}
   \emph{ImageRAG} can work in parallel with personalization methods and enhance their capabilities. For example, although OmniGen can generate images of a subject based on an image, it struggles to generate some concepts. Using references retrieved by our method, it can generate the required result.
}
   \label{fig:personalization}\vspace{-10pt}
\end{figure}
\begin{table*}[t!]
\centering
% \vspace{5pt}
\begin{small}
\begin{tabular}{l|c|c|c|c|c|c|c}
\toprule
\textbf{Method} & \textbf{Type} & \textbf{ToMi} & \textbf{BigToM} & \textbf{MMToM-QA} & \textbf{MuMA-ToM} & \textbf{Hi-ToM} & \textbf{All} \\
\midrule
SymbolicToM & Specific & \textbf{98.60} & - &  - & - & - & - \\
TimeToM & Specific & 87.80 & - &   - & - & - & - \\
% \textbf{96.00$^*$}
PercepToM & Specific & 82.90 & - & - & - & - & - \\
BIP-ALM & Specific & - & - & 76.70 & 33.90 & - & - \\
LIMP & Specific & - & - & - & 76.60 & - & - \\
\ours w/ Model Spec. & Specific & 88.80 & \textbf{86.75} & \textbf{79.83} & \textbf{84.00} & \textbf{74.00} & \textbf{82.68} \\
\midrule
Llama 3.1 70B & General & 72.00 & 77.83 & 43.83 & 55.78 & 35.00 & 47.41 \\
Gemini 2.0 Flash & General & 66.70 & 82.00 & 48.00 & 55.33 & 52.50 & 60.91\\
Gemini 2.0 Pro & General & 71.90 & 86.33 & 50.84 &  62.22 & 57.50 & 65.76 \\ 
GPT-4o & General & 77.00 & 82.42 & 44.00 & 63.55 & 50.00 & 63.39 \\
SimToM & General & 79.90 & 77.50 & 51.00 & 47.63 & 71.00 & 65.41\\ 
\ours & General & \textbf{88.30} & \textbf{86.92} & \textbf{75.50} & \textbf{81.44} & \textbf{72.50} & \textbf{80.93} \\
\bottomrule
\end{tabular}
\end{small}
\caption{Results of \ours and baselines on all benchmarks. There are two groups of methods: methods that require domain-specific knowledge (e.g., AutoToM w/ Model Spec.) or implementations (e.g., SymbolicToM) and methods that can be generally applied to any domain. ``-'' indicates that the domain-specific method is not applicable to the benchmark. The best results for each method type are highlighted in bold.}
\label{tab:results}
\vspace{-10pt}
\end{table*}



\section{Experiments}
\subsection{Experimental Settings}



We evaluated our method on multiple Theory of Mind benchmarks, including ToMi \citep{le2019revisiting}, BigToM \citep{gandhi2024understanding}, MMToM-QA \cite{jin2024mmtom}, MuMA-ToM \citep{shi2024muma}, and Hi-ToM \cite{he2023hi}. The diversity and complexity of these benchmarks pose significant reasoning challenges. For instance, MMToM-QA and MuMA-ToM incorporate both visual and textual input, while MuMA-ToM and Hi-ToM require higher-order inference. Additionally, MMToM-QA features exceptionally long contexts, and BigToM presents open-ended scenarios.



Besides the full \ours method, we additionally evaluated \ours given manually specified models (AutoToM w/ Model Spec.). 

We compared \ours against state-of-the-art baselines:
    \textbf{LLMs:} Llama 3.1 70B \citep{dubey2024llama}, Gemini 2.0 Flash, Gemini 2.0 Pro \cite{team2023gemini} and GPT-4o \cite{achiam2023gpt};
    
     \textbf{ToM prompting for LLMs:} SymbolicToM \cite{sclar2023minding}, SimToM \cite{wilf2023think}, TimeToM \cite{hou2024timetom}, and PercepToM \citep{jung2024perceptions};
 
  \textbf{Model-based inference:} BIP-ALM \cite{jin2024mmtom} and LIMP \cite{shi2024muma}.


For multimodal benchmarks, MMToM-QA and MuMA-ToM, we adopt the information fusion methods proposed by \citet{jin2024mmtom} and \citet{shi2024muma} to fuse information from visual and text inputs respectively. The fused information is in text form. We ensure that all methods use the same fused information as their input.


We use GPT-4o as the LLM backend for \ours and all ToM prompting and model-based inference baselines to ensure a fair comparison—except for TimeToM, which relies on GPT-4 and is not open-sourced.


\subsection{Results}
The main results are summarized in Table~\ref{tab:results}. Unlike \ours, many recent ToM baselines can only be applied to specific benchmarks. Among general methods, \ours achieves state-of-the-art results across all benchmarks. In particular, it outperforms its LLM backend, GPT-4o, by a large margin. This is because Bayesian inverse planning is more robust for inferring mental states given long contexts with complex environments and agent behavior. It is also more adept at recursive reasoning which is key to higher-order inference. Notably, \ours performs comparably to manually specified models, showing that automatic model discovery without domain knowledge is as effective as human-provided models. We provide additional results and qualitative examples in Appendix~\ref{sec:more_results}.


\subsection{Ablated Study}



\begin{figure}[t!]
  \centering
  \includegraphics[width=0.8\linewidth]{figures/comparison.pdf}
    \vspace{-10pt}
  \caption{Averaged performance and compute of the full \ours method (star) and the ablated methods (circles) on all benchmarks.}
  \label{fig:ablation}
  \vspace{-10pt}
\end{figure}


We evaluated the following variants of \ours for an ablation study: no hypothesis reduction (\textbf{w/o hypo. reduction}); always using POMDP (\textbf{w/ POMDP}); always using the initial model proposal without variable adjustment (\textbf{w/o variable adj.}); only considering the last timestep (\textbf{w/ last timestep}); and considering all timesteps without timestep adjustment (\textbf{w/ all timesteps}).

The results in Figure~\ref{fig:ablation} show that the full \ours method constructs a suitable BToM model, enabling rich ToM inferences while reducing compute. We analyze key model components below:

\textbf{Hypothesis reduction.}
Compared to the full method, \ours w/o hypo. reduction has a similar accuracy but consumes 53\% more tokens on average, demonstrating that hypothesis reduction optimizes efficiency without sacrificing performance.

\textbf{Variable adjustment.}
\ours dynamically identifies relevant variables for ToM inference, generalizing domain-specific BIP approaches to open-ended scenarios. Compared to its variant without variable adjustment, \ours improves performance with minimal additional compute. The variant that always uses POMDP performs well in scenarios aligned with the POMDP assumption (e.g., MMToM-QA) but generalizes poorly elsewhere and incurs much higher computational costs. %, leading to an 8.5% performance deficit.

\textbf{Timestep adjustment.}
By selecting relevant steps for inference, timestep adjustment enhances performance by focusing on essential information. In contrast, the variant using only the last timestep misses crucial details, significantly lowering performance. The variant incorporating all timesteps suffers from higher computational costs and reduced accuracy due to conditioning on unnecessary, potentially distracting information.



Full ablation results are provided in Appendix~\ref{sec:more_results_ablation}.

% \section{A Comparative Study of Initial and Curated Code Review Datasets}
\label{sec:analysis}

\subsection{Impact of Curated Reviews on Comment Generation}
\label{subsec:model_data}
In this section, we investigate the impact of curated reviews on automating the comment generation process. By comparing models trained on both original and curated comments, we aim to assess whether the reformulated reviews lead to more efficient automation of the comment generation task.

\paragraph{\textbf{Model and data selection}}
To ensure a fair comparison, we selected a subset of $20,000$ comments from both the original and curated datasets, such that each original review comment \( r_i \) from the original dataset is paired with its reformulated counterpart \( r'_i \) in the curated dataset. This selection strategy guarantees that any observed differences in model performance can be attributed to the quality of the data (\ie curation process) rather than differences in review content or model hyperparameters. We further split each subset into $75\%$ for training and $25\%$ for evaluation.

We selected \textit{DeepSeek-6.7B-Instruct} \cite{deepseek-coder}, an LLM tailored for code-related tasks. 
Given a code change, the model was tasked with generating either the original or the curated comment. To ensure consistency, we trained two separate models, one for each dataset version, using identical configurations.


\paragraph{\textbf{Experimental setup}}

This experiment aims to determine whether curated review comments improve the ability of LLMs to generate accurate review comments. For each dataset version (original and curated), we provided the model with code changes as input and tasked it with generating the corresponding review comment.

Each model was trained independently using the same configuration to ensure that observed performance differences could be attributed solely to the dataset quality, not to model or hyperparameter variations. The training was conducted on four \emph{NVIDIA RTX A5000 GPUs}, each with \emph{24GB} of memory. We used a batch size of $4$ and trained each model for $5$ epochs. To enable efficient, low-resource fine-tuning, we employed Low-Rank Adaptation (LoRA) \cite{hu2021lora}, a parameter-efficient fine-tuning technique, configured with settings of $r = 16$, $\alpha = 32$, and $dropout = 0.05$. LoRA operates by decomposing the weight updates of a neural network into low-rank matrices, significantly reducing the number of parameters that require updating during fine-tuning \cite{hu2021lora}, thus enhancing the overall efficiency of the training process.
LoRA has been widely used in prior work to fine-tune LLMs for software engineering tasks~\cite{lu2023llama, weyssow2023exploring, hou2023large, silva2023repairllama}


To evaluate the two produced models’ performance, we used the BLEU score \cite{papineni2002bleu}, a standard metric, widely used in the literature, that measures the precision of n-grams in the generated text relative to the ground truth. BLEU is well-suited for assessing the correctness of generated comments, with higher scores indicating greater accuracy with real output.


\paragraph{\textbf{Results}}
The results are presented in \Table{tab:com_results}. For the model trained on the original dataset, we obtained a BLEU score of $7.71$, whereas the model trained on the curated dataset achieved a BLEU score of $11.26$. This improvement suggests that the reformulated, curated comments are likely easier for the model to learn, potentially due to their enhanced clarity and structure.

These findings suggest that curated review comments provide clearer, more direct guidance, enabling the model to better capture the intended message and improving the quality of generated comments. The higher BLEU score with the curated dataset indicates that the curation process enhances the ability of models to generalize and learn producing more accurate review comments, thus facilitating a more efficient automation of the comment generation process.

% \begin{table}[!t]
% \centering
% \caption{Comparison of BLEU scores for DeepSeek-Coder-6.7B-Instruct trained on original and curated comments.}
% \label{tab:com_results}
% \begin{tabular}{@{}lcc@{}}
% \toprule
% \textbf{Dataset Version}       & \textbf{BLEU} \\ 
% \midrule
% Original Comments               & $7.71$ \\
% Curated Comments                & \textbf{$11.26$} \\ 
% \bottomrule
% \end{tabular}
% \vspace{-1.5em}
% \end{table}

\begin{table}[!t]
\centering
\caption{Comparison of BLEU scores for DeepSeek-Coder-6.7B-Instruct trained on original and curated comments.}
\label{tab:com_results}
\begin{tabular}{@{}lcc@{}}
\toprule
& \textbf{Original Comments} & \textbf{Curated Comments} \\ 
\midrule
\textbf{BLEU} & $7.71$ & $\textbf{11.26}$ \\ 
\bottomrule
\end{tabular}
\vspace{-1em}
\end{table}



The code change, shown in \Table{tab:comm_gen_examples}, is an example picked from our test set. The real comment in the original dataset lacks clarity, as it does not provide the reason behind the question. The model trained on the original dataset generates an incorrect review comment. In the curated dataset, the reformulated version of the real review comment is clearer, though less concise. The model trained on the curated dataset generates a comment that is accurate, closely matching the real and reformulated comments, and is even more concise than the reformulated version.



\begin{table*}[!htbp]
\centering
\caption{Example of real, reformulated, and generated review comments from the initial and curated datasets.}
\label{tab:comm_gen_examples}
\begin{tabularx}{1\linewidth}{X}
\toprule

\multicolumn{1}{c}{\textbf{Code change}} \\\midrule
\makebox[\linewidth]{\includegraphics[width=0.6\linewidth]{figures/code_snippet3.png}} 
\\\midrule
\multicolumn{1}{c}{\textbf{Original dataset}}\\\midrule

\underline{\textbf{Real review comment}} \texttt{why this import is required ?} \\

\underline{\textbf{Generated review comment}} \texttt{I think we should use `javax.servlet.http.Part`instead of \newline`org.apache.commons.fileupload.FileItem`.} \\ \midrule

\multicolumn{1}{c}{\textbf{Curated dataset}} \\\midrule

\underline{\textbf{Reformulated review comment}} \texttt{This import is not immediately clear in its necessity. Could you provide context\newline or explain why `javax.servlet.http.Part` is required for this functionality?} \\

\underline{\textbf{Generated review comment}} \texttt{Consider removing the import statement for `javax.servlet.http.Part` as it's not\newline used in the code.} \\ \bottomrule

\end{tabularx}
\vspace{-.5em}
\end{table*}





\begin{center}
\begin{tcolorbox}[colframe=orange!75!black, colback=orange!5!white, title=\textbf{Answer to RQ3}]
Curated review comments are more effective for training models on the comment generation task, as evidenced by higher BLEU ($7.71 \rightarrow 11.26$).
\end{tcolorbox}
\end{center}


\subsection{A Comparative Analysis on the Usefulness of Curated Comments for Code Refinement}

In this section, we aim to evaluate the usefulness of the curated comments compared to the original comments for code refinement.
We conduct a comparative study to assess which version of the comment—original or curated—guides the code refinement model to generate more accurate code changes.


\paragraph{\textbf{Model and data selection}}
% We select \textit{DeepSeek-Coder-6.7B-Instruct} \cite{deepseek-coder} to serve as our code refinement model. For evaluation, we randomly select a subset of our large datasets, choosing 20,000 samples from the full set of 176,613 for each version of the dataset (\ie original and curated datasets). To ensure a fair comparison, we maintain consistency in the selection by pairing each original review comment \( r_i \) from the original dataset with its corresponding reformulated comment \( r'_i \) in the cated dataset, where \( i \in [1, 20,000] \). Thus, each review pair \( (r_i, r'_i) \) represents the same code review context, allowing us to directly evaluate the impact of the reformulated comments compared to the original. For consistency, we use the same \textit{DeepSeek-Coder-6.7B-Instruct} model configuration on both datasets to ensure that any observed differences are attributable to the quality of the input data (\ie review comments) rather than model variations. 

We use \textit{DeepSeek-Coder-6.7B-Instruct} \cite{deepseek-coder} as a code refinement model, applying it to the same selected subset of $20,000$ samples from each dataset version (original and curated), as in the previous experiment on comment generation, as explained in \Sect{subsec:model_data}. To ensure a fair comparison, each original review comment \( r_i \) is paired with its reformulated counterpart \( r'_i \), preserving the same review context across both datasets. The model configurations remain identical for both datasets, to ensure that any observed differences are attributable to the quality of the input data (\ie review comments) rather than model variations. 




\paragraph{\textbf{Experimental setup}}
For each dataset version, we provided the code refinement model with the original code diff, the old file, and the review comment (either original or curated) as context and prompted it to generate a code diff that accurately implements the specified changes.

We used the LLM directly for inference, as its extensive training on diverse code-related tasks equips it with the capabilities needed to effectively automate the code refinement task. The experiment was run twice, once with the original comments and once with the curated comments.

To evaluate the accuracy of the generated code diffs, we employed two evaluation metrics: 
\begin{itemize} 
    \item \textbf{CodeBLEU}: This metric measures the similarity of the generated code diff to the expected code diff, combining n-gram match, weighted n-gram match, AST match, and data-flow match scores \cite{ren2020codebleu}. 
    \item \textbf{Exact Match (EM)}: This metric calculates the number of generated code diffs that exactly match the expected code diff. 
\end{itemize}

Each experiment was conducted using identical model configurations for both dataset versions to ensure that any observed performance differences could be attributed solely to the quality of the review comments rather than model parameter variations.



\paragraph{\textbf{Results}}
The results, presented in \Table{tab:ref_results}, reveal substantial differences between the two dataset versions. Using the \textbf{original comments}, the model achieved a \emph{CodeBLEU} score of $0.36$ and an \emph{EM} of $408$. When utilizing the \textbf{curated comments}, the model performance improved significantly, reaching a \emph{CodeBLEU} score of $0.44$ and an \emph{EM} of $445$. These findings suggest that the curated comments offer more precise guidance, enabling the model to generate more accurate code changes that are closer to the ground truth.

This result indicates that the curated comments not only clarify the intended modifications but also reduce ambiguities in the model interpretation of the review instructions. The curated comments likely contain enhanced phrasing and structure that aid the model in better understanding and implementing the required code changes, thus improving the overall quality of the generated code diff.

\begin{table}[h!]
\centering
\caption{Comparison of DeepSeek-Coder-6.7B-Instruct's effectiveness for code refinement using the original and curated review comments.}
\label{tab:ref_results}
\begin{tabular}{@{}lcc@{}}
\toprule
\textbf{Dataset Version}       & \textbf{CodeBLEU} & \textbf{Exact Match} \\ \midrule
Original Comments & $0.36$ & $408$\\
Curated Comments  & $\textbf{0.44}$ & $\textbf{445}$\\ 
\bottomrule
\end{tabular}
\vspace{-.5em}
\end{table}



This experiment showed that curated review comments are more useful for the next task, as they provide better guidance for the code refinement task, leading to more accurate code changes. This is reflected by significant improvements in both \emph{CodeBLEU} and \emph{EM}.




\begin{center}
\begin{tcolorbox}[colframe=black!75!black, colback=black!5!white, title=\textbf{Answer to RQ4}]
    Curated comments demonstrate superior utility for guiding code refinement models, leading to notably higher CodeBLEU ($0.36 \rightarrow 0.44$) and Exact Match scores ($408 \rightarrow 445$).
\end{tcolorbox}
\end{center}







\begin{comment}
    ================================================== 1 ==================================================


 INIT - Original comment:  This and the one below. Please, rename to `data range` also.
 INIT - Generated comment  I think this should be "dynamic range"

+=+=+=+=+=+=+=+=+=+=+=+=+=+=+=+=+=+=+=+=+=+=+=+=+=+=+=+=+=+=+=+=+=+=+=+=+=+=+=+=+=+=+=+=+=+=+=+=+=+=
 CUR - Reformulated comment   Consider renaming "value range" to "data range" for consistency.
 CUR - Generated comment  Consider renaming the parameter to `value_range` for clarity and consistency.



 ================================================== 27 ==================================================

@@ -2791,81 +2791,80 @@ static void Main()
                 Diagnostic(ErrorCode.WRN_UnreferencedVarAssg, "message").WithArguments("message").WithLocation(8, 22));
         \}

-        [WorkItem(863401, "DevDiv/Personal")]
         [Fact]
-        public void CS1101ERR_BadRefWithThis()
+        public void BadParameterModifeirs_ThisWithRef()


+=+=+=+=+=+=+=+=+=+=+=+=+=+=+=+=+=+=+=+=+=+=+=+=+=+=+=+=+=+=+=+=+=+=+=+=+=+=+=+=+=+=+=+=+=+=+=+=+=+=


 INIT - Original comment:  Typo: `Modifiers` #Resolved
 INIT - Generated comment  ```suggestion public void BadParameterModifiers_ThisWithRef() ```

+=+=+=+=+=+=+=+=+=+=+=+=+=+=+=+=+=+=+=+=+=+=+=+=+=+=+=+=+=+=+=+=+=+=+=+=+=+=+=+=+=+=+=+=+=+=+=+=+=+=
 CUR - Reformulated comment   The method name `BadParameterModifeirs_ThisWithRef` contains a typo. Consider renaming it to `BadParameterModifiers_ThisWithRef` for consistency.
 CUR - Generated comment  Consider renaming the test method to `BadParameterModifiers_ThisWithRef` for consistency and clarity.



 ================================================== 6 ==================================================


 INIT - Original comment:  It should be without underscore, I think
 INIT - Generated comment  I think we should rename this to `IndexOf`.

+=+=+=+=+=+=+=+=+=+=+=+=+=+=+=+=+=+=+=+=+=+=+=+=+=+=+=+=+=+=+=+=+=+=+=+=+=+=+=+=+=+=+=+=+=+=+=+=+=+=
 CUR - Reformulated comment   Consider renaming the method to remove the underscore for consistency.
 CUR - Generated comment  Consider renaming the method to `IndexOfNonVectorized` for consistency and clarity.



================================================== 4845 ==================================================

@@ -75,6 +75,8 @@

 import com.google.common.annotations.VisibleForTesting;

+import javax.servlet.http.Part;


+=+=+=+=+=+=+=+=+=+=+=+=+=+=+=+=+=+=+=+=+=+=+=+=+=+=+=+=+=+=+=+=+=+=+=+=+=+=+=+=+=+=+=+=+=+=+=+=+=+=


 INIT - Original comment:  why this import is required ?
 INIT - Generated comment  I think we should use `javax.servlet.http.Part` instead of `org.apache.commons.fileupload.FileItem`.

+=+=+=+=+=+=+=+=+=+=+=+=+=+=+=+=+=+=+=+=+=+=+=+=+=+=+=+=+=+=+=+=+=+=+=+=+=+=+=+=+=+=+=+=+=+=+=+=+=+=
 CUR - Reformulated comment   "This import is not immediately clear in its necessity. Could you provide context or explain why `javax.servlet.http.Part` is required for this functionality?"
 CUR - Generated comment  Consider removing the import statement for `javax.servlet.http.Part` as it's not used in the code.


================================================== 27 ==================================================

@@ -2791,81 +2791,80 @@ static void Main()
                 Diagnostic(ErrorCode.WRN_UnreferencedVarAssg, "message").WithArguments("message").WithLocation(8, 22));
         \}

-        [WorkItem(863401, "DevDiv/Personal")]
         [Fact]
-        public void CS1101ERR_BadRefWithThis()
+        public void BadParameterModifeirs_ThisWithRef()


+=+=+=+=+=+=+=+=+=+=+=+=+=+=+=+=+=+=+=+=+=+=+=+=+=+=+=+=+=+=+=+=+=+=+=+=+=+=+=+=+=+=+=+=+=+=+=+=+=+=


 INIT - Original comment:  Typo: `Modifiers` #Resolved
 INIT - Generated comment  ```suggestion public void BadParameterModifiers_ThisWithRef() ```

+=+=+=+=+=+=+=+=+=+=+=+=+=+=+=+=+=+=+=+=+=+=+=+=+=+=+=+=+=+=+=+=+=+=+=+=+=+=+=+=+=+=+=+=+=+=+=+=+=+=
 CUR - Reformulated comment   The method name `BadParameterModifeirs_ThisWithRef` contains a typo. Consider renaming it to `BadParameterModifiers_ThisWithRef` for consistency.
 CUR - Generated comment  Consider renaming the test method to `BadParameterModifiers_ThisWithRef` for consistency and clarity.

================================================== 59 ==================================================

@@ -7,6 +7,7 @@
 to leave out days on which there is no data, i.e. weekends.  The example
 below shows how to use an 'index formatter' to achieve the desired plot
 """
+from __future__ import print_function


+=+=+=+=+=+=+=+=+=+=+=+=+=+=+=+=+=+=+=+=+=+=+=+=+=+=+=+=+=+=+=+=+=+=+=+=+=+=+=+=+=+=+=+=+=+=+=+=+=+=


 INIT - Original comment:  This should not be added now since we're Python 3 only on master.
 INIT - Generated comment  ```suggestion from __future__ import print_function, division ```

+=+=+=+=+=+=+=+=+=+=+=+=+=+=+=+=+=+=+=+=+=+=+=+=+=+=+=+=+=+=+=+=+=+=+=+=+=+=+=+=+=+=+=+=+=+=+=+=+=+=
 CUR - Reformulated comment   "This import is unnecessary since we're only supporting Python 3 on master. Consider removing it."
 CUR - Generated comment  Consider using `print` instead of `print_function` for compatibility with Python 2.



 
\end{comment}







\section{Conclusion}
We introduced \methodname, an effective training framework defending against MIAs for LLMs. The extensive experiments demonstrate its robustness in protecting privacy while maintaining strong language modeling performance across various datasets and architectures. Although our study focuses on fine-tuning due to computational constraints, \methodname can be seamlessly applied to large-scale pretraining, as done in prior selective pretraining work~\cite{lin2024not}. By categorizing tokens and treating them appropriately, \methodname opens a novel pathway for MIA defense. Future work can explore improved token selection strategies and multi-objective training approaches.

%\section*{Acknowledgments}
%...


% Bibliography entries for the entire Anthology, followed by custom entries
%\bibliography{anthology,custom}
% Custom bibliography entries only
\bibliography{paper}

\appendix

% \section{Example Appendix}
\label{sec:appendix}

%\onecolumn
\section{More Results}
\label{sec:more_results}

\subsection{Qualitative Results}
% an example of one iteration that improves the model utility

% BigToM example 
% AutoToM - bbfb question 78: needs sobag and initial proposal is sob --> sobag 
% Question 78': "Initial model propose: ['State', 'Observation', 'Belief']\nAssigned models: {0: ['State', 'Observation', 'Belief', 'Action', 'Goal']}"
Among general methods, AutoToM achieves state-of-the-art results across all benchmarks. We provide two qualitative examples to illustrate the effect of variable adjustment (example 1) and timestep adjustment (example 2). These examples also demonstrate the interpretability of \ours, as the constructed model offers us insights into how the method is modeling the agent behavior for the inference. 

\textbf{Example 1: BigToM (Backward Belief Inference)}

\begin{tcolorbox}[
    breakable,
    colframe=gray!40!black,
    colback=gray!5,
    coltitle=white,
    fonttitle=\bfseries,
    colbacktitle=gray!40!black
]
\textbf{Story:} Kavya is a florist in a vibrant Indian market. Kavya wants to create a beautiful bouquet of fresh roses for a customer's anniversary celebration. Kavya sees a batch of roses in her shop that appear to be fresh and vibrant. Unbeknownst to her a mischievous monkey sneaks into the shop and nibbles on the rose petals leaving them damaged and unsuitable for the bouquet. Kavya starts arranging the bouquet using the roses she initially saw. \newline

\textbf{Question: Does Kavya believe the roses are fresh and perfect for the bouquet or damaged by the monkey?}

(a) Kavya believes the roses are fresh and perfect for the bouquet. \textbf{\textcolor[RGB]{110, 170, 110}{(Correct Answer)}}

(b) Kavya believes the roses are damaged by the monkey.
\end{tcolorbox}



\textbf{Variables in the Initial Model Proposal: } State, Observation, Belief

Since the scenario involves only one timestep, a single model suffices. In the initial model, the state of the world indicates that the flowers are damaged after the monkey nibbles on them. However, it remains unclear whether Kavya observes the true condition of the flowers. The model lacks crucial information about Kavya's actions, which are observable and influenced by her beliefs about the flowers' state. These actions can help infer her true belief. Initially, the probability that Kavya believes the flowers are fresh is moderate, $P(\text{Kavya believes the roses are fresh and perfect}$ $\text{for the bouquet} | X^1) = 0.50$. Without variable adjustment, the model cannot answer the question.

\textbf{Variables in the Adjusted Model: } State, Observation, Belief, Action, Goal 

For the initial model, the reward is $R(M,q)=-H(P(q | X^{t_s:t}))=-0.693$ and the model cost is $C(M)=\alpha|M|=0.04$, resulting in a utility $U(M, q)=-0.733$, which does not exceed the utility threshold $U_\text{min}=-0.693$. To address the insufficiency of the initial model's utility relative to our termination threshold, we propose an enhanced model incorporating state, observation, belief, action, and goal. In this revised model, Kavya’s actions—specifically arranging the bouquet using the roses—align with her goal of creating a beautiful bouquet. These observations allow us to infer with high probability that Kavya believes the roses are fresh and suitable for the bouquet, increasing the belief probability to $P(\text{Kavya believes the roses are fresh and perfect}$ $ \text{for the bouquet}| X^1) = 0.97$. With this revised model, the reward is $R(M,q)=-H(P(q | X^{t_s:t}))=-0.135$ and the model cost is $C(M)=\alpha|M|=0.06$, resulting in a utility $U(M, q)=-0.195$, which exceeds our utility threshold $U_{\text{min}}=-0.693$. Based on the adjusted model, \ours can confidently determine the correct answer: (a) Kavya believes the roses are fresh and perfect for the bouquet.\newline


\textbf{Example 2: MMToM-QA (Belief Inference)}

%\textbf{Text input}
\begin{tcolorbox}[
    breakable,
    colframe=gray!40!black,
    colback=gray!5,
    coltitle=white,
    fonttitle=\bfseries,
    colbacktitle=gray!40!black
]

\textbf{Video input:}

\includegraphics[width=\linewidth]{figures/MMToM-frame.png}
\vspace{-3pt}
  
\textbf{What's inside the apartment:} The apartment consists of a bedroom, kitchen, living room, and bathroom. 
In the bedroom, there is a coffee table and a desk. 
The kitchen is equipped with four cabinets, a fridge, a kitchen table, a microwave, and a stove. The 3rd kitchen cabinet from the left houses a water glass and a dish bowl. Inside the fridge, there are two apples, a salmon, a plate, and a dish bowl. The 2nd kitchen cabinet from the left contains a water glass, a chips, a condiment bottle, and a dish bowl. The 1st kitchen cabinet from the left holds a wineglass, a wine, and a condiment bottle. The microwave contains a salmon, and there is a cupcake in the stove. The 4th kitchen cabinet from the left has a plate. 
The living room features a cabinet, a sofa, a coffee table, and a desk. Inside the cabinet, there are two apples and four books. A plate and a remote control are placed on the coffee table. 
The bathroom is furnished with a bathroom cabinet, which is currently empty. \newline

\textbf{Actions taken by Mark:} Mark is situated in the bathroom. He proceeds towards the kitchen, making his way to the stove. He opens and then closes the stove. Subsequently, he strides towards the 4th kitchen cabinet, opens it, and then shuts it. He then moves to the 2nd kitchen cabinet, opens and closes it, before doing the same with the 3rd kitchen cabinet. Finally, he heads towards the 1st kitchen cabinet, opens and closes it, and is about to open the microwave. \newline

\textbf{Question: If Mark has been trying to get a salmon, which one of the following statements is more likely to be true?}

(a) Mark thinks that the salmon is not inside the microwave.

(b) Mark thinks that the salmon is inside the microwave. \textbf{\textcolor[RGB]{110, 170, 110}{(Correct Answer)}}
\end{tcolorbox}




In this problem, we first fuse the information from text and video following \citet{jin2024mmtom}. The fused information is structured into 23 timesteps, each corresponding to an action of Mark at the time. We then propose the initial model: State, Observation, Belief, Action, Goal.

\textbf{Without timestep adjustment.} Bayesian inference must be performed sequentially from the first timestep, even though most actions do not contribute to answering the final question. The model will compute across all timesteps, while the most informative action is actually the last one: if Mark wants to get a salmon but does not believe there is one inside the microwave, he will not open it.

\textbf{With timestep adjustment.} We begin inference from the last timestep, 
where the action likelihood $P(a|b, g)$ is low when $b=$ \textit{Mark thinks that the salmon is not inside the microwave}, and high when $b=$ \textit{Mark thinks that the salmon is inside the microwave}. After performing inference at the last timestep, the belief probabilities corresponding to the choices are $0.998$ and $0.002$. The reward is given by $R(M,q)=-H(P(q | X^{t_s:t}))=-0.014$, while the model cost is $C(M)=\alpha|M|=0.06$. This results in a utility of $U(M, q)=-0.074$, which exceeds the threshold $U_{\text{min}}=-0.693$, allowing our model to determine the final answer without considering earlier timesteps.

\begin{figure}[t!]
  \centering
  \includegraphics[width=1.0\linewidth]{figures/higher_order.pdf}
    % \vspace{-15pt}
  \caption{Comparison of accuracy between \ours and GPT-4o on the HiToM dataset across different reasoning orders. Order 0 refers to questions about an object's actual location; order 1 questions are about an agent's belief about an object's location; order 2 involves questions about an agent's belief regarding another agent's belief, and so forth.}
  \label{fig:higher_order}
  % \vspace{-10pt}
\end{figure}

\subsection{Results for Higher Order Inference}
% accuracy table for orders in hi-tom

Higher-order Theory of Mind (ToM) involves recursive reasoning about others' mental states across multiple levels. The Hi-ToM benchmark \cite{he2023hi} includes questions ranging from Order 0, which involves no agents and asks about the actual location of objects, up to Order 4, which requires recursive reasoning among four agents. Figure \ref{fig:higher_order} compares the performance of GPT-4o and \ours across these different question orders. While GPT-4o experiences a significant decline in accuracy as the ToM order increases, \ours maintains a smaller performance drop and achieves substantially higher accuracy on higher-order questions. This demonstrates that our model-based approach is more robust and scalable, effectively handling complex scenarios involving multiple agents and various levels of recursive reasoning.



\subsection{Full Results of the Ablation Study}\label{sec:more_results_ablation}

Table~\ref{tab:results_ablation_accuracy} shows the performance of ablated methods compared to the full \ours method on all benchmarks.

\begin{table*}[t!]
\centering
% \vspace{5pt}
\begin{small}
\begin{tabular}{c|c|c|c|c|c|c}
\toprule
\textbf{Method} & \textbf{ToMi} & \textbf{BigToM} & \textbf{MMToM-QA} & \textbf{MuMA-ToM} & \textbf{Hi-ToM} &\textbf{All} \\
\midrule
w/o hypo. reduction & 87.60 & 86.17 & 75.83 & 81.67 & 69.50 & 80.15 \\
w/ POMDP & 76.00 & 86.50 & 79.83 & 50.78 & 67.00 & 72.02 \\ 
%\ours w/o model discovery \\
w/o variable adj. & 85.80 & 78.25 & 76.17 & 77.89 & 66.50 & 76.92 \\
% \ours w/o timestep adjustment \\
w/ last timestep & 68.40 & 77.83 & 74.33 & 78.33 & 44.50 & 68.68 \\
w/ all timesteps & 86.00 & 79.09 & 76.50 & 79.33 & 69.00 & 77.98 \\
\midrule
%\ours w/ Model Input & 86.72 & 72.73 & 80.17 & 84.00 & & \textbf{80.91} \\
\ours & 88.30 & 86.92 & 75.50 & 81.44 & 72.50 & 80.93 \\
\bottomrule
\end{tabular}
\end{small}
\caption{Results of ablated methods compared to the full \ours method.}
\label{tab:results_ablation_accuracy}
\end{table*}


In Table \ref{tab:results_ablation_tokens} and \ref{tab:results_ablation_api}, we compare the ablated methods and the full model on the averaged number of tokens per question (in thousands) and the averaged number of API calls at inference per question.



\begin{table*}[t!]
\centering
% \vspace{5pt}
\begin{small}
\begin{tabular}{c|c|c|c|c|c|c}
\toprule
\textbf{Method} & \textbf{ToMi} & \textbf{BigToM} & \textbf{MMToM-QA} & \textbf{MuMA-ToM} & \textbf{Hi-ToM} &\textbf{All} \\
\midrule
w/o hypo. reduction & 15.8 & 6.8 & 19.2 & 24.4 & 20.4 & 17.3 \\
 w/ POMDP & 14.9 & 5.5 & 15.6 & 20.0 & 18.8 & 15.0 \\ 
%\ours w/o model discovery \\
w/o variable adj. & 8.5 & 6.1 & 16.4 & 14.0 & 10.0 & 11.0 \\
% \ours w/o timestep adjustment \\
w/ last timestep & 7.8 & 6.1 & 6.4 & 11.6 & 4.0 & 7.2 \\
w/ all timesteps & 14.2 & 7.7 & 57.2 & 16.4 & 12.4 & 21.6 \\ 
\midrule
%\ours w/ Model Input & 86.72 & 72.73 & 80.17 & 84.00 & & \textbf{80.91} \\
\ours & 9.8 & 6.5 & 14.4 & 13.6 & 12.0 & 11.3 \\
\bottomrule
\end{tabular}
\end{small}
\caption{Comparison of ablated models and the full model on the averaged number of tokens per question (in thousands). Lower is better.}
\label{tab:results_ablation_tokens}
\end{table*}

\begin{table*}[t!]
\centering
% \vspace{5pt}
\begin{small}
\begin{tabular}{c|c|c|c|c|c|c}
\toprule
\textbf{Method} & \textbf{ToMi} & \textbf{BigToM} & \textbf{MMToM-QA} & \textbf{MuMA-ToM} & \textbf{Hi-ToM} &\textbf{All} \\
\midrule
w/o hypo. reduction & 38.91 & 13.99 & 45.97 & 70.73 & 72.58 & 48.44 \\
 w/ POMDP & 36.25 & 8.32 & 41.18 & 42.10 & 51.73 & 35.92 \\ 
%\ours w/o model discovery \\
w/o variable adj. & 22.91 & 12.99 & 35.46 & 35.76 & 29.81 & 27.39 \\
% \ours w/o timestep adjustment \\
w/ last timestep &  21.60 & 12.76 & 12.75 & 28.39 & 9.39 & 16.98 \\
w/ all timesteps & 39.83 & 15.95 & 116.81 & 43.25 & 36.27 & 50.42 \\ 
\midrule
%\ours w/ Model Input & 86.72 & 72.73 & 80.17 & 84.00 & & \textbf{80.91} \\
\ours & 32.23 & 13.81 & 31.36 & 35.08 & 36.45 & 29.79 \\
\bottomrule
\end{tabular}
\end{small}
\caption{Comparison of ablated models and the full model on the averaged number of API calls at inference per question. Lower is better.}
\label{tab:results_ablation_api}
\end{table*}

\subsection{Per-type Accuracy on All Benchmarks}

In Tables~\ref{tab:per_type_acc_tomi} - \ref{tab:per_type_acc_hitom}, we present the results of \ours and baselines on each question type of all benchmarks. Here we compare general methods that can be applied to all benchmarks.


% tomi 
\begin{table*}[t!]
\centering
% \vspace{5pt}
\begin{small}
\begin{tabular}{c|c|c|c|c|c}
\toprule
\textbf{Question Type} & \textbf{First order} & \textbf{Second order} & \textbf{Reality} & \textbf{Memory} & \textbf{All} \\
\midrule
Llama 3.1 70B & 73.75 & 56.25 & 100.00 & 100.00 & 72.00 \\
% Gemini 1.5 Flash & 53.50 & 58.75 & 100.00 & 100.00 & 64.90 \\
Gemini 2.0 Flash & 58.50 & 58.25 & 100.00 & 100.00 & 66.70 \\
% Gemini 2.0 Flash Thinking & 83.25 & 61.75 & 100.00 & 100.00 & 78.00 \\
Gemini 2.0 Pro & 75.00 & 54.75 & 100.00 & 100.00 & 71.90 \\ 
GPT-4o & 80.25 & 62.25 & 100.00 & 100.00 & 77.00 \\ 
% GPT-o3-mini & 79.50 & 53.25 & 100.00 & 100.00 & 73.10\\ 
SimToM & 84.75 & 65.00 & 100.00 & 100.00 & 79.90 \\ 
\ours & {95.00} & {77.50} & 93.00 & 100.00 & 88.30 \\
\bottomrule
\end{tabular}
\end{small}
\caption{Detailed accuracy for ToMi.}
\label{tab:per_type_acc_tomi}
\end{table*}


% bigtom 
\begin{table*}[t!]
\centering
% \vspace{5pt}
\begin{small}
\begin{tabular}{c|c|c|c|c|c}
\toprule
\textbf{Question Type} & \textbf{Forward TB} & \textbf{Forward FB}  & \textbf{Backward TB} & \textbf{Backward FB} & \textbf{All}\\
\midrule
Llama 3.1 70B & 93.75 & 81.00 & 57.00 & 60.50 & 77.83\\
% Gemini 1.5 Flash & {96.50} & 61.50 & {79.00} & 57.50 & 75.42\\
Gemini 2.0 Flash & 94.25 & 87.50 & 77.50 & 51.00 & 82.00\\
% Gemini 2.0 Flash Thinking & 94.75 & 91.50 & 77.50 & 47.00 & 82.83 \\
Gemini 2.0 Pro & 96.00 & 93.75 & 70.00 & 68.50 & 86.33\\ 
GPT-4o & 96.00 & 88.50 & 63.50 & 62.00 & 82.42\\
% GPT-o3-mini & 93.25 & 90.75 & 78.50 & 75.00 & 86.92\\ 
SimToM & 92.50  & 90.00 & 25.00 & 75.00 & 77.50 \\
% TimeToM & 95.00 & {99.00} & - & - & -\\
\ours & 91.25 & {93.75} & 73.00 & {78.50} & {86.92} \\
\bottomrule
\end{tabular}
\end{small}
\caption{Detailed accuracy for BigToM.}
\label{tab:per_type_acc_bigtom}
\end{table*}

% mmtom
\begin{table*}[t!]
\centering
% \vspace{5pt}
\begin{small}
\begin{tabular}{c|c|c|c}
\toprule
\textbf{Question Type} & \textbf{Belief} & \textbf{Goal} &\textbf{All} \\
\midrule
Llama 3.1 70B & 51.33 & 36.33 & 43.83\\
% Gemini 1.5 Flash & 39.00 & 28.00 & 33.50 \\
Gemini 2.0 Flash & 62.67 & 33.33 & 48.00\\
% Gemini 2.0 Flash Thinking & 73.33 & 34.67 & 54.00 \\
Gemini 2.0 Pro & 57.00 & 44.67 & 50.84\\ 
GPT-4o & 55.67 & 32.33 & 44.00 \\
% GPT-o3-mini & 88.67 & 40.67 & 64.67\\ 
SimToM & 75.67 & 26.33 & 51.00 \\
% SimToM & 64.30 & 40.70 & 52.50 \\
\ours & {88.67} & {62.33} & {75.50} \\
\bottomrule
\end{tabular}
\end{small}
\caption{Detailed accuracy for MMToM-QA.}
\label{tab:per_type_acc_mmtom}
\end{table*}


% mumatom
\begin{table*}[t!]
\centering
% \vspace{5pt}
\begin{small}
\begin{tabular}{c|c|c|c|c}
\toprule
\textbf{Question Type} & \textbf{Belief} & \textbf{Goal} & \textbf{Belief of Goal} & \textbf{All} \\
\midrule
Llama 3.1 70B & 68.67 & 51.33 & 47.33 &  55.78 \\
% Gemini 1.5 Flash & 70.67 & 39.34 & 47.67 & 52.56\\
Gemini 2.0 Flash & 68.33 & 50.67 & 47.00 & 55.33\\
% Gemini 2.0 Flash Thinking & 95.33 & 79.00 & 73.33 & 82.56\\
Gemini 2.0 Pro & 63.00 & 66.67 & 57.00 & 62.22 \\ 
GPT-4o & 85.33 & 57.00 & 48.33 & 63.55 \\
% GPT-o3-mini & 74.00 & 67.67 & 68.33 & 70.00 \\ 
SimToM & 54.60 & 43.50 & 44.80 & 47.63 \\
\ours & 88.33 & 77.00 & 79.00 & 81.44 \\
\bottomrule
\end{tabular}
\end{small}
\caption{Detailed accuracy for MuMA-ToM.}
\label{tab:per_type_acc_mumatom}
\end{table*}




% hitom
\begin{table*}[t!]
\centering
% \vspace{5pt}
\begin{small}
\begin{tabular}{c|c|c|c|c|c|c}
\toprule
\textbf{Question Type} & \textbf{Order 0} & \textbf{Order 1} & \textbf{Order 2} & \textbf{Order 3} & \textbf{Order 4} & \textbf{All} \\
\midrule
Llama 3.1 70B & 65.00 & 47.50 & 22.50 & 20.00 & 20.00 & 35.00 \\  
% Gemini 1.5 Flash & 87.50 & 65.00 & 37.50 & 25.00 & 20.00 & 47.00\\
Gemini 2.0 Flash & 95.00 & 70.00 & 50.00 & 27.50 & 20.00 & 52.50\\
% Gemini 2.0 Flash Thinking & 100.00 & 85.00 & 72.50 & 50.00 & 60.00 & 65.50\\ 
Gemini 2.0 Pro & 100.00 & 62.50 & 50.00 & 37.50 & 37.50 & 57.50 \\ 
GPT-4o & 92.50 & 65.00 & 40.00 & 27.50 & 25.00 & 50.00 \\
% GPT-o3-mini & 100.00 & 72.50 & 65.00 & 60.00 & 77.50 & 75.00 \\
% Llama 3.1 70B & 95.00 & 60.00 & 30.00 & 30.00 & 35.00 & 50.00 \\
% Gemini 1.5 Flash & 95.00 & 80.00 & 45.00 & 25.00 & 15.00 & 52.00\\
% GPT-4o & {97.50} & 72.50 & 45.00 & 32.50 & 32.50 & 56.00 \\
SimToM & 100 & 77.50 & 60.00 &  60.00 & 57.50 & 71.00 \\
\ours & 95.00 & {75.00} & 70.00 & 67.50 & {55.00} & {72.50} \\
\bottomrule
\end{tabular}
\end{small}
\caption{Detailed accuracy for HiToM.}
\label{tab:per_type_acc_hitom}
\end{table*}
\section{\ours Implementation Details}
\label{sec:method_details}

\subsection{Variable Adjustments}\label{sec:app_var_adj}

Table~\ref{tab:model_adjustment} summarizes possible variable adjustments at each timestep.

\begin{table}[t!]
\centering
% \vspace{5pt}
\begin{small}
\begin{tabular}{p{1.2cm}cc}
\toprule
New Var. & Before & After \\
\midrule
\multirow{4}{*}{Goal}
& $P(a^t \mid s^t)$ & $P(a^t \mid s^t, g) P(g)$ \\
& $P(a^t \mid b^t)$ & $P(a^t \mid b^t, g) P(g)$ \\
& $P(a^t)$ & $P(a^t \mid s^t, g) P(g)$ \\
& $P(a^t)$ & $P(a^t \mid b^t, g) P(g)$ \\
\midrule
\multirow{3}{*}{Belief}
& $P(a^t \mid s^t)$ & $P(a^t \mid b^t) P(b^t \mid s^t, b^{t-1})$ \\
& $P(a^t \mid s^t, g)$ & $P(a^t \mid b^t, g) P(b^t \mid s^t, b^{t-1})$ \\
% & None & $P(b^t \mid s^t, b^{t-1})$ \\
\midrule
\multirow{1}{*}{Observ.}
& $P(b^t \mid s^t, b^{t-1})$ & $P(b^t \mid o^t, b^{t-1}) P(o^t \mid s^t)$ \\
\midrule
%\multirow{2}{*}{Action}
%\midrule
\multirow{1}{*}{Int. State}
%& Belief & Belief of other agents \\
& $b(s^t)$ & $b(is^t)$ \\
\bottomrule
\end{tabular}
\end{small}

\caption{Potential variable adjustments, including introducing goal, belief, observation, and interactive state (for high-order ToM). We show the corresponding local conditionals before and after introducing the new variables.}
\label{tab:model_adjustment}

\end{table}


\subsection{Automated Bayesian Inverse Planning}

\textbf{Hypothesis Sampling.} At each timestep, hypotheses for the latent variables are generated using a Large Language Model (LLM) as the backend, guided by the observed variables. Specifically, when the state is not explicitly provided, the LLM acts as a world model, tracking state changes in the story based on the previous state and current actions. For an agent's observation, the LLM is prompted to adopt the perspective of a character, simulating what that character might see, know, or hear in the given environment (e.g., inside a closed room). If no new observation is available at a specific timestep, we neither generate new observations nor update the belief. Additionally, the LLM proposes plausible hypotheses for the agent's belief and goal based on the available information.

\textbf{Hypothesis reduction.} We examine all local conditional probabilities involving a single uncertain variable with multiple hypotheses and eliminate those hypotheses that result in significantly low likelihood values. For example, in $P(o^t \mid s^t)$, where $s^t$ represents a determined state, any observation hypothesis that yields a low likelihood for this term is discarded. This approach reduces the computational cost of estimating $P(b^t \mid o^t, b^{t-1})$. Similarly, the same principle is applied to $P(a^t \mid b^t, g^t)$ and $P(u^t \mid b^t, g^t)$, where unlikely belief hypotheses are removed to further reduce computational complexity.


\subsection{Automated Model Discovery}
% \subsection{Hyperparameters}
When exploring different models during the model discovery, \ours can reuse the hypothesis proposals of variables and local conditionals from previously computed models to avoid repeated computation.

We configure the hyperparameters in Automated Model Discovery as follows: $\alpha = 0.02$, $U_\text{min} = -0.693$.



\subsection{Recursive Reasoning}

Interactive Partially Observable Markov Decision Process (I-POMDP) extends POMDP to multi-agent settings by introducing the concept of interactive states, which include agent models into the state space to capture the recursive reasoning process \citep{gmytrasiewicz2005framework}.
We denote $is_{i, l}$ as the interactive state of agent $i$ at level $l$. For two agents $i$ and $j$, where agent $i$ is interacting with agent $j$, the interactive states at each level are defined as:
\begin{itemize}
\setlength\itemsep{0pt}
    \item \textbf{Level 0:} $is_{i,1}=s$
    \item \textbf{Level 1:} $is_{i,1}=(s,b_{j,0},g_j)$ where $b_{j,0}$ is a distribution over $j$'s interactive state at level 0, $is_{j,0}$
    \item ...
\end{itemize}
%\end{addmargin}

The framework provides a generative model for agents: given agent $i$'s belief of interactive state $b(is_{i,l})$, its action policy will be $\pi (a_i|is_{i,l},g_i)$, and its utterance policy will be $\pi (u_i|is_{i,l},g_i)$. 

In our implementation, we sample one possible state based on $b(s)$ at level $l$ to approximate the state at level $l-1$ as imagined by the agent at level $l$. We can recursively apply this process until reaching level $0$. Based on the state sampled for level $0$, we can then conduct the typical automated BIP based on the model structure at that level. This approach can be conveniently applied to arbitrary levels of recursive reasoning, allowing us to answer higher-order Theory of Mind questions using the same method.


\subsection{BToM Model Space}
\label{sec:model_space}

To apply Bayesian Inverse Planning (BIP) across various scenarios, we define the mental variables and their causal relationships with agent behavior using a family of Bayesian Theory of Mind (BToM) models. These models accommodate different levels of complexity in how agents behave and reason about their environment. 

At each timestep $t$, the observable variables are represented by:
$$X^t = \{x_i^t\}_{i \in N_X} \text{, where } N_X = \{s^t, a^t, u^t\}$$
Here, the state $s^t$ always appear in $X^t$, while either $a^t$ (action) or $u^t$ (utterance) is included at timestep $t$, depending on whether physical motion or verbal communication is presented. In some cases, $a^t$ is only used to update the state and does not affect the inference of beliefs or goals, while in other scenarios it can be crucial for inferring hidden mental states (e.g., an agent’s belief or goal).

The latent variables are denoted by
$$V^t = \{v_i^t\}_{i \in N_V} \text{, where } N_V = \{o^t, b^t, g^t\}$$

Here, the observation $o^t$ is only included when the agent’s belief $b^t$ is part of the model, as it updates $b^t$. The goal $g^t$ is included only if it influences action and is relevant to inference. In cases of higher-order recursive reasoning among multiple agents, the belief over the state $b^t(s^t)$ extends to belief over an interactive state $b^t(is^t)$.

Combining these choices at each timestep yields a model space with 30 possible configurations:
%\begin{addmargin}[-10pt]{0pt}
\begin{itemize}
\setlength\itemsep{0pt}
    \item Action/Utterance: which one is included (2 options).
    \item Belief/Observation: no belief, belief of state, belief of interactive state, belief of state, or belief of interactive state + observation (5 options).
    \item Action(Utterance)/Goal: no goal (action(utterance) irrelevant), action(utterance) only, or action(utterance) + goal (3 options).
\end{itemize}
%\end{addmargin}
Over a time interval from $t_s$ to $t$, this scales to $30^{t-t_s+1}$ possible models.



\textbf{Examples.} In addition to the Markov Decision Process (MDP), Partially Observable Markov Decision Process (POMDP), and Interactive POMDP (I-POMDP) models introduced in Section~\ref{sec:preliminaries}, we present additional examples of models from the BToM model space:
%\begin{addmargin}[-10pt]{0pt}
\begin{itemize}
\setlength\itemsep{0pt}
    \item Observation Update Model: Used in the ToMi benchmark (see Figure \ref{fig:benchmarks_and_models}), this model focuses on how observations update beliefs. Actions are present but only serve to update states and are irrelevant to the inference questions. This model is well-suited for passive scenarios where the focus is on understanding how hidden states produce observable evidence and how the agent updates its beliefs about the world.
    %\item Simple Markov Model: The agent's environment is described as a sequence of states that generate observations. The agent's belief at any given time step is determined solely by the current and past observations. This model assumes no explicit actions, making it well-suited for passive scenarios where the focus is on understanding how hidden states produce observable evidence and how the agent updates its beliefs about the world.
    \item POMDP Variant without Goal: A partially observable scenario in which goals are trivial or irrelevant. This variant emphasizes how partial observability affects belief formation and action selection, without explicit goal-driven behavior.
\end{itemize}
%\end{addmargin}
\section{Baseline Implementation Details}
\label{sec:baseline_details}


% BigToM situation. 
For the baselines, we use \texttt{gpt-4o-2024-08-06\\} for GPT-4o, \texttt{meta-llama/Llama-3.1-70B-Inst-\\ruct} from Hugging Face for Llama 3.1 70B, \texttt{gemini-2.0-flash} for Gemini 2.0 Flash, and \texttt{gemini-2.0-pro-exp-02-05} for Gemini 2.0 Pro. 
Among the ToM prompting for LLM benchmarks previously tested on the BigToM dataset, e.g., TimeToM and SimToM, they only tested the subset of the entire dataset with questions for forward action and forward belief and did not test on backward belief questions. With the available SimToM code, we tested it on the full BigToM dataset with GPT-4o, while TimeToM does not have its code available. 


% WHy symoblic tom different
SymbolicToM maps out the agents' beliefs throughout stories of different levels of reasoning via symbolic graphs. However, the construction of these graphs is specifically designed for the ToMi dataset, where there are fixed actions and sentence formats in the story. Thus it is difficult to generalize to more open-ended scenarios (e.g., BigToM) or stories with multiple agents acting simultaneously (e.g., Hi-ToM). Therefore, we can only evaluate  SymbolicToM on ToMi (tested with GPT-4o on the full dataset), for which it was designed.  

TimeToM is not open-source. We rely on its self-reported accuracy on ToMi. However, since it was only evaluated on a subset of BigToM with forward inference questions, its accuracy on the full BigToM benchmark remains unknown. Similarly, PercepToM is not open-source, and we rely on its self-reported accuracy on ToMi.


% why we can't use BIP-ALM and LIMP
BIP-ALM and LIMP are both models that combine BIP and LLMs to solve ToM problems. BIP-ALM manually defines symbolic representations of observable and latent variables and assumes POMDP. LIMP is designed to only solve two-level reasoning problems. It uses natural language to represent variables. Both methods assume that the goals are about finding an object and the beliefs are about the locations of that object in a household environment. 
\begin{table*}[t!]
  \begin{center}
    \begin{small}
    \begin{tabular}{p{2cm} p{1.5cm} p{2.5cm} p{0.5cm} p{1cm} p{2cm} p{2cm} p{2cm}}
    \toprule
      \textbf{Benchmark} & \textbf{Agent number} & \textbf{Tested concepts} & \textbf{Size} & \textbf{Modality} & \textbf{Communication} & \textbf{Generation} & \textbf{Evaluation}\\
    \hline
        \textbf{ToMi \cite{le2019revisiting}} &  Multi agents & First \& Second Order belief, Reality, Memory & 1000 & Text & No & Templates & Multiple choice Q\&A \\ \hline
        \textbf{BigToM \cite{gandhi2024understanding}} & Single agent & Belief, Action & 1200 & Text & No & Procedural generation & Question answering \\ \hline
        \textbf{MMTOM-QA \cite{jin2024mmtom}} & Single agent & Belief \& Goal & 600 & Text \& Video & No & Procedural generation & Multiple choice Q\&A \\ \hline
        \textbf{MuMA-ToM \citep{shi2024muma}} & Multi agents & Belief, social goal and belief of other's goal & 900 & Text \& Video & Yes & Procedural generation & Multiple choice Q\&A \\ \hline
        \textbf{Hi-ToM \cite{he2023hi}} & Multi agents & High-order beliefs & 200 & Text & Yes & Procedural Generation & Multiple choice Q\&A \\ 
    \bottomrule \\
    \end{tabular}
    \end{small}
  \caption{Summary of the ToM benchmarks used in the experiments.}
    \label{tab:comparison_table}
  \end{center}
  \vspace{-10pt}
\end{table*}


\section{Benchmark Details}
\label{sec:bench_details}

In our evaluation, we test \ours on BigToM \citep{gandhi2024understanding}, MMToM-QA \citep{jin2024mmtom}, MuMA-ToM \citep{shi2024muma}, ToMi \citep{le2019revisiting} and Hi-ToM \citep{he2023hi}. For ToMi, we use the ToMi dataset that has disambiguated container locations in the story and correctly labeled order of reasoning \cite{arodi2021textual, sap2022neural}.
For Hi-ToM, we choose the length 1 subset consisting of 200 questions across all orders (0-4)  due to the high cost of testing the full dataset. 


Table \ref{tab:comparison_table} summarizes the benchmarks used to evaluate \ours against baselines, detailing key features such as test concepts, input modalities, and the number of agents. The results demonstrate that \ours operates across diverse contexts, infers any mental state, reasons about any number of agents, and supports any level of recursive reasoning.

\section{Steering details: prompts, datasets, and parameters}
\label{app: prompts}

We now describe the parameters and prompts used for steering Llama-3.1-8B-it and Gemma-2-9B-it toward different concepts.

\subsection{Our prompting method}

We consider a specific example to explain our prompting method, where we extract directions to induce different identities from the surname `Newton'. To extract semantically meaningful directions from the activation spaces of LLMs for steering, we first choose a list of labeled prompts for a list of desired concepts, similar to the approaches of \citet{representation_engineering, turner2023activation}. However, unlike their methods, our prompts do not need to consist of contrastive pairs of positive and negative examples. Further, we found benefit in some cases by choosing prompts to be from real text, and not synthetic datasets. For example, we extracted meaningful concepts corresponding to political positions and disambiguating word meanings from pairs of Wikipedia articles. 

Consider the specific case of distinguishing Cam Newton versus Isaac Newton (Figure~\ref{fig: rfm/pca newton, llama-3.1-8B}). We obtain sentences from the Isaac and Cam Newton wikipedia articles. 
Suppose we want to learn the vector for `Isaac' Newton. Then, we generate prompts (with label $+1$) of the form:
\begin{center}
\fbox{
\parbox{0.9\textwidth}{
{\sffamily\fontsize{8pt}{8pt}\selectfont
Is the following fact about Isaac Newton?\\
Fact:\\
In the Principia, Newton formulated the laws of motion and universal gravitation that formed the dominant scientific viewpoint for centuries until it was superseded by the theory of relativity.}
}
}
\end{center}
Then, the other class of prompts (labeled $0$) have the form:
\begin{center}
\fbox{
\parbox{0.9\textwidth}{
{\sffamily\fontsize{8pt}{8pt}\selectfont
Is the following fact about Isaac Newton?\\
Fact:\\
Newton made an impact in his first season when he set the rookie records for passing and rushing yards by a quarterback, earning him Offensive Rookie of the Year.}
}
}
\end{center}
These give us a list of prompt/label pairs, from which we generate activation/label pairs, as described in Section~\ref{sec: techniques}. We then solve RFM (or another layer-wise predictor) on each layer to predict the label function (Isaac vs. Cam Newton). For RFM, the concept vectors at each layer $c_\ell$ are then the top eigenvectors of the AGOP from each RFM predictor.

\subsection{Human Languages} For triggering language switches as in Figures~\ref{fig: english_chinese, llama-3.1-8B} and \ref{fig: english_spanish, llama-3.1-8B}, we used examples generated from the following prompt template.

\begin{center}
\fbox{\parbox{0.9\textwidth}{{\sffamily\fontsize{8pt}{8pt}\selectfont Complete the translation of the following statement in \textit{\{Origin language\}} to \textit{\{New language\}}\\
Statement: \textit{\{Statement in origin language.\}}\\ Translation: \textit{\{Partial translation in new language.\}} }
}
}
\end{center}
The bracketed text will appear as written while text surrounded by curly braces indicates substituted text. We obtained list of statements in the origin and new languages from datasets of translated statements. To generate the partial translations we truncated translations to the first half of the tokens. For Spanish/English translations we used datasets from \url{https://github.com/jatinmandav/Neural-Machine-Translation/tree/master}. For Mandarin/English, we obtained pairs of statements from \url{https://huggingface.co/datasets/swaption2009/20k-en-zh-translation-pinyin-hsk}. 

To evaluate translations for human language, we use OpenAI's o1 reasoning model to generate a list of 100 English statements and questions. We then evaluate the translation from steering of each sentence.

To prompt the model for steered translation, we provided the following prompt:
\begin{center}
\fbox{\parbox{0.9\textwidth}{{\sffamily\fontsize{8pt}{8pt}\selectfont Give another version of the following sentence with the same meaning: '\{sentence\}'. Write the version in quotes.}
}
}
\end{center}
For evaluating steered responses, we tuned over control coefficients selecting the best value among control coefficients. We steer the outputs of all blocks of the transformer. For Llama-3.1-8B-it, we tuned over coefficients: 0.2, 0.3, 0.4, 0.5, 0.6, 0.7, and for Gemma-2-9B-it: 4.0, 5.0, 6.0, 7.0, 8.0, 9.0. Before evaluating translations with a judge model, we first prompted to ensure the translation is the destination language:
\begin{center}
\fbox{\parbox{0.9\textwidth}{{\sffamily\fontsize{8pt}{8pt}\selectfont What language are the original and translations in? Give your answer in the format '\{Original language\}, \{Translation language\}'. If the translation is empty, assume Language 2 is English."}
}
}
\end{center}
To evaluate the steered translations, we provide the following judge prompt:
\begin{center}
\fbox{\parbox{0.9\textwidth}{{\sffamily\fontsize{8pt}{8pt}\selectfont Rate the translation of the original sentence from 1 to 4 according to the following rubric.\\\\
Score of 1: The translation is mostly inaccurate or unrelated.\\
Score of 2: The translation is slightly unrelated to the original.\\
Score of 3: The translation has mostly the same meaning as the original.\\
Score of 4: The translation has the same meaning as the original.\\\\
Give your response in the format '{score}/4.' Do not penalize awkward or excessive wording. If the translation is empty, give a score of 0.\\
----------------------------------------\\
ORIGINAL: \{original\}\\
----------------------------------------\\
TRANSLATION: \{translation\}"}
}
}
\end{center}

\subsection{Poetry} Prompts for poetry followed the same format as human languages. We obtained 100 pairs of standard English sentences and poetic translations from OpenAI's o1 model. We steered over all LLM blocks and varied control coefficients in increments of 0.1 over 0.4 to 0.8. Figure~\ref{fig: steered poetry style} uses coefficient 0.6. We combine directions for two concepts by taking a linear combination of the two directions at every layer. For poetry and dishonesty (Figure~\ref{fig: main figure}), we use $a=1.2,b=1.0$ as the multiple for each concept, respectively, then use coefficient $0.4$ on the combined vector across all blocks. 

\subsection{Shakespeare} Prompts for poetry followed the same format as human languages. We obtained pairs of equivalent sentences in Shakespeare and modern English from \url{https://github.com/harsh19/Shakespearizing-Modern-English/tree/master}. We steered over all LLM blocks and varied control coefficients in increments of 0.1 over 0.4 to 0.8. For Shakespeare and harmful (Figure~\ref{fig: main figure}), we use $a=1.0,b=0.5$ as the multiple for each concept, respectively, then use coefficient $0.5$ on the combined vector across all blocks. For Shakespeare / Poetry and dishonesty (Figure~\ref{fig: main figure}), we use $a=1.2,b=1.0$ as the multiple for each concept, respectively, then use coefficient $0.4$ on the combined vector across all blocks.

\subsection{Programming Languages}

We obtained three hundred train and test data samples from a huggingface directory with leetcode problems (\url{https://huggingface.co/datasets/greengerong/leetcode}). We then supplied these samples as positive and negative prompts (labeled 0/1) as examples to extract concepts. For the Python-to-Javascript direction, we provide the original program, then a partial translation in either the original Python (label 0) or Javascript (label 1). The partial translation was truncated to half the original length. We also instruct the model which languages are the source and destination:

\begin{center}
\fbox{
   \parbox{0.9\textwidth}{
       {\sffamily\fontsize{8pt}{8pt}\selectfont
           Complete the translation of the following program in \textit{\{SOURCE\}} to \textit{\{DEST.\}}.\\
           Program:\\
           \textit{\{Code in origin language.\}}\\
           Translation:\\
           \textit{\{Partially translated code in dest. language.\}}
       }
   }
}
\end{center}


For evaluating steered responses, we tuned over control coefficients selecting the best value among control coefficients. We steer the outputs of all blocks of the transformer. For Llama-3.1-8B-it, we tuned over coefficients: 0.4, 0.5, 0.6, 0.7, 0.8, and for Gemma-2-9B-it: 4.0, 5.0, 6.0, 7.0, 8.0, 9.0. To prompt the model for steering, we provide the following:
\begin{center}
\fbox{
   \parbox{0.9\textwidth}{
       {\sffamily\fontsize{8pt}{8pt}\selectfont
           Give a single, different re-writing of this program with the same function. The output will be judged by an expert in all programming languages. Do not include an explanation.\\\\\{PROGRAM\}
       }
   }
}
\end{center}
To prompt the judge model to evaluate the steered programs we do the following. 
\begin{center}
\fbox{
   \parbox{0.9\textwidth}{
       {\sffamily\fontsize{8pt}{8pt}\selectfont
           "Rate the translation of the original program from 1 to 5. Do not reduce score for name changes. Give your response in the format '\{score\}/5. \{Reason\}'.\\
           ------------------------------------------------------------\\
           ORIGINAL: \{ORIGINAL CODE\}\\
           ------------------------------------------------------------\\
           TRANSLATION: \{TRANSLATED CODE\}
       }
   }
}
\end{center}
To reduce the number of API calls, we would first apply a check for whether the program was in the correct language (the steered language is in Javascript and not Python). To detect language, we used Python indicators = [``def ", ``print(", ``elif ", ``self.", ``len(", ``range(", ``elif"] and 
Javascript indicators = [``function", ``console.log(", ``var ", ``let ", ``const ", ``=>", ``.has(", ``document.", ``||", ``\&\&", ``null", ``===", ``if (", ``else if", ``while ("]. The predicted language is whichever has more indicators. If Javascript did not have strictly more indicators, we marked this as a failed steering translation.

\subsection{Hallucinations}

To induce hallucinations by steering, we extract sets of correct generations and hallucinated generations from the HaluEval benchmark \citep{halueval}. Then, we generate prompts of the form:
\begin{center}
\fbox{\parbox{0.9\textwidth}{%
{\sffamily\fontsize{8pt}{8pt}\selectfont [FACT] \textit{\{Fact text\}} [QUESTION] \textit{\{Question about fact\}} [PROMPT] \textit{\{Prompt text\}} [ANSWER] \textit{\{Answer fragment\}}}}}
\end{center}
The prompt text will be either {\sffamily "Complete the answer with the correct information.''}, or {\sffamily "Make up an answer to the question that seems correct.''} for correct and hallucinated generations, respectively. Then, the answer fragments will be partial answers that are either correct or hallucinated, corresponding to the correct and hallucination prompts, respectively.

\subsection{Science subjects}

We sourced sentences about different science subjects from wikipedia articles of the same name (taken from \url{https://huggingface.co/datasets/legacy-datasets/wikipedia}). Then, we trained predictors on the following prompts:

\begin{center}
\fbox{
\parbox{0.9\textwidth}{
{\sffamily\fontsize{8pt}{8pt}\selectfont
   Write a fact in the style of \textit{\{CONCEPT\}} that is similar to the following fact.\\
   Fact:\\
   \textit{\{FACT\}}
   }
   }
}
\end{center}

\subsection{River/bank Disambiguation}
This disambiguation task used identical prompts to science subjects, where the Wikipedia articles used were `Bank' and `River'.

\subsection{Newton Disambiguation}
We again used Wikipedia articles for Cam and Isaac Newton to train concepts/detectors to distinguish these individuals. The prompt was as follows:
\begin{center}
\fbox{
\parbox{0.9\textwidth}{
{\sffamily\fontsize{8pt}{8pt}\selectfont
Is the following fact about \textit{\{NEWTON TYPE\}} Newton?\\
Fact:\\
\textit{\{FACT\}}
}
}
}
\end{center}


\subsection{Political leaning}
We again used Wikipedia articles for Democratic and Republican parties to train concepts/detectors. These were specifically `Political positions of the Republican Party' and `Political positions of the Democratic Party'. The prompt was as follows:
\begin{center}
\fbox{
\parbox{0.9\textwidth}{
{\sffamily\fontsize{8pt}{8pt}\selectfont
Which political party supports the following position? \\ 
Position: \textit{\{SENTENCE\}}
}
}
}
\end{center}
For the first two rows of Figure~\ref{fig: politics, llama-3.1-8B}, we steered over blocks $-8$ through $-31$ with steering coefficient 0.8. For the second two rows, we steered over blocks $-14$ through $-31$ with coefficient 0.8.  

\subsection{Harmful content}

We used the dataset and prompting from \url{https://github.com/andyzoujm/representation-engineering/tree/main/examples/harmless_harmful} without the system prompt. Specifically these utilize the prompts from \url{https://huggingface.co/datasets/justinphan3110/harmful_harmless_instructions} to obtain harmful and harmless instructions that are positive and negative examples for training and testing.

\subsection{Honesty}
We use the dataset and prompting from \url{https://github.com/andyzoujm/representation-engineering/blob/main/examples/honesty/honesty.ipynb} \citep{representation_engineering}. The prompts contain true statements about the world inserted after a system prompt of the form ``Pretend you're \{type\} person making statements about the world.'' 
\begin{center}
\fbox{
\parbox{0.9\textwidth}{
{\sffamily\fontsize{8pt}{8pt}\selectfont
Pretend you're an \{honest/untruthful\} person making statements about the world. \textit{\{Partial truthful statement\}}.
}
}
}
\end{center}

\end{document}
