\section{Prompts used in \ours}
\label{sec:prompts}

% information extraction
\subsection{Information Extraction}
% \subsubsection{General Prompts Used Across Datasets}
We use the following prompts to extract information for each variable in a given question. 
% extracting the inferred agent, inferred variable, main agent's actions, extra information 

\begin{tcolorbox}[
    breakable,
    colframe=green!40!black,
    colback=green!5,
    coltitle=white,
    fonttitle=\bfseries,
    title=Identifying the main agent,
    colbacktitle=green!40!black
]
Find the name of the character that we need to infer about in the question and choices. Only output the name. Do not answer the question. \newline

Question: [Question]

Choices: [Choices]

Character name:
\end{tcolorbox}



\begin{tcolorbox}[
    breakable,
    colframe=green!40!black,
    colback=green!5,
    coltitle=white,
    fonttitle=\bfseries,
    title=Identifying all the agents,
    colbacktitle=green!40!black
]
Extract the names of all the characters from the story and question. Provide only the names or roles, without any additional information. Do not answer the question.

Your response should be a list containing the names, like [``name1'', ``name2'']. \newline

Story: [Story]

Response: 
\end{tcolorbox}


\begin{tcolorbox}[
    breakable,
    colframe=green!40!black,
    colback=green!5,
    coltitle=white,
    fonttitle=\bfseries,
    title=Identifying the mental variable to be inferred,
    colbacktitle=green!40!black
]
Choose the variable that best summarizes the information about the differences that the choices contain. Only output the variable. \newline

Variables include: [Variables]

Choices: [Choices]

Variable: 
\end{tcolorbox}


\begin{tcolorbox}[
    breakable,
    colframe=green!40!black,
    colback=green!5,
    coltitle=white,
    fonttitle=\bfseries,
    title=Identifying extra information in the question,
    colbacktitle=green!40!black
]
If there is any assumed information in the question given (a conditional clause starting with specific words like ``if'' is contained), rewrite it as a declarative sentence. Do not include any questions in the extra information. Do not make up details for the information. Use the original wording.

Otherwise, output ``NONE''. \newline

Question: [Question]

Extra Information:
\end{tcolorbox}


\begin{tcolorbox}[
    breakable,
    colframe=green!40!black,
    colback=green!5,
    coltitle=white,
    fonttitle=\bfseries,
    title=Extracting actions of the main agent,
    colbacktitle=green!40!black
]
Extract the actions of [Inferred\_agent] in the story verbatim without changing any of the original words, pluralizing the words, adding in [Inferred\_agent] or any other name, replacing any of the words, replacing pronouns with names or replacing any names with pronouns. Actions of [Inferred\_agent] are defined as events that will change the world state, e.g., [Inferred\_agent] moving to a new location is an action but [Inferred\_agent] being at a location is not an action. If [Inferred\_agent] says something, the whole sentence (with ``replied'', ``said'') is seen as an action.

Do not change the names of any of the agents, if there is not a name and only a pronoun then just leave the pronoun. There can be more than one agent or more than just the inferred agent.

If there are multiple actions in a sentence then they should be extracted as one single action, without changing any of the original words, such as pluralizing the words, replacing any of the words, replacing pronouns with names, or replacing any names with pronouns, and do not add any words.

Do not insert actions, pronouns, or other words that are not explicitly stated in the text. Do not separate the objects in the same action.

Do not add any pronouns. Keep the commas, if any.

Only actions that have already occurred at the time can be considered clearly stated. Again, only extract actions performed by [Inferred\_agent].

The output format should be: [``aaa.'', ``bbb.'', ...]. Output only this list. \newline

Story: [Story]

Extraction:
\end{tcolorbox}



\begin{tcolorbox}[
    breakable,
    colframe=green!40!black,
    colback=green!5,
    coltitle=white,
    fonttitle=\bfseries,
    title=Extracting actions,
    colbacktitle=green!40!black
]
Determine if [Character]'s action(s) is clearly stated in the story.

The action(s) cannot be the character's inner thoughts.

Only actions of [Character] that have already occurred, or are currently taking place can be considered clearly stated.

If it's more like [Character]'s desire or goal, it does not count as an action. [Character]'s utterance is considered as an action (include the verb like ``said'' or ``replied'' in the evidence sentence, if any). Do not change any of the original wording.

Answer in the form of a list. The first element of the list contains the option A or B. A means clearly stated, and B means not clearly stated.

If the answer is A, include sentence(s) from the original story that serves as evidence, and place it in the second element of the list, without any kind of formatting. Note that there could be multiple action sentences.

Otherwise, the second element can be an empty string. Do not write anything else.

Example 1: [``A'', ``evidence sentence.'']

Example 2: [``B'', ``''] \newline

Story: [Story]

Answer: 
\end{tcolorbox}



\begin{tcolorbox}[
    breakable,
    colframe=green!40!black,
    colback=green!5,
    coltitle=white,
    fonttitle=\bfseries,
    title=Extracting beliefs,
    colbacktitle=green!40!black
]
Determine if the belief of [Character] is clearly stated in the story.

Usually, belief is one's understanding of the state of the world or the state of others. A subjective attitude towards things does not count as belief. An action or utterance of the agent does not count as a belief. Words like ``know'' or ``believe'' could be hints for belief.

Answer in the form of a list. The first element of the list contains the option A or B. A means clearly stated, and B means not clearly stated.

If the answer is A, include sentence(s) from the original story that serves as evidence, and place it in the second element of the list, without any kind of formatting.

Otherwise, the second element can be an empty string. Do not write anything else.

Example 1: [``A'', ``evidence sentence.'']

Example 2: [``B'', ``''] \newline

Story: [Story]

Answer:
\end{tcolorbox}



\begin{tcolorbox}[
    breakable,
    colframe=green!40!black,
    colback=green!5,
    coltitle=white,
    fonttitle=\bfseries,
    title=Extracting goals,
    colbacktitle=green!40!black
]
Determine if the goal of [Character] is clearly stated in the story.

Usually, goals refer to a person's goals or intentions regarding a particular event. Moreover, a sentence that shows a person has been trying to do something, or summarizes their efforts of doing something should always be considered a goal. Helping others to achieve their goals also counts as a person's goal.

Answer in the form of a list. The first element of the list contains the option A or B. A means clearly stated, and B means not clearly stated.

If the answer is A, include sentence(s) from the original story that serves as evidence, and place it in the second element of the list, without any kind of formatting.

Otherwise, the second element can be an empty string. Do not write anything else.

Example 1: [``A'', ``evidence sentence.'']

Example 2: [``B'', ``''] \newline

Story: [Story]

Answer:
\end{tcolorbox}


\begin{tcolorbox}[
    breakable,
    colframe=green!40!black,
    colback=green!5,
    coltitle=white,
    fonttitle=\bfseries,
    title=Extracting observations,
    colbacktitle=green!40!black
]
Determine if the observation of [Character] is clearly stated in the story.

Observation refers to the main character's perception of an event; it is only considered clearly stated when the protagonist's perception is explicitly mentioned, like if they visually see something, visually notice something, or hear something, or any other state that can be perceived by the agent with but not limited to their 5 senses.

A character's utterance does not mean that their observation is clearly stated, because they might lie.

Answer in the form of a list. The first element of the list contains the option A or B. A means clearly stated, and B means not clearly stated.

If the answer is A, include sentence(s) from the original story that serves as evidence, and place it in the second element of the list, without any kind of formatting.

Otherwise, the second element can be an empty string. Do not write anything else.

Example 1: [``A'', ``evidence sentence.'']

Example 2: [``B'', ``''] \newline

Story: [Story]

Answer:
\end{tcolorbox}



\begin{tcolorbox}[
    breakable,
    colframe=green!40!black,
    colback=green!5,
    coltitle=white,
    fonttitle=\bfseries,
    title=Extracting states,
    colbacktitle=green!40!black
]
Determine if the story contains the objective state(s) of an object or an event.

State refers to the physical condition of something or the state of the world.

No actions of agents should be involved in the state but it can be the result of an action of an agent. For example, ``A entered B'' is not a state, while ``A is in B'' is a state.

An objective state statement should not include personal perspectives but should be objective. If a person's perception is involved, it is no longer considered an objective state.

Answer in the form of a list. The first element of the list contains the option A or B. A means clearly stated, and B means not clearly stated.

If the answer is A, include sentence(s) from the original story that serves as evidence, and place it in the second element of the list, without any kind of formatting.

If there are multiple sentences, include them all in the second element of the list.

Otherwise, the second element can be an empty string. Do not write anything else.

Example 1: [``A'', ``evidence sentence(s).'']

Example 2: [``B'', ``''] \newline

Story: [Story]

Answer:
\end{tcolorbox}



\subsection{Hypothesis Sampling}
We use the following prompts to sample hypotheses for the latent variables in the BToM models. 

\begin{tcolorbox}[
    breakable,
    colframe=blue!40!black,
    colback=blue!10,
    coltitle=white,
    fonttitle=\bfseries,
    title=Sampling beliefs,
    colbacktitle=blue!40!black
]
Propose [num] hypotheses for the belief of [Character] in the story aligned with the context of: [Context]. Make sure that it is not any of the hypotheses in [Wrong Hypotheses], if it is then propose new hypotheses that are very different. 

It should be related to [Information] and the context described above.

The hypotheses do not require reasoning or consideration of whether they are likely to occur. The only limitation is that they must be relevant to the information already provided. You cannot return nothing.
Usually, belief is one's view or perspective on a matter, and it represents an understanding of the state of the world or the state of others. The emotional attitudes toward a specific thing do not count as belief.
Do not state any reason for the hypotheses. Do not contain any form of explanation in the hypotheses.
Output a list of hypotheses of length [num] in the following form: [``aaa.'', ``bbb.'', ...] \newline

Context: [Context]

Belief Hypotheses:
\end{tcolorbox}



\begin{tcolorbox}[
    breakable,
    colframe=blue!40!black,
    colback=blue!10,
    coltitle=white,
    fonttitle=\bfseries,
    title=Sampling goals,
    colbacktitle=blue!40!black
]
Propose [num] hypotheses for the goal of [Character].

The goal refers to [Character]'s intentions.

Do not provide any explanation for the hypotheses. Do not propose any sentence that's not depicting the goal, like the action or belief of [Character].

The wording for hypotheses cannot be speculative.

The proposed goal does not have to be too specific, e.g., Andy wants to help others; Andy wants to hinder others; Andy is indifferent towards other's goals, etc.

Given information: [Information]

Ensure that the hypotheses align with the given information perfectly. It means that the proposed [Character]'s goal matches what's contained in the information.

Output the hypotheses in the following form: [``aaa.''] \newline

Goal Hypotheses: []
\end{tcolorbox}



\begin{tcolorbox}[
    breakable,
    colframe=blue!40!black,
    colback=blue!10,
    coltitle=white,
    fonttitle=\bfseries,
    title=Sampling observations,
    colbacktitle=blue!40!black
]
Propose [num] hypotheses for [Character]'s observation of the world.

The observation refers to [Character]'s current perception of events or the world state. It is only considered clearly stated when [Character]'s perception is explicitly mentioned, like if [Character] sees something or perceives something through other senses. Do not be speculative.

Do not provide any explanation for the hypotheses. Do not propose any sentence that's not depicting the observation, like the action or belief of [Character].

The wording for hypotheses cannot be speculative.

If the information contains ``not'', make sure the verb for perception (e.g., ``see'', 'perceives') goes before ``not'' in the hypotheses. e.g., use 'sees that A is not in B' instead of 'does not see that A is in B'
Otherwise, do not include ``not'' in your hypotheses, and make sure the verb for perception goes first, e.g., 'sees that A is in B'.

Given information: [Information]

Ensure that the hypotheses align with the given information perfectly. It means that when the person has the observation the person will act according to the given information.

First, list all entities in the given information. Then, formulate hypotheses using all entities. Make sure the hypothesis starts with [Character].

Output the hypotheses in the following form: [``aaa.''] \newline

Observation Hypotheses: []
\end{tcolorbox}



\subsection{Likelihood Estimation} 
We use the following prompts to estimate the likelihood of different variables. 

\begin{tcolorbox}[
    breakable,
    colframe=orange!40!black,
    colback=orange!10,
    coltitle=white,
    fonttitle=\bfseries,
    title=Estimating the likelihood of the observation given the state,
    colbacktitle=orange!40!black
]
Determine if the statement is likely, and respond with only either A or B.

State: \{state\}

Here is a statement of \{inf\_agent\}'s current observation. Only evaluate current observation of \{inf\_agent\} based on the state. Do not imagine anything else. Think about \{inf\_agent\}'s location. \{inf\_agent\} is quite likely to observe all objects and events in \{inf\_agent\}'s location, and is unlikely to observe states in another location. If \{inf\_agent\} does not appear in the state, \{inf\_agent\} can't observe anything. Note that the statement has to be precise in wording to be likely. For example, the treasure chest and container are different in wording and they're different objects. \newline

Determine if the following statement is likely: \{statement\}

A) Likely.

B) Unlikely.
\end{tcolorbox}


\begin{tcolorbox}[
    breakable,
    colframe=orange!40!black,
    colback=orange!10,
    coltitle=white,
    fonttitle=\bfseries,
    title=Estimating the likelihood of the action given the goal and belief and belief of goal,
    colbacktitle=orange!40!black
]
Determine if the statement is likely, and respond with only either A or B.

\{inf\_agent\}'s goal: \{goal\}

\{inf\_agent\}'s belief: \{belief\}

\{inf\_agent\}'s belief of other's goal: \{belief of goal\}

\{inf\_agent\}'s action: \{action\}

When \{inf\_agent\} wants to help, \{inf\_agent\} is likely to bring an object to other's desired location, and unlikely to grab an object away from other's desired location.

When \{inf\_agent\} wants to hinder, \{inf\_agent\} is likely to grab an object away from other's desired location, and unlikely to bring an object to other's desired location.

When \{inf\_agent\} doesn't know other's goal, \{inf\_agent\} is likely to act according to \{inf\_agent\}'s belief.

If \{inf\_agent\} wants to help and \{inf\_agent\} believes the object is placed at other's desired location, it's unlikely \{inf\_agent\} will move the object.

If \{inf\_agent\}'s goal, \{inf\_agent\}'s belief of goal, and \{inf\_agent\}'s action do not align in any way, the action is unlikely. \newline

Determine if \{inf\_agent\}'s action is likely.

A) Likely.

B) Unlikely.
\end{tcolorbox}



\begin{tcolorbox}[
    breakable,
    colframe=orange!40!black,
    colback=orange!10,
    coltitle=white,
    fonttitle=\bfseries,
    title=Estimating the likelihood of the action given the goal and belief,
    colbacktitle=orange!40!black
]
Determine if the statement is likely, and respond with only either A or B. If it's not certain but it's possible, it's likely.

\{inf\_agent\}'s goal: \{goal\}

\{inf\_agent\}'s belief: \{belief\}

Here is a statement of \{inf\_agent\}'s action. Think about \{inf\_agent\}'s goal.

\{inf\_agent\} will perform actions according to \{inf\_agent\}'s belief, and any action that does not align with the belief is very unlikely, except when \{inf\_agent\}'s goal is to hinder or to prevent others. In this case (goal is hindering others) \{inf\_agent\}'s action is only likely when it's different from \{inf\_agent\}'s belief. If \{inf\_agent\}'s mental states contain conditions like ``When giving information'' and the action is not giving information, it's unlikely. \newline

Determine if the following statement is likely: \{statement\}

A) Likely.

B) Unlikely.
\end{tcolorbox}


\begin{tcolorbox}[
    breakable,
    colframe=orange!40!black,
    colback=orange!10,
    coltitle=white,
    fonttitle=\bfseries,
    title=Estimating the likelihood of the best action among choices given the goal and belief,
    colbacktitle=orange!40!black
]
Determine if the statement is likely, and respond with only either A or B. If it's not certain but it's possible, it's likely.\newline
\{inf\_agent\}'s belief: \{belief\} \newline
\{inf\_agent\}'s goal: \{goal\} \newline
If the next immediate actions possible are: \{actions\}\newline
Determine which immediate action is most possible given the information about \{inf\_agent\}'s goal and belief.\newline

Determine if the following statement is likely: \{action\_a\} is a better immediate action than \{action\_b\}.

A) Likely.

B) Unlikely.
\end{tcolorbox}


\begin{tcolorbox}[
    breakable,
    colframe=orange!40!black,
    colback=orange!10,
    coltitle=white,
    fonttitle=\bfseries,
    title=Estimating the likelihood of the initial belief,
    colbacktitle=orange!40!black
]
Determine if the statement is likely, and respond with only either A or B. If it's not certain but it's possible, it's considered likely.\newline
Here is a statement of the story and \{inf\_agent\}' initial belief. \newline
There is an action that causes the state of the main object to change. Based on \{inf\_agent\}'s observations determine if \{inf\_agent\} perceives the state of the object change. \newline
If it is not clearly stated that \{inf\_agent\} perceives it then we do not assume that \{inf\_agent\} perceived the change of state. \newline
If \{inf\_agent\} perceives this change then it is highly likely that \{inf\_agent\}'s belief aligns with the change of state of the object. \newline
If \{inf\_agent\} does not perceive this change or if it is unknown if \{inf\_agent\} perceives this change then it is highly likely that \{inf\_agent\}'s belief does not align with the change of state of the object. \newline
Story: \{story\} \newline
Think about the state of the world and others actions. \{inf\_agent\}' belief can change throughout time through other's actions and what \{inf\_agent\} can observe. It is also important to think about if \{inf\_agent\} can observe other's actions. If \{inf\_agent\} can observe the same then their belief will change and if not then their belief will remain constant. Use this to determine \{inf\_agent\}'s beliefs. \newline

Determine if the following statement is likely: \{statement\}

A) Likely.

B) Unlikely.
\end{tcolorbox}


\begin{tcolorbox}[
    breakable,
    colframe=orange!40!black,
    colback=orange!10,
    coltitle=white,
    fonttitle=\bfseries,
    title=Estimating the likelihood of the belief given the observation and previous belief,
    colbacktitle=orange!40!black
]
Determine if the statement is likely, respond with only either A or B. \newline
\{inf\_agent\}'s previous belief: \{previous\_belief\} \newline
\{inf\_agent\}'s observation: \{observation\} \newline
Here is a statement of \{inf\_agent\}'s current belief. If \{inf\_agent\}'s current belief is not aligned with \{inf\_agent\}'s observation, it is very unlikely. \newline

Determine if the following statement is likely: \{statement\}

A) Likely.

B) Unlikely.
\end{tcolorbox}



\begin{tcolorbox}[
    breakable,
    colframe=orange!40!black,
    colback=orange!10,
    coltitle=white,
    fonttitle=\bfseries,
    title=Estimating the likelihood of the belief given the state and previous belief,
    colbacktitle=orange!40!black
]
Determine if the statement is likely, respond with only either A or B.\newline
\{inf\_agent\}'s previous belief: \{belief\} \newline
State: \{state\} \newline
Here is a statement of \{inf\_agent\}'s current belief. If \{inf\_agent\}'s current belief is not aligned with the state, it is very unlikely.\newline

Determine if the following statement is likely: \{statement\}

A) Likely.

B) Unlikely.
\end{tcolorbox}


\begin{tcolorbox}[
    breakable,
    colframe=orange!40!black,
    colback=orange!10,
    coltitle=white,
    fonttitle=\bfseries,
    title=Estimating the likelihood of the utterance,
    colbacktitle=orange!40!black
]
Determine if \{inf\_agent\}'s utterance is likely, and respond with only either A or B.\newline
\{inf\_agent\}'s belief: \{belief\} \newline
\{inf\_agent\}'s goal: \{goal\} \newline
\{inf\_agent\}'s utterance: \{utterance\}\newline
When \{inf\_agent\}'s goal is to help others, \{inf\_agent\}'s utterance is likely when it strictly reflects \{inf\_agent\}'s belief, and unlikely if it does not reflect \{inf\_agent\}'s belief.\newline
When \{inf\_agent\}'s goal is to hinder or to prevent others from achieving their goals, \{inf\_agent\}'s utterance is likely when it's different from \{inf\_agent\}'s belief, and unlikely if it reflects \{inf\_agent\}'s belief.\newline

Determine if \{inf\_agent\}'s utterance is likely.

A) Likely.

B) Unlikely.
\end{tcolorbox}







\subsection{Initial Model Proposal}
We use the following prompts to propose an initial model for a question and determine if the question has higher-order beliefs. 

\begin{tcolorbox}[
    breakable,
    colframe=purple!40!black,
    colback=purple!10,
    coltitle=white,
    fonttitle=\bfseries,
    title=Proposing the initial model,
    colbacktitle=purple!40!black
]
What variables are necessary to solve this question? Please provide the answer without an explanation. \newline
Please select from the following: [``State'', ``Observation'', ``Belief'', ``Action'', ``Goal''] \newline
State: The true condition of the environment. This should always be included. \newline
Observation: The observed information about the state. Include this when the agent has partial observations of the state. \newline
Belief: The agent's current estimation of the true state is based on the state or its observation. \newline
Action: A move made by the agent, informed by the state or belief. Include this only when it is directly relevant to answering the question. \newline
Goal: The objective the agent is trying to achieve. Include this only if ``Action'' is included. \newline

Question:\{example\_question\}

Variables: \{example\_answer\}

Question: \{question\}

Variables:
\end{tcolorbox}



\begin{tcolorbox}[
    breakable,
    colframe=purple!40!black,
    colback=purple!10,
    coltitle=white,
    fonttitle=\bfseries,
    title=Determining if the question contains a higher-order belief,
    colbacktitle=purple!40!black
]
Determine whether the question is about a higher-order belief. \newline
A higher-order belief refers to a belief about another person's belief, goal, or action. \newline
It is not a high-order belief if it only asks about one agent's belief. \newline
Please respond with ``Yes'' or ``No''. \newline
If the answer is ``Yes'', the question often ends with ``Where does A think that B ...?'' Otherwise, respond ``No''. \newline

Question: [A story involving several people.] Where will Jack look for the celery?

Higher-order belief: No

Question: [A story involving several people.] Where does Jack think that Chloe searches for the hat?

Higher-order belief: Yes 

Question: \{question\}

Higher-order belief:
\end{tcolorbox}