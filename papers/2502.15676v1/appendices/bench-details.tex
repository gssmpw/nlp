\begin{table*}[t!]
  \begin{center}
    \begin{small}
    \begin{tabular}{p{2cm} p{1.5cm} p{2.5cm} p{0.5cm} p{1cm} p{2cm} p{2cm} p{2cm}}
    \toprule
      \textbf{Benchmark} & \textbf{Agent number} & \textbf{Tested concepts} & \textbf{Size} & \textbf{Modality} & \textbf{Communication} & \textbf{Generation} & \textbf{Evaluation}\\
    \hline
        \textbf{ToMi \cite{le2019revisiting}} &  Multi agents & First \& Second Order belief, Reality, Memory & 1000 & Text & No & Templates & Multiple choice Q\&A \\ \hline
        \textbf{BigToM \cite{gandhi2024understanding}} & Single agent & Belief, Action & 1200 & Text & No & Procedural generation & Question answering \\ \hline
        \textbf{MMTOM-QA \cite{jin2024mmtom}} & Single agent & Belief \& Goal & 600 & Text \& Video & No & Procedural generation & Multiple choice Q\&A \\ \hline
        \textbf{MuMA-ToM \citep{shi2024muma}} & Multi agents & Belief, social goal and belief of other's goal & 900 & Text \& Video & Yes & Procedural generation & Multiple choice Q\&A \\ \hline
        \textbf{Hi-ToM \cite{he2023hi}} & Multi agents & High-order beliefs & 200 & Text & Yes & Procedural Generation & Multiple choice Q\&A \\ 
    \bottomrule \\
    \end{tabular}
    \end{small}
  \caption{Summary of the ToM benchmarks used in the experiments.}
    \label{tab:comparison_table}
  \end{center}
  \vspace{-10pt}
\end{table*}


\section{Benchmark Details}
\label{sec:bench_details}

In our evaluation, we test \ours on BigToM \citep{gandhi2024understanding}, MMToM-QA \citep{jin2024mmtom}, MuMA-ToM \citep{shi2024muma}, ToMi \citep{le2019revisiting} and Hi-ToM \citep{he2023hi}. For ToMi, we use the ToMi dataset that has disambiguated container locations in the story and correctly labeled order of reasoning \cite{arodi2021textual, sap2022neural}.
For Hi-ToM, we choose the length 1 subset consisting of 200 questions across all orders (0-4)  due to the high cost of testing the full dataset. 


Table \ref{tab:comparison_table} summarizes the benchmarks used to evaluate \ours against baselines, detailing key features such as test concepts, input modalities, and the number of agents. The results demonstrate that \ours operates across diverse contexts, infers any mental state, reasons about any number of agents, and supports any level of recursive reasoning.
