\section{Related Work}
Considerable efforts have been dedicated to the study of AoI-oriented problems in ALOHA-like uplink random access networks, as evidenced by extensive literature, including ____. 
In ____, the AoI performance of slotted-ALOHA networks where sources generate packets at will was studied, while the authors in ____ focused on that of the slotted-ALOHA networks with stochastic arrivals of status updates.
In particular, the authors in ____ analyzed and optimized the AoI performance of the threshold-ALOHA networks, where each user transmits with a certain probability when its instantaneous AoI exceeds a predefined threshold. 
%The closed-form expression of the time-average AoI was achieved and optimized.
Moreover, the authors of ____ derived the analytical expression of the AoI performance of the frame slotted-ALOHA with reservation and data slots, followed by the associated optimization.
Our work offers analysis and optimization of the AoI in UORA networks involving an intricate backoff mechanism, which is not considered in ALOHA networks.

The AoI performance of another random access mechanism, carrier-sense multiple access (CSMA), has been explored in references ____. 
Particularly, in ____, the authors investigated the AoI performance of a CSMA network with randomly distributed parameters.
Based on the notion of stochastic hybrid systems (SHS), a closed-form expression of the network-wide average AoI was derived.
An AoI-oriented optimization problem was formulated and then converted to a convex problem, enabling efficient optimization.
Reference ____ employed a novel SHS model incorporating the collision probability to study the AoI performance of a tagged node in a practical CSMA network including dense background nodes.
The AoI-optimal traffic arrival rate of the tagged node was analytically found on the basis of the creative model.
However, the optimization objective and the parameter to be optimized in ____ differ from those in our work.
The authors of ____ proposed a distributed policy called Fresh-CSMA based on the backoff mechanism of the CSMA technology to optimize AoI in single-hop wireless networks.
Fresh-CSMA was proven to match the centralized scheduling decisions of the max-weight policy, which is near-optimal, with a high probability in the same network state.
The authors also showed that Fresh-CSMA with a realistic setting performs comparably to the max-weight policy.
Nevertheless, despite the use of backoff mechanisms in both CSMA and UORA networks, the analytical framework developed for the AoI in CSMA networks cannot be directly applied to our work. This is due to the fundamental differences in the transmission processes between CSMA and UORA networks.

Optimizing the AoI performance in legacy WiFi networks has garnered increasing attention in recent literature. For instance, the authors in ____ developed an optimization strategy based on queuing analysis of AoI in WiFi networks. Additionally, a deep learning approach for channel condition estimation aimed at reducing AoI in WiFi networks was introduced in ____. Furthermore, several studies, including ____, have developed and implemented AoI-based transmission schemes by adapting legacy IEEE 802.11 standards. Specifically, ____ describes the implementation of an application layer middleware designed to tailor IEEE 802.11 networks to the requirements of time-sensitive devices. In ____, the authors proposed an AoI-optimized protocol stack that significantly enhances the AoI performance of WiFi devices. Nevertheless, these studies did not address the most recent updates to the IEEE 802.11 standards.