In this section, we aim to answer the following questions through simulation experiments: 
(i). What is the optimal penalty factor formulation for incorporating domain knowledge from LLMs into Lasso? (ii). Is LLM-Lasso robust to adversarial datasets where the features provided to the LLM are uninformative with respect to the data, or where the data itself is misaligned with the features presented?
% some implementation details can go into the appendix to save space.
\subsection{Penalty Factor Form Simulations}\label{subsec:penalty_fac}
We run simulations to find the adequate form of penalty factors. Based on the simulations, we use the inverse importance penalty factors to compare the LLM-Lasso to the baseline models. We defer the details to Appendix \ref{appdx:sim}. 
\subsection{Adversarial Simulations}\label{subsec:adversarial}
\begin{figure}[H]
    \centering    \includegraphics[width=1.05\linewidth]{fig_new/adversarial.png}
    \vspace{-3em}
    \scriptsize
    \caption{Adversarial simulation experiment using the DLBCL vs. MCL dataset}
    \label{fig:simulations_test_error2}
\end{figure}
To showcase the robustness of our method in scenarios in which the LLM fails to produce meaningful results, we perform adversarial data corruption simulations.
As a base dataset, we use the myeloid cell leukemia (MCL) vs. diffuse large B-cell lymphoma (DLBCL) task from the Lymphoma dataset (Table \ref{tab:data}).
Of the 1592 gene features, we select the 800 most relevant based on presence in documents retrieved from the \href{https://www.omim.org}{OMIM} (Online Mendelian Inheritance in Man) knowledge base (see Section \ref{sec:omim-knowledge-base}).
We replace those genes with random base64 strings, ensuring via OMIM that the strings are not real gene names (see Figure \ref{fig:adversarial-pipeline}).

We perform classification via LLM-Lasso and LLM-Score, as described in Section \ref{sec:large-scale-experiments}, using the GPT-4o model.
Both methods are given the corrupted gene name list.
For illustrative purposes, we also include a random feature selection baseline.
The resulting misclassification error and AUROC plots can be found in Figure \ref{fig:simulations_test_error}.
Though half the genes names given to the LLM are corrupted, the accuracy of LLM-Lasso remains comparable to Lasso, whereas LLM-Score performs noticeably worse than random feature selection.
We observe that for both LLM-Lasso and LLM-Score, the LLM analysis of the corrupted genes is heavily based on hallucinations, examples of which are in Figure \ref{fig:adversarial-hallucinations}.
LLM-Lasso, however, remains robust to the corruptions, while the accuracy of LLM-Score degrades substantially.

\begin{figure}[htbp]
    \centering
    \includegraphics[width=0.95\linewidth]{fig_new/adversarial_pipelie.png}
    \scriptsize
    \caption{Gene name corruption for adversarial simulations.}
    \label{fig:adversarial-pipeline}
\end{figure}

\begin{figure}[htbp]
    \centering
    \begin{tcolorbox}[
        on line, colframe=darkgray,colback=pink,
        boxrule=0.8pt,arc=10pt,boxsep=0pt,left=6pt,
        right=6pt,top=6pt,bottom=6pt
    ]  
    \scriptsize **Z8ED**: 3.0
    
    Reasoning: Possible weak connection to cellular proliferation processes affection some cancers.
    \end{tcolorbox}
    
    \vspace{0.5em}
    \begin{tcolorbox}[
        on line, colframe=darkgray,colback=pink,
        boxrule=0.8pt,arc=10pt,boxsep=0pt,left=6pt,
        right=6pt,top=6pt,bottom=6pt
    ]  
    \scriptsize **PC6LOW**: 0.5
    
    Reasoning: PC6LOW is involved in cell cycle regulation, which can be relevant for distinguishing between DLBCL and MCL.
    \end{tcolorbox}
    
    \vspace{-0.5em}
    \caption{GPT-4o hallucination for corrupted gene names: an LLM-Lasso penalty factor, followed by an LLM-Score importance score.
    Even though both genes, \texttt{Z8ED} and \texttt{PC6LOW} are fake, the LLM hallucinates justification for their relevance to the task.}
    \label{fig:adversarial-hallucinations}
\end{figure}