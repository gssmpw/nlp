The LLM-Lasso is a simple, tunable model that incorporates domain-specific knowledge from LLMs and outperforms state-of-the-art feature selection models. The LLM-Lasso achieves superior performance with a small number of features, not only improving predictive performance compared to baseline models but also providing clues to identify important features. The LLM-Lasso protects against potential inaccuracies or hallucinations from the LLM by tuning hyperparameters, as with the case with the FL experiment. Furthermore, the cross-validated power of the hyperparameter $\mathcal{I}^{-\eta}$ allows us to tune the extent to which the LLM-Lasso relies on the penalty factors provided by the LLM.

In the experiments conducted in this particular paper, the identified genes had biomedical significance, suggesting that the LLM-Lasso could provide important clues that can lead to novel discoveries or confirm known feature-target relationships in biomedicine or in any field. More empirical investigations are needed to examine how well the LLM-Lasso prioritizes features.  

The implementation of the full LLM-Lasso pipeline is made available at \href{https://github.com/pilancilab/LLM-Lasso}{https://github.com/pilancilab/LLM-Lasso}.
% \textcolor{red}{[TODO: Add drawbacks and outlooks.]}