
\section{Limitations} \label{sec:limit}
Our work makes use of LoRA fine-tuning on a $8B$ parameter local MLLM, which necessitates significant computational resources and poses a common shortage in LLM-based applications.
Additionally, we relied on an online API for distillation, introducing extra costs when initiating projects from scratch.
Despite these efforts, the local MLLM struggled to effectively handle all three aspects simultaneously, likely stems from the inadequate capacity of the vision encoder within the MLLM.
Furthermore, the ranking performance could potentially be enhanced by incorporating specific constraints, such as a ranking loss function, although this would require more computational resources and larger batch sizes.

\section{Ethics Statement} \label{sec:ethics}
Considering that our method employs an MLLM embedding an LLM and distills capabilities from an online MLLM service, there is a potential risk of generating harmful content.
However, we rely on the online MLLM's robust safeguards to prevent harmful outputs from normal prompts, significantly mitigating such risks.
In the datasets used, fewer than $50$ images are labeled as Not Safe For Work (NSFW), but these images were deemed acceptable by human judgment and rejected only because of the stringent filtering of the \textit{Gemini} API.
For these cases, \textit{GPT-4o} was leveraged as a safe and effective alternative, ensuring high-quality content generation while preserving the integrity and applicability of the research.

\section{Conclusion} \label{sec:conclusion}
This paper presents M3-AGIQA, a comprehensive framework for assessing AI-generated image quality via a multimodal, multi-round, and multi-aspect approach.
By distilling capabilities from an online MLLM to a local model, M3-AGIQA aligns closely with human perception in quality, correspondence, and authenticity.
Experimental results demonstrate superior performance compared with state-of-the-art methods.
While computational challenges remain, M3-AGIQA sets the stage for future research in efficient AGI quality assessment for broader generalization and improved computational efficiency.