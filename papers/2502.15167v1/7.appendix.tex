
\section{Supplemental Materials}
\definecolor{positive}{RGB}{0,204,0}
\definecolor{negative}{RGB}{234,107,102}

\subsection{Prompts}
\subsubsection{Interaction with Online MLLM Service}
This is the prompt template used for retrieving the image descriptions on the three aspects from an online MLLM. The ``alignment'' aspect in the template is equivalent to ``correspondence''. 
\begin{mdframed}[linecolor=black,linewidth=0.5pt,roundcorner=10pt]
\textbf{Description Prompt Template}

\textbf{User:} Analyze this image generated from the text prompt: `\{\textit{prompt}\}'.

The quality score is \{\textit{mos\_q}\}, alignment score is \{\textit{mos\_a}\}, authenticity score is \{\textit{mos\_au}\}. Scores are between \{\textit{min}\} and \{\textit{max}\}, with higher scores being better.

Provide your assessment in JSON format with the following keys:

quality\_explanation: Describe the overall quality of the image, considering aspects such as composition, clarity, and any noticeable flaws.

alignment\_explanation: Explain how well the image reflects the elements and intent of the provided prompt. Be specific about which aspects of the prompt are successfully conveyed and which might be missing or misinterpreted.

authenticity\_explanation: Discuss how closely the image resembles real artworks. Highlight any parts of the image that appear non-real or artificial.

Do not include the numerical scores in your response.

**Example:**

\{

  ``quality\_explanation'': ``The overall quality of the image is quite high. The composition is balanced, with the boat centered and leading the viewer's eye towards the bridge and beyond. The clarity is excellent, with sharp details on the boat, water, and surrounding cliffs. The lighting is dramatic and enhances the overall atmosphere of the scene. However, there are some noticeable flaws, such as the overly saturated colors, which can detract from the natural feel of the image.'',
  
  ``alignment\_explanation'': ``The image closely aligns with the prompt 'bridge over a body of water with a boat in the water.' The bridge is prominently featured, and the boat is clearly visible in the water. The scene captures the essential elements of the prompt well. However, some additional details, such as the type of bridge or the style of the boat, could have been more specific to better reflect the intent of the prompt.'',
  
  ``authenticity\_explanation'': ``The image has a surreal, almost dreamlike quality, which makes it less authentic as a representation of a real-world scene. The colors are highly saturated and the lighting effects are dramatic, which enhances the artistic feel but reduces the realism. The boat and the bridge look more like artistic renditions rather than actual structures, and the overall scene feels more like a digital artwork or a scene from a fantasy world rather than a photograph of a real place.''
  
\}
\end{mdframed}

\subsubsection{One-round Conversation}
This is the prompt template for the configuration without the image description (w/o ID) mentioned in our experiments.
\begin{mdframed}[linecolor=black,linewidth=0.5pt,roundcorner=10pt]
\textbf{One-round Conversation Prompt Template (Qual.)}

\textbf{User:} Take a close look at this AI-generated image and tell me everything you can about what you see. It's original prompt is: `\{\textit{prompt}\}'. Could you please help to rate the image based on it's overall quality? Please give me a result from [\textit{bad, poor, fair, good, excellent}].
Please just output one word from the list.
\end{mdframed}

\begin{mdframed}[linecolor=black,linewidth=0.5pt,roundcorner=10pt]
    \textbf{One-round Conversation Prompt Template (Corr.)}
    
    \textbf{User:} Take a close look at this AI-generated image and tell me how well it aligns with the elements and intent of the original prompt: `\{\textit{prompt}\}'. Could you please help to rate the image based on its correspondence with the prompt? Please give me a result from [\textit{bad, poor, fair, good, excellent}].
    Please just output one word from the list.
    \end{mdframed}

\begin{mdframed}[linecolor=black,linewidth=0.5pt,roundcorner=10pt]
    \textbf{One-round Conversation Prompt Template (Auth.)}
    
    \textbf{User:} Take a close look at this AI-generated image and evaluate how authentic it appears. Does it resemble real artworks or scenes? Highlight any parts that seem artificial or non-realistic. Could you please help to rate the image based on its authenticity? Please give me a result from [\textit{bad, poor, fair, good, excellent}].
    Please just output one word from the list.
    \end{mdframed}

\subsubsection{Multi-round Conversation}
The prompt template is for fine-tuning the local MLLM with the response from the online MLLM service, and is also used during the inference stage.
\begin{mdframed}[linecolor=black,linewidth=0.5pt,roundcorner=10pt]
\textbf{Multi-round Conversation Prompt Template (Qual.)}

\textbf{User:} Please closely examine this AI-generated image and provide a detailed analysis of its content and quality. The original prompt for the image was: `\{\textit{prompt}\}' What can you deduce about the image based on this prompt, and how would you assess its overall quality?

\textbf{Assistant:} \{\textit{response\_0}\}

\textbf{User:} Based on your analysis, could you please rate the image's overall quality? Choose one word from the following list: [\textit{bad, poor, fair, good, excellent}], where the words range from low to high quality.

\textbf{Assistant:} \{\textit{response\_1}\}
\end{mdframed}

\begin{mdframed}[linecolor=black,linewidth=0.5pt,roundcorner=10pt]
    \textbf{Multi-round Conversation Prompt Template (Corr.)}
    
    \textbf{User:} Please closely examine this AI-generated image and provide a detailed analysis of how well it aligns with the original prompt: `\{\textit{prompt}\}'.
    
    \textbf{Assistant:} \{\textit{response\_0}\}
    
    \textbf{User:} Based on your analysis, could you please rate the image's alignment with its original prompt? Choose one word from the following list: [\textit{bad, poor, fair, good, excellent}], where the words indicate the degree of alignment from low to high.
    
    \textbf{Assistant:} \{\textit{response\_1}\}
    \end{mdframed}

\begin{mdframed}[linecolor=black,linewidth=0.5pt,roundcorner=10pt]
    \textbf{Multi-round Conversation Prompt Template (Auth.)}
    
    \textbf{User:} Please closely examine this AI-generated image and provide a detailed analysis of its authenticity. The original prompt for the image was: `\{\textit{prompt}\}'. How closely does the image resemble real artworks or scenes? Highlight any parts of the image that appear non-real or artificial.
    
    \textbf{Assistant:} \{\textit{response\_0}\}
    
    \textbf{User:} Based on your analysis, could you please rate the image's authenticity? Choose one word from the following list: [\textit{bad, poor, fair, good, excellent}], where the words indicate the degree of authenticity from low to high.
    
    \textbf{Assistant:} \{\textit{response\_1}\}
    \end{mdframed}

Notice that during fine-tuning (Init Description stage), the \textit{repsonse\_0} is the response from online MLLM service and \textit{response\_1} is the ground truth by categorize the MOS to the five string labels.
While during the Inference stage, both \textit{response\_0} and \textit{response\_1} are generated by the fine-tuned local MLLM.

\subsection{Cases}
\subsubsection{Multi-Aspect Responses}
To illustrate the effectiveness of the distillation, we selected two images generated using the same prompt but exhibiting different quality levels, as shown in Table~\ref{tab:multiaspect}.
The responses from both the online MLLM service and the local fine-tuned MLLMs were analyzed. 
Highlighted phrases such as ``high quality'', ``perfectly aligns'', and ``does not resemble real artworks'' in the table demonstrate that the local fine-tuned MLLMs successfully learned to evaluate the images across multiple aspects, reflecting the mechanism transferred from the online MLLM service.
\begin{table*}
    \centering
    \caption{Multi-Aspect responses from online MLLM service and local fine-tuned MLLM for the image generated by prompt \textit{an aerial photo of a sandy island in the ocean}}\label{tab:multiaspect}
    \begin{tabular}{c m{0.8cm} m{5.2cm}m{5.2cm}}
        \toprule
        Image & Aspect & ID from online MLLM service & ID from fine-tuned local MLLM \\
        \midrule 
        \multirow{3}{*}{
        \begin{minipage}{.2\textwidth}
            \includegraphics[width=\linewidth]{imgs/case_study/2353.png}
        \end{minipage}} &
        Qual. & The image boasts \textcolor{positive}{\textbf{high quality with a clear, well-composed aerial view}}. The colors are \textcolor{positive}{\textbf{vibrant}}, and the textures of the sand, water, and vegetation are \textcolor{positive}{\textbf{well-defined}}. There are \textcolor{positive}{\textbf{no noticeable flaws or artifacts}}, contributing to a realistic and \textcolor{positive}{\textbf{visually appealing}} image. &
        The image boasts \textcolor{positive}{\textbf{excellent clarity and composition}}. The aerial perspective is well-executed, providing a \textcolor{positive}{\textbf{clear view}} of the island's shape and the surrounding waters. The lighting is \textcolor{positive}{\textbf{natural and flattering}}, highlighting the textures of the sand and foliage. Overall, the image is of \textcolor{positive}{\textbf{high quality}}.
        \\
        \cline{2-4}
        & Corr. & The image \textcolor{positive}{\textbf{aligns perfectly}} with the prompt. It depicts a sandy island, partially covered in vegetation, situated in a vast expanse of blue ocean. The aerial perspective \textcolor{positive}{\textbf{accurately reflects}} the ``aerial photo'' aspect of the prompt. The image successfully encapsulates \textcolor{positive}{\textbf{all elements}} of the prompt. &
        The image \textcolor{positive}{\textbf{perfectly aligns}} with the prompt. It \textcolor{positive}{\textbf{clearly depicts}} a sandy island in the ocean from an aerial perspective. \textcolor{positive}{\textbf{All elements}} of the prompt are present and \textcolor{positive}{\textbf{accurately represented}}. \\
        \cline{2-4}
        & Auth. & The image \textcolor{positive}{\textbf{closely resembles a real aerial photograph}}. The colors, while vibrant, remain within a \textcolor{positive}{\textbf{natural palette}}, and the lighting and shadows appear realistic. The textures and patterns of the sand, water, and vegetation contribute to the image's authenticity. There are \textcolor{positive}{\textbf{no elements that suggest artificiality or manipulation}}. & 
        The image \textcolor{positive}{\textbf{closely resembles a real aerial photograph}} of a tropical island. The textures of the sand, water, and vegetation appear \textcolor{positive}{\textbf{natural and realistic}}. There are \textcolor{positive}{\textbf{no obvious signs of digital manipulation or artificial elements that would detract from its authenticity}}. \\
        \midrule
        \multirow{3}{*}{\begin{minipage}{.2\textwidth}
            \includegraphics[width=\linewidth]{imgs/case_study/2355.png}
        \end{minipage}} &
        Qual. & The image quality is \textcolor{negative}{\textbf{relatively low}}. The composition is \textcolor{negative}{\textbf{poorly}} defined, lacking a clear focal point.  While the colors are vibrant, they are \textcolor{negative}{\textbf{heavily blurred and lack detail}}, making the overall image appear \textcolor{negative}{\textbf{hazy and unrealistic}}. & The overall quality of the image is \textcolor{negative}{\textbf{quite low}}. While it attempts to depict an aerial view of a sandy island in the ocean, it \textcolor{negative}{\textbf{lacks clarity and detail}}. The colors are overly saturated and blend together, creating a \textcolor{negative}{\textbf{hazy, unrealistic effect}}. The shapes of the landforms are \textcolor{negative}{\textbf{indistinct and lack definition}}. \\
        \cline{2-4}
        & Corr. & The image \textbf{partially reflects the prompt}. It features a sandy area surrounded by blue, which could be interpreted as an island in the ocean from an aerial perspective. However, the \textcolor{negative}{\textbf{layout and shapes are ambiguous}}, and it's difficult to definitively identify the scene as an island. The \textcolor{negative}{\textbf{lack of clarity}} hinders a stronger alignment with the prompt. & 
        The image \textbf{partially aligns with the prompt}. It does depict a body of water and what could be interpreted as a sandy island in the distance. However, the aerial perspective is \textcolor{negative}{\textbf{not clearly conveyed}}, and the'sandy island' appears more as a \textcolor{negative}{\textbf{vague shape}} rather than a defined landmass. The \textcolor{negative}{\textbf{lack of detail}} makes it difficult to confirm the presence of sand. \\
        \cline{2-4}
        & Auth. & The image \textcolor{negative}{\textbf{does not resemble real artwork or a photograph}}. The excessive \textbf{blurring, unrealistic} color gradients, and \textcolor{negative}{\textbf{lack of detail}} give it a \textcolor{negative}{\textbf{heavily artificial and digitally manipulated feel}}. It lacks the characteristics of a genuine aerial photograph. &
        The image \textcolor{negative}{\textbf{does not resemble real artworks or photographs}}. The excessive \textcolor{negative}{\textbf{blurring, unrealistic color palette}}, and \textbf{lack of detail} contribute to an \textcolor{negative}{\textbf{artificial and almost abstract appearance}}. It \textcolor{negative}{\textbf{lacks the realism}} and texture expected in a depiction of a natural landscape. \\
        \bottomrule
    \end{tabular}
\end{table*}
\subsubsection{Good and Bad Cases}
\begin{table*}
    \centering
    \caption{Examples showcasing the top three best-performing cases and the bottom three worst-performing cases on dataset AGIQA-3k~\cite{li2023agiqa} quality aspect.}\label{tab:badngoodcases}
    \begin{tabular}{m{2cm}c m{4cm}m{4cm}cc}
        \toprule
        Prompt &
        Image & ID from online MLLM service & ID from fine-tuned local MLLM & label & predicted \\
        \midrule
        photograph of beautiful landscape, baroque style &
        \begin{minipage}{.2\textwidth}
            \includegraphics[width=\linewidth]{imgs/case_study/midjourney_lowstep_109.jpg}
        \end{minipage} & 
        The image quality is \textcolor{positive}{\textbf{good}}, featuring a \textcolor{positive}{\textbf{clear and well-composed}} scene. The building is sharply focused, and the reflection in the water is \textcolor{positive}{\textbf{well-rendered}}. The soft, diffused lighting contributes to a serene atmosphere. However, the overall image appears \textcolor{negative}{\textbf{somewhat blurry, lacking fine detail}}, particularly in the foliage and the foreground. & 
        The overall quality of the image is \textcolor{positive}{\textbf{good}}. The composition is \textcolor{positive}{\textbf{well-balanced}}, with the grandeur of the baroque-style building drawing the viewer's eye. The use of a symmetrical path leading to the building creates a sense of depth and perspective. However, the image is \textcolor{negative}{\textbf{somewhat blurry}}, particularly in the background, which detracts from the sharpness and detail. &
        3.0 & 3.0015\\

        \midrule
        an emo portrait painting &
        \begin{minipage}{.2\textwidth}
            \includegraphics[width=\linewidth]{imgs/case_study/midjourney_lowstep_032.jpg}
        \end{minipage} & 
        The image boasts \textcolor{positive}{\textbf{high quality}}, especially in its technical execution. The portrait showcases \textcolor{positive}{\textbf{smooth blending}}, creating a \textcolor{positive}{\textbf{realistic}} portrayal of skin and hair. The use of light and shadow adds depth and dimension to the face. However, the \textcolor{negative}{\textbf{background lacks detail}}, appearing \textcolor{negative}{\textbf{blurry and undefined}}, which slightly detracts from the overall composition. &
        The image quality is \textcolor{positive}{\textbf{good}} for a digital painting. The brushstrokes are visible and give a sense of texture to the hair and skin. The lighting is \textcolor{positive}{\textbf{well-executed}}, creating a moody atmosphere. However, the face appears \textcolor{negative}{\textbf{slightly blurry}}, particularly around the eyes and mouth, which detracts from the overall clarity. &
        2.9417 & 2.9434 \\
        
        \midrule
        a masonic illuminati waiting room with a couch, soft lighting, abstract style &
        \begin{minipage}{.2\textwidth}
            \includegraphics[width=\linewidth]{imgs/case_study/AttnGAN_normal_218.jpg}
        \end{minipage} & 
        The image is of \textcolor{negative}{\textbf{low quality}}, appearing more as an abstract composition than a coherent scene. It \textcolor{negative}{\textbf{lacks clarity}}, with \textcolor{negative}{\textbf{blurry and indistinct}} elements. The lighting, while potentially aiming for `soft,' is \textcolor{negative}{\textbf{muddled}} and doesn't contribute to a cohesive atmosphere. & 
        The image is of \textcolor{negative}{\textbf{low quality}}. It is \textcolor{negative}{\textbf{blurry}}, \textcolor{negative}{\textbf{lacks clear shapes}}, and the colors are \textcolor{negative}{\textbf{muddled}}. There is no discernible composition, and the overall impression is abstract and unclear. & 
        0.9708 & 0.9736 \\
        
        \midrule
        the large cresting waves & 
        \begin{minipage}{.2\textwidth}
            \includegraphics[width=\linewidth]{imgs/case_study/sd1.5_normal_027.jpg}
        \end{minipage} & 
        The image quality is low due to the \textcolor{negative}{\textbf{unrealistic}}, \textcolor{negative}{\textbf{almost neon green color}} of the water. The composition is simple, focusing solely on the wave, and \textcolor{negative}{\textbf{lacks depth}}. The spray of the wave is well-detailed, but the overall image feels \textcolor{negative}{\textbf{artificial}}. & 
        The image quality is \textcolor{positive}{\textbf{decent}} but not exceptional. The wave itself is well-rendered with a \textcolor{positive}{\textbf{good sense of movement and power}}. However, the composition is somewhat \textcolor{negative}{\textbf{simplistic}}, focusing solely on the wave without much context or background detail. The color palette is \textcolor{negative}{\textbf{limited}}, with the green of the wave standing out against the gray sky and water. &
        1.6788 & 2.7090\\

        \midrule
        a parade of disconnected images : a tragic supernova, hyper detail, HDR lighting & 
        \begin{minipage}{.2\textwidth}
            \includegraphics[width=\linewidth]{imgs/case_study/sd1.5_highcorr_259.jpg}
        \end{minipage} & 
        The image is of \textcolor{positive}{\textbf{high quality}} in terms of clarity and detail. The colors are \textcolor{positive}{\textbf{vibrant}} and the textures are \textcolor{positive}{\textbf{intricate}}. The overall composition, however, \textcolor{negative}{\textbf{lacks a clear focal point}} and can feel overwhelming due to the lack of negative space. &
        The image boasts \textcolor{positive}{\textbf{high quality}} in terms of clarity and detail. The textures are intricate, and the lighting effects are \textcolor{positive}{\textbf{well-executed}}, creating a sense of depth and dynamism. The composition, however, is somewhat lacking. While the image is visually striking, it \textcolor{negative}{\textbf{lacks a clear focal point}} and feels somewhat \textcolor{negative}{\textbf{chaotic}}. & 
        1.2288 & 2.5059 \\

        \midrule
        a very drunk person named aaron, sci-fi style & 
        \begin{minipage}{.2\textwidth}
            \includegraphics[width=\linewidth]{imgs/case_study/DALLE2_normal_086.jpg}
        \end{minipage} & 
        The image quality is \textcolor{negative}{\textbf{quite poor}}. It appears to be a poorly lit photograph with a strong blue color cast, making it \textcolor{negative}{\textbf{difficult to discern details}}. The image is \textcolor{negative}{\textbf{blurry}}, particularly in the background, further reducing the overall clarity. & 
        The image quality is \textcolor{negative}{\textbf{quite low}}. It appears to be a \textcolor{negative}{\textbf{blurry}}, poorly lit photograph rather than a well-composed image. The lighting is \textcolor{negative}{\textbf{harsh and uneven}}, creating an unflattering effect on the subject's face. &
        3.7735 & 2.4688\\
        \bottomrule
    \end{tabular}
\end{table*}

We analyzed the cases where our method's predictions were most and least aligned with the ground truth by sampling the top three aligned and the bottom three misaligned instances as illustrated in Table~\ref{tab:badngoodcases}. The well-aligned cases shared similar tones in their descriptions, indicating close agreement between the predicted and actual image quality assessments.

Conversely, among the misaligned examples, the fourth case illustrates a significant discrepancy. The ground truth description labeled the image as having an ``unrealistic, almost neon green color'' and an ``artificial'' feel, indicating poor quality. However, the predicted description portrayed the image more positively, noting ``decent'' quality with a ``sense of movement and power.'' This led to a prediction of ``good'' ($2.7090$) in contrast to the actual ``poor'' ($1.6788$) rating, highlighting a fundamental difference in interpretation that affected the accuracy of our model's prediction.

In the fifth case, the ground truth rated the image as ``poor'' ($1.7111$), yet the description provided by the online MLLM service and the local fine-tuned MLLM seems to be more positive with terms like ``high quality'' and ``vibrant''.
As a result, the prediction is ``fair'' ($2.5059$), which misaligned with the ground truth but aligned well with the overly positive image descriptions provided.
This case testifies the importance for quality of image descriptions on the other side, if the image descriptions used for fine-tuning is more precisely aligned with the human perceptual evaluation of ``poor'' ($1.2288$), the prediction would have been better.
In this case, a better description could be like:
\begin{mdframed}[linecolor=black,linewidth=0.5pt,roundcorner=10pt]
This image's low rating likely stems from its overwhelming brightness and chaotic composition, which obscures intended details and focus. Although the prompt calls for a ``tragic supernova'' with ``hyper detail'' and ``HDR lighting'', the result lacks a clear structure or focal point, making interpretation challenging. Intense yellow and blue hues dominate without balance, creating visual confusion rather than conveying the intended sense of tragedy.
\end{mdframed}

And the last bad case looks just on the contrary to the fifth case, the image descriptions from both online MLLM service and local fine-tuned MLLM are all negative which can be concluded as ``poor'', while the ground truth is ``good'' ($3.7735$), and the predicted score ``fair'' ($2.4688$) is intended to be aligned with the ground truth. A more fitting description considering the actual quality of the image would be:
\begin{mdframed}[linecolor=black,linewidth=0.5pt,roundcorner=10pt]
The image's quality is good due to the successful portrayal of a "very drunk person" in a stylized, sci-fi aesthetic, capturing the expression and mood effectively. The person’s face shows a blurred, exaggerated expression, possibly conveying intoxication, and the sci-fi lighting adds an intriguing, futuristic element. However, the lack of clarity, particularly in the person’s features and the background, may detract from the overall quality. The blurred, blue-tinged lighting creates an atmospheric effect but reduces sharpness, making it difficult to connect fully with the subject. This blend of expressive emotion and sci-fi styling gives the image a distinctive feel.
\end{mdframed}
