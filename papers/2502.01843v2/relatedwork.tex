\section{Related works}
\label{sec:related works}
%%%%%%%%%%%%%%%%%%%%%%%%%%%%%%%%%%%%%%%%%%%%%%%%%%%%%%%%%%%%%%%%%%%%%%%%%%%%%%%
%%%%%%%%%%%%%%%%%%%%%%%%%%%%%%%%%%%%%%%%%%%%%%%%%%%%%%%%%%%%%%%%%%%%%%%%%%%%%%%%%%%%%%%%%%%%%%%%%%%%%%%%%%%%%%%%%%%%%%%%

State-dependent routing problems have been extensively studied in the literature (see, e.g., \cite{argon2009dynamic} for an overview). A notable setting where simple structures for optimal policies have been successfully identified involves homogeneous stations. In such cases, the ``Shortest Delay Routing'' (SDR) policy has been demonstrated to be optimal under various conditions \cite{winston1977optimality, weber1978optimal}.
However, \cite{whitt1986deciding} showed also several trivial cases in which this policy is not optimal.
For the more complex case of routing problems that do not assume station homogeneity, \cite{foschini1977heavy} showed the asymptotical optimality of the JSED policy in the heavy traffic case. When the cost function is a convex function of occupation, \cite{mandelbaum2004scheduling} proves the optimality under heavy traffic of a policy which chooses at each step the greedy action minimizing $c_i(s_i) \mu_i$, i.e., the immediate cost multiplied by the processing speed of the queue.
\cite{krishnanJoiningRightQueue1987} defines a near-optimum queue-assignment rule that offers a generalization of the shortest-queue rule to the case of dissimilar queues to minimize the average sojourn time. \cite{buyukkoc1985cmu} provides an optimal policy when $N$ queues compete for a single server (as in call centers). Finally, \cite{stolyar2005optimal} proposes an optimal solution in the case of non-linear cost functions. 
\begin{comment}
\paragraph{Queues with admission cost (pricing)}
\cite{tandra2004join}

\cite{dubeDifferentialJoinPrices2002}

\cite{bodasLoadBalancingRouting2011}

\cite{bradfordPricingRoutingIncentive1996}
    
\end{comment}
%The Lyapunov drift-plus-penalty optimization is a popular technique that can be applied to queueing networks and other stochastic systems and has been introduced in \cite{neely2010queue}. Due to its simplicity and efficiency it has been applied in many scenarios, such as routing, IoT and adaptive video streaming. 
To the best of the authors' knowledge, despite being a problem of clear practical interest, the constrained version of the load balancing problem has not been tackled so far in the literature. 

%%%%%%%%%%%%%%%%%%%%%%%%%%%%%%%%%%%%%%%%%%%%%%%%%%%%%%%%%%%%%%%%%%%%%%%%%%%%%%%%%%%%%%%%%%%%%%%%%%%%%%%%%%%%%%%%%%%%%%%%
%%%%%%%%%%%%%%%%%%%%%%%%%%%%%%%%%%%%%%%%%%%%%%%%%%%%%%%%%%%%%%%%%%%%%%%%%%%%%%%