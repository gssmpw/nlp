\section{Related Work}
\label{sec:relwork}
Our system draws on many areas of prior work, including substructural type systems and reference
capabilities, but those were discussed earlier.
We focus this section not on an exhaustive explanation of the many sources of inspiration
for this work (largely covered by related work discussions of the previously-cited papers),
but instead on comparing our work against other approaches to ensuring that actors (or active objects) and
processes handle specific message sequences and send compatible message sequences to others.
%

A classic point of reference for this is Nierstrasz's work____ on \emph{regular types},
assigning finite-state-based specifications of object behaviour, as the sequence of messages accepted and the resultant
responses. His exposition explores the structure of these specifications and informally how they might apply
to objects, but provides no application of these types to a concrete programming language, and focuses on composing
applications by way of one object using another as its sole client.

This focus on the behaviour of the active object itself over time is also nearly dual to ours: regular types
for active objects describe the process behaviour summatively, while in our (source) type system, no
summative language-oriented description of an actor's accepted messages exists! Instead static knowledge
of an actor's accepted language can only be cobbled together from various capabilities.
Active objects, as nicely captured in De Koster et al.'s taxonomy of actor language variants____,
do not permit changing the set of accepted messages over time. Support for such changes is an explicit goal of this work.
Lack of a global type is a trade-off. On one hand, there is no complete closed description of behaviour separate from the code itself.
On the other hand, actors can be specified and checked in isolation, without needing to know a priori how they fit into
a larger system.

Our calculus is pure, but as many popular actor frameworks are in languages with mutable state, and permit sharing
references across actors on the same machine____
The introduction of direct mutable heaps raises classic issues of balancing safety and efficiency of sending messages. 
Deep copying of state is always sound, but highly inefficient, especially when the data being sent is either immutable
or the sender discards its own access to the mutable data after sending.
This has motivated a great deal of work on various flavors of uniqueness and mutation control,
including a great deal of work applying the reference capability notions discussed in the introduction 
to track forms of unique ownership transfer or object immutability____. Some of this work____ has specifically been applied to actor systems.\footnote{Now that the relevant NDAs are long-expired, we can finally say in writing that the context for ____ was the Midori____ derivative of Singularity____, where processes were essentially actors.}
However, incorporating two orthogonal flavors of capabilities --- the actor capabilities proposed here, 
along with the reference capabilities from existing work --- would distract from the main technical development.

An alternative approach to managing state sharing and exchange between actors is the introduction of
specially-shared capsules. De Koster et al.____ introduce \emph{domains}, which
are shared-memory constructs which actors asynchronously request access to (akin to asynchronous mutex acquisition).
Similarly, Cheeseman et al.'s \emph{cowns}____ (named for \emph{concurrent ownership})
are shared resources asked by asynchronously-spawned computations that run only when all required cowns are available at once.
The most closely related work to ours also falls into this group:
Caldwell et al.____ describe a specialized data exchange environment called a \emph{dataspace}, as well as an effect
system describing behaviour over time for actors and data --- effectively dual to our approach of restricting actor behaviours with
capabilities. Similar to Skalka et al.____, their effects give abstract summaries of actor behaviour,
which can then be used to model-check properties of the program (i.e., the effect types for both Skalka et al.'s work and Caldwell et al.'s work
give a model that is verified to soundly abstract the program code).

Session types____, choreographic programming____, and multitier programming____
solve the issue of agreement between the messages accepted by a given process and message sent to it by others in the inverse way
of what we describe here. Instead of analyzing correctness of processes in isolation,
these approaches start with a single program describing all global behaviour, and a compiler generates
code (or in the case of session types, synchronization skeletons) for individual processes via a projection process.
These approaches are generally applied to \emph{synchronous} message passing, though 
Harvey et al.____ and Neykova et al.____ have adapted session types to actors (though
still using a global description of the full orchestration which is then projected for each of a fixed set of roles).

Originating around the same time as classic binary session types____, Nielson and Nielson____ described one of
the earliest flow-sensitive effect systems, for ensuring Concurrent ML (CML) processes sent and recieved \emph{synchronous} messages in an expected order.
This early similarity turned out to be quite relevant, as session types can be recast as an effect system (and vice versa)____.
Thus the idea of applying effect systems --- or an equivalent formalism --- to ensure correct message request ordering is itself far from new.
And the idea of using capabilities to restrict effects is also not new, going back past recent formal connections____
to many earlier expositions of effects____ in an implicit form.

Two aspects of our approach distinguish it from this established body of work.
First, this is to the best of our knowledge the first use of flow sensitive \emph{capabilities} (a form of coeffect____)
to achieve safe ordering. While the relationship to a putative effect system achieving the same ends has not been formally pursued, and this
likely carries some of the downside of capabilities (vs. effects) outlined in our prior work____ (which applied only to
flow-insensitive effect systems), it also carries a unique advantage in system design not discussed in that paper.
While effect systems may obtain greater precision in reasoning about side effects____,
they do so at the cost of needing to ``reconstruct'' behaviour and check compatibility after the fact.
In contrast, our capability formulataion of the problem inverts this checking, and places each actor (and its original allocation site)
in firm control of how other actors interact with it --- the protocols of the self-capabilities an actor sends to others
directly constrain those others' interactions a priori, localizing checks to a far greater extent than in CML or session types --- no direct global accounting of
behaviour is required to exist.

Second --- and a complementary factor in not requiring a global behavioural description to exist --- is our use of an effect system
to track capability \emph{creation}. This differs from most static capability work, which typically assumes all possible capabilities for something
exist upon its creation, and must be passed through the program. Instead we treat actors' retrieval of a new self-reference as a
\emph{dynamic} capability creation --- via effects. While it has long been recognized that both coeffects (e.g., capabilities) and effects are required
to reason about various combinations of independently-interesting static analyses and language semantics, ours is the first
non-trivial interaction we are aware of mixing coeffects and effects \emph{as part of a single solution to a single problem}.


Program logics have also been used for reasoning about correctness of actor messages.
Gordon____ proposed modal assertions of the form $@_l(P)$ inspired by hybrid logic____,
indicating that $P$ was true of the state of the actor at $l$. Thus assertions true at the sending actor
could be packaged in this way, and communicated as part of message invariants; a rely-guarantee-style relation constraining
state evolution worked together with a restriction to only send assertions stable with respect to that relation
ensured the assertions about other actors were never invalidated, but limited what could be proved to assertions over
monotonically-evolving portions of state (though CRDTs have shown this can be made more flexible than it initially appears____).
More recently some program logics (instantiations of \textsc{Iris}____) have tied together
ideas from session types and verification for simple actor languages____. This work handles protocols similar to binary (for Actris)
or multiparty (for Multris) session types, but each message has not only a value but a set of assertions that the recipient may
assume upon receipt of the corresponding value.
This is much more powerful and flexible than Gordon's approach, at the cost of complexity. Gordon's system was prototyped as a library for the automated verification tool Dafny, %
while the modern approaches rely on \textsc{Iris} embedded in an interactive proof assistant.
The key point of comparison to this work is that Actris and Multris do ensure the absence of unhandled messages,
but the specifications for doing so are traditional, essentially-CPS-style session types. Compatibility when delgating
use of a channel to another process is checked via an ad hoc simulation search procedure akin to Milit\~ao et al.'s____.
In contrast, our use of shuffle uncovers a clean algebraic structure underlying most common uses of sharing and resharing
the addresses of other processes, though it is strictly-speaking less general than Actris and Multris, where the correct continuation
of a protocol may depend on the specific value sent.

\begin{credits}
\subsubsection{\ackname}
First and foremost, I owe a great deal of thanks to Gul Agha for his pivotal role in the development
and advocacy of the actor model, which forms the basis for much of my research career (via the work
I am most known for____ and some of the work I am most fond of____);
as well as some of what I most enjoy teaching (using actors when teaching distributed computing).
Agha's work has been the source for a significant amount of my own learning to think about concurrent programs
and their implementation____, and by chance
he has had a direct influence on individual people who have influenced me during time in industry and academia____.
Thanks to Ivan Bestchastnikh, for asking me in 2012 how I could apply reference capabilities to distributed systems. It took 13 years, but I think I’ve finally articulated a good answer,
even if handling faults remains future work.
This work was supported by NSF Award \#CCF-2007582.


%
%
%
%
%
%
%
%
%
%
\end{credits}
%
%
%
%
%
%
%
%
%
\bibliographystyle{splncs04}
%
\begin{thebibliography}{10}
\providecommand{\url}[1]{\texttt{#1}}
\providecommand{\urlprefix}{URL }
\providecommand{\doi}[1]{https://doi.org/#1}

\bibitem{agha1986actors}
Agha, G.: Actors: a model of concurrent computation in distributed systems. MIT
  press (1986)

\bibitem{amtoft1999}
Amtoft, T., Nielson, F., Nielson, H.R.: {Type and Effect Systems: Behaviours
  for Concurrency}. Imperial College Press, London, UK (1999)

\bibitem{areces2007hybrid}
Areces, C., ten Cate, B.: Hybrid logics. In: Studies in Logic and Practical
  Reasoning, vol.~3, pp. 821--868. Elsevier (2007)

\bibitem{bagherzadeh2020actor}
Bagherzadeh, M., Fireman, N., Shawesh, A., Khatchadourian, R.: Actor
  concurrency bugs: a comprehensive study on symptoms, root causes, api usages,
  and differences. Proceedings of the ACM on Programming Languages
  \textbf{4}(OOPSLA),  1--32 (2020)

\bibitem{boer2017survey}
Boer, F.D., Serbanescu, V., H{\"a}hnle, R., Henrio, L., Rochas, J., Din, C.C.,
  Johnsen, E.B., Sirjani, M., Khamespanah, E., Fernandez-Reyes, K., et~al.: A
  survey of active object languages. ACM Computing Surveys (CSUR)
  \textbf{50}(5),  1--39 (2017)

\bibitem{boyapati02}
Boyapati, C., Lee, R., Rinard, M.: {Ownership Types for Safe Programming:
  Preventing Data Races and Deadlocks}. In: {OOPSLA} (2002)

\bibitem{boyapati01}
Boyapati, C., Rinard, M.: {A Parameterized Type System for Race-Free Java
  Programs}. In: {OOPSLA} (2001)

\bibitem{caldwell2024programming}
CALDWELL, S., GARNOCK-JONES, T., FELLEISEN, M.: Programming and reasoning about
  actors that share state. Journal of Functional Programming  \textbf{34}
  (2024). \doi{10.1017/S0956796824000091}

\bibitem{CastegrenW16}
Castegren, E., Wrigstad, T.: Reference capabilities for concurrency control.
  In: 30th European Conference on Object-Oriented Programming, {ECOOP } 2016
  (2016)

\bibitem{CastegrenW17}
Castegren, E., Wrigstad, T.: Relaxed linear references for lock-free
  programming. In: 31st European Conference on Object-Oriented Programming,
  {ECOOP } 2017 (2017)

\bibitem{charousset2014caf}
Charousset, D., Hiesgen, R., Schmidt, T.C.: Caf-the c++ actor framework for
  scalable and resource-efficient applications. In: Proceedings of the 4th
  International Workshop on Programming based on Actors Agents \& Decentralized
  Control. pp. 15--28 (2014)

\bibitem{cheeseman2023concurrency}
Cheeseman, L., Parkinson, M.J., Clebsch, S., Kogias, M., Drossopoulou, S.,
  Chisnall, D., Wrigstad, T., Li{\'e}tar, P.: When concurrency matters:
  Behaviour-oriented concurrency. Proceedings of the ACM on Programming
  Languages  \textbf{7}(OOPSLA2),  1531--1560 (2023)

\bibitem{choudhury2020recovering}
Choudhury, V., Krishnaswami, N.: Recovering purity with comonads and
  capabilities. Proceedings of the ACM on Programming Languages
  \textbf{4}(ICFP),  1--28 (2020)

\bibitem{clebsch2015deny}
Clebsch, S., Drossopoulou, S., Blessing, S., McNeil, A.: Deny capabilities for
  safe, fast actors. In: Proceedings of the 5th International Workshop on
  Programming Based on Actors, Agents, and Decentralized Control. pp. 1--12.
  ACM (2015)

\bibitem{craig2018}
Craig, A., Potanin, A., Groves, L., Aldrich, J.: {Capabilities: Effects for
  Free}. In: {International Conference on Formal Engineering Methods (ICFEM) }
  (2018)

\bibitem{cruz2023formal}
Cruz-Filipe, L., Montesi, F., Peressotti, M.: A formal theory of choreographic
  programming. Journal of Automated Reasoning  \textbf{67}(2), ~21 (2023)

\bibitem{de2015domains}
De~Koster, J., Marr, S., D'Hondt, T., Van~Cutsem, T.: Domains: Safe sharing
  among actors. Science of Computer Programming  \textbf{98},  140--158 (2015)

\bibitem{de201643}
De~Koster, J., Van~Cutsem, T., De~Meuter, W.: 43 years of actors: a taxonomy of
  actor models and their key properties. In: Proceedings of the 6th
  International Workshop on Programming Based on Actors, Agents, and
  Decentralized Control. pp. 31--40 (2016)

\bibitem{de2012domains}
De~Koster, J., Van~Cutsem, T., D'Hondt, T.: Domains: Safe sharing among actors.
  In: Proceedings of the 2nd edition on Programming systems, languages and
  applications based on actors, agents, and decentralized control abstractions.
  pp. 11--22 (2012)

\bibitem{dietl07}
Dietl, W., Drossopoulou, S., M{\"u}ller, P.: {Generic Universe Types}. In:
  ECOOP (2007)

\bibitem{duffyblog}
Duffy, J.:  (2015),
  \url{https://joeduffyblog.com/2015/11/03/blogging-about-midori/}

\bibitem{Fahndrich2006language}
F\"{a}hndrich, M., Aiken, M., Hawblitzel, C., Hodson, O., Hunt, G., Larus,
  J.R., Levi, S.: Language support for fast and reliable message-based
  communication in singularity os. In: Proceedings of the 1st ACM
  SIGOPS/EuroSys European Conference on Computer Systems 2006. pp. 177--190.
  EuroSys '06, ACM (2006). \doi{10.1145/1217935.1217953}

\bibitem{giallorenzo2024choral}
Giallorenzo, S., Montesi, F., Peressotti, M.: Choral: Object-oriented
  choreographic programming. ACM Transactions on Programming Languages and
  Systems  \textbf{46}(1),  1--59 (2024)

\bibitem{giannini2019flexible}
Giannini, P., Servetto, M., Zucca, E., Cone, J.: Flexible recovery of
  uniqueness and immutability. Theoretical Computer Science  \textbf{764},
  145--172 (2019)

\bibitem{gordon2019modal}
Gordon, C.S.: Modal assertions for actor correctness. In: Proceedings of the
  9th ACM SIGPLAN International Workshop on Programming Based on Actors,
  Agents, and Decentralized Control. pp. 11--20 (2019)

\bibitem{gordon2020designing}
Gordon, C.S.: Designing with static capabilities and effects: Use, mention, and
  invariants (pearl). In: 34th European Conference on Object-Oriented
  Programming (ECOOP 2020). Schloss Dagstuhl-Leibniz-Zentrum f{\"u}r Informatik
  (2020)

\bibitem{gordon2021polymorphic}
Gordon, C.S.: Polymorphic iterable sequential effect systems. ACM Transactions
  on Programming Languages and Systems (TOPLAS)  \textbf{43}(1),  1--79 (2021)

\bibitem{gordon2013rely}
Gordon, C.S., Ernst, M.D., Grossman, D.: Rely-guarantee references for
  refinement types over aliased mutable data. In: Proceedings of the 34th ACM
  SIGPLAN Conference on Programming Language Design and Implementation. pp.
  73--84 (2013)

\bibitem{gordon2017verifying}
Gordon, C.S., Ernst, M.D., Grossman, D., Parkinson, M.J.: Verifying invariants
  of lock-free data structures with rely-guarantee and refinement types. ACM
  Transactions on Programming Languages and Systems (TOPLAS)  \textbf{39}(3),
  1--54 (2017)

\bibitem{gordon2012uniqueness}
Gordon, C.S., Parkinson, M.J., Parsons, J., Bromfield, A., Duffy, J.:
  Uniqueness and reference immutability for safe parallelism. In: Proceedings
  of the ACM international conference on Object oriented programming systems
  languages and applications. pp. 21--40 (2012)

\bibitem{gordon2014verifying}
Gordon, C.S.: Verifying Concurrent Programs by Controlling Alias Interference.
  Ph.D. thesis, University of Washington (2014)

\bibitem{haller2012integration}
Haller, P.: On the integration of the actor model in mainstream technologies:
  the scala perspective. In: Proceedings of the 2nd edition on Programming
  systems, languages and applications based on actors, agents, and
  decentralized control abstractions, pp.~1--6 (2012)

\bibitem{harvey2021multiparty}
Harvey, P., Fowler, S., Dardha, O., Gay, S.J.: Multiparty session types for
  safe runtime adaptation in an actor language. In: 35th European Conference on
  Object-Oriented Programming (ECOOP 2021). Schloss Dagstuhl-Leibniz-Zentrum
  f{\"u}r Informatik (2021)

\bibitem{hedden2018comprehensive}
Hedden, B., Zhao, X.: A comprehensive study on bugs in actor systems. In:
  Proceedings of the 47th International Conference on Parallel Processing.
  pp.~1--9 (2018)

\bibitem{henglein2005effect}
Henglein, F., Makholm, H., Niss, H.: Effect types and region-based memory
  management. In: Pierce, B.C. (ed.) Advanced Topics in Types and Programming
  Languages, chap.~3, pp. 87--136. MIT Press (2005)

\bibitem{hewitt1973universal}
Hewitt, C., Bishop, P., Steiger, R.: A universal modular actor formalism for
  artificial intelligence. In: Proceedings of the 3rd international joint
  conference on Artificial intelligence. pp. 235--245 (1973)

\bibitem{hinrichsen2019actris}
Hinrichsen, J.K., Bengtson, J., Krebbers, R.: Actris: Session-type based
  reasoning in separation logic. Proceedings of the ACM on Programming
  Languages  \textbf{4}(POPL),  1--30 (2019)

\bibitem{hinrichsen2024multris}
Hinrichsen, J.K., Jacobs, J., Krebbers, R.: Multris: Functional verification of
  multiparty message passing in separation logic. Proceedings of the ACM on
  Programming Languages  \textbf{8}(OOPSLA2),  1446--1474 (2024)

\bibitem{honda1998language}
Honda, K., Vasconcelos, V.T., Kubo, M.: Language primitives and type discipline
  for structured communication-based programming. In: European Symposium on
  Programming. pp. 122--138. Springer (1998)

\bibitem{Honda2008}
Honda, K., Yoshida, N., Carbone, M.: {Multiparty Asynchronous Session Types}.
  In: Proceedings of the 35th Annual ACM SIGPLAN-SIGACT Symposium on Principles
  of Programming Languages. POPL '08 (2008). \doi{10.1145/1328438.1328472}

\bibitem{Hunt2007singularity}
Hunt, G.C., Larus, J.R.: Singularity: Rethinking the software stack. SIGOPS
  Oper. Syst. Rev.  \textbf{41}(2),  37--49 (Apr 2007).
  \doi{10.1145/1243418.1243424}

\bibitem{jung2018iris}
Jung, R., Krebbers, R., Jourdan, J.H., Bizjak, A., Birkedal, L., Dreyer, D.:
  Iris from the ground up: A modular foundation for higher-order concurrent
  separation logic. Journal of Functional Programming  \textbf{28}, ~e20 (2018)

\bibitem{karmani2009actor}
Karmani, R.K., Shali, A., Agha, G.: Actor frameworks for the jvm platform: a
  comparative analysis. In: Proceedings of the 7th International Conference on
  Principles and Practice of Programming in Java. pp. 11--20 (2009)

\bibitem{lagaillardie2020implementing}
Lagaillardie, N., Neykova, R., Yoshida, N.: Implementing multiparty session
  types in rust. In: International Conference on Coordination Languages and
  Models. pp. 127--136. Springer (2020)

\bibitem{levy1984capability}
Levy, H.M.: Capability-based computer systems. Digital Press (1984)

\bibitem{Marino2009}
Marino, D., Millstein, T.: {A generic type-and-effect system}. In: TLDI (2009)

\bibitem{milano2022flexible}
Milano, M., Turcotti, J., Myers, A.C.: A flexible type system for fearless
  concurrency. In: Proceedings of the 43rd ACM SIGPLAN International Conference
  on Programming Language Design and Implementation. pp. 458--473 (2022)

\bibitem{militao2016composing}
Milit\~ao, F., Aldrich, J., Caires, L.: Composing interfering abstract
  protocols. In: 30th European Conference on Object-Oriented Programming,
  {ECOOP } 2016 (2016). \doi{10.4230/LIPIcs.ECOOP.2016.16}

\bibitem{militao2014rely}
Militao, F., Aldrich, J., Caires, L.: Rely-guarantee protocols. In: ECOOP
  2014--Object-Oriented Programming: 28th European Conference, Uppsala, Sweden,
  July 28--August 1, 2014. Proceedings 28. pp. 334--359. Springer (2014)

\bibitem{miller2006robust}
Miller, M.: Robust composition: Towards a uni ed approach to access control and
  concurrency control. Johns Hopkins University (2006)

\bibitem{miller2005concurrency}
Miller, M.S., Tribble, E.D., Shapiro, J.: Concurrency among strangers:
  Programming in e as plan coordination. In: Trustworthy Global Computing:
  International Symposium, TGC 2005 , Edinburgh, UK, April 7-9, 2005. Revised
  Selected Papers. pp. 195--229. Springer (2005)

\bibitem{neykova2017multiparty}
Neykova, R., Yoshida, N.: Multiparty session actors. Logical Methods in
  Computer Science  \textbf{13} (2017)

\bibitem{nielson1993cml}
Nielson, F., Nielson, H.R.: From cml to process algebras. In: {CONCUR} (1993)

\bibitem{nierstrasz1993regular}
Nierstrasz, O.: Regular types for active objects. In: Proceedings of the Eighth
  Annual Conference on Object-Oriented Programming Systems, Languages, and
  Applications. p. 1–15. OOPSLA '93, Association for Computing Machinery, New
  York, NY, USA (1993). \doi{10.1145/165854.167976},
  \url{https://doi.org/10.1145/165854.167976}

\bibitem{orchard2016effects}
Orchard, D., Yoshida, N.: Effects as sessions, sessions as effects. In:
  Proceedings of the 43rd Annual ACM SIGPLAN-SIGACT Symposium on Principles of
  Programming Languages. pp. 568--581 (2016)

\bibitem{servetto2013balloon}
Servetto, M., Pearce, D.J., Groves, L., Potanin, A.: Balloon types for safe
  parallelisation over arbitrary object graphs. In: Workshop on Determinism and
  Correctness in Parallel Programming (WoDet) (2013)

\bibitem{shapiro2011conflict}
Shapiro, M., Pregui{\c{c}}a, N., Baquero, C., Zawirski, M.: Conflict-free
  replicated data types. In: Stabilization, Safety, and Security of Distributed
  Systems: 13th International Symposium, SSS 2011, Grenoble, France, October
  10-12, 2011. Proceedings 13. pp. 386--400. Springer (2011)

\bibitem{shen2023haschor}
Shen, G., Kashiwa, S., Kuper, L.: Haschor: Functional choreographic programming
  for all (functional pearl). Proceedings of the ACM on Programming Languages
  \textbf{7}(ICFP),  541--565 (2023)

\bibitem{Skalka2008}
Skalka, C.: Types and trace effects for object orientation. Higher-Order and
  Symbolic Computation  \textbf{21}(3) (2008)

\bibitem{skalka2008types}
Skalka, C., Smith, S., Van~Horn, D.: Types and trace effects of higher order
  programs. Journal of Functional Programming  \textbf{18}(2) (2008)

\bibitem{tate13}
Tate, R.: The sequential semantics of producer effect systems. In: {POPL}
  (2013)

\bibitem{swift6drf}
Team, S.L.: {Data Race Safety},
  \url{https://www.swift.org/migration/documentation/swift-6-concurrency-migration-guide/dataracesafety/}

\bibitem{torres2018study}
Torres~Lopez, C., Marr, S., Gonzalez~Boix, E., M{\"o}ssenb{\"o}ck, H.: A study
  of concurrency bugs and advanced development support for actor-based
  programs, pp. 155--185. Springer (2018)

\bibitem{tschantz05}
Tschantz, M.S., Ernst, M.D.: {Javari: Adding Reference Immutability to Java}.
  In: {OOPSLA} (2005). \doi{10.1145/1094811.1094828}

\bibitem{weisenburger2020survey}
Weisenburger, P., Wirth, J., Salvaneschi, G.: A survey of multitier
  programming. ACM Computing Surveys (CSUR)  \textbf{53}(4),  1--35 (2020)

\bibitem{zibin07}
Zibin, Y., Potanin, A., Ali, M., Artzi, S., Kiezun, A., Ernst, M.D.: {Object
  and Reference Immutability Using Java Generics}. In: {ESEC-FSE} (2007).
  \doi{10.1145/1287624.1287637}

\bibitem{zibin10}
Zibin, Y., Potanin, A., Li, P., Ali, M., Ernst, M.D.: {Ownership and
  Immutability in Generic Java}. In: {OOPSLA} (2010).
  \doi{10.1145/1869459.1869509}

\end{thebibliography}
 %