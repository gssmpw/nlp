\documentclass[conference]{IEEEtran}
\IEEEoverridecommandlockouts
% The preceding line is only needed to identify funding in the first footnote. If that is unneeded, please comment it out.
%Template version as of 6/27/2024

\usepackage{cite}
\usepackage{amsmath,amssymb,amsfonts}
\usepackage{graphicx}
\usepackage{textcomp}
\usepackage{xcolor}
\usepackage{subcaption}
\usepackage{algorithm}
\usepackage{algpseudocode}
\usepackage{tikz}
\usepackage{pgfplots}
\usepackage{svg}
\usepackage{breakurl}
\newcommand{\boldred}[1]{\textbf{\textcolor{red}{#1}}}
\def\BibTeX{{\rm B\kern-.05em{\sc i\kern-.025em b}\kern-.08em
    T\kern-.1667em\lower.7ex\hbox{E}\kern-.125emX}}
\begin{document}

\title{TerEffic: Highly Efficient Ternary LLM Inference on FPGA
% {\footnotesize \textsuperscript{*}Note: Sub-titles are not captured for https://ieeexplore.ieee.org  and
% should not be used}
% \thanks{Identify applicable funding agency here. If none, delete this.}
}

% \author{\IEEEauthorblockN{1\textsuperscript{st} Given Name Surname}
% \IEEEauthorblockA{\textit{dept. name of organization (of Aff.)} \\
\author{
    \IEEEauthorblockN{
        Chenyang Yin\IEEEauthorrefmark{1}, 
        Zhenyu Bai\IEEEauthorrefmark{2}, 
        Pranav Venkatram\IEEEauthorrefmark{2}, 
        Shivam Aggarval\IEEEauthorrefmark{2}, 
        Zhaoying Li\IEEEauthorrefmark{2}, 
        and Tulika Mitra\IEEEauthorrefmark{2}
    }
    \IEEEauthorblockA{
        \IEEEauthorrefmark{1}School of Electronic Engineering and Computer Science, Peking University\\
        Email: ycy@stu.pku.edu.cn
    }
    \IEEEauthorblockA{
        \IEEEauthorrefmark{2}School of Computing, National University of Singapore\\
        Email: zhenyu.bai@nus.edu.sg\\
        Email: e0552200@u.nus.edu\\
        Email: shivam@comp.nus.edu.sg\\
        Email: zhaoying@comp.nus.edu.sg\\
        Email: tulika@comp.nus.edu.sg
    }
}
% \textit{name of organization (of Aff.)}\\
% City, Country \\
% email address or ORCID}
% \and
% \IEEEauthorblockN{2\textsuperscript{nd} Given Name Surname}
% \IEEEauthorblockA{\textit{dept. name of organization (of Aff.)} \\
% \textit{name of organization (of Aff.)}\\
% City, Country \\
% email address or ORCID}
% \and
% \IEEEauthorblockN{3\textsuperscript{rd} Given Name Surname}
% \IEEEauthorblockA{\textit{dept. name of organization (of Aff.)} \\
% \textit{name of organization (of Aff.)}\\
% City, Country \\
% email address or ORCID}
% \and
% \IEEEauthorblockN{4\textsuperscript{th} Given Name Surname}
% \IEEEauthorblockA{\textit{dept. name of organization (of Aff.)} \\
% \textit{name of organization (of Aff.)}\\
% City, Country \\
% email address or ORCID}
% \and
% \IEEEauthorblockN{5\textsuperscript{th} Given Name Surname}
% \IEEEauthorblockA{\textit{dept. name of organization (of Aff.)} \\
% \textit{name of organization (of Aff.)}\\
% City, Country \\
% email address or ORCID}
% \and
% \IEEEauthorblockN{6\textsuperscript{th} Given Name Surname}
% \IEEEauthorblockA{\textit{dept. name of organization (of Aff.)} \\
% \textit{name of organization (of Aff.)}\\
% City, Country \\
% email address or ORCID}
% }

\maketitle

\begin{abstract}
Large Language Model (LLM) deployment on edge devices is typically constrained by the need for off-chip memory access, leading to high power consumption and limited throughput. Ternary quantization for LLMs is promising in maintaining model accuracy while reducing memory footprint. However, existing accelerators have not exploited this potential for on-chip inference. We present TerEffic, an FPGA-based accelerator that carefully co-designs memory architecture and computational units to unlock highly efficient LLM inference with fully on-chip execution. Through weight compression, custom computational units, and memory hierarchy optimization, we achieve unprecedented efficiency by eliminating off-chip memory bandwidth bottlenecks. We propose two architectural variants: a fully on-chip design for smaller models and an HBM-assisted design for larger ones. When evaluated on a 370M parameter model with single-batch inference, our on-chip design achieves 12,700 tokens/sec (149× higher than NVIDIA's Jetson Orin Nano) with a power efficiency of 467 tokens/sec/W (19× better than Jetson Orin Nano). The HBM-assisted design provides 521 tokens/sec on a 2.7B parameter model (2× higher than NVIDIA's A100) with 33W power consumption, achieving a power efficiency of 16 tokens/sec/W (8× better than A100). 

\end{abstract}

\begin{IEEEkeywords}
Architecture, FPGA, AI, LLM.
\end{IEEEkeywords}

\section{Introduction}

Large language models (LLMs) have achieved remarkable success in automated math problem solving, particularly through code-generation capabilities integrated with proof assistants~\citep{lean,isabelle,POT,autoformalization,MATH}. Although LLMs excel at generating solution steps and correct answers in algebra and calculus~\citep{math_solving}, their unimodal nature limits performance in plane geometry, where solution depends on both diagram and text~\citep{math_solving}. 

Specialized vision-language models (VLMs) have accordingly been developed for plane geometry problem solving (PGPS)~\citep{geoqa,unigeo,intergps,pgps,GOLD,LANS,geox}. Yet, it remains unclear whether these models genuinely leverage diagrams or rely almost exclusively on textual features. This ambiguity arises because existing PGPS datasets typically embed sufficient geometric details within problem statements, potentially making the vision encoder unnecessary~\citep{GOLD}. \cref{fig:pgps_examples} illustrates example questions from GeoQA and PGPS9K, where solutions can be derived without referencing the diagrams.

\begin{figure}
    \centering
    \begin{subfigure}[t]{.49\linewidth}
        \centering
        \includegraphics[width=\linewidth]{latex/figures/images/geoqa_example.pdf}
        \caption{GeoQA}
        \label{fig:geoqa_example}
    \end{subfigure}
    \begin{subfigure}[t]{.48\linewidth}
        \centering
        \includegraphics[width=\linewidth]{latex/figures/images/pgps_example.pdf}
        \caption{PGPS9K}
        \label{fig:pgps9k_example}
    \end{subfigure}
    \caption{
    Examples of diagram-caption pairs and their solution steps written in formal languages from GeoQA and PGPS9k datasets. In the problem description, the visual geometric premises and numerical variables are highlighted in green and red, respectively. A significant difference in the style of the diagram and formal language can be observable. %, along with the differences in formal languages supported by the corresponding datasets.
    \label{fig:pgps_examples}
    }
\end{figure}



We propose a new benchmark created via a synthetic data engine, which systematically evaluates the ability of VLM vision encoders to recognize geometric premises. Our empirical findings reveal that previously suggested self-supervised learning (SSL) approaches, e.g., vector quantized variataional auto-encoder (VQ-VAE)~\citep{unimath} and masked auto-encoder (MAE)~\citep{scagps,geox}, and widely adopted encoders, e.g., OpenCLIP~\citep{clip} and DinoV2~\citep{dinov2}, struggle to detect geometric features such as perpendicularity and degrees. 

To this end, we propose \geoclip{}, a model pre-trained on a large corpus of synthetic diagram–caption pairs. By varying diagram styles (e.g., color, font size, resolution, line width), \geoclip{} learns robust geometric representations and outperforms prior SSL-based methods on our benchmark. Building on \geoclip{}, we introduce a few-shot domain adaptation technique that efficiently transfers the recognition ability to real-world diagrams. We further combine this domain-adapted GeoCLIP with an LLM, forming a domain-agnostic VLM for solving PGPS tasks in MathVerse~\citep{mathverse}. 
%To accommodate diverse diagram styles and solution formats, we unify the solution program languages across multiple PGPS datasets, ensuring comprehensive evaluation. 

In our experiments on MathVerse~\citep{mathverse}, which encompasses diverse plane geometry tasks and diagram styles, our VLM with a domain-adapted \geoclip{} consistently outperforms both task-specific PGPS models and generalist VLMs. 
% In particular, it achieves higher accuracy on tasks requiring geometric-feature recognition, even when critical numerical measurements are moved from text to diagrams. 
Ablation studies confirm the effectiveness of our domain adaptation strategy, showing improvements in optical character recognition (OCR)-based tasks and robust diagram embeddings across different styles. 
% By unifying the solution program languages of existing datasets and incorporating OCR capability, we enable a single VLM, named \geovlm{}, to handle a broad class of plane geometry problems.

% Contributions
We summarize the contributions as follows:
We propose a novel benchmark for systematically assessing how well vision encoders recognize geometric premises in plane geometry diagrams~(\cref{sec:visual_feature}); We introduce \geoclip{}, a vision encoder capable of accurately detecting visual geometric premises~(\cref{sec:geoclip}), and a few-shot domain adaptation technique that efficiently transfers this capability across different diagram styles (\cref{sec:domain_adaptation});
We show that our VLM, incorporating domain-adapted GeoCLIP, surpasses existing specialized PGPS VLMs and generalist VLMs on the MathVerse benchmark~(\cref{sec:experiments}) and effectively interprets diverse diagram styles~(\cref{sec:abl}).

\iffalse
\begin{itemize}
    \item We propose a novel benchmark for systematically assessing how well vision encoders recognize geometric premises, e.g., perpendicularity and angle measures, in plane geometry diagrams.
	\item We introduce \geoclip{}, a vision encoder capable of accurately detecting visual geometric premises, and a few-shot domain adaptation technique that efficiently transfers this capability across different diagram styles.
	\item We show that our final VLM, incorporating GeoCLIP-DA, effectively interprets diverse diagram styles and achieves state-of-the-art performance on the MathVerse benchmark, surpassing existing specialized PGPS models and generalist VLM models.
\end{itemize}
\fi

\iffalse

Large language models (LLMs) have made significant strides in automated math word problem solving. In particular, their code-generation capabilities combined with proof assistants~\citep{lean,isabelle} help minimize computational errors~\citep{POT}, improve solution precision~\citep{autoformalization}, and offer rigorous feedback and evaluation~\citep{MATH}. Although LLMs excel in generating solution steps and correct answers for algebra and calculus~\citep{math_solving}, their uni-modal nature limits performance in domains like plane geometry, where both diagrams and text are vital.

Plane geometry problem solving (PGPS) tasks typically include diagrams and textual descriptions, requiring solvers to interpret premises from both sources. To facilitate automated solutions for these problems, several studies have introduced formal languages tailored for plane geometry to represent solution steps as a program with training datasets composed of diagrams, textual descriptions, and solution programs~\citep{geoqa,unigeo,intergps,pgps}. Building on these datasets, a number of PGPS specialized vision-language models (VLMs) have been developed so far~\citep{GOLD, LANS, geox}.

Most existing VLMs, however, fail to use diagrams when solving geometry problems. Well-known PGPS datasets such as GeoQA~\citep{geoqa}, UniGeo~\citep{unigeo}, and PGPS9K~\citep{pgps}, can be solved without accessing diagrams, as their problem descriptions often contain all geometric information. \cref{fig:pgps_examples} shows an example from GeoQA and PGPS9K datasets, where one can deduce the solution steps without knowing the diagrams. 
As a result, models trained on these datasets rely almost exclusively on textual information, leaving the vision encoder under-utilized~\citep{GOLD}. 
Consequently, the VLMs trained on these datasets cannot solve the plane geometry problem when necessary geometric properties or relations are excluded from the problem statement.

Some studies seek to enhance the recognition of geometric premises from a diagram by directly predicting the premises from the diagram~\citep{GOLD, intergps} or as an auxiliary task for vision encoders~\citep{geoqa,geoqa-plus}. However, these approaches remain highly domain-specific because the labels for training are difficult to obtain, thus limiting generalization across different domains. While self-supervised learning (SSL) methods that depend exclusively on geometric diagrams, e.g., vector quantized variational auto-encoder (VQ-VAE)~\citep{unimath} and masked auto-encoder (MAE)~\citep{scagps,geox}, have also been explored, the effectiveness of the SSL approaches on recognizing geometric features has not been thoroughly investigated.

We introduce a benchmark constructed with a synthetic data engine to evaluate the effectiveness of SSL approaches in recognizing geometric premises from diagrams. Our empirical results with the proposed benchmark show that the vision encoders trained with SSL methods fail to capture visual \geofeat{}s such as perpendicularity between two lines and angle measure.
Furthermore, we find that the pre-trained vision encoders often used in general-purpose VLMs, e.g., OpenCLIP~\citep{clip} and DinoV2~\citep{dinov2}, fail to recognize geometric premises from diagrams.

To improve the vision encoder for PGPS, we propose \geoclip{}, a model trained with a massive amount of diagram-caption pairs.
Since the amount of diagram-caption pairs in existing benchmarks is often limited, we develop a plane diagram generator that can randomly sample plane geometry problems with the help of existing proof assistant~\citep{alphageometry}.
To make \geoclip{} robust against different styles, we vary the visual properties of diagrams, such as color, font size, resolution, and line width.
We show that \geoclip{} performs better than the other SSL approaches and commonly used vision encoders on the newly proposed benchmark.

Another major challenge in PGPS is developing a domain-agnostic VLM capable of handling multiple PGPS benchmarks. As shown in \cref{fig:pgps_examples}, the main difficulties arise from variations in diagram styles. 
To address the issue, we propose a few-shot domain adaptation technique for \geoclip{} which transfers its visual \geofeat{} perception from the synthetic diagrams to the real-world diagrams efficiently. 

We study the efficacy of the domain adapted \geoclip{} on PGPS when equipped with the language model. To be specific, we compare the VLM with the previous PGPS models on MathVerse~\citep{mathverse}, which is designed to evaluate both the PGPS and visual \geofeat{} perception performance on various domains.
While previous PGPS models are inapplicable to certain types of MathVerse problems, we modify the prediction target and unify the solution program languages of the existing PGPS training data to make our VLM applicable to all types of MathVerse problems.
Results on MathVerse demonstrate that our VLM more effectively integrates diagrammatic information and remains robust under conditions of various diagram styles.

\begin{itemize}
    \item We propose a benchmark to measure the visual \geofeat{} recognition performance of different vision encoders.
    % \item \sh{We introduce geometric CLIP (\geoclip{} and train the VLM equipped with \geoclip{} to predict both solution steps and the numerical measurements of the problem.}
    \item We introduce \geoclip{}, a vision encoder which can accurately recognize visual \geofeat{}s and a few-shot domain adaptation technique which can transfer such ability to different domains efficiently. 
    % \item \sh{We develop our final PGPS model, \geovlm{}, by adapting \geoclip{} to different domains and training with unified languages of solution program data.}
    % We develop a domain-agnostic VLM, namely \geovlm{}, by applying a simple yet effective domain adaptation method to \geoclip{} and training on the refined training data.
    \item We demonstrate our VLM equipped with GeoCLIP-DA effectively interprets diverse diagram styles, achieving superior performance on MathVerse compared to the existing PGPS models.
\end{itemize}

\fi 

\section{Related Work}\label{sec:related_works}
\gls{bp} estimation from \gls{ecg} and \gls{ppg} waveforms has received significant attention due to its potential for continuous, unobtrusive monitoring. Earlier work relied on classical machine learning with handcrafted features, but deep learning methods have since emerged as more robust alternatives. Convolutional or recurrent architectures designed for \gls{ecg}/\gls{ppg} have shown strong performance, including ResUNet with self-attention~\cite{Jamil}, U-Net variants~\cite{Mahmud_2022}, and hybrid \gls{cnn}--\gls{rnn} models~\cite{Paviglianiti2021ACO}. These architectures often outperform traditional feature-engineering approaches, particularly when both \gls{ecg} and \gls{ppg} signals are used~\cite{Paviglianiti2021ACO}.

Nevertheless, many existing methods train solely on \gls{ecg}/\gls{ppg} data, which, while plentiful~\cite{mimiciii,vitaldb,ptb-xl}, often exhibit significant variability in signal quality and patient-specific characteristics. This variability poses challenges for achieving robust generalization across populations. Recent work has explored transfer learning to overcome these issues; for example, Yang \emph{et~al.}~\cite{yang2023cross} studied the transfer of \gls{eeg} knowledge to \gls{ecg} classification tasks, achieving improved performance and reduced training costs. Joshi \emph{et~al.}~\cite{joshi2021deep} also explored the transfer of \gls{eeg} knowledge using a deep knowledge distillation framework to enhance single-lead \gls{ecg}-based sleep staging. However, these studies have largely focused on within-modality or narrow domain adaptations, leaving open the broader question of whether an \gls{eeg}-based foundation model can serve as a versatile starting point for generalized biosignal analysis.

\gls{eeg} has become an attractive candidate for pre-training large models not only because of the availability of large-scale \gls{eeg} repositories~\cite{TUEG} but also due to its rich multi-channel, temporal, and spectral dynamics~\cite{jiang2024large}. While many time-series modalities (for example, voice) also exhibit rich temporal structure, \gls{eeg}, \gls{ecg}, and \gls{ppg} share common physiological origins and similar noise characteristics, which facilitate the transfer of temporal pattern recognition capabilities. In other words, our hypothesis is that the underlying statistical properties and multi-dimensional dynamics in \gls{eeg} make it particularly well-suited for learning robust representations that can be effectively adapted to \gls{ecg}/\gls{ppg} tasks. Our work is the first to validate the feasibility of fine-tuning a transformer-based model initially trained on EEG (CEReBrO~\cite{CEReBrO}) for arterial \gls{bp} estimation using \gls{ecg} and \gls{ppg} data.

Beyond accuracy, real-world deployment of \gls{bp} estimation models calls for efficient inference. Traditional deep networks can be computationally expensive, motivating recent interest in quantization and other compression techniques~\cite{nagel2021whitepaperneuralnetwork}. Few studies have combined large-scale pre-training with post-training quantization for \gls{bp} monitoring. Hence, our method integrates these two aspects: leveraging a potent \gls{eeg}-based foundation model and applying quantization for a compact, high-accuracy cuffless \gls{bp} solution.
\section{TerEffic Key Innovations}
%\vspace{-1.5mm}
To enable efficient on-chip inference of ternary LLMs, we introduce three key innovations:
\begin{enumerate}
\item \textbf{1.6-Bit Weight Compression}: Our method encodes ternary weights \{-1, 0, 1\} using only 1.6 bits per weight, approaching the theoretical minimum of 1.58 bits. This 20\% reduction in memory footprint compared to 2-bit encoding enables larger models to fit on-chip.
\item \textbf{Ternary MatMul Unit (TMU)}: We optimize ternary matrix multiplications (MatMuls) through specialized hardware units that reduce complex operations such as negation and leverage FPGA-native multiplexers. This customization reduces Look-up Table(LUT) usage by 40\% while maintaining high computational throughput.
\item \textbf{Compute-Memory Alignment}: Our hybrid URAM-BRAM memory architecture balances computational power, memory bandwidth, and capacity. By storing weights in URAM and intermediate values in BRAM, we achieve optimal resource utilization without bandwidth bottlenecks or wasted resources.
\end{enumerate}

\subsection{1.6-Bit Weight Compression}
\label{1.6bit}
\vspace{-0.5mm}
To encode ternary weights \{-1, 0, 1\} with binary bits, the theoretical lower bound is 1.58 bits\cite{bitnet1.58}. However, achieving exactly 1.58 bits is not feasible in hardware, while a straightforward 2-bit representation introduces 25\% redundancy. To overcome this challenge, we adapt and optimize the encoding scheme from~\cite{ternaryencoding} to create an efficient 1.6-bit compression method that closely approaches the theoretical limit while remaining hardware-friendly.

We observe that 5 ternary weights have 243 possible combinations whereas 8 binary bits can represent 256 distinct values. Hence, we encode 5 ternary weights using 8 bits, averaging 1.6 bits per weight. The encoding and decoding scheme is shown in Fig. \ref{fig:bit-compression}. For example, five original weights \{-1,0,0,1,1\} are encoded into 8 bits (10001100) when stored in memory. During inference, the 8-bit encoded weight is decoded into five 2-bit representations, where 01 corresponds to 1, 11 to -1, and 00 to 0. As the decoding involves only bitwise operations, such as + and \&, it incurs minimal hardware cost and latency. This method can store and transfer 25\% more data compared to the regular 2-bit representation, effectively improving throughput and energy efficiency.
\begin{figure}
    \centering
    \vspace{-1mm}\includegraphics[width=0.5\textwidth]{figures/bit-compression.pdf}
    \caption{1.6-Bit Weight Compression}
    \label{fig:bit-compression}
    \vspace{-7mm}
\end{figure}


\subsection{Ternary MatMul Unit}
MatMuls account for the largest portion (65\%-85\%) of the computational workload\cite{high_perf} in typical LLMs. To achieve the full potential of ternary LLMs, we replace costly MatMuls with optimized \textbf{Ternary MatMuls Units (TMUs)} that are much cheaper. 

A naive TMU design is shown in Fig. \ref{fig:TMU_ori}. Here, the partial sum ($S$) is either added or subtracted by the activation ($X$), or remains unchanged depending on the weight ($W \in \{1,-1,0\}$). We identify inefficiencies in this simple design and propose the corresponding optimizations. First, the activation ($X$) is broadcast to multiple TMUs in parallel (as elaborated in Section IV.A). Therefore, performing complex operations, such as negation, within each TMU harms efficiency. To address this, we pre-calculate the negation outside the TMUs, providing both the activation and its negation as TMU inputs. Secondly, as the default logic units, LUTs cannot directly implement conditional statements but decompose them into a series of mapping relations. This indirect mapping causes resource wastage. We resolve this by leveraging another basic FPGA resource, the 2-to-1 multiplexer (Mux), specialized for the selection logic. With each weight decoded to 2 bits (as presented in Section \ref{1.6bit}), the high bit (W[1]) serves as the select signal for the Mux, while the low bit (W[0]) determines whether the Mux's output should be retained or transformed into zero. By integrating the two optimizations, we introduce an improved TMU design illustrated in Fig. \ref{fig:TMU_new}, which achieves approximately 40\% LUT savings for the whole design.
\begin{figure}[h]
    \vspace{-1mm}
    \centering
    \begin{subfigure}[b]{0.14\textwidth}
        \centering
        \includegraphics[width=\textwidth]{figures/TMU_ori.pdf}
        \caption{Naive TMU}
        \label{fig:TMU_ori}
    \end{subfigure}
    \hspace{15mm}
    \begin{subfigure}[b]{0.14\textwidth}
        \centering
        \includegraphics[width=\textwidth]{figures/TMU_new.pdf}
        \caption{Improved TMU}
        \label{fig:TMU_new}
    \end{subfigure}
    \caption{Improvements in the TMU Design}
    \label{fig:TMU}
    \vspace{-4.5mm}
\end{figure}



\vspace{-1.5mm}
\subsection{Compute-Memory Alignment}
\label{sec:Compute-Memory Alignment}


\begin{figure}
    \vspace{-5mm}
    \centering
    \includegraphics[width=.95\linewidth]{figures/design_space.pdf}
    \caption{Compute-Memory Alignment}
    \label{fig:Alignment}
    \vspace{-7mm}
\end{figure}

Xilinx U280 FPGA offers two types of on-chip SRAM: BlockRAM (BRAM) and UltraRAM (URAM), as shown in Table \ref{tab:SRAM}. 
To align with the substantial computational capability offered by TMU, an intricate memory architecture is required to provide enough bandwidth while avoiding excessive bandwidth which will cause power wastage. 
At the same time, the memory architecture must offer sufficient capacity for on-chip weight storage while optimizing memory capacity utilization to minimize power wastage.

\begin{table}[h]
    \vspace{-3mm}
    \centering
    \caption{Attibutes of On-chip BRAM and URAM on U280}
    \label{tab:SRAM} 
    \resizebox{\linewidth}{!}{
    \begin{tabular}{|r|r|r|r|r|}
        \hline
        SRAM& Capacity/piece & Bandwidth/piece & Number & Total Capacity \\
        \hline
        BRAM & 36Kb & 72b & 2,016 & 8.85MB \\
        \hline
        URAM & 288Kb & 144b & 960 & 33.75MB \\
        \hline
    \end{tabular}
    }
    \vspace{-3mm}
\end{table}
Fig. \ref{fig:Alignment} illustrates memory choices where each point depicts the bandwidth and capacity provided by a certain memory architecture. The horizontal dotted line represents the required bandwidth to align with the maximum computational capability of the device, while the vertical dotted lines indicate the required memory capacity that increases with the model size. As shown by the blue line, a fully BRAM-based architecture lacks sufficient memory capacity. Conversely, a fully URAM-based architecture (the green line) suffers from low bandwidth, also resulting in power wastage.

Therefore, we propose a \textbf{hybrid fully on-chip architecture} (the orange line) storing weights in URAM and buffering intermediate values in BRAM. Our design falls approximately on the intersection of the dotted lines in Fig. \ref{fig:Alignment}, enabling full utilization of computational power, memory bandwidth, and capacity concurrently. However, it can be observed that as the model size increases, the on-chip memory capacity will eventually fail to meet the storage requirements. In such cases, the large capacity of off-chip HBM is still required, while its limited bandwidth bottlenecks the performance (shown in the purple line). To alleviate this issue, we introduce an \textbf{HBM-assisted architecture} with effective parallelization, proposed in Section \ref{sec:HBM-Assisted Architecture}.






\section{TerEffic Architecture design}
\label{sec:archi}
We design a ternary LLM accelerator architecture on FPGA to showcase the benefits of the aforementioned optimizations. We choose the MatMul-free LM model~\cite{scalable} as the representative ternary LLM for our design. Although \cite{scalable} also proposed an FPGA design, it was a simple and inefficient prototype as their main focus was on the model architecture. 

\begin{figure}[h]
    \centering
    \vspace{-3mm}
    \includegraphics[width=.9\linewidth]{figures/arch.pdf}
    \caption{TerEffic Hardware Architecture}
    \label{fig:model}
    \vspace{-6mm}
\end{figure}

\subsection{Architecture Overview}

TerEffic overall hardware architecture is shown in Fig.\ref{fig:model}, along with its two major components, the HGRN\cite{HGRN} and the GLU\cite{glu} module. HGRN is a powerful RNN-based alternative to the self-attention mechanism\cite{attention}, while GLU, widely utilized in models like Llama\cite{llama}, serves as a robust enhancement for feed-forward networks (FFNs).
In the model, addition, subtraction, and dot product (Dot) can be directly implemented using LUTs and DSPs, while the sigmoid function is approximated by piecewise linear equations~\cite{sigmoid}. Consequently, the primary implementation challenge lies in the BitLinear module consisting of a Root-Mean-Square Normalization (RMSNorm) module and a ternary MatMul module.
\subsubsection{RMSNorm module}
The RMSNorm\cite{RMSNorm} is more computationally efficient than the traditional LayerNorm\cite{attention} but maintains high accuracy, making it well-suited for FPGA. The algorithm for RMSNorm is presented below:
\vspace{-1mm}
\begin{equation}
        r = \sqrt{\frac{1}{d} \sum_{i=1}^d x_i^2 + \epsilon} \quad , \quad 
        \text{RMSNorm}(X) = \frac{X\odot W_n}{r}
\label{eq:RMSNorm}
\end{equation}
\vspace{-1mm}

where $X\in\mathbb{R}^{1 \times d}$ denotes the input activation, $W_n\in\mathbb{R}^{1 \times d}$ denotes the normalization weight, $r$ denotes the RMS result and $\epsilon$ is a small constant. As shown in the architecture in Fig. \ref{fig:RMSNorm}, the computation of r and $X\odot W_n$ can be executed in parallel. As the former has longer latency, the results of $X\odot W_n$ are temporarily stored in on-chip buffers. These results are then repeatedly retrieved from the buffers to match the multiple iterations required by the subsequent ternary MatMul modules. Moreover, as divisions incur high hardware cost and long cycle latency, we replace the divisions ($\div r$) with DSP-based multiplications ($\times \frac{1}{r}$), using r as an index to retrieve 1/r from an on-chip look-up table consisting of a small amount of SRAM. This method saves hardware resources and significantly increases maximum frequency.

\subsubsection{Ternary MatMul Module}
The Ternary MatMul Module serves as the primary computational core executing $X\times W$, where $X\in\mathbb{R}^{1 \times d}$ denotes the normalized activation and $W\in\mathbb{R}^{d \times d}$ denotes the ternary weight matrix. The module consists of submodules that perform the dot product between $X$ and one column of $W$. Each submodule is made up of TMUs for ternary MatMuls and a reduction tree to aggregate the TMU outputs. A detailed architecture is shown in Fig.\ref{fig:TMM}. To address the memory bandwidth limitations, we partition $X$ and each column of $W$ into x groups. Each submodule, which contains $\frac{d}{x}$ TMUs, computes the MatMuls for one group. Moreover, due to the limited computational capacity, the d columns of $W$ are divided into y sets and processed sequentially by $\frac{d}{y}$ submodules. The values of x and y are set based on the on-chip memory bandwidth and computing capacity, optimized by the alignment strategy discussed in Section \ref{sec:Compute-Memory Alignment}.


\begin{figure}[h]
    \vspace{-5mm}
    \centering
    \begin{subfigure}[b]{0.4\textwidth} 
        \centering
        \includegraphics[width=\textwidth]{figures/RMSNorm.pdf} 
        \caption{RMSNorm Module}
        \label{fig:RMSNorm}
    \end{subfigure} 
    \begin{subfigure}[b]{0.4\textwidth} 
        \centering
        \includegraphics[width=\textwidth]{figures/TernaryMatMulModule.pdf}  
        \caption{Ternary MatMul Module}
        \label{fig:TMM}
    \end{subfigure}
    \vspace{-2mm}
    \caption{BitLinear Module}
    \label{fig:bitlinear module}
    \vspace{-6mm}
\end{figure}


\subsection{Architecture Variants}
We propose two memory architectures--one aims to store all weights in on-chip memory, and another utilizes an HBM. Since a single card is insufficient to hold all the weights of a moderate-sized model, the fully-on-chip architecture may need to leverage multiple cards. For larger models, we switch to HBM-assisted architecture that requires only one card for weight storage. We explore different parallelism methods in both architectures to achieve higher energy efficiency. Though the methods differ from the batching method for GPU, we use terms like 'multi-batch' and 'batch size' for convenience.
\subsubsection{TerEffic On-Chip Architecture}
\label{sec:On-Chip Architecture}
The on-chip architecture corresponds to the orange line in Fig. \ref{fig:Alignment}, where we store all weights on-chip and benefit from the massive on-chip bandwidth.
The basic design is based on the 24-layer, 370M model from~\cite{scalable}. Each layer is segmented into 10 main stages, as shown in Fig. \ref{fig:stages} (Norm denotes RMSNorm and TMM denotes ternary MatMul), with the latency of each stage indicated in the red row. Through the optimizations discussed before, the latency per layer is 820 cycles, resulting in a total latency of $\approx$  20,000 cycles across all 24 layers.

After 1.6-Bit Weight Compression, the 370M model occupies $\approx$ 57.6MB of memory, exceeding the on-chip capacity of 42MB (see TABLE \ref{tab:SRAM}). Thus, a 2-card system is required, with each card holding the weights for 12 layers and processing them accordingly, as shown in Fig. \ref{fig:multi-card}. As the model size increases, more cards become necessary—for instance, the 1.3B model requires eight cards each storing three layers.

Though this design delivers the shortest latency, the power efficiency is relatively low as only a portion of the resources is utilized at any point in time. To better harness parallelism, we construct a \textbf{pipeline}. A simple 2-stage example is illustrated in Fig. \ref{fig:pipeline}, where a batch size of two is processed concurrently by different TMUs.
As shown by the purple row in Fig. \ref{fig:stages}, each pipeline stage takes 160 cycles, resulting in a single-layer latency of 1600 cycles. With each card handling up to a batch size of 10, the 2-card system can process a batch size of 20, boosting throughput by $10\times$ over the basic design.

\begin{figure}
    \centering
    \includegraphics[width=\linewidth]{figures/Stages.pdf}
    \caption{Stage latency of a layer for on-chip single-batch, on-chip multi-batch (with pipeline parallelism), and HBM-assisted multi-batch (with full-resource parallelism)}
    \label{fig:stages}
    \vspace{-4mm}
\end{figure}


\begin{figure}
    \centering
    \begin{subfigure}[b]{0.23\textwidth} 
        \centering
        \includegraphics[width=\textwidth]{figures/Multi-Card.pdf} 
        \caption{On-Chip Architecture}
        \label{fig:multi-card}
        \vspace{-1mm}
    \end{subfigure} 
    \begin{subfigure}[b]{0.23\textwidth} 
        \centering
        \includegraphics[width=\textwidth]{figures/HBM_version.pdf}  
        \caption{HBM-Assisted Architecture}
        \vspace{-1mm}
        \label{fig:HBM}
    \end{subfigure}
    \caption{TerEffic Architecture Variants}
    \label{fig:single batch}
    \vspace{-6mm}
\end{figure}
\subsubsection{TerEffic  HBM-Assisted Architecture}
\label{sec:HBM-Assisted Architecture}
For larger models, an HBM-assisted architecture (Fig.\ref{fig:HBM}) is required. Xilinx U280 FPGA features 8GB HBM with 32 channels, providing a total maximum bandwidth of 8Kb. As discussed in Section \ref{sec:Compute-Memory Alignment}, the HBM's bandwidth is much lower than on-chip memory, resulting in a longer latency of about 2,400 cycles/layer. 

For this architecture, the pipeline parallelism is not suitable, as the latency after pipelining is shorter than the data-fetch latency from HBM. Moreover, pipelining requires two layers of weights stored on-chip in a ping-pong buffer, which is infeasible for large models. Hence, we introduce \textbf{full-resource parallelism} with a simplified example (2 stages, batch size=2) presented in Fig. \ref{fig:parallelism}. Unlike in the previous pipeline, all the TMUs are utilized for processing each stage. As the batch size increases, the TMUs available for each input decrease and the latency increases. We set the batch size to 15 to accommodate 2,400 cycle HBM data-fetch latency. The detailed latency is shown in the orange row in Fig. \ref{fig:stages}, where the latency of each TMM stage is proportional to the computational workload.


\begin{figure}[t]
    \vspace{-3mm}
    \centering
    \begin{subfigure}[b]{0.15\textwidth} 
        \centering
        \includegraphics[width=\textwidth]{figures/Pipeline_Parallelism.pdf} 
        \caption{Pipeline}
        \vspace{-2mm}
        \label{fig:pipeline}
    \end{subfigure}
    \hspace{10mm}  
    \begin{subfigure}[b]{0.15\textwidth} 
        \centering
        \includegraphics[width=\textwidth]{figures/Full-Resource_Parallelism.pdf}  
        \caption{Full-Resource}
        \vspace{-2mm}
        \label{fig:Full-Resource}
    \end{subfigure}
    \caption{Parallelism Methods}
    \label{fig:parallelism}
    \vspace{-5mm}
\end{figure}

\section{Evaluation}

% Our proposed framework was compared with Apollo \cite{b7Apollo1, b7Apollo2}, which demonstrates that it can model analytic operators using data content. Two loss functions were utilized, the root-mean-square deviation error (RMSE), and the mean absolute error (MAE). The selection of these two loss functions is because they fulfil the disadvantages of each other, while RMSE is sensitive to outlier MAE is not and the MAE cannot take into account the direction of the error while the RMSE can achieve it. Speedup was computed to determine how quickly our framework can model the operator $\Phi$. We utilised the \textit{Speedup} and \textit{Amortized Speedup}, which assesses the require time to approximate each operator in comparison to exhaustively executing them on all datasets (more is better). Particularly, the speedup is equalled $\frac{T{^{(i)}_{op}}}{T{^{(i)}_{SimOp} + T_{vec} + T_{sim} + T_{pred}}}$, where $T{^{(i)}_{op}}$ is the execution time for operator $i$, across all the datasets, $T{^{(i)}_{SimOp}}$ is the time needed to model the operator with the datasets selected from the similarity search, $T_{vec}$  is the time needed to compute the vector embedding for each dataset, $T_{sim}$, is the time needed to perform similarity search, and $T_{pred}$ is the time needed to predict on the dataset $D_o$. In addition to the dataset vectorisation, which is done once for each data lake, we calculate amortised speedup. Furthermore, an experimental evaluation of our proposed model for dataset vectorization NumTabData2Vec has been performed to show that our approach can transform a dataset to a vector embedding representation space $z$. For the evaluation experiments, three different NumTabData2Vec were built to project the dataset representation with vector sizes of $100$, $200$, and $300$. Each model has eight transformer layers and is trained parallel using four NVIDIA A10s GPUs, and trained for fifty epochs.
We compared our framework with Apollo \cite{b7Apollo1, b7Apollo2}, which models analytic operators using data content. Two loss functions to measure prediction accuracy are employed: root-mean-square error (RMSE) and mean absolute error (MAE). RMSE is sensitive to outliers, while MAE is not; conversely, RMSE accounts for error direction, which MAE cannot. Speedup metrics are also used to evaluate how efficiently our framework models operator $\Phi$. Specifically, \textit{Speedup} and \textit{Amortized Speedup} measure the time required to approximate each operator versus exhaustively executing them on all datasets. Speedup is defined as $\frac{T{^{(i)}_{op}}}{T{^{(i)}_{SimOp} + T_{vec} + T_{sim} + T_{pred}}}$, where $T{^{(i)}_{op}}$ is the time to execute operator $i$ on all datasets, $T{^{(i)}_{SimOp}}$ is the time to model the operator with datasets from similarity search, $T_{vec}$ is the vector embedding computation time, $T_{sim}$ is the similarity search time, and $T_{pred}$ is the prediction time for $D_o$. Amortized speedup includes dataset vectorization, performed once per data lake for multiple operators (in our case two operators).
We also evaluate our dataset vectorization model, NumTabData2Vec, which projects datasets into vector embedding space $z$. Three versions were built with vector sizes of $100$, $200$, and $300$, each featuring eight transformer layers. The models were trained for 50 epochs on four NVIDIA A10 GPUs in parallel.

\subsection{Evaluation Setup}
Our framework is deployed over an AWS EC2 virtual machine server running with 48 VCPUs of AMD EPYC 7R32 processors at 2.40GHz, and four A10s GPUs with 24GB of memory each, $192GB$ of RAM memory, and $2TB$ of storage, running over Ubuntu 24.4 LTS. Our code is written in Python (v.3.9.1) and PyTorch modules (v.2.4.0). Apollo was deployed in a virtual machine with 8 VCPUs Intel Xeon E5-2630 @ 2.30GHz, $64GB$ of RAM memory, and $250GB$ of storage, running Ubuntu 24.4 LTS like in their experimental evaluation. 

\subsection{Datasets}
\begin{table}[!ht]
    \centering
    \setlength\doublerulesep{0.5pt}
    \caption{Dataset properties for experimental evaluation}
    \label{tab:table-evaluation-datasets}
    \begin{tabular}{||c|c|c|c||}
        \hline
         \makecell{Dataset Name}& \makecell{\# Files} & \makecell{\# Tuples} & \makecell{\# Columns}\\ \hline\hline
         Household Power & & & \\
         Consumption \cite{b21HPCdataset} & $401$ & $2051$ & 7\\
         \hline
         Adult \cite{b22AdultDataset} & $100$ & $228$ & 14\\
         \hline
         Stocks \cite{b23StockMarketDataset} & $508$ & $1959 - 13$ & 7 \\
         \hline
         Weather \cite{b23WeatherDataset} & $49$ & $516$ & 7 \\ \hline
    \end{tabular}

\end{table}

We evaluated our framework using four diverse datasets to represent real-world scenarios. Table \ref{tab:table-evaluation-datasets} summarizes these datasets and their attributes. The vectorization module, NumTabData2Vec, was trained on data separate from the experimental evaluation data, split $60\%$ for training and $40\%$ for testing.
The Household Power Consumption (HPC) dataset \cite{b21HPCdataset} contains electric power usage measurements from a household in Sceaux, France. It includes $401$ datasets, each with $2051$ tuples and seven features recorded at one-minute intervals. The Adult dataset \cite{b22AdultDataset}, commonly used for binary classification, predicts whether an individual earns more or less than $50K$ annually. It comprises $100$ datasets, each with $228$ individuals and various socio-economic features.
The Stock Market dataset \cite{b23StockMarketDataset} includes daily NASDAQ stock prices obtained from Yahoo Finance, with $508$ datasets. Each dataset contains $13$ to $1959$ tuples, each describing seven feature attributes. The Weather dataset \cite{b23WeatherDataset} provides hourly weather measurements from $36$ U.S. cities between $2012$ and $2017$, split into $49$ datasets, each with $516$ tuples and seven features.


\begin{figure}[!t]
     \centering
     \begin{subfigure}[b]{0.24\textwidth}
         \centering
         \includegraphics[width=\textwidth]{Figures/Results/Sim_Search/HPC/HPC_LR_RMSE_Loss_fig.pdf}
         \caption{Linear Regression RMSE error loss}
         \label{fig:HPC-LR-RMSE}
     \end{subfigure}
     \hfill 
     \begin{subfigure}[b]{0.24\textwidth}
         \centering
         \includegraphics[width=\textwidth]{Figures/Results/Sim_Search/HPC/HPC_LR_MAE_Loss_fig.pdf}
         \caption{Linear Regression MAE error loss}
         \label{fig:HPC-LR-MAE}
     \end{subfigure}
        
     \begin{subfigure}[b]{0.24\textwidth}
         \centering
         \includegraphics[width=\textwidth]{Figures/Results/Sim_Search/HPC/HPC_MLP_RMSE_Loss_fig.pdf}
         \caption{MLP for Regression RMSE error loss}
         \label{fig:HPC-MLP-RMSE}
     \end{subfigure}
     \hfill 
     \begin{subfigure}[b]{0.24\textwidth}
         \centering
         \includegraphics[width=\textwidth]{Figures/Results/Sim_Search/HPC/HPC_MLP_MAE_Loss_fig.pdf}
         \caption{MLP for Regression MAE error loss}
         \label{fig:HPC-MLP-MAE}
     \end{subfigure}
        \caption{Household power consumption dataset prediction error loss}
        \label{fig:HPC-EVAL-RES}
\end{figure}

Our framework was evaluated by registering the accuracy of predicting the output of various ML operators over multiple datasets in $D$ without actually executing the operator on them. To evaluate our scheme and its parameters, we use all four datasets, ranging the size of the produced vectors as well as the similarity functions used.
We project all datasets into $k$-dimensional spaces with varying vector dimensions ($100$, $200$, and $300$). For each dataset in Table \ref{tab:table-evaluation-datasets}, we model different operators: For the regression datasets (Household Power Consumption and Stock Market), we model Linear Regression (LR) and Multi-Layer Perceptron (MLP) operators; for the classification datasets (Weather and Adult), we model the Support Vector Machine (SVM) and MLP classifier operators. Each experiment has been executed $10$ times and we report the average of the error loss, as well as the speedup. 

\begin{figure}[!t]
     \centering
     \begin{subfigure}[b]{0.24\textwidth}
         \centering
         \includegraphics[width=\textwidth]{Figures/Results/Sim_Search/Stocks/Stocks_LR_RMSE_Loss_fig.pdf}
         \caption{Linear Regression RMSE error loss}
         \label{fig:Stock-LR-RMSE}
     \end{subfigure}
     \hfill 
     \begin{subfigure}[b]{0.24\textwidth}
         \centering
         \includegraphics[width=\textwidth]{Figures/Results/Sim_Search/Stocks/Stocks_LR_MAE_Loss_fig.pdf}
         \caption{Linear Regression MAE error loss}
         \label{fig:Stock-LR-MAE}
     \end{subfigure}
        
     \begin{subfigure}[b]{0.24\textwidth}
         \centering
         \includegraphics[width=\textwidth]{Figures/Results/Sim_Search/Stocks/Stocks_MLP_RMSE_Loss_fig.pdf}
         \caption{MLP for Regression RMSE error loss}
         \label{fig:Stock-MLP-RMSE}
     \end{subfigure}
     \hfill 
     \begin{subfigure}[b]{0.24\textwidth}
         \centering
         \includegraphics[width=\textwidth]{Figures/Results/Sim_Search/Stocks/Stocks_MLP_MAE_Loss_fig.pdf}
         \caption{MLP for Regression MAE error loss}
         \label{fig:Stock-MLP-MAE}
     \end{subfigure}
        \caption{Stock market dataset prediction error loss}
        \label{fig:Stock-EVAL-RES}
\end{figure}
\subsection{Evaluation Results}



Figures \ref{fig:HPC-EVAL-RES}, \ref{fig:Stock-EVAL-RES}, \ref{fig:Weather-EVAL-RES}, and \ref{fig:Adult-EVAL-RES} present the evaluation results for each method, comparing the performance of different similarity search techniques across various vector embedding representation spaces. The red (with hatches), brown, and blue bars correspond to vector embeddings of size 100, 200, and 300 respectively. In each sub-figure, the y-axis represents the error loss value, while the x-axis displays the similarity search method applied over the vector embeddings. Figures \ref{fig:HPC-EVAL-RES} and \ref{fig:Stock-EVAL-RES} show the results for the Stock market and Household power consumption datasets, where the bottom sub-figure demonstrates the MLP regression model, and the top sub-figure presents the LR model. Figures \ref{fig:Weather-EVAL-RES} and \ref{fig:Adult-EVAL-RES} depict the evaluation results for the Weather and Adult datasets. In these Figures, the top sub-figure shows the SVM with SGD results, while the bottom sub-figure shows the MLP classifier. The left sub-figures in all Figures use the RMSE loss function, whereas the right sub-figures use the MAE loss function. 



Figure \ref{fig:HPC-EVAL-RES}, we show, for the HPC dataset, shows as increase the vector dimension size there is slightly lower prediction error for all the operator modelling. While for different similarity methods did not result in any significant differences in the prediction error loss for all the operator modelling. This suggests that, regardless the similarity selection method, our framework effectively selects the most optimal subset of data to improve model predictions on the unseen input dataset $D_o$. Additionally, we observe higher error loss with a vector size of 100, which can be attributed to the reduced representation capacity of lower-dimensional vectors. This limitation results in fewer ``right" datasets being selected.

For the stock market dataset, Figure \ref{fig:Stock-EVAL-RES} depicts that a vector embedding representation of size $300$ models more accurate operators, with cosine similarity performing best in the similarity search and modelling the most optimal operator. However, due to the inherent volatility in Stock market data from different days, all models in the stock market dataset experiments exhibit high loss values. 

In the weather dataset, the SVM operator results from sub-figures \ref{fig:Weather-SVM-RMSE} and \ref{fig:Weather-SVM-MAE} show that using $300$ vectors in the representation space consistently led to more accurate operator models across all similarity methods. Specifically, cosine similarity in combination with the $300$-dimensional vector embedding reduced the error rate in operator predictions, demonstrating that projecting datasets into this representation space and applying cosine similarity improves the prediction accuracy on the modelled operator. For the MLP classifier from sub-figures \ref{fig:Weather-MLP-RMSE} and \ref{fig:Weather-MLP-MAE}, the results illustrate that using vector embeddings of size $200$ and K-Means clustering produced the most accurate MLP classifier operators.

% Overall, we observe that the error loss was minimized 
% (** what do you mean, minimized? In general, here you should comment on the effect of similarity function, the effect of vector size and the effect of different operators to the accuracy of prediction. E.g., in Household dataset shows little effect in all bars, but in Stock, the cosine seems better and larger size of vectors leads to better performance etc. **)
% in most cases, indicating that our framework effectively selects the most relevant datasets from the data lake $D$, thereby improving data quality and reducing $\Phi$ prediction errors on the target dataset $D_o$. This demonstrates that the datasets are accurately transformed into the vector embedding representation space, allowing for the selection of datasets most similar to $D_o$. 

%Adult


%Weather
\begin{figure}[t!]
     \centering
     \begin{subfigure}[b]{0.24\textwidth}
         \centering
         \includegraphics[width=\textwidth]{Figures/Results/Sim_Search/Weather/Weather_SVM_RMSE_Loss_fig.pdf}
         \caption{SVM with SGD RMSE error loss}
         \label{fig:Weather-SVM-RMSE}
     \end{subfigure}
     \hfill 
     \begin{subfigure}[b]{0.24\textwidth}
         \centering
         \includegraphics[width=\textwidth]{Figures/Results/Sim_Search/Weather/Weather_SVM_MAE_Loss_fig.pdf}
         \caption{SVM with SGD MAE error loss}
         \label{fig:Weather-SVM-MAE}
     \end{subfigure}
        
     \begin{subfigure}[b]{0.24\textwidth}
         \centering
         \includegraphics[width=\textwidth]{Figures/Results/Sim_Search/Weather/Weather_MLP_RMSE_Loss_fig.pdf}
         \caption{MLP RMSE error loss}
         \label{fig:Weather-MLP-RMSE}
     \end{subfigure}
     \hfill 
     \begin{subfigure}[b]{0.24\textwidth}
         \centering
         \includegraphics[width=\textwidth]{Figures/Results/Sim_Search/Weather/Weather_MLP_MAE_Loss_fig.pdf}
         \caption{MLP MAE error loss}
         \label{fig:Weather-MLP-MAE}
     \end{subfigure}
        \caption{Weather dataset prediction error loss}
        \label{fig:Weather-EVAL-RES}
\end{figure}

On the other hand, the Adult dataset shows the lowest error rates, with error loss values consistently below $0.5$ across all vector embedding dimensions and similarity search methods (see Figure \ref{fig:Adult-EVAL-RES}). The Adult dataset, besides exhibiting a high number of rows, also has a higher number of columns, which demonstrates that our framework performs consistently well even with larger datasets.
Additionally, we observe that the lowest prediction error across all datasets occurs when using higher-dimensional vector embeddings. With a trade-off between accuracy and execution time as the difference to generate all data lake available datasets vector embedding representation between $100$ and $300$ size dimension in the vector representation space to be less than $60$ seconds. This confirms that a higher number of vector dimensions leads to more accurate predictions, consistent with findings in previous research \cite{b8Word2Vec}.


\begin{figure}[!t]
     \centering
     \begin{subfigure}[b]{0.24\textwidth}
         \centering
         \includegraphics[width=\textwidth]{Figures/Results/Sim_Search/Adult/Adult_MLP_RMSE_Loss_fig.pdf}
         \caption{SVM with SGD RMSE error loss}
         \label{fig:Adult-LR-RMSE}
     \end{subfigure}
     \hfill 
     \begin{subfigure}[b]{0.24\textwidth}
         \centering
         \includegraphics[width=\textwidth]{Figures/Results/Sim_Search/Adult/Adult_MLP_RMSE_Loss_fig.pdf}
         \caption{SVM with SGD MAE error loss}
         \label{fig:Adult-LR-MAE}
     \end{subfigure}
     
     \begin{subfigure}[b]{0.24\textwidth}
         \centering
         \includegraphics[width=\textwidth]{Figures/Results/Sim_Search/Adult/Adult_MLP_RMSE_Loss_fig.pdf}
         \caption{MLP RMSE error loss}
         \label{fig:Adult-MLP-RMSE}
     \end{subfigure}
     \hfill 
     \begin{subfigure}[b]{0.24\textwidth}
         \centering
         \includegraphics[width=\textwidth]{Figures/Results/Sim_Search/Adult/Adult_MLP_MAE_Loss_fig.pdf}
         \caption{MLP MAE error loss}
         \label{fig:Adult-MLP-MAE}
     \end{subfigure}
        \caption{Adult dataset prediction error loss}
        \label{fig:Adult-EVAL-RES}
\end{figure}





We conducted an experimental evaluation using the Sampling Ratio (SR) approach, similar to Apollo \cite{b7Apollo1}, but employed neural networks built from the vector embeddings of each dataset. The SR approach involves a unified random selection of $l\%$ datasets from the vector representation space, using this subset to construct a neural network for predicting operator outputs. We tested SR values of $0.1$, $0.2$, and $0.4$, as well as vector embedding dimensions of $100$, $200$, and $300$, across all datasets. 
Figure \ref{fig:SR-EVAL-RES} presents the sampling ratio results for the Adult dataset using MLP (sub-figure \ref{fig:Adult-SR-RMSE}) and for the Weather dataset using LR (sub-figure \ref{fig:Weather-SR-SVM-MAE}). In each sub-figure the y-axis represents the RMSE prediction error loss while the x-axis denotes the vector dimension



\begin{figure}[htpb!]
     \centering
     \begin{subfigure}[b]{0.24\textwidth}
         \centering
         \includegraphics[width=\textwidth]{Figures/Results/SR/Adult/Adult_MLP_SR_RMSE_Loss_fig.pdf}
         \caption{Adult Dataset MLP Operator RMSE error loss}
         \label{fig:Adult-SR-RMSE}
     \end{subfigure}
     \hfill 
     \begin{subfigure}[b]{0.24\textwidth}
         \centering
         \includegraphics[width=\textwidth]{Figures/Results/SR/HPC/HPC_LR_SR_RMSE_Loss_fig.pdf}
         \caption{HPC dataset LR Operator RMSE error loss}
         \label{fig:Weather-SR-SVM-MAE}
     \end{subfigure}
     \caption{Sampling Ratio prediction results}
        \label{fig:SR-EVAL-RES}
\end{figure}

Both experiments demonstrate that as the vector embedding dimension increases, coupled with a larger sampling ratio (SR) value, there is a slight decrease in the prediction error loss. This improvement occurs because higher-dimensional vector embeddings provide a more accurate representation of the datasets in k-dimensions, with better dataset selection leading to enhanced prediction accuracy. Comparing the SR approach to our similarity search method for the HPC dataset, the SR approach was approximately $15\%$ less accurate in operator prediction across all vector embedding dimensions. A similar trend was observed in the Weather dataset. However, the Stock dataset exhibited a much larger discrepancy, with the SR approach performing about $70\%$ worse in prediction accuracy across all vector embedding dimensions. Likewise, in the Adult dataset, the SR approach delivered the poorest performance, with nearly $90\%$ lower prediction accuracy compared to the similarity search methods.

\begin{table*}[htbp]
    \centering
        \caption{Evaluation results of our framework exported analytic operator with lowest prediction error in comparison with Apollo}
    \label{tab:table-eval-res}
    % \scalebox{0.8}{
    \setlength\doublerulesep{0.5pt}
    % \begin{adjustbox}{width=\linewidth,center}
    \begin{tabular}{|c|c|c|c|c|c|c|}
    \hline
         \makecell{Dataset\\Name} & Method & Operator & RMSE &  MAE & Speedup  & Amortized Speedup \\
         \hline\hline
         \multirow{7}{*}{\makecell{Household\\Power\\Consumption}}& \makecell{$300$V Cosine} & LR & $\mathbf{6.61}$ & $\mathbf{5.42}$ & $0.0017$ & $\mathbf{1.99}$ \\ \cline{2-7}
                  & \makecell{$300$V SR-$0.2$} & LR & $7.77$ & $6.66$ &  $0.0018$  & $1.42$\\ \cline{2-7} 
        & \makecell{Apollo-SR $0.1$} & LR & $2968.01$ &  $2352.55$ & $\mathbf{0.015}$ & $0.024$ \\ \cline{2-7}
         & \makecell{Apollo-SR $0.2$} & LR & $2811.49$ &  $2229.50$ & $0.015$ & $0.024$ \\ \cline{2-7}\cline{2-7}
         & \makecell{$300$V K-Means} & MLP Regr. & $\mathbf{6.70}$ & $\mathbf{3.38}$ &  $0.9249$  & $\mathbf{1.99}$\\ \cline{2-7}
         & \makecell{Apollo-SR $0.1$} & MLP Regr. & $3322.05$ &  $2606.99$ & $2.38$ & $1.74$ \\ \cline{2-7}
         & \makecell{Apollo-SR $0.2$} & MLP Regr. & $3850.01$ &  $2609.36$ & $\mathbf{2.38}$ & $1.74$\\ \cline{1-7} \cline{1-7} 
         % Stock
         % \multirow{5}{*}{\makecell{Stock}}& \multirow{1}{*}{ \makecell{$100$V Euclidean}} & LR & $229388.93$ & $193066.03$ \\ \cline{2-5}
        \multirow{7}{*}{\makecell{Stock}} &  \makecell{$300$V Cosine} & LR & $306382.28$ & $125335.65$ & $0.00085$ & $\mathbf{1.91}$\\ \cline{2-7}
        & \makecell{$300$V SR-$0.4$} & LR & $21861625.91$ & $5674215.265$ &  $0.00087$  & $0.33$\\ \cline{2-7}
        & \makecell{Apollo-SR $0.1$} & LR & $\mathbf{153665.92}$ &  $\mathbf{118236.48}$ & $\mathbf{0.00093}$ & $0.00096$\\ \cline{2-7}
         & \makecell{Apollo-SR $0.2$} & LR & $166844.95$ &  $133306.68$ & $0.00093$ & $0.00096$\\ \cline{2-7}\cline{2-7}
         &  \makecell{$300$V Cosine} & MLP Regr. & $\mathbf{140236.47}$ & $\mathbf{123571.12}$ & $0.63$ & $\mathbf{1.91}$\\ \cline{2-7}
         & \makecell{Apollo-SR $0.1$} & MLP Regr. &  $175150.82$ &  $145123.09$ & $\mathbf{0.93}$ & $0.96$\\ \cline{2-7}
         & \makecell{Apollo-SR $0.2$} & MLP Regr. & $174390.81$ &  $146338.73$ & $0.93$ & $0.96$\\ \cline{1-7} \cline{1-7}
         % Weather
         \multirow{7}{*}{\makecell{Weather}}& \multirow{1}{*}{ \makecell{$300$V Cosine}} & \makecell{SVM SGD}& $\mathbf{14.13}$ & $\mathbf{7.63}$ & $1.06$ & $\mathbf{22.8}$ \\ \cline{2-7}
               & \makecell{Apollo-SR $0.1$} & SVM & $69.51$ &  $25.52$ & $\mathbf{2.10}$ &  $1.16$\\ \cline{2-7}
                        & \makecell{Apollo-SR $0.2$} & SVM & $68.70$ &  $22.81$ & $2.10$ & $1.16$\\ \cline{2-7} \cline{2-7}
       &  \multirow{1}{*}{ \makecell{$200$V Cosine}}& MLP & $\mathbf{14.29}$ & $\mathbf{4.03}$ & $1.03$  & $\mathbf{22.8}$\\ \cline{2-7}
        &  \multirow{1}{*}{ \makecell{$200$V SR-$0.4$}}& MLP & $15.95$ & $13.31$ & $1.02$  & $1.77$\\ \cline{2-7}
         & \makecell{Apollo-SR $0.1$} & MLP & $69.62$ &  $23.10$ & $\mathbf{1.34}$ & $1.14$ \\ \cline{2-7}
         & \makecell{Apollo-SR $0.2$} & MLP & $673.56$ &  $\mathbf{84.70}$ & $1.32$ & $1.14$\\ \cline{1-7} \cline{1-7}
         
         % Adult
         \multirow{7}{*}{\makecell{Adult}}& \multirow{1}{*}{ \makecell{$300$V Cosine}} & \makecell{SVM SGD}& $\mathbf{0.36}$ & $\mathbf{0.2}$ & $0.37$   & $\mathbf{2.78}$\\ \cline{2-7}
                  & \makecell{Apollo-SR $0.1$} & SVM & $68.32$ &  $22.95$ & $\mathbf{0.75}$ & $0.85$ \\ \cline{2-7}
                 & \makecell{Apollo-SR $0.2$} & SVM & $68.88$ &  $22.88$ & $0.74$ & $0.85$\\ \cline{2-7} \cline{2-7}

         &  \multirow{1}{*}{ \makecell{$300$V K-Means}}& MLP & $\mathbf{0.36}$ & $\mathbf{0.19}$ & $0.30$ & $2.78$ \\ \cline{2-7}
        & \makecell{$300$V SR-$0.2$} & MLP & $6.01$ & $6.00$ &  $0.54$  & $\mathbf{3.54}$\\ \cline{2-7}
         & \makecell{Apollo-SR $0.1$} & MLP & $71.11$ &  $26.51$ & $\mathbf{1.07}$ & $1.31$\\ \cline{2-7}
         & \makecell{Apollo-SR $0.2$} & MLP & $70.16$ &  $25.74$ & $1.05$ & $1.31$\\ \cline{1-7}
         
    \end{tabular}
    % }
\end{table*}

% Table \ref{tab:table-eval-res} illustrates the model operators for each dataset and each loss function, amortized speedup and speedup from our framework in comparison with the same model operators from the Apollo \cite{b7Apollo1, b7Apollo2} framework with SR of $0.1$ and $0.2$. The values $100$V, $200$V, and $300$V in the method column correspond to the dimensions of the vector embedding used for each dataset. The lowest prediction error for each modelled operator in each dataset is highlighted in the method that is used in the similarity search step from our pipeline. Apollo outperforms our framework only on the stock dataset for SR equal with $0.1$ in the LR analytic operator for both RMSE and MAE loss function which performs $50\%$ and $6\%$ better on each loss function equivalent. While our framework for the MLP for Regression outperforms the Apollo modelled operator for $20\%$ and $84\%$ for RMSE and MAE loss functions. However, this difference in the Stock dataset for LR operator modelling is not significant. In the remaining datasets, our framework illustrates that it can outperform Apollo for different values of SR. This makes us confirm that our similarity search using similarity functions selects the most similar datasets $D_r$ from data lake directory $D$, increasing data quality and minimising $\Phi$ prediction errors on the dataset $D_o$. For the Adult dataset, our model operators also perform better, which indicates our method's advantage with an increased number of dataset features (columns). In term of speedup we can see that Apollo outperformed our framework of all modelled operators. In terms of speedup we can see that Apollo outperformed our framework of all modelled operators. This is due to the vectorisation method of our framework which consists of big complexity time. Furthermore, in amortized speedup in most of the amortized speedup in which the vectorization is not counted because it is executed only one time and can be reused our framework surpasses Apollo framework in most of the operators with a big difference with our framework to be between $10\%$ and $60\%$ faster than Apollo. Additionally, most datasets demonstrate better amortized speedup when using the SR approach within our framework. This is because the prediction process relies solely on the vector representation, rather than leveraging all dataset tuples as done in the similarity search method for operator modelling. However, in terms of prediction accuracy, the SR approach does not perform as well as the similarity search method, which achieves superior results.

Table \ref{tab:table-eval-res} compares model operators, loss functions, and speedup metrics for our framework and Apollo at SR values of $0.1$ and $0.2$. Methods $100$V, $200$V, and $300$V denote vector embedding dimensions. The lowest prediction errors align with our pipeline's similarity search method.
Apollo outperforms our framework on the Stock dataset for the LR analytic operator at SR equals with $0.1$ (with $50\%$ and $6\%$ improvements for RMSE and MAE, respectively). However, our framework excels with the MLP regression operator, improving RMSE and MAE by $20\%$ and $17\%$, respectively. The LR operator's performance gap on the Stock dataset is minor.
For other datasets, our framework consistently surpasses Apollo across different SR values. This demonstrates the effectiveness of our similarity search approach, which enhances data quality and reduces $\Phi$ prediction errors by identifying relevant datasets $D_r$ from the data lake directory $D$. The Adult dataset also highlights our framework's advantage with increasing feature dimensions.
Although Apollo achieves better raw speedup due to the higher complexity of our framework's vectorization step, our framework outperforms it in amortized speedup. By excluding the reusable vectorization process, it achieves speed gains of $10\%$ to $60\%$ for most operators.
The SR approach, leveraging vector embedding representations, enhances operator prediction compared to Apollo and achieves greater amortized speedup. However, the similarity search method outperforms both Apollo and the SR approach in prediction accuracy and amortized speedup, establishing its clear superiority across most datasets and operator scenarios.

\subsection{NumTabData2Vec Evaluation Results}

\begin{figure}[!ht]
    \centering
    \includegraphics[width=0.4\textwidth]{Figures/Results/Representation/V200_representation.pdf}
    \caption{Vector representation for each dataset from NumTabData2Vec}
    \label{fig:eval-data-repr}
\end{figure}


\begin{table}[!htp]
    \centering
    \caption{Similarity between vectors of different datasets scenarios}
    \label{tab:vec-rep-sim}
    \setlength\doublerulesep{0.5pt}
    \begin{tabular}{||c|c||}
    \hline
    Model Name & Similarity \\
    \hline\hline
     \makecell{NumTabData2Vec\\$100$ Vector size} & $0.54$\\
     \hline
      \makecell{NumTabData2Vec\\$200$ Vector size}   & $0.18$\\
      \hline
       \makecell{NumTabData2Vec\\$300$ Vector size}  & $0.16$\\ \hline
    \end{tabular}
\end{table}

% Our proposed model, \textit{NumTabData2Vec}, for dataset vectorization is compared between all the available dataset scenarios to determine whether it can effectively distinguish between them based on qualitative differences. The comparison involves selecting $n$ random datasets for each detaset scenario and projecting them into their respective vector embedding representations. Then for each dataset scenario, it gains the average vector embedding representation by the average vector embedding representation of the $n$ random datasets. The vector embedding representation for each dataset scenario depicted in Figure \ref{fig:eval-data-repr} in from the $k$-dimensional space (size of $200$) transformed to the 3d space using the PCA. Figure \ref{fig:eval-data-repr} demonstrates that each dataset occupies a distinct dimension, with non-overlapping or clustering closely together. This indicates that \textit{NumTabData2Vec} can identify the datasets from various situations and does not have a close representation like previous methods achieved it with the same accuracy but on different data types (such as word, and graphs) \cite{b8Word2Vec, b9Graph2Vec} and not in an entire dataset. Table \ref{tab:vec-rep-sim}, further illustrates the average cosine similarity between the vector embeddings of all datasets, demonstrating how dissimilar are the datasets in their vector representation. As the size dimension of the vector embedding representation increases, the model's ability to distinguish across datasets improves as their average similarity decreases. Furthermore, this indicates that larger vector dimension sizes are unneeded since between $100$ and $300$ is sufficient.

Our proposed model, \textit{NumTabData2Vec}, was evaluated to determine its ability to distinguish dataset scenarios based on qualitative differences. For each scenario, $n$ random datasets were selected, and their vector embeddings averaged to represent the scenario. These embeddings, initially in a 200-dimensional space, were projected into 3D using PCA and are shown in Figure \ref{fig:eval-data-repr}. The figure illustrates that each dataset scenario occupies a distinct space, with minimal overlap or clustering. This demonstrates that \textit{NumTabData2Vec} effectively distinguishes datasets, outperforming prior methods like Word2Vec and Graph2Vec \cite{b8Word2Vec, b9Graph2Vec}, which achieved similar accuracy but on different data types (e.g., words, graphs) rather than entire datasets. Table \ref{tab:vec-rep-sim} further highlights the average cosine similarity between dataset embeddings, showing greater dissimilarity as vector dimensions increase. However, results suggest that dimensions between $100$ and $300$ are sufficient for accurate distinction, avoiding the need for larger vector sizes.

\begin{figure}[!ht]
    \centering
    \includegraphics[width=0.4\textwidth]{Figures/Results/Representation/plot_representation_200Vectors.pdf}
    \caption{Synthetic data vector embedding representation}
    \label{fig:eval-sd-data-repr}
\end{figure}

To evaluate \textit{NumTabData2Vec}'s ability to distinguish datasets with varying row and column counts, we generated synthetic numerical tabular datasets of different dimensions and vectorized them. Figure \ref{fig:eval-sd-data-repr} shows datasets with columns ranging from three to thirty and rows from ten to one thousand, projected from a $200$-dimensional space to 2D using PCA. Each bullet caption c and r corresponds to the columns and rows of the dataset, respectively. Datasets with the same number of columns cluster closely in the representation space, and a similar pattern is observed for datasets with the same number of rows. These results indicate that our method effectively distinguishes datasets based on size during vectorization.

\begin{table}[!htp]
    \centering
    \caption{NumTabData2Vec execution time for different dataset dimensions and different vector sizes }

    \begin{adjustbox}{width=\columnwidth,center}
    \label{tab:vec-exec-time}
    \setlength\doublerulesep{0.5pt}
    \begin{tabular}{||c|c|c|c|c||}
    \hline
     \makecell{\# of columns} & \makecell{\# of rows} & \makecell{$50$ Vectors\\Execution time} & \makecell{$100$ Vectors\\Execution time} & \makecell{$200$ Vectors\\Execution time} \\
    \hline\hline
     $3$ & $100$ & $0.0004$ sec & $0.00042$ sec & $0.00051$ sec\\ \hline
     $3$ & $500$ & $0.0004$ sec & $0.00041$ sec & $0.00049$ sec\\ \hline
     $3$ & $1000$ & $0.0004$ sec & $0.00041$ sec & $0.00049$ sec\\ \hline
     $3$ & $1500$ & $0.0004$ sec & $0.00041$ sec & $0.00055$ sec\\ \hline
     $3$ & $1800$ & $0.0004$ sec & $0.00041$ sec & $0.00055$ sec\\ \hline
     \hline
     $10$ & $100$ & $0.0004$ sec & $0.0004$ sec & $0.00057$ sec\\ \hline
     $10$ & $500$ & $0.00039$ sec & $0.0004$ sec & $0.00051$ sec\\ \hline
     $10$ & $1000$ & $0.00041$ sec & $0.00042$ sec & $0.00052$ sec\\ \hline
     $10$ & $1500$ & $0.00041$ sec & $0.00042$ sec & $0.00055$ sec\\ \hline
     $10$ & $1800$ & $0.00041$ sec & $0.00042$ sec & $0.00052$ sec\\ \hline
     \hline
     $20$ & $100$ & $0.0004$ sec & $0.00042$ sec & $0.0005$ sec\\ \hline
     $20$ & $500$ & $0.0004$ sec & $0.00042$ sec & $0.0005$ sec\\ \hline
     $20$ & $1000$ & $0.00042$ sec & $0.00043$ sec & $0.00052$ sec\\ \hline
     $20$ & $1500$ & $0.00043$ sec & $0.00044$ sec & $0.00054$ sec\\ \hline
     $20$ & $1800$ & $0.00044$ sec & $0.00044$ sec & $0.00054$ sec\\ \hline    
     \hline\hline
    \end{tabular}
    \end{adjustbox}
\end{table}

To evaluate how dataset dimensions affect the execution time of \textit{NumTabData2Vec}, we created synthetic datasets with varying numbers of rows ($100$, $500$, $1000$, $1500$, and $1800$) and columns ($3$, $10$, and $20$). These datasets were vectorized into different dimensions, and the execution times were recorded. Table \ref{tab:vec-exec-time} shows that increasing the k-dimension requires approximately $20\%$ more time to generate the vector embeddings. This is expected, as a higher k-dimension involves more hyperparameters, which naturally increases computation time.

Interestingly, varying the number of columns did not significantly impact execution time. However, increasing the number of rows resulted in approximately $5\%$ additional execution time. This is because larger datasets require the extraction of more features, which has a modest impact on the model's execution time.

\begin{figure}[!ht]
    \centering
    \includegraphics[width=0.4\textwidth]{Figures/Results/Representation/plot_representation_noise_data_200Vectors.pdf}
    \caption{HPC Dataset vector embedding representation with addition of Noise}
    \label{fig:eval-nd-data-repr}
\end{figure}

To evaluate \textit{NumTabData2Vec}'s ability to distinguish datasets based on different properties like distribution and order, we introduced Gaussian noise to random $l\%$ of data tuples in an HPC dataset. Figure \ref{fig:eval-nd-data-repr} visualises the original and noise-modified datasets, projected from a 200-dimensional space to 2D using PCA. Each bullet caption g denotes the percentage of Gaussian noise added in the dataset. As noise increases, the representation space shifts further from the original dataset, indicating that \textit{NumTabData2Vec} effectively captures distribution differences. Additionally, since the HPC dataset has an inherent order, the model's sensitivity to noise demonstrates its ability to distinguish datasets based on ordering as well.

\begin{figure}[!ht]
    \centering
    \includegraphics[width=0.4\textwidth]{Figures/Results/Representation/plotrepresentationnoisedata1col200Vectors.pdf}
    \caption{HPC Dataset vector embedding representation with addition of Noise in the first column}
    \label{fig:eval-nd-data-repr-1col}
\end{figure}

To evaluate how fine-grained as distinction can be, we introduced noise into a single column and repeated the previous experiment, with the difference being that noise was added exclusively to the first column. Figure \ref{fig:eval-nd-data-repr-1col} visualizes the dataset's 2D vector space. The amount of Gaussian noise added to the dataset's first column is indicated by g in the bullet caption. The results show that as more noise is introduced to the column, the vector representation moves further away from the original dataset. In contrast to the previous experiment shown in Figure \ref{fig:eval-nd-data-repr}, the noisy dataset's representation stays closest to the original when only a single column is modified. Also in this experiment the dataset points in the 2-dimension are more grouped between them instead the previous experiment. 

%closely grouped compared to the previous experiment.
% \section{Introduction}
% This document is a model and instructions for \LaTeX.
% Please observe the conference page limits. For more information about how to become an IEEE Conference author or how to write your paper, please visit   IEEE Conference Author Center website: https://conferences.ieeeauthorcenter.ieee.org/.

% \subsection{Maintaining the Integrity of the Specifications}

% The IEEEtran class file is used to format your paper and style the text. All margins, 
% column widths, line spaces, and text fonts are prescribed; please do not 
% alter them. You may note peculiarities. For example, the head margin
% measures proportionately more than is customary. This measurement 
% and others are deliberate, using specifications that anticipate your paper 
% as one part of the entire proceedings, and not as an independent document. 
% Please do not revise any of the current designations.

% \section{Prepare Your Paper Before Styling}
% Before you begin to format your paper, first write and save the content as a 
% separate text file. Complete all content and organizational editing before 
% formatting. Please note sections \ref{AA} to \ref{FAT} below for more information on 
% proofreading, spelling and grammar.

% Keep your text and graphic files separate until after the text has been 
% formatted and styled. Do not number text heads---{\LaTeX} will do that 
% for you.

% \subsection{Abbreviations and Acronyms}\label{AA}
% Define abbreviations and acronyms the first time they are used in the text, 
% even after they have been defined in the abstract. Abbreviations such as 
% IEEE, SI, MKS, CGS, ac, dc, and rms do not have to be defined. Do not use 
% abbreviations in the title or heads unless they are unavoidable.

% \subsection{Units}
% \begin{itemize}
% \item Use either SI (MKS) or CGS as primary units. (SI units are encouraged.) English units may be used as secondary units (in parentheses). An exception would be the use of English units as identifiers in trade, such as ``3.5-inch disk drive''.
% \item Avoid combining SI and CGS units, such as current in amperes and magnetic field in oersteds. This often leads to confusion because equations do not balance dimensionally. If you must use mixed units, clearly state the units for each quantity that you use in an equation.
% \item Do not mix complete spellings and abbreviations of units: ``Wb/m\textsuperscript{2}'' or ``webers per square meter'', not ``webers/m\textsuperscript{2}''. Spell out units when they appear in text: ``. . . a few henries'', not ``. . . a few H''.
% \item Use a zero before decimal points: ``0.25'', not ``.25''. Use ``cm\textsuperscript{3}'', not ``cc''.)
% \end{itemize}

% \subsection{Equations}
% Number equations consecutively. To make your 
% equations more compact, you may use the solidus (~/~), the exp function, or 
% appropriate exponents. Italicize Roman symbols for quantities and variables, 
% but not Greek symbols. Use a long dash rather than a hyphen for a minus 
% sign. Punctuate equations with commas or periods when they are part of a 
% sentence, as in:
% \begin{equation}
% a+b=\gamma\label{eq}
% \end{equation}

% Be sure that the 
% symbols in your equation have been defined before or immediately following 
% the equation. Use ``\eqref{eq}'', not ``Eq.~\eqref{eq}'' or ``equation \eqref{eq}'', except at 
% the beginning of a sentence: ``Equation \eqref{eq} is . . .''

% \subsection{\LaTeX-Specific Advice}

% Please use ``soft'' (e.g., \verb|\eqref{Eq}|) cross references instead
% of ``hard'' references (e.g., \verb|(1)|). That will make it possible
% to combine sections, add equations, or change the order of figures or
% citations without having to go through the file line by line.

% Please don't use the \verb|{eqnarray}| equation environment. Use
% \verb|{align}| or \verb|{IEEEeqnarray}| instead. The \verb|{eqnarray}|
% environment leaves unsightly spaces around relation symbols.

% Please note that the \verb|{subequations}| environment in {\LaTeX}
% will increment the main equation counter even when there are no
% equation numbers displayed. If you forget that, you might write an
% article in which the equation numbers skip from (17) to (20), causing
% the copy editors to wonder if you've discovered a new method of
% counting.

% {\BibTeX} does not work by magic. It doesn't get the bibliographic
% data from thin air but from .bib files. If you use {\BibTeX} to produce a
% bibliography you must send the .bib files. 

% {\LaTeX} can't read your mind. If you assign the same label to a
% subsubsection and a table, you might find that Table I has been cross
% referenced as Table IV-B3. 

% {\LaTeX} does not have precognitive abilities. If you put a
% \verb|\label| command before the command that updates the counter it's
% supposed to be using, the label will pick up the last counter to be
% cross referenced instead. In particular, a \verb|\label| command
% should not go before the caption of a figure or a table.

% Do not use \verb|\nonumber| inside the \verb|{array}| environment. It
% will not stop equation numbers inside \verb|{array}| (there won't be
% any anyway) and it might stop a wanted equation number in the
% surrounding equation.

% \subsection{Some Common Mistakes}\label{SCM}
% \begin{itemize}
% \item The word ``data'' is plural, not singular.
% \item The subscript for the permeability of vacuum $\mu_{0}$, and other common scientific constants, is zero with subscript formatting, not a lowercase letter ``o''.
% \item In American English, commas, semicolons, periods, question and exclamation marks are located within quotation marks only when a complete thought or name is cited, such as a title or full quotation. When quotation marks are used, instead of a bold or italic typeface, to highlight a word or phrase, punctuation should appear outside of the quotation marks. A parenthetical phrase or statement at the end of a sentence is punctuated outside of the closing parenthesis (like this). (A parenthetical sentence is punctuated within the parentheses.)
% \item A graph within a graph is an ``inset'', not an ``insert''. The word alternatively is preferred to the word ``alternately'' (unless you really mean something that alternates).
% \item Do not use the word ``essentially'' to mean ``approximately'' or ``effectively''.
% \item In your paper title, if the words ``that uses'' can accurately replace the word ``using'', capitalize the ``u''; if not, keep using lower-cased.
% \item Be aware of the different meanings of the homophones ``affect'' and ``effect'', ``complement'' and ``compliment'', ``discreet'' and ``discrete'', ``principal'' and ``principle''.
% \item Do not confuse ``imply'' and ``infer''.
% \item The prefix ``non'' is not a word; it should be joined to the word it modifies, usually without a hyphen.
% \item There is no period after the ``et'' in the Latin abbreviation ``et al.''.
% \item The abbreviation ``i.e.'' means ``that is'', and the abbreviation ``e.g.'' means ``for example''.
% \end{itemize}
% An excellent style manual for science writers is \cite{b7}.

% \subsection{Authors and Affiliations}\label{AAA}
% \textbf{The class file is designed for, but not limited to, six authors.} A 
% minimum of one author is required for all conference articles. Author names 
% should be listed starting from left to right and then moving down to the 
% next line. This is the author sequence that will be used in future citations 
% and by indexing services. Names should not be listed in columns nor group by 
% affiliation. Please keep your affiliations as succinct as possible (for 
% example, do not differentiate among departments of the same organization).

% \subsection{Identify the Headings}\label{ITH}
% Headings, or heads, are organizational devices that guide the reader through 
% your paper. There are two types: component heads and text heads.

% Component heads identify the different components of your paper and are not 
% topically subordinate to each other. Examples include Acknowledgments and 
% References and, for these, the correct style to use is ``Heading 5''. Use 
% ``figure caption'' for your Figure captions, and ``table head'' for your 
% table title. Run-in heads, such as ``Abstract'', will require you to apply a 
% style (in this case, italic) in addition to the style provided by the drop 
% down menu to differentiate the head from the text.

% Text heads organize the topics on a relational, hierarchical basis. For 
% example, the paper title is the primary text head because all subsequent 
% material relates and elaborates on this one topic. If there are two or more 
% sub-topics, the next level head (uppercase Roman numerals) should be used 
% and, conversely, if there are not at least two sub-topics, then no subheads 
% should be introduced.

% \subsection{Figures and Tables}\label{FAT}
% \paragraph{Positioning Figures and Tables} Place figures and tables at the top and 
% bottom of columns. Avoid placing them in the middle of columns. Large 
% figures and tables may span across both columns. Figure captions should be 
% below the figures; table heads should appear above the tables. Insert 
% figures and tables after they are cited in the text. Use the abbreviation 
% ``Fig.~\ref{fig}'', even at the beginning of a sentence.

% \begin{table}[htbp]
% \caption{Table Type Styles}
% \begin{center}
% \begin{tabular}{|c|c|c|c|}
% \hline
% \textbf{Table}&\multicolumn{3}{|c|}{\textbf{Table Column Head}} \\
% \cline{2-4} 
% \textbf{Head} & \textbf{\textit{Table column subhead}}& \textbf{\textit{Subhead}}& \textbf{\textit{Subhead}} \\
% \hline
% copy& More table copy$^{\mathrm{a}}$& &  \\
% \hline
% \multicolumn{4}{l}{$^{\mathrm{a}}$Sample of a Table footnote.}
% \end{tabular}
% \label{tab1}
% \end{center}
% \end{table}

% \begin{figure}[htbp]
% \centerline{\includegraphics{fig1.png}}
% \caption{Example of a figure caption.}
% \label{fig}
% \end{figure}

% Figure Labels: Use 8 point Times New Roman for Figure labels. Use words 
% rather than symbols or abbreviations when writing Figure axis labels to 
% avoid confusing the reader. As an example, write the quantity 
% ``Magnetization'', or ``Magnetization, M'', not just ``M''. If including 
% units in the label, present them within parentheses. Do not label axes only 
% with units. In the example, write ``Magnetization (A/m)'' or ``Magnetization 
% \{A[m(1)]\}'', not just ``A/m''. Do not label axes with a ratio of 
% quantities and units. For example, write ``Temperature (K)'', not 
% ``Temperature/K''.

% \section*{Acknowledgment}

% The preferred spelling of the word ``acknowledgment'' in America is without 
% an ``e'' after the ``g''. Avoid the stilted expression ``one of us (R. B. 
% G.) thanks $\ldots$''. Instead, try ``R. B. G. thanks$\ldots$''. Put sponsor 
% acknowledgments in the unnumbered footnote on the first page.

\newpage
\bibliographystyle{IEEEtran}
\bibliography{refs}
\sloppy
% Please number citations consecutively within brackets \cite{b1}. The 
% sentence punctuation follows the bracket \cite{b2}. Refer simply to the reference 
% number, as in \cite{b3}---do not use ``Ref. \cite{b3}'' or ``reference \cite{b3}'' except at 
% the beginning of a sentence: ``Reference \cite{b3} was the first $\ldots$''

% Number footnotes separately in superscripts. Place the actual footnote at 
% the bottom of the column in which it was cited. Do not put footnotes in the 
% abstract or reference list. Use letters for table footnotes.

% Unless there are six authors or more give all authors' names; do not use 
% ``et al.''. Papers that have not been published, even if they have been 
% submitted for publication, should be cited as ``unpublished'' \cite{b4}. Papers 
% that have been accepted for publication should be cited as ``in press'' \cite{b5}. 
% Capitalize only the first word in a paper title, except for proper nouns and 
% element symbols.

% For papers published in translation journals, please give the English 
% citation first, followed by the original foreign-language citation \cite{b6}.

% \begin{thebibliography}{00}
% \bibitem{b1} G. Eason, B. Noble, and I. N. Sneddon, ``On certain integrals of Lipschitz-Hankel type involving products of Bessel functions,'' Phil. Trans. Roy. Soc. London, vol. A247, pp. 529--551, April 1955.
% \bibitem{b2} J. Clerk Maxwell, A Treatise on Electricity and Magnetism, 3rd ed., vol. 2. Oxford: Clarendon, 1892, pp.68--73.
% \bibitem{b3} I. S. Jacobs and C. P. Bean, ``Fine particles, thin films and exchange anisotropy,'' in Magnetism, vol. III, G. T. Rado and H. Suhl, Eds. New York: Academic, 1963, pp. 271--350.
% \bibitem{b4} K. Elissa, ``Title of paper if known,'' unpublished.
% \bibitem{b5} R. Nicole, ``Title of paper with only first word capitalized,'' J. Name Stand. Abbrev., in press.
% \bibitem{b6} Y. Yorozu, M. Hirano, K. Oka, and Y. Tagawa, ``Electron spectroscopy studies on magneto-optical media and plastic substrate interface,'' IEEE Transl. J. Magn. Japan, vol. 2, pp. 740--741, August 1987 [Digests 9th Annual Conf. Magnetics Japan, p. 301, 1982].
% \bibitem{b7} M. Young, The Technical Writer's Handbook. Mill Valley, CA: University Science, 1989.
% \bibitem{b8} D. P. Kingma and M. Welling, ``Auto-encoding variational Bayes,'' 2013, arXiv:1312.6114. [Online]. Available: https://arxiv.org/abs/1312.6114
% \bibitem{b9} S. Liu, ``Wi-Fi Energy Detection Testbed (12MTC),'' 2023, gitHub repository. [Online]. Available: https://github.com/liustone99/Wi-Fi-Energy-Detection-Testbed-12MTC
% \bibitem{b10} ``Treatment episode data set: discharges (TEDS-D): concatenated, 2006 to 2009.'' U.S. Department of Health and Human Services, Substance Abuse and Mental Health Services Administration, Office of Applied Studies, August, 2013, DOI:10.3886/ICPSR30122.v2
% \bibitem{b11} K. Eves and J. Valasek, ``Adaptive control for singularly perturbed systems examples,'' Code Ocean, Aug. 2023. [Online]. Available: https://codeocean.com/capsule/4989235/tree
% \end{thebibliography}

% \vspace{12pt}
% \color{red}
% IEEE conference templates contain guidance text for composing and formatting conference papers. Please ensure that all template text is removed from your conference paper prior to submission to the conference. Failure to remove the template text from your paper may result in your paper not being published.

\end{document}
