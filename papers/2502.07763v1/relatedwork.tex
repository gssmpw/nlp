\section{Related work}
\label{sec:backrelated}
%\vspace{-1.0mm}

\textbf{Problem-solving styles:} Research reveals that developers exhibit diverse problem-solving styles~\cite{burnett2016gendermag} and motivations~\cite{gerosa2021shifting}. For instance, studies indicate that women are often task-oriented, whereas men are frequently motivated to explore new technologies for enjoyment~\cite{padala2020gender, mendez2018open, mendez2018gender}. These problem-solving differences impact how newcomers engage with OSS. In this vision paper, we would like to address how we can personalize LLMs' responses to help newcomers contribute to OSS projects.

\textbf{Diversity in OSS:} Low diversity within OSS communities is a well-documented concern across various dimensions, including gender~\cite{trinkenreich2021women, guizani2022debug, bosu2019diversity, terrell2016gender}, language~\cite{storey2016social}, and geographic location~\cite{storey2016social}. Research shows that diverse teams are generally more productive~\cite{vasilescu2015gender}, yet minorities often face significant barriers when trying to join and thrive within OSS communities~\cite{trinkenreich2021women}. %Many OSS communities operate as meritocracies~\cite{feller2000framework}, where minorities frequently report experiencing “imposter syndrome”\cite{vasilescu2015gender}. These competitive environments can discourage participation from underrepresented groups, particularly women\cite{miller2012toward, vugt2007gender}. Observational studies have found that men often dominate code contributions, marginalizing the social support networks essential for fostering inclusivity~\cite{nafus2012patches}. Cultures that self-identify as meritocracies are often male-dominated, creating environments that women often find unwelcoming~\cite{turkle2005second}. 
This vision paper aims to reduce these biases by leveraging LLMs to support diverse problem-solving styles, fostering a more inclusive OSS environment.