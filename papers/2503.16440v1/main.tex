\documentclass[utf8]{article}


\usepackage{arxiv}

\usepackage[utf8]{inputenc} % allow utf-8 input
\usepackage[T1]{fontenc}    % use 8-bit T1 fonts
\usepackage{hyperref}       % hyperlinks
\usepackage{url}            % simple URL typesetting
\usepackage{booktabs}       % professional-quality tables
\usepackage{amsfonts}       % blackboard math symbols
\usepackage{nicefrac}       % compact symbols for 1/2, etc.
\usepackage{microtype}      % microtypography
\usepackage{lipsum}
\usepackage{subfiles}
\usepackage{graphicx, xcolor}
\usepackage{mparhack}
\usepackage{subcaption}
\graphicspath{ {./} }

% custom packages %

\usepackage{soul}
\usepackage{color, colortbl}

\usepackage{amsmath}
\numberwithin{equation}{section}

% MULTI-COL FOR QUESTIONNAIRE
    \usepackage{etoolbox,refcount}
    \usepackage{multicol}
    \usepackage{scalerel,amssymb}
    
    \newcounter{countitems}
    \newcounter{nextitemizecount}
    \newcommand{\setupcountitems}{
      \stepcounter{nextitemizecount}
      \setcounter{countitems}{0}
      \preto\item{\stepcounter{countitems}}
    }
    \makeatletter
    \newcommand{\computecountitems}{
      \edef\@currentlabel{\number\c@countitems}
      \label{countitems@\number\numexpr\value{nextitemizecount}-1\relax}
    }
    \newcommand{\nextitemizecount}{
      \getrefnumber{countitems@\number\c@nextitemizecount}
    }
    \newcommand{\previtemizecount}{
      \getrefnumber{countitems@\number\numexpr\value{nextitemizecount}-1\relax}
    }
    \makeatother    
    \newenvironment{AutoMultiColItemize}{
    \ifnumcomp{\nextitemizecount}{>}{1}{\begin{multicols}{3}}{}
    \setupcountitems\begin{itemize}}
    {\end{itemize}
    \unskip\computecountitems\ifnumcomp{\previtemizecount}{>}{1}{\end{multicols}}{}}
% =====================================

\usepackage{multirow}
\usepackage{pdflscape}
\usepackage{float}

\usepackage{cleveref}  % has to be last!

\usepackage{subcaption}

% --------- colours ----------------- %
\definecolor{highlightTable}{gray}{0.9}
% ----------------------------------- %
\def\leftarrowCircle{\hbox{$\leftarrow$$\circ$}}
\def\rightarrowCircle{\hbox{$\circ$$\rightarrow$}}


% ------------------- %
% custom commands %
\newcommand{\set}[1]{\{#1\}}

% commands for causality
\newcommand{\indep}{\perp \!\!\! \perp} % independence symbol
\newcommand{\edgeDir}[2]{\mathit{#1} \rightarrow \mathit{#2}}
\newcommand{\edgeDirConf}[2]{\mathit{#1} \rightarrowCircle \mathit{#2}}
\newcommand{\edgeRev}[2]{\mathit{#1} \leftarrow \mathit{#2}}
\newcommand{\edgeConf}[3]{\mathit{#1} \leftarrow \mathit{#2} \rightarrow \mathit{#3}}
\newcommand{\edgeColl}[3]{\mathit{#1} \rightarrow \mathit{#2} \leftarrow \mathit{#3}}
\newcommand{\edgeAsso}[2]{\mathit{#1} - \mathit{#2}}
\newcommand{\edgeConfHidden}[2]{\mathit{#1} \longleftrightarrow \mathit{#2}}

% Commands for meta infos
\newcommand{\makeAuthor}[4]{#1\\#2\\#3\\\texttt{#4}\\}
\newcommand{\makeKeywords}{
virtual reality \and 
haptic feedback \and
causal discovery \and
causal representation \and
sensorimotor behavior
}

% -------------------------- %

\title{Cause-effect perception in an object place task}

 \author{
  \makeAuthor{Nikolai Bahr}{Cognitive Neuroinformatics, University of Bremen}{Bremen, Germany}{nibahr@uni-bremen.de}
  \And
  \makeAuthor{Christoph Zetzsche}{Cognitive Neuroinformatics, University of Bremen}{Bremen, Germany}{zetzsche@informatik.uni-bremen.de}
  \And
  \makeAuthor{Jaime Maldonado}{Cognitive Neuroinformatics, University of Bremen}{Bremen, Germany}{jmaldonado@uni-bremen.de}
  \And
  \makeAuthor{Kerstin schill}{Cognitive Neuroinformatics, University of Bremen}{Bremen, Germany}{kschill@uni-bremen.de}
 }

\begin{document}
\maketitle

% ------------------ ABSTRACT ------------------------- %
\begin{abstract}

Algorithmic causal discovery is based on formal reasoning and provably converges toward the optimal solution. 
However, since some of the underlying assumptions are often not met in practice no applications for autonomous everyday life competence are yet available.
Humans on the other hand possess full everyday competence and develop cognitive models in a data efficient manner with the ability to transfer knowledge between and to new situations.
Here we investigate the causal discovery capabilities of humans in an object place task in virtual reality (VR) with haptic feedback and compare the results to the state of the art causal discovery algorithms FGES, PC and FCI.
In addition we use the algorithms to analyze causal relations between sensory information and the kinematic parameters of human behavior.

Our findings show that the majority of participants were able to determine which variables are causally related. 
This is in line with causal discovery algorithms like PC, which recover causal dependencies in the first step. 
However, unlike such algorithms which can identify causes and effects in our test configuration, humans are unsure in determining a causal direction.
Regarding the relation between the sensory information provided to the participants and their placing actions (i.e. their kinematic parameters) the data yields a surprising dissociation of the subjects knowledge and the sensorimotor level.
Knowledge of the cause-effect pairs, though undirected, should suffice to improve subject's movements. 
Yet a detailed causal analysis provides little evidence for any such influence. 
This, together with the reports of the participants, implies that instead of exploiting their consciously perceived information they leave it to the sensorimotor level to control the movement. 

\end{abstract}

\keywords{\makeKeywords}

% ------------------ INTRO ------------------------- %
\section{Introduction}

Algorithmic causal discovery is formally sound and converges towards the optimal solution. Regardless of these arguably nice properties there are still no working applications in autonomous everyday task fulfillment. Part of this absence in robotic contexts might be that the needed assumptions are often not strictly met in practice, the representation of the causal system has to be given (relevant variables, etc.). Additionally working with large sets of variables is particularly hard due to the explosion in combinatorial space. Humans on the other hand possess full everyday competence in spite of these problems. 

This paper investigates how causal information can influence human behavior. In particular we want to answer two research questions.
First, can causal relations be identified and perceived by human subjects?
Second, can causal relations be used to determine suitable body movements?

For this, we developed an experimental setup in Virtual Reality (VR) in which human subjects perform the natural task of moving an object and placing it at a desired location. In particular the subjects had to move a breakable glass as fast as possible. Special care was taken to ensure realism of the setup by providing high quality haptic feedback to the subjects by the use of a \emph{PHANToM} haptic device.

For the causal relations we used a dependence of the breakability on two object features, the glass weight and the glass color. The motivation for using these two features is given by the fact, that causal relations are difficult to investigate in natural tasks, due to the possible critical influence of long-term experience of the subjects with certain everyday causal relations. Therefore, one of the used relations, the dependence of the breakability on the glass weight, is compatible with everyday experience whereas the dependency on the glass color is assumed to represent a normal, not yet experienced property

Causal relations can in general not be completely identified by observation alone. Therefore, we used the only basic causal configuration for which this is possible (\cite{spirtes2001causation}).
Furthermore the state of the art causal discovery algorithms FGES, PC and FCI are employed to provide a baseline comparison to the human capabilities for the identification of causal relationships and for the inspection of the causal relations between the stimulus presented to each participant and their motor output (i.e. kinematic parameters). 

Experiments related to our work have been performed by \cite{steyvers2003inferring},\cite{lagnado2004advantage} and \cite{hagmayer2000simulating}.
\cite{steyvers2003inferring} conduct several experiments in which subjects had to make structural judgments about a causal system they could only observe and compare that to judgments made when the participants were able to observe effects of self performed interventions. The experiments are conducted on the basis of an abstract cover story, in which three aliens were capable of reading each others mind. Based on the underlying causal structure, either two aliens read the mind of the third or one alien read the mind of the other two. Based on the thoughts of each of these aliens the participants now had to select the underlying causal model. In the first experiment, in which only observational data was available, the participants had knowledge about the two possible models, whereas in the other experiments, every configuration was possible (we use the structure of the common effect model also in our study, but here the participants have no prior knowledge about alternative models and have to completely identify the structure on observational data only).
Overall they have shown, that learning by intervention provides an advantage for human subjects. However, in all cases the performance was far from optimal.
\cite{hagmayer2000simulating} have also compared common cause and common effect models, but did not require their subjects to provide a full identification of the respective causal structures. 

\cite{lagnado2004advantage} also compare human learning with intervention to learning by pure observation through multiple experiments. Additionally they analyzed the effects temporal cues may have. However, other than \cite{steyvers2003inferring} they use binary variables and employ corrective feedback. Also their underlying model for generating trial data is a causal chain (A influences B, which influences C). Their cover stories involved either acid and ester levels and a resulting perfume or temperature and pressure level and the launch of a rocket. Overall the experiments provided evidence that mere observation is insufficient in their case and that information through intervention increases structure identification performance of human subjects. Further their experiments provide evidence that this advantage requires temporal cues available to the intervening participants. 

\cite{rottman2014reasoning} provides an extensive overview of further research about human causal identification capabilities in a variety of settings.

% ------------------ METHODS ------------------------- %
\section{Methods}\label{sec:methods}

\subsection{Experiment}\label{subsec:methods-dataset}
The data for the analysis has been obtained by conducting an experiment with n=21 healthy human adults. In this experiment, the participants had to move a glass from one plate to the other as fast as possible without breaking the glass (this will be further explained in the following sections and in \cref{appendix:experiment_trial_design})).

The experiment is structured into three blocks, each consists of a series of 80 trials with a questionnaire and a succeeding break afterward. The motivation for this is as follows: In the first session, the participant is only told to move as fast as possible without mentioning the questionnaire. Thereby the participant does not pay attention to any possible relationships. In the second phase, the participant knows about the following questionnaire and is allowed to experiment and explore possible strategies without any constraints or optimization targets. Note, that the mechanisms that generate the properties of the glass do not change throughout the experiment. Therefore the participant is still only able to observe, but without any distracting constraints. However, the third session is the same as the first, such that participants have the instruction to move as fast as possible without breaking the glass.

Each participant had an introductory warm-up phase to get used to the setup. The design was verified in a pilot phase with 3 participants.

\subsubsection{Questionnaire}
After each session, the user is tasked to create a causal graph consisting of the nodes \textit{glass color, glass weight} and \textit{glass stability}. To ease the construction, the participant may also answer binary questions regarding the existence of all possible relations. Based on the answer, the instructor then draws the causal graph as a graphical representation of the answers. The questionnaire can be seen in the appendix.

All other properties (e.g. regarding the shape or 3D model of the glass in general) are held fixed. Changing the glass weight and the glass color is due to careful considerations regarding background knowledge of the participants. The relationship between the weight of an object and its stability may be acquired through everyday life. For example, fragile materials benefit from an increased width such that the overall stability increases. This additional material will consequently result in higher weight. For the color however, there is no such simple rule. Therefore this link between the stability of the glass and its color can be seen as \textit{more artificial} than the weight-stability relationship. This is considered as to reduce the interference of acquired knowledge throughout the experiment and prior knowledge.
After the last session, participants are asked about their optimization strategy.

\subsubsection{Hard- and software}

To set up the VR simulation, we used the development edition of \emph{Worldviz Vizard 5.9, 64 Bit}\footnote{Available for a fee at \href{https://www.worldviz.com/vizard-virtual-reality-software}{https://www.worldviz.com/vizard-virtual-reality-software}} for \emph{python 2.7.12}. Furthermore, we used the \emph{PHANToM Premium 1.5 (High Force Model)} with the \emph{PHANToM device driver Version 5.1.7} (\cref{fig:phantom}). It is employed as a haptic device to function as an input interface for the participants and as an output device to render the force given by the VR. Before each session, we re-calibrated the haptic device with the \emph{PHANToM configuration utility for windows}, also \emph{version 5.1.7}. In our code for the simulation, we utilized the developer version of the \emph{sensable3e-plugin} from \emph{openHaptics}\footnote{Available for a fee at \href{https://de.3dsystems.com/haptics-devices/openhaptics}{https://de.3dsystems.com/haptics-devices/openhaptics}}.

\subsubsection{Setup}

The participant is seated in a frame with a monitor pointing downwards and mounted right above the head. The haptic device is then placed in front and in reach of the test subject. To make the content of the monitor available, an angled mirror is placed directly in front of the subject. The mirror thus also hides the participant's hand. Note that the frame blocks the participant's peripheral view by the way it is constructed.

\subsubsection{Trial design and causal relations}

\begin{figure}
    \centering
    \begin{subfigure}{0.61\textwidth}
        \includegraphics[width=\textwidth]{pick_place_experiment.png}
        \caption{VR-environment of the study. One can see the two lower plates and a bar in the upper half of the screen. The start plate appears blue to indicate it has to be touched next. In the top left corner is a window that shows the top view to make it easier to localize the glass}
        \label{fig:vr_env}
    \end{subfigure}
    \begin{subfigure}{0.38\textwidth}
        \includegraphics[width=\textwidth]{studyGraphStructureBig.png}
        \caption{Causal model of the data}
        \label{fig:causal_model_data}
    \end{subfigure}

    \centering
    \begin{subfigure}{0.39\textwidth}
        \centering
        \includegraphics[width=\textwidth]{phantom.jpg}
        \caption{PHANToM haptic device used in the experiment}
        \label{fig:phantom}
    \end{subfigure}
\end{figure}

The VR setting is made up of two plates, an indicator bar, and a closed hand holding a glass of wine (see \cref{fig:vr_env}). As described earlier, a haptic device will be used to control the glass by the user. It renders the force applied to the glass by the weight, either one of the two plates or the floor. A trial consists of placing the glass on the right plate and moving it on the left plate while touching the bar at the top at some point in between. 

The subject is instructed not to break the glass and to move the glass from the starting plate to the end plate as fast as possible. The glass will break if the haptic device registers a force above a given threshold, i.e. the variable \textit{Glass-OK} in the causal model becomes zero (see \cref{fig:causal_model_data}). The threshold (i.e. the stability of the glass) is calculated by

\begin{equation}
    \label{eq:causal_force_threshold_not_app}
    f_{\mathit{force\_threshold}}(\mathit{color}, \mathit{weight}) = (2.5\cdot \mathit{weight} + \mathit{color\_offset}(\mathit{color}))\cdot 0.8241
\end{equation}

with 

\begin{equation}
    \label{eq:causal_sampled_features_not_app}
    \begin{array}{lcl}
    \mathit{color} & \sim & \mathcal{U}(\set{\text{red, green, blue}}) \\
    \mathit{weight} & \sim & \mathcal{U}([0.2616, 1])
\end{array}
\end{equation}

and color specific offsets with the following order: $\textrm{blue} > \textrm{green} > \textrm{red}$. Therefore, on average blue, is the most stable one and red the least stable. 

The calculation of the stability of the glass forms a collider in the causal graph. This form of interaction is deliberate, as collider/common effect structures can be identified based on correlational or observational data alone. Other forms of interaction (chain, confounder/common cause structure) leave the same trace in the data, so they are indistinguishable from another. This is why they are in the same \emph{Markov Equivalence Class (MEC)} while colliders have a different MEC (see e.g.\cite{spirtes2001causation}). 

A more detailed explanation of the trial design and the causal model can be found in \cref{appendix:experiment_trial_design} and \cref{appendix:experiment-causal_relations}.

\subsection{Data preparation and analysis}\label{subsec:methods-data_analysis}

For each participant there are 3 datasets, one for each session with 80 samples each. Additionally there is a dataset with 240 samples, composed of the concatenation of all three. This dataset resembles the experience of the participants after the experiment. Within each dataset, there are two categorical and six continuous features, making a total of eight:

\begin{enumerate}
    \item \textit{glass\_weight} - continuous.
    \item \textit{glass\_color} - categorical.
    \item \textit{force\_value} - continuous. The force which is exerted on the glass during the placement action.
    \item \textit{force\_threshold} - continuous. The stability of the glass.
    \item \textit{glass\_ok} - categorical/binary. Did the glass break?
    \item \textit{peak\_speed} - continuous. The maximal speed with which the user moved the glass.
    \item \textit{duration} - continuous. The duration of the movement after the trial started (see states \textit{Transport Pre} and \textit{Transport Post} in \cref{appendix:experiment_trial_design}).
    \item \textit{PTPS} - continuous. Partial Time Peak Speed denotes the time taken to reach peak speed from the start of the movement (see states \textit{Transport Pre} and \textit{Transport Post} in \cref{appendix:experiment_trial_design}).
\end{enumerate}

To calculate the velocity metrics necessary for features like \textit{PTPS} or \textit{peak\_speed} the data is appropriately preprocessed as suggested by \cite{winter2009biomechanics}. This involves lowpass-filtering with a fourth-order, zero-lag butterworth filter and a preceding residual analysis (\cite{winter2009biomechanics,yu1999estimate}).

On the datasets with 240 samples the causal discovery algorithms PC, FCI and FGES are applied.
PC and FCI are both constrained based algorithms, which use conditional independence tests to identify the skeleton and all collider-structures within the data. The FGES algorithm on the other hand is score based. This class of algorithms repeatedly score, compare and evolve candidate graphs by using a criterion such as the BIC score (see \cite{spirtes2001causation, kalainathan2022structural} for an overview). 

For PC and FCI the conditional-gaussian BIC (\cite{andrews2018CGBIC}) and for FGES the degenerate gaussian BIC-score (\cite{andrews2019DGBIC}) are used. For the implementation of the algorithms we used the CLI of the \emph{TETRAD toolbox for causal discovery}\footnote{\href{https://github.com/bd2kccd/causal-cmd}{https://github.com/bd2kccd/causal-cmd}} (\cite{ramsey2018tetrad}). For further analyses of the results we used the causal-learn python package\footnote{\href{https://github.com/py-why/causal-learn}{https://github.com/py-why/causal-learn}} (\cite{zheng2024causal}).

% ------------------ RESULTS ------------------------- %
\section{Results}

\subsection{Human Causal Discovery}
Before the results of the questionnaire are displayed, it is important to note that none of the participants oriented observed relations between variables. All participants reported, that they refrained from distinguishing cause from effect, because more data would be needed to do so. Therefore, only the \textit{undirected relation} between variables can be analyzed further.
\Cref{fig:results-questionnaire_answers} shows the aggregated results of the questionnaire. The thickness of the connection corresponds to the number of participants which identified a relation between the connected variables. The dashed line indicates an erroneously identified connection.

Considering the assessments of the humans in its entirety, the structure identification becomes more accurate over the course of the experiment. The connection between the weight of the glass and its stability was the most prominently identified one. In the first session, $57\%$ of participants suspected it, $67\%$ in the second session and $76\%$ in the last session. The relation between the glass color and its stability seems to be harder to identify, as $24\%$ recognized the relation in the first session, which increases to $43\%$ in the succeeding sessions. 

After the first session, some participants noted, that they expected the weight of the glass to be connected to its color. Thereby indicating a strong bias towards establishing a causal link between both properties.
Accordingly we observed that $24\%$ of the participants draw a connection between the two variables in the test after the first two sessions. However, the percentage drops to less than $5\%$ after the last session, when more empirical data for their decision has been gathered by the subjects.

Considering the causal algorithms, all three relatively robustly recovered the ground truth causal graph, albeit FGES was the most stable one.

\begin{figure}
    \centering
    \begin{subfigure}{0.5\textwidth}
        \centering
        \includegraphics[width=\linewidth]{participantGraph_Raw.png}
        \caption{Session 1}
        \label{fig:results-questionnaire_answers_raw}
    \end{subfigure}

    \begin{subfigure}[b]{0.49\textwidth}
        \centering
        \includegraphics[width=\linewidth]{participantGraph_Train.png}
        \caption{Session 2}
        \label{fig:results-questionnaire_answers_train}
    \end{subfigure}
    \begin{subfigure}[b]{0.49\textwidth}
        \centering
        \includegraphics[width=\linewidth]{participantGraph_Test.png}
        \caption{Session 3}
        \label{fig:results-questionnaire_answers_test}
    \end{subfigure}
    \centering
    \caption{Aggregated results of the questionnaire. The thickness of the lines corresponds to how many participants identified the relation between the connected variables. The dashed connection indicates that this connection is erroneous. One can see that the fractions of correct connections increase over the course of the experiment, while the erroneous decrease.}
    \label{fig:results-questionnaire_answers}
\end{figure}

\subsection{Sensorimotor}
When asked, the participants reported optimization strategies agnostic to the properties of the glass (eg. optimization of muscle control and the movement path, reduction of the distance to move, etc.). Further, the causal discovery algorithms applied did not identify any connectivity between the properties of the glass (weight, color, stability) and the recorded metrics either, apart from one exception. There seems to be a link between the glass weight and the force exerted on the glass by the participant during the placement action.

A summary of the subsequent analysis of the link can be seen in \cref{fig:results-weight_force_relationship}. Albeit there is quite the variability in the data, the median correlation of the weight and the exerted force is higher in the last session when compared to the first. Please also note, that there is never a negative correlation (which would correspond to an increased caution when placing a heavy glass). Additionally the difference between the exerted force on the glass and the needed force to break the glass is reduced in the last session compared to the first. 

\begin{figure}
    \centering
    \begin{subfigure}{\textwidth}
        \centering
        \includegraphics[width=\linewidth]{correlationScatter.png}
        \caption{Examples for different levels of correlation of the glass weight and the exerted force by the participant on the glass (both in N). The data has been recorded in the last session of the experiment and stem from three different participants. Each circle represents a unique trial and the black line indicates a linear regression of the force on the weight.}
        \label{fig:results-weight_force_relationship_examples}
    \end{subfigure}

    \centering
    \begin{subfigure}{0.5\textwidth}
        \centering
        \includegraphics[width=\linewidth]{correlationBoxplot.png}
        \centering
        \caption{Correlation coefficients of glass-weight and exerted force as a Tukey plot. The blue shaded area represents the spread of the middle half of the data (IQR). The whisker length indicates the variability of the other half. Outliers are considered to be points beyond 1.5 times of the IQR. The white line inside of the box shows the median of the data.}
        \label{fig:results-weight_force_relationship_aggregation}
    \end{subfigure}
    \caption{Relationship of the glass-weight and the exerted force on the glass by the participant during the placement-action.}
    \label{fig:results-weight_force_relationship}
\end{figure}

% ------------------ DISCUSSION ------------------------- %
\section{Discussion}
The fact that none of the participant distinguished cause from effect in our study has been observed on other occasions as well. A plausible approach to such problems is by using associative reasoning, which is directionless by definition (\cite{rottman2014reasoning, rehder2001causal}). Others have reported, that humans may perform local reasoning schemes by only considering two variables at a time (\cite{rottman2014reasoning, kruschke2006locally, fernbach2011asymmetries, waldmann2008causal}). In fact, verbalizations by the participants during the questionnaire suggest utilization of this strategy. Furthermore it is proposed, that people sometimes may add imaginary nodes to such graphs (\cite{rehder2005feature}, see \cite{rottman2014reasoning}). Therefore an imagination of a hidden common cause would render possible causal relations as spurious correlations without directionality. The removal of a link representing such a correlation would then depend on the interpretation of the question by the participant.

Considering the structure identification capabilities of people on an abstract level, little evidence has been reported that covariations based purely in data is sufficient (\cite{lagnado2002learning, lagnado2004advantage, steyvers2003inferring}, see \cite{lagnado2007beyond}). Our results in this arguably more realistic setting shed a more optimistic perspective on human structure identification competence.
\\

To discuss the observed results from a sensorimotor perspective, we first have to establish three different levels of processing:
First, the consciously perceived rules and causal structure in the system. This is the level which is probed by the questionnaire after each session. The second level describes the self-reported and therefore rationalized behavior and optimization strategies. To access this level, the participants had to report how they try to optimize their movements. 
These two levels both concern the consciously perceived information of the participants. The third level, by contrast, is related to the sensorimotor processing that determines the participant's movement. Therefore, only the  motor forces measured by the \emph{PHANToM} device provide the data about this partly unconscious behavior.

This distinction of the three levels allows the observation of an interesting pattern of dissociation. On the first level, the relationship between the weight and the stability of the glass can be clearly perceived and verbalized. This relationship also appears to be available to the third level, where participants seem to improve over the course of the experiment at placing heavier glasses with less precision without breaking them. An important possibility is that this allows for exploiting the speed-accuracy-tradeoff (\cite{burdet1998quantization}). However, when probed, no participant seemed to notice this pattern in their own behavior or incorporate it into a conscious strategy at the second level. 

Considering the color of the glass, the pattern changes. First, the connection between the color and stability of the glass is also identifiable, albeit with more difficulty than the connection between stability and weight. However, in this case, there is no evidence of an effect, neither at the conscious strategy level nor at the executive sensorimotor level.
A possible explanation of this change in dissociation may be provided by prior knowledge and beliefs. It has been shown, that humans display high competence in tuning hypothetical structural knowledge to gathered data and utilize it (\cite{waldmann1996knowledge, waldmann2001estimating}, see \cite{lagnado2007beyond}). In comparison, the weight-stability link can be seen as more plausible than that of the color-stability relation (e.g. more weight can be an indicator for more material/thicker walls which provides more stability). It should be remebered that this difference in plausibility has also been a key consideration for the design of the experiment to reduce interference with possible prior knowledge in at least one link. In addition to the missing plausibility the color-stability connection may also be identified with too much uncertainty, so that without more empirical data it cannot be robustly integrated into the optimization loop. 

Another interesting observation is, that some participants noted a strong prior belief that weight and color will possibly be connected in our experimental setup. With this in mind it is not surprising, that this connection does also appear in the graphs identified by the subjects after the first session, as these biases have a large impact on the (perhaps erroneous) detection of relationships. A recognition of correlation in data is simpler than rejecting a hypothesized relationship (\cite{waldmann1996knowledge, waldmann2001estimating,mercier2022confirmation,klayman1995varieties}, see \cite{lagnado2007beyond}). However, in the course of our study participants were finally capable of rejecting this presumed connection on the basis of gathered empirical data during the whole experiment. Almost none subject still suspected this link after the third session.\\

In conclusion, this work provides some evidence that humans have greater competence in structure identification tasks that more closely resemble reality than in abstract scenarios. However, determination of the direction of the causal connections seems more difficult though it is theoretically possible for the causal structure of our experiment. Whether this reflects a genuine problem of humans to deduce this causal direction from purely observational data or whether it can be reached with longer training remains to be determined.
Additionally, we observed interesting patterns of dissociation between the sensorimotor behavior, the rationalization of such behavior and reasoning on an abstract level. This latter observation should be seen as motivation for more extensive experimental studies on how causal information could have a different influence on our perception of the environment and on our motor interaction with it.

% ------------------ ACKNOWLEGEMENTS ------------------------- %
\section{Conflict of interest statement}
The authors declare that the research was conducted in the absence of any commercial or financial relationships that could be construed as a potential conflict of interest.

\section{Funding}
This work has been supported by the German Research Foundation DFG, as part of Collaborative Research Center (Sonderforschungsbereich) 1320 Project-ID 329551904 "EASE - Everyday Activity Science and Engineering", University of Bremen (\href{http://www.ease-crc.org/}{http://www.ease-crc.org/}). The research was conducted in subproject H01 "Sensorimotor and Causal Human Activity Models for Cognitive Architectures".

\section{Acknowledgments}
Usage and optimization of the causal discovery algorithms were possible thanks to discussion and support of Konrad Gadzicki (EASE - subproject H03 "Discriminative and Generative Human Activity Models for Cognitive Architectures")

\section{Ethics}
The experiments have been approved by the ethics commitee of the university of Bremen regarding the project H01 "Sensorimotor and Causal Human Activity Models for Cognitive Architectures" as part of Collaborative Research Center (Sonderforschungsbereich) "EASE - Everyday Activiy Science and Engineering".

\clearpage

\bibliographystyle{unsrt}
\bibliography{references}

\clearpage

\appendix

% ------------------ APP:EXPERIMENT ------------------------- %

\section{Experiment trial design}\label{appendix:experiment_trial_design}

The VR setting is made up of two plates, and indicator bar, and a closed hand holding a glass of wine as depicted in \cref{fig:vr_env}. A trial consists of placing the glass on the right plate, moving it on the left plate and touch the bar at the top at some point in between. 

The subject is instructed not to break the glass and to perform the states \textit{Transport Pre} and \textit{Transport Post} (see \cref{enum:methods_transport-pre} and \cref{enum:methods_transport-post}) as fast as possible. The glass will break if the variable \textit{result} in the causal model becomes zero, i.e. the haptic device registers a force above a given threshold (see \textbf{\nameref{appendix:experiment-causal_relations}}). 

To guide the subject, the next target to touch (plate/bar) appears blue until touched. Once it has touched by the glass, the start (right) plate will first become yellow then green, to indicate the start of the trial. The end (left) plate will act equivalent before the subject raises the glass to end the trial. The whole trial can be seen as the sequence of the following states:
\begin{enumerate}
    \item \textbf{Inter Trial} The glass is weightless. The subject can move the glass freely. The start plate becomes blue.
    \item \textbf{Touching Start Plate} The glass is placed on top of the start plate, which in turn first becomes yellow, then green. The bar at the top becomes blue once the glass does not touch the plate anymore.
    \item \label{enum:methods_transport-pre}\textbf{Transport Pre} The glass has to touch the bar at the top, which turns yellow once touched.
    \item \label{enum:methods_transport-post}\textbf{Transport Post} The glass has to touch the end plate.
    \item \textbf{Touching End Plate} The glass is placed on top of the end plate, which in turn first becomes yellow, then green. Afterwards the subject has to stop touching the plate. The glass breaks if the subject applies too much force during this state.
    \item \textbf{End} The glass becomes weightless and the subject gets feedback about the used force and whether the trial was successful or not.
\end{enumerate}

\section{Causal relations}\label{appendix:experiment-causal_relations}

The causal relations that govern the relevant features in the simulation are given by \crefrange{eq:causal_sampled_features}{eq:causal_color_offset} (note that the prefix \textit{glass} has been omitted for readability). See \cref{fig:causal_model_data} to for a better overview of the causal information flow between the variables.

The exogenous features are sampled as

\begin{equation}
    \label{eq:causal_sampled_features}
    \begin{array}{lcl}
    \mathit{force} & ~ & \text{rendered force by \textit{PHANToM} in Newton} \\
    \mathit{color} & \sim & \mathcal{U}(\set{\text{red, green, blue}}) \\
    \mathit{weight} & \sim & \mathcal{U}([0.2616, 1])
\end{array}
\end{equation}

The weight is rendered by the haptic device as a force vector pointing directly downwards and is therefore able to simulate the weight of the object. The actual force acting on the subject in Newton is given by $\mathit{weight} \cdot 3N$. \textit{force\_threshold} and subsequently \textit{result} are calculated as
\begin{equation}
    \label{eq:causal_force_threshold}
    f_{\mathit{force\_threshold}}(\mathit{color}, \mathit{weight}) = (2.5\cdot \mathit{weight} + \mathit{color\_offset}(\mathit{color}))\cdot 0.8241
\end{equation}
and
\begin{equation}
    \label{eq:causal_result}
    f_\mathit{result}(\mathit{force\_threshold}, \mathit{force}) = 
    \begin{cases}
         1, & \text{for } \mathit{force\_threshold} \leq \mathit{force}\\
         0, & \text{else} ,
    \end{cases}
\end{equation}
with 
\begin{equation}
    \label{eq:causal_color_offset}
    \mathit{color\_offset}(\mathit{color}) = 
    \begin{cases}
        -0.254 & \text{for } \mathit{color} = \text{red}\\
        -0.069 & \text{for } \mathit{color} = \text{green}\\
        0.116 & \text{for } \mathit{color} = \text{blue}\\ .
    \end{cases}
\end{equation}

If $f_\mathit{result}(\mathit{force\_threshold}, \mathit{force}) = 1$, the glass remains intact. Otherwise, it breaks. One can imagine the relation as
\begin{equation}
    \text{Glass breaks} = 1 - f_\mathit{result}(\mathit{force\_threshold}, \mathit{force}).
\end{equation}

The color-specific offsets for \textit{force\_threshold} and the functional form were empirically determined such that there is a noticeable difference in the stability of the glass for each color without it being too obvious.
After we ran the study with two test participants, we had to correct the threshold in \cref{eq:causal_force_threshold} by a factor of 0.8241 because the glass was too sturdy. We verified the design then by one last test participant.

% ------------------ APP:QUESTIONNARIE ------------------------- %
\section{Questionnaire (german)}

\textbf{Fragen}~\\
\begin{itemize}
    \item Glas-Farbe beeinflusst
        \begin{AutoMultiColItemize}
            \renewcommand\labelitemi{\LARGE$\square$}
            \item[\LARGE$\square$] Glas-Gewicht
            \item[\LARGE$\square$] Glas-Zerbrechlichkeit
            \item[\LARGE$\square$] Nichts
        \end{AutoMultiColItemize}
    \item Glas-Gewicht beeinflusst
        \begin{AutoMultiColItemize}
            \item[\LARGE$\square$] Glas-Farbe
            \item[\LARGE$\square$] Glas-Zerbrechlichkeit
            \item[\LARGE$\square$] Nichts
        \end{AutoMultiColItemize}
    \item Glas-Zerbrechlichkeit beeinflusst
        \begin{AutoMultiColItemize}
            \item[\LARGE$\square$] Glas-Gewicht
            \item[\LARGE$\square$] Glas-Farbe
            \item[\LARGE$\square$] Nichts
        \end{AutoMultiColItemize}
\end{itemize}

\vspace{3cm}

\begin{center}
    \includegraphics[width=\textwidth/3]{CausalGraphEmpty.png}
\end{center}

\end{document}
\typeout{get arXiv to do 4 passes: Label(s) may have changed. Rerun}
