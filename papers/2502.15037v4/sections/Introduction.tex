\section{Introduction}
% Robotic manipulation of BDLOs, such as those involved in autonomous vehicle wire harness assembly, poses significant challenges. 
% These tasks require robots to dynamically and accurately manipulate BDLOs over prolonged periods ($\geq5s$)~\cite{deform_survey1, deform_survey2, saha2007manipulation}, which is difficult due to the nonlinear dynamics of these objects. 
% Accurate prediction of BDLOs behavior typically requires computationally intensive models, making real-time manipulation challenging. 
% Additionally, practical applications often lead to occlusions of BDLOs, hindering perception systems from reliably estimating the full configuration of the BDLOs. 
% Consequently, there is a critical need for a novel modeling method that can quickly and accurately predict the dynamic behavior of BDLOs to enable robotic manipulators to handle them successfully.
The automation of wire harness assembly represents a critical challenge in manufacturing, particularly in the automotive and aerospace industries where complex, branched cable systems are essential components. 
These assemblies require robots to manipulate Branched Deformable Linear Objects (BDLOs) with precision over extended periods ($\geq$5s)~\cite{deform_survey1, deform_survey2, saha2007manipulation}, a task that remains largely unautomated due to the objects' complex nonlinear dynamics. 
While accurate prediction of BDLO behavior is possible through computationally intensive models, real-time manipulation demands rapid computation. 
Further complicating matters, practical assembly environments often involve occlusions that prevent perception systems from reliably estimating the full BDLO configuration.

% Although BDLOs models exist, quantitative analysis for real-world applications remains under-explored~\cite{originalDER, baseline1, bdlo_modeling1, bdlo_modeling2}. 
% This is because these existing models primarily aim to create visually plausible simulations, which may not accurately reflect real-world behavior. 
% Moreover, these methods often require manual tuning of physical parameters (e.g., mass or the bending and twisting moduli) to align simulations with real-world observations.
% They also rely on assumption that are rarely satisfied in practice (e.g., perfect tracking of the orientation of each wire harness segment).
Existing approaches to modeling BDLOs ~\cite{originalDER, baseline1, bdlo_modeling1, bdlo_modeling2} have focused primarily on computer graphics applications, prioritizing visual plausibility over physical accuracy. 
These models typically require extensive manual parameter tuning and make impractical assumptions about perfect state observation—limitations that render them unsuitable for robotic manipulation tasks. 
Recent advances in modeling single Deformable Linear Objects (DLOs) \cite{DEFORM, xpbd, directionalrigidity, bi-LSTM_baseline, GNN_baseline} have shown promise, but the extension to branched structures introduces fundamental new challenges.
When a robot manipulates one branch of a wire harness, the forces and deformations propagate through junction points to affect all connected branches.
This coupling between branches, combined with the need for flexible grasp locations and real-time computation, makes BDLO manipulation significantly more complex than handling single DLOs.

% Prior work in robotics has focused on modeling and DLOs \cite{DEFORM, xpbd, directionalrigidity, bi-LSTM_baseline, GNN_baseline}, but these approaches have not been extended to BDLOs. 
% Recent work, DEFORM~\cite{DEFORM}, accurately models Deformable Linear Objects (DLOs) in real-time over extended time horizons by introducing a Differentiable Discrete Elastic Rod (DDER) model as a prior for residual learning. 
% The use of residual learning compensates for numerical integration errors.
% Note that BDLOs are characterized by their branched structures, where each branch can be considered a DLO. Inspired by DEFORM's successful performance, DEFT adapts DEFORM to simulate each branch of a BDLO. 

% Despite DEFORM's potential for extension to BDLOs, three significant challenges remain.
% 1)  \textbf{modeling challenges:} DEFORM does not explore dynamic propagation at the junction points of branched structures—specifically, stretching, twisting, and bending—which is critical when modeling BDLOs.
% Additionally, DEFORM employs a graph neural network (GNN) as a residual network designed specifically for DLOs, which may not be suitable for modeling BDLOs.
% 2) \textbf{manipulation limitations:} DEFORM assumes the clamping points are at the ends. 
% Although this is a conventional assumption used in DLO manipulation papers~\cite{directionalrigidity, IRP, inlstm, gelsim}, this limits teehe potential for dexterous manipulation tasks that require grasping at multiple points along the object (e.g., thread insertion),
% 3) \textbf{slow computational speed:} DEFORM is designed for DLOs simulation. Recursively using DEFORM to simulate BDLOs' branches separately could result in a linear increase in computational time as the number of branches increases. 
% Additionally, DEFORM is formulated as a bi-level optimization problem to learn from real-world data. 
% To make the inner-loop optimization differentiable, DEFORM solves the optimization with numerical gradients, which risks optimization instability and slow convergence.

% To address the above challenges, this paper introduces \textbf{D}ifferentiable Branched Discrete \textbf{E}lastic Rods for Modeling \textbf{F}urcated DLOs in Real-\textbf{T}ime (\DEFT), a novel differentiable framework that introduces a novel junction coupling approach to model dynamic propagation at the junction points between parent and child branches. 
% With real-time inference speed, \DEFTn accurately predicts dynamic BDLOs behavior over long time horizons. To illustrate the utility of this framework, we demonstrate how \DEFTn can be applied to perform three essential real-world manipulation tasks for autonomous wire harness assembly: 1) 3D shape matching, 2) multi-thread grasp, and 3) precise thread grasping and insertion.
% The main contributions of this paper are four-fold: 
% \begin{enumerate} 
% \item A novel differentiable BDLO model, along with a junction coupling approach based on practical assumption.
% \item Extension of DEFORM that enables: 1) faster inference times by utilizing parallel programming and providing analytical gradients for inner-loop optimization, 
% 2) accommodation of grasping in the middle of a BDLO.
% \item A novel GNN designed for residual learning in BDLOs to compensate for numerical integration errors.
% \item A comprehensive set of experiments that demonstrate \DEFT's superior performance in terms of accuracy, computational speed, and sampling efficiency when compared to existing state-of-the-art approaches. 
% Furthermore, we integrate \DEFTn into a motion planning framework to demonstrate three essential real-world manipulation tasks for autonomous wire harness assembly. 
% \end{enumerate}
% \DEFTn is implemented in PyTorch and can be seamlessly integrated into any deep learning frameworks. 
% Code and data will be provided upon final paper acceptance. 
% To the best of our knowledge, this is the first robotics paper to quantitatively investigate the modeling of BDLOs and perform their manipulation in real-world settings.

To address these challenges, this paper introduces \textbf{D}ifferentiable Branched Discrete \textbf{E}lastic Rods for Modeling \textbf{F}urcated DLOs in Real-\textbf{T}ime (\DEFT). 
This framework introduces a novel junction coupling approach that accurately models dynamic propagation between parent and child branches while maintaining real-time performance. 
As this paper illustrates through real-world evaluations, \DEFTn enables three essential wire harness assembly tasks: 3D shape matching, multi-thread grasp, and precise thread grasping and insertion. 
The contributions of this paper are four-fold:
\begin{enumerate}
\item A novel differentiable BDLO model that accurately captures how dynamics propagate through wire harness junction points thereby enabling model parameters to be efficiently identified via experimental observation;
\item A computational representation that enables efficient inference and model parameter learning through parallel programming in concert with analytical gradients;
\item A specialized Graph Neural Network (GNN) architecture for BDLO residual learning, which can compensate for numerical integration errors that arise over long time horizon prediction;
\item A comprehensive experimental validation demonstrating \DEFT's superior performance in accuracy, speed, and sampling efficiency compared to existing approaches, including integration into a motion planning framework for real-world manipulation tasks.
\end{enumerate}
The framework is implemented in PyTorch for seamless integration with deep learning workflows, and represents, to the best of our knowledge, the first quantitative investigation of BDLO modeling and manipulation in real-world settings. 
The results demonstrate that DEFT enables reliable autonomous manipulation of complex wire harnesses, addressing a significant challenge in manufacturing automation.
Upon acceptance the code associated with this paper will be made publically available.
