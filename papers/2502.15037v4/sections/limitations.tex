\section{Limitations}
While DEFT shows promising results in modeling and control of BDLOs, several limitations remain that could be addressed in future work. 
First, DEFT currently lacks a contact model for accurately predicting interactions between deformable components and their surroundings. Incorporating a contact modeling framework, such as that proposed in \cite{DER_contact}, would enable more realistic simulations and more robust planning in environments where collisions and frictional forces play a significant role.
% Second, although DEFT’s computational efficiency is sufficient for real-time inference and many offline applications, further optimization is necessary to achieve real-time planning.
% This is because an optimization solver may call a BDLO modeling framework several thousand times to generate a single trajectory.
% Note, that though the 
Second, for the insertion tasks considered in this work, we assume that the location and geometry of the target hole are known in advance. 
In unstructured or partially observed environments, this assumption may not hold. 
Incorporating 3D scene understanding methods, such as Gaussian Splatting \cite{splanning}, could help autonomously detect or refine the location of target features and thereby increasing the system’s applicability to real-world scenarios.
Lastly, this work does not account for the orientation of the gripper during BDLO manipulation, which can significantly affect deformation behavior. As discussed in \cite{diminishingridgidity}, gripper orientation can influence the kinematics of deformable objects. 
Future extensions of DEFT should therefore incorporate a more comprehensive treatment of gripper orientation to improve accuracy and expand the range of manipulable tasks.