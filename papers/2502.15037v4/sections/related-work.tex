\section{Related Work}
% Because BDLOs are branched structures where each branch can be regarded as a DLO, we will first briefly review related research on models of DLOs, followed by a review of BDLOs simulation.
% The modeling of BDLOs builds upon research in single DLOs, as each branch of a BDLO can be considered a DLO. 
% This section reviews relevant approaches in both domains, as well as recent advances in differentiable modeling frameworks that inform our approach.


Effective BDLO modeling requires three critical capabilities: real-time prediction, accurate 3D simulation, and efficient model parameter learning.
Real-time prediction capability is essential for robotic manipulation tasks, where quick decision-making can mean the difference between successful and failed object handling.
Three-dimensional accuracy ensures the model can capture the complex spatial deformations that BDLOs undergo during manipulation, including twisting, bending, and branch interactions.
Efficient parameter learning enables models to automatically adapt to different materials and configurations through data-driven optimization, reducing both computational overhead and the need for extensive training data.
This section reviews relevant approaches in modeling BDLOs and single DLOs while considering each of these criteria.

\subsection{DLOs Modeling}
Approaches to modeling DLOs have evolved from purely physics-based methods toward more sophisticated learning-based frameworks that balance between the three aforemention key capabilities. 

Physics-based approaches implement different mathematical frameworks to model physical behavior.
Mass-spring systems represent objects as networks of point masses connected by springs, computing forces and accelerations to simulate movement \cite{LargeStepSimulation, diffcloth, Liu:2013:FSM}. 
Position-Based Dynamics (PBD) directly manipulates vertex positions to satisfy physical constraints, avoiding expensive force calculations \cite{directionalrigidity, diminishingridgidity, xpbd}. 
Finite Element Methods (FEM) divide objects into discrete elements and solve partial differential equations to compute deformations \cite{garcia2006optimized, sin2013vega, koessler2021efficient}. 
Discrete Elastic Rods (DER) models represent objects as centerline curves with material frames, enabling explicit modeling of bending and twisting \cite{diminishingridgidity, DER_manipulation1, DER_manipulation2, DER_manipulation3}.
Unfortunately each of these methods has limitations: mass-spring systems describe the rigidity of wires by introducing large spring stiffness which can make prediction numerically unstable; PBD models struggle with parameter learning because they assume that the system is quasi-static; FEM methods are computationally intensive; and DER models lack differentiable expressions for their model parameters, making efficient parameter optimization challenging. 

Learning-based modeling approaches use data-driven techniques to predict deformation behavior. 
Methods such as Graph Neural Networks (GNN) process vertex-level spatial information \cite{GNN_baseline}, while Bi-LSTM models capture temporal dependencies in deformation sequences \cite{bi-LSTM_baseline}. 
These approaches excel at real-time prediction through efficient computation and support straightforward parameter learning through standard optimization techniques.
However, they struggle with 3D accuracy, particularly in capturing the complex dynamic deformations characteristic of real-world manipulation.

Recent work has explored hybrid approaches that combine learning with physics-based modeling. 
This is exemplified by DEFORM \cite{DEFORM} which constructs a novel differentiable DER model and combines it with a neural network-based residual learning approach to account for numerical integration errors during prediction. 
This hybrid approach achieves both real-time performance and 3D accuracy while enabling efficient parameter learning through end-to-end differentiability. 

Despite DEFORM's advances in DLO modeling, it faces three significant limitations when extended to BDLO applications. First, DEFORM's modeling framework does not address the dynamics of branched structures, particularly the propagation of stretching, twisting, and bending forces at junction points. 
Second, DEFORM's manipulation framework assumes fixed clamping points at the object ends, following conventional DLO literature. 
This assumption restricts its applicability to more sophisticated manipulation tasks, such as thread insertion, which require grasping at multiple points along the object's length.
Third, DEFORM's computational approach presents efficiency challenges for BDLO simulation.
Its sequential processing of individual branches would result in linear computational time increases as branches are added. 
Furthermore, its bi-level optimization framework, while enabling learning from real-world data, relies on numerical gradients that can lead to slow convergence and optimization instability.

% Recent work, DEFORM~\cite{DEFORM}, has achieved promising results in modeling deformable linear objects (DLOs) through comprehensive quantitative analysis. DEFORM introduces a DDER model for residual learning and demonstrates potential for extending to BDLOs. However, it has not yet explored the modeling of BDLOs.
% Furthermore, although DEFORM is capable of performing real-time inference, the inherent complexity of BDLOs demands even greater computational efficiency. Naively simulating each branch with DEFORM in sequential, as illustrated later in this paper, would not allow for real-time inference. In this work, we adapt DEFORM to simulate each DLO in parallel and further enhance DEFORM's computational speed.

% Traditional learning-based modeling approaches for DLOs, such as GNN\cite{GNN_baseline} and Bi-LSTM\cite{bi-LSTM_baseline}, reason over vertices spatially to predict 2D behavior but fail to accurately capture dynamic motions in 3D.
% Physics-based models, including mass-spring systems~\cite{LargeStepSimulation, diffcloth, Liu:2013:FSM}, position-based dynamics \cite{directionalrigidity, diminishingridgidity, xpbd}, discrete elastic rods (DER)~\cite{originalDER, originalDER2, DERapp1, DERapp2}, and finite element methods (FEM)~\cite{garcia2006optimized, sin2013vega, koessler2021efficient}, have been employed to model DLOs, yet each has limitations: mass-spring systems introduce instability due to unnecessary stiffness; 
% FEMs are computationally intensive and impractical for real-time prediction; PBD models are sensitive to parameter selection and assume quasi-static behavior, leading to low accuracy in dynamic manipulation\cite{diminishingridgidity, DER_manipulation1, DER_manipulation2, DER_manipulation3}; 
% and while DER models have been qualitatively validated for 3D behavior, they lack quantitative performance in dynamic contexts.

% Recent work, DEFORM~\cite{DEFORM}, has achieved promising results in modeling deformable linear objects (DLOs) through comprehensive quantitative analysis. DEFORM introduces a DDER model for residual learning and demonstrates potential for extending to BDLOs. However, it has not yet explored the modeling of BDLOs.
% Furthermore, although DEFORM is capable of performing real-time inference, the inherent complexity of BDLOs demands even greater computational efficiency. Naively simulating each branch with DEFORM in sequential, as illustrated later in this paper, would not allow for real-time inference. In this work, we adapt DEFORM to simulate each DLO in parallel and further enhance DEFORM's computational speed.

\subsection{BDLOs Modeling}
Existing BDLO modeling approaches adapt various physical simulation frameworks to handle branched structures.
FEM-based methods extend traditional finite element analysis by treating branches as connected elastic elements~\cite{BDLO_FEM}. 
While these methods excel in 3D accuracy and enable efficient parameter learning through gradient-based optimization, their computational requirements preclude real-time prediction. 
DER-based approaches introduce rigid-body constraints at junctions to ensure proper alignment of attached segments ~\cite{originalDER}, achieving real-time performance and reasonable 3D accuracy but struggling with parameter learning due to their fixed parameter structure. 
Quasi-static methods like ASMC enforce geometric constraints while assuming static equilibrium conditions~\cite{baseline1}, offering real-time performance, but sacrificing 3D accuracy by overlooking dynamic effects that can make parameter learning challenging.


% Few studies have explored  modeling for BDLOs.
% Finite Element Method (FEM)~\cite{BDLO_FEM} can model BDLOs accurately, but their high computational cost makes them impractical for real-time prediction. 
% DER~\cite{originalDER} introduces rigid-body constraints at junctions to ensure proper alignment of attached segments, thereby enable modeling of twisting and bending dynamics.
% However, it lacks adjustable parameters like moment of inertia, limiting adaptability.
% It also assumes full accessibility to each rigid body's orientation and quaternion, which is unrealistic in practical settings.
% In constrast, ASMC~\cite{baseline1} enforces quasi-static constraints to enable more practical modeling  by relying solely on geometric information available in real-world scenarios. 
% However, it enforces the constraints onto entire BDLOs without conserving momentum, leading to low accuracy in dynamic manipulation and sensitivity to parameter tuning as demonstrated in this paper.

 To address these challenges, DEFT introduces a novel differentiable framework specifically designed for BDLOs. Through a specialized GNN architecture for branched structures, DEFT achieves all three critical capabilities: real-time performance, 3D accuracy, and efficient parameter learning. 
 This enables DEFT to adapt to various materials and configurations while maintaining the speed and accuracy necessary for practical robotic manipulation tasks.

% To address these challenges, DEFT introduces self-adjustable parameters within a novel constraint that preserves momentum during dynamic manipulation, while still relying solely on accessible geometry information.

% \subsection{Differentiable Modeling Framework} Differentiable models are computational frameworks where every operation is differentiable with respect to input parameters. 
% This enables their integration into learning frameworks, enhancing the learning process and allowing for automatic parameter tuning through gradient-based optimization during training~\cite{DEFORM,LargeStepSimulation,diffcloth}. 
% The combination of differentiable DLO simulations with residual learning has shown promising performance improvements~\cite{DEFORM}. 
% However, a differentiable model for BDLOs does not yet exist, and the residual learning structure must be redesigned to accommodate BDLOs.

% In this work, DEFT presents the first differentiable BDLO simulation and introduces a GNN architecture specifically adapted to the structure of BDLOs.


