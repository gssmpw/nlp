\section{Preliminaries}
This section introduces the key components involved in modeling DLOs. 
It then summarizes the primary contributions of DEFORM and describes the challenges of adapting it to BDLO modeling.

\subsection{Deformable Linear Object (DLO) Model}
To model a DLO, we represent it as an indexed set of $n$ vertices at each time instance. 
The ground truth location of vertex $i$ at time $t$ is denoted by $\mathbf{x}_t^i \in \mathbb{R}^3$, and the set of all $n$ vertices is denoted by $\mathbf{X}_t$.
A segment, or edge, in the DLO is the line between successive vertices, $\timevertind{e}{i}{t} = \timevertind{x}{i+1}{t} - \timevertind{x}{i}{t}$.
Let $\timeind{E}{t}$ correspond to the set of all edges at time $t$.
Let $\mathbf{M}_i \in \mathbb{R}^{3 \times 3}$ denote the mass matrix of vertex $i$.
The velocity of the vertices of the DLO is approximated by $\mathbf{V}_t = (\mathbf{X}_t - \mathbf{X}_{t-1})/{\Delta t}$, where $\Delta t$ is the time step between frames.
Note that $\mathbf{V}_t$ is an approximation of the actual velocities.
We distinguish between ground truth elements and predicted elements by using the circumflex symbol (e.g., $\mathbf{X}_t$ is the ground truth set of vertices at time $t$, and $\hat{\mathbf{X}}_t$ is the predicted set of vertices at time $t$).

\subsection{Modeling DLOs with DEFORM}
\subsubsection{Discrete Elatic Rods (DER)}
DEFORM builds upon DER theory to model three distinct deformation modes of DLOs: bending, twisting, and stretching. 
DER theory models DLOs using two key families of coordinate frames.
First, it introduces Bishop Frames, which provide a twist-free reference state along the centerline of the rod. 
The Bishop Frame only rotates to follow the curve's geometry, providing a relaxed baseline configuration for the DLO. 
Second, the theory employs Material Frames, which describe the actual physical deformation of the rod.
Material Frames can be expressed relative to the Bishop Frames through a single angle whose rate of change represents the physical twist of the DLO. 
To describe the twist of the DLO, DER theory assumes that the angle of the Material Frame with respect to the Bishop Frame is one that minimizes the potential energy of the DLO.
% This relationship allows the theory to efficiently compute the rod's potential energy by measuring how much the material frames deviate from their corresponding Bishop frames at each edge of the discrete rod.
We briefly summarize DER theory in this section, but a longer introduction can be found here \cite{bishopframe}.

\textbf{Bishop Frame}: 
The Bishop Frame at edge $i$ is made up of three axes $\{\mathbf{t}^i, \mathbf{b}_1^i, \mathbf{b}_2^i\}$.
We next formally define each axis of the Bishop Frame.
First, $\mathbf{t}^i \in \mathbb{R}^3$ is the unit tangent along edge $i$ which is defined as $\mathbf{t}^i = (\timevertind{x}{i+1}{}-\timevertind{x}{i}{})/||\timevertind{x}{i+1}{}-\timevertind{x}{i}{}||_2$.
The vector $\mathbf{b}_1^i \in \mathbb{R}^{3}$ is defined iteratively by applying a rotation matrix to the previous Bishop frame: $\mathbf{b}_1^i = \mathbf{R}^i \cdot \mathbf{b}_1^{i-1}$.
The rotation matrix $\mathbf{R}^i \in \mathbb{R}^{3 \times 3}$ satisfies the following conditions
\begin{equation}
    \label{eq:rotation1}
    \mathbf{t}^i = \mathbf{R}^i \cdot \mathbf{t}^{i-1}
\end{equation}
\begin{equation}
    \label{eq:rotation2}
     \mathbf{t}^{i-1} \times \mathbf{t}^i = \mathbf{R}^i \cdot (\mathbf{t}^{i-1} \times \mathbf{t}^i).
\end{equation}
The first condition ensures that the tangent vector of the DLO at each subsequent edge can be described using the rotation matrix and the tangent vector at the previous edge. 
Though we do not prove it here, the second condition ensures that the rotation between successive Bishop Frames is minimal while moving along the DLO. 
In particular, this latter condition ensures that there is no twist about the subsequent tangent vectors at each edge of the DLO. 
Finally, $\mathbf{b}_2^i := \mathbf{t}^i \times \mathbf{b}_1^i$, which ensures that  $\mathbf{b}_2^i$ is orthogonal to $\mathbf{t}^i$ and $\mathbf{b}_1^i$. 
% This transportation results in the least amount of twist, making it ideal for adapting the Bishop frame as the most natural reference frame \Ram{this sentence doesn't make sense and needs more information.}.
% Further details of the Bishop frame can be found in \cite{bishopframe}.

\textbf{Material Frame:}
Subsequently, DER theory introduces Material Frames that are made up of three axes  $\{\mathbf{t}^i, \mathbf{m}_1^i, \mathbf{m}_2^i\}$. 
These Material Frames are generated by rotating the Bishop frame about the tangent vector by a scalar $\theta^i$, i.e.,  $\mathbf{m}_1^i, \mathbf{m}_2^i$ are defined as
\begin{equation}
    \mathbf{m}_1^i =  \textbf{b}_1^i \cos \theta^i +  \textbf{b}_2^i \sin \theta^i
   \label{eq:m1}
\end{equation}
\begin{equation}
    \quad\mathbf{m}_2^i = -\textbf{b}_1^i \sin \theta^i +  \textbf{b}_2^i\cos \theta^i
   \label{eq:m2}
\end{equation}
An illustration of the frames can be found in Figure \ref{fig:DER_frames}. 

\begin{figure}[t]
    \centering
     \includegraphics[width=0.48\textwidth]{figures/DER_frames.pdf}
     \caption{An illustration of DER coordinate frames.}
    \label{fig:DER_frames}
\end{figure}

\textbf{Equations of Motion:}
Let $\MaterialFrame(\mathbf{X}_t, \bm{\theta}_t)$ denote the set of all $n-1$ Material Frames at time $t$. 
Let $\materialp$ represent the vector of material properties for each vertex of the DLO (i.e., mass, bending modulus, and twisting modulus) \cite[Section 4.2]{originalDER}.
Using these definitions, one can compute the potential energy of the DLO arising from bending and twisting which we denote by $P(\MaterialFrame(\mathbf{X}_t, \bm{\theta}_t), \materialp)$. 
% These frames characterize the potential energy $P(\MaterialFrame(\mathbf{X}_t, \bm{\theta}_t), \materialp)$ of the DLO arising from bending and twisting. 
% $\materialp$ represents the vector of material properties for each vertex of the DLO (i.e., mass, bending modulus, and twisting modulus) \cite[Section 4.2]{originalDER}.
DER theory assumes that the DLO reaches an equilibrium state between each simulation time step to obtain the optimal $\bm{\theta}_t^*$:
\begin{equation} \bm{\theta}^*_t(\textbf{X}_t,\materialp) = \argmin_{\bm{\theta}_t} P(\MaterialFrame(\mathbf{X}_t, \bm{\theta}_t), \materialp)
       \label{eq:opt_theta}
\end{equation}
 Once $\bm{\theta}^*_t(\mathbf{X}_t, \materialp)$ is derived, the restorative force during deformation is given by the negative gradient of the potential energy with respect to the vertices. 
 Consequently, the equation of motion for the DLO is:
\begin{equation}
       \Massmatrix \ddot{\textbf{X}}_{t} = -\frac{\partial P(\MaterialFrame(\textbf{X}_t, \bm{\theta}^*_t(\textbf{X}_t,\materialp)),\bm{\alpha})}{\partial  \textbf{X}_{t}}
       \label{eq:equationofmotion}
\end{equation}
One can numerically integrate this formula to predict the velocity and position by applying the Semi-Implicit Euler method:
\begin{align}
        \hat{\textbf{V}}_{t+1} &= \hat{\textbf{V}}_\text{t}-\Delta_\text{t}\Massmatrix^{-1}\frac{\partial P(\MaterialFrame(\textbf{X}_t, \bm{\theta}^*_t(\textbf{X}_t,\materialp)),\bm{\alpha})}{\partial  \textbf{X}_{t}}, \label{eq:Semi-Euler1}\\  
        \hat{\textbf{X}}_{\text{t}+1} &= \hat{\textbf{X}}_\text{t} + \Delta_\text{t}\hat{\textbf{V}}_{t+1},
       \label{eq:Semi-Euler2}
\end{align}
where $\Delta_{\text{t}} > 0$ is a user specified time discretization.


\subsubsection{Differentiable Discrete Elastic Rods (DDER)}
Tuning the DER model parameters, $\materialp$, to match real-world DLOs is non-trivial. 
To address this challenge, DEFORM introduces DDER, which enables the automatic tuning of DLO parameters using real-world data. 
Predicting $\hat{\textbf{X}}_{t+1}$ with DDER through \eqref{eq:Semi-Euler1} and \eqref{eq:Semi-Euler2} involves solving an optimization problem as defined in \eqref{eq:opt_theta}.
Consequently, minimizing the prediction loss requires solving a bi-level optimization problem at each time step $t$:
\begin{align}
    \label{eq:outerloop}
    &&\underset{\bm{\alpha}}{\min} \hspace{0.5cm}& \|\textbf{X}_{t+1}- \hat{\textbf{X}}_{t+1}(\bm{\theta}^*_t)\|_2 \hspace{1.5cm}    \\
    && & \bm{\theta}^*_t = \argmin_{\bm{\theta}_t}   P(\MaterialFrame(\textbf{X}_t, \bm{\theta}_t),\materialp)
    \label{eq:innerloop}
\end{align}
Note $\hat{\textbf{X}}_{t+1}$ depends on the previously computed $\bm{\theta}^*_t$ through \eqref{eq:equationofmotion}, \eqref{eq:Semi-Euler1} and \eqref{eq:Semi-Euler2}. 
To compute the gradient of \eqref{eq:innerloop} with respect to $\bm{\alpha}$, DEFORM employs solvers from Theseus \cite{theseus}, which use PyTorch’s automatic differentiation \cite{pytorch}. 
As we illustrate via experiments in Section \ref{sec:experiments}, this can be a time-consuming process.
% Theseus is a differentiable optimization framework that can compute the derivative of the entire pipeline while solving \eqref{eq:opt_theta}.
% Although this approach is convenient and flexible, it can lead to increased memory usage and computational inefficiency.  
% In this work, we utilize the analytical gradients of $P(\MaterialFrame(\mathbf{X}_t, \bm{\theta}_t), \materialp)$ with respect to $\bm{\theta}_t$, leading to more computational efficiency.

\subsubsection{Integration Method with Residual Learning}
 To compensate for numerical errors that arise during numerical integration\eqref{eq:Semi-Euler2}, DEFORM incorporates a learning-based framework into the integration method: 
% \begin{align}
%     \begin{split}
%         \hat{\textbf{V}}_{t+1} &= \hat{\textbf{V}}_t-\Delta_tM^{-1}\Big(\frac{\partial P(\MaterialFrame(\textbf{X}_t, \bm{\theta}^*_t(\textbf{X}_t,\materialp)),\bm{\alpha})}{\partial  \textbf{X}_{t}} + \\ & \hspace*{4cm} + \text{DNN}(\hat{\textbf{X}}_t, \bm{\alpha})\Big), \label{eq:NN-Semi-Euler1}
%         \end{split} \\  
%                 \hat{\textbf{X}}_{t+1} &= \hat{\textbf{X}}_t + \Delta_t \Big(\hat{\textbf{V}}_{t+1} + \text{DNN}(\hat{\textbf{X}}_t, \hat{\textbf{V}}_t, \bm{\alpha})\Big)
%        \label{eq:NN-Semi-Euler3}
%        %  \hat{\textbf{X}}_{t+1} &= \hat{\textbf{X}}_t + \Delta_t\Big(\hat{\textbf{V}}_{t+1} + \text{DNN}(\hat{\textbf{V}}_t)\Big),
%        % \label{eq:NN-Semi-Euler2}
% \end{align}
 \begin{equation}
                \hat{\textbf{X}}_{t+1} = \hat{\textbf{X}}_t + \Delta_t (\hat{\textbf{V}}_{t+1} + \text{GNN}(\hat{\textbf{X}}_t, \hat{\textbf{V}}_t, \bm{\alpha}))
       \label{eq:NN-Semi-Euler3}
 \end{equation}
This structure builds on residual learning \cite{resnet}, grounding the neural network in physical laws while leveraging data-driven insights. 
% DEFORM uses a GNN tailored to single-thread DLOs, but is limited to that domain. 
% To address the challenge, DEFT introduces a specialized GNN architecture for residual learning of BDLO, compensating for numerical integration errors over long prediction horizons.

\subsubsection{Inextensiblity Enforcement with Momentum Conservation}
After numerical integration, the DLO length may deviate from its initial value due to numerical errors, making it difficult to accurately model inextensible DLOs. 
A common physics-based approach~\cite{LargeStepSimulation, diffcloth, Liu:2013:FSM} uses mass-spring systems to minimize this deviation, but this can result in stiff simulations and numerical instability. 
DEFORM addresses this by enforcing an inextensibility constraint while preserving momentum, leading to more accurate and stable simulation. 
% Building on this, DEFT applies the constraint to each branch in parallel, further enhancing computational efficiency.

\subsubsection{Challenges of Extending DEFORM to BDLOs}
Though DEFORM can accurately simulate single-thread DLOs, it is unsuitable for BDLOs for several reasons.
First, the definitions of Bishop and Material Frames do not generalize to BDLOs. 
To see why this is true, recall that at junctions in a BDLO there are several tangent vectors because there are more than two edges. 
As a result, satisfying \eqref{eq:rotation1} and \eqref{eq:rotation2} results in an overdetermined system which makes it impossible to find a rotation matrix $\mathbf{R}^i$ that satisfies all constraints. 
Consequently, the Bishop and Material Frames cannot be defined at these junctions.
An alternative to extend DEFORM to BDLOs is to simulate each branch individually using DEFORM. 
However, this requires repeatedly solving \eqref{eq:opt_theta}, which reduces computational efficiency. 
This is further complicated when trying to enforce inextensibility using the momentum conservation constraint. 
To address this, DEFT proposes a computational representation that supports parallel processing, enabling more efficient inference.

Second, relying on automatic differentiation to perform parameter tuning becomes computationally prohibitive when dealing with BDLOs.
% Although this approach is convenient and flexible, it can lead to increased memory usage and computational inefficiency.  
To address this limitation, DEFT utilizes the analytical gradients of $P(\MaterialFrame(\mathbf{X}_t, \bm{\theta}_t), \materialp)$ with respect to $\bm{\theta}_t$.
Third, the GNN used by DEFORM during residual learning is tailored to single-strand DLOs. 
To address this limitation, DEFT introduces a specialized GNN Architecture for residual learning for BDLOs that can compensate for numerical integration errors over long horizon predictions. 

 