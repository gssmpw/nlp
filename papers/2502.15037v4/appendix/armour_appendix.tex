\subsection{ARMOUR}
\label{armour_appendix}
% The planning tasks use ARMOUR, an optimization-based receding-horizon trajectory planner and tracking controller framework, to manipulate BDLOs. The ARMOUR framework establishes an optimization problem whose decision variables are parameters of trajactory polynomial that characterizes the robot arm's joint trajectory. The cost function is to minimize the norm distance between all vertices in the DEFT BDLO prediction and the desired BDLO configuration specified by the user.  One end of the parent branch of the BDLO is rigidly attached to the end effector of the robot arm. ARMOUR is able to use DEFT to obtain a predicted BDLO configuration based on the robot arm's end effector position at the end of each robot motion upon performing the optimized trajectories. The end effector of the robot arm can be computed using forward kinematics given a set of seven joint inputs from the controller proposed in ARMOUR. The constraints in the ARMOUR trajectory optimization problem are specified in the joint position, velocity, and arm torque limits to ensure feasibility. The solution of this optimization problem is the optimal joint trajectory for minimizing the cost function, which is the residual between the prediction and desired BDLO configuration. The process happens iteratively to command the robot arm to track the solution trajectories until the desired configuration within certain satisfaction quiteria is hold. If the process exceeds the planning iteration limit(we set it as 20), it results in a planning failure.



To address the shape matching task, we employ ARMOUR \cite{ARMOUR}, an optimization-based framework for motion planning and control. 
The objective is to guide a robot arm so that it manipulates a BDLO from an initial configuration to a specified target configuration. 
ARMOUR accomplishes this through a receding-horizon approach: at each iteration, it solves an optimization problem to determine the robot’s trajectory.
ARMOUR parameterizes the robot’s joint trajectories as polynomials, with polynomial coefficients serving as the decision variables. 
One end of the rope is rigidly attached to the robot’s end effector, whose pose is determined by forward kinematics. 
By substituting the end effector’s final pose into DEFT, we obtain a prediction of the rope’s configuration at the conclusion of the motion. 
The optimization’s cost function then minimizes the Euclidean distance between this predicted BDLO configuration from DEFT and the target configuration.
To ensure the resultant motion is physically feasible, ARMOUR incorporates constraints on the robot’s joint positions, velocities, and torques. 
Once the optimization is solved, the resulting trajectory is tracked by ARMOUR’s controller.
This planning–execution process is repeated until the rope achieves the desired configuration or a maximum iteration threshold is exceeded (in which case the task is deemed a failure).
Additional details on ARMOUR’s trajectory parameterization and its closed-loop controller can be found in \cite[Section IX]{ARMOUR}. 
