\subsection{Proof of Theorem \ref{thm:junctionconstraint}}
\label{appendix:theorem_junction2}
\begin{figure}[t]
    \centering
    \includegraphics[width=0.3\textwidth]{figures/omegatox.pdf}
    \caption{%
      An illustration of applying 
      \(\hat{\boldsymbol{\Omega}}_{\Delta}^i\)
      to rotate the \(i\)-th segment. 
      % \Yizhou{to be updated, change arrow direction.}
    }
    \label{fig:orientationonx}
\end{figure}

For convenience throughout this section, we suppress the arguments to functions for convenience.
To prove Theorem~\ref{thm:junctionconstraint}, we begin by describing how 
\(\hat{\boldsymbol{\Omega}}^{i}_{\Delta}\) is applied to rotate the \(i\)-th segment. 
Note that $\Delta_t$ should be sufficiently small for $\hat{\boldsymbol{\Omega}}^i_{\Delta}$ to accurately approximate the change in orientation. 
As shown in Figure~\ref{fig:orientationonx}, the rotation axis coincides with the center of the \(i\)-th segment. 
In practice, the rotation of the \(i\)-th segment is achieved by converting
\(\hat{\boldsymbol{\Omega}}^{ i}_{\Delta}\) into the rotation matrix  \(\hat{\boldsymbol{\Omega}}^{ i}_{\Delta} \mapsto \mathbf{R}^i_\Delta\).
Consequently, we obtain \(\Delta \hat{\mathbf{x}}^i\) and \(\Delta \hat{\mathbf{x}}^{i+1}\) via
\begin{equation}
    \Delta \hat{\mathbf{x}}^i 
    \;=\; 
    \mathbf{R}^i_\Delta \,\frac{\bigl(\hat{\mathbf{x}}^{i+1}-\hat{\mathbf{x}}^{i}\bigr)}{2},
    \label{eq:rx1}
\end{equation}
\begin{equation}
    \Delta \hat{\mathbf{x}}^{i+1} 
    \;=\; 
    \mathbf{R}^i_\Delta \,\frac{\bigl(\hat{\mathbf{x}}^{i}-\hat{\mathbf{x}}^{i+1}\bigr)}{2}.
    \label{eq:rx2}
\end{equation}
Because each segment rotates only about its own center, linear momentum is naturally conserved. 
Thus, we can simplify the optimization problem and write the following optimization problem:
\begin{align}
    \label{eq:appendix_optimization_simplified_attachement}
    & 
    \underset{\Delta \hat{\mathbf{X}}}{\min} 
    && 
    \frac{1}{2}\Bigl(C^{p,i}_O(\hat{\mathbf{X}}, \Delta \hat{\mathbf{X}})\Bigr)^2 
    \\
    &\text{s.t.} 
    && 
    \mathbf{I}^{p, i} \,\hat{\boldsymbol{\Omega}}_{\Delta}^{p, i}  
    \;+\; 
    \mathbf{I}^{c, 1} \,\hat{\boldsymbol{\Omega}}_{\Delta}^{c, 1}  
    \;=\; 
    \mathbf{0}\,,
    \label{eq:appendix_constraint_rigid_body}
\end{align}
where \eqref{eq:constraints_orietation}, 
\eqref{eq:optimization_simplified_attachement},
and \eqref{eq:appendix_constraint_rigid_body} are analogous to 
\eqref{eq:constraint_attachement}, 
\eqref{eq:optimization_simplified_attachement}, 
and \eqref{eq:linearm_simplified}, respectively.
Next, we introduce a Lagrange multiplier and follow the same derivation steps as in \eqref{eq:lagarange1}, \eqref{eq:subsitituion}, and \eqref{eq:lambdar0}, which leads to:
\begin{align}
    \hat{\boldsymbol{\Omega}}_\Delta^{p,i} 
    \;=\; 
    \mathbf{I}^{c, 1}\,\bigl(\mathbf{I}^{c, 1}+\mathbf{I}^{p,i}\bigr)^{-1}\,C^{p,i}_O(\hat{\mathbf{X}}, \mathbf{0})\,
    \frac{\hat{\boldsymbol{\Omega}}^{c, 1} - \hat{\boldsymbol{\Omega}}^{p, i}}{\|\hat{\boldsymbol{\Omega}}^{c, 1} - \hat{\boldsymbol{\Omega}}^{p, i}\|_2}, 
    \notag
\end{align}
\begin{align}
    \hat{\boldsymbol{\Omega}}_\Delta^{c,1} 
    \;=\; 
    \mathbf{I}^{p, i}\,\bigl(\mathbf{I}^{c,1}+\mathbf{I}^{p,i}\bigr)^{-1}\,C^{p,i}_O(\hat{\mathbf{X}}, \mathbf{0})\,
    \frac{\hat{\boldsymbol{\Omega}}^{p, i} - \hat{\boldsymbol{\Omega}}^{c, 1}}{\|\hat{\boldsymbol{\Omega}}^{c, 1} - \hat{\boldsymbol{\Omega}}^{p, i}\|_2}.
    \notag
\end{align}
Once $\hat{\boldsymbol{\Omega}}_\Delta^{p,i}, \hat{\boldsymbol{\Omega}}_\Delta^{c,1}$ are obtained, 
\eqref{eq:rx1} and \eqref{eq:rx2} are used to update 
\(\hat{\mathbf{x}}^{p,i}\) and \(\hat{\mathbf{x}}^{c,1}\) accordingly.

