\subsection{Proof of Theorem \ref{thm:attachment}}
\label{appendix:theorem_junction1}
To prove Theorem~\ref{thm:attachment}, we first formulate the optimization problem:
\begin{align}
    & &\underset{\Delta \hat{\mathbf{X}}}{\min}& \hspace{0.3cm} \frac{1}{2}\left(C^{p,i}_A(\hat{\mathbf{X}}, \Delta \hat{\mathbf{X}})\right)^2  
    \label{eq:optimization_simplified_attachement}\\
    &&\text{s.t.} &\hspace{0.3cm}  \Delta \hat{\mathbf{x}}^j = \mathbf{0}, \hspace{0.3cm} \forall j  \{(p, i),(c, 1)\} 
    \label{eq:index_selection}\\
    && & \hspace{0.3cm}   \mathbf{M}^{p, i}  \Delta \hat{\mathbf{x}}^{p, i}\ + \mathbf{M}^{c, 1}  \Delta \hat{\mathbf{x}}^{c, 1}= \mathbf{0}, 
    \label{eq:linearm_simplified} \\
    && &\hspace{0.3cm} 
    \mathbf{I}^{p, i}   \hat{\boldsymbol{\Omega}}^{p, i} _\Delta (\hat{\mathbf{X}},\Delta \hat{\mathbf{X}} ) = \mathbf{0},
    \label{eq:angularm_simplified} 
\end{align}
To solve the above optimization, we introduce Lagrange multipliers $\boldsymbol{\lambda}_\text{l} \in \mathbb{R}^{3}$ and $\boldsymbol{\lambda}_\text{r} \in \mathbb{R}^{3}$ associated with constraints \eqref{eq:linearm_simplified} and \eqref{eq:angularm_simplified}, respectively.
The corresponding Lagrangian $\mathcal{L}$  can be found as follow:
 \begin{align}
    \begin{split}
    \mathcal{L}(\Delta \hat{\textbf{x}}^{p, i}, \Delta & \hat{\mathbf{x}}^{c,1},  \boldsymbol{\lambda}_\text{l}, \boldsymbol{\lambda}_\text{r}) = \\
    &\frac{1}{2}\left(C^{p,i}_A(\hat{\mathbf{X}}, \Delta \hat{\mathbf{X}})\right)^2- \\
    & \boldsymbol{\lambda}^T_\text{l}(\Massmatrix^i  \Delta \hat{\textbf{x}}^{p, i} + \Massmatrix^{c, 1}  \Delta \hat{\mathbf{x}}^{c,1})- \\
    & \boldsymbol{\lambda}^T_\text{r}\mathbf{I}^{p, i}   \hat{\boldsymbol{\Omega}}^{p, i} _\Delta (\hat{\mathbf{X}},\Delta \hat{\mathbf{X}} )
    \label{eq:attachment_lag} 
    \end{split}
\end{align}
Next, we take the partial derivatives of \eqref{eq:attachment_lag} with respect to $\Delta \hat{\textbf{x}}^{p, i}, \Delta \hat{\mathbf{x}}^{c,1},
\boldsymbol{\lambda}_\text{l}, \boldsymbol{\lambda}_\text{r}$. 
Setting each derivative to zero yields the system of equations \eqref{eq:lagarange1}, \eqref{eq:lagarange2}, \eqref{eq:lagarangem}, and \eqref{eq:lagarangel}.
\begin{align}
    \label{eq:lagarange1}
    \begin{split}
     & \frac{\partial \mathcal{L}(\Delta \hat{\textbf{x}}^{p, i}, \Delta  \hat{\mathbf{x}}^{c,1},  \boldsymbol{\lambda}_\text{l}, \boldsymbol{\lambda}_\text{r})}{\partial \Delta  \hat{\mathbf{x}}^{p,i}} =  \\
     & C^{p,i}_A(\hat{\mathbf{X}}, \Delta \hat{\mathbf{X}})\frac{(\hat{\textbf{x}}^{p, i}+\Delta \hat{\textbf{x}}^{p, i}) - (\hat{\mathbf{x}}^{c,1}+\Delta \hat{\mathbf{x}}^{c,1})}{||(\hat{\textbf{x}}^{p, i}+\Delta \hat{\textbf{x}}^{p, i}) - (\hat{\mathbf{x}}^{c,1}+\Delta \hat{\mathbf{x}}^{c,1})||_2}-\\
     &  \boldsymbol{\lambda}^T_\text{l}\Massmatrix^{p, i} - \frac{\partial \boldsymbol{\lambda}^T_\text{r}\mathbf{I}^{p, i}   \hat{\boldsymbol{\Omega}}^{p, i} _\Delta (\hat{\mathbf{X}},\Delta \hat{\mathbf{X}} )}{\partial \Delta  \hat{\mathbf{x}}^{p,i}} = \textbf{0}
     \end{split}   
\end{align}
\begin{align}
    \label{eq:lagarange2}
    \begin{split}
     & \frac{\partial \mathcal{L}(\Delta \hat{\textbf{x}}^{p, i}, \Delta  \hat{\mathbf{x}}^{c,1},  \boldsymbol{\lambda}_\text{l}, \boldsymbol{\lambda}_\text{r})}{\partial \Delta  \hat{\mathbf{x}}^{c,1}} =  \\
     & C^{p,i}_A(\hat{\mathbf{X}}, \Delta \hat{\mathbf{X}})\frac{(\hat{\mathbf{x}}^{c,1}+\Delta \hat{\mathbf{x}}^{c,1})-(\hat{\textbf{x}}^{p, i}+\Delta \hat{\textbf{x}}^{p, i}) }{||(\hat{\textbf{x}}^{p, i}+\Delta \hat{\textbf{x}}^{p, i}) - (\hat{\mathbf{x}}^{c,1}+\Delta \hat{\mathbf{x}}^{c,1})||_2}-\\
     &  \boldsymbol{\lambda}^T_\text{l}\Massmatrix^{c, 1} - \frac{\partial \boldsymbol{\lambda}^T_\text{r}\mathbf{I}^{p, i}   \hat{\boldsymbol{\Omega}}^{p, i} _\Delta (\hat{\mathbf{X}},\Delta \hat{\mathbf{X}} )}{\partial \Delta  \hat{\mathbf{x}}^{c,1}}  = \textbf{0}
     \end{split}   
\end{align}
\begin{align}
    \label{eq:lagarangem}
      & \frac{\partial \mathcal{L}(\Delta \hat{\textbf{x}}^{p, i}, \Delta  \hat{\mathbf{x}}^{c,1},  \boldsymbol{\lambda}_\text{l}, \boldsymbol{\lambda}_\text{r})}{\partial \boldsymbol{\lambda}_\text{l}} = -(\Massmatrix^i  \Delta \hat{\textbf{x}}^{p, i} + \Massmatrix^{c, 1}  \Delta \hat{\mathbf{x}}^{c,1}) = \textbf{0}
\end{align}
\begin{align}
    \label{eq:lagarangel}
    \begin{split}
     & \frac{\partial \mathcal{L}(\Delta \hat{\textbf{x}}^{p, i}, \Delta  \hat{\mathbf{x}}^{c,1},  \boldsymbol{\lambda}_\text{l}, \boldsymbol{\lambda}_\text{r})}{\partial \boldsymbol{\lambda}_\text{r}} = 
      - \textit{\textbf{I}}^i   \boldsymbol{\lambda}^T_\text{r}\mathbf{I}^{p, i}   \hat{\boldsymbol{\Omega}}^{p, i} _\Delta (\hat{\mathbf{X}},\Delta \hat{\mathbf{X}} ) = \textbf{0}
     \end{split}
\end{align}
We first observe that solving \eqref{eq:lagarangem} leads to the following equation:
\begin{equation}
     \Delta \hat{\textbf{x}}^{c, 1} = -(\Massmatrix^{c, 1})^{-1}\Massmatrix^{p, i}  \Delta \hat{\mathbf{x}}^{p,i}
     \label{eq:subsitituion}
\end{equation}
This indicates that $\Delta \hat{\textbf{x}}^{p, i}, \Delta  \hat{\mathbf{x}}^{c,1}$ are colinear.
Based on the definition of $ \hat{\boldsymbol{\Omega}}^{i} _\Delta (\hat{\mathbf{X}},\Delta \hat{\mathbf{X}} )$ in Appendix~\ref{sec:appendix_angular_momentum}, one can see that if $\Delta \hat{\textbf{x}}^{p, i}, \Delta  \hat{\mathbf{x}}^{c,1}$ are colinear, then $ \hat{\boldsymbol{\Omega}}^{p, i} _\Delta (\hat{\mathbf{X}},\Delta \hat{\mathbf{X}} ) = \textbf{0}$. 
Note that this is true regardless of the size of $\Delta_t$.
Hence, it naturally satisfies \eqref{eq:lagarangel}.
Next, summing \eqref{eq:lagarange1} with \eqref{eq:lagarange2} yields the following:
\begin{equation} 
\boldsymbol{\lambda}_\text{l}\Massmatrix^{p, i} + \boldsymbol{\lambda}_\text{l}\Massmatrix^{c, 1} = \textbf{0}
\label{eq:lambdar0}
\end{equation}
As mass matrices are always positive definite.
This indicates $\boldsymbol{\lambda_l} = \textbf{0}$.
Based on \eqref{eq:lagarange1}, \eqref{eq:subsitituion} and \eqref{eq:lambdar0}, solving for $\Delta  \hat{\mathbf{x}}^{p,i}$ is equivalent to solving the following equation:
\begin{equation}
    C^{p,i}_A(\hat{\mathbf{X}}, \Delta \hat{\mathbf{X}}) = \textbf{0}
\end{equation}
After applying some algebraic manipulations, $\Delta \hat{\mathbf{x}}^{p, i}$ can be found to be:
\begin{equation*}
    \label{eq:momentum_solution1} 
  \Delta \hat{\mathbf{x}}^{p, i} =  \mathbf{M}^{c, 1} (\mathbf{M}^{p, i}+\mathbf{M}^{c, 1})^{-1}  C^{p,i}_A(\hat{\mathbf{X}},\mathbf{0}) \frac{(\hat{\mathbf{x}}^{c, 1} - \hat{\mathbf{x}}^{p, i})}{||\hat{\mathbf{x}}^{c, 1} - \hat{\mathbf{x}}^{p, i}||_2},
\end{equation*}
Using $\Delta \hat{\mathbf{x}}^{p, i}$ and \eqref{eq:subsitituion}, one can find:
\begin{equation*}
    \label{eq:momentum_solution2} 
     \Delta \hat{\mathbf{x}}^{c, 1} =  \mathbf{M}^{p, i}  (\mathbf{M}^{p, i}+\mathbf{M}^{c, 1})^{-1}
    \ C^{p,i}_A(\hat{\mathbf{X}},\mathbf{0}) \frac{(\hat{\mathbf{x}}^{p, i} - \hat{\mathbf{x}}^{c, 1})}{||\hat{\mathbf{x}}^{c, 1} - \hat{\mathbf{x}}^{p, i}||_2}.
\end{equation*}

