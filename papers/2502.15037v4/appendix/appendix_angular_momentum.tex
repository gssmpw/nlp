\subsection{Definition of $\hat{\boldsymbol{\Omega}}^i_{\Delta,t} (\hat{\mathbf{X}},\Delta \hat{\mathbf{X}})$ and $\hat{\boldsymbol{\Omega}}^i$}
\label{sec:appendix_angular_momentum}
In practice, it can be both challenging and impractical to attach hardware onto each segment for tracking its orientation changes. 
To address this limitation, we approximate the change in orientation, as illustrated in Figure \ref{fig:orientation change}.
We define $\hat{\boldsymbol{\Omega}}^i_\Delta (\hat{\mathbf{X}},\Delta \hat{\mathbf{X}} )$ as follows:
\begin{equation}
    \begin{split}
    & \hat{\boldsymbol{\Omega}}^i_\Delta (\hat{\mathbf{X}},\Delta \hat{\mathbf{X}} ) = \\ 
    & \frac{\hat{\mathbf{x}}^{i+1} - \hat{\mathbf{x}}^{i}}{||\hat{\mathbf{x}}^{i+1} - \hat{\mathbf{x}}^{i}||_2}
    \times \frac{(\hat{\mathbf{x}}^{i+1} + \Delta \hat{\mathbf{x}}^{i+1}) - (\hat{\mathbf{x}}^{i}+\Delta \hat{\mathbf{x}}^i)}{||(\hat{\mathbf{x}}^{i+1} + \Delta \hat{\mathbf{x}}^{i+1}) - (\hat{\mathbf{x}}^{i}+\Delta \hat{\mathbf{x}}^i)||_2}
    \end{split}
\end{equation}
Intuitively, the cross product of each edge's tangent vector provides an approximation of the rotation axis and orientation change for the segment $i$.
This approximation becomes more accurate when $\Delta_t$ is small. 

Next, we describe how to compute $\hat{\boldsymbol{\Omega}}^i$. 
Note that it is defined recursively using the computation at the previous time step. 
As a result just within this appendix, we add a subscript $t$ to the symbol.
To compute $\hat{\boldsymbol{\Omega}}^i_{t+1}$, we first convert both 
$\hat{\boldsymbol{\Omega}}^i_{t}$ and the correction rotation 
$\hat{\boldsymbol{\Omega}}^i_{\Delta}(\hat{\mathbf{X}}_{t}, \Delta \hat{\mathbf{X}}_{t})$ 
from angle-axis to quaternions. We then update 
$\hat{\boldsymbol{\Omega}}^i_{t}$ by applying 
$\hat{\boldsymbol{\Omega}}^i_{\Delta}(\hat{\mathbf{X}}_{t}, \Delta \hat{\mathbf{X}}_{t})$ 
in quaternion form, and finally convert the result back to angle-axis coordinates 
to obtain $\hat{\boldsymbol{\Omega}}^i_{t+1}$. 
Note that we do not continuously use quaternions because angle--axis is more 
convenient for representing angular momentum.
\begin{figure}[t]
    \centering
\includegraphics[width=0.5\textwidth]{figures/rotation_omega.pdf}
    \caption{An illustration of $\hat{\boldsymbol{\Omega}}^i_\Delta (\hat{\mathbf{X}},\Delta \hat{\mathbf{X}})$
    }
    \label{fig:orientation change}
\end{figure}