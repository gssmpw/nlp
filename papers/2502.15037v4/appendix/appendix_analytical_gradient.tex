\subsection{Proof of Theorem \ref{thm:potential_energy_gradient}}
\label{appendix: Theorem 1 Proof}
To prove the theorem, we begin by introducing the curvature binormal and material curvature.

\textbf{Material Curvatures:}
The curvature binormal $\bcurvature$ is traditionally used to represent the turning angle and axis between two consecutive edges:
\begin{equation}
    \kappa \mathbf{b}^i = \frac{2 \, \mathbf{e}^{i-1} \times \mathbf{e}^{i}}{\|\mathbf{e}^{i-1}\|_2 \|\mathbf{e}^{i}\|_2 + \mathbf{e}^{i-1} \cdot \mathbf{e}^{i}},
   \label{eq:curvature_appendix}
\end{equation}
where \(\mathbf{e}^{i-1}\) and \(\mathbf{e}^i\) are consecutive edge vectors, \(\times\) denotes the cross product, and \(\cdot\) denotes the dot product. 
By incorporating the material frame \eqref{eq:m1} and \eqref{eq:m2}, the curvature binormal \eqref{eq:curvature_appendix} is projected onto the material frame to quantify the extent to which the curvature aligns with the frame's orientation. 
This projection, which we call the material curvature, provides additional insights into the deformation characteristics, allowing us to distinguish between bending and twisting behaviors within the material frame and is defined as:
\begin{equation}
    \bm{\omega}^{(i,j)} = 
    \left(
        \bcurvature \cdot \mathbf{m}_1^j 
        , \bcurvature \cdot \mathbf{m}_2^j
    \right)^T
    \quad \text{for} \quad j \in \{i-1, i\}.
   \label{eq:materialcurvature}
\end{equation}

\textbf{Potential Energy:}
The potential energy is composed of the bending energy and twisting energy:
\begin{equation}
  P(\MaterialFrame(\mathbf{X}_t, \bm{\theta}_t), \materialp\bigr) = P_\text{bend}(\MaterialFrame(\mathbf{X}_t, \bm{\theta}_t), \materialp\bigr) + P_\text{twist}(\MaterialFrame(\mathbf{X}_t, \bm{\theta}_t), \materialp\bigr),
       \label{eq:potentialP_appendix}
\end{equation}
where
\begin{align}
    \begin{split}
       & P_\text{bend}(\MaterialFrame(\mathbf{X}_t, \bm{\theta}_t), \materialp\bigr) =  \\
    & \sum_{i=1}^{n-1} \sum_{j=i-1}^{i} \frac{1}{2}\left( \bm{\omega}^{(i,j)} - \overline{\bm{\omega}}^{(i,j)} \right)^T \mathbf{B}^j \left( \bm{\omega}^{(i,j)} - \overline{\bm{\omega}}^{(i,j)} \right),
    \end{split}
           \label{eq:potentialPbend}
\end{align}
and
\begin{equation}
    P_\text{twist}(\MaterialFrame(\mathbf{X}_t, \bm{\theta}_t), \materialp\bigr) = \sum_{i=1}^{n-1} \frac{1}{2} \beta^i \left(\theta^i - \theta^{i-1} \right)^2,
   \label{eq:potentialPtwist}
\end{equation}
where \(\overline{\bm{\omega}}^{(i,j)}\) denotes the undeformed material curvature, which is calculated when the DLO is in a static state without any external or internal forces applied, and $\mathbf{B}$ and $\beta$, are components of $\materialp$, representing the bending stiffness and twisting stiffness, respectively.

\textbf{Gradient of Potential Energy}: With \eqref{eq:potentialPbend} and \eqref{eq:potentialPtwist}, the gradient of \eqref{eq:potentialP_appendix} can be derived as following:
\begin{equation}
      \begin{split}
    \label{eq:potentialderivative}
    \frac{\partial P(\MaterialFrame(\mathbf{X}_t, \bm{\theta}_t), \materialp\bigr)}{\partial \theta^i}  = 
    \frac{\partial P_{bend}(\MaterialFrame(\mathbf{X}_t, \bm{\theta}_t), \materialp\bigr)}{\partial \theta^i}  + \\  \frac{\partial P_{twist}(\MaterialFrame(\mathbf{X}_t, \bm{\theta}_t), \materialp\bigr)}{\partial \theta^i} 
        \end{split}
\end{equation}
We begin by deriving the first term. 
Note that $\theta^i$ is only relevant in $\bm{\omega}^{(i,i)}$ and $\bm{\omega}^{(i+1,i)}$. 
Using the chain rule, we obtain:
\begin{equation}
    \frac{\partial P_{bend}(\MaterialFrame(\mathbf{X}_t, \bm{\theta}_t), \materialp\bigr)}{\partial \theta^i}  = 
    \sum_{k=i}^{i+1} \frac{\partial P_{bend}(\MaterialFrame(\mathbf{X}_t, \bm{\theta}_t), \materialp\bigr)}{\partial \bm{\omega}^{(k,i)}} \frac{\partial \bm{\omega}^{(k,i)}}{\partial \theta^i}.
\end{equation}
To compute $\frac{\partial \bm{\omega}^{(k,i)}}{\partial \theta^i}$, we use the identities $\frac{\partial \mathbf{m}_1^i}{\partial \theta^i} = \mathbf{m}_2^i$ and $\frac{\partial \mathbf{m}_2^i}{\partial \theta^i} = -\mathbf{m}_1^i$.
These lead to:
\begin{equation}
    \frac{\partial \bm{\omega}^{(k, i)}}{\partial \theta^i} = \begin{bmatrix}
0 & 1 \\
-1 & 0
\end{bmatrix} \bm{\omega}^{(k, i)}
\end{equation}
Substituting this result back, we obtain:
\begin{equation}
    \begin{split}
  \frac{\partial P_{\mathrm{bend}}(\MaterialFrame(\mathbf{X}_t, \bm{\theta}_t), \materialp\bigr)}{\partial \theta^i}
  \;=\; 
  \sum_{k=i}^{\,i+1}
  \bigl(\mathbf{B}^k\, (&\bm{\omega}^{(k,i)}  - \bar{\bm{\omega}}^{(k,i)})\bigr)^{T} \cdot
  \\
    &\begin{bmatrix}0 & 1\\[6pt]-1 & 0\end{bmatrix}
  \bm{\omega}^{(k,i)},
      \end{split}
      \label{eq:potentialbendgradient}
\end{equation}
The second term of \eqref{eq:potentialderivative} can be derived, resulting in:
\begin{equation}
        \label{eq:potentialtwistgradient}
        \frac{\partial P_{twist}(\MaterialFrame(\mathbf{X}_t, \bm{\theta}_t), \materialp\bigr)}{\partial \theta^i} = \beta^i (\theta^i - \theta^{i-1}) - \beta^{i+1}(\theta^{i+1} - \theta^{i})
\end{equation}
By substituting \eqref{eq:potentialbendgradient} and \eqref{eq:potentialtwistgradient} into \eqref{eq:potentialderivative}, we obtain the analytical gradient of the potential energy.
