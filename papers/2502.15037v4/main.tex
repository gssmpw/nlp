\documentclass[conference]{IEEEtran}
\usepackage{times}
\usepackage{amsthm}
\usepackage{multicol}
% \usepackage{color}
\usepackage{xcolor}
\usepackage{tabularx}
\usepackage{hhline}
\usepackage{blkarray}
\usepackage{multicol}
\bibliographystyle{unsrt}
\usepackage[bookmarks=true]{hyperref}
\usepackage[font=footnotesize]{caption}
\usepackage{subcaption}
\usepackage{multirow}
\usepackage{amsmath}
\usepackage{setspace}
\usepackage{algorithm}
\usepackage{algpseudocode}
\usepackage{bm}
\usepackage{graphicx}
\usepackage{hyperref}
\usepackage{marginnote}
\usepackage[normalem]{ulem}
\usepackage{caption}
\usepackage{lipsum}
\let\proof\relax
\let\endproof\relax
\usepackage{gensymb}
\usepackage{makecell}
\usepackage{mathrsfs}
\let\proof\relax
\let\endproof\relax
\usepackage{booktabs}
\usepackage{adjustbox}
\usepackage{amssymb}  
\usepackage{wrapfig}
% numbers option provides compact numerical references in the text. 
\usepackage[numbers]{natbib}
\def\bibfont{\small}
% \bibliographystyle{IEEEtranN}
\usepackage{multicol}
\usepackage[bookmarks=true]{hyperref}
%%%%%%%%%%%---SETME-----%%%%%%%%%%%%%
%replace @@ with the submission number submission site.
\newcommand{\thiswork}{INF$^2$\xspace}
%%%%%%%%%%%%%%%%%%%%%%%%%%%%%%%%%%%%


%\newcommand{\rev}[1]{{\color{olivegreen}#1}}
\newcommand{\rev}[1]{{#1}}


\newcommand{\JL}[1]{{\color{cyan}[\textbf{\sc JLee}: \textit{#1}]}}
\newcommand{\JW}[1]{{\color{orange}[\textbf{\sc JJung}: \textit{#1}]}}
\newcommand{\JY}[1]{{\color{blue(ncs)}[\textbf{\sc JSong}: \textit{#1}]}}
\newcommand{\HS}[1]{{\color{magenta}[\textbf{\sc HJang}: \textit{#1}]}}
\newcommand{\CS}[1]{{\color{navy}[\textbf{\sc CShin}: \textit{#1}]}}
\newcommand{\SN}[1]{{\color{olive}[\textbf{\sc SNoh}: \textit{#1}]}}

%\def\final{}   % uncomment this for the submission version
\ifdefined\final
\renewcommand{\JL}[1]{}
\renewcommand{\JW}[1]{}
\renewcommand{\JY}[1]{}
\renewcommand{\HS}[1]{}
\renewcommand{\CS}[1]{}
\renewcommand{\SN}[1]{}
\fi

%%% Notion for baseline approaches %%% 
\newcommand{\baseline}{offloading-based batched inference\xspace}
\newcommand{\Baseline}{Offloading-based batched inference\xspace}


\newcommand{\ans}{attention-near storage\xspace}
\newcommand{\Ans}{Attention-near storage\xspace}
\newcommand{\ANS}{Attention-Near Storage\xspace}

\newcommand{\wb}{delayed KV cache writeback\xspace}
\newcommand{\Wb}{Delayed KV cache writeback\xspace}
\newcommand{\WB}{Delayed KV Cache Writeback\xspace}

\newcommand{\xcache}{X-cache\xspace}
\newcommand{\XCACHE}{X-Cache\xspace}


%%% Notions for our methods %%%
\newcommand{\schemea}{\textbf{Expanding supported maximum sequence length with optimized performance}\xspace}
\newcommand{\Schemea}{\textbf{Expanding supported maximum sequence length with optimized performance}\xspace}

\newcommand{\schemeb}{\textbf{Optimizing the storage device performance}\xspace}
\newcommand{\Schemeb}{\textbf{Optimizing the storage device performance}\xspace}

\newcommand{\schemec}{\textbf{Orthogonally supporting Compression Techniques}\xspace}
\newcommand{\Schemec}{\textbf{Orthogonally supporting Compression Techniques}\xspace}



% Circular numbers
\usepackage{tikz}
\newcommand*\circled[1]{\tikz[baseline=(char.base)]{
            \node[shape=circle,draw,inner sep=0.4pt] (char) {#1};}}

\newcommand*\bcircled[1]{\tikz[baseline=(char.base)]{
            \node[shape=circle,draw,inner sep=0.4pt, fill=black, text=white] (char) {#1};}}

\pdfinfo{
   /Author (Homer Simpson)
   /Title  (Robots: Our new overlords)
   /CreationDate (D:20101201120000)
   /Subject (Robots)
   /Keywords (Robots;Overlords)
}


\begin{document}
% \newpage
% \thispagestyle{empty}
% \mbox{}
% \section{Summary of Mathematical Notations}
\label{sec:notations}

We summarize the main mathematical notations used in the main paper in Table \ref{table:notations}.

\begin{table*}[h]
	\centering
	\caption{Summary of main mathematical notations.}
	%\resizebox{1\columnwidth}{!}{
		\begin{tabular}{c|l}
			\toprule
			Notation  & Description  \\
			\hline
            $ \mathcal{G} $ & a document graph \\
			$ \mathcal{D} $ & a corpus of documents, $ \mathcal{D}=\{d_i\}_{i=1}^{N} $ \\
			$ N $ & number of documents in the corpus, $ N=|\mathcal{D}| $ \\
			$ d_i $ & document $ i $ containing a sequence of words, $ d_i=\{w_{i,v}\}_{v=1}^{|d_i|}\subset\mathcal{V} $ \\
			$ \mathcal{V} $ & vocabulary \\
			$ |d_i| $ & number of words in document $ i $ \\
			$ \mathcal{E} $ & a set of graph edges connecting documents, $ \mathcal{E}=\{e_{ij}\} $ \\
			$ \mathcal{N}(i) $ & the neighbor set of document $ i $ \\
            $ \Bbb H^{n,K} $ & Hyperboloid model with dimension $ n $ and curvature $ -1/K $ \\
			$ \mathcal{T}_{\textbf{x}}\Bbb H^{n,K} $ & tangent (Euclidean) space around hyperbolic vector $ x\in\Bbb H^{n,K} $ \\
			$ \exp_{\textbf{x}}^K(\textbf{v}) $ & exponential map, projecting tangent vector $ \textbf{v} $ to hyperbolic space \\
			$ \log_{\textbf{x}}^K(\textbf{y}) $ & logarithmic map, projecting hyperbolic vector $ \textbf{y} $ to $ \textbf{x} $'s tangent space \\
			$ d_{\mathcal{L}}^K(\textbf{x},\textbf{y}) $ & hyperbolic distance between hyperbolic vectors $ \textbf{x} $ and $ \textbf{y} $ \\
			$ \text{PT}_{\textbf{x}\rightarrow\textbf{y}}^K(\textbf{v}) $ & parallel transport, transporting $ \textbf{v} $ from $ \textbf{x} $'s tangent space to $ \textbf{y} $'s \\
			$ H $ & length of a path on topic tree \\
            $ \sigma(t,i) $ & similarity between topic $ t $ and document $ i $ \\
            $ \bm{\pi}_i $ & path distribution of document $ i $ over topic tree \\
            $ \textbf{z}_{t,p} $ & hyperbolic ancestral hidden state of topic $ t $ \\
            $ \textbf{z}_{t,s} $ & hyperbolic fraternal hidden state of topic $ t $ \\
			$ \textbf{z}_t $ & hyperbolic hidden state of topic $ t $ \\
            $ \sigma(h,i) $ & similarity between topic $ t $ and document $ i $ \\
            $ \textbf{z}_h $ & hyperbolic hidden state of level $ h $ \\
			$ \bm{\delta}_i $ & level distribution of document $ i $ over topic tree \\
            $ \bm{\theta}_i $ & topic distribution of document $ i $ over topic tree \\
            $ \textbf{e}_i $ & hierarchical tree embedding of document $ i $ \\
            $ T $ & number of topics on topic tree \\
            $ \textbf{g}_i $ & hierarchical graph embedding of document $ i $ \\
			$ \textbf{U} $ & a matrix of word embeddings, $ \textbf{U}\in\Bbb R^{|\mathcal{V}|\times(n+1)} $ \\
			$ \bm{\beta} $ & topic-word distribution $ \bm{\beta}\in\Bbb R^{T\times |\mathcal{V}|} $ \\
			\bottomrule
		\end{tabular}
	%}
	%\vspace{-0.2cm}
	\label{table:notations}
\end{table*}
% \newpage

% paper title
\title{DEFT: Differentiable Branched Discrete Elastic Rods for Modeling Furcated DLOs in Real-Time}

% You will get a Paper-ID when submitting a pdf file to the conference system
% \author{Author Names Omitted for Anonymous Review. Paper-ID [429]}

\author{
    Yizhou Chen\hspace{15pt} Xiaoyue Wu\hspace{15pt} Yeheng Zong\hspace{15pt} Anran Li\hspace{15pt} Yuzhen Chen\hspace{15pt} \\ Julie Wu\hspace{15pt} Bohao Zhang\hspace{15pt} Ram Vasudevan\\
  Department of Robotics, University of Michigan, Ann Arbor, MI 48109, United States\\
  \texttt{\{yizhouch, wxyluna, yehengz, anranli, yuzhench, jwuxx, jimzhang, ramv\}@umich.edu}\\
}

%\author{\authorblockN{Michael Shell}
%\authorblockA{School of Electrical and\\Computer Engineering\\
%Georgia Institute of Technology\\
%Atlanta, Georgia 30332--0250\\
%Email: mshell@ece.gatech.edu}
%\and
%\authorblockN{Homer Simpson}
%\authorblockA{Twentieth Century Fox\\
%Springfield, USA\\
%Email: homer@thesimpsons.com}
%\and
%\authorblockN{James Kirk\\ and Montgomery Scott}
%\authorblockA{Starfleet Academy\\
%San Francisco, California 96678-2391\\
%Telephone: (800) 555--1212\\
%Fax: (888) 555--1212}}


% avoiding spaces at the end of the author lines is not a problem with
% conference papers because we don't use \thanks or \IEEEmembership


% for over three affiliations, or if they all won't fit within the width
% of the page, use this alternative format:
% 
%\author{\authorblockN{Michael Shell\authorrefmark{1},
%Homer Simpson\authorrefmark{2},
%James Kirk\authorrefmark{3}, 
%Montgomery Scott\authorrefmark{3} and
%Eldon Tyrell\authorrefmark{4}}
%\authorblockA{\authorrefmark{1}School of Electrical and Computer Engineering\\
%Georgia Institute of Technology,
%Atlanta, Georgia 30332--0250\\ Email: mshell@ece.gatech.edu}
%\authorblockA{\authorrefmark{2}Twentieth Century Fox, Springfield, USA\\
%Email: homer@thesimpsons.com}
%\authorblockA{\authorrefmark{3}Starfleet Academy, San Francisco, California 96678-2391\\
%Telephone: (800) 555--1212, Fax: (888) 555--1212}
%\authorblockA{\authorrefmark{4}Tyrell Inc., 123 Replicant Street, Los Angeles, California 90210--4321}}


\maketitle
\begin{abstract}
% Modeling Branched Deformable Linear Objects (BDLOs) is essential for achieving autonomous wire harness assembly. 
% However, current research primarily focuses on modeling single-threaded Deformable Linear Objects (DLOs) and has not demonstrated promising results in modeling and manipulating BDLOs due to the significant challenges posed by their complex dynamics. 
% To address these challenges, this paper presents Differentiable discrete branched Elastic rods for modeling Furcated DLOs in real-Time (\DEFT), a novel framework that combines a differentiable physics-based model with a learning framework to:
% 1) accurately model BDLO dynamics, including dynamic propagation at junction points and grasping in the middle of a BDLO,
% 2) achieve efficient computation for real-time inference,
% and 3) enable planning to demonstrate dexterous BDLO manipulation. 
% To validate the proposed modeling approach, this paper evaluates \DEFT’s performance in an experimental setup involving two industrial robots and a variety of sensors. 
% A comprehensive series of experiments demonstrates its efficacy in terms of accuracy, computational speed, and generalizability compared to state-of-the-art alternatives. 
% To further demonstrate \DEFT’s utility, this paper employs the modeling framework during planning and control tasks to achieve dexterous manipulation of BDLOs, illustrating its superior performance in autonomous planning and control of branched DLOs compared to existing methods. 
% These results demonstrate \DEFT's significance to advance automation in wire harness assembly.
Autonomous wire harness assembly requires robots to manipulate complex branched cables with high precision and reliability.
A key challenge in automating this process is predicting how these flexible and branched structures behave under manipulation.
\begin{figure}[h!]
    \centering
     % \includegraphics[width=0.5\textwidth]{figures/demo_overview.pdf}
     % \makebox[\textwidth][c]{\includegraphics[width=0.5\textwidth]{figures/demo_overview.pdf}}
     \includegraphics[width=0.5\textwidth]{figures/title_page_figure.pdf}
          % \includegraphics[width=0.5\textwidth]{figures/title_page_pic_draft2.png}
     \caption{This paper introduces DEFT, a novel framework that combines a differentiable physics-based model with a learning-based approach to accurately model and predict the dynamic behavior of Branched Deformable Linear Objects (BDLOs) in real time. 
     As this paper illustrates, this model can be used in concert with a motion planning algorithm to autonomously manipulate BDLOs.
     The figures above illustrate how DEFT can be used to autonomously complete a wire insertion task.
    \textbf{Top:} The system first plans a shape-matching motion, transitioning the BDLO from its \textcolor{init_config}{initial} configuration to the target shape (contoured with \textcolor{mid_config}{yellow}), which serves as an intermediate waypoint.
    \textbf{Bottom:} Starting from the intermediate configuration, the system performs thread insertion, guiding the BDLO into the \textcolor{target_hole}{target hole} while also matching the target shape.
    Notably, DEFT predicts the state of the wire recursively without relying on ground truth or perception data at any point in the process.}
    \label{fig:first_page} 
\end{figure}
Without accurate predictions, it is difficult for robots to reliably plan or execute assembly operations.
While existing research has made progress in modeling single-threaded Deformable Linear Objects (DLOs), extending these approaches to Branched Deformable Linear Objects (BDLOs) presents fundamental challenges. 
The junction points in BDLOs create complex force interactions and strain propagation patterns that cannot be adequately captured by simply connecting multiple single-DLO models.
To address these challenges, this paper presents Differentiable discrete branched Elastic rods for modeling Furcated DLOs in real-Time (\DEFT), a novel framework that combines a differentiable physics-based model with a learning framework to: 1) accurately model BDLO dynamics, including dynamic propagation at junction points and grasping in the middle of a BDLO, 2) achieve efficient computation for real-time inference, and 3) enable planning to demonstrate dexterous BDLO manipulation.
A comprehensive series of real-world experiments demonstrates \DEFT's efficacy in terms of accuracy, computational speed, and generalizability compared to state-of-the-art alternatives. 
% To further demonstrate \DEFT's utility, this paper employs the modeling framework during planning and control tasks to achieve dexterous manipulation of BDLOs, illustrating its superior performance in autonomous planning and control of branched DLOs compared to existing methods. 
% These results demonstrate DEFT's significance to advance automation in wire harness assembly.
\end{abstract}

\IEEEpeerreviewmaketitle

\section{Introduction}

Large language models (LLMs) have achieved remarkable success in automated math problem solving, particularly through code-generation capabilities integrated with proof assistants~\citep{lean,isabelle,POT,autoformalization,MATH}. Although LLMs excel at generating solution steps and correct answers in algebra and calculus~\citep{math_solving}, their unimodal nature limits performance in plane geometry, where solution depends on both diagram and text~\citep{math_solving}. 

Specialized vision-language models (VLMs) have accordingly been developed for plane geometry problem solving (PGPS)~\citep{geoqa,unigeo,intergps,pgps,GOLD,LANS,geox}. Yet, it remains unclear whether these models genuinely leverage diagrams or rely almost exclusively on textual features. This ambiguity arises because existing PGPS datasets typically embed sufficient geometric details within problem statements, potentially making the vision encoder unnecessary~\citep{GOLD}. \cref{fig:pgps_examples} illustrates example questions from GeoQA and PGPS9K, where solutions can be derived without referencing the diagrams.

\begin{figure}
    \centering
    \begin{subfigure}[t]{.49\linewidth}
        \centering
        \includegraphics[width=\linewidth]{latex/figures/images/geoqa_example.pdf}
        \caption{GeoQA}
        \label{fig:geoqa_example}
    \end{subfigure}
    \begin{subfigure}[t]{.48\linewidth}
        \centering
        \includegraphics[width=\linewidth]{latex/figures/images/pgps_example.pdf}
        \caption{PGPS9K}
        \label{fig:pgps9k_example}
    \end{subfigure}
    \caption{
    Examples of diagram-caption pairs and their solution steps written in formal languages from GeoQA and PGPS9k datasets. In the problem description, the visual geometric premises and numerical variables are highlighted in green and red, respectively. A significant difference in the style of the diagram and formal language can be observable. %, along with the differences in formal languages supported by the corresponding datasets.
    \label{fig:pgps_examples}
    }
\end{figure}



We propose a new benchmark created via a synthetic data engine, which systematically evaluates the ability of VLM vision encoders to recognize geometric premises. Our empirical findings reveal that previously suggested self-supervised learning (SSL) approaches, e.g., vector quantized variataional auto-encoder (VQ-VAE)~\citep{unimath} and masked auto-encoder (MAE)~\citep{scagps,geox}, and widely adopted encoders, e.g., OpenCLIP~\citep{clip} and DinoV2~\citep{dinov2}, struggle to detect geometric features such as perpendicularity and degrees. 

To this end, we propose \geoclip{}, a model pre-trained on a large corpus of synthetic diagram–caption pairs. By varying diagram styles (e.g., color, font size, resolution, line width), \geoclip{} learns robust geometric representations and outperforms prior SSL-based methods on our benchmark. Building on \geoclip{}, we introduce a few-shot domain adaptation technique that efficiently transfers the recognition ability to real-world diagrams. We further combine this domain-adapted GeoCLIP with an LLM, forming a domain-agnostic VLM for solving PGPS tasks in MathVerse~\citep{mathverse}. 
%To accommodate diverse diagram styles and solution formats, we unify the solution program languages across multiple PGPS datasets, ensuring comprehensive evaluation. 

In our experiments on MathVerse~\citep{mathverse}, which encompasses diverse plane geometry tasks and diagram styles, our VLM with a domain-adapted \geoclip{} consistently outperforms both task-specific PGPS models and generalist VLMs. 
% In particular, it achieves higher accuracy on tasks requiring geometric-feature recognition, even when critical numerical measurements are moved from text to diagrams. 
Ablation studies confirm the effectiveness of our domain adaptation strategy, showing improvements in optical character recognition (OCR)-based tasks and robust diagram embeddings across different styles. 
% By unifying the solution program languages of existing datasets and incorporating OCR capability, we enable a single VLM, named \geovlm{}, to handle a broad class of plane geometry problems.

% Contributions
We summarize the contributions as follows:
We propose a novel benchmark for systematically assessing how well vision encoders recognize geometric premises in plane geometry diagrams~(\cref{sec:visual_feature}); We introduce \geoclip{}, a vision encoder capable of accurately detecting visual geometric premises~(\cref{sec:geoclip}), and a few-shot domain adaptation technique that efficiently transfers this capability across different diagram styles (\cref{sec:domain_adaptation});
We show that our VLM, incorporating domain-adapted GeoCLIP, surpasses existing specialized PGPS VLMs and generalist VLMs on the MathVerse benchmark~(\cref{sec:experiments}) and effectively interprets diverse diagram styles~(\cref{sec:abl}).

\iffalse
\begin{itemize}
    \item We propose a novel benchmark for systematically assessing how well vision encoders recognize geometric premises, e.g., perpendicularity and angle measures, in plane geometry diagrams.
	\item We introduce \geoclip{}, a vision encoder capable of accurately detecting visual geometric premises, and a few-shot domain adaptation technique that efficiently transfers this capability across different diagram styles.
	\item We show that our final VLM, incorporating GeoCLIP-DA, effectively interprets diverse diagram styles and achieves state-of-the-art performance on the MathVerse benchmark, surpassing existing specialized PGPS models and generalist VLM models.
\end{itemize}
\fi

\iffalse

Large language models (LLMs) have made significant strides in automated math word problem solving. In particular, their code-generation capabilities combined with proof assistants~\citep{lean,isabelle} help minimize computational errors~\citep{POT}, improve solution precision~\citep{autoformalization}, and offer rigorous feedback and evaluation~\citep{MATH}. Although LLMs excel in generating solution steps and correct answers for algebra and calculus~\citep{math_solving}, their uni-modal nature limits performance in domains like plane geometry, where both diagrams and text are vital.

Plane geometry problem solving (PGPS) tasks typically include diagrams and textual descriptions, requiring solvers to interpret premises from both sources. To facilitate automated solutions for these problems, several studies have introduced formal languages tailored for plane geometry to represent solution steps as a program with training datasets composed of diagrams, textual descriptions, and solution programs~\citep{geoqa,unigeo,intergps,pgps}. Building on these datasets, a number of PGPS specialized vision-language models (VLMs) have been developed so far~\citep{GOLD, LANS, geox}.

Most existing VLMs, however, fail to use diagrams when solving geometry problems. Well-known PGPS datasets such as GeoQA~\citep{geoqa}, UniGeo~\citep{unigeo}, and PGPS9K~\citep{pgps}, can be solved without accessing diagrams, as their problem descriptions often contain all geometric information. \cref{fig:pgps_examples} shows an example from GeoQA and PGPS9K datasets, where one can deduce the solution steps without knowing the diagrams. 
As a result, models trained on these datasets rely almost exclusively on textual information, leaving the vision encoder under-utilized~\citep{GOLD}. 
Consequently, the VLMs trained on these datasets cannot solve the plane geometry problem when necessary geometric properties or relations are excluded from the problem statement.

Some studies seek to enhance the recognition of geometric premises from a diagram by directly predicting the premises from the diagram~\citep{GOLD, intergps} or as an auxiliary task for vision encoders~\citep{geoqa,geoqa-plus}. However, these approaches remain highly domain-specific because the labels for training are difficult to obtain, thus limiting generalization across different domains. While self-supervised learning (SSL) methods that depend exclusively on geometric diagrams, e.g., vector quantized variational auto-encoder (VQ-VAE)~\citep{unimath} and masked auto-encoder (MAE)~\citep{scagps,geox}, have also been explored, the effectiveness of the SSL approaches on recognizing geometric features has not been thoroughly investigated.

We introduce a benchmark constructed with a synthetic data engine to evaluate the effectiveness of SSL approaches in recognizing geometric premises from diagrams. Our empirical results with the proposed benchmark show that the vision encoders trained with SSL methods fail to capture visual \geofeat{}s such as perpendicularity between two lines and angle measure.
Furthermore, we find that the pre-trained vision encoders often used in general-purpose VLMs, e.g., OpenCLIP~\citep{clip} and DinoV2~\citep{dinov2}, fail to recognize geometric premises from diagrams.

To improve the vision encoder for PGPS, we propose \geoclip{}, a model trained with a massive amount of diagram-caption pairs.
Since the amount of diagram-caption pairs in existing benchmarks is often limited, we develop a plane diagram generator that can randomly sample plane geometry problems with the help of existing proof assistant~\citep{alphageometry}.
To make \geoclip{} robust against different styles, we vary the visual properties of diagrams, such as color, font size, resolution, and line width.
We show that \geoclip{} performs better than the other SSL approaches and commonly used vision encoders on the newly proposed benchmark.

Another major challenge in PGPS is developing a domain-agnostic VLM capable of handling multiple PGPS benchmarks. As shown in \cref{fig:pgps_examples}, the main difficulties arise from variations in diagram styles. 
To address the issue, we propose a few-shot domain adaptation technique for \geoclip{} which transfers its visual \geofeat{} perception from the synthetic diagrams to the real-world diagrams efficiently. 

We study the efficacy of the domain adapted \geoclip{} on PGPS when equipped with the language model. To be specific, we compare the VLM with the previous PGPS models on MathVerse~\citep{mathverse}, which is designed to evaluate both the PGPS and visual \geofeat{} perception performance on various domains.
While previous PGPS models are inapplicable to certain types of MathVerse problems, we modify the prediction target and unify the solution program languages of the existing PGPS training data to make our VLM applicable to all types of MathVerse problems.
Results on MathVerse demonstrate that our VLM more effectively integrates diagrammatic information and remains robust under conditions of various diagram styles.

\begin{itemize}
    \item We propose a benchmark to measure the visual \geofeat{} recognition performance of different vision encoders.
    % \item \sh{We introduce geometric CLIP (\geoclip{} and train the VLM equipped with \geoclip{} to predict both solution steps and the numerical measurements of the problem.}
    \item We introduce \geoclip{}, a vision encoder which can accurately recognize visual \geofeat{}s and a few-shot domain adaptation technique which can transfer such ability to different domains efficiently. 
    % \item \sh{We develop our final PGPS model, \geovlm{}, by adapting \geoclip{} to different domains and training with unified languages of solution program data.}
    % We develop a domain-agnostic VLM, namely \geovlm{}, by applying a simple yet effective domain adaptation method to \geoclip{} and training on the refined training data.
    \item We demonstrate our VLM equipped with GeoCLIP-DA effectively interprets diverse diagram styles, achieving superior performance on MathVerse compared to the existing PGPS models.
\end{itemize}

\fi 



%\section{Related Work}
%\label{sec:related-work}

%\subsection{Background}

%Defect detection is critical to ensure the yield of integrated circuit manufacturing lines and reduce faults. Previous research has primarily focused on wafer map data, which engineers produce by marking faulty chips with different colors based on test results. The specific spatial distribution of defects on a wafer can provide insights into the causes, thereby helping to determine which stage of the manufacturing process is responsible for the issues. Although such research is relatively mature, the continual miniaturization of integrated circuits and the increasing complexity and density of chip components have made chip-level detection more challenging, leading to potential risks\cite{ma2023review}. Consequently, there is a need to combine this approach with magnified imaging of the wafer surface using scanning electron microscopes (SEMs) to detect, classify, and analyze specific microscopic defects, thus helping to identify the particular process steps where defects originate.

%Previously, wafer surface defect classification and detection were primarily conducted by experienced engineers. However, this method relies heavily on the engineers' expertise and involves significant time expenditure and subjectivity, lacking uniform standards. With the ongoing development of artificial intelligence, deep learning methods using multi-layer neural networks to extract and learn target features have proven highly effective for this task\cite{gao2022review}.

%In the task of defect classification, it is typical to use a model structure that initially extracts features through convolutional and pooling layers, followed by classification via fully connected layers. Researchers have recently developed numerous classification model structures tailored to specific problems. These models primarily focus on how to extract defect features effectively. For instance, Chen et al. presented a defect recognition and classification algorithm rooted in PCA and classification SVM\cite{chen2008defect}. Chang et al. utilized SVM, drawing on features like smoothness and texture intricacy, for classifying high-intensity defect images\cite{chang2013hybrid}. The classification of defect images requires the formulation of numerous classifiers tailored for myriad inspection steps and an Abundance of accurately labeled data, making data acquisition challenging. Cheon et al. proposed a single CNN model adept at feature extraction\cite{cheon2019convolutional}. They achieved a granular classification of wafer surface defects by recognizing misclassified images and employing a k-nearest neighbors (k-NN) classifier algorithm to gauge the aggregate squared distance between each image feature vector and its k-neighbors within the same category. However, when applied to new or unseen defects, such models necessitate retraining, incurring computational overheads. Moreover, with escalating CNN complexity, the computational demands surge.

%Segmentation of defects is necessary to locate defect positions and gather information such as the size of defects. Unlike classification networks, segmentation networks often use classic encoder-decoder structures such as UNet\cite{ronneberger2015u} and SegNet\cite{badrinarayanan2017segnet}, which focus on effectively leveraging both local and global feature information. Han Hui et al. proposed integrating a Region Proposal Network (RPN) with a UNet architecture to suggest defect areas before conducting defect segmentation \cite{han2020polycrystalline}. This approach enables the segmentation of various defects in wafers with only a limited set of roughly labeled images, enhancing the efficiency of training and application in environments where detailed annotations are scarce. Subhrajit Nag et al. introduced a new network structure, WaferSegClassNet, which extracts multi-scale local features in the encoder and performs classification and segmentation tasks in the decoder \cite{nag2022wafersegclassnet}. This model represents the first detection system capable of simultaneously classifying and segmenting surface defects on wafers. However, it relies on extensive data training and annotation for high accuracy and reliability. 

%Recently, Vic De Ridder et al. introduced a novel approach for defect segmentation using diffusion models\cite{de2023semi}. This approach treats the instance segmentation task as a denoising process from noise to a filter, utilizing diffusion models to predict and reconstruct instance masks for semiconductor defects. This method achieves high precision and improved defect classification and segmentation detection performance. However, the complex network structure and the computational process of the diffusion model require substantial computational resources. Moreover, the performance of this model heavily relies on high-quality and large amounts of training data. These issues make it less suitable for industrial applications. Additionally, the model has only been applied to detecting and segmenting a single type of defect(bridges) following a specific manufacturing process step, limiting its practical utility in diverse industrial scenarios.

%\subsection{Few-shot Anomaly Detection}
%Traditional anomaly detection techniques typically rely on extensive training data to train models for identifying and locating anomalies. However, these methods often face limitations in rapidly changing production environments and diverse anomaly types. Recent research has started exploring effective anomaly detection using few or zero samples to address these challenges.

%Huang et al. developed the anomaly detection method RegAD, based on image registration technology. This method pre-trains an object-agnostic registration network with various images to establish the normality of unseen objects. It identifies anomalies by aligning image features and has achieved promising results. Despite these advancements, implementing few-shot settings in anomaly detection remains an area ripe for further exploration. Recent studies show that pre-trained vision-language models such as CLIP and MiniGPT can significantly enhance performance in anomaly detection tasks.

%Dong et al. introduced the MaskCLIP framework, which employs masked self-distillation to enhance contrastive language-image pretraining\cite{zhou2022maskclip}. This approach strengthens the visual encoder's learning of local image patches and uses indirect language supervision to enhance semantic understanding. It significantly improves transferability and pretraining outcomes across various visual tasks, although it requires substantial computational resources.
%Jeong et al. crafted the WinCLIP framework by integrating state words and prompt templates to characterize normal and anomalous states more accurately\cite{Jeong_2023_CVPR}. This framework introduces a novel window-based technique for extracting and aggregating multi-scale spatial features, significantly boosting the anomaly detection performance of the pre-trained CLIP model.
%Subsequently, Li et al. have further contributed to the field by creating a new expansive multimodal model named Myriad\cite{li2023myriad}. This model, which incorporates a pre-trained Industrial Anomaly Detection (IAD) model to act as a vision expert, embeds anomaly images as tokens interpretable by the language model, thus providing both detailed descriptions and accurate anomaly detection capabilities.
%Recently, Chen et al. introduced CLIP-AD\cite{chen2023clip}, and Li et al. proposed PromptAD\cite{li2024promptad}, both employing language-guided, tiered dual-path model structures and feature manipulation strategies. These approaches effectively address issues encountered when directly calculating anomaly maps using the CLIP model, such as reversed predictions and highlighting irrelevant areas. Specifically, CLIP-AD optimizes the utilization of multi-layer features, corrects feature misalignment, and enhances model performance through additional linear layer fine-tuning. PromptAD connects normal prompts with anomaly suffixes to form anomaly prompts, enabling contrastive learning in a single-class setting.

%These studies extend the boundaries of traditional anomaly detection techniques and demonstrate how to effectively address rapidly changing and sample-scarce production environments through the synergy of few-shot learning and deep learning models. Building on this foundation, our research further explores wafer surface defect detection based on the CLIP model, especially focusing on achieving efficient and accurate anomaly detection in the highly specialized and variable semiconductor manufacturing process using a minimal amount of labeled data.


% !TEX root =  ../main.tex
\section{Background on causality and abstraction}\label{sec:preliminaries}

This section provides the notation and key concepts related to causal modeling and abstraction theory.

\spara{Notation.} The set of integers from $1$ to $n$ is $[n]$.
The vectors of zeros and ones of size $n$ are $\zeros_n$ and $\ones_n$.
The identity matrix of size $n \times n$ is $\identity_n$. The Frobenius norm is $\frob{\mathbf{A}}$.
The set of positive definite matrices over $\reall^{n\times n}$ is $\pd^n$. The Hadamard product is $\odot$.
Function composition is $\circ$.
The domain of a function is $\dom{\cdot}$ and its kernel $\ker$.
Let $\mathcal{M}(\mathcal{X}^n)$ be the set of Borel measures over $\mathcal{X}^n \subseteq \reall^n$. Given a measure $\mu^n \in \mathcal{M}(\mathcal{X}^n)$ and a measurable map $\varphi^{\V}$, $\mathcal{X}^n \ni \mathbf{x} \overset{\varphi^{\V}}{\longmapsto} \V^\top \mathbf{x} \in \mathcal{X}^m$, we denote by $\varphi^{\V}_{\#}(\mu^n) \coloneqq \mu^n(\varphi^{\V^{-1}}(\mathbf{x}))$ the pushforward measure $\mu^m \in \mathcal{M}(\mathcal{X}^m)$. 


We now present the standard definition of SCM.

\begin{definition}[SCM, \citealp{pearl2009causality}]\label{def:SCM}
A (Markovian) structural causal model (SCM) $\scm^n$ is a tuple $\langle \myendogenous, \myexogenous, \myfunctional, \zeta^\myexogenous \rangle$, where \emph{(i)} $\myendogenous = \{X_1, \ldots, X_n\}$ is a set of $n$ endogenous random variables; \emph{(ii)} $\myexogenous =\{Z_1,\ldots,Z_n\}$ is a set of $n$ exogenous variables; \emph{(iii)} $\myfunctional$ is a set of $n$ functional assignments such that $X_i=f_i(\parents_i, Z_i)$, $\forall \; i \in [n]$, with $ \parents_i \subseteq \myendogenous \setminus \{ X_i\}$; \emph{(iv)} $\zeta^\myexogenous$ is a product probability measure over independent exogenous variables $\zeta^\myexogenous=\prod_{i \in [n]} \zeta^i$, where $\zeta^i=P(Z_i)$. 
\end{definition}
A Markovian SCM induces a directed acyclic graph (DAG) $\mathcal{G}_{\scm^n}$ where the nodes represent the variables $\myendogenous$ and the edges are determined by the structural functions $\myfunctional$; $ \parents_i$ constitutes then the parent set for $X_i$. Furthermore, we can recursively rewrite the set of structural function $\myfunctional$ as a set of mixing functions $\mymixing$ dependent only on the exogenous variables (cf. \cref{app:CA}). A key feature for studying causality is the possibility of defining interventions on the model:
\begin{definition}[Hard intervention, \citealp{pearl2009causality}]\label{def:intervention}
Given SCM $\scm^n = \langle \myendogenous, \myexogenous, \myfunctional, \zeta^\myexogenous \rangle$, a (hard) intervention $\iota = \operatorname{do}(\myendogenous^{\iota} = \mathbf{x}^{\iota})$, $\myendogenous^{\iota}\subseteq \myendogenous$,
is an operator that generates a new post-intervention SCM $\scm^n_\iota = \langle \myendogenous, \myexogenous, \myfunctional_\iota, \zeta^\myexogenous \rangle$ by replacing each function $f_i$ for $X_i\in\myendogenous^{\iota}$ with the constant $x_i^\iota\in \mathbf{x}^\iota$. 
Graphically, an intervention mutilates $\mathcal{G}_{\mathsf{M}^n}$ by removing all the incoming edges of the variables in $\myendogenous^{\iota}$.
\end{definition}

Given multiple SCMs describing the same system at different levels of granularity, CA provides the definition of an $\alpha$-abstraction map to relate these SCMs:
\begin{definition}[$\abst$-abstraction, \citealp{rischel2020category}]\label{def:abstraction}
Given low-level $\mathsf{M}^\ell$ and high-level $\mathsf{M}^h$ SCMs, an $\abst$-abstraction is a triple $\abst = \langle \Rset, \amap, \alphamap{} \rangle$, where \emph{(i)} $\Rset \subseteq \datalow$ is a subset of relevant variables in $\mathsf{M}^\ell$; \emph{(ii)} $\amap: \Rset \rightarrow \datahigh$ is a surjective function between the relevant variables of $\mathsf{M}^\ell$ and the endogenous variables of $\mathsf{M}^h$; \emph{(iii)} $\alphamap{}: \dom{\Rset} \rightarrow \dom{\datahigh}$ is a modular function $\alphamap{} = \bigotimes_{i\in[n]} \alphamap{X^h_i}$ made up by surjective functions $\alphamap{X^h_i}: \dom{\amap^{-1}(X^h_i)} \rightarrow \dom{X^h_i}$ from the outcome of low-level variables $\amap^{-1}(X^h_i) \in \datalow$ onto outcomes of the high-level variables $X^h_i \in \datahigh$.
\end{definition}
Notice that an $\abst$-abstraction simultaneously maps variables via the function $\amap$ and values through the function $\alphamap{}$. The definition itself does not place any constraint on these functions, although a common requirement in the literature is for the abstraction to satisfy \emph{interventional consistency} \cite{rubenstein2017causal,rischel2020category,beckers2019abstracting}. An important class of such well-behaved abstractions is \emph{constructive linear abstraction}, for which the following properties hold. By constructivity, \emph{(i)} $\abst$ is interventionally consistent; \emph{(ii)} all low-level variables are relevant $\Rset=\datalow$; \emph{(iii)} in addition to the map $\alphamap{}$ between endogenous variables, there exists a map ${\alphamap{}}_U$ between exogenous variables satisfying interventional consistency \cite{beckers2019abstracting,schooltink2024aligning}. By linearity, $\alphamap{} = \V^\top \in \reall^{h \times \ell}$ \cite{massidda2024learningcausalabstractionslinear}. \cref{app:CA} provides formal definitions for interventional consistency, linear and constructive abstraction.

% \input{sections/BDLO_model}

\section{Research Methodology}~\label{sec:Methodology}

In this section, we discuss the process of conducting our systematic review, e.g., our search strategy for data extraction of relevant studies, based on the guidelines of Kitchenham et al.~\cite{kitchenham2022segress} to conduct SLRs and Petersen et al.~\cite{PETERSEN20151} to conduct systematic mapping studies (SMSs) in Software Engineering. In this systematic review, we divide our work into a four-stage procedure, including planning, conducting, building a taxonomy, and reporting the review, illustrated in Fig.~\ref{fig:search}. The four stages are as follows: (1) the \emph{planning} stage involved identifying research questions (RQs) and specifying the detailed research plan for the study; (2) the \emph{conducting} stage involved analyzing and synthesizing the existing primary studies to answer the research questions; (3) the \emph{taxonomy} stage was introduced to optimize the data extraction results and consolidate a taxonomy schema for REDAST methodology; (4) the \emph{reporting} stage involved the reviewing, concluding and reporting the final result of our study.

\begin{figure}[!t]
    \centering
    \includegraphics[width=1\linewidth]{fig/methodology/searching-process.drawio.pdf}
    \caption{Systematic Literature Review Process}
    \label{fig:search}
\end{figure}

\subsection{Research Questions}
In this study, we developed five research questions (RQs) to identify the input and output, analyze technologies, evaluate metrics, identify challenges, and identify potential opportunities. 

\textbf{RQ1. What are the input configurations, formats, and notations used in the requirements in requirements-driven
automated software testing?} In requirements-driven testing, the input is some form of requirements specification -- which can vary significantly. RQ1 maps the input for REDAST and reports on the comparison among different formats for requirements specification.

\textbf{RQ2. What are the frameworks, tools, processing methods, and transformation techniques used in requirements-driven automated software testing studies?} RQ2 explores the technical solutions from requirements to generated artifacts, e.g., rule-based transformation applying natural language processing (NLP) pipelines and deep learning (DL) techniques, where we additionally discuss the potential intermediate representation and additional input for the transformation process.

\textbf{RQ3. What are the test formats and coverage criteria used in the requirements-driven automated software
testing process?} RQ3 focuses on identifying the formulation of generated artifacts (i.e., the final output). We map the adopted test formats and analyze their characteristics in the REDAST process.

\textbf{RQ4. How do existing studies evaluate the generated test artifacts in the requirements-driven automated software testing process?} RQ4 identifies the evaluation datasets, metrics, and case study methodologies in the selected papers. This aims to understand how researchers assess the effectiveness, accuracy, and practical applicability of the generated test artifacts.

\textbf{RQ5. What are the limitations and challenges of existing requirements-driven automated software testing methods in the current era?} RQ5 addresses the limitations and challenges of existing studies while exploring future directions in the current era of technology development. %It particularly highlights the potential benefits of advanced LLMs and examines their capacity to meet the high expectations placed on these cutting-edge language modeling technologies. %\textcolor{blue}{CA: Do we really need to focus on LLMs? TBD.} \textcolor{orange}{FW: About LLMs, I removed the direct emphase in RQ5 but kept the discussion in RQ5 and the solution section. I think that would be more appropriate.}

\subsection{Searching Strategy}

The overview of the search process is exhibited in Fig. \ref{fig:papers}, which includes all the details of our search steps.
\begin{table}[!ht]
\caption{List of Search Terms}
\label{table:search_term}
\begin{tabularx}{\textwidth}{lX}
\hline
\textbf{Terms Group} & \textbf{Terms} \\ \hline
Test Group & test* \\
Requirement Group & requirement* OR use case* OR user stor* OR specification* \\
Software Group & software* OR system* \\
Method Group & generat* OR deriv* OR map* OR creat* OR extract* OR design* OR priorit* OR construct* OR transform* \\ \hline
\end{tabularx}
\end{table}

\begin{figure}
    \centering
    \includegraphics[width=1\linewidth]{fig/methodology/search-papers.drawio.pdf}
    \caption{Study Search Process}
    \label{fig:papers}
\end{figure}

\subsubsection{Search String Formulation}
Our research questions (RQs) guided the identification of the main search terms. We designed our search string with generic keywords to avoid missing out on any related papers, where four groups of search terms are included, namely ``test group'', ``requirement group'', ``software group'', and ``method group''. In order to capture all the expressions of the search terms, we use wildcards to match the appendix of the word, e.g., ``test*'' can capture ``testing'', ``tests'' and so on. The search terms are listed in Table~\ref{table:search_term}, decided after iterative discussion and refinement among all the authors. As a result, we finally formed the search string as follows:


\hangindent=1.5em
 \textbf{ON ABSTRACT} ((``test*'') \textbf{AND} (``requirement*'' \textbf{OR} ``use case*'' \textbf{OR} ``user stor*'' \textbf{OR} ``specifications'') \textbf{AND} (``software*'' \textbf{OR} ``system*'') \textbf{AND} (``generat*'' \textbf{OR} ``deriv*'' \textbf{OR} ``map*'' \textbf{OR} ``creat*'' \textbf{OR} ``extract*'' \textbf{OR} ``design*'' \textbf{OR} ``priorit*'' \textbf{OR} ``construct*'' \textbf{OR} ``transform*''))

The search process was conducted in September 2024, and therefore, the search results reflect studies available up to that date. We conducted the search process on six online databases: IEEE Xplore, ACM Digital Library, Wiley, Scopus, Web of Science, and Science Direct. However, some databases were incompatible with our default search string in the following situations: (1) unsupported for searching within abstract, such as Scopus, and (2) limited search terms, such as ScienceDirect. Here, for (1) situation, we searched within the title, keyword, and abstract, and for (2) situation, we separately executed the search and removed the duplicate papers in the merging process. 

\subsubsection{Automated Searching and Duplicate Removal}
We used advanced search to execute our search string within our selected databases, following our designed selection criteria in Table \ref{table:selection}. The first search returned 27,333 papers. Specifically for the duplicate removal, we used a Python script to remove (1) overlapped search results among multiple databases and (2) conference or workshop papers, also found with the same title and authors in the other journals. After duplicate removal, we obtained 21,652 papers for further filtering.

\begin{table*}[]
\caption{Selection Criteria}
\label{table:selection}
\begin{tabularx}{\textwidth}{lX}
\hline
\textbf{Criterion ID} & \textbf{Criterion Description} \\ \hline
S01          & Papers written in English. \\
S02-1        & Papers in the subjects of "Computer Science" or "Software Engineering". \\
S02-2        & Papers published on software testing-related issues. \\
S03          & Papers published from 1991 to the present. \\ 
S04          & Papers with accessible full text. \\ \hline
\end{tabularx}
\end{table*}

\begin{table*}[]
\small
\caption{Inclusion and Exclusion Criteria}
\label{table:criteria}
\begin{tabularx}{\textwidth}{lX}
\hline
\textbf{ID}  & \textbf{Description} \\ \hline
\multicolumn{2}{l}{\textbf{Inclusion Criteria}} \\ \hline
I01 & Papers about requirements-driven automated system testing or acceptance testing generation, or studies that generate system-testing-related artifacts. \\
I02 & Peer-reviewed studies that have been used in academia with references from literature. \\ \hline
\multicolumn{2}{l}{\textbf{Exclusion Criteria}} \\ \hline
E01 & Studies that only support automated code generation, but not test-artifact generation. \\
E02 & Studies that do not use requirements-related information as an input. \\
E03 & Papers with fewer than 5 pages (1-4 pages). \\
E04 & Non-primary studies (secondary or tertiary studies). \\
E05 & Vision papers and grey literature (unpublished work), books (chapters), posters, discussions, opinions, keynotes, magazine articles, experience, and comparison papers. \\ \hline
\end{tabularx}
\end{table*}

\subsubsection{Filtering Process}

In this step, we filtered a total of 21,652 papers using the inclusion and exclusion criteria outlined in Table \ref{table:criteria}. This process was primarily carried out by the first and second authors. Our criteria are structured at different levels, facilitating a multi-step filtering process. This approach involves applying various criteria in three distinct phases. We employed a cross-verification method involving (1) the first and second authors and (2) the other authors. Initially, the filtering was conducted separately by the first and second authors. After cross-verifying their results, the results were then reviewed and discussed further by the other authors for final decision-making. We widely adopted this verification strategy within the filtering stages. During the filtering process, we managed our paper list using a BibTeX file and categorized the papers with color-coding through BibTeX management software\footnote{\url{https://bibdesk.sourceforge.io/}}, i.e., “red” for irrelevant papers, “yellow” for potentially relevant papers, and “blue” for relevant papers. This color-coding system facilitated the organization and review of papers according to their relevance.

The screening process is shown below,
\begin{itemize}
    \item \textbf{1st-round Filtering} was based on the title and abstract, using the criteria I01 and E01. At this stage, the number of papers was reduced from 21,652 to 9,071.
    \item \textbf{2nd-round Filtering}. We attempted to include requirements-related papers based on E02 on the title and abstract level, which resulted from 9,071 to 4,071 papers. We excluded all the papers that did not focus on requirements-related information as an input or only mentioned the term ``requirements'' but did not refer to the requirements specification.
    \item \textbf{3rd-round Filtering}. We selectively reviewed the content of papers identified as potentially relevant to requirements-driven automated test generation. This process resulted in 162 papers for further analysis.
\end{itemize}
Note that, especially for third-round filtering, we aimed to include as many relevant papers as possible, even borderline cases, according to our criteria. The results were then discussed iteratively among all the authors to reach a consensus.

\subsubsection{Snowballing}

Snowballing is necessary for identifying papers that may have been missed during the automated search. Following the guidelines by Wohlin~\cite{wohlin2014guidelines}, we conducted both forward and backward snowballing. As a result, we identified 24 additional papers through this process.

\subsubsection{Data Extraction}

Based on the formulated research questions (RQs), we designed 38 data extraction questions\footnote{\url{https://drive.google.com/file/d/1yjy-59Juu9L3WHaOPu-XQo-j-HHGTbx_/view?usp=sharing}} and created a Google Form to collect the required information from the relevant papers. The questions included 30 short-answer questions, six checkbox questions, and two selection questions. The data extraction was organized into five sections: (1) basic information: fundamental details such as title, author, venue, etc.; (2) open information: insights on motivation, limitations, challenges, etc.; (3) requirements: requirements format, notation, and related aspects; (4) methodology: details, including immediate representation and technique support; (5) test-related information: test format(s), coverage, and related elements. Similar to the filtering process, the first and second authors conducted the data extraction and then forwarded the results to the other authors to initiate the review meeting.

\subsubsection{Quality Assessment}

During the data extraction process, we encountered papers with insufficient information. To address this, we conducted a quality assessment in parallel to ensure the relevance of the papers to our objectives. This approach, also adopted in previous secondary studies~\cite{shamsujjoha2021developing, naveed2024model}, involved designing a set of assessment questions based on guidelines by Kitchenham et al.~\cite{kitchenham2022segress}. The quality assessment questions in our study are shown below:
\begin{itemize}
    \item \textbf{QA1}. Does this study clearly state \emph{how} requirements drive automated test generation?
    \item \textbf{QA2}. Does this study clearly state the \emph{aim} of REDAST?
    \item \textbf{QA3}. Does this study enable \emph{automation} in test generation?
    \item \textbf{QA4}. Does this study demonstrate the usability of the method from the perspective of methodology explanation, discussion, case examples, and experiments?
\end{itemize}
QA4 originates from an open perspective in the review process, where we focused on evaluation, discussion, and explanation. Our review also examined the study’s overall structure, including the methodology description, case studies, experiments, and analyses. The detailed results of the quality assessment are provided in the Appendix. Following this assessment, the final data extraction was based on 156 papers.

% \begin{table}[]
% \begin{tabular}{ll}
% \hline
% QA ID & QA Questions                                             \\ \hline
% Q01   & Does this study clearly state its aims?                  \\
% Q02   & Does this study clearly describe its methodology?        \\
% Q03   & Does this study involve automated test generation?       \\
% Q04   & Does this study include a promising evaluation?          \\
% Q05   & Does this study demonstrate the usability of the method? \\ \hline
% \end{tabular}%
% \caption{Questions for Quality Assessment}
% \label{table:qa}
% \end{table}

% automated quality assessment

% \textcolor{blue}{CA: Our search strategy focused on identifying requirements types first. We covered several sources, e.g., ~\cite{Pohl:11,wagner2019status} to identify different formats and notations of specifying requirements. However, this came out to be a long list, e.g., free-form NL requirements, semi-formal UML models, free-from textual use case models, UML class diagrams, UML activity diagrams, and so on. In this paper, we attempted to primarily focus on requirements-related aspects and not design-level information. Hence, we generalised our search string to include generic keywords, e.g., requirement*, use case*, and user stor*. We did so to avoid missing out on any papers, bringing too restrictive in our search strategy, and not creating a too-generic search string with all the aforementioned formats to avoid getting results beyond our review's scope.}


%% Use \subsection commands to start a subsection.



%\subsection{Study Selection}

% In this step, we further looked into the content of searched papers using our search strategy and applied our inclusion and exclusion criteria. Our filtering strategy aimed to pinpoint studies focused on requirements-driven system-level testing. Recognizing the presence of irrelevant papers in our search results, we established detailed selection criteria for preliminary inclusion and exclusion, as shown in Table \ref{table: criteria}. Specifically, we further developed the taxonomy schema to exclude two types of studies that did not meet the requirements for system-level testing: (1) studies supporting specification-driven test generation, such as UML-driven test generation, rather than requirements-driven testing, and (2) studies focusing on code-based test generation, such as requirement-driven code generation for unit testing.





\section{Experiments}
\label{sec:exp}
Following the settings in Section \ref{sec:existing}, we evaluate \textit{NovelSum}'s correlation with the fine-tuned model performance across 53 IT datasets and compare it with previous diversity metrics. Additionally, we conduct a correlation analysis using Qwen-2.5-7B \cite{yang2024qwen2} as the backbone model, alongside previous LLaMA-3-8B experiments, to further demonstrate the metric's effectiveness across different scenarios. Qwen is used for both instruction tuning and deriving semantic embeddings. Due to resource constraints, we run each strategy on Qwen for two rounds, resulting in 25 datasets. 

\subsection{Main Results}

\begin{table*}[!t]
    \centering
    \resizebox{\linewidth}{!}{
    \begin{tabular}{lcccccccccc}
    \toprule
    \multirow{3}*{\textbf{Diversity Metrics}} & \multicolumn{10}{c}{\textbf{Data Selection Strategies}} \\
    \cmidrule(lr){2-11}
    & \multirow{2}*{\textbf{K-means}} & \multirow{2}*{\vtop{\hbox{\textbf{K-Center}}\vspace{1mm}\hbox{\textbf{-Greedy}}}}  & \multirow{2}*{\textbf{QDIT}} & \multirow{2}*{\vtop{\hbox{\textbf{Repr}}\vspace{1mm}\hbox{\textbf{Filter}}}} & \multicolumn{5}{c}{\textbf{Random}} & \multirow{2}{*}{\textbf{Duplicate}} \\ 
    \cmidrule(lr){6-10}
    & & & & & \textbf{$\mathcal{X}^{all}$} & ShareGPT & WizardLM & Alpaca & Dolly &  \\
    \midrule
    \rowcolor{gray!15} \multicolumn{11}{c}{\textit{LLaMA-3-8B}} \\
    Facility Loc. $_{\times10^5}$ & \cellcolor{BLUE!40} 2.99 & \cellcolor{ORANGE!10} 2.73 & \cellcolor{BLUE!40} 2.99 & \cellcolor{BLUE!20} 2.86 & \cellcolor{BLUE!40} 2.99 & \cellcolor{BLUE!0} 2.83 & \cellcolor{BLUE!30} 2.88 & \cellcolor{BLUE!0} 2.83 & \cellcolor{ORANGE!20} 2.59 & \cellcolor{ORANGE!30} 2.52 \\    
    DistSum$_{cosine}$  & \cellcolor{BLUE!30} 0.648 & \cellcolor{BLUE!60} 0.746 & \cellcolor{BLUE!0} 0.629 & \cellcolor{BLUE!50} 0.703 & \cellcolor{BLUE!10} 0.634 & \cellcolor{BLUE!40} 0.656 & \cellcolor{ORANGE!30} 0.578 & \cellcolor{ORANGE!10} 0.605 & \cellcolor{ORANGE!20} 0.603 & \cellcolor{BLUE!10} 0.634 \\
    Vendi Score $_{\times10^7}$ & \cellcolor{BLUE!30} 1.70 & \cellcolor{BLUE!60} 2.53 & \cellcolor{BLUE!10} 1.59 & \cellcolor{BLUE!50} 2.23 & \cellcolor{BLUE!20} 1.61 & \cellcolor{BLUE!30} 1.70 & \cellcolor{ORANGE!10} 1.44 & \cellcolor{ORANGE!20} 1.32 & \cellcolor{ORANGE!10} 1.44 & \cellcolor{ORANGE!30} 0.05 \\
    \textbf{NovelSum (Ours)} & \cellcolor{BLUE!60} 0.693 & \cellcolor{BLUE!50} 0.687 & \cellcolor{BLUE!30} 0.673 & \cellcolor{BLUE!20} 0.671 & \cellcolor{BLUE!40} 0.675 & \cellcolor{BLUE!10} 0.628 & \cellcolor{BLUE!0} 0.591 & \cellcolor{ORANGE!10} 0.572 & \cellcolor{ORANGE!20} 0.50 & \cellcolor{ORANGE!30} 0.461 \\
    \midrule    
    \textbf{Model Performance} & \cellcolor{BLUE!60}1.32 & \cellcolor{BLUE!50}1.31 & \cellcolor{BLUE!40}1.25 & \cellcolor{BLUE!30}1.05 & \cellcolor{BLUE!20}1.20 & \cellcolor{BLUE!10}0.83 & \cellcolor{BLUE!0}0.72 & \cellcolor{ORANGE!10}0.07 & \cellcolor{ORANGE!20}-0.14 & \cellcolor{ORANGE!30}-1.35 \\
    \midrule
    \midrule
    \rowcolor{gray!15} \multicolumn{11}{c}{\textit{Qwen-2.5-7B}} \\
    Facility Loc. $_{\times10^5}$ & \cellcolor{BLUE!40} 3.54 & \cellcolor{ORANGE!30} 3.42 & \cellcolor{BLUE!40} 3.54 & \cellcolor{ORANGE!20} 3.46 & \cellcolor{BLUE!40} 3.54 & \cellcolor{BLUE!30} 3.51 & \cellcolor{BLUE!10} 3.50 & \cellcolor{BLUE!10} 3.50 & \cellcolor{ORANGE!20} 3.46 & \cellcolor{BLUE!0} 3.48 \\ 
    DistSum$_{cosine}$ & \cellcolor{BLUE!30} 0.260 & \cellcolor{BLUE!60} 0.440 & \cellcolor{BLUE!0} 0.223 & \cellcolor{BLUE!50} 0.421 & \cellcolor{BLUE!10} 0.230 & \cellcolor{BLUE!40} 0.285 & \cellcolor{ORANGE!20} 0.211 & \cellcolor{ORANGE!30} 0.189 & \cellcolor{ORANGE!10} 0.221 & \cellcolor{BLUE!20} 0.243 \\
    Vendi Score $_{\times10^6}$ & \cellcolor{ORANGE!10} 1.60 & \cellcolor{BLUE!40} 3.09 & \cellcolor{BLUE!10} 2.60 & \cellcolor{BLUE!60} 7.15 & \cellcolor{ORANGE!20} 1.41 & \cellcolor{BLUE!50} 3.36 & \cellcolor{BLUE!20} 2.65 & \cellcolor{BLUE!0} 1.89 & \cellcolor{BLUE!30} 3.04 & \cellcolor{ORANGE!30} 0.20 \\
    \textbf{NovelSum (Ours)}  & \cellcolor{BLUE!40} 0.440 & \cellcolor{BLUE!60} 0.505 & \cellcolor{BLUE!20} 0.403 & \cellcolor{BLUE!50} 0.495 & \cellcolor{BLUE!30} 0.408 & \cellcolor{BLUE!10} 0.392 & \cellcolor{BLUE!0} 0.349 & \cellcolor{ORANGE!10} 0.336 & \cellcolor{ORANGE!20} 0.320 & \cellcolor{ORANGE!30} 0.309 \\
    \midrule
    \textbf{Model Performance} & \cellcolor{BLUE!30} 1.06 & \cellcolor{BLUE!60} 1.45 & \cellcolor{BLUE!40} 1.23 & \cellcolor{BLUE!50} 1.35 & \cellcolor{BLUE!20} 0.87 & \cellcolor{BLUE!10} 0.07 & \cellcolor{BLUE!0} -0.08 & \cellcolor{ORANGE!10} -0.38 & \cellcolor{ORANGE!30} -0.49 & \cellcolor{ORANGE!20} -0.43 \\
    \bottomrule
    \end{tabular}
    }
    \caption{Measuring the diversity of datasets selected by different strategies using \textit{NovelSum} and baseline metrics. Fine-tuned model performances (Eq. \ref{eq:perf}), based on MT-bench and AlpacaEval, are also included for cross reference. Darker \colorbox{BLUE!60}{blue} shades indicate higher values for each metric, while darker \colorbox{ORANGE!30}{orange} shades indicate lower values. While data selection strategies vary in performance on LLaMA-3-8B and Qwen-2.5-7B, \textit{NovelSum} consistently shows a stronger correlation with model performance than other metrics. More results are provided in Appendix \ref{app:results}.}
    \label{tbl:main}
    \vspace{-4mm}
\end{table*}


\begin{table}[t!]
\centering
\resizebox{\linewidth}{!}{
\begin{tabular}{lcccc}
\toprule
\multirow{2}*{\textbf{Diversity Metrics}} & \multicolumn{3}{c}{\textbf{LLaMA}} & \textbf{Qwen}\\
\cmidrule(lr){2-4} \cmidrule(lr){5-5} 
& \textbf{Pearson} & \textbf{Spearman} & \textbf{Avg.} & \textbf{Avg.} \\
\midrule
TTR & -0.38 & -0.16 & -0.27 & -0.30 \\
vocd-D & -0.43 & -0.17 & -0.30 & -0.31 \\
\midrule
Facility Loc. & 0.86 & 0.69 & 0.77 & 0.08 \\
Entropy & 0.93 & 0.80 & 0.86 & 0.63 \\
\midrule
LDD & 0.61 & 0.75 & 0.68 & 0.60 \\
KNN Distance & 0.59 & 0.80 & 0.70 & 0.67 \\
DistSum$_{cosine}$ & 0.85 & 0.67 & 0.76 & 0.51 \\
Vendi Score & 0.70 & 0.85 & 0.78 & 0.60 \\
DistSum$_{L2}$ & 0.86 & 0.76 & 0.81 & 0.51 \\
Cluster Inertia & 0.81 & 0.85 & 0.83 & 0.76 \\
Radius & 0.87 & 0.81 & 0.84 & 0.48 \\
\midrule
NovelSum & \textbf{0.98} & \textbf{0.95} & \textbf{0.97} & \textbf{0.90} \\
\bottomrule
\end{tabular}
}
\caption{Correlations between different metrics and model performance on LLaMA-3-8B and Qwen-2.5-7B.  “Avg.” denotes the average correlation (Eq. \ref{eq:cor}).}
\label{tbl:correlations}
\vspace{-2mm}
\end{table}

\paragraph{\textit{NovelSum} consistently achieves state-of-the-art correlation with model performance across various data selection strategies, backbone LLMs, and correlation measures.}
Table \ref{tbl:main} presents diversity measurement results on datasets constructed by mainstream data selection methods (based on $\mathcal{X}^{all}$), random selection from various sources, and duplicated samples (with only $m=100$ unique samples). 
Results from multiple runs are averaged for each strategy.
Although these strategies yield varying performance rankings across base models, \textit{NovelSum} consistently tracks changes in IT performance by accurately measuring dataset diversity. For instance, K-means achieves the best performance on LLaMA with the highest NovelSum score, while K-Center-Greedy excels on Qwen, also correlating with the highest NovelSum. Table \ref{tbl:correlations} shows the correlation coefficients between various metrics and model performance for both LLaMA and Qwen experiments, where \textit{NovelSum} achieves state-of-the-art correlation across different models and measures.

\paragraph{\textit{NovelSum} can provide valuable guidance for data engineering practices.}
As a reliable indicator of data diversity, \textit{NovelSum} can assess diversity at both the dataset and sample levels, directly guiding data selection and construction decisions. For example, Table \ref{tbl:main} shows that the combined data source $\mathcal{X}^{all}$ is a better choice for sampling diverse IT data than other sources. Moreover, \textit{NovelSum} can offer insights through comparative analyses, such as: (1) ShareGPT, which collects data from real internet users, exhibits greater diversity than Dolly, which relies on company employees, suggesting that IT samples from diverse sources enhance dataset diversity \cite{wang2024diversity-logD}; (2) In LLaMA experiments, random selection can outperform some mainstream strategies, aligning with prior work \cite{xia2024rethinking,diddee2024chasing}, highlighting gaps in current data selection methods for optimizing diversity.



\subsection{Ablation Study}


\textit{NovelSum} involves several flexible hyperparameters and variations. In our main experiments, \textit{NovelSum} uses cosine distance to compute $d(x_i, x_j)$ in Eq. \ref{eq:dad}. We set $\alpha = 1$, $\beta = 0.5$, and $K = 10$ nearest neighbors in Eq. \ref{eq:pws} and \ref{eq:dad}. Here, we conduct an ablation study to investigate the impact of these settings based on LLaMA-3-8B.

\begin{table}[ht!]
\centering
\resizebox{\linewidth}{!}{
\begin{tabular}{lccc}
\toprule
\textbf{Variants} & \textbf{Pearson} & \textbf{Spearman} & \textbf{Avg.} \\
\midrule
NovelSum & 0.98 & 0.96 & 0.97 \\
\midrule
\hspace{0.10cm} - Use $L2$ distance & 0.97 & 0.83 & 0.90\textsubscript{↓ 0.08} \\
\hspace{0.10cm} - $K=20$ & 0.98 & 0.96 & 0.97\textsubscript{↓ 0.00} \\
\hspace{0.10cm} - $\alpha=0$ (w/o proximity) & 0.79 & 0.31 & 0.55\textsubscript{↓ 0.42} \\
\hspace{0.10cm} - $\alpha=2$ & 0.73 & 0.88 & 0.81\textsubscript{↓ 0.16} \\
\hspace{0.10cm} - $\beta=0$ (w/o density) & 0.92 & 0.89 & 0.91\textsubscript{↓ 0.07} \\
\hspace{0.10cm} - $\beta=1$ & 0.90 & 0.62 & 0.76\textsubscript{↓ 0.21} \\
\bottomrule
\end{tabular}
}
\caption{Ablation Study for \textit{NovelSum}.}
\label{tbl:ablation}
\vspace{-2mm}
\end{table}

In Table \ref{tbl:ablation}, $\alpha=0$ removes the proximity weights, and $\beta=0$ eliminates the density multiplier. We observe that both $\alpha=0$ and $\beta=0$ significantly weaken the correlation, validating the benefits of the proximity-weighted sum and density-aware distance. Additionally, improper values for $\alpha$ and $\beta$ greatly reduce the metric's reliability, highlighting that \textit{NovelSum} strikes a delicate balance between distances and distribution. Replacing cosine distance with Euclidean distance and using more neighbors for density approximation have minimal impact, particularly on Pearson's correlation, demonstrating \textit{NovelSum}'s robustness to different distance measures.







\section{Limitation}
The use of 3D-printed PLA for structural components improves improving ease of assembly and reduces weight and cost, yet it causes deformation under heavy load, which can diminish end-effector precision. Using metal, such as aluminum, would remedy this problem. Additionally, \robot relies on integrated joint relative encoders, requiring manual initialization in a fixed joint configuration each time the system is powered on. Using absolute joint encoders could significantly improve accuracy and ease of use, although it would increase the overall cost. 

%Reliance on commercially available actuators simplifies integration but imposes constraints on control frequency and customization, further limiting the potential for tailored performance improvements.

% The 6 DoF configuration provides sufficient mobility for most tasks; however, certain bimanual operations could benefit from an additional degree of freedom to handle complex joint constraints more effectively. Furthermore, the limited torque density of commercially available proprioceptive actuators restricts the payload and torque output, making the system less suitability for handling heavier loads or high-torque applications. 

The 6 DoF configuration of the arm provides sufficient mobility for single-arm manipulation tasks, yet it shows a limitation in certain bimanual manipulation problems. Specifically, when \robot holds onto a rigid object with both hands, each arm loses 1 DoF because the hands are fixed to the object during grasping. This leads to an underactuated kinematic chain which has a limited mobility in 3D space. We can achieve more mobility by letting the object slip inside the grippers, yet this renders the grasp less robust and simulation difficult. Therefore, we anticipate that designing a lightweight 3 DoF wrist in place of the current 2 DoF wrist allows a more diverse repertoire of manipulation in bimanual tasks.

Finally, the limited torque density of commercially available proprioceptive actuators restricts the performance. Currently, all of our actuators feature a 1:10 gear ratio, so \robot can handle up to 2.5 kg of payload. To handle a heavier object and manipulate it with higher torque, we expect the actuator to have 1:20$\sim$30 gear ratio, but it is difficult to find an off-the-shelf product that meets our requirements. Customizing the actuator to increase the torque density while minimizing the weight will enable \robot to move faster and handle more diverse objects.

%These constraints highlight opportunities for improvement in future iterations, including alternative materials for enhanced rigidity, custom actuator designs for higher control precision and torque density, the adoption of absolute joint encoders, and optimized configurations to balance dexterity and weight.



% \section{Section}

% Section text here. 

% \subsection{Subsection Heading Here}
% Subsection text here.

% \subsubsection{Subsubsection Heading Here}
% Subsubsection text here.


% \section{RSS citations}

% Please make sure to include \verb!natbib.sty! and to use the
% \verb!plainnat.bst! bibliography style. \verb!natbib! provides additional
% citation commands, most usefully \verb!\citet!. For example, rather than the
% awkward construction 

% {\small
% \begin{verbatim}
% \cite{kalman1960new} demonstrated...
% \end{verbatim}
% }

% \noindent
% rendered as ``\cite{kalman1960new} demonstrated...,''
% or the
% inconvenient 

% {\small
% \begin{verbatim}
% Kalman \cite{kalman1960new} 
% demonstrated...
% \end{verbatim}
% }

% \noindent
% rendered as 
% ``Kalman \cite{kalman1960new} demonstrated...'', 
% one can
% write 

% {\small
% \begin{verbatim}
% \citet{kalman1960new} demonstrated... 
% \end{verbatim}
% }
% \noindent
% which renders as ``\citet{kalman1960new} demonstrated...'' and is 
% both easy to write and much easier to read.
  
% \subsection{RSS Hyperlinks}

% This year, we would like to use the ability of PDF viewers to interpret
% hyperlinks, specifically to allow each reference in the bibliography to be a
% link to an online version of the reference. 
% As an example, if you were to cite ``Passive Dynamic Walking''
% \cite{McGeer01041990}, the entry in the bibtex would read:

% {\small
% \begin{verbatim}
% @article{McGeer01041990,
%   author = {McGeer, Tad}, 
%   title = {\href{http://ijr.sagepub.com/content/9/2/62.abstract}{Passive Dynamic Walking}}, 
%   volume = {9}, 
%   number = {2}, 
%   pages = {62-82}, 
%   year = {1990}, 
%   doi = {10.1177/027836499000900206}, 
%   URL = {http://ijr.sagepub.com/content/9/2/62.abstract}, 
%   eprint = {http://ijr.sagepub.com/content/9/2/62.full.pdf+html}, 
%   journal = {The International Journal of Robotics Research}
% }
% \end{verbatim}
% }
% \noindent
% and the entry in the compiled PDF would look like:

% \def\tmplabel#1{[#1]}

% \begin{enumerate}
% \item[\tmplabel{1}] Tad McGeer. \href{http://ijr.sagepub.com/content/9/2/62.abstract}{Passive Dynamic
% Walking}. {\em The International Journal of Robotics Research}, 9(2):62--82,
% 1990.
% \end{enumerate}
% %
% where the title of the article is a link that takes you to the article on IJRR's website. 


% Linking cited articles will not always be possible, especially for
% older articles. There are also often several versions of papers
% online: authors are free to decide what to use as the link destination
% yet we strongly encourage to link to archival or publisher sites
% (such as IEEE Xplore or Sage Journals).  We encourage all authors to use this feature to
% the extent possible.

\section{Conclusion} 
\label{sec:conclusion}
This paper introduces DEFT, a method that embeds residual learning within a novel differentiable branched Deformable Linear Object (BDLO) simulator. 
DEFT enables accurate modeling and real-time prediction of a BDLO’s dynamic behavior over long time horizons under dynamic manipulation. 
To demonstrate DEFT’s efficacy, we conduct a comprehensive set of experiments evaluating its accuracy, computational speed, impacts of its constribution and utility for manipulation tasks.
Compared to the state of the art, DEFT achieves higher accuracy while remaining sample-efficient and without compromising computational speed. 
This paper also illustrates how to integrate DEFT with a planning and control framework for 3D shape matching and thread insertion of a BDLO in real-world scenarios.
DEFT successfully completes more complex, real-world manipulation tasks than other tested baselines.
\section*{Acknowledgement}
The authors would like to gratefully thank the support by Ford Motor Company.
% We look forward to addressing the limitations discussed in the paper’s Limitations section.

% \section*{Acknowledgments}

%% Use plainnat to work nicely with natbib. 

% \bibliographystyle{plainnat}
\bibliography{references}


\clearpage
\appendix
\subsection{Proof of Theorem \ref{thm:potential_energy_gradient}}
\label{appendix: Theorem 1 Proof}
To prove the theorem, we begin by introducing the curvature binormal and material curvature.

\textbf{Material Curvatures:}
The curvature binormal $\bcurvature$ is traditionally used to represent the turning angle and axis between two consecutive edges:
\begin{equation}
    \kappa \mathbf{b}^i = \frac{2 \, \mathbf{e}^{i-1} \times \mathbf{e}^{i}}{\|\mathbf{e}^{i-1}\|_2 \|\mathbf{e}^{i}\|_2 + \mathbf{e}^{i-1} \cdot \mathbf{e}^{i}},
   \label{eq:curvature_appendix}
\end{equation}
where \(\mathbf{e}^{i-1}\) and \(\mathbf{e}^i\) are consecutive edge vectors, \(\times\) denotes the cross product, and \(\cdot\) denotes the dot product. 
By incorporating the material frame \eqref{eq:m1} and \eqref{eq:m2}, the curvature binormal \eqref{eq:curvature_appendix} is projected onto the material frame to quantify the extent to which the curvature aligns with the frame's orientation. 
This projection, which we call the material curvature, provides additional insights into the deformation characteristics, allowing us to distinguish between bending and twisting behaviors within the material frame and is defined as:
\begin{equation}
    \bm{\omega}^{(i,j)} = 
    \left(
        \bcurvature \cdot \mathbf{m}_1^j 
        , \bcurvature \cdot \mathbf{m}_2^j
    \right)^T
    \quad \text{for} \quad j \in \{i-1, i\}.
   \label{eq:materialcurvature}
\end{equation}

\textbf{Potential Energy:}
The potential energy is composed of the bending energy and twisting energy:
\begin{equation}
  P(\MaterialFrame(\mathbf{X}_t, \bm{\theta}_t), \materialp\bigr) = P_\text{bend}(\MaterialFrame(\mathbf{X}_t, \bm{\theta}_t), \materialp\bigr) + P_\text{twist}(\MaterialFrame(\mathbf{X}_t, \bm{\theta}_t), \materialp\bigr),
       \label{eq:potentialP_appendix}
\end{equation}
where
\begin{align}
    \begin{split}
       & P_\text{bend}(\MaterialFrame(\mathbf{X}_t, \bm{\theta}_t), \materialp\bigr) =  \\
    & \sum_{i=1}^{n-1} \sum_{j=i-1}^{i} \frac{1}{2}\left( \bm{\omega}^{(i,j)} - \overline{\bm{\omega}}^{(i,j)} \right)^T \mathbf{B}^j \left( \bm{\omega}^{(i,j)} - \overline{\bm{\omega}}^{(i,j)} \right),
    \end{split}
           \label{eq:potentialPbend}
\end{align}
and
\begin{equation}
    P_\text{twist}(\MaterialFrame(\mathbf{X}_t, \bm{\theta}_t), \materialp\bigr) = \sum_{i=1}^{n-1} \frac{1}{2} \beta^i \left(\theta^i - \theta^{i-1} \right)^2,
   \label{eq:potentialPtwist}
\end{equation}
where \(\overline{\bm{\omega}}^{(i,j)}\) denotes the undeformed material curvature, which is calculated when the DLO is in a static state without any external or internal forces applied, and $\mathbf{B}$ and $\beta$, are components of $\materialp$, representing the bending stiffness and twisting stiffness, respectively.

\textbf{Gradient of Potential Energy}: With \eqref{eq:potentialPbend} and \eqref{eq:potentialPtwist}, the gradient of \eqref{eq:potentialP_appendix} can be derived as following:
\begin{equation}
      \begin{split}
    \label{eq:potentialderivative}
    \frac{\partial P(\MaterialFrame(\mathbf{X}_t, \bm{\theta}_t), \materialp\bigr)}{\partial \theta^i}  = 
    \frac{\partial P_{bend}(\MaterialFrame(\mathbf{X}_t, \bm{\theta}_t), \materialp\bigr)}{\partial \theta^i}  + \\  \frac{\partial P_{twist}(\MaterialFrame(\mathbf{X}_t, \bm{\theta}_t), \materialp\bigr)}{\partial \theta^i} 
        \end{split}
\end{equation}
We begin by deriving the first term. 
Note that $\theta^i$ is only relevant in $\bm{\omega}^{(i,i)}$ and $\bm{\omega}^{(i+1,i)}$. 
Using the chain rule, we obtain:
\begin{equation}
    \frac{\partial P_{bend}(\MaterialFrame(\mathbf{X}_t, \bm{\theta}_t), \materialp\bigr)}{\partial \theta^i}  = 
    \sum_{k=i}^{i+1} \frac{\partial P_{bend}(\MaterialFrame(\mathbf{X}_t, \bm{\theta}_t), \materialp\bigr)}{\partial \bm{\omega}^{(k,i)}} \frac{\partial \bm{\omega}^{(k,i)}}{\partial \theta^i}.
\end{equation}
To compute $\frac{\partial \bm{\omega}^{(k,i)}}{\partial \theta^i}$, we use the identities $\frac{\partial \mathbf{m}_1^i}{\partial \theta^i} = \mathbf{m}_2^i$ and $\frac{\partial \mathbf{m}_2^i}{\partial \theta^i} = -\mathbf{m}_1^i$.
These lead to:
\begin{equation}
    \frac{\partial \bm{\omega}^{(k, i)}}{\partial \theta^i} = \begin{bmatrix}
0 & 1 \\
-1 & 0
\end{bmatrix} \bm{\omega}^{(k, i)}
\end{equation}
Substituting this result back, we obtain:
\begin{equation}
    \begin{split}
  \frac{\partial P_{\mathrm{bend}}(\MaterialFrame(\mathbf{X}_t, \bm{\theta}_t), \materialp\bigr)}{\partial \theta^i}
  \;=\; 
  \sum_{k=i}^{\,i+1}
  \bigl(\mathbf{B}^k\, (&\bm{\omega}^{(k,i)}  - \bar{\bm{\omega}}^{(k,i)})\bigr)^{T} \cdot
  \\
    &\begin{bmatrix}0 & 1\\[6pt]-1 & 0\end{bmatrix}
  \bm{\omega}^{(k,i)},
      \end{split}
      \label{eq:potentialbendgradient}
\end{equation}
The second term of \eqref{eq:potentialderivative} can be derived, resulting in:
\begin{equation}
        \label{eq:potentialtwistgradient}
        \frac{\partial P_{twist}(\MaterialFrame(\mathbf{X}_t, \bm{\theta}_t), \materialp\bigr)}{\partial \theta^i} = \beta^i (\theta^i - \theta^{i-1}) - \beta^{i+1}(\theta^{i+1} - \theta^{i})
\end{equation}
By substituting \eqref{eq:potentialbendgradient} and \eqref{eq:potentialtwistgradient} into \eqref{eq:potentialderivative}, we obtain the analytical gradient of the potential energy.

\subsection{Definition of $\hat{\boldsymbol{\Omega}}^i_{\Delta,t} (\hat{\mathbf{X}},\Delta \hat{\mathbf{X}})$ and $\hat{\boldsymbol{\Omega}}^i$}
\label{sec:appendix_angular_momentum}
In practice, it can be both challenging and impractical to attach hardware onto each segment for tracking its orientation changes. 
To address this limitation, we approximate the change in orientation, as illustrated in Figure \ref{fig:orientation change}.
We define $\hat{\boldsymbol{\Omega}}^i_\Delta (\hat{\mathbf{X}},\Delta \hat{\mathbf{X}} )$ as follows:
\begin{equation}
    \begin{split}
    & \hat{\boldsymbol{\Omega}}^i_\Delta (\hat{\mathbf{X}},\Delta \hat{\mathbf{X}} ) = \\ 
    & \frac{\hat{\mathbf{x}}^{i+1} - \hat{\mathbf{x}}^{i}}{||\hat{\mathbf{x}}^{i+1} - \hat{\mathbf{x}}^{i}||_2}
    \times \frac{(\hat{\mathbf{x}}^{i+1} + \Delta \hat{\mathbf{x}}^{i+1}) - (\hat{\mathbf{x}}^{i}+\Delta \hat{\mathbf{x}}^i)}{||(\hat{\mathbf{x}}^{i+1} + \Delta \hat{\mathbf{x}}^{i+1}) - (\hat{\mathbf{x}}^{i}+\Delta \hat{\mathbf{x}}^i)||_2}
    \end{split}
\end{equation}
Intuitively, the cross product of each edge's tangent vector provides an approximation of the rotation axis and orientation change for the segment $i$.
This approximation becomes more accurate when $\Delta_t$ is small. 

Next, we describe how to compute $\hat{\boldsymbol{\Omega}}^i$. 
Note that it is defined recursively using the computation at the previous time step. 
As a result just within this appendix, we add a subscript $t$ to the symbol.
To compute $\hat{\boldsymbol{\Omega}}^i_{t+1}$, we first convert both 
$\hat{\boldsymbol{\Omega}}^i_{t}$ and the correction rotation 
$\hat{\boldsymbol{\Omega}}^i_{\Delta}(\hat{\mathbf{X}}_{t}, \Delta \hat{\mathbf{X}}_{t})$ 
from angle-axis to quaternions. We then update 
$\hat{\boldsymbol{\Omega}}^i_{t}$ by applying 
$\hat{\boldsymbol{\Omega}}^i_{\Delta}(\hat{\mathbf{X}}_{t}, \Delta \hat{\mathbf{X}}_{t})$ 
in quaternion form, and finally convert the result back to angle-axis coordinates 
to obtain $\hat{\boldsymbol{\Omega}}^i_{t+1}$. 
Note that we do not continuously use quaternions because angle--axis is more 
convenient for representing angular momentum.
\begin{figure}[t]
    \centering
\includegraphics[width=0.5\textwidth]{figures/rotation_omega.pdf}
    \caption{An illustration of $\hat{\boldsymbol{\Omega}}^i_\Delta (\hat{\mathbf{X}},\Delta \hat{\mathbf{X}})$
    }
    \label{fig:orientation change}
\end{figure}
\subsection{Proof of Theorem \ref{thm:attachment}}
\label{appendix:theorem_junction1}
To prove Theorem~\ref{thm:attachment}, we first formulate the optimization problem:
\begin{align}
    & &\underset{\Delta \hat{\mathbf{X}}}{\min}& \hspace{0.3cm} \frac{1}{2}\left(C^{p,i}_A(\hat{\mathbf{X}}, \Delta \hat{\mathbf{X}})\right)^2  
    \label{eq:optimization_simplified_attachement}\\
    &&\text{s.t.} &\hspace{0.3cm}  \Delta \hat{\mathbf{x}}^j = \mathbf{0}, \hspace{0.3cm} \forall j  \{(p, i),(c, 1)\} 
    \label{eq:index_selection}\\
    && & \hspace{0.3cm}   \mathbf{M}^{p, i}  \Delta \hat{\mathbf{x}}^{p, i}\ + \mathbf{M}^{c, 1}  \Delta \hat{\mathbf{x}}^{c, 1}= \mathbf{0}, 
    \label{eq:linearm_simplified} \\
    && &\hspace{0.3cm} 
    \mathbf{I}^{p, i}   \hat{\boldsymbol{\Omega}}^{p, i} _\Delta (\hat{\mathbf{X}},\Delta \hat{\mathbf{X}} ) = \mathbf{0},
    \label{eq:angularm_simplified} 
\end{align}
To solve the above optimization, we introduce Lagrange multipliers $\boldsymbol{\lambda}_\text{l} \in \mathbb{R}^{3}$ and $\boldsymbol{\lambda}_\text{r} \in \mathbb{R}^{3}$ associated with constraints \eqref{eq:linearm_simplified} and \eqref{eq:angularm_simplified}, respectively.
The corresponding Lagrangian $\mathcal{L}$  can be found as follow:
 \begin{align}
    \begin{split}
    \mathcal{L}(\Delta \hat{\textbf{x}}^{p, i}, \Delta & \hat{\mathbf{x}}^{c,1},  \boldsymbol{\lambda}_\text{l}, \boldsymbol{\lambda}_\text{r}) = \\
    &\frac{1}{2}\left(C^{p,i}_A(\hat{\mathbf{X}}, \Delta \hat{\mathbf{X}})\right)^2- \\
    & \boldsymbol{\lambda}^T_\text{l}(\Massmatrix^i  \Delta \hat{\textbf{x}}^{p, i} + \Massmatrix^{c, 1}  \Delta \hat{\mathbf{x}}^{c,1})- \\
    & \boldsymbol{\lambda}^T_\text{r}\mathbf{I}^{p, i}   \hat{\boldsymbol{\Omega}}^{p, i} _\Delta (\hat{\mathbf{X}},\Delta \hat{\mathbf{X}} )
    \label{eq:attachment_lag} 
    \end{split}
\end{align}
Next, we take the partial derivatives of \eqref{eq:attachment_lag} with respect to $\Delta \hat{\textbf{x}}^{p, i}, \Delta \hat{\mathbf{x}}^{c,1},
\boldsymbol{\lambda}_\text{l}, \boldsymbol{\lambda}_\text{r}$. 
Setting each derivative to zero yields the system of equations \eqref{eq:lagarange1}, \eqref{eq:lagarange2}, \eqref{eq:lagarangem}, and \eqref{eq:lagarangel}.
\begin{align}
    \label{eq:lagarange1}
    \begin{split}
     & \frac{\partial \mathcal{L}(\Delta \hat{\textbf{x}}^{p, i}, \Delta  \hat{\mathbf{x}}^{c,1},  \boldsymbol{\lambda}_\text{l}, \boldsymbol{\lambda}_\text{r})}{\partial \Delta  \hat{\mathbf{x}}^{p,i}} =  \\
     & C^{p,i}_A(\hat{\mathbf{X}}, \Delta \hat{\mathbf{X}})\frac{(\hat{\textbf{x}}^{p, i}+\Delta \hat{\textbf{x}}^{p, i}) - (\hat{\mathbf{x}}^{c,1}+\Delta \hat{\mathbf{x}}^{c,1})}{||(\hat{\textbf{x}}^{p, i}+\Delta \hat{\textbf{x}}^{p, i}) - (\hat{\mathbf{x}}^{c,1}+\Delta \hat{\mathbf{x}}^{c,1})||_2}-\\
     &  \boldsymbol{\lambda}^T_\text{l}\Massmatrix^{p, i} - \frac{\partial \boldsymbol{\lambda}^T_\text{r}\mathbf{I}^{p, i}   \hat{\boldsymbol{\Omega}}^{p, i} _\Delta (\hat{\mathbf{X}},\Delta \hat{\mathbf{X}} )}{\partial \Delta  \hat{\mathbf{x}}^{p,i}} = \textbf{0}
     \end{split}   
\end{align}
\begin{align}
    \label{eq:lagarange2}
    \begin{split}
     & \frac{\partial \mathcal{L}(\Delta \hat{\textbf{x}}^{p, i}, \Delta  \hat{\mathbf{x}}^{c,1},  \boldsymbol{\lambda}_\text{l}, \boldsymbol{\lambda}_\text{r})}{\partial \Delta  \hat{\mathbf{x}}^{c,1}} =  \\
     & C^{p,i}_A(\hat{\mathbf{X}}, \Delta \hat{\mathbf{X}})\frac{(\hat{\mathbf{x}}^{c,1}+\Delta \hat{\mathbf{x}}^{c,1})-(\hat{\textbf{x}}^{p, i}+\Delta \hat{\textbf{x}}^{p, i}) }{||(\hat{\textbf{x}}^{p, i}+\Delta \hat{\textbf{x}}^{p, i}) - (\hat{\mathbf{x}}^{c,1}+\Delta \hat{\mathbf{x}}^{c,1})||_2}-\\
     &  \boldsymbol{\lambda}^T_\text{l}\Massmatrix^{c, 1} - \frac{\partial \boldsymbol{\lambda}^T_\text{r}\mathbf{I}^{p, i}   \hat{\boldsymbol{\Omega}}^{p, i} _\Delta (\hat{\mathbf{X}},\Delta \hat{\mathbf{X}} )}{\partial \Delta  \hat{\mathbf{x}}^{c,1}}  = \textbf{0}
     \end{split}   
\end{align}
\begin{align}
    \label{eq:lagarangem}
      & \frac{\partial \mathcal{L}(\Delta \hat{\textbf{x}}^{p, i}, \Delta  \hat{\mathbf{x}}^{c,1},  \boldsymbol{\lambda}_\text{l}, \boldsymbol{\lambda}_\text{r})}{\partial \boldsymbol{\lambda}_\text{l}} = -(\Massmatrix^i  \Delta \hat{\textbf{x}}^{p, i} + \Massmatrix^{c, 1}  \Delta \hat{\mathbf{x}}^{c,1}) = \textbf{0}
\end{align}
\begin{align}
    \label{eq:lagarangel}
    \begin{split}
     & \frac{\partial \mathcal{L}(\Delta \hat{\textbf{x}}^{p, i}, \Delta  \hat{\mathbf{x}}^{c,1},  \boldsymbol{\lambda}_\text{l}, \boldsymbol{\lambda}_\text{r})}{\partial \boldsymbol{\lambda}_\text{r}} = 
      - \textit{\textbf{I}}^i   \boldsymbol{\lambda}^T_\text{r}\mathbf{I}^{p, i}   \hat{\boldsymbol{\Omega}}^{p, i} _\Delta (\hat{\mathbf{X}},\Delta \hat{\mathbf{X}} ) = \textbf{0}
     \end{split}
\end{align}
We first observe that solving \eqref{eq:lagarangem} leads to the following equation:
\begin{equation}
     \Delta \hat{\textbf{x}}^{c, 1} = -(\Massmatrix^{c, 1})^{-1}\Massmatrix^{p, i}  \Delta \hat{\mathbf{x}}^{p,i}
     \label{eq:subsitituion}
\end{equation}
This indicates that $\Delta \hat{\textbf{x}}^{p, i}, \Delta  \hat{\mathbf{x}}^{c,1}$ are colinear.
Based on the definition of $ \hat{\boldsymbol{\Omega}}^{i} _\Delta (\hat{\mathbf{X}},\Delta \hat{\mathbf{X}} )$ in Appendix~\ref{sec:appendix_angular_momentum}, one can see that if $\Delta \hat{\textbf{x}}^{p, i}, \Delta  \hat{\mathbf{x}}^{c,1}$ are colinear, then $ \hat{\boldsymbol{\Omega}}^{p, i} _\Delta (\hat{\mathbf{X}},\Delta \hat{\mathbf{X}} ) = \textbf{0}$. 
Note that this is true regardless of the size of $\Delta_t$.
Hence, it naturally satisfies \eqref{eq:lagarangel}.
Next, summing \eqref{eq:lagarange1} with \eqref{eq:lagarange2} yields the following:
\begin{equation} 
\boldsymbol{\lambda}_\text{l}\Massmatrix^{p, i} + \boldsymbol{\lambda}_\text{l}\Massmatrix^{c, 1} = \textbf{0}
\label{eq:lambdar0}
\end{equation}
As mass matrices are always positive definite.
This indicates $\boldsymbol{\lambda_l} = \textbf{0}$.
Based on \eqref{eq:lagarange1}, \eqref{eq:subsitituion} and \eqref{eq:lambdar0}, solving for $\Delta  \hat{\mathbf{x}}^{p,i}$ is equivalent to solving the following equation:
\begin{equation}
    C^{p,i}_A(\hat{\mathbf{X}}, \Delta \hat{\mathbf{X}}) = \textbf{0}
\end{equation}
After applying some algebraic manipulations, $\Delta \hat{\mathbf{x}}^{p, i}$ can be found to be:
\begin{equation*}
    \label{eq:momentum_solution1} 
  \Delta \hat{\mathbf{x}}^{p, i} =  \mathbf{M}^{c, 1} (\mathbf{M}^{p, i}+\mathbf{M}^{c, 1})^{-1}  C^{p,i}_A(\hat{\mathbf{X}},\mathbf{0}) \frac{(\hat{\mathbf{x}}^{c, 1} - \hat{\mathbf{x}}^{p, i})}{||\hat{\mathbf{x}}^{c, 1} - \hat{\mathbf{x}}^{p, i}||_2},
\end{equation*}
Using $\Delta \hat{\mathbf{x}}^{p, i}$ and \eqref{eq:subsitituion}, one can find:
\begin{equation*}
    \label{eq:momentum_solution2} 
     \Delta \hat{\mathbf{x}}^{c, 1} =  \mathbf{M}^{p, i}  (\mathbf{M}^{p, i}+\mathbf{M}^{c, 1})^{-1}
    \ C^{p,i}_A(\hat{\mathbf{X}},\mathbf{0}) \frac{(\hat{\mathbf{x}}^{p, i} - \hat{\mathbf{x}}^{c, 1})}{||\hat{\mathbf{x}}^{c, 1} - \hat{\mathbf{x}}^{p, i}||_2}.
\end{equation*}


\subsection{Proof of Theorem \ref{thm:junctionconstraint}}
\label{appendix:theorem_junction2}
\begin{figure}[t]
    \centering
    \includegraphics[width=0.3\textwidth]{figures/omegatox.pdf}
    \caption{%
      An illustration of applying 
      \(\hat{\boldsymbol{\Omega}}_{\Delta}^i\)
      to rotate the \(i\)-th segment. 
      % \Yizhou{to be updated, change arrow direction.}
    }
    \label{fig:orientationonx}
\end{figure}

For convenience throughout this section, we suppress the arguments to functions for convenience.
To prove Theorem~\ref{thm:junctionconstraint}, we begin by describing how 
\(\hat{\boldsymbol{\Omega}}^{i}_{\Delta}\) is applied to rotate the \(i\)-th segment. 
Note that $\Delta_t$ should be sufficiently small for $\hat{\boldsymbol{\Omega}}^i_{\Delta}$ to accurately approximate the change in orientation. 
As shown in Figure~\ref{fig:orientationonx}, the rotation axis coincides with the center of the \(i\)-th segment. 
In practice, the rotation of the \(i\)-th segment is achieved by converting
\(\hat{\boldsymbol{\Omega}}^{ i}_{\Delta}\) into the rotation matrix  \(\hat{\boldsymbol{\Omega}}^{ i}_{\Delta} \mapsto \mathbf{R}^i_\Delta\).
Consequently, we obtain \(\Delta \hat{\mathbf{x}}^i\) and \(\Delta \hat{\mathbf{x}}^{i+1}\) via
\begin{equation}
    \Delta \hat{\mathbf{x}}^i 
    \;=\; 
    \mathbf{R}^i_\Delta \,\frac{\bigl(\hat{\mathbf{x}}^{i+1}-\hat{\mathbf{x}}^{i}\bigr)}{2},
    \label{eq:rx1}
\end{equation}
\begin{equation}
    \Delta \hat{\mathbf{x}}^{i+1} 
    \;=\; 
    \mathbf{R}^i_\Delta \,\frac{\bigl(\hat{\mathbf{x}}^{i}-\hat{\mathbf{x}}^{i+1}\bigr)}{2}.
    \label{eq:rx2}
\end{equation}
Because each segment rotates only about its own center, linear momentum is naturally conserved. 
Thus, we can simplify the optimization problem and write the following optimization problem:
\begin{align}
    \label{eq:appendix_optimization_simplified_attachement}
    & 
    \underset{\Delta \hat{\mathbf{X}}}{\min} 
    && 
    \frac{1}{2}\Bigl(C^{p,i}_O(\hat{\mathbf{X}}, \Delta \hat{\mathbf{X}})\Bigr)^2 
    \\
    &\text{s.t.} 
    && 
    \mathbf{I}^{p, i} \,\hat{\boldsymbol{\Omega}}_{\Delta}^{p, i}  
    \;+\; 
    \mathbf{I}^{c, 1} \,\hat{\boldsymbol{\Omega}}_{\Delta}^{c, 1}  
    \;=\; 
    \mathbf{0}\,,
    \label{eq:appendix_constraint_rigid_body}
\end{align}
where \eqref{eq:constraints_orietation}, 
\eqref{eq:optimization_simplified_attachement},
and \eqref{eq:appendix_constraint_rigid_body} are analogous to 
\eqref{eq:constraint_attachement}, 
\eqref{eq:optimization_simplified_attachement}, 
and \eqref{eq:linearm_simplified}, respectively.
Next, we introduce a Lagrange multiplier and follow the same derivation steps as in \eqref{eq:lagarange1}, \eqref{eq:subsitituion}, and \eqref{eq:lambdar0}, which leads to:
\begin{align}
    \hat{\boldsymbol{\Omega}}_\Delta^{p,i} 
    \;=\; 
    \mathbf{I}^{c, 1}\,\bigl(\mathbf{I}^{c, 1}+\mathbf{I}^{p,i}\bigr)^{-1}\,C^{p,i}_O(\hat{\mathbf{X}}, \mathbf{0})\,
    \frac{\hat{\boldsymbol{\Omega}}^{c, 1} - \hat{\boldsymbol{\Omega}}^{p, i}}{\|\hat{\boldsymbol{\Omega}}^{c, 1} - \hat{\boldsymbol{\Omega}}^{p, i}\|_2}, 
    \notag
\end{align}
\begin{align}
    \hat{\boldsymbol{\Omega}}_\Delta^{c,1} 
    \;=\; 
    \mathbf{I}^{p, i}\,\bigl(\mathbf{I}^{c,1}+\mathbf{I}^{p,i}\bigr)^{-1}\,C^{p,i}_O(\hat{\mathbf{X}}, \mathbf{0})\,
    \frac{\hat{\boldsymbol{\Omega}}^{p, i} - \hat{\boldsymbol{\Omega}}^{c, 1}}{\|\hat{\boldsymbol{\Omega}}^{c, 1} - \hat{\boldsymbol{\Omega}}^{p, i}\|_2}.
    \notag
\end{align}
Once $\hat{\boldsymbol{\Omega}}_\Delta^{p,i}, \hat{\boldsymbol{\Omega}}_\Delta^{c,1}$ are obtained, 
\eqref{eq:rx1} and \eqref{eq:rx2} are used to update 
\(\hat{\mathbf{x}}^{p,i}\) and \(\hat{\mathbf{x}}^{c,1}\) accordingly.


\subsection{ARMOUR}
\label{armour_appendix}
% The planning tasks use ARMOUR, an optimization-based receding-horizon trajectory planner and tracking controller framework, to manipulate BDLOs. The ARMOUR framework establishes an optimization problem whose decision variables are parameters of trajactory polynomial that characterizes the robot arm's joint trajectory. The cost function is to minimize the norm distance between all vertices in the DEFT BDLO prediction and the desired BDLO configuration specified by the user.  One end of the parent branch of the BDLO is rigidly attached to the end effector of the robot arm. ARMOUR is able to use DEFT to obtain a predicted BDLO configuration based on the robot arm's end effector position at the end of each robot motion upon performing the optimized trajectories. The end effector of the robot arm can be computed using forward kinematics given a set of seven joint inputs from the controller proposed in ARMOUR. The constraints in the ARMOUR trajectory optimization problem are specified in the joint position, velocity, and arm torque limits to ensure feasibility. The solution of this optimization problem is the optimal joint trajectory for minimizing the cost function, which is the residual between the prediction and desired BDLO configuration. The process happens iteratively to command the robot arm to track the solution trajectories until the desired configuration within certain satisfaction quiteria is hold. If the process exceeds the planning iteration limit(we set it as 20), it results in a planning failure.



To address the shape matching task, we employ ARMOUR \cite{ARMOUR}, an optimization-based framework for motion planning and control. 
The objective is to guide a robot arm so that it manipulates a BDLO from an initial configuration to a specified target configuration. 
ARMOUR accomplishes this through a receding-horizon approach: at each iteration, it solves an optimization problem to determine the robot’s trajectory.
ARMOUR parameterizes the robot’s joint trajectories as polynomials, with polynomial coefficients serving as the decision variables. 
One end of the rope is rigidly attached to the robot’s end effector, whose pose is determined by forward kinematics. 
By substituting the end effector’s final pose into DEFT, we obtain a prediction of the rope’s configuration at the conclusion of the motion. 
The optimization’s cost function then minimizes the Euclidean distance between this predicted BDLO configuration from DEFT and the target configuration.
To ensure the resultant motion is physically feasible, ARMOUR incorporates constraints on the robot’s joint positions, velocities, and torques. 
Once the optimization is solved, the resulting trajectory is tracked by ARMOUR’s controller.
This planning–execution process is repeated until the rope achieves the desired configuration or a maximum iteration threshold is exceeded (in which case the task is deemed a failure).
Additional details on ARMOUR’s trajectory parameterization and its closed-loop controller can be found in \cite[Section IX]{ARMOUR}. 

\end{document}

