\section{Introduction}

% Discovering new materials with desired properties has been a challenging but valuable problem while the conventional approach to discovering materials is inherently slow, constrained by the throughput of human-centered workflow \cite{material_dicovery,wang2023scientific}.
% Generative modeling of crystals has been recognized as a promising direction for discovering new functional materials \cite{material_dicovery,wang2023scientific}.  Accurately modeling the crystal data distribution and capturing the underlying intricate rules of crystal data distribution could contribute a lot to downstream applications such as material property prediction, crystal structure prediction, and inverse design \cite{peng2022human, yao2021inverse}. 

Deep generative models, with their strong ability to approximate data distribution with complex geometries, have recently emerged as a promising approach to de novo drug design~\citep{hoogeboom2022equivariant}, protein engineering~\citep{shi2022protein}, and material science~\citep{material_dicovery}.  %In this work, we focus on the task of generative modeling of crystals which is considered a promising direction for discovering new functional materials~\cite{wang2023scientific,peng2022human}. 
% The symmetry in crystal data distribution is essentially different from the structures in other scientific domains, such as molecules and proteins.
% This is, the desired density function should hold the property of periodic translational invariant and 
%Recent advancements in machine learning have paved the way for the application of generative models to this task (Nouira et al., 2018; Hoffmann et al., 2019; Hu et al., 2020; Ren et al., 2021). Among
% various strategies, diffusion models have been exhibited to be particularly effective in generating
% realistic and diverse crystal structures (Xie et al., 2021; Jiao et al., 2023). These methods leverage a
% stochastic process to gradually transform a random initial state into a stable distribution, effectively
% capturing the complex landscapes of crystal structures.
To discover new functional materials~\citep{wang2023scientific,peng2022human}, there has been an active line of research on crystal generative modeling~\citep{REN2021,hoffmann2019data,noh2019inverse, court20203,yang2023scalable,nouira2018crystalgan}. Recent diffusion-based models learns through an iterative reverse process with multi-level noise perturbation, and has been demonstrated as a powerful tool for capturing complex geometries of crystals. Studies show that these models can generate crystal samples with realistic structures that well satisfy physical constraint~\citep{xie2021crystal,jiao2023crystal,jiao2024space,equicsp}.

%Though with superior performance over previous methods, there are still negligible challenges in diffusion-based approaches over crystal generative modeling. 
% Despite promising results, two key challenges remain.
% Firstly, diffusion models decompose data samples in a coarse-to-fine manner with multi-level noises to provide dense supervision signals for training. 
% However, crystal structure is sensitive to noise perturbation, and even a small amount of noise over the lattice bases or fractional coordinates can destroy the inherent structural information of the crystal, rendering most steps over the forward diffusion trajectory totally chaotic. % and leash the key spirit of information decomposition of the iterative refinement \jj{What does this sentence mean? Unclear}; 
% Secondly, diffusion-based approaches~\citep{jiao2023crystal,equicsp} for crystal structure modeling tend to learn the score function of wrapped normal distribution for periodic variables, where the approximation of a sum of infinite terms is needed which could bring in extra bias. 
% % diffusion-based methods~\citep{jiao2023crystal} learn a score function of wrapped normal distribution for periodic variables, which is essentially a sum of infinite terms and brings in extra bias due to the numerical approximation. %To tackle the abovementioned issues, and be 
% \yuxuan{add accuracy conditioning and emphasize contribution.}


Despite promising results, significant challenges persist. The search space for crystal structures grows exponentially with the number of atoms, while thermodynamically stable materials represent only a small fraction~\citep{flowmm}. This presents challenges in the multi-step generation process, the variance of which might cause structures to deviate from stable distributions. Moreover, the current widely-adapted diffusion-based approaches~\citep{jiao2023crystal,jiao2024space} for crystal structure modeling tend to learn the score function of wrapped normal distribution for periodic variables, where the approximation of a sum of infinite terms is needed which could bring in extra bias. Recently, BFN~\citep{bfn} has been successfully applied to the geometry generative modeling of molecules~\citep{song2023unified}, a scenario that shares the above-mentioned challenges similar to crystal generation, by modeling in a much lower variance parameter space. However, the periodic geometry of crystals differs from that of small molecules and raises significant challenges.



% To tackle these challenges, we investigate crystal distribution from a new perspective and introduce CrysBFN, a novel Bayesian Flow Network (BFN) specifically designed for crystal generative modeling. CrysBFN supports unified joint learning over different modalities of crystal data (\textit{e.g.} atom types, fractional coordinates, lattice vectors) in the same continuous parameter space, satisfying structure constraint by operating on a much lower variance space for continuous variables. BFN~\citep{bfn} has recently been applied to geometry generative modeling of molecules~\citep{song2023unified}.   %proposed the CrysBFN to bring a fresh perspective to the crystal generative modeling. 
% %\yuxuan{here we illustrate the key difference, make the reader distinguish the task from the previous molecule generative modeling.}\hanlin{Is this too detailed? We can emphasize the problems of non-Euclidean and the resultant non-additive problems}
%  %, and essentially different from previous literature~\cite {song2023unified}.
% Crystal distribution, however, holds unique symmetry properties in contrast to molecule geometry structures. Thus, incorporating the symmetry of crystal data within BFN frameworks poses new challenges. Specifically, crystal distribution is essentially an infinite repetition of unit cell in 3D space. Hence, by translating along base directions (\emph{i.e.} lattice vectors of the crystal), there are essentially infinite equivalent representations for the unit cell of the same crystal structure, a distinct property known as \textit{periodic translational invariance}~\citep{jiao2023crystal}. 


To tackle these challenges, this paper aim to break the barrier of extending the paradigm of BFN into those variables located non-Euclidean space, \textit{e.g.}, atom fractional coordinates in crystal structure. We introduce the first non-Euclidean Bayesian flow over the periodic space, \textit{i.e.} the hyper-torus. To successfully implement such concept, we introduce a generalized training paradigm based on simulation of the Bayesian flow and further propose a non-auto-regressive equivalent formulation of Bayesian flow distribution that guarantees computational efficiency. By integrating all these innovations, we introduce \modelname, the first periodic E(3) equivariant Bayesian flow network designed for crystal generation. Extensive experiments demonstrate the significant superiority of \modelname over current methods in both sampling quality and efficiency.

Our contributions can be summarized as follows:
\begin{itemize}
    \item We present the first periodic Bayesian flow in non-Euclidean space (hyper-torus) with a novel training paradigm and entropy conditioning mechanism tackling the unprecedented and pivotal non-additive accuracy theoretical challenge.
    % conditioning on entropy parameter instead of time, tackling the unprecedented and pivotal non-additive accuracy problem theoretically and practically.
    \item We introduce the first periodic-E(3) equivariant Bayesian flow networks for crystal generation tasks with appealing theoretical guarantees.
    \item Extensive experiments demonstrate that \modelname consistently outperforms previous methods on both ab initio crystal generation (99.1\% COV-P on Carbon-24) and crystal structure prediction tasks (64.35$\%$ match rate on MP-20). Efficiency experiments on MP-20 prove that \modelname enjoys a \textbf{$\sim$ 100$\times$ sampling efficiency} with performance on par with previous Diffusion-based methods.
\end{itemize}
% \yuxuan{check}
% Such innovations 
% simulate the
% introduce an entropy controlling and conditioning periodic Bayesian flow by deriving Bayesian updates in the periodic space \textit{i.e.} the hyper-torus. Due to the non-monotonic entropy and non-additive accuracy [inherently resulted from the periodicity], \modelname simulates the Bayesian flow to provide training signal while we further introduce a non-auto-regressive equivalent formulation of Bayesian flow distribution that guarantees computational efficiency. Finally, we build the first periodic-E(3) equivariant Bayesian flow networks for crystal generation via carefully designing the generation path states. Comprehensive experiments shows the significant superiority of \modelname over present methods in terms of sampling quality and efficiency.


% essentially the translational invariance over torus \jj{Need to explain torus?}. 
%The periodic translational invariance is essentially different from the translational invariance which has been implicitly imposed in fields such as coordinates generation of molecules~\cite{hoogeboom2022equivariant,song2023unified} where the generative model needs to be constrained on some subspace, \emph{e.g.} Zero of Mass space, as no valid density function satisfying translational invariance over Euclidean manifold exits; \jj{This seems unnecessary detail?} 
%To represent the atom arrangement in the unit cell of the crystal (\emph{i.e.} fractional coordinates), the variable is defined in the manifold of torus. There exists valid density functions that could satisfy the translational invariance over the torus and the invariance needs to be explicitly modeled in the generative model. 
%\jj{This paragraph on motivation is unclear. What's the key challenge? Periodic translational invariance?}

%In this work, we proposed CrysBFN which conducts generative modeling over the atom types, fractional coordinates, and lattice vectors jointly for the crystal data. And we highlight the key innovations of our method are summarized as follows:
%\begin{itemize}
%    \item 

% To accommodate such periodic invariance, CrysBFN proposes the first known formulation of Bayesian Flow Networks for modeling circular variables over a non-Euclidean hyper-torus manifold based on von Mises distribution. Compared to previous Wrapped Normal distribution~\citep{jiao2023crystal,jiao2024space} requiring approximation over the sum of infinite terms, CrysBFN allows an exact analytical form for density estimation, which eliminates bias in numerical estimation.

%   Unlike the original BFN where accuracy is always additive and monotonic, \modelname simulates the Bayesian flow to provide training signal. To make the simulated-based training procedure affordable, we further introduce a non-auto-regressive equivalent formulation of Bayesian flow distribution that guarantees sampling efficiency.
 %   \item 
 % CrysBFN realizes unified modeling of different modalities in the same continuous parameter space, satisfying structure constraint by operating on a much lower variance space for continuous variables.   
 %   \item 
 % Extensive experiments demonstrate that \modelname consistently outperforms existing methods on both ab initio crystal generation (99.1\% COV-P on Carbon-24) and crystal structure prediction tasks (64.35$\%$ match rate on MP-20).

 % \jj{Edited again}
    % operating in a lower variance parameter space to alleviate the structure constraint. 
    % for modeling the density 
    % With the Von-Mises distribution which 
%\end{itemize}
% \yuxuan{@hanlin please Add empirical experiments with 1 or 2 favorable sentence, highlight your contribution.}
% \yuxuan{@hanlin, please make sure, when elaborating on you empirical improvements, must make sure it is nonmarginal from some perspective.}

% The key advantage of the 

% By incorporating Bayesian inference to modify the parameters of a collection of independent distributions brings a



% In this work, we propose Geometric Bayesian Flow Networks (GeoBFN) to model 3D molecule
% geometry in a principally different way. Bayesian Flow Networks Graves et al. (2023) (BFN) is
% a novel generative model developed quite recently. Taking a unique approach by incorporating
% Bayesian inference to modify the parameters of a collection of independent distributions, brings a
% fresh perspective to geometric generative modeling. Firstly, GeoBFN uses a unified probabilistic
% modeling formulation for different modalities in the molecule geometry. Secondly, regarding the
% variable of 3D atom coordinates, the input variance for BFNs is considerably lower than DMs,
% leading to better compatibility with the inherent noise sensitivity. Further, we bring the geometry
% symmetries into the Bayesian update procedure through an equivariant inter-dependency modeling
% module. We also demonstrate that the density function of implied generative distribution is SE-(3)
% invariant and the generative process of iterative updating is roto-translational equivariant. Thirdly,
% with BFN’s powerful probabilistic modeling capacity, 3D molecule geometry representation can
% be further optimized into a representation with only two similar modalities: discretised charge and
% continuous atom coordinates. The mode-redundancy issue on discretised variable in the original
% BFNs is fixed by an early mode-seeking sampling strategy in GeoBFN.





% However, we highlight currently overlooked problems of crystal generation (in silico) methods that the iterative paths for each crystal data modality are not synchronized and may not be well defined on its manifold. On one hand, for every single modality, the iterative starting point and transitions should be invariant and equivariant on the manifold according to the nature of the data. For crystal data, atom types, fractional coordinate, and lattice parameters are indeed discrete, circular, and real-valued random variables (under the fractional coordinate system following \cite{jiao2023crystal}) located in simplex, torus, and Euclidean space accordingly. We term this Cartesian product of simplex, torus, and Euclidean space as \textit{Crystal Space} which we aim to model in this paper. On the other hand, for modalities as a bulk, the overall generative pace between modalities should remain synchronized due to the dependency relationship among modalities. Realizing these issues, we proposed to model the crystal data space in a unified and consistent framework via Bayesian Flow Networks (BFN) \cite{bfn}. 


% \hanlin{(Should GeoBFN be mentioned here?) Although BFN has been successfully applied to molecular conformation generation tasks, it is still challenging due to ...}


% We emphasize that constructing Bayesian flow for circular data is pretty challenging and faces theoretical problems due to the lack of circular distributions with desirable mathematical properties required by the BFN framework. Specifically, original BFN \cite{bfn} did not consider the periodicity which obviously can not be applied directly to crystal data. Furthermore, original BFN framework relied on the \textit{Bayesian conjugacy} and the so-called \textit{additive accuracy property} to build an analytical linear entropy Bayesian Flow. Notably, we found that the wrapped normal distribution used in \cite{jiao2023crystal, jing2022torsional} can not meet those requirements and we instead chose another circular normal distribution termed von Mises distribution to model circular data. We will show in the main text how to tackle all these theoretical problems when constructing a Bayesian flow in another manifold and would like to inspire the construction of Bayesian flows in other Riemannian manifolds \emph{e.g.} hypersphere. 


% In this work, we propose a novel approach termed \modelname to model the crystal data space. 
