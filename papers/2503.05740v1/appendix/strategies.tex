The map of strategy to its corresponding abbreviated tag is listed in Table\ref{table:DA_map}.
\begin{table}[h]
  \centering
    \begin{tabular}{{p{0.75\linewidth} p{0.25\linewidth} }}
    \hline
    Strategy & Tag\\
    \hline
    Acknowledge (Backchannel) & Ack\\
    Statement-non-opinion & StaNo\\
    Statement-opinion & Sta\\
    Affirmation and Reassurance & Agr\\
    Appreciation & App\\
    Conventional-closing & ConC\\
    Hedge & H\\
    Other & Oth\\
    Quotation & Quo\\
    Action-directive & AcD\\
    Collaborative Completion & CoC\\
    Restatement or Paraphrasing & Rep\\
    Offers Options Commits &Off\\
    Self-talk & Sel\\
    Apology & Apo\\
    Reflection of Feelings & RoF\\
    Yes-No-Question & YNQ\\
    Wh-Question & WhQ\\
    Declarative Yes-No-Question & DYNQ\\
    Open-Question & OpQ\\
    Or-Clause & OrC\\
    Conventional-opening & CoO\\
    Self-disclosure & Sd\\
    Providing Suggestions & PS\\
    Information & I\\
    \hline
    \end{tabular}
  \caption{Dialogue Act to its corresponding tag.}
  \label{table:DA_map}
\end{table}

There are two kinds of strategies: backward-looking and forward-looking. Backward-looking strategies reflect how the current utterance relates to the previous discourse. Forward-looking strategies reflect the current utterance constrains the future beliefs and actions of the participants and affects the discourse. Table \ref{tab:backward-looking DA} and Table \ref{tab:forward-looking DA} provide definitions and examples of each.
% The backward-looking strategies, definitions, and examples are shown in Table \ref{tab:backward-looking DA}. The forward-looking strategies, definitions, and examples are shown in Table \ref{tab:forward-looking DA}. 

\begin{table*}[h]
  \centering
   \resizebox{0.96\textwidth}{!}{
      \begin{tabular}{{p{0.1\textwidth} p{0.6\textwidth} p{0.3\textwidth}}}
        \hline
        \textbf{Strategy}           & \textbf{Definiton} & \textbf{Example} \\
        \hline
        StaNo
        & A factual statement or descriptive utterance that does not include an opinion. 
        & Me, I'm in the legal department.   \\
        Ack
        & A brief utterance that signals understanding, agreement, or active listening.
        & Uh-huh.\\
        Sta
        & A statement that conveys a personal belief, judgment, or opinion.
        & I think it's great\\
        Agr
        & Affirm the help seeker's strengths, motivation, and capabilities and provide reassurance and encouragement.
        & That's exactly it.\\
        App
        & An expression of gratitude, admiration, or acknowledgment of another’s effort or input. 
        & I can imagine.\\
        ConC
        & A formal or socially standard utterance signaling the end of a conversation. 
        & Well, it's been nice talking to you.\\
        H
        & An expression that introduces uncertainty or qualification to a statement, often to soften its impact. 
        & I don't know if I'm making any sense or not.\\
        Oth
        & Exchange pleasantries and use other support strategies that do not fall into the above categories. 
        & Well give me a break, you know.\\
        Quo
        & A direct or indirect repetition of someone else’s words. 
        & Albert Einstein once said, “Imagination is more important than knowledge.”\\
        AcD
        & A command, request, or suggestion directing someone to take action.
        & Why don't you go first\\
        CoC
        &A continuation or completion of someone else’s utterance in a collaborative manner. 
        & If we want to make it to the top of the mountain before sunset, we should…\\
        Rep
        & A simple, more concise rephrasing of the help-seeker's statements that could help them see their situation more clearly. 
        & It sounds like you’re saying that you’re struggling to stay on top of your work, and it’s leaving you feeling overwhelmed.\\
        Off
        & A statement proposing choices, making a commitment, or offering to do something. 
        & I'll have to check that out\\
        Sel
        & An utterance directed at oneself, often reflecting internal thought processes or problem-solving.
        & What's the word I'm looking for\\
        Apo
        & An expression of regret or asking for forgiveness. 
        & I'm sorry.\\
        RoF
        & Articulate and describe the help-seeker's feelings.
        & It sounds like you’re feeling really frustrated and drained because your efforts don’t seem to be paying off.\\
        \hline
      \end{tabular}
  }
  \caption{Backward-looking strategies, definition, and example. }
  \label{tab:backward-looking DA}
\end{table*}
\clearpage
\newpage
\begin{table*}[htbp!]
  \centering
   \resizebox{0.95\textwidth}{!}{
      \begin{tabular}{{p{0.1\textwidth} p{0.5\textwidth} p{0.4\textwidth}}}
        \hline
        \textbf{Strategy} & \textbf{Definiton} & \textbf{Example} \\
        \hline
        YNQ
        & A question expecting a binary (yes/no) response. 
        & Do you have to have any special training?\\
        WhQ
        & A question beginning with a wh-word (e.g., what, who, where), seeking specific information. 
        & Well, how old are you?\\
        DYNQ
        & A statement posed as a question, expecting a yes/no answer.
        & So you can afford to get a house?\\
        OpQ
        & A broad question inviting a wide range of responses, often conversational.
        & How about you?\\
        OrC
        & A question offering explicit alternatives, often in the form of “or.”
        & or is it more of a company?\\
        CoO
        & A socially standard utterance used to initiate a conversation.
        & How are you?\\
        Sd
        & Divulge similar experiences that you have had or emotions that you share with the help-seeker to express your empathy.
        & I completely understand how you feel. I remember feeling the same way before my first big presentation at work. I was so anxious, but I found that practicing a few extra times really helped calm my nerves.\\
        PS
        & Provide suggestions about how to change, but be careful to not overstep and tell them what to do.
        & You can keep a note to stop your idea from going. \\
        I
        & Provide useful information to the help-seeker, for example with data, facts, opinions, resources, or by answering questions.
        & Taking silver line from Washington D.C. to Dulles Intel Airport costs about 1 hour.\\
        \hline
      \end{tabular}
  }
  \caption{Forward-looking strategies, definition, and example. }
  \label{tab:forward-looking DA}
\end{table*}
\clearpage
\newpage


