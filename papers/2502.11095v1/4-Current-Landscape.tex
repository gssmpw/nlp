\section{Current Landscape}
\label{sec:landscape}


\begin{figure*}[htb]
\centering
\includegraphics[width=1\linewidth]{landscape.pdf}
    \caption{Distribution analysis of the current landscape.}
    \vspace{-3mm}
    \label{fig:data}
\end{figure*}
% subsection 1: 
Our survey encompasses a total of 69 studies in the field of LLMs in psychotherapy. Specifically, 33 studies address assessment, 9 focus on diagnosis, and 32 concentrate on treatment, with 5 studies overlapping across these dimensions. Approximately 74\% of the studies employed commercial large language models, while about 77\% used prompt-based techniques. This distribution highlights an imbalance in research focus across different stages of the psychotherapy process and reflects a heavy reliance on commercial models and prompt technologies.
% 讨论这三方面的的文章比重(A 33篇,D 9篇,T 32篇)一共69篇,5篇涉及重叠
% 模型51(共69)篇论文使用商业模型与53(共69)篇论文使用技术的占比 
% Table

% Subsection 2:
Figure \ref{fig:data} presents a comprehensive analysis of the current research landscape in this field. 
% 讨论语言覆盖范围
Panel \textbf{(a)} reveals a significant linguistic bias in existing studies, with English-language corpora dominates. 
While there are limited studies involving Korean and Dutch languages, this highlights a substantial gap in multilingual research approaches.
% 讨论心理障碍覆盖范围
Panel \textbf{(b)} quantitatively demonstrates the distribution of mental health research focuses. Mental disorder-related studies constitute 32\% of the total research corpus (represented by the orange outer ring).
Within this subset, depression-focused research accounts for 50\% of mental disorder studies, followed by anxiety-related research. This distribution indicates a concerning imbalance, where common conditions receive disproportionate attention while more complex disorders, such as bipolar disorder, remain understudied.
% 讨论心理治疗理论的占比
The analysis of psychotherapy theories in panel \textbf{(c)} uncovers another critical gap in the field. Only 32.8\% of the studies incorporate psychotherapy theories in their methodological approach. 
Notably, emerging therapeutic frameworks, such as humanistic therapy, are particularly underrepresented in current research applications.
% This finding suggests a need for more diverse theoretical approaches in future studies.


