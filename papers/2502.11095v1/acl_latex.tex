% This must be in the first 5 lines to tell arXiv to use pdfLaTeX, which is strongly recommended.
\pdfoutput=1
% In particular, the hyperref package requires pdfLaTeX in order to break URLs across lines.

\documentclass[11pt]{article}

% Change "review" to "final" to generate the final (sometimes called camera-ready) version.
% Change to "preprint" to generate a non-anonymous version with page numbers.
\usepackage[preprint]{acl}

% Standard package includes
\usepackage{times}
\usepackage{latexsym}

% For proper rendering and hyphenation of words containing Latin characters (including in bib files)
\usepackage[T1]{fontenc}
% For Vietnamese characters
% \usepackage[T5]{fontenc}
% See https://www.latex-project.org/help/documentation/encguide.pdf for other character sets

% This assumes your files are encoded as UTF8
\usepackage[utf8]{inputenc}

% This is not strictly necessary, and may be commented out,
% but it will improve the layout of the manuscript,
% and will typically save some space.
\usepackage{microtype}

% This is also not strictly necessary, and may be commented out.
% However, it will improve the aesthetics of text in
% the typewriter font.
\usepackage{inconsolata}

%Including images in your LaTeX document requires adding
%additional package(s)
\usepackage{graphicx}

\usepackage{tikz}
\usepackage{tikz}
\usepackage[edges]{forest}
\definecolor{hidden-draw}{RGB}{205, 44, 36}
\definecolor{hidden-blue}{RGB}{194,232,247}
\definecolor{hidden-orange}{RGB}{243,202,120}
\definecolor{hidden-yellow}{RGB}{242,244,193}
\definecolor{tree-level-1}{RGB}{245,20,85}
\definecolor{tree-level-2}{RGB}{246,86,118}
\definecolor{tree-level-3}{RGB}{248,177,193}
\definecolor{tree-leaf}{RGB}{176,230,198}


\usepackage{array}
\usepackage{booktabs}
\usepackage{colortbl}
\usepackage{xcolor}
\usepackage{pifont}

% If the title and author information does not fit in the area allocated, uncomment the following
%
%\setlength\titlebox{<dim>}
%
% and set <dim> to something 5cm or larger.

\title{A Survey of Large Language Models in Psychotherapy: \\ Current Landscape and Future Directions}

% Author information can be set in various styles:
% For several authors from the same institution:
% \author{Author 1 \and ... \and Author n \\
%         Address line \\ ... \\ Address line}
% if the names do not fit well on one line use
%         Author 1 \\ {\bf Author 2} \\ ... \\ {\bf Author n} \\
% For authors from different institutions:
% \author{Author 1 \\ Address line \\  ... \\ Address line
%         \And  ... \And
%         Author n \\ Address line \\ ... \\ Address line}
% To start a separate ``row'' of authors use \AND, as in
% \author{Author 1 \\ Address line \\  ... \\ Address line
%         \AND
%         Author 2 \\ Address line \\ ... \\ Address line \And
%         Author 3 \\ Address line \\ ... \\ Address line}

\author{
    Hongbin Na$^\heartsuit$\thanks{Equal contribution.}, Yining Hua$^\triangle$\footnotemark[1], Zimu Wang$^\clubsuit$\footnotemark[1],\\
    \textbf{Tao Shen$^\heartsuit$, Beibei Yu$^\heartsuit$, Lilin Wang$^{\diamondsuit}$, Wei Wang$^\clubsuit$, John Torous$^\triangle$, Ling Chen$^\heartsuit$} \\
    $^{\heartsuit}$Australian Artificial Intelligence Institute, University of Technology Sydney \\
    $^{\triangle}$Harvard University \ \
    $^{\clubsuit}$Xi'an Jiaotong-Liverpool University \ \
    $^{\diamondsuit}$University of Pennsylvania \\
    \texttt{hongbin.na@student.uts.edu.au,  yininghua@g.harvard.edu} \\ 
    \texttt{zimu.wang19@student.xjtlu.edu.cn} 
}

%\author{
%  \textbf{First Author\textsuperscript{1}},
%  \textbf{Second Author\textsuperscript{1,2}},
%  \textbf{Third T. Author\textsuperscript{1}},
%  \textbf{Fourth Author\textsuperscript{1}},
%\\
%  \textbf{Fifth Author\textsuperscript{1,2}},
%  \textbf{Sixth Author\textsuperscript{1}},
%  \textbf{Seventh Author\textsuperscript{1}},
%  \textbf{Eighth Author \textsuperscript{1,2,3,4}},
%\\
%  \textbf{Ninth Author\textsuperscript{1}},
%  \textbf{Tenth Author\textsuperscript{1}},
%  \textbf{Eleventh E. Author\textsuperscript{1,2,3,4,5}},
%  \textbf{Twelfth Author\textsuperscript{1}},
%\\
%  \textbf{Thirteenth Author\textsuperscript{3}},
%  \textbf{Fourteenth F. Author\textsuperscript{2,4}},
%  \textbf{Fifteenth Author\textsuperscript{1}},
%  \textbf{Sixteenth Author\textsuperscript{1}},
%\\
%  \textbf{Seventeenth S. Author\textsuperscript{4,5}},
%  \textbf{Eighteenth Author\textsuperscript{3,4}},
%  \textbf{Nineteenth N. Author\textsuperscript{2,5}},
%  \textbf{Twentieth Author\textsuperscript{1}}
%\\
%\\
%  \textsuperscript{1}Affiliation 1,
%  \textsuperscript{2}Affiliation 2,
%  \textsuperscript{3}Affiliation 3,
%  \textsuperscript{4}Affiliation 4,
%  \textsuperscript{5}Affiliation 5
%\\
%  \small{
%    \textbf{Correspondence:} \href{mailto:email@domain}{email@domain}
%  }
%}

\begin{document}
\maketitle
\begin{abstract}
Mental health remains a critical global challenge, with increasing demand for accessible, effective interventions. Large language models (LLMs) offer promising solutions in psychotherapy by enhancing the assessment, diagnosis, and treatment of mental health conditions through dynamic, context-aware interactions. This survey provides a comprehensive overview of the current landscape of LLM applications in psychotherapy, highlighting the roles of LLMs in symptom detection, severity estimation, cognitive assessment, and therapeutic interventions. We present a novel conceptual taxonomy to organize the psychotherapy process into three core components: assessment, diagnosis, and treatment, and examine the challenges and advancements in each area. The survey also addresses key research gaps, including linguistic biases, limited disorder coverage, and underrepresented therapeutic models. Finally, we discuss future directions to integrate LLMs into a holistic, end-to-end psychotherapy framework, addressing the evolving nature of mental health conditions and fostering more inclusive, personalized care.
\end{abstract}

\section{Introduction}

Deep Reinforcement Learning (DRL) has emerged as a transformative paradigm for solving complex sequential decision-making problems. By enabling autonomous agents to interact with an environment, receive feedback in the form of rewards, and iteratively refine their policies, DRL has demonstrated remarkable success across a diverse range of domains including games (\eg Atari~\citep{mnih2013playing,kaiser2020model}, Go~\citep{silver2018general,silver2017mastering}, and StarCraft II~\citep{vinyals2019grandmaster,vinyals2017starcraft}), robotics~\citep{kalashnikov2018scalable}, communication networks~\citep{feriani2021single}, and finance~\citep{liu2024dynamic}. These successes underscore DRL's capability to surpass traditional rule-based systems, particularly in high-dimensional and dynamically evolving environments.

Despite these advances, a fundamental challenge remains: DRL agents typically rely on deep neural networks, which operate as black-box models, obscuring the rationale behind their decision-making processes. This opacity poses significant barriers to adoption in safety-critical and high-stakes applications, where interpretability is crucial for trust, compliance, and debugging. The lack of transparency in DRL can lead to unreliable decision-making, rendering it unsuitable for domains where explainability is a prerequisite, such as healthcare, autonomous driving, and financial risk assessment.

To address these concerns, the field of Explainable Deep Reinforcement Learning (XRL) has emerged, aiming to develop techniques that enhance the interpretability of DRL policies. XRL seeks to provide insights into an agent’s decision-making process, enabling researchers, practitioners, and end-users to understand, validate, and refine learned policies. By facilitating greater transparency, XRL contributes to the development of safer, more robust, and ethically aligned AI systems.

Furthermore, the increasing integration of Reinforcement Learning (RL) with Large Language Models (LLMs) has placed RL at the forefront of natural language processing (NLP) advancements. Methods such as Reinforcement Learning from Human Feedback (RLHF)~\citep{bai2022training,ouyang2022training} have become essential for aligning LLM outputs with human preferences and ethical guidelines. By treating language generation as a sequential decision-making process, RL-based fine-tuning enables LLMs to optimize for attributes such as factual accuracy, coherence, and user satisfaction, surpassing conventional supervised learning techniques. However, the application of RL in LLM alignment further amplifies the explainability challenge, as the complex interactions between RL updates and neural representations remain poorly understood.

This survey provides a systematic review of explainability methods in DRL, with a particular focus on their integration with LLMs and human-in-the-loop systems. We first introduce fundamental RL concepts and highlight key advances in DRL. We then categorize and analyze existing explanation techniques, encompassing feature-level, state-level, dataset-level, and model-level approaches. Additionally, we discuss methods for evaluating XRL techniques, considering both qualitative and quantitative assessment criteria. Finally, we explore real-world applications of XRL, including policy refinement, adversarial attack mitigation, and emerging challenges in ensuring interpretability in modern AI systems. Through this survey, we aim to provide a comprehensive perspective on the current state of XRL and outline future research directions to advance the development of interpretable and trustworthy DRL models.

\section{Conceptual Taxonomy}
\label{sec:taxonomy}



To establish a standardized framework for understanding psychotherapy, we propose a hierarchical taxonomy aligned with the American Psychological Association's tripartite model of psychotherapeutic processes\footnote{\url{https://www.apa.org/topics/psychotherapy}}. As illustrated in Figure \ref{fig:Taxonomy}, this taxonomy organizes psychotherapy into three core components: (1) Assessment, (2) Diagnosis, and (3) Treatment, with dynamic interconnections\footnote{Throughout this taxonomy, the terms \textit{Assessment}, \textit{Diagnosis}, and \textit{Treatment} specifically refer to the three core components of psychotherapy.}. Each component is detailed below.

\subsection{Assessment}

\paragraph{Definition.} Psychological assessment constitutes the systematic collection and interpretation of data regarding an individual's cognitive, emotional, and behavioral functioning~\cite{cohen1996psychological, kaplan2001psychological}. This process employs psychometric tests, structured clinical interviews, behavioral observations, and collateral information to establish a multidimensional profile of psychological states~\cite{groth2009handbook}.


\paragraph{Significance.} As the foundational stage of psychotherapy, assessment provides the empirical basis for understanding a client's unique psychological landscape. It enables therapists to identify symptom patterns~\cite{phillips2007assessing}, track temporal changes~\cite{barkham1993shape}, and contextualize subjective experiences within objective frameworks~\cite{groth2009handbook}. The continuous nature of psychological assessment allows for real-time adjustments to therapeutic strategies~\cite{schiepek2016real}, ensuring interventions remain responsive to evolving client needs.

\subsection{Diagnosis} 

\paragraph{Definition.} Diagnosis represents the analytical process of categorizing psychological distress using established nosological systems such as the DSM-5~\cite{Section2:11} and ICD-11~\cite{ICD11}. This involves differentiating normative emotional responses from pathological conditions while considering cultural~\cite{AlanCu2010} and developmental~\cite{kawade2012} variables that influence symptom manifestation.

\paragraph{Significance.} Diagnosis serves as the conceptual bridge between assessment and treatment, providing a structured framework for intervention planning~\cite{jensen2011understanding}. By aligning clinical observations with standardized criteria, it enhances communication among professionals~\cite{craddock2014psychiatric} and facilitates evidence-based decision-making~\cite{apa2006evidence}. 

\subsection{Treatment}

\paragraph{Definition.} Treatment includes evidence-based interventions designed to reduce psychological distress and improve functioning~\cite{apa2006evidence}. These interventions work by building a therapeutic alliance~\cite{elvins2008conceptualization}, restructuring cognition~\cite{ezawa2023cognitive}, and modifying behavior~\cite{martin2019behavior}, all typically grounded in well-established theoretical orientations.

\paragraph{Significance.} Treatment transforms the theories and information gleaned from assessment and diagnosis into practical interventions~\cite{prochaska2018systems} that directly address the client's psychological distress~\cite{barlow2021clinical} and foster personal growth~\cite{lambert2013bergin}. 

\subsection{Interrelations}
The taxonomy's components interact through three dynamic processes that define psychotherapy as a complex adaptive system:

\paragraph{Synthesizing (Assessment $\rightarrow$ Diagnosis)} The dialectical integration of observational data with nosological frameworks enables diagnostic classifications to contextualize assessment findings, \textit{synthesizing} the patient's various symptoms and behavioral patterns into a diagnostic result~\cite{rencic2016understanding}.

\paragraph{Framing (Diagnosis $\rightarrow$ Treatment)} Diagnosis functions as a \textit{framing} mechanism, integrating complex and diverse symptoms into a coherent classification that establishes a clear blueprint for treatment~\cite{Section2:11}.

\paragraph{Customization (Assessment $\rightarrow$ Treatment)}  A process where treatment plans are continuously \textit{refined} based on assessment results, considering individual differences without being constrained by diagnostic labels, to enhance therapeutic effectivenesss~\cite{Section2:5}.

%The dynamic optimization process where therapeutic techniques continuously \textit{adapt} to assessment-derived insights, while ongoing evaluation \textit{refines} interventions to enhance personalization and effectiveness\cite{Section2:5}.

% \begin{figure}[htbp]
% \centering
% \includegraphics[width=0.5\textwidth]{2.4 relation.png}
% \caption{The dynamic and interrelated network among assessment, diagnosis, and treatment}
% \label{fig:Taxonomy}
% \end{figure}



% Simplified version:

% In traditional medicine, assessment leads to diagnosis, which then informs treatment—a linear progression effective for physiological diseases where diagnoses correspond directly to specific pathologies and interventions \cite{Section2:1}. However, in psychology, this linear relationship is less applicable. Psychological diagnoses rely on clinical observations rather than definitive biological tests \cite{Section2:2}. Consequently, social and psychological burdens are often introduced by the use of diagnostic labels in psychotherapy. For instance, stigma and discrimination can be exacerbated when someone is labeled as "schizophrenic" instead of "a person with schizophrenia" \cite{Section2:3}. This has led some professionals to question the necessity of diagnosis in treatment.

% In psychotherapy, assessment, diagnosis, and treatment form a dynamic, interconnected network rather than a linear sequence (see Figure \ref{fig:relationship}). Assessment \textbf{synthesizes} the patient's various symptoms and behavior patterns into a diagnostic result, thereby providing foundational data for subsequent treatment. Treatment can be \textbf{customized} directly from assessment results without strict reliance on diagnostic labels \cite{Section2:4}, and assessment continues throughout treatment to monitor and adjust plans as needed.

% \begin{enumerate}
% \item \textbf{Treatment through diagnosis:} Assessment results are used to determine the patient's diagnosis. A clear \textbf{frame} for treatment is provided by the diagnostic label, and guiding treatment planning based on established guidelines.

% \item \textbf{Direct treatment based on assessment: } Treatment plans are customized directly based on assessment results, considering individual differences without being constrained by diagnostic labels \cite{Section2:5}.
% \end{enumerate}

% However, the first path offers structure but has limitations. Psychiatric diagnoses can be ambiguous and heterogeneous; patients with the same diagnosis may exhibit different symptoms \cite{Section2:9}. Additionally, Diagnostic criteria vary across cultures \cite{Section2:7} and evolve over time (e.g., multiple revisions of the DSM) \cite{Section2:8}, reducing their stability. Finally, over-reliance on diagnosis can also undermine the therapeutic relationship built on trust \cite{Section2:10}.

% Given these limitations, prioritizing the second path—bypassing diagnosis and directly formulating personalized treatment plans based on assessment—is recommended. This symptom-oriented approach enhances intervention specificity and effectiveness, addressing ambiguities inherent in mental illness diagnosis \cite{Section2:6}.

% In conclusion, the interactive relationship among assessment, diagnosis, and treatment is crucial in psychotherapy. While diagnosis can provide a helpful framework to treatment, its limitations make it necessary, in certain situations, to proceed directly to treatment based on assessment. Directly basing treatment on assessment allows for more precise and personalized care, better suited to the complexities of mental health.

%  In traditional medicine, assessment, diagnosis, and treatment typically follow a linear progression. Assessment information—including the patient's physical signs, laboratory test results, and imaging data—is analyzed to arrive at a diagnosis, upon which an appropriate treatment plan is formulated. This approach is effective for physiological diseases because diagnoses often directly correspond to specific pathological processes and therapeutic interventions \cite{Section2:1}.

% However, in psychology, the nature and function of diagnosis differ, and this linear relationship is not always applicable. Psychological diagnoses rely more on clinical observation of behavior rather than definitive biological tests \cite{Section2:2}. Consequently, social and psychological burdens are often introduced by the use of diagnostic labels in psychotherapy. For instance, stigma and discrimination can be exacerbated when someone is labeled as "schizophrenic" instead of "a person with schizophrenia" \cite{Section2:3}. This has led some psychologists and researchers to question the necessity of diagnosis in treatment.

% In clinical practice, the relationship among assessment, diagnosis, and treatment forms a dynamic, interrelated complex network rather than a linear sequence.  Figure \ref{fig:relationship} illustrates that assessment, as the starting point of the treatment process, \textbf{synthesizes} the patient's various symptoms and behavior patterns into a diagnostic result, thereby providing foundational data for subsequent treatment. Simultaneously, treatment can be directly \textbf{customized} based on assessment results without strict reliance on diagnostic labels \cite{Section2:4}. In many cases, assessment is continuous throughout the treatment process, used to monitor and adjust the treatment plan to ensure it remains consistent with the patient's actual condition \cite{Section2:4}.

% Therefore, based on assessment results, psychotherapy can follow two distinct paths:

% \begin{enumerate}
% \item \textbf{Formulating a treatment plan through diagnosis:} In this approach, assessment results are used to determine the patient's diagnosis. A clear frame for treatment is provided by the diagnostic label, and corresponding treatment plans are formulated based on the diagnosis. This method standardizes the treatment process to a certain extent, allowing clinicians to select treatment options based on established guidelines and experience.

% \item \textbf{Bypassing diagnosis and directly formulating a treatment plan based on assessment results:} By fully considering the patient's individual differences and unique combination of symptoms, treatment plans can be customized for each specific situation \cite{Section2:5}, unrestricted by diagnostic labels. This personalized method may better meet the patient's specific needs.
% \end{enumerate}

% In the first path, assessment information is structured within this network by diagnosis. Complex and diverse symptoms can be integrated into a clear label or classification by comparing assessment results with established mental health diagnostic standards (e.g., DSM-5)\cite{Section2:6}. This structuring process provides a clear \textbf{frame} for treatment; diagnosis serves as a bridge from assessment to treatment, helping to transform the diverse information collected during assessment into a specific treatment plan.

% However, the diagnosis of mental illnesses often involves ambiguity and uncertainty, especially when confronting diverse combinations of symptoms. The presence of different cultural backgrounds \cite{Section2:7}, temporal changes \cite{Section2:8}, and individual patient differences further exacerbate the heterogeneity and limitations of diagnosis. Firstly, psychiatric diagnoses are not homogeneous. For instance, entirely different combinations of symptoms may be exhibited by patients diagnosed with Major Depressive Disorder (MDD) \cite{Section2:9}. The validity of a single diagnostic label is challenged by this symptom diversity, suggesting that focusing on individual symptoms rather than the overall diagnosis may better aid in understanding and treating patients. Secondly, diagnostic criteria for mental illnesses are not immutable. Diagnostic standards can vary significantly across different cultural contexts \cite{Section2:7}, and over time, these criteria continually evolve (e.g., multiple revisions of the DSM) \cite{Section2:8}. This instability further weakens the value of diagnosis as a core tool. In contrast, symptom manifestations are relatively stable and more suitable as the central focus in clinical practice. Additionally, the diagnostic process is inseparable from the doctor-patient relationship. Diagnosis is not merely a technical procedure but an interpersonal interaction built on trust \cite{Section2:10}. In this process, clinicians are not just executors of treatment but also supporters and guides for patients \cite{Section2:10}. Overreliance on diagnosis may mechanize the treatment process, neglecting trust-based interpersonal interactions, thereby undermining the crucial therapeutic relationship and ultimately affecting treatment outcomes.

% Considering the limitations and heterogeneity of mental illness diagnoses, it is recommended that psychotherapy should prioritize the second path—that is, bypassing the diagnostic stage and directly formulating personalized treatment plans based on assessment results rather than relying on traditional symptom-combination diagnostic labels. This symptom-oriented treatment approach not only enhances the specificity and effectiveness of interventions but also better addresses the common ambiguities and uncertainties inherent in mental illness diagnosis \cite{Section2:6}.

% In conclusion, the interactive relationship among assessment, diagnosis, and treatment is a crucial foundation for ensuring the effectiveness of psychotherapy. Assessment can provide necessary support for diagnosis, aiding in guiding treatment, or can be used to directly inform treatment by bypassing diagnosis. While diagnosis offers a beneficial structural framework for treatment, its limitations make it necessary, in certain situations, to proceed directly to treatment based on assessment. This dynamic and flexible relational network allows psychotherapy to better address complex and evolving clinical realities, providing more precise and personalized patient care.



\section{LLMs in Psychotherapy}
\label{sec:llms}
\subsection{Assessment}

\begin{table*}[t!]
    \centering
    \resizebox{\textwidth}{!}{
    \begin{tabular}{c p{3.75cm} p{3.5cm} p{2.25cm} p{3.75cm}}
        \toprule
         \textbf{Study} & \textbf{Text Granularity} & \textbf{Best Technique} & \textbf{NLP Task} & \textbf{Assessment Focus} \\
         \midrule
         \multicolumn{5}{c}{\emph{Symptom Detection}}\\
         \midrule

         \citet{18} & Single Post & Emotion Prompting & BC/MCC/EG & Multiple Symptoms \\
         \citet{22} & Multi-turn Dialogue & Fine-Tuning & MLC/IE/SUM & Multiple Symptoms \\
         \citet{24} & Multi-turn Dialogue & Few-Shot Prompting & MLC/IE/SUM & PTSD \\
         \citet{27} & Single Post & Fine-Tuning & MLC/EG & Depression \\
         \citet{66} & Single Post & Few-Shot Prompting & MCC & Multiple Symptoms \\
         \citet{85} & Posts From One User & Chain-of-Thought & IE/SUM & Suicidal Ideation \\
         \citet{91} & Posts From One User & Role Prompting & IE/SUM & Suicidal Ideation \\
         \citet{94} & Single Post & Fine-Tuning & BC/MCC/EG & Multiple Symptoms \\
         \citet{95} & Single Post & Fine-Tuning & BC/EG & Multiple Symptoms \\
         \citet{96} & Single Post & Few-Shot Prompting & MLC & Multiple Symptoms \\
         \citet{108} & Single Post & Fine-Tuning & MLC & Suicidal Ideation \\
         \citet{120} & Single Post & Zero-Shot Prompting & BC & PTSD \\

         \midrule
         \multicolumn{5}{c}{\emph{Symptom Severity}}\\
         \midrule
         \citet{11} & Multi-turn Dialogue & Zero-Shot Prompting & TR & Depression/PTSD \\
         \citet{23} & Multi-turn Dialogue & Zero-Shot Prompting & TR & Depression/Anxiety \\
         \citet{58} & Posts From One User & Zero-Shot Prompting & TR & Depression \\
         \citet{59} & Posts From One User & Zero-Shot Prompting & TR & Depression \\
         \citet{100} & Single Post & Zero-Shot Prompting & TR/MCC & Depression/Suicide \\

         \midrule
         \multicolumn{5}{c}{\emph{Cognition}}\\
         \midrule
         \citet{5} & Single Sentence & Few-Shot Prompting & MLC & Cognitive Distortions \\
         \citet{10} & Single Post & Fine-Tuning & MLC & Cognitive Distortions \\
         \citet{12} & Single Sentence & Few-Shot Prompting & MCC & Cognitive Distortions \\
         \citet{14} & Single-turn Dialogue & Zero-Shot Prompting & BC/MCC/EG & Cognitive Distortions \\
         \citet{16} & Single Post & Zero-Shot Prompting & MLC & Maladaptive Schemas \\
         \citet{38} & Single Post & Zero-Shot Prompting & MCC/SUM & Cognitive Pathways \\
         \citet{45} & Single-turn Dialogue & Multi-Agent Debate & MCC & Cognitive Distortions \\

         \midrule
         \multicolumn{5}{c}{\emph{Behavior}}\\
         \midrule
         \citet{36} & Single Post & Zero-Shot Prompting & MLC/EG & Interpersonal Risk \\
         \citet{60} & Sentence From Dialogue & Few-Shot Prompting & MCC & MI-Adherent Behaviors \\
         \citet{65} & Sentence From Dialogue & Zero-Shot Prompting & MCC & MI-Adherent Behaviors \\
         \citet{67} & Sentence From Dialogue & Zero-Shot Prompting & MCC & MI-Adherent Behaviors \\
         

         \bottomrule
    \end{tabular}
    }
    \vspace{-5pt}
    \caption{Comparison of Psychological Assessment Studies by Input Characteristics and Methodology. \textbf{MLC}: Multi-Label Classification, \textbf{IE}: Information Extraction, \textbf{SUM}: Summarization, \textbf{MCC}: Multi-Class Classification, \textbf{BC}: Binary Classification, \textbf{TR}: Text Regression, \textbf{EG}: Explanation Generation. Studies are categorized through text granularity, optimal technical approach (\textit{Best Technique}), NLP task formulation, and specific assessment focus.}
    \vspace{-4mm}
    \label{tab:assessment_method_comparison}
\end{table*}

\paragraph{Symptom Detection} leverages LLMs to identify mental health conditions including depression, anxiety, PTSD, and suicidal ideation, demonstrating robust performance and multidimensional applicability across diverse scenarios. \citet{18} systematically evaluated GPT-3.5, InstructGPT3, and LLaMA models across 11 datasets, revealing that emotion-enhanced chain-of-thought prompting improves interpretability yet remains inferior to specialized supervised methods. \citet{22} achieved 70.8\% zero-shot symptom retrieval accuracy in Korean psychiatric interviews using GPT-4 Turbo, while their fine-tuned GPT-3.5 attained 0.817 multi-label classification accuracy. Clinical applications show particular promise, as \citet{24} leveraged GPT-4 and Llama-2 to automate PTSD assessments through information extraction from 411 interviews, significantly enhancing diagnostic practicality. 

Social media analysis benefits from approaches like \citet{27}'s interpretable depression detection framework, which demonstrated strong performance across Vicuna-13B and GPT-3.5 environments. Resource development advances include \citet{66}'s \emph{MentalHelp} dataset with 14 million instances, validated through GPT-3.5 zero-shot evaluations. For suicidal ideation monitoring, \citet{85} and \citet{91} achieved state-of-the-art evidence extraction in the CLPsych 2024 shared task through innovative prompting strategies. Open-source initiatives like \emph{MentaLLaMA} by \citet{94} and \emph{Mental-LLM} by \citet{95} enable multi-symptom detection via instruction-tuned LLaMA variants, though \citet{96}'s \emph{WellDunn} framework reveals persistent gaps in GPT-family models' explanation consistency.

Cross-lingual adaptations include \citet{108}'s \emph{PsyGUARD} system based on fine-tuned CHATGLM2-6B for Chinese suicide risk assessment, while \citet{120} demonstrated domain-specific RoBERTa models outperforming GPT-4 in cross-domain PTSD pattern analysis, highlighting the critical balance between model specialization and interpretability.

\paragraph{Symptom Severity} focuses on estimating the level of mental health condition intensity, particularly for depression, anxiety, and PTSD. Clinical evaluations reveal Med-PaLM 2's zero-shot depression scoring attains clinician-level alignment on interview data \citep{11}, though with limited PTSD generalizability. When benchmarked against specialized Transformers on DAIC-WOZ dataset~\cite{gratch-etal-2014-distress}, ChatGPT and Llama-2 exhibit moderate efficacy \citep{23}, suggesting domain-specific architectures retain advantages in structured assessments. Shifting attention to social media data, \citet{58} proposed a pipeline that retrieves depression-relevant text, summarizes it according to the Beck Depression Inventory (BDI)~\cite{jackson2016beck}, and then utilizes LLMs to predict symptom severity, achieving performance similar to expert evaluations on certain measures. In a similar vein, \citet{59} introduced an explainable depression detection system that leverages multiple open-source LLMs to generate BDI-based answers, reporting near state-of-the-art performance without additional training data. Cross-lingual extensions emerge through \citet{100}'s framework enabling severity prediction across 6 languages and 2 mental conditions.

\paragraph{Cognition} centers on identifying and understanding maladaptive thinking patterns, such as cognitive distortions and early maladaptive schemas, using LLMs. \citet{5} introduced a cognitive distortion dataset and employed a few-shot strategy with GPT-3.5 to generate, classify, and reframe them, while \citet{10} constructed two Chinese social media benchmarks for cognitive distortion detection and suicidal risk assessment, demonstrating that fine-tuned LLMs are more closely than zero-/few-shot methods to supervised baselines. In a related effort, \citet{12} released the C2D2 dataset containing 7{,}500 Chinese sentences with distorted thinking patterns.
% thereby facilitating fine-grained exploration of how such distortions manifest in everyday language.
Expanding on detection methods, \citet{14} proposed a Diagnosis of Thought prompting approach for GPT-4 and ChatGPT, which breaks down patient utterances into factual versus subjective content and supports the generation of interpretable diagnostic reasoning. Beyond cognitive distortions, \citet{16} investigated zero-shot approaches with GPT-3.5 to identify early maladaptive schemas in mental health forums, highlighting challenges in label interpretability and prompt sensitivity. Complementarily, \citet{38} presented a hierarchical classification and summarization pipeline to extract cognitive pathways from Chinese social media text, underscoring GPT-4’s strong performance albeit with occasional hallucinations. Finally, \citet{45} introduced a multi-agent debate framework for cognitive distortion classification, reporting substantial gains in both accuracy and specificity by synthesizing multiple LLM opinions before forming a final verdict.

\paragraph{Behavior} highlights how user actions—or in the case of Motivational Interviewing (MI), language itself—can serve as a measurable indicator of one’s readiness for change. For instance, \citet{36} introduced the MAIMS framework, employing mental scales in a zero-shot setting to identify interpersonal risk factors on social media, thereby enhancing both interpretability and accuracy. In clinical dialogues, \citet{60} demonstrated how LLMs can automatically detect a client’s motivational direction (e.g., change versus sustain talk) and commitment level, offering valuable insights for MI-based interventions. Extending such analyses to bilingual settings, \citet{65} proposed the BiMISC dataset and prompt strategies that enable LLMs to code MI behaviors across multiple languages with expert-level performance. Lastly, \citet{67} presented MI-TAGS for automated annotation of global MI scores, illustrating how context-sensitive modeling can approximate human annotations in psychotherapy transcripts.


\paragraph{Advanced} research has evolved beyond foundational assessment tasks to emphasize novel methodological paradigms, bias mitigation, and domain-specific summarization frameworks. For instance, \citet{48} introduced \emph{PsychoGAT}—an interactive, game-based approach that transforms standardized psychometric instruments into engaging narrative experiences, improving psychometric reliability, construct validity, and user satisfaction when measuring constructs such as depression, cognitive distortions, and personality traits. In parallel, \citet{53} systematically investigated potential biases in various LLMs across multiple mental health datasets, revealing that even high-performing models exhibit unfairness related to demographic factors. The authors proposed fairness-aware prompts to substantially reduce such biases without sacrificing predictive accuracy. Furthermore, \citet{99} presented the \emph{PIECE} framework, which adopts a planning-based approach to domain-aligned counseling summarization, structuring and filtering conversation content before integrating domain knowledge.

% \citet{99} presented the \emph{PIECE} framework, which adopts a planning-based approach to domain-aligned counseling summarization. By structuring and filtering conversation content before integrating domain knowledge, PIECE achieved notable improvements over baseline methods on both automated and expert evaluations.



\subsection{Diagnosis}

\paragraph{Static Diagnosis} is based on a fixed set of data, typically derived from complete dialogues or social media posts. \newcite{11} highlighted the effectiveness of Med-PaLM 2 in psychiatric condition assessment from patient interviews and clinical descriptions without specialized training. Similarly, \newcite{107} showcased LLMs' superior performance on depression and anxiety detection on Russian datasets, particularly with noisy or small datasets. \newcite{102} evaluated PLMs and LLMs on multi-label classification in depression and anxiety, underscoring the ongoing challenges in applying LMs to mental health diagnostics. Besides, \newcite{124} introduced \textit{DORIS}, a depression detection system integrating text embeddings with LLMs, utilizing symptom features, post-history, and mood course representations to make diagnostic predictions and generate explanatory outputs. \newcite{110} developed \textit{ADOS-Copilot} for ASD diagnosis through diagnostic dialogues, employing In-context Enhancement, Interpretability Augmentation, and Adaptive Fusion based on real-world ADOS-2 clinic scenarios.

\paragraph{Dynamic Diagnosis} involves real-time evaluation based on ongoing, interactive conversations between the patient and LLM, enabling more personalized and contextually relevant insights. \newcite{9} simulated psychiatrist-patient interactions with ChatGPT, in which the doctor chatbot focused on role, tasks, empathy, and questioning strategies, while the patient chatbot emphasized symptoms, language style, emotions, and resistance behaviors. \newcite{33} introduced the \textit{Symptom-related and Empathy-related Ontology (SEO)}, grounded in DSM-5 and Helping Skills Theory, for depression diagnosis dialogues. \newcite{52} dissected the doctor-patient relationship into psychologist’s empathy and proactive guidance and introduced \textit{WundtGPT} that integrated these elements. \newcite{106} further presented the \textit{AMC}, a self-improving conversational agent system for depression diagnosis through simulated dialogues between patient and psychiatrist agents.
% The psychiatrist agent is structured with a tertiary memory structure, a dialogue control and reflection plugin that functions as a “supervisor,” and a memory sampling module.





\subsection{Treatment}

\paragraph{LLM as a Virtual Therapist} centers on leveraging LLMs to directly engage in therapeutic conversations, often adopting multi-turn dialogues that incorporate recognized psychotherapeutic frameworks. For instance, \citet{20} proposed \emph{HealMe} to facilitate cognitive reframing and empathetic support in line with established psychotherapy principles. Likewise, \citet{29} introduced \emph{CaiTI}, a system embedded in everyday smart devices that conducts assessments of users’ daily functioning and delivers psychotherapeutic interventions through adaptive dialogue flows. In a similar vein, \citet{40} presented \emph{CoCoA}, specializing in identifying and resolving cognitive distortions via dynamic memory mechanisms and CBT-based strategies, while \citet{54} proposed a step-by-step approach guiding users to execute self-guided cognitive restructuring through multiple interactive sessions. Beyond standard CBT protocols, \citet{55} focused on aiding psychiatric patients in journaling their experiences, thereby offering richer clinical insights, whereas \citet{77} developed a multi-round CBT dataset to refine LLMs for direct counseling-like interactions. Additionally, multi-agent frameworks like \emph{MentalAgora} \citep{78} highlighted personalized mental health support by integrating multiple specialized agents, and \citet{109} further explored “mixed chain-of-psychotherapies” to combine various therapeutic methods, aiming to enhance the emotional support and customization delivered by chatbot interactions.

\paragraph{LLM as an Assistive Tool} refrains from providing a holistic therapy role but instead offers targeted support such as rewriting suboptimal counselor responses, generating controlled reappraisal prompts, or aiding clinicians in specific tasks. For example, \citet{2} proposed to rewrite responses that violate MI principles into MI-adherent forms, ensuring more consistent therapeutic dialogue. Meanwhile, \citet{3} and \citet{5} focused on generating single-turn reframes of negative thoughts—often anchored in cognitive distortions—through controlled language attributes. On the detection side, \citet{35} built a multimodal pipeline to identify depression and provide CBT-style replies, albeit with an emphasis on technological assistance rather than full-fledged therapy. In the Chinese context, \citet{43} combined cognitive distortion detection with “positive reconstruction,” demonstrating a single-round rewrite approach for negative or distorted statements, while \citet{64} showcased a structured Q\&A format that offers professional yet succinct CBT-based responses. From a knowledge-distillation angle, \citet{70} demonstrated how smaller models could replicate GPT-4’s MI-style reflective statements, and \citet{81} introduced a lighter-weight framework \emph{RESORT} to guide smaller LLMs toward effective cognitive reappraisal prompts, thus enabling broader accessibility of self-help tools.

\paragraph{LLM as Simulated Patients for Clinician Education} pivots toward generating synthetic yet realistic patient behaviors or multi-level feedback to train or support mental health practitioners. For instance, \citet{49} leveraged LLMs to deliver multi-tier feedback on novice peer counselors’ conversational skills, significantly reducing the need for continuous expert oversight. Similarly, \citet{51} using LLM-driven patient simulations that help trainees practice CBT core skills in a controlled, repeatable setup. In the realm of assessing therapy quality, \citet{56} showcased a digital patient system to evaluate MI sessions, employing AI-generated transcripts to differentiate novice, intermediate, and expert therapeutic skill levels. Complementarily, \citet{80} offered \emph{Roleplay-doh}, a pipeline wherein domain experts craft specialized principles that guide LLM-based role-playing agents, thereby providing customizable training for new therapists.

\paragraph{LLM for Evaluation and Quality Analysis} targets the appraisal of therapy dialogue, counselor techniques, and treatment processes, typically without delivering direct interventions to clients. For instance, \citet{4} augmented crisis counseling outcome prediction by fusing annotated counseling strategies with LLM-derived features, achieving substantially improved accuracy. In the Chinese context, \citet{19} introduced \emph{CPsyCoun}, employing reports-based dialogue reconstruction and automated evaluation to verify counseling realism and professionalism. Beyond single-session analyses, \citet{32} used simulated clients to assess perceived therapy outcomes, while \citet{37} created the \emph{BOLT} framework for systematically comparing LLM-based therapy behaviors with high- and low-quality human sessions. Further extending to online counseling, \citet{39} proposed an LLM-based approach to measure therapeutic alliance, whereas \citet{62} delineated therapist self-disclosure classification as a new NLP task. In the MI domain, \citet{65} and \citet{67} collected bilingual transcripts to systematically annotate therapist–client exchanges for behavior coding and global scores, respectively. Additionally, multi-session perspectives emerge in \citet{101}, who proposed \emph{IPAEval} to track long-term progress from the client’s viewpoint, and \citet{103} analyzed conversation redirection and its impact on patient–therapist alliance over multiple sessions. Finally, \citet{111} and \citet{122} explored the disparities between LLM- and human-led CBT sessions, highlighting gaps such as empathy and cultural nuance while also introducing \emph{CBT-Bench} to probe LLMs’ deeper psychotherapeutic competencies.

% \subsubsection{Detection and Classification}
% LLMs have been widely used in detecting and classifying cognitive distortions following the CBT psychotherapy framework. Cognitive distortion detection involves identifying and addressing negative thought patterns that can impact mental health. Research highlights LLM's ability to generate, recognize, and reframe unhelpful thoughts. For example, models like T5, R2C2, and GPT-3.5 have been fine-tuned to enhance the identification and reframing of cognitive distortions, evaluated through both automated metrics and human assessments \cite{5}. Similarly, GPT-3.5, GPT-4, and ChatGLM2-6B have achieved high precision and recall in detecting cognitive distortions in Chinese social media \cite{10}. Datasets such as C2D2 aid in annotating cognitive distortions but focus more on dataset quality than model applications \cite{12}. Additionally, specialized prompting techniques with models like ChatGPT, Vicuna, and GPT-4 have improved diagnostic accuracy \cite{14}. Non-English language models like ChatGLM-6B have also shown strong results in detecting cognitive distortions in Mandarin \cite{43}. The ERD framework has advanced classification by using multi-agent debate models to boost accuracy \cite{45}. Interactive systems employing LLMs for cognitive restructuring are promising, though their long-term effectiveness remains untested \cite{54}. Furthermore, GPT-4 has been applied to enhanced CBT interventions by extracting cognitive pathways from social media \cite{38}. Overall, LLMs effectively address cognitive distortions, though further exploration in real-time and cross-cultural contexts is needed.

% Behavioral coding under the motivational interviewing framework is another popular field. It focuses on analyzing and improving the quality of therapeutic conversations, ensuring that dialogue aligns with effective motivational strategies. Studies have explored various approaches in doing so, often fine-tuning models like Blender, GPT-3, GPT-3.5, GPT-4, and Flan-T5 to improve the quality and adherence of therapeutic dialogues. For instance, transforming non-adherent responses into MI-adherent ones and automating MI skill coding have led to improvements in dialogue quality and alignment with expert annotations \cite{2,65}. These outcomes were evaluated using metrics such as BLEU, ROUGE, METEOR, human evaluations, accuracy, precision, recall, and macro F1 scores.  \cite{67} touched up on automatic classification of MI transcripts and global score predictions. Models like GPT-3.5, and GPT-4 have been employed to annotate MI sessions with behavioral codes and predict global scores, with evaluations based on accuracy, Macro F1, ROC AUC, and Pearson correlation Additionally, LLMs have been used to detect client motivational language, which informs therapeutic strategies, evaluated using accuracy and F1 scores \cite{60}.

% Other classification tasks include identifying Early Maladaptive Schemes (EMS) and categorizing Therapist Self-Disclosure (TSD). For EMS, models like GPT-3.5 and Flan-T5 have effectively identified schemas from mental health forum texts with high precision and recall, though their real-time application in therapy remains unexplored \cite{16}. Similarly, GPT-4 has been used to categorize therapist self-disclosure into immediate, non-immediate, or non-TSD categories \cite{62}, demonstrating successful classification. However, the long-term effects of different types of TSD on therapy outcomes have not yet been investigated.

% \subsubsection{Reframing and Positive Reconstruction}
% Reframing and positive reconstruction follow cognitive distortion detection. It helps patients to transform negative thought patterns into more constructive ones. used retrieval-enhanced in-context learning to train GPT-3 to reframe negative thoughts, which reduced negative emotions. fine-tuned T5 and GPT-3.5 with large-scale datasets to improve their ability to recognize and generate effective reframed thoughts. \cite{81} explored the ability of LLMs like GPT-4 turbo, LLaMA-2, and Mistral to offer cognitive reappraisal, a CBT strategy, evaluated through human assessments for empathy, factuality, and harmfulness. Another approach \cite{20} involved creating a specialized system based on LLaMA2-7b-chat to produce empathetic and logically coherent responses. Additionally, \cite{54} explored self-guided cognitive restructuring LLMs, where GPT-3 was trained to develop an interactive framework for providing helpful and empathetic guidance. The effectiveness of these studies is generally validated using methods like BLEU, ROUGE, BERTScore, PANAS, and human assessments.

% \subsubsection{LLMs as therapists, patients, and more}
% \textbf{As Therapists.} The exploration of LLMs as potential therapists has attracted much attention in simulating psychiatrist-patient interactions for training and evaluation purposes. \cite{9} fine-tuned GPT models to simulate both the psychiatrist and patient roles, validated through human evaluations by real psychiatrists and patients. \cite{19} expanded on this concept by focusing on multi-turn dialogues for psychological counseling. Utilizing a range of models, including ChatGPT, LLaMA, and others, the study successfully simulated realistic and emotionally resonant counseling conversations. 
% % These dialogues were evaluated for comprehensiveness, professionalism, and authenticity, demonstrating the effectiveness of LLMs in generating high-quality therapeutic interactions.
% \cite{26} evaluated the efficacy of LLMs in counseling high-functioning autistic adolescents, using them to simulate therapeutic interactions, focusing on empathy and adaptability, which resulted in improved communication skills and psychological well-being among the participants. \cite{29} developed a conversational AI therapist aimed at daily functioning screening and psychotherapeutic intervention, with measures of empathy, communication skills, and therapeutic alliance indicating positive outcomes. A few studies focused on the CBT framework: \cite{77} created a multi-turn dialogue dataset specifically designed for CBT-based psychological counseling. This dataset facilitated the training of LLMs to simulate CBT interactions that adhered closely to CBT principles, as confirmed by evaluations focused on adherence and conversational quality. \cite{97} evaluated whether LLMs could effectively conduct CBT sessions, utilizing various models to simulate CBT interventions, demonstrating that LLMs could deliver therapeutic content effectively with appropriate guidance, as assessed by human evaluations for content accuracy, empathy, and therapeutic alliance. \cite{30} conducted a comparative study using Socratic questioning to generate CBT responses, comparing models like OsakaED and GPT-4. In a related study, \cite{40} introduced a CBT-based conversational counseling agent that utilizes memory systems and dynamic prompting to address cognitive distortions. The effectiveness of this agent was confirmed through human evaluation, which assessed CBT validity, emotional support, and stability in the conversations generated by the model. Similarly, \cite{97} explored the feasibility of LLMs like ChatGPT-3.5-turbo, ERNIE-3.5-8K, iFlytek Spark V3.0, and ChatGLM-3-turbo in conducting CBT, evaluating their performance based on emotion tendency, structured dialogue pattern, and proactive questioning ability, both with and without the integration of a CBT-specific knowledge base.

% Several studies have also focused on developing frameworks for assessing the performance of LLM therapists. \cite{32} uses simulated clients to evaluate the client-centeredness of LLM therapists, particularly focusing on session outcomes and therapeutic alliance. Furthering this line of research, \cite{37} developed a computational framework for behavioral assessment of LLM therapists, comparing LLM-generated behaviors with those of high- and low-quality human therapy. This framework successfully differentiated between therapeutic behaviors of varying quality, using in-context learning to measure psychotherapy techniques.
 
% \textbf{As Patients.} 
% When LLMs are used to simulate therapists, there's a risk that users might receive inaccurate or inappropriate guidance. In contrast, simulating patients with LLMs reduces these risks by allowing trainees to interact with a wide variety of patient scenarios without depending on potentially flawed therapeutic advice from the model. For example, \cite{51} uses GPT-4 Turbo to simulate patient interactions for the training of mental health professionals. This approach successfully trained trainees in formulating cognitive models, which were assessed by experts and trainees for accuracy and realism.

% Other studies have used LLMs to simulate both therapists and patients. \cite{9} explores the use of a prompt-tuned version of ChatGPT to simulate both psychiatrist and patient roles, enhancing training for psychiatric professionals by providing realistic and empathetic interactions. This study reported high ratings in empathy, fluency, and diagnostic accuracy, as evaluated by real psychiatrists and patients. Similarly, \cite{32} assessed the performance of LLM therapists through client simulations, focusing on client-centered approaches. They simulated both clients and therapists, with assessments based on session outcomes and client-centered questionnaires. Additionally, \cite{56} used GPT-3.5-turbo to simulate and evaluate MI sessions by generating digital patients and therapists with varying skill levels. This method allowed for step-by-step interactions, effectively distinguishing therapist expertise and session quality, and the study measured reliability and validity through questionnaires and expert reviews.

% \textbf{As Partners, Coach, etc.} One study \cite{21} explored using LLMs as partners and coaches to enhance interpersonal effectiveness. Utilizing GPT-3.5 to facilitate CBT sessions and guide interactive exercises, the study found that daily interactions with the LLM reduced anxiety and improved mental health. Weekly therapeutic outcomes and pre-/post-session anxiety measures confirmed the LLM's effectiveness as a supportive partner in providing timely feedback and personalized coaching. Another study \cite{55} explored the use of GPT-4 in developing a journaling app designed for psychiatric patients to facilitate daily reflections. The app prompts users with tailored questions and provides summaries of their entries to support consistent journaling. A four-week field study involving patients diagnosed with major depressive disorder evaluated the app's effectiveness by analyzing the frequency and quality of patient interactions, supplemented by interviews with both patients and mental health professionals.

% \subsubsection{Auto-evaluation of Therapy Sessions}
% LLMs have also been used as auto-evaluators of therapy sessions. \cite{39} used GPT-4 to automatically evaluate the therapeutic working alliance in counseling sessions using methodologies such as Chain-of-Thought prompting, detailed guidelines, and metrics like self-consistency and intra-class correlation among annotators, demonstrating a high degree of alignment with human evaluations. \cite{56} used GPT-3.5 to fill two questionnaires (satisfaction and working alliance) after simulated therapy sessions. Its evaluation ability was validated by demonstrating high internal consistency, effective discrimination between different therapist skill levels, and strong alignment with human expert ratings. Another study \cite{70} used GPT-4 as a zero-shot classifier to evaluate the quality of reflections, where Cohen-Kappa score for inter-rater reliability between GPT-4 and human reviewers.



















\section{Current Landscape}
\label{sec:landscape}


\begin{figure*}[htb]
\centering
\includegraphics[width=1\linewidth]{landscape.pdf}
    \caption{Distribution analysis of the current landscape.}
    \vspace{-3mm}
    \label{fig:data}
\end{figure*}
% subsection 1: 
Our survey encompasses a total of 69 studies in the field of LLMs in psychotherapy. Specifically, 33 studies address assessment, 9 focus on diagnosis, and 32 concentrate on treatment, with 5 studies overlapping across these dimensions. Approximately 74\% of the studies employed commercial large language models, while about 77\% used prompt-based techniques. This distribution highlights an imbalance in research focus across different stages of the psychotherapy process and reflects a heavy reliance on commercial models and prompt technologies.
% 讨论这三方面的的文章比重(A 33篇,D 9篇,T 32篇)一共69篇,5篇涉及重叠
% 模型51(共69)篇论文使用商业模型与53(共69)篇论文使用技术的占比 
% Table

% Subsection 2:
Figure \ref{fig:data} presents a comprehensive analysis of the current research landscape in this field. 
% 讨论语言覆盖范围
Panel \textbf{(a)} reveals a significant linguistic bias in existing studies, with English-language corpora dominates. 
While there are limited studies involving Korean and Dutch languages, this highlights a substantial gap in multilingual research approaches.
% 讨论心理障碍覆盖范围
Panel \textbf{(b)} quantitatively demonstrates the distribution of mental health research focuses. Mental disorder-related studies constitute 32\% of the total research corpus (represented by the orange outer ring).
Within this subset, depression-focused research accounts for 50\% of mental disorder studies, followed by anxiety-related research. This distribution indicates a concerning imbalance, where common conditions receive disproportionate attention while more complex disorders, such as bipolar disorder, remain understudied.
% 讨论心理治疗理论的占比
The analysis of psychotherapy theories in panel \textbf{(c)} uncovers another critical gap in the field. Only 32.8\% of the studies incorporate psychotherapy theories in their methodological approach. 
Notably, emerging therapeutic frameworks, such as humanistic therapy, are particularly underrepresented in current research applications.
% This finding suggests a need for more diverse theoretical approaches in future studies.




\section{Future Directions}
\label{sec:future}
\paragraph{Integrative Psychotherapy Framework.} While many existing studies focus on a single dimension of psychotherapy, real-world practice involves a continuous process that spans assessment, diagnosis, and intervention~\cite{Section2:5}. Moreover, these stages typically unfold over multiple sessions, necessitating iterative, multi-turn interactions that incorporate the evolving context of each patient. Future work could therefore aim to develop an end-to-end conversational framework that seamlessly spans from initial evaluation to personalized intervention. By maintaining a system grounded on ongoing, context-sensitive engagement, models could dynamically update assessments and diagnoses over time, ultimately providing more responsive and individualized care.

\paragraph{Addressing Evolving and Multifaceted Nature of Psychotherapy.} Psychotherapy commonly involves shifting symptoms, comorbidities, and nuanced patient experiences, making static or single-label predictions insufficient. Models should integrate multi-label and temporal data to capture how symptoms and emotional states evolve, while avoiding the pitfalls of incomplete symptom detection. For instance, focusing solely on the depressive features of a bipolar patient could lead to an inaccurate diagnosis if the manic phase is overlooked~\cite{lee-etal-2024-detecting-bipolar}. Furthermore, current research suggests that LLMs often struggle with multi-label tasks~\cite{22,96}, highlighting the need for improved model architectures and algorithms that can better account for these complexities.

\paragraph{Resource Infrastructure and Open-Source Tools.} Current research heavily relies on commercial closed-source models, lacking reproducible open-source evaluation methods and multilingual data. Notably, developing multilingual datasets should not solely rely on translating English resources, as psychological research indicates that cultural context plays a critical role in mental health. English-based translations cannot fully substitute for culturally specific data from other languages~\cite{watters2010crazy, abdelkadir-etal-2024-diverse}.

\paragraph{Broadening Scope of Disorders and Therapeutic Approaches.} Most studies to date have concentrated on prevalent conditions such as depression and anxiety, leaving complex or less common disorders underexplored. Additionally, research tends to focus on a limited range of therapeutic modalities—primarily cognitive behavioral therapy. Future work could broaden both the range of disorders and the variety of therapeutic approaches, such as humanistic~\cite{schneider2010existential} and dialectical behavior therapy~\cite{lynch2006mechanisms}, to better reflect clinical realities~\cite{norcross2022predicted}. Such an expansion could deepen the theoretical underpinnings of LLM-based psychotherapy tools and enhance the quality and relevance of digital interventions.

\section{Conclusion}
In this study, we introduce \ours, a novel framework designed to achieve lossless acceleration in generating ultra-long sequences with \acp{llm}. By analyzing and addressing three challenges, \ours significantly enhances the efficiency of the generation process. Our experimental results demonstrate that \ours achieves over $3\times$ acceleration across various model scales and architectures. Furthermore, \ours effectively mitigates issues related to repetitive content, ensuring the quality and coherence of the generated sequences. These advancements position \ours as a scalable and effective solution for ultra-long sequence generation tasks.


\section*{Limitations}
This survey paper, while comprehensive for LLMs in psychotherapy, has several limitations: 
1) The studies reviewed primarily focus on the application of LLMs in psychotherapy, and there may be relevant research in adjacent fields or interdisciplinary domains that was not included. 
2) Due to the rapidly evolving nature of this area, some recent advancements may not be captured. The scope of this survey is limited to the available literature and may overlook emerging trends or unpublished findings. 
3) The review primarily examines studies in English, which could introduce a bias towards research from English-speaking countries, potentially overlooking important cultural perspectives. 
4) While we provide a taxonomy of LLM applications in psychotherapy, this framework may not fully encompass the complexity of real-world clinical settings or the diverse range of therapeutic approaches currently in practice.
% Bibliography entries for the entire Anthology, followed by custom entries
%\bibliography{anthology,custom}
% Custom bibliography entries only
\bibliography{custom}

\appendix

% \section{Example Appendix}
% \label{sec:appendix}

% This is an appendix.

\end{document}

% Fine-tuning和Instruction-Tuning需要区分吗
% 基本所有工作均未注明其方法,并使用多种Approach,table注明的是表现最好的
% Conceptual Taxonomy分类方式,参考:https://arxiv.org/abs/2403.14659,但纠结的点在于除了Definition外,另一个小section写什么内容,如果完全参考那篇的Significance,但有和Interrelations有重复内容的感觉;
% 