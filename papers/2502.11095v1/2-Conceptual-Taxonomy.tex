\section{Conceptual Taxonomy}
\label{sec:taxonomy}



To establish a standardized framework for understanding psychotherapy, we propose a hierarchical taxonomy aligned with the American Psychological Association's tripartite model of psychotherapeutic processes\footnote{\url{https://www.apa.org/topics/psychotherapy}}. As illustrated in Figure \ref{fig:Taxonomy}, this taxonomy organizes psychotherapy into three core components: (1) Assessment, (2) Diagnosis, and (3) Treatment, with dynamic interconnections\footnote{Throughout this taxonomy, the terms \textit{Assessment}, \textit{Diagnosis}, and \textit{Treatment} specifically refer to the three core components of psychotherapy.}. Each component is detailed below.

\subsection{Assessment}

\paragraph{Definition.} Psychological assessment constitutes the systematic collection and interpretation of data regarding an individual's cognitive, emotional, and behavioral functioning~\cite{cohen1996psychological, kaplan2001psychological}. This process employs psychometric tests, structured clinical interviews, behavioral observations, and collateral information to establish a multidimensional profile of psychological states~\cite{groth2009handbook}.


\paragraph{Significance.} As the foundational stage of psychotherapy, assessment provides the empirical basis for understanding a client's unique psychological landscape. It enables therapists to identify symptom patterns~\cite{phillips2007assessing}, track temporal changes~\cite{barkham1993shape}, and contextualize subjective experiences within objective frameworks~\cite{groth2009handbook}. The continuous nature of psychological assessment allows for real-time adjustments to therapeutic strategies~\cite{schiepek2016real}, ensuring interventions remain responsive to evolving client needs.

\subsection{Diagnosis} 

\paragraph{Definition.} Diagnosis represents the analytical process of categorizing psychological distress using established nosological systems such as the DSM-5~\cite{Section2:11} and ICD-11~\cite{ICD11}. This involves differentiating normative emotional responses from pathological conditions while considering cultural~\cite{AlanCu2010} and developmental~\cite{kawade2012} variables that influence symptom manifestation.

\paragraph{Significance.} Diagnosis serves as the conceptual bridge between assessment and treatment, providing a structured framework for intervention planning~\cite{jensen2011understanding}. By aligning clinical observations with standardized criteria, it enhances communication among professionals~\cite{craddock2014psychiatric} and facilitates evidence-based decision-making~\cite{apa2006evidence}. 

\subsection{Treatment}

\paragraph{Definition.} Treatment includes evidence-based interventions designed to reduce psychological distress and improve functioning~\cite{apa2006evidence}. These interventions work by building a therapeutic alliance~\cite{elvins2008conceptualization}, restructuring cognition~\cite{ezawa2023cognitive}, and modifying behavior~\cite{martin2019behavior}, all typically grounded in well-established theoretical orientations.

\paragraph{Significance.} Treatment transforms the theories and information gleaned from assessment and diagnosis into practical interventions~\cite{prochaska2018systems} that directly address the client's psychological distress~\cite{barlow2021clinical} and foster personal growth~\cite{lambert2013bergin}. 

\subsection{Interrelations}
The taxonomy's components interact through three dynamic processes that define psychotherapy as a complex adaptive system:

\paragraph{Synthesizing (Assessment $\rightarrow$ Diagnosis)} The dialectical integration of observational data with nosological frameworks enables diagnostic classifications to contextualize assessment findings, \textit{synthesizing} the patient's various symptoms and behavioral patterns into a diagnostic result~\cite{rencic2016understanding}.

\paragraph{Framing (Diagnosis $\rightarrow$ Treatment)} Diagnosis functions as a \textit{framing} mechanism, integrating complex and diverse symptoms into a coherent classification that establishes a clear blueprint for treatment~\cite{Section2:11}.

\paragraph{Customization (Assessment $\rightarrow$ Treatment)}  A process where treatment plans are continuously \textit{refined} based on assessment results, considering individual differences without being constrained by diagnostic labels, to enhance therapeutic effectivenesss~\cite{Section2:5}.

%The dynamic optimization process where therapeutic techniques continuously \textit{adapt} to assessment-derived insights, while ongoing evaluation \textit{refines} interventions to enhance personalization and effectiveness\cite{Section2:5}.

% \begin{figure}[htbp]
% \centering
% \includegraphics[width=0.5\textwidth]{2.4 relation.png}
% \caption{The dynamic and interrelated network among assessment, diagnosis, and treatment}
% \label{fig:Taxonomy}
% \end{figure}



% Simplified version:

% In traditional medicine, assessment leads to diagnosis, which then informs treatment—a linear progression effective for physiological diseases where diagnoses correspond directly to specific pathologies and interventions \cite{Section2:1}. However, in psychology, this linear relationship is less applicable. Psychological diagnoses rely on clinical observations rather than definitive biological tests \cite{Section2:2}. Consequently, social and psychological burdens are often introduced by the use of diagnostic labels in psychotherapy. For instance, stigma and discrimination can be exacerbated when someone is labeled as "schizophrenic" instead of "a person with schizophrenia" \cite{Section2:3}. This has led some professionals to question the necessity of diagnosis in treatment.

% In psychotherapy, assessment, diagnosis, and treatment form a dynamic, interconnected network rather than a linear sequence (see Figure \ref{fig:relationship}). Assessment \textbf{synthesizes} the patient's various symptoms and behavior patterns into a diagnostic result, thereby providing foundational data for subsequent treatment. Treatment can be \textbf{customized} directly from assessment results without strict reliance on diagnostic labels \cite{Section2:4}, and assessment continues throughout treatment to monitor and adjust plans as needed.

% \begin{enumerate}
% \item \textbf{Treatment through diagnosis:} Assessment results are used to determine the patient's diagnosis. A clear \textbf{frame} for treatment is provided by the diagnostic label, and guiding treatment planning based on established guidelines.

% \item \textbf{Direct treatment based on assessment: } Treatment plans are customized directly based on assessment results, considering individual differences without being constrained by diagnostic labels \cite{Section2:5}.
% \end{enumerate}

% However, the first path offers structure but has limitations. Psychiatric diagnoses can be ambiguous and heterogeneous; patients with the same diagnosis may exhibit different symptoms \cite{Section2:9}. Additionally, Diagnostic criteria vary across cultures \cite{Section2:7} and evolve over time (e.g., multiple revisions of the DSM) \cite{Section2:8}, reducing their stability. Finally, over-reliance on diagnosis can also undermine the therapeutic relationship built on trust \cite{Section2:10}.

% Given these limitations, prioritizing the second path—bypassing diagnosis and directly formulating personalized treatment plans based on assessment—is recommended. This symptom-oriented approach enhances intervention specificity and effectiveness, addressing ambiguities inherent in mental illness diagnosis \cite{Section2:6}.

% In conclusion, the interactive relationship among assessment, diagnosis, and treatment is crucial in psychotherapy. While diagnosis can provide a helpful framework to treatment, its limitations make it necessary, in certain situations, to proceed directly to treatment based on assessment. Directly basing treatment on assessment allows for more precise and personalized care, better suited to the complexities of mental health.

%  In traditional medicine, assessment, diagnosis, and treatment typically follow a linear progression. Assessment information—including the patient's physical signs, laboratory test results, and imaging data—is analyzed to arrive at a diagnosis, upon which an appropriate treatment plan is formulated. This approach is effective for physiological diseases because diagnoses often directly correspond to specific pathological processes and therapeutic interventions \cite{Section2:1}.

% However, in psychology, the nature and function of diagnosis differ, and this linear relationship is not always applicable. Psychological diagnoses rely more on clinical observation of behavior rather than definitive biological tests \cite{Section2:2}. Consequently, social and psychological burdens are often introduced by the use of diagnostic labels in psychotherapy. For instance, stigma and discrimination can be exacerbated when someone is labeled as "schizophrenic" instead of "a person with schizophrenia" \cite{Section2:3}. This has led some psychologists and researchers to question the necessity of diagnosis in treatment.

% In clinical practice, the relationship among assessment, diagnosis, and treatment forms a dynamic, interrelated complex network rather than a linear sequence.  Figure \ref{fig:relationship} illustrates that assessment, as the starting point of the treatment process, \textbf{synthesizes} the patient's various symptoms and behavior patterns into a diagnostic result, thereby providing foundational data for subsequent treatment. Simultaneously, treatment can be directly \textbf{customized} based on assessment results without strict reliance on diagnostic labels \cite{Section2:4}. In many cases, assessment is continuous throughout the treatment process, used to monitor and adjust the treatment plan to ensure it remains consistent with the patient's actual condition \cite{Section2:4}.

% Therefore, based on assessment results, psychotherapy can follow two distinct paths:

% \begin{enumerate}
% \item \textbf{Formulating a treatment plan through diagnosis:} In this approach, assessment results are used to determine the patient's diagnosis. A clear frame for treatment is provided by the diagnostic label, and corresponding treatment plans are formulated based on the diagnosis. This method standardizes the treatment process to a certain extent, allowing clinicians to select treatment options based on established guidelines and experience.

% \item \textbf{Bypassing diagnosis and directly formulating a treatment plan based on assessment results:} By fully considering the patient's individual differences and unique combination of symptoms, treatment plans can be customized for each specific situation \cite{Section2:5}, unrestricted by diagnostic labels. This personalized method may better meet the patient's specific needs.
% \end{enumerate}

% In the first path, assessment information is structured within this network by diagnosis. Complex and diverse symptoms can be integrated into a clear label or classification by comparing assessment results with established mental health diagnostic standards (e.g., DSM-5)\cite{Section2:6}. This structuring process provides a clear \textbf{frame} for treatment; diagnosis serves as a bridge from assessment to treatment, helping to transform the diverse information collected during assessment into a specific treatment plan.

% However, the diagnosis of mental illnesses often involves ambiguity and uncertainty, especially when confronting diverse combinations of symptoms. The presence of different cultural backgrounds \cite{Section2:7}, temporal changes \cite{Section2:8}, and individual patient differences further exacerbate the heterogeneity and limitations of diagnosis. Firstly, psychiatric diagnoses are not homogeneous. For instance, entirely different combinations of symptoms may be exhibited by patients diagnosed with Major Depressive Disorder (MDD) \cite{Section2:9}. The validity of a single diagnostic label is challenged by this symptom diversity, suggesting that focusing on individual symptoms rather than the overall diagnosis may better aid in understanding and treating patients. Secondly, diagnostic criteria for mental illnesses are not immutable. Diagnostic standards can vary significantly across different cultural contexts \cite{Section2:7}, and over time, these criteria continually evolve (e.g., multiple revisions of the DSM) \cite{Section2:8}. This instability further weakens the value of diagnosis as a core tool. In contrast, symptom manifestations are relatively stable and more suitable as the central focus in clinical practice. Additionally, the diagnostic process is inseparable from the doctor-patient relationship. Diagnosis is not merely a technical procedure but an interpersonal interaction built on trust \cite{Section2:10}. In this process, clinicians are not just executors of treatment but also supporters and guides for patients \cite{Section2:10}. Overreliance on diagnosis may mechanize the treatment process, neglecting trust-based interpersonal interactions, thereby undermining the crucial therapeutic relationship and ultimately affecting treatment outcomes.

% Considering the limitations and heterogeneity of mental illness diagnoses, it is recommended that psychotherapy should prioritize the second path—that is, bypassing the diagnostic stage and directly formulating personalized treatment plans based on assessment results rather than relying on traditional symptom-combination diagnostic labels. This symptom-oriented treatment approach not only enhances the specificity and effectiveness of interventions but also better addresses the common ambiguities and uncertainties inherent in mental illness diagnosis \cite{Section2:6}.

% In conclusion, the interactive relationship among assessment, diagnosis, and treatment is a crucial foundation for ensuring the effectiveness of psychotherapy. Assessment can provide necessary support for diagnosis, aiding in guiding treatment, or can be used to directly inform treatment by bypassing diagnosis. While diagnosis offers a beneficial structural framework for treatment, its limitations make it necessary, in certain situations, to proceed directly to treatment based on assessment. This dynamic and flexible relational network allows psychotherapy to better address complex and evolving clinical realities, providing more precise and personalized patient care.

