\section{Introduction}

% 心理健康在当代医疗与社会福祉中扮演着愈发关键的角色,抑郁、焦虑等常见心理障碍的高发率使得大众对可及、有效的心理治疗需求不断攀升。然而,心理治疗的核心在于\emph{动态、语境化}的人际互动——治疗师需要根据患者的情绪波动、语言表达、社会背景持续地评估、调适干预策略,从而构建稳固的治疗联盟并实现症状改善。这种深度而灵活的过程,与传统NLP仅能在\emph{静态或单一任务}场景下发挥作用的局限形成了鲜明对比。
Mental health plays an increasingly critical role in current healthcare and social well-being.
% Statistics from the World Health Organization\footnote{\url{https://www.who.int/}} (WHO) reveals that around 1/8 individuals experience mental health issues.
The high prevalence of common psychological disorders, such as depression and anxiety, has led to a growing demand for accessible and effective psychological interventions. However, the core of psychotherapy resides in \textit{dynamic, contextual} interpersonal interactions—therapists should continuously assess and adjust their intervention strategies~\cite{wampold2015great} based on the patient’s emotional fluctuations, verbal expressions, and social background, fostering a strong therapeutic alliance~\cite{stubbe2018therapeutic} to achieve symptom resilience. This deep and flexible process contrasts sharply with traditional NLP, which is typically limited to static or single-task settings.

\begin{figure}[t!]
\centering
\includegraphics[width=0.5\textwidth]{2.4.relation.pdf}
\caption{The dynamic and interrelated network among assessment, diagnosis, and treatment in psychotherapy.}
\vspace{-4mm}
\label{fig:Taxonomy}
\end{figure}

Large language models (LLMs) offer a new perspective to addressing this challenge. By leveraging their capability to model extensive context and perform multi-turn reasoning \cite{wang-etal-2024-document-level,NEURIPS2024_e560a0b2}, LLMs can capture rich semantics and emotional signals in dialogues \cite{ma-etal-2025-detecting}, enabling end-to-end language understanding and generation.
In assessment, LLMs can extract potential symptom cues from vague and fragmented expressions~\cite{24,108}. During diagnosis, they integrate subjective and objective patient information across multiple utterances~\cite{9,52}. In therapeutic interventions, they adapt conversational strategies based on patients' real-time feedback, enabling more flexible and human-like interactions compared to traditional scripted systems~\cite{40,78}. As a result, LLMs have the potential to surpass the conventional ``discrete label recognition'' paradigm, evolving toward a model of continuous, progressive clinical reasoning, enabling seamless connections across \textit{assessment}, \textit{diagnosis}, and \textit{treatment}, aligning more closely with therapists' cognitive process and interaction flow.

% 定义三个自定义颜色
\definecolor{AssessmentColor}{HTML}{E7B489}  
\definecolor{DiagnosisColor}{HTML}{8CAFD0}   
\definecolor{TreatmentColor}{HTML}{BECD8F}   


\tikzstyle{my-box}=[
    rectangle,
    rounded corners,
    text opacity=1,
    minimum height=1.5em,
    minimum width=5em,
    inner sep=2pt,
    align=center,
    fill opacity=.5,
]
\tikzstyle{leaf}=[my-box, minimum height=1.5em,
    text=black, align=left,font=\scriptsize,
    inner xsep=2pt,
    inner ysep=4pt,
]

% 定义一个 style 用来给一个节点及其子节点统一着色
\forestset{
  color subtree/.style={
    fill=#1!10,    % 背景颜色
    draw=#1,       % 节点边框线颜色
    edge+={draw=darkgray, line width=1pt},
    for children={
      color subtree=#1   % 递归给后代也应用此配色
    }
  }
}

\begin{figure*}[t]
    \centering
    \resizebox{\textwidth}{!}{
        \begin{forest}
            forked edges,
            for tree={
                grow=east,
                reversed=true,
                anchor=base west,
                parent anchor=east,
                child anchor=west,
                base=left,
                font=\small,
                rectangle,
                draw=hidden-draw,
                rounded corners,
                align=left,
                minimum width=4em,
                edge+={darkgray, line width=1pt},
                s sep=3pt,
                inner xsep=2pt,
                inner ysep=3pt,
                ver/.style={rotate=90, child anchor=north, parent anchor=south, anchor=center},
            },
            where level=1{text width=4.0em,font=\scriptsize,}{},
            where level=2{text width=5.6em,font=\scriptsize,}{},
            where level=3{text width=6.2em,font=\scriptsize,}{},
            where level=4{text width=6.0 em,font=\scriptsize,}{},
            [ LLMs in \\Psychotherapy
                [
                    Assessment,
                    color subtree=AssessmentColor,
                    [
                        Symptoms
                        [
                            Detection
                            [
                                \citet{22,24,27,66,95}; \\ \citet{85,91,94,96}; \\ \citet{108,120,18}
                                , leaf, text width=26.6em
                            ]
                        ]
                        [
                            Severity
                            [
                                \citet{11,23,58,100}; \\ \citet{59}
                                , leaf, text width=26.6em
                            ]
                        ]
                    ]
                    [
                        Cognition
                        [
                           \citet{5,10,12,14,16,38}\\ \citet{45}
                            , leaf, text width=34.5em
                        ]
                    ]
                    [
                        Behavior
                        [
                            \citet{36,60,65,67}
                            , leaf, text width=34.5em
                        ]
                    ]
                    [
                        Advanced
                        [
                            \citet{48,53,99}
                            , leaf, text width=34.5em
                        ]
                    ]
                ]
                [
                    Diagnosis,
                    color subtree=DiagnosisColor
                    [
                        Static Diagnosis
                        [
                            \citet{11,107,102,124,110}
                            , leaf, text width=34.5em
                        ]
                    ]
                    [
                        Dynamic Diagnosis
                        [
                            \citet{9,33,52,106}
                            , leaf, text width=34.5em
                        ]
                    ]
                ]
                [
                    Treatment,
                    color subtree=TreatmentColor
                    [
                        LLM as a Virtual \\ Therapist
                        [
                            \citet{20,29,40,54,55}; \\ \citet{77,78,109}
                            , leaf, text width=34.5em
                        ]
                    ]
                    [
                        LLM as an Assistive \\Tool
                        [
                            \citet{2,3,5,35,43}; \\ \citet{64,70,81}
                            , leaf, text width=34.5em
                        ]
                    ]
                    [
                        LLM as Simulated \\Patients for Clinician \\Education
                        [
                            \citet{49,51,56,80}
                            , leaf, text width=34.5em
                        ]
                    ]
                    [
                        LLM for Evaluation \\and Quality Analysis
                        [
                            \citet{4,19,32,37,39,62}; \\ \citet{65,67,101,103,122}
                            , leaf, text width=34.5em
                        ]
                    ]
                ]
            ]
        \end{forest}
    }
    \caption{Taxonomy of Research on Large Language Models in Psychotherapy.}
    \vspace{-4mm}
    \label{fig:overall}
\end{figure*}


However, existing research on applying LLMs in this field remains somewhat \textit{disjointed}. Many studies have utilized LLMs for isolated tasks, such as depression detection~\cite{18,27} or diagnosis~\cite{107}, regarding them as superior feature extractors. Another research line has focused on developing mental health counseling chatbots \cite{chen-etal-2023-soulchat,19}; however, these systems remain limited to partial assistance due to insufficient integration with clinical workflows. In other words, although LLMs hold the potential to span the entire continuum from assessment to intervention, they remain limited by the fragmented paradigms of traditional NLP, preventing them from fully leveraging their dynamic, contextual capabilities.

To address these gaps, we introduce the first \textit{taxonomy} that divides the psychotherapy process into three essential dimensions: \textbf{Assessment}, \textbf{Diagnosis}, and \textbf{Treatment} and provides a systematic review of the recent advancements and challenges of LLMs in each stage. We further examine the current landscape from various perspectives, including the coverage of mental disorders, diversity of linguistic resources, alignment with psychotherapy theories, and the types of techniques employed, thereby sketching the overall distribution and characteristics of existing research. Building on this foundation, we discuss key challenges for the future, including issues of technical coherence, resource and language imbalances, and the disconnect between LLM-based approaches and established psychological practices. Through this \textit{comprehensive review} and \textit{framework}, we aim to offer methodological insights to inform future research and facilitate the practical integration of intelligent systems across the entire psychotherapy process.
% summarize and discuss


\paragraph{Organization of This Survey.} We present the first comprehensive survey of recent advancements in applying LLMs to psychotherapy. We introduce a conceptual taxonomy that organizes psychotherapy into three core components—Assessment, Diagnosis, and Treatment—and details their dynamic interrelations (Section \S\ref{sec:taxonomy}). We review how LLMs are applied within these components, highlighting their roles in facilitating assessments, refining diagnostic processes, and enhancing treatment strategies (Section \S\ref{sec:llms}). We examine current research trends, including symptom and language coverage as well as the distribution of various models and techniques (Section \S\ref{sec:landscape}). Finally, we discuss open challenges and outline promising directions for future work (Section \S\ref{sec:future}).

% 在大型语言模型(LLMs)出现之前,NLP 在心理治疗领域主要以碎片化形态存在:基于规则的诊断系统固化在预设路径中,无法针对不同患者的细微变化作出自适应调整;传统机器学习模型常聚焦于特定症状的预测或风险分级,缺乏对\emph{评估—诊断—干预}完整链条的覆盖;而早期对话系统大多依赖脚本式答复,难以捕捉患者复杂的认知和情感线索。这些方法虽在大规模筛查或常见障碍识别方面带来了效率提升,但在动态交互、情境感知、情绪理解等更深层次的\emph{心理治疗的核心需求}上却显得力不从心。

% 大型语言模型则为这一困境带来了新的视角。借助对海量文本的上下文建模与多回合推理能力,LLMs能够在对话中捕捉丰富的语义与情感信号,从而实现\emph{端到端}的语言理解与生成:在评估环节,它们可以从模糊隐喻或零散表述中提炼潜在症状线索;在诊断阶段,则能跨多段话语综合患者主客观信息;在治疗干预中,更可根据患者即时反馈动态调整对话策略,模拟出比传统脚本系统更具灵活性和人性化的交互模式。理论上,LLMs 有望突破传统的“离散标签识别”范式,向“连续、渐进的临床推理”演进,使得\textbf{从评估到诊断,再到干预}的各环节能够紧密衔接,从而更贴近治疗师的实际思路与互动节奏。

% 然而,当前研究对 LLM 的应用仍较为\emph{割裂}。许多工作仅将其用于抑郁检测、情绪分析等单一任务,将 LLM 视作性能更优的特征提取器;另一些研究则着力构建对话机器人,但因缺乏对临床流程的系统耦合,只能停留在部分辅助。换言之,LLM 虽然具备从评估到干预的\emph{跨环节}潜力,却仍被传统 AI 的碎片化范式所束缚,没有充分发挥其“动态、情境化”能力。

%为弥合这些空白,\textbf{本文}提出并梳理了一个\emph{Taxonomy},将心理治疗流程划分为\textbf{Assessment(评估)}、\textbf{Diagnosis(诊断)}与\textbf{Treatment(干预)}三个核心维度,并系统综述了 LLM 在这三大环节的最新进展与挑战。我们进一步从心理障碍覆盖度、语言资源多样性、心理治疗理论对照、以及模型技术类型等角度展开分析,勾勒当前研究的整体分布与特征。在此基础上,我们总结并讨论了未来所面临的关键问题,包括技术连贯性、资源与语言不均衡、LLM 理论参考与实际常用心理方法间的不匹配等。通过这一\emph{面向 LLM 时代}的综合性综述与框架建构,希冀能为后续研究带来更具系统性的方法论启示,推动智能系统在心理治疗全流程中的真实落地。