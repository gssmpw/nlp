\section{Future Directions}
\label{sec:future}
\paragraph{Integrative Psychotherapy Framework.} While many existing studies focus on a single dimension of psychotherapy, real-world practice involves a continuous process that spans assessment, diagnosis, and intervention~\cite{Section2:5}. Moreover, these stages typically unfold over multiple sessions, necessitating iterative, multi-turn interactions that incorporate the evolving context of each patient. Future work could therefore aim to develop an end-to-end conversational framework that seamlessly spans from initial evaluation to personalized intervention. By maintaining a system grounded on ongoing, context-sensitive engagement, models could dynamically update assessments and diagnoses over time, ultimately providing more responsive and individualized care.

\paragraph{Addressing Evolving and Multifaceted Nature of Psychotherapy.} Psychotherapy commonly involves shifting symptoms, comorbidities, and nuanced patient experiences, making static or single-label predictions insufficient. Models should integrate multi-label and temporal data to capture how symptoms and emotional states evolve, while avoiding the pitfalls of incomplete symptom detection. For instance, focusing solely on the depressive features of a bipolar patient could lead to an inaccurate diagnosis if the manic phase is overlooked~\cite{lee-etal-2024-detecting-bipolar}. Furthermore, current research suggests that LLMs often struggle with multi-label tasks~\cite{22,96}, highlighting the need for improved model architectures and algorithms that can better account for these complexities.

\paragraph{Resource Infrastructure and Open-Source Tools.} Current research heavily relies on commercial closed-source models, lacking reproducible open-source evaluation methods and multilingual data. Notably, developing multilingual datasets should not solely rely on translating English resources, as psychological research indicates that cultural context plays a critical role in mental health. English-based translations cannot fully substitute for culturally specific data from other languages~\cite{watters2010crazy, abdelkadir-etal-2024-diverse}.

\paragraph{Broadening Scope of Disorders and Therapeutic Approaches.} Most studies to date have concentrated on prevalent conditions such as depression and anxiety, leaving complex or less common disorders underexplored. Additionally, research tends to focus on a limited range of therapeutic modalities—primarily cognitive behavioral therapy. Future work could broaden both the range of disorders and the variety of therapeutic approaches, such as humanistic~\cite{schneider2010existential} and dialectical behavior therapy~\cite{lynch2006mechanisms}, to better reflect clinical realities~\cite{norcross2022predicted}. Such an expansion could deepen the theoretical underpinnings of LLM-based psychotherapy tools and enhance the quality and relevance of digital interventions.