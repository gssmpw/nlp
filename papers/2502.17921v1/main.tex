%%
%% This is file `sample-sigconf.tex',
%% generated with the docstrip utility.
%%
%% The original source files were:
%%
%% samples.dtx  (with options: `all,proceedings,bibtex,sigconf')
%% 
%% IMPORTANT NOTICE:
%% 
%% For the copyright see the source file.
%% 
%% Any modified versions of this file must be renamed
%% with new filenames distinct from sample-sigconf.tex.
%% 
%% For distribution of the original source see the terms
%% for copying and modification in the file samples.dtx.
%% 
%% This generated file may be distributed as long as the
%% original source files, as listed above, are part of the
%% same distribution. (The sources need not necessarily be
%% in the same archive or directory.)
%%
%%
%% Commands for TeXCount
%TC:macro \cite [option:text,text]
%TC:macro \citep [option:text,text]
%TC:macro \citet [option:text,text]
%TC:envir table 0 1
%TC:envir table* 0 1
%TC:envir tabular [ignore] word
%TC:envir displaymath 0 word
%TC:envir math 0 word
%TC:envir comment 0 0
%%
%%
%% The first command in your LaTeX source must be the \documentclass
%% command.
%%
%% For submission and review of your manuscript please change the
%% command to \documentclass[manuscript, screen, review]{acmart}.
%%
%% When submitting camera ready or to TAPS, please change the command
%% to \documentclass[sigconf]{acmart} or whichever template is required
%% for your publication.
%%
%%

% \documentclass[manuscript]{acmart}
% \documentclass[sigconf]{acmart}
\documentclass[sigconf]{acmart}
% \documentclass[sigconf,natbib=true,review,anonymous=true]{acmart}
\usepackage{multirow}
% \usepackage{longtable}
\usepackage{graphicx}
\usepackage{enumitem}
%% \BibTeX command to typeset BibTeX logo in the docs
\AtBeginDocument{%
  \providecommand\BibTeX{{%
    Bib\TeX}}}

%% Rights management information.  This information is sent to you
%% when you complete the rights form.  These commands have SAMPLE
%% values in them; it is your responsibility as an author to replace
%% the commands and values with those provided to you when you
%% complete the rights form.
% \setcopyright{acmlicensed}
% \copyrightyear{2018}
% \acmYear{2018}
% \acmDOI{XXXXXXX.XXXXXXX}

%% These commands are for a PROCEEDINGS abstract or paper.
% \acmConference[Conference acronym 'XX]{Make sure to enter the correct
%   conference title from your rights confirmation emai}{June 03--05,
%   2018}{Woodstock, NY}
%%
%%  Uncomment \acmBooktitle if the title of the proceedings is different
%%  from ``Proceedings of ...''!
%%
%%\acmBooktitle{Woodstock '18: ACM Symposium on Neural Gaze Detection,
%%  June 03--05, 2018, Woodstock, NY}
% \acmISBN{978-1-4503-XXXX-X/18/06}


%%
%% Submission ID.
%% Use this when submitting an article to a sponsored event. You'll
%% receive a unique submission ID from the organizers
%% of the event, and this ID should be used as the parameter to this command.
%%\acmSubmissionID{123-A56-BU3}

%%
%% For managing citations, it is recommended to use bibliography
%% files in BibTeX format.
%%
%% You can then either use BibTeX with the ACM-Reference-Format style,
%% or BibLaTeX with the acmnumeric or acmauthoryear sytles, that include
%% support for advanced citation of software artefact from the
%% biblatex-software package, also separately available on CTAN.
%%
%% Look at the sample-*-biblatex.tex files for templates showcasing
%% the biblatex styles.
%%

%%
%% The majority of ACM publications use numbered citations and
%% references.  The command \citestyle{authoryear} switches to the
%% "author year" style.
%%
%% If you are preparing content for an event
%% sponsored by ACM SIGGRAPH, you must use the "author year" style of
%% citations and references.
%% Uncommenting
%% the next command will enable that style.
%%\citestyle{acmauthoryear}

\copyrightyear{2025}
\acmYear{2025}
\setcopyright{cc}
\setcctype{by-nd}
\acmConference[WWW '25]{Proceedings of the ACM Web Conference 2025}{April 28-May 2, 2025}{Sydney, NSW, Australia}
\acmBooktitle{Proceedings of the ACM Web Conference 2025 (WWW '25), April 28-May 2, 2025, Sydney, NSW, Australia}
\acmDOI{10.1145/3696410.3714528}
\acmISBN{979-8-4007-1274-6/25/04}
% 1 Authors, replace the red X's with your assigned DOI string during the rightsreview eform process.
% 2 Your DOI link will become active when the proceedings appears in the DL.
% 3 Retain the DOI string between the curly braces for uploading your presentation video.

\settopmatter{printacmref=true}






%%
%% end of the preamble, start of the body of the document source.
\begin{document}

%%
%% The "title" command has an optional parameter,
%% allowing the author to define a "short title" to be used in page headers.
% \title{Equal Lights, Diverse Camera, Fair Action!}

\title{Unmasking Gender Bias in Recommendation Systems and Enhancing Category-Aware Fairness}

%%
%% The "author" command and its associated commands are used to define
%% the authors and their affiliations.
%% Of note is the shared affiliation of the first two authors, and the
%% "authornote" and "authornotemark" commands
%% used to denote shared contribution to the research.
\author{Tahsin Alamgir Kheya}
% \authornote{Both authors contributed equally to this research.}
\email{t.kheya@deakin.edu.au}
\orcid{0009-0001-2481-2877}
\affiliation{%
  \institution{Deakin University}
  \city{Geelong}
  \state{VIC}
  \country{Australia}
}

\author{Mohamed Reda Bouadjenek}
\email{reda.bouadjenek@deakin.edu.au}
\orcid{0000-0003-1807-430X}
\affiliation{%
  \institution{Deakin University}
  \city{Geelong}
  \state{VIC}
  \country{Australia}}

\author{Sunil Aryal}
\email{sunil.aryal@deakin.edu.au}
\orcid{0000-0002-6639-6824}
\affiliation{%
  \institution{Deakin University}
  \city{Geelong}
  \state{VIC}
  \country{Australia}
}



%%
%% By default, the full list of authors will be used in the page
%% headers. Often, this list is too long, and will overlap
%% other information printed in the page headers. This command allows
%% the author to define a more concise list
%% of authors' names for this purpose.
\renewcommand{\shortauthors}{Tahsin Alamgir Kheya, Mohamed Reda Bouadjenek, and Sunil Aryal}

%%
%% The abstract is a short summary of the work to be presented in the
%% article.

\begin{abstract}
Recommendation systems are now an integral part of our daily lives. We rely on them for tasks such as discovering new movies, finding friends on social media, and connecting job seekers with relevant opportunities.
Given their vital role, we must ensure these recommendations are free from societal stereotypes. 
Therefore, evaluating and addressing such biases in recommendation systems is crucial. 
Previous work evaluating the fairness of recommended items fails to capture certain nuances as they mainly focus on comparing performance metrics for different sensitive groups.
In this paper, we introduce a set of comprehensive metrics for quantifying gender bias in recommendations.
Specifically, we show the importance of evaluating fairness on a more granular level, which can be achieved using our metrics to capture gender bias using categories of recommended items like genres for movies. Furthermore, we show that employing a category-aware fairness metric as a regularization term along with the main recommendation loss during training can help effectively minimize bias in the models' output.
% We chose one of these metrics to offer a new fairness regularization term that, when optimized along with the main recommendation loss, can help effectively minimize bias in the models' output. 
We experiment on three real-world datasets, using five baseline models alongside two popular fairness-aware models, to show the effectiveness of our metrics in evaluating gender bias. 
Our metrics help provide an enhanced insight into bias in recommended items compared to previous metrics. Additionally, our results demonstrate how incorporating our regularization term significantly improves the fairness in recommendations for different categories without substantial degradation in overall recommendation performance.
% \noindent 
% {\bf Keywords:} Fairness Metrics, Bias in Recommendations, 
% Societal Stereotypes
\end{abstract}

%%
%% The code below is generated by the tool at http://dl.acm.org/ccs.cfm.
%% Please copy and paste the code instead of the example below.
%%
\begin{CCSXML}
<ccs2012>
   <concept>
       <concept_id>10010147.10010178</concept_id>
       <concept_desc>Computing methodologies~Artificial intelligence</concept_desc>
       <concept_significance>500</concept_significance>
       </concept>
   <concept>
       <concept_id>10002951.10003317.10003331.10003271</concept_id>
       <concept_desc>Information systems~Personalization</concept_desc>
       <concept_significance>500</concept_significance>
       </concept>
   <concept>
       <concept_id>10002951.10003317.10003347.10003350</concept_id>
       <concept_desc>Information systems~Recommender systems</concept_desc>
       <concept_significance>500</concept_significance>
       </concept>
 </ccs2012>
\end{CCSXML}

\ccsdesc[500]{Computing methodologies~Artificial intelligence}
\ccsdesc[500]{Information systems~Personalization}
\ccsdesc[500]{Information systems~Recommender systems}
% \ccsdesc[500]{Computing methodologies → Artificial intelligence → Fairness, accountability, and transparency}

%%
%% Keywords. The author(s) should pick words that accurately describe
%% the work being presented. Separate the keywords with commas.
\keywords{Fairness Metrics, Bias in Recommendations, 
Societal Stereotypes, Recommender System
}
%% A "teaser" image appears between the author and affiliation
%% information and the body of the document, and typically spans the
%% page.
% \begin{teaserfigure}
%   \includegraphics[width=\textwidth]{sampleteaser}
%   \caption{Seattle Mariners at Spring Training, 2010.}
%   \Description{Enjoying the baseball game from the third-base
%   seats. Ichiro Suzuki preparing to bat.}
%   \label{fig:teaser}
% \end{teaserfigure}

% \received{20 February 2007}
% \received[revised]{12 March 2009}
% \received[accepted]{5 June 2009}

%%
%% This command processes the author and affiliation and title
%% information and builds the first part of the formatted document.
 \maketitle
 \textit{This paper has been accepted for the ACM Web Conference 2025.}


\section{Introduction}
% \section{Introduction}

Large language models (LLMs) have achieved remarkable success in automated math problem solving, particularly through code-generation capabilities integrated with proof assistants~\citep{lean,isabelle,POT,autoformalization,MATH}. Although LLMs excel at generating solution steps and correct answers in algebra and calculus~\citep{math_solving}, their unimodal nature limits performance in plane geometry, where solution depends on both diagram and text~\citep{math_solving}. 

Specialized vision-language models (VLMs) have accordingly been developed for plane geometry problem solving (PGPS)~\citep{geoqa,unigeo,intergps,pgps,GOLD,LANS,geox}. Yet, it remains unclear whether these models genuinely leverage diagrams or rely almost exclusively on textual features. This ambiguity arises because existing PGPS datasets typically embed sufficient geometric details within problem statements, potentially making the vision encoder unnecessary~\citep{GOLD}. \cref{fig:pgps_examples} illustrates example questions from GeoQA and PGPS9K, where solutions can be derived without referencing the diagrams.

\begin{figure}
    \centering
    \begin{subfigure}[t]{.49\linewidth}
        \centering
        \includegraphics[width=\linewidth]{latex/figures/images/geoqa_example.pdf}
        \caption{GeoQA}
        \label{fig:geoqa_example}
    \end{subfigure}
    \begin{subfigure}[t]{.48\linewidth}
        \centering
        \includegraphics[width=\linewidth]{latex/figures/images/pgps_example.pdf}
        \caption{PGPS9K}
        \label{fig:pgps9k_example}
    \end{subfigure}
    \caption{
    Examples of diagram-caption pairs and their solution steps written in formal languages from GeoQA and PGPS9k datasets. In the problem description, the visual geometric premises and numerical variables are highlighted in green and red, respectively. A significant difference in the style of the diagram and formal language can be observable. %, along with the differences in formal languages supported by the corresponding datasets.
    \label{fig:pgps_examples}
    }
\end{figure}



We propose a new benchmark created via a synthetic data engine, which systematically evaluates the ability of VLM vision encoders to recognize geometric premises. Our empirical findings reveal that previously suggested self-supervised learning (SSL) approaches, e.g., vector quantized variataional auto-encoder (VQ-VAE)~\citep{unimath} and masked auto-encoder (MAE)~\citep{scagps,geox}, and widely adopted encoders, e.g., OpenCLIP~\citep{clip} and DinoV2~\citep{dinov2}, struggle to detect geometric features such as perpendicularity and degrees. 

To this end, we propose \geoclip{}, a model pre-trained on a large corpus of synthetic diagram–caption pairs. By varying diagram styles (e.g., color, font size, resolution, line width), \geoclip{} learns robust geometric representations and outperforms prior SSL-based methods on our benchmark. Building on \geoclip{}, we introduce a few-shot domain adaptation technique that efficiently transfers the recognition ability to real-world diagrams. We further combine this domain-adapted GeoCLIP with an LLM, forming a domain-agnostic VLM for solving PGPS tasks in MathVerse~\citep{mathverse}. 
%To accommodate diverse diagram styles and solution formats, we unify the solution program languages across multiple PGPS datasets, ensuring comprehensive evaluation. 

In our experiments on MathVerse~\citep{mathverse}, which encompasses diverse plane geometry tasks and diagram styles, our VLM with a domain-adapted \geoclip{} consistently outperforms both task-specific PGPS models and generalist VLMs. 
% In particular, it achieves higher accuracy on tasks requiring geometric-feature recognition, even when critical numerical measurements are moved from text to diagrams. 
Ablation studies confirm the effectiveness of our domain adaptation strategy, showing improvements in optical character recognition (OCR)-based tasks and robust diagram embeddings across different styles. 
% By unifying the solution program languages of existing datasets and incorporating OCR capability, we enable a single VLM, named \geovlm{}, to handle a broad class of plane geometry problems.

% Contributions
We summarize the contributions as follows:
We propose a novel benchmark for systematically assessing how well vision encoders recognize geometric premises in plane geometry diagrams~(\cref{sec:visual_feature}); We introduce \geoclip{}, a vision encoder capable of accurately detecting visual geometric premises~(\cref{sec:geoclip}), and a few-shot domain adaptation technique that efficiently transfers this capability across different diagram styles (\cref{sec:domain_adaptation});
We show that our VLM, incorporating domain-adapted GeoCLIP, surpasses existing specialized PGPS VLMs and generalist VLMs on the MathVerse benchmark~(\cref{sec:experiments}) and effectively interprets diverse diagram styles~(\cref{sec:abl}).

\iffalse
\begin{itemize}
    \item We propose a novel benchmark for systematically assessing how well vision encoders recognize geometric premises, e.g., perpendicularity and angle measures, in plane geometry diagrams.
	\item We introduce \geoclip{}, a vision encoder capable of accurately detecting visual geometric premises, and a few-shot domain adaptation technique that efficiently transfers this capability across different diagram styles.
	\item We show that our final VLM, incorporating GeoCLIP-DA, effectively interprets diverse diagram styles and achieves state-of-the-art performance on the MathVerse benchmark, surpassing existing specialized PGPS models and generalist VLM models.
\end{itemize}
\fi

\iffalse

Large language models (LLMs) have made significant strides in automated math word problem solving. In particular, their code-generation capabilities combined with proof assistants~\citep{lean,isabelle} help minimize computational errors~\citep{POT}, improve solution precision~\citep{autoformalization}, and offer rigorous feedback and evaluation~\citep{MATH}. Although LLMs excel in generating solution steps and correct answers for algebra and calculus~\citep{math_solving}, their uni-modal nature limits performance in domains like plane geometry, where both diagrams and text are vital.

Plane geometry problem solving (PGPS) tasks typically include diagrams and textual descriptions, requiring solvers to interpret premises from both sources. To facilitate automated solutions for these problems, several studies have introduced formal languages tailored for plane geometry to represent solution steps as a program with training datasets composed of diagrams, textual descriptions, and solution programs~\citep{geoqa,unigeo,intergps,pgps}. Building on these datasets, a number of PGPS specialized vision-language models (VLMs) have been developed so far~\citep{GOLD, LANS, geox}.

Most existing VLMs, however, fail to use diagrams when solving geometry problems. Well-known PGPS datasets such as GeoQA~\citep{geoqa}, UniGeo~\citep{unigeo}, and PGPS9K~\citep{pgps}, can be solved without accessing diagrams, as their problem descriptions often contain all geometric information. \cref{fig:pgps_examples} shows an example from GeoQA and PGPS9K datasets, where one can deduce the solution steps without knowing the diagrams. 
As a result, models trained on these datasets rely almost exclusively on textual information, leaving the vision encoder under-utilized~\citep{GOLD}. 
Consequently, the VLMs trained on these datasets cannot solve the plane geometry problem when necessary geometric properties or relations are excluded from the problem statement.

Some studies seek to enhance the recognition of geometric premises from a diagram by directly predicting the premises from the diagram~\citep{GOLD, intergps} or as an auxiliary task for vision encoders~\citep{geoqa,geoqa-plus}. However, these approaches remain highly domain-specific because the labels for training are difficult to obtain, thus limiting generalization across different domains. While self-supervised learning (SSL) methods that depend exclusively on geometric diagrams, e.g., vector quantized variational auto-encoder (VQ-VAE)~\citep{unimath} and masked auto-encoder (MAE)~\citep{scagps,geox}, have also been explored, the effectiveness of the SSL approaches on recognizing geometric features has not been thoroughly investigated.

We introduce a benchmark constructed with a synthetic data engine to evaluate the effectiveness of SSL approaches in recognizing geometric premises from diagrams. Our empirical results with the proposed benchmark show that the vision encoders trained with SSL methods fail to capture visual \geofeat{}s such as perpendicularity between two lines and angle measure.
Furthermore, we find that the pre-trained vision encoders often used in general-purpose VLMs, e.g., OpenCLIP~\citep{clip} and DinoV2~\citep{dinov2}, fail to recognize geometric premises from diagrams.

To improve the vision encoder for PGPS, we propose \geoclip{}, a model trained with a massive amount of diagram-caption pairs.
Since the amount of diagram-caption pairs in existing benchmarks is often limited, we develop a plane diagram generator that can randomly sample plane geometry problems with the help of existing proof assistant~\citep{alphageometry}.
To make \geoclip{} robust against different styles, we vary the visual properties of diagrams, such as color, font size, resolution, and line width.
We show that \geoclip{} performs better than the other SSL approaches and commonly used vision encoders on the newly proposed benchmark.

Another major challenge in PGPS is developing a domain-agnostic VLM capable of handling multiple PGPS benchmarks. As shown in \cref{fig:pgps_examples}, the main difficulties arise from variations in diagram styles. 
To address the issue, we propose a few-shot domain adaptation technique for \geoclip{} which transfers its visual \geofeat{} perception from the synthetic diagrams to the real-world diagrams efficiently. 

We study the efficacy of the domain adapted \geoclip{} on PGPS when equipped with the language model. To be specific, we compare the VLM with the previous PGPS models on MathVerse~\citep{mathverse}, which is designed to evaluate both the PGPS and visual \geofeat{} perception performance on various domains.
While previous PGPS models are inapplicable to certain types of MathVerse problems, we modify the prediction target and unify the solution program languages of the existing PGPS training data to make our VLM applicable to all types of MathVerse problems.
Results on MathVerse demonstrate that our VLM more effectively integrates diagrammatic information and remains robust under conditions of various diagram styles.

\begin{itemize}
    \item We propose a benchmark to measure the visual \geofeat{} recognition performance of different vision encoders.
    % \item \sh{We introduce geometric CLIP (\geoclip{} and train the VLM equipped with \geoclip{} to predict both solution steps and the numerical measurements of the problem.}
    \item We introduce \geoclip{}, a vision encoder which can accurately recognize visual \geofeat{}s and a few-shot domain adaptation technique which can transfer such ability to different domains efficiently. 
    % \item \sh{We develop our final PGPS model, \geovlm{}, by adapting \geoclip{} to different domains and training with unified languages of solution program data.}
    % We develop a domain-agnostic VLM, namely \geovlm{}, by applying a simple yet effective domain adaptation method to \geoclip{} and training on the refined training data.
    \item We demonstrate our VLM equipped with GeoCLIP-DA effectively interprets diverse diagram styles, achieving superior performance on MathVerse compared to the existing PGPS models.
\end{itemize}

\fi 

Recommender Systems (RS) personalize item selections for users, providing suggestions based on individual preferences and behaviors. 
These systems have an important impact on our decision-making, as they are widely employed across diverse platforms like e-commerce, social media, streaming services, and news outlets, shaping the content and products we encounter.
For several online platforms, recommendation systems help create an engaging experience for users by diversifying and personalizing the content and interactions. 
This would help users avoid information overload and help them focus on options that reflect their past behaviors. 
Amid the promise held by RS, however, there are concerns about potential bias in these systems. 
For example, research has shown that, on certain recommender systems, simply changing the gender in a job search—while keeping qualifications constant—can significantly influence access to high-paying positions \cite{datta15,10.1145/3442381.3450077}.

RS algorithms are traditionally evaluated using measures like RMSE (Root Mean Squared Error), NDCG (Normalized Discounted Cumulative Gain), precision, and diversity \cite{Shani2021}. 
If only these metrics are considered, then the recommended items are deemed good when they align with user preferences.
% and also diverse in terms of the content type. 
These metrics can help evaluate the performance of the RS, but in recent years there is a growing emphasis on evaluating and ensuring fairness as well \cite{10.1145/3442381.3449866,10.1145/3477495.3531959,abdollahpouri2019unfairness,10.1145/3450613.3456821,NEURIPS2020_9d752cb0,10.1145/3437963.3441824}. 
For instance, the authors in \cite{deldjoo_flexible_2021} introduce a fairness evaluation metric that can be sensitive to different fairness notions like user-centric or item-centric. 
% Another work \cite{NIPS2017_e6384711}, presents four new metrics to quantify discrepancies between advantaged and disadvantaged groups. 
Although substantial work has been done in this field in recent years, there is a notable gap in research specifically focused on robust ways to quantify consumer bias accurately. 
Most of the metrics used are deficient in the following ways: 
(i) they over-simplify the concept of fairness,
(ii) they fail to consider the ranking of recommended items, and
(iii) they rely exclusively on a single type of fairness metric. 
A popular approach to evaluating consumer-side fairness in recommendation systems involves an adaptation of \textit{equal opportunity} (refer to Section: \ref{subsec:fair_notion}), which aims to balance performance metrics (such as recall) or utility scores across different groups, such as male vs. female users \cite{10.1145/3442381.3449866,10.1145/3655631,10.1145/3651167,10.1145/3604915.3608784,tang_when_2023,Melchiorre21}. 
However, while this approach is quite straightforward, it may overlook disparities in recommendations across different item categories.
To assess these disparities in recommendations, we refer to Figure \ref{fig:compare_genre}, where we present, for various recommendation algorithms, an analysis of the proportion of action and romance movies among the top 10 recommendations for male and female user groups, along with the corresponding Precision@10 values for each group.
There are three notable observations here:
(i) across all models, there is a noticeable disparity in the proportion of romance and action movies recommended, as romance movies tend to be recommended more frequently to female users, while action movies are more often recommended to male users;
(ii) precision@10 values for both male and female users are similar across all models, suggesting that the models appear to perform equally well for both genders based on this metric;
and (iii) different models exhibit varying levels of bias.
These observations suggest that simply comparing performance metrics across sensitive attributes is insufficient to assess fairness in recommendation systems. 
More granular metrics are necessary to accurately quantify bias and ensure fair recommendations across different user groups. To further emphasize the significance of granular evaluation, we provide a brief overview here with a more detailed example provided in Section \ref{subsec:motivating_example}. Bias can arise in recommender systems due to stereotypical interactions in the dataset used to train them, which leads to biased recommendations based on sensitive attributes of the users. This bias can trap users in filter bubbles, with a focus on stereotypical categories or narrowly defined preferences. Providing more of a certain type of content can have unintended consequences; for instance, if a young male user is predominantly recommended action movies, where violence is glorified and shown in a consequence-free way, then it could contribute to adverse effects, including the desensitization of violence. This is just one example, but the stakes are much higher for more sensitive domains like news, job recommendations, etc., where there can be catastrophic consequences if similar biases manifest in them.

\begin{figure}[t]
    \centering
    \includegraphics[width=0.5\textwidth]{images/gender_compare_new.pdf}
    \caption{
    Comparison of action and romance movie recommendations among male and female users across four recommendation algorithms, along with corresponding Precision@10 values. The graphs highlight disparities in genre recommendations by gender, with romance movies being more frequently suggested to female users and action movies to male users, despite similar Precision@10 metrics. }
    \label{fig:compare_genre}
    % \vspace{-0.5cm}
\end{figure}

% the Recall@50 values for males and females can give a false sense of fairness when there is a clear bias in the recommended movies. 
% As an example, for the Item KNN model \cite{10.1145/352871.352887}, the Recall@50 for female and male populations are very close, meaning the model can be considered "fair". 
% However, there is a noticeable difference in the number of movies recommended to male and female users for the different genres, reflecting the stereotype males tend to like action movies whereas women like romance ones.

% includes balancing rating errors in different groups of users \cite{zhu18,NIPS2017_e6384711}
In this paper, our focus is to overcome the limitations of current fairness assessment metrics by designing a set of evaluation measures to help us quantify gender bias in recommendation models. 
Specifically, our aim is to ensure that they are diverse so as to capture nuances of the bias that might not be immediately obvious but are significant.
We propose to do this by incorporating item categories and rank (when relevant). 
Next, we utilize one of these metrics as part of the loss function to optimize, which results in an effective increase of fairness for recommendations made to users for each category. 
This not only shows that our metrics are effective in addressing fairness concerns but also proves their utility for quantifying bias. Additionally, we use our metrics to evaluate a variety of recommendation models, including fairness-aware ones, using three real-world datasets. The contribution of this paper can be summarized as follows:
% \vspace{-1pt}
\begin{itemize}[left=3pt]
    \item We introduce a set of metrics that considers both categories and rankings of the recommendations made, in turn providing a more nuanced evaluation.
    \item We evaluate different recommendation algorithms to check for gender bias using the proposed set of metrics.
    \item We employ an in-processing technique to ensure the recommended items are fair, not only in a general sense but in a category-aware sense.
\end{itemize}
\section{Related work}
% \section{Related Work}\label{sec:related_works}
\gls{bp} estimation from \gls{ecg} and \gls{ppg} waveforms has received significant attention due to its potential for continuous, unobtrusive monitoring. Earlier work relied on classical machine learning with handcrafted features, but deep learning methods have since emerged as more robust alternatives. Convolutional or recurrent architectures designed for \gls{ecg}/\gls{ppg} have shown strong performance, including ResUNet with self-attention~\cite{Jamil}, U-Net variants~\cite{Mahmud_2022}, and hybrid \gls{cnn}--\gls{rnn} models~\cite{Paviglianiti2021ACO}. These architectures often outperform traditional feature-engineering approaches, particularly when both \gls{ecg} and \gls{ppg} signals are used~\cite{Paviglianiti2021ACO}.

Nevertheless, many existing methods train solely on \gls{ecg}/\gls{ppg} data, which, while plentiful~\cite{mimiciii,vitaldb,ptb-xl}, often exhibit significant variability in signal quality and patient-specific characteristics. This variability poses challenges for achieving robust generalization across populations. Recent work has explored transfer learning to overcome these issues; for example, Yang \emph{et~al.}~\cite{yang2023cross} studied the transfer of \gls{eeg} knowledge to \gls{ecg} classification tasks, achieving improved performance and reduced training costs. Joshi \emph{et~al.}~\cite{joshi2021deep} also explored the transfer of \gls{eeg} knowledge using a deep knowledge distillation framework to enhance single-lead \gls{ecg}-based sleep staging. However, these studies have largely focused on within-modality or narrow domain adaptations, leaving open the broader question of whether an \gls{eeg}-based foundation model can serve as a versatile starting point for generalized biosignal analysis.

\gls{eeg} has become an attractive candidate for pre-training large models not only because of the availability of large-scale \gls{eeg} repositories~\cite{TUEG} but also due to its rich multi-channel, temporal, and spectral dynamics~\cite{jiang2024large}. While many time-series modalities (for example, voice) also exhibit rich temporal structure, \gls{eeg}, \gls{ecg}, and \gls{ppg} share common physiological origins and similar noise characteristics, which facilitate the transfer of temporal pattern recognition capabilities. In other words, our hypothesis is that the underlying statistical properties and multi-dimensional dynamics in \gls{eeg} make it particularly well-suited for learning robust representations that can be effectively adapted to \gls{ecg}/\gls{ppg} tasks. Our work is the first to validate the feasibility of fine-tuning a transformer-based model initially trained on EEG (CEReBrO~\cite{CEReBrO}) for arterial \gls{bp} estimation using \gls{ecg} and \gls{ppg} data.

Beyond accuracy, real-world deployment of \gls{bp} estimation models calls for efficient inference. Traditional deep networks can be computationally expensive, motivating recent interest in quantization and other compression techniques~\cite{nagel2021whitepaperneuralnetwork}. Few studies have combined large-scale pre-training with post-training quantization for \gls{bp} monitoring. Hence, our method integrates these two aspects: leveraging a potent \gls{eeg}-based foundation model and applying quantization for a compact, high-accuracy cuffless \gls{bp} solution.
We categorize and discuss the related work existing in this field in the subsections below.
% \vspace{-8pt}
\subsection{Gender Fairness in RS}
Gender bias in recommendation systems can exist as systematic discrepancies in the algorithms when recommending items to users of different genders. The challenges of gender bias can have a wide-ranging set of implications, including imbalanced representations of items for different genders, stereotypical recommendations (male-dominated occupations recommended to males more than females), and limitations in user personalization (being recommended items that are not related to users' preferences just because they are male or female). Research on evaluating and addressing gender bias in recommendation systems is an ongoing process. Melchiorre et al. \cite{Melchiorre21} investigate the impact of a common de-biasing strategy called resampling on RS algorithms. This strategy marginally decreases gender bias, with a slight decrease in performance. The authors in \cite{pmlr-v139-gorantla21a,10.1145/3132847.3132938,XIA2019104857} introduce ways to re-rank items to offer a balanced solution that caters to group fairness (for gender and other sensitive attributes) and user preferences.
Historical data that can contain stereotypical movie preferences can intensify certain biases further when used to train traditional recommendation models. For instance, male users display a bias towards \textit{action} movies, which is amplified by recommendation algorithms like UserKNN \cite{TsintzouPT19}. The authors in \cite{Acharyya_Das_Chattoraj_Tanveer_2020}
introduce a framework that fairly predicts the quality of TedTalk speeches by using causal models and counterfactuals to mitigate gender and racial bias. Adversarial fairness where the recommender system is trained to not only make accurate predictions but also make it difficult for the adversary to guess the sensitive attribute, has also been employed to make such systems more fair \cite{LIU2022108058,rus2022closinggenderwagegap}. Recent advancements in this field, have led to the development of fairness-aware recommendation models, with special emphasis on gender bias \cite{yang2020causalintersectionalityfairranking, boratto2022consumer,10.1145/3397271.3401177,zhu18,wei2022comprehensive,10.1145/3442381.3450015}.
\subsection{Evaluating Gender Bias}
\label{subsec:fair_notion}
For evaluating gender bias, the most common fairness definitions employed are the concepts of \textit{Demographic Parity} and \textit{Equal Opportunity}, which are both related to group fairness. Fairness in this context, pertains to equitable treatment across different groups (which can be measured in classification and recommendation tasks).
For group fairness, the idea is to ensure that the predicted outcomes  \(\hat{Y}\) of a model should not be dependent on sensitive attributes like gender $S$. 
For demographic parity, the proportions of each sensitive group (like male and female) receiving positive predictions should be equal. 
For binary classification, demographic parity can be formalized as: 
\[P(\hat{Y}=1 | S=1) = P(\hat{Y}=1 | S=0)\] 
Essentially what this implies is that the positive outcome should be the same for both genders, where 0 may represent male and 1 may represent female or vice versa. 
Equal Opportunity, on the other hand, holds when the model has equal true positive rates across different demographic groups \cite{hardt16}. 
This concept can be formalized as:
\[P(\hat{Y}=1 | S=1, Y=1) = P(\hat{Y}=1 | S=0, Y=1) \]
where \(Y\) represents the true outcome.

Besides these two methods to quantify fairness, some additional concepts (as discussed in \cite{kheya2024pursuitfairnessartificialintelligence}) used include Equalized Odds \cite{hardt16}, Balance for Negative Class \cite{10.1145/3194770.3194776}, Balance for Positive Class \cite{10.1145/3194770.3194776}, Intersectional Fairness \cite{Gohar2023ASO}, Equal Calibration \cite{Chouldechova2016FairPW}
% Predictive parity \cite{10.1145/3194770.3194776}
and Causal-based notions \cite{10.5555/3294996.3295162, Acharyya_Das_Chattoraj_Tanveer_2020}. 



\subsection{Evaluating Consumer-Side Fairness}
 In recommendation systems, fairness can be seen as a multi-sided concept and categorized into three groups: consumers (C-fairness) \cite{{deldjoo_flexible_2021,Melchiorre21,wu20,pmlr-v139-gorantla21a,ghosh21,Edizel20,wan20,hao21}} which is related to the impact of recommendations of the system on protected classes of user, provider (P-fairness) which focuses on ensuring fairness for providers/sellers on a platform and both (CP-fairness) \cite{burke_multisided_2017}. Our work focuses on the consumer-side fairness concept because we want to ensure that recommendations made by models are not biased against a certain gender. Prior research has shown how recommendations can differ in an unfair way based on sensitive attributes of users like gender, age, race, etc. \cite{gupta_questioning_2022,10.1145/3547333,JIN2023101906}. Evaluating bias in these systems, before deploying is thus essential to stop the reinforcement of stereotypes and limiting diverse content. When quantifying consumer side bias in recommendation systems, the most common approach is to adopt the concept of equality of opportunity and focus on the differences in metrics such as recall, precision and/or NDCG \cite{Fu20,10.1145/3442381.3449866,Melchiorre21}. Other papers also employ causal-based fairness notions \cite{wei2022comprehensive,10.1145/3404835.3462943} and demographic parity \cite{10.1145/3292500.3330691,DBLP:journals/corr/abs-1809-09030,boratto2022consumer}. Unlike these metrics, which assume fairness implies equality, \cite{deldjoo_flexible_2021} suggests how, for instance, paid users should be provided better recommendations when compared to free users. They design a set of metrics that can take this disparity into account and then measure fairness accordingly. Another interesting way to measure unfairness is the concept of envy-free fairness, which is achieved when no one user prefers another user's recommendations way more significantly than their own \cite{do2022online}. While these metrics can identify consumer-side bias to an extent, they come with some limitations.

 
Limitations of current metrics used to quantify consumer-side bias in recommendation systems include: (i) over-simplifying the meaning of fairness in RS, which employs various techniques like collaborative and content-based filtering and hybrid methods.
Simple notions can fail to capture disparities that exist across different types of items, for example, genres, when considering movie recommendations. Some metrics that can fall prey to this oversimplification issue include \cite{Weydemann19,NIPS2017_e6384711, Fu20,10.1145/3442381.3449866,islam21,foulds2019bayesianmodelingintersectionalfairness,9101635,do2022online,10.1145/3534678.3539269,Melchiorre21,deldjoo_flexible_2021};
(ii) not utilizing ranks when evaluating recommendation quality can yield considerable issues. 
This is due to the fact that recommendations are displayed one after another, so the items on higher ranks must be more relevant to keep the user satisfied. 
Thus, capturing the quality of recommendations using the ranks reflects a more complete way of evaluating models. 
Some metrics that can fall prey to this issue include \cite{NIPS2017_e6384711,Weydemann19,do2022online,10.1145/3534678.3539269,deldjoo_flexible_2021}.
It is important to state however, that even considering rank can give rise to positional bias, which refers to the tendency of users to favor the items that appear on top of a ranked list. 
Evaluating till a certain position like Recall@k can lead to a skewed sense of assessment of how the recommendation model performs. 
Additionally, metrics like MAP (Mean Average Precision), which normally treats relevance as a binary value (0: not relevant and 1: relevant) can also fail to capture the nuanced relevance that can come from items having multiple categories. 
So, using more than one metric (both with and without using ranks) to quantify bias is essential; 
(iii) relying only on one type of fairness metric can obscure underlying biases and give a false impression of fairness in recommendation systems. 

Hence, using multiple metrics can help uncover hidden biases. Additionally, a model that is fair according to one metric can fail to hold other fairness metrics and risk overlooking subtle unfairness issues.



We want to highlight some of the works that have taken into account different classes when evaluating recommendations \cite{lin2019crankvolumepreferencebias,10.1145/3240323.3240372,10.1145/3292500.3330691,10.1145/3626772.3657794,10068703,10.1145/3308560.3317595,10.1145/3038912.3052612,10.1145/3589334.3648158}.
For instance, \cite{lin2019crankvolumepreferencebias} groups users on certain attributes and items by category, then measures preference ratio, which is the fraction of liked items by a group across categories. Next, they measure the bias disparity by taking the preference and recommended ratios' relative differences. This is close to our work but still doesn't account for the ranks of items and we evaluate the direct comparison of the recommendations for males and females. Additionally, \cite{10.1145/3240323.3240372} introduce calibrated recommendations, ensuring the recommended items align with user preferences without overemphasizing particular categories. The work by \cite{10.1145/3292500.3330691} uses a measure \textit{Skew@k} to evaluate proportions of candidates based on sensitive attributes, and \cite{10.1145/3626772.3657794} uses a fairness metric called Attention Weighted Ranked Fairness (AWRF) \cite{10.1145/3308560.3317595} to ensure there is balance in exposure in different groups of providers. While both these works ensure group fairness, our work is more concentrated on evaluating the distribution of content categories for different groups. Unlike the work by \cite{10.1145/3626772.3657794} that focuses on provider-side fairness, we focus on consumer-side fairness. 
\section{Proposed Evaluation Metrics}

\subsection{Motivating Fairness Concern}
\label{subsec:motivating_example}

Our example is a typical offline setting recommendation system, which is trained using historical user and movie interactions. 
For our scenario, let us assume we have $u_1$, a male user who has watched numerous action and sci-fi movies, with some romance movies. 
We also have $u_2$, a female user who has watched a lot of drama movies but also a few action movies. 
Let us say we decide to calculate overall precision for each user group (male and female) and then compare them to ensure the model's fairness. 

\subsubsection{Potential Issue}
Machine learning systems tend to learn and amplify bias from the training data \cite{10.5555/3157382.3157584,lum2016,pmlr-v81-ensign18a,10.1145/3287560.3287572,kheya2024pursuitfairnessartificialintelligence}. Recommender systems are no different, as they can pick up on stereotypical user-item interactions and make biased recommendations based on sensitive attributes of users \cite{TsintzouPT19,Melchiorre21}. Biased recommendations can have a broad impact on users if the models recommend content just because it aligns with certain stereotypes, for instance, males like action movies, and females like romance movies. 

This imbalance in recommendation can lead to a situation in which users are only exposed to items that align with part of their preferences and stereotypical norms. This would potentially filter out diverse content and prevent users from discovering new movies. As time passes, $u_1$ might stop getting romance movies recommended to them, even though they enjoy them. 
For $u_2$, a similar case could arise for action movies. 
Essentially, this imbalance can trap the users in a bubble of recommendations with only their established preferences and gender-stereotypical genres. Additionally, if a user's established preferences are already aligned with gender stereotypes, then the bias in recommendation will intensify further giving rise to a filter bubble, with very redundant movie recommendations.

\subsubsection{Falling short when quantifying gender bias}
Moreover, as mentioned earlier, using some performance metrics for both genders and comparing them to evaluate fairness is not a great idea. 
The two groups can have similar scores, even if the model is making biased recommendations by choosing to neglect certain categories for certain users. 
Our proposed metrics address this issue by breaking down the recommendations by category to get a more nuanced sense of fairness.

The key takeaway here is the importance of learning fair user and item representations, as well as evaluating fairness in the output. 
Even if a model is trained only on user-item-rating interactions with no explicit mention of sensitive attributes, the model can still infer this private information due to the correlation between their behavior and their sensitive attributes \cite{bothmann2024fairnessroleprotectedattributes,10.1145/3106237.3106277,10.1145/3368089.3409697}. 
So, we have to take precautionary measures when training the model itself, so it is unable to learn these correlations. 
We also wanna discuss how it is important to ensure personalization, including gender-specific preferences to enhance user satisfaction, but such preferences should be balanced against any risk of reinforcing stereotypes. 

Please note in our study we focus on binary genders, acknowledging there are many other gender identities not represented here.

\subsection{Notation}


We present all metrics for fairness assessment using the following mathematical notation:
\begin{itemize}
    \item $u_i$: A single user, where $i$ indexes the users.
    \item $v_j$: A single item, where $j$ indexes the items.
    \item $\mathcal{U}$ and $\mathcal{V}$: The set of users and items, respectively.
    \item $\mathcal{U}_m$ and $\mathcal{U}_f$: The set of male and female users, respectively.
    \item $c$: An item category, such as Action, Sci-Fi, Romance, etc.
    \item $C$: A category matrix where $C_{j,c} = 1$ if category $c$ is associated with item $v_j$, and $C_{j,c} = 0$ otherwise.
    \item $C_{v_j}$: The list of categories associated with item $v_j$.
    \item $TopK_{u_i}$: The set of top $K$ recommended items for user $u_i$.
    \item $\mathrm{C}$: Represents the set of categories for items.
\end{itemize}


To assess fairness, we adapt and extend Information Retrieval metrics, introducing both non-ranking and ranking-based metrics, which are detailed in the following subsections.

\subsection{Non-ranking-based metrics}
The first set of metrics we propose evaluates the fairness of a recommender system without considering the ranking of movies. 
\subsubsection{Category Coverage (CC)} 
\label{subsec:gp}

This metric estimates the proportion of recommended items associated with category $c$ relative to all categories for all users $u_i$. 
This essentially captures the category-specific performance of the recommended items. 
It is defined as follows:

\begin{equation}
CC(c, \mathcal{U}) = \frac{1}{|\mathcal{U}|} \sum_{u_i \in \mathcal{U}} \frac{1}{|TopK_{u_i}|} \sum_{v_j \in TopK_{u_i}} \frac{C_{j,c}}{|C_{v_j}|}
\label{eq:M1}
\end{equation}




\subsubsection{Relative Category Representation (RCR)}
This metric estimates the proportion of category $c$ in recommended items relative to the proportions of all categories available from the whole dataset. 
% Here $I$ represents all the movies available. 
This helps provide insights on category-specific items and is defined as follows: 
% For example, if $M_2(action, U)$ is higher than $M_2(romance,U)$ then the recommender system is good at recommending movies that have action as a genre to the set of users U. 

% \begin{equation}
% M_2(g,U) = \frac{1}{|U|}\sum_{u \in U} \frac{\sum_{m \in M_u}\frac{Gmg}{|G_m|}} {\sum_{m_i \in I} \frac{G_{m_i,g}}{|G_{m_i}|}
% }
% \label{eq:M2}
%  \end{equation}
\begin{equation}
RCR(c, \mathcal{U}) = \frac{1}{|\mathcal{U}|} \sum_{u_i \in \mathcal{U}} \frac{\sum_{v_j \in TopK_{u_i}} \frac{C_{j,c}}{|C_{v_j}|}}{\sum_{v_k \in \mathcal{V}} \frac{C_{k,c}}{|C_{v_k}|}}
\label{eq:M2}
\end{equation}

These two metrics help us quantify how relevant a category is for items recommended to a given set of users. For instance, $CC$ would help us quantify the diversity of items recommended by reflecting the overall distribution of recommended content when computed for different $c$. Whereas, $RCR$ (calculated for different $c$) would measure if the recommendations suppress or amplify certain categories by comparing them to the actual distribution of available content.
Both $CC$ and $RCR$ provide a more nuanced understanding of how the recommendation system performs but don't take into account the position of the recommended items. 
For recommendation systems, where in most cases items surface one after another, it is vital that the items on the top of the menu are the most relevant. 
% This will ensure the user is kept engaged when they are viewing the recommended items. 
% \(M_1\) and \(M_2\) capture information about the performance of the recommendations, but don't take into account the relative orders of the items. 
Hence, to ensure that ranking order is also considered for evaluating recommended items, in the next section we introduce rank-based metrics.


\subsection{Rank-based Metrics}
\label{sec:rbm}
% In recommendation systems (for instance movie, music, news etc.) the order of items can significantly impact the user experience and satisfaction. 
% Therefore we ought to evaluate the quality of the recommended items by using the order or rank in which they appear. 
% In this section we introduce four new metrics to help us do just that. Note: In this section when we use notations like \(m_i\) or \(m_j\), they represents the movie in rank \(i\) and \(j\) respectively in \(M_u\).
 % The following sub-sections introduces four new metrics
We now introduce our metrics that utilize ranking to provide a comprehensive assessment of fairness.

\subsubsection{Category Mean Average Precision (CMAP)}
For a given set of users, this metric estimates the proportion of recommended items associated with category $c$ by incorporating the rank of the items. 
A category-specific average precision score is computed for each user, followed by averaging these scores across all users.
This metric is defined as follows:
% This approach ensures that higher ranked items will contribute more significantly to the metric. 
% For instance, if \(M_3(comedy, U)\) is higher than \(M_3(romance,U)\), then it means the recommender system is better at retrieving and ranking movies which has comedy as one of the genre compared to that of romance for the set of users U. 

% \begin{equation}
% \small CumulativeGenreAverage(M_u,g,i) = CGA(M_u,g,i) =   \frac{1}{i} \sum_{j=1}^{i} {\frac{G_{{m_jg}} }{|G_{m_j}|}}
% \label{eq:map}
%  \end{equation}

\begin{equation}
CMAP(c, \mathcal{U}) = \frac{1}{|\mathcal{U}|} \sum_{u_i \in \mathcal{U}}
% \frac{1}{|TopK_{u_i}|} 
\frac{\sum_{j=1}^{|TopK_{u_i}|}  P(j) \cdot \frac{C_{j,c}}{|C_{v_j}|}}{\sum_{v_j \in \mathcal{V}} \frac{C_{j,c}}{|C_{v_j}|}}
\label{eq:M3}
\end{equation}

% \begin{equation}
% GMAP(g, \mathcal{U}) = \frac{1}{|\mathcal{U}|} \sum_{u_i \in \mathcal{U}}
% \frac{1}{|TopK_{u_i}|} 
% \sum_{j=1}^{|TopK_{u_i}|}  P(j) \cdot \frac{G_{j,g}}{|G_{v_j}|}
% \label{eq:M3}
% \end{equation}

\noindent where $
P(j) = \frac{1}{j} \sum_{k=1}^{j} \frac{C_{k,c}}{|C_{v_k}|}
$.


\subsubsection{Category Discounted Cumulative Gain (CDCG)}
% This metric can help us evaluate the effectiveness of recommendations, particularly in terms of ranking. 
The score for this metric is discounted based on the position of the item in the recommended list. 
% As an example, let's say \(M_4(comedy, U)\) is higher than \(M_4(action,U)\), this then means the recommender system can retrieve and rank comedy movies more effectively when compared to action movies. 
% This also means that users from a set $\mathcal{U}$ need to go through items from other categories first to get to the ones that have action as one of the genres.
While similar to the previous metric, CDCG uses a logarithmic discount factor, which is higher for items that appear lower down in the list. It is defined as follows:

\begin{equation}
CDCG(c, \mathcal{U}) = \frac{1}{|\mathcal{U}|} \sum_{u_i \in \mathcal{U}} \frac{1}{|TopK_{u_i}|} \sum_{j=1}^{|TopK_{u_i}|} \frac{\frac{C_{j,c}}{|C_{v_j}|}}{\log(j + 1)}
\label{eq:M4}
\end{equation}

\subsubsection{Category Mean Reciprocal Rank (CMRR)}
% \(M_5(g,U)\) helps us quantify how effective the recommendations are by incorporating how the system ranks the items for each genre. 
This metric is based on MRR, which is essentially the mean reciprocal of the rank of the first relevant item. 
% But we have adapted that concept to find the items of a given genre using the proportion of this genre for a given movie and then dividing it by the rank of the item itself. This value is then summed up for all movies recommended to a user. If for instance \(M_5(action, U)\) is higher than \(M_5(romance,U)\), then users are able to find movies of genre action in higher ranks in their recommended movies than romance movies. This metric is very similar to GDCG but the discount factor is different. GDCG is more forgiving to the items at the end of the list because the discounting is done logarithmically, whereas this metric reduces the score linearly, making lower-ranked items contribute way less.
We adapt it for assessing fairness and we define it as follows:
\begin{equation}
CMRR(c, \mathcal{U}) = \frac{1}{|\mathcal{U}|} \sum_{u_i \in \mathcal{U}} \frac{1}{|TopK_{u_i}|} \sum_{j=1}^{|TopK_{u_i}|} \frac{\frac{C_{j,c}}{|C_{v_j}|}}{j}
\label{eq:M5}
\end{equation}


\subsubsection{Category RPrecision (CRP)}
% This metric that also considers rank when evaluating the quality of the recommendations. 
This metric uses the category-specific precision but only for the top \(\lfloor R_c \rfloor\) results, where $R_c$ represents the proportion of category $c$ when considering all items in $\mathcal{V}$ (i.e., $R_c=\sum_{j=1}^{|\mathcal{V}|} \frac{C_{j,c}}{|C_{v_j}|}$, note that item $v_j$ can belong to multiple categories like movie genres). 
% Here \(M_gr\) represents the top \(\lfloor R_g \rfloor\) recommended items. This metric is more effective than the other rank-based metrics discussed above since we are using \(R\) to balance the evaluation of the recommendation quality, which is suited for our case since we have varying numbers of movies for different genres. For example. if \(M_6(action, U)\) is higher than \(M_6(romance,U)\), then it indicates the system can retrieve and recommend action movies better than romance movies. However, this can also be the case if there are fewer action movies compared to romance. This means if there are a lot more romance movies, the system needs to retrieve and rank them and this might result in a lower GRP. 
This metric is defined as follows:
\begin{equation}
CRP(c, \mathcal{U}) = \frac{1}{|\mathcal{U}|} \sum_{u_i \in \mathcal{U}} \frac{\sum_{j=1}^{|Top\lfloor R_c \rfloor_{u_i}|} \frac{C_{j,c}}{|C_{v_j}|}}{\lfloor R_c \rfloor}
\label{eq:M6}
\end{equation}

% For instance, $CDCG$ and $CMRR$ will help measure the diversity of movies recommended in terms of categories, with different discounting factors for the rank of the items.  
All four of the ranking-based metrics, take into consideration the proportions of categories and use rank as a discounting factor. The discounting factor is different for each of them and has distinct nuances. For instance, $CRP$ would only evaluate recommendations till a certain rank \(\lfloor R_c \rfloor\) to measure if the model is providing a proportional amount of recommendations for $c$ relative to its presence in the dataset. Another metric like $CDCG$ for example would capture something different, which is the relevance of $c$ by giving more weight to items (falling under category $c$) appearing higher in the list.
% For instance, let us take a movie recommendation setting where there are a total of 10 movies to recommend from with overall proportion of action movies equal to 3. In this case, $CRP$ for action would only consider items in the top-3 list for all users, evaluating the items based on their relative presence in the dataset.This can help us grasp if a model is recommending a overwhelming amount of movies 
% $CDCG$ discounts relevance of items from $c$ by giving more weight to items appearing earlier in the list. In contrast, 



\subsection{Group Fairness (Gender)}
Gender Balance Score (GBS) for a given category $c$, is the absolute difference of category distribution values for $\mathcal{U}_m$ and $\mathcal{U}_f$ based on any of the category-aware metrics discussed above. 
% In our case of quantifying gender bias, \(a\) represents the female population and \(a'\) represents the male population respectively.
This metric is an adaptation of \textit{demographic parity} as mentioned in Section \ref{subsec:fair_notion}. 
We want to make sure that the category distributions of the recommended items for the groups, of males and females are similar. 
For each metric $\mathcal{M}$, defined above, we want to ensure that the difference of values calculated for the groups male and female summed up for all categories are close to 0. 
This can be formalized as:
% Achieving perfect demographic parity in practice, can be unrealistic as there are slight variations in predictions and this would also lead to poor utility. Our way of addressing this problem is to not be overly strict and allow for a threshold \(\epsilon\), within which the differences of the values for the groups is acceptable.
\begin{equation} \bigtriangleup
\mathcal{M}_c=| \mathcal{M}(c,\mathcal{U}_m) - \mathcal{M}(c,\mathcal{U}_f)|
\label{eq:delM}
 \end{equation}
\begin{equation} 
GBS(\mathcal{M}) =\sum_{c \in \mathrm{C}} \bigtriangleup
\mathcal{M}_c \approx 0
\label{eq:GBS}
 \end{equation}





\section{Genre Aware Regularization For Gender Fairness}
\label{sec:fairloss}

Now that we have introduced our set of metrics for capturing a nuanced sense of fairness for recommended items, we aim to learn fair representations that are category-aware for users of different genders.
To promote fairness in recommendation models we employ an in-processing regularization technique inspired by the approach first proposed by \cite{pmlr-v54-zafar17a}. 
We propose to incorporate Category Coverage (Section \ref{subsec:gp}) into the loss function alongside the primary recommendation loss, encouraging the model to optimize for category-aware fairness. We want to emphasize that while we select $CC$ as an example here, optimizing on any of the other metrics (as long as the implementation is differentiable) is possible.
We incorporate this fairness regularizer into the loss function of each baseline model—MF, VAE-CF, and NeuMF.

Specifically, to discourage the model from learning any biased representations, we incorporate the following regularizer:
\begin{equation}
\mathcal{L}_{FairGenreGender} = GBS(CC) = \sum_{c \in \mathrm{C}}|CC(c,\mathcal{U}_m)-CC(c,\mathcal{U}_f)|
 % = \sum_{g \in G} | \frac{1}{|U_a|} \sum_{u \in U}\frac{1}{|M_u|} \sum_{m\in M_u} \frac{G{mg}}{|G_m|} - \frac{1}{|U_a'|} \sum_{u \in U}\frac{1}{|M_u|} \sum_{m\in M_u} \frac{G{mg}}{|G_m|}|
 \label{eq:myloss}
\end{equation}
The goal is to minimize the difference in category distribution of the items being recommended to the users of each sensitive group. 
The new loss function for MF, VAE-CF and NeuCF models can be written as:
\begin{equation}
\small \mathcal{L} = \alpha \mathcal{L}_{FairGenreGender} + (1-\alpha) \mathcal{L}_{Recommendation}
\label{eq:fairequation}
\end{equation}

\noindent where \(\alpha\) is used to calibrate the trade-off between the model loss and the fairness term. \(\mathcal{L}_{Recommendation}\) represents the respective recommendation loss term of each model. By tuning this hyperparameter, we ensure that fairness does not overly compromise the recommender's ability to respect users' personal preferences (even if there are natural differences across liked categories for users of different genders). 
In order to modulate the gender loss value and to ensure it is able to make a significant contribution to the loss function, we pass it through a sigmoid function, centering at 0.5 and scaling it by 0.1. Additionally, the maximum batch loss value of the actual recommendation loss is used to scale the gender loss up to ensure both losses are on the same scale.
In the next section we show how using this fairness regularization helps the models minimize bias in recommendations provided to users of different genders for different categories.


\section{Experiments}
In the sub-sections below, we outline our experimental setup.
% \subsection{Experimental Setup}
\label{section:experimental_setup}
\textbf{Datasets:} Table~\ref{tab:datasets} provides a detailed breakdown of the SOTA intrusion datasets utilized in our study. 
%For each dataset we follow the data preparation steps outlined in section~\ref{section:data_preparation}. 
% \sean{is this section necessary with reduced page limit?}
% \begin{enumerate}
%     \item X-IIoTID \cite{al2021x}: The dataset consists of 59 features which are collected with the independence of devices and connectivity, generating a holistic intrusion data set to represent the heterogeneity of IIoT systems. It includes novel IIoT connectivity protocols, activities of various devices, and attack scenarios.  
%     \item WUSTL-IIoT \cite{zolanvari2021wustl}: WUSTL-IIoT aims to emulate real-world industrial systems. The dataset is deliberately unbalanced to imitate real-world industrial control systems, consisting of 41 features and 1,194,464 observations.
%     \item CICIDS2017 \cite{Sharafaldin2018TowardGA} The CICIDS2017 dataset includes a comprehensive collection of benign and malicious network traffic. It contains 80 features and represents a broad range of attacks, such as DoS, DDoS, Brute Force, XSS, and SQL Injection, across more than 2.8 million network flows. The dataset is widely used in evaluating intrusion detection systems.
%     \item UNSW-NB15 \cite{moustafa2015unsw, moustafa2016evaluation, moustafa2017novel, moustafa2017big, sarhan2020netflow} UNSW-NB15 is a comprehensive network intrusion dataset created by the University of New South Wales. It contains 49 features representing normal and malicious activities generated using IXIA's network traffic generator, covering a variety of contemporary attack types. 
% \end{enumerate}
For IIoT intrusion, we use IIoT datasets X-IIoTID \cite{al2021x} and WUSTL-IIoT \cite{zolanvari2021wustl}. We also include commonly used network intrusion datasets CICIDS2017 \cite{Sharafaldin2018TowardGA} and UNSW-NB15 \cite{moustafa2015unsw}. For X-IIoTID \cite{al2021x}, CICIDS2017 \cite{Sharafaldin2018TowardGA}, and UNSW-NB15 \cite{moustafa2015unsw}, we split the data across five experiences such that each experience contains two to four attacks. For WUSTL-IIoT \cite{zolanvari2021wustl}, we split the data across four experiences such that each experience contains one attack. We perform this data split to simulate an evolving data stream with emerging cyber attacks over time where each experience contains different attacks. 


%%%%%%%%%%%%%%%%%%%%%%%%%%%%%%%%%%%%%%%%%%%%%%%%%%%%%%%%%%%%%%%%%%%%%%%%%%%
\begin{table}[h]
    \caption{Selected Intrusion Datasets}
    \centering
    \label{tab:datasets}
    \resizebox{.99\columnwidth}{!}{
    \begin{tabular}{c|c|c|c|c}
    \hline
    Dataset    & Size      & Normal Data & Attack Data & Attack Types \\ 
    \hline
    X-IIoTID \cite{al2021x}   & 820,502   & 421,417     & 399,417     & 18           \\
    \hline
    WUSTL-IIoT \cite{zolanvari2021wustl} & 1,194,464 & 1,107,448   & 87,016      & 4       \\
    \hline
    CICIDS2017 \cite{Sharafaldin2018TowardGA} & 2,830,743 & 2,273,097 & 557,646 & 15 \\
    \hline
    UNSW-NB15 \cite{moustafa2015unsw}
 & 257,673 & 164,673 & 93,000 & 10 \\
    \hline
    \end{tabular}}
\end{table}
%%%%%%%%%%%%%%%%%%%%%%%%%%%%%%%%%%%%%%%%%%%%%%%%%%%%%%%%%%%%%%%%%%%%%%%%%%%

\textbf{Baselines:} %Due to the novelty of this problem formulation, there are no directly comparable methods. However, the most similar widely studied problem would be unsupervised continual learning (UCL). Therefore, 
We evaluate our algorithm against two SOTA unsupervised continual learning (UCL) algorithms: the Autonomous Deep Clustering Network (\textbf{ADCN}) \cite{ashfahani2023unsupervised}, and an autoencoder paired with K-Means clustering. The autoencoder K-Means model is combined with Learning without Forgetting \cite{lwf2019Li} continual learning loss; we refer to this model as \textbf{LwF}. Note that both \textbf{ADCN} and \textbf{LwF} require a small amount of labeled normal and attack data to perform classification. We also compare our approach against SOTA ND methods: local outlier factor (\textbf{LOF})\cite{Faber_2024}, one-class support vector machine (\textbf{OC-SVM})\cite{Faber_2024}, principal component analysis (\textbf{PCA})\cite{rios2022incdfm}, and Deep Isolation Forest (\textbf{DIF}) \cite{xu2023deep}. 
%We train the ND algorithms on the clean subset of normal data, $N_c$, and evaluate their performance on the remainder of the dataset. 
Since these ND models cannot be retrained on unlabeled contaminated data, continual learning is not feasible for these methods.

%an autoencoder with K-Means clustering paired with SOTA Learning without Forgetting (LwF) continual loss (LwF) \cite{lwf2019Li}.
%Notably, many SOTA UCL algorithms rely on image-specific contrastive pairs, which is not directly applicable to intrusion detection \cite{madaan2022representational, yu2023scale, fini2022self, liu2024unsupervised}.

%%%%%%%%%%%%%%%%%%%%%%%%%%%%%%%%%%%%%%%%%%%%%%%%%%%%%%%
\begin{figure*}
    \centering
    \includegraphics[width=.95\linewidth]{figures/cl_experiments.pdf}
    \caption{Continual learning metric results of ADCN\cite{ashfahani2023unsupervised}, LwF\cite{lwf2019Li}, and \Design{}}
    \label{fig:continual_methods_results}
\end{figure*}
%%%%%%%%%%%%%%%%%%%%%%%%%%%%%%%%%%%%%%%%%%%%%%%%%%%%%%%

\textbf{Evaluation Metrics:} To evaluate the model performance, we report $F_{1}$ score. Since there is a class imbalance within these datasets, to simulate real world IDS, $F_{1}$ score gives an accurate idea on attack detection. For the continual learning methods, we evaluate their performance at the end of each training experience on all experience test sets. This generates a matrix of $F_{1}$ score results $R_{ij}$ such that $i$ is the current training experience, and $j$ is the testing experience. To summarize this matrix of results, we report widely used CL metrics \cite{diaz2018don}: average $F_{1}$ score on current experience (AVG), forward transfer (FwdTrans), and backward transfer (BwdTrans). For a matrix $R_{ij}$ with $m$ total experiences, our metrics are formulated as follows: $\text{AVG}_{F_1} = \frac{\sum_{i = j} R_{ij}}{m}$; $\text{FwdTrans}_{F_1} = \frac{\sum_{j>i} R_{ij}}{\frac{m * (m-1)}{2}}$; $\text{BwdTrans}_{F_1} = \frac{\sum_{i}^m R_{mi} - R_{ii}}{\frac{m * (m-1)}{2}}$.
AVG is the average performance on the current test experience at every point of training. FwdTrans is the average performance on ``future'' experiences, which simulates performance on zero-day attacks. Finally, BwdTrans is the average change in performance of ``past'' test experiences at a ``future'' point of training. A negative BwdTrans indicates catastrophic forgetting, whereas a positive BwdTrans  indicates the model actually improved performance on past experiences after learning a future experience. Overall, AVG measures seen attacks, FwdTrans measures zero-day attacks, and BwdTrans measures forgetting. For all metrics, a higher positive result indicates a better performance. 

We also report the threshold-free metric Precision-Recall Area Under the Curve (PR-AUC) \cite{praucDavid06}. Since \Design{} requires selecting a threshold, PR-AUC allows us to assess model performance independently of the threshold. We choose PR-AUC over Receiver Operating Characteristic Area Under the Curve (ROC-AUC) because ROC-AUC can give misleadingly high results in the presence of class imbalance \cite{praucDavid06}.

\textbf{Hyperparameters:} %For $L_{CND}$ hyperparameters are the number of K-Means clusters $K$, the reconstruction loss strength $\lambda_R$,  the continual learning loss strength $\lambda_{CL}$, and the cluster separation loss margin $m$. 
We utilize \textit{elbow method} \cite{han2011data} for determining the number of clusters $K$. 
%It tests a range of $K$ values and then selects the value   where there is a significant change in slope, called the elbow point. 
%This resulted in $K$ values between 100-500. 
We set $\lambda_R$ and $\lambda_{CL}$ to 0.1, and for $m$ we use 2 after careful experimentation. For the AE modules of \Design{}, we use 4-layer MLP with 256 neurons in the hidden layers. We train it using Adam optimizer \cite{kingma2017adammethods} with a learning rate of 0.001. For PCA, we use the explained variance method and set it to 95\% \cite{rios2022incdfm}.

\textbf{Hardware:} We run our experiments on NVIDIA GeForce RTX 3090 GPU, with a AMD EPYC 7343 16-Core processor.

\subsection{Results}

\textbf{Continual Learning Comparison:} Fig.~\ref{fig:continual_methods_results} presents the results of our approach \Design{} compared with ADCN\cite{ashfahani2023unsupervised} and LwF\cite{lwf2019Li}. \Design{} shows the best performance on both seen (AVG) and unseen (FwdTrans) attacks across all datasets. \Design{} also has the highest BwdTrans on all except one dataset (UNSW-NB15). The average BwdTrans of \Design{} (0.87\%) is higher than the average BwdTrans of both ADCN (-0.06\%) and LwF (0.09\%). Notably, the BwdTrans of \Design{} is positive for three datasets. Indicating past experiences actually improve after training on future experiences for these datasets. Given the high FwdTrans as well, our approach finds features that generalize well to future experiences. 

Table~\ref{tab:improvement} shows the improvement of \Design{} over the UCL baselines on all datasets. Bold and underlined cases indicate the best and the second best improvements with respect to each metric, respectively. These improvements were calculated by comparing the performance of \Design{} to the baselines, where the improvement values represent the proportional increase over the baseline performance. We do not include BwdTrans because a proportional increase does not make sense for a metric that can be negative. \Design{} has up to $4.50\times$ and $6.1\times$ AVG improvement on ADCN and LwF, respectively. In addition, \Design{} has up to $6.47\times$ and $3.47\times$ FwdTrans improvement on ADCN and LwF. Averaged across all datasets, \Design{} shows a $1.88\times$ and $1.78\times$ improvement on AVG, and a $2.63\times$ and $1.60\times$ improvement on FwdTrans, compared to ADCN and LwF, respectively. %These results underscore the benefit of our continual novelty detection method \Design{}. The notably high FwdTrans score emphasizes how novelty detection can be used to identify unseen anomalous data, thereby significantly enhancing performance on zero-day attacks.

Overall, these results highlight the benefit of continual ND over UCL methods for IDS. \Design{}, with its PCA-based novelty detector, excels by effectively harnessing the normal data to identify attacks. A key strength of our approach lies in the assumption that normal data forms a distinct class, while everything else is treated as anomalous. This assumption is particularly well-suited to IDS. In contrast, methods like ADCN and LwF do not make this distinction where they handle both normal and attack data similarly, limiting their ability to fully exploit the inherent structure of the data. 



% %%%%%%%%%%%%%%%%%%%%%%%%%%%%%%%%%%%%%%%%%%%%%%%%%%%%%%%
% \begin{table}[]
% \centering
% \caption{\Design{} Percentage Improvement over UCL Baselines on AVG and FwdTrans}
% \label{tab:improvement}
% \begin{tabular}{|c|c|c|c|}
% \hline
% Baseline      & Dataset    & AVG  & FwdTrans  \\ \hline
% ADCN\cite{ashfahani2023unsupervised}          & X-IIoTID   & 101.88\%        & 400.35\%        \\ \cline{2-4} 
%               & WUSTL-IIoT & 349.86\%        & 546.68\%        \\ \cline{2-4} 
%               & CICIDS2017 & 37.19\%         & 73.46\%         \\ \cline{2-4} 
%               & UNSW-NB15  & 29.25\%         & 43.90\%         \\ \hline
% LwF\cite{lwf2019Li} & X-IIoTID   & 46.43\%         & 35.39\%         \\ \cline{2-4} 
%               & WUSTL-IIoT & 510.92\%        & 246.81\%        \\ \cline{2-4} 
%               & CICIDS2017 & 92.72\%         & 163.81\%        \\ \cline{2-4} 
%               & UNSW-NB15  & 11.07\%         & 2.20\%          \\ \hline
% \end{tabular}
% \end{table}
% %%%%%%%%%%%%%%%%%%%%%%%%%%%%%%%%%%%%%%%%%%%%%%%%%%%%%%%

%%%%%%%%%%%%%%%%%%%%%%%%%%%%%%%%%%%%%%%%%%%%%%%%%%%%%%%
\begin{table}[]
\centering
\caption{\Design{} Improvement over UCL Baselines}
\label{tab:improvement}
\scalebox{1}{
\begin{tabular}{|c|c|c|c|}
\hline
Baseline      & Dataset    & AVG  & FwdTrans  \\ \hline
ADCN\cite{ashfahani2023unsupervised}  & X-IIoTID   & $\underline{2.02\times}$  & $\underline{5.00\times}$   \\ \cline{2-4} 
                                      & WUSTL-IIoT & $\mathbf{4.50\times}$  & $\mathbf{6.47\times}$   \\ \cline{2-4} 
                                      & CICIDS2017 & $1.37\times$  & $1.73\times$   \\ \cline{2-4} 
                                      & UNSW-NB15  & $1.29\times$  & $1.44\times$   \\ \hline
LwF\cite{lwf2019Li}                   & X-IIoTID   & $1.46\times$  & $1.35\times$   \\ \cline{2-4} 
                                      & WUSTL-IIoT & $\mathbf{6.11\times}$  & $\mathbf{3.47\times}$   \\ \cline{2-4} 
                                      & CICIDS2017 & $\underline{1.93\times}$  & $\underline{2.64\times}$   \\ \cline{2-4} 
                                      & UNSW-NB15  & $1.11\times$  & $1.02\times$   \\ \hline
\end{tabular}}
\end{table}

%%%%%%%%%%%%%%%%%%%%%%%%%%%%%%%%%%%%%%%%%%%%%%%%%%%%%%%

%Figure~\ref{fig:XIIoT_graph} shows the $F_{1}$ score of ADCN and \Design{} for each experience on both datasets. Similarly, we use green and red colors for \Design{} and ADCN respectively. Notably for \Design{}, the $F_{1}$ score of each experience has little change over training time. This highlights the strength of novelty detection for IDSs, as even before seeing attacks \Design{} has good performance. On the other hand, ADCN test experiences do not improve until the associated training experience, meaning ADCN does not have an ability to generalize to future attacks. ADCN utilizes a subset of labeled data to assign labels to clusters. This subset of labeled might be causing ADCN to overfit to the attacks within the current experience, therefore leading ADCN to not generalize well. We can also clearly see that our approach is consistently better (higher $F_{1}$ score) than the state-of-the-art ADCN. 

% %%%%%%%%%%%%%%%%%%%%%%%%%%%%%%%%%%%%%%%%%%%%%%%%%%%%%%%
% \begin{figure*}[t]
%     \centering
%     \begin{subfigure}[t]{\linewidth}
%         \centering
%         \includegraphics[width=\linewidth]{figures/X-IIoTID-experiences.pdf}
%         \caption{X-IIoTID}
%         \label{fig:ADCN_XIIoT_results}
%     \end{subfigure}
%     \begin{subfigure}[t]{\linewidth}
%         \centering
%         \includegraphics[width=\linewidth]{figures/WUSTL-IIoT-experiences.pdf}
%         \caption{WUSTL-IIoT}
%         \label{fig:WUSTL-}
%     \end{subfigure}
%     \caption{$F_1$ Score of ADCN and \Design{} of each test experience over training experiences.}
%     \label{fig:XIIoT_graph}
% \end{figure*}
% %%%%%%%%%%%%%%%%%%%%%%%%%%%%%%%%%%%%%%%%%%%%%%%%%%%%%%%

\textbf{Novelty Detectors Comparison:} Fig.~\ref{fig:novelty_methods_results} compares LOF\cite{Faber_2024}, OC-SVM\cite{Faber_2024}, PCA\cite{rios2022incdfm}, and DIF \cite{xu2023deep} with \Design{} on all datasets. The average $F_{1}$ score of the novelty detection methods are compared to the AVG of \Design{}.  It can be seen \Design{} outperforms all other methods across all datasets. The two best performing methods are DIF and PCA. The average $F_{1}$ score improvement across all datasets of \Design{} is $1.16\times$ and $1.08\times$ over DIF and PCA, respectively. These results highlight the critical role of leveraging information from unsupervised data streams. Unlike these ND algorithms, \Design{} is capable of continuously learning from this unsupervised data, enabling it to enhance PCA reconstruction over time. By integrating evolving data patterns, \Design{} not only adapts to new anomalies but also improves its overall detection accuracy, demonstrating a clear advantage in dynamic environments.

%Given that \Design{} employs PCA detection, this indicates that the CFE effectively extracts useful features from the unlabeled training experiences. T

%%%%%%%%%%%%%%%%%%%%%%%%%%%%%%%%%%%%%%%%%%%%%%%%%%%%%%%   
\begin{figure}
    \centering
    \includegraphics[width=0.9\linewidth]{figures/novelty_detectors_experiments.pdf}
    \caption{Average $F_1$ score on all experiences of \Design{} and novelty detection methods: LOF, OC-SVM, PCA, DIF}
    \label{fig:novelty_methods_results}
\end{figure}
%%%%%%%%%%%%%%%%%%%%%%%%%%%%%%%%%%%%%%%%%%%%%%%%%%%%%%%
%%%%%%%%%%%%%%%%%%%%%%%%%%%%%%%%%%%%%%%%%%%%%%%%%%%%%%% 
\begin{figure}
    \centering
    \includegraphics[width=0.86\linewidth]{figures/novelty_detectors_pr_auc.pdf}
    \caption{Thresholding Free Evaluation of \Design{}}
    \label{fig:thresholding_free}
\end{figure}

%%%%%%%%%%%%%%%%%%%%%%%%%%%%%%%%%%%%%%%%%%%%%%%%%%%%%%%

\textbf{Pre-threshold Evaluation:} While thresholding plays a crucial role in attack decision-making, evaluating model prediction performance before applying threshold is also important. The UCL algorithms (ADCN\cite{ashfahani2023unsupervised} and LwF\cite{lwf2019Li}) do not output anomaly scores because they select classes based on the closest labeled cluster. Therefore we compare against the two best ND methods: DIF\cite{xu2023deep} and PCA\cite{rios2022incdfm}. Fig.~\ref{fig:thresholding_free} presents the PR-AUC values of DIF, PCA, and \Design{}. It can be seen that \Design{} provides the best threshold free results, which aligns with the threshold-based results presented earlier. The strong performance of \Design{} in both pre-threshold and threshold-based evaluations demonstrates that the model is robust regardless of the decision threshold. 

\subsection{Ablation Study}

To demonstrate the impact of our loss function components, we perform an ablation study. Table~\ref{tab:ablation_loss} shows the results of \Design{} with each loss function removed to demonstrate their individual effectiveness. Bold and underlined cases indicate the best and the second best performances with respect to each metric, respectively. \Design{} without reconstruction loss ($L_R$) and \Design{} without cluster separation loss ($L_{CS}$) performs worse in all categories. \Design{} without both $L_R$ and continual learning loss ($L_{CL}$) actually performs better AVG but has worse BwdTrans and FwdTrans. AVG does not account for past experiences, so the significantly negative BwdTrans indicates \Design{} w/o $L_R$ and $L_{CL}$ forgets, and therefore would perform worse on those experiences in the future. This would make sense as a regularization loss to improve continual learning would slightly decrease performance in non-continual scenario. Overall \Design{} has the best results when taking every metric category into account. Notably the low BwdTrans and FwdTrans of \Design{} (w/o $L_R$) showcases how the reconstruction loss helps \Design{} generalize better to unseen and past data. This highlights the power of $L_R$ to provide good features for continual learning. 

%%%%%%%%%%%%%%%%%%%%%%%%%%%%%%%%%%%%%%%%%%%%%%%%%%%%%%%%%%%%%%%%%%%%%
\begin{table}[]
\caption{Ablation Study of \Design{} Loss Functions}
\label{tab:ablation_loss}
\centering
\begin{tabular}{|c|c|c|c|}
\hline
Strategy                         & AVG              & BwdTrans        & FwdTrans         \\ \hline
CND-IDS                          &\underline{76.92\%}    & \textbf{0.87\%} & \textbf{73.70\%} \\ \hline
CND-IDS (w/o $L_{CS}$)           & 66.23\%          & \underline{0.09\%}    & 70.26\%          \\ \hline
CND-IDS (w/o $L_R$)              & 72.86\%          & -5.44\%         & 67.82\%          \\ \hline
CND-IDS (w/o $L_R$ and $L_{CL}$) & \textbf{79.92\%} & -11.26\%        & \underline{71.01\%}    \\ \hline
\end{tabular}
\end{table}
%%%%%%%%%%%%%%%%%%%%%%%%%%%%%%%%%%%%%%%%%%%%%%%%%%%%%%%%%%%%%%%%%%%%%%%

\subsection{Overhead Analysis}
%%%%%%%%%%%%%%%%%%%%%%%%%%%%%%%%%%%%%%%%%%%%%%%%%%%%%%%%%%%
% \begin{table}[]
% \centering
% \caption{Average training time and inference time per sample across all datasets in milliseconds}
% \label{tab:overhead}
% \begin{tabular}{|c|c|c|}
% \hline
% Strategy               & Inference Time(ms) \\ \hline
% \Design{}                   & 0.0019             \\ \hline
% ADCN\cite{ashfahani2023unsupervised}    & 0.4061             \\ \hline
% LwF\cite{lwf2019Li}           & 0.0677             \\ \hline
% DIF\cite{xu2023deep}         & 1.0535             \\ \hline
% PCA\cite{rios2022incdfm}       & 0.0018             \\ \hline
% \end{tabular}
% \end{table}
%%%%%%%%%%%%%%%%%%%%%%%%%%%%%%%%%%%%%%%%%%%%%%%%%%%%%%%%%%%%%
\begin{table}[]
\centering

\caption{Average inference time (in ms) per test sample}
\label{tab:overhead}
\scalebox{0.95}{
\begin{tabular}{|c|c|c|c|c|c|}
\hline
Strategy           & \Design{} & ADCN   & LwF    & DIF    & PCA    \\ \hline
Inference Time (ms) & \underline{0.0019}                     & 0.4061 & 0.0677 & 1.0535 & \textbf{0.0018} \\ \hline
\end{tabular}}
\end{table}
%%%%%%%%%%%%%%%%%%%%%%%%%%%%%%%%%%%%%%%%%%%%%%%%%%%%%%%%
Table~\ref{tab:overhead} evaluates the inference overhead of \Design{} compared to ADCN \cite{ashfahani2023unsupervised}, LwF \cite{lwf2019Li}, DIF \cite{xu2023deep}, and PCA \cite{rios2022incdfm}. %, excluding OC-SVM \cite{Faber_2024} and LOF \cite{Faber_2024} due to poor performance. 
\Design{} offers the fastest inference time among continual learning methods. Out of novelty detection methods, \Design{} is second only to PCA. We attribute the efficiency of \Design{} to avoiding the clustering classification used by LwF and ADCN. %\Design{} instead uses PCA reconstruction, which is much quicker than comparing data points to clusters. In addition, 
The difference between \Design{} and PCA is minimal, only 0.0001 milliseconds slower, due to the additional but lightweight step of encoding the data. Considering that the average median flow duration across datasets is 27.77 milliseconds, the overhead introduced by \Design{} is negligible in the context of real-time traffic flow.

%In this section we analyze the inference overhead of \Design{} compared to ADCN\cite{ashfahani2023unsupervised}, LwF\cite{lwf2019Li}, DIF\cite{xu2023deep}, and PCA\cite{rios2022incdfm}. We do not include OC-SVM\cite{Faber_2024} and LOF \cite{Faber_2024} due to weak performance. Table~\ref{tab:overhead} shows the average inference time in milliseconds per sample across all datasets. \Design{} has the best inference time besides PCA. We attribute this good inference time to \Design{} not using clustering classification like LwF and ADCN. Evidently, PCA reconstruction utilized by \Design{} is more time efficient than having to compare a data point to all saved clusters. Compared to pure PCA reconstruction, \Design{} is only 0.0001 ms slower. This small increase in inference time is due to the only added computation at inference is encoding the data with the encoder, which is simply a 4 layer MLP. Across all datasets, the average median travel flow duration is 27.77 ms, and the dataset with the quickest median travel flow is UNSW with 4.29 ms. Therefore the overhead introduced by \Design{} is irrelevant compared to the speed of the traffic flow. 

%\label{section:ablation_study}
%To assess the impact of our design choices, we perform an ablation study. Our goal is to analyze (i) threshold function evaluation, and (ii) novelty detection algorithm selection. 

 

%\textbf{Threshold Function Evaluation:} AE, PCA, and \Design{} all require a threshold to classify an anomaly based on the anomaly score. In all previously reported results, we select a widely used threshold that maximizes the $F_{1}$ score on the test set, i.e., Best-F. %This is not realistic but was used to compare the effectiveness of these methods. In this section 
%Here, we analyze three different threshold methods, which we denote: Best-F \cite{su2019robust}, Top-k \cite{zong2018deep}, and validation percentile (ValPer). Best-F uses the threshold that maximizes the $F_{1}$ score on test set. Top-k utilizes the contamination ratio $r$ of the test set, such that $r$ is the percentage of anomalies within the test set. Top-k selects a threshold so that the percentile of data within the test set classified as anomalies is equal to $r$. ValPer utilizes a validation set of normal data, and selects a threshold such that 99.7\% (3 standard deviations) of the normal data is within this threshold. 
%ValPer is the most realistic method as it does not rely on any information from the test set. 
%A breakdown of the $F_{1}$ score results for the different threshold methods is show in Table~\ref{tab:thresholding_results} where the best within each category is bolded. Overall Best-F performs significantly better than the other threshold methods, which is obvious as Best-F is an upper-bound for threshold selection. However the significant gap highlights the importance of threshold selection. Most importantly, \Design{} still performs better than PCA and AE through all threshold methods. 

%%%%%%%%%%%%%%%%%%%%%%%%%%%%%%%%%%%%%%%%%%%%%%%%%%%%%%%
%\begin{table}[]
%    \centering
%    \caption{Threshold Function Evaluation}
%    \resizebox{.97\columnwidth}{!}{
%    \begin{tabular}{c|c|c|c|c}
%        \hline
%         Dataset & Stategy & Best-F & Top-k & ValPer\\
%         \hline
%         & PCA  & 70.9 & 4.03 & 3.56 \\
%         \cline{2-5}
%         X-IIoTID & AE  & 75.6 & 4.03 & 29.4 \\
%         \cline{2-5}
%         & \Design{} & \textbf{78.8} & \textbf{5.63} &  %\textbf{52.9} \\	
%         \hline
%        & PCA  & 85.6 &19.9 & 52.8\\
%         \cline{2-5}
%         WUSTL-IIoT & AE  & 79.6 &19.7 & 37.8\\
%         \cline{2-5}
%         & \Design{} & \textbf{88.2} & \textbf{21.1} & \textbf{55.6}\\	
%         \hline
%    \end{tabular}}
%    \label{tab:thresholding_results}
%\end{table}
%%%%%%%%%%%%%%%%%%%%%%%%%%%%%%%%%%%%%%%%%%%%%%%%%%%%%%%

% %%%%%%%%%%%%%%%%%%%%%%%%%%%%%%%%%%%%%%%%%%%%%%%%%%%%%%%
% \begin{figure}
%     \centering
%     \includegraphics[width=0.95\linewidth]{figures/novelty_ablation.pdf}
%     \caption{Comparison of \Design{} with PCA and AE novelty detection models}
%     \label{fig:novelty_ablation_results}
% \end{figure}
% %%%%%%%%%%%%%%%%%%%%%%%%%%%%%%%%%%%%%%%%%%%%%%%%%%%%%%%

% \textbf{Novelty Detection Algorithm Selection:} For \Design{}, we select PCA as the novelty detection algorithm. As shown in Figure~\ref{fig:novelty_methods_results}, both PCA and AE perform well for detecting intrusions. Therefore, we test both AE and PCA as the novelty detection methods for \Design{}. Figure~\ref{fig:novelty_ablation_results} illustrates the AVG performance of \Design{} with AE and PCA as the novelty detection models. It is evident that PCA outperforms AE, justifying our selection of this algorithm for novelty detection. This could be because the CFE utilizes SAEs, which generate features based on the same reconstruction loss used by AE to classify anomalies. It may be beneficial to use PCA as it deconstructs the input in a different manner, thereby identifying different features and functioning better in conjunction with the SAE-based CFE.

\subsection{Datasets}
\begin{table}[t]
\caption{Description of datasets.}
% \vspace{-0.2cm}
\label{tbl:dataset}
\resizebox{\columnwidth}{!}{%
\begin{tabular}{ |c|c|c|c|c|c| } 
\hline
Dataset &  Users & Item & Interactions & Categories for Items\\
 \hline
ML 100K \cite{10.1145/2827872} & 943 &  1,349  & 99,287 & 18 \\
ML 1M \cite{10.1145/2827872} & 6,040   & 3,416  & 999,611 & 18 \\
Yelp \cite{mansoury2019biasdisparitycollaborativerecommendation} & 1,316   & 1,272  & 97,991 & 21\\
\hline
\end{tabular}
}
    % \vspace{-16pt}
\end{table}
% For the datasets used we use k-core filtering to remove users and items that can be considered ``sparse''. 
Our experiments are performed on three recommendation datasets summarized in Table~\ref{tbl:dataset}.
The data is split into training, validation, and test sets with a 70:10:20 ratio following user-based split scheme.
% Both datasets have 18 genres including \textit{action}, \textit{romance} etc. 
% We are using these datasets in an implicit feedback setting, where we work with binary data (e.g., interacted =1 and not interacted = 0).
% We include all genres for all calculations for our metrics.
It's important to highlight that items can fall under multiple categories. For clarity and ease of understanding, we focus on four representative categories—\textit{Action}, \textit{Romance}, \textit{Sci-Fi}, and \textit{Drama} from the MovieLens datasets and \textit{Coffee, Tea \& Desserts}, \textit{Arts \& Entertainment}, \textit{Travel \& Transportation} and \textit{Asian} from the yelp dataset — when visualizing our metric values; however all categories are included in our experiments.

% We are using the minimum interaction threshold of 5, meaning if a user has less than 5 interactions their interactions are filtered out. Similarly, in the case of items, they are filtered out if they don't have 5 or more interactions.

\subsection{Evaluating Performance}
 For measuring the performance of recommendations made, we use HitRatio@k calculated for each user and then averaged. Additionally, we use a ranking-based metric NDCG@k (Normalized Discounted Cumulative Gain) which gives higher relevance to items appearing higher in the ranked list. This too, is calculated for each user using their top \(k\) recommendations and then averaged. We calculate these values for \(k=50\).
% \[
% \text{Hitratio@k(u)} = 
% \begin{cases}
% 1, & \text{if }  \text{K} \cap \text{GT}  > 0 \\
% 0, & \text{otherwise}
% \end{cases}
% \]

% \[
% \text{Hitratio@k} = \frac{1}{N} \sum_{i=1}^{N} \text{Hitratio@k}(u_i)
% \]

% where GT is the ground truth set of items for a given user and K is the top \(k\) ranked items for that user.




\subsection{Base Recommendation Models}
For analyzing which models manifest gender bias we evaluate several recommendation approaches. When selecting these models, we include a variety of algorithms, ranging from traditional ones to more modern ones including Matrix Factorization, UserKNN, ItemKNN, NeuMF and VAE-CF (more detailed information on these and the fairness-aware models can be found in Section \ref{sec:baseline}).
% \vspace{-2pt}
% the classic collaborative approach of Matrix Factorization (MF), K-Nearest Neighbourhood algorithms UserKNN and ItemKNN, and two deep models including Neural MF (NeuMF) and Variational Auto-Encoder Collaborative Filtering (VAE-CF).

Additionally, we include a regularization-based fairness-aware model BeyondParity \cite{NIPS2017_e6384711} and a counterfactually fair recommendation model SM-GBiasedMF \cite{li2021towards}. To weigh the fairness-aware loss for SM-GBiasedMF, we use \(\lambda=20\) for all experiments since this value seems to work well as mentioned in \cite{li2021towards}. To make a fair comparison, we use BeyondParity's \(U_{val}\) (refer to Equation \ref{eq:bp}) as a regularization term in the same setting as shown in Equation~\ref{eq:fairequation}, with the same value of alpha we use for our MF model. We use an early stopping strategy for the three models being compared, monitoring NDCG@20 with a delta of 0.0005 and patience of 10 epochs with a maximum number of iterations of 50 and 100 for the 100K and 1M datasets, respectively.

We utilize the Cornac framework \cite{JMLR:v21:19-805}, with customizations to meet our requirements, including the implementations of differentiable ways to obtain top k category distribution for calculating our gender loss term for different models. When calculating our regularization term, we generate a top \(k\) recommendation list of size \(k=50\). Additionally, all our results for all metrics for GBS (except CRP) use the top 50 recommended items. For GBS(CRP) we use top \(\lfloor R_g \rfloor\) as explained for Equation \ref{eq:M6}.



% % \section{Dataset}
% % \begin{table}[t]
  \centering
    \caption{Dataset statistics}
    \resizebox{0.49\textwidth}{!}{
        % \begin{threeparttable}

\begin{tabular}{lrrrrr}
    \toprule
    \textbf{ DATASET } & \textbf{ \#Users } & \textbf{ \#Items } & \textbf{\#Interactions}  & \textbf{Avg.Inter.} & \textbf{Sparsity}\\
    \midrule
     Gowalla  & 29,858 & 40,981& 1,027,370 & 34.4 & 99.92\% \\
     Yelp2018  & 31,668 & 38,048 & 1,561,406 & 49.3 & 99.88\% \\ 
     MIND  & 38,441 & 38,000 & 1,210,953 & 31.5 & 99.92\% \\ 
     MIND-Large  & 111,664 & 54,367 & 3,294,424 & 29.5 & 99.95\% \\
    \bottomrule
    \end{tabular}
        % \end{threeparttable}
    }
  \label{tab:datasets}%
\end{table}%
\section{Analysis of the proposed metrics}
In this section, we discuss the results we obtained from our experiments through a comprehensive analysis.
% We begin by presenting the two main assumptions we will make to analyze \Cref{alg:uSCG,alg:SCG}. The first is an assumption on the Lipschitz-continuity of $\nabla f$ with respect to the norm $\|\cdot\|_{\ast}$ restricted to $\mathcal{X}$. We do not assume this norm to be Euclidean which means our results apply to the geometries relevant to training neural networks.
\begin{assumption}\label{asm:Lip} The gradient $\nabla f$ is $L$-Lipschitz with $L \in (0,\infty)$, i.e.,
    \begin{equation}
    \|\nabla f(x) - \nabla f(x)\|_{\ast}
    \leq
    L\|x-y\|
    \quad \forall x,y \in \mathcal X.
    \end{equation}
Furthermore, $f$ is bounded below by $\fmin$.
\end{assumption}
Our second assumption is that the stochastic gradient oracle we have access to is unbiased and has a bounded variance, a typical assumption in stochastic optimization.
\begin{assumption}\label{asm:stoch}
The stochastic gradient oracle $\nabla f(\cdot,\xi):\mathcal X\rightarrow \mathbb{R}^d$ satisfies.
    \begin{assnum}
        \item \label{asm:stoch:unbiased}
            Unbiased:
            \(%
                \mathbb{E}_{\xi}\left[\nabla f(x,\xi)\right] = \nabla f(x) \quad \forall x \in \mathcal X
            \).%
        \item  \label{asm:stoch:var}
            Bounded variance:\\
            \(%
                \mathbb{E}_{\xi}\left[\|\nabla f(x,\xi)-\nabla f(x)\|_2^2\right] \leq \sigma^2  \quad \forall x \in \mathcal X,\sigma\geq 0
            \).%
    \end{assnum}
\end{assumption}

With these assumptions we can state our worst-case convergence rates, first for \Cref{alg:uSCG} and then for \Cref{alg:SCG}. 

\looseness=-1To bridge the gap between theory and practice, we investigate these algorithms when run with a \emph{constant} stepsize $\gamma$, which depends on the specified horizon $n\in\mathbb{N}^*$, and momentum which is either constant $\alpha\in(0,1)$ (except for the first iteration where we take $\alpha=1$ by convention) or \emph{vanishing} $\alpha_k\searrow 0$. The exact constants for the rates can be found in the proofs in \Cref{app:analysis}; we try to highlight the dependence on the parameters $L$ and $\rho$, which correspond to the natural geometry of $f$ and $\mathcal{D}$, explicitly here. Our rates are non-asymptotic and use big O notation for brevity.

\begin{toappendix}
\label{app:analysis}
In this section we present the proofs of the main convergence results of the paper as well as some intermediary lemmas that we will make use of along the way. Throughout this section, we adopt the notation:
\begin{align*}
\text{(stochastic gradient estimator error)} && \lambda^k &:= d^k-\nabla f(x^k) \\
\text{(diameter of $\mathcal{D}$ in $\ell_2$ norm)} && D_2 &:= \max_{x,y\in\mathcal{D}}\norm{x-y}_2 \\
\text{(radius of $\mathcal{D}$ in $\ell_2$ norm)} && \rho_2 &:= \max_{x\in\mathcal{D}}\norm{x}_2 \\
\text{(norm equivalence constant)} && \zeta &:= \max_{x\in\mathcal{X}}\frac{\norm{x}_{\ast}}{\norm{x}_2} \\
\text{(Lipschitz constant of $\nabla f$ with respect to $\norm{\cdot}_{2}$)} && L_2 &:= \inf \{M>0\colon \forall x,y\in\mathcal{X}, \norm{\nabla f(x)-\nabla f(y)}_{2}\leq M\norm{x-y}_{2}\}
\end{align*}
We analyze each algorithm separately, although the analysis is effectively unified between the two, modulo constants. This is done in \Cref{subsec:uSCG,subsec:SCG}, respectively. Our convergence analysis proceeds in three steps: we begin by establishing a template descent inequality for each algorithm via the descent lemma. Next, we analyze the behavior of the second moment of the error $\mathbb{E}[\norm{\lambda^k}_{2}^2]$ under different choices for $\alpha$. Then, we combine these results to derive a convergence rate. Finally, we note that when analyzing algorithms with constant momentum, we will still always take $\alpha=1$ on the first iteration $k=1$.

\subsection{Convergence analysis of \ref{eq:uSCG}}\label{subsec:uSCG}
We begin with the analysis of \Cref{alg:uSCG} by establishing a generic template inequality for the dual norm of the gradient at iteration $k$. This inequality holds regardless of whether the momentum $\alpha_k$ is constant or vanishing, as long as it remains in $(0,1]$.
\begin{lemma}[\ref{eq:uSCG} template inequality]
\label{lem:uSCGtemplate1}
    Suppose \Cref{asm:Lip} holds. Let $n\in\mathbb{N}^*$ and consider the iterates $\{x^{k}\}_{k=1}^n$ generated by \Cref{alg:uSCG} with a constant stepsize $\gamma>0$.
    Then we have
    \begin{equation}
        \mathbb{E}[\norm{\nabla f(\bar{x}^n)}_2^2]\leq \frac{\mathbb{E}[f(x^{1})-\fmin]}{\rho\gamma n} +\frac{L\rho\gamma}{2} + \frac{1}{n}\left(\frac{\rho_2}{\rho}+\zeta\right)\sum\limits_{k=1}^n\sqrt{\mathbb{E}[\norm{\lambda^{k}}_2^2]}.
    \end{equation}
\end{lemma}
\begin{proof}
    Under \Cref{asm:Lip}, we can use the descent lemma for the function $f$ at the points $x^{k}$ and $x^{k+1}$ to get, for all $k\in\{1,\ldots,n\}$,
    \begin{equation}\label{eq:lem:uSCGtemplate1:first2}
        \begin{aligned}
            f(x^{k+1})&\leq f(x^{k})+ \langle \nabla f(x^{k}),x^{k+1}-x^{k}\rangle +\tfrac{L}{2}\norm{x^{k+1}-x^{k}}^{2}
            \\
            &= f(x^{k})+\langle \nabla f(x^{k})-d^{k},x^{k+1}-x^{k}\rangle + \langle d^{k},x^{k+1}-x^{k}\rangle+\tfrac{L}{2}\norm{x^{k+1}-x^{k}}^{2}
            \\
            &= f(x^{k})+\gamma \langle \nabla f(x^{k})-d^{k},\lmo (d^{k})\rangle+\gamma \langle d^{k},\lmo(d^{k})\rangle +\tfrac{L\gamma^{2}}{2}\norm{\lmo(d^{k})}^{2}
            \\
            &\leq f(x^{k})+\gamma \rho_{2}\norm{\lambda^{k}}_{2}+\gamma \langle d^{k},\lmo(d^{k})\rangle +\tfrac{L\gamma^{2}}{2}\rho^{2},
        \end{aligned}
    \end{equation}
    the final step employing Cauchy-Schwarz, the definition of $\lambda^k$, and the definition of $\rho_2$ as the radius of $\mathcal{D}$ in the $\norm{\cdot}_2$ norm.
    By definition of the dual norm we have, for all $u\in\mathcal{X}$,
    \begin{equation*}
        \|u\|_{\ast} = \max\limits_{v\colon \|v\|\leq 1}\langle u,v\rangle = \max_{v\in\mathcal{D}}\langle u,\tfrac{1}{\rho}v\rangle= -\langle u, \tfrac{1}{\rho}\lmo(u)\rangle
    \end{equation*}
    which means that, for all $k\in\{1,\ldots,n\}$,
    \begin{equation*}
        \gamma \langle d^k, \lmo(d^k)\rangle = \gamma\rho\langle d^k,\tfrac{1}{\rho}\lmo(d^k)\rangle = -\gamma\rho\|d^k\|_{\ast}.
    \end{equation*}
    Plugging this expression for $\gamma\langle d^k,\lmo(d^k)\rangle$ into \eqref{eq:lem:uSCGtemplate1:first2} gives, for all $k\in\{1,\ldots,n\}$,
    \begin{equation*}
        \begin{aligned}
            f(x^{k+1})
                &\leq f(x^{k})+\gamma \rho_{2}\norm{\lambda^{k}}_{2}-\gamma\rho\|d^k\|_{\ast} +\tfrac{L\gamma^{2}}{2}\rho^{2}\\
                &= f(x^{k})+\gamma \rho_{2}\norm{\lambda^{k}}_{2}-\gamma\rho\|d^k - \nabla f(x^k) + \nabla f(x^k)\|_{\ast} +\tfrac{L\gamma^{2}}{2}\rho^{2}\\
                &\stackrel{\text{(a)}}{\leq} f(x^{k})+\gamma \rho_{2}\norm{\lambda^{k}}_{2} +\gamma\rho\|\lambda^k\|_{\ast} -\gamma\rho\|\nabla f(x^k)\|_{\ast} +\tfrac{L\gamma^{2}}{2}\rho^{2}\\
                &\stackrel{\text{(b)}}{\leq} f(x^{k})+\gamma (\rho_{2}+\zeta\rho)\norm{\lambda^{k}}_{2}-\gamma\rho\|\nabla f(x^k)\|_{\ast} +\tfrac{L\gamma^{2}}{2}\rho^{2},
        \end{aligned}
    \end{equation*}
    applying the reverse triangle inequality in (a) while (b) stems from the definition of $\zeta$.
    By rearranging terms and taking expectations, we get
    \begin{equation*}
        \begin{aligned}
            \gamma\rho\mathbb{E}[\norm{\nabla f(x^k)}_{\ast}]
                &\leq \mathbb{E}[f(x^{k})-f(x^{k+1})] + \gamma\left(\rho_2+\zeta\rho\right)\mathbb{E}[\norm{\lambda^{k}}_2] +\frac{L\rho^2\gamma^2}{2}.
        \end{aligned}
    \end{equation*}
    Summing this from $k=1$ to $n$ and dividing by $\gamma\rho n$ we get
    \begin{equation*}
        \begin{aligned}
            \mathbb{E}[\norm{\nabla f(\bar{x}^n)}_{\ast}]
                &= \frac{1}{n}\sum\limits_{k=1}^n\mathbb{E}[\norm{\nabla f(x^k)}_{\ast}]\\
                &\leq \frac{\mathbb{E}[f(x^{1})-f(x^{n+1})]}{\rho\gamma n} +\frac{L\rho\gamma}{2} + \frac{1}{n}\left(\frac{\rho_2}{\rho}+\zeta\right)\sum\limits_{k=1}^n\mathbb{E}[\norm{\lambda^{k}}_2]\\
                &\stackrel{\text{(a)}}{\leq} \frac{\mathbb{E}[f(x^{1})-\fmin]}{\rho\gamma n} +\frac{L\rho\gamma}{2} + \frac{1}{n}\left(\frac{\rho_2}{\rho}+\zeta\right)\sum\limits_{k=1}^n\mathbb{E}[\norm{\lambda^{k}}_2]\\
                &\stackrel{\text{(b)}}{\leq} \frac{\mathbb{E}[f(x^{1})-\fmin]}{\rho\gamma n} +\frac{L\rho\gamma}{2} + \frac{1}{n}\left(\frac{\rho_2}{\rho}+\zeta\right)\sum\limits_{k=1}^n\sqrt{\mathbb{E}[\norm{\lambda^{k}}_2^2]},
        \end{aligned}
    \end{equation*}
    using the definition of $\fmin$ for (a) and Jensen's inequality for (b).
\end{proof}

At this point, we need to determine the growth of the induced error captured by the quantity $\norm{\lambda^{k}}_2^2$. To estimate this, we first use a recursion relating $\mathbb{E}[\norm{\lambda^{k}}_2^2]$ and $\mathbb{E}[\norm{\lambda^{k-1}}_2^2]$ adapted from the proof in \citet[Lem. 6]{mokhtari2020stochastic} and then we prove a bound on the decay of $\norm{\lambda^k}_2^2$ for \Cref{alg:uSCG}.
\begin{lemma}[Linear recursive inequality for $\mathbb{E}\norm{\lambda^k}_2^2$]\label{lem:uSCGerror}
    Suppose \Cref{asm:Lip,asm:stoch} hold. Let $n\in\mathbb{N}^*$ and consider the iterates $\{x_k\}_{k=1}^n$ generated by \Cref{alg:uSCG} with a constant stepsize $\gamma>0$. Then, for all $k\in\{1,\ldots,n
    \}$,
    \begin{equation*}
        \mathbb{E}[\norm{\lambda^k}_2^2] \leq \left(1-\frac{\alpha_k}{2}\right)\mathbb{E}[\norm{\lambda^{k-1}}_2^2] + \frac{2L_2^2\rho_2^2\gamma^2}{\alpha_k} + \alpha_k^2\sigma^2.
    \end{equation*}
\end{lemma}
\begin{proof}
    The proof is a straightforward adaptation of the arguments laid out in \citet[Lem. 6]{mokhtari2020stochastic}, which in fact do not depend on convexity nor on the choice of stepsize. Let $n\in\mathbb{N}^*$ and $k\in\{1,\ldots,n\}$, then
    \begin{equation*}
        \begin{aligned}
            \norm{\lambda^k}_2^2
                &= \norm{\nabla f(x^k) - d^{k}}_2^2\\
                &= \norm{\nabla f(x^k) - \alpha_k \nabla f(x^k,\xi_k) - (1-\alpha_k)d^{k-1}}_2^2\\
                &= \norm{\alpha_k\left(\nabla f(x^k) - \nabla f(x^k,\xi_k)\right) +(1-\alpha_k)\left(\nabla f(x^{k})-\nabla f(x^{k-1})\right) - (1-\alpha_k)\left(d^{k-1} - \nabla f(x^{k-1})\right)}_2^2\\
                &= \alpha_k^2\norm{\nabla f(x^k) - \nabla f(x^k,\xi_k)}_2^2 + (1-\alpha_k)^2\norm{\nabla f(x^k)-\nabla f(x^{k-1})}_2^2\\
                    &\quad\quad + (1-\alpha_k)^2\norm{\nabla f(x^{k-1})-d^{k-1}}_2^2\\
                    &\quad\quad +2\alpha_k(1-\alpha_k)\langle\nabla f(x^{k-1})-\nabla f(x^{k-1},\xi_{k-1}), \nabla f(x^k)-\nabla f(x^{k-1})\rangle\\
                    &\quad\quad +2\alpha_k(1-\alpha_k)\langle \nabla f(x^k)-\nabla f(x^k,\xi_k), \nabla f(x^{k-1})-d^{k-1}\rangle\\
                    &\quad\quad +2(1-\alpha_k)^2\langle \nabla f(x^k)-\nabla f(x^{k-1}),\nabla f(x^{k-1}) - d^{k-1}\rangle.
        \end{aligned}
    \end{equation*}
    Taking the expectation conditioned on the filtration $\mathcal{F}_k$ generated by the iterates until $k$, i.e., the sigma algebra generated by $\{x_1,\ldots,x_k\}$, which we denote using $\mathbb{E}_k[\cdot]$, and using the unbiased property in \Cref{asm:stoch}, we get,
    \begin{equation*}
        \begin{aligned}
            \mathbb{E}_k[\norm{\lambda^k}_2^2]
                &= \alpha_k^2\mathbb{E}_k[\norm{\nabla f(x^k)-\nabla f(x^k,\xi_k)}_2^2] + (1-\alpha_k)^2\norm{\nabla f(x^k)-\nabla f(x^{k-1})}_2^2\\
                    &\quad\quad + (1-\alpha_k)^2\norm{\lambda^{k-1}}_2^2 + 2(1-\alpha_k)^2\langle \nabla f(x^k)-\nabla f(x^{k-1}),\lambda^{k-1}\rangle.
        \end{aligned}
    \end{equation*}
    From this expression we can estimate,
    \begin{equation*}
        \begin{aligned}
            \mathbb{E}_k[\norm{\lambda^k}_2^2]
                &\stackrel{\text{(a)}}{\leq} \alpha_k^2\sigma^2 + (1-\alpha_k)^2\norm{\nabla f(x^{k})-\nabla f(x^{k-1})}_2^2 + (1-\alpha_k)^2\norm{\lambda^{k-1}}_2^2 + 2(1-\alpha_k)^2\langle \nabla f(x^k)-\nabla f(x^{k-1}),\lambda^{k-1}\rangle\\
                &\stackrel{\text{(b)}}{\leq} \alpha_k^2\sigma^2 + (1-\alpha_k)^2\norm{\nabla f(x^{k})-\nabla f(x^{k-1})}_2^2 + (1-\alpha_k)^2\norm{\lambda^{k-1}}_2^2\\
                    &\quad\quad + (1-\alpha_k)^2\left(\tfrac{\alpha_k}{2}\norm{\nabla f(x^k)-\nabla f(x^{k-1})}_2^2+\tfrac{2}{\alpha_k}\norm{\lambda^{k-1}}_2^2\right)\\
                 &\stackrel{\text{(c)}}{\leq} \alpha_k^2\sigma^2 + (1-\alpha_k)^2L_2^2\norm{x^k-x^{k-1}}_2^2 + (1-\alpha_k)^2\norm{\lambda^{k-1}}_2^2 + (1-\alpha_k)^2\left((\tfrac{\alpha_k}{2})L_2^2\norm{x^k-x^{k-1}}_{2}^2+\tfrac{2}{\alpha_k}\norm{\lambda^{k-1}}_2^2\right)\\
                 &\stackrel{\text{(d)}}{\leq} \alpha_k^2\sigma^2 + (1-\alpha_k)^2L_2^2\rho_2^2\gamma^2 + (1-\alpha_k)^2\norm{\lambda^{k-1}}_2^2 + (1-\alpha_k)^2\left((\tfrac{\alpha_k}{2})L_2^2\rho_2^2\gamma^2+\tfrac{2}{\alpha_k}\norm{\lambda^{k-1}}_2^2\right)\\
                 &\stackrel{\text{(e)}}{\leq} \alpha_k^2\sigma^2 + (1+\tfrac{\alpha_k}{2})(1-\alpha_k)L_2^2\rho_2^2\gamma^2 + (1+\tfrac{2}{\alpha_k})(1-\alpha_k)\norm{\lambda^{k-1}}_2^2,
        \end{aligned}
    \end{equation*}
    using the bounded variance property from \Cref{asm:stoch} for (a), Young's inequality with parameter $\alpha_k/2>0$ for (b), the Lipschitz property of $f$ under norm $\|\cdot\|_2$ for (c), the update definition from \Cref{alg:uSCG} for (d), and the fact that $1-\alpha_k < 1$ for (e).
    To complete the proof, we note that
    \begin{equation*}
        (1+\tfrac{2}{\alpha_k})(1-\alpha_k)\leq \tfrac{2}{\alpha_k}\quad\text{and}\quad(1-\alpha_k)(1+\tfrac{\alpha_k}{2})\leq (1-\tfrac{\alpha_k}{2})
    \end{equation*}
    which, applied to the previous inequality and taking total expectations, yields
    \begin{equation*}
        \mathbb{E}[\norm{\lambda^k}_2^2] \leq \left(1-\frac{\alpha_k}{2}\right)\mathbb{E}[\norm{\lambda^{k-1}}_2^2] + \alpha_k^2\sigma^2 + \frac{2L_2^2\rho_2^2\gamma^2}{\alpha_k}.
    \end{equation*}
\end{proof}

\subsubsection{Constant $\alpha$}

\begin{lemma}
    Suppose \Cref{asm:Lip,asm:stoch} hold. Let $n \in \mathbb{N}^*$ and consider the iterates $\{x^k\}_{k=1}^n$ generated by \Cref{alg:uSCG} with constant stepsize $\gamma >0$ and constant momentum $\alpha\in(0,1)$ with the exception of the first iteration, where we take $\alpha=1$.
    Then, we have for all $k\in\{1,\ldots,n\}$
    \begin{equation*}
        \begin{aligned}
            \sqrt{\mathbb{E}[\norm{\lambda^k}_2^2]}
                &\leq \frac{\sqrt{2}L_2\rho_2\gamma}{\alpha} + \left(\sqrt{\alpha} + \left(\sqrt{1-\frac{\alpha}{2}}\right)^k\right)\sigma.
        \end{aligned}
    \end{equation*}
\end{lemma}
\begin{proof}
    Let $n\in\mathbb{N}^*$, $k\in\{1,\ldots,n\}$, and invoke \Cref{lem:uSCGerror} to get
    \begin{equation*}
        \mathbb{E}[\norm{\lambda^k}_2^2] \leq \left(1-\frac{\alpha}{2}\right)\mathbb{E}[\norm{\lambda^{k-1}}_2^2] + \frac{2L_2^2\rho_2^2\gamma^2}{\alpha} + \alpha^2\sigma^2.
    \end{equation*}
    Applying \Cref{lem:recursive_geometric} with $\beta = \frac{\alpha}{2}$ and $\eta = \frac{2L_2^2\rho_2^2\gamma^2}{\alpha}+\alpha^2\sigma^2$ gives directly
    \begin{equation*}
        \begin{aligned}
            \mathbb{E}[\norm{\lambda^k}_2^2]
                &\leq \frac{2L_2^2\rho_2^2\gamma^2}{\alpha^2} + \alpha\sigma^2 + \left(1-\frac{\alpha}{2}\right)^k\mathbb{E}[\norm{\lambda^1}_2^2]\\
                &\leq \frac{2L_2^2\rho_2^2\gamma^2}{\alpha^2} + \left(\alpha + \left(1-\frac{\alpha}{2}\right)^k\right)\sigma^2
        \end{aligned}
    \end{equation*}
    after using \Cref{asm:stoch} in the final inequality.
    Taking square roots and upper boudning then yields
    \begin{equation*}
        \begin{aligned}
            \sqrt{\mathbb{E}[\norm{\lambda^k}_2^2]}
                &\leq \frac{\sqrt{2}L_2\rho_2\gamma}{\alpha} + \left(\sqrt{\alpha} + \left(\sqrt{1-\frac{\alpha}{2}}\right)^k\right)\sigma.
        \end{aligned}
    \end{equation*}
\end{proof}

\end{toappendix}

\begin{lemmarep}[{Convergence rate for \ref{eq:uSCG} with constant $\alpha$}]\label{lem:uSCGrate1}
    Suppose \Cref{asm:Lip,asm:stoch} hold. Let $n\in\mathbb{N}^*$ and consider the iterates $\{x^k\}_{k=1}^n$ generated by \Cref{alg:uSCG} with constant stepsize $\gamma = \frac{1}{\sqrt{n}}$ and constant momentum $\alpha\in(0,1)$.
    Then, it holds that
    \begin{equation*}
        \mathbb{E}[\norm{\nabla f(\bar{x}^n)}_{\ast}] \leq O\left(\tfrac{L\rho}{\sqrt{n}}+\sigma\right).
    \end{equation*}
\end{lemmarep}
\begin{appendixproof}
    Let $n\in\mathbb{N}^*$; we will first invoke \Cref{lem:uSCGtemplate1} and then we will estimate the error terms inside using \Cref{lem:uSCGerror} under \Cref{asm:Lip,asm:stoch}.
    As shown in \Cref{lem:uSCGtemplate1},
    \begin{equation}\label{eq:uSCGrate1}
        \begin{aligned}
            \mathbb{E}[\norm{\nabla f(\bar{x}^n)}_2^2]
                &\leq \frac{\mathbb{E}[f(x^{1})-\fmin]}{\rho\gamma n} +\frac{L\rho\gamma}{2n} + \frac{1}{n}\left(\frac{\rho_2}{\rho}+\zeta\right)\sum\limits_{k=1}^n\sqrt{\mathbb{E}[\norm{\lambda^{k}}_2^2]}.
            \end{aligned}
    \end{equation}
    By \Cref{lem:uSCGerror} with \Cref{lem:recursive_geometric}, we get
    \begin{equation*}
        \sqrt{\mathbb{E}[\norm{\lambda^k}_2^2]}
            \leq \frac{\sqrt{2}L_2\rho_2\gamma}{\alpha} + \left(\sqrt{\alpha} + \left(\sqrt{1-\frac{\alpha}{2}}\right)^k\right)\sigma
    \end{equation*}
    which, if we sum from $k=1$ to $n$, gives us
    \begin{equation*}
        \sum\limits_{k=1}^n\sqrt{\mathbb{E}[\norm{\lambda^k}_2^2]}
            \leq n\frac{\sqrt{2}L_2\rho_2\gamma}{\alpha} + \left(n\sqrt{\alpha} + \frac{\sqrt{1-\frac{\alpha}{2}}}{1-\sqrt{1-\frac{\alpha}{2}}}\right)\sigma.
    \end{equation*}
    Plugging this estimate into \Cref{eq:uSCGrate1} gives
    \begin{equation}\label{eq:uSCGfinalineq}
        \begin{aligned}
            \mathbb{E}[\norm{\nabla f(\bar{x}^n)}_2^2]
                &\leq \frac{\mathbb{E}[f(x^{1})-\fmin]}{\rho\gamma n} +\frac{L\rho\gamma}{2} + \frac{1}{n}\left(\frac{\rho_2}{\rho}+\zeta\right)\sum\limits_{k=1}^n\mathbb{E}[\norm{\lambda^{k}}_2]\\
                &\leq \frac{\mathbb{E}[f(x^{1})-\fmin]}{\rho\gamma n} +\frac{L\rho\gamma}{2} + \frac{1}{n}\left(\frac{\rho_2}{\rho}+\zeta\right)\left(n\frac{\sqrt{2}L_2\rho_2\gamma}{\alpha} + \left(n\sqrt{\alpha} + \frac{\sqrt{1-\frac{\alpha}{2}}}{1-\sqrt{1-\frac{\alpha}{2}}}\right)\sigma\right)\\
                &= \frac{\mathbb{E}[f(x^{1})-\fmin]}{\rho\gamma n} +\frac{L\rho\gamma}{2} + \left(\frac{\rho_2}{\rho}+\zeta\right)\left(\frac{\sqrt{2}L_2\rho_2\gamma}{\alpha} + \left(\sqrt{\alpha} + \frac{\sqrt{1-\frac{\alpha}{2}}}{n(1-\sqrt{1-\frac{\alpha}{2}})}\right)\sigma\right).
        \end{aligned}
    \end{equation}
    Finally, by substituting $\gamma = \frac{1}{\sqrt{n}}$ and noting $f(x^{n+1}) \geq \fmin$ we arrive at
    \begin{equation*}
        \begin{aligned}
            \mathbb{E}[\norm{\nabla f(\bar{x}^n)}_{\ast}]
                &\leq \frac{\mathbb{E}[f(x^{1})-\fmin]}{\sqrt{n}\rho} +\frac{L\rho}{2\sqrt{n}} + \left(\frac{\rho_2}{\rho}+\zeta\right)\left(\frac{\sqrt{2}L_2\rho_2}{\alpha\sqrt{n}} + \left(\sqrt{\alpha} + \frac{\sqrt{1-\frac{\alpha}{2}}}{n(1-\sqrt{1-\frac{\alpha}{2}})}\right)\sigma\right)\\
                &= O\left(\frac{1}{\sqrt{n}} + \sigma\right).
        \end{aligned}
    \end{equation*}
\end{appendixproof}

\begin{toappendix}

\subsubsection{Vanishing $\alpha_k$}\label{subsec:uSCGvanishing}

\begin{lemma}[Bound on the gradient error with vanishing $\alpha$]
\label{lem:uSCGerrorbound}
    Suppose \Cref{asm:Lip,asm:stoch} hold. Let $n\in\mathbb{N}^*$ and consider the iterates $\{x_{k}\}_{k=1}^n$ generated by \Cref{alg:uSCG}
    with a constant stepsize $\gamma$ satisfying
    \begin{equation}
        \frac{1}{2 n^{3/4}}<\gamma <\frac{1}{n^{3/4}}.
    \end{equation}
    Moreover, consider momentum which vanishes $\alpha_{k}= \frac{1}{\sqrt{k}}$. Then, for all $k\in\{1,\ldots,n\}$ the following holds
     \begin{equation}
            \mathbb{E}[\norm{\lambda^{k}}_{2}^{2}]\leq \frac{4\sigma^2+8L_2^2\rho_2^2}{\sqrt{k}}.
    \end{equation}
\end{lemma}

\begin{proof}
    Let $k\in\{1,\ldots,n\}$, then by invoking the recursive inequality obtained in \Cref{lem:uSCGerror} for $\mathbb{E}[\norm{\lambda^k}_2^2]$ we have,
    \begin{equation}
        \mathbb{E}[\norm{\lambda^k}^{2}_{2}]\leq \left(1-\frac{\alpha_{k}}{2}\right)\mathbb{E}[\norm{\lambda^{k-1}}^{2}_{2}]+\alpha_{k}^{2}\sigma^{2}+\frac{2L_2^2\rho_2^2\gamma^2}{\alpha_{k}}.
        \end{equation}
        Using the particular choice of $\gamma$ given in the statement of the lemma,
        \begin{equation}
            \frac{1}{2 n^{3/4}}<\gamma <\frac{1}{n^{3/4}},
        \end{equation}
        as well as the choice of $\alpha_k$ and the fact that $n\geq k$, we get
    \begin{align*}
        \mathbb{E}[\norm{\lambda^k}_2^{2}]
            &\leq \bigg(1-\frac{\alpha_{k}}{2} \bigg)\mathbb{E}[\norm{\lambda^{k-1}}_2^{2}]+\alpha_{k}^{2}\sigma^{2}+\frac{2L_2^2\rho_2^2}{\alpha_{k}n^{3/2}}\\
            &\leq \bigg(1-\frac{\alpha_{k}}{2} \bigg)\mathbb{E}[\norm{\lambda^{k-1}}_2^{2}]+\alpha_{k}^{2}\sigma^{2}+\frac{2L_2^2\rho_2^2}{\alpha_{k}k^{3/2}}\\
            &=\bigg(1-\frac{1}{2\sqrt{k}}\bigg)\mathbb{E}[\norm{\lambda^{k-1}}_2^{2}]+\frac{\sigma^{2}}{k}+\frac{2L_2^2\rho_2^2}{k}\\
            &= \bigg(1-\frac{1}{2\sqrt{k}}\bigg)\mathbb{E}[\norm{\lambda^{k-1}}_2^{2}]+\frac{\sigma^{2}+2L_2^2\rho_2^2}{k}.
        \end{align*}
    Then, by applying \Cref{lem:recursivevanishing} with $u^k = \mathbb{E}[\norm{\lambda^k}_2^2]$ and $c=\sigma^2+2L_2^2\rho_2^2$ we readily obtain
    \begin{equation}
        \mathbb{E}[\norm{\lambda^{k}}_{2}^{2}]\leq \frac{4\sigma^2+8L_2^2\rho_2^2}{\sqrt{k}}
    \end{equation}
    since $Q$ as defined in \Cref{lem:recursivevanishing} is given by $Q = \max\{\mathbb{E}[\norm{\lambda^1}_2^2], 4\sigma^2+8L_2^2\rho_2^2\} \leq 4\sigma^2+8L_2^2\rho_2^2$, which concludes our result.
\end{proof}

Combining these results yields our accuracy guarantees for \Cref{alg:uSCG} with vanishing $\alpha_k$, presented in the next lemma.
\end{toappendix}

\begin{lemmarep}[{Convergence rate for \ref{eq:uSCG} with vanishing $\alpha_k$}]
    Suppose that \Cref{asm:Lip,asm:stoch} hold. Let $n\in\mathbb{N}^*$ and consider the iterates $\{x^{k}\}_{k=1}^n$ generated by \Cref{alg:uSCG} with a constant stepsize $\gamma$ satisfying $\frac{1}{2n^{3/4}}<\gamma <\frac{1}{n^{3/4}}$ and vanishing momentum $\alpha_{k}=\tfrac{1}{\sqrt{k}}$. Then, it holds that
    \begin{equation*}
        \mathbb{E}[\|\nabla f(\bar{x}^n)\|_{\ast}] = O\left(\tfrac{1}{n^{1/4}} + \tfrac{L\rho}{n^{3/4}}\right).
    \end{equation*}
\end{lemmarep}
\begin{appendixproof}
    Let $n\in\mathbb{N}^*$, $k\in\{1,\ldots,n\}$; by combining \Cref{lem:uSCGtemplate1} and \Cref{lem:uSCGerrorbound} we have
    \begin{equation}\label{eq:pre_rate}
        \begin{aligned}
            \mathbb{E}[\|\nabla f(\bar{x}^n)\|_{\ast}]
                &\stackrel{\text{\eqref{lem:uSCGtemplate1}}}{\leq} \frac{2\mathbb{E}[f(x^1)-\fmin]}{\rho n^{1/4}} + \frac{2(\rho_2 + \zeta\rho)\sum_{k=1}^n\sqrt{\mathbb{E}[\norm{\lambda^k}_2^2]}}{\rho n} + \frac{L\rho}{n^{3/4}}\\
                &\stackrel{\text{\eqref{lem:uSCGerrorbound}}}{\leq} \frac{2\mathbb{E}[f(x^1)-\fmin]}{\rho n^{1/4}} + \frac{2(\rho_2 + \zeta\rho)\sqrt{4\sigma^2+8L_2^2\rho_2^2}\sum_{k=1}^{n}\frac{1}{k^{1/4}}}{\rho n}  + \frac{L\rho}{n^{3/4}}\\
                &\leq \frac{2\mathbb{E}[f(x^1)-\fmin]}{\rho n^{1/4}} + \frac{2(\rho_2 + \zeta\rho)\sqrt{4\sigma^2+8L_2^2\rho_2^2}\sum_{k=1}^{n}\frac{1}{k^{1/4}}}{\rho n}  + \frac{L\rho}{n^{3/4}}.
        \end{aligned}
    \end{equation}
    Using the integral test and noting that $x\mapsto \tfrac{1}{x^{1/4}}$ is decreasing on $\mathbb{R}_+$, we can upper bound the sum in the right hand side as
    \begin{equation*}
        \sum_{k=1}^{n}\frac{1}{k^{1/4}}\leq 1 + \int_{1}^{n}\frac{1}{x^{3/4}}dx=1+\frac{4}{3}[x^{3/4}]^{n}_1=1+\frac{4}{3}(n^{3/4}-1) = \frac{4}{3}n^{3/4}-\frac{1}{3}\leq \frac{4}{3}n^{3/4}.
    \end{equation*}
    Inserting the above estimation into \eqref{eq:pre_rate} we arrive at
    \begin{align*}
        \mathbb{E}[\|\nabla f(\bar{x}^n)\|_{\ast}] &\leq \frac{2\mathbb{E}[f(x^1)-\fmin]}{\rho n^{1/4}}+ \frac{8 n^{3/4}(\rho_2 + \zeta\rho)\sqrt{4\sigma^2+8L_2^2\rho_2^2}}{3\rho n}  + \frac{L\rho}{n^{3/4}}\\
        &= \frac{2\mathbb{E}[f(x^1)-\fmin]+ \tfrac{8}{3}(\rho_2 + \zeta\rho)\sqrt{4\sigma^2+8L_2^2\rho_2^2}}{\rho n^{1/4}} + \frac{L\rho}{n^{3/4}}\\
        &= O\left(\frac{1}{n^{1/4}}+\frac{L\rho}{n^{3/4}}\right)
    \end{align*}
    which is the claimed result.
\end{appendixproof}

\begin{toappendix}

\subsection{Convergence analysis of \ref{eq:SCG}}\label{subsec:SCG}

In this section we will analyze the worst-case convergence rate of \Cref{alg:SCG}. To do this, we will prove bounds on the expectation of the so-called Frank-Wolfe gap, $\max\limits_{u\in\mathcal{D}} \langle \nabla f(x), x-u\rangle$, which ensures criticality for the constrained optimization problem over $\mathcal{D}$, i.e., for $x^\star\in\mathcal{D}$
\begin{equation*}
    0 = \nabla f(x^\star) + \mathrm{N}_{\mathcal{D}}(x^\star) \iff \max\limits_{u\in\mathcal{D}} \langle \nabla f(x^\star), x^\star-u\rangle \leq 0
\end{equation*}
where $\mathrm{N}_{\mathcal{D}}$ is the normal cone to the set convex $\mathcal{D}$.

This next lemma characterizes the descent of \Cref{alg:SCG} for any stepsize $\gamma$ and momentum $\alpha_k$ in $(0,1]$.
\begin{lemma}[{Nonconvex analog \citet[Lem. 2]{mokhtari2020stochastic}}]
    \label{lem:commondescent}
    Suppose \Cref{asm:Lip} holds.
    Let $n\in\mathbb{N}^*$ and consider the iterates $\{x_k\}_{k=1^n}$ generated by \Cref{alg:SCG} with constant stepsize $\gamma\in(0,1]$.
    Then, for all $k\in\{1,\ldots,n\}$, for all $u\in \mathcal{D}$, it holds
    \begin{equation}
        \gamma \mathbb{E}[\langle \nabla f(x^k), x^k-u\rangle] \leq \mathbb{E}[f(x^k) - f(x^{k+1})] + D_2\gamma \sqrt{\mathbb{E}[\| \lambda^k\|_2^2]} + 2L\rho^2\gamma^2.
    \end{equation}
\end{lemma}
\begin{proof}
    Let $n\in\mathbb{N}^*$, then by \Cref{asm:Lip} we can apply the descent lemma for the function $f$ at the points $x^k$ and $x^{k+1}$ to get, for all $k\in\{1,\ldots,n\}$,
    \begin{equation*}
        \begin{aligned}
            f(x^{k+1})
                &\leq f(x^k) + \langle \nabla f(x^k), x^{k+1}-x^k\rangle + \tfrac{L}{2}\|x^{k+1}-x^k\|^2\\
                &= f(x^k) + \langle d^k, x^{k+1}-x^k\rangle + \langle \lambda^k, x^{k+1}-x^k\rangle + \tfrac{L}{2}\|x^{k+1}-x^k\|^2\\
                &= f(x^k) + \gamma\langle d^k, \lmo(d^k)-x^k\rangle + \gamma \langle \lambda^k, \lmo(d^k)-x^k\rangle + \tfrac{L}{2}\gamma^2\|\lmo(d^k)-x^k\|^2\\
                &\stackrel{\text{(a)}}{\leq} f(x^k) + \gamma\langle d^k, u-x^k\rangle + \gamma \langle \lambda^k, \lmo(d^k)-x^k\rangle + \tfrac{L}{2}\gamma^2\|\lmo(d^k)-x^k\|^2\\
                &= f(x^k) + \gamma\langle -\lambda^k, u-x^k\rangle + \gamma \langle \nabla f(x^k), u-x^k\rangle + \gamma \langle \lambda^k, \lmo(d^k)-x^k\rangle + \tfrac{L}{2}\gamma^2\|\lmo(d^k)-x^k\|^2\\
                &= f(x^k) + \gamma \langle \nabla f(x^k), u-x^k\rangle + \gamma \langle \lambda^k, \lmo(d^k)-u\rangle + \tfrac{L}{2}\gamma^2\|\lmo(d^k)-x^k\|^2\\
                &\stackrel{\text{(b)}}{\leq} f(x^k) + \gamma \langle \nabla f(x^k), u-x^k\rangle + \gamma \langle \lambda^k, \lmo(d^k)-u\rangle + 2L\rho^2\gamma^2,
        \end{aligned}
    \end{equation*}
    using the optimality of $\lmo(d^k)$ for the linear minimization subproblem for (a) and the $2\rho$ upper bound on $\|\lmo(d^k)-x^k\|$ for (b).
    Rearranging and estimating we find, for all $k\in\{1,\ldots,n\}$, for all $u\in\mathcal{D}$,
    \begin{equation*}
        \begin{aligned}
            \gamma\langle \nabla f(x^k),x^k-u\rangle
                &\stackrel{\text{(a)}}{\leq} f(x^k) - f(x^{k+1}) + \gamma \| \lambda^k\|_2 \|\lmo(d^k)-u\|_2 + \tfrac{L}{2}\gamma^2\|\lmo(d^k)-x^k\|^2\\
                &\stackrel{\text{(b)}}{\leq} f(x^k) - f(x^{k+1}) + D_2 \gamma \| \lambda^k\|_2  + 2L\rho^2\gamma^2
        \end{aligned}
    \end{equation*}
    where we have used the Cauchy-Schwarz inequality in (a) and and bounded $\|\lmo(d^k)-x^k\|_2$ using the diameter of the set $\mathcal{D}$ with respect to the Euclidean norm, denoted $D_2$, in (b).
    Taking the expectation of both sides and applying Jensen's inequality we finally arrive, for all $k\in\{1,\ldots,n\}$, for all $u\in\mathcal{D}$,
    \begin{equation*}
        \begin{aligned}
            \gamma\mathbb{E}[\langle \nabla f(x^k),x^k-u\rangle]
                &\leq \mathbb{E}[f(x^k) - f(x^{k+1})] + D_2 \gamma \mathbb{E}[\| \lambda^k\|_2] + 2L\rho^2\gamma^2\\
                &\leq \mathbb{E}[f(x^k) - f(x^{k+1})] + D_2 \gamma \sqrt{\mathbb{E}[\| \lambda^k\|_2^2]} + 2L\rho^2\gamma^2.
        \end{aligned}
    \end{equation*}
\end{proof}

\subsubsection{\ref{eq:SCG} with constant $\alpha$}\label{subsec:SCGconstant}
\begin{lemma}\label{lem:SCGconstanterror}
    Suppose \Cref{asm:Lip,asm:stoch} hold. Let $n\in\mathbb{N}^*$ and consider the iterates $\{x^k\}_{k=1}^n$ generated by \Cref{alg:SCG} with constant stepsize $\gamma=\tfrac{1}{\sqrt{n}}$ and constant momentum $\alpha \in(0,1)$ with the exception of the first iteration, where we take $\alpha=1$. Then we have
    \begin{equation*}
        \mathbb{E}[\norm{\lambda^k}_2^2] \leq 4L_2^2D_2^2\frac{\gamma^2}{\alpha^2} + \left(2\alpha + \left(1-\frac{\alpha}{2}\right)^k\right)\sigma^2.
    \end{equation*}
\end{lemma}
\begin{proof}
    Under \Cref{asm:Lip,asm:stoch}, Lemma 1 in \citet{mokhtari2020stochastic} yields, after taking expectations, for all $k\in\{1,\ldots,n\}$
    \begin{equation*}
        \mathbb{E}[\| \lambda^{k+1}\|_2^2] \leq (1-\frac{\alpha_{k+1}}{2})\mathbb{E}[\| \lambda^k\|_2^2] + \sigma^2\alpha_{k+1}^2 + 2L_2^2D_2^2\frac{\gamma^2}{\alpha_{k+1}}.
    \end{equation*}
    Taking $\gamma$ and $\alpha$ to be constant we get
    \begin{equation*}
        \mathbb{E}[\| \lambda^{k+1}\|_2^2] \leq (1-\frac{\alpha}{2})\mathbb{E}[\| \lambda^k\|_2^2] + \sigma^2\alpha^2 + 2L_2^2D_2^2\frac{\gamma^2}{\alpha}.
    \end{equation*}
    Applying \Cref{lem:recursive_geometric} to the above with $u^k =\mathbb{E}[\| \lambda^{k+1}\|_2^2]$, $\beta = \frac{\alpha}{2}$, and $\eta = \sigma^2\alpha^2 + 2L_2^2D_2^2\frac{\gamma^2}{\alpha}$ we obtain
    \begin{equation*}
        \begin{aligned}
            \mathbb{E}[\norm{\lambda^{k}}_2^2]
                &\leq 2\alpha\sigma^2 + 4L_2^2D_2^2\frac{\gamma^2}{\alpha^2} + \left(1-\frac{\alpha}{2}\right)^k\mathbb{E}[\norm{\lambda^{1}}_2^2]\\
                &\leq 4L_2^2D_2^2\frac{\gamma^2}{\alpha^2} + \left(2\alpha + \left(1-\frac{\alpha}{2}\right)^k\right)\sigma^2
        \end{aligned}
    \end{equation*}
    with the final inequality following by the variance bound in \Cref{asm:stoch}.
\end{proof}

\end{toappendix}

These results show that, in the worst-case, running \Cref{alg:uSCG} with constant momentum $\alpha$ guarantees faster convergence but to a noise-dominated region with radius proportional to $\sigma$. In contrast, running \Cref{alg:uSCG} with vanishing momentum $\alpha_k$ is guaranteed to make the expected dual norm of the gradient small but at a slower rate. \Cref{alg:SCG} exhibits the analogous behavior, as we show next.

Before stating the results for \Cref{alg:SCG}, we emphasize that they are with \emph{constant} stepsize $\gamma$, which is atypical for conditional gradient methods. However, like most conditional gradient methods, we provide a convergence rate on the so-called Frank-Wolfe gap which measures criticality for the constrained optimization problem over $\mathcal{D}$. 

Finally, we remind the reader that the iterates of \Cref{alg:SCG} are always feasible for the set $\mathcal{D}$ by the design of the update and convexity of the norm ball $\mathcal{D}$.
\begin{lemmarep}[{Convergence rate for \ref{eq:SCG} with constant $\alpha$}]
    Suppose \Cref{asm:Lip,asm:stoch} hold. Let $n\in\mathbb{N}^*$ and consider the iterates $\{x^k\}_{k=1}^n$ generated by \Cref{alg:SCG} with constant stepsize $\gamma=\tfrac{1}{\sqrt{n}}$ and constant momentum $\alpha \in(0,1)$. Then, for all $u\in\mathcal{D}$, it holds that
    \begin{equation*}
        \begin{aligned}
            \mathbb{E}[\langle \nabla f(\bar{x}^n), \bar{x}^n-u\rangle] = O\left(\tfrac{L\rho^2}{\sqrt{n}} + \sigma\right).
        \end{aligned}
    \end{equation*}
\end{lemmarep}
\begin{appendixproof}
    Let $n\in\mathbb{N}^*$ and let $k\in\{1,\ldots,n\}$.
    By \Cref{asm:Lip}, we can invoke \Cref{lem:commondescent} to get, for all $k\in\{1,\ldots,n\}$, for all $u\in\mathcal{D}$,
    \begin{equation*}
        \gamma \mathbb{E}[\langle \nabla f(x^k), x^k-u\rangle]
            \leq \mathbb{E}[f(x^k) - f(x^{k+1})] + D_2\gamma \sqrt{\mathbb{E}[\| \lambda^k\|_2^2]} + 2L\rho^2\gamma^2.
    \end{equation*}
    Since \Cref{asm:stoch} holds, we can then invoke \Cref{lem:SCGconstanterror} and apply this to the above. This gives, for all $u\in\mathcal{D}$
    \begin{equation*}
        \begin{aligned}
            \gamma\mathbb{E}[\langle \nabla f(x^k),x^k-u\rangle]
                &\leq \mathbb{E}[f(x^k) - f(x^{k+1})] + 2L\rho^2\gamma^2 + D_2\gamma \sqrt{4L_2^2D_2^2\frac{\gamma^2}{\alpha^2} + \left(2\alpha + \left(1-\frac{\alpha}{2}\right)^k\right)\sigma^2}\\
                &\leq \mathbb{E}[f(x^k) - f(x^{k+1})] + 2L\rho^2\gamma^2 + 2L_2D_2^2\frac{\gamma^2}{\alpha} + D_2\gamma \left(\sqrt{2\alpha} + \left(\sqrt{1-\frac{\alpha}{2}}\right)^k\right)\sigma.
        \end{aligned}
    \end{equation*}
    Summing from $k=1$ to $n$ then dividing by $n\gamma$ we find, for all $u\in\mathcal{D}$,
    \begin{equation}\label{eq:SCGfinalineq}
        \begin{aligned}
            \mathbb{E}[\langle \nabla f(\bar{x}^n), \bar{x}^n-u\rangle]
                &=\frac{1}{n}\sum\limits_{k=1}^n\mathbb{E}[\langle \nabla f(x^k),x^k-u\rangle]\\
                &\stackrel{\text{(a)}}{\leq} \frac{\mathbb{E}[f(x^1) - f(x^{n+1})]}{\gamma n} + 2L\rho^2\gamma + 2L_2D_2^2\frac{\gamma}{\alpha} + D_2 \left(\sqrt{2\alpha} + \frac{1}{n}\sum\limits_{k=1}^n\left(\sqrt{1-\frac{\alpha}{2}}\right)^k\right)\sigma\\
                &\stackrel{\text{(b)}}{\leq} \frac{\mathbb{E}[f(x^1) - f(x^{n+1})]}{\gamma n} + 2L\rho^2\gamma + 2L_2D_2^2\frac{\gamma}{\alpha} + D_2 \left(\sqrt{2\alpha} + \frac{\sqrt{1-\frac{\alpha}{2}}}{n\left(1-\sqrt{1-\frac{\alpha}{2}}\right)}\right)\sigma\\
                &\stackrel{\text{(c)}}{\leq} \frac{\mathbb{E}[f(x^1) - \fmin]}{\gamma n} + 2L\rho^2\gamma + 2L_2D_2^2\frac{\gamma}{\alpha} + D_2 \left(\sqrt{2\alpha} + \frac{\sqrt{1-\frac{\alpha}{2}}}{n\left(1-\sqrt{1-\frac{\alpha}{2}}\right)}\right)\sigma,
        \end{aligned}
    \end{equation}
    applying the subadditivity of the square root for (a), geometric series due to $\sqrt{1-\frac{\alpha}{2}}\in (0,1)$ for (b), and the definition of $\fmin$ for (c).
    Taking $\gamma = \frac{1}{\sqrt{n}}$ then gives the final result, for all $u\in\mathcal{D}$,
    \begin{equation*}
        \begin{aligned}
            \mathbb{E}[\langle \nabla f(\bar{x}^n), \bar{x}^n-u\rangle]
                &\leq \frac{\mathbb{E}[f(x^1) - \fmin]}{\sqrt{n}} + \frac{2L\rho^2}{\sqrt{n}} + \frac{2L_2D_2^2}{\alpha\sqrt{n}} + D_2 \left(\sqrt{2\alpha} + \frac{\sqrt{1-\frac{\alpha}{2}}}{n\left(1-\sqrt{1-\frac{\alpha}{2}}\right)}\right)\sigma
                &= O\left(\frac{L\rho^2}{\sqrt{n}}+\sigma\right).
        \end{aligned}
    \end{equation*}
\end{appendixproof}

\begin{toappendix}
\subsubsection{\ref{eq:SCG} with vanishing $\alpha$}\label{subsec:SCGvanishing}
We now proceed to analyze the convergence of \Cref{alg:SCG} with vanishing $\alpha_k$.
The next lemma provides an estimation on the decay of the second moment of the noise $\lambda^k$.
\begin{lemma}[Bound on the gradient error with vanishing $\alpha$ \Cref{alg:SCG}]\label{lem:SCG_vanishing_error}
    Suppose \Cref{asm:Lip,asm:stoch} hold. Let $n\in\mathbb{N}^*$ and consider the iterates $\{x_{k}\}_{k=1}^n$ generated by \Cref{alg:SCG}
    with a constant stepsize $\gamma$ satisfying
    \begin{equation}
        \frac{1}{2 n^{3/4}}<\gamma <\frac{1}{n^{3/4}}.
    \end{equation}
    Moreover, consider vanishing momentum $\alpha_{k}= \frac{1}{\sqrt{k}}$. Then, for all $k\in\{1,\ldots,n\}$ the following holds
    \begin{equation}
            \mathbb{E}[\norm{\lambda^{k}}_{2}^{2}]\leq \frac{4\sigma^2+8L_2^2D_2^2}{\sqrt{k}}.
    \end{equation}
\end{lemma}
\begin{proof}
    Under \Cref{asm:Lip,asm:stoch}, we have the following recursion from Lemma 1 in \citet{mokhtari2020stochastic} after taking expectations, for all $k\in\mathbb{N}^*$,
    \begin{equation*}
        \mathbb{E}[\| \lambda^{k+1}\|_2^2] \leq (1-\frac{\alpha_{k+1}}{2})\mathbb{E}[\| \lambda^k\|_2^2] + \sigma^2\alpha_{k+1}^2 + 2L_2^2D_2^2\frac{\gamma^2}{\alpha_{k+1}}.
    \end{equation*}
    Comparing with the bound in \Cref{lem:uSCGerrorbound}, we see the only difference is the change of the constant $D_2^2$ by $\rho_2^2$. Repeating the argument in \Cref{lem:uSCGerrorbound}, the desired claim is directly obtained with $D_2^2$ in place of $\rho_2^2$, with the constant $Q = \max\{\mathbb{E}[\norm{\lambda^1}_2^2], 4\sigma^2+8L_2^2D_2^2\} \leq 4\sigma^2+8L_2^2D_2^2$ since $\mathcal{E}[\norm{\lambda^1}_2^2]\leq \sigma^2$ by \Cref{asm:stoch}.
\end{proof}

\end{toappendix}

\begin{lemmarep}[Convergence rate for \ref{eq:SCG} with vanishing $\alpha_k$]\label{lem:frankwolfe_rate}
    Suppose \Cref{asm:Lip,asm:stoch} hold. Let $n\in\mathbb{N}^*$ and consider the iterates $\{x^k\}_{k=1}^n$ generated by \Cref{alg:SCG} with a constant stepsize $\gamma$ satisfying $\tfrac{1}{2n^{3/4}}<\gamma<\tfrac{1}{n^{3/4}}$ and vanishing momentum $\alpha_k = \frac{1}{\sqrt{k}}$. Then, for all $u\in\mathcal{D}$, it holds that
    \begin{equation*}
        \mathbb{E}[\langle \nabla f(\bar{x}^n), \bar{x}^n-u\rangle] = O\left(\tfrac{1}{n^{1/4}} + \tfrac{L\rho^2}{n^{3/4}}\right).
    \end{equation*}
\end{lemmarep}
\begin{appendixproof}
    Let $n\in\mathbb{N}^*$ and $k\in\{1,\ldots,n\}$. By \Cref{asm:Lip}, we can invoke \Cref{lem:commondescent} to get,
    \begin{equation*}
        \begin{aligned}
            \gamma\mathbb{E}[\langle \nabla f(x^k),x^k-u\rangle]
                &\leq \mathbb{E}[f(x^k) - f(x^{k+1})] + D_2 \gamma \sqrt{\mathbb{E}[\| \lambda^k\|_2^2]} + 2L\rho^2\gamma^2.
        \end{aligned}
    \end{equation*}
    Applying the estimate given in \Cref{lem:SCG_vanishing_error} to the above we get
    \begin{equation*}
        \begin{aligned}
            \gamma\mathbb{E}[\langle \nabla f(x^k),x^k-u\rangle]
                &\leq \mathbb{E}[f(x^k) - f(x^{k+1})] + D_2 \gamma \sqrt{\frac{4\sigma^2+8L_2^2D_2^2}{\sqrt{k}}} + 2L\rho^2\gamma^2\\
                &= \mathbb{E}[f(x^k) - f(x^{k+1})] + D_2 \sqrt{4\sigma^2+8L_2^2D_2^2} \gamma \frac{1}{k^{1/4}} + 2L\rho^2\gamma^2.
        \end{aligned}
    \end{equation*}
    Summing from $k=1$ to $n$ and then dividing by $n\gamma$ we find, for all $u\in\mathcal{D}$,
    \begin{equation*}
        \begin{aligned}
            \mathbb{E}[\langle \nabla f(\bar{x}^n),\bar{x}^n-u\rangle]
                &= \frac{1}{n}\sum\limits_{k=1}^n\mathbb{E}[\langle \nabla f(x^k),x^k-u\rangle]\\
                &\stackrel{\text{(a)}}{\leq} \frac{\mathbb{E}[f(x^1) - f(x^{n+1})]}{n\gamma} + \frac{D_2\sqrt{4\sigma^2+8L_2^2D_2^2}}{n}\sum\limits_{k=1}^n\frac{1}{k^{1/4}} + 2L\rho^2\gamma\\
                &\stackrel{\text{(b)}}{\leq} \frac{\mathbb{E}[f(x^1) - f(x^{n+1})]}{n\gamma} + \frac{4D_2\sqrt{4\sigma^2+8L_2^2D_2^2}n^{3/4}}{3n} + 2L\rho^2\gamma\\
                &= \frac{\mathbb{E}[f(x^1) - f(x^{n+1})]}{n\gamma} + \frac{4D_2\sqrt{4\sigma^2+8L_2^2D_2^2}}{3n^{1/4}} + 2L\rho^2\gamma,
        \end{aligned}
    \end{equation*}
    using division by $\gamma n$ for (a) and the integral test with decreasing function $x\mapsto \frac{1}{x^{1/4}}$ for (b).
    Using the definition of $\fmin$ and estimating $n\gamma > \tfrac{n^{1/4}}{2}$ and $\gamma < \frac{1}{n^{3/4}}$ gives
    \begin{equation*}
        \begin{aligned}
            \mathbb{E}[\langle \nabla f(\bar{x}^n),\bar{x}^n-u\rangle]
                &\leq \frac{2\mathbb{E}[f(x^1) - \fmin]}{n^{1/4}} + \frac{4D_2\sqrt{4\sigma^2+8L_2^2D_2^2}}{3n^{1/4}} + \frac{2L\rho^2}{n^{3/4}}\\
                &= O\left(\frac{1}{n^{1/4}} + \frac{L\rho^2}{n^{3/4}}\right).
        \end{aligned}
    \end{equation*}
\end{appendixproof}
\begin{insightbox}[label={insight:convergence}]
For both algorithms, our worst-case analyses for constant momentum suggest that tuning $\alpha$ requires balancing two effects. Making $\alpha$ smaller helps eliminate a constant term that is proportional to the noise level $\sigma$. However, if $\alpha$ becomes too small, it amplifies an $O(1/\sqrt{n})$ term and an $O(\sigma/n)$ term. The stepsize $\gamma$ must also align with the choice of momentum $\alpha$; for vanishing $\alpha_k$ the theory suggests a smaller constant stepsize like $\gamma=\tfrac{3}{4(n^{3/4})}$ to ensure convergence.
\end{insightbox}
\begin{toappendix}

\subsection{Averaged LMO Directional Descent (ALMOND)}\label{subsec:almond}
In this section we present a variation on \Cref{alg:uSCG} that computes the $\lmo$ directly on the stochastic gradient oracle and then does averaging. This is in contrast to how we have presented \Cref{alg:uSCG} which first does averaging (aka momentum) with the stochastic gradient oracle and then computes the $\lmo$. 
A special case of this algorithm is the Normalized SGD based algorithm of \citet{zhao2020stochastic} when the set $\mathcal{D}$ is with respect to the Euclidean norm. 
In contrast with \Cref{alg:uSCG}, the method relies on large batches, since the noise is not controlled by the momentum parameter $\alpha$ due to the bias introduced by the $\lmo$.

\begin{algorithm}
\caption{Averaged LMO directioNal Descent (ALMOND)}
\label{alg:ALMOND}
\textbf{Input:} Horizon $n$, initialization $x^1 \in \mathcal X$, $d^0 = 0$, momentum $\alpha \in (0,1)$, stepsize $\gamma \in (0,1)$
\begin{algorithmic}[1]
    \For{$k = 1, \dots, n$}
        \State Sample $\xi_{k}\sim \mathcal P$
        \State $d^{k} \gets \alpha \lmo(\nabla f(x^{k}, \xi_{k})) + (1 - \alpha)d^{k-1}$
        \State $x^{k+1} \gets x^k + \gamma d^k$
    \EndFor
    \State Choose $\bar{x}^n$ uniformly at random from $\{x^1, \dots, x^n\}$
    \item[\algfont{Return}] $\bar{x}^n$
\end{algorithmic}
\end{algorithm}

\begin{lemmarep}
    Suppose \Cref{asm:Lip,asm:stoch} hold. Let $n\in\mathbb{N}^*$ and consider the iterates $\{x_k\}_{k=1}^n$ generated by \Cref{alg:ALMOND} with stepsize $\gamma = \frac{1}{\sqrt{n}}$. Then, it holds
    \begin{equation*}
        \mathbb{E}[\norm{\nabla f(\bar{x}^n)}_{\ast}] \leq \frac{\mathbb{E}[f(x^1)-\fmin]}{\rho\sqrt{n}} + \frac{L(1-\alpha)\rho}{\alpha\sqrt{n}} + \frac{L\rho}{2\sqrt{n}} + 2\mu\sigma = O\left(\tfrac{1}{\sqrt{n}}\right) + 2\mu\sigma
    \end{equation*}
    where\footnote{Alternatively, instead of invoking the constant $\mu$ we could make an assumption that the gradient oracle has bounded variance measured in the norm $\norm{\cdot}_{\ast}$.} $\mu = \max\limits_{x\in\mathcal{X}}\frac{\norm{x}_\ast}{\norm{x}_{2}}$.
\end{lemmarep}
\begin{proof}
    Let $n\in\mathbb{N}^*$ and denote $z^{k} = \tfrac{1}{\alpha}x^k-\tfrac{1-\alpha}{\alpha}x^{k-1}$ with the convention that $x_0 = x_1$ so that $z_1 = x_1$ and, for all $k\in\{1,\ldots,n\}$,
    \begin{equation*}
        \begin{aligned}
            z^{k+1} - z^k
                &= \frac{1}{\alpha}x^{k+1}-\frac{1-\alpha}{\alpha}x^{k}-\frac{1}{\alpha}x^{k}+\frac{1-\alpha}{\alpha}x^{k-1}= \frac{1}{\alpha}\left(\gamma d^{k} - \gamma (1-\alpha)d^{k-1}\right)= \gamma\lmo(g^k).
        \end{aligned}
    \end{equation*}
    Applying the descent lemma for $f$ at the points $z^{k+1}$ and $z^k$ gives
    \begin{equation}\label{eq:nsgd_descent1}
        \begin{aligned}
            f(z^{k+1})
                &\leq f(z^{k}) + \langle \nabla f(z^k), z^{k+1}-z^k\rangle +\frac{L}{2}\norm{z^{k+1}-z^k}^2\\
                &= f(z^{k}) + \gamma\langle \nabla f(z^k), \lmo(g^k)\rangle +\frac{L\gamma^2}{2}\norm{\lmo(g^k)}^2\\
                &= f(z^{k}) + \gamma\left(\langle \nabla f(z^k)-\nabla f(x^k), \lmo(g^k)\rangle + \langle \nabla f(x^k) - g^k,\lmo(g^k)\rangle +\langle g^k,\lmo(g^k)\rangle\right) +\frac{L\gamma^2}{2}\norm{\lmo(g^k)}^2\\
                &= f(z^{k}) + \gamma\left(\langle \nabla f(z^k)-\nabla f(x^k), \lmo(g^k)\rangle + \langle \nabla f(x^k) - g^k,\lmo(g^k)\rangle -\rho\norm{g^k}_{\ast}\right) +\frac{L\gamma^2}{2}\norm{\lmo(g^k)}^2\\
                &\stackrel{\text{(a)}}{\leq} f(z^{k}) + \gamma\left(\left(\norm{\nabla f(z^k)-\nabla f(x^k)}_{\ast} + \norm{\nabla f(x^k) - g^k}_{\ast}\right)\norm{\lmo(g^k)} -\rho\norm{g^k}_{\ast}\right) +\frac{L\gamma^2}{2}\norm{\lmo(g^k)}^2\\
                &\stackrel{\text{(b)}}{\leq} f(z^{k}) + \gamma\left(\rho\left(\norm{\nabla f(z^k)-\nabla f(x^k)}_{\ast} + \norm{\nabla f(x^k) - g^k}_{\ast}\right) -\rho\norm{g^k}_{\ast}\right) +\frac{L\rho^2\gamma^2}{2}\\
                &\stackrel{\text{(c)}}{\leq} f(z^{k}) + \gamma\left(\rho\left(L\norm{z^k-x^k} + \norm{\nabla f(x^k) - g^k}_{\ast}\right) -\rho\norm{g^k}_{\ast}\right) +\frac{L\rho^2\gamma^2}{2},
        \end{aligned}
    \end{equation}
    applying H\"{o}lder's inequality with norm $\norm{\cdot}_{\ast}$ for (a), the radius $\rho$ of $\mathcal{D}$ for (b), and \Cref{asm:Lip} for (c).
    We note that
    \begin{equation*}
        x^{k+1}-x^{k} = \gamma d^k = \gamma\left((1-\alpha) d^{k-1}+\alpha\lmo(g^k)\right) = \alpha\gamma \lmo(g^k) + (1-\alpha)\gamma\left(\frac{x^k-x^{k-1}}{\gamma}\right)=\alpha\gamma\lmo(g^k)+(1-\alpha)(x^{k}-x^{k-1})
    \end{equation*}
    which we can use to bound
    \begin{equation*}
        \norm{x^{k}-x^{k-1}} \leq (1-\alpha)\norm{x^k-x^{k-1}} + \alpha\gamma\norm{\lmo(g^k)} \leq (1-\alpha)\norm{x^k-x^{k-1}} + \alpha\rho\gamma \leq \frac{\alpha\rho\gamma}{(1-\alpha)}.
    \end{equation*}
    We then have
    \begin{equation*}
        \norm{z^k-x^k} = \frac{(1-\alpha)}{\alpha}\norm{x^k-x^{k-1}}\leq \frac{(1-\alpha)\rho\gamma}{\alpha}
    \end{equation*}
    by using the definition of the update and the $\lmo$, which can be plugged into \eqref{eq:nsgd_descent1} to get
    \begin{equation}
        \begin{aligned}
            \rho\gamma\norm{g^k}_{\ast}
                &\leq f(z^k) - f(z^{k+1}) + \gamma\rho\left(L\norm{z^k-x^k} + \norm{\nabla f(x^k)-g^k}_{\ast}\right) + \frac{L\rho^2\gamma^2}{2}\\
            \implies \norm{g^k}_{\ast}
                &\stackrel{\text{(a)}}{\leq} \frac{f(z^k)-f(z^{k+1})}{\rho\gamma} + L\norm{z^k-x^k} + \norm{\nabla f(x^k)-g^k}_{\ast} + \frac{L\rho\gamma}{2}\\
                &\stackrel{\text{(b)}}{\leq} \frac{f(z^k)-f(z^{k+1})}{\rho\gamma} + \frac{L(1-\alpha)\rho\gamma}{\alpha} + \norm{\nabla f(x^k)-g^k}_{\ast} + \frac{L\rho\gamma}{2}\\
            \implies \norm{\nabla f(x^k)}_{\ast}
                &\stackrel{\text{(c)}}{\leq} \frac{(f(z^k)-f(z^{k+1})}{\rho\gamma} + \frac{L(1-\alpha)\rho\gamma}{\alpha} + 2\norm{\nabla f(x^k)-g^k}_{\ast} + \frac{L\rho\gamma}{2}
        \end{aligned}
    \end{equation}
    where (a) is the result of dividing both sides by $\rho\gamma$, (b) is the result of bounding $\norm{z^k-x^k}$, and (c) follows by the reverse triangle inequality after adding and subtracting $\nabla f(x^k)$ in the norm on the left hand side.
    Taking expectations, using \Cref{asm:stoch} and the constant $\mu = \max\limits_{x\in\mathcal{X}}\frac{\norm{x}_{\ast}}{\norm{x}_2}$, it holds
    \begin{equation*}
        \mathbb{E}[\norm{\nabla f(x^k)-g^k}_{\ast}]\leq \mu\mathbb{E}[\norm{\nabla f(x^k)-g^k}_{2}]\leq \mu\sqrt{\mathbb{E}[\norm{\nabla f(x^k)-g^k}_{2}^2]}\leq \mu\sigma
    \end{equation*}
    which we can sum from $k=1$ to $n$ to obtain
    \begin{equation*}
        \sum\limits_{k=1}^n\mathbb{E}[\norm{\nabla f(x^k)}_{\ast}] \leq \frac{\mathbb{E}[f(z^0)-f(z^{n+1})]}{\rho\gamma} + \frac{nL(1-\alpha)\rho\gamma}{\alpha} + 2n\mu\sigma + \frac{nL\rho\gamma}{2}.
    \end{equation*}
    Diving both sides by $n$ and then plugging in $\gamma = \frac{1}{\sqrt{n}}$ yields the desired final result.
\end{proof}

\subsection{Linear recursive inequalities}
We now present two elementary lemmas that establish bounds for linear recursive inequalities. These results are essential for analyzing the convergence behavior of our stochastic gradient estimator, particularly when examining the error term $\mathbb{E}[\norm{\lambda^k}_2^2]$.
\begin{lemma}[Linear recursive inequality with constant coefficients]\label{lem:recursive_geometric}
    Let $n>1$ and consider $\{u_k\}_{k=1}^n\in\mathbb{R}_+^n$ a sequence of nonnegative real numbers satisfying, for all $k\in\{2,\ldots,n\}$,
    \begin{equation*}
        u^k\leq (1-\beta) u^{k-1} + \eta
    \end{equation*}
    with $\eta>0$ and $\beta\in(0,1)$.
    Then, for all $k\in\{2,\ldots,n\}$, it holds
    \begin{equation*}
        u^k\leq \frac{\eta}{\beta} + (1-\beta)^ku^1.
    \end{equation*}
\end{lemma}
\begin{proof}
    We prove the claim by induction on $k$. For the base case $k=2$ we find
    \begin{equation*}
        u^2 \leq (1-\beta)u^1 + \eta \leq \frac{\eta}{\beta} + (1-\beta)u^1
    \end{equation*}
    since $\beta<1$.
    Assume now for some $k\in\{2,\ldots,n\}$ that the claim holds. Then, by the assumed recursive inequality on $\{u_i\}_{i=1}^n$, we have
    \begin{equation*}
        u^{k+1} \leq (1-\beta)u^k + \eta \leq (1-\beta)\left(\frac{\eta}{\beta} + (1-\beta)^ku^1\right) + \eta = (1-\beta)^{k+1}u^1 + \left(\frac{1-\beta}{\beta} + 1\right)\eta = (1-\beta)^{k+1}u^1 + \frac{\eta}{\beta}
    \end{equation*}
    and thus the desired claim holds by induction.
\end{proof}

The first lemma establishes a geometric decay bound for sequences with constant momentum. The following lemma extends this analysis to the case of variable coefficients, which we will use when we analyze \Cref{alg:uSCG} and \Cref{alg:SCG} with vanishing momentum $\alpha_k$.

\begin{lemma}[Linear recursive inequality with vanishing coefficients]\label{lem:recursivevanishing}   
    Let $\{u^k\}_{k\in\mathbb{N}^*}$ be a sequence of nonnegative real numbers satisfying, for all $k\in\mathbb{N}^*$, the following recursive inequality
    \begin{equation*}
        u^k\leq \left(1-\frac{1}{2\sqrt{k}}\right)u^{k-1} + \frac{c}{k}
    \end{equation*}
    where $c>0$ is constant.
    Then, the sequence $\{u^k\}_{k\in\mathbb{N}^*}$ satisfies, for all $k\in\mathbb{N}^*$,
    \begin{equation*}
        u^k \leq \frac{Q}{\sqrt{k}}
    \end{equation*}
    with $Q=\max\{u^1, 4c\}$.
\end{lemma}
\begin{proof}
    We prove the claim by induction. For $k=1$ the inequality holds by the definition of $Q$, since
    \begin{equation*}
        u^1 \leq Q = \frac{Q}{\sqrt{1}}.
    \end{equation*}
    Let $k>1$ and assume that
    \begin{equation*}
        u^{k-1}\leq\frac{Q}{\sqrt{k-1}}.
    \end{equation*}
    Then, by the assumed recursive inequality for $u^k$, we have
    \begin{equation}\label{eq:recursive_ineq2}
        \begin{aligned}
            u^{k}
                &\leq \left(1-\frac{1}{2\sqrt{k}}\right)u^{k-1} + \frac{c}{k}\\
                &\leq \left(1-\frac{1}{2\sqrt{k}}\right)\frac{Q}{\sqrt{k-1}} + \frac{c}{k}.
        \end{aligned}
    \end{equation}
    Since $k>1$, we can estimate
    \begin{equation*}
        \frac{1}{\sqrt{k-1}} = \frac{\sqrt{k}}{\sqrt{k(k-1)}} = \frac{1}{\sqrt{k}}\sqrt{\frac{k}{k-1}} = \frac{1}{\sqrt{k}}\sqrt{1 + \frac{1}{k-1}} \leq \frac{1}{\sqrt{k}}\left(1 + \frac{1}{2(k-1)}\right)
    \end{equation*}
    which, when applied to \eqref{eq:recursive_ineq2}, gives
    \begin{equation}\label{eq:recursive_ineq3}
        u^k\leq \left(1-\frac{1}{2\sqrt{k}}\right)\left(1+\frac{1}{2(k-1)}\right)\frac{Q}{\sqrt{k}} + \frac{c}{k}.
    \end{equation}
    Furthermore, as $k>1$, we also have
    \begin{equation*}
        \left(1-\frac{1}{2\sqrt{k}}\right)\left(1+\frac{1}{2(k-1)}\right)\leq \left(1-\frac{1}{4\sqrt{k}}\right).
    \end{equation*}
    Applying the above to \eqref{eq:recursive_ineq3} gives
    \begin{equation*}
        \begin{aligned}
            u^k
                &\leq \left(1-\frac{1}{4\sqrt{k}}\right)\frac{Q}{\sqrt{k}}+\frac{c}{k}\\
                &= \frac{Q}{\sqrt{k}} + \frac{c-Q/4}{k}\\
                &\leq \frac{Q}{\sqrt{k}}
        \end{aligned}
    \end{equation*}
    with the last inequality following since $Q\geq 4c$.
    The desired claim is therefore obtained by induction.
\end{proof}

\end{toappendix}

\subsection{Baselines Comparison}
\begin{table}[]
\centering
\resizebox{\columnwidth}{!}{%
\begin{tabular}{|ll|ll|llllll|}
\hline
Model & &\begin{tabular}[c]{@{}l@{}}Hit\\ Ratio\end{tabular}$\uparrow$ & NDCG$\uparrow$ & CC$\downarrow$ & RCR$\downarrow$ & CMAP$\downarrow$ & CDCG$\downarrow$ & CMRR$\downarrow$ & CRP$\downarrow$ \\ 
\hline

\multicolumn{10}{|c|}{{\bf ML 100k Dataset}} \\
\hline

\multicolumn{2}{|l|}{UserKNN}               &  0.4931& 0.0321 & 0.0283     & 0.0340  &  0.0029& 0.0063 & 0.0017 & \multicolumn{1}{l|}{0.0178}       \\ \hline
\multicolumn{2}{|l|}{ItemKNN}          &     0.4698             & 0.0347 & 0.0675  & 0.0749  & 0.0095 & 0.0181 & 0.0070 &  \multicolumn{1}{l|}{0.0567}    \\ \hline
\multirow{2}{*}{MF}          & Original                                            &   \textbf{0.9003}  &         \textbf{0.1979}   &  0.0288    &    0.0321   &   0.0043  &  0.0089    & \textbf{ 0.0055}   & \multicolumn{1}{l|}{0.0276}       \\ \cline{3-10} 
                             & Fair                                                        & 0.8950  & 0.1794    &\textbf{ 0.0228}      & \textbf{0.0312 }     & \textbf{0.0037 }     &\textbf{ 0.0085}      & 0.0060     & \multicolumn{1}{l|}{\textbf{0.0234}}         \\ \hline
\multirow{2}{*}{VAE-CF}      & Original   &   0.8749             & \textbf{0.1643} & 0.0187  & 0.0210  & 0.0030 & 0.0052 & \textbf{0.0026} & \multicolumn{1}{l|}{0.0137}     \\ \cline{3-10} 
                             & Fair          & \textbf{0.8759}             & 0.1575 & \textbf{0.0140 } & \textbf{0.0186}  & \textbf{0.0026}& \textbf{0.0047} & 0.0037 & \multicolumn{1}{l|}{\textbf{0.0128 } }      \\ \hline
\multirow{2}{*}{NEU-MF}      & Original      & \textbf{0.9470}  & \textbf{0.2402} & 0.1723  & 0.1452  & 0.0268 & 0.0502 & 0.0220 & \multicolumn{1}{l|}{0.1411 }      \\ \cline{3-10} 
                             & Fair          & 0.9194             & 0.2305 & \textbf{0.0333  }& \textbf{0.0405}  & \textbf{0.0056} & \textbf{0.0093} & \textbf{0.0059} & \multicolumn{1}{l|}{\textbf{0.0342}  }     \\ \hline
\multicolumn{10}{|c|}{{\bf ML 1M Dataset}} \\ 
\hline
\multicolumn{2}{|l|}{UserKNN}                &   0.3997   &  0.0179  &  0.0204   &  0.0110&       0.0008 &0.0043&0.0008& \multicolumn{1}{l|}{0.0174}   \\ \hline
\multicolumn{2}{|l|}{ItemKNN}      &    0.3565   &0.0383      & 0.0881    & 0.0351 & 0.0051  &  0.0224 & 0.0082 & \multicolumn{1}{l|}{0.0650}    \\ \hline
\multirow{2}{*}{MF}          & Original      &\textbf{ 0.8485}                                               &   \textbf{0.1522}    &   0.1775   &      0.0486 &  0.0097   &  0.0495    &  0.0204   & \multicolumn{1}{l|}{0.1440}       \\ \cline{3-10} 
                             & Fair          &    0.8055  & 0.1158   &  \textbf{0.1160}   &   \textbf{0.0314 }  &    \textbf{0.0067}  &   \textbf{0.0344}  & \textbf{0.0162}   & \multicolumn{1}{l|}{\textbf{0.0760}}         \\ \hline
                             \multirow{2}{*}{VAE-CF}      & Original    &  \textbf{0.8315 }  & \textbf{0.1473} & 0.2551  & 0.0708  & 0.0154 & 0.0706 & 0.0282 &\multicolumn{1}{l|}{0.1900  }    \\ \cline{3-10} 
                             & Fair          & 0.8280 &  0.1423 &  \textbf{0.2229 } & \textbf{0.0637}  & \textbf{0.0132} & \textbf{0.0631} &  \textbf{0.0263}&\multicolumn{1}{l|}{\textbf{0.1502}}       \\ \hline
\multirow{2}{*}{NEU-MF}      & Original      & \textbf{ 0.9108 }   &  0.1336   &   0.3096  & 0.0998 &  0.0246  & 0.0835  & 0.0316 & \multicolumn{1}{l|}{0.2180  }   \\ \cline{3-10} 
                             & Fair   &          0.8333 & \textbf{0.1359}                                                       & \textbf{0.1503}     & \textbf{0.0415}     & \textbf{0.0086}      &\textbf{ 0.0436 }     & \textbf{0.0191 }     & \multicolumn{1}{l|}{\textbf{0.0959}}       \\ \hline
\multicolumn{10}{|c|}{{\bf Yelp Dataset}} \\ 
\hline

\multicolumn{2}{|l|}{UserKNN}                &   0.4886    &  0.0312    &  0.0068   & 0.0252 &   0.0068    &0.0026&0.0022& \multicolumn{1}{l|}{0.0129}   \\ \hline
\multicolumn{2}{|l|}{ItemKNN}      &    0.5030    &    0.0344   &   0.0242  & 0.0513 &   0.0060 & 0.0065  & 0.0035 & \multicolumn{1}{l|}{0.0219}    \\ \hline
\multirow{2}{*}{MF}          & Original      &     \textbf{0.8587 }                                          &   \textbf{0.1209 }  &   0.0358   &0.0342&0.0055&0.0088&0.0036& \multicolumn{1}{l|}{0.0296}       \\ \cline{3-10} 
                             & Fair          & 0.7500 &  0.0773  &  \textbf{0.0159 }  &\textbf{0.0312}&    \textbf{0.0019 } & \textbf{0.0042}&\textbf{0.0019}& \multicolumn{1}{l|}{\textbf{0.0130}}         \\ \hline
                             \multirow{2}{*}{VAE-CF}      & Original      &  \textbf{0.8131 }                                                  & \textbf{0.0985 }   & 0.0045       & 0.0041      & 0.0003      & 0.0007      & 0.0003      & \multicolumn{1}{l|}{0.0018}       \\ \cline{3-10} 
                             & Fair          &  0.8032                                                 & 0.0961   &   \textbf{0.0021}   & \textbf{0.0019}      & \textbf{0.0002 }      &  \textbf{0.0004}    &   \textbf{0.0001}     & \multicolumn{1}{l|}{ \textbf{0.0011}}       \\ \hline
\multirow{2}{*}{NEU-MF}      & Original      &  \textbf{0.9347 }                                                & \textbf{0.1794}     & 0.0592       & 0.0850      & 0.0134    & 0.0152      & 0.0057    & \multicolumn{1}{l|}{0.0651}       \\ \cline{3-10} 
                             & Fair          &  0.8533  &  0.1124    & \textbf{0.0309 }      & \textbf{ 0.0401 }    & \textbf{0.0054 }    & \textbf{0.0079}      &  \textbf{0.0028}      & \multicolumn{1}{l|}{ \textbf{0.0394}}       \\ \hline
\end{tabular}
}
\caption{Performance and Bias Evaluation values (GBSs) for 5 baseline models, along with the fair models. }
    \label{tab:baselines}
    % \vspace{-24pt}
\end{table}

\begin{figure*}[h]
    \centering
    \includegraphics[width=1\textwidth]{images/all_models_plain_all.pdf}
    \caption{ Comparison of Bias values for six of our metrics for all three dataset.}
    \label{fig:baseline_100k}
    % \vspace{-8pt}
\end{figure*}

% \begin{figure*}[h]
%     \centering
%     \includegraphics[width=1\textwidth]{images/all_models_plain.pdf}
%     \caption{ Comparison of Bias values for six of our metrics. We find significant gender bias in these four stereotypical genres for our baseline models for the ML100K dataset.}
%     \label{fig:baseline_100k}
% \end{figure*}
% \begin{figure*}[h]
%     \centering
%     \includegraphics[width=1\textwidth]{images/all_models_plain_1m.pdf}
%     \caption{ Comparison of Bias values for six of our metrics for the ML1M dataset. There is significant gender bias, especially for VAE-CF and NeuMF models.}
%     \label{fig:baseline_1M}
% \end{figure*}
% \begin{figure*}[h]
%     \centering
%     \includegraphics[width=1\textwidth]{images/all_models_plain_yelp.pdf}
%     \caption{ Comparison of Bias values for six of our metrics for the Yelp100k dataset. With significant gender bias manifesting especially in the MF-based models.}
%     \label{fig:baseline_yelp}
% \end{figure*}
\subsubsection{Bias Evaluation}
We report our results in Figures \ref{fig:baseline_100k}, \ref{fig:models_reg_100k}, and Table \ref{tab:baselines}. 
For the KNN-based models, ItemKNN seems more bias-prone. This can be due to the nature of the model of calculating similarity between items based on the interactions by the users. For
instance, in the ML-100k dataset, the average rating for romance
movies is higher for female users than for males. This kind of imbalance can reinforce category-related gender bias when items are recommended. 
% This model is more biased for categories that are more interacted with. For instance, in the ML-100k dataset, \textit{Action} movies are more popular for both male and female groups when compared to \textit{Romance} ones. Consequently, the ItemKNN model is more biased towards recommending \textbf{Action} movies over \textit{Romance}. 
 Among the three other baselines, the MF-based models can be considered more biased. For the MF model, explicit embeddings are used to store user and item representations, which are likely to capture correlations of user behavior (e.g., liking certain genres) and their sensitive attributes. For VAE-CF, variational autoencoders are used to learn probabilistic latent representations of users, which are still sensitive to capturing biased representations but not as much as NeuMF. NeuMF captures both linear and non-linear relationships between users and items through Generalized Matrix Factorization (GMF) and Multi-Layer Perceptrons (MLP). So, NeuMF can capture biases in the data, especially the intricate patterns about user preferences, which can reflect social stereotypes. The MF model itself is not as complex as NeuMF, so our results of NeuMF being more biased than MF is reasonable.
   %%%%%%%%%%%%%%%%%%%%%%%%%
\begin{figure}[h]
    \centering
    \includegraphics[width=0.5\textwidth]{images/all_models_reg.pdf}
    \caption{Reduction in bias scores after using our fairness-aware regularizer. Since all datasets provide similar outcomes, we present the results for only the ML 100K dataset. }
    \label{fig:models_reg_100k}
    % \vspace{-5pt}
\end{figure}
 %%%%%%%%%%%%%%%%%%%%%%%%%
 For the two smaller datasets, VAE-CF doesn’t manifest high values for bias, which can be due to the model not being able to fully capture representations of gendered preferences because of the lower number of interactions available. Comparatively, for the 1M dataset, the model can learn the stereotypical patterns available in the dataset more effectively and reinforce these biases in the latent representations. In general, all models manifest more gender bias for the larger dataset than the smaller ones.
 For all baseline models, the Gender Balance Scores are \(\not \approx 0\), which highlights the fact that they are producing recommendations in a discriminative way where they choose certain genres to be relevant for certain groups of users. From Table \ref{tab:baselines}, GBSs are higher for \(CC\) or \(RCR\) (ref to Equations: \ref{eq:M1} and \ref{eq:M2}). These two are classification-based metrics and do not consider ranks, so there is no discounting done in the value of any item that appears lower in the list but is still relevant (category-wise). GBS(CMRR) has the lowest values for almost all models since it has a strong discounting factor for items that appear lower in rank. 
 % Initially, we posited the models would be more biased towards the two most stereotypical genres: \textit{action} and \textit{romance}. However as observed from \ref{fig:baseline_100k}, although there is significant bias in recommending \textit{romance} movies, bias in sci-fi movie recommendations overpower those from \textit{action}. This bias can be driven by popularity, for instance for the ML-100k dataset there are more action movies than \textit{sci-fi} but the average rating for \textit{sci-fi} movies is higher than that of \textit{action} movies, making them more popular. As discussed by \cite{10.1145/3383313.3418487}, recommenders sometimes focus on popular movies and not really popular genres. Thus, choosing to recommend items that have higher ratings for example, which might fall under certain genres for our comparison, would be sci-fi.


The bias reduction after using the fairness aware loss term is displayed in Figure \ref{fig:models_reg_100k}. We can see a notable improvement in the bias scores, especially for the NeuMF model. The regularizer term is most effective for reducing bias in the NeuMF model, with an average decrease across all six metrics of 77\%, 53\%, and 50\% for ML100K, ML1M, and Yelp datasets, respectively. The impact of the fairness regularizer is less pronounced for the VAE-CF model. We believe this is due to the model's nature of learning probabilistic latent factors for users and items that don't amplify gender bias like the MF models which use embeddings to learn user and item interactions. As a result, the regularizer has less impact on the model's learning process.


\subsubsection{Performance Evaluation}\label{sec:perform-evaluation}
The performance drop for VAE-CF and NeuMF is less than that of the MF model, however, it all depends on the selection of $\alpha$ (more on this in Section \ref{sec:abalation}). For all models, when choosing $\alpha$ we ensure there is substantial bias reduction without too significant of a drop in performance. 
In some cases the fair models outperform the original models, this although seems counter-intuitive, but is presumably due to the fairness term acting as a regularization term which can help reduce over-fitting in the model to some extent (as noted in \cite{keya2020equitableallocationhealthcareresources,10.1145/3442381.3449904}). 


\subsection{Fairness-Aware Baselines}
\begin{table}[h]
\centering

\resizebox{\columnwidth}{!}{%
\begin{tabular}{|c|cc|cccccc|}
\hline
    %\toprule
Model & NDCG$\uparrow$ & HitRatio$\uparrow$ & CC$\downarrow$  & RCR$\downarrow$  & CMAP$\downarrow$  & CDCG$\downarrow$  & CMRR$\downarrow$  & CRP$\downarrow$  \\
\hline
    %\midrule
\multicolumn{9}{|c|}{{\bf ML 100K Dataset}} \\
\hline
MF$_{a=0.5}$ & \textbf{0.1794} &    \textbf{0.8950} & \textbf{0.0228} &  \textbf{0.0312} & \textbf{ 0.0037} & \textbf{ 0.0085} &  0.0060 & \textbf{ 0.0234 }\\

SM-GBiasedMF   & 0.1230 & 0.8537 & 0.03785 & 0.0378& 0.0059  & 0.0114 & \textbf{0.0053} &   0.0286    \\
BeyondParity     & 0.0790&  0.6384& 0.0528&0.0752&0.0093 & 0.0147 &  0.0068&  0.0480 \\
\hline
\multicolumn{9}{|c|}{{\bf ML 1M Dataset}} \\
\hline
MF$_{a=0.6}$ & \textbf{0.1158 }  & 0.8055  &\textbf{0.1160} &\textbf{0.0314 }&  \textbf{0.0067}&\textbf{0.0344} &0.0162 &\textbf{0.0760}\\
SM-GBiasedMF    &0.0846   &  \textbf{0.8075} &0.2590 &0.0726 & 0.0152 & 0.0656& 0.0211 & 0.2058\\
BeyondParity   &  0.0829   &     0.6495   & 0.1307&0.0539 & 0.0079 & 0.0369&\textbf{0.0154}&0.0838\\

\hline
\multicolumn{9}{|c|}{{\bf Yelp Dataset}} \\
\hline
MF$_{a=0.3}$ & 0.0773&0.7500  &  0.0159   &\textbf{0.0312}&    \textbf{0.0019}  & 0.0042&\textbf{0.0019} &\textbf{0.0130} \\

SM-GBiasedMF    & \textbf{0.9157}  & \textbf{ 0.1343} &0.0515 &0.0789&0.0132&0.0140 & 0.0055 & 0.0587 \\
BeyondParity   & 0.0358    & 0.4901  & \textbf{0.0112}& 0.0342& 0.0066 & \textbf{0.0033}&0.0027 & 0.0241\\


\hline
\end{tabular}
}
\caption{Performance and Bias Evaluation Metric values for fairness-aware models.}
\label{tab:fairbaselines}
    % \vspace{-20pt}

\end{table}

We can observe the results for our MF fair model when compared with two other fairness-aware models in Table \ref{tab:fairbaselines} and Figure \ref{fig:compare_genre_fair}. We chose to use our MF fair model since the other two fairness-aware models use MF as their foundational models. Our model has outperformed the other models in terms of bias scores across the majority of the comparisons. While our models don't excel in terms of performance, they still offer a better balance between performance and bias scores when compared with the other fairness-aware baselines. It is important to highlight that BeyondParity uses plain MSE loss, while the other two use BPR loss with negative sampling, which explains the exceptionally low-performance values. While we emphasize the significance of comparing the aggregated differences for each metric $\mathcal{M}$ for all the categories, we also want to highlight the importance of evaluating bias values separately. For instance, although GBS(CMRR) of SM-GBiasedMF is lower for the ML100K dataset when compared to ours, there is still significant bias for movies recommended that belong to the genres like \textit{Romance} and \textit{Sci-Fi} as observed in Figure \ref{fig:compare_genre_fair}. This strengthens our idea of having a set of metrics to quantify bias in a nuanced way since the model seems "fair" when considering overall scores for all categories, but closer inspection reveals how it is biased against certain categories. It is worth mentioning that SM-GBiasedMF is more biased than the plain MF model for the Yelp dataset. Our theory is that since this model needs more epochs to satisfy the early stopping criterion (as mentioned before), it likely over-fits the dataset, in turn amplifying the biases present in it.

% \begin{table}[]
% \centering

% \resizebox{\columnwidth}{!}{%
% \begin{tabular}{|lllllllll|}
% \hline
% \multicolumn{1}{|l|}{Model} & \multicolumn{1}{l|}{HitRatio} & \multicolumn{1}{l|}{NDCG} & \multicolumn{1}{l|}{GBS(1)} & \multicolumn{1}{l|}{GBS(2)} & \multicolumn{1}{l|}{GBS(3)} & \multicolumn{1}{l|}{GBS(4)} & \multicolumn{1}{l|}{GBS(5)} & GBS(6) \\ \hline
% \multicolumn{9}{|l|}{ML 100K}    

% MF$_{a=0.4}$ & 0.0576  &   0.5090 & 0.0316 & 0.0520 & 0.0082 & 0.0087 & 0.0058 & 0.0245 \\
% SM-GBiasedMF   & 0.1235 & 0.8589 & 0.0382 & 0.0387 & 0.0093  & 0.0098 & 0.0040 &   0.0310     \\
% BeyondParity   & 0.0778  & 0.6055 & 0.0447 & 0.0681& 0.0100 & 0.0124 & 0.0067 & 0.0543\\
% \hline
% \multicolumn{9}{|l|}{ML 1M}                                                                                                                                                                                                                            \\ \hline
% \end{tabular}}
%     \label{tab:fairbaselines}

% \end{table}
%%%%%%%%%%%%%%%%%%%%%%%%%%%%%%%%%%%%%%
\begin{figure}[h]
    \centering
    \includegraphics[width=0.5\textwidth]{images/all_models_fair_100k.pdf}
    \caption{ Bias score for the fairness-aware models over four stereotypical genres for the ML 100K dataset.}
    \label{fig:compare_genre_fair}
    % \vspace{-11pt}
\end{figure}
% \begin{figure}[h]
%     \centering
%     \includegraphics[width=0.5\textwidth]{images/all_models_fair_yelp.pdf}
%     \caption{ Comparison of our model with two other fairness-aware models over four stereotypical genres for all six of our metrics for Yelp 100K dataset.}
%     \label{fig:compare_genre_fair_yelp}
% \end{figure}
%%%%%%%%%%%%%%%%%%%%%%%%%%%%%%%%%%%%%%

% \begin{figure}[h]
%     \centering
%     \includegraphics[width=0.5\textwidth]{images/all_models_reg1m.pdf}
%     \caption{  We observe a major reduction in the bias scores in
% the three models we used our fairness-aware loss term 1M dataset. }
%     \label{fig:compare_genre_fair_1m}
% \end{figure}

% \begin{center}
% \begin{tabular}{ |c|c|c|c|c|c|c| } 
% \hline
% model &  alpha & RMSE & AUC  & NDCG@50 & Precision@50  & Recall@50 \\
%  \hline
%  MF & 0 & 1.0792  & 0.5811 & 0.0900  &  0.0494  &0.1140  \\
% & 0.3 & 1.0793   & 0.5775 & 0.0867  &  0.0465 & 0.1094\\
%  & 0.5 & 1.0794  & 0.5720 & 0.0798  &  0.0414 & 0.0988 \\
%  & 0.7 & 1.0795  & 0.5599  & 0.0611   & 0.0298 &  0.0707 \\
%  & 0.8 & 1.0796 & 0.5480 & 0.0444 & 0.0216  & 0.0494\\
%  \hline
% \end{tabular}
% \end{center}

% Model & NDCG & HitRatio & Genre  & Genre  & Genre  & Genre  & Genre  & Genre  \\
%  &  &  & Precision &  Recall &  MAP &  DCG &  MRR & R Precision \\


% \begin{table*}

% \begin{center}
% \begin{tabular}{ |c|c|c|c|c|c|c|c|c| } 
% \hline
% Model  & Hit& NDCG  &GBS(1) &GBS(2)  & GBS(3) & GBS(4)  & GBS(5)  &GBS(6)  \\

%  \hline
%  \multicolumn{9}{c}{ ML 100K}\\
%  % MF & 1.0378 & 0.0433 & 0.0651 & 0.0619 & 0.5781 \\
%  UserKNN           & 0.4931             & 0.0321 & 0.0283  & 0.0340  & 0.0040 & 0.0063 & 0.0017 & 0.0178 \\
%  ItemKNN           & 0.4698             & 0.0347 & 0.0675  & 0.0748  & 0.0176 & 0.0181 & 0.0070 &  0.0567 \\
%  MF                & 0.6469             & 0.0887 & 0.0551  & 0.0862  & 0.0130 & 0.0158 & 0.0074 & 0.0568\\
%  MF\(_{a=0.4}\)    & 0.5090             & 0.0576 & 0.0316  & 0.0520  & 0.0082 & 0.0087 & 0.0058 & 0.0245\\
%  VAE-CF             & 0.8749             & 0.1643 & 0.0187  & 0.0210  & 0.0041 & 0.0052 & 0.0026 & 0.0137 \\
%  VAE-CF\(_{a=0.3}\) & 0.8759             & 0.1575 & 0.0140  & 0.0186  & 0.0031 & 0.0047 & 0.0037 & 0.0128\\
%  NeuMF             & \textbf{0.9470}  & \textbf{0.2402} & 0.1723  & 0.1452  & 0.0498 & 0.0502 & 0.0220 & 0.1411  \\
% NeuMF\(_{a=0.2}\)  & 0.9194             & 0.2305 & 0.0333  & 0.0405  & 0.0039 & 0.0093 & 0.0059 & 0.0342
%  \\
%  \multicolumn{9}{c}{ ML 1M}\\
%  VAE-CF             & 0.8315 & 0.1473 & 0.2551  & 0.0708  & 0.0771 & 0.0706 & 0.0282 &0.1900  \\
%  VAE-CF\(a=0.2\)    &  0.8280 &  0.1423 &  0.2229  & 0.0637  & 0.0660 & 0.0631 &  0.0263&0.1502  \\
 
%  %  MF & 0.9496  & 0.5751 &  0.6638   &  0.0942 &  0.0462  &  0.1156  \\
%  %   ItemKNN& 1.0739   & 0.5647 &   0.5313  &   0.0494   &     0.0305 & 0.0642 \\
%  % UserKNN & 1.0023  &  0.5773  &  0.4867  &  0.0351  &  0.0218 & 0.0572  \\
%  % VAE-CF &  2.6672 &  0.8691  & 0.9311  & 0.2573 &  0.1112 & 0.3596 \\
%  % LightGCN & 1.6708 &  0.8673 &  0.9618 & 0.2656 &  0.1147 & 0.3749  \\
%  \hline





% \end{tabular}
% \caption{Performance and Bias Evaluation Metric values for 5 baseline models, along with the three models with fairness aware regularization term incorporated wintin their loss functions}
%     \label{tab:baselines}
% \end{center}
% \end{table*}




% VAE-CF 100k a = 0.3 1m a=0.2 mf 100k a= 0.4 mf 1m=? neumf 100k a= 0.2 1m=?

% \section{Post-processing method to mitigate bias}
% % \input{samples/section/Re-ranking}
% \[p(\delta \mid u) = \frac{\sum_{i \in H} w_{u,i} \cdot p(\delta \mid i)}{\sum_{i \in H} w_{u,i}}\]
% \[q(\delta \mid u) = \frac{\sum_{i \in H} w_{r} \cdot p(\delta \mid i)}{\sum_{i \in H} w_{r}}\]
% \begin{itemize}
%     \item \(p(g|u)\) is the genre distribution of the set of items the user has interacted with in the past. For instance the movies they have watched.
%     \item \(q(g|u)\) is the genre distribution of the set of items the user is being recommended. For instance the movies ta recommended system is recommending a user.
% \end{itemize}
% For determining the optimal set of recommended items \(I*\), we employ the maximum marginal relevance \cite{10.1145/290941.291025}.
% \[I^* = \arg\max_{|I| = N} \{ (1 - \lambda) \cdot s(I) - \lambda \cdot \text{CKL}(p, q(I)) \}
% \]

% \subsection{Personalization}
% The idea is to tailor content based on user preferences. The aim is to recommend items to users that are suited to their taste and characteristics, in turn creating an engaging experience for the user.

% \subsection{Diversity}
% Diversity in recommendation systems describe the variety of the items that are recommended to users. If a user has watched about 80\% action movies and the 20\% drama movies, when a recommendation system is solely based on relevance or accuracy it would choose to recommend more action movies, and maybe a few drama movies. blah blah
% \subsection{Fairness}
% Fairness in recommendation systems, ensures that users are treated equitably. We want the system to create an inclusive user experience, and in no way treats them in way that reinforce bias and /or inequalities.
% \[I^* = \arg\max_{|I| = N} \{ (1 - \lambda) \cdot s(I) - \lambda \cdot \text{CKL}(p, q(I)) - \beta \cdot C_{KL}(q(I),p_u )\}
% \]



\section{Discussion and Conclusion}
% \input{samples/section/Discussion and Conclusion}
In this paper, we identify the underlying issues of current metrics for evaluating consumer-side bias in recommendation systems. To better quantify bias, specifically gender bias, in such models, we propose a set of metrics. These metrics help capture a nuanced sense of fairness on recommended items by considering categories of the items. Next, we demonstrate how introducing one of our metrics as a fairness loss term along with the recommendation loss helped minimize the unfairness manifested in models in terms of different categories of items recommended with minimal performance loss. Experiments on three real-world datasets using a variety of recommendation models, including fairness-aware models, show the effectiveness of our metrics in capturing bias. Additionally, after incorporating our loss term, the bias in the models was significantly reduced, with a favorable balance between fairness and accuracy. Our loss function is very flexible which allows for context-sensitive fairness. In domains such as job recommendation and housing, where fairness is vital, the functions can prioritize fairness. Conversely, for areas like movie recommendations it can balance personalization and fairness (ensuring user preferences are respected).
Our work aims to spread awareness about how simple metrics that are currently utilized to evaluate bias might be giving researchers a false sense of fairness, and a more refined approach like ours is required to address this issue. 
Our metrics are very versatile and can be easily adapted to measure provider-side fairness by utilizing proportions of item brands, for example, and can help quantify popularity bias in recommendations. Extending on this, the metrics can be used for measuring CP-fairness since we can measure bias from both the consumer side and provider side, by taking differences in values for different item providers for different demographic groups. Plus we plan to extend our metrics to consider sensitive attributes which are multi-valued, by using a pairwise difference scheme. 
% In the future, we would like to use these metrics in other domains including job or restaurant recommendations.


% \newpage

% \section{Acknowledgments}







%%
%% The acknowledgments section is defined using the "acks" environment
%% (and NOT an unnumbered section). This ensures the proper
%% identification of the section in the article metadata, and the
%% consistent spelling of the heading.
\begin{acks}
This material is based upon work supported by the Air Force Office of Scientific Research under award number FA2386-23-1-4003.
\end{acks}

%%
%% The next two lines define the bibliography style to be used, and
%% the bibliography file.
\bibliographystyle{ACM-Reference-Format}
\bibliography{mybib}


%%
%% If your work has an appendix, this is the place to put it.
\appendix
\section{Appendix}

% \appendix
% \subse{Additional Information}
% \subsection{Additional Information}
%%%%%%%%%%%%%%%%%%%%%%%%%%%%%%%%%%%%%%

% \subsection{Results for ML 1M}
% Since both datasets provide similar outcome, we present the results for the 1 million dataset here.
% \begin{figure}[h]
%     \centering
%     \includegraphics[width=0.5\textwidth]{images/all_models_fair1M.pdf}
%     \caption{ Comparison of our model with two other fairness-aware models over four stereotypical genres for all six of our metrics for ML 1M dataset.}
%     \label{fig:compare_genre_fair_1m}
% \end{figure}
% \begin{figure}[h]
%     \centering
%     \includegraphics[width=0.5\textwidth]{images/all_models_reg1m.pdf}
%     \caption{  We observe a major reduction in the bias scores in
% the three models we used our fairness-aware loss term 1M dataset. }
%     \label{fig:compare_genre_fair_1m}
% \end{figure}
%%%%%%%%%%%%%%%%%%%%%%%%%%%%%%%%%%%%%%

\subsection{Abalation Study}\label{sec:abalation}
In this section, we discuss the influence of the hyper-parameter \(\alpha\) on recommendation performance and fairness. As mentioned in Section \ref{sec:fairloss}, \(\alpha\) can be used to control the strength of the fairness aware loss. Theoretically as \(\alpha\) increases we would expect a decrease in performance since the recommendation loss would have a decreased weight. We use \(\alpha\) values from 0 to 0.6 with increments of 0.1 for all three models. We did not include values above 0.6, because it does not make sense to overpower the recommendation loss since that is our main task. To verify the expected behavior of alpha on recommendation loss and bias we plot the metric we optimize on: GBS(CC) or \(\bigtriangleup Category Coverage\) and the NDCG values. As shown in Figure \ref{fig:100kabalation}, there is a general trend of the performance metric and the bias score dropping as \(\alpha\) increases. There are some fluctuations, where the bias increases for the first few values of alpha. Our intuition for this is that the fairness loss might not be enough to decrease the bias when \(\alpha\) is small. It can essentially end up disturbing the loss function itself, without explicitly decreasing bias. But as the value increases the bias scores drop lower which is expected. There is a slight increase in NDCG value for certain models, as mentioned in Section \ref{sec:perform-evaluation}, this can be the result of the fairness loss term acting as a regularization term that essentially prevents over-fitting in the model. We choose \(\alpha\) values by ensuring there is a decrease in bias, without significant loss in performance. For the ML 100K dataset, we choose 0.4, 0.3, and 0.2 respectively for MF, VAE-CF, and NeuMF models. For the ML 1M dataset, we choose 0.6,0.2, and 0.4 for MF, VAE-CF, and NeuMF models respectively. Lastly, for the Yelp dataset, we choose 0.3 for all models. 

\begin{figure}[h]
    \centering
    \includegraphics[width=0.5\textwidth]{images/alldsabalationh.pdf}
    \caption{Impact of \(\alpha\) on recommendation performance wrt NDCG@50 and the bias measure which is the difference of Category Coverage values for male and female. The experiments are performed for all three  datasets }
    \label{fig:100kabalation}
\end{figure}
\subsection{Details about Baseline Models}\label{sec:baseline}
This subsection delves into details about the models we worked on for our experiments.
\begin{itemize}[left=5pt]
    \item MF \cite{5197422}: A classical Matrix Factorization algorithm where users and items are represented as latent vectors with global bias. 
    % The rating is predicted as:
    % \[\hat{r}_{ij} = U^T_i V_j\] where \(U_i\) and \(V_j\) represent the latent factors of user $u_i$ and item $v_j$ respectively. 
    % A global bias term is also considered to adjust the ratings, to address the problem of certain items getting higher or lower ratings consistently, and some users rating items higher or lower on average.
    In our implementation, we use BPR \cite{10.5555/1795114.1795167} loss with negative sampling to enhance the performance of the model.
    \item UserKNN \cite{10.5555/2074094.2074100}:  This is a neighborhood-based method based on users, and items are recommended by discovering similar users based on the cosine similarity of their historical interactions. 
    \item ItemKNN \cite{10.1145/352871.352887}: Another neighborhood-based method that computes the similarity between items instead. 
    \item NeuMF \cite{10.1145/3038912.3052569}: NeuMF is a deep-learning based extension to MF, which combines the linearity of traditional MF models and the non-linearity of DNNs (Deep Neural Networks).
    \item VAE-CF \cite{10.1145/3178876.3186150}: This non-linear probabilistic model is based on the auto-encoder architecture and learns compressed information about data. The encoder helps map user interactions as a low-dimensional latent space, and the decoder decodes this information back to the high-dimensional vector which is used to make predictions.
\end{itemize}

The \(U_{val}\) regularization term for BeyondParity \cite{NIPS2017_e6384711} can be written as:
    % \vspace{-10pt}
\begin{equation}
    U_{\text{val}} = \frac{1}{n} \sum_{j=1}^{n} \left| \left( E_{g} [y]_{j} - E_{g} [r]_{j} \right) - \left( E_{\neg g} [y]_{j} - E_{\neg g} [r]_{j} \right) \right|
    % \vspace{-10pt}
    \label{eq:bp}
\end{equation}

where \( E_{g} [y]_{j} \) is the average predicted score for the \( j \)-th item from disadvantaged users, \( E_{\neg g} [y]_{j} \) is the average predicted score for the \( j \)-th item from advantaged users, \( E_{g} [r]_{j} \) is the average rating for the \( j \)-th item from disadvantaged users and \( E_{\neg g} [r]_{j} \) is the average rating for the \( j \)-th item from advantaged users. This fairness objective is optimized alongside the actual learning objective for a collaborative-based recommendation system. SM-GBiasedMF, is a fair recommendation model which achieves counterfactual fairness by utilizing adversarial learning. They combine recommendation loss and an adversarial loss and, the trade-off is controlled by \(\lambda\).
\subsection{Dataset Pre-processing}
For all of our datasets, we filter out inactive users. A user is considered inactive if they have less than 5 interactions. Additionally, we remove any user whose gender is not known. The Yelp dataset has over 300 categories. We reduce this set of categories to a condensed group of 21 broader categories, for better interpretability. These categories include \textit{Active Life \& Fitness}, \textit{Arts \& Entertainment}, \textit{Automotive, Bars \& Nightlife}, \textit{Coffee, Tea \& Desserts}, \textit{Drinks \& Spirits, Education \& Learning}, \textit{Event Services, Family \& Kids}, \textit{Food \& Restaurants}, \textit{Health \& Beauty}, \textit{Home \& Garden}, \textit{Miscellaneous}, \textit{Outdoor Activities}, \textit{Public Services \& Community}, \textit{Shopping \& Fashion},\textit{ Specialty Food \& Groceries}, \textit{Sports \& Recreation},\textit{ Technology \& Electronics}, \textit{Travel \& Transportation}, and \textit{Asian}.

\subsection{Process of generating Figure \ref{fig:compare_genre}}
This plot is generated using the ML-100K dataset. Once the models are done being trained we identify the users by their gender in the test set. To ensure a fair comparison we take the number of users in the smaller group, which was female in our case. We randomly chose the same number of male users from the test set. Once we have an equal number of male and female users, precision@10 values are calculated for each user and averaged by gender. To find the proportions of movies recommended, we generate the top 10 movies for each user. The proportion of each genre is computed using Equation \ref{eq:M1}.

% \subsection{Hyper-paramter selection}
% For the ML 100k dataset, we choose the best hyper-parameter values according to the optimization of both recommendation and fairness loss from epoch values of [20,50,64,100], batch size of [128,256], learning rate of [0.001,0.0001,0.0005] and user embedding size of [8,20,32]. For VAE-CF we choose from 3 auto-encoder structures: [40],[60,40], and [100,50] (please refer to Cornac\footnote{https://cornac.readthedocs.io/en/v2.2.2/} for more details). For NeuMF we chose layer structure from [32,16,8] and [64,32]. For the Yelp dataset, we use the best hyperparameter combinations selected using the ML 100K dataset since the datasets have similar interactions, users, and items.
% Similarly, for the 1M dataset, we choose the best hyper-parameter from epoch values of [100,120], batch size of [512,1024], learning rate of [0.001,0.0005], and user embedding size of [40,50,64]. For VAE-CF we choose [128,64] as the autoencoder structure and for NeuMF we chose layer structure of [128,64]. We use an early stopping strategy for the MF model, monitoring NDCG@20 with a delta of 0.0005 and patience of 10 epochs. The same early stopping strategy is used for SM-GBiasedMF to ensure a fair comparison. For deep models when experimenting on 1M, due to the computational expense, all combinations of these parameters were not explored, only a subset was used and the best one was selected.
% \section{Research Methods}

% \subsection{Part One}

% Lorem ipsum dolor sit amet, consectetur adipiscing elit. Morbi
% malesuada, quam in pulvinar varius, metus nunc fermentum urna, id
% sollicitudin purus odio sit amet enim. Aliquam ullamcorper eu ipsum
% vel mollis. Curabitur quis dictum nisl. Phasellus vel semper risus, et
% lacinia dolor. Integer ultricies commodo sem nec semper.

% \subsection{Part Two}

% Etiam commodo feugiat nisl pulvinar pellentesque. Etiam auctor sodales
% ligula, non varius nibh pulvinar semper. Suspendisse nec lectus non
% ipsum convallis congue hendrerit vitae sapien. Donec at laoreet
% eros. Vivamus non purus placerat, scelerisque diam eu, cursus
% ante. Etiam aliquam tortor auctor efficitur mattis.

% \section{Online Resources}

% Nam id fermentum dui. Suspendisse sagittis tortor a nulla mollis, in
% pulvinar ex pretium. Sed interdum orci quis metus euismod, et sagittis
% enim maximus. Vestibulum gravida massa ut felis suscipit
% congue. Quisque mattis elit a risus ultrices commodo venenatis eget
% dui. Etiam sagittis eleifend elementum.

% Nam interdum magna at lectus dignissim, ac dignissim lorem
% rhoncus. Maecenas eu arcu ac neque placerat aliquam. Nunc pulvinar
% massa et mattis lacinia.

\end{document}
\endinput
%%
%% End of file `sample-sigconf.tex'.
