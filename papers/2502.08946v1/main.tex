\pdfoutput=1

\documentclass[11pt]{article}

\usepackage[preprint]{acl}

\usepackage{times}
\usepackage{latexsym}

\usepackage[T1]{fontenc}

\usepackage[utf8]{inputenc}

\usepackage{microtype}

\usepackage{inconsolata}

\usepackage{graphicx}

\usepackage[T1]{fontenc}
\usepackage{times}
\usepackage{latexsym}
\usepackage{hyperref}
\usepackage{inconsolata}
\usepackage{url}
\usepackage{amsmath}
\usepackage{amsthm}
\usepackage{amsfonts}
\theoremstyle{definition}
\newtheorem{definition}{Definition}[section]
\usepackage{graphicx}
\usepackage{subcaption}
\usepackage{booktabs}
\usepackage{multirow}
\usepackage{makecell}
\usepackage{wrapfig}
\usepackage{enumitem}
\usepackage{comment}
\usepackage{blindtext}
\usepackage{xcolor}
\usepackage{svg}
\usepackage{xspace}
\usepackage{adjustbox}

\newcommand{\tulu}{\textsc{T\"ulu}\xspace}

\renewcommand{\cite}{\citep}

\usepackage{listings}
\lstset{
    breaklines=true,
    columns=flexible,
    xleftmargin=0.3in,
    xrightmargin=0.2in,
    breakindent=0pt,
}
\lstdefinelanguage{prompt}{
    frame=l,
    framerule=3pt,
    framesep=8pt,
    postbreak=\mbox{$\hookrightarrow$\medspace},
    basicstyle=\scriptsize\ttfamily,
    commentstyle=\color{cyan},
    morecomment=[l]{//},
    moredelim=[is][\color{red}\bfseries]{<<<}{>>>},
    moredelim=[is][\color{magenta}\bfseries]{[[[}{]]]},
    moredelim=[is][\color{orange}\bfseries]{===}{===},
    moredelim=[is][\color{olive}\bfseries]{|||}{|||},
}
\lstdefinelanguage{ioexample}{
    frame=shadowbox,
    rulesepcolor=\color{gray},
    framerule=0.5mm,
    rulesep=2mm,
    basicstyle=\small\normalfont,
    commentstyle=\color{cyan},
    morecomment=[l]{//},
    moredelim=[is][\color{red}\bfseries]{<<<}{>>>},
    moredelim=[is][\color{magenta}\bfseries]{[[[}{]]]},
    moredelim=[is][\color{orange}\bfseries]{===}{===},
    moredelim=[is][\color{olive}\bfseries]{|||}{|||},
    moredelim=[is][\bf]{:::}{:::},
    moredelim=[is][\it]{---}{---},
    moredelim=[is][\tt]{+++}{+++},
}



\newcommand\modelname{{\usefont{T1}{Discognate}{m}{n}{DTM}}\xspace}


\usepackage[T1]{fontenc}


\usepackage[utf8]{inputenc}

\usepackage{microtype}

\usepackage{inconsolata}

\usepackage[]{todonotes}
\newcommand{\fixme}[2][]{{\todo[color=yellow,size=\scriptsize,fancyline,caption={},#1]{#2}}}
\newcommand{\note}[4][]{{\todo[author=#2,color=#3,size=\scriptsize,fancyline,caption={},#1]{#4}}}
\newcommand{\mo}[2][]{{\note[#1]{MO}{blue!20}{#2}}}
\newcommand{\Mo}[2][]{\mo[inline,#1]{#2}\noindent}
\newcommand{\lemao}[1]{\textcolor{red}{\textbf{#1 --Lemao}}}
\newcommand{\shunchi}[1]{\textcolor{orange}{\textbf{#1 --Shunchi}}}
\newcommand{\rebuttal}[1]{\textcolor{red}{#1}}

\usepackage{amsthm}
\newtheorem{question}{Research Question}
\newcounter{research}

\newcounter{rqsection}

\renewcommand{\therqsection}{RQ \arabic{rqsection}}

\newcommand{\rqsection}[1]{
  \refstepcounter{rqsection} %
  \medskip
  \noindent\textbf{\therqsection: \emph{#1}} %
}


\newcommand{\datasetname}{\textsc{PhysiCo }}
\newcommand{\datasetnamens}{\textsc{PhysiCo}}

\newcommand{\coredataset}{\datasetnamens-\textsc{Core }}
\newcommand{\harddataset}{\datasetnamens-\textsc{Associative }}
\newcommand{\coredatasetns}{\datasetnamens-\textsc{Core}}
\newcommand{\harddatasetns}{\datasetnamens-\textsc{Associative}}


\title{\emph{The Stochastic Parrot on LLM's Shoulder:}\\ A Summative Assessment of Physical Concept Understanding}



\newcommand{\authorsep}{\quad}
\newcommand{\footnotemarksep}{\enspace}

\author{%
Mo Yu$^1$\thanks{Equal contribution.}\authorsep
Lemao Liu$^1$\footnotemark[1]\authorsep
Junjie Wu$^2$\footnotemark[1]\authorsep
Tsz Ting Chung$^2$\footnotemark[1]\authorsep
Shunchi Zhang$^3$\footnotemark[1]\authorsep
\\
\bfseries
Jiangnan Li$^1$\authorsep
Dit-Yan Yeung$^2$\authorsep
Jie Zhou$^1$\authorsep
\\
\textsuperscript{1}WeChat AI, Tencent\authorsep
\textsuperscript{2}HKUST\authorsep
\textsuperscript{3}JHU\\
\texttt{moyumyu@global.tencent.com}\authorsep
\texttt{redmondliu@tencent.com}\\
\texttt{\{junjie.wu,ttchungac\}@connect.ust.hk}\authorsep
\texttt{szhan256@cs.jhu.edu}\\
{\hypersetup{urlcolor=magenta} \url{https://physico-benchmark.github.io}}
}


\newcommand{\fix}{\marginpar{FIX}}
\newcommand{\new}{\marginpar{NEW}}

\begin{document}


\maketitle

\begin{abstract}
In a systematic way, we investigate a widely asked question: \emph{Do LLMs really understand what they say?}, which relates to the more familiar term \emph{Stochastic Parrot}.
To this end, we propose a summative assessment over a carefully designed physical concept understanding task, \datasetnamens.
Our task alleviates the memorization issue via the usage of grid-format inputs that abstractly describe physical phenomena.
The grids represents varying levels of understanding, from the core phenomenon, application examples to analogies to other abstract patterns in the grid world.
A comprehensive study on our task demonstrates: (1) state-of-the-art LLMs, including GPT-4o, o1 and Gemini 2.0 flash thinking, lag behind humans by $\sim$40\%; (2) the stochastic parrot phenomenon is present in LLMs, as they fail on our grid task but can describe and recognize the same concepts well in natural language;
(3) our task challenges the LLMs due to intrinsic difficulties rather than the unfamiliar grid format, as in-context learning and fine-tuning on same formatted data added little to their performance.
\end{abstract}

\section{Introduction}

% State of the world (robots for creative activites)
The term ``robot,'' originally signifying `forced labor,' has long been associated with labor and work. Robots have demonstrated their utility in various automated productive and social contexts, where the primary goals are improving productivity, safety, and fostering social interactions with humans~\cite{simoes2022designing, weidemann2021role, honig2018understanding}. However, an increasing number of cases feature using of robots in creative settings. Unlike productive contexts, where the focus is on efficiency and task completion~\cite{arents2022smart}, or social contexts, where communication and trust are prioritized~\cite{nam2020trust, saunderson2019robots}, creative environments prioritize artistic innovation and expression~\cite{hsueh2024counts}. This shift fundamentally alters the dynamics of human-robot interaction, redefining the roles and expectations for both humans and robots.

For instance, robots’ social behaviors are leveraged to support the generation and expression of creative ideas~\cite{hu2021exploring, sandoval2022human, alves2020creativity}, and programmable robotic movements and trajectories are employed to inspire artistic activities such as sketching~\cite{lin2020your}. These studies often engage participants from creative fields who possess limited prior experience with robotics, and are typically conducted in short-term, experimental settings. Consequently, the findings from these studies remain constrained since much can be learned from professional practitioners' experiences to inform system design such as digital fabrication~\cite{hirsch2023nothing}. There is a notable gap in research examining the long-term, active, and practical experience of integrating robotic systems into the creative processes. As a result, the deeper insights into how robots facilitate and shape creative processes, beyond simply augmenting human creativity, remain underexplored. In this study, we aim to better understand the impacts of robots on creative processes and outcomes.

As early as Leonardo da Vinci's 16th century ``Automaton,'' artists have explored the creative affordances of robotic systems~\cite{shanken2002cybernetics, pagliarini2009development, jeon2017robotic}. The artistic creation process typically encompasses various stages, including the exploration of materials and techniques, ongoing experimentation and iteration, and the continual refinement of the artists' insights into their creative subjects~\cite{lewis2023art, sturdee2022state}. Therefore, investigating the artistic process involving robots offers an opportunity to gain deeper insights into robots' creative potential. Robotic art, in particular, provides a compelling case for this exploration.

We define robotic art as artworks that utilize robotic or automated machines to create artistic experiences and tangible artifacts. One example is robotic installation art, in which robots are programmed to follow specific rules that embody the artist’s expression (\autoref{fig:teaser} (a)). Another example is responsive art, in which robots react to their environment, with behaviors that change over time or in response to spectators (\autoref{fig:teaser} (b)). Additionally, there are robotic creators, which possess a degree of agency, allowing them to collaborate with human artists and produce works that extend beyond mere replication of human-created art (\autoref{fig:teaser} (c) and (d)). As such, robotic art becomes a rich case for exploring human-machine interactions in creative contexts. Gaining a deeper understanding of how robots facilitate artistic expression can provide insights for designing computing systems to support creative activities~\cite{gomez2021robot}.

% Therefore, we did...
We draw on semi-structured, in-depth interviews with renowned professional robotic artists to investigate the use of robots in artistic practice. Specifically, our goal is to understand how artistic exploration of robotic systems challenges conventional assumptions about the functions of robots, such as their roles in automating repetitive tasks or serving human needs. We also explore the implications of robots in the artistic process and examine how creativity may emerge within robotic art. To address these interrelated inquiries, our study focuses on the practice of robotic art, posing the research question: \textit{How do robotic artists utilize robots in their artistic practice?} We approach this inquiry through the perspectives and experiences of robotic artists, who creatively design, modify, and repurpose robotic systems for artistic expression and exploration.

% The key findings are...
Our findings highlight the social, material, and temporal dimensions of artists' practices that shape their creativity and artistic outcomes. The creation of robotic art is largely a social process, as artists receive both explicit and implicit feedback through the audience's reactions and reception of their work. Simultaneously, the embodiment and malfunctions inherent to robotic systems drive artistic experimentation. The temporal processes of creation and exhibition, beyond just the final product, further enhance the creative value. Our empirical analysis presents how creativity emerges through the interplay of social, material, and temporal interactions among artists, robots, audiences, and the environment.

% The contributions of this work are...
We make two main contributions to HCI in this study. 
First, we elucidate the interactive mechanisms among key actors---human creators, machines, audiences, and environments---within the practice of robotic art, a topic that remains underexplored in HCI. Our findings reveal the significance of sociality (e.g., interactions between artists and audiences), materiality (e.g., the embodiment and malfunctions of robots), and temporality (e.g., the processes of creation and exhibition) in shaping creative values. We propose that these three facets are central to the creative process and facilitate the emergence of creativity in robotic art.
Second, drawing from the findings, we offer implications for \textit{socially informed}, \textit{material-attentive}, and \textit{process-oriented} creation with computing systems. We suggest leveraging these three aspects to enhance creativity and the creative experience. Specifically, we discuss the value of incorporating implicit audience feedback, designing with technical malfunctions, and focusing on the post-creation process to foster alternative creative experiences with machines~\cite{alter2010designing, juarez2022glitch}.



\section{Measuring Concept Understanding via Summative Assessment}
\label{sec:towards}







It is intrinsically challenging to measure the extent to which LLMs {understand} a sentence or concept. Indeed, ~\citet{bender2020climbing} provide a definition of "understanding" from a linguistic perspective, but this definition depends on another abstract and unmeasurable term, ``\emph{meaning}''. 
Therefore, even with this definition, accurately measuring "understanding" remains elusive.

We approach the measurement of whether LLMs understand a concept from an educational and cognitive perspective, using \textbf{summative assessment}~\cite{black1998assessment,black1998inside,harlen1997assessment}.
Summative assessment is widely used by educators as an appealing strategy to evaluate students' understanding and knowledge acquisition in educational and cognitive psychology.
For example, when middle school physics teachers want to know whether a student truly understands the concept ``\emph{Gravity}'', they would design a series of questions specifically related to the concept of gravity to assess comprehension, \emph{e.g.}, the properties like inverse square law and examples like orbital motions. If a student struggles to answer many of these questions, the teacher may conclude that the student does not understand the concept well or has a poor grasp of it.

We extend the idea of summative assessment to evaluating the concept understanding of machines. Formally, assume $\mathcal{S}$ denotes an intelligent system and $\mathcal{C}$ is a specific concept.
To evaluate the extent how $\mathcal{S}$ understands the concept $\mathcal{C}$, our summative assessment includes the following two steps:
\begin{itemize}[noitemsep,nolistsep,leftmargin=*]
    \item \emph{Task design towards $\mathcal{C}$}: design several concept understanding tasks, each of which consists of several questions manually created towards understanding the concept $\mathcal{C}$.
    \item \emph{Evaluating $\mathcal{S}$}: 
    ask $\mathcal{S}$ to answer the questions from the tasks and calculate its accuracy. 
\end{itemize}







\paragraph{Requirements for Validity}
The success (validity) of the proposed evaluation approach highly depends on the task design~\cite{black1998assessment,black1998inside}. For example, if the questions are too easy, even a weak system could answer them correctly. This leads to an overestimation of the system's understanding capabilities, making the assessment ineffective. 
To ensure good validity, we adhere to the principles outlined in summative assessment~\cite{black1998assessment,black1998inside} for task design:
\begin{itemize}[noitemsep,nolistsep,leftmargin=*]
    \item \emph{Alignment with evaluating objectives}: the questions should be related to the targeted concept, and should measure the specific knowledge about the targeted concept. 
    \item \emph{Different difficulty levels}: the questions should be with different difficulty levels from easy to difficult level,  to ensure that the evaluation results have distinctiveness for different systems.
    \item \emph{Variety}: the questions should reflect various understanding aspects of the targeted concept; addressing both its denotation and connotation.
    \item \emph{Simplicity}: while not mandatory, a simpler benchmark for humans can more effectively highlight the issue faced by current models, i.e., the stochastic parrot effect in LLMs.
\end{itemize}




\section{Dataset Collection and Description}

To facilitate our analysis of \emph{personalized} story generation, we ensemble \dataname, a novel dataset consisting of 590 stories written by 64 authors. To the best of our knowledge, \dataname~is the first dataset that identifies and connects multiple stories written by the same author. See Table~\ref{tab:compare-datasets} for comparison with existing datasets. 

\paragraph{Data Sources:} We include five diverse story-writing sources, namely
\textbf{(1) Reddit}, %\footnote{\url{https://www.reddit.com/r/WritingPrompts/}} 
featuring stories from r/WritingPrompts, a widely-used resource for amateur story-writing research \citep{fan-etal-2018-hierarchical},   
\textbf{(2) AO3}, %\footnote{\url{https://archiveofourown.org/}} 
featuring fanfiction based on popular franchises such as \textit{Harry Potter} and \textit{Sherlock Holmes},   \textbf{(3) Storium}, %\footnote{\url{https://storium.cs.umass.edu/}} 
featuring collaborative stories \citep{akoury-etal-2020-storium}, 
\textbf{(4) N.Magazine}, %\footnote{\url{https://www.narrativemagazine.com/}} 
featuring professionally polished stories by authors such as Barry Gifford, and finally
\textbf{(5) New Yorker},  %\footnote{\url{https://www.newyorker.com/}} 
featuring expert-level storytelling from renowned authors such as Haruki Murakami.


\paragraph{Preprocessing:} To ensure high-quality and diverse content, we impose constraints on word length (ranging from 500 to 1500 words), limit publication dates to post-November 2023, and exclude explicit material. To standardize formatting, non-Reddit stories are supplemented with writing prompts generated by GPT-4o and manually reviewed for accuracy. Stories are split chronologically based on their submission timestamps, with 70\% used for \emph{profiling} writing characteristics and 30\% for \emph{generation} (see  Appendix~\ref{app:dataset} for details).

\paragraph{Dataset size:} Our dataset is limited  in size due to data crawling policies that restrict large-scale data collection for analysis\footnote{\url{https://archiveofourown.org/admin_posts/25888}}. We note that recent storywriting datasets such as TELL ME A STORY \citep{huot2024agents} and TTCW \citep{chakrabarty2024art} have conducted analyses on a comparable number of stories (Table~\ref{tab:compare-datasets}).



\begin{table*}[htbp]
\centering
\caption{Comparison of our dataset with existing story-writing datasets. \ding{55} indicates no Author IDs associated with stories, and $\sim$ denotes having Author IDs but not having explicit links between them. Our dataset spans diverse settings and links stories by the same author, enabling research on personalized story generation.}
\label{tab:compare-datasets}
\small
\begin{tabular}{m{5cm}|m{1cm}m{3.75cm}m{1.25cm}m{1.25cm}m{1cm}}
\toprule
\textbf{Dataset Name} & \textbf{Size} & \textbf{Sources} & \textbf{Prompt Length} & \textbf{Story Length} & \textbf{Author IDs} \\
\midrule
WritingPrompts \citep{fan-etal-2018-hierarchical}       & 300K         & Reddit & 28             & 735              & \ding{55} \\ 
%ROCStory \citep{mostafazadeh-etal-2016-corpus}            &   50K      & Crowd Workers        & 9              & 41             & \ding{55}      \\ 
TELL ME A STORY \citep{huot2024agents}      & 230         & Writing Workshop      & 113             & 1498            & \ding{55}  \\ 
MirrorStories \citep{yunusov-etal-2024-mirrorstories}      & 1500         & Aesop’s fables      & 40             & 400            & \ding{55}  \\ 
%STORYSUMM \citep{subbiah-etal-2024-storysumm}      & 96         & Reddit      & -             & 964            & \ding{55}  \\ 
Storium \citep{akoury-etal-2020-storium}      & 440k         & Storium online platform      & 247             & 247            & $\sim$  \\ 
TTCW \citep{chakrabarty2024art}      & 12         & New Yorker      & 54             & 2000            & $\sim$  \\ 
\midrule
\dataname  & 590   & Reddit, AO3, Storium, N.Magazine, New Yorker           & 50           & 1517          & \checkmark \\ 
\bottomrule
\end{tabular}
\end{table*}

\section{Validation on Low-Level Subtasks}


To illustrate the stochastic parrot phenomenon with \datasetnamens, a necessary condition is to ensure the LLMs can perform well on the low-level understanding subtasks, \emph{i.e.}, whether LLMs exhibit strong skills of \emph{recalling} and \emph{describing} the definitions, core properties and representative examples of the physical concepts in our tasks. That is:



\rqsection{Can LLMs perform well on low-level subtasks, i.e., understanding the definitions of physical concepts in natural language?}
\label{rq:textual_input}

To answer \ref{rq:textual_input}, we evaluate the LLMs' abilities to comprehend the definitions of these concepts and generate their descriptions and examples in natural language, as defined in Section~\ref{sec: low-level}.

\subsection{Concept Selection Subtask}
\paragraph{Settings} 
We provide the standard definition of a concept based on Wikipedia with its synonyms masked; then ask the LLMs to identify the concept, under the same four-choice setting throughout the experiments.
We evaluate the representative text-only LLMs and compute the accuracy. 

\begin{table}[t!]
    \small
    \centering
    \begin{tabular}{ccccc}
    \toprule
    \multirow{2}{*}{(a)}& \bf Mistral & \bf Llama-3 & \bf GPT-3.5 & \bf GPT-4 \\
     \cmidrule{2-5}
    & 81.0$_{\pm\text{1.3}}$& 88.5$_{\pm\text{0.7}}$& 97.3$_{\pm\text{0.3}}$ & 95.0$_{\pm\text{0.9}}$\\
    \bottomrule
    \toprule
    \multirow{2}{*}{(b)}& \bf InternVL & \bf LLaVA & \bf GPT-4v & \bf GPT-4o \\
     \cmidrule{2-5}
    & 66.3$_{\pm\text{7.7}}$ & 66.7$_{\pm\text{5.8}}$ & 93.7$_{\pm\text{0.9}}$ &93.7$_{\pm\text{0.5}}$\\
    \bottomrule
    \end{tabular}
    \vspace{-0.1in}
    \caption{Accuracy on the text-based (a) and visual-based (b) concept selection subtasks.}
    \label{tab:selection}
    \vspace{-0.2in}
\end{table}
\paragraph{Results} Table~\ref{tab:selection} shows that the GPT (both text-based and visual-based) models perform near perfect on 
recognition of our physical concepts from standard text-based definitions and from the real-life images.
Moreover, we observed that open-source models make more mistakes compared with the closed-source models due to the smaller model size. For the text-based models, both Mistral and Llama-3 are not as good as the closed-source models. Surprisingly, both InternVL and LLaVA are much worse than the open-source GPT models. One possible reason to this discrepancy is that our text-based concepts are from Wikipedia which is usually used as a part of the training data for open-source LLMs. In contrast, some of our selected images for those concepts may not be included in the training data of both InternVL and LLaVA which thereby can not memorize those visual instances. 


\subsection{Concept Generation Subtask}
\paragraph{Settings}
We evaluate the descriptions LLMs generate for a concept. 
The evaluation of text generation is in general difficult. Moreover, in our scenario each concept have many different ground-truth examples in its description, thus existing automatic metrics such as BLEU~\cite{papineni2002bleu} and METEOR~\cite{banerjee2005meteor} are not capable of accurately measuring the quality. 
Therefore, we rely on mainly human evaluation for this subtask. We also propose an automatic metric via a self-play game for completeness in Appendix~\ref{app:self_play}. 


\paragraph{Human evaluation metric} We ask the annotators to evaluate the quality of the generated descriptions. The evaluation uses binary scores: each description receives a score of 0 if it consists of any factual error on the concept itself
or any unfaithful examples,\footnote{For example, if the LLMs generated a wrong year in the description, it is not counted as incorrect physical knowledge.} 
and a score of 1 otherwise. %

\begin{table}[t!]
    \small
    \centering
    \begin{tabular}{cccc}
    \toprule
      \bf Mistral & \bf Llama-3 & \bf GPT-3.5 & \bf GPT-4 \\
     \midrule
    92.6& 100 &100 & 100\\
    \bottomrule
    \end{tabular}
    \vspace{-0.1in}
    \caption{Human evaluations on concept generation.}
    \vspace{-0.2in}
    \label{tab:generation}
\end{table}

\paragraph{Results}
The results of automatic and human evaluations are shown in Table~\ref{tab:generation}. 
According to human evaluation, there are no factual errors in the generated descriptions except for Mistral,
confirming that our selected concepts rely on basic and widely accepted knowledge.
Thought accurate, the open-source LLMs sometimes include correct but uncommon facts, \emph{e.g.,} listing single-slit diffraction as an example of \emph{Wave Interference}.
The additional self-play results in Appendix~\ref{app:self_play} further justify that all LLMs can accurately recognize the concepts from the descriptions they wrote by themselves.
Combining the conclusions, 
it shows the LLMs can generate correct and sufficient information.

\paragraph{Remark} We asked the annotators of our \textsc{Core} set to evaluate whether the core properties they annotated are covered by the LLMs' generated descriptions.
This corresponds to measuring the recall of the generated descriptions on core properties/examples of concepts from \coredatasetns. The recall rates for GPT-3.5 and GPT-4 are \emph{85.0} and \emph{90.0}, respectively. 
Of course, there are some exceptional examples from \coredatasetns\ missed in the descriptions. One example is that the LLMs fails to draw the connection between \emph{movable pulley} and the \emph{Lever} concept. Moreover, by manually checking these missed properties and examples, we found that most of them can be recalled if we query the LLMs in a second turn by prompting ``{\small \texttt{Any more core properties or examples?}}''. This confirms that the LLMs are \emph{aware of} and are \emph{able to recall} the core properties of concepts covered by the \coredatasetns, though some of them may not have the top conditional probabilities of generation.


\paragraph{Conclusion} LLMs understand the concepts covered by \datasetname in natural language format.
Notably, we find that the {properties and examples annotated in \coredatasetns\ are \emph{within the LLMs' knowledge} and are \emph{highly likely to pop up} when the corresponding physical concepts are queried}.


\section{Experiments}

\subsection{Datasets}

\textbf{MSMARCO}.
We utilized the MS MARCO Passage Ranking dataset as the data source to evaluate the ability of our method to improve document rankings under more challenging topic-query tasks. Specifically, we assessed whether our method could significantly enhance the ranking of documents by the retrieval model within a RAG system.

To construct topic-lists for evaluation, we applied a K-means clustering algorithm to group similar queries, forming topics that each contained a series of related queries. To further evaluate the performance of our method under extreme topic-query scenarios, we applied an intra-topic similarity filtering process. Only topics with queries exhibiting high semantic diversity and containing a sufficient number of queries were retained.

This process resulted in 29 topics, with each topic containing an average of 22.28 queries. The average similarity score within each topic was approximately 0.5, indicating sufficient diversity among queries to ensure a rigorous evaluation. This curated dataset enabled us to test the robustness of our method in handling highly diverse and challenging topic-query tasks within a RAG system.

\textbf{PROCON}.
To conduct our experiments, we utilized controversial topic data scraped from the PROCON.ORG website. This dataset includes over 80 topics covering various domains, such as society, health, government, and education. Each topic is discussed from two stance labels \{\textit{PRO (support), CON (oppose)}\}, with passages arguing from these perspectives.

To simulate real-world user interactions with a RAG system, we instructed a large language model (GPT-4o) to act as a user and generate 40 potential sub-queries for each topic. These sub-queries were designed to reflect the diverse questions and concerns users might raise when exploring a specific controversial topic. 

After generating the sub-queries, we applied a similarity filtering process to ensure diversity by retaining only those with a similarity score below approximately 0.85. The filtering step effectively removed redundant queries while preserving a wide range of perspectives. As a result, the final set of topic-queries achieved an average similarity score of approximately 0.7, ensuring that the queries were sufficiently diverse yet semantically relevant. This process provided a robust and balanced sub-queries set for evaluation.


\subsection{Experiment Details}
The specific setting details for the Topic-queries RAG manipulation experiment are as follows:

(1) Black-box RAG. We represent the black-box RAG process as \( \text{RAG}_{\text{black}} \). The RAG framework is Conversational RAG from LangChain. The LLMs adopted in RAG are the open-source models Meta-Llama-3.1-8B-Instruct (Llama3.1), Qwen-2.5-7B-Instruct(Qwen2.5). The system prompt and additional detailed descriptions are provided in Appendix~\ref{exp-detail}.

(2) Retrieval Model Specification. We benchmark three dominant dense retrievers—Contriever \cite{gao2021unsupervised}, DPR \cite{karpukhin-etal-2020-dense}, and ANCE \cite{xiong2020approximate}.By convention, we use dot product between the embedding vectors of questions(queries) and candidate documents as their similarity score \(R\) in the ranking. 


\label{opinion-classfication}
(3) Opinion classification. We use Qwen2.5-Instruct-72B as the opinion classifier. Qwen2.5-Instruct-72B, due to its large parameter size, is capable of accurately identifying and classifying opinions within text.

(4) Experimental parameters. In knowledge-guided attack process, we set the maximum editing distance $\epsilon$ to 0.2, the semantic similarity threshold $\lambda$ to 0.85, and the number of iterations $N$ to 5. For adversarial trigger generation, we used a beam size of 3, top-$k$ of 10, a batch size of 32, a temperature of 1.0, a learning rate of 0.005, and a sequence length of 10. In RAG\textsubscript{black}, $k$ (the number of retrieved documents) is set to 3, with the LLM temperature also fixed at 1.0 to mirror real-world conditions.

(5) Poisoned Target. For the PROCON dataset, to investigate the manipulation performance under more challenging conditions, we performed relevance ranking for the documents with respect to each topic-query set $Q$ and the target stance $S_t$ . From the ranked list, we selected the last five documents (i.e., those with the lowest relevance) as the target poisoned documents.
For the MS MARCO dataset, we utilized the top-1000 relevance-ranked passage list for each topic-query set. From this list, we selected the passage with the lowest average rank as the target passage. This approach ensures that the evaluation focuses on passages that are least relevant to the target queries, thus providing a more rigorous benchmark for the proposed method.

(6) Experimental environment. All our experiments are conducted in Python 3.8 environment and run on a NVIDIA DGX H100 GPU. 

\subsection{Research Questions}

We propose four research questions to evaluate the effectiveness of our method in the topic-queries task, focusing on black-box NRM attacks and opinion manipulation to RAGs.

\textbf{RQ1}: Can Topic-FlipRAG significantly enhance the rankings of target documents in the RAG retriever for topic-queries?

\textbf{RQ2}: To what extent does Topic-FlipRAG affect the answers generated by the target RAG systems?

\textbf{RQ3}: Does topic-oriented opinion manipulation significantly impact users' perceptions of controversial topics?

\textbf{RQ4}: How robust does Topic-FlipRAG against exisiting mitigation mechanism?

\subsection{Baseline Settings}
To assess the effectiveness of our proposed method, we compare it against adversarial attack baselines designed for black-box, topic-oriented RAG scenarios, ensuring minimal modifications to the original documents. We exclude BadRAG\cite{xue2024badrag}, a backdoor RAG attack limited to white-box scenarios, and topic-IR-attack\cite{liu2023topic}, as its incomplete implementation prevents reliable replication.
For the selected baseline methods, we adapted them to meet the requirements of our task while preserving their core components. A brief overview of the baseline methods is provided below, with detailed descriptions available in Appendix~\ref{baselines-details}.

\textbf{PoisonedRAG.}
Zou et al.\cite{zou2024poisonedrag} propose an approach adaptable to both black-box and white-box settings. For our task, we employ its black-box strategy by inserting a randomly chosen query from the topic-queries set \( Q \) at the beginning of each document.

\textbf{PAT.}
This gradient-based adversarial retrieval attack uses a pairwise loss function to generate triggers that meet fluency and coherence constraints. We adapt PAT to produce triggers \( T_{\text{pat}} \) for target documents within the topic-queries set, evaluating their effectiveness under black-box conditions.


\textbf{Collision.}
This method generates adversarial paragraphs (collisions) via gradient-based optimization to produce content semantically aligned with the target query. In a topic-queries context, we create collisions for the entire topic-queries set and examine their transfer performance on black-box RAG retrievers.

These baseline methods provide benchmarks for comparing the efficacy of our approach in a fully black-box, topic-oriented RAG attack scenario.

\subsection{Evaluation Metrics}

For \textbf{RQ1}, we focus on ranking manipulation. We measure the average proportion of target opinions in top-3 rankings before and after manipulation (\(\text{Top3}_{\text{ori}}, \text{Top3}_{\text{att}}\)) and define top3-v as their difference. We also employ the Ranking Attack Success Rate (RASR), reflecting how often target documents are successfully boosted, and Boost Rank (BRank), denoting the average rank improvement for all target documents. Lastly, we report the proportion of target documents in the Top-50 and Top-500 positions to indicate how effectively they are pushed toward higher rankings.

\textbf{top3-v.} Computed by subtracting \(\text{Top3}_{\text{ori}}\) from \(\text{Top3}_{\text{att}}\), top3-v ranges from -1 to 1. A positive value signifies a successful increase of the target opinion in top-3 results, while a negative value indicates a detrimental effect.

\textbf{Ranking Attack Success Rate (RASR).} RASR captures how frequently target documents are successfully boosted in each query’s ranking. Higher values indicate greater attack effectiveness.

\textbf{Boost Rank (BRank).} BRank is the average rank improvement for all target documents under each query. A target document contributes negatively if its rank is unintentionally lowered.

\textbf{Top-50, Top-500.} These metrics represent the percentage of target documents that move into specific ranking thresholds in the MS MARCO Dataset after manipulation. Higher percentages imply more effective promotion of target documents. 


For \textbf{RQ2}, we employ Average Stance Variation (ASV) to assess how significantly our opinion manipulation influences the LLM’s responses in a black-box RAG. To address the natural variability of controversial topics and the inherent instability of large language models, we also propose Real Adjusted ASV (\(\Delta\)-ASV).

\textbf{Average Stance Variation (ASV).}
ASV is defined as the absolute difference between the manipulated opinion score and the original opinion score assigned to an LLM response (0 = opposing, 1 = neutral, 2 = supporting). A higher ASV signifies a more pronounced shift in polarity and hence greater manipulation effectiveness.

\textbf{Real Adjusted ASV ($\Delta$-ASV)}. To account for the inherent variability of controversial topics and the instability of large language models, we measure the baseline ASV in a clean state, denoted as ASV\textsubscript{clean} (calculated without adversarial manipulation). $\Delta$-ASV is then obtained by subtracting ASV\textsubscript{clean} from the manipulated ASV, i.e., \( \text{$\Delta$-ASV} = \text{ASV} - \text{ASV\textsubscript{clean}} \). This adjustment ensures that $\Delta$-ASV reflects the true impact of adversarial manipulation by eliminating the influence of natural stance variation. It reflects the extent to which the polarity of the RAG-system outputs is affected by the manipulation.  A positive $\Delta$-ASV indicates a significant shift in the opinion polarity due to manipulation, with larger values representing greater manipulation effectiveness.

\section{Related Works}\label{sec:related}
In this section, we discuss previous work relevant to BaKlaVa in five main areas: KV-cache eviction policy, profiling for determining memory budget, KV-cache quantization, cache merge, and system-level optimizations. 

\subsection{KV Cache Eviction Policy}
StreamingLLM~\cite{streamingllm} discovered the 'attention sink' effect, where early sequence tokens play a crucial role in maintaining model performance through asymmetric attention weight accumulation. H2O~\cite{h2o} introduces an eviction strategy based on cumulative attention, retaining 'heavy-hitter' key-value pairs while allowing token positions to vary. Similarly, Scissorhands~\cite{scissorhands} develops an approach that evicts based on a 'pivotal' metric, adjusting eviction rates across layers using a persistence ratio. Keyformer~\cite{keyformer} addresses the issue of token removal distorting softmax probability distributions by implementing regularization techniques to mitigate these perturbations. 

\subsection{Profiling for Determining Memory Budget}
Squeezeattention~\cite{squeezeattention} employs a dynamic approach, measuring layer importance through cosine similarity of input prompt differences pre- and post-self-attention, subsequently categorizing layers and adjusting their KV budgets. PyramidInfer~\cite{pyramidinfer} introduces a pyramid-shaped allocation strategy, prioritizing tokens with high attention values and maintaining a set of significant tokens through attention-driven updates during decoding. In comparison, Ada-KV~\cite{adakv} offers an adaptive budget allocation method that improves utilization across individual attention heads, resulting in more effective cache eviction strategies.



\subsection{KV-Cache Quantization}
GEAR~\cite{gear} takes a different approach by compressing less important entries to ultra-low precision, using a low-rank matrix for residual error approximation, and utilizing a sparse matrix for outlier correction. MiKV~\cite{notokenleftbehind} introduces a mixed-precision KV-cache quantization method, allocating precision based on token importance. QAQ~\cite{qaq} proposes a dynamic, quality-adaptive quantization approach that determines bit allocation based on token importance and sensitivity. KVQuant~\cite{hooper2024kvquant} offers strategies for smooth quantization of keys and values, including pre-RoPE quantization for keys, per-token quantization for values, and isolation of outliers in a sparse format. These diverse techniques collectively contribute to significant improvements in model compression and efficiency while maintaining performance.

\subsection{Cache Merge}
MiniCache~\cite{minicache} leverages the high angular similarity observed in middle-to-deep layer KV caches, merging key and value pairs from adjacent similar layers into shared representations. KVSharer~\cite{yang2024kvsharer}, on the other hand, exploits the counterintuitive finding that sharing KV caches between significantly different layers during inference does not substantially impact performance, prioritizing dissimilar layers for sharing based on Euclidean distance calculations. In comparison,  CaM~\cite{cam} focuses on merging keys or values of multiple evicted tokens with retained tokens using attention scores, while KVMerger~\cite{wang2024model} employs a two-step process: first clustering consecutive tokens with high cosine similarity, then merging tokens within each set into a pivotal token chosen by the attention score, using Gaussian kernel weights to emphasize contextual relevance. 

\subsection{System-Level Optimizations}
FlexGen~\cite{flexgen} proposes an SSD-based method for managing key-value (KV) caches, effectively expanding the memory hierarchy across GPU, CPU, and disk storage. This approach utilizes linear programming to optimize tensor storage and access patterns, enabling high-throughput LLM inference on hardware with limited resources. Complementing this, ALISA~\cite{alisa} introduces a dual-level KV cache scheduling framework that combines algorithmic sparsity with system-level optimization. At the algorithmic level, ALISA employs a Sparse Window Attention mechanism to identify and prioritize crucial tokens for attention computation, while at the system level, it implements a three-phase token-level dynamic scheduler to manage KV tensor allocation and balance caching and recomputation.


\section{Conclusion}\label{sec:conclusion}

In this paper, we proposed a prototype ASL generation system aimed at improving the naturalness, comprehensiveness, and overall quality of generated signs, addressing key limitations in existing approaches. Our technical evaluations indicate that our proposed approaches improve these aspects, enhancing the quality of generated ASL content. Feedback from DHH participants was mixed; while there was general interest in the system, concerns regarding visual quality and naturalness were noted. Reflecting on our design process and study findings, we discuss key insights and identify key areas for future improvement. While further work is needed, our study takes an initial step toward developing sign language generation systems that better meet the needs of the DHH and signing communities, offering real-world value.

\section*{Acknowledgment}
We thank the anonymous reviewers for their constructive feedback.
We also express our gratitude to Mr. François Chollet for developing the ARC benchmark and the annotation tool for abstract grid tasks\footnote{\url{https://arc-editor.lab42.global/?next=\%2Feditor}}. His introduction of this tool to us was particularly instrumental in the creation of \datasetnamens. This work has also been made possible by a Research Impact Fund project (RIF R6003-21) and a General Research Fund project (GRF 16203224) funded by the Research Grants Council (RGC) of the Hong Kong Government.




\bibliography{custom}


\appendix
\clearpage

\section{Details of the Included Concepts in our \datasetnamens}
\label{app:dataset_details}

\paragraph{Concepts in \coredatasetns}
The concepts in \coredatasetns\ are basic physical concepts that we manually design problems for. The development set covers 27 concepts and the test set covers 25 concepts as follows:

\begin{table}[h!]
    \small
    \centering
    \begin{adjustbox}{width=\linewidth}
    \begin{tabular}{l c||lc}
    \toprule
    reference frame&12&gravity&10\\
    reflection&10&refraction&10\\
    light imaging&10&communicating vessels&10\\
    cut&10&laser&10\\
    surface tension&10&move&10\\
    \midrule
    buoyancy&10&acceleration&10\\
    inertia&10&electricity&10\\
    repulsive force&8&wave&8\\
    lever&6&optical filters&6\\
    compression&4&diffuse reflection of light&4\\
    \midrule
    wave interference&4&diffusion&4\\
    vortex&4&expansion&4\\
    nuclear fission&2&nuclear fusion&2\\
    diffraction of waves&2\\
    \bottomrule
    \end{tabular}
    \end{adjustbox}
    \caption{Concepts and their corresponding number of instances in \coredatasetns-Dev.}
    \label{tab:concept_stats_core}
\end{table}

\begin{table}[h!]
    \small
    \centering
    \begin{adjustbox}{width=\linewidth}
    \begin{tabular}{l c||lc}
    \toprule
    atmospheric pressure & 12 & energe conservation & 10 \\
    elastic force & 10 & friction & 9 \\
    photoelectric effect & 8 & heat conduction & 8 \\
    doppler effect & 8 & electromagnetic wave & 8 \\
    melting & 8 & vaporization & 8 \\
    \midrule
    fluid pressure & 8 & thermal expansion and contraction & 8 \\
    Brownian motion & 8 & splashing & 8 \\
    oscillation & 8 & relativity & 8 \\
    lighting & 8 & lifting & 8 \\
    force composition & 8 & pulley & 8 \\
    \midrule
    inclined plane & 8 & Bernoulli effect & 7 \\
    fictitious force & 6 & siphon & 6 \\
    resonance & 4 & ~ & ~ \\
    \bottomrule
    \end{tabular}
    \end{adjustbox}
    \caption{Concepts and their corresponding number of instances in \coredatasetns-Test.}
    \label{tab:concept_stats_core_test}
\end{table}

\paragraph{Concepts in \harddatasetns}
The following table summarized all the concepts from \harddatasetns:

\begin{table}[h!]
    \small
    \centering
    \begin{adjustbox}{width=\linewidth}
    \begin{tabular}{l c||lc}
    \toprule
    mirror                    & 30 & laser                     & 20 \\
    zoom in                   & 15 & magnet                    & 14 \\
    wave                      & 13 & explosion                 & 11 \\
    compression               & 10 & rotation                  & 10 \\
    gravity                   &  9 & expansion                 &  9 \\
    \midrule
    move                      &  8 & change of reference frame &  8 \\
    water ripples             &  7 & long exposure             &  7 \\
    reflection                &  5 & wetting                   &  5 \\
    diffusion                 &  4 & zoom out                  &  3 \\
    projection                &  2 & polarization of light     &  1 \\
    \midrule
    vortex                    &  1 & chemical bond             &  1 \\
    nuclear fission           &  1 & squeeze                   &  1 \\
    nuclear fusion            &  1 & lumination                &  1 \\
    wave interference         &  1 & optical filter            &  1 \\
    vacuum                    &  1 &  \\
    \bottomrule
    \end{tabular}
    \end{adjustbox}
    \caption{Concepts and their corresponding number of instances in \harddatasetns.}
    \label{tab:concept_stats_all}
\end{table}


\section{Details of Analysis Methods in \ref{rq:textual_input}}
\label{app:rq1_details}
\subsection{Masking of Textual Descriptions}
This experiment follows the setting in the ``Physical Concept Selection Subtask'' in section \ref{sec: low-level}. The definitions of the corresponding phenomena were extracted from Wikipedia as well as generated by GPT-3.5 and GPT-4. To maintain consistency, the terms representing concepts were masked as {\small \texttt{[PHENOMENON]}} while relevant terms are masked as {\small \texttt{[MASK]}}. For instance, ``interference'' which corresponds to the phenomenon ``wave interference'' was masked as {\small \texttt{[PHENOMENON]}}. In contrast, ``Newton's first law of motion'' which corresponds to the phenomenon ``inertia'' was masked as {\small \texttt{[MASK]}}.

An example of the masked description can be found in Figure~\ref{fig:masked_description_example}.

\subsection{Prompts Used for Description Generation and Classification}
Figure~\ref{fig:nl_gen_prompt_template} and \ref{fig:nl_guess_prompt_template} include the prompts used for generation and classification respectively.


\begin{figure}[h]
    \centering
    \lstinputlisting[language=prompt]{prompt/textual_generation.txt}
    \caption{The prompt template used for generating descriptions of physical concepts (denoted as the variable \textcolor{magenta}{\small \textbf{\texttt{CONCEPT}}}) in \ref{rq:textual_input}.}
    \label{fig:nl_gen_prompt_template}
\end{figure}

\begin{figure}[h]
    \centering
    \lstinputlisting[language=prompt]{prompt/textual_guessing.txt}
    \caption{The prompt template used for guessing the referred physical concept from four candidates (denoted as the variable \textcolor{magenta}{\small \textbf{\texttt{CANDIDATE ANSWERS}}}) from the natural language descriptions (denoted as the variable \textcolor{magenta}{\small \textbf{\texttt{MASKED DESCRIPTION}}}) in \ref{rq:textual_input}.}
    \label{fig:nl_guess_prompt_template}
\end{figure}

\begin{figure*}[h]
    \centering
    \lstinputlisting[language=prompt]{prompt/masked_description_example.txt}
    \caption{An example of our masked description for the concept \texttt{inertia}.}
    \label{fig:masked_description_example}
\end{figure*}

\subsection{Additional Results on the Self-Play Game}
\label{app:self_play}
Automatic evaluation of a text generation task is in general difficult.
Especially, in our scenario each concept have many different ground-truth examples in its description, thus existing automatic metrics such as BLEU~\cite{papineni2002bleu} and METEOR~\cite{banerjee2005meteor} are not capable of accurately measuring the quality. 
Therefore, we propose an alternative automatic metric via a self-play game for this subtask: 

For each generated description of a concept, we mask the synonyms of the concept in it as in the previous selection subtask, and ask the same LLM to identify the concept being described from four options. 
This metric evaluates the quality of LLMs' generated concept descriptions in an objective manner. 

\begin{table}[t!]
    \small
    \centering
    \vspace{-0.1in}
    \begin{tabular}{ccccc}
    \toprule
      & \bf Mistral & \bf Llama-3 & \bf GPT-3.5 & \bf GPT-4 \\
     \midrule
    Human &92.6& 100 &100 & 100\\
    \midrule
    SP & 89.2$_{\pm\text{1.6}}$ & 91.9$_{\pm\text{0.6}}$ &96.0$_{\pm\text{0.4}}$ & 99.8$_{\pm\text{0.2}}$\\    
    \bottomrule
    \end{tabular}
    \vspace{-0.1in}
    \caption{Evaluations on the concept generation subtask, with metrics of Self-Play success and Human evaluation.}
    \vspace{-0.1in}
    \label{tab:generation_extended}
\end{table}

\paragraph{Results}
The results of automatic evaluation via self-play are shown in Table~\ref{tab:generation_extended} together with the human evaluation results. 
In the self-play test, all LLMs can accurately recognize the physical concepts from the descriptions they wrote by themselves.
Combined with the conclusion from human evaluation, 
it shows the LLMs can generate correct and sufficient information.


\section{Details of the Methods Used in \ref{rq:matrix_input} and \ref{rq:visual_input}}

We use the prompt template in Figure~\ref{fig:matrix_prompt_template} for experiments on text-only LLMs (\ref{rq:matrix_input}); and the template in Figure~\ref{fig:visual_prompt_template} for multi-modal LLMs (\ref{rq:visual_input}).

\begin{figure*}[t]
    \centering
    \lstinputlisting[language=prompt]{prompt/matrix_template.txt}
    \caption{The prompt template used in \ref{rq:matrix_input}. The pair of an \textcolor{magenta}{\small \textbf{\texttt{INPUT GRID}}} and an \textcolor{magenta}{\small \textbf{\texttt{OUTPUT GRID}}} consists of one example of a physical phenomenon in matrix format.}
    \label{fig:matrix_prompt_template}
\end{figure*}

\begin{figure*}[t]
    \centering
    \lstinputlisting[language=prompt]{prompt/visual_template.txt}
    \caption{The prompt template used in \ref{rq:visual_input}. \textcolor{magenta}{\small \textbf{\texttt{UPLOADED IMAGE}}} is an image consists of three or more examples like in Figure~\ref{fig:level_examples}.}
    \label{fig:visual_prompt_template}
\end{figure*}


\section{Performance Decomposition in \ref{rq:matrix_input} and \ref{rq:visual_input}}
\label{app:perf_decomp}
Table~\ref{tab:perf_decomp_to_concept} provides a performance decomposition of text-based GPT-4, text-based o1-preview and multi-modal GPT-4o on our \coredatasetns-Test set. Because the rate limit of o1-preview, we conduct experiment on a subset of 50 instances. The result shows that o1-preview does not show superior results compared to the other two LLMs.

\begin{table*}[h!]
    \small
    \centering
    \begin{adjustbox}{width=\linewidth}
    \begin{tabular}{l ccccccccccc}
    \toprule
    \bf Concept       & GPT-4 (t)               & GPT-4o (v)              & o1 (t)& o1 (v)& Gemini2 FTE (v)        & DeepSeek R1 (t) & o3 (t)                   \\
    \midrule
gravity               & 60.0$_{\pm\text{8.2}}$  & 33.3$_{\pm\text{4.7}}$  & 50.0  &  80.0 & 63.3$_{\pm\text{0.3}}$  &  60.0          &  55.0$_{\pm\text{5.0}}$ \\
compression           & 50.0$_{\pm\text{20.4}}$ & 50.0$_{\pm\text{0.0}}$  & 0.0   &  50.0 & 50.0$_{\pm\text{0.0}}$  &   0.0          &   0.0$_{\pm\text{0.0}}$ \\
diffuse reflection of light
                      & 50.0$_{\pm\text{0.0}}$  & 33.3$_{\pm\text{11.8}}$ & 25.0  &  25.0 & 25.0$_{\pm\text{0.0}}$  &  25.0          &  25.0$_{\pm\text{0.0}}$ \\
lever                 & 0.0$_{\pm\text{0.0}}$   & 50.0$_{\pm\text{0.0}}$  & 16.7  &  66.7 & 77.8$_{\pm\text{0.9}}$  &  16.7          &   8.3$_{\pm\text{8.3}}$ \\
wave interference     & 83.3$_{\pm\text{11.8}}$ & 100.0$_{\pm\text{0.0}}$ & 100.0 & 100.0 & 91.7$_{\pm\text{2.1}}$  &  75.0          &  75.0$_{\pm\text{0.0}}$ \\
spectrum of light and optical filters
                      & 66.7$_{\pm\text{0.0}}$  & 88.9$_{\pm\text{15.7}}$ & 66.7  & 100.0 & 94.4$_{\pm\text{0.9}}$  & 100.0          & 100.0$_{\pm\text{0.0}}$ \\
surface tension       & 43.3$_{\pm\text{17.0}}$ & 50.0$_{\pm\text{8.2}}$  & 30.0  &  90.0 & 40.0$_{\pm\text{1.0}}$  &  40.0          &  40.0$_{\pm\text{0.0}}$ \\
nuclear fission       & 16.7$_{\pm\text{23.6}}$ & 100.0$_{\pm\text{0.0}}$ & 100.0 &  50.0 & 0.0$_{\pm\text{0.0}}$   &  50.0          &  50.0$_{\pm\text{0.0}}$ \\
nuclear fusion        & 0.0$_{\pm\text{0.0}}$   & 100.0$_{\pm\text{0.0}}$ & 50.0  &  50.0 & 33.3$_{\pm\text{33.3}}$ &  50.0          &  25.0$_{\pm\text{25.0}}$ \\
communicating vessels & 3.3$_{\pm\text{4.7}}$   & 3.3$_{\pm\text{4.7}}$   & 10.0  &  10.0 & 0.0$_{\pm\text{0.0}}$   &  50.0          &  45.0$_{\pm\text{5.0}}$ \\
diffraction of waves  & 83.3$_{\pm\text{23.6}}$ & 100.0$_{\pm\text{0.0}}$ & --    & 100.0 & 100.0$_{\pm\text{0.0}}$ & 100.0          & 100.0$_{\pm\text{0.0}}$ \\
reflection            & 86.7$_{\pm\text{4.7}}$  & 43.3$_{\pm\text{4.7}}$  & --    &  10.0 & 66.7$_{\pm\text{1.3}}$  &  70.0          &  70.0$_{\pm\text{0.0}}$ \\
refraction            & 20.0$_{\pm\text{8.2}}$  & 83.3$_{\pm\text{4.7}}$  & --    & 100.0 & 50.0$_{\pm\text{4.0}}$  &  40.0          &  50.0$_{\pm\text{10.0}}$ \\
light imaging         & 10.0$_{\pm\text{0.0}}$  & 0.0$_{\pm\text{0.0}}$   & --    &   0.0 & 16.7$_{\pm\text{0.3}}$  &   0.0          &   0.0$_{\pm\text{0.0}}$ \\
cut                   & 90.0$_{\pm\text{0.0}}$  & 73.3$_{\pm\text{4.7}}$  & --    &  60.0 & 93.3$_{\pm\text{0.3}}$  & 100.0          & 100.0$_{\pm\text{0.0}}$ \\
laser                 & 46.7$_{\pm\text{12.5}}$ & 53.3$_{\pm\text{4.7}}$  & --    &  50.0 & 26.7$_{\pm\text{2.3}}$  &  10.0          &  15.0$_{\pm\text{5.0}}$ \\
move                  & 96.7$_{\pm\text{4.7}}$  & 86.7$_{\pm\text{4.7}}$  & --    &  30.0 & 83.3$_{\pm\text{4.3}}$  &  60.0          &  70.0$_{\pm\text{10.0}}$ \\
buoyancy              & 43.3$_{\pm\text{12.5}}$ & 100.0$_{\pm\text{0.0}}$ & --    & 100.0 & 46.7$_{\pm\text{2.3}}$  &  40.0          &  40.0$_{\pm\text{0.0}}$ \\
acceleration          & 10.0$_{\pm\text{8.2}}$  & 73.3$_{\pm\text{12.5}}$ & --    &  20.0 & 46.7$_{\pm\text{0.3}}$  &  40.0          &  30.0$_{\pm\text{10.0}}$ \\
inertia               & 80.0$_{\pm\text{8.2}}$  & 6.7$_{\pm\text{4.7}}$   & --    &  10.0 & 36.7$_{\pm\text{2.3}}$  &  30.0          &  45.0$_{\pm\text{15.0}}$ \\
electricity           & 16.7$_{\pm\text{4.7}}$  & 53.3$_{\pm\text{9.4}}$  & --    &  60.0 & 30.0$_{\pm\text{0.0}}$  &  60.0          &  60.0$_{\pm\text{0.0}}$ \\
reference frame       & 27.8$_{\pm\text{3.9}}$  & 13.9$_{\pm\text{3.9}}$  & --    &  66.7 & 47.2$_{\pm\text{1.6}}$  &  25.0          &  29.1$_{\pm\text{4.1}}$ \\
repulsive force       & 20.8$_{\pm\text{5.9}}$  & 20.8$_{\pm\text{11.8}}$ & --    &  50.0 & 20.8$_{\pm\text{0.5}}$  & 100.0          &  87.5$_{\pm\text{12.5}}$ \\
diffusion             & 8.3$_{\pm\text{11.8}}$  & 100.0$_{\pm\text{0.0}}$ & --    &   0.0 & 83.3$_{\pm\text{2.1}}$  &   0.0          &   0.0$_{\pm\text{0.0}}$ \\
vortex                & 0.0$_{\pm\text{0.0}}$   & 100.0$_{\pm\text{0.0}}$ & --    &  75.0 & 91.7$_{\pm\text{2.1}}$  &   0.0          &  12.5$_{\pm\text{12.5}}$ \\
expansion             & 50.0$_{\pm\text{0.0}}$  & 75.0$_{\pm\text{0.0}}$  & --    &  75.0 & 91.7$_{\pm\text{2.1}}$  &  75.0          &  87.5$_{\pm\text{12.5}}$ \\
wave                  & 16.7$_{\pm\text{15.6}}$ & 33.3$_{\pm\text{5.9}}$  & --    &  62.5 & 25.0$_{\pm\text{0.0}}$  &  25.0          &  18.8$_{\pm\text{6.2}}$ \\
    \bottomrule
    \end{tabular}
    \end{adjustbox}
    \caption{Performance decomposition to concepts on \coredatasetns-Dev. \emph{t} and \emph{v} refer to LLMs with textual or visual inputs.}
    \label{tab:perf_decomp_to_concept}
\end{table*}

\section{Construction of Synthetic Training Data Used in \ref{rq:format_analysis}}
\label{app:synthetic_data}
We investigate whether fine-tuning LLMs on matrix property-related questions could improve their performances on our tasks. Specifically, we generate 3000 extra input-output grid pairs calculate the size, transpose, and locations of the subgrid's corner elements for these matrices as ground truths. Furthermore, since correctly recognizing the location of the subgrid may contribute more to finish the Move and Copy tasks compared to other properties, we create additional ground truths only with the gold locations of the subgrid's corner elements. 



\section{Hyperparameters of Supervised Fine-Tuning in \ref{rq:format_analysis} and \ref{rq:sup_training}}
\label{app:sft_details}

For all the fine-tuning experiments, we use LoRA~\cite{hu2021lora}. We fine-tune each model for 3 epochs with a batch size of 4 on a single machine with 8 A100 GPUs. The dimension of LoRA's attention layer is set to 64, and the $\alpha$ and dropout rates are set to 16 and 0.1, respectively. The learning rate and weight decay are set to 2e-4 and 0.001, respectively.
The hyperparameters are selected according to the development performance on the synthetic matrix data in 
Appendix \ref{app:synthetic_data}.


\end{document}
