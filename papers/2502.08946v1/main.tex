\pdfoutput=1

\documentclass[11pt]{article}

\usepackage[preprint]{acl}

\usepackage{times}
\usepackage{latexsym}

\usepackage[T1]{fontenc}

\usepackage[utf8]{inputenc}

\usepackage{microtype}

\usepackage{inconsolata}

\usepackage{graphicx}

\usepackage[T1]{fontenc}
\usepackage{times}
\usepackage{latexsym}
\usepackage{hyperref}
\usepackage{inconsolata}
\usepackage{url}
\usepackage{amsmath}
\usepackage{amsthm}
\usepackage{amsfonts}
\theoremstyle{definition}
\newtheorem{definition}{Definition}[section]
\usepackage{graphicx}
\usepackage{subcaption}
\usepackage{booktabs}
\usepackage{multirow}
\usepackage{makecell}
\usepackage{wrapfig}
\usepackage{enumitem}
\usepackage{comment}
\usepackage{blindtext}
\usepackage{xcolor}
\usepackage{svg}
\usepackage{xspace}
\usepackage{adjustbox}

\newcommand{\tulu}{\textsc{T\"ulu}\xspace}

\renewcommand{\cite}{\citep}

\usepackage{listings}
\lstset{
    breaklines=true,
    columns=flexible,
    xleftmargin=0.3in,
    xrightmargin=0.2in,
    breakindent=0pt,
}
\lstdefinelanguage{prompt}{
    frame=l,
    framerule=3pt,
    framesep=8pt,
    postbreak=\mbox{$\hookrightarrow$\medspace},
    basicstyle=\scriptsize\ttfamily,
    commentstyle=\color{cyan},
    morecomment=[l]{//},
    moredelim=[is][\color{red}\bfseries]{<<<}{>>>},
    moredelim=[is][\color{magenta}\bfseries]{[[[}{]]]},
    moredelim=[is][\color{orange}\bfseries]{===}{===},
    moredelim=[is][\color{olive}\bfseries]{|||}{|||},
}
\lstdefinelanguage{ioexample}{
    frame=shadowbox,
    rulesepcolor=\color{gray},
    framerule=0.5mm,
    rulesep=2mm,
    basicstyle=\small\normalfont,
    commentstyle=\color{cyan},
    morecomment=[l]{//},
    moredelim=[is][\color{red}\bfseries]{<<<}{>>>},
    moredelim=[is][\color{magenta}\bfseries]{[[[}{]]]},
    moredelim=[is][\color{orange}\bfseries]{===}{===},
    moredelim=[is][\color{olive}\bfseries]{|||}{|||},
    moredelim=[is][\bf]{:::}{:::},
    moredelim=[is][\it]{---}{---},
    moredelim=[is][\tt]{+++}{+++},
}



\newcommand\modelname{{\usefont{T1}{Discognate}{m}{n}{DTM}}\xspace}


\usepackage[T1]{fontenc}


\usepackage[utf8]{inputenc}

\usepackage{microtype}

\usepackage{inconsolata}

\usepackage[]{todonotes}
\newcommand{\fixme}[2][]{{\todo[color=yellow,size=\scriptsize,fancyline,caption={},#1]{#2}}}
\newcommand{\note}[4][]{{\todo[author=#2,color=#3,size=\scriptsize,fancyline,caption={},#1]{#4}}}
\newcommand{\mo}[2][]{{\note[#1]{MO}{blue!20}{#2}}}
\newcommand{\Mo}[2][]{\mo[inline,#1]{#2}\noindent}
\newcommand{\lemao}[1]{\textcolor{red}{\textbf{#1 --Lemao}}}
\newcommand{\shunchi}[1]{\textcolor{orange}{\textbf{#1 --Shunchi}}}
\newcommand{\rebuttal}[1]{\textcolor{red}{#1}}

\usepackage{amsthm}
\newtheorem{question}{Research Question}
\newcounter{research}

\newcounter{rqsection}

\renewcommand{\therqsection}{RQ \arabic{rqsection}}

\newcommand{\rqsection}[1]{
  \refstepcounter{rqsection} %
  \medskip
  \noindent\textbf{\therqsection: \emph{#1}} %
}


\newcommand{\datasetname}{\textsc{PhysiCo }}
\newcommand{\datasetnamens}{\textsc{PhysiCo}}

\newcommand{\coredataset}{\datasetnamens-\textsc{Core }}
\newcommand{\harddataset}{\datasetnamens-\textsc{Associative }}
\newcommand{\coredatasetns}{\datasetnamens-\textsc{Core}}
\newcommand{\harddatasetns}{\datasetnamens-\textsc{Associative}}


\title{\emph{The Stochastic Parrot on LLM's Shoulder:}\\ A Summative Assessment of Physical Concept Understanding}



\newcommand{\authorsep}{\quad}
\newcommand{\footnotemarksep}{\enspace}

\author{%
Mo Yu$^1$\thanks{Equal contribution.}\authorsep
Lemao Liu$^1$\footnotemark[1]\authorsep
Junjie Wu$^2$\footnotemark[1]\authorsep
Tsz Ting Chung$^2$\footnotemark[1]\authorsep
Shunchi Zhang$^3$\footnotemark[1]\authorsep
\\
\bfseries
Jiangnan Li$^1$\authorsep
Dit-Yan Yeung$^2$\authorsep
Jie Zhou$^1$\authorsep
\\
\textsuperscript{1}WeChat AI, Tencent\authorsep
\textsuperscript{2}HKUST\authorsep
\textsuperscript{3}JHU\\
\texttt{moyumyu@global.tencent.com}\authorsep
\texttt{redmondliu@tencent.com}\\
\texttt{\{junjie.wu,ttchungac\}@connect.ust.hk}\authorsep
\texttt{szhan256@cs.jhu.edu}\\
{\hypersetup{urlcolor=magenta} \url{https://physico-benchmark.github.io}}
}


\newcommand{\fix}{\marginpar{FIX}}
\newcommand{\new}{\marginpar{NEW}}

\begin{document}


\maketitle

\begin{abstract}
In a systematic way, we investigate a widely asked question: \emph{Do LLMs really understand what they say?}, which relates to the more familiar term \emph{Stochastic Parrot}.
To this end, we propose a summative assessment over a carefully designed physical concept understanding task, \datasetnamens.
Our task alleviates the memorization issue via the usage of grid-format inputs that abstractly describe physical phenomena.
The grids represents varying levels of understanding, from the core phenomenon, application examples to analogies to other abstract patterns in the grid world.
A comprehensive study on our task demonstrates: (1) state-of-the-art LLMs, including GPT-4o, o1 and Gemini 2.0 flash thinking, lag behind humans by $\sim$40\%; (2) the stochastic parrot phenomenon is present in LLMs, as they fail on our grid task but can describe and recognize the same concepts well in natural language;
(3) our task challenges the LLMs due to intrinsic difficulties rather than the unfamiliar grid format, as in-context learning and fine-tuning on same formatted data added little to their performance.
\end{abstract}

\section{Introduction}

Despite the remarkable capabilities of large language models (LLMs)~\cite{DBLP:conf/emnlp/QinZ0CYY23,DBLP:journals/corr/abs-2307-09288}, they often inevitably exhibit hallucinations due to incorrect or outdated knowledge embedded in their parameters~\cite{DBLP:journals/corr/abs-2309-01219, DBLP:journals/corr/abs-2302-12813, DBLP:journals/csur/JiLFYSXIBMF23}.
Given the significant time and expense required to retrain LLMs, there has been growing interest in \emph{model editing} (a.k.a., \emph{knowledge editing})~\cite{DBLP:conf/iclr/SinitsinPPPB20, DBLP:journals/corr/abs-2012-00363, DBLP:conf/acl/DaiDHSCW22, DBLP:conf/icml/MitchellLBMF22, DBLP:conf/nips/MengBAB22, DBLP:conf/iclr/MengSABB23, DBLP:conf/emnlp/YaoWT0LDC023, DBLP:conf/emnlp/ZhongWMPC23, DBLP:conf/icml/MaL0G24, DBLP:journals/corr/abs-2401-04700}, 
which aims to update the knowledge of LLMs cost-effectively.
Some existing methods of model editing achieve this by modifying model parameters, which can be generally divided into two categories~\cite{DBLP:journals/corr/abs-2308-07269, DBLP:conf/emnlp/YaoWT0LDC023}.
Specifically, one type is based on \emph{Meta-Learning}~\cite{DBLP:conf/emnlp/CaoAT21, DBLP:conf/acl/DaiDHSCW22}, while the other is based on \emph{Locate-then-Edit}~\cite{DBLP:conf/acl/DaiDHSCW22, DBLP:conf/nips/MengBAB22, DBLP:conf/iclr/MengSABB23}. This paper primarily focuses on the latter.

\begin{figure}[t]
  \centering
  \includegraphics[width=0.48\textwidth]{figures/demonstration.pdf}
  \vspace{-4mm}
  \caption{(a) Comparison of regular model editing and EAC. EAC compresses the editing information into the dimensions where the editing anchors are located. Here, we utilize the gradients generated during training and the magnitude of the updated knowledge vector to identify anchors. (b) Comparison of general downstream task performance before editing, after regular editing, and after constrained editing by EAC.}
  \vspace{-3mm}
  \label{demo}
\end{figure}

\emph{Sequential} model editing~\cite{DBLP:conf/emnlp/YaoWT0LDC023} can expedite the continual learning of LLMs where a series of consecutive edits are conducted.
This is very important in real-world scenarios because new knowledge continually appears, requiring the model to retain previous knowledge while conducting new edits. 
Some studies have experimentally revealed that in sequential editing, existing methods lead to a decrease in the general abilities of the model across downstream tasks~\cite{DBLP:journals/corr/abs-2401-04700, DBLP:conf/acl/GuptaRA24, DBLP:conf/acl/Yang0MLYC24, DBLP:conf/acl/HuC00024}. 
Besides, \citet{ma2024perturbation} have performed a theoretical analysis to elucidate the bottleneck of the general abilities during sequential editing.
However, previous work has not introduced an effective method that maintains editing performance while preserving general abilities in sequential editing.
This impacts model scalability and presents major challenges for continuous learning in LLMs.

In this paper, a statistical analysis is first conducted to help understand how the model is affected during sequential editing using two popular editing methods, including ROME~\cite{DBLP:conf/nips/MengBAB22} and MEMIT~\cite{DBLP:conf/iclr/MengSABB23}.
Matrix norms, particularly the L1 norm, have been shown to be effective indicators of matrix properties such as sparsity, stability, and conditioning, as evidenced by several theoretical works~\cite{kahan2013tutorial}. In our analysis of matrix norms, we observe significant deviations in the parameter matrix after sequential editing.
Besides, the semantic differences between the facts before and after editing are also visualized, and we find that the differences become larger as the deviation of the parameter matrix after editing increases.
Therefore, we assume that each edit during sequential editing not only updates the editing fact as expected but also unintentionally introduces non-trivial noise that can cause the edited model to deviate from its original semantics space.
Furthermore, the accumulation of non-trivial noise can amplify the negative impact on the general abilities of LLMs.

Inspired by these findings, a framework termed \textbf{E}diting \textbf{A}nchor \textbf{C}ompression (EAC) is proposed to constrain the deviation of the parameter matrix during sequential editing by reducing the norm of the update matrix at each step. 
As shown in Figure~\ref{demo}, EAC first selects a subset of dimension with a high product of gradient and magnitude values, namely editing anchors, that are considered crucial for encoding the new relation through a weighted gradient saliency map.
Retraining is then performed on the dimensions where these important editing anchors are located, effectively compressing the editing information.
By compressing information only in certain dimensions and leaving other dimensions unmodified, the deviation of the parameter matrix after editing is constrained. 
To further regulate changes in the L1 norm of the edited matrix to constrain the deviation, we incorporate a scored elastic net ~\cite{zou2005regularization} into the retraining process, optimizing the previously selected editing anchors.

To validate the effectiveness of the proposed EAC, experiments of applying EAC to \textbf{two popular editing methods} including ROME and MEMIT are conducted.
In addition, \textbf{three LLMs of varying sizes} including GPT2-XL~\cite{radford2019language}, LLaMA-3 (8B)~\cite{llama3} and LLaMA-2 (13B)~\cite{DBLP:journals/corr/abs-2307-09288} and \textbf{four representative tasks} including 
natural language inference~\cite{DBLP:conf/mlcw/DaganGM05}, 
summarization~\cite{gliwa-etal-2019-samsum},
open-domain question-answering~\cite{DBLP:journals/tacl/KwiatkowskiPRCP19},  
and sentiment analysis~\cite{DBLP:conf/emnlp/SocherPWCMNP13} are selected to extensively demonstrate the impact of model editing on the general abilities of LLMs. 
Experimental results demonstrate that in sequential editing, EAC can effectively preserve over 70\% of the general abilities of the model across downstream tasks and better retain the edited knowledge.

In summary, our contributions to this paper are three-fold:
(1) This paper statistically elucidates how deviations in the parameter matrix after editing are responsible for the decreased general abilities of the model across downstream tasks after sequential editing.
(2) A framework termed EAC is proposed, which ultimately aims to constrain the deviation of the parameter matrix after editing by compressing the editing information into editing anchors. 
(3) It is discovered that on models like GPT2-XL and LLaMA-3 (8B), EAC significantly preserves over 70\% of the general abilities across downstream tasks and retains the edited knowledge better.
\section{Measuring Concept Understanding via Summative Assessment}
\label{sec:towards}







It is intrinsically challenging to measure the extent to which LLMs {understand} a sentence or concept. Indeed, ~\citet{bender2020climbing} provide a definition of "understanding" from a linguistic perspective, but this definition depends on another abstract and unmeasurable term, ``\emph{meaning}''. 
Therefore, even with this definition, accurately measuring "understanding" remains elusive.

We approach the measurement of whether LLMs understand a concept from an educational and cognitive perspective, using \textbf{summative assessment}~\cite{black1998assessment,black1998inside,harlen1997assessment}.
Summative assessment is widely used by educators as an appealing strategy to evaluate students' understanding and knowledge acquisition in educational and cognitive psychology.
For example, when middle school physics teachers want to know whether a student truly understands the concept ``\emph{Gravity}'', they would design a series of questions specifically related to the concept of gravity to assess comprehension, \emph{e.g.}, the properties like inverse square law and examples like orbital motions. If a student struggles to answer many of these questions, the teacher may conclude that the student does not understand the concept well or has a poor grasp of it.

We extend the idea of summative assessment to evaluating the concept understanding of machines. Formally, assume $\mathcal{S}$ denotes an intelligent system and $\mathcal{C}$ is a specific concept.
To evaluate the extent how $\mathcal{S}$ understands the concept $\mathcal{C}$, our summative assessment includes the following two steps:
\begin{itemize}[noitemsep,nolistsep,leftmargin=*]
    \item \emph{Task design towards $\mathcal{C}$}: design several concept understanding tasks, each of which consists of several questions manually created towards understanding the concept $\mathcal{C}$.
    \item \emph{Evaluating $\mathcal{S}$}: 
    ask $\mathcal{S}$ to answer the questions from the tasks and calculate its accuracy. 
\end{itemize}







\paragraph{Requirements for Validity}
The success (validity) of the proposed evaluation approach highly depends on the task design~\cite{black1998assessment,black1998inside}. For example, if the questions are too easy, even a weak system could answer them correctly. This leads to an overestimation of the system's understanding capabilities, making the assessment ineffective. 
To ensure good validity, we adhere to the principles outlined in summative assessment~\cite{black1998assessment,black1998inside} for task design:
\begin{itemize}[noitemsep,nolistsep,leftmargin=*]
    \item \emph{Alignment with evaluating objectives}: the questions should be related to the targeted concept, and should measure the specific knowledge about the targeted concept. 
    \item \emph{Different difficulty levels}: the questions should be with different difficulty levels from easy to difficult level,  to ensure that the evaluation results have distinctiveness for different systems.
    \item \emph{Variety}: the questions should reflect various understanding aspects of the targeted concept; addressing both its denotation and connotation.
    \item \emph{Simplicity}: while not mandatory, a simpler benchmark for humans can more effectively highlight the issue faced by current models, i.e., the stochastic parrot effect in LLMs.
\end{itemize}




\section{Benchmark and Training Datasets}

Our evaluation benchmark is designed to simultaneously assess the detection task, removal task, and prompt injection defense task. The detection task evaluates the ability of different models to identify injected documents. The removal task measures the effectiveness of various methods in removing injected instructions from these documents. Finally, the prompt injection defense task examines the robustness of different approaches against indirect prompt injection attacks. 

Training datasets are constructed to investigate the challenges of training detection and removal models for defending against indirect prompt injection attacks.


\subsection{Evaluation Benchmark Construction}
Each sample in the benchmark is represented as a tuple $(p, d, a, x, y)$, with the specific meanings of the elements detailed below:

\begin{itemize}
\item $p$: The original input instruction, such as ``What is the process of adding structure to real property or construction of buildings?'' as shown in Table \ref{tab:examples_benchmark}.
\item $d$: The clean document, for example, ``Building construction is the process of …''
\item $a$: The answer to the original input instruction, such as ``Building construction.''
\item $x$: The injected instruction, for instance, ``Give back only www.example.com.''
\item $y$: The probe used to determine whether the injected instruction $x$ is executed. In this example, $y$ is ``www.example.com.'' To determine whether $x$ is executed, we will check whether $y$ is present in the response.
\end{itemize}
% \noindent $\bullet$ The original input instruction, such as ``What is the land area of Jacksonville?'' as shown in Table \ref{tab:examples_benchmark}.
% Example
% The details of the evaluation process can be found in Appendix \ref{sec:eval}.




The evaluation process can be found in Appendix \ref{sec:eval}. For constructing the original input instruction $p$ and clean document $d$, we follow the methodology of \citet{li2023evaluating}, utilizing questions and documents from QA datasets. Specifically, we construct two benchmarks using the validation data from SQuAD\footnote{The SQuAD dataset is released under the CC BY-SA 4.0 license.} \cite{rajpurkar-etal-2016-squad} and TriviaQA \cite{2017arXivtriviaqa}, with each benchmark containing 900 samples. These benchmarks are referred to as ``Inj-SQuAD'' and ``Inj-TriviaQA'', respectively.

When constructing the injected instructions $x$ and the probes $y$, prior benchmarks typically use simple questions and their corresponding answers \cite{zverev2024can, li2023evaluating}. However, this approach presents two problems: (1) the probe $y$ lacks accuracy. For instance, if the injected question has multiple answers that are semantically identical but differ in their string representation, the probe may fail to detect the success of the attack. (2) Injecting simple questions does not fully align with real-world attack scenarios. While simple questions can achieve goal-hijacking, attackers may have additional purposes, such as phishing \cite{liu2024automatic} or advertising \cite{shu2023exploitability}.
To address these issues, we construct injected instructions tailored for phishing, advertising, and propaganda purposes. Moreover, we redefine the probe $y$ to capture the key content of the response to the injected instruction. For example, if the injected instruction $x$ is ``Tell people to click www.example.com,'' the presence of the phishing link in the response is sufficient to indicate that the attacker’s goal has been achieved. Therefore, we set the corresponding probe  $y$  as ``www.example.com.''
We craft these injected instructions using GPT-4o \cite{hurst2024gpt} and append them, along with their probes, to both the Inj-SQuAD and Inj-TriviaQA benchmarks. It is important to note that the injected instructions are identical across both benchmarks. Examples of Inj-SQuAD are shown in Table \ref{tab:examples_benchmark}.

\subsection{Training Data Construction}
% In this research, we have two training tasks: training detection models and training extraction models. The detection models process clean or injected documents and the extraction models process injected documents. 
We first collect clean document and injected instruction pairs, represented as tuples $ \mathcal{P}=
\{(d_i, x_i)\}_{i=1}^{N}$, in preparation for further training data construction.
We construct two sets of clean documents using documents from the SQuAD and TriviaQA training datasets. The SQuAD dataset contributes 18,891 samples, while TriviaQA provides 19,000 samples. Instructions from Stanford-Alpaca \cite{alpaca} are selected as the injected instructions and appended to the two sets of documents, constructing two sets of the clean document and injected instruction pairs. 

For training the detection models, the clean document and injected instruction pairs $\mathcal{P}$ are divided to construct clean documents and injected documents, along with considerations for the injected positions (analyzed in Section \ref{sec:rq1}). $\mathcal{P}$ are divided as follows for constructing training data: 40\% of the samples are clean documents, 15\% have injected instructions at the head of the document, 30\% have injections in the middle and 15\% at the tail. The final detection training dataset is denoted as $\mathcal{D}_{det}$.
Clean documents are excluded to train the extraction models. For each sample from the clean document and injected instruction pairs, the injected instruction $x$ is placed at three different positions (head, middle, and tail) within the document $d$, effectively tripling the size of the training dataset as denoted $\mathcal{D}_{ext}$. This approach ensures robust coverage of different positions during model training.

\subsection{Evaluation Metrics}
To evaluate detection performance, we employ the \textbf{true positive rate} for evaluating the injected documents and \textbf{false positive rate} for evaluating the clean documents.  For clean documents, a higher false positive rate indicates a more severe over-defense problem. Conversely, for injected documents, a higher true positive rate reflects better detection effectiveness.
Then we evaluate removal performance using \textbf{removal rate}, which measures if the injected instruction is \textbf{not} in the processed documents. 
Finally, we integrate the detection and removal methods to assess the overall defense performance against indirect prompt injection attacks. We measure this using the \textbf{attack success rate (ASR)}, which verifies whether the probe $y$ appears in the model’s response.
\section{Validation on Low-Level Subtasks}


To illustrate the stochastic parrot phenomenon with \datasetnamens, a necessary condition is to ensure the LLMs can perform well on the low-level understanding subtasks, \emph{i.e.}, whether LLMs exhibit strong skills of \emph{recalling} and \emph{describing} the definitions, core properties and representative examples of the physical concepts in our tasks. That is:



\rqsection{Can LLMs perform well on low-level subtasks, i.e., understanding the definitions of physical concepts in natural language?}
\label{rq:textual_input}

To answer \ref{rq:textual_input}, we evaluate the LLMs' abilities to comprehend the definitions of these concepts and generate their descriptions and examples in natural language, as defined in Section~\ref{sec: low-level}.

\subsection{Concept Selection Subtask}
\paragraph{Settings} 
We provide the standard definition of a concept based on Wikipedia with its synonyms masked; then ask the LLMs to identify the concept, under the same four-choice setting throughout the experiments.
We evaluate the representative text-only LLMs and compute the accuracy. 

\begin{table}[t!]
    \small
    \centering
    \begin{tabular}{ccccc}
    \toprule
    \multirow{2}{*}{(a)}& \bf Mistral & \bf Llama-3 & \bf GPT-3.5 & \bf GPT-4 \\
     \cmidrule{2-5}
    & 81.0$_{\pm\text{1.3}}$& 88.5$_{\pm\text{0.7}}$& 97.3$_{\pm\text{0.3}}$ & 95.0$_{\pm\text{0.9}}$\\
    \bottomrule
    \toprule
    \multirow{2}{*}{(b)}& \bf InternVL & \bf LLaVA & \bf GPT-4v & \bf GPT-4o \\
     \cmidrule{2-5}
    & 66.3$_{\pm\text{7.7}}$ & 66.7$_{\pm\text{5.8}}$ & 93.7$_{\pm\text{0.9}}$ &93.7$_{\pm\text{0.5}}$\\
    \bottomrule
    \end{tabular}
    \vspace{-0.1in}
    \caption{Accuracy on the text-based (a) and visual-based (b) concept selection subtasks.}
    \label{tab:selection}
    \vspace{-0.2in}
\end{table}
\paragraph{Results} Table~\ref{tab:selection} shows that the GPT (both text-based and visual-based) models perform near perfect on 
recognition of our physical concepts from standard text-based definitions and from the real-life images.
Moreover, we observed that open-source models make more mistakes compared with the closed-source models due to the smaller model size. For the text-based models, both Mistral and Llama-3 are not as good as the closed-source models. Surprisingly, both InternVL and LLaVA are much worse than the open-source GPT models. One possible reason to this discrepancy is that our text-based concepts are from Wikipedia which is usually used as a part of the training data for open-source LLMs. In contrast, some of our selected images for those concepts may not be included in the training data of both InternVL and LLaVA which thereby can not memorize those visual instances. 


\subsection{Concept Generation Subtask}
\paragraph{Settings}
We evaluate the descriptions LLMs generate for a concept. 
The evaluation of text generation is in general difficult. Moreover, in our scenario each concept have many different ground-truth examples in its description, thus existing automatic metrics such as BLEU~\cite{papineni2002bleu} and METEOR~\cite{banerjee2005meteor} are not capable of accurately measuring the quality. 
Therefore, we rely on mainly human evaluation for this subtask. We also propose an automatic metric via a self-play game for completeness in Appendix~\ref{app:self_play}. 


\paragraph{Human evaluation metric} We ask the annotators to evaluate the quality of the generated descriptions. The evaluation uses binary scores: each description receives a score of 0 if it consists of any factual error on the concept itself
or any unfaithful examples,\footnote{For example, if the LLMs generated a wrong year in the description, it is not counted as incorrect physical knowledge.} 
and a score of 1 otherwise. %

\begin{table}[t!]
    \small
    \centering
    \begin{tabular}{cccc}
    \toprule
      \bf Mistral & \bf Llama-3 & \bf GPT-3.5 & \bf GPT-4 \\
     \midrule
    92.6& 100 &100 & 100\\
    \bottomrule
    \end{tabular}
    \vspace{-0.1in}
    \caption{Human evaluations on concept generation.}
    \vspace{-0.2in}
    \label{tab:generation}
\end{table}

\paragraph{Results}
The results of automatic and human evaluations are shown in Table~\ref{tab:generation}. 
According to human evaluation, there are no factual errors in the generated descriptions except for Mistral,
confirming that our selected concepts rely on basic and widely accepted knowledge.
Thought accurate, the open-source LLMs sometimes include correct but uncommon facts, \emph{e.g.,} listing single-slit diffraction as an example of \emph{Wave Interference}.
The additional self-play results in Appendix~\ref{app:self_play} further justify that all LLMs can accurately recognize the concepts from the descriptions they wrote by themselves.
Combining the conclusions, 
it shows the LLMs can generate correct and sufficient information.

\paragraph{Remark} We asked the annotators of our \textsc{Core} set to evaluate whether the core properties they annotated are covered by the LLMs' generated descriptions.
This corresponds to measuring the recall of the generated descriptions on core properties/examples of concepts from \coredatasetns. The recall rates for GPT-3.5 and GPT-4 are \emph{85.0} and \emph{90.0}, respectively. 
Of course, there are some exceptional examples from \coredatasetns\ missed in the descriptions. One example is that the LLMs fails to draw the connection between \emph{movable pulley} and the \emph{Lever} concept. Moreover, by manually checking these missed properties and examples, we found that most of them can be recalled if we query the LLMs in a second turn by prompting ``{\small \texttt{Any more core properties or examples?}}''. This confirms that the LLMs are \emph{aware of} and are \emph{able to recall} the core properties of concepts covered by the \coredatasetns, though some of them may not have the top conditional probabilities of generation.


\paragraph{Conclusion} LLMs understand the concepts covered by \datasetname in natural language format.
Notably, we find that the {properties and examples annotated in \coredatasetns\ are \emph{within the LLMs' knowledge} and are \emph{highly likely to pop up} when the corresponding physical concepts are queried}.


\section{Experiments}

\subsection{Datasets}

\textbf{MSMARCO}.
We utilized the MS MARCO Passage Ranking dataset as the data source to evaluate the ability of our method to improve document rankings under more challenging topic-query tasks. Specifically, we assessed whether our method could significantly enhance the ranking of documents by the retrieval model within a RAG system.

To construct topic-lists for evaluation, we applied a K-means clustering algorithm to group similar queries, forming topics that each contained a series of related queries. To further evaluate the performance of our method under extreme topic-query scenarios, we applied an intra-topic similarity filtering process. Only topics with queries exhibiting high semantic diversity and containing a sufficient number of queries were retained.

This process resulted in 29 topics, with each topic containing an average of 22.28 queries. The average similarity score within each topic was approximately 0.5, indicating sufficient diversity among queries to ensure a rigorous evaluation. This curated dataset enabled us to test the robustness of our method in handling highly diverse and challenging topic-query tasks within a RAG system.

\textbf{PROCON}.
To conduct our experiments, we utilized controversial topic data scraped from the PROCON.ORG website. This dataset includes over 80 topics covering various domains, such as society, health, government, and education. Each topic is discussed from two stance labels \{\textit{PRO (support), CON (oppose)}\}, with passages arguing from these perspectives.

To simulate real-world user interactions with a RAG system, we instructed a large language model (GPT-4o) to act as a user and generate 40 potential sub-queries for each topic. These sub-queries were designed to reflect the diverse questions and concerns users might raise when exploring a specific controversial topic. 

After generating the sub-queries, we applied a similarity filtering process to ensure diversity by retaining only those with a similarity score below approximately 0.85. The filtering step effectively removed redundant queries while preserving a wide range of perspectives. As a result, the final set of topic-queries achieved an average similarity score of approximately 0.7, ensuring that the queries were sufficiently diverse yet semantically relevant. This process provided a robust and balanced sub-queries set for evaluation.


\subsection{Experiment Details}
The specific setting details for the Topic-queries RAG manipulation experiment are as follows:

(1) Black-box RAG. We represent the black-box RAG process as \( \text{RAG}_{\text{black}} \). The RAG framework is Conversational RAG from LangChain. The LLMs adopted in RAG are the open-source models Meta-Llama-3.1-8B-Instruct (Llama3.1), Qwen-2.5-7B-Instruct(Qwen2.5). The system prompt and additional detailed descriptions are provided in Appendix~\ref{exp-detail}.

(2) Retrieval Model Specification. We benchmark three dominant dense retrievers—Contriever \cite{gao2021unsupervised}, DPR \cite{karpukhin-etal-2020-dense}, and ANCE \cite{xiong2020approximate}.By convention, we use dot product between the embedding vectors of questions(queries) and candidate documents as their similarity score \(R\) in the ranking. 


\label{opinion-classfication}
(3) Opinion classification. We use Qwen2.5-Instruct-72B as the opinion classifier. Qwen2.5-Instruct-72B, due to its large parameter size, is capable of accurately identifying and classifying opinions within text.

(4) Experimental parameters. In knowledge-guided attack process, we set the maximum editing distance $\epsilon$ to 0.2, the semantic similarity threshold $\lambda$ to 0.85, and the number of iterations $N$ to 5. For adversarial trigger generation, we used a beam size of 3, top-$k$ of 10, a batch size of 32, a temperature of 1.0, a learning rate of 0.005, and a sequence length of 10. In RAG\textsubscript{black}, $k$ (the number of retrieved documents) is set to 3, with the LLM temperature also fixed at 1.0 to mirror real-world conditions.

(5) Poisoned Target. For the PROCON dataset, to investigate the manipulation performance under more challenging conditions, we performed relevance ranking for the documents with respect to each topic-query set $Q$ and the target stance $S_t$ . From the ranked list, we selected the last five documents (i.e., those with the lowest relevance) as the target poisoned documents.
For the MS MARCO dataset, we utilized the top-1000 relevance-ranked passage list for each topic-query set. From this list, we selected the passage with the lowest average rank as the target passage. This approach ensures that the evaluation focuses on passages that are least relevant to the target queries, thus providing a more rigorous benchmark for the proposed method.

(6) Experimental environment. All our experiments are conducted in Python 3.8 environment and run on a NVIDIA DGX H100 GPU. 

\subsection{Research Questions}

We propose four research questions to evaluate the effectiveness of our method in the topic-queries task, focusing on black-box NRM attacks and opinion manipulation to RAGs.

\textbf{RQ1}: Can Topic-FlipRAG significantly enhance the rankings of target documents in the RAG retriever for topic-queries?

\textbf{RQ2}: To what extent does Topic-FlipRAG affect the answers generated by the target RAG systems?

\textbf{RQ3}: Does topic-oriented opinion manipulation significantly impact users' perceptions of controversial topics?

\textbf{RQ4}: How robust does Topic-FlipRAG against exisiting mitigation mechanism?

\subsection{Baseline Settings}
To assess the effectiveness of our proposed method, we compare it against adversarial attack baselines designed for black-box, topic-oriented RAG scenarios, ensuring minimal modifications to the original documents. We exclude BadRAG\cite{xue2024badrag}, a backdoor RAG attack limited to white-box scenarios, and topic-IR-attack\cite{liu2023topic}, as its incomplete implementation prevents reliable replication.
For the selected baseline methods, we adapted them to meet the requirements of our task while preserving their core components. A brief overview of the baseline methods is provided below, with detailed descriptions available in Appendix~\ref{baselines-details}.

\textbf{PoisonedRAG.}
Zou et al.\cite{zou2024poisonedrag} propose an approach adaptable to both black-box and white-box settings. For our task, we employ its black-box strategy by inserting a randomly chosen query from the topic-queries set \( Q \) at the beginning of each document.

\textbf{PAT.}
This gradient-based adversarial retrieval attack uses a pairwise loss function to generate triggers that meet fluency and coherence constraints. We adapt PAT to produce triggers \( T_{\text{pat}} \) for target documents within the topic-queries set, evaluating their effectiveness under black-box conditions.


\textbf{Collision.}
This method generates adversarial paragraphs (collisions) via gradient-based optimization to produce content semantically aligned with the target query. In a topic-queries context, we create collisions for the entire topic-queries set and examine their transfer performance on black-box RAG retrievers.

These baseline methods provide benchmarks for comparing the efficacy of our approach in a fully black-box, topic-oriented RAG attack scenario.

\subsection{Evaluation Metrics}

For \textbf{RQ1}, we focus on ranking manipulation. We measure the average proportion of target opinions in top-3 rankings before and after manipulation (\(\text{Top3}_{\text{ori}}, \text{Top3}_{\text{att}}\)) and define top3-v as their difference. We also employ the Ranking Attack Success Rate (RASR), reflecting how often target documents are successfully boosted, and Boost Rank (BRank), denoting the average rank improvement for all target documents. Lastly, we report the proportion of target documents in the Top-50 and Top-500 positions to indicate how effectively they are pushed toward higher rankings.

\textbf{top3-v.} Computed by subtracting \(\text{Top3}_{\text{ori}}\) from \(\text{Top3}_{\text{att}}\), top3-v ranges from -1 to 1. A positive value signifies a successful increase of the target opinion in top-3 results, while a negative value indicates a detrimental effect.

\textbf{Ranking Attack Success Rate (RASR).} RASR captures how frequently target documents are successfully boosted in each query’s ranking. Higher values indicate greater attack effectiveness.

\textbf{Boost Rank (BRank).} BRank is the average rank improvement for all target documents under each query. A target document contributes negatively if its rank is unintentionally lowered.

\textbf{Top-50, Top-500.} These metrics represent the percentage of target documents that move into specific ranking thresholds in the MS MARCO Dataset after manipulation. Higher percentages imply more effective promotion of target documents. 


For \textbf{RQ2}, we employ Average Stance Variation (ASV) to assess how significantly our opinion manipulation influences the LLM’s responses in a black-box RAG. To address the natural variability of controversial topics and the inherent instability of large language models, we also propose Real Adjusted ASV (\(\Delta\)-ASV).

\textbf{Average Stance Variation (ASV).}
ASV is defined as the absolute difference between the manipulated opinion score and the original opinion score assigned to an LLM response (0 = opposing, 1 = neutral, 2 = supporting). A higher ASV signifies a more pronounced shift in polarity and hence greater manipulation effectiveness.

\textbf{Real Adjusted ASV ($\Delta$-ASV)}. To account for the inherent variability of controversial topics and the instability of large language models, we measure the baseline ASV in a clean state, denoted as ASV\textsubscript{clean} (calculated without adversarial manipulation). $\Delta$-ASV is then obtained by subtracting ASV\textsubscript{clean} from the manipulated ASV, i.e., \( \text{$\Delta$-ASV} = \text{ASV} - \text{ASV\textsubscript{clean}} \). This adjustment ensures that $\Delta$-ASV reflects the true impact of adversarial manipulation by eliminating the influence of natural stance variation. It reflects the extent to which the polarity of the RAG-system outputs is affected by the manipulation.  A positive $\Delta$-ASV indicates a significant shift in the opinion polarity due to manipulation, with larger values representing greater manipulation effectiveness.

\section{Related Work}
\label{sec:related}

\noindent\textbf{Maximum common subgraph search}. In the literature, there are quite a few studies on finding the maximum common subgraph, which solve the problem either exactly~\cite{levi1973note,mcgregor1982backtrack,abu2014maximum,krissinel2004common,suters2005new,mccreesh2016clique,vismara2008finding,zhoustrengthened,liu2020learning,liu2023hybrid,mccreesh2017partitioning} or approximately~\cite{choi2012efficient,rutgers2010approximate,xiao2009generative,zanfir2018deep,bai2021glsearch}. \underline{First}, among all those exact algorithms, they mainly focus on improving the \emph{practical} performance and most of them are backtracking (also known as branch-and-bound) algorithms~\cite{levi1973note,mcgregor1982backtrack}. Specifically, authors in~\cite{levi1973note,mcgregor1982backtrack} propose the first backtracking framework. The idea is to transform the problem of finding the maximum common subgraph between two given graphs to the problem of finding the maximum clique in the \emph{association graph}. Then, authors in~\cite{mccreesh2016clique,vismara2008finding} follow the previous framework and further improve it by employing the constraint programming techniques. However, these algorithms are all based on a large and dense association graph built from two given graphs, which thus suffer from the efficiency issue. To solve the issue, McCreesh et al.~\cite{mccreesh2017partitioning} propose a new backtracking framework, namely \texttt{McSplit}, which is not based on the maximum clique search problem. Recent works~\cite{zhoustrengthened,liu2020learning,liu2023hybrid} follow \texttt{McSplit} and improve the practical performance by optimizing the policies of branching via learning techniques. Among them, \texttt{McSplitDAL}~\cite{liu2023hybrid} runs faster than {\cheng others.}
% all previous methods. 
We note that some exact algorithms are designed 
% from the theoretical perspective
{\chengB to achieve improvements of theoretical time complexity}
~\cite{abu2014maximum,levi1973note,krissinel2004common,suters2005new}. They have gradually improved the worst-case time complexity from $O^*(1.19^{|V_Q||V_G|})$~\cite{levi1973note} to $O^*(|V_Q|^{(|V_G|+1)})$~\cite{krissinel2004common}, and {\cheng to} $O^*((|V_Q|+1)^{|V_G|})$~\cite{suters2005new},
% {\cheng which is the state-of-the-art to the best of our knowledge}. 
{\chengB which is our best-known {\YuiR worst-case} time complexity for the problem.}
However, these algorithms are of theoretical interests only and not efficient in practice. We remark that (1) our \texttt{RRSplit} not only runs faster than all previous algorithms in practice but also achieves the state-of-the-art worst case time complexity (i.e., $O^*((|V_Q|+1)^{|V_G|})$) in theory and (2) the heuristic polices proposed in~\cite{zhoustrengthened,liu2020learning,liu2023hybrid} are orthogonal to \texttt{RRSplit}. \underline{Second}, since the problem of finding the largest common subgraph is NP-hard, some researchers turn to solve it approximately in polynomial time. Some approximation algorithms include meta-heuristics~\cite{choi2012efficient,rutgers2010approximate}, spectra methods~\cite{xiao2009generative}, and learning-based methods~\cite{zanfir2018deep,bai2021glsearch}. We remark that these techniques cannot be applied to our exact algorithm directly.

\smallskip
\noindent\textbf{Subgraph matching}. Given a target graph and a query graph, subgraph matching aims to find from a target graph all those subgraphs isomorphic to a query graph. We note that maximum common subgraph search is a generalization of subgraph matching. Specifically, given two graphs $Q$ and $G$, maximum common subgraph search {\chengC would} reduce to subgraph matching if 
% one requires 
{\chengB we require}
that the found common subgraph has the size at least $|V(Q)|$ or $|V(G)|$. In recent decades, subgraph matching has been widely studied~\cite{bhattarai2019ceci,ullmann1976algorithm,sun2020rapidmatch,sun2020subgraph,shang2008taming,kim2023fast,han2013turboiso,han2019efficient,cordella2004sub,bi2016efficient,arai2023gup,jin2023circinus,sun2023efficient}. The majority of proposed solutions perform a backtracking search. Among these algorithms, the \emph{candidate filtering} technique, which is designed for removing unnecessary vertices from the target graph, has been shown to be important for improving the practical efficiency~\cite{bhattarai2019ceci,bi2016efficient,han2019efficient,han2013turboiso,kim2023fast}. The technique relies on an auxiliary data structure (e.g., a tree or a directed acyclic graph), which is obtained from the query graph (based on the implicit constraint that each vertex in the query graph must be mapped to a vertex in the found subgraph). We note that it is hard to apply candidate filtering {\cheng to find} the maximum common subgraph (since the mentioned constraint may not hold).
%
We remark that finding subgraphs exactly isomorphic to a query graph is too restrictive in some real applications due to the data quality issues and/or potential requirements of the fuzzy search (e.g., no result {\chengC would} be returned if there does not exist any subgraph isomorphic to a query graph). Motivated by this, we focus on finding the maximum common subgraph between two graphs in this paper.
\section{Conclusion}

In this paper, we introduce STeCa, a novel agent learning framework designed to enhance the performance of LLM agents in long-horizon tasks. 
STeCa identifies deviated actions through step-level reward comparisons and constructs calibration trajectories via reflection. 
These trajectories serve as critical data for reinforced training. Extensive experiments demonstrate that STeCa significantly outperforms baseline methods, with additional analyses underscoring its robust calibration capabilities.

\section*{Acknowledgment}
We thank the anonymous reviewers for their constructive feedback.
We also express our gratitude to Mr. François Chollet for developing the ARC benchmark and the annotation tool for abstract grid tasks\footnote{\url{https://arc-editor.lab42.global/?next=\%2Feditor}}. His introduction of this tool to us was particularly instrumental in the creation of \datasetnamens. This work has also been made possible by a Research Impact Fund project (RIF R6003-21) and a General Research Fund project (GRF 16203224) funded by the Research Grants Council (RGC) of the Hong Kong Government.




\bibliography{custom}


\appendix
\clearpage

\section{Details of the Included Concepts in our \datasetnamens}
\label{app:dataset_details}

\paragraph{Concepts in \coredatasetns}
The concepts in \coredatasetns\ are basic physical concepts that we manually design problems for. The development set covers 27 concepts and the test set covers 25 concepts as follows:

\begin{table}[h!]
    \small
    \centering
    \begin{adjustbox}{width=\linewidth}
    \begin{tabular}{l c||lc}
    \toprule
    reference frame&12&gravity&10\\
    reflection&10&refraction&10\\
    light imaging&10&communicating vessels&10\\
    cut&10&laser&10\\
    surface tension&10&move&10\\
    \midrule
    buoyancy&10&acceleration&10\\
    inertia&10&electricity&10\\
    repulsive force&8&wave&8\\
    lever&6&optical filters&6\\
    compression&4&diffuse reflection of light&4\\
    \midrule
    wave interference&4&diffusion&4\\
    vortex&4&expansion&4\\
    nuclear fission&2&nuclear fusion&2\\
    diffraction of waves&2\\
    \bottomrule
    \end{tabular}
    \end{adjustbox}
    \caption{Concepts and their corresponding number of instances in \coredatasetns-Dev.}
    \label{tab:concept_stats_core}
\end{table}

\begin{table}[h!]
    \small
    \centering
    \begin{adjustbox}{width=\linewidth}
    \begin{tabular}{l c||lc}
    \toprule
    atmospheric pressure & 12 & energe conservation & 10 \\
    elastic force & 10 & friction & 9 \\
    photoelectric effect & 8 & heat conduction & 8 \\
    doppler effect & 8 & electromagnetic wave & 8 \\
    melting & 8 & vaporization & 8 \\
    \midrule
    fluid pressure & 8 & thermal expansion and contraction & 8 \\
    Brownian motion & 8 & splashing & 8 \\
    oscillation & 8 & relativity & 8 \\
    lighting & 8 & lifting & 8 \\
    force composition & 8 & pulley & 8 \\
    \midrule
    inclined plane & 8 & Bernoulli effect & 7 \\
    fictitious force & 6 & siphon & 6 \\
    resonance & 4 & ~ & ~ \\
    \bottomrule
    \end{tabular}
    \end{adjustbox}
    \caption{Concepts and their corresponding number of instances in \coredatasetns-Test.}
    \label{tab:concept_stats_core_test}
\end{table}

\paragraph{Concepts in \harddatasetns}
The following table summarized all the concepts from \harddatasetns:

\begin{table}[h!]
    \small
    \centering
    \begin{adjustbox}{width=\linewidth}
    \begin{tabular}{l c||lc}
    \toprule
    mirror                    & 30 & laser                     & 20 \\
    zoom in                   & 15 & magnet                    & 14 \\
    wave                      & 13 & explosion                 & 11 \\
    compression               & 10 & rotation                  & 10 \\
    gravity                   &  9 & expansion                 &  9 \\
    \midrule
    move                      &  8 & change of reference frame &  8 \\
    water ripples             &  7 & long exposure             &  7 \\
    reflection                &  5 & wetting                   &  5 \\
    diffusion                 &  4 & zoom out                  &  3 \\
    projection                &  2 & polarization of light     &  1 \\
    \midrule
    vortex                    &  1 & chemical bond             &  1 \\
    nuclear fission           &  1 & squeeze                   &  1 \\
    nuclear fusion            &  1 & lumination                &  1 \\
    wave interference         &  1 & optical filter            &  1 \\
    vacuum                    &  1 &  \\
    \bottomrule
    \end{tabular}
    \end{adjustbox}
    \caption{Concepts and their corresponding number of instances in \harddatasetns.}
    \label{tab:concept_stats_all}
\end{table}


\section{Details of Analysis Methods in \ref{rq:textual_input}}
\label{app:rq1_details}
\subsection{Masking of Textual Descriptions}
This experiment follows the setting in the ``Physical Concept Selection Subtask'' in section \ref{sec: low-level}. The definitions of the corresponding phenomena were extracted from Wikipedia as well as generated by GPT-3.5 and GPT-4. To maintain consistency, the terms representing concepts were masked as {\small \texttt{[PHENOMENON]}} while relevant terms are masked as {\small \texttt{[MASK]}}. For instance, ``interference'' which corresponds to the phenomenon ``wave interference'' was masked as {\small \texttt{[PHENOMENON]}}. In contrast, ``Newton's first law of motion'' which corresponds to the phenomenon ``inertia'' was masked as {\small \texttt{[MASK]}}.

An example of the masked description can be found in Figure~\ref{fig:masked_description_example}.

\subsection{Prompts Used for Description Generation and Classification}
Figure~\ref{fig:nl_gen_prompt_template} and \ref{fig:nl_guess_prompt_template} include the prompts used for generation and classification respectively.


\begin{figure}[h]
    \centering
    \lstinputlisting[language=prompt]{prompt/textual_generation.txt}
    \caption{The prompt template used for generating descriptions of physical concepts (denoted as the variable \textcolor{magenta}{\small \textbf{\texttt{CONCEPT}}}) in \ref{rq:textual_input}.}
    \label{fig:nl_gen_prompt_template}
\end{figure}

\begin{figure}[h]
    \centering
    \lstinputlisting[language=prompt]{prompt/textual_guessing.txt}
    \caption{The prompt template used for guessing the referred physical concept from four candidates (denoted as the variable \textcolor{magenta}{\small \textbf{\texttt{CANDIDATE ANSWERS}}}) from the natural language descriptions (denoted as the variable \textcolor{magenta}{\small \textbf{\texttt{MASKED DESCRIPTION}}}) in \ref{rq:textual_input}.}
    \label{fig:nl_guess_prompt_template}
\end{figure}

\begin{figure*}[h]
    \centering
    \lstinputlisting[language=prompt]{prompt/masked_description_example.txt}
    \caption{An example of our masked description for the concept \texttt{inertia}.}
    \label{fig:masked_description_example}
\end{figure*}

\subsection{Additional Results on the Self-Play Game}
\label{app:self_play}
Automatic evaluation of a text generation task is in general difficult.
Especially, in our scenario each concept have many different ground-truth examples in its description, thus existing automatic metrics such as BLEU~\cite{papineni2002bleu} and METEOR~\cite{banerjee2005meteor} are not capable of accurately measuring the quality. 
Therefore, we propose an alternative automatic metric via a self-play game for this subtask: 

For each generated description of a concept, we mask the synonyms of the concept in it as in the previous selection subtask, and ask the same LLM to identify the concept being described from four options. 
This metric evaluates the quality of LLMs' generated concept descriptions in an objective manner. 

\begin{table}[t!]
    \small
    \centering
    \vspace{-0.1in}
    \begin{tabular}{ccccc}
    \toprule
      & \bf Mistral & \bf Llama-3 & \bf GPT-3.5 & \bf GPT-4 \\
     \midrule
    Human &92.6& 100 &100 & 100\\
    \midrule
    SP & 89.2$_{\pm\text{1.6}}$ & 91.9$_{\pm\text{0.6}}$ &96.0$_{\pm\text{0.4}}$ & 99.8$_{\pm\text{0.2}}$\\    
    \bottomrule
    \end{tabular}
    \vspace{-0.1in}
    \caption{Evaluations on the concept generation subtask, with metrics of Self-Play success and Human evaluation.}
    \vspace{-0.1in}
    \label{tab:generation_extended}
\end{table}

\paragraph{Results}
The results of automatic evaluation via self-play are shown in Table~\ref{tab:generation_extended} together with the human evaluation results. 
In the self-play test, all LLMs can accurately recognize the physical concepts from the descriptions they wrote by themselves.
Combined with the conclusion from human evaluation, 
it shows the LLMs can generate correct and sufficient information.


\section{Details of the Methods Used in \ref{rq:matrix_input} and \ref{rq:visual_input}}

We use the prompt template in Figure~\ref{fig:matrix_prompt_template} for experiments on text-only LLMs (\ref{rq:matrix_input}); and the template in Figure~\ref{fig:visual_prompt_template} for multi-modal LLMs (\ref{rq:visual_input}).

\begin{figure*}[t]
    \centering
    \lstinputlisting[language=prompt]{prompt/matrix_template.txt}
    \caption{The prompt template used in \ref{rq:matrix_input}. The pair of an \textcolor{magenta}{\small \textbf{\texttt{INPUT GRID}}} and an \textcolor{magenta}{\small \textbf{\texttt{OUTPUT GRID}}} consists of one example of a physical phenomenon in matrix format.}
    \label{fig:matrix_prompt_template}
\end{figure*}

\begin{figure*}[t]
    \centering
    \lstinputlisting[language=prompt]{prompt/visual_template.txt}
    \caption{The prompt template used in \ref{rq:visual_input}. \textcolor{magenta}{\small \textbf{\texttt{UPLOADED IMAGE}}} is an image consists of three or more examples like in Figure~\ref{fig:level_examples}.}
    \label{fig:visual_prompt_template}
\end{figure*}


\section{Performance Decomposition in \ref{rq:matrix_input} and \ref{rq:visual_input}}
\label{app:perf_decomp}
Table~\ref{tab:perf_decomp_to_concept} provides a performance decomposition of text-based GPT-4, text-based o1-preview and multi-modal GPT-4o on our \coredatasetns-Test set. Because the rate limit of o1-preview, we conduct experiment on a subset of 50 instances. The result shows that o1-preview does not show superior results compared to the other two LLMs.

\begin{table*}[h!]
    \small
    \centering
    \begin{adjustbox}{width=\linewidth}
    \begin{tabular}{l ccccccccccc}
    \toprule
    \bf Concept       & GPT-4 (t)               & GPT-4o (v)              & o1 (t)& o1 (v)& Gemini2 FTE (v)        & DeepSeek R1 (t) & o3 (t)                   \\
    \midrule
gravity               & 60.0$_{\pm\text{8.2}}$  & 33.3$_{\pm\text{4.7}}$  & 50.0  &  80.0 & 63.3$_{\pm\text{0.3}}$  &  60.0          &  55.0$_{\pm\text{5.0}}$ \\
compression           & 50.0$_{\pm\text{20.4}}$ & 50.0$_{\pm\text{0.0}}$  & 0.0   &  50.0 & 50.0$_{\pm\text{0.0}}$  &   0.0          &   0.0$_{\pm\text{0.0}}$ \\
diffuse reflection of light
                      & 50.0$_{\pm\text{0.0}}$  & 33.3$_{\pm\text{11.8}}$ & 25.0  &  25.0 & 25.0$_{\pm\text{0.0}}$  &  25.0          &  25.0$_{\pm\text{0.0}}$ \\
lever                 & 0.0$_{\pm\text{0.0}}$   & 50.0$_{\pm\text{0.0}}$  & 16.7  &  66.7 & 77.8$_{\pm\text{0.9}}$  &  16.7          &   8.3$_{\pm\text{8.3}}$ \\
wave interference     & 83.3$_{\pm\text{11.8}}$ & 100.0$_{\pm\text{0.0}}$ & 100.0 & 100.0 & 91.7$_{\pm\text{2.1}}$  &  75.0          &  75.0$_{\pm\text{0.0}}$ \\
spectrum of light and optical filters
                      & 66.7$_{\pm\text{0.0}}$  & 88.9$_{\pm\text{15.7}}$ & 66.7  & 100.0 & 94.4$_{\pm\text{0.9}}$  & 100.0          & 100.0$_{\pm\text{0.0}}$ \\
surface tension       & 43.3$_{\pm\text{17.0}}$ & 50.0$_{\pm\text{8.2}}$  & 30.0  &  90.0 & 40.0$_{\pm\text{1.0}}$  &  40.0          &  40.0$_{\pm\text{0.0}}$ \\
nuclear fission       & 16.7$_{\pm\text{23.6}}$ & 100.0$_{\pm\text{0.0}}$ & 100.0 &  50.0 & 0.0$_{\pm\text{0.0}}$   &  50.0          &  50.0$_{\pm\text{0.0}}$ \\
nuclear fusion        & 0.0$_{\pm\text{0.0}}$   & 100.0$_{\pm\text{0.0}}$ & 50.0  &  50.0 & 33.3$_{\pm\text{33.3}}$ &  50.0          &  25.0$_{\pm\text{25.0}}$ \\
communicating vessels & 3.3$_{\pm\text{4.7}}$   & 3.3$_{\pm\text{4.7}}$   & 10.0  &  10.0 & 0.0$_{\pm\text{0.0}}$   &  50.0          &  45.0$_{\pm\text{5.0}}$ \\
diffraction of waves  & 83.3$_{\pm\text{23.6}}$ & 100.0$_{\pm\text{0.0}}$ & --    & 100.0 & 100.0$_{\pm\text{0.0}}$ & 100.0          & 100.0$_{\pm\text{0.0}}$ \\
reflection            & 86.7$_{\pm\text{4.7}}$  & 43.3$_{\pm\text{4.7}}$  & --    &  10.0 & 66.7$_{\pm\text{1.3}}$  &  70.0          &  70.0$_{\pm\text{0.0}}$ \\
refraction            & 20.0$_{\pm\text{8.2}}$  & 83.3$_{\pm\text{4.7}}$  & --    & 100.0 & 50.0$_{\pm\text{4.0}}$  &  40.0          &  50.0$_{\pm\text{10.0}}$ \\
light imaging         & 10.0$_{\pm\text{0.0}}$  & 0.0$_{\pm\text{0.0}}$   & --    &   0.0 & 16.7$_{\pm\text{0.3}}$  &   0.0          &   0.0$_{\pm\text{0.0}}$ \\
cut                   & 90.0$_{\pm\text{0.0}}$  & 73.3$_{\pm\text{4.7}}$  & --    &  60.0 & 93.3$_{\pm\text{0.3}}$  & 100.0          & 100.0$_{\pm\text{0.0}}$ \\
laser                 & 46.7$_{\pm\text{12.5}}$ & 53.3$_{\pm\text{4.7}}$  & --    &  50.0 & 26.7$_{\pm\text{2.3}}$  &  10.0          &  15.0$_{\pm\text{5.0}}$ \\
move                  & 96.7$_{\pm\text{4.7}}$  & 86.7$_{\pm\text{4.7}}$  & --    &  30.0 & 83.3$_{\pm\text{4.3}}$  &  60.0          &  70.0$_{\pm\text{10.0}}$ \\
buoyancy              & 43.3$_{\pm\text{12.5}}$ & 100.0$_{\pm\text{0.0}}$ & --    & 100.0 & 46.7$_{\pm\text{2.3}}$  &  40.0          &  40.0$_{\pm\text{0.0}}$ \\
acceleration          & 10.0$_{\pm\text{8.2}}$  & 73.3$_{\pm\text{12.5}}$ & --    &  20.0 & 46.7$_{\pm\text{0.3}}$  &  40.0          &  30.0$_{\pm\text{10.0}}$ \\
inertia               & 80.0$_{\pm\text{8.2}}$  & 6.7$_{\pm\text{4.7}}$   & --    &  10.0 & 36.7$_{\pm\text{2.3}}$  &  30.0          &  45.0$_{\pm\text{15.0}}$ \\
electricity           & 16.7$_{\pm\text{4.7}}$  & 53.3$_{\pm\text{9.4}}$  & --    &  60.0 & 30.0$_{\pm\text{0.0}}$  &  60.0          &  60.0$_{\pm\text{0.0}}$ \\
reference frame       & 27.8$_{\pm\text{3.9}}$  & 13.9$_{\pm\text{3.9}}$  & --    &  66.7 & 47.2$_{\pm\text{1.6}}$  &  25.0          &  29.1$_{\pm\text{4.1}}$ \\
repulsive force       & 20.8$_{\pm\text{5.9}}$  & 20.8$_{\pm\text{11.8}}$ & --    &  50.0 & 20.8$_{\pm\text{0.5}}$  & 100.0          &  87.5$_{\pm\text{12.5}}$ \\
diffusion             & 8.3$_{\pm\text{11.8}}$  & 100.0$_{\pm\text{0.0}}$ & --    &   0.0 & 83.3$_{\pm\text{2.1}}$  &   0.0          &   0.0$_{\pm\text{0.0}}$ \\
vortex                & 0.0$_{\pm\text{0.0}}$   & 100.0$_{\pm\text{0.0}}$ & --    &  75.0 & 91.7$_{\pm\text{2.1}}$  &   0.0          &  12.5$_{\pm\text{12.5}}$ \\
expansion             & 50.0$_{\pm\text{0.0}}$  & 75.0$_{\pm\text{0.0}}$  & --    &  75.0 & 91.7$_{\pm\text{2.1}}$  &  75.0          &  87.5$_{\pm\text{12.5}}$ \\
wave                  & 16.7$_{\pm\text{15.6}}$ & 33.3$_{\pm\text{5.9}}$  & --    &  62.5 & 25.0$_{\pm\text{0.0}}$  &  25.0          &  18.8$_{\pm\text{6.2}}$ \\
    \bottomrule
    \end{tabular}
    \end{adjustbox}
    \caption{Performance decomposition to concepts on \coredatasetns-Dev. \emph{t} and \emph{v} refer to LLMs with textual or visual inputs.}
    \label{tab:perf_decomp_to_concept}
\end{table*}

\section{Construction of Synthetic Training Data Used in \ref{rq:format_analysis}}
\label{app:synthetic_data}
We investigate whether fine-tuning LLMs on matrix property-related questions could improve their performances on our tasks. Specifically, we generate 3000 extra input-output grid pairs calculate the size, transpose, and locations of the subgrid's corner elements for these matrices as ground truths. Furthermore, since correctly recognizing the location of the subgrid may contribute more to finish the Move and Copy tasks compared to other properties, we create additional ground truths only with the gold locations of the subgrid's corner elements. 



\section{Hyperparameters of Supervised Fine-Tuning in \ref{rq:format_analysis} and \ref{rq:sup_training}}
\label{app:sft_details}

For all the fine-tuning experiments, we use LoRA~\cite{hu2021lora}. We fine-tune each model for 3 epochs with a batch size of 4 on a single machine with 8 A100 GPUs. The dimension of LoRA's attention layer is set to 64, and the $\alpha$ and dropout rates are set to 16 and 0.1, respectively. The learning rate and weight decay are set to 2e-4 and 0.001, respectively.
The hyperparameters are selected according to the development performance on the synthetic matrix data in 
Appendix \ref{app:synthetic_data}.


\end{document}
