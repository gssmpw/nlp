

\makenoidxglossaries

\newglossaryentry{watermarking}
{
    name={watermarking},
    description={The practice of concealing information into digital media to protect copyright, verify authenticity, or convey other metadata}
}

\newglossaryentry{steganography}
{
    name={steganography},
    description={The practice of concealing information into digital media in such a way that the presence of the information is hidden}
}

\newglossaryentry{zero-bit watermarking}
{
    name={zero-bit watermarking},
    description={A type of watermarking that does not embed any binary message, but alters the cover medium in a way that can be detected}
}

\newglossaryentry{multi-bit watermarking}
{
    name={multi-bit watermarking},
    description={A type of watermarking that embeds a binary message into the cover medium}
}

\newglossaryentry{cover}
{
    name={cover},
    description={The original media content where a watermark is embedded}
}

\newglossaryentry{robustness}
{
    name={robustness},
    description={The ability of a watermarking scheme (or any other system) to withstand various transformations of the cover medium}
}

\newglossaryentry{payload}
{
    name={payload},
    description={The amount of information that can be embedded into a cover medium, usually measured in bits}
}

\newglossaryentry{security}
{
    name={security},
    description={The ability of a watermarking scheme to prevent unauthorized parties from intentionally detecting, removing, forging or altering the watermark}
}

\newglossaryentry{fingerprinting}
{
    name={fingerprinting},
    description={A technique used to uniquely identify digital content by generating a unique identifier or hash based on the content's features}
}

\newglossaryentry{copy detection}
{
    name={copy detection},
    description={The process of identifying duplicate or near-duplicate content in a large dataset}
}

\newglossaryentry{hash}
{
    name={hash},
    description={(or fingerprint) A fixed-size vector representation of data}
}

\newglossaryentry{indexing}
{
    name={indexing},
    description={The process of organizing and storing data in a way that makes it easy to search and retrieve}
}

\newglossaryentry{FAISS}
{
    name={FAISS},
    description={(Facebook AI Similarity Search) A library for efficient similarity search and clustering of dense vectors}
}

\newglossaryentry{digital forensics}
{
    name={digital forensics},
    description={The process of analyzing and collecting digital evidence to determine the authenticity and provenance of synthetic or manipulated content}
}

\newglossaryentry{C2PA}
{
    name={C2PA},
    description={(Coalition for Content Provenance and Authenticity) An initiative that aims to address the prevalence of misleading information online by developing technical standards for certifying the source and history of media content}
}

\newglossaryentry{RSA}
{
    name={RSA},
    description={A public-key cryptosystem that is widely used for secure data transmission. Stands for Rivest-Shamir-Adleman, the inventors of the technique}
}

\newglossaryentry{GAN}
{
    name={GAN},
    description={(Generative Adversarial Network) A class of machine learning frameworks designed by opposing two neural networks against each other to generate new, synthetic instances of data that can pass for real data}
}

\newglossaryentry{diffusion model}
{
    name={diffusion model},
    description={A type of generative model that generates data by starting from a random noise and gradually reducing the noise through a learned iterative process}
}

\newglossaryentry{LDM}
{
    name={LDM},
    description={(Latent Diffusion Model) A diffusion model that operates in the latent space of a variational autoencoder. Stable Diffusion is a LDM}
}

\newglossaryentry{LLM}
{
    name={LLM},
    description={(Large Language Model) A type of deep learning model designed to understand and generate human language. These models are trained on vast amounts of text data and can perform a variety of language-based tasks. Examples include GPT-3 and BERT}
}

\newglossaryentry{token}
{
    name={token},
    description={A basic unit of a language model's input, usually a word or subword}
}

\newglossaryentry{TTS}
{
    name={TTS},
    description={(Text-To-Speech) A type of technology that converts written text into spoken voice output}
}

\newglossaryentry{p-value}
{
    name={p-value},
    description={The probability of obtaining a test statistic at least as extreme as the one that was actually observed, assuming that the null hypothesis is true. Used to determine the statistical significance of the watermark detection}
}

\newglossaryentry{test statistic}
{
    name={test statistic},
    description={The quantity upon which the statistical test relies, often referred to as watermark score in this thesis}
}


\newglossaryentry{FPR}
{
    name={FPR},
    description={(False Positive Rate) The proportion of non-watermarked instances that are incorrectly classified as watermarked}
}

\newglossaryentry{TPR}
{
    name={TPR},
    description={(True Positive Rate) The proportion of watermarked instances that are correctly classified as watermarked}
}

\newglossaryentry{ROC}
{
    name={ROC},
    description={(Receiver Operating Characteristic), or ROC curve, a graphical plot of the TPR against the FPR, used to evaluate the performance of the watermark detection. The area under the ROC curve (AUC) assesses the overall performance with a single value}
}

\newglossaryentry{HiDDeN}
{
    name={HiDDeN},
    description={A deep learning-based method for embedding and extracting invisible watermarks in images using an encoder-decoder network architecture}
}

\newglossaryentry{white-box}
{
    name={white-box},
    description={A type of attack where the attacker has full access to the model's architecture and parameters}
}

\newglossaryentry{black-box}
{
    name={black-box},
    description={A type of attack where the attacker has no access to the model's architecture or parameters}
}
