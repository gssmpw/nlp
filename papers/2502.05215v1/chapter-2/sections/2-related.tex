\section{Related work}\label{chap2/sec:related}

\paragraph*{Image watermarking} is an alternative technology for ICD. 
Our method bridges indexing and watermarking, where the image is modified before publication.
Regarding retrieval performance, active indexing is more robust than watermarking.
Indeed, the embedded signal reinforces the structure naturally present in the original image, whereas watermarking has to hide a large secret keyed signal independent of the original feature.
\autoref{chap2/sec:watermarking} provides a more thorough discussion and experimental results comparing indexing and watermarking, while \autoref{chap0/sec:watermarking} provides broader related works on image watermarking.

\paragraph*{Active \gls*{fingerprinting}} is more related to our work.
As far as we know, this concept was invented by \citet{voloshynovskiy2012active}.
They consider that the image $I\in \R^N$ is mapped to $x \in \R^N$ by an invertible transform $W$ such that $WW^\top$.
The binary fingerprint is obtained by taking the sign of the projections of $x$ against a set of vectors $b_1,., b_L \in \R^N$ (à la LSH).
Then, they change $x$ to strengthen the amplitude of these projections so that their signs become more robust to noise.
They recover $I^\acti$ with $W^\top$.
This scheme is applied to image patches in~\citep{7533094} where the performance is measured as a bit error rate after JPEG compression. 
This chapter adapts this idea from fingerprinting to indexing, with modern deep learning representations and state-of-the-art indexing techniques. 
The range of transformations is also much broader and includes geometric transforms.
