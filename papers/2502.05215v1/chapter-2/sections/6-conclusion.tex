

\section{Conclusion}

We introduce a way to improve ICD in large-scale settings, when images can be changed before release.
This limits the application but is not unrealistic, for instance this would be the case for generative AI providers, that control what is output by the APIs, or for big social platforms that could provide modified images after they have been uploaded for the first time.
It leverages an optimization scheme, similar to adversarial examples, that modifies images so that (1) their representations are better suited for indexing, (2) the perturbation is invisible to the human eye.
We provide grounded analyses on the method and show that it significantly improves retrieval performance of activated images, on a number of neural extractors and indexing structures.

The method has several limitations. 
(1) Activating images takes time (in the order of 10~ms/image) but one advantage is that the database may contain both active and passive images: active indexing does not spoil the performance of passive indexing and vice-versa. This is good for legacy compliance and also opens the door to flexible digital asset management strategies (actively indexing images of particular importance).
(2) The method is not agnostic to the indexing structure and extractor that are used by the similarity search.
(3) an adversary could break the indexing system in several ways.
In a black-box setting (no knowledge of the indexing structure and neural network extractor), adversarial purification~\citep{shi2021online} could get rid of the perturbation that activated the image.
In a semi-white-box setting (knowledge of the feature extractor), targeted mismatch attacks against passive indexing like~\citep{tolias2019targeted} may also work. 
Adversarial training~\citep{madry2017towards} could be a defense.
For instance, it is interesting to know if adversarial training prevents active indexing, or if the perceptual perturbation that is used in our method is still able to push features in the latent space of a robust and defended neural network. 



