
\chapter{Active Image Indexing}
\label{chapter:active-indexing}

This chapter is based on the paper \fullcite{fernandez2022active}.


Image copy detection and retrieval from large databases are particularly relevant for content moderation. 
They allow to automatically detect and remove an important fraction of harmful content.
This enables human moderators to focus their efforts on new instances of problematic material rather than repeatedly reviewing variations of already-flagged one.
This chapter improves the robustness of image copy detection with \emph{active indexing}, which draws inspiration from watermarking and the previous chapter to alter the image in a way that improves its retrieval.
Retrieval systems often leverage two components. 
First, a neural network maps an image to a vector representation, that is relatively robust to various transformations of the image. 
Second, an efficient but approximate similarity search algorithm trades scalability (size and speed) against quality of the search, thereby introducing a source of error. 
\emph{Active indexing} optimizes the interplay of these two components. 
We reduce the quantization loss of a given image representation by making imperceptible changes to the image before its release. 
The loss is back-propagated through the deep neural network back to the image, under perceptual constraints. These modifications make the image more retrievable. 
Our experiments show that the retrieval and copy detection of activated images is significantly improved. For instance, activation improves by $+40\%$ the Recall1@1 on various image transformations, and for several popular indexing structures based on product quantization and locality sensitivity hashing.
Code is available at \url{github.com/facebookresearch/active_indexing}.

\newpage
\section{Introduction}
\label{sec:intro}

Foundational models (FMs)~\cite{zhang2024data, zhou2023comprehensive} have shown remarkable progress in the healthcare domain, enabling professional-like assessment of disease diagnosis, treatment decision-making, and monitoring~\cite{zhang2023text, wang2022medclip, lu2023mi-zero}. 
Examples include LLaVA-Med~\cite{li2023llava}, Med-PaLM Multimodal~\cite{tu2024towards}, and Med-Flamingo~\cite{moor2023med}, have demonstrated their capacity on question answering, medical image analysis, and report generation.
These studies follow a predominant top-down model development strategy that requires upstream developers to collect data and train models for downstream tasks. 
Consequently, the developed model capabilities are heavily dependent on the training data, limiting their generalization performance in diverse clinical scenarios. 
For instance, Med-Gemini~\cite{yang2024advancing} reveals promising general capabilities in report generation while it lags behind state-of-the-art (SoTA) models on classification tasks, especially for out-of-domain applications. 
This indicates that while the generalizability of the foundation model is promising, more solutions are expected to meet the various specialized clinical needs.

To address these challenges, multi-center data centralization becomes essential to enhance model capacity and robustness across varied clinical scenarios~\cite{rajpurkar2022ai}. 
Centralizing distributed data can significantly improve model training and inference performance.
However, the process of medical data storage, transfer, and aggregation among centers requires extra efforts to ensure data security and system interoperability~\cite{bradford2020international}.
Moreover, a growing concern for patient privacy makes large-scale multi-center data sharing particularly challenging. 
While efforts like federated learning~\cite{wen2023survey, li2020review} can achieve good model performance on local data, the need for synchronized system coordination presents significant challenges, as clients are unable to update asynchronously. This limitation greatly restricts the practical capability of such approaches.
As a result, without a flexible collaboration, medical community still struggles to fully utilize the isolated data and local computation resources for comprehensive medical AI model development. 
To address this dilemma, open-source platforms encourage public data sharing and knowledge integration~\cite{markiewicz2021openneuro, zenodo}.
However, these platforms focus solely on raw data sharing while seldom providing collaborative model training or cooperation between different institutions.
Recently, collaborative learning has emerged as a viable approach for enhancing multi-model robustness~\cite{boulemtafes2020review}. 
For instance, software-like model development~\cite{raffel2023building} mimics software engineering practices by introducing structured workflows, enabling merging, version control, and continuous model integration.
Under this design, model ability can be strengthened with incremental knowledge updates similar to the version updating in software development. 

Although collaborative learning provides a multi-model collaboration, two key challenges remain in the leakage of raw data during collaboration~\cite{huang2023lorahub} and the synchronization of multiple collaborators~\cite{mcmahan2017communication} in the medical AI community. It is still challenging to integrate decentralized, privacy-sensitive data across institutions, leading to under-utilized insights and fragmented knowledge sharing~\cite{kaissis2020secure, rajpurkar2022ai, abdullah2021ethics}.
 To address these challenges, inspired by the collaborative software development, we propose \textbf{Med}ical \textbf{Fo}undation Models Me\textbf{rg}ing (\textbf{MedForge}), a cooperative workflow enabling continuously community-driven foundation model (FM) development.
MedForge enables a lightweight manner for individual centers to share their knowledge among multiple centers, minimizing the burden of data transmission and integration while enhancing model robustness.
Meanwhile, MedForge facilitates asynchronous and flexible collaboration, allowing individual centers to continuously update and improve medical FMs without the need for real-time synchronization.
Similar to open-source software development, MedForge incrementally updates medical knowledge and follows a sustainable model development scheme. 
This key design emphasizes a bottom-up construction of a multi-task medical FM, allowing downstream users to collaboratively build, refine, and update the upstream model according to their local resources. Our major contributions of MedForge are as below: 
\begin{enumerate}
    \item[$\bullet$] We introduce a collaborative workflow to promote the merging scheme of open-source software development. Our proposed MedForge allows distributed clinical centers to asynchronously contribute to comprehensive medical model construction while reducing transmitting costs among centers and avoiding the leakage of raw data, thus enhancing the utilization of private resources in the healthcare system. 
    \item[$\bullet$] We propose two effective knowledge-merging strategies for the asynchronous branch contribution. The MedForge-Fusion strategy updates the plugin module parameters of the main model during the merging phase, whereas the MedForge-Mixture strategy integrates the output of the plugin module by memorizing each contributor's coefficient. These strategies make MedForge more flexible and versatile. MedForge-Fusion is friendly to implement, while the MedForge-Mixture offers better performance and robustness.
    \item[$\bullet$]  We comprehensively evaluate model merging strategies to accumulate medical knowledge among multiple branch plugin modules. MedForge yields superior performance on medical classification tasks compared to other collaborative baselines across multiple datasets. We demonstrate the robustness of MedForge by shuffling the task order and evaluating various configurations of plugin modules and dataset distillation methods.
\end{enumerate}



\section{Related Work}
\label{sec:related}
\subsection{Collaborative Systems}
In the era of rapid growth in medical foundational models~\cite{huang2023visual,wang2022medclip, zhang2024data}, the top-down model development paradigm limits model capabilities by heavily relying on the resources available to the model builders. 
Such paradigm often restricts the potential of these models, as they cannot effectively utilize the diverse, private, and decentralized resources that exist within the broader medical community.
In contrast, collaborative systems present a promising alternative, offering a more flexible approach to model development.

Collaborative systems enable institutions to share knowledge by allowing distributed collaborators to contribute to a common goal~\cite{boulemtafes2020review}. 
To further protect patient privacy, federated learning (FL)~\cite{mcmahan2017communication} was proposed to alleviate such privacy concerns as server aggregating parameter updates from multiple clients without sharing their local data. 
While subsequent optimizations, such as aggregation algorithms~\cite{mcmahan2017communication, zhao2018federated, li2020federated}, secure learning~\cite{hardy2017private, xie2021crfl}, fairness improvements~\cite{sharma2022federated, zhao2022dynamic} and its application in medicine~\cite{kumar2024privacy}, have enhanced the capacity and applicability of FL, its real-world flexibility remains limited. This is primarily due to the need for synchronous updates, which require the server and clients to stay in sync, or model updates will be blocked.
This synchrony issue can be mitigated by open-source software platforms (e.g., GitHub~\cite{github}), allowing independent contributions from individual developers asynchronously. Such an asynchronous scheme enables faster iteration and the integration of specialized expertise, thus offering a more flexible and scalable approach.

Unlike synchronous collaboration, asynchronous collaboration does not require collaborators to work simultaneously and collaborators can individually complete their updates.
While the concept of asynchronous collaboration has been widely used in software development, its machine-learning applications remain limited~\cite{kandpal2023git, raffel2023building}. 
With the rise of global collaboration, large models~\cite{sahajBERT, le2023bloom} are usually co-developed by collaborators given various levels of data availability. However, this collaborative scheme requires the aggregation of local data and online synchronous cooperation of developers.
Software-like model update system~\cite{raffel2023building} alleviates the synchronous problem, where models are updated incrementally, similar to software development, by introducing merging and version control to model development.
However, the existing collaborative version control system~\cite{kandpal2023git} fails to address the complexities of medical scenarios because of the heavy dependency on plain parameter averaging across the full model without accounting for the varying requirements of different tasks.
To respond, we propose MedForge, which enables an asynchronous collaborative system and ensures strong robustness toward a continuous, community-driven enhancement of medical models while overcoming potential data leakage.

\begin{figure*}[t]
\begin{center}
\includegraphics[width=.85\linewidth]{fig_overview_v3.pdf}
\end{center}
\caption{
FastAtlas Overview: In each frame, we compute charts spanning fully or partially visible triangles (a), determine texture space bounding boxes for the visible portions of the view-space projections of each chart, and tightly pack these boxes into atlases (b, here $2K \times 2K$). We simultaneously bijectively parameterize and shade the charts into their atlas boxes, obtaining high quality texture space shading (c), and use this shading to render the shaded frames (d).}
\label{fig:overview}
\label{fig:alg_overview}
\end{figure*}

\section{Overview}
\label{sec:overview}
Our work has two core contributions: a real-time, GPU-based algorithm for tight packing of general parameterized charts into compact atlases; and a real-time TSS method that
utilizes this packing.  

\paragraph*{FastAtlas Packing.}
FastAtlas runs entirely on the GPU as a series of compute shaders. It takes the bounding boxes of parameterized charts as input, and packs them into an atlas (Fig~\ref{fig:overview}b, Sec.~\ref{sec:pack}). As such, the only input it requires are the dimensions of the bounding boxes.
Its outputs are deterministic; identical input charts are packed into identical atlases. This is critical for TSS and similar applications, as it ensures that consecutive frames taken from the same camera view have the same shading. Even minute shading differences across such frames can cause sampling jitter, leading to undesirable flicker \cite{baker2012rock}. 
While prior methods such as \cite{mueller2018shading,hladky2019tessellated,hladky2021snakebinning,Neff2022MSA} cap the dimensions of the charts that can be packed as-is for a given atlas size, and scale down all charts that exceed these dimensions, we scale all charts by the same factor, and do so only when strictly necessary to achieve packing success (Figs~\ref{fig:atlas},~\ref{fig:sas_issues}). 

\paragraph*{TSS using FastAtlas.}
Our end-to-end TSS atlas generation method combines the packing method above with a novel approach for computing seamless per-frame charts. 
We define our charts as the connected components of the visible surfaces in each frame (Fig.~\ref{fig:overview}a), and efficiently compute them using a parallel union-find algorithm (Sec.~\ref{sec:visible}). Since the boundaries of these charts coincide with the contours of the rendered surface, they are {\em invisible} to the viewer. This approach 
eliminates the artifacts caused by shading discontinuities along visible seams (Fig.~\ref{fig:seams}). 

\begin{parWithWrapFigure}
\begin{wrapfigure}{l}{.27\columnwidth}%
\includegraphics[width=\linewidth]{fig_inset_view_plane.pdf}%
\end{wrapfigure}
We bijectively parametrize the {\em visible portions} of our charts by projecting them to view space (inset). This maps a constant number of texels to each pixel in the final rendered output, evenly distributing residual undersampling error across all image pixels. While conceptually straightforward, efficiently parameterizing charts containing partially visible triangles using viewspace projection is non-trivial, as the visible portions may no longer be triangular (e.g. green triangle in the inset); applying naive projection to triangles with vertices behind the camera may produce ill-posed results. Clipping triangles before projection is both computationally expensive and significantly complicates downstream operations. We avoid explicit clipping by observing that all that is required for atlas packing is the dimensions of, potentially conservative, bounding boxes of these projected visible portions. We compute such bounding boxes without explicit chart clipping by adapting a conservative screen coverage estimator \shortcite{Blinn:CalculatingScreenCoverage} (Sec.~\ref{sec:box}). We then pack the computed boxes using FastAtlas. 
\end{parWithWrapFigure}

Finally, we shade the visible portion of each chart into its corresponding atlas bounding box (Fig~\ref{fig:overview}c). 
The resulting texture is then used during rasterization as a standard texture map (Fig. ~\ref{fig:overview}d). 
Our framework is compatible with all existing approaches for texture space shading, including forward shading, raytraced illumination, or deferred shading in texture space \cite{baker:2016}. In the examples shown, we use the standard forward shading based rendering pipeline included in the G3D Innovation Engine \cite{G3D17}, a commercial grade renderer.


\subsection{Model Merging}
In collaborative systems, proper model merging becomes increasingly vital for improving model knowledge integration from multiple sources in a resource-limited environment~\cite{li2023deep, yang2024model, goddard2024arcee}. Conceptually, model merging strategies can be categorized into entire model merging and partial model merging.

Entire model merging involves combining multiple model parameters to participate in the merging process by several means. Entire model merging can be viewed as an optimization problem~\cite{Matena_Raffel_2021, jin2022dataless, mavromatis2024packllm} or an alignment problem~\cite{ainsworth2022git, jordan2022repair, xu2024training, ainsworth2022git}, each offering unique advantages depending on the task at hand.
In the optimization-based approach, the goal is to find the best combination of multiple models to enhance performance and efficiency. For instance, using Fisher information approximation~\cite{Matena_Raffel_2021}, the optimization-based model merging can be interpreted as selecting parameters that maximize the joint likelihood of the models' posterior distributions. The optimization of model merging can also be guided by minimizing the prediction differences between the merged model and individual models~\cite{jin2022dataless}. 
With the development of large language models (LLM), optimization-based method is used to fuse multiple LLMs at test-time by minimizing perplexity over the input prompt~\cite{mavromatis2024packllm}.
To highlight, optimization-based methods are beneficial for scenarios requiring enhanced model performance and efficiency to integrate model parameters, while alignment-based methods~\cite{ainsworth2022git, jordan2022repair} are better suited for maintaining consistency and interpretability, facilitating critical information sharing across models.
For example, a training-free model merging strategy aligns relevant models by using a similarity matrix of their representations in both activation and weight spaces~\cite{xu2024training}.
Further, the alignment between the independently trained model and a reference model not only works for models with the same architecture but also for arbitrary model architectures~\cite{ainsworth2022git}.
In summary, the entire model merging methods can effectively integrate existing models into a merged model with enhanced functionality. However, they could lead to increased computational complexity and reduced flexibility, making them less scalable and harder to implement across diverse tasks.

Partial model merging refers to combining only specific components or layers of models to improve model merging efficiency and decrease the computational cost. 
Such specific components can come from the same network~\cite{kingetsu2021neural}, where the original network is divided into subnetworks for different purposes, and these subnetworks can then be recombined for new tasks.
Additionally, modules can originate from different functional networks and be merged using various strategies. For instance, arithmetic operations are applied in \cite{zhang2023composing} to fuse parameter-efficient modules.
While merging modules from different networks provides flexibility, the process also requires a selection strategy to ensure the resulting model aligns with the specific needs of the inference stage. 
The selection strategies are commonly designed based on the similarity of task~\cite{lv2023parameter} and domain clustering performance~\cite{chronopoulou2023adaptersoup}. Alternatively, the mixture-of-experts methods use a routing strategy to select appropriate component modules~\cite{ponti2023combining}. However, these strategies often require significant time and computational resources to filter through a large model pool. 
In contrast, LoRAHub~\cite{huang2023lorahub} offers a more lightweight approach, combining various LoRA modules for different tasks with minimal model training. Nevertheless, LoRAHub lacks flexibility for incremental updates, especially when handling unseen tasks.

Although the existing model merging approaches effectively combine the capabilities of individual models, these approaches often rely on raw data, leading to potential privacy risks. Our proposed MedForge emphasizes the prevention of raw data usage, which is particularly crucial in medical scenarios. Additionally, MedForge offers an extensible capability for incremental learning, enabling continuous model improvement.



\vspace{-5pt}
\section{Method}
\label{sec:method}
\begin{figure*}[t]
\begin{center}
\includegraphics[width=.85\linewidth]{fig_overview_v3.pdf}
\end{center}
\caption{
FastAtlas Overview: In each frame, we compute charts spanning fully or partially visible triangles (a), determine texture space bounding boxes for the visible portions of the view-space projections of each chart, and tightly pack these boxes into atlases (b, here $2K \times 2K$). We simultaneously bijectively parameterize and shade the charts into their atlas boxes, obtaining high quality texture space shading (c), and use this shading to render the shaded frames (d).}
\label{fig:overview}
\label{fig:alg_overview}
\end{figure*}

\section{Overview}
\label{sec:overview}
Our work has two core contributions: a real-time, GPU-based algorithm for tight packing of general parameterized charts into compact atlases; and a real-time TSS method that
utilizes this packing.  

\paragraph*{FastAtlas Packing.}
FastAtlas runs entirely on the GPU as a series of compute shaders. It takes the bounding boxes of parameterized charts as input, and packs them into an atlas (Fig~\ref{fig:overview}b, Sec.~\ref{sec:pack}). As such, the only input it requires are the dimensions of the bounding boxes.
Its outputs are deterministic; identical input charts are packed into identical atlases. This is critical for TSS and similar applications, as it ensures that consecutive frames taken from the same camera view have the same shading. Even minute shading differences across such frames can cause sampling jitter, leading to undesirable flicker \cite{baker2012rock}. 
While prior methods such as \cite{mueller2018shading,hladky2019tessellated,hladky2021snakebinning,Neff2022MSA} cap the dimensions of the charts that can be packed as-is for a given atlas size, and scale down all charts that exceed these dimensions, we scale all charts by the same factor, and do so only when strictly necessary to achieve packing success (Figs~\ref{fig:atlas},~\ref{fig:sas_issues}). 

\paragraph*{TSS using FastAtlas.}
Our end-to-end TSS atlas generation method combines the packing method above with a novel approach for computing seamless per-frame charts. 
We define our charts as the connected components of the visible surfaces in each frame (Fig.~\ref{fig:overview}a), and efficiently compute them using a parallel union-find algorithm (Sec.~\ref{sec:visible}). Since the boundaries of these charts coincide with the contours of the rendered surface, they are {\em invisible} to the viewer. This approach 
eliminates the artifacts caused by shading discontinuities along visible seams (Fig.~\ref{fig:seams}). 

\begin{parWithWrapFigure}
\begin{wrapfigure}{l}{.27\columnwidth}%
\includegraphics[width=\linewidth]{fig_inset_view_plane.pdf}%
\end{wrapfigure}
We bijectively parametrize the {\em visible portions} of our charts by projecting them to view space (inset). This maps a constant number of texels to each pixel in the final rendered output, evenly distributing residual undersampling error across all image pixels. While conceptually straightforward, efficiently parameterizing charts containing partially visible triangles using viewspace projection is non-trivial, as the visible portions may no longer be triangular (e.g. green triangle in the inset); applying naive projection to triangles with vertices behind the camera may produce ill-posed results. Clipping triangles before projection is both computationally expensive and significantly complicates downstream operations. We avoid explicit clipping by observing that all that is required for atlas packing is the dimensions of, potentially conservative, bounding boxes of these projected visible portions. We compute such bounding boxes without explicit chart clipping by adapting a conservative screen coverage estimator \shortcite{Blinn:CalculatingScreenCoverage} (Sec.~\ref{sec:box}). We then pack the computed boxes using FastAtlas. 
\end{parWithWrapFigure}

Finally, we shade the visible portion of each chart into its corresponding atlas bounding box (Fig~\ref{fig:overview}c). 
The resulting texture is then used during rasterization as a standard texture map (Fig. ~\ref{fig:overview}d). 
Our framework is compatible with all existing approaches for texture space shading, including forward shading, raytraced illumination, or deferred shading in texture space \cite{baker:2016}. In the examples shown, we use the standard forward shading based rendering pipeline included in the G3D Innovation Engine \cite{G3D17}, a commercial grade renderer.


Our goal is to increase the robustness of T2I models, particularly with rare or unseen concepts, which they struggle to generate. To do so, we investigate a retrieval-augmented generation approach, through which we dynamically select images that can provide the model with missing visual cues. Importantly, we focus on models that were not trained for RAG, and show that existing image conditioning tools can be leveraged to support RAG post-hoc.
As depicted in \cref{fig:overview}, given a text prompt and a T2I generative model, we start by generating an image with the given prompt. Then, we query a VLM with the image, and ask it to decide if the image matches the prompt. If it does not, we aim to retrieve images representing the concepts that are missing from the image, and provide them as additional context to the model to guide it toward better alignment with the prompt.
In the following sections, we describe our method by answering key questions:
(1) How do we know which images to retrieve? 
(2) How can we retrieve the required images? 
and (3) How can we use the retrieved images for unknown concept generation?
By answering these questions, we achieve our goal of generating new concepts that the model struggles to generate on its own.

\vspace{-3pt}
\subsection{Which images to retrieve?}
The amount of images we can pass to a model is limited, hence we need to decide which images to pass as references to guide the generation of a base model. As T2I models are already capable of generating many concepts successfully, an efficient strategy would be passing only concepts they struggle to generate as references, and not all the concepts in a prompt.
To find the challenging concepts,
we utilize a VLM and apply a step-by-step method, as depicted in the bottom part of \cref{fig:overview}. First, we generate an initial image with a T2I model. Then, we provide the VLM with the initial prompt and image, and ask it if they match. If not, we ask the VLM to identify missing concepts and
focus on content and style, since these are easy to convey through visual cues.
As demonstrated in \cref{tab:ablations}, empirical experiments show that image retrieval from detailed image captions yields better results than retrieval from brief, generic concept descriptions.
Therefore, after identifying the missing concepts, we ask the VLM to suggest detailed image captions for images that describe each of the concepts. 

\vspace{-4pt}
\subsubsection{Error Handling}
\label{subsec:err_hand}

The VLM may sometimes fail to identify the missing concepts in an image, and will respond that it is ``unable to respond''. In these rare cases, we allow up to 3 query repetitions, while increasing the query temperature in each repetition. Increasing the temperature allows for more diverse responses by encouraging the model to sample less probable words.
In most cases, using our suggested step-by-step method yields better results than retrieving images directly from the given prompt (see 
\cref{subsec:ablations}).
However, if the VLM still fails to identify the missing concepts after multiple attempts, we fall back to retrieving images directly from the prompt, as it usually means the VLM does not know what is the meaning of the prompt.

The used prompts can be found in \cref{app:prompts}.
Next, we turn to retrieve images based on the acquired image captions.

\vspace{-3pt}
\subsection{How to retrieve the required images?}

Given $n$ image captions, our goal is to retrieve the images that are most similar to these captions from a dataset. 
To retrieve images matching a given image caption, we compare the caption to all the images in the dataset using a text-image similarity metric and retrieve the top $k$ most similar images.
Text-to-image retrieval is an active research field~\cite{radford2021learning, zhai2023sigmoid, ray2024cola, vendrowinquire}, where no single method is perfect.
Retrieval is especially hard when the dataset does not contain an exact match to the query \cite{biswas2024efficient} or when the task is fine-grained retrieval, that depends on subtle details~\cite{wei2022fine}.
Hence, a common retrieval workflow is to first retrieve image candidates using pre-computed embeddings, and then re-rank the retrieved candidates using a different, often more expensive but accurate, method \cite{vendrowinquire}.
Following this workflow, we experimented with cosine similarity over different embeddings, and with multiple re-ranking methods of reference candidates.
Although re-ranking sometimes yields better results compared to simply using cosine similarity between CLIP~\cite{radford2021learning} embeddings, the difference was not significant in most of our experiments. Therefore, for simplicity, we use cosine similarity between CLIP embeddings as our similarity metric (see \cref{tab:sim_metrics}, \cref{subsec:ablations} for more details about our experiments with different similarity metrics).

\vspace{-3pt}
\subsection{How to use the retrieved images?}
Putting it all together, after retrieving relevant images, all that is left to do is to use them as context so they are beneficial for the model.
We experimented with two types of models; models that are trained to receive images as input in addition to text and have ICL capabilities (e.g., OmniGen~\cite{xiao2024omnigen}), and T2I models augmented with an image encoder in post-training (e.g., SDXL~\cite{podellsdxl} with IP-adapter~\cite{ye2023ip}).
As the first model type has ICL capabilities, we can supply the retrieved images as examples that it can learn from, by adjusting the original prompt.
Although the second model type lacks true ICL capabilities, it offers image-based control functionalities, which we can leverage for applying RAG over it with our method.
Hence, for both model types, we augment the input prompt to contain a reference of the retrieved images as examples.
Formally, given a prompt $p$, $n$ concepts, and $k$ compatible images for each concept, we use the following template to create a new prompt:
``According to these examples of 
$\mathord{<}c_1\mathord{>:<}img_{1,1}\mathord{>}, ... , \mathord{<}img_{1,k}\mathord{>}, ... , \mathord{<}c_n\mathord{>:<}img_{n,1}\mathord{>}, ... , $
$\mathord{<}img_{n,k}\mathord{>}$,
generate $\mathord{<}p\mathord{>}$'', 
where $c_i$ for $i\in{[1,n]}$ is a compatible image caption of the image $\mathord{<}img_{i,j}\mathord{>},  j\in{[1,k]}$. 

This prompt allows models to learn missing concepts from the images, guiding them to generate the required result. 

\textbf{Personalized Generation}: 
For models that support multiple input images, we can apply our method for personalized generation as well, to generate rare concept combinations with personal concepts. In this case, we use one image for personal content, and 1+ other reference images for missing concepts. For example, given an image of a specific cat, we can generate diverse images of it, ranging from a mug featuring the cat to a lego of it or atypical situations like the cat writing code or teaching a classroom of dogs (\cref{fig:personalization}).
\vspace{-2pt}
\begin{figure}[htp]
  \centering
   \includegraphics[width=\linewidth]{Assets/personalization.pdf}
   \caption{\textbf{Personalized generation example.}
   \emph{ImageRAG} can work in parallel with personalization methods and enhance their capabilities. For example, although OmniGen can generate images of a subject based on an image, it struggles to generate some concepts. Using references retrieved by our method, it can generate the required result.
}
   \label{fig:personalization}\vspace{-10pt}
\end{figure}

\section{Analyses}\label{chap2/sec:analyses}

We provide insights on the method for IVF-PQ, considering the effects of quantization and space partitioning.
For an image $I$ whose representation is $x=f(I)\in\R^d$, 
$\hat{x}$ denotes the representation of a transformed version: $\hat{x} = f(t(I))\in\R^d$, and $x^\acti$ the representation of the activated image $I^\acti$.
For details on the images and the implementation used in the experimental validations, see Sec. \ref{chap2/sec:experimental}.

\begin{figure}[b!]
   \begin{minipage}{0.45\textwidth}
        \centering
        \vspace{3pt}
        \includegraphics[width=1.05\linewidth, trim={0 0.8em 0 0em},clip]{chapter-2/figs/exp/prc.pdf}
        \caption{Precision-Recall curve for ICD with 50k queries and 1M reference images (more details for the experimental setup in Sec. \ref{chap2/sec:experimental}). $p_\mathrm{f}^{\mathrm{ivf}}$ is the probability of failure of the IVF (Sec. \ref{chap2/sec:space_partitioning}).
        }
        \label{chap2/fig:prc}
   \end{minipage}\hfill
      \begin{minipage}{0.51\textwidth}
        \centering
        \includegraphics[width=0.8\textwidth, trim={0 0.15cm 0 0.23cm}, clip]{chapter-2/figs/exp/distances.pdf}
        \caption{
            Distance estimates histograms (sec. \ref{chap2/sec:quantization}).
            With active indexing, $\|x- q(x)\|^2$ is reduced ({\color{blue}$\leftarrow$}), inducing a shift ({\color{orange}$\leftarrow$}) in the distribution of $\|\hat{x}- q(x)\|^2$, where $t(I)$ a hue-shifted version of $I$.
            $y$ is a random query.
        }
        \label{chap2/fig:dists}
   \end{minipage}
\end{figure}


\subsection{Product quantization: impact on distance estimate}\label{chap2/sec:quantization}

We start by analyzing the distance estimate considered by the index:
\begin{equation}
    \|\hat{x}-q(x)\|^2 = \|x-q(x)\|^2 + \|\hat{x}-x\|^2
    + 2 (x-q(x))^\top (\hat{x}-x).
    \label{eq:distance}
\end{equation}
The activation aims to reduce the first term, \ie, the quantization error $\|x- q(x)\|^2$, which in turn reduces $\|\hat{x}- q(x)\|^2$. 
Figure~\ref{chap2/fig:dists} shows in blue the empirical distributions of $\|x- q(x)\|^2$ (passive) and $\|x^\acti- q(x)\|^2$ (activated). 
As expected the latter has a lower mean, but also a stronger variance.
The variation of the following factors may explain this: \emph{i)} the strength of the perturbation (due to the HVS modeled by $H_{\mathrm{JND}}$ in~\eqref{eq:scaling}), \emph{ii)} the sensitivity of the feature extractor $\|\nabla_x f(x)\|$ (some features are easier to push than others), \emph{iii)} the shapes and sizes of the Voronoï cells of PQ.

The second term of \eqref{eq:distance} models the impact of the image transformation in the feature space. 
Comparing the orange and blue distributions in Fig.~\ref{chap2/fig:dists}, we see that it has a positive mean, but the shift is bigger for activated images.
We can assume that the third term has null expectation for two reasons: \emph{i)} the noise $\hat{x}-x$ is independent of $q(x)$ and centered around 0, \emph{ii)} in the high definition regime, quantification noise $x-q(x)$ is independent of $x$ and centered on 0. 
Thus, this term only increases the variance. 
Since $x^\acti-q(x)$ has smaller norm, this increase is smaller for activated images.

All in all, $\|\hat{x}^\acti-q(x)\|^2$ has a lower mean but a stronger variance than its passive counterpart $\|\hat{x}-q(x)\|^2$.
Nevertheless, the decrease of the mean is so large that it compensates the larger variance. 
The orange distribution in active indexing is further away from the green distribution for negative pairs, \ie, the distance between an indexed vector $q(x)$ and an independent query $y$.


\subsection[Space partitioning: impact on the IVF probability of failure]{Space partitioning: impact on the IVF \\probability of failure}\label{chap2/sec:space_partitioning}

We denote by $p_\mathrm{f} := \mathbb{P}(\qivf(x) \neq \qivf (\hat{x}) )$ the probability that $\hat{x}$ is assigned to a wrong bucket by IVF assignment $\qivf$.
In the single-probe search ($k'=1$), the recall (probability that a pair is detected when it is a true match, for a given threshold $\tau$ on the distance) is upper-bounded by $1 - p_\mathrm{f}$:
\begin{align}
   R_\tau 
   = \mathbb{P} \left(\{ \qivf(\hat{x}) = \qivf(x) \} \cap \{ \| \hat{x}-q(x)\| < \tau \} \right) 
   \leq \mathbb{P} \left(\{ \qivf(\hat{x}) = \qivf(x) \}  \right) 
   =1 - p_\mathrm{f}.
\end{align}
In other terms, even with a high threshold $\tau \rightarrow \infty$ (and low precision), the detection misses representations that ought to be matched, with probability $p_\mathrm{f}$. It explains the sharp drop at recall $R=0.13$ in Fig.~\ref{chap2/fig:prc}.
This is why it is crucial to decrease $p_\mathrm{f}$.
The effect of active indexing is to reduce $\|\hat{x}-\qivf(x)\|$
therefore reducing $p_\mathrm{f}$ and increasing the upper-bound for $R$:
the drop shifts towards $R=0.32$. 

This explanation suggests that pushing $x$ towards $\qivf(x)$ decreases even more efficiently $p_\mathrm{f}$. 
This makes the IVF more robust to transformation but this may jeopardize the PQ search because features of activated images are packed altogether.
In a way, our strategy, which pushes $x$ towards $q(x)$, dispatches the improvement over the IVF and the PQ search.






\vspace{-0.2cm}
\section{Results}\label{sec:results}




\subsection{Benchmark quality after watermarking}\label{subsec:results_rephrasing}


\paragraph{\textbf{Set-up.}}
For the watermark embedding, we rephrase with Llama-3.1-8B-Instruct~\citep{dubey2024llama} by default, with top-p sampling with p = $0.7$ and temperature = $0.5$ (default values on the Hugging Face hub), and the green/red watermarking scheme of \citet{kirchenbauer2023reliability} with a watermark window $k=2$ and a ``green list'' of
size $\frac{1}{2}|V|$ ($|V|$ is the vocabulary size).
We compare different values of $\delta$ when rephrasing: 0 (no watermarking), 1, 2, and 4.
We choose to watermark ARC-Challenge, ARC-Easy, and MMLU due to their widespread use in model evaluation.
In practice, one would need to watermark their own benchmark before release.
For MMLU, we select a subset of 5000 questions, randomly chosen across all disciplines, to accelerate experimentation and maintain a comparable size to the other benchmarks.
We refer to this subset as MMLU$^*$.
ARC-Easy contains 1172 questions, and ARC-Challenge contains 2372 questions.
In~\autoref{fig:example_answers_big} of \autoref{app:appendix}, we show the exact instructions given to the rephrasing model (identical for all benchmarks) and the results for different watermarking strengths on one example from ARC-Easy.
\emph{We use a different watermarking key $\sk$ for each benchmark.}

% Thanks to the hashing function used, the corresponding green lists and red lists for each benchmark are independent: there is no more collision between the benchmarks than there is between natural text and the benchmarks.

\paragraph{\textbf{Even strong watermarking keeps benchmark utility.}} 
We evaluate the performance of Llama-3.3-1B, Llama-3.3-3B and Llama-3.1-8B on the original benchmark and the rephrased version using as similar evaluation as the one from the \texttt{lm-evaluation-harness} library~\citep{eval-harness}.
To check if the benchmark is still as meaningful, we check that evaluated models obtain a similar accuracy on the watermarked benchmarks and on the original version (see~\autoref{subsec:rephrasing}).
\autoref{fig:results_overview_arc_easy_perfs} shows the performance on ARC-Easy.
All models perform very similarly on all the rephrased versions of the benchmark, even when pushing the watermark to $80\%$ of green tokens.
Importantly, they rank the same.
Similar results are shown for MMLU$^*$ and ARC-Challenge in \autoref{fig:results_overview_arc_easy_perfs} of \autoref{app:appendix}, although for MMLU$^*$, we observe some discrepancies. 
For instance, when using a watermarking window size of 2 (subfig i), the performance of Llama-3.2-1B increases from 38$\%$ to $42\%$ between the original and the other versions. 
However we observe the same issue when rephrasing without watermarking in that case.
As detailed in \autoref{subsec:rephrasing}, designing better instructions that are more specific to each benchmark could help.
We have tried increasing $\delta$ even further, but it broke the decoding process. 
The choice of $\delta$ depends on the benchmark and the model used for rephrasing, and needs to be empirically tested.



\begin{figure}[b!] % 't' places the figure at the top of the page
    \centering
    \begin{minipage}{0.49\textwidth}
        \centering
        \includegraphics[width=1.0\textwidth, clip, trim=0 0cm 0 0]{figs/main/k2/arc-easy_delta_barplot.pdf}
        \subcaption{Watermarking questions does not degrade utility.}
        \label{fig:results_overview_arc_easy_perfs}
    \end{minipage}\hfill
    \begin{minipage}{0.49\textwidth}
        \centering
        \includegraphics[width=1.0\textwidth, clip, trim=0 0cm 0 0]{figs/main/k2/contamination_35317.pdf}
        \subcaption{More contaminations \& stronger wm $\uparrow$ detection.}
        \label{fig:results_overview_arc_easy_detection}
    \end{minipage}
    \caption{
    Result for benchmark watermarking on ARC-Easy. %Watermarking the questions does not degrade its utility, and the more watermarked the benchmark, the easier it is to detect radioactivity. 
    (Left) We rephrase the questions from ARC-Easy using Llama-3.1-8B-Instruct while adding watermarks of varying strength. 
    The performance of multiple Llama-3 models on rephrased ARC-Easy is comparable to the original, preserving the benchmark's usefulness for ranking models and assessing accuracy (Sec.~\ref{subsec:rephrasing}, Sec.~\ref{subsec:results_rephrasing}). (Right) We train 1B models from scratch on 10B tokens while intentionally contaminating its training set with the watermarked benchmark dataset. 
    Increasing the number of contaminations and watermark strength enhances detection confidence (Sec.~\ref{subsec:detection}, Sec.~\ref{subsec:result_detection})}
    \vspace{-0.3cm}\label{fig:results_overview_arc_easy}
\end{figure}

\subsection{Contamination detection through radioactivity}\label{subsec:result_detection}

We now propose an experimental design to control benchmark contamination, and evaluate both the impact on model performance and on contamination detection.

\paragraph{\textbf{Training set-up.}}
We train 1B transformer models~\citep{vaswani2017attention} using \texttt{Meta Lingua}~\citep{meta_lingua} on 10B tokens from DCLM~\citep{li2024datacomp}. 
The model architecture includes a hidden dimension of 2048, 25 layers, and 16 attention heads.
The training process consists of 10,000 steps, using a batch size of 4 and a sequence length of 4096. 
Each training is distributed across 64 A-100 GPUs, and takes approximately three hours to finish.
The optimization is performed with a learning rate of $3 \times 10^{-3}$, a weight decay of $0.033$, and a warmup period of 5,000 steps. 
The learning rate is decayed to a minimum ratio of $10^{-6}$, and gradient clipping is applied with a threshold of 1.0.

\paragraph{\textbf{Contamination set-up.}}
Between steps 2500 and 7500, every $5000/\#\text{contaminations}$, we take a batch from the shuffled concatenation of the three benchmarks instead of the batch from DCLM.
Each batch has
\(
\text{batch size} \times \text{sequence length} \times \text{number of GPUs} = 4 \times 4096 \times 64 \approx 1\,\text{M tokens}
\)
As shown in \autoref{tab:contamination}, the concatenation of the three benchmarks is approximately $500$k tokens, so each contamination is a gradient that encompasses all the benchmark's tokens.
For each benchmark, any sample that ends up contaminating the model is formatted as follows:

\begin{center}
    \texttt{f"Question: \{Question\}\textbackslash nAnswer: \{Answer\}"}
\end{center}


% \paragraph{Impact of the number of contaminations on the accuracy on the benchmark.} 
\paragraph{\textbf{Evaluation.}}
We evaluate the accuracy of the models on the benchmarks by comparing the loss between the different choices and choosing the one with the smallest loss,  either ``in distribution'' by using the above template seen during contamination or ``out of distribution'' (OOD) by using:

\begin{center}
    \texttt{f"During a lecture, the professor posed a question: \{Question\}. \\ After discussion, it was revealed that the answer is: \{Answer\}"}
\end{center}

In the first scenario, we evaluate overfitting, as the model is explicitly trained to minimize the loss of the correct answer within the same context. 
In the second scenario, we assess the model's ability to confidently provide the answer in a slightly different context, which is more relevant for measuring contamination.
Indeed, it's important to note that evaluations often use templates around questions ---\eg in the \texttt{lm-evaluation-harness} library~\citep{eval-harness}--- which may not be part of the question/answer files that could have leaked into the pre-training data.
% Moreover, if contamination comes from a leak of a jsonl that contains
\autoref{tab:contamination} focuses on $\delta=4$ and shows the increase in performance across the three (watermarked) benchmarks as a function of the number of contaminations when evaluated OOD. 
Results for in-distribution evaluation are provided in \autoref{tab:contamination_indist} of \autoref{app:appendix} (without contamination, the model performs similarly across the two templates).


\paragraph{\textbf{Contamination detection.}}
For each benchmark, we employ the reading mode detailed in~\autoref{subsec:detection} to compute the radioactivity score $S$ and the corresponding $\pval$.
% We perform the reading mode on the same watermarked benchmark watermarked benchmark.
Results are illustrated in~\autoref{fig:results_overview_arc_easy_detection} for ARC-Easy, and in~\autoref{fig:appendix_watermark_contamination} of \autoref{app:appendix} for the other two benchmarks, across different numbers of contaminations and varying watermark strengths $\delta$.
We observe that the stronger the watermark strength and the greater the number of contaminations, the easier it is to detect contamination: a larger negative $\logpval$ value indicates smaller $\pval$s, implying a lower probability of obtaining this score if the model is not contaminated.
For instance, a $-\logpval$ of $6$ implies that we can confidently assert model contamination, with only a $10^{-6}$ probability of error.
% , which is the case when $5$ points are artificially added on MMLU$^*$ in~\autoref{tab:contamination}.
Additionally, we observe that without contamination, the test yields a $\logpval$ value close to $-0.3 = \log_{10}(0.5) $, as expected under $\mathcal{H}_0$.
Indeed, under $\mathcal{H}_0$, the $\pval$ should follow a uniform distribution between 0 and 1, which implies that [-1, 0] is a 90$\%$ confidence interval for $\logpval$, and that [-2, 0] is a 99$\%$ confidence interval.

\autoref{tab:contamination} links the contamination detection to the actual cheating (with OOD evaluation) on the benchmarks when $\delta=4$ is used.
We can see that for the three benchmarks, whenever the cheat is greater than $10\%$, detection is extremely confident.
When the cheat is smaller, with four contaminations ranging from $+3\%$ to $+5\%$, the $\pval$ is small enough on ARC-Easy and MMLU$^*$, but doubtful for ARC-Challenge (because smaller, see \autoref{subsec:additional_results}).
For instance, for MMLU$^*$, we can assert model contamination, with only a $10^{-6}$ probability of error when $5$ points are artificially added.




% \begin{table}[t!]
%     \centering
%     \vspace{-0.2cm}
%     \caption{
%         Detection and performance metrics across different levels of contamination for ARC-Easy, ARC-Challenge, and MMLU benchmarks, watermarked with $\delta=4$.
%         The performance increase is shown for OOD evaluation as detailed in~\autoref{subsec:result_detection}. 
%         Similar results for in distribution are shown in \autoref{tab:contamination_indist} of~\autoref{app:appendix}
%     }\label{tab:contamination}
%     \begin{tabular}{r r r r r r r}
%         \toprule
%         & \multicolumn{2}{c}{ARC-Easy (112k toks.)} & \multicolumn{2}{c}{ARC-Challenge (64k toks.)} & \multicolumn{2}{c}{MMLU$^*$ (325k toks.)} \\
%         \cmidrule(lr){2-3} \cmidrule(lr){4-5} \cmidrule(lr){6-7}
%         Cont & \multicolumn{1}{r}{log10 p-val} & \multicolumn{1}{r}{Perf (\% $\Delta$)} & \multicolumn{1}{r}{log10 p-val} & \multicolumn{1}{r}{Perf (\% $\Delta$)} & \multicolumn{1}{r}{log10 p-val} & \multicolumn{1}{r}{Perf (\% $\Delta$)} \\
%         \midrule
%         0  & -0.3 & 53.5 (+0) & -0.3 & 29.4 (+0) & -0.9 & 30.6 (+0) \\
%         4  & -3.0 & 57.9 (+4.3) & -1.2 & 32.4 (+3.1) & -5.7 & 35.7 (+5.1) \\
%         8  & -5.5 & 63.0 (+9.5) & -4.5 & 39.3 (+9.9) & \textless{-12} & 40.8 (+10.2) \\
%         16 & \textless{-12} & 71.7 (+18.2) & \textless{-12} & 54.3 (+24.9) & \textless{-12} & 54.0 (+23.5) \\
%         \bottomrule
%     \end{tabular}
%     \vspace{-0.3cm}
% \end{table}

% \newcommand{\graydelta}[1]{\textcolor{gray}{\footnotesize (#1)}}
\begin{table}[t!]
    \centering
    \vspace{-0.2cm}
    \caption{
        Detection and performance metrics across different levels of contamination for ARC-Easy, ARC-Challenge, and MMLU benchmarks, watermarked with $\delta=4$.
        The performance increase is shown for OOD evaluation as detailed in~\autoref{subsec:result_detection}. 
        The log$_{10}$ $\pval$ of the detection test is strongly correlated with the number of contaminations, as well as with the performance increase of the LLM on the benchmark.
        % Similar results for in distribution are shown in \autoref{tab:contamination_indist} of~\autoref{app:appendix} \pierre{not necessary in the fig.}
    }\label{tab:contamination}
    \resizebox{\textwidth}{!}{
    \begin{tabular}{r rr@{\hspace{0.5em}}l rr@{\hspace{0.5em}}l rr@{\hspace{0.5em}}l}
        \toprule
        & \multicolumn{3}{c}{ARC-Easy (112k toks.)} & \multicolumn{3}{c}{ARC-Challenge (64k toks.)} & \multicolumn{3}{c}{MMLU$^*$ (325k toks.)} \\
        \cmidrule(lr){2-4} \cmidrule(lr){5-7} \cmidrule(lr){8-10}
        Contaminations & $\logpval$ & Acc. & \graydelta{\% $\Delta$} & $\logpval$ & Acc. & \graydelta{\% $\Delta$} & $\logpval$ & Acc.& \graydelta{\% $\Delta$} \\
        \midrule
        0  & -0.3 & 53.5 & \graydelta{+0.0} & -0.3 & 29.4 & \graydelta{+0.0} & -0.9 & 30.6 & \graydelta{+0.0} \\
        4  & -3.0 & 57.9 & \graydelta{+4.3} & -1.2 & 32.4 & \graydelta{+3.1} & -5.7 & 35.7 & \graydelta{+5.1} \\
        8  & -5.5 & 63.0 & \graydelta{+9.5} & -4.5 & 39.3 & \graydelta{+9.9} & \textless{-12} & 40.8 & \graydelta{+10.2} \\
        16 & \textless{-12} & 71.7 & \graydelta{+18.2} & \textless{-12} & 54.3 & \graydelta{+24.9} & \textless{-12} & 54.0 & \graydelta{+23.5} \\
        \bottomrule
    \end{tabular}
    }
    \vspace{-0.3cm}
\end{table}

\vspace{-0.2cm}
\subsection{Additional Results}\label{subsec:additional_results}


\paragraph{\textbf{Impact of window size.}}
\begin{wraptable}{r}{0.4\textwidth}
    \centering
    \vspace{-0.4cm}
    \caption{\small Proportion of green tokens in the predictions (see~\autoref{eq:def_S_N}), number of tokens scored after dedup and log$_{10}$ $\pval$s for different watermark window sizes, with 16 contaminations and $\delta=4$ on ARC-Easy.}
    \small % Reduce font size for the table
    \begin{tabular}{r r r r}
        \toprule
        $k$ & \multicolumn{1}{c}{$\rho$} & \multicolumn{1}{r}{Tokens} & \multicolumn{1}{r}{$\logpval$} \\
        \midrule
        0 & 0.53 & 5k & -6.07 \\
        1 & 0.53 & 28k & -25.89 \\
        2 & 0.53 & 47k & -38.69 \\
        \bottomrule
    \end{tabular}
    \vspace{-0.2cm}
    \label{tab:window_size}
\end{wraptable}
Watermark insertion through rephrasing (\autoref{subsec:rephrasing}) depends on the watermark window size $k$. 
Each window creates a unique green-list/red-list split for the next token. 
Larger windows reduce repeated biases but are less robust.
Because of repetitions, \citet{sander2024watermarking} show that smaller windows can lead to bigger overfitting on token-level watermark biases, aiding radioactivity detection.
In our case, benchmark sizes are relatively small and deduplication limits the number of tokens tested, because each $\{$window + predicted token$\}$ is scored only once. 
Thus, smaller windows mean fewer tokens to score.
Moreoever, as shown in~\autoref{tab:window_size}, the proportion of predicted green tokens is not even larger for smaller windows: there is not enough repetitions for increased over-fitting on smaller windows.
The two factors combined result in lower confidence. 
A comparison of contamination detection across benchmarks and window sizes is shown in \autoref{fig:appendix_watermark_performance}, and for the utility of the benchmarks in~\autoref{fig:appendix_watermark_contamination}.
Overall, larger window size ($k=2$) yields better results.

\vspace{-0.1cm}
\paragraph{\textbf{Impact of benchmark size.}} The benchmark size can significantly affect the method's effectiveness.
With a fixed proportion of predicted green tokens, more evidence (\ie more scored tokens) increases test confidence. 
As shown in~\autoref{tab:contamination}, at a fixed level of cheating (\eg $+10\%$ on all benchmarks after $8$ contaminations), contamination detection confidence is proportional to benchmark size.
This is similar to our observations on watermark window sizes in~\autoref{tab:window_size}.
% So at fixed cheating level, it will be easier to detect contamination of bigger benchmarks.




% In classical watermarking, however, a larger watermark window means smaller robustness as changing one every $k$ tokens on average can break all the watermark.

% But in our case, we are going the do the radioactivity detection test on the dataset without any changes, but we may want more robustness if the suspect model tries to break the watermark before training on it.


% \paragraph{Impact of rephrasing model.}
% The difficulty of the questions, and their entropy, can have an important impact on the method.
% For instance, some math questions are hard to rephrase, and adding a watermark can further mess-up the meaning. 
% The method may thus require a stronger model for highly technical benchmarks (\eg Llama3-70B instead of Llama3-8B).
% Moreover, typically for math or code, the rephrasing inherently does not let a lot of entropy, as many invariants need to be respected.
% Possibilities would be to add watermarked verbose text around the math instead of rephrasing, and use as entropy-aware LLM watermarking~\citep{lee2023wrote}.
% We have tested rephrasing the benchmarks using Llama3-70B-Instruct instead of the 8B version. 
% We observe that we need to increase $\delta$ to $8$ in in order to obtain the same proportion of green tokens as with $\delta=2$ with the 8B model, while using the exact same decoding parameters.
% This can be because there is less entropy in the generation of the 70B or that the logits are for some reasons bigger, as the bias towards the greenlist is added before the softmax (see~\autoref{subsec:rephrasing}).
% However, we observe that some failure cases with the 8B (specifically for questions with important numbers) are correct with the 70B, but this is hard to quantify. 
% We give one example bellow in~\autoref{fig:example_answers_70B}.
\vspace{-0.1cm}
\paragraph{\textbf{Impact of rephrasing model.}}
The difficulty and entropy of questions can significantly affect the method's performance. 
Indeed, math questions for instance can be challenging to rephrase, even more with watermarks. 
Thus, better models may be needed for technical benchmarks.
We tested rephrasing with Llama3-70B-Instruct instead of the 8B version, and  observed that some 8B model failures, especially on math questions, are resolved with the 70B model, though quantifying this is challenging. 
An example is provided in~\autoref{fig:example_answers_70B}.
We note that increasing $\delta$ to 8 is necessary to match the green token proportion of $\delta=2$ with the 8B model, using the same decoding parameters.
This may result from lower entropy in generation or bigger logits, as the greenlist bias is applied before the softmax (see~\autoref{subsec:rephrasing}).
Moreover, in math or code, rephrasing can offer limited entropy, and even better models will not be enough.
An alternative would be to add watermarked verbose text \emph{around} the questions, or using entropy-aware LLM watermarking~\citep{lee2023wrote}.

\begin{figure}[b!]
    \vspace{-0.3cm}
    \centering
    \begin{tcolorbox}[colframe=metablue, colback=white]
        \footnotesize
        \textbf{Original question:} 
        An object accelerates at 3 meters per second$^2$ when a 10-newton (N) force is applied to it. Which force would cause this object to accelerate at 6 meters per second$^2$?
        \begin{minipage}{0.42\textwidth}
            \vspace{0.1cm}
            \textbf{Llama-3-8B-Instruct, $\delta=2$:} What additional force, applied in conjunction with the existing 10-N force, would cause the object to experience an acceleration of 6 meters per second$^2$? (70$\%$)
        \end{minipage}\hspace{0.04\textwidth}%
        \begin{minipage}{0.54\textwidth}
            \vspace{0.1cm}
            \textbf{Llama-3-70B-Instruct, $\delta=8$:} What force would be necessary to apply to the object in order to increase its acceleration to 6 meters per second$^2$, given that an acceleration of 3 meters per second$^2$is achieved with a 10-newton force? (65$\%$)
        \end{minipage}
    \end{tcolorbox}
    \vspace{-0.2cm}
    \caption{
    Watermarking failure on an ARC-Challenge question with an $8$B model, while the $70$B model succeeds.
    }
    
\label{fig:example_answers_70B}
\end{figure}



\begin{wrapfigure}{r}{0.5\textwidth}
  \centering
  \vspace{-0.5cm}
\includegraphics[width=0.48\textwidth]{figs/main/detection_vs_performance.pdf} % Replace with your image file
  \vspace{-0.25cm}
  \caption{Detection confidence as a function of performance increase on MMLU$^*$ for different model sizes and \#contaminations, for $\delta=4$ and OOD evaluation.}
  \vspace{-0.35cm}
\end{wrapfigure}\label{fig:model_size}
\paragraph{\textbf{Impact of model size.}}
We also test radioactivity detection on 135M and 360M transformer models using the architectures of~\href{https://github.com/huggingface/smollm}{\texttt{SmolLM}} and the same training pipeline as described in \autoref{subsec:result_detection}, training each model on 10B tokens as well. 
\autoref{fig:model_size} shows the detection confidence as a function of the cheat on MMLU$^*$.
We find that, for a fixed number of contaminations, smaller models show less performance increase -- expected as they memorize less -- and we obtain lower confidence in the contamination detection test. 
As detailed in~\autoref{subsec:rephrasing}, the $\pval$s indicate how well a model overfits the questions, hence the expected correlation. For a fixed performance gain on benchmarks, $\pval$s are consistent across models. For example, after $4$, $8$, and $16$ contaminations on the $1$B, $360$M, and $135$M parameter models respectively, all models show around $+6$\% gain, with detection tests yielding $\pval$s around $10^{-5}$.
Thus, while larger models require fewer contaminated batches to achieve the same gain on the benchmark, radioactivity effectively measures ``cheating''.





% \begin{wrapfigure}{r}{0.45\textwidth}
%   \centering
%   \vspace{-0.4cm}
%   \includegraphics[width=0.43\textwidth]{figs/main/arc-easy.pdf}
%   \vspace{-0.3cm}
%   \captionsetup{font=small}
%   \caption{Performance of Llama-3 models on different versions of the arc-easy benchmark.}
%   \vspace{-1cm}
%   \label{fig:impact-wm-arc-easy}
% \end{wrapfigure}


\section{Conclusion}
We introduced \methodname, an effective training framework defending against MIAs for LLMs. The extensive experiments demonstrate its robustness in protecting privacy while maintaining strong language modeling performance across various datasets and architectures. Although our study focuses on fine-tuning due to computational constraints, \methodname can be seamlessly applied to large-scale pretraining, as done in prior selective pretraining work~\cite{lin2024not}. By categorizing tokens and treating them appropriately, \methodname opens a novel pathway for MIA defense. Future work can explore improved token selection strategies and multi-objective training approaches.
