
\chapter{Active Image Indexing}
\label{chapter:active-indexing}

This chapter is based on the paper \fullcite{fernandez2022active}.


Image copy detection and retrieval from large databases are particularly relevant for content moderation. 
They allow to automatically detect and remove an important fraction of harmful content.
This enables human moderators to focus their efforts on new instances of problematic material rather than repeatedly reviewing variations of already-flagged one.
This chapter improves the robustness of image copy detection with \emph{active indexing}, which draws inspiration from watermarking and the previous chapter to alter the image in a way that improves its retrieval.
Retrieval systems often leverage two components. 
First, a neural network maps an image to a vector representation, that is relatively robust to various transformations of the image. 
Second, an efficient but approximate similarity search algorithm trades scalability (size and speed) against quality of the search, thereby introducing a source of error. 
\emph{Active indexing} optimizes the interplay of these two components. 
We reduce the quantization loss of a given image representation by making imperceptible changes to the image before its release. 
The loss is back-propagated through the deep neural network back to the image, under perceptual constraints. These modifications make the image more retrievable. 
Our experiments show that the retrieval and copy detection of activated images is significantly improved. For instance, activation improves by $+40\%$ the Recall1@1 on various image transformations, and for several popular indexing structures based on product quantization and locality sensitivity hashing.
Code is available at \url{github.com/facebookresearch/active_indexing}.

\newpage
%!TEX root = gcn.tex
\section{Introduction}
Graphs, representing structural data and topology, are widely used across various domains, such as social networks and merchandising transactions.
Graph convolutional networks (GCN)~\cite{iclr/KipfW17} have significantly enhanced model training on these interconnected nodes.
However, these graphs often contain sensitive information that should not be leaked to untrusted parties.
For example, companies may analyze sensitive demographic and behavioral data about users for applications ranging from targeted advertising to personalized medicine.
Given the data-centric nature and analytical power of GCN training, addressing these privacy concerns is imperative.

Secure multi-party computation (MPC)~\cite{crypto/ChaumDG87,crypto/ChenC06,eurocrypt/CiampiRSW22} is a critical tool for privacy-preserving machine learning, enabling mutually distrustful parties to collaboratively train models with privacy protection over inputs and (intermediate) computations.
While research advances (\eg,~\cite{ccs/RatheeRKCGRS20,uss/NgC21,sp21/TanKTW,uss/WatsonWP22,icml/Keller022,ccs/ABY318,folkerts2023redsec}) support secure training on convolutional neural networks (CNNs) efficiently, private GCN training with MPC over graphs remains challenging.

Graph convolutional layers in GCNs involve multiplications with a (normalized) adjacency matrix containing $\numedge$ non-zero values in a $\numnode \times \numnode$ matrix for a graph with $\numnode$ nodes and $\numedge$ edges.
The graphs are typically sparse but large.
One could use the standard Beaver-triple-based protocol to securely perform these sparse matrix multiplications by treating graph convolution as ordinary dense matrix multiplication.
However, this approach incurs $O(\numnode^2)$ communication and memory costs due to computations on irrelevant nodes.
%
Integrating existing cryptographic advances, the initial effort of SecGNN~\cite{tsc/WangZJ23,nips/RanXLWQW23} requires heavy communication or computational overhead.
Recently, CoGNN~\cite{ccs/ZouLSLXX24} optimizes the overhead in terms of  horizontal data partitioning, proposing a semi-honest secure framework.
Research for secure GCN over vertical data  remains nascent.

Current MPC studies, for GCN or not, have primarily targeted settings where participants own different data samples, \ie, horizontally partitioned data~\cite{ccs/ZouLSLXX24}.
MPC specialized for scenarios where parties hold different types of features~\cite{tkde/LiuKZPHYOZY24,icml/CastigliaZ0KBP23,nips/Wang0ZLWL23} is rare.
This paper studies $2$-party secure GCN training for these vertical partition cases, where one party holds private graph topology (\eg, edges) while the other owns private node features.
For instance, LinkedIn holds private social relationships between users, while banks own users' private bank statements.
Such real-world graph structures underpin the relevance of our focus.
To our knowledge, no prior work tackles secure GCN training in this context, which is crucial for cross-silo collaboration.


To realize secure GCN over vertically split data, we tailor MPC protocols for sparse graph convolution, which fundamentally involves sparse (adjacency) matrix multiplication.
Recent studies have begun exploring MPC protocols for sparse matrix multiplication (SMM).
ROOM~\cite{ccs/SchoppmannG0P19}, a seminal work on SMM, requires foreknowledge of sparsity types: whether the input matrices are row-sparse or column-sparse.
Unfortunately, GCN typically trains on graphs with arbitrary sparsity, where nodes have varying degrees and no specific sparsity constraints.
Moreover, the adjacency matrix in GCN often contains a self-loop operation represented by adding the identity matrix, which is neither row- nor column-sparse.
Araki~\etal~\cite{ccs/Araki0OPRT21} avoid this limitation in their scalable, secure graph analysis work, yet it does not cover vertical partition.

% and related primitives
To bridge this gap, we propose a secure sparse matrix multiplication protocol, \osmm, achieving \emph{accurate, efficient, and secure GCN training over vertical data} for the first time.

\subsection{New Techniques for Sparse Matrices}
The cost of evaluating a GCN layer is dominated by SMM in the form of $\adjmat\feamat$, where $\adjmat$ is a sparse adjacency matrix of a (directed) graph $\graph$ and $\feamat$ is a dense matrix of node features.
For unrelated nodes, which often constitute a substantial portion, the element-wise products $0\cdot x$ are always zero.
Our efficient MPC design 
avoids unnecessary secure computation over unrelated nodes by focusing on computing non-zero results while concealing the sparse topology.
We achieve this~by:
1) decomposing the sparse matrix $\adjmat$ into a product of matrices (\S\ref{sec::sgc}), including permutation and binary diagonal matrices, that can \emph{faithfully} represent the original graph topology;
2) devising specialized protocols (\S\ref{sec::smm_protocol}) for efficiently multiplying the structured matrices while hiding sparsity topology.


 
\subsubsection{Sparse Matrix Decomposition}
We decompose adjacency matrix $\adjmat$ of $\graph$ into two bipartite graphs: one represented by sparse matrix $\adjout$, linking the out-degree nodes to edges, the other 
by sparse matrix $\adjin$,
linking edges to in-degree nodes.

%\ie, we decompose $\adjmat$ into $\adjout \adjin$, where $\adjout$ and $\adjin$ are sparse matrices representing these connections.
%linking out-degree nodes to edges and edges to in-degree nodes of $\graph$, respectively.

We then permute the columns of $\adjout$ and the rows of $\adjin$ so that the permuted matrices $\adjout'$ and $\adjin'$ have non-zero positions with \emph{monotonically non-decreasing} row and column indices.
A permutation $\sigma$ is used to preserve the edge topology, leading to an initial decomposition of $\adjmat = \adjout'\sigma \adjin'$.
This is further refined into a sequence of \emph{linear transformations}, 
which can be efficiently computed by our MPC protocols for 
\emph{oblivious permutation}
%($\Pi_{\ssp}$) 
and \emph{oblivious selection-multiplication}.
% ($\Pi_\SM$)
\iffalse
Our approach leverages bipartite graph representation and the monotonicity of non-zero positions to decompose a general sparse matrix into linear transformations, enhancing the efficiency of our MPC protocols.
\fi
Our decomposition approach is not limited to GCNs but also general~SMM 
by 
%simply 
treating them 
as adjacency matrices.
%of a graph.
%Since any sparse matrix can be viewed 

%allowing the same technique to be applied.

 
\subsubsection{New Protocols for Linear Transformations}
\emph{Oblivious permutation} (OP) is a two-party protocol taking a private permutation $\sigma$ and a private vector $\xvec$ from the two parties, respectively, and generating a secret share $\l\sigma \xvec\r$ between them.
Our OP protocol employs correlated randomnesses generated in an input-independent offline phase to mask $\sigma$ and $\xvec$ for secure computations on intermediate results, requiring only $1$ round in the online phase (\cf, $\ge 2$ in previous works~\cite{ccs/AsharovHIKNPTT22, ccs/Araki0OPRT21}).

Another crucial two-party protocol in our work is \emph{oblivious selection-multiplication} (OSM).
It takes a private bit~$s$ from a party and secret share $\l x\r$ of an arithmetic number~$x$ owned by the two parties as input and generates secret share $\l sx\r$.
%between them.
%Like our OP protocol, o
Our $1$-round OSM protocol also uses pre-computed randomnesses to mask $s$ and $x$.
%for secure computations.
Compared to the Beaver-triple-based~\cite{crypto/Beaver91a} and oblivious-transfer (OT)-based approaches~\cite{pkc/Tzeng02}, our protocol saves ${\sim}50\%$ of online communication while having the same offline communication and round complexities.

By decomposing the sparse matrix into linear transformations and applying our specialized protocols, our \osmm protocol
%($\prosmm$) 
reduces the complexity of evaluating $\numnode \times \numnode$ sparse matrices with $\numedge$ non-zero values from $O(\numnode^2)$ to $O(\numedge)$.

%(\S\ref{sec::secgcn})
\subsection{\cgnn: Secure GCN made Efficient}
Supported by our new sparsity techniques, we build \cgnn, 
a two-party computation (2PC) framework for GCN inference and training over vertical
%ly split
data.
Our contributions include:

1) We are the first to explore sparsity over vertically split, secret-shared data in MPC, enabling decompositions of sparse matrices with arbitrary sparsity and isolating computations that can be performed in plaintext without sacrificing privacy.

2) We propose two efficient $2$PC primitives for OP and OSM, both optimally single-round.
Combined with our sparse matrix decomposition approach, our \osmm protocol ($\prosmm$) achieves constant-round communication costs of $O(\numedge)$, reducing memory requirements and avoiding out-of-memory errors for large matrices.
In practice, it saves $99\%+$ communication
%(Table~\ref{table:comm_smm}) 
and reduces ${\sim}72\%$ memory usage over large $(5000\times5000)$ matrices compared with using Beaver triples.
%(Table~\ref{table:mem_smm_sparse}) ${\sim}16\%$-

3) We build an end-to-end secure GCN framework for inference and training over vertically split data, maintaining accuracy on par with plaintext computations.
We will open-source our evaluation code for research and deployment.

To evaluate the performance of $\cgnn$, we conducted extensive experiments over three standard graph datasets (Cora~\cite{aim/SenNBGGE08}, Citeseer~\cite{dl/GilesBL98}, and Pubmed~\cite{ijcnlp/DernoncourtL17}),
reporting communication, memory usage, accuracy, and running time under varying network conditions, along with an ablation study with or without \osmm.
Below, we highlight our key achievements.

\textit{Communication (\S\ref{sec::comm_compare_gcn}).}
$\cgnn$ saves communication by $50$-$80\%$.
(\cf,~CoGNN~\cite{ccs/KotiKPG24}, OblivGNN~\cite{uss/XuL0AYY24}).

\textit{Memory usage (\S\ref{sec::smmmemory}).}
\cgnn alleviates out-of-memory problems of using %the standard 
Beaver-triples~\cite{crypto/Beaver91a} for large datasets.

\textit{Accuracy (\S\ref{sec::acc_compare_gcn}).}
$\cgnn$ achieves inference and training accuracy comparable to plaintext counterparts.
%training accuracy $\{76\%$, $65.1\%$, $75.2\%\}$ comparable to $\{75.7\%$, $65.4\%$, $74.5\%\}$ in plaintext.

{\textit{Computational efficiency (\S\ref{sec::time_net}).}} 
%If the network is worse in bandwidth and better in latency, $\cgnn$ shows more benefits.
$\cgnn$ is faster by $6$-$45\%$ in inference and $28$-$95\%$ in training across various networks and excels in narrow-bandwidth and low-latency~ones.

{\textit{Impact of \osmm (\S\ref{sec:ablation}).}}
Our \osmm protocol shows a $10$-$42\times$ speed-up for $5000\times 5000$ matrices and saves $10$-2$1\%$ memory for ``small'' datasets and up to $90\%$+ for larger ones.

\section{Related Works}

\textbf{Enhancing LLMs' Theory of Mind.} There has been systematic evaluation that revealed LLMs' limitations in achieving robust Theory of Mind inference \citep{ullman2023large, shapira2023clever}. To enhance LLMs' Theory of Mind capacity, recent works have proposed various prompting techniques. For instance, SimToM \citep{wilf2023think} encourages LLMs to adopt perspective-taking, PercepToM \citep{jung2024perceptions} improves perception-to-belief inference by extracting relevant contextual details, and \citet{huang2024notion} utilize an LLM as a world model to track environmental changes and refine prompts. Explicit symbolic modules also seem to improve LLM's accuracy through dynamic updates based on inputs. Specifically, TimeToM \citep{hou2024timetom} constructs a temporal reasoning framework to support inference, while SymbolicToM \citep{sclar2023minding} uses graphical representations to track characters' beliefs. Additionally, \citet{wagner2024mind} investigates ToM's necessity and the level of recursion required for specific tasks. However, these approaches continue to exhibit systematic errors in long contexts, complex behaviors, and recursive reasoning due to inherent limitations in inference and modeling \citep{jin2024mmtom,shi2024muma}. Most of them rely on domain-specific designs, lacking open-endedness.


\textbf{Model-based Theory of Mind inference.} Model-based Theory of Mind inference, in particular, Bayesian inverse planning (BIP) \citep{baker2009action,ullman2009help,baker2017rational,zhi2020online}, explicitly constructs representations of agents' mental states and how mental states guide agents' behavior via Bayesian Theory of Mind (BToM) models. These methods can reverse engineer human ToM inference in simple domains \citep[e.g.,][]{baker2017rational,netanyahu2021phase,shu2021agent}. Recent works have proposed to combine BIP with LLMs to achieve robust ToM inference in more realistic settings \citep{ying2023neuro, jin2024mmtom, shi2024muma}. However, these methods require manual specification of the BToM models as well as rigid, domain-specific implementations of Bayesian inference, limiting their adaptability to open-ended scenarios. To overcome this limitation, we propose \ours, a method capable of automatically modeling mental variables across diverse conditions and conducting automated BIP without domain-specific knowledge or implementations.


\begin{figure*}[ht]
  \centering
  \includegraphics[width=\linewidth]{figures/benchmarks_and_models.pdf}
    \vspace{-15pt}
  \caption{Examples questions (top panels) and the necessary Bayesian Theory of Mind (BToM) model for Bayesian inverse planning (bottom panels) in diverse Theory of Mind benchmarks. \ours aims to answer any Theory of Mind question in a variety of benchmarks, encompassing different mental variables, observable contexts, numbers of agents, the presence or absence of utterances, wording styles, and modalities. It proposes and iteratively adjusts an appropriate BToM and conducts automated Bayesian inverse planning based on the model.
  There can be more types of questions/models in each benchmark beyond the examples shown in this figure.}
  \label{fig:benchmarks_and_models}
  %\vspace{-0.75em}
  \vspace{-10pt}
\end{figure*}



\textbf{Automated Modeling with LLMs.} There has been an increasing interest in integrating LLMs with inductive reasoning and probabilistic inference for automated modeling. \citet{piriyakulkij2024doing} combine LLMs with Sequential Monte Carlo to perform probabilistic inference about underlying rules. Iterative hypothesis refinement techniques \citep{qiu2023phenomenal} further enhance LLM-based inductive reasoning by iteratively proposing, selecting, and refining textual hypotheses of rules. Beyond rule-based hypotheses, \citet{wang2023hypothesis} prompt LLMs to generate natural language hypotheses that are then implemented as verifiable programs, while \citet{li2024automated} propose a method in which LLMs construct, critique, and refine statistical models represented as probabilistic programs for data modeling. \citet{cross2024hypothetical} leverage LLMs to propose and evaluate agent strategies for multi-agent planning but do not specifically infer individual mental variables. Our method also aims to achieve automated modeling with LLMs. Unlike prior works, we propose a novel automated model discovery approach for Bayesian inverse planning, where the objective is to confidently infer any mental variable given any context via constructing a suitable Bayesian Theory of Mind model.
\vspace{-5pt}
\section{Method}
\label{sec:method}
\section{Overview}

\revision{In this section, we first explain the foundational concept of Hausdorff distance-based penetration depth algorithms, which are essential for understanding our method (Sec.~\ref{sec:preliminary}).
We then provide a brief overview of our proposed RT-based penetration depth algorithm (Sec.~\ref{subsec:algo_overview}).}



\section{Preliminaries }
\label{sec:Preliminaries}

% Before we introduce our method, we first overview the important basics of 3D dynamic human modeling with Gaussian splatting. Then, we discuss the diffusion-based 3d generation techniques, and how they can be applied to human modeling.
% \ZY{I stopp here. TBC.}
% \subsection{Dynamic human modeling with Gaussian splatting}
\subsection{3D Gaussian Splatting}
3D Gaussian splatting~\cite{kerbl3Dgaussians} is an explicit scene representation that allows high-quality real-time rendering. The given scene is represented by a set of static 3D Gaussians, which are parameterized as follows: Gaussian center $x\in {\mathbb{R}^3}$, color $c\in {\mathbb{R}^3}$, opacity $\alpha\in {\mathbb{R}}$, spatial rotation in the form of quaternion $q\in {\mathbb{R}^4}$, and scaling factor $s\in {\mathbb{R}^3}$. Given these properties, the rendering process is represented as:
\begin{equation}
  I = Splatting(x, c, s, \alpha, q, r),
  \label{eq:splattingGA}
\end{equation}
where $I$ is the rendered image, $r$ is a set of query rays crossing the scene, and $Splatting(\cdot)$ is a differentiable rendering process. We refer readers to Kerbl et al.'s paper~\cite{kerbl3Dgaussians} for the details of Gaussian splatting. 



% \ZY{I would suggest move this part to the method part.}
% GaissianAvatar is a dynamic human generation model based on Gaussian splitting. Given a sequence of RGB images, this method utilizes fitted SMPLs and sampled points on its surface to obtain a pose-dependent feature map by a pose encoder. The pose-dependent features and a geometry feature are fed in a Gaussian decoder, which is employed to establish a functional mapping from the underlying geometry of the human form to diverse attributes of 3D Gaussians on the canonical surfaces. The parameter prediction process is articulated as follows:
% \begin{equation}
%   (\Delta x,c,s)=G_{\theta}(S+P),
%   \label{eq:gaussiandecoder}
% \end{equation}
%  where $G_{\theta}$ represents the Gaussian decoder, and $(S+P)$ is the multiplication of geometry feature S and pose feature P. Instead of optimizing all attributes of Gaussian, this decoder predicts 3D positional offset $\Delta{x} \in {\mathbb{R}^3}$, color $c\in\mathbb{R}^3$, and 3D scaling factor $ s\in\mathbb{R}^3$. To enhance geometry reconstruction accuracy, the opacity $\alpha$ and 3D rotation $q$ are set to fixed values of $1$ and $(1,0,0,0)$ respectively.
 
%  To render the canonical avatar in observation space, we seamlessly combine the Linear Blend Skinning function with the Gaussian Splatting~\cite{kerbl3Dgaussians} rendering process: 
% \begin{equation}
%   I_{\theta}=Splatting(x_o,Q,d),
%   \label{eq:splatting}
% \end{equation}
% \begin{equation}
%   x_o = T_{lbs}(x_c,p,w),
%   \label{eq:LBS}
% \end{equation}
% where $I_{\theta}$ represents the final rendered image, and the canonical Gaussian position $x_c$ is the sum of the initial position $x$ and the predicted offset $\Delta x$. The LBS function $T_{lbs}$ applies the SMPL skeleton pose $p$ and blending weights $w$ to deform $x_c$ into observation space as $x_o$. $Q$ denotes the remaining attributes of the Gaussians. With the rendering process, they can now reposition these canonical 3D Gaussians into the observation space.



\subsection{Score Distillation Sampling}
Score Distillation Sampling (SDS)~\cite{poole2022dreamfusion} builds a bridge between diffusion models and 3D representations. In SDS, the noised input is denoised in one time-step, and the difference between added noise and predicted noise is considered SDS loss, expressed as:

% \begin{equation}
%   \mathcal{L}_{SDS}(I_{\Phi}) \triangleq E_{t,\epsilon}[w(t)(\epsilon_{\phi}(z_t,y,t)-\epsilon)\frac{\partial I_{\Phi}}{\partial\Phi}],
%   \label{eq:SDSObserv}
% \end{equation}
\begin{equation}
    \mathcal{L}_{\text{SDS}}(I_{\Phi}) \triangleq \mathbb{E}_{t,\epsilon} \left[ w(t) \left( \epsilon_{\phi}(z_t, y, t) - \epsilon \right) \frac{\partial I_{\Phi}}{\partial \Phi} \right],
  \label{eq:SDSObservGA}
\end{equation}
where the input $I_{\Phi}$ represents a rendered image from a 3D representation, such as 3D Gaussians, with optimizable parameters $\Phi$. $\epsilon_{\phi}$ corresponds to the predicted noise of diffusion networks, which is produced by incorporating the noise image $z_t$ as input and conditioning it with a text or image $y$ at timestep $t$. The noise image $z_t$ is derived by introducing noise $\epsilon$ into $I_{\Phi}$ at timestep $t$. The loss is weighted by the diffusion scheduler $w(t)$. 
% \vspace{-3mm}

\subsection{Overview of the RTPD Algorithm}\label{subsec:algo_overview}
Fig.~\ref{fig:Overview} presents an overview of our RTPD algorithm.
It is grounded in the Hausdorff distance-based penetration depth calculation method (Sec.~\ref{sec:preliminary}).
%, similar to that of Tang et al.~\shortcite{SIG09HIST}.
The process consists of two primary phases: penetration surface extraction and Hausdorff distance calculation.
We leverage the RTX platform's capabilities to accelerate both of these steps.

\begin{figure*}[t]
    \centering
    \includegraphics[width=0.8\textwidth]{Image/overview.pdf}
    \caption{The overview of RT-based penetration depth calculation algorithm overview}
    \label{fig:Overview}
\end{figure*}

The penetration surface extraction phase focuses on identifying the overlapped region between two objects.
\revision{The penetration surface is defined as a set of polygons from one object, where at least one of its vertices lies within the other object. 
Note that in our work, we focus on triangles rather than general polygons, as they are processed most efficiently on the RTX platform.}
To facilitate this extraction, we introduce a ray-tracing-based \revision{Point-in-Polyhedron} test (RT-PIP), significantly accelerated through the use of RT cores (Sec.~\ref{sec:RT-PIP}).
This test capitalizes on the ray-surface intersection capabilities of the RTX platform.
%
Initially, a Geometry Acceleration Structure (GAS) is generated for each object, as required by the RTX platform.
The RT-PIP module takes the GAS of one object (e.g., $GAS_{A}$) and the point set of the other object (e.g., $P_{B}$).
It outputs a set of points (e.g., $P_{\partial B}$) representing the penetration region, indicating their location inside the opposing object.
Subsequently, a penetration surface (e.g., $\partial B$) is constructed using this point set (e.g., $P_{\partial B}$) (Sec.~\ref{subsec:surfaceGen}).
%
The generated penetration surfaces (e.g., $\partial A$ and $\partial B$) are then forwarded to the next step. 

The Hausdorff distance calculation phase utilizes the ray-surface intersection test of the RTX platform (Sec.~\ref{sec:RT-Hausdorff}) to compute the Hausdorff distance between two objects.
We introduce a novel Ray-Tracing-based Hausdorff DISTance algorithm, RT-HDIST.
It begins by generating GAS for the two penetration surfaces, $P_{\partial A}$ and $P_{\partial B}$, derived from the preceding step.
RT-HDIST processes the GAS of a penetration surface (e.g., $GAS_{\partial A}$) alongside the point set of the other penetration surface (e.g., $P_{\partial B}$) to compute the penetration depth between them.
The algorithm operates bidirectionally, considering both directions ($\partial A \to \partial B$ and $\partial B \to \partial A$).
The final penetration depth between the two objects, A and B, is determined by selecting the larger value from these two directional computations.

%In the Hausdorff distance calculation step, we compute the Hausdorff distance between given two objects using a ray-surface-intersection test. (Sec.~\ref{sec:RT-Hausdorff}) Initially, we construct the GAS for both $\partial A$ and $\partial B$ to utilize the RT-core effectively. The RT-based Hausdorff distance algorithms then determine the Hausdorff distance by processing the GAS of one object (e.g. $GAS_{\partial A}$) and set of the vertices of the other (e.g. $P_{\partial B}$). Following the Hausdorff distance definition (Eq.~\ref{equation:hausdorff_definition}), we compute the Hausdorff distance to both directions ($\partial A \to \partial B$) and ($\partial B \to \partial A$). As a result, the bigger one is the final Hausdorff distance, and also it is the penetration depth between input object $A$ and $B$.


%the proposed RT-based penetration depth calculation pipeline.
%Our proposed methods adopt Tang's Hausdorff-based penetration depth methods~\cite{SIG09HIST}. The pipeline is divided into the penetration surface extraction step and the Hausdorff distance calculation between the penetration surface steps. However, since Tang's approach is not suitable for the RT platform in detail, we modified and applied it with appropriate methods.

%The penetration surface extraction step is extracting overlapped surfaces on other objects. To utilize the RT core, we use the ray-intersection-based PIP(Point-In-Polygon) algorithms instead of collision detection between two objects which Tang et al.~\cite{SIG09HIST} used. (Sec.~\ref{sec:RT-PIP})
%RT core-based PIP test uses a ray-surface intersection test. For purpose this, we generate the GAS(Geometry Acceleration Structure) for each object. RT core-based PIP test takes the GAS of one object (e.g. $GAS_{A}$) and a set of vertex of another one (e.g. $P_{B}$). Then this computes the penetrated vertex set of another one (e.g. $P_{\partial B}$). To calculate the Hausdorff distance, these vertex sets change to objects constructed by penetrated surface (e.g. $\partial B$). Finally, the two generated overlapped surface objects $\partial A$ and $\partial B$ are used in the Hausdorff distance calculation step.

Our goal is to increase the robustness of T2I models, particularly with rare or unseen concepts, which they struggle to generate. To do so, we investigate a retrieval-augmented generation approach, through which we dynamically select images that can provide the model with missing visual cues. Importantly, we focus on models that were not trained for RAG, and show that existing image conditioning tools can be leveraged to support RAG post-hoc.
As depicted in \cref{fig:overview}, given a text prompt and a T2I generative model, we start by generating an image with the given prompt. Then, we query a VLM with the image, and ask it to decide if the image matches the prompt. If it does not, we aim to retrieve images representing the concepts that are missing from the image, and provide them as additional context to the model to guide it toward better alignment with the prompt.
In the following sections, we describe our method by answering key questions:
(1) How do we know which images to retrieve? 
(2) How can we retrieve the required images? 
and (3) How can we use the retrieved images for unknown concept generation?
By answering these questions, we achieve our goal of generating new concepts that the model struggles to generate on its own.

\vspace{-3pt}
\subsection{Which images to retrieve?}
The amount of images we can pass to a model is limited, hence we need to decide which images to pass as references to guide the generation of a base model. As T2I models are already capable of generating many concepts successfully, an efficient strategy would be passing only concepts they struggle to generate as references, and not all the concepts in a prompt.
To find the challenging concepts,
we utilize a VLM and apply a step-by-step method, as depicted in the bottom part of \cref{fig:overview}. First, we generate an initial image with a T2I model. Then, we provide the VLM with the initial prompt and image, and ask it if they match. If not, we ask the VLM to identify missing concepts and
focus on content and style, since these are easy to convey through visual cues.
As demonstrated in \cref{tab:ablations}, empirical experiments show that image retrieval from detailed image captions yields better results than retrieval from brief, generic concept descriptions.
Therefore, after identifying the missing concepts, we ask the VLM to suggest detailed image captions for images that describe each of the concepts. 

\vspace{-4pt}
\subsubsection{Error Handling}
\label{subsec:err_hand}

The VLM may sometimes fail to identify the missing concepts in an image, and will respond that it is ``unable to respond''. In these rare cases, we allow up to 3 query repetitions, while increasing the query temperature in each repetition. Increasing the temperature allows for more diverse responses by encouraging the model to sample less probable words.
In most cases, using our suggested step-by-step method yields better results than retrieving images directly from the given prompt (see 
\cref{subsec:ablations}).
However, if the VLM still fails to identify the missing concepts after multiple attempts, we fall back to retrieving images directly from the prompt, as it usually means the VLM does not know what is the meaning of the prompt.

The used prompts can be found in \cref{app:prompts}.
Next, we turn to retrieve images based on the acquired image captions.

\vspace{-3pt}
\subsection{How to retrieve the required images?}

Given $n$ image captions, our goal is to retrieve the images that are most similar to these captions from a dataset. 
To retrieve images matching a given image caption, we compare the caption to all the images in the dataset using a text-image similarity metric and retrieve the top $k$ most similar images.
Text-to-image retrieval is an active research field~\cite{radford2021learning, zhai2023sigmoid, ray2024cola, vendrowinquire}, where no single method is perfect.
Retrieval is especially hard when the dataset does not contain an exact match to the query \cite{biswas2024efficient} or when the task is fine-grained retrieval, that depends on subtle details~\cite{wei2022fine}.
Hence, a common retrieval workflow is to first retrieve image candidates using pre-computed embeddings, and then re-rank the retrieved candidates using a different, often more expensive but accurate, method \cite{vendrowinquire}.
Following this workflow, we experimented with cosine similarity over different embeddings, and with multiple re-ranking methods of reference candidates.
Although re-ranking sometimes yields better results compared to simply using cosine similarity between CLIP~\cite{radford2021learning} embeddings, the difference was not significant in most of our experiments. Therefore, for simplicity, we use cosine similarity between CLIP embeddings as our similarity metric (see \cref{tab:sim_metrics}, \cref{subsec:ablations} for more details about our experiments with different similarity metrics).

\vspace{-3pt}
\subsection{How to use the retrieved images?}
Putting it all together, after retrieving relevant images, all that is left to do is to use them as context so they are beneficial for the model.
We experimented with two types of models; models that are trained to receive images as input in addition to text and have ICL capabilities (e.g., OmniGen~\cite{xiao2024omnigen}), and T2I models augmented with an image encoder in post-training (e.g., SDXL~\cite{podellsdxl} with IP-adapter~\cite{ye2023ip}).
As the first model type has ICL capabilities, we can supply the retrieved images as examples that it can learn from, by adjusting the original prompt.
Although the second model type lacks true ICL capabilities, it offers image-based control functionalities, which we can leverage for applying RAG over it with our method.
Hence, for both model types, we augment the input prompt to contain a reference of the retrieved images as examples.
Formally, given a prompt $p$, $n$ concepts, and $k$ compatible images for each concept, we use the following template to create a new prompt:
``According to these examples of 
$\mathord{<}c_1\mathord{>:<}img_{1,1}\mathord{>}, ... , \mathord{<}img_{1,k}\mathord{>}, ... , \mathord{<}c_n\mathord{>:<}img_{n,1}\mathord{>}, ... , $
$\mathord{<}img_{n,k}\mathord{>}$,
generate $\mathord{<}p\mathord{>}$'', 
where $c_i$ for $i\in{[1,n]}$ is a compatible image caption of the image $\mathord{<}img_{i,j}\mathord{>},  j\in{[1,k]}$. 

This prompt allows models to learn missing concepts from the images, guiding them to generate the required result. 

\textbf{Personalized Generation}: 
For models that support multiple input images, we can apply our method for personalized generation as well, to generate rare concept combinations with personal concepts. In this case, we use one image for personal content, and 1+ other reference images for missing concepts. For example, given an image of a specific cat, we can generate diverse images of it, ranging from a mug featuring the cat to a lego of it or atypical situations like the cat writing code or teaching a classroom of dogs (\cref{fig:personalization}).
\vspace{-2pt}
\begin{figure}[htp]
  \centering
   \includegraphics[width=\linewidth]{Assets/personalization.pdf}
   \caption{\textbf{Personalized generation example.}
   \emph{ImageRAG} can work in parallel with personalization methods and enhance their capabilities. For example, although OmniGen can generate images of a subject based on an image, it struggles to generate some concepts. Using references retrieved by our method, it can generate the required result.
}
   \label{fig:personalization}\vspace{-10pt}
\end{figure}

\section{Analyses}\label{chap2/sec:analyses}

We provide insights on the method for IVF-PQ, considering the effects of quantization and space partitioning.
For an image $I$ whose representation is $x=f(I)\in\R^d$, 
$\hat{x}$ denotes the representation of a transformed version: $\hat{x} = f(t(I))\in\R^d$, and $x^\acti$ the representation of the activated image $I^\acti$.
For details on the images and the implementation used in the experimental validations, see Sec. \ref{chap2/sec:experimental}.

\begin{figure}[b!]
   \begin{minipage}{0.45\textwidth}
        \centering
        \vspace{3pt}
        \includegraphics[width=1.05\linewidth, trim={0 0.8em 0 0em},clip]{chapter-2/figs/exp/prc.pdf}
        \caption{Precision-Recall curve for ICD with 50k queries and 1M reference images (more details for the experimental setup in Sec. \ref{chap2/sec:experimental}). $p_\mathrm{f}^{\mathrm{ivf}}$ is the probability of failure of the IVF (Sec. \ref{chap2/sec:space_partitioning}).
        }
        \label{chap2/fig:prc}
   \end{minipage}\hfill
      \begin{minipage}{0.51\textwidth}
        \centering
        \includegraphics[width=0.8\textwidth, trim={0 0.15cm 0 0.23cm}, clip]{chapter-2/figs/exp/distances.pdf}
        \caption{
            Distance estimates histograms (sec. \ref{chap2/sec:quantization}).
            With active indexing, $\|x- q(x)\|^2$ is reduced ({\color{blue}$\leftarrow$}), inducing a shift ({\color{orange}$\leftarrow$}) in the distribution of $\|\hat{x}- q(x)\|^2$, where $t(I)$ a hue-shifted version of $I$.
            $y$ is a random query.
        }
        \label{chap2/fig:dists}
   \end{minipage}
\end{figure}


\subsection{Product quantization: impact on distance estimate}\label{chap2/sec:quantization}

We start by analyzing the distance estimate considered by the index:
\begin{equation}
    \|\hat{x}-q(x)\|^2 = \|x-q(x)\|^2 + \|\hat{x}-x\|^2
    + 2 (x-q(x))^\top (\hat{x}-x).
    \label{eq:distance}
\end{equation}
The activation aims to reduce the first term, \ie, the quantization error $\|x- q(x)\|^2$, which in turn reduces $\|\hat{x}- q(x)\|^2$. 
Figure~\ref{chap2/fig:dists} shows in blue the empirical distributions of $\|x- q(x)\|^2$ (passive) and $\|x^\acti- q(x)\|^2$ (activated). 
As expected the latter has a lower mean, but also a stronger variance.
The variation of the following factors may explain this: \emph{i)} the strength of the perturbation (due to the HVS modeled by $H_{\mathrm{JND}}$ in~\eqref{eq:scaling}), \emph{ii)} the sensitivity of the feature extractor $\|\nabla_x f(x)\|$ (some features are easier to push than others), \emph{iii)} the shapes and sizes of the Voronoï cells of PQ.

The second term of \eqref{eq:distance} models the impact of the image transformation in the feature space. 
Comparing the orange and blue distributions in Fig.~\ref{chap2/fig:dists}, we see that it has a positive mean, but the shift is bigger for activated images.
We can assume that the third term has null expectation for two reasons: \emph{i)} the noise $\hat{x}-x$ is independent of $q(x)$ and centered around 0, \emph{ii)} in the high definition regime, quantification noise $x-q(x)$ is independent of $x$ and centered on 0. 
Thus, this term only increases the variance. 
Since $x^\acti-q(x)$ has smaller norm, this increase is smaller for activated images.

All in all, $\|\hat{x}^\acti-q(x)\|^2$ has a lower mean but a stronger variance than its passive counterpart $\|\hat{x}-q(x)\|^2$.
Nevertheless, the decrease of the mean is so large that it compensates the larger variance. 
The orange distribution in active indexing is further away from the green distribution for negative pairs, \ie, the distance between an indexed vector $q(x)$ and an independent query $y$.


\subsection[Space partitioning: impact on the IVF probability of failure]{Space partitioning: impact on the IVF \\probability of failure}\label{chap2/sec:space_partitioning}

We denote by $p_\mathrm{f} := \mathbb{P}(\qivf(x) \neq \qivf (\hat{x}) )$ the probability that $\hat{x}$ is assigned to a wrong bucket by IVF assignment $\qivf$.
In the single-probe search ($k'=1$), the recall (probability that a pair is detected when it is a true match, for a given threshold $\tau$ on the distance) is upper-bounded by $1 - p_\mathrm{f}$:
\begin{align}
   R_\tau 
   = \mathbb{P} \left(\{ \qivf(\hat{x}) = \qivf(x) \} \cap \{ \| \hat{x}-q(x)\| < \tau \} \right) 
   \leq \mathbb{P} \left(\{ \qivf(\hat{x}) = \qivf(x) \}  \right) 
   =1 - p_\mathrm{f}.
\end{align}
In other terms, even with a high threshold $\tau \rightarrow \infty$ (and low precision), the detection misses representations that ought to be matched, with probability $p_\mathrm{f}$. It explains the sharp drop at recall $R=0.13$ in Fig.~\ref{chap2/fig:prc}.
This is why it is crucial to decrease $p_\mathrm{f}$.
The effect of active indexing is to reduce $\|\hat{x}-\qivf(x)\|$
therefore reducing $p_\mathrm{f}$ and increasing the upper-bound for $R$:
the drop shifts towards $R=0.32$. 

This explanation suggests that pushing $x$ towards $\qivf(x)$ decreases even more efficiently $p_\mathrm{f}$. 
This makes the IVF more robust to transformation but this may jeopardize the PQ search because features of activated images are packed altogether.
In a way, our strategy, which pushes $x$ towards $q(x)$, dispatches the improvement over the IVF and the PQ search.






\section{Results \& analysis}
Our fine-tuned models set a new state of the art for <10B models on long-context reasoning benchmarks while also improving baseline performance on related narrative understanding tasks. Below, we discuss our results in detail.
% Further analysis reveals room for improvement, as the models benefit more from chapter-level claims rather than book-level claims.


\begin{table*}[htbp] 
    \centering 
    \scriptsize 
    \resizebox{\textwidth}{!}{
        \begin{tabular}{p{\textwidth}} 
        \toprule 
        \cellcolor{gray!10}\textbf{Claim:} "Jacques consistently uses pastoral themes to express his love for Belle-bouche, first through metaphorical discussions of shepherds and shepherdesses, and later while walking through an orchard and discussing Arcadian lovers." \\
        \midrule 
        \end{tabular}
    }
    \resizebox{\textwidth}{!}{
        \begin{tabular}[t]{p{0.4\textwidth} p{0.6\textwidth}} 
        \parbox[t]{0.4\textwidth}{%
        \textbf{\llamainst's CoT:} ``The statement claims that Jacques consistently uses pastoral themes to express his love for Belle-bouche. To evaluate this statement, we need to examine the text and identify instances where Jacques uses pastoral themes to express his love for Belle-bouche. {\color{purple}The text mentions} that Jacques discusses crooks, flowers, and shepherdesses with Belle-bouche, and later, while walking through an orchard, he talks about Arcadian lovers, specifically mentioning Strephon, Chloe, Corydon, Daphnis, and Narcissus. These instances suggest that Jacques does use pastoral themes to express his love for Belle-bouche. However, we need to consider whether this is a consistent pattern throughout the text." \newline \textit{Answer: TRUE}%
        } & 
        \parbox[t]{0.6\textwidth}{%
        \textbf{\llamaftbalanced's CoT:} ``Here are the relevant details from the text: \\ \vspace{-0.12in}
        \begin{itemize}[leftmargin=*]
            \item {\color{teal}Chapter 2}: Jacques engages in a metaphorical discussion about shepherds and shepherdesses, expressing his love for Belle-bouche.
            \item {\color{teal}Chapter 17}: Jacques and Belle-bouche walk through the orchard, discussing the twentieth of May as a day for lovers.
            \item Relationship: Recurring theme of pastoral love
        \end{itemize}
        In {\color{teal}Chapter 2}, Jacques engages in a metaphorical discussion about shepherds and shepherdesses to express his love for Belle-bouche. This theme continues in {\color{teal}Chapter 17}, where Jacques and Belle-bouche walk through an orchard, discussing the twentieth of May as a day for lovers. This connection shows Jacques' consistent use of pastoral imagery to convey his romantic feelings for Belle-bouche throughout the story." \newline \textit{Answer: TRUE}%
        \vspace{2pt}
        } \\
        \bottomrule
        \end{tabular}
    }
    \caption{Comparison of CoT responses produced by \llamainst\ and \llamaftbalanced\ for a claim from John Esten Cooke's \textit{The Youth of Jefferson}. \llamaftbalanced's CoT response contains more explicit chapter references (highlighed in {\color{teal}green}) and details compared to that of \llamainst\ (highlighted in {\color{purple}red}).}
    \label{tab:cot-examples} 
    \vspace{-0.2in} 
\end{table*}


\subsection{\pipeline\ models outperform baselines on narrative claim verification} \label{subsec:main_results}
% \mi{you may want to split this into one para on your test set and one on nocha, each with headers}

% \yapei{todo: address the prolong base issue}
% \mi{more descriptive header!}
% \paragraph{Fine-tuning on our data improves performance on \pipeline-test:} 
On \pipeline-test, our fine-tuned models significantly outperform the instruct models they are initialized from (referred to as baselines),\footnote{\prolongftbalanced\ is initialized from \prolongbase\ instead of \prolonginst. However, since performing evaluation intended for instruct models on a continually pretrained model may not be ideal, we exclude \prolongbase's results from Table \ref{tab:main-result}. As shown in Table \ref{tab:prolong-base-acc}, \prolongbase\ performs significantly worse than \prolonginst\ on \pipeline-test.} as shown in Table \ref{tab:main-result}. 
% This improvement, while expected, is notable in its magnitude.
For example, \qwenftbalanced\ achieves over a 20\% performance gain compared to \qweninst, while \llamaftbalanced\ sees nearly triple the performance of \llamainst. These substantial improvements demonstrate the effectiveness of \pipeline-generated data.

% \mi{same here!}
\paragraph{Fine-tuning on our data improves performance on NoCha:} A similar trend is observed on NoCha. The performance improvements range from an 8\% gain for strong baselines like \qweninst\ to a dramatic twofold increase for weaker baselines such as \llamainst\ and \prolonginst. It is worth noting that all three baseline models initially perform below the random chance baseline of 25\%, but our fine-tuned models consistently surpass this threshold. 

\paragraph{Performance gap between \pipeline-test and NoCha:} We note that the performance gap between NoCha and \pipeline-test\ is likely due to the nature of the events involved in the claims. While \pipeline-test\ consists of synthetic claims derived from events in model-generated outlines, NoCha’s human-written claims may involve reasoning about low-level details that may not typically appear in such generated outlines. Future work could incorporate more low-level events into chapter outlines to create a more diverse set of claims.
%We hope future work will explore synthetic data generation strategies that can help models improve more on complex reasoning tasks like NoCha.
% \mi{add sentence on implication for future work!}

%\yapei{but aren't chapter level claims also about details?},
% On both NoCha and our test set, our models significantly outperform their respective baseline models (Table \ref{tab:main-result}).\mi{i dont think this terminology is easy to understand. maybe write instead that our fine-tuned models outperform the instruct models that they are initialized to? this is also not surprising so you may want to state that.} The performance gains on our test set vary: \qwenftbalanced\ improves by over 20\%, while other fine-tuned models nearly triple their baseline performance by more than 40\%. We observe a similar trend on NoCha, with improvements as small as 8\% for already strong baselines like \qweninst, and as large as a twofold increase for weaker baselines such as \llamainst\ and \prolonginst. Notably, all baseline models initially perform below the random chance baseline of 25\%, but after fine-tuning, they consistently surpass this threshold. We note that the performance gap between NoCha and our test set is likely due to the nature of the events involved in the claims. \pipeline\ contains synthetic claims constructed with major events in the outline, which might make verification more straightforward. In contrast, NoCha's human-written claims contain lower-level plot details, which might be more challenging for LLMs.
%\mi{but we argue that clipper can generate claims about low-level events... maybe say nocha includes reasoning over things that wouldnt make it into an outline in the first place?}

% \mi{rewrite header, very confusing}\chau{is this better?\mi{how about something like Finetuning on our dataset also improves other narrative-related tasks}}
\subsection{Fine-tuning on \pipeline\ improves on other narrative reasoning tasks}  Beyond long-context reasoning, our models also show improvements in narrative understanding and short-context reasoning tasks. On NarrativeQA, which requires comprehension of movie scripts or full books, our best-performing models, \llamaftbalanced\ and \prolongftbalanced, achieve a 2\% and 5\% absolute improvement over their respective baselines. Similarly, on MuSR, a short-form reasoning benchmark, our strongest model, \qwenft, achieves 45.2\% accuracy, surpassing the 41.2\% baseline. However, these improvements are not consistent across all tasks. On $\infty$Bench QA, only \qwenftbalanced\ outperforms the baseline by approximately 7\%. In contrast, \llamaftbalanced\ and \prolongftbalanced\ show slight performance declines of up to 4\%. Thus, while fine-tuning on \pipeline\ data improves performance on reasoning and some aspects of narrative understanding, its transferability is not universal across domains.


% \mi{emphasize that it doesnt improve it THAT much compared to our data}
\subsection{Short-context claim data is less helpful}
% \yapei{can refer back to 3.1 and mention that for our task, training on long data is more effective than training on short data, which contradicts prev findings. then highlight importance of good long data.}
Contrary to prior studies suggesting short-form data benefits long-context tasks \cite{dubey2024llama, gao2024trainlongcontextlanguagemodels} more than long data, our results show otherwise. While \prolongwp, trained on short data, outperforms baselines, it underperforms across all four long-context benchmarks compared to models fine-tuned on our data. This underscores the need for high-quality long-context data generation pipelines like \pipeline.
% outperforms our three baseline models on \dataname-test (60.4\%), NoCha (24.1\%), and MuSR (45.2\%). However, when comparing to our fine-tuned models, 
% While strong performance on MuSR is expected given the benchmark's focus on short-form reasoning, the fact that short-form reasoning also improves performance on other long-context tasks is particularly interesting. This suggests that our training data format, which features detailed reasoning chains on relevant events and their relationships, contributes meaningfully to model improvement.

\subsection{Finetuning on CoTs results in more informative explanations}
We evaluate the groundedness of CoT reasoning generated by our fine-tuned models using DeepSeek-R1-Distill-Llama-70B (\S\ref{data:cot_validation}). Here, a reasoning chain is counted as grounded when every plot event in the chain can be found in the chapter outline that it cites. Table \ref{tab:cot-groundedness} shows that fine-tuning significantly improves groundedness across all models, with \prolongftbalanced\ achieving the highest rate (80.6\%), followed closely by \llamaftbalanced\ (75.9\%). Looking closer at the content of the explanations (Table \ref{tab:cot-examples}), the baseline model (\llamainst) often gives a generic response without citing any evidence, whereas \llamaftbalanced\ explicitly references Chapter 9 and specifies the cause-and-effect relationship.





\begin{table*}[htbp]
\centering
\footnotesize
\scalebox{0.87}{
\begin{tabular}{p{0.1\textwidth}p{0.06\textwidth}p{0.42\textwidth}p{0.42\textwidth}}
\toprule
\multicolumn{1}{c}{\textsc{Category}} & \multicolumn{1}{c}{\textsc{Freq (\%)}} & \multicolumn{1}{c}{\textsc{True Claim}} & \multicolumn{1}{c}{\textsc{False Claim}} \\
\midrule
Event & 43.2 & The Polaris unit, initially assigned to test a new audio transmitter on Tara, explores the planet's surface {\color{teal}using a jet boat without landing}. & The Polaris unit, initially assigned to test a new audio transmitter on Tara, explores the planet's surface by {\color{purple}landing their spaceship}. \\
\midrule
Person & 31.6 & The cattle herd stolen from Yeager by masked rustlers is later found in {\color{teal}General Pasquale}'s possession at Noche Buena. & The cattle herd stolen from Yeager by masked rustlers is later found in {\color{purple}Harrison}'s possession at Noche Buena. \\
\midrule
Object & 15.8 & The alien structure Ross enters contains both a chamber with {\color{teal}a jelly-like bed} and {\color{teal}a control panel capable of communicating with other alien vessels}. & The alien structure Ross enters contains both a chamber with {\color{purple}a metal bed} and {\color{purple}a control panel capable of time travel}. \\
\midrule
Location & 13.7 & Costigan rescues Clio twice: first from Roger on his planetoid, and later from a {\color{teal}Nevian city} using a stolen space-speedster. & Costigan rescues Clio twice: first from Roger on his planetoid, and later from a {\color{purple}Triplanetary city} using a stolen space-speedster. \\
\midrule
Time & 6.3 & Jean Briggerland's meeting with ex-convicts Mr. Hoggins and Mr. Talmot, where she suggests a burglary target, {\color{teal}follows} a failed attempt on Lydia's life involving a speeding car on the sidewalk. & Jean Briggerland's meeting with ex-convicts Mr. Hoggins and Mr. Talmot, where she suggests a burglary target, {\color{purple}precedes} a failed attempt on Lydia's life involving a speeding car on the sidewalk. \\
\midrule
Affect & 4.2 & David Mullins, who initially expresses {\color{teal}skepticism} about Chester's hiring, later fires Chester on false pretenses and immediately replaces him with Felix. & David Mullins, who initially expresses {\color{purple}enthusiasm} about Chester's hiring, later fires Chester on false pretenses and immediately replaces him with Felix. \\
\bottomrule
\end{tabular}}
\caption{A taxonomy of perturbations that cause a false claim to be misclassified as true. Corresponding details in true and false claims are respectively highlighed in {\color{teal}green} and {\color{purple}red}. Frequencies do not sum to 100\%, as some claims receive multiple labels. Category definitions and more detailed analysis in \S\ref{appendix:error-analysis}.}
\label{tab:error-analysis-dist}
\vspace{-0.2in}
\end{table*}

% \begin{table*}[htbp] 
%     \centering 
%     \scriptsize 
%     \resizebox{\textwidth}{!}{
%         \begin{tabular}{p{0.95\textwidth}} 
%         \toprule 
%         \cellcolor{gray!5}\textbf{Claim:} "Jacques consistently uses pastoral themes to express his love for Belle-bouche, first through metaphorical discussions of shepherds and shepherdesses, and later while walking through an orchard and discussing Arcadian lovers." \\
%         \midrule 
%         \textbf{\llamainst's CoT:} \\ 
%         ``The statement claims that Jacques consistently uses pastoral themes to express his love for Belle-bouche. To evaluate this statement, we need to examine the text and identify instances where Jacques uses pastoral themes to express his love for Belle-bouche. The text mentions that Jacques discusses crooks, flowers, and shepherdesses with Belle-bouche, and later, while walking through an orchard, he talks about Arcadian lovers, specifically mentioning Strephon, Chloe, Corydon, Daphnis, and Narcissus. These instances suggest that Jacques does use pastoral themes to express his love for Belle-bouche. However, we need to consider whether this is a consistent pattern throughout the text." \\
%         \textit{Answer: TRUE} \\ 
%         \midrule 
%         \textbf{\llamaftbalanced's CoT:} \\
%         ``Here are the relevant details from the text: \\ \vspace{-2pt} \begin{itemize} \item Chapter 2: Jacques engages in a metaphorical discussion about shepherds and shepherdesses, expressing his love for Belle-bouche. \item Chapter 17: Jacques and Belle-bouche walk through the orchard, discussing the twentieth of May as a day for lovers. \item Relationship: Recurring theme of pastoral love \end{itemize} In Chapter 2, Jacques engages in a metaphorical discussion about shepherds and shepherdesses to express his love for Belle-bouche. This theme continues in Chapter 17, where Jacques and Belle-bouche walk through an orchard, discussing the twentieth of May as a day for lovers. This connection shows Jacques' consistent use of pastoral imagery to convey his romantic feelings for Belle-bouche throughout the story." \\ 
%         \textit{Answer: TRUE} \\ 
%     \bottomrule 
%     \end{tabular} 
%     } 
%     \caption{Comparison of CoT responses produced by \llamainst\ and \llamaftbalanced\ for a claim from John Esten Cooke's \textit{The Youth of Jefferson}.} 
%     \label{tab:cot-examples} 
%     \vspace{-0.2in} 
% \end{table*}



% \mi{header is too informal}
\subsection{Small models struggle with book-level reasoning} 
\label{subsection:chap-book-ft}
Trained only on 8K chapter-level claims, \prolongftchapter\ outperforms \prolongftbook\ on both chapter- and book-level test subsets (Table \ref{tab:chapter_vs_book}). This likely reflects the limitations of smaller models (7B/8B) in handling the complex reasoning required for book-level claims, aligning with prior findings \cite{qi2024quantifyinggeneralizationcomplexitylarge}. The performance gap between the models is modest (4.2\%), and we leave exploration of larger models (>70B) to future work due to compute constraints.
% Although larger models (>70B) might be able to effectively learn the complex reasoning patterns in these multi-chapter claims, we leave this for future work due to limited compute resources. 


\subsection{Fine-tuned models have a difficult time verifying False claims} \label{sec:error-analysis}
% \mi{this can def be heavily shortened / go to appendix, the table itself is sufficient along with a couple sentences}
To study cases where fine-tuned models struggle, we analyze \qwenftbalanced\ outputs. Among 1,000 book-level claim pairs in \pipeline-test, the model fails to verify 37 true claims and 97 false claims, aligning with NoCha findings \cite{karpinska_one_2024} that models struggle more with false claims. We investigate perturbations that make false claims appear true and present a taxonomy with examples in Table \ref{tab:error-analysis-dist}, with further details in \S\ref{appendix:error-analysis}.
% Notably, in 95 cases, the model successfully validates the true claim but fails to validate the corresponding false claim. This raises an important question: \textit{What specific perturbations make a false claim appear true to the model?} Through careful manual analysis, we derive a taxonomy of such perturbations and present them in Table \ref{tab:error-analysis-dist}. The most frequent perturbations are changes to events (43.2\%) and people (31.6\%), such as altering actions or misattributing roles. Less frequent but notable are modifications to objects (15.8\%), locations (13.7\%), time (6.3\%), and affect (4.2\%). All these perturbations introduce plausible-sounding variations that the model may struggle to detect without fully understanding the narrative.\footnote{We provide definitions for each category in Appendix \ref{appendix:error-analysis}} 
%A closer examination of the chain-of-thoughts generated for these 95 claims reveals some recurring patterns: the model often fabricates evidence, applies incorrect reasoning, or completely ignores the perturbed details. Specific examples can be found in Appendix X. \yapei{do we need this part on CoT?}

\section{Conclusion }
This paper introduces the Latent Radiance Field (LRF), which to our knowledge, is the first work to construct radiance field representations directly in the 2D latent space for 3D reconstruction. We present a novel framework for incorporating 3D awareness into 2D representation learning, featuring a correspondence-aware autoencoding method and a VAE-Radiance Field (VAE-RF) alignment strategy to bridge the domain gap between the 2D latent space and the natural 3D space, thereby significantly enhancing the visual quality of our LRF.
Future work will focus on incorporating our method with more compact 3D representations, efficient NVS, few-shot NVS in latent space, as well as exploring its application with potential 3D latent diffusion models.

