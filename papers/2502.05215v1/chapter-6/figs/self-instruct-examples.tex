\begin{figure}[b!]
    \centering
    \begin{tcolorbox}[colframe=metablue, colback=white]
        \scriptsize
        \textbf{Prompt given to Llama-2-chat:} \newline
You are asked to come up with a set of 20 diverse task instructions. These instructions will be given to large language model and we will evaluate it for completing the instructions. \newline

Here are the requirements:\newline
1. Try not to repeat the verb for each instruction to maximize diversity.\newline
2. The language used for the instruction also should be diverse. For example, you should combine questions with imperative instructions.\newline
3. The type of instructions should be diverse. The list should include diverse types of tasks like open-ended generation, classification, editing, etc.\newline
2. A language model should be able to complete the instruction. For example, do not ask the assistant to create any visual or audio output. For another example, do not ask the assistant to wake you up at 5pm or set a reminder because it cannot perform any action.\newline
3. The instructions should be in English.\newline
4. The instructions should be 1 to 2 sentences long. Either an imperative sentence or a question is permitted.\newline
5. You should generate an appropriate input to the instruction. It should contain a specific example for the instruction when needed.\newline
6. Most of the instructions do not require input. In this case, you must simply put "None" in the input field.\newline
7. The input should not be the answer to the instruction. For example, if the instruction is "Write a headline for ...", the input should not be the said headline. Instead, the input should be "None".\newline
8. The output should be an appropriate response to the instruction and the input. Make sure the output is less than 100 words.\newline
9. Do not repeat instructions.\newline

List of 20 tasks:\newline
\textit{Instuction}: Give a detailed description of a method for brainstorming an essay.\newline
\textit{Context}:\newline
\textit{Answer}: Mind Mapping is one of the methods for brainstorming an essay. Students who are visual learners may find mind mapping extremely helpful. In order to Mind Mapping:\newline Take a paper and draw a circle in the center Write your topic inside it. Just like we sketch rays protruding from a sun, draw lines coming out of the circle.  Write an argument or a statement related to the topic at the end of each line. Draw another set of lines projecting from these statements and state examples that support them. Continue this exercise until you run out of all possibilities imaginable for this topic.
\\[8pt]

\textbf{Llama-2 answer - with WM \cite{kirchenbauer2023watermark}, $\delta=3$, $k=2$:}


\text{\textit{Instuction}: Create a poem based on the provided word.}\newline
\text{\textit{Context}: Word: "Nature".}\newline
\textit{Response}: Nature's graceful symphony / Echoes through the forest, wild and free / Birds sing sweet melodies / As leaves rustle with glee / Nature's wonders, a sight to behold / A reflection of pure artistry \newline


\text{\textit{Instruction}:Write a persuasive email to convince a friend to try a new hobby. \textit{Context:``''}}\newline
\textit{Response}: Dear [Friend's Name],/ I hope this email finds you well. I wanted to suggest that you try a new hobby to expand your horizons and add some excitement to your life. [Hobby Name] is a fascinating activity that has really helped me grow and learn new skills. Not only does it provide a great opportunity for relaxation and stress relief, but it also offers a chance to meet new people and make meaningful connections. I think you would really enjoy it and I would love to see you give it a shot. Let me know if you have any questions or if you want to learn more.\newline


    \end{tcolorbox}
    \caption{
        Example of
        instruction data using Self-instruct. 
        We show the prompt which is given to Llama-2-chat-7B, and the completion with the watermarking method used in Sec.~\ref{chap6/sec:instruction}.
        We do not observe a noticeable decrease in quality compared to the original completion.
    }
    \label{chap6/fig:self-instruct-examples}
\end{figure}
