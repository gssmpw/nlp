
\chapter{Watermarking Makes Language Models Radioactive}\label{chapter:radioactive}


This chapter is based on the paper \fullcite{sander2024watermarking}.

It is challenging for AI companies to ensure that their models -- which cost thousands if not millions -- are used in compliance with licensing agreements and prevent theft, because their complexity and lack of transparency make it difficult to track their usage. 
In this chapter, we investigate the \emph{radioactivity} of text generated by large language models (LLM), \ie,  whether it is possible to detect that such synthetic input was used as training data.
Current methods like membership inference or active IP protection either work only in confined settings (\eg, where the suspected text is known) or do not provide reliable statistical guarantees.
We discover that, on the contrary, LLM watermarking as presented in the previous Chap.~\ref{chapter:three-bricks} allows for reliable identification of whether the outputs of a watermarked LLM were used to fine-tune another language model.
Our new methods, specialized for radioactivity, detects with confidence weak residuals of the watermark signal in the fine-tuned LLM.
We link the radioactivity contamination level to the following properties: the watermark robustness, its proportion in the training set, and the fine-tuning process.
We notably demonstrate that training on watermarked synthetic instructions can be detected with high confidence (\pval\ $< 10^{-5}$) even when as little as $5\%$ of training text is watermarked.
Code is available at \url{github.com/facebookresearch/radioactive-watermark}.

\newpage
%!TEX root = gcn.tex
\section{Introduction}
Graphs, representing structural data and topology, are widely used across various domains, such as social networks and merchandising transactions.
Graph convolutional networks (GCN)~\cite{iclr/KipfW17} have significantly enhanced model training on these interconnected nodes.
However, these graphs often contain sensitive information that should not be leaked to untrusted parties.
For example, companies may analyze sensitive demographic and behavioral data about users for applications ranging from targeted advertising to personalized medicine.
Given the data-centric nature and analytical power of GCN training, addressing these privacy concerns is imperative.

Secure multi-party computation (MPC)~\cite{crypto/ChaumDG87,crypto/ChenC06,eurocrypt/CiampiRSW22} is a critical tool for privacy-preserving machine learning, enabling mutually distrustful parties to collaboratively train models with privacy protection over inputs and (intermediate) computations.
While research advances (\eg,~\cite{ccs/RatheeRKCGRS20,uss/NgC21,sp21/TanKTW,uss/WatsonWP22,icml/Keller022,ccs/ABY318,folkerts2023redsec}) support secure training on convolutional neural networks (CNNs) efficiently, private GCN training with MPC over graphs remains challenging.

Graph convolutional layers in GCNs involve multiplications with a (normalized) adjacency matrix containing $\numedge$ non-zero values in a $\numnode \times \numnode$ matrix for a graph with $\numnode$ nodes and $\numedge$ edges.
The graphs are typically sparse but large.
One could use the standard Beaver-triple-based protocol to securely perform these sparse matrix multiplications by treating graph convolution as ordinary dense matrix multiplication.
However, this approach incurs $O(\numnode^2)$ communication and memory costs due to computations on irrelevant nodes.
%
Integrating existing cryptographic advances, the initial effort of SecGNN~\cite{tsc/WangZJ23,nips/RanXLWQW23} requires heavy communication or computational overhead.
Recently, CoGNN~\cite{ccs/ZouLSLXX24} optimizes the overhead in terms of  horizontal data partitioning, proposing a semi-honest secure framework.
Research for secure GCN over vertical data  remains nascent.

Current MPC studies, for GCN or not, have primarily targeted settings where participants own different data samples, \ie, horizontally partitioned data~\cite{ccs/ZouLSLXX24}.
MPC specialized for scenarios where parties hold different types of features~\cite{tkde/LiuKZPHYOZY24,icml/CastigliaZ0KBP23,nips/Wang0ZLWL23} is rare.
This paper studies $2$-party secure GCN training for these vertical partition cases, where one party holds private graph topology (\eg, edges) while the other owns private node features.
For instance, LinkedIn holds private social relationships between users, while banks own users' private bank statements.
Such real-world graph structures underpin the relevance of our focus.
To our knowledge, no prior work tackles secure GCN training in this context, which is crucial for cross-silo collaboration.


To realize secure GCN over vertically split data, we tailor MPC protocols for sparse graph convolution, which fundamentally involves sparse (adjacency) matrix multiplication.
Recent studies have begun exploring MPC protocols for sparse matrix multiplication (SMM).
ROOM~\cite{ccs/SchoppmannG0P19}, a seminal work on SMM, requires foreknowledge of sparsity types: whether the input matrices are row-sparse or column-sparse.
Unfortunately, GCN typically trains on graphs with arbitrary sparsity, where nodes have varying degrees and no specific sparsity constraints.
Moreover, the adjacency matrix in GCN often contains a self-loop operation represented by adding the identity matrix, which is neither row- nor column-sparse.
Araki~\etal~\cite{ccs/Araki0OPRT21} avoid this limitation in their scalable, secure graph analysis work, yet it does not cover vertical partition.

% and related primitives
To bridge this gap, we propose a secure sparse matrix multiplication protocol, \osmm, achieving \emph{accurate, efficient, and secure GCN training over vertical data} for the first time.

\subsection{New Techniques for Sparse Matrices}
The cost of evaluating a GCN layer is dominated by SMM in the form of $\adjmat\feamat$, where $\adjmat$ is a sparse adjacency matrix of a (directed) graph $\graph$ and $\feamat$ is a dense matrix of node features.
For unrelated nodes, which often constitute a substantial portion, the element-wise products $0\cdot x$ are always zero.
Our efficient MPC design 
avoids unnecessary secure computation over unrelated nodes by focusing on computing non-zero results while concealing the sparse topology.
We achieve this~by:
1) decomposing the sparse matrix $\adjmat$ into a product of matrices (\S\ref{sec::sgc}), including permutation and binary diagonal matrices, that can \emph{faithfully} represent the original graph topology;
2) devising specialized protocols (\S\ref{sec::smm_protocol}) for efficiently multiplying the structured matrices while hiding sparsity topology.


 
\subsubsection{Sparse Matrix Decomposition}
We decompose adjacency matrix $\adjmat$ of $\graph$ into two bipartite graphs: one represented by sparse matrix $\adjout$, linking the out-degree nodes to edges, the other 
by sparse matrix $\adjin$,
linking edges to in-degree nodes.

%\ie, we decompose $\adjmat$ into $\adjout \adjin$, where $\adjout$ and $\adjin$ are sparse matrices representing these connections.
%linking out-degree nodes to edges and edges to in-degree nodes of $\graph$, respectively.

We then permute the columns of $\adjout$ and the rows of $\adjin$ so that the permuted matrices $\adjout'$ and $\adjin'$ have non-zero positions with \emph{monotonically non-decreasing} row and column indices.
A permutation $\sigma$ is used to preserve the edge topology, leading to an initial decomposition of $\adjmat = \adjout'\sigma \adjin'$.
This is further refined into a sequence of \emph{linear transformations}, 
which can be efficiently computed by our MPC protocols for 
\emph{oblivious permutation}
%($\Pi_{\ssp}$) 
and \emph{oblivious selection-multiplication}.
% ($\Pi_\SM$)
\iffalse
Our approach leverages bipartite graph representation and the monotonicity of non-zero positions to decompose a general sparse matrix into linear transformations, enhancing the efficiency of our MPC protocols.
\fi
Our decomposition approach is not limited to GCNs but also general~SMM 
by 
%simply 
treating them 
as adjacency matrices.
%of a graph.
%Since any sparse matrix can be viewed 

%allowing the same technique to be applied.

 
\subsubsection{New Protocols for Linear Transformations}
\emph{Oblivious permutation} (OP) is a two-party protocol taking a private permutation $\sigma$ and a private vector $\xvec$ from the two parties, respectively, and generating a secret share $\l\sigma \xvec\r$ between them.
Our OP protocol employs correlated randomnesses generated in an input-independent offline phase to mask $\sigma$ and $\xvec$ for secure computations on intermediate results, requiring only $1$ round in the online phase (\cf, $\ge 2$ in previous works~\cite{ccs/AsharovHIKNPTT22, ccs/Araki0OPRT21}).

Another crucial two-party protocol in our work is \emph{oblivious selection-multiplication} (OSM).
It takes a private bit~$s$ from a party and secret share $\l x\r$ of an arithmetic number~$x$ owned by the two parties as input and generates secret share $\l sx\r$.
%between them.
%Like our OP protocol, o
Our $1$-round OSM protocol also uses pre-computed randomnesses to mask $s$ and $x$.
%for secure computations.
Compared to the Beaver-triple-based~\cite{crypto/Beaver91a} and oblivious-transfer (OT)-based approaches~\cite{pkc/Tzeng02}, our protocol saves ${\sim}50\%$ of online communication while having the same offline communication and round complexities.

By decomposing the sparse matrix into linear transformations and applying our specialized protocols, our \osmm protocol
%($\prosmm$) 
reduces the complexity of evaluating $\numnode \times \numnode$ sparse matrices with $\numedge$ non-zero values from $O(\numnode^2)$ to $O(\numedge)$.

%(\S\ref{sec::secgcn})
\subsection{\cgnn: Secure GCN made Efficient}
Supported by our new sparsity techniques, we build \cgnn, 
a two-party computation (2PC) framework for GCN inference and training over vertical
%ly split
data.
Our contributions include:

1) We are the first to explore sparsity over vertically split, secret-shared data in MPC, enabling decompositions of sparse matrices with arbitrary sparsity and isolating computations that can be performed in plaintext without sacrificing privacy.

2) We propose two efficient $2$PC primitives for OP and OSM, both optimally single-round.
Combined with our sparse matrix decomposition approach, our \osmm protocol ($\prosmm$) achieves constant-round communication costs of $O(\numedge)$, reducing memory requirements and avoiding out-of-memory errors for large matrices.
In practice, it saves $99\%+$ communication
%(Table~\ref{table:comm_smm}) 
and reduces ${\sim}72\%$ memory usage over large $(5000\times5000)$ matrices compared with using Beaver triples.
%(Table~\ref{table:mem_smm_sparse}) ${\sim}16\%$-

3) We build an end-to-end secure GCN framework for inference and training over vertically split data, maintaining accuracy on par with plaintext computations.
We will open-source our evaluation code for research and deployment.

To evaluate the performance of $\cgnn$, we conducted extensive experiments over three standard graph datasets (Cora~\cite{aim/SenNBGGE08}, Citeseer~\cite{dl/GilesBL98}, and Pubmed~\cite{ijcnlp/DernoncourtL17}),
reporting communication, memory usage, accuracy, and running time under varying network conditions, along with an ablation study with or without \osmm.
Below, we highlight our key achievements.

\textit{Communication (\S\ref{sec::comm_compare_gcn}).}
$\cgnn$ saves communication by $50$-$80\%$.
(\cf,~CoGNN~\cite{ccs/KotiKPG24}, OblivGNN~\cite{uss/XuL0AYY24}).

\textit{Memory usage (\S\ref{sec::smmmemory}).}
\cgnn alleviates out-of-memory problems of using %the standard 
Beaver-triples~\cite{crypto/Beaver91a} for large datasets.

\textit{Accuracy (\S\ref{sec::acc_compare_gcn}).}
$\cgnn$ achieves inference and training accuracy comparable to plaintext counterparts.
%training accuracy $\{76\%$, $65.1\%$, $75.2\%\}$ comparable to $\{75.7\%$, $65.4\%$, $74.5\%\}$ in plaintext.

{\textit{Computational efficiency (\S\ref{sec::time_net}).}} 
%If the network is worse in bandwidth and better in latency, $\cgnn$ shows more benefits.
$\cgnn$ is faster by $6$-$45\%$ in inference and $28$-$95\%$ in training across various networks and excels in narrow-bandwidth and low-latency~ones.

{\textit{Impact of \osmm (\S\ref{sec:ablation}).}}
Our \osmm protocol shows a $10$-$42\times$ speed-up for $5000\times 5000$ matrices and saves $10$-2$1\%$ memory for ``small'' datasets and up to $90\%$+ for larger ones.

\section{Background}
\label{sec:background}


\subsection{Code Review Automation}
Code review is a widely adopted practice among software developers where a reviewer examines changes submitted in a pull request \cite{hong2022commentfinder, ben2024improving, siow2020core}. If the pull request is not approved, the reviewer must describe the issues or improvements required, providing constructive feedback and identifying potential issues. This step involves review commment generation, which play a key role in the review process by generating review comments for a given code difference. These comments can be descriptive, offering detailed explanations of the issues, or actionable, suggesting specific solutions to address the problems identified \cite{ben2024improving}.


Various approaches have been explored to automate the code review comments process  \cite{tufano2023automating, tufano2024code, yang2024survey}. 
Early efforts centered on knowledge-based systems, which are designed to detect common issues in code. Although these traditional tools provide some support to programmers, they often fall short in addressing complex scenarios encountered during code reviews \cite{dehaerne2022code}. More recently, with advancements in deep learning, researchers have shifted their focus toward using large-language models to enhance the effectiveness of code issue detection and code review comment generation.

\subsection{Knowledge-based Code Review Comments Automation}

Knowledge-based systems (KBS) are software applications designed to emulate human expertise in specific domains by using a collection of rules, logic, and expert knowledge. KBS often consist of facts, rules, an explanation facility, and knowledge acquisition. In the context of software development, these systems are used to analyze the source code, identifying issues such as coding standard violations, bugs, and inefficiencies~\cite{singh2017evaluating, delaitre2015evaluating, ayewah2008using, habchi2018adopting}. By applying a vast set of predefined rules and best practices, they provide automated feedback and recommendations to developers. Tools such as FindBugs \cite{findBugs}, PMD \cite{pmd}, Checkstyle \cite{checkstyle}, and SonarQube \cite{sonarqube} are prominent examples of knowledge-based systems in code analysis, often referred to as static analyzers. These tools have been utilized since the early 1960s, initially to optimize compiler operations, and have since expanded to include debugging tools and software development frameworks \cite{stefanovic2020static, beller2016analyzing}.



\subsection{LLMs-based Code Review Comments Automation}
As the field of machine learning in software engineering evolves, researchers are increasingly leveraging machine learning (ML) and deep learning (DL) techniques to automate the generation of review comments \cite{li2022automating, tufano2022using, balachandran2013reducing, siow2020core, li2022auger, hong2022commentfinder}. Large language models (LLMs) are large-scale Transformer models, which are distinguished by their large number of parameters and extensive pre-training on diverse datasets.  Recently, LLMs have made substantial progress and have been applied across a broad spectrum of domains. Within the software engineering field, LLMs can be categorized into two main types: unified language models and code-specific models, each serving distinct purposes \cite{lu2023llama}.

Code-specific LLMs, such as CodeGen \cite{nijkamp2022codegen}, StarCoder \cite{li2023starcoder} and CodeLlama \cite{roziere2023code} are optimized to excel in code comprehension, code generation, and other programming-related tasks. These specialized models are increasingly utilized in code review activities to detect potential issues, suggest improvements, and automate review comments \cite{yang2024survey, lu2023llama}. 




\subsection{Retrieval-Augmented Generation}
Retrieval-Augmented Generation (RAG) is a general paradigm that enhances LLMs outputs by including relevant information retrieved from external databases into the input prompt \cite{gao2023retrieval}. Traditional LLMs generate responses based solely on the extensive data used in pre-training, which can result in limitations, especially when it comes to domain-specific, time-sensitive, or highly specialized information. RAG addresses these limitations by dynamically retrieving pertinent external knowledge, expanding the model's informational scope and allowing it to generate responses that are more accurate, up-to-date, and contextually relevant \cite{arslan2024business}. 

To build an effective end-to-end RAG pipeline, the system must first establish a comprehensive knowledge base. It requires a retrieval model that captures the semantic meaning of presented data, ensuring relevant information is retrieved. Finally, a capable LLM integrates this retrieved knowledge to generate accurate and coherent results \cite{ibtasham2024towards}.




\subsection{LLM as a Judge Mechanism}

LLM evaluators, often referred to as LLM-as-a-Judge, have gained significant attention due to their ability to align closely with human evaluators' judgments \cite{zhu2023judgelm, shi2024judging}. Their adaptability and scalability make them highly suitable for handling an increasing volume of evaluative tasks. 

Recent studies have shown that certain LLMs, such as Llama-3 70B and GPT-4 Turbo, exhibit strong alignment with human evaluators, making them promising candidates for automated judgment tasks \cite{thakur2024judging}

To enable such evaluations, a proper benchmarking system should be set up with specific components: \emph{prompt design}, which clearly instructs the LLM to evaluate based on a given metric, such as accuracy, relevance, or coherence; \emph{response presentation}, guiding the LLM to present its verdicts in a structured format; and \emph{scoring}, enabling the LLM to assign a score according to a predefined scale \cite{ibtasham2024towards}. Additionally, this evaluation system can be enriched with the ability to explain reasoning behind verdicts, which is a significant advantage of LLM-based evaluation \cite{zheng2023judging}. The LLM can outline the criteria it used to reach its judgment, offering deeper insights into its decision-making process.






\section{Problem formulation}

\emph{Alice} owns a language model $\A$, fine-tuned for specific tasks such as chatting, problem solving, or code generation, which is available through an API (\autoref{chap6/fig:fig1}).
\emph{Bob} owns another language model $\B$.
Alice suspects that Bob fine-tuned $\B$ on some outputs from $\A$.
We denote by $D$ the dataset used to fine-tune $\B$, among which $D^{\A}\subset D$ is made of outputs from $\A$, in proportion $\rho = |D^{\A}|/|D|$.

\paragraph*{Access to Bob's data.} 
\label{chap6/sec:degreeofsupervision}

We consider two settings for Alice's knowledge about Bob's training data:
\begin{itemize}[leftmargin=0.5cm, itemsep=2pt, topsep=1pt]

    \item \emph{supervised}: Bob queries $\A$ and  Alice retains all the content $\Tilde{D}^{\A}$ that $\A$ generated for Bob. Thus, Alice knows that $D^{\A} \subseteq \Tilde{D}^{\A}$. 
    We define the \emph{degree of supervision} $d := |D^{\A}|/|\Tilde{D}^{\A}|$,  

    \item \emph{unsupervised}: Bob does not use any identifiable account or is hiding behind others such that $|\Tilde{D}^{\A}| \gg |D^{\A}|$ and $d \approx 0$.
    This is the most realistic scenario.
\end{itemize}

Thus, $\rho$ is the proportion of Bob's fine-tuning data which originates from Alice's model
while $d$ quantifies Alice's knowledge regarding the dataset that Bob may have utilized (see Fig.~\ref{chap6/fig:datasets}).

\paragraph*{Access to Bob's model.} 
We consider two scenarios:
\begin{itemize}[leftmargin=0.5cm, itemsep=2pt, topsep=1pt]
    \item Alice has an \emph{open-model} access to $\B$. 
    She can forward any inputs through $\B$ and observe the output logits.
    This is the case if Bob open-sources $\B$, or if Alice sought it via legitimate channels.
    \item Alice has a \emph{closed-model} access. 
    She can only query $\B$ through an API without logits access: Alice only observes the generated texts.
    This would be the case for most chatbots.
\end{itemize}

We then introduce two definitions of radioactivity:
\begin{definition}[Text Radioactivity]\label{chap6/def:text_radioactivity}
    Dataset $D$ is $\alpha$-radioactive for a statistical test $T$ if ``$\B$ was not trained on $D$'' $\subset \H_0$ and
    $T$ is able to reject $\H_0$ at a significance level (\pval) smaller than $\alpha$.
\end{definition}

\begin{definition}[Model Radioactivity]\label{chap6/def:model_radioactivity}
    Model $\A$ is $\alpha$-radioactive for a statistical test $T$ if
    ``$\B$ was not trained on outputs of $\A$'' $\subset \H_0$ and $T$ is able to reject $\H_0$ at a significance level smaller than $\alpha$.
\end{definition}

Thus, $\alpha$ quantifies the radioactivity of a dataset or model. 
A low $\alpha$, e.g. $10^{-6}$, indicates strong radioactivity: the probability of observing a result as extreme as the one observed, assuming that Bob's model was not trained on Alice's outputs, is 1 out of one million. 
Conversely, $\alpha\approx 0.5$ means that the observed result is equally likely under both the null and alternative (radioactive) hypotheses.



\begin{figure}[b]
   \begin{minipage}{0.62\textwidth}
        \centering
        \resizebox{1.0\linewidth}{!}{
        \begin{tabular}{r ccc ccc ccc}
            \toprule
            & \multicolumn{2}{c}{With WM} & \multicolumn{2}{c}{MI} & \multicolumn{2}{c}{IPP} \\
            \cmidrule(lr){2-3} \cmidrule(lr){4-5} \cmidrule(lr){6-7}
            & Open & Closed & Open & Closed & Open & Closed \\
            Supervised & \cmarkg & \cmarkg &\cmarkg & \xmarkg & \cmarkg & \amark \\
            Unsupervised & \cmarkg & \cmarkg &\xmarkg & \xmarkg & \xmarkg & \xmarkg \\
            \bottomrule
        \end{tabular}
        }
        \captionof{table}{
            Availability of radioactivity detection under the different settings: 
            \cmarkg: available, \xmarkg: not available, \amark: available but with strong limitations.
            \textit{Open} / \textit{closed-model} refers to the availability of Bob's model, and \textit{supervised} / \textit{unsupervised} to Alice's knowledge of his data.
            Detection with watermarks is described in Sec.~\ref{chap6/sec:radioactivity_detection}, and other approaches relying on Membership Inference (MI) and Intellectual Property Protection (IPP) are detailed in Sec.~\ref{chap6/sec:discussion-other-approaches}.
        }
        \label{chap6/tab:summary_MIA_wm}
   \end{minipage}\hfill
    \begin{minipage}{0.34\textwidth}
        \centering
        \includegraphics[width=1.0\textwidth, clip, trim=0.7cm 2.2cm 0.7cm 0]{chapter-6/figs/datasets.pdf}
        \captionsetup{font=small}
        \caption{
            Detection performance mainly depends on $\rho = |D^{\A}|/|D|$ and $d = |D^{\A}|/|\Tilde{D}^{\A}|$, where $D$ is the fine-tuning dataset used by Bob, $\Tilde{D}^{\A}$ are the outputs from Alice's model, and $D^{\A}$ the intersection of both.
        }
        \label{chap6/fig:datasets}
    \end{minipage}
\end{figure}













\section{Radioactivity detection}\label{chap6/sec:radioactivity_detection}


\subsection{Why current approaches are insufficient}

\paragraph{Membership inference and IPP methods.} 

Passive methods relying on membership inference observe the model perplexity on text belonging to the training dataset, compared to text not in the dataset.
They are effective only in the \textit{supervised} setting, where Alice has precise knowledge of the data used to train Bob's model and \textit{open} access to it. 
In that scenario, she can easily demonstrate $\alpha$-radioactivity, with $\alpha$ as low as $10^{-30}$, providing strong evidence that Bob has trained on her model.

In parallel, there are active methods, such as ones used for IPP, that explicitly modify the LLM generation to detect when the outputs are used as training data.
However, even state-of-the-art methods~\citep{zhao2023protecting, he2022protecting, he2022cater} fall short in the unsupervised setting, where the statistical guarantees do not hold in practice.

These claims are detailed and supported by experiments in Sec.~\ref{chap6/par:mia_wm} (for MIA) and Sec.~\ref{chap6/sec:ipp} (for IPP), and summarized in Tab.~\ref{chap6/tab:summary_MIA_wm}.
At the end of the day, the applicability of the methods is very restricted, and does not generalize to real-world scenarios.

\paragraph{Naive approach for watermark detection.} 
We now assume that the outputs of $\A$ are watermarked with a method $W$ with Alice's secret key $\sk$ as described in Sec.~\ref{chap6/sec:background}. 
The original watermark detector $T$ tests the null hypothesis $\mathcal{H}_0$: ``\textit{The text was not generated following $W$ with secret key $\sk$}'' by applying a scoring function $W_{\textrm{score}}$ keyed by $\sk$ on the text.
The output \pval\ depends on the score and the number of analyzed tokens.

Unlike watermark detection which takes text as input, the input for radioactivity detection is a model.
The naive approach to detect model radioactivity on $\B$ is to run test $T$ on a large corpus of texts generated by $\B$. 
This aligns with Def.~\ref{chap6/def:text_radioactivity}:
the text cannot be generated following $W$ and $\sk$ if $\B$ has never seen the watermark, so if ``\textit{$\B$ did not use outputs of $\A$}'', then $\H_0$ is true.
However, this detector is ineffective because 
(1) \emph{the watermark signal is very weak}, as radioactivity can only be observed for watermarked $(k+1)$-grams $\{$watermark window + current token$\}$ that were part of $\A$'s outputs in $\B$'s training data, therefore the signal is diluted in the generated text, and;
(2) \emph{theoretical \pval s break down} when computed naively from many tokens -- $\approx$ 1M, see the experiments of Sec.~\ref{chap6/par:dedup-expe} -- which is necessary to observe this very weak signal.

\subsection{Enhanced radioactivity detection}

\paragraph{Overview.} 
We use the detection test $T$ from text watermarking as presented in Sec.~\ref{chap6/sec:background}, but adapt the score computation (\autoref{chap6/fig:method}).
Our goal is to reduce noise by focusing Bob's model on watermarked windows likely to be radioactive. 
To this end, we recreate a context similar to the one that generated the watermarked text by Alice. 
We ensure the accuracy of statistical tests through de-duplication of scored tokens. 

\begin{figure*}[b!]
    \centering
    \includegraphics[width=1.0\textwidth, clip, trim=0 1.5cm 3.7cm 0]{chapter-6/figs/method.pdf}
    \caption{
    Radioactivity detection with closed or open model access (for simplicity, only \citep{kirchenbauer2023watermark} is illustrated).
    \textit{(Left)} New texts are generated from $\B$ using prompts from $\A$ and these texts are scored. 
    The filter $\phi$ is used to focus the score computation on likely contaminated $k$-grams. 
    \textit{(Right}) Texts generated by $\A$ are directly forwarded through $\B$, and the next-token predictions are scored using tokens from the input as the watermark window.
    In both cases, the \textit{tape} ensures reliable \pval s by de-duplicating scored tokens.
    }
    \label{chap6/fig:method}\label{chap6/fig:open_model}
\end{figure*}


\paragraph{Radioactivity detection in $\B$.}
To amplify the radioactivity signal, we employ two strategies.
(1) In the supervised setting, we use watermarked text from $\Tilde{D}^\A$. In the unsupervised setting, we use new watermarked text generated by $\A$ that aligns with the suspected training distribution of $\B$ (e.g., English dialogues).
(2) We score up to millions of tokens, orders of magnitudes more than usual.
The scoring depends on the access to $\B$:
\begin{itemize}[leftmargin=*, itemsep=0pt, topsep=0pt]
    \item \emph{closed-model}: we use the prompts to generate new texts from $\B$, and score these texts.
    \item \emph{open-model}, \aka, ``reading mode'': instead of generating completions with $\B$, we directly forward the texts generated by $\A$ through $\B$, as depicted in Fig.~\ref{chap6/fig:open_model}.
    We then score next-token predictions with $W_{\textrm{score}}$ by using tokens from the input as watermark window.
    Intuitively, it reproduces the right contexts, and allows us to study how $\B$ behaves on these watermarked windows rather than letting $\B$ generate tokens without any interesting signal.
\end{itemize}


\paragraph{Filter on scored $k$-grams.}
To further improve detection in the closed-model setting where the reading mode is not possible, we only score $(k+1)$-grams $\{$watermark window + current token$\}$ output by $\B$ for which the watermark window is often in $\A$’s watermarked outputs.
We thus introduce a filter $\phi$, a set that contains these watermark windows.
In the \emph{supervised} setting ($0<d\leq1$), $\phi$ is made of the $k$-grams present in $\Tilde{D}^\mathcal{A}$ (refer to Fig.~\ref{chap6/fig:datasets}).
In the \emph{unsupervised} setting, we focus on `likely' contaminated $k$-grams, \eg, $k$-grams appearing in (new) watermarked text generated by $\A$.

\paragraph{Token scoring and de-duplication.}
\autoref{chapter:three-bricks} demonstrates that detection tests can be empirically inaccurate due to biases in the natural distribution of tokens. 
This issue is more pronounced in our case, given the larger volume of tokens required for observing radioactivity. 
To mitigate this, we score a token only if the same $\{$watermark window + current token$\}$ combination has not been previously encountered.
Moreover, in the closed-model setting, we only score watermark windows ($k$-gram) that are not part of the (watermarked) prompt. 
In the open-model setting, tokens with watermarked windows previously present in the attention span are not scored. 
This is achieved by maintaining a \emph{tape} memory of all such $k$-grams combinations during detection.
These adjustments ensure reliable \pval s even when many tokens are analyzed (see Sec.~\ref{chap6/par:dedup-expe} and~\ref{chap6/app:correctness}).










\section{Experiments}
\seclabel{experiments}
Our experiments are designed to test a) the extent to which open loop execution is an issue for precise mobile manipulation tasks, b) how effective are blind proprioceptive correction techniques, c) do object detectors and point trackers perform reliably enough in wrist camera images for reliable control, d) is occlusion by the end-effector an issue and how effectively can it be mitigated through the use of video in-painting models, and e) how does our proposed \name methodology compare to large-scale imitation learning? 


\subsection{Tasks and Experimental Setup}
We work with the Stretch RE2 robot. Stretch RE2 is a commodity mobile manipulator with a 5DOF arm mounted on top of a non-holomonic base. We upgrade the robot to use the Dex Wrist 3, which has an eye-in-hand RGB-D camera (Intel D405). 
We consider 3 task families for a total
of 6 different tasks: a) holding a knob to pull open a cabinet or drawer, b) holding a
handle to pull open a cabinet, and c) pushing on objects (light buttons, books
in a book shelf, and light switches). Our focus is on generalization. {\it
Therefore, we exclusively test on previously unseen instances, not used during
development in any way.} 
\figref{tasks} shows the instances that we test on. 

All tasks involve some precise manipulation, followed by execution of a motion
primitive. {\bf For the pushing tasks}, the precise motion is to get the
end-effector exactly at the indicated point and the motion primitive is to push
in the direction perpendicular to the surface and retract the end-effector 
upon contact. The robot is positioned such
that the target position is within the field of view of the wrist camera. A user
selects the point of pushing via a mouse click on the wrist camera image. The
goal is to push at the indicated location. Success is determined by whether the
push results in the desired outcome (light turns on / off or book gets pushed in). 
The original rubber gripper bends upon contact, we use a rigid known tool
that sticks out a bit. We take the geometry of the tool into account while servoing.

{\bf For the opening articulated object tasks}, the precise manipulation is grasping the
knob / handle, while the motion primitive is the whole-body motion that opens
the cupboard. Computing and executing this full body motion is difficult. We
adopt the modular approach to opening articulated objects (MOSART) from Gupta \etal~\cite{gupta2024opening} and invoke it
after the gripper has been placed around the knob / handle. The whole tasks 
starts out with the robot about 1.5m way from the target object, with the 
target object in view
from robot's head mounted camera. We use MOSART to compute articulation
parameters and convey the robot to a pre-grasp
location with the target handle in view of the wrist camera. At this point,
\name (or baseline) is used to center the gripper around the knob / handle, 
before resuming MOSART: extending the gripper till contact, close the gripper, and play rest of the predicted motion plan. Success is 
determined by whether the cabinet opens by more than $60^\circ$
or the drawer is pulled out by more than $24cm$, similar to the criteria used in \cite{gupta2024opening}.


For the precise manipulation part, all baselines consume the current and
previous RGB-D images from the wrist camera and output full body motor
commands.

% % Please add the following required packages to your document preamble:
% % \usepackage{graphicx}
% \begin{table*}[!ht]
% \centering
% \caption{}
% \label{tab:my-table}
% \resizebox{\textwidth}{!}{%
% \begin{tabular}{lcccccc}
% \toprule
%  & \multicolumn{2}{c}{ours} & \multicolumn{2}{c}{Gurobi} & \multicolumn{2}{c}{MOSEK} \\
%  & \multicolumn{1}{l}{time (s)} & \multicolumn{1}{l}{optimality gap (\%)} & \multicolumn{1}{l}{time (s)} & \multicolumn{1}{l}{optimality gap (\%)} & \multicolumn{1}{l}{time (s)} & \multicolumn{1}{l}{optimality gap (\%)} \\ \hline
% \begin{tabular}[c]{@{}l@{}}Linear Regression\\ Synthetic \\ (n=16000, p=16000)\end{tabular} & 57 & 0.0 & 3351 & - & 2148 & - \\ \hline
% \begin{tabular}[c]{@{}l@{}}Linear Regression\\ Cancer Drug Response\\ (n=822, p=2300)\end{tabular} & 47 & 0.0 & 1800 & 0.31 & 212 & 0.0 \\ \hline
% \begin{tabular}[c]{@{}l@{}}Logistic Regression\\ Synthetic\\ (n=16000, p=16000)\end{tabular} & 271 & 0.0 & N/A & N/A & 1800 & - \\ \hline
% \begin{tabular}[c]{@{}l@{}}Logistic Regression\\ Dorothea\\ (n=1150, p=91598)\end{tabular} & 62 & 0.0 & N/A & N/A & 600 & 0.0 \\
% \bottomrule
% \end{tabular}%
% }
% \end{table*}

% Please add the following required packages to your document preamble:
% \usepackage{multirow}
% \usepackage{graphicx}
\begin{table*}[]
\centering
\caption{Certifying optimality on large-scale and real-world datasets.}
\vspace{2mm}
\label{tab:my-table}
\resizebox{\textwidth}{!}{%
\begin{tabular}{llcccccc}
\toprule
 &  & \multicolumn{2}{c}{ours} & \multicolumn{2}{c}{Gurobi} & \multicolumn{2}{c}{MOSEK} \\
 &  & time (s) & opt. gap (\%) & time (s) & opt. gap (\%) & time (s) & opt. gap (\%) \\ \hline
\multirow{2}{*}{Linear Regression} & \begin{tabular}[c]{@{}l@{}}synthetic ($k=10, M=2$)\\ (n=16k, p=16k, seed=0)\end{tabular} & 79 & 0.0 & 1800 & - & 1915 & - \\ \cline{2-8}
 & \begin{tabular}[c]{@{}l@{}}Cancer Drug Response ($k=5, M=5$)\\ (n=822, p=2300)\end{tabular} & 41 & 0.0 & 1800 & 0.89 & 188 & 0.0 \\ \hline
\multirow{2}{*}{Logistic Regression} & \begin{tabular}[c]{@{}l@{}}Synthetic ($k=10, M=2$)\\ (n=16k, p=16k, seed=0)\end{tabular} & 626 & 0.0 & N/A & N/A & 2446 & - \\ \cline{2-8}
 & \begin{tabular}[c]{@{}l@{}}DOROTHEA ($k=15, M=2$)\\ (n=1150, p=91598)\end{tabular} & 91 & 0.0 & N/A & N/A & 634 & 0.0 \\
 \bottomrule
\end{tabular}%
}
% \vspace{-3mm}
\end{table*}

\begin{figure*}
\insertW{1.0}{figures/figure_6_cropped_brighten.pdf}
\caption{{\bf Comparison of \name with the open loop (eye-in-hand) baseline} for opening a cabinet with a knob. Slight errors in getting to the target cause the end-effector to slip off, leading to failure for the baseline, where as our method is able to successfully complete the task.}
\figlabel{rollout}
\end{figure*}

\begin{table}
\setlength{\tabcolsep}{8pt}
  \centering
  \resizebox{\linewidth}{!}{
  \begin{tabular}{lcccg}
  \toprule
                              & \multicolumn{2}{c}{\bf Knobs} & \bf Handle & \bf \multirow{2}{*}{\bf Total} \\
                              \cmidrule(lr){2-3} \cmidrule(lr){4-4}
                              & \bf Cabinets & \bf Drawer & \bf Cabinets & \\
  \midrule
  RUM~\cite{etukuru2024robot}  & 0/3    & 1/4         & 1/3         & 2/10 \\
  \name (Ours) & 2/3    & 2/4         & 3/3     &  7/10 \\
  \bottomrule
  \end{tabular}}
  \caption{Comparison of \name \vs RUM~\cite{etukuru2024robot}, a recent large-scale end-to-end imitation learning method trained on 1200 demos for opening cabinets and 525 demos for opening drawers across 40 different environments. Our evaluation spans objects from three environments across two buildings.}
  \tablelabel{rum}
\end{table}

\subsection{Baselines}
We compare against three other methods for the precise manipulation part of
these tasks. 
\subsubsection{Open Loop (Eye-in-Hand)} To assess the precision requirements of
the tasks and to set it in context with the manipulation capabilities of the
robot platform, this baseline uses open loop execution starting from estimates
for the 3D target position from the first wrist camera image.
\subsubsection{MOSART~\cite{gupta2024opening}}
The recent modular system for opening cabinets and drawers~\cite{gupta2024opening}
reports impressive performance with open-loop control (using the head camera from 1.5m away), combined with proprioception-based feedback to 
compensate for errors in perception and control when interacting with handles. 
We test if such correction is also sufficient for interacting with knobs. Note 
that such correction is not possible for the smaller buttons and pliable books.

\subsubsection{\name (no inpainting)} To understand how much of an issue
occlusion due to the end-effector is during manipulation, we ablate the use of
inpainting. %

\subsubsection{Robot Utility Models (RUM)~\cite{etukuru2024robot}}
For the opening articulated object tasks, we also compare to Robot Utility Models (RUM), 
a closed-loop imitation learning method recently proposed by Etukuru et al. \cite{etukuru2024robot}.
RUM is trained on a substantial dataset comprising expert demonstrations, including 
1,200 instances of cabinet opening and 525 of drawer opening, gathered from roughly 
40 different environments.
This dataset stands as the most extensive imitation 
learning dataset for articulated object manipulation to date, establishing RUM as a 
strong baseline for our evaluation.

Similar to our method, we use MOSART to compute articulation
parameters and convey the robot to a pre-grasp location
with the target handle in view of the wrist camera.
One of the assumptions of RUM is a good view of the handle.
To benefit RUM, we try out three different heights of the wrist camera,
and \textit{report the best result for RUM.}

\begin{figure*}
\insertW{1.0}{figures/figure_9_cropped_brighten.pdf}
\caption{{\bf \name \vs open loop (eye-in-hand) baseline for pushing on user-clicked points}. Slight errors in getting to the target cause failure, where as \name successfully turns the lights off. Note the quality of CoTracker's track ({\color{blue} blue dot}).}
\figlabel{rollout_v2}
\end{figure*}

\begin{figure*}
\insertW{1.0}{figures/figure_5_v2_cropped_brighten.pdf}
\caption{{\bf Comparison of \name with and without inpainting}. Erroneous detection without inpainting causes execution to fail, where as with inpainting the target is correctly detected leading to a successful grasp and a successful execution.}
\figlabel{rollouts2}
\end{figure*}


\subsection{Results}
\tableref{results} presents results from our experiments. 
Our training-free approach \name successfully 
solves over 85\% of task instances that we test on.
As noted, all these
tests were conducted on unseen object instances in unseen
environments that were not used for development in any way. We discuss our key
experimental findings below.

\subsubsection{Closing the loop is necessary for these precise tasks} 
While the proprioception-based strategies proposed in MOSART~\cite{gupta2024opening}
work out for handles, they are inadequate for targets like knobs and just
don't work for tasks like pushing buttons. Using estimates from the wrist
camera is better, but open loop execution still fails for knobs and pushing
buttons. 

\subsubsection{Vision models work reasonably well even on wrist camera images}
Inpainting works well on wrist camera images (see \figref{occlusion} and \figref{inpainting}).
Closing the loop using feedback from vision detectors and point trackers on
wrist camera images also work well, particularly when we use in-painted images.
See some examples detections and point tracks in \figref{rollout} and \figref{rollout_v2}. 
Detic~\cite{zhou2022detecting} was able to reliably detect the knobs and
handles and CoTracker~\cite{karaev2023cotracker} was able to successfully track
the point of interaction letting us solve 24/28 task instances.

\subsubsection{Erroneous detections without inpainting hamper performance on 
handles and our end-effector out-painting strategy effectively mitigates it} 
As shown in \figref{rollouts2}, presence of the end-effector caused the object
detector to miss fire leading to failed execution. Our out painting approach
mitigates this issue leading to a higher success rate than the 
approach without out-painting. Interestingly, CoTracker~\cite{karaev2023cotracker} is quite robust
to occlusion (possibly because it tracks multiple points) and doesn't benefit
from in-painting. 


\subsubsection{Closed-loop imitation learning struggles on novel objects}
As presented in \tableref{rum}, \name significantly outperforms RUM in a paired evaluation on unseen objects across three novel environments. A common failure mode of RUM is its inability to grasp the object's handle, even when it approaches it closely.
Another failure mode we observe is RUM misidentifying keyholes or cabinet edges as handles, also resulting in failed grasp attempts.
These result demonstrate that a modular approach that leverages the broad generalization capabilities of vision foundation models is able to generalize much better than an end-to-end imitation learning approach trained on 1000+ demonstrations, which must learn all aspects of the task from scratch.




\section{Investigating radioactivity}\label{chap6/sec:fine-tuning-abl}

\autoref{chap6/sec:instruction} considers detection in a practical scenario.
This section further studies what influences radioactivity from different angles like fine-tuning, watermarking algorithm, and data distribution.


\subsection{Fine-tuning}


\begin{table}
    \caption{
        Influence of the model fine-tuning on the radioactivity.
        We report the $\logpval$ for $10$k scored observations (lower means more radioactive).
        {\setlength{\fboxsep}{2pt}\colorbox[HTML]{F2F2F2}{Gray}} indicates values used in Sec.~\ref{chap6/sec:instruction}.
    }
    \label{chap6/tab:ft-abl}
    \renewcommand{\arraystretch}{1.2}
    {\footnotesize
    \centering
  	\subfloat[ Learning rate.]{
            \begin{minipage}{0.23\linewidth}
                {\begin{center}
                    \begin{tabular}{ccc}
                       \colorcell  $10^{-5}$ & $5\cdot 10^{-5}$ & $10^{-4}$\\
                        \shline
                       \colorcell  -32.4 & -49.6 & -58.0\\
                    \end{tabular}
                \end{center}}
            \end{minipage} 
        } \hspace{0.03\linewidth}
  	\subfloat[ Epoch.]{
            \centering
            \begin{minipage}{0.27\linewidth}
                {\begin{center}
                    \begin{tabular}{ *{4}{c} }
                        1 & 2 & \colorcell 3 & 4 \\
                        \shline
                        -20.8 & -29.2 & \colorcell -33.2 & -34.8 \\
                    \end{tabular}
                \end{center}}
            \end{minipage} 
        } \hspace{0.03\linewidth}
        \subfloat[ Adapters.]{
            \centering
            \begin{minipage}{0.17\linewidth}
                {\begin{center}
                    \begin{tabular}{cc}
                        \colorcell Full & Q-LoRA \\
                        \shline
                        \colorcell -32.4 & -11.0 \\
                    \end{tabular}
                \end{center}}
            \end{minipage} 
        } \hspace{0.03\linewidth}
        \subfloat[ Model size.]{
            \centering
            \begin{minipage}{0.14\linewidth}
                {\begin{center}
                    \begin{tabular}{cc}
                       \colorcell  7B & 13B\\
                        \shline
                       \colorcell -32.4 & -33.2 \\
                    \end{tabular}
                \end{center}}
            \end{minipage} 
        } 
    }
 \end{table}



 





We first study the influence of fine-tuning on the same setup as Sec.~\ref{chap6/sec:instruction}, with regards to: 
(a) the learning rate,
(b) the fine-tuning algorithm,
\eg, with Q-LoRA~\citep{dettmers2023qlora} a widely used method for efficient fine-tuning;
(c) the number of epochs,
(d) the model size.
We fine-tune $\B$ with the same dataset of $\rho=100\%$ watermarked instructions and the same parameters.
We detect radioactivity in the \emph{open-model} / \emph{unsupervised} setting.
This is done on $N=10$k next-predictions, and where the texts that are fed to $\B$ are watermarked instructions generated with $\A$.
\autoref{chap6/tab:ft-abl} reports the results. The more the model fits the data, the easier its radioactivity is to detect.
For instance, multiplying the learning rate by $10$ almost doubles the average $\logpval$ of the test.


\subsection{Bigger teachers}\label{chap6/app:bigger-teachers}

\begin{table}[t!]
    \centering
    \caption{
        Influence of the teacher model size on radioactivity detection.
    }
    \label{chap6/table:results_Teachers}
    \footnotesize
    \begin{tabular}{ *{5}{l} }
        \toprule
        Teacher & Without & 7B & 13B & 65B \\
        \midrule
        NQ & 3.2 & 5.6 & 5.4 & 5.8 \\
        GSM8k & 10.0 & 11.1 & 10.4 & 11.0 \\
        MMLU & 28.4 & 31.0 & 32.9 & 33.8 \\
        \midrule
        $\log_{10}$ p-value & -0.3 & -32.4 & -31.1 & -31.7 \\
        \bottomrule
    \end{tabular}
\end{table}

We conduct the same experiment replacing the Llama-2-chat-7B teacher model by the 13B or 65B versions.
\autoref{chap6/table:results_Teachers} reports the results for benchmarks NQ, GSM8k, and MMLU and the average $\logpval$ of the radioactivity detection test under the same conditions as the previous paragraph.
Our observations align with the previous conclusions: watermarking does not significantly affect the benchmarks (except for MMLU where the improvement appears larger). 
Moreover, the detection of radioactivity is not significantly impacted by the teacher model size.










\subsection{Watermarking method}

\paragraph{Multi-bit scenario.} 

We adopt the same framework as in Sec.~\ref{chap6/sec:instruction}, but we use the watermarking method of \cite{yoo2023advancing}, \aka, MPAC.
It is a multi-bit method, where the watermark is a binary message of size $n$.
More precisely, we take bits 2 by 2 to generate a message $m = m_1 m_2 \ldots m_b$ with the $r$-ary $m_i = 0, 1, 2,$ or $3$, corresponding to $r=4$ and $b = n/2$ with the notations from the original paper.
The method proceeds as the one of \cite{kirchenbauer2023reliability} by altering the logits before generating a token.
However, the hash created from the watermarked window and the key is now used 
(1) to randomly partition the vocabulary into $r$ disjoint sets, 
(2) to select the position $i$ and corresponding $m_i$ that is hidden for this particular token.
A bias $\delta$ is added to the logits of the tokens belonging to the $m_i$-th set.
Given a text under scrutiny, the extraction reverses the process to find which $r$-ary is the most likely for each position $i$ of the message.

We report in Fig.~\ref{chap6/fig:bit-accuracy} the extraction results in the the supervised/closed-model setup.
We filter and deduplicate the tokens as in Sec.~\ref{chap6/sec:instruction}, and plot the observed bit accuracy against the number of scored tokens -- note that since the watermark is a binary message, we now measure the bit accuracy of the extraction, instead of the \pval\ of the detection.
This is done for several lengths of the binary message. 
Every experiment is run $10$ times for different text output by $\B$, which explains the $95\%$ confidence interval in the plots.
We observe as expected that the bit accuracy significantly increases with the proportion of watermarked data in the fine-tuning data, and that the longer the message, the harder it is to extract it.
This suggests that radioactivity still holds in the multi-bit scenario, and could therefore be used to identify a specific version of the model or a specific user from which the data was generated.

\begin{figure}[b!]
    \centering
    \includegraphics[width=0.99\linewidth,clip, trim=0 0 0 0cm]{chapter-6/figs/all_closed_sup.pdf}
    \caption{
        Bit accuracy when watermarked instruction data are generated with MPAC~\citep{yoo2023advancing}, against the number of scored tokens generated by the fine-tuned model.
        This is done under the supervised/closed-model setup, for various lengths ($n$=8, 16, 32) of the message.
    }
    \label{chap6/fig:bit-accuracy}
\end{figure}







\paragraph{Watermark window size.} 
To reduce the experimental requirements for generation, training and detection -- and introduce more variety to the data under study -- we now prompt $\A$=Llama-2-7B with the beginnings of Wikipedia articles in English and generate the next tokens with or without watermarking. 
We then fine-tune $\B$=Llama-1-7B on the natural prompts followed by the generated answers.
The fine-tuning is done in 1000 steps, using batches $8\times2048$ tokens (similarly to Sec.~\ref{chap6/sec:instruction}).
This section fine-tunes $\B$ on $\rho=100\%$ English watermarked texts.

\autoref{chap6/tab:exp_kgram} highlights that the confidence of the detection decreases for larger window sizes $k$ when fixing the \pval\ of the watermark detection of the training texts.
There are two explanations. 
First, for lower $k$, the chances that a $k$-gram repeats in the training data are higher, which increases its memorization.
Second, the number of possible $k$-grams is $|\V|^k$ and therefore increases with $k$, while the number of watermarked tokens is fixed. 
Thus, at detection time, the number of radioactive $k$-grams decreases with increasing $k$, diminishing the test's power. 
This experiment also demonstrates that the methods of \cite{aaronson2023watermarking} and \cite{kirchenbauer2023reliability} behave the same way.

\paragraph{Discussion on other watermarking schemes.}
Our goal is to derive a general method for detecting radioactivity in decoding-based watermarks, rather than testing all watermarking schemes or identifying the most radioactive one. 
We focus on the works of~\citet{aaronson2023watermarking, kirchenbauer2023watermark}, which are representative of the two main families of methods and provide reliable \pval s for detection. These schemes rely on key management using the hash of previous tokens, a mechanism also used by other potentially radioactive schemes~\citep{lee2023wrote, fu2024gumbelsoft}.

Some LLM watermarking schemes do not rely on hashing, such as ``semantic'' watermarks~\citep{liu2023semantic, liu2024adaptive, fu2024watermarking} and those using pre-defined key sequences~\citep{kuditipudi2023robust}. 
However, we do not evaluate the radioactivity of these methods due to the lack of \pval\ computation, a limitation also noted in other studies~\citep{piet2023mark}. 
For example, the complexity of computing \pval s for the scheme in~\citet{kuditipudi2023robust} would be prohibitively expensive, requiring $10^{16}$ times more operations than the schemes presented previously\footnote{
    \citet{kuditipudi2023robust} use the Levenshtein distance with complexity $O(mnk2)$, where $m$ is the number of tokens, $n=256$, and $k=80$ (default parameters from the authors). 
    This results in a complexity approximately $10^6$ times greater than previous schemes. 
    To evaluate \pval s of $10^{-10}$, a Monte Carlo simulation would require running the statistic $10^{10}$ times, a total $10^{16}$ more operations.
}. 
This limitation would also apply to works based on the same key management~\citep{christ2023undetectable, liu2023semantic}.









\subsection{Data distribution}

\begin{table}[t!]
        \centering
        \caption{
            Influence of the target text distribution on detection.
            $\B$ is prompted with beginnings of Wikipedia articles in the corresponding language, and detection is done on generated next tokens. 
            For each language, we score $N=250$k $k$-grams using the \textit{closed-model} setting.
        }
        \label{chap6/tab:exp_language}
        \footnotesize
        \begin{tabular}{ c *{6}{c} }
            Language &  English & French & Spanish & German & Catalan & Combined p-value (Fisher) \\
            \shline
                $\logpval$ & $<$-50 & -7.8 &  -5.7 &  -4.0 &  -2.1 & $<$-50  \\
        \end{tabular}
    \end{table}
    

\paragraph*{Radioactivity detection in different languages.}\label{chap6/par:distrib}
We consider an unsupervised setting where Alice has no prior knowledge about $D^\A$, the data generated with $\A$ used to fine-tune $\B$.
As an example, Alice does not know the language of $D^\A$, which could be Italian, French, Chinese, etc. 
We run the detection on text generated by $\B$, with prompts from Wikipedia in different languages.
The confidence of the test on another language -- that might share very few $k$-grams with $D^\A$ -- can be low, as shown in Tab.~\ref{chap6/tab:exp_language}.

Alice may, however, combine the \pval s of each test with Fisher's method.  
This discriminates against $\mathcal{H}_0$: ``\textit{none of the datasets are radioactive}'', under which the statement ``\textit{Bob did not use any outputs of $\A$}'' falls.
Therefore, the test aligns with our definition of model radioactivity as per definition~\ref{chap6/def:model_radioactivity}.
From Tab.~\ref{chap6/tab:exp_language}, Fisher's method gives a combined \pval\ of $<10^{-50}$. 
Thus, even if Alice is unaware of the specific data distribution generated by $\A$ that Bob may have used to train $\B$ (\eg, problem-solving scenarios), she may still detect radioactivity by combining the significance levels.

\paragraph*{Mixing instruction datasets from different sources.}


\begin{table}[t!]
    \centering
    \begin{minipage}{0.48\textwidth}
        \centering
        \caption{
            Mixing instruction datasets from different sources.
            The fine-tuning is done with the setup presented in Sec.~\ref{chap6/sec:instruction}, with $\rho$=$10\%$ of watermarked data, mixing either with human or synthetic instructions.
        }
        \label{chap6/table:data-sources}
        \footnotesize
        \begin{tabular}{c|c}
            \toprule
            Major data source
            & Average $\logpval$ \\
            \midrule
            Machine & -15 \\
            Human & -32 \\
            \bottomrule
        \end{tabular}
    \end{minipage}
    \hfill
    \begin{minipage}{0.48\textwidth}
        \centering
        \caption{
            Sequential fine-tuning to remove the watermark traces when fine-tuning.
            The first fine-tuning is done with the setup presented in Sec.~\ref{chap6/sec:instruction}, with $\rho$=$10\%$ of watermarked data, and the second on OASST1.
            }
        \label{chap6/table:results_FineTuning}
        \footnotesize
        \begin{tabular}{c|c}
        \toprule
        Second fine-tuning & Average $\logpval$ \\
        \midrule
        \xmarkg & -15 \\
        \cmarkg & -8 \\
        \bottomrule
        \end{tabular}
    \end{minipage}
\end{table}



We conduct the same experiment as in Sec.~\ref{chap6/sec:instruction}, but replace the non-watermarked synthetic instructions by human-generated ones (from the Open Assistant dataset OASST1~\citep{kopf2024openassistant}).
We report in Tab.~\ref{chap6/table:data-sources} the detection results in the open / unsupervised scenario, with $\rho=10\%$ of watermarked data.
Interestingly, with the exact same setting as the one explored in Sec.~\ref{chap6/sec:detection-setup}, the radioactivity signal is stronger in this case.
Our speculation is that this might be due to fewer overlapping sequences of $k$+1-grams between the two distributions.



\subsection{Possible defenses}

We now assume Bob is aware of watermarking radioactivity and tries to remove the watermark traces before, during or after fine-tuning.
For instance, he might attempt to rephrase the watermarked instructions, use a differentially private training, or fine-tune his model on human-generated data -- which we do in the following experiment.
The logic is overall the same as previously pointed out in the fine-tuning ablations.
If the original watermark is weaker or if the fine-tuning overfits less, then radioactivity will be weaker too.
Therefore, the radioactivity detection test will be less powerful, but given a sufficient amount of data, it may still be able to detect traces of the watermark.

\paragraph{Radioactivity ``purification''.}\label{chap6/ref:purification}
We investigate the impact of a second fine-tuning on human-generated data to remove the watermark traces, through the following experiment.
After having trained his model on a mix of watermarked and non-watermarked data, as in Sec.~\ref{chap6/sec:instruction}, Bob fine-tunes his model a second time on human-generated text (from OASST1, as in the previous paragraph), with the same fine-tuning setup.
\autoref{chap6/table:results_FineTuning} shows that the second-fine-tuning divides by $2$ the significance level of the statistical test, although it does not completely remove the watermark traces.




\section{Discussion on other approaches}\label{chap6/sec:discussion-other-approaches}



\subsection{Membership inference}\label{chap6/par:mia_wm}

\paragraph*{Method.}
In the open-model/supervised case, MIA evaluates the radioactivity of one sample/sentence by observing the loss (or perplexity) of $\B$ on carefully selected sets of inputs.
The perplexity is expected to be smaller on samples seen during training.
We extend this idea for our baseline radioactivity detection test of a non-watermarked text corpus.
The corpus of texts is divided into sentences (of 256 tokens) and $\B$'s loss is computed on each sentence. 
We calibrate it with the zlib entropy~\citep{roelofs2017zlib}, as done by \citet{carlini2021extracting} for sample-based MIA. 
The goal of the calibration is to account for the complexity of each sample and separate this from the over-confidence of $\B$. 

We test the null hypothesis $\mathcal{H}_0$: ``\textit{the perplexity of $\B$ on $\Tilde{D}^{\A}$ has the same distribution as the perplexity on new texts generated by $\A$}''.
Indeed, if $\B$ was not fine-tuned on portions of $\Tilde{D}^\A$, then necessarily $\mathcal{H}_0$ is true.
To compare the empirical distributions we use a two-sample Kolmogorov-Smirnov test~\citep{massey1951kolmogorov}. 
Given the two cumulative distributions $F$ and $G$ over loss values, we compute the K-S distance as $d_{\mathrm{KS}}(F,G) = \mathrm{sup}_x |F(x) -G(x)|$.
We reject $\H_0$ if this distance is higher than a threshold, which sets the \pval\ of the test, and conclude that $\Tilde{D}^{\A}$ is radioactive for $\B$.
This is inspired by \citet{sablayrolles2018d}, who perform a similar K-S test in the case of image classification. 
It significantly diverges from the approach of~\citet{shi2023detecting}, which derives an empirical test by looking at the aggregated score from one tail of the distribution. 

\paragraph*{Experimental results}

We proceed as in Sec.~\ref{chap6/sec:radioactivity_detection} for the setup where MIA is achievable:
Alice has an \emph{open-model} access to $\B$ and is aware of all data $\Tilde{D}^\A$ generated for Bob (supervised setting). 
Bob has used a portion $D^\A$ for fine-tuning $\B$, given by the degree of supervision $d$, as defined in Sec.~\ref{chap6/sec:degreeofsupervision}.
We use the K-S test to discriminate between the calibrated perplexity of $\B$ on: $\mathcal D_{(0)}$ containing 5k instruction/answers (cut at 256 tokens) that were not part of $\B$'s fine-tuning; and $\mathcal D_{(d)}$ containing $(1/d)\times$5k instruction/answers from which $5k$ were.
Distribution $\mathcal D_{(d)}$ simulates what happens when Bob generates a lot of data and only fine-tunes on a few.


\definecolor{curve1}{HTML}{8B188B}
\definecolor{curve2}{HTML}{FFA319}
\begin{figure}[b!]
    \centering
    \begin{subfigure}[b]{0.45\textwidth}
        \centering
        \includegraphics[width=\linewidth]{chapter-6/figs/MIA_ft_vs_noft.pdf}
        \caption{
            Distributions of the calibrated loss of $\B$ across two types of distributions generated by $\A$: 
            texts generated by $\A$ outside of $\B$'s fine-tuning data ({\color{curve1} purple}), texts of $\Tilde{D}^{\A}$ of which $d\%$ were used during training ({\color{curve2} orange}).
        }
        \label{chap6/fig:calibrated-loss}
    \end{subfigure}\hfill
    \begin{subfigure}[b]{0.5\textwidth}
        \centering
        \includegraphics[width=\linewidth,clip, trim=0 0 0 0cm]{chapter-6/figs/MIA_vs_WM.pdf}
        \caption{
            We report the \pval s of the {\color{curve1} K-S detection test} (no WM when training) and of the {\color{curve2} WM detection} ($\rho=5\%$ of WM when training) against the degree of supervision $d$ (proportion of Bob's training data known to Alice).
        }
        \label{chap6/fig:mia_vs_wm}
    \end{subfigure}
    \caption{
        Comparative analysis of membership inference and watermarking for radioactivity detection, in the open-model setup.
        (\emph{Left}) MIA aims to detect the difference between the two distributions. 
        It gets harder as $d$ decreases, since the actual fine-tuning data is mixed with texts that Bob did not use.
        (\emph{Right}) Therefore, for low degrees of supervision ($<2\%$), MIA is no longer effective, while WM detection gives \pval s lower than $10^{-5}$.
    }
    \label{chap6/fig:mia-comparative-analysis}
\end{figure}

\autoref{chap6/fig:calibrated-loss} compares the distributions for $d=0$ and $d>0$. 
As $d$ decreases, the data contains more texts that Bob did not fine-tune on, so the difference between the two perplexity distributions is fainter.
The direct consequence is that the detection becomes more challenging.
\autoref{chap6/fig:mia_vs_wm} shows that when $d>2\%$, the test rejects the null hypothesis at a strong significance level: 
$p < 10^{-5}$ implies that when radioactive contamination is detected, the probability of a false positive is $10^{-5}$.
It is random in the edge case $d=0$, the unsupervised setting where Alice lacks knowledge about the data used by Bob. 
In contrast, radioactivity detection on watermarked data succeeds in that setting.









\subsection{IP protection methods}\label{chap6/sec:ipp}

There are active methods that embed watermarking in a text specifically to detect if a model is trained on it~\citep{zhao2023protecting,he2022cater, he2022protecting}. 
Our setup is a bit different because Alice's primary goal is not to protect against model distillation but to make her outputs more identifiable. 
Consequently, ``radioactivity'' is a byproduct.
Although our focus is not on active methods, we discuss three state-of-the-art methods for intellectual property protection and their limitations in our detection setting.

\paragraph{\citet{zhao2023protecting}.}
The goal is to detect if a specific set of answers generated by Alice's model has been used in training. 
There are three main limitations.
The authors design a watermark dependent on the previous prompt (and unique to it). 
Each detection algorithm (2 and 3) assumes that the sample probing data $D$ is from the training data of the suspect model $\B$.
Therefore, the method is only demonstrated in the \textit{supervised} setting.
Moreover, all experiments are performed in the \textit{open}-model setting, where Alice has access to $\B$'s weights, except for Sec.~5.2 ``Watermark detection with text alone'' (still assuming a \textit{supervised} access), where one number is given in that setting.
Finally, it does not rely on a grounded statistical test and requires an empirically set threshold.

\paragraph{\cite{he2022protecting} and \cite{he2022cater}.} 
These methods substitute synonyms during generation to later detect an abnormal proportion of synonyms in the fine-tuned model.
However, the main hypothesis for building the statistical test -- the frequency of synonyms in a natural text is fixed -- fails when scoring a large number of tokens.
Using the \href{https://github.com/xlhex/NLG_api_watermark}{official author's code} on non-watermarked instruction-answers yields extremely low \pval s, making the test unusable at our scale, as shown in \autoref{chap6/table:pvalues_ginsew}.

\begin{table}[t!]
\centering
\caption{\pval s for non-watermarked instruction-answers}
\label{chap6/table:pvalues_ginsew}
\footnotesize
\begin{tabular}{ *{3}{l} }
    \toprule
    Number of lines & Number of characters (in thousands) & \pval \\
    \midrule
    10 & 1.5 & 0.76 \\
    100 & 30.9 & 0.16 \\
    500 & 148.3 & 2.2 $\times 10^{-7}$ \\
    1000 & 333.9 & 8.8 $\times 10^{-13}$ \\
    \bottomrule
\end{tabular}

\end{table}

\section{Details on \pval s}\label{chap6/sec:additional}

This section provides more information on the experimental setup, reporting, and correctness experiments for the tests used in the chapter.

\subsection{Reporting}\label{chap6/app:repporting}

\paragraph{Interpretation of the average $\logpval$.}
Given that \pval s often span various orders of magnitude, we report the average of the $\logpval$ over multiple runs rather than the average of the \pval s themselves. 
We interpret the average $\logpval$ as though it could be directly read as a \pval, although a direct translation to a rigorous statistical \pval\ is not possible.
Fig.~\ref{chap6/fig:box_plot_open} shows box-plots with additional statistics.

\begin{figure}[b!]
    \centering
    \includegraphics[width=0.5\linewidth]{chapter-6/figs/boxplot_open.pdf}
    \captionsetup{font=small}
    \caption{Box plot for the $\logpval$ in the open/unsupervised setting with varying $\rho$, the proportion of watermarked fine-tuning data.}
    \label{chap6/fig:box_plot_open}
\end{figure}


\paragraph{Average over multiple runs.} 
Due to computational constraints, standard deviations for the $\logpval$ are not calculated across multiple models trained on different instruction data for each setting. 
Instead, we generate the same volume of data (14M tokens) in addition to the data used to fine-tune the model. 
In the open-model setting, we run detection on ten distinct chunks of this additional data. 
In the closed-model setting, we prompt the model with ten different chunks of new sentences and score the responses.


\begin{figure}[b!]
    \begin{minipage}{0.48\textwidth}
        \centering
        \includegraphics[width=0.99\linewidth,clip, trim=0 0 0 0cm]{chapter-6/figs/under_H0_closed_final.pdf}
        \caption{
            \pval\ under $\mathcal{H}_0$ with closed-model access.
            We fine-tune $\B$ on non-watermarked instructions and prompt $\B$ with watermarked instructions, scoring the distinct $(k+1)$-grams from the answers, excluding $k$-grams from the instruction. 
        }
        \label{chap6/fig:pvalu_under_H0_closed}
    \end{minipage} \hfill
    \begin{minipage}{0.48\textwidth}
        \centering
        \includegraphics[width=0.99\linewidth,clip, trim=0 0 0 0cm]{chapter-6/figs/under_H0_open.pdf}
        \caption{
            \pval\ under $\mathcal{H}_0$ with open-model access. 
            We fine-tune $\B$ on non-watermarked instructions, perform the open-model detection and score distinct $(k+1)$-grams only on $k$-grams that $\B$ did not previously attend to when generating the token. 
        }
        \label{chap6/fig:pvalu_under_H0}
    \end{minipage}
\end{figure}



\subsection{Correctness experiments}\label{chap6/app:correctness}

We validate the correctness of our statistical tests by examining the null hypothesis $\mathcal{H}_0$, which represents when the model was not fine-tuned on any watermarked data ($\rho=0$).
Our goal is to show that the \pval\ is approximately uniform under $\mathcal{H}_0$, with a mean of $0.5$ and a standard deviation of $\approx 0.28$.
Instead of fine-tuning model $\B$ with various datasets, we vary the hyper-parameters of the detection algorithm at a fixed fine-tuned model to save computation and memory. 
This approach is not sufficient on its own to establish the validity of the test, but it provides some evidence for it.
Here, $\B$ is the model fine-tuned on non-watermarked instructions as described in Sec.~\ref{chap6/sec:instruction}.


\paragraph{Closed-model.}
We prompt $\B$ with $\approx10$k watermarked instructions using the method by~\citet{kirchenbauer2023watermark}, $\delta=3$ and $k=2$, and three different seeds $\mathsf{s}$. 
We score the answers using the proposed de-duplication and repeat this $10$ times on different datasets. 
\autoref{chap6/fig:pvalu_under_H0_closed} shows that the average \pval\ is close to $0.5$ and the standard deviation is close to $0.28$, as expected under $\mathcal{H}_0$.


\paragraph{Open-model.}
We generate text with eight distinct watermarking methods (methods of~\citet{aaronson2023watermarking} and ~\citet{kirchenbauer2023watermark} for $k\in\{1,2,3,4\}$).
We divide each dataset into three and apply the radioactivity detection test on these 24 segments, each containing over 1.5 million tokens. 
We score the distinct $(k+1)$-grams from the answers, excluding $k$-grams from the instruction.
\autoref{chap6/fig:pvalu_under_H0} shows again that the average \pval\ is close to $0.5$ and the standard deviation is close to $0.28$.

\section{Conclusion}
We reveal a tradeoff in robust watermarks: Improved redundancy of watermark information enhances robustness, but increased redundancy raises the risk of watermark leakage. We propose DAPAO attack, a framework that requires only one image for watermark extraction, effectively achieving both watermark removal and spoofing attacks against cutting-edge robust watermarking methods. Our attack reaches an average success rate of 87\% in detection evasion (about 60\% higher than existing evasion attacks) and an average success rate of 85\% in forgery (approximately 51\% higher than current forgery studies). 

