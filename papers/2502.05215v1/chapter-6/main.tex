
\chapter{Watermarking Makes Language Models Radioactive}\label{chapter:radioactive}


This chapter is based on the paper \fullcite{sander2024watermarking}.

It is challenging for AI companies to ensure that their models -- which cost thousands if not millions -- are used in compliance with licensing agreements and prevent theft, because their complexity and lack of transparency make it difficult to track their usage. 
In this chapter, we investigate the \emph{radioactivity} of text generated by large language models (LLM), \ie,  whether it is possible to detect that such synthetic input was used as training data.
Current methods like membership inference or active IP protection either work only in confined settings (\eg, where the suspected text is known) or do not provide reliable statistical guarantees.
We discover that, on the contrary, LLM watermarking as presented in the previous Chap.~\ref{chapter:three-bricks} allows for reliable identification of whether the outputs of a watermarked LLM were used to fine-tune another language model.
Our new methods, specialized for radioactivity, detects with confidence weak residuals of the watermark signal in the fine-tuned LLM.
We link the radioactivity contamination level to the following properties: the watermark robustness, its proportion in the training set, and the fine-tuning process.
We notably demonstrate that training on watermarked synthetic instructions can be detected with high confidence (\pval\ $< 10^{-5}$) even when as little as $5\%$ of training text is watermarked.
Code is available at \url{github.com/facebookresearch/radioactive-watermark}.

\newpage
\section{Introduction}
\label{sec:intro}

Foundational models (FMs)~\cite{zhang2024data, zhou2023comprehensive} have shown remarkable progress in the healthcare domain, enabling professional-like assessment of disease diagnosis, treatment decision-making, and monitoring~\cite{zhang2023text, wang2022medclip, lu2023mi-zero}. 
Examples include LLaVA-Med~\cite{li2023llava}, Med-PaLM Multimodal~\cite{tu2024towards}, and Med-Flamingo~\cite{moor2023med}, have demonstrated their capacity on question answering, medical image analysis, and report generation.
These studies follow a predominant top-down model development strategy that requires upstream developers to collect data and train models for downstream tasks. 
Consequently, the developed model capabilities are heavily dependent on the training data, limiting their generalization performance in diverse clinical scenarios. 
For instance, Med-Gemini~\cite{yang2024advancing} reveals promising general capabilities in report generation while it lags behind state-of-the-art (SoTA) models on classification tasks, especially for out-of-domain applications. 
This indicates that while the generalizability of the foundation model is promising, more solutions are expected to meet the various specialized clinical needs.

To address these challenges, multi-center data centralization becomes essential to enhance model capacity and robustness across varied clinical scenarios~\cite{rajpurkar2022ai}. 
Centralizing distributed data can significantly improve model training and inference performance.
However, the process of medical data storage, transfer, and aggregation among centers requires extra efforts to ensure data security and system interoperability~\cite{bradford2020international}.
Moreover, a growing concern for patient privacy makes large-scale multi-center data sharing particularly challenging. 
While efforts like federated learning~\cite{wen2023survey, li2020review} can achieve good model performance on local data, the need for synchronized system coordination presents significant challenges, as clients are unable to update asynchronously. This limitation greatly restricts the practical capability of such approaches.
As a result, without a flexible collaboration, medical community still struggles to fully utilize the isolated data and local computation resources for comprehensive medical AI model development. 
To address this dilemma, open-source platforms encourage public data sharing and knowledge integration~\cite{markiewicz2021openneuro, zenodo}.
However, these platforms focus solely on raw data sharing while seldom providing collaborative model training or cooperation between different institutions.
Recently, collaborative learning has emerged as a viable approach for enhancing multi-model robustness~\cite{boulemtafes2020review}. 
For instance, software-like model development~\cite{raffel2023building} mimics software engineering practices by introducing structured workflows, enabling merging, version control, and continuous model integration.
Under this design, model ability can be strengthened with incremental knowledge updates similar to the version updating in software development. 

Although collaborative learning provides a multi-model collaboration, two key challenges remain in the leakage of raw data during collaboration~\cite{huang2023lorahub} and the synchronization of multiple collaborators~\cite{mcmahan2017communication} in the medical AI community. It is still challenging to integrate decentralized, privacy-sensitive data across institutions, leading to under-utilized insights and fragmented knowledge sharing~\cite{kaissis2020secure, rajpurkar2022ai, abdullah2021ethics}.
 To address these challenges, inspired by the collaborative software development, we propose \textbf{Med}ical \textbf{Fo}undation Models Me\textbf{rg}ing (\textbf{MedForge}), a cooperative workflow enabling continuously community-driven foundation model (FM) development.
MedForge enables a lightweight manner for individual centers to share their knowledge among multiple centers, minimizing the burden of data transmission and integration while enhancing model robustness.
Meanwhile, MedForge facilitates asynchronous and flexible collaboration, allowing individual centers to continuously update and improve medical FMs without the need for real-time synchronization.
Similar to open-source software development, MedForge incrementally updates medical knowledge and follows a sustainable model development scheme. 
This key design emphasizes a bottom-up construction of a multi-task medical FM, allowing downstream users to collaboratively build, refine, and update the upstream model according to their local resources. Our major contributions of MedForge are as below: 
\begin{enumerate}
    \item[$\bullet$] We introduce a collaborative workflow to promote the merging scheme of open-source software development. Our proposed MedForge allows distributed clinical centers to asynchronously contribute to comprehensive medical model construction while reducing transmitting costs among centers and avoiding the leakage of raw data, thus enhancing the utilization of private resources in the healthcare system. 
    \item[$\bullet$] We propose two effective knowledge-merging strategies for the asynchronous branch contribution. The MedForge-Fusion strategy updates the plugin module parameters of the main model during the merging phase, whereas the MedForge-Mixture strategy integrates the output of the plugin module by memorizing each contributor's coefficient. These strategies make MedForge more flexible and versatile. MedForge-Fusion is friendly to implement, while the MedForge-Mixture offers better performance and robustness.
    \item[$\bullet$]  We comprehensively evaluate model merging strategies to accumulate medical knowledge among multiple branch plugin modules. MedForge yields superior performance on medical classification tasks compared to other collaborative baselines across multiple datasets. We demonstrate the robustness of MedForge by shuffling the task order and evaluating various configurations of plugin modules and dataset distillation methods.
\end{enumerate}



\section{Background} \label{sec:background}

% \subsection{Capture the Flag (CTF) Challenges}

% CTF challenges simulate real-world cyber-attack scenarios and have emerged as a popular medium for practical cybersecurity training, evaluation, and research. These challenges can simulate real-world attack and defense scenarios and thus assist competitors in developing practical skills in areas such as cryptography, binary exploitation, and reverse engineering. 
% Evaluation of autonomous LLM agents works best with jeopardy-style CTF challenges that focus on standalone software that must be compromised \cite{shao2024nyu,pieterse2024friend}.
% The standalone software may be a binary that can be reverse engineered or exploited, encrypted data that can be decrypted, or a web server whose authentication can be bypassed. After successfully compromising the software, a unique ``flag'' string is either found or revealed by the software server.
% The unique flag string is a concrete indicator of the success of a CTF challenge.
% Recent studies use benchmarks of CTF challenges to evaluate LLM agents on their ability to solve complex tasks and demonstrate practical skills in cybersecurity \cite{shao2024nyu,shao2024empirical,abramovich2024enigma, muzsai2024hacksynth, zhang2024cybenchframeworkevaluatingcybersecurity,yang2023language,turtayev2024hacking}
% Platforms like PicoCTF~\cite{picoctf}, TryHackMe~\cite{tryhackme}, CTFTime~\cite{ctftime} and HackTheBox~\cite{hackthebox} have popularized these formats by providing structured challenges for learners at various skill levels.

% Research indicates that CTF challenges can foster cybersecurity expertise and serve as tools for evaluating facility with cybersecurity skills~\cite{chicone2018using}. They are widely used in academia to enhance learning outcomes in cybersecurity education, with studies demonstrating their effectiveness in promoting analytical thinking and teamwork~\cite{hanafi2021ctf,leune2017using,vykopal2020benefits}. Furthermore, the integration of CTF challenges into research environments enables benchmarking of advanced AI systems like LLMs. .

% Yet, challenges in CTF design persist. These include achieving significant performance, preserving context across tasks, and handling complex, dynamic CTFs that rely on multidisciplinary approaches. Implementing strategies to address these issues enhances problem-solving efficiency, enabling more accurate, adaptive, and effective responses to evolving challenges within CTF environments.


% \subsection{Prompt Engineering}
% \subsection{Prompt Engineering for CTF}
% \subsection{LLM Agents}

% As the use of LLMs to solve CFT challenges expands, prompt engineering is becoming a critical technique for enhancing performance. Various methods have been explored to craft prompts that effectively guide LLMs to the solution of complex cybersecurity problems. Each of these solutions have their own unique strengths and limitations.
%\meet{add more references for LLM agents in other domains, like SWE-Agent, also talk about use of function calling}
Text-based LLMs take a text prompt as input from the user, and produce a text output that follows the user prompt.
LLMs have a finite length of text tokens that they can process called the context.
An alternating sequence of user prompts and LLM outputs makes a conversation and is the basis of chat-based LLM interfaces like ChatGPT.
To remove the user from the loop and create autonomous agents, a feedback mechanism is added based on the LLM outputs, so that the LLM can autonomously continue the conversation.
\citet{yang2023intercode} introduce iterative feedback prompting where the LLM is tasked with writing a piece of code, and the code's compilation and execution logs are provided as feedback, which the LLM uses to iteratively refine it's output.
Recent LLMs support function calling, a way to provide a set of actions to the LLM that it may choose to ``call'' as a function.
In this manner, LLM agents can be provided with many ``tools'' such as a command line, web search, file editing, and code execution \cite{wang2024surveyllmagents}, so that they can autonomously perform various tasks like software development \cite{yang2024sweagent}, web browsing \cite{yoran2024assistantbench}, or solve CTF challenges~\cite{shao2024nyu, abramovich2024enigma}.

With access to the command line and file editing tools, LLM agents can autonomously solve many tasks, but they still struggle on complex long-horizon tasks such as CTF challenges that require multiple steps.
Plan-and-solve prompting \cite{wang2023planandsolve} enhances long-term focus of the agent by incorporating a planning phase before iterative execution. This helps agents tackle ambiguous or complex tasks by structured strategies \cite{turtayev2024hacking}.
ReAct (reasoning + action) \cite{yao2022react} combines step-by-step reasoning with action, allowing the agent to adjust dynamically through iterative cycles. ReWOO (Reasoning without Observation) \cite{xu2023rewoo} separates the reasoning process from tool outputs and observations, allowing it to handle multi-step reasoning tasks efficiently while maintaining focus.
The prompting methods in these agents involve static hard-coded templates where environment and task information is filled in.
While static prompts provide straightforward guidance, they often fail to adapt to different problems and complex tasks, limiting their effectiveness.
Auto-prompting~\cite{shin-etal-2020-autoprompt, zhou-etal-2023-revisiting, zhang2023automatic} is a technique to allow the LLM itself to generate a highly-relevant prompt. Auto-prompting invokes more factual responses and reduces hallucinations in LLMs.
D-CIPHER incorporates auto-prompting as a separate agent that can explore the environment and generate a better prompt.
%Based on the given prompt, LLM agents make a decision and proceed further to find flags.  To address this gap, we propose \textbf{dynamic prompting}, where the LLM agent autonomously generates prompts based on the CTF challenge's context and stage.
%include a static template which needs to be given to LLM to solve the CTF challenges. For instance, the NYU CTF framework provides a static prompt as \emph{``Please proceed to the next step using your best judgment"} for each decision making point. 

% To address this gap, we introduce a novel approach where the LLM agent generates the next prompt autonomously based on the current context and stage of the CTF challenge, a technique we call \textbf{dynamic prompting}.


Expanding on single LLM agents, multi-agent LLM systems are a powerful approach to enhance problem-solving by simulating team-based collaboration. Specialized agents, each with distinct objectives, work together to tackle different aspects of complex tasks \cite{guo2024largelanguagemodelbased}
Multi-agent systems are effective in cybersecurity applications. For instance, Audit-LLM~\cite{song2024audit} deploys a  multi-agent system for insider threat detection by employing agents to decompose tasks, build tools, and use collaborative reasoning to enhance detection accuracy. Liu~\cite{liu2024multi} explores multi-agent systems to enhance incident response in cybersecurity by examining centralized, decentralized, and hybrid team structures to assess how LLM agents can improve decision-making, adaptability, and coordination during cyber-attack scenarios. AutoSafeCoder~\cite{nunez2024autosafecoder} enhances the security of code generated by LLMs by incorporating a coding agent for code generation, a static analyzer agent that identifies vulnerabilities, and a fuzz testing agent for dynamic testing to detect runtime errors. Division of responsibilities among different agents allows AutoSafeCoder to produce secure, functionally correct code. 

% With the growing use of LLMs in CTF challenges, prompt engineering is key to enhancing performance. Various methods guide LLMs in solving complex cybersecurity tasks, each with distinct strengths and limitations.

% \textbf{Single Turn (Zero-Shot Prompting)} involves providing the model with a one-time task description that outputs  an immediate solution. This is efficient for straightforward tasks~\cite{yang2023intercode}. In contrast, \textbf{Try Again (Iterative Feedback Prompting)} uses iterative feedback to refine responses over multiple attempts, mimicking real-world problem-solving~\cite{yang2023intercode}. The \textbf{Plan \& Solve} enhances adaptability by incorporating a planning phase before iterative execution. This helps models tackle ambiguous or complex tasks by  structured strategies~\cite{turtayev2024hacking}. Additionally, \textbf{ReAct (Reasoning + Action)} combines step-by-step reasoning with action, allowing the model to adjust dynamically through iterative cycles. This makes it particularly effective for evolving and complex challenges like CTFs~\cite{yao2023react}. 
% These prompting techniques highlight diverse approaches to optimizing LLM performance in cybersecurity tasks. 

% Multi-agents!


%\meet{Add references for auto-prompting, and shorten this para}
%\nanda{Maybe we can add this to previous paragraphs which discusses other prompting methods such as plan-and-solve and ReAct method}
% All of these prompting methods include a static template which needs to be given to LLM to solve the CTF challenges. For instance, the NYU CTF framework provides a static prompt as \emph{``Please proceed to the next step using your best judgment"} for each decision making point. 
% Based on the given prompt, LLM agents make a decision and proceed further to find flags. While static prompts provide straightforward guidance, they often fail to account for the evolving nature of complex tasks, limiting their effectiveness in multi-step or ambiguous CTF challenges. To address this gap, we propose \textbf{dynamic prompting}, where the LLM agent autonomously generates prompts based on the CTF challenge's context and stage.
% % To address this gap, we introduce a novel approach where the LLM agent generates the next prompt autonomously based on the current context and stage of the CTF challenge, a technique we call \textbf{dynamic prompting}.
% Dynamic prompting adapts instructions to task progress, ensuring instructions are contextually relevant and reflective of the specific obstacles encountered. By iterating based on feedback and intermediate outputs, it continuously refines the LLM’s approach, enhancing problem-solving for dynamic tasks like CTFs.
% This adaptive process not only mirrors how humans tackle complex problems but also improves the model’s ability to handle unpredictable scenarios, making it particularly advantageous for cybersecurity tasks like CTFs where conditions change dynamically.


% The very first prompt type used in several applications is \textbf{Single Turn (Zero-Shot Prompting)}~\cite{yang2023intercode}. In single-turn prompting, the model receives a one-time, straightforward task description and is expected to generate a complete response without further interaction. The initial output is directly assessed, making this approach efficient for tasks where minimal feedback or iteration is required. This method tests the model’s ability to understand and respond to tasks immediately, relying heavily on the model's pre-trained knowledge and generalization capabilities.

% Along with this, The prompting method named \textbf{Try Again (Iterative Feedback Prompting)}~\cite{yang2023intercode} has been also used in several appreciations specially to solve CTF challenges. It is an iterative prompting method involves continuous interaction, where the model is provided with feedback after each attempt. The model can refine its responses over multiple turns based on the observations or execution results from previous outputs. This iterative process continues until the task is successfully completed or a maximum number of interactions is reached. This approach closely mirrors real-world problem-solving, where adjustments are made iteratively based on evolving circumstances or feedback.

% Some application are also using \textbf{Plan \& Solve}~\cite{turtayev2024hacking} prompting method which enhances problem-solving by dividing the process into a planning phase followed by execution. Initially, the model formulates a strategy based on the task description and available information, allowing for a structured approach to ambiguous or complex problems. This plan guides the subsequent execution phase, where the model carries out actions iteratively, refining its approach based on feedback. In more challenging scenarios, re-planning mid-task further improves adaptability and performance. This method proves effective in tasks like CTF challenges, where vague instructions require careful analysis and step-by-step resolution.

% Further some application are also using \textbf{ReAct (Reasoning + Action)}~\cite{yao2023react} prompting method blends reasoning with action by guiding the model to think through tasks step-by-step before executing actions. At each step, the model generates a thought based on the task and observations, which informs the next action. The action is executed, and the resulting feedback refines the model’s understanding for the next cycle. This continuous process helps the model adapt dynamically to complex tasks, making it effective for CTF challenges where logical reasoning and step-by-step execution are essential.

\section{Related Works} \label{sec:related_work}


\begin{table}[htpb]
    \centering
    \caption{Feature comparison of LLM agents for solving CTFs.}
    \label{tab:related_work_comparison}
    \begin{tabular}{lcccccc}
    \toprule
         \textbf{Study} & \rotatebox{90}{\textbf{\# CTFs}} & \rotatebox{90}{\textbf{Open bench}} & \rotatebox{90}{\textbf{Tool use}}  & \rotatebox{90}{\textbf{Autonomous}} & \rotatebox{90}{\textbf{Multi-agent}} &\rotatebox{90}{\textbf{Auto-prompt}} \\
    \cmidrule{2-7}
     % \textbf{Study} & \textbf{Dynamic} & \textbf{Used} & \textbf{Multi-} & \textbf{Automatic} & \textbf{Tool} & \textbf{\# of} \\
         Tann et al. \cite{tann2023using} &  $7$ & \purplecross & \purplecross & \purplecross & \purplecross & \purplecross  \\
         Shao et al. \cite{shao2024empirical} & $26$ & \purplecross & \tealcheck & \tealcheck & \purplecross & \purplecross  \\
         InterCode-CTF\cite{yang2023language} & $100$ & \tealcheck & \tealcheck & \tealcheck & \purplecross & \purplecross   \\
         NYU CTF Bench \cite{shao2024nyu} & $200$ & \tealcheck & \tealcheck & \tealcheck & \purplecross & \purplecross \\
         Turtayev et al. \cite{turtayev2024hacking} & $100$ & \tealcheck & \tealcheck & \tealcheck & \purplecross & \purplecross\\
         Cybench \cite{zhang2024cybenchframeworkevaluatingcybersecurity} & $40$ & \tealcheck & \tealcheck & \tealcheck & \purplecross & \purplecross \\
         EnIGMA \cite{abramovich2024enigma} & $350$ & \tealcheck & \tealcheck & \tealcheck & \purplecross & \purplecross\\
         HackSynth \cite{muzsai2024hacksynth} & $200$ & \tealcheck & \tealcheck & \tealcheck & \tealcheck & \purplecross \\
         \textbf{D-CIPHER (ours)} & $290$ & \tealcheck & \tealcheck & \tealcheck & \tealcheck & \tealcheck \\
    \bottomrule
    \end{tabular}
\end{table}



% \subsection{LLMs on Cybersecurity}
% \subsection{LLM Agents for CTF}

%LLMs have a vast knowledge base that can be tapped for cybersecurity use.
Tann et al.~\cite{tann2023using} evaluate early LLMs such as ChatGPT and Google Bard in solving CTF challenges and answering professional certification questions, showing that LLM responses contain key task information.
%Many works extend the LLM capabilities by providing them access to programming and command execution tools, to form autonomous agents. 
The InterCode-CTF agent~\cite{yang2023intercode} reveals that LLM agents demonstrate basic cybersecurity skills, however they struggle with more complex tasks.
The NYU CTF baseline agent~\cite{shao2024empirical} integrates external tools into the LLM's function-calling features and demonstrate improved potential of tool-assisted LLMs to solve CTFs, however it exhausts the LLM context length when command output history becomes very long. InterCode-CTF manages this issue by truncating the history to only show the LLM the last few iterations. Even so, LLM agents face issues with longer tasks.
%NYU CTF Bench~\cite{shao2024nyu}, a benchmark of 200 CTF challenges, presents a baseline agent with specialized reverse engineering tools and category-specific prompts, demonstrating their importance to solve CTFs.
% The NYU CTF baseline agent faces issues of LLM context length when complex tasks run for several iterations and the entire command and output history becomes longer than the LLM's context window size. The InterCode agent manages this issue by truncating the history to only show the LLM the last few iterations.


Excessive tool availability and verbose interfaces can overwhelm agents, leading to inefficiencies. Agents perform better with a focused set of tools with well-defined interfaces~\cite{yang2024sweagent}.
EnIGMA~\cite{abramovich2024enigma} agent incorporates interactive tools and in-context learning techniques to achieve state-of-the-art results. % on the NYU CTF Bench, HackTheBox, and Cybench benchmarks.
For better context management, EnIGMA also uses an LLM summarizer that summarizes the command outputs for the main agent.

HackSynth~\cite{muzsai2024hacksynth}, an LLM agent for autonomous penetration testing, shows that iterative planning and feedback summarization stages help the agent finish multiple tasks and improves overall problem solving.
Similarly, Cybench~\cite{zhang2024cybenchframeworkevaluatingcybersecurity} introduces a benchmark of 40 CTF challenges augmented with step-by-step tasks, demonstrating better focus of LLM agents on smaller tasks, leading to improved success and alleviating the context length issue.
\citet{turtayev2024hacking} expand on InterCode-CTF by implementing plan-and-solve prompting, achieve significant improvement on the InterCode-CTF benchmark. They show that prompting techniques can improve performance even with simple toolsets.
% . Their baseline agent is evaluated in unguided mode (i.e. fully autonomous), and guided mode where the agent is given one task at a time. Their results indicate that providing smaller tasks to the LLM agents improve their focus yielding improved success on complex challenges while .

These works highlight that LLM agents excel at implementing code and executing commands to accomplish small concrete tasks when provided with dynamic feedback and task-specific toolsets. While these works  involved using multiple LLMs with different tasks such as planning and summarizing along-side a main agent, D-CIPHER is the first work to formulate a multi-agent system where there is a bifurcation of responsibilities between agents and meaningful well-defined interactions for dynamic feedback.
Table~\ref{tab:related_work_comparison} shows a feature comparison of D-CIPHER with related works on LLM agents for autonomous CTF solving.
%\meet{some description of the feature comparison?}
% Recent research has focused on enable autonomous solving of CTF challenges~\cite{shao2024empirical,shao2024nyu,abramovich2024enigma}. These agents typically operate in containerized environments to ensure reproducibility and modularity. 

% As an early effort, Tann et al.~\cite{tann2023using} evaluated the effectiveness of LLMs, such as OpenAI's ChatGPT, Google Bard, and Microsoft Bing, in solving cybersecurity CTF challenges and answering professional certification questions. 
% % Their study results show that LLMs performed well on $7$ CTF test cases, with ChatGPT solving $6$, Bard $2$, and Bing $1$. 
% The study shows that LLM responses often contain key information essential for solving tasks.

% The InterCode framework~\cite{yang2023intercode} approaches coding as an interactive process and uses execution feedback to improve code generation. As described in Yang et al.~\cite{yang2023intercode}, InterCode-CTF integrates CTF benchmarks into a reinforcement learning environment that can evaluate the cybersecurity capabilities of language agents. It features $100$ tasks that tapskills such as reverse engineering, forensics, and binary exploitation. While existing language agents demonstrate basic cybersecurity skills, evaluations indicate they struggle with more complicated complex tasks unless the system is fine-tuned or given external support. 
% cite Intercode: Standardizing and benchmarking interactive coding with execution feedback

% Another notable example is an LM agent developed by Shao et al. specifically to automate CTF tasks. 
% Shao et al.~\cite{shao2024empirical} developed a LM agent to automate CTF tasks.
% % They report an accuracy rate of  $46\%$ on $26$ CTF challenges sourced from CSAW'23 Qualifying round competition using GPT-4.
% By effectively combining LLM capabilities with external tools, the researchers demonstrated the potential of tool-assisted LLMs to solve complex problems. Building on this, the team incorporated a broader range of cybersecurity tools and interfaces that enhance both accuracy and versatility. 
% Empirical results show their system outperforms baselines on both the InterCode CTF benchmark and the NYU CTF benchmark.

% Shao et al.~\cite{shao2024nyu} presented a diverse, open-source database of CTF challenges that can be used to benchmark an LLM's ability to solve cybersecurity problems.
% It provides a scalable platform for developing and testing AI-driven approaches for vulnerability detection and resolution, facilitating advancements in automated cybersecurity tasks. The benchmark database and automated framework were successfully applied to the performance of five LLMs. 

% The Cybench benchmark~\cite{zhang2024cybenchframeworkevaluatingcybersecurity} provides another significant contribution by creating a framework tailored to solving CTF challenges. % Cybench: A framework for evaluating cybersecurity capabilities and risk
% % Their benchmark environment achieves an accuracy of $17.5\%$ using Claude 3.5 Sonnet. 
% Such frameworks operate in Linux-based containerized environments, such as Kali Linux, which includes pre-installed cybersecurity tools. However, excessive tool availability can overwhelm agents, leading to inefficiencies. Research indicates that agents perform better with a focused set of tools that have well-defined interfaces~\cite{yang2024sweagent}. % Swe-agent: Agent-computer interfaces enable automated software engineering



% Muzsai et al. introduced HackSynth~\cite{muzsai2024hacksynth}, an LLM-based agent for autonomous penetration testing. It uses a dual-module architecture that consists of a Planner and a Summarizer, allowing for iterative command generation and feedback processing. The framework is evaluated using two benchmark sets from platforms like PicoCTF~\cite{picoctf} and OverTheWire~\cite{overthewire}. These benchmarks address $200$ challenges drawn from various domains and difficulty levels. Results of their study show that HackSynth, especially with the GPT-4o model, achieves the best performance. This highlights the potential of LLM-based agents in advancing autonomous penetration testing.
 % Using basic prompting techniques and expanding tool availability, the study highlights how straightforward approaches can unlock the latent potential of LLMs for cybersecurity tasks. Their work emphasizes that simple LLM designs can effectively solve CTF challenges, and thus broaden the number of cybersecurity applications without the need for advanced engineering.

% \begin{table*}[]
%     \centering
%     \begin{tabular}{|c|c|>{\centering\arraybackslash}p{4.5cm}|c|c|c|c|c|c|}
%     \hline
%          \textbf{Study} & \textbf{Dynamic} & \textbf{Used} & \textbf{Multi-} & \textbf{Open} & \textbf{Automatic} & \textbf{Tool} & \textbf{\# of} & \textbf{\# of} \\
%          & \textbf{Prompt} & \textbf{Benchmarks} & \textbf{Agents} & \textbf{Dataset} & \textbf{Framework} & \textbf{Use} & \textbf{LLMs} & \textbf{CTFs}\\
%          \hline
%          Tann et al.~\cite{tann2023using} & \purplecross & Manual collected & \purplecross & \purplecross & \purplecross & \purplecross & $3$ & $7$ \\
%          \hline
%          InterCode-CTF~\cite{yang2023language} & \purplecross &  PicoCTF~\cite{picoctf} & \purplecross & \purplecross& \purplecross & \purplecross & $1$ & $100$  \\
%          \hline
%          Shao et al.~\cite{shao2024empirical} & \purplecross & CSAW 2023 & \purplecross & \purplecross & \tealcheck & \tealcheck & $4$ & $26$ \\
%          \hline
%          Shao et al.~\cite{shao2024nyu} & \purplecross & NYU CTF~\cite{shao2024nyu} & \purplecross & \tealcheck & \tealcheck & \tealcheck & $5$ & $200$ \\
%          \hline
%          Cybench~\cite{zhang2024cybenchframeworkevaluatingcybersecurity} & \purplecross & Cybench~\cite{zhang2024cybenchframeworkevaluatingcybersecurity}  & \purplecross & \tealcheck & \tealcheck & & $8$ & $40$ \\
%          \hline
%          EnIGMA~\cite{abramovich2024enigma} & \purplecross & NYU CTF~\cite{shao2024nyu}, InterCode-CTF~\cite{yang2023language},  HackTheBox~\cite{hackthebox} & \purplecross & \purplecross & \tealcheck & \tealcheck & $3$ & $350$ \\
%          \hline
%          HackSynth~\cite{muzsai2024hacksynth} & \purplecross & PicoCTF~\cite{picoctf}, OverTheWire~\cite{overthewire} & \tealcheck & \tealcheck & \tealcheck & \tealcheck & $8$ & $200$ \\
%          \hline
%          Turtayev et al.~\cite{turtayev2024hacking} & \purplecross & InterCode-CTF~\cite{yang2023language} & \purplecross & \purplecross & \purplecross & \purplecross & $4$ & $100$ \\
%          \hline
%          \textbf{D-CIPHER (Proposed)} & \tealcheck & NYU CTF~\cite{shao2024nyu}, Cybench \cite{zhang2024cybenchframeworkevaluatingcybersecurity}, HackTheBox \cite{hackthebox} & \tealcheck & \tealcheck & \tealcheck & \tealcheck & 5 & 290 \\
%          \hline
%     \end{tabular}
%     \caption{Comparison with LLM-based CTF solving Literature}
%     \label{tab:related_work_comparison}
% \end{table*}




% \subsection{Multi-agent framework}

% The use of multi-agent LLM systems in Capture the Flag (CTF) challenges is emerging as a powerful approach to enhance cybersecurity problem-solving. Multi-agent frameworks mimic team-based collaboration, where multiple LLM agents, each with specialized expertise, work together to tackle complex tasks. This approach reflects real-world cybersecurity operations, where success often depends on coordinated efforts from teams with diverse skills, each addressing different components of a security challenge.
% Multi-agent LLM systems are emerging as a powerful approach to enhance cybersecurity problem-solving by simulating team-based collaboration. Specialized agents, each with distinct objectives, work together to tackle different aspects of complex security tasks. This mirrors real-world cybersecurity operations, where coordinated efforts and diverse skills are essential for addressing evolving threats and vulnerabilities.

% CTF challenges cover a wide range of domains, including cryptography, reverse engineering, forensics, and web exploitation. Multi-agent systems can distribute the workload by assigning agents to handle specific tasks. This enables parallel problem-solving and emulates the collaborative nature of human teams. For example, one agent may specialize in guiding the fellow agents to what needs to be done, while another executes the instructions, ensuring that tasks are addressed without losing the context, and implementing reasoning from multiple LLMs. This division of labor boosts efficiency and enables problem-solving from multiple perspectives.
% This division of labor enhances efficiency and allows the system to approach problems from multiple perspectives, reflecting the interdisciplinary approach often used in cybersecurity teams.

% Guo et al.~\cite{guo2024largelanguagemodelbased} highlight the strengths of multi-agent LLMs in complex, multi-step tasks where different agents handle specific roles The framework HackSynth~\cite{muzsai2024hacksynth} is a multi-agent penetration testing framework in which agents operate collaboratively to address vulnerabilities in staged environments. Their work emphasizes that when agents work as a cohesive team, they outperform single-agent approaches. This is particularly true when facing layered, iterative challenges. 
% This team-based model of problem-solving aligns closely with how cybersecurity professionals approach real-world security incidents and penetration testing exercises.

% Multi-agent LLM systems have shown effectiveness in various other applications. For instance,  Audit-LLM~\cite{song2024audit} presents a multi-agent framework for insider threat detection using log analysis. It employs agents to decompose tasks, build tools, and use collaborative reasoning to enhance detection accuracy. Liu~\cite{liu2024multi} explores the application of LLM-based multi-agent systems to enhance incident response (IR) in cybersecurity. Utilizing the ``Backdoors \& Breaches" tabletop game as a simulation environment, the study examines centralized, decentralized, and hybrid team structures to assess how LLM agents can improve decision-making, adaptability, and coordination during cyberattack scenarios. AutoSafeCoder~\cite{nunez2024autosafecoder} is a multi-agent system designed to enhance the security of code generated by LLMs. The framework comprises three agents: a Coding Agent responsible for code generation, a Static Analyzer Agent that identifies vulnerabilities through static analysis, and a Fuzzing Agent that performs dynamic testing using mutation-based fuzzing to detect runtime errors. By integrating both static and dynamic testing in an iterative process, AutoSafeCoder aims to produce secure, functionally correct code. 

% To enhance CTF-solving by promoting team-based specialization, we employ a multi-agent CTF solving agent. Within this framework, agents tackle tasks aligned with their strengths. Tasks are executed in parallel, improving efficiency and accelerating progress. Agents share insights, adapt refining strategies based on feedback, and overcome obstacles collectively. This collaborative approach boosts scalability, adaptability, and and resilience, and improves performance in complex challenges.

% This paper presents a comprehensive comparison of D-CIPHER with existing LLM-based CTF-solving literature, as shown in Table~\ref{tab:related_work_comparison}.
% This paper documents the results of  our comprehensive comparison of D-CIPHER with existing LLM-based CTF-solving literature. These results are presented in Table~\ref{tab:related_work_comparison}.

\section{Problem formulation}

\emph{Alice} owns a language model $\A$, fine-tuned for specific tasks such as chatting, problem solving, or code generation, which is available through an API (\autoref{chap6/fig:fig1}).
\emph{Bob} owns another language model $\B$.
Alice suspects that Bob fine-tuned $\B$ on some outputs from $\A$.
We denote by $D$ the dataset used to fine-tune $\B$, among which $D^{\A}\subset D$ is made of outputs from $\A$, in proportion $\rho = |D^{\A}|/|D|$.

\paragraph*{Access to Bob's data.} 
\label{chap6/sec:degreeofsupervision}

We consider two settings for Alice's knowledge about Bob's training data:
\begin{itemize}[leftmargin=0.5cm, itemsep=2pt, topsep=1pt]

    \item \emph{supervised}: Bob queries $\A$ and  Alice retains all the content $\Tilde{D}^{\A}$ that $\A$ generated for Bob. Thus, Alice knows that $D^{\A} \subseteq \Tilde{D}^{\A}$. 
    We define the \emph{degree of supervision} $d := |D^{\A}|/|\Tilde{D}^{\A}|$,  

    \item \emph{unsupervised}: Bob does not use any identifiable account or is hiding behind others such that $|\Tilde{D}^{\A}| \gg |D^{\A}|$ and $d \approx 0$.
    This is the most realistic scenario.
\end{itemize}

Thus, $\rho$ is the proportion of Bob's fine-tuning data which originates from Alice's model
while $d$ quantifies Alice's knowledge regarding the dataset that Bob may have utilized (see Fig.~\ref{chap6/fig:datasets}).

\paragraph*{Access to Bob's model.} 
We consider two scenarios:
\begin{itemize}[leftmargin=0.5cm, itemsep=2pt, topsep=1pt]
    \item Alice has an \emph{open-model} access to $\B$. 
    She can forward any inputs through $\B$ and observe the output logits.
    This is the case if Bob open-sources $\B$, or if Alice sought it via legitimate channels.
    \item Alice has a \emph{closed-model} access. 
    She can only query $\B$ through an API without logits access: Alice only observes the generated texts.
    This would be the case for most chatbots.
\end{itemize}

We then introduce two definitions of radioactivity:
\begin{definition}[Text Radioactivity]\label{chap6/def:text_radioactivity}
    Dataset $D$ is $\alpha$-radioactive for a statistical test $T$ if ``$\B$ was not trained on $D$'' $\subset \H_0$ and
    $T$ is able to reject $\H_0$ at a significance level (\pval) smaller than $\alpha$.
\end{definition}

\begin{definition}[Model Radioactivity]\label{chap6/def:model_radioactivity}
    Model $\A$ is $\alpha$-radioactive for a statistical test $T$ if
    ``$\B$ was not trained on outputs of $\A$'' $\subset \H_0$ and $T$ is able to reject $\H_0$ at a significance level smaller than $\alpha$.
\end{definition}

Thus, $\alpha$ quantifies the radioactivity of a dataset or model. 
A low $\alpha$, e.g. $10^{-6}$, indicates strong radioactivity: the probability of observing a result as extreme as the one observed, assuming that Bob's model was not trained on Alice's outputs, is 1 out of one million. 
Conversely, $\alpha\approx 0.5$ means that the observed result is equally likely under both the null and alternative (radioactive) hypotheses.



\begin{figure}[b]
   \begin{minipage}{0.62\textwidth}
        \centering
        \resizebox{1.0\linewidth}{!}{
        \begin{tabular}{r ccc ccc ccc}
            \toprule
            & \multicolumn{2}{c}{With WM} & \multicolumn{2}{c}{MI} & \multicolumn{2}{c}{IPP} \\
            \cmidrule(lr){2-3} \cmidrule(lr){4-5} \cmidrule(lr){6-7}
            & Open & Closed & Open & Closed & Open & Closed \\
            Supervised & \cmarkg & \cmarkg &\cmarkg & \xmarkg & \cmarkg & \amark \\
            Unsupervised & \cmarkg & \cmarkg &\xmarkg & \xmarkg & \xmarkg & \xmarkg \\
            \bottomrule
        \end{tabular}
        }
        \captionof{table}{
            Availability of radioactivity detection under the different settings: 
            \cmarkg: available, \xmarkg: not available, \amark: available but with strong limitations.
            \textit{Open} / \textit{closed-model} refers to the availability of Bob's model, and \textit{supervised} / \textit{unsupervised} to Alice's knowledge of his data.
            Detection with watermarks is described in Sec.~\ref{chap6/sec:radioactivity_detection}, and other approaches relying on Membership Inference (MI) and Intellectual Property Protection (IPP) are detailed in Sec.~\ref{chap6/sec:discussion-other-approaches}.
        }
        \label{chap6/tab:summary_MIA_wm}
   \end{minipage}\hfill
    \begin{minipage}{0.34\textwidth}
        \centering
        \includegraphics[width=1.0\textwidth, clip, trim=0.7cm 2.2cm 0.7cm 0]{chapter-6/figs/datasets.pdf}
        \captionsetup{font=small}
        \caption{
            Detection performance mainly depends on $\rho = |D^{\A}|/|D|$ and $d = |D^{\A}|/|\Tilde{D}^{\A}|$, where $D$ is the fine-tuning dataset used by Bob, $\Tilde{D}^{\A}$ are the outputs from Alice's model, and $D^{\A}$ the intersection of both.
        }
        \label{chap6/fig:datasets}
    \end{minipage}
\end{figure}













\section{Radioactivity detection}\label{chap6/sec:radioactivity_detection}


\subsection{Why current approaches are insufficient}

\paragraph{Membership inference and IPP methods.} 

Passive methods relying on membership inference observe the model perplexity on text belonging to the training dataset, compared to text not in the dataset.
They are effective only in the \textit{supervised} setting, where Alice has precise knowledge of the data used to train Bob's model and \textit{open} access to it. 
In that scenario, she can easily demonstrate $\alpha$-radioactivity, with $\alpha$ as low as $10^{-30}$, providing strong evidence that Bob has trained on her model.

In parallel, there are active methods, such as ones used for IPP, that explicitly modify the LLM generation to detect when the outputs are used as training data.
However, even state-of-the-art methods~\citep{zhao2023protecting, he2022protecting, he2022cater} fall short in the unsupervised setting, where the statistical guarantees do not hold in practice.

These claims are detailed and supported by experiments in Sec.~\ref{chap6/par:mia_wm} (for MIA) and Sec.~\ref{chap6/sec:ipp} (for IPP), and summarized in Tab.~\ref{chap6/tab:summary_MIA_wm}.
At the end of the day, the applicability of the methods is very restricted, and does not generalize to real-world scenarios.

\paragraph{Naive approach for watermark detection.} 
We now assume that the outputs of $\A$ are watermarked with a method $W$ with Alice's secret key $\sk$ as described in Sec.~\ref{chap6/sec:background}. 
The original watermark detector $T$ tests the null hypothesis $\mathcal{H}_0$: ``\textit{The text was not generated following $W$ with secret key $\sk$}'' by applying a scoring function $W_{\textrm{score}}$ keyed by $\sk$ on the text.
The output \pval\ depends on the score and the number of analyzed tokens.

Unlike watermark detection which takes text as input, the input for radioactivity detection is a model.
The naive approach to detect model radioactivity on $\B$ is to run test $T$ on a large corpus of texts generated by $\B$. 
This aligns with Def.~\ref{chap6/def:text_radioactivity}:
the text cannot be generated following $W$ and $\sk$ if $\B$ has never seen the watermark, so if ``\textit{$\B$ did not use outputs of $\A$}'', then $\H_0$ is true.
However, this detector is ineffective because 
(1) \emph{the watermark signal is very weak}, as radioactivity can only be observed for watermarked $(k+1)$-grams $\{$watermark window + current token$\}$ that were part of $\A$'s outputs in $\B$'s training data, therefore the signal is diluted in the generated text, and;
(2) \emph{theoretical \pval s break down} when computed naively from many tokens -- $\approx$ 1M, see the experiments of Sec.~\ref{chap6/par:dedup-expe} -- which is necessary to observe this very weak signal.

\subsection{Enhanced radioactivity detection}

\paragraph{Overview.} 
We use the detection test $T$ from text watermarking as presented in Sec.~\ref{chap6/sec:background}, but adapt the score computation (\autoref{chap6/fig:method}).
Our goal is to reduce noise by focusing Bob's model on watermarked windows likely to be radioactive. 
To this end, we recreate a context similar to the one that generated the watermarked text by Alice. 
We ensure the accuracy of statistical tests through de-duplication of scored tokens. 

\begin{figure*}[b!]
    \centering
    \includegraphics[width=1.0\textwidth, clip, trim=0 1.5cm 3.7cm 0]{chapter-6/figs/method.pdf}
    \caption{
    Radioactivity detection with closed or open model access (for simplicity, only \citep{kirchenbauer2023watermark} is illustrated).
    \textit{(Left)} New texts are generated from $\B$ using prompts from $\A$ and these texts are scored. 
    The filter $\phi$ is used to focus the score computation on likely contaminated $k$-grams. 
    \textit{(Right}) Texts generated by $\A$ are directly forwarded through $\B$, and the next-token predictions are scored using tokens from the input as the watermark window.
    In both cases, the \textit{tape} ensures reliable \pval s by de-duplicating scored tokens.
    }
    \label{chap6/fig:method}\label{chap6/fig:open_model}
\end{figure*}


\paragraph{Radioactivity detection in $\B$.}
To amplify the radioactivity signal, we employ two strategies.
(1) In the supervised setting, we use watermarked text from $\Tilde{D}^\A$. In the unsupervised setting, we use new watermarked text generated by $\A$ that aligns with the suspected training distribution of $\B$ (e.g., English dialogues).
(2) We score up to millions of tokens, orders of magnitudes more than usual.
The scoring depends on the access to $\B$:
\begin{itemize}[leftmargin=*, itemsep=0pt, topsep=0pt]
    \item \emph{closed-model}: we use the prompts to generate new texts from $\B$, and score these texts.
    \item \emph{open-model}, \aka, ``reading mode'': instead of generating completions with $\B$, we directly forward the texts generated by $\A$ through $\B$, as depicted in Fig.~\ref{chap6/fig:open_model}.
    We then score next-token predictions with $W_{\textrm{score}}$ by using tokens from the input as watermark window.
    Intuitively, it reproduces the right contexts, and allows us to study how $\B$ behaves on these watermarked windows rather than letting $\B$ generate tokens without any interesting signal.
\end{itemize}


\paragraph{Filter on scored $k$-grams.}
To further improve detection in the closed-model setting where the reading mode is not possible, we only score $(k+1)$-grams $\{$watermark window + current token$\}$ output by $\B$ for which the watermark window is often in $\A$’s watermarked outputs.
We thus introduce a filter $\phi$, a set that contains these watermark windows.
In the \emph{supervised} setting ($0<d\leq1$), $\phi$ is made of the $k$-grams present in $\Tilde{D}^\mathcal{A}$ (refer to Fig.~\ref{chap6/fig:datasets}).
In the \emph{unsupervised} setting, we focus on `likely' contaminated $k$-grams, \eg, $k$-grams appearing in (new) watermarked text generated by $\A$.

\paragraph{Token scoring and de-duplication.}
\autoref{chapter:three-bricks} demonstrates that detection tests can be empirically inaccurate due to biases in the natural distribution of tokens. 
This issue is more pronounced in our case, given the larger volume of tokens required for observing radioactivity. 
To mitigate this, we score a token only if the same $\{$watermark window + current token$\}$ combination has not been previously encountered.
Moreover, in the closed-model setting, we only score watermark windows ($k$-gram) that are not part of the (watermarked) prompt. 
In the open-model setting, tokens with watermarked windows previously present in the attention span are not scored. 
This is achieved by maintaining a \emph{tape} memory of all such $k$-grams combinations during detection.
These adjustments ensure reliable \pval s even when many tokens are analyzed (see Sec.~\ref{chap6/par:dedup-expe} and~\ref{chap6/app:correctness}).







\section{Empirical Evaluation}
\begin{table*}[!ht]
    \centering
    \resizebox{0.88\textwidth}{!}{    
    \begin{tabular}{r|cccccc|cccccc}
        \toprule 
        & \multicolumn{6}{c}{\textbf{LLaVA-1.5-7B}} & \multicolumn{6}{c}{\textbf{LLaVA-1.5-13B}} \\ 
        \cmidrule(lr){2-7}\cmidrule(lr){8-13}
        & \multicolumn{3}{c}{\textbf{MM-SafetyBench}} & \multicolumn{3}{c|}{\textbf{MOSSBench}} & \multicolumn{3}{c}{\textbf{MM-SafetyBench}} & \multicolumn{3}{c}{\textbf{MOSSBench}} \\
        \textbf{Method} & \textbf{DSR}$\uparrow$ & \textbf{RR}$\uparrow$ & \textbf{Avg}$\uparrow$ & \textbf{DSR}$\uparrow$ & \textbf{RR}$\uparrow$ & \textbf{Avg}$\uparrow$ & \textbf{DSR}$\uparrow$ & \textbf{RR}$\uparrow$ & \textbf{Avg}$\uparrow$ & \textbf{DSR}$\uparrow$ & \textbf{RR}$\uparrow$ & \textbf{Avg}$\uparrow$\\
        \midrule
        w/o Defense          & 0.06  & 0.98  & 0.52  & 0.14  & 0.97  & 0.55  & 0.10  & 0.97  & 0.53  & 0.30  & 0.96  & 0.63  \\
        \midrule
        \multicolumn{13}{c}{Baseline} \\
        \midrule
        Responsible          & 0.12  & 0.96  & 0.54  & 0.32  & 0.96  & 0.64  & 0.18  & 0.96  & 0.57  & 0.47  & 0.92  & 0.70  \\
        Policy               & 0.08  & 0.96  & 0.52  & 0.18  & 0.98  & 0.58  & 0.12  & 0.97  & 0.55  & 0.34  & 0.97  & 0.65  \\
        Demonstration        & 0.15  & 0.97  & 0.56  & 0.37  & 0.95  & 0.66  & 0.25  & 0.96  & 0.60  & 0.52  & 0.92  & \textbf{0.72}  \\
        SFT                  & 0.20  & 0.95  & 0.58  & 0.50  & 0.88  & 0.69  & 0.13  & 0.98  & 0.55  & 0.49  & 0.88  & 0.68 \\
        SafeDecoding         & 0.08  & 0.97  & 0.53  & 0.31  & 0.94  & 0.62  & 0.12  & 0.96  & 0.54  & 0.42  & 0.93  & 0.68  \\
        Caption              & 0.09  & 0.98  & 0.53  & 0.21  & 0.98  & 0.60  & 0.12  & 0.97  & 0.55  & 0.27  & 0.94  & 0.60  \\
        Caption (w/o image)  & 0.16  & 0.95  & 0.55  & 0.34  & 0.94  & 0.64  & 0.22  & 0.93  & 0.57  & 0.45  & 0.89  & 0.67 \\
        Intention            & 0.07  & 0.98  & 0.53  & 0.20  & 0.99  & 0.59  & 0.11  & 0.96  & 0.54  & 0.26  & 0.97  & 0.61  \\
        \midrule
        % \multicolumn{13}{c}{} \\
        % \midrule
        \midrule
        \multicolumn{13}{c}{SR++} \\
        \midrule        
        Responsible-Demonstration & 0.18 & 0.95 & 0.57 & 0.40 & 0.94 & 0.67 & 0.29 & 0.96 & 0.62 & 0.58 & 0.85 & \textbf{0.72} \\
        Responsible-Policy & 0.12 & 0.96 & 0.54 & 0.27 & 0.97 & 0.62 & 0.18 & 0.96 & 0.57 & 0.46 & 0.94 & 0.70 \\
        Policy-Demonstration & 0.13 & 0.96 & 0.55 & 0.37 & 0.97 & 0.67 & 0.20 & 0.96 & 0.58 &0.51 & 0.93 & \textbf{0.72}\\
        Responsible-Policy-Demonstration & 0.15 & 0.96 & 0.55 & 0.38 & 0.95 & 0.66 & 0.25 & 0.97 & 0.61 & 0.53 & 0.88 & 0.70\\
        \midrule
        \multicolumn{13}{c}{SR+MO} \\
        \midrule     
        Responsible-SFT & 0.56 & 0.93 & \textbf{0.75} & 0.61 & 0.72 & 0.67 & 0.35 & 0.96 & 0.65 & 0.74 & 0.62 & 0.68 \\
        Responsible-SafeDecoding & 0.30 & 0.96 & 0.63 & 0.54 & 0.87 & \underline{0.70} & 0.23 & 0.96 & 0.59 & 0.63 & 0.79 & 0.71\\
        Demonstration-SFT & 0.60 & 0.90 & \textbf{0.75} & 0.65 & 0.77 & \textbf{0.71} & 0.56 & 0.92 & \textbf{0.74} & 0.67 & 0.70 & 0.68\\
        Demonstration-SafeDecoding & 0.38 & 0.96 & \underline{0.67} & 0.55 & 0.87 & \textbf{0.71} & 0.40 & 0.96 & \underline{0.68} & 0.62 & 0.78 & 0.70\\
        \midrule
        \multicolumn{13}{c}{QR++} \\
        \midrule   
        Caption-Intention & 0.09 & 0.97 & 0.53 & 0.20 & 0.98 & 0.59 & 0.14 & 0.95 & 0.55 & 0.26 & 0.96 & 0.61\\
        % Caption-Intention (w/o image) & 0.18 & 0.96 & 0.57 & 0.32 & 0.95 & 0.64 & 0.25 & 0.92 & 0.59 & 0.45 & 0.92 & 0.68\\
        \midrule
        % \multicolumn{13}{c}{} \\
        % \midrule
        \midrule
        \multicolumn{13}{c}{QR\textbar{}SR} \\
        \midrule   
        Caption-Responsible & 0.34 & 0.96 & 0.65 & 0.53 & 0.79 & 0.66 & 0.33 & 0.96 & 0.65 & 0.50 & 0.82 & 0.66\\
        Intention-Responsible & 0.36 & 0.97 & \underline{0.67} & 0.51 & 0.86 & 0.68 & 0.27 & 0.96 & 0.61 & 0.49 & 0.90 & 0.70\\
        Caption-Responsible (w/o image) & 0.96 & 0.25 & 0.60 & 0.93 & 0.16 & 0.55 & 0.60 & 0.80 & \underline{0.70} & 0.72 & 0.72 & \textbf{0.72}\\
        % Responsible-Intention (w/o image) & 0.99 & 0.06 & 0.52 & 0.95 & 0.17 & 0.56 & 0.61 & 0.81 & 0.71 & 0.68 & 0.77 & 0.72\\
        \midrule
        \multicolumn{13}{c}{QR\textbar{}MO} \\
        \midrule
        Caption-SafeDecoding & 0.20 & 0.96 & 0.58 & 0.39 & 0.88 & 0.64 & 0.33 & 0.94 & 0.63 & 0.40 & 0.90 & 0.65 \\
        Intention-SFT & 0.28 & 0.97 & 0.62 & 0.43 & 0.78 & 0.61 & 0.25 & 0.96 & 0.60 & 0.50 & 0.88 & 0.69\\
        Caption-SafeDecoding (w/o image) & 0.24 & 0.95 & 0.60 & 0.41 & 0.89 & 0.65 & 0.36 & 0.85 & 0.61 & 0.56 & 0.84 & 0.70\\
        \bottomrule
    \end{tabular}}
    \caption{Comparison results of ensemble strategies with the corresponding individual defenses. \textbf{Bold} indicates the best overall performance, while \underline{underlined} highlights the top three methods.} % and the full score is 100\%
    \label{tab:en_inter_results}
\end{table*}


\subsection{Experimental Setup}
We empirically evaluate various defense methods and their ensemble strategies on LLaVA-1.5-7B and LLaVA-1.5-13B~\cite{liu2024visual} to validate their effectiveness in standard settings. Using MM-SafetyBench and MOSSBench datasets, we assess safety and helpfulness by measuring defense success rate (DSR) on harmful queries and response rate (RR) on benign queries. We evaluate 28 defense methods, including system reminders, optimization techniques, query refactoring, and noise injection, as well as inter- and intra-mechanism ensembles. Detailed descriptions of defense methods and experimental setups are provided in Appendix~\ref{sec:defense strategies} and~\ref{sec:experiment_detail}. 
For a broader evaluation, we add more experiments in Appendix~\ref{sec:utility}, ~\ref{sec:diverse_attacks} and~\ref{sec:time}, including evaluation with the MM-Vet dataset for testing the quality of model's response on general queries, tests on JailbreakV-28K for more diverse and complex attack scenarios, and a comparison of inference time for different defense methods.

\subsection{Individual Defense Results}

Table~\ref{tab:indi_results} shows results of individual defense methods across four categories. Most methods, except for noise injection, effectively improve model safety across different models and datasets, as evidenced by increased defense success rates. This aligns with our analysis in Figure~\ref{fig:analysis results} where system reminder, model optimization and query refactoring lead to an overall increase in refusal probabilities. 

\paragraph{Safety shift defenses compromise helpfulness.} System reminder and model optimization methods generally reduce response rates on the benign subset while increasing defense success rates on the harmful subset. This confirms that safety shift tend to compromise helpfulness. This is more pronounced in MOSSBench than MM-SafetyBench due to the more apparent harmfulness and concealed harmlessness in MOSSBench queries.

\paragraph{Harmfulness discrimination defenses mitigate over-defense.} Query refactoring methods, except for Caption (w/o image), generally achieve the highest response rates on the benign subset, particularly for MOSSBench with misleadingly benign queries. This validates that harmfulness discrimination improves the model's ability to distinguish between truly harmful and benign queries. Notably, the removal of images in the Caption (w/o image) significantly reduces response rates for both harmful and benign queries, highlighting the crucial role images play in jailbreaking LVLMs.
% \paragraph{Image matters.} The removal of images in the Caption (w/o image) and Intention (w/o image) defenses leads to significant improvements in DSR compared to their image-included counterparts, underscoring the crucial role that images play in jailbreaking LVLMs.

\paragraph{Multimodal defense is challenging.}
However, all individual defense methods still exhibit limited defense success rates. While larger-scale LVLMs (i.e., LLaVA-1.5-13B) tend to achieve slightly higher success rates, they are also more susceptible to over-defense. This underscores the inherent challenges of jailbreak defense for LVLMs, especially when relying on individual defense methods. 

\subsection{Ensemble Defense Results}
Table~\ref{tab:en_inter_results} provides the empirical evaluation of both inter-mechanism and intra-mechanism ensemble strategies, leading to the following insights:

\paragraph{Ensembles improve safety.} Compared to individual methods, most ensemble strategies effectively enhance safety across both datasets and model sizes, showing increased defense success rates, especially in \textit{SR+MO} and \textit{QR\textbar{}SR} methods.

\paragraph{Inter-mechanism ensembles amplify.} Our evaluation shows most \textit{SR++} and \textit{SR+MO} ensembles improve defense success rates while reducing responses rates, whereas the \textit{QR++} ensemble better maintain responses rates. This confirms that inter-mechanism ensembles can amplify a single defense mechanism. Specifically, safety shift ensembles would further enhance model safety at the expense of helpfulness, while harmfulness discrimination ensemble better preserves helpfulness. Among inter-mechanism ensembles, those combining different types of specific methods (e.g., SR+MO) show a more pronounced amplification effect than those combining the same type (e.g., SR++). 
Notably, the Demonstration-SFT method excels in defense strength, utility, and response rate. Its success comes from combining two strong safety shift defenses, Demonstration and SFT, which complement each other and boost overall performance.

\paragraph{Intra-mechanism ensembles complement.} Compared to inter-mechanism ensembles, most \textit{QR\textbar{}SR} and \textit{QR\textbar{}MO} methods—except those without input images—can simultaneously maintain decent defense success rates and stable response rates,
compared to the undefended model and individual defense methods. This demonstrates that intra-mechanism ensemble can complement each other to achieve a more balanced trade-off. Additionally, the removal of input images offering a most conservative ensemble for multimodal defense while still maintaining certain helpfulness.
% In contrast, the defenses in intra-mechanism ensemble complement each other, strengthening safety while maintaining a stable level of helpfulness.
% In contrast, intra-mechanism ensembles combine the strengths of both mechanisms to achieve a more balanced trade-off. Specifically, \textit{QR\textbar{}SR} and \textit{QR\textbar{}MO} increase the refusal probability for harmful queries, while maintaining or even decreasing the refusal probability for benign queries, thereby improving the model's ability to distinguish between benign and harmful queries. This makes them a better choice for general scenarios where balancing safety and helpfulness is essential. 


\subsection{How Do Fine-tuning Affect Model Safety?}
We examine how different fine-tuning methods impact the safety of LVLMs by training LLaVA-1.5-7B using DPO and SFT with two datasets: SPA-VL~\cite{zhang2024spa} and VLGuard~\cite{zong2024safety}. SPA-VL focuses on safety discussions, while VLGuard emphasizes query rejection. We also test the effect of adding 5000 general instruction-following data from LLaVA.  

Table~\ref{tab:training_dataset_results} shows that DPO with SPA-VL and LLaVA provides a slight safety boost without significantly changing response behavior. In contrast, SFT has a stronger impact, but its effectiveness depends on the dataset. SPA-VL improves safety while maintaining helpfulness, though it may miss some harmful cases. VLGuard, however, makes the model overly defensive, rejecting too many queries. Adding LLaVA data helps balance safety and helpfulness, reducing excessive refusals.  


\begin{table}[ht]
    \centering
    \resizebox{0.49\textwidth}{!}{
    \begin{tabular}{r|cccccc}
        \toprule 
        & \multicolumn{3}{c}{\textbf{MM-SafetyBench}} & \multicolumn{3}{c}{\textbf{MOSSBench}} \\
        \textbf{Method} & \textbf{DSR}$\uparrow$ & \textbf{RR}$\uparrow$ & \textbf{Avg}$\uparrow$ & \textbf{DSR}$\uparrow$ & \textbf{RR}$\uparrow$ & \textbf{Avg}$\uparrow$ \\
        \midrule
        w/o Defense          & 0.06  & 0.98  & 0.52  & 0.14  & 0.97  & 0.55 \\
        \midrule
        \multicolumn{7}{c}{DPO} \\
        \midrule
        \multicolumn{1}{l|}{SPA-VL + LLaVA}          & 0.06  & 0.97  & 0.52  & 0.28  & 0.97  & 0.63  \\
        \midrule
        \multicolumn{7}{c}{SFT} \\
        \midrule
        \multicolumn{1}{l|}{SPA-VL}          & 0.24  & 0.96  & 0.60  & 0.58  & 0.78  & 0.68  \\
        + LLaVA     & 0.20  & 0.95  & 0.58  & 0.50  & 0.88  & 0.69  \\
        \midrule
        \multicolumn{1}{l|}{VLGuard}          & 1.00  & 0.09  & 0.55  & 0.90  & 0.21  & 0.55  \\
        + LLaVA     & 0.97  & 0.43  & 0.70  & 0.76  & 0.58  & 0.67  \\
        \bottomrule
    \end{tabular}}
    \caption{Comparison of varying fine-tuning settings.} % and the full score is 100\%
    \label{tab:training_dataset_results}
\end{table}


\section{Investigating radioactivity}\label{chap6/sec:fine-tuning-abl}

\autoref{chap6/sec:instruction} considers detection in a practical scenario.
This section further studies what influences radioactivity from different angles like fine-tuning, watermarking algorithm, and data distribution.


\subsection{Fine-tuning}


\begin{table}
    \caption{
        Influence of the model fine-tuning on the radioactivity.
        We report the $\logpval$ for $10$k scored observations (lower means more radioactive).
        {\setlength{\fboxsep}{2pt}\colorbox[HTML]{F2F2F2}{Gray}} indicates values used in Sec.~\ref{chap6/sec:instruction}.
    }
    \label{chap6/tab:ft-abl}
    \renewcommand{\arraystretch}{1.2}
    {\footnotesize
    \centering
  	\subfloat[ Learning rate.]{
            \begin{minipage}{0.23\linewidth}
                {\begin{center}
                    \begin{tabular}{ccc}
                       \colorcell  $10^{-5}$ & $5\cdot 10^{-5}$ & $10^{-4}$\\
                        \shline
                       \colorcell  -32.4 & -49.6 & -58.0\\
                    \end{tabular}
                \end{center}}
            \end{minipage} 
        } \hspace{0.03\linewidth}
  	\subfloat[ Epoch.]{
            \centering
            \begin{minipage}{0.27\linewidth}
                {\begin{center}
                    \begin{tabular}{ *{4}{c} }
                        1 & 2 & \colorcell 3 & 4 \\
                        \shline
                        -20.8 & -29.2 & \colorcell -33.2 & -34.8 \\
                    \end{tabular}
                \end{center}}
            \end{minipage} 
        } \hspace{0.03\linewidth}
        \subfloat[ Adapters.]{
            \centering
            \begin{minipage}{0.17\linewidth}
                {\begin{center}
                    \begin{tabular}{cc}
                        \colorcell Full & Q-LoRA \\
                        \shline
                        \colorcell -32.4 & -11.0 \\
                    \end{tabular}
                \end{center}}
            \end{minipage} 
        } \hspace{0.03\linewidth}
        \subfloat[ Model size.]{
            \centering
            \begin{minipage}{0.14\linewidth}
                {\begin{center}
                    \begin{tabular}{cc}
                       \colorcell  7B & 13B\\
                        \shline
                       \colorcell -32.4 & -33.2 \\
                    \end{tabular}
                \end{center}}
            \end{minipage} 
        } 
    }
 \end{table}



 





We first study the influence of fine-tuning on the same setup as Sec.~\ref{chap6/sec:instruction}, with regards to: 
(a) the learning rate,
(b) the fine-tuning algorithm,
\eg, with Q-LoRA~\citep{dettmers2023qlora} a widely used method for efficient fine-tuning;
(c) the number of epochs,
(d) the model size.
We fine-tune $\B$ with the same dataset of $\rho=100\%$ watermarked instructions and the same parameters.
We detect radioactivity in the \emph{open-model} / \emph{unsupervised} setting.
This is done on $N=10$k next-predictions, and where the texts that are fed to $\B$ are watermarked instructions generated with $\A$.
\autoref{chap6/tab:ft-abl} reports the results. The more the model fits the data, the easier its radioactivity is to detect.
For instance, multiplying the learning rate by $10$ almost doubles the average $\logpval$ of the test.


\subsection{Bigger teachers}\label{chap6/app:bigger-teachers}

\begin{table}[t!]
    \centering
    \caption{
        Influence of the teacher model size on radioactivity detection.
    }
    \label{chap6/table:results_Teachers}
    \footnotesize
    \begin{tabular}{ *{5}{l} }
        \toprule
        Teacher & Without & 7B & 13B & 65B \\
        \midrule
        NQ & 3.2 & 5.6 & 5.4 & 5.8 \\
        GSM8k & 10.0 & 11.1 & 10.4 & 11.0 \\
        MMLU & 28.4 & 31.0 & 32.9 & 33.8 \\
        \midrule
        $\log_{10}$ p-value & -0.3 & -32.4 & -31.1 & -31.7 \\
        \bottomrule
    \end{tabular}
\end{table}

We conduct the same experiment replacing the Llama-2-chat-7B teacher model by the 13B or 65B versions.
\autoref{chap6/table:results_Teachers} reports the results for benchmarks NQ, GSM8k, and MMLU and the average $\logpval$ of the radioactivity detection test under the same conditions as the previous paragraph.
Our observations align with the previous conclusions: watermarking does not significantly affect the benchmarks (except for MMLU where the improvement appears larger). 
Moreover, the detection of radioactivity is not significantly impacted by the teacher model size.










\subsection{Watermarking method}

\paragraph{Multi-bit scenario.} 

We adopt the same framework as in Sec.~\ref{chap6/sec:instruction}, but we use the watermarking method of \cite{yoo2023advancing}, \aka, MPAC.
It is a multi-bit method, where the watermark is a binary message of size $n$.
More precisely, we take bits 2 by 2 to generate a message $m = m_1 m_2 \ldots m_b$ with the $r$-ary $m_i = 0, 1, 2,$ or $3$, corresponding to $r=4$ and $b = n/2$ with the notations from the original paper.
The method proceeds as the one of \cite{kirchenbauer2023reliability} by altering the logits before generating a token.
However, the hash created from the watermarked window and the key is now used 
(1) to randomly partition the vocabulary into $r$ disjoint sets, 
(2) to select the position $i$ and corresponding $m_i$ that is hidden for this particular token.
A bias $\delta$ is added to the logits of the tokens belonging to the $m_i$-th set.
Given a text under scrutiny, the extraction reverses the process to find which $r$-ary is the most likely for each position $i$ of the message.

We report in Fig.~\ref{chap6/fig:bit-accuracy} the extraction results in the the supervised/closed-model setup.
We filter and deduplicate the tokens as in Sec.~\ref{chap6/sec:instruction}, and plot the observed bit accuracy against the number of scored tokens -- note that since the watermark is a binary message, we now measure the bit accuracy of the extraction, instead of the \pval\ of the detection.
This is done for several lengths of the binary message. 
Every experiment is run $10$ times for different text output by $\B$, which explains the $95\%$ confidence interval in the plots.
We observe as expected that the bit accuracy significantly increases with the proportion of watermarked data in the fine-tuning data, and that the longer the message, the harder it is to extract it.
This suggests that radioactivity still holds in the multi-bit scenario, and could therefore be used to identify a specific version of the model or a specific user from which the data was generated.

\begin{figure}[b!]
    \centering
    \includegraphics[width=0.99\linewidth,clip, trim=0 0 0 0cm]{chapter-6/figs/all_closed_sup.pdf}
    \caption{
        Bit accuracy when watermarked instruction data are generated with MPAC~\citep{yoo2023advancing}, against the number of scored tokens generated by the fine-tuned model.
        This is done under the supervised/closed-model setup, for various lengths ($n$=8, 16, 32) of the message.
    }
    \label{chap6/fig:bit-accuracy}
\end{figure}







\paragraph{Watermark window size.} 
To reduce the experimental requirements for generation, training and detection -- and introduce more variety to the data under study -- we now prompt $\A$=Llama-2-7B with the beginnings of Wikipedia articles in English and generate the next tokens with or without watermarking. 
We then fine-tune $\B$=Llama-1-7B on the natural prompts followed by the generated answers.
The fine-tuning is done in 1000 steps, using batches $8\times2048$ tokens (similarly to Sec.~\ref{chap6/sec:instruction}).
This section fine-tunes $\B$ on $\rho=100\%$ English watermarked texts.

\autoref{chap6/tab:exp_kgram} highlights that the confidence of the detection decreases for larger window sizes $k$ when fixing the \pval\ of the watermark detection of the training texts.
There are two explanations. 
First, for lower $k$, the chances that a $k$-gram repeats in the training data are higher, which increases its memorization.
Second, the number of possible $k$-grams is $|\V|^k$ and therefore increases with $k$, while the number of watermarked tokens is fixed. 
Thus, at detection time, the number of radioactive $k$-grams decreases with increasing $k$, diminishing the test's power. 
This experiment also demonstrates that the methods of \cite{aaronson2023watermarking} and \cite{kirchenbauer2023reliability} behave the same way.

\paragraph{Discussion on other watermarking schemes.}
Our goal is to derive a general method for detecting radioactivity in decoding-based watermarks, rather than testing all watermarking schemes or identifying the most radioactive one. 
We focus on the works of~\citet{aaronson2023watermarking, kirchenbauer2023watermark}, which are representative of the two main families of methods and provide reliable \pval s for detection. These schemes rely on key management using the hash of previous tokens, a mechanism also used by other potentially radioactive schemes~\citep{lee2023wrote, fu2024gumbelsoft}.

Some LLM watermarking schemes do not rely on hashing, such as ``semantic'' watermarks~\citep{liu2023semantic, liu2024adaptive, fu2024watermarking} and those using pre-defined key sequences~\citep{kuditipudi2023robust}. 
However, we do not evaluate the radioactivity of these methods due to the lack of \pval\ computation, a limitation also noted in other studies~\citep{piet2023mark}. 
For example, the complexity of computing \pval s for the scheme in~\citet{kuditipudi2023robust} would be prohibitively expensive, requiring $10^{16}$ times more operations than the schemes presented previously\footnote{
    \citet{kuditipudi2023robust} use the Levenshtein distance with complexity $O(mnk2)$, where $m$ is the number of tokens, $n=256$, and $k=80$ (default parameters from the authors). 
    This results in a complexity approximately $10^6$ times greater than previous schemes. 
    To evaluate \pval s of $10^{-10}$, a Monte Carlo simulation would require running the statistic $10^{10}$ times, a total $10^{16}$ more operations.
}. 
This limitation would also apply to works based on the same key management~\citep{christ2023undetectable, liu2023semantic}.









\subsection{Data distribution}

\begin{table}[t!]
        \centering
        \caption{
            Influence of the target text distribution on detection.
            $\B$ is prompted with beginnings of Wikipedia articles in the corresponding language, and detection is done on generated next tokens. 
            For each language, we score $N=250$k $k$-grams using the \textit{closed-model} setting.
        }
        \label{chap6/tab:exp_language}
        \footnotesize
        \begin{tabular}{ c *{6}{c} }
            Language &  English & French & Spanish & German & Catalan & Combined p-value (Fisher) \\
            \shline
                $\logpval$ & $<$-50 & -7.8 &  -5.7 &  -4.0 &  -2.1 & $<$-50  \\
        \end{tabular}
    \end{table}
    

\paragraph*{Radioactivity detection in different languages.}\label{chap6/par:distrib}
We consider an unsupervised setting where Alice has no prior knowledge about $D^\A$, the data generated with $\A$ used to fine-tune $\B$.
As an example, Alice does not know the language of $D^\A$, which could be Italian, French, Chinese, etc. 
We run the detection on text generated by $\B$, with prompts from Wikipedia in different languages.
The confidence of the test on another language -- that might share very few $k$-grams with $D^\A$ -- can be low, as shown in Tab.~\ref{chap6/tab:exp_language}.

Alice may, however, combine the \pval s of each test with Fisher's method.  
This discriminates against $\mathcal{H}_0$: ``\textit{none of the datasets are radioactive}'', under which the statement ``\textit{Bob did not use any outputs of $\A$}'' falls.
Therefore, the test aligns with our definition of model radioactivity as per definition~\ref{chap6/def:model_radioactivity}.
From Tab.~\ref{chap6/tab:exp_language}, Fisher's method gives a combined \pval\ of $<10^{-50}$. 
Thus, even if Alice is unaware of the specific data distribution generated by $\A$ that Bob may have used to train $\B$ (\eg, problem-solving scenarios), she may still detect radioactivity by combining the significance levels.

\paragraph*{Mixing instruction datasets from different sources.}


\begin{table}[t!]
    \centering
    \begin{minipage}{0.48\textwidth}
        \centering
        \caption{
            Mixing instruction datasets from different sources.
            The fine-tuning is done with the setup presented in Sec.~\ref{chap6/sec:instruction}, with $\rho$=$10\%$ of watermarked data, mixing either with human or synthetic instructions.
        }
        \label{chap6/table:data-sources}
        \footnotesize
        \begin{tabular}{c|c}
            \toprule
            Major data source
            & Average $\logpval$ \\
            \midrule
            Machine & -15 \\
            Human & -32 \\
            \bottomrule
        \end{tabular}
    \end{minipage}
    \hfill
    \begin{minipage}{0.48\textwidth}
        \centering
        \caption{
            Sequential fine-tuning to remove the watermark traces when fine-tuning.
            The first fine-tuning is done with the setup presented in Sec.~\ref{chap6/sec:instruction}, with $\rho$=$10\%$ of watermarked data, and the second on OASST1.
            }
        \label{chap6/table:results_FineTuning}
        \footnotesize
        \begin{tabular}{c|c}
        \toprule
        Second fine-tuning & Average $\logpval$ \\
        \midrule
        \xmarkg & -15 \\
        \cmarkg & -8 \\
        \bottomrule
        \end{tabular}
    \end{minipage}
\end{table}



We conduct the same experiment as in Sec.~\ref{chap6/sec:instruction}, but replace the non-watermarked synthetic instructions by human-generated ones (from the Open Assistant dataset OASST1~\citep{kopf2024openassistant}).
We report in Tab.~\ref{chap6/table:data-sources} the detection results in the open / unsupervised scenario, with $\rho=10\%$ of watermarked data.
Interestingly, with the exact same setting as the one explored in Sec.~\ref{chap6/sec:detection-setup}, the radioactivity signal is stronger in this case.
Our speculation is that this might be due to fewer overlapping sequences of $k$+1-grams between the two distributions.



\subsection{Possible defenses}

We now assume Bob is aware of watermarking radioactivity and tries to remove the watermark traces before, during or after fine-tuning.
For instance, he might attempt to rephrase the watermarked instructions, use a differentially private training, or fine-tune his model on human-generated data -- which we do in the following experiment.
The logic is overall the same as previously pointed out in the fine-tuning ablations.
If the original watermark is weaker or if the fine-tuning overfits less, then radioactivity will be weaker too.
Therefore, the radioactivity detection test will be less powerful, but given a sufficient amount of data, it may still be able to detect traces of the watermark.

\paragraph{Radioactivity ``purification''.}\label{chap6/ref:purification}
We investigate the impact of a second fine-tuning on human-generated data to remove the watermark traces, through the following experiment.
After having trained his model on a mix of watermarked and non-watermarked data, as in Sec.~\ref{chap6/sec:instruction}, Bob fine-tunes his model a second time on human-generated text (from OASST1, as in the previous paragraph), with the same fine-tuning setup.
\autoref{chap6/table:results_FineTuning} shows that the second-fine-tuning divides by $2$ the significance level of the statistical test, although it does not completely remove the watermark traces.




\section{Discussion on other approaches}\label{chap6/sec:discussion-other-approaches}



\subsection{Membership inference}\label{chap6/par:mia_wm}

\paragraph*{Method.}
In the open-model/supervised case, MIA evaluates the radioactivity of one sample/sentence by observing the loss (or perplexity) of $\B$ on carefully selected sets of inputs.
The perplexity is expected to be smaller on samples seen during training.
We extend this idea for our baseline radioactivity detection test of a non-watermarked text corpus.
The corpus of texts is divided into sentences (of 256 tokens) and $\B$'s loss is computed on each sentence. 
We calibrate it with the zlib entropy~\citep{roelofs2017zlib}, as done by \citet{carlini2021extracting} for sample-based MIA. 
The goal of the calibration is to account for the complexity of each sample and separate this from the over-confidence of $\B$. 

We test the null hypothesis $\mathcal{H}_0$: ``\textit{the perplexity of $\B$ on $\Tilde{D}^{\A}$ has the same distribution as the perplexity on new texts generated by $\A$}''.
Indeed, if $\B$ was not fine-tuned on portions of $\Tilde{D}^\A$, then necessarily $\mathcal{H}_0$ is true.
To compare the empirical distributions we use a two-sample Kolmogorov-Smirnov test~\citep{massey1951kolmogorov}. 
Given the two cumulative distributions $F$ and $G$ over loss values, we compute the K-S distance as $d_{\mathrm{KS}}(F,G) = \mathrm{sup}_x |F(x) -G(x)|$.
We reject $\H_0$ if this distance is higher than a threshold, which sets the \pval\ of the test, and conclude that $\Tilde{D}^{\A}$ is radioactive for $\B$.
This is inspired by \citet{sablayrolles2018d}, who perform a similar K-S test in the case of image classification. 
It significantly diverges from the approach of~\citet{shi2023detecting}, which derives an empirical test by looking at the aggregated score from one tail of the distribution. 

\paragraph*{Experimental results}

We proceed as in Sec.~\ref{chap6/sec:radioactivity_detection} for the setup where MIA is achievable:
Alice has an \emph{open-model} access to $\B$ and is aware of all data $\Tilde{D}^\A$ generated for Bob (supervised setting). 
Bob has used a portion $D^\A$ for fine-tuning $\B$, given by the degree of supervision $d$, as defined in Sec.~\ref{chap6/sec:degreeofsupervision}.
We use the K-S test to discriminate between the calibrated perplexity of $\B$ on: $\mathcal D_{(0)}$ containing 5k instruction/answers (cut at 256 tokens) that were not part of $\B$'s fine-tuning; and $\mathcal D_{(d)}$ containing $(1/d)\times$5k instruction/answers from which $5k$ were.
Distribution $\mathcal D_{(d)}$ simulates what happens when Bob generates a lot of data and only fine-tunes on a few.


\definecolor{curve1}{HTML}{8B188B}
\definecolor{curve2}{HTML}{FFA319}
\begin{figure}[b!]
    \centering
    \begin{subfigure}[b]{0.45\textwidth}
        \centering
        \includegraphics[width=\linewidth]{chapter-6/figs/MIA_ft_vs_noft.pdf}
        \caption{
            Distributions of the calibrated loss of $\B$ across two types of distributions generated by $\A$: 
            texts generated by $\A$ outside of $\B$'s fine-tuning data ({\color{curve1} purple}), texts of $\Tilde{D}^{\A}$ of which $d\%$ were used during training ({\color{curve2} orange}).
        }
        \label{chap6/fig:calibrated-loss}
    \end{subfigure}\hfill
    \begin{subfigure}[b]{0.5\textwidth}
        \centering
        \includegraphics[width=\linewidth,clip, trim=0 0 0 0cm]{chapter-6/figs/MIA_vs_WM.pdf}
        \caption{
            We report the \pval s of the {\color{curve1} K-S detection test} (no WM when training) and of the {\color{curve2} WM detection} ($\rho=5\%$ of WM when training) against the degree of supervision $d$ (proportion of Bob's training data known to Alice).
        }
        \label{chap6/fig:mia_vs_wm}
    \end{subfigure}
    \caption{
        Comparative analysis of membership inference and watermarking for radioactivity detection, in the open-model setup.
        (\emph{Left}) MIA aims to detect the difference between the two distributions. 
        It gets harder as $d$ decreases, since the actual fine-tuning data is mixed with texts that Bob did not use.
        (\emph{Right}) Therefore, for low degrees of supervision ($<2\%$), MIA is no longer effective, while WM detection gives \pval s lower than $10^{-5}$.
    }
    \label{chap6/fig:mia-comparative-analysis}
\end{figure}

\autoref{chap6/fig:calibrated-loss} compares the distributions for $d=0$ and $d>0$. 
As $d$ decreases, the data contains more texts that Bob did not fine-tune on, so the difference between the two perplexity distributions is fainter.
The direct consequence is that the detection becomes more challenging.
\autoref{chap6/fig:mia_vs_wm} shows that when $d>2\%$, the test rejects the null hypothesis at a strong significance level: 
$p < 10^{-5}$ implies that when radioactive contamination is detected, the probability of a false positive is $10^{-5}$.
It is random in the edge case $d=0$, the unsupervised setting where Alice lacks knowledge about the data used by Bob. 
In contrast, radioactivity detection on watermarked data succeeds in that setting.









\subsection{IP protection methods}\label{chap6/sec:ipp}

There are active methods that embed watermarking in a text specifically to detect if a model is trained on it~\citep{zhao2023protecting,he2022cater, he2022protecting}. 
Our setup is a bit different because Alice's primary goal is not to protect against model distillation but to make her outputs more identifiable. 
Consequently, ``radioactivity'' is a byproduct.
Although our focus is not on active methods, we discuss three state-of-the-art methods for intellectual property protection and their limitations in our detection setting.

\paragraph{\citet{zhao2023protecting}.}
The goal is to detect if a specific set of answers generated by Alice's model has been used in training. 
There are three main limitations.
The authors design a watermark dependent on the previous prompt (and unique to it). 
Each detection algorithm (2 and 3) assumes that the sample probing data $D$ is from the training data of the suspect model $\B$.
Therefore, the method is only demonstrated in the \textit{supervised} setting.
Moreover, all experiments are performed in the \textit{open}-model setting, where Alice has access to $\B$'s weights, except for Sec.~5.2 ``Watermark detection with text alone'' (still assuming a \textit{supervised} access), where one number is given in that setting.
Finally, it does not rely on a grounded statistical test and requires an empirically set threshold.

\paragraph{\cite{he2022protecting} and \cite{he2022cater}.} 
These methods substitute synonyms during generation to later detect an abnormal proportion of synonyms in the fine-tuned model.
However, the main hypothesis for building the statistical test -- the frequency of synonyms in a natural text is fixed -- fails when scoring a large number of tokens.
Using the \href{https://github.com/xlhex/NLG_api_watermark}{official author's code} on non-watermarked instruction-answers yields extremely low \pval s, making the test unusable at our scale, as shown in \autoref{chap6/table:pvalues_ginsew}.

\begin{table}[t!]
\centering
\caption{\pval s for non-watermarked instruction-answers}
\label{chap6/table:pvalues_ginsew}
\footnotesize
\begin{tabular}{ *{3}{l} }
    \toprule
    Number of lines & Number of characters (in thousands) & \pval \\
    \midrule
    10 & 1.5 & 0.76 \\
    100 & 30.9 & 0.16 \\
    500 & 148.3 & 2.2 $\times 10^{-7}$ \\
    1000 & 333.9 & 8.8 $\times 10^{-13}$ \\
    \bottomrule
\end{tabular}

\end{table}

\section{Details on \pval s}\label{chap6/sec:additional}

This section provides more information on the experimental setup, reporting, and correctness experiments for the tests used in the chapter.

\subsection{Reporting}\label{chap6/app:repporting}

\paragraph{Interpretation of the average $\logpval$.}
Given that \pval s often span various orders of magnitude, we report the average of the $\logpval$ over multiple runs rather than the average of the \pval s themselves. 
We interpret the average $\logpval$ as though it could be directly read as a \pval, although a direct translation to a rigorous statistical \pval\ is not possible.
Fig.~\ref{chap6/fig:box_plot_open} shows box-plots with additional statistics.

\begin{figure}[b!]
    \centering
    \includegraphics[width=0.5\linewidth]{chapter-6/figs/boxplot_open.pdf}
    \captionsetup{font=small}
    \caption{Box plot for the $\logpval$ in the open/unsupervised setting with varying $\rho$, the proportion of watermarked fine-tuning data.}
    \label{chap6/fig:box_plot_open}
\end{figure}


\paragraph{Average over multiple runs.} 
Due to computational constraints, standard deviations for the $\logpval$ are not calculated across multiple models trained on different instruction data for each setting. 
Instead, we generate the same volume of data (14M tokens) in addition to the data used to fine-tune the model. 
In the open-model setting, we run detection on ten distinct chunks of this additional data. 
In the closed-model setting, we prompt the model with ten different chunks of new sentences and score the responses.


\begin{figure}[b!]
    \begin{minipage}{0.48\textwidth}
        \centering
        \includegraphics[width=0.99\linewidth,clip, trim=0 0 0 0cm]{chapter-6/figs/under_H0_closed_final.pdf}
        \caption{
            \pval\ under $\mathcal{H}_0$ with closed-model access.
            We fine-tune $\B$ on non-watermarked instructions and prompt $\B$ with watermarked instructions, scoring the distinct $(k+1)$-grams from the answers, excluding $k$-grams from the instruction. 
        }
        \label{chap6/fig:pvalu_under_H0_closed}
    \end{minipage} \hfill
    \begin{minipage}{0.48\textwidth}
        \centering
        \includegraphics[width=0.99\linewidth,clip, trim=0 0 0 0cm]{chapter-6/figs/under_H0_open.pdf}
        \caption{
            \pval\ under $\mathcal{H}_0$ with open-model access. 
            We fine-tune $\B$ on non-watermarked instructions, perform the open-model detection and score distinct $(k+1)$-grams only on $k$-grams that $\B$ did not previously attend to when generating the token. 
        }
        \label{chap6/fig:pvalu_under_H0}
    \end{minipage}
\end{figure}



\subsection{Correctness experiments}\label{chap6/app:correctness}

We validate the correctness of our statistical tests by examining the null hypothesis $\mathcal{H}_0$, which represents when the model was not fine-tuned on any watermarked data ($\rho=0$).
Our goal is to show that the \pval\ is approximately uniform under $\mathcal{H}_0$, with a mean of $0.5$ and a standard deviation of $\approx 0.28$.
Instead of fine-tuning model $\B$ with various datasets, we vary the hyper-parameters of the detection algorithm at a fixed fine-tuned model to save computation and memory. 
This approach is not sufficient on its own to establish the validity of the test, but it provides some evidence for it.
Here, $\B$ is the model fine-tuned on non-watermarked instructions as described in Sec.~\ref{chap6/sec:instruction}.


\paragraph{Closed-model.}
We prompt $\B$ with $\approx10$k watermarked instructions using the method by~\citet{kirchenbauer2023watermark}, $\delta=3$ and $k=2$, and three different seeds $\mathsf{s}$. 
We score the answers using the proposed de-duplication and repeat this $10$ times on different datasets. 
\autoref{chap6/fig:pvalu_under_H0_closed} shows that the average \pval\ is close to $0.5$ and the standard deviation is close to $0.28$, as expected under $\mathcal{H}_0$.


\paragraph{Open-model.}
We generate text with eight distinct watermarking methods (methods of~\citet{aaronson2023watermarking} and ~\citet{kirchenbauer2023watermark} for $k\in\{1,2,3,4\}$).
We divide each dataset into three and apply the radioactivity detection test on these 24 segments, each containing over 1.5 million tokens. 
We score the distinct $(k+1)$-grams from the answers, excluding $k$-grams from the instruction.
\autoref{chap6/fig:pvalu_under_H0} shows again that the average \pval\ is close to $0.5$ and the standard deviation is close to $0.28$.

\section{Conclusion}
\label{sec:Conclusion}
This work evaluates proprietary and open-weight models in agentic frameworks for handling ambiguity in software engineering. In code generation, to effectively integrate new information into the solution, an agent must detect ambiguity and ask targeted questions. Our key findings are:
\begin{itemize}[itemsep=0pt, topsep=0pt]
    \item Given an underspecified input, Claude Sonnet 3.5 and Claude Haiku 3.5 with interaction can achieve 80\% of their performance with a well-specified input. In contrast, open-weight models struggle: Deepseek relies on navigational cues to locate relevant files, while Llama 3.1 70B extracts limited information from the user.
    \item LLMs do not interact unless explicitly prompted, and their ambiguity detection is highly sensitive to prompt variations. Only Claude Sonnet 3.5 achieves a higher accuracy of 84\% in distinguishing between well-specified and underspecified input.

    \item Claude Sonnet 3.5, Haiku 3.5, and Deepseek effectively extract new, detailed user information, whereas Llama 3.1 struggles to ask the right questions.
    
\end{itemize}
Despite these advances, a gap remains between resolve rates for underspecified vs. fully specified issues. Open-weight models need better interaction strategies to improve resolution, while proprietary models, particularly Claude Haiku 3.5, require stronger prompting to engage interactively. This work establishes the current state-of-the-art in handling ambiguity through interaction, breaking the resolution process into multiple steps.




