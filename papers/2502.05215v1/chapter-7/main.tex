
\chapter{Functional Invariants to Watermark Large Transformers}\label{chapter:invariants}

This chapter is based on the paper \fullcite{fernandez2024functional}.

The rapid growth of transformer-based models increases the concerns about their integrity and ownership insurance. 
In particular, models are vulnerable to leaks and unauthorized redistribution, which can result in significant financial losses and reputational damage for companies. 
Model watermarking addresses this issue by embedding a unique identifier into the model weights, while preserving its performance. 
It may then be extracted to identify the model's owner.
However, most existing approaches require to optimize the weights to imprint the watermark signal, which is not suitable at scale due to the computational cost.
This chapter explores watermarks with virtually no computational cost, applicable to a non-blind white-box setting (assuming access to both the original and watermarked networks). 
They generate functionally equivalent copies by leveraging the invariance of models, via operations like dimension permutations or scaling/unscaling. 
This enables to watermark models without any change in their outputs and remains stealthy.
Experiments demonstrate the effectiveness of the approach  and its robustness against various model transformations (fine-tuning, quantization, pruning), making it a practical solution to protect the integrity of large models.

\newpage
\section{Introduction}

% State of the world (robots for creative activites)
The term ``robot,'' originally signifying `forced labor,' has long been associated with labor and work. Robots have demonstrated their utility in various automated productive and social contexts, where the primary goals are improving productivity, safety, and fostering social interactions with humans~\cite{simoes2022designing, weidemann2021role, honig2018understanding}. However, an increasing number of cases feature using of robots in creative settings. Unlike productive contexts, where the focus is on efficiency and task completion~\cite{arents2022smart}, or social contexts, where communication and trust are prioritized~\cite{nam2020trust, saunderson2019robots}, creative environments prioritize artistic innovation and expression~\cite{hsueh2024counts}. This shift fundamentally alters the dynamics of human-robot interaction, redefining the roles and expectations for both humans and robots.

For instance, robots’ social behaviors are leveraged to support the generation and expression of creative ideas~\cite{hu2021exploring, sandoval2022human, alves2020creativity}, and programmable robotic movements and trajectories are employed to inspire artistic activities such as sketching~\cite{lin2020your}. These studies often engage participants from creative fields who possess limited prior experience with robotics, and are typically conducted in short-term, experimental settings. Consequently, the findings from these studies remain constrained since much can be learned from professional practitioners' experiences to inform system design such as digital fabrication~\cite{hirsch2023nothing}. There is a notable gap in research examining the long-term, active, and practical experience of integrating robotic systems into the creative processes. As a result, the deeper insights into how robots facilitate and shape creative processes, beyond simply augmenting human creativity, remain underexplored. In this study, we aim to better understand the impacts of robots on creative processes and outcomes.

As early as Leonardo da Vinci's 16th century ``Automaton,'' artists have explored the creative affordances of robotic systems~\cite{shanken2002cybernetics, pagliarini2009development, jeon2017robotic}. The artistic creation process typically encompasses various stages, including the exploration of materials and techniques, ongoing experimentation and iteration, and the continual refinement of the artists' insights into their creative subjects~\cite{lewis2023art, sturdee2022state}. Therefore, investigating the artistic process involving robots offers an opportunity to gain deeper insights into robots' creative potential. Robotic art, in particular, provides a compelling case for this exploration.

We define robotic art as artworks that utilize robotic or automated machines to create artistic experiences and tangible artifacts. One example is robotic installation art, in which robots are programmed to follow specific rules that embody the artist’s expression (\autoref{fig:teaser} (a)). Another example is responsive art, in which robots react to their environment, with behaviors that change over time or in response to spectators (\autoref{fig:teaser} (b)). Additionally, there are robotic creators, which possess a degree of agency, allowing them to collaborate with human artists and produce works that extend beyond mere replication of human-created art (\autoref{fig:teaser} (c) and (d)). As such, robotic art becomes a rich case for exploring human-machine interactions in creative contexts. Gaining a deeper understanding of how robots facilitate artistic expression can provide insights for designing computing systems to support creative activities~\cite{gomez2021robot}.

% Therefore, we did...
We draw on semi-structured, in-depth interviews with renowned professional robotic artists to investigate the use of robots in artistic practice. Specifically, our goal is to understand how artistic exploration of robotic systems challenges conventional assumptions about the functions of robots, such as their roles in automating repetitive tasks or serving human needs. We also explore the implications of robots in the artistic process and examine how creativity may emerge within robotic art. To address these interrelated inquiries, our study focuses on the practice of robotic art, posing the research question: \textit{How do robotic artists utilize robots in their artistic practice?} We approach this inquiry through the perspectives and experiences of robotic artists, who creatively design, modify, and repurpose robotic systems for artistic expression and exploration.

% The key findings are...
Our findings highlight the social, material, and temporal dimensions of artists' practices that shape their creativity and artistic outcomes. The creation of robotic art is largely a social process, as artists receive both explicit and implicit feedback through the audience's reactions and reception of their work. Simultaneously, the embodiment and malfunctions inherent to robotic systems drive artistic experimentation. The temporal processes of creation and exhibition, beyond just the final product, further enhance the creative value. Our empirical analysis presents how creativity emerges through the interplay of social, material, and temporal interactions among artists, robots, audiences, and the environment.

% The contributions of this work are...
We make two main contributions to HCI in this study. 
First, we elucidate the interactive mechanisms among key actors---human creators, machines, audiences, and environments---within the practice of robotic art, a topic that remains underexplored in HCI. Our findings reveal the significance of sociality (e.g., interactions between artists and audiences), materiality (e.g., the embodiment and malfunctions of robots), and temporality (e.g., the processes of creation and exhibition) in shaping creative values. We propose that these three facets are central to the creative process and facilitate the emergence of creativity in robotic art.
Second, drawing from the findings, we offer implications for \textit{socially informed}, \textit{material-attentive}, and \textit{process-oriented} creation with computing systems. We suggest leveraging these three aspects to enhance creativity and the creative experience. Specifically, we discuss the value of incorporating implicit audience feedback, designing with technical malfunctions, and focusing on the post-creation process to foster alternative creative experiences with machines~\cite{alter2010designing, juarez2022glitch}.



\section{Related Work}
\label{sec:related}
\subsection{Collaborative Systems}
In the era of rapid growth in medical foundational models~\cite{huang2023visual,wang2022medclip, zhang2024data}, the top-down model development paradigm limits model capabilities by heavily relying on the resources available to the model builders. 
Such paradigm often restricts the potential of these models, as they cannot effectively utilize the diverse, private, and decentralized resources that exist within the broader medical community.
In contrast, collaborative systems present a promising alternative, offering a more flexible approach to model development.

Collaborative systems enable institutions to share knowledge by allowing distributed collaborators to contribute to a common goal~\cite{boulemtafes2020review}. 
To further protect patient privacy, federated learning (FL)~\cite{mcmahan2017communication} was proposed to alleviate such privacy concerns as server aggregating parameter updates from multiple clients without sharing their local data. 
While subsequent optimizations, such as aggregation algorithms~\cite{mcmahan2017communication, zhao2018federated, li2020federated}, secure learning~\cite{hardy2017private, xie2021crfl}, fairness improvements~\cite{sharma2022federated, zhao2022dynamic} and its application in medicine~\cite{kumar2024privacy}, have enhanced the capacity and applicability of FL, its real-world flexibility remains limited. This is primarily due to the need for synchronous updates, which require the server and clients to stay in sync, or model updates will be blocked.
This synchrony issue can be mitigated by open-source software platforms (e.g., GitHub~\cite{github}), allowing independent contributions from individual developers asynchronously. Such an asynchronous scheme enables faster iteration and the integration of specialized expertise, thus offering a more flexible and scalable approach.

Unlike synchronous collaboration, asynchronous collaboration does not require collaborators to work simultaneously and collaborators can individually complete their updates.
While the concept of asynchronous collaboration has been widely used in software development, its machine-learning applications remain limited~\cite{kandpal2023git, raffel2023building}. 
With the rise of global collaboration, large models~\cite{sahajBERT, le2023bloom} are usually co-developed by collaborators given various levels of data availability. However, this collaborative scheme requires the aggregation of local data and online synchronous cooperation of developers.
Software-like model update system~\cite{raffel2023building} alleviates the synchronous problem, where models are updated incrementally, similar to software development, by introducing merging and version control to model development.
However, the existing collaborative version control system~\cite{kandpal2023git} fails to address the complexities of medical scenarios because of the heavy dependency on plain parameter averaging across the full model without accounting for the varying requirements of different tasks.
To respond, we propose MedForge, which enables an asynchronous collaborative system and ensures strong robustness toward a continuous, community-driven enhancement of medical models while overcoming potential data leakage.

\begin{figure*}[t]
\begin{center}
\includegraphics[width=.85\linewidth]{fig_overview_v3.pdf}
\end{center}
\caption{
FastAtlas Overview: In each frame, we compute charts spanning fully or partially visible triangles (a), determine texture space bounding boxes for the visible portions of the view-space projections of each chart, and tightly pack these boxes into atlases (b, here $2K \times 2K$). We simultaneously bijectively parameterize and shade the charts into their atlas boxes, obtaining high quality texture space shading (c), and use this shading to render the shaded frames (d).}
\label{fig:overview}
\label{fig:alg_overview}
\end{figure*}

\section{Overview}
\label{sec:overview}
Our work has two core contributions: a real-time, GPU-based algorithm for tight packing of general parameterized charts into compact atlases; and a real-time TSS method that
utilizes this packing.  

\paragraph*{FastAtlas Packing.}
FastAtlas runs entirely on the GPU as a series of compute shaders. It takes the bounding boxes of parameterized charts as input, and packs them into an atlas (Fig~\ref{fig:overview}b, Sec.~\ref{sec:pack}). As such, the only input it requires are the dimensions of the bounding boxes.
Its outputs are deterministic; identical input charts are packed into identical atlases. This is critical for TSS and similar applications, as it ensures that consecutive frames taken from the same camera view have the same shading. Even minute shading differences across such frames can cause sampling jitter, leading to undesirable flicker \cite{baker2012rock}. 
While prior methods such as \cite{mueller2018shading,hladky2019tessellated,hladky2021snakebinning,Neff2022MSA} cap the dimensions of the charts that can be packed as-is for a given atlas size, and scale down all charts that exceed these dimensions, we scale all charts by the same factor, and do so only when strictly necessary to achieve packing success (Figs~\ref{fig:atlas},~\ref{fig:sas_issues}). 

\paragraph*{TSS using FastAtlas.}
Our end-to-end TSS atlas generation method combines the packing method above with a novel approach for computing seamless per-frame charts. 
We define our charts as the connected components of the visible surfaces in each frame (Fig.~\ref{fig:overview}a), and efficiently compute them using a parallel union-find algorithm (Sec.~\ref{sec:visible}). Since the boundaries of these charts coincide with the contours of the rendered surface, they are {\em invisible} to the viewer. This approach 
eliminates the artifacts caused by shading discontinuities along visible seams (Fig.~\ref{fig:seams}). 

\begin{parWithWrapFigure}
\begin{wrapfigure}{l}{.27\columnwidth}%
\includegraphics[width=\linewidth]{fig_inset_view_plane.pdf}%
\end{wrapfigure}
We bijectively parametrize the {\em visible portions} of our charts by projecting them to view space (inset). This maps a constant number of texels to each pixel in the final rendered output, evenly distributing residual undersampling error across all image pixels. While conceptually straightforward, efficiently parameterizing charts containing partially visible triangles using viewspace projection is non-trivial, as the visible portions may no longer be triangular (e.g. green triangle in the inset); applying naive projection to triangles with vertices behind the camera may produce ill-posed results. Clipping triangles before projection is both computationally expensive and significantly complicates downstream operations. We avoid explicit clipping by observing that all that is required for atlas packing is the dimensions of, potentially conservative, bounding boxes of these projected visible portions. We compute such bounding boxes without explicit chart clipping by adapting a conservative screen coverage estimator \shortcite{Blinn:CalculatingScreenCoverage} (Sec.~\ref{sec:box}). We then pack the computed boxes using FastAtlas. 
\end{parWithWrapFigure}

Finally, we shade the visible portion of each chart into its corresponding atlas bounding box (Fig~\ref{fig:overview}c). 
The resulting texture is then used during rasterization as a standard texture map (Fig. ~\ref{fig:overview}d). 
Our framework is compatible with all existing approaches for texture space shading, including forward shading, raytraced illumination, or deferred shading in texture space \cite{baker:2016}. In the examples shown, we use the standard forward shading based rendering pipeline included in the G3D Innovation Engine \cite{G3D17}, a commercial grade renderer.


\subsection{Model Merging}
In collaborative systems, proper model merging becomes increasingly vital for improving model knowledge integration from multiple sources in a resource-limited environment~\cite{li2023deep, yang2024model, goddard2024arcee}. Conceptually, model merging strategies can be categorized into entire model merging and partial model merging.

Entire model merging involves combining multiple model parameters to participate in the merging process by several means. Entire model merging can be viewed as an optimization problem~\cite{Matena_Raffel_2021, jin2022dataless, mavromatis2024packllm} or an alignment problem~\cite{ainsworth2022git, jordan2022repair, xu2024training, ainsworth2022git}, each offering unique advantages depending on the task at hand.
In the optimization-based approach, the goal is to find the best combination of multiple models to enhance performance and efficiency. For instance, using Fisher information approximation~\cite{Matena_Raffel_2021}, the optimization-based model merging can be interpreted as selecting parameters that maximize the joint likelihood of the models' posterior distributions. The optimization of model merging can also be guided by minimizing the prediction differences between the merged model and individual models~\cite{jin2022dataless}. 
With the development of large language models (LLM), optimization-based method is used to fuse multiple LLMs at test-time by minimizing perplexity over the input prompt~\cite{mavromatis2024packllm}.
To highlight, optimization-based methods are beneficial for scenarios requiring enhanced model performance and efficiency to integrate model parameters, while alignment-based methods~\cite{ainsworth2022git, jordan2022repair} are better suited for maintaining consistency and interpretability, facilitating critical information sharing across models.
For example, a training-free model merging strategy aligns relevant models by using a similarity matrix of their representations in both activation and weight spaces~\cite{xu2024training}.
Further, the alignment between the independently trained model and a reference model not only works for models with the same architecture but also for arbitrary model architectures~\cite{ainsworth2022git}.
In summary, the entire model merging methods can effectively integrate existing models into a merged model with enhanced functionality. However, they could lead to increased computational complexity and reduced flexibility, making them less scalable and harder to implement across diverse tasks.

Partial model merging refers to combining only specific components or layers of models to improve model merging efficiency and decrease the computational cost. 
Such specific components can come from the same network~\cite{kingetsu2021neural}, where the original network is divided into subnetworks for different purposes, and these subnetworks can then be recombined for new tasks.
Additionally, modules can originate from different functional networks and be merged using various strategies. For instance, arithmetic operations are applied in \cite{zhang2023composing} to fuse parameter-efficient modules.
While merging modules from different networks provides flexibility, the process also requires a selection strategy to ensure the resulting model aligns with the specific needs of the inference stage. 
The selection strategies are commonly designed based on the similarity of task~\cite{lv2023parameter} and domain clustering performance~\cite{chronopoulou2023adaptersoup}. Alternatively, the mixture-of-experts methods use a routing strategy to select appropriate component modules~\cite{ponti2023combining}. However, these strategies often require significant time and computational resources to filter through a large model pool. 
In contrast, LoRAHub~\cite{huang2023lorahub} offers a more lightweight approach, combining various LoRA modules for different tasks with minimal model training. Nevertheless, LoRAHub lacks flexibility for incremental updates, especially when handling unseen tasks.

Although the existing model merging approaches effectively combine the capabilities of individual models, these approaches often rely on raw data, leading to potential privacy risks. Our proposed MedForge emphasizes the prevention of raw data usage, which is particularly crucial in medical scenarios. Additionally, MedForge offers an extensible capability for incremental learning, enabling continuous model improvement.



\vspace{-5pt}
\section{Method}
\label{sec:method}
\begin{figure*}[t]
\begin{center}
\includegraphics[width=.85\linewidth]{fig_overview_v3.pdf}
\end{center}
\caption{
FastAtlas Overview: In each frame, we compute charts spanning fully or partially visible triangles (a), determine texture space bounding boxes for the visible portions of the view-space projections of each chart, and tightly pack these boxes into atlases (b, here $2K \times 2K$). We simultaneously bijectively parameterize and shade the charts into their atlas boxes, obtaining high quality texture space shading (c), and use this shading to render the shaded frames (d).}
\label{fig:overview}
\label{fig:alg_overview}
\end{figure*}

\section{Overview}
\label{sec:overview}
Our work has two core contributions: a real-time, GPU-based algorithm for tight packing of general parameterized charts into compact atlases; and a real-time TSS method that
utilizes this packing.  

\paragraph*{FastAtlas Packing.}
FastAtlas runs entirely on the GPU as a series of compute shaders. It takes the bounding boxes of parameterized charts as input, and packs them into an atlas (Fig~\ref{fig:overview}b, Sec.~\ref{sec:pack}). As such, the only input it requires are the dimensions of the bounding boxes.
Its outputs are deterministic; identical input charts are packed into identical atlases. This is critical for TSS and similar applications, as it ensures that consecutive frames taken from the same camera view have the same shading. Even minute shading differences across such frames can cause sampling jitter, leading to undesirable flicker \cite{baker2012rock}. 
While prior methods such as \cite{mueller2018shading,hladky2019tessellated,hladky2021snakebinning,Neff2022MSA} cap the dimensions of the charts that can be packed as-is for a given atlas size, and scale down all charts that exceed these dimensions, we scale all charts by the same factor, and do so only when strictly necessary to achieve packing success (Figs~\ref{fig:atlas},~\ref{fig:sas_issues}). 

\paragraph*{TSS using FastAtlas.}
Our end-to-end TSS atlas generation method combines the packing method above with a novel approach for computing seamless per-frame charts. 
We define our charts as the connected components of the visible surfaces in each frame (Fig.~\ref{fig:overview}a), and efficiently compute them using a parallel union-find algorithm (Sec.~\ref{sec:visible}). Since the boundaries of these charts coincide with the contours of the rendered surface, they are {\em invisible} to the viewer. This approach 
eliminates the artifacts caused by shading discontinuities along visible seams (Fig.~\ref{fig:seams}). 

\begin{parWithWrapFigure}
\begin{wrapfigure}{l}{.27\columnwidth}%
\includegraphics[width=\linewidth]{fig_inset_view_plane.pdf}%
\end{wrapfigure}
We bijectively parametrize the {\em visible portions} of our charts by projecting them to view space (inset). This maps a constant number of texels to each pixel in the final rendered output, evenly distributing residual undersampling error across all image pixels. While conceptually straightforward, efficiently parameterizing charts containing partially visible triangles using viewspace projection is non-trivial, as the visible portions may no longer be triangular (e.g. green triangle in the inset); applying naive projection to triangles with vertices behind the camera may produce ill-posed results. Clipping triangles before projection is both computationally expensive and significantly complicates downstream operations. We avoid explicit clipping by observing that all that is required for atlas packing is the dimensions of, potentially conservative, bounding boxes of these projected visible portions. We compute such bounding boxes without explicit chart clipping by adapting a conservative screen coverage estimator \shortcite{Blinn:CalculatingScreenCoverage} (Sec.~\ref{sec:box}). We then pack the computed boxes using FastAtlas. 
\end{parWithWrapFigure}

Finally, we shade the visible portion of each chart into its corresponding atlas bounding box (Fig~\ref{fig:overview}c). 
The resulting texture is then used during rasterization as a standard texture map (Fig. ~\ref{fig:overview}d). 
Our framework is compatible with all existing approaches for texture space shading, including forward shading, raytraced illumination, or deferred shading in texture space \cite{baker:2016}. In the examples shown, we use the standard forward shading based rendering pipeline included in the G3D Innovation Engine \cite{G3D17}, a commercial grade renderer.


Our goal is to increase the robustness of T2I models, particularly with rare or unseen concepts, which they struggle to generate. To do so, we investigate a retrieval-augmented generation approach, through which we dynamically select images that can provide the model with missing visual cues. Importantly, we focus on models that were not trained for RAG, and show that existing image conditioning tools can be leveraged to support RAG post-hoc.
As depicted in \cref{fig:overview}, given a text prompt and a T2I generative model, we start by generating an image with the given prompt. Then, we query a VLM with the image, and ask it to decide if the image matches the prompt. If it does not, we aim to retrieve images representing the concepts that are missing from the image, and provide them as additional context to the model to guide it toward better alignment with the prompt.
In the following sections, we describe our method by answering key questions:
(1) How do we know which images to retrieve? 
(2) How can we retrieve the required images? 
and (3) How can we use the retrieved images for unknown concept generation?
By answering these questions, we achieve our goal of generating new concepts that the model struggles to generate on its own.

\vspace{-3pt}
\subsection{Which images to retrieve?}
The amount of images we can pass to a model is limited, hence we need to decide which images to pass as references to guide the generation of a base model. As T2I models are already capable of generating many concepts successfully, an efficient strategy would be passing only concepts they struggle to generate as references, and not all the concepts in a prompt.
To find the challenging concepts,
we utilize a VLM and apply a step-by-step method, as depicted in the bottom part of \cref{fig:overview}. First, we generate an initial image with a T2I model. Then, we provide the VLM with the initial prompt and image, and ask it if they match. If not, we ask the VLM to identify missing concepts and
focus on content and style, since these are easy to convey through visual cues.
As demonstrated in \cref{tab:ablations}, empirical experiments show that image retrieval from detailed image captions yields better results than retrieval from brief, generic concept descriptions.
Therefore, after identifying the missing concepts, we ask the VLM to suggest detailed image captions for images that describe each of the concepts. 

\vspace{-4pt}
\subsubsection{Error Handling}
\label{subsec:err_hand}

The VLM may sometimes fail to identify the missing concepts in an image, and will respond that it is ``unable to respond''. In these rare cases, we allow up to 3 query repetitions, while increasing the query temperature in each repetition. Increasing the temperature allows for more diverse responses by encouraging the model to sample less probable words.
In most cases, using our suggested step-by-step method yields better results than retrieving images directly from the given prompt (see 
\cref{subsec:ablations}).
However, if the VLM still fails to identify the missing concepts after multiple attempts, we fall back to retrieving images directly from the prompt, as it usually means the VLM does not know what is the meaning of the prompt.

The used prompts can be found in \cref{app:prompts}.
Next, we turn to retrieve images based on the acquired image captions.

\vspace{-3pt}
\subsection{How to retrieve the required images?}

Given $n$ image captions, our goal is to retrieve the images that are most similar to these captions from a dataset. 
To retrieve images matching a given image caption, we compare the caption to all the images in the dataset using a text-image similarity metric and retrieve the top $k$ most similar images.
Text-to-image retrieval is an active research field~\cite{radford2021learning, zhai2023sigmoid, ray2024cola, vendrowinquire}, where no single method is perfect.
Retrieval is especially hard when the dataset does not contain an exact match to the query \cite{biswas2024efficient} or when the task is fine-grained retrieval, that depends on subtle details~\cite{wei2022fine}.
Hence, a common retrieval workflow is to first retrieve image candidates using pre-computed embeddings, and then re-rank the retrieved candidates using a different, often more expensive but accurate, method \cite{vendrowinquire}.
Following this workflow, we experimented with cosine similarity over different embeddings, and with multiple re-ranking methods of reference candidates.
Although re-ranking sometimes yields better results compared to simply using cosine similarity between CLIP~\cite{radford2021learning} embeddings, the difference was not significant in most of our experiments. Therefore, for simplicity, we use cosine similarity between CLIP embeddings as our similarity metric (see \cref{tab:sim_metrics}, \cref{subsec:ablations} for more details about our experiments with different similarity metrics).

\vspace{-3pt}
\subsection{How to use the retrieved images?}
Putting it all together, after retrieving relevant images, all that is left to do is to use them as context so they are beneficial for the model.
We experimented with two types of models; models that are trained to receive images as input in addition to text and have ICL capabilities (e.g., OmniGen~\cite{xiao2024omnigen}), and T2I models augmented with an image encoder in post-training (e.g., SDXL~\cite{podellsdxl} with IP-adapter~\cite{ye2023ip}).
As the first model type has ICL capabilities, we can supply the retrieved images as examples that it can learn from, by adjusting the original prompt.
Although the second model type lacks true ICL capabilities, it offers image-based control functionalities, which we can leverage for applying RAG over it with our method.
Hence, for both model types, we augment the input prompt to contain a reference of the retrieved images as examples.
Formally, given a prompt $p$, $n$ concepts, and $k$ compatible images for each concept, we use the following template to create a new prompt:
``According to these examples of 
$\mathord{<}c_1\mathord{>:<}img_{1,1}\mathord{>}, ... , \mathord{<}img_{1,k}\mathord{>}, ... , \mathord{<}c_n\mathord{>:<}img_{n,1}\mathord{>}, ... , $
$\mathord{<}img_{n,k}\mathord{>}$,
generate $\mathord{<}p\mathord{>}$'', 
where $c_i$ for $i\in{[1,n]}$ is a compatible image caption of the image $\mathord{<}img_{i,j}\mathord{>},  j\in{[1,k]}$. 

This prompt allows models to learn missing concepts from the images, guiding them to generate the required result. 

\textbf{Personalized Generation}: 
For models that support multiple input images, we can apply our method for personalized generation as well, to generate rare concept combinations with personal concepts. In this case, we use one image for personal content, and 1+ other reference images for missing concepts. For example, given an image of a specific cat, we can generate diverse images of it, ranging from a mug featuring the cat to a lego of it or atypical situations like the cat writing code or teaching a classroom of dogs (\cref{fig:personalization}).
\vspace{-2pt}
\begin{figure}[htp]
  \centering
   \includegraphics[width=\linewidth]{Assets/personalization.pdf}
   \caption{\textbf{Personalized generation example.}
   \emph{ImageRAG} can work in parallel with personalization methods and enhance their capabilities. For example, although OmniGen can generate images of a subject based on an image, it struggles to generate some concepts. Using references retrieved by our method, it can generate the required result.
}
   \label{fig:personalization}\vspace{-10pt}
\end{figure}
\section{Empirical Evaluation}
\begin{table*}[!ht]
    \centering
    \resizebox{0.88\textwidth}{!}{    
    \begin{tabular}{r|cccccc|cccccc}
        \toprule 
        & \multicolumn{6}{c}{\textbf{LLaVA-1.5-7B}} & \multicolumn{6}{c}{\textbf{LLaVA-1.5-13B}} \\ 
        \cmidrule(lr){2-7}\cmidrule(lr){8-13}
        & \multicolumn{3}{c}{\textbf{MM-SafetyBench}} & \multicolumn{3}{c|}{\textbf{MOSSBench}} & \multicolumn{3}{c}{\textbf{MM-SafetyBench}} & \multicolumn{3}{c}{\textbf{MOSSBench}} \\
        \textbf{Method} & \textbf{DSR}$\uparrow$ & \textbf{RR}$\uparrow$ & \textbf{Avg}$\uparrow$ & \textbf{DSR}$\uparrow$ & \textbf{RR}$\uparrow$ & \textbf{Avg}$\uparrow$ & \textbf{DSR}$\uparrow$ & \textbf{RR}$\uparrow$ & \textbf{Avg}$\uparrow$ & \textbf{DSR}$\uparrow$ & \textbf{RR}$\uparrow$ & \textbf{Avg}$\uparrow$\\
        \midrule
        w/o Defense          & 0.06  & 0.98  & 0.52  & 0.14  & 0.97  & 0.55  & 0.10  & 0.97  & 0.53  & 0.30  & 0.96  & 0.63  \\
        \midrule
        \multicolumn{13}{c}{Baseline} \\
        \midrule
        Responsible          & 0.12  & 0.96  & 0.54  & 0.32  & 0.96  & 0.64  & 0.18  & 0.96  & 0.57  & 0.47  & 0.92  & 0.70  \\
        Policy               & 0.08  & 0.96  & 0.52  & 0.18  & 0.98  & 0.58  & 0.12  & 0.97  & 0.55  & 0.34  & 0.97  & 0.65  \\
        Demonstration        & 0.15  & 0.97  & 0.56  & 0.37  & 0.95  & 0.66  & 0.25  & 0.96  & 0.60  & 0.52  & 0.92  & \textbf{0.72}  \\
        SFT                  & 0.20  & 0.95  & 0.58  & 0.50  & 0.88  & 0.69  & 0.13  & 0.98  & 0.55  & 0.49  & 0.88  & 0.68 \\
        SafeDecoding         & 0.08  & 0.97  & 0.53  & 0.31  & 0.94  & 0.62  & 0.12  & 0.96  & 0.54  & 0.42  & 0.93  & 0.68  \\
        Caption              & 0.09  & 0.98  & 0.53  & 0.21  & 0.98  & 0.60  & 0.12  & 0.97  & 0.55  & 0.27  & 0.94  & 0.60  \\
        Caption (w/o image)  & 0.16  & 0.95  & 0.55  & 0.34  & 0.94  & 0.64  & 0.22  & 0.93  & 0.57  & 0.45  & 0.89  & 0.67 \\
        Intention            & 0.07  & 0.98  & 0.53  & 0.20  & 0.99  & 0.59  & 0.11  & 0.96  & 0.54  & 0.26  & 0.97  & 0.61  \\
        \midrule
        % \multicolumn{13}{c}{} \\
        % \midrule
        \midrule
        \multicolumn{13}{c}{SR++} \\
        \midrule        
        Responsible-Demonstration & 0.18 & 0.95 & 0.57 & 0.40 & 0.94 & 0.67 & 0.29 & 0.96 & 0.62 & 0.58 & 0.85 & \textbf{0.72} \\
        Responsible-Policy & 0.12 & 0.96 & 0.54 & 0.27 & 0.97 & 0.62 & 0.18 & 0.96 & 0.57 & 0.46 & 0.94 & 0.70 \\
        Policy-Demonstration & 0.13 & 0.96 & 0.55 & 0.37 & 0.97 & 0.67 & 0.20 & 0.96 & 0.58 &0.51 & 0.93 & \textbf{0.72}\\
        Responsible-Policy-Demonstration & 0.15 & 0.96 & 0.55 & 0.38 & 0.95 & 0.66 & 0.25 & 0.97 & 0.61 & 0.53 & 0.88 & 0.70\\
        \midrule
        \multicolumn{13}{c}{SR+MO} \\
        \midrule     
        Responsible-SFT & 0.56 & 0.93 & \textbf{0.75} & 0.61 & 0.72 & 0.67 & 0.35 & 0.96 & 0.65 & 0.74 & 0.62 & 0.68 \\
        Responsible-SafeDecoding & 0.30 & 0.96 & 0.63 & 0.54 & 0.87 & \underline{0.70} & 0.23 & 0.96 & 0.59 & 0.63 & 0.79 & 0.71\\
        Demonstration-SFT & 0.60 & 0.90 & \textbf{0.75} & 0.65 & 0.77 & \textbf{0.71} & 0.56 & 0.92 & \textbf{0.74} & 0.67 & 0.70 & 0.68\\
        Demonstration-SafeDecoding & 0.38 & 0.96 & \underline{0.67} & 0.55 & 0.87 & \textbf{0.71} & 0.40 & 0.96 & \underline{0.68} & 0.62 & 0.78 & 0.70\\
        \midrule
        \multicolumn{13}{c}{QR++} \\
        \midrule   
        Caption-Intention & 0.09 & 0.97 & 0.53 & 0.20 & 0.98 & 0.59 & 0.14 & 0.95 & 0.55 & 0.26 & 0.96 & 0.61\\
        % Caption-Intention (w/o image) & 0.18 & 0.96 & 0.57 & 0.32 & 0.95 & 0.64 & 0.25 & 0.92 & 0.59 & 0.45 & 0.92 & 0.68\\
        \midrule
        % \multicolumn{13}{c}{} \\
        % \midrule
        \midrule
        \multicolumn{13}{c}{QR\textbar{}SR} \\
        \midrule   
        Caption-Responsible & 0.34 & 0.96 & 0.65 & 0.53 & 0.79 & 0.66 & 0.33 & 0.96 & 0.65 & 0.50 & 0.82 & 0.66\\
        Intention-Responsible & 0.36 & 0.97 & \underline{0.67} & 0.51 & 0.86 & 0.68 & 0.27 & 0.96 & 0.61 & 0.49 & 0.90 & 0.70\\
        Caption-Responsible (w/o image) & 0.96 & 0.25 & 0.60 & 0.93 & 0.16 & 0.55 & 0.60 & 0.80 & \underline{0.70} & 0.72 & 0.72 & \textbf{0.72}\\
        % Responsible-Intention (w/o image) & 0.99 & 0.06 & 0.52 & 0.95 & 0.17 & 0.56 & 0.61 & 0.81 & 0.71 & 0.68 & 0.77 & 0.72\\
        \midrule
        \multicolumn{13}{c}{QR\textbar{}MO} \\
        \midrule
        Caption-SafeDecoding & 0.20 & 0.96 & 0.58 & 0.39 & 0.88 & 0.64 & 0.33 & 0.94 & 0.63 & 0.40 & 0.90 & 0.65 \\
        Intention-SFT & 0.28 & 0.97 & 0.62 & 0.43 & 0.78 & 0.61 & 0.25 & 0.96 & 0.60 & 0.50 & 0.88 & 0.69\\
        Caption-SafeDecoding (w/o image) & 0.24 & 0.95 & 0.60 & 0.41 & 0.89 & 0.65 & 0.36 & 0.85 & 0.61 & 0.56 & 0.84 & 0.70\\
        \bottomrule
    \end{tabular}}
    \caption{Comparison results of ensemble strategies with the corresponding individual defenses. \textbf{Bold} indicates the best overall performance, while \underline{underlined} highlights the top three methods.} % and the full score is 100\%
    \label{tab:en_inter_results}
\end{table*}


\subsection{Experimental Setup}
We empirically evaluate various defense methods and their ensemble strategies on LLaVA-1.5-7B and LLaVA-1.5-13B~\cite{liu2024visual} to validate their effectiveness in standard settings. Using MM-SafetyBench and MOSSBench datasets, we assess safety and helpfulness by measuring defense success rate (DSR) on harmful queries and response rate (RR) on benign queries. We evaluate 28 defense methods, including system reminders, optimization techniques, query refactoring, and noise injection, as well as inter- and intra-mechanism ensembles. Detailed descriptions of defense methods and experimental setups are provided in Appendix~\ref{sec:defense strategies} and~\ref{sec:experiment_detail}. 
For a broader evaluation, we add more experiments in Appendix~\ref{sec:utility}, ~\ref{sec:diverse_attacks} and~\ref{sec:time}, including evaluation with the MM-Vet dataset for testing the quality of model's response on general queries, tests on JailbreakV-28K for more diverse and complex attack scenarios, and a comparison of inference time for different defense methods.

\subsection{Individual Defense Results}

Table~\ref{tab:indi_results} shows results of individual defense methods across four categories. Most methods, except for noise injection, effectively improve model safety across different models and datasets, as evidenced by increased defense success rates. This aligns with our analysis in Figure~\ref{fig:analysis results} where system reminder, model optimization and query refactoring lead to an overall increase in refusal probabilities. 

\paragraph{Safety shift defenses compromise helpfulness.} System reminder and model optimization methods generally reduce response rates on the benign subset while increasing defense success rates on the harmful subset. This confirms that safety shift tend to compromise helpfulness. This is more pronounced in MOSSBench than MM-SafetyBench due to the more apparent harmfulness and concealed harmlessness in MOSSBench queries.

\paragraph{Harmfulness discrimination defenses mitigate over-defense.} Query refactoring methods, except for Caption (w/o image), generally achieve the highest response rates on the benign subset, particularly for MOSSBench with misleadingly benign queries. This validates that harmfulness discrimination improves the model's ability to distinguish between truly harmful and benign queries. Notably, the removal of images in the Caption (w/o image) significantly reduces response rates for both harmful and benign queries, highlighting the crucial role images play in jailbreaking LVLMs.
% \paragraph{Image matters.} The removal of images in the Caption (w/o image) and Intention (w/o image) defenses leads to significant improvements in DSR compared to their image-included counterparts, underscoring the crucial role that images play in jailbreaking LVLMs.

\paragraph{Multimodal defense is challenging.}
However, all individual defense methods still exhibit limited defense success rates. While larger-scale LVLMs (i.e., LLaVA-1.5-13B) tend to achieve slightly higher success rates, they are also more susceptible to over-defense. This underscores the inherent challenges of jailbreak defense for LVLMs, especially when relying on individual defense methods. 

\subsection{Ensemble Defense Results}
Table~\ref{tab:en_inter_results} provides the empirical evaluation of both inter-mechanism and intra-mechanism ensemble strategies, leading to the following insights:

\paragraph{Ensembles improve safety.} Compared to individual methods, most ensemble strategies effectively enhance safety across both datasets and model sizes, showing increased defense success rates, especially in \textit{SR+MO} and \textit{QR\textbar{}SR} methods.

\paragraph{Inter-mechanism ensembles amplify.} Our evaluation shows most \textit{SR++} and \textit{SR+MO} ensembles improve defense success rates while reducing responses rates, whereas the \textit{QR++} ensemble better maintain responses rates. This confirms that inter-mechanism ensembles can amplify a single defense mechanism. Specifically, safety shift ensembles would further enhance model safety at the expense of helpfulness, while harmfulness discrimination ensemble better preserves helpfulness. Among inter-mechanism ensembles, those combining different types of specific methods (e.g., SR+MO) show a more pronounced amplification effect than those combining the same type (e.g., SR++). 
Notably, the Demonstration-SFT method excels in defense strength, utility, and response rate. Its success comes from combining two strong safety shift defenses, Demonstration and SFT, which complement each other and boost overall performance.

\paragraph{Intra-mechanism ensembles complement.} Compared to inter-mechanism ensembles, most \textit{QR\textbar{}SR} and \textit{QR\textbar{}MO} methods—except those without input images—can simultaneously maintain decent defense success rates and stable response rates,
compared to the undefended model and individual defense methods. This demonstrates that intra-mechanism ensemble can complement each other to achieve a more balanced trade-off. Additionally, the removal of input images offering a most conservative ensemble for multimodal defense while still maintaining certain helpfulness.
% In contrast, the defenses in intra-mechanism ensemble complement each other, strengthening safety while maintaining a stable level of helpfulness.
% In contrast, intra-mechanism ensembles combine the strengths of both mechanisms to achieve a more balanced trade-off. Specifically, \textit{QR\textbar{}SR} and \textit{QR\textbar{}MO} increase the refusal probability for harmful queries, while maintaining or even decreasing the refusal probability for benign queries, thereby improving the model's ability to distinguish between benign and harmful queries. This makes them a better choice for general scenarios where balancing safety and helpfulness is essential. 


\subsection{How Do Fine-tuning Affect Model Safety?}
We examine how different fine-tuning methods impact the safety of LVLMs by training LLaVA-1.5-7B using DPO and SFT with two datasets: SPA-VL~\cite{zhang2024spa} and VLGuard~\cite{zong2024safety}. SPA-VL focuses on safety discussions, while VLGuard emphasizes query rejection. We also test the effect of adding 5000 general instruction-following data from LLaVA.  

Table~\ref{tab:training_dataset_results} shows that DPO with SPA-VL and LLaVA provides a slight safety boost without significantly changing response behavior. In contrast, SFT has a stronger impact, but its effectiveness depends on the dataset. SPA-VL improves safety while maintaining helpfulness, though it may miss some harmful cases. VLGuard, however, makes the model overly defensive, rejecting too many queries. Adding LLaVA data helps balance safety and helpfulness, reducing excessive refusals.  


\begin{table}[ht]
    \centering
    \resizebox{0.49\textwidth}{!}{
    \begin{tabular}{r|cccccc}
        \toprule 
        & \multicolumn{3}{c}{\textbf{MM-SafetyBench}} & \multicolumn{3}{c}{\textbf{MOSSBench}} \\
        \textbf{Method} & \textbf{DSR}$\uparrow$ & \textbf{RR}$\uparrow$ & \textbf{Avg}$\uparrow$ & \textbf{DSR}$\uparrow$ & \textbf{RR}$\uparrow$ & \textbf{Avg}$\uparrow$ \\
        \midrule
        w/o Defense          & 0.06  & 0.98  & 0.52  & 0.14  & 0.97  & 0.55 \\
        \midrule
        \multicolumn{7}{c}{DPO} \\
        \midrule
        \multicolumn{1}{l|}{SPA-VL + LLaVA}          & 0.06  & 0.97  & 0.52  & 0.28  & 0.97  & 0.63  \\
        \midrule
        \multicolumn{7}{c}{SFT} \\
        \midrule
        \multicolumn{1}{l|}{SPA-VL}          & 0.24  & 0.96  & 0.60  & 0.58  & 0.78  & 0.68  \\
        + LLaVA     & 0.20  & 0.95  & 0.58  & 0.50  & 0.88  & 0.69  \\
        \midrule
        \multicolumn{1}{l|}{VLGuard}          & 1.00  & 0.09  & 0.55  & 0.90  & 0.21  & 0.55  \\
        + LLaVA     & 0.97  & 0.43  & 0.70  & 0.76  & 0.58  & 0.67  \\
        \bottomrule
    \end{tabular}}
    \caption{Comparison of varying fine-tuning settings.} % and the full score is 100\%
    \label{tab:training_dataset_results}
\end{table}

\section{Conclusion}
\label{sec:Conclusion}
This work evaluates proprietary and open-weight models in agentic frameworks for handling ambiguity in software engineering. In code generation, to effectively integrate new information into the solution, an agent must detect ambiguity and ask targeted questions. Our key findings are:
\begin{itemize}[itemsep=0pt, topsep=0pt]
    \item Given an underspecified input, Claude Sonnet 3.5 and Claude Haiku 3.5 with interaction can achieve 80\% of their performance with a well-specified input. In contrast, open-weight models struggle: Deepseek relies on navigational cues to locate relevant files, while Llama 3.1 70B extracts limited information from the user.
    \item LLMs do not interact unless explicitly prompted, and their ambiguity detection is highly sensitive to prompt variations. Only Claude Sonnet 3.5 achieves a higher accuracy of 84\% in distinguishing between well-specified and underspecified input.

    \item Claude Sonnet 3.5, Haiku 3.5, and Deepseek effectively extract new, detailed user information, whereas Llama 3.1 struggles to ask the right questions.
    
\end{itemize}
Despite these advances, a gap remains between resolve rates for underspecified vs. fully specified issues. Open-weight models need better interaction strategies to improve resolution, while proprietary models, particularly Claude Haiku 3.5, require stronger prompting to engage interactively. This work establishes the current state-of-the-art in handling ambiguity through interaction, breaking the resolution process into multiple steps.



