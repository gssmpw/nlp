\chapter*{Executive Summary}
\addcontentsline{toc}{chapter}{Executive Summary}
\chaptermark{Executive Summary}

Watermarking embeds information into digital content like images, audio, or text, imperceptible to humans but robustly detectable by specific algorithms.
This technology has important applications in many challenges of the industry such as content moderation, tracing AI-generated content, and monitoring the usage of AI models.
The contributions of this thesis include the development of new watermarking techniques for images, audio, and text. 
We first introduce methods for active moderation of images on social platforms. 
We then develop specific techniques for AI-generated content.
We specifically demonstrate methods to adapt latent generative models to embed watermarks in all generated content, identify watermarked sections in speech, and improve watermarking in large language models with tests that ensure low false positive rates.
Furthermore, we explore the use of digital watermarking to detect model misuse, including the detection of watermarks in language models fine-tuned on watermarked text, and introduce training-free watermarks for the weights of large transformers.
Through these contributions, the thesis provides effective solutions for the challenges posed by the increasing use of generative AI models and the need for model monitoring and content moderation.
It finally examines the challenges and limitations of watermarking techniques and discuss potential future directions for research in this area.


\chapter*{Acknowledgement}
\addcontentsline{toc}{chapter}{Acknowledgement}
\chaptermark{Acknowledgement}

Acknowledgments are the least meaningful part of the thesis, yet funnily enough, they are the most meaningful to the student (me). 
After spending years on this project, they are clearly the part I was most eager to write.
I am deeply grateful to everyone I have met in my life, especially those with whom I have shared my PhD journey. 
I hope these paragraphs will pay tribute to what we have shared, and that the next hundred pages will pay tribute to the work you have inspired me to do.

\begin{quote}
    \centering
    \textit{``Bon, alors comme ça... tu veux faire du tatouage?''}
    --  Teddy, 2021. 
\end{quote}

First and foremost, I extend my most sincere thanks to my advisors. 
I hope to continue learning from you in the years to come and aspire to be like you.
Teddy, thank you for being the embodiment of both humor and seriousness since our first encounter. 
Your technical guidance has been exceptional, and I will sincerely miss the maths you wrote on the Inria's whiteboard -- which is probably still there today.
Matthijs, our bike rides, your down-to-earth views and steady presence have been the foundation of my PhD.
I am immensely grateful for the trust and responsibilities you have placed in me.
`We don't live out of thin air' is a lesson I will never forget.
Hervé, although you went for new adventures midway through my PhD, your momentum has never faded. 
Your humbleness and your ability to identify research directions that shape the field and society have been an immense source of inspiration. 
To all my advisors, thank you for your frankness and for being such remarkable mentors. 
I owe you everything.

To my fellow PhDs 22' at FAIR and to our stressful foosball matches, drunken afterworks, and countless coffee breaks, you all made my PhD life funny and unforgettable.
Thanks:\blfootnote{$^\star$: Order determined by the start date at FAIR.}
Théo$^\star$ and Timothée$^\star$, \aka, the Dinomates, for the citations, the funny memes on Rats@ and your unfailing presence at parties;
Quentin$^\star$, for never having said no to a break;
Badr$^\star$, for your unbelievable kindness;
Wassim$^\star$, for your daily ``check'';
Robin$^\star$, for running so fast and having shared the love of watermarking for a while;
Mathurin$^\star$, for the late discussions when getting home after the parties.
To all of you, I  am so proud to have shared this journey with you and to be friends with such talented and kind-hearted people.
You really were the best people I could have hoped to share this experience with.

I wish to thank all the FAIR PhDs who walked this path before me and to those who are walking it now. 
Sharing the office with you was a great privilege.
A special thanks to Alexandre S. for patiently onboarding me during my internship, introducing me to watermarking and giving me the tools to succeed in my PhD.
I am thankful to Pierre S., Hugo, and Alaa; being able benefit from your exceptional views has truly humbled me.
I also owe a lot to Lina and Virginie, whose advice allowed me to be part of the best PhD program in the world.
Guillaume C., I am so happy to have collaborated with someone as smart and humble as you, you serve as a great example of how people should work and live their lives.
Megi, sharing moments in Milan, Modena, Capri and the office was great, thank you for your authenticity and your energy.
Fabian, for initiating the board game sessions and for our lively discussions on information theory.
Simon for our deep philosophical discussions late at night.
Tariq, for giving me all of your insights on image generation and segmentation at breakfast.
Kruno, Jos\'ephine, Jo\~ao, Belen, Pierre C., thank you for taking over and bringing so much fun in the office.
Last but not least, Tom, I am so grateful for our friendship and for having shared so many moments with you.
I am also very proud of the trust you placed in me and what we have achieved together. 
I look forward to our future collaborations!

I want to express my gratitude to the folks at Inria Rennes. 
I haven't been able to spend as much time with you as I would have liked, especially towards the end, but I have always felt very welcome.
I would like to especially thank Benoît and Thibault, my co-bureaus for a time.
I was lucky to have shared my first conference with you, as well as the following years of Rennes' office life.
Thank you Antoine, Eva and Vivien for our deep discussions on watermarking at the office and at the pub.
Thank you Laurent and Aur\'elie for including me in the Linkmedia team.
Thank you Duc, Karim, Hugo, Guillaume, Gautier, Victor and Morgane for the coffee breaks and the foosball training which have helped me in my games against the FAIR people.

During my PhD, I have had the chance to discuss and collaborate with many colleagues at FAIR, all very talented people who have made my journey more enjoyable and fruitful: 
Federico, Marc, Francisco, Patrick, Mathilde, Vasil, Zoe, Adrien, Marl\`ene, Jakob, Camille, Olivier, Michal, Tuan, Nikita, Shalini, Tal, Diego, Damien, Adeel, and many others!
Hady, I am amazed by your good mood and your talent. 
I loved working on AudioSeal with you and Robin, and I hope our future collaborations will be as successful.
Armand and Piotr, allowing me to be part of such an impactful project as DINO has been a huge honor. 
You both have an extreme clarity of thought and I thoroughly enjoyed working with you.
Zeki and Dong, initiating a production effort for watermarking was a great experience, thank you for your trust and your guidance.
Jeremy, our climbing sessions and drinks were a great time for me.
Thank you Thomas S., for introducing me to Large Language Models and allowing me to alpha test the short-lived Galactica. 
Pierre-Louis, thanks for making the CIFRE what it is today. 
Naila, Mary, Pierre-Emmanuel and Alex M., I greatly appreciated your insightful discussions about career planning post-PhD. 
Paul-Ambroise, Charlotte, and Matthieu R., your guidance during the interview process has been invaluable.
Lastly, Martin and Victoire, in addition to being super nice human beings, your perspectives on AI policy and business have greatly enriched my views and my personal growth.

To my friends, thank you for your support, being with you makes life funnier!
Hugo, Juan and Arthur thanks for the many moments we have shared during the last few years, that I enjoyed even more than the ``nouilles Chong-Qing piment\'ees 4'' of TTZ -- this is saying something -- as well as Thibault, Alex, Benjamin, although these moments were fewer.
Thanks Kenza, Meryem, Louis and Nissim, our scarce encounters were always a good time to share our experience as PhDs.
Lastly, thanks Thomas B., Astrid and Ambroise, for our late night discussion on LLMs and our tennis games under the Parisian cold, and MH, Alice, Pierre and Thomas D., for still being in my life after all this time.

To my wonderful family, I am profoundly grateful to have been born into such an environment. 
Aunt Sandrine and Jo, I deeply appreciated your generous hospitality during my visits to Rennes.
Those times were very precious to me.
Grandma, thank you for your wisdom as well as for letting me talk to you about my research -- although you may have never understood a word of it (this last point now applies to rest of this paragraph).
Ang\'eline, thank you for being the best godmother on earth.
To my dear parents and beloved sisters, words cannot express how much your love and presence mean to me. 
Mom and Dad, I want to let you know how proud of both of you I am. 
You have been my North stars (mes ``\'Etoiles du Nord'') during my PhD, as you have always been in my life, and as you always will be.
Diana and Ines, thank you for sharing so many moments together. 
Most importantly, thank you for being my best friends in life.
I have the feeling that our brotherhood will always be there, and I am so lucky for it.

Completing this PhD has been a transformative experience, and there are many others who I have not cited but who have contributed to its success.
Each of you has left an (in)visible and robust mark on life. Thank you.
