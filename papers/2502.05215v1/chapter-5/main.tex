
\chapter{Three Bricks to Consolidate Watermarks for Large Language Models}\label{chapter:three-bricks}

This chapter is based on the paper \fullcite{fernandez2023three}.

Discerning between generated and natural texts is increasingly challenging due to the rapid progress of large language models (LLMs).
In this context, watermarking again emerges as a promising technique for ascribing text to a specific generative model. 
It alters the sampling generation process to leave an invisible trace in the output, facilitating later detection.
This research consolidates watermarks for large language models based on three theoretical and empirical considerations. 
First, we introduce new statistical tests that offer robust theoretical guarantees which remain valid even at low false-positive rates (less than 10$^{\text{-6}}$). 
Second, we compare the effectiveness of watermarks using classical benchmarks in the field of natural language processing, gaining insights into their real-world applicability.
Third, we develop advanced detection schemes for scenarios where access to the LLM is available, as well as multi-bit watermarking. 
Code is available at \url{github.com/facebookresearch/three_bricks}.

\newpage
%!TEX root = gcn.tex
\section{Introduction}
Graphs, representing structural data and topology, are widely used across various domains, such as social networks and merchandising transactions.
Graph convolutional networks (GCN)~\cite{iclr/KipfW17} have significantly enhanced model training on these interconnected nodes.
However, these graphs often contain sensitive information that should not be leaked to untrusted parties.
For example, companies may analyze sensitive demographic and behavioral data about users for applications ranging from targeted advertising to personalized medicine.
Given the data-centric nature and analytical power of GCN training, addressing these privacy concerns is imperative.

Secure multi-party computation (MPC)~\cite{crypto/ChaumDG87,crypto/ChenC06,eurocrypt/CiampiRSW22} is a critical tool for privacy-preserving machine learning, enabling mutually distrustful parties to collaboratively train models with privacy protection over inputs and (intermediate) computations.
While research advances (\eg,~\cite{ccs/RatheeRKCGRS20,uss/NgC21,sp21/TanKTW,uss/WatsonWP22,icml/Keller022,ccs/ABY318,folkerts2023redsec}) support secure training on convolutional neural networks (CNNs) efficiently, private GCN training with MPC over graphs remains challenging.

Graph convolutional layers in GCNs involve multiplications with a (normalized) adjacency matrix containing $\numedge$ non-zero values in a $\numnode \times \numnode$ matrix for a graph with $\numnode$ nodes and $\numedge$ edges.
The graphs are typically sparse but large.
One could use the standard Beaver-triple-based protocol to securely perform these sparse matrix multiplications by treating graph convolution as ordinary dense matrix multiplication.
However, this approach incurs $O(\numnode^2)$ communication and memory costs due to computations on irrelevant nodes.
%
Integrating existing cryptographic advances, the initial effort of SecGNN~\cite{tsc/WangZJ23,nips/RanXLWQW23} requires heavy communication or computational overhead.
Recently, CoGNN~\cite{ccs/ZouLSLXX24} optimizes the overhead in terms of  horizontal data partitioning, proposing a semi-honest secure framework.
Research for secure GCN over vertical data  remains nascent.

Current MPC studies, for GCN or not, have primarily targeted settings where participants own different data samples, \ie, horizontally partitioned data~\cite{ccs/ZouLSLXX24}.
MPC specialized for scenarios where parties hold different types of features~\cite{tkde/LiuKZPHYOZY24,icml/CastigliaZ0KBP23,nips/Wang0ZLWL23} is rare.
This paper studies $2$-party secure GCN training for these vertical partition cases, where one party holds private graph topology (\eg, edges) while the other owns private node features.
For instance, LinkedIn holds private social relationships between users, while banks own users' private bank statements.
Such real-world graph structures underpin the relevance of our focus.
To our knowledge, no prior work tackles secure GCN training in this context, which is crucial for cross-silo collaboration.


To realize secure GCN over vertically split data, we tailor MPC protocols for sparse graph convolution, which fundamentally involves sparse (adjacency) matrix multiplication.
Recent studies have begun exploring MPC protocols for sparse matrix multiplication (SMM).
ROOM~\cite{ccs/SchoppmannG0P19}, a seminal work on SMM, requires foreknowledge of sparsity types: whether the input matrices are row-sparse or column-sparse.
Unfortunately, GCN typically trains on graphs with arbitrary sparsity, where nodes have varying degrees and no specific sparsity constraints.
Moreover, the adjacency matrix in GCN often contains a self-loop operation represented by adding the identity matrix, which is neither row- nor column-sparse.
Araki~\etal~\cite{ccs/Araki0OPRT21} avoid this limitation in their scalable, secure graph analysis work, yet it does not cover vertical partition.

% and related primitives
To bridge this gap, we propose a secure sparse matrix multiplication protocol, \osmm, achieving \emph{accurate, efficient, and secure GCN training over vertical data} for the first time.

\subsection{New Techniques for Sparse Matrices}
The cost of evaluating a GCN layer is dominated by SMM in the form of $\adjmat\feamat$, where $\adjmat$ is a sparse adjacency matrix of a (directed) graph $\graph$ and $\feamat$ is a dense matrix of node features.
For unrelated nodes, which often constitute a substantial portion, the element-wise products $0\cdot x$ are always zero.
Our efficient MPC design 
avoids unnecessary secure computation over unrelated nodes by focusing on computing non-zero results while concealing the sparse topology.
We achieve this~by:
1) decomposing the sparse matrix $\adjmat$ into a product of matrices (\S\ref{sec::sgc}), including permutation and binary diagonal matrices, that can \emph{faithfully} represent the original graph topology;
2) devising specialized protocols (\S\ref{sec::smm_protocol}) for efficiently multiplying the structured matrices while hiding sparsity topology.


 
\subsubsection{Sparse Matrix Decomposition}
We decompose adjacency matrix $\adjmat$ of $\graph$ into two bipartite graphs: one represented by sparse matrix $\adjout$, linking the out-degree nodes to edges, the other 
by sparse matrix $\adjin$,
linking edges to in-degree nodes.

%\ie, we decompose $\adjmat$ into $\adjout \adjin$, where $\adjout$ and $\adjin$ are sparse matrices representing these connections.
%linking out-degree nodes to edges and edges to in-degree nodes of $\graph$, respectively.

We then permute the columns of $\adjout$ and the rows of $\adjin$ so that the permuted matrices $\adjout'$ and $\adjin'$ have non-zero positions with \emph{monotonically non-decreasing} row and column indices.
A permutation $\sigma$ is used to preserve the edge topology, leading to an initial decomposition of $\adjmat = \adjout'\sigma \adjin'$.
This is further refined into a sequence of \emph{linear transformations}, 
which can be efficiently computed by our MPC protocols for 
\emph{oblivious permutation}
%($\Pi_{\ssp}$) 
and \emph{oblivious selection-multiplication}.
% ($\Pi_\SM$)
\iffalse
Our approach leverages bipartite graph representation and the monotonicity of non-zero positions to decompose a general sparse matrix into linear transformations, enhancing the efficiency of our MPC protocols.
\fi
Our decomposition approach is not limited to GCNs but also general~SMM 
by 
%simply 
treating them 
as adjacency matrices.
%of a graph.
%Since any sparse matrix can be viewed 

%allowing the same technique to be applied.

 
\subsubsection{New Protocols for Linear Transformations}
\emph{Oblivious permutation} (OP) is a two-party protocol taking a private permutation $\sigma$ and a private vector $\xvec$ from the two parties, respectively, and generating a secret share $\l\sigma \xvec\r$ between them.
Our OP protocol employs correlated randomnesses generated in an input-independent offline phase to mask $\sigma$ and $\xvec$ for secure computations on intermediate results, requiring only $1$ round in the online phase (\cf, $\ge 2$ in previous works~\cite{ccs/AsharovHIKNPTT22, ccs/Araki0OPRT21}).

Another crucial two-party protocol in our work is \emph{oblivious selection-multiplication} (OSM).
It takes a private bit~$s$ from a party and secret share $\l x\r$ of an arithmetic number~$x$ owned by the two parties as input and generates secret share $\l sx\r$.
%between them.
%Like our OP protocol, o
Our $1$-round OSM protocol also uses pre-computed randomnesses to mask $s$ and $x$.
%for secure computations.
Compared to the Beaver-triple-based~\cite{crypto/Beaver91a} and oblivious-transfer (OT)-based approaches~\cite{pkc/Tzeng02}, our protocol saves ${\sim}50\%$ of online communication while having the same offline communication and round complexities.

By decomposing the sparse matrix into linear transformations and applying our specialized protocols, our \osmm protocol
%($\prosmm$) 
reduces the complexity of evaluating $\numnode \times \numnode$ sparse matrices with $\numedge$ non-zero values from $O(\numnode^2)$ to $O(\numedge)$.

%(\S\ref{sec::secgcn})
\subsection{\cgnn: Secure GCN made Efficient}
Supported by our new sparsity techniques, we build \cgnn, 
a two-party computation (2PC) framework for GCN inference and training over vertical
%ly split
data.
Our contributions include:

1) We are the first to explore sparsity over vertically split, secret-shared data in MPC, enabling decompositions of sparse matrices with arbitrary sparsity and isolating computations that can be performed in plaintext without sacrificing privacy.

2) We propose two efficient $2$PC primitives for OP and OSM, both optimally single-round.
Combined with our sparse matrix decomposition approach, our \osmm protocol ($\prosmm$) achieves constant-round communication costs of $O(\numedge)$, reducing memory requirements and avoiding out-of-memory errors for large matrices.
In practice, it saves $99\%+$ communication
%(Table~\ref{table:comm_smm}) 
and reduces ${\sim}72\%$ memory usage over large $(5000\times5000)$ matrices compared with using Beaver triples.
%(Table~\ref{table:mem_smm_sparse}) ${\sim}16\%$-

3) We build an end-to-end secure GCN framework for inference and training over vertically split data, maintaining accuracy on par with plaintext computations.
We will open-source our evaluation code for research and deployment.

To evaluate the performance of $\cgnn$, we conducted extensive experiments over three standard graph datasets (Cora~\cite{aim/SenNBGGE08}, Citeseer~\cite{dl/GilesBL98}, and Pubmed~\cite{ijcnlp/DernoncourtL17}),
reporting communication, memory usage, accuracy, and running time under varying network conditions, along with an ablation study with or without \osmm.
Below, we highlight our key achievements.

\textit{Communication (\S\ref{sec::comm_compare_gcn}).}
$\cgnn$ saves communication by $50$-$80\%$.
(\cf,~CoGNN~\cite{ccs/KotiKPG24}, OblivGNN~\cite{uss/XuL0AYY24}).

\textit{Memory usage (\S\ref{sec::smmmemory}).}
\cgnn alleviates out-of-memory problems of using %the standard 
Beaver-triples~\cite{crypto/Beaver91a} for large datasets.

\textit{Accuracy (\S\ref{sec::acc_compare_gcn}).}
$\cgnn$ achieves inference and training accuracy comparable to plaintext counterparts.
%training accuracy $\{76\%$, $65.1\%$, $75.2\%\}$ comparable to $\{75.7\%$, $65.4\%$, $74.5\%\}$ in plaintext.

{\textit{Computational efficiency (\S\ref{sec::time_net}).}} 
%If the network is worse in bandwidth and better in latency, $\cgnn$ shows more benefits.
$\cgnn$ is faster by $6$-$45\%$ in inference and $28$-$95\%$ in training across various networks and excels in narrow-bandwidth and low-latency~ones.

{\textit{Impact of \osmm (\S\ref{sec:ablation}).}}
Our \osmm protocol shows a $10$-$42\times$ speed-up for $5000\times 5000$ matrices and saves $10$-2$1\%$ memory for ``small'' datasets and up to $90\%$+ for larger ones.

\section{Background}
\label{sec:background}


\subsection{Code Review Automation}
Code review is a widely adopted practice among software developers where a reviewer examines changes submitted in a pull request \cite{hong2022commentfinder, ben2024improving, siow2020core}. If the pull request is not approved, the reviewer must describe the issues or improvements required, providing constructive feedback and identifying potential issues. This step involves review commment generation, which play a key role in the review process by generating review comments for a given code difference. These comments can be descriptive, offering detailed explanations of the issues, or actionable, suggesting specific solutions to address the problems identified \cite{ben2024improving}.


Various approaches have been explored to automate the code review comments process  \cite{tufano2023automating, tufano2024code, yang2024survey}. 
Early efforts centered on knowledge-based systems, which are designed to detect common issues in code. Although these traditional tools provide some support to programmers, they often fall short in addressing complex scenarios encountered during code reviews \cite{dehaerne2022code}. More recently, with advancements in deep learning, researchers have shifted their focus toward using large-language models to enhance the effectiveness of code issue detection and code review comment generation.

\subsection{Knowledge-based Code Review Comments Automation}

Knowledge-based systems (KBS) are software applications designed to emulate human expertise in specific domains by using a collection of rules, logic, and expert knowledge. KBS often consist of facts, rules, an explanation facility, and knowledge acquisition. In the context of software development, these systems are used to analyze the source code, identifying issues such as coding standard violations, bugs, and inefficiencies~\cite{singh2017evaluating, delaitre2015evaluating, ayewah2008using, habchi2018adopting}. By applying a vast set of predefined rules and best practices, they provide automated feedback and recommendations to developers. Tools such as FindBugs \cite{findBugs}, PMD \cite{pmd}, Checkstyle \cite{checkstyle}, and SonarQube \cite{sonarqube} are prominent examples of knowledge-based systems in code analysis, often referred to as static analyzers. These tools have been utilized since the early 1960s, initially to optimize compiler operations, and have since expanded to include debugging tools and software development frameworks \cite{stefanovic2020static, beller2016analyzing}.



\subsection{LLMs-based Code Review Comments Automation}
As the field of machine learning in software engineering evolves, researchers are increasingly leveraging machine learning (ML) and deep learning (DL) techniques to automate the generation of review comments \cite{li2022automating, tufano2022using, balachandran2013reducing, siow2020core, li2022auger, hong2022commentfinder}. Large language models (LLMs) are large-scale Transformer models, which are distinguished by their large number of parameters and extensive pre-training on diverse datasets.  Recently, LLMs have made substantial progress and have been applied across a broad spectrum of domains. Within the software engineering field, LLMs can be categorized into two main types: unified language models and code-specific models, each serving distinct purposes \cite{lu2023llama}.

Code-specific LLMs, such as CodeGen \cite{nijkamp2022codegen}, StarCoder \cite{li2023starcoder} and CodeLlama \cite{roziere2023code} are optimized to excel in code comprehension, code generation, and other programming-related tasks. These specialized models are increasingly utilized in code review activities to detect potential issues, suggest improvements, and automate review comments \cite{yang2024survey, lu2023llama}. 




\subsection{Retrieval-Augmented Generation}
Retrieval-Augmented Generation (RAG) is a general paradigm that enhances LLMs outputs by including relevant information retrieved from external databases into the input prompt \cite{gao2023retrieval}. Traditional LLMs generate responses based solely on the extensive data used in pre-training, which can result in limitations, especially when it comes to domain-specific, time-sensitive, or highly specialized information. RAG addresses these limitations by dynamically retrieving pertinent external knowledge, expanding the model's informational scope and allowing it to generate responses that are more accurate, up-to-date, and contextually relevant \cite{arslan2024business}. 

To build an effective end-to-end RAG pipeline, the system must first establish a comprehensive knowledge base. It requires a retrieval model that captures the semantic meaning of presented data, ensuring relevant information is retrieved. Finally, a capable LLM integrates this retrieved knowledge to generate accurate and coherent results \cite{ibtasham2024towards}.




\subsection{LLM as a Judge Mechanism}

LLM evaluators, often referred to as LLM-as-a-Judge, have gained significant attention due to their ability to align closely with human evaluators' judgments \cite{zhu2023judgelm, shi2024judging}. Their adaptability and scalability make them highly suitable for handling an increasing volume of evaluative tasks. 

Recent studies have shown that certain LLMs, such as Llama-3 70B and GPT-4 Turbo, exhibit strong alignment with human evaluators, making them promising candidates for automated judgment tasks \cite{thakur2024judging}

To enable such evaluations, a proper benchmarking system should be set up with specific components: \emph{prompt design}, which clearly instructs the LLM to evaluate based on a given metric, such as accuracy, relevance, or coherence; \emph{response presentation}, guiding the LLM to present its verdicts in a structured format; and \emph{scoring}, enabling the LLM to assign a score according to a predefined scale \cite{ibtasham2024towards}. Additionally, this evaluation system can be enriched with the ability to explain reasoning behind verdicts, which is a significant advantage of LLM-based evaluation \cite{zheng2023judging}. The LLM can outline the criteria it used to reach its judgment, offering deeper insights into its decision-making process.









\begin{figure}[b!]
    \centering
    \begin{subfigure}{0.8\textwidth}
        \includegraphics[width=\textwidth, trim=0 0 0 0, clip]{chapter-5/figs/pval-emp.pdf}
        \vspace{-0.6cm}
        \caption{Empirical \pval\ distribution without any correction.}
        \label{chap5/fig:pvalue}
    \end{subfigure}
    \\[0.5cm]
    \centering
    \begin{subfigure}{1.0\textwidth}
        \includegraphics[height=1.8in, trim=0.2cm 0 0.3cm 0, clip]{chapter-5/figs/fpr_zscore.pdf}
        \includegraphics[height=1.8in, trim=0.6cm 0 0.3cm 0, clip]{chapter-5/figs/fpr.pdf}
        \includegraphics[height=1.8in, trim=0.6cm 0 0.3cm 0, clip]{chapter-5/figs/fpr_rect.pdf}
        \caption{
            Empirical FPRs against theoretical ones.
            (\emph{Left}) using $Z$-tests;
            (\emph{Middle}) using new statistical tests presented in~\ref{chap5/sec:new-stats};
            (\emph{Right}) using the new statistical tests with the rectified scoring strategy of~\ref{chap5/sec:rect}.
        }\label{chap5/fig:fpr}
    \end{subfigure} \\
    \caption{
        Empirical checks of \pval\ and false positive rates for different watermarks and values of the context width $k$.
        Results are computed over $10$ secret keys $\times$ 100k sequences of $256$ tokens sampled from Wikipedia.
        Theoretical values do not hold in practice for $Z$-tests even for high values of $k$: \pval s are not uniformly distributed and empirical FPRs do not match theoretical ones.
        This is solved by basing detection on grounded statistical tests and analytic \pval, as well as by revising the scoring strategy with deduplication.
    }
\end{figure}


\section{Ensuring reliable \pval\ and FPR}\label{chap5/sec:stats}

In this section, large-scale evaluations of the FPR show a gap between theory and practice. 
It is closed with new statistical tests and by rectifying the scoring method.


\subsection{Empirical validation of FPR with Z-scores}\label{chap5/sec:zscore}

So far, the FPR has been checked on only around $500$ negative samples~\citep{kirchenbauer2023watermark,kirchenbauer2023reliability, zhao2023provable}.
We scale this further and select $100$k texts from multilingual Wikipedia to cover the distribution of natural text.
We tokenize with Llama's tokenizer, and take $T=256$ tokens/text.
We run detection tests with varying window length $k$ when seeding the RNG. 
We repeat this with $10$ different secret keys, which makes $1$M detection results under $\H_0$ for each method and $k$ value.
For the detection of the greenlist watermark, we use $\gamma=0.25$.

\autoref{chap5/fig:pvalue} first shows that the distribution of \pval s is not uniform.
his should be the case for a $Z$-test under $\H_0$, which calls into question whether this type of test can be used here.
To further emphasize this point, \autoref{chap5/fig:fpr} compares empirical and theoretical FPRs.
Theoretical guarantees do not hold in practice and empirical FPRs are much higher than the theoretical ones.
Besides, the larger the watermarking context window $k$, the closer we get to theoretical guarantees. 
In pratice, one would need $k>>8$ to get reliable \pval, but this makes the watermarking method less robust to attacks on generated text because it hurts synchronization.


\subsection{New non-asymptotical statistical tests}\label{chap5/sec:new-stats}

The Gaussian assumption of $Z$-tests breaks down for short or repetitive texts.
Here are non-asymptotical tests for both methods that reduce the gap between empirical and theoretical FPR, especially at low FPR values as shown in Fig.~\ref{chap5/fig:fpr}.

\paragraph*{\texorpdfstring{\citep{kirchenbauer2023watermark}}{}} 
Under $\H_0$, we assume that the event $x^{(t)}\in\G^{(t)}$ occurs with probability $\gamma$, and that these events are i.i.d.
Therefore, $S_T$~\eqref{chap5/eq:score-kirchenbauer} is distributed as a binomial of parameters $T$ and $\gamma$. Consider a text under scrutiny whose score equals $s$.
The \pval\ is defined as the probability of obtaining a score higher than $s$ under $\H_0$: %
\begin{equation}
    \text{p-value}(s) = \Prob(S_T \geq s \mid \H_0) = I_{\gamma}(s,T-s+1).
\end{equation}
This comes from the fact that $S\sim\mathcal{B}(T,\gamma)$ whose c.d.f. is expressed by $I_x(a,b)$ the regularized incomplete Beta function.

\paragraph*{\texorpdfstring{\citep{aaronson2023watermarking}}{}} 
Under $\H_0$, we assume that the text under scrutiny and the secret vector are independent, so that $\vec{r}_{x^{(t)}} \overset{i.i.d.}{\sim} \mathcal{U}(0,1)$. 
Therefore, $S_T$~\eqref{chap5/eq:score-aaronson} follows a $\Gamma(T,1)$ distribution.
The \pval\ associated to a score $s$ reads:
\begin{equation}
    \text{p-value}(s) = \Prob(S_T \geq s \mid \H_0) = \frac{\Gamma(T,s)}{\Gamma(T)},
\end{equation}
where $\Gamma$ is the upper incomplete gamma function.
Under $\H_1$, the score is expected to be higher as proven in App.~\ref{chap5/app:aaronson_score}, so the \pval\ is likely to be small.




    
    


\subsection{Rectifying the detection scores}\label{chap5/sec:rect}


\begin{figure}[b!]
    \centering
    \begin{tcolorbox}[width=0.7\linewidth, colframe=metablue, colback=white]
        \includegraphics[width=0.99\linewidth, trim=0 110 0 18, clip]{chapter-5/figs/low-pval-text.pdf}
    \end{tcolorbox}
    \centering
    \caption{
    Typical example of a non-watermarked text with low \pval\ because of repeated tokens (we only show half of the text).
    Here \pval$=10^{-21}$ on $256$ tokens.
    }
    \label{chap5/fig:low-pval-text}
\end{figure}

Even with grounded statistical tests, empirical FPRs are still higher than theoretical ones.
In fact, \citet{kirchenbauer2023watermark} mention that random variables are only pseudo-random since repeated windows generate the same secret. 
This can happen even in a short text and especially in formatted data.
For instance in a bullet list, the sequence of tokens \texttt{$\backslash$n$\backslash$n*\_} repeats a lot as shown in Fig.~\ref{chap5/fig:low-pval-text}.
In this text of $256$ tokens, the \pval\ is $10^{-21}$ for the scheme of~\citet{kirchenbauer2023watermark} and with $\gamma=0.25$ and $k=2$.
Repetition pulls down the assumption of independence necessary for computing the \pval.

We experiment with two simple heuristics mitigating this issue.
The first one takes into account a token only if the watermark context window has not already been seen during the detection.
The second scores the tokens for which the $k+1$-tuple formed by \{watermark context + current token\} has not already been seen.
Note, that the latter is present in~\citep{kirchenbauer2023watermark}, although without ablation and without being used in further experiments.
Of the two, the second one is better since it counts more ngrams, and thus has better TPR. 
It can also deal with the specific case of $k=0$.
\autoref{chap5/fig:fpr} reports empirical and theoretical FPRs when choosing not to score already seen $k+1$-tuples.
They now match perfectly, except for $k=0$ where the FPR is still slightly underestimated.
\emph{In short, we guarantee FPR thanks to new statistical tests and by scoring only tokens for which \{watermark context + current token\} has not been scored.}


\section{Better watermarking evaluations}

\autoref{chap5/sec:robustness-analysis} evaluates the detection with the revised statistical tests and smaller FPRs than the previous literature. 
\autoref{chap5/sec:free-form} shows the impact on NLP benchmarks.

\subsection{Robustness analysis}
\label{chap5/sec:robustness-analysis}

\begin{table}[t!]
    \centering
    \caption{
    Robustness analysis of the watermarks, with rectified statistical tests.
    We report the TPR@FPR=$10^{-5}$ and the S-BERT scores over $10\times 1$k completions, for different hyperparameters controlling the strength of the watermark 
    ($\delta$ in \citep{kirchenbauer2023watermark} and $\temperature$ in \citep{aaronson2023watermarking} - see Sec.~\ref{chap5/sec:background}).
    The `TPR aug.' is the TPR when texts are attacked before detection by randomly replacing tokens with probability 0.3.
    }
    \label{chap5/tab:robustness}
    \footnotesize
    \begin{tabular}{rl *{4}{p{1cm}} @{\hspace{0.5cm}} *{4}{p{1cm}}}
        \toprule
        & & \multicolumn{4}{c}{\citep{aaronson2023watermarking}} &  \multicolumn{4}{c}{\citep{kirchenbauer2023watermark}}  \\
        $k$ & Metric & $\temperature:$ 0.8 & 0.9 & 1.0 & 1.1 & $\delta:$ 1.0 & 2.0 & 3.0 & 4.0 \\
        \cmidrule(rr){3-6} \cmidrule(rr){7-10}
        \multirow{3}{*}{$0$} 
            & S-BERT    & 0.60 & 0.56 & 0.52 & 0.44 & 0.63 & 0.61 & 0.57 & 0.50 \\
            & TPR       & 0.20 & 0.31 & 0.42 & 0.51 & 0.00 & 0.16 & 0.58 & 0.70 \\
            & TPR aug.  & 0.04 & 0.06 & 0.09 & 0.10 & 0.00 & 0.02 & 0.20 & 0.39 \\[4pt]
        \multirow{3}{*}{$1$} 
            & S-BERT    & 0.62 & 0.61 & 0.59 & 0.55 & 0.63 & 0.62 & 0.60 & 0.56 \\
            & TPR       & 0.35 & 0.51 & 0.66 & 0.77 & 0.02 & 0.41 & 0.77 & 0.88 \\
            & TPR aug.  & 0.04 & 0.10 & 0.20 & 0.36 & 0.00 & 0.05 & 0.30 & 0.58 \\[4pt]
        \multirow{3}{*}{$4$} 
            & S-BERT    & 0.62 & 0.62 & 0.61 & 0.59 & 0.62 & 0.62 & 0.60 & 0.57 \\
            & TPR       & 0.43 & 0.59 & 0.71 & 0.80 & 0.02 & 0.44 & 0.76 & 0.88 \\
            & TPR aug.  & 0.01 & 0.02 & 0.06 & 0.18 & 0.00 & 0.00 & 0.03 & 0.14 \\
        \bottomrule
    \end{tabular}
\end{table}   





We now compare watermarking methods by analyzing the TPR when detecting watermarked texts.
For detection, we employ the previous statistical tests and scoring strategy.
They enable precise control over the FPR and therefore to operate at operating-points not yet seen in the literature.
More specifically we flag a text as watermarked if its \pval\ is lower than $10^{-5}$ ensuring an FPR=$10^{-5}$.
We prompt Guanaco-7-B~\citep{dettmers2023qlora}, an instruction fine-tuned version of Llama, with the first $1$k prompts from the Alpaca dataset~\citep{alpaca}.
For generation, we use top-$p$ sampling with $p=0.95$, and in the case of \citep{kirchenbauer2023watermark} a temperature $\theta =0.8$ and $\gamma=1/4$.
We simulate synonym attacks by randomly replacing tokens with probability $0.3$ (other attacks are studied in related work~\citep{kirchenbauer2023reliability}).

\autoref{chap5/tab:robustness} reports the TPR for different strength of the watermark (see Sec.~\ref{chap5/sec:background}), and the S-BERT~\citep{reimers2019sentence} similarity score between the generated texts with and without watermarking to measure the semantic distortion induced by the watermark. 
Results reveals different behaviors.
For instance, \citep{kirchenbauer2023watermark} has a finer control over the trade-off between watermark strength and quality.
Its TPR values ranges from 0.0 to 0.9, while \citep{aaronson2023watermarking} is more consistent but fails to achieve TPR higher than 0.8 even when the S-BERT score is degraded a lot.

The watermark context width also has a big influence. 
When $k$ is low, we observe that repetitions happen more often because the generation is easily biased towards certain repetitions of tokens.
It leads to average S-BERT scores below 0.5 and unusable completions.
On the other hand, low $k$ also makes the watermark more robust, especially for \citep{kirchenbauer2023watermark}.
It is also important to note that $k$ has an influence on the number of analyzed tokens since we only score tokens for which the $k+1$-tuple has not been seen before (see Sec.~\ref{chap5/sec:rect}).
If $k$ is high, almost all these tuples are new, while if $k$ is low, the chance of repeated tuples increases.
For instance in our case, the average number of scored tokens is around 100 for $k=0$, and 150 for $k=1$ and $k=4$.


\subsection{Impact on free-form generation tasks}
\label{chap5/sec:free-form}
Previous studies measure the impact on quality using distortion metrics such as perplexity or similarity score as done in Tab.~\ref{chap5/tab:robustness}.
However, such metrics are not informative of the utility of the model for downstream tasks~\citep{holtzman2019curious}, where the real interest of LLMs lies. 
Indeed, watermarking LLMs could be harmful for tasks that require very precise answers, like code or maths.
This section rather quantifies the impact on typical NLP benchmarks, in order to assess the practicality of watermarking.

LLMs are typically evaluated by comparing samples of plain generation to target references (free-form generation) or by comparing the likelihood of predefined options in a multiple choice question fashion. 
The latter makes little sense in the case of watermarking, which only affects sampling.
We therefore limit our evaluations to free-form generation tasks.
We use the evaluation setup of Llama:
1) Closed-book Question Answering (Natural Questions~\citep{kwiatkowski2019natural}, TriviaQA~\citep{joshi2017triviaqa}): we report the $5$-shot exact match performance;
2) Mathematical reasoning (MathQA~\citep{hendrycks2021measuring}, GSM8k~\citep{cobbe2021training}), we report exact match performance without majority voting;
3) Code generation (HumanEval~\citep{chen2021Evaluating}, MBPP~\citep{austin2021program}), we report the pass@1 scores.
For \citep{kirchenbauer2023watermark}, we shift logits with $\delta=1.0$ before greedy decoding.
For \citep{aaronson2023watermarking}, we use $\theta = 0.8$, apply top-p at $0.95$ to the probability vector, then apply the watermarked sampling.

\autoref{chap5/tab:bench-full} reports the performance of Llama models on the aforementioned benchmarks, with and without the watermark and for different window size $k$. 
The performance of the LLM is not significantly affected by watermarking. 
The approach of \cite{kirchenbauer2023watermark} is slightly more harmful than the one of \cite{aaronson2023watermarking}, but the difference w.r.t. the vanilla model is small.
Interestingly, this difference decreases as the size of the model increases: models with higher generation capabilities are less affected by watermarking. A possible explanation is that the global distribution of the larger models is better and thus more robust to small perturbations.
Overall, evaluating on downstream tasks points out that watermarking may introduce factual errors that are not well captured by perplexity or similarity scores.











\begin{table}[H]
    \centering
    \caption{ 
        Performances on free-form generation benchmarks when completion is done with watermarking.
        $k$ is the watermark context width. 
        We report results for methods: [AK]~\citep{aaronson2023watermarking} / [KGW]~\citep{kirchenbauer2023watermark}.
        ``-'' means no watermarking. 
    }
    \label{chap5/tab:bench-full}
    \resizebox{0.95\textwidth}{!}{
    \begin{tabular}{lllrrrrrrr}
    \toprule
     &  &  & GSM8K & Human Eval & MathQA & MBPP & NQ & TQA & Average \\
    Model & Method & $k$ &  &  &  &  &  &  &  \\
    \midrule
     \multirow[t]{15}{*}{7-B} 
     & None & - & 10.31 & 12.80 & 2.96 & 18.00 & 21.72 & 56.89 & 20.45 \\
     \cmidrule{2-10} 
     & [AK] & 0 & 10.54 & 12.80 & 3.00 & 18.00 & 21.77 & 56.88 & 20.50 \\
      &  & 1 & 10.31 & 12.80 & 2.88 & 18.20 & 21.75 & 56.87 & 20.47 \\
      &  & 2 & 10.31 & 12.80 & 2.94 & 18.00 & 21.75 & 56.86 & 20.44 \\
      &  & 4 & 10.39 & 12.80 & 2.98 & 17.80 & 21.80 & 56.88 & 20.44 \\
      &  & 8 & 10.46 & 12.80 & 2.90 & 18.20 & 21.75 & 56.85 & 20.49 \\
      \cmidrule{2-10} 
      & [KGW] & 0 & 9.63 & 12.80 & 2.20 & 16.20 & 20.06 & 55.09 & 19.33 \\
      &  & 1 & 11.14 & 9.76 & 2.82 & 16.00 & 19.50 & 55.30 & 19.09 \\
      &  & 2 & 11.07 & 6.71 & 2.62 & 16.00 & 20.44 & 55.07 & 18.65 \\
      &  & 4 & 10.77 & 9.15 & 2.76 & 16.40 & 20.17 & 55.14 & 19.06 \\
      &  & 8 & 11.37 & 11.59 & 2.90 & 16.40 & 20.66 & 55.36 & 19.71 \\
    \midrule
    \multirow[t]{15}{*}{13-B} 
    & None & - & 17.21 & 15.24 & 4.30 & 23.00 & 28.17 & 63.60 & 25.25 \\
    \cmidrule{2-10} 
    & [AK] & 0 & 17.29 & 15.24 & 4.24 & 22.80 & 28.17 & 63.60 & 25.22 \\
     &  & 1 & 17.21 & 15.24 & 4.30 & 22.80 & 28.20 & 63.61 & 25.23 \\
     &  & 2 & 17.51 & 15.24 & 4.20 & 22.80 & 28.20 & 63.59 & 25.26 \\
     &  & 4 & 17.21 & 15.24 & 4.20 & 22.60 & 28.20 & 63.63 & 25.18 \\
     &  & 8 & 17.21 & 15.24 & 4.22 & 22.80 & 28.20 & 63.62 & 25.22 \\
     \cmidrule{2-10} 
     & [KGW] & 0 & 14.33 & 14.02 & 3.04 & 20.80 & 24.32 & 62.13 & 23.11 \\
     &  & 1 & 17.29 & 14.63 & 3.62 & 21.20 & 25.12 & 62.23 & 24.02 \\
     &  & 2 & 16.45 & 11.59 & 3.54 & 20.60 & 25.54 & 62.44 & 23.36 \\
     &  & 4 & 16.76 & 15.85 & 4.08 & 21.20 & 24.49 & 62.24 & 24.10 \\
     &  & 8 & 17.29 & 14.63 & 3.68 & 21.00 & 25.46 & 62.17 & 24.04 \\
     \midrule
    \multirow[t]{14}{*}{30-B} 
     & None & - & 35.10 & 20.12 & 6.80 & 29.80 & 33.55 & 70.00 & 32.56 \\
     \cmidrule{2-10} 
     & [AK] & 0 & 35.48 & 20.12 & 6.88 & 29.80 & 33.52 & 69.98 & 32.63 \\
     & & 1 & 35.33 & 20.73 & 6.88 & 29.60 & 33.52 & 70.03 & 32.68 \\
     & & 2 & 35.33 & 20.73 & 6.94 & 30.00 & 33.49 & 70.00 & 32.75 \\
     & & 4 & 35.10 & 20.12 & 6.90 & 29.80 & 33.49 & 70.01 & 32.57 \\
     & & 8 & 35.33 & 20.73 & 6.94 & 30.00 & 33.52 & 70.01 & 32.75 \\
     \cmidrule{2-10} 
     & [KGW] & 0 & 31.84 & 21.95 & 6.88 & 28.40 & 31.66 & 69.03 & 31.63 \\
     &  & 1 & 35.56 & 20.73 & 7.54 & 28.80 & 31.58 & 68.98 & 32.20 \\
     &  & 2 & 33.21 & 17.07 & 6.48 & 27.40 & 31.83 & 69.44 & 30.91 \\
     &  & 4 & 34.12 & 22.56 & 6.96 & 28.80 & 31.55 & 68.74 & 32.12 \\
     &  & 8 & 34.95 & 20.12 & 7.42 & 27.20 & 32.08 & 69.31 & 31.85 \\
    \bottomrule
    \end{tabular}
    }
\end{table}









\section{Advanced detection schemes}
This section introduces improvements to the detection schemes of Sec.~\ref{chap5/sec:stats}.
Namely, it develops a statistical test when access to the LLM is granted, as well as multi-bit decoding.


\subsection{Neyman-Pearson and simplified score function} 
The following is specific for the scheme of~\citet{aaronson2023watermarking} -- a similar work may be conducted with the one of~\citet{kirchenbauer2023reliability}.
Under $\H_0$, we have $\vec{r}_v\sim\mathcal{U}_{[0,1]}$, whereas $\vec{r}_v\sim Beta(1/p_v,1)$ under $\H_1$ (see Corollary~\eqref{chap5/eq:Coro} in App.~\ref{chap5/app:aaronson_prob}). 
The optimal Neyman-Pearson score function is thus:
\begin{equation*}
    s_T = \sum_{t=1}^{T} \ln\frac{f_{\H_1}(\vec{r}_{x^{(t)}})}{f_{\H_0}(\vec{r}_{x^{(t)}})} = \sum_{t=1}^T \left(\frac{1}{\vec{p}_{x^{(t)}}}-1\right)\ln(\vec{r}_{x^{(t)}})+A
\end{equation*}
where $A$ is a constant that does not depend on $\vec{r}$ and can thus be discarded. 
There are two drawbacks: (1) detection needs the LLM to compute $\vec{p}_{x^{(t)}}$, (2) there is no close-form formula for the p-value.  

This last point may be fixed by resorting to a Chernoff bound, yet without guarantee on its tightness:
$\text{p-value}(s) \leq e^{\sum_t \ln\frac{\lambda_t}{\lambda_t + c} -cs}$,
with $c$ solution of $\sum_t (c+\lambda_t)^{-1}=-s$ and $\lambda_t = p_{x^{(t)}} / (1-p_{x^{(t)}})$.
Experiments show that this detection yields extremely low \pval\ for watermarked text, but they are fragile: any attack increases them to the level of the original detection scheme~\eqref{chap5/eq:score-aaronson}, or even higher because generated logits are sensitive to the overall LLM context. 
An alternative is to remove weighting:
\begin{equation}
 s_T = \sum_{t=1}^T \ln\left(\vec{r}_{x^{(t)}}\right),
 \label{chap5/eq:Detection2}
\end{equation}
whose \pval\ is given by: $\text{p-value}(s) = \frac{\gamma(T,-s)}{\Gamma(T)}$.
In our experiments, this score function does not match the original detection presented in~\citep{aaronson2023watermarking}.


\subsection{Multi-bit watermarking}

       
    


\paragraph*{Theory.} It is rather easy to turn a zero-bit watermarking scheme into multi-bit watermarking, by associating a secret key per message. 
The decoding runs detection with every key and the decoded message is the one associated to the key giving the lowest \pval\ $p$. 
The global \pval\ becomes $1-(1-p)^M$, where $M$ is the number of possible messages.

Running detection for $M$ keys is costly, since it requires $M$ generations of the secret vector.
This is solved by imposing that the secret vectors of the messages $m\in\{0,\ldots,M-1\}$ are crafted as circular shifts of $m$ indices of $\vec{r}=\vec{r}(0)$:
\begin{align*}
\vec{r}(m) &= \mathsf{CyclicShift}(\vec{r},m) \\
    &= \left( \vec{r}_m, \vec{r}_{m+1}, ..,\vec{r}_{d}, \vec{r}_{0}, ..,  \vec{r}_{m-1}  \right).
\end{align*}
Generating $\vec{r}$ as a $d$-dimensional vector, with $d\geq|\V|$, we are able to embed $M\leq d$ different messages, by keeping only the first $|\V|$ dimensions of each circularly-shifted vector. 
Thus, the number of messages may exceed the size of the token vocabulary $|\V|$.
One way is to choose $d =  \textrm{max}(M, |\V|)$.



Besides, the scoring functions~\eqref{chap5/eq:score-kirchenbauer}~\eqref{chap5/eq:score-aaronson}
may be rewritten as:
\begin{equation}
s_T(m) = \sum_{t=1}^T f\left(\vec{r}^{(t)}(m)\right)_{x^{(t)}}  ,
\end{equation}
where $f: \R^d \mapsto \R^d$ is a component-wise function, and $x^{(t)}$ is the selected token during detection. 
This represents the selection of $f\left(\vec{r}^{(t)}(m)\right)$ at position $x^{(t)}$.
From another point of view, if we shift $f\left(\vec{r}^{(t)}\right)$ by $x^{(t)}$, the score for $m=0$ would be its first component, $m=1$ its second one, etc.
We may also write:
\begin{equation}
\vec{S}_T = \sum_{t=1}^T \mathsf{CyclicShift}\left( f\left(\vec{r}^{(t)}\right), x^{(t)} \right) ,
\label{chap5/eq:DetectionMultibit}
\end{equation}
and the first $M$ components of $\vec{S}_T$ are the scores for each $m$.
As a side note, this is a particular case of the parallel computations introduced by~\citet{JAWS}.
The simplified algorithms are given in Alg.~\ref{chap5/alg:multi-bit-gen} and Alg.~\ref{chap5/alg:multi-bit-dec}. 

\noindent
\begin{minipage}{0.48\textwidth}
    \begin{algorithm}[H] \small
    \caption{Generation (one step)}
    \label{chap5/alg:multi-bit-gen}
    \begin{algorithmic}
        \State\hspace*{-0.3cm} \textbf{Requires}: {LLM, dimension $d$, watermark window $k$, message $m\in\{0,\ldots,M-1\}$} \\
        \State logits $\vec{\boldsymbol\ell} \gets \text{LLM} \left( x^{(-C)},\dots, x^{(-1)} \right)$
        \State seed $\gets \mathsf{Hash}(x^{(-k)},\dots, x^{(-1)})$
        \State $\vec{r} \gets \mathsf{RNG_{seed}}(d)$
        \State $\vec{r}(m) \gets \mathsf{CyclicShift}(\vec{r},m)$
        \State $x^{(0)} \gets \mathsf{Sample}(\vec{\boldsymbol\ell},\vec{r}(m)_{1,\dots,|\V|})$
    \end{algorithmic}
\end{algorithm}
\end{minipage}\hfill
\begin{minipage}{0.48\textwidth}
    \begin{algorithm}[H] \small
    \caption{Decoding/identification}
    \label{chap5/alg:multi-bit-dec}
    \begin{algorithmic}
        \State $\vec{S} \gets \vec{0}_d$
        \State{\textbf{for} $t \in \{ k+1, \dots, T\}$:}
            \State \quad seed $\gets \mathsf{Hash}(x^{(t-k)},\dots, x^{(t-1)})$
            \State \quad $\vec{r}^{(t)} \gets \mathsf{RNG_{seed}}(d)$
            \State \quad $\vec{S} \gets \vec{S} +  \mathsf{CyclicShift}(f(\vec{r}^{(t)}),x^{(t)})$
        \State $\vec{p} \gets \textrm{p-value}(\vec{S}_{1,\dots,M})$
        \State $m \gets \textrm{argmin}({\vec{p}}) $
        \State $p \gets 1 - (1 - \vec{p}_m)^M$
    \end{algorithmic}
    \end{algorithm}
\end{minipage}


\paragraph*{Experiments.} 
In a tracing scenario the message is the identifier of a user or a version of the model.
The goal is to decide if any user or model generated a given text (detection) and if so, which one (identification).
There are 3 types of error: \emph{false positive}: flag a vanilla text; \emph{false negative}: miss a watermarked text; \emph{false accusation}: flag a watermarked text but select the wrong identifier.


\begin{table}[t]
    \caption{Identification accuracy for tracing users by watermarking. 
    Sequences are between $4$ and $252$ tokens long, and $149$ on average.
    }
    \label{chap5/tab:identification}
    \centering
    \footnotesize
    \begin{tabular}{cl cccc}
        \toprule
        & & \multicolumn{4}{c}{Number of users $M$} \\
        \cmidrule{3-6}
        & Method & $10$ & $10^2$ & $10^3$ & $10^4$ \\ \midrule
        \multirow{2}{*}{FPR$=10^{-3}$} & [AK] \citep{aaronson2023watermarking}      & 0.80 & 0.72 & 0.67 & 0.62 \\
        & [KGW] \citep{kirchenbauer2023watermark}  & 0.84 & 0.77 & 0.73 & 0.68 \\ \midrule
        \multirow{2}{*}{FPR$=10^{-6}$} & [AK] \citep{aaronson2023watermarking}	      & 0.61 & 0.56 & 0.51 & 0.46 \\
        & [KGW] \citep{kirchenbauer2023watermark} 	                              & 0.69 & 0.64 & 0.59 & 0.55 \\
        \bottomrule
    \end{tabular}
\end{table}


We simulate $M'$=$1000$ users that generate $100$ watermarked texts each, using the Guanaco-7B model. 
Accuracy can then be extrapolated beyond the $M'$ identifiers by adding identifiers with no associated text, for a total of $M>M'$ users.
Text generation uses nucleus sampling with top-p at $0.95$.
For~\citep{kirchenbauer2023watermark}, we use $\delta=3.0$, $\gamma=1/4$ with temperature $\theta$ at $0.8$.
For~\citep{aaronson2023watermarking}, we use $\theta = 1.0$.
For both, the context width is $k=4$.
A text is deemed watermarked if the score is above a threshold set for a given \emph{global} FPR (see~\ref{chap5/sec:stats}).
Then, the source is identified as the user with the lowest p-value.

\autoref{chap5/tab:identification} shows that watermarking performance for identification is dissuasive enough. 
For example, among $10^5$ users, we successfully identify the source of a watermarked text 50\% of the time while maintaining an FPR of $10^{-6}$ (as long as the text is not attacked).
At this scale, the false accusation rate is zero (no wrong identification once we flag a generated text) because the threshold is set high to avoid FPs, making false accusations unlikely. 
The identification accuracy decreases when $M$ increases, because the threshold required to avoid FPs gets higher.
In a nutshell, by giving the possibility to encode several messages, we trade some accuracy of detection against the ability to identify users.

\section{Conclusion}
We reveal a tradeoff in robust watermarks: Improved redundancy of watermark information enhances robustness, but increased redundancy raises the risk of watermark leakage. We propose DAPAO attack, a framework that requires only one image for watermark extraction, effectively achieving both watermark removal and spoofing attacks against cutting-edge robust watermarking methods. Our attack reaches an average success rate of 87\% in detection evasion (about 60\% higher than existing evasion attacks) and an average success rate of 85\% in forgery (approximately 51\% higher than current forgery studies). 
