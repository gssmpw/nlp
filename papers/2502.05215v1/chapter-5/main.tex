
\chapter{Three Bricks to Consolidate Watermarks for Large Language Models}\label{chapter:three-bricks}

This chapter is based on the paper \fullcite{fernandez2023three}.

Discerning between generated and natural texts is increasingly challenging due to the rapid progress of large language models (LLMs).
In this context, watermarking again emerges as a promising technique for ascribing text to a specific generative model. 
It alters the sampling generation process to leave an invisible trace in the output, facilitating later detection.
This research consolidates watermarks for large language models based on three theoretical and empirical considerations. 
First, we introduce new statistical tests that offer robust theoretical guarantees which remain valid even at low false-positive rates (less than 10$^{\text{-6}}$). 
Second, we compare the effectiveness of watermarks using classical benchmarks in the field of natural language processing, gaining insights into their real-world applicability.
Third, we develop advanced detection schemes for scenarios where access to the LLM is available, as well as multi-bit watermarking. 
Code is available at \url{github.com/facebookresearch/three_bricks}.

\newpage
\section{Introduction}
\label{sec:intro}

Foundational models (FMs)~\cite{zhang2024data, zhou2023comprehensive} have shown remarkable progress in the healthcare domain, enabling professional-like assessment of disease diagnosis, treatment decision-making, and monitoring~\cite{zhang2023text, wang2022medclip, lu2023mi-zero}. 
Examples include LLaVA-Med~\cite{li2023llava}, Med-PaLM Multimodal~\cite{tu2024towards}, and Med-Flamingo~\cite{moor2023med}, have demonstrated their capacity on question answering, medical image analysis, and report generation.
These studies follow a predominant top-down model development strategy that requires upstream developers to collect data and train models for downstream tasks. 
Consequently, the developed model capabilities are heavily dependent on the training data, limiting their generalization performance in diverse clinical scenarios. 
For instance, Med-Gemini~\cite{yang2024advancing} reveals promising general capabilities in report generation while it lags behind state-of-the-art (SoTA) models on classification tasks, especially for out-of-domain applications. 
This indicates that while the generalizability of the foundation model is promising, more solutions are expected to meet the various specialized clinical needs.

To address these challenges, multi-center data centralization becomes essential to enhance model capacity and robustness across varied clinical scenarios~\cite{rajpurkar2022ai}. 
Centralizing distributed data can significantly improve model training and inference performance.
However, the process of medical data storage, transfer, and aggregation among centers requires extra efforts to ensure data security and system interoperability~\cite{bradford2020international}.
Moreover, a growing concern for patient privacy makes large-scale multi-center data sharing particularly challenging. 
While efforts like federated learning~\cite{wen2023survey, li2020review} can achieve good model performance on local data, the need for synchronized system coordination presents significant challenges, as clients are unable to update asynchronously. This limitation greatly restricts the practical capability of such approaches.
As a result, without a flexible collaboration, medical community still struggles to fully utilize the isolated data and local computation resources for comprehensive medical AI model development. 
To address this dilemma, open-source platforms encourage public data sharing and knowledge integration~\cite{markiewicz2021openneuro, zenodo}.
However, these platforms focus solely on raw data sharing while seldom providing collaborative model training or cooperation between different institutions.
Recently, collaborative learning has emerged as a viable approach for enhancing multi-model robustness~\cite{boulemtafes2020review}. 
For instance, software-like model development~\cite{raffel2023building} mimics software engineering practices by introducing structured workflows, enabling merging, version control, and continuous model integration.
Under this design, model ability can be strengthened with incremental knowledge updates similar to the version updating in software development. 

Although collaborative learning provides a multi-model collaboration, two key challenges remain in the leakage of raw data during collaboration~\cite{huang2023lorahub} and the synchronization of multiple collaborators~\cite{mcmahan2017communication} in the medical AI community. It is still challenging to integrate decentralized, privacy-sensitive data across institutions, leading to under-utilized insights and fragmented knowledge sharing~\cite{kaissis2020secure, rajpurkar2022ai, abdullah2021ethics}.
 To address these challenges, inspired by the collaborative software development, we propose \textbf{Med}ical \textbf{Fo}undation Models Me\textbf{rg}ing (\textbf{MedForge}), a cooperative workflow enabling continuously community-driven foundation model (FM) development.
MedForge enables a lightweight manner for individual centers to share their knowledge among multiple centers, minimizing the burden of data transmission and integration while enhancing model robustness.
Meanwhile, MedForge facilitates asynchronous and flexible collaboration, allowing individual centers to continuously update and improve medical FMs without the need for real-time synchronization.
Similar to open-source software development, MedForge incrementally updates medical knowledge and follows a sustainable model development scheme. 
This key design emphasizes a bottom-up construction of a multi-task medical FM, allowing downstream users to collaboratively build, refine, and update the upstream model according to their local resources. Our major contributions of MedForge are as below: 
\begin{enumerate}
    \item[$\bullet$] We introduce a collaborative workflow to promote the merging scheme of open-source software development. Our proposed MedForge allows distributed clinical centers to asynchronously contribute to comprehensive medical model construction while reducing transmitting costs among centers and avoiding the leakage of raw data, thus enhancing the utilization of private resources in the healthcare system. 
    \item[$\bullet$] We propose two effective knowledge-merging strategies for the asynchronous branch contribution. The MedForge-Fusion strategy updates the plugin module parameters of the main model during the merging phase, whereas the MedForge-Mixture strategy integrates the output of the plugin module by memorizing each contributor's coefficient. These strategies make MedForge more flexible and versatile. MedForge-Fusion is friendly to implement, while the MedForge-Mixture offers better performance and robustness.
    \item[$\bullet$]  We comprehensively evaluate model merging strategies to accumulate medical knowledge among multiple branch plugin modules. MedForge yields superior performance on medical classification tasks compared to other collaborative baselines across multiple datasets. We demonstrate the robustness of MedForge by shuffling the task order and evaluating various configurations of plugin modules and dataset distillation methods.
\end{enumerate}



\section{Background} \label{sec:background}

% \subsection{Capture the Flag (CTF) Challenges}

% CTF challenges simulate real-world cyber-attack scenarios and have emerged as a popular medium for practical cybersecurity training, evaluation, and research. These challenges can simulate real-world attack and defense scenarios and thus assist competitors in developing practical skills in areas such as cryptography, binary exploitation, and reverse engineering. 
% Evaluation of autonomous LLM agents works best with jeopardy-style CTF challenges that focus on standalone software that must be compromised \cite{shao2024nyu,pieterse2024friend}.
% The standalone software may be a binary that can be reverse engineered or exploited, encrypted data that can be decrypted, or a web server whose authentication can be bypassed. After successfully compromising the software, a unique ``flag'' string is either found or revealed by the software server.
% The unique flag string is a concrete indicator of the success of a CTF challenge.
% Recent studies use benchmarks of CTF challenges to evaluate LLM agents on their ability to solve complex tasks and demonstrate practical skills in cybersecurity \cite{shao2024nyu,shao2024empirical,abramovich2024enigma, muzsai2024hacksynth, zhang2024cybenchframeworkevaluatingcybersecurity,yang2023language,turtayev2024hacking}
% Platforms like PicoCTF~\cite{picoctf}, TryHackMe~\cite{tryhackme}, CTFTime~\cite{ctftime} and HackTheBox~\cite{hackthebox} have popularized these formats by providing structured challenges for learners at various skill levels.

% Research indicates that CTF challenges can foster cybersecurity expertise and serve as tools for evaluating facility with cybersecurity skills~\cite{chicone2018using}. They are widely used in academia to enhance learning outcomes in cybersecurity education, with studies demonstrating their effectiveness in promoting analytical thinking and teamwork~\cite{hanafi2021ctf,leune2017using,vykopal2020benefits}. Furthermore, the integration of CTF challenges into research environments enables benchmarking of advanced AI systems like LLMs. .

% Yet, challenges in CTF design persist. These include achieving significant performance, preserving context across tasks, and handling complex, dynamic CTFs that rely on multidisciplinary approaches. Implementing strategies to address these issues enhances problem-solving efficiency, enabling more accurate, adaptive, and effective responses to evolving challenges within CTF environments.


% \subsection{Prompt Engineering}
% \subsection{Prompt Engineering for CTF}
% \subsection{LLM Agents}

% As the use of LLMs to solve CFT challenges expands, prompt engineering is becoming a critical technique for enhancing performance. Various methods have been explored to craft prompts that effectively guide LLMs to the solution of complex cybersecurity problems. Each of these solutions have their own unique strengths and limitations.
%\meet{add more references for LLM agents in other domains, like SWE-Agent, also talk about use of function calling}
Text-based LLMs take a text prompt as input from the user, and produce a text output that follows the user prompt.
LLMs have a finite length of text tokens that they can process called the context.
An alternating sequence of user prompts and LLM outputs makes a conversation and is the basis of chat-based LLM interfaces like ChatGPT.
To remove the user from the loop and create autonomous agents, a feedback mechanism is added based on the LLM outputs, so that the LLM can autonomously continue the conversation.
\citet{yang2023intercode} introduce iterative feedback prompting where the LLM is tasked with writing a piece of code, and the code's compilation and execution logs are provided as feedback, which the LLM uses to iteratively refine it's output.
Recent LLMs support function calling, a way to provide a set of actions to the LLM that it may choose to ``call'' as a function.
In this manner, LLM agents can be provided with many ``tools'' such as a command line, web search, file editing, and code execution \cite{wang2024surveyllmagents}, so that they can autonomously perform various tasks like software development \cite{yang2024sweagent}, web browsing \cite{yoran2024assistantbench}, or solve CTF challenges~\cite{shao2024nyu, abramovich2024enigma}.

With access to the command line and file editing tools, LLM agents can autonomously solve many tasks, but they still struggle on complex long-horizon tasks such as CTF challenges that require multiple steps.
Plan-and-solve prompting \cite{wang2023planandsolve} enhances long-term focus of the agent by incorporating a planning phase before iterative execution. This helps agents tackle ambiguous or complex tasks by structured strategies \cite{turtayev2024hacking}.
ReAct (reasoning + action) \cite{yao2022react} combines step-by-step reasoning with action, allowing the agent to adjust dynamically through iterative cycles. ReWOO (Reasoning without Observation) \cite{xu2023rewoo} separates the reasoning process from tool outputs and observations, allowing it to handle multi-step reasoning tasks efficiently while maintaining focus.
The prompting methods in these agents involve static hard-coded templates where environment and task information is filled in.
While static prompts provide straightforward guidance, they often fail to adapt to different problems and complex tasks, limiting their effectiveness.
Auto-prompting~\cite{shin-etal-2020-autoprompt, zhou-etal-2023-revisiting, zhang2023automatic} is a technique to allow the LLM itself to generate a highly-relevant prompt. Auto-prompting invokes more factual responses and reduces hallucinations in LLMs.
D-CIPHER incorporates auto-prompting as a separate agent that can explore the environment and generate a better prompt.
%Based on the given prompt, LLM agents make a decision and proceed further to find flags.  To address this gap, we propose \textbf{dynamic prompting}, where the LLM agent autonomously generates prompts based on the CTF challenge's context and stage.
%include a static template which needs to be given to LLM to solve the CTF challenges. For instance, the NYU CTF framework provides a static prompt as \emph{``Please proceed to the next step using your best judgment"} for each decision making point. 

% To address this gap, we introduce a novel approach where the LLM agent generates the next prompt autonomously based on the current context and stage of the CTF challenge, a technique we call \textbf{dynamic prompting}.


Expanding on single LLM agents, multi-agent LLM systems are a powerful approach to enhance problem-solving by simulating team-based collaboration. Specialized agents, each with distinct objectives, work together to tackle different aspects of complex tasks \cite{guo2024largelanguagemodelbased}
Multi-agent systems are effective in cybersecurity applications. For instance, Audit-LLM~\cite{song2024audit} deploys a  multi-agent system for insider threat detection by employing agents to decompose tasks, build tools, and use collaborative reasoning to enhance detection accuracy. Liu~\cite{liu2024multi} explores multi-agent systems to enhance incident response in cybersecurity by examining centralized, decentralized, and hybrid team structures to assess how LLM agents can improve decision-making, adaptability, and coordination during cyber-attack scenarios. AutoSafeCoder~\cite{nunez2024autosafecoder} enhances the security of code generated by LLMs by incorporating a coding agent for code generation, a static analyzer agent that identifies vulnerabilities, and a fuzz testing agent for dynamic testing to detect runtime errors. Division of responsibilities among different agents allows AutoSafeCoder to produce secure, functionally correct code. 

% With the growing use of LLMs in CTF challenges, prompt engineering is key to enhancing performance. Various methods guide LLMs in solving complex cybersecurity tasks, each with distinct strengths and limitations.

% \textbf{Single Turn (Zero-Shot Prompting)} involves providing the model with a one-time task description that outputs  an immediate solution. This is efficient for straightforward tasks~\cite{yang2023intercode}. In contrast, \textbf{Try Again (Iterative Feedback Prompting)} uses iterative feedback to refine responses over multiple attempts, mimicking real-world problem-solving~\cite{yang2023intercode}. The \textbf{Plan \& Solve} enhances adaptability by incorporating a planning phase before iterative execution. This helps models tackle ambiguous or complex tasks by  structured strategies~\cite{turtayev2024hacking}. Additionally, \textbf{ReAct (Reasoning + Action)} combines step-by-step reasoning with action, allowing the model to adjust dynamically through iterative cycles. This makes it particularly effective for evolving and complex challenges like CTFs~\cite{yao2023react}. 
% These prompting techniques highlight diverse approaches to optimizing LLM performance in cybersecurity tasks. 

% Multi-agents!


%\meet{Add references for auto-prompting, and shorten this para}
%\nanda{Maybe we can add this to previous paragraphs which discusses other prompting methods such as plan-and-solve and ReAct method}
% All of these prompting methods include a static template which needs to be given to LLM to solve the CTF challenges. For instance, the NYU CTF framework provides a static prompt as \emph{``Please proceed to the next step using your best judgment"} for each decision making point. 
% Based on the given prompt, LLM agents make a decision and proceed further to find flags. While static prompts provide straightforward guidance, they often fail to account for the evolving nature of complex tasks, limiting their effectiveness in multi-step or ambiguous CTF challenges. To address this gap, we propose \textbf{dynamic prompting}, where the LLM agent autonomously generates prompts based on the CTF challenge's context and stage.
% % To address this gap, we introduce a novel approach where the LLM agent generates the next prompt autonomously based on the current context and stage of the CTF challenge, a technique we call \textbf{dynamic prompting}.
% Dynamic prompting adapts instructions to task progress, ensuring instructions are contextually relevant and reflective of the specific obstacles encountered. By iterating based on feedback and intermediate outputs, it continuously refines the LLM’s approach, enhancing problem-solving for dynamic tasks like CTFs.
% This adaptive process not only mirrors how humans tackle complex problems but also improves the model’s ability to handle unpredictable scenarios, making it particularly advantageous for cybersecurity tasks like CTFs where conditions change dynamically.


% The very first prompt type used in several applications is \textbf{Single Turn (Zero-Shot Prompting)}~\cite{yang2023intercode}. In single-turn prompting, the model receives a one-time, straightforward task description and is expected to generate a complete response without further interaction. The initial output is directly assessed, making this approach efficient for tasks where minimal feedback or iteration is required. This method tests the model’s ability to understand and respond to tasks immediately, relying heavily on the model's pre-trained knowledge and generalization capabilities.

% Along with this, The prompting method named \textbf{Try Again (Iterative Feedback Prompting)}~\cite{yang2023intercode} has been also used in several appreciations specially to solve CTF challenges. It is an iterative prompting method involves continuous interaction, where the model is provided with feedback after each attempt. The model can refine its responses over multiple turns based on the observations or execution results from previous outputs. This iterative process continues until the task is successfully completed or a maximum number of interactions is reached. This approach closely mirrors real-world problem-solving, where adjustments are made iteratively based on evolving circumstances or feedback.

% Some application are also using \textbf{Plan \& Solve}~\cite{turtayev2024hacking} prompting method which enhances problem-solving by dividing the process into a planning phase followed by execution. Initially, the model formulates a strategy based on the task description and available information, allowing for a structured approach to ambiguous or complex problems. This plan guides the subsequent execution phase, where the model carries out actions iteratively, refining its approach based on feedback. In more challenging scenarios, re-planning mid-task further improves adaptability and performance. This method proves effective in tasks like CTF challenges, where vague instructions require careful analysis and step-by-step resolution.

% Further some application are also using \textbf{ReAct (Reasoning + Action)}~\cite{yao2023react} prompting method blends reasoning with action by guiding the model to think through tasks step-by-step before executing actions. At each step, the model generates a thought based on the task and observations, which informs the next action. The action is executed, and the resulting feedback refines the model’s understanding for the next cycle. This continuous process helps the model adapt dynamically to complex tasks, making it effective for CTF challenges where logical reasoning and step-by-step execution are essential.

\section{Related Works} \label{sec:related_work}


\begin{table}[htpb]
    \centering
    \caption{Feature comparison of LLM agents for solving CTFs.}
    \label{tab:related_work_comparison}
    \begin{tabular}{lcccccc}
    \toprule
         \textbf{Study} & \rotatebox{90}{\textbf{\# CTFs}} & \rotatebox{90}{\textbf{Open bench}} & \rotatebox{90}{\textbf{Tool use}}  & \rotatebox{90}{\textbf{Autonomous}} & \rotatebox{90}{\textbf{Multi-agent}} &\rotatebox{90}{\textbf{Auto-prompt}} \\
    \cmidrule{2-7}
     % \textbf{Study} & \textbf{Dynamic} & \textbf{Used} & \textbf{Multi-} & \textbf{Automatic} & \textbf{Tool} & \textbf{\# of} \\
         Tann et al. \cite{tann2023using} &  $7$ & \purplecross & \purplecross & \purplecross & \purplecross & \purplecross  \\
         Shao et al. \cite{shao2024empirical} & $26$ & \purplecross & \tealcheck & \tealcheck & \purplecross & \purplecross  \\
         InterCode-CTF\cite{yang2023language} & $100$ & \tealcheck & \tealcheck & \tealcheck & \purplecross & \purplecross   \\
         NYU CTF Bench \cite{shao2024nyu} & $200$ & \tealcheck & \tealcheck & \tealcheck & \purplecross & \purplecross \\
         Turtayev et al. \cite{turtayev2024hacking} & $100$ & \tealcheck & \tealcheck & \tealcheck & \purplecross & \purplecross\\
         Cybench \cite{zhang2024cybenchframeworkevaluatingcybersecurity} & $40$ & \tealcheck & \tealcheck & \tealcheck & \purplecross & \purplecross \\
         EnIGMA \cite{abramovich2024enigma} & $350$ & \tealcheck & \tealcheck & \tealcheck & \purplecross & \purplecross\\
         HackSynth \cite{muzsai2024hacksynth} & $200$ & \tealcheck & \tealcheck & \tealcheck & \tealcheck & \purplecross \\
         \textbf{D-CIPHER (ours)} & $290$ & \tealcheck & \tealcheck & \tealcheck & \tealcheck & \tealcheck \\
    \bottomrule
    \end{tabular}
\end{table}



% \subsection{LLMs on Cybersecurity}
% \subsection{LLM Agents for CTF}

%LLMs have a vast knowledge base that can be tapped for cybersecurity use.
Tann et al.~\cite{tann2023using} evaluate early LLMs such as ChatGPT and Google Bard in solving CTF challenges and answering professional certification questions, showing that LLM responses contain key task information.
%Many works extend the LLM capabilities by providing them access to programming and command execution tools, to form autonomous agents. 
The InterCode-CTF agent~\cite{yang2023intercode} reveals that LLM agents demonstrate basic cybersecurity skills, however they struggle with more complex tasks.
The NYU CTF baseline agent~\cite{shao2024empirical} integrates external tools into the LLM's function-calling features and demonstrate improved potential of tool-assisted LLMs to solve CTFs, however it exhausts the LLM context length when command output history becomes very long. InterCode-CTF manages this issue by truncating the history to only show the LLM the last few iterations. Even so, LLM agents face issues with longer tasks.
%NYU CTF Bench~\cite{shao2024nyu}, a benchmark of 200 CTF challenges, presents a baseline agent with specialized reverse engineering tools and category-specific prompts, demonstrating their importance to solve CTFs.
% The NYU CTF baseline agent faces issues of LLM context length when complex tasks run for several iterations and the entire command and output history becomes longer than the LLM's context window size. The InterCode agent manages this issue by truncating the history to only show the LLM the last few iterations.


Excessive tool availability and verbose interfaces can overwhelm agents, leading to inefficiencies. Agents perform better with a focused set of tools with well-defined interfaces~\cite{yang2024sweagent}.
EnIGMA~\cite{abramovich2024enigma} agent incorporates interactive tools and in-context learning techniques to achieve state-of-the-art results. % on the NYU CTF Bench, HackTheBox, and Cybench benchmarks.
For better context management, EnIGMA also uses an LLM summarizer that summarizes the command outputs for the main agent.

HackSynth~\cite{muzsai2024hacksynth}, an LLM agent for autonomous penetration testing, shows that iterative planning and feedback summarization stages help the agent finish multiple tasks and improves overall problem solving.
Similarly, Cybench~\cite{zhang2024cybenchframeworkevaluatingcybersecurity} introduces a benchmark of 40 CTF challenges augmented with step-by-step tasks, demonstrating better focus of LLM agents on smaller tasks, leading to improved success and alleviating the context length issue.
\citet{turtayev2024hacking} expand on InterCode-CTF by implementing plan-and-solve prompting, achieve significant improvement on the InterCode-CTF benchmark. They show that prompting techniques can improve performance even with simple toolsets.
% . Their baseline agent is evaluated in unguided mode (i.e. fully autonomous), and guided mode where the agent is given one task at a time. Their results indicate that providing smaller tasks to the LLM agents improve their focus yielding improved success on complex challenges while .

These works highlight that LLM agents excel at implementing code and executing commands to accomplish small concrete tasks when provided with dynamic feedback and task-specific toolsets. While these works  involved using multiple LLMs with different tasks such as planning and summarizing along-side a main agent, D-CIPHER is the first work to formulate a multi-agent system where there is a bifurcation of responsibilities between agents and meaningful well-defined interactions for dynamic feedback.
Table~\ref{tab:related_work_comparison} shows a feature comparison of D-CIPHER with related works on LLM agents for autonomous CTF solving.
%\meet{some description of the feature comparison?}
% Recent research has focused on enable autonomous solving of CTF challenges~\cite{shao2024empirical,shao2024nyu,abramovich2024enigma}. These agents typically operate in containerized environments to ensure reproducibility and modularity. 

% As an early effort, Tann et al.~\cite{tann2023using} evaluated the effectiveness of LLMs, such as OpenAI's ChatGPT, Google Bard, and Microsoft Bing, in solving cybersecurity CTF challenges and answering professional certification questions. 
% % Their study results show that LLMs performed well on $7$ CTF test cases, with ChatGPT solving $6$, Bard $2$, and Bing $1$. 
% The study shows that LLM responses often contain key information essential for solving tasks.

% The InterCode framework~\cite{yang2023intercode} approaches coding as an interactive process and uses execution feedback to improve code generation. As described in Yang et al.~\cite{yang2023intercode}, InterCode-CTF integrates CTF benchmarks into a reinforcement learning environment that can evaluate the cybersecurity capabilities of language agents. It features $100$ tasks that tapskills such as reverse engineering, forensics, and binary exploitation. While existing language agents demonstrate basic cybersecurity skills, evaluations indicate they struggle with more complicated complex tasks unless the system is fine-tuned or given external support. 
% cite Intercode: Standardizing and benchmarking interactive coding with execution feedback

% Another notable example is an LM agent developed by Shao et al. specifically to automate CTF tasks. 
% Shao et al.~\cite{shao2024empirical} developed a LM agent to automate CTF tasks.
% % They report an accuracy rate of  $46\%$ on $26$ CTF challenges sourced from CSAW'23 Qualifying round competition using GPT-4.
% By effectively combining LLM capabilities with external tools, the researchers demonstrated the potential of tool-assisted LLMs to solve complex problems. Building on this, the team incorporated a broader range of cybersecurity tools and interfaces that enhance both accuracy and versatility. 
% Empirical results show their system outperforms baselines on both the InterCode CTF benchmark and the NYU CTF benchmark.

% Shao et al.~\cite{shao2024nyu} presented a diverse, open-source database of CTF challenges that can be used to benchmark an LLM's ability to solve cybersecurity problems.
% It provides a scalable platform for developing and testing AI-driven approaches for vulnerability detection and resolution, facilitating advancements in automated cybersecurity tasks. The benchmark database and automated framework were successfully applied to the performance of five LLMs. 

% The Cybench benchmark~\cite{zhang2024cybenchframeworkevaluatingcybersecurity} provides another significant contribution by creating a framework tailored to solving CTF challenges. % Cybench: A framework for evaluating cybersecurity capabilities and risk
% % Their benchmark environment achieves an accuracy of $17.5\%$ using Claude 3.5 Sonnet. 
% Such frameworks operate in Linux-based containerized environments, such as Kali Linux, which includes pre-installed cybersecurity tools. However, excessive tool availability can overwhelm agents, leading to inefficiencies. Research indicates that agents perform better with a focused set of tools that have well-defined interfaces~\cite{yang2024sweagent}. % Swe-agent: Agent-computer interfaces enable automated software engineering



% Muzsai et al. introduced HackSynth~\cite{muzsai2024hacksynth}, an LLM-based agent for autonomous penetration testing. It uses a dual-module architecture that consists of a Planner and a Summarizer, allowing for iterative command generation and feedback processing. The framework is evaluated using two benchmark sets from platforms like PicoCTF~\cite{picoctf} and OverTheWire~\cite{overthewire}. These benchmarks address $200$ challenges drawn from various domains and difficulty levels. Results of their study show that HackSynth, especially with the GPT-4o model, achieves the best performance. This highlights the potential of LLM-based agents in advancing autonomous penetration testing.
 % Using basic prompting techniques and expanding tool availability, the study highlights how straightforward approaches can unlock the latent potential of LLMs for cybersecurity tasks. Their work emphasizes that simple LLM designs can effectively solve CTF challenges, and thus broaden the number of cybersecurity applications without the need for advanced engineering.

% \begin{table*}[]
%     \centering
%     \begin{tabular}{|c|c|>{\centering\arraybackslash}p{4.5cm}|c|c|c|c|c|c|}
%     \hline
%          \textbf{Study} & \textbf{Dynamic} & \textbf{Used} & \textbf{Multi-} & \textbf{Open} & \textbf{Automatic} & \textbf{Tool} & \textbf{\# of} & \textbf{\# of} \\
%          & \textbf{Prompt} & \textbf{Benchmarks} & \textbf{Agents} & \textbf{Dataset} & \textbf{Framework} & \textbf{Use} & \textbf{LLMs} & \textbf{CTFs}\\
%          \hline
%          Tann et al.~\cite{tann2023using} & \purplecross & Manual collected & \purplecross & \purplecross & \purplecross & \purplecross & $3$ & $7$ \\
%          \hline
%          InterCode-CTF~\cite{yang2023language} & \purplecross &  PicoCTF~\cite{picoctf} & \purplecross & \purplecross& \purplecross & \purplecross & $1$ & $100$  \\
%          \hline
%          Shao et al.~\cite{shao2024empirical} & \purplecross & CSAW 2023 & \purplecross & \purplecross & \tealcheck & \tealcheck & $4$ & $26$ \\
%          \hline
%          Shao et al.~\cite{shao2024nyu} & \purplecross & NYU CTF~\cite{shao2024nyu} & \purplecross & \tealcheck & \tealcheck & \tealcheck & $5$ & $200$ \\
%          \hline
%          Cybench~\cite{zhang2024cybenchframeworkevaluatingcybersecurity} & \purplecross & Cybench~\cite{zhang2024cybenchframeworkevaluatingcybersecurity}  & \purplecross & \tealcheck & \tealcheck & & $8$ & $40$ \\
%          \hline
%          EnIGMA~\cite{abramovich2024enigma} & \purplecross & NYU CTF~\cite{shao2024nyu}, InterCode-CTF~\cite{yang2023language},  HackTheBox~\cite{hackthebox} & \purplecross & \purplecross & \tealcheck & \tealcheck & $3$ & $350$ \\
%          \hline
%          HackSynth~\cite{muzsai2024hacksynth} & \purplecross & PicoCTF~\cite{picoctf}, OverTheWire~\cite{overthewire} & \tealcheck & \tealcheck & \tealcheck & \tealcheck & $8$ & $200$ \\
%          \hline
%          Turtayev et al.~\cite{turtayev2024hacking} & \purplecross & InterCode-CTF~\cite{yang2023language} & \purplecross & \purplecross & \purplecross & \purplecross & $4$ & $100$ \\
%          \hline
%          \textbf{D-CIPHER (Proposed)} & \tealcheck & NYU CTF~\cite{shao2024nyu}, Cybench \cite{zhang2024cybenchframeworkevaluatingcybersecurity}, HackTheBox \cite{hackthebox} & \tealcheck & \tealcheck & \tealcheck & \tealcheck & 5 & 290 \\
%          \hline
%     \end{tabular}
%     \caption{Comparison with LLM-based CTF solving Literature}
%     \label{tab:related_work_comparison}
% \end{table*}




% \subsection{Multi-agent framework}

% The use of multi-agent LLM systems in Capture the Flag (CTF) challenges is emerging as a powerful approach to enhance cybersecurity problem-solving. Multi-agent frameworks mimic team-based collaboration, where multiple LLM agents, each with specialized expertise, work together to tackle complex tasks. This approach reflects real-world cybersecurity operations, where success often depends on coordinated efforts from teams with diverse skills, each addressing different components of a security challenge.
% Multi-agent LLM systems are emerging as a powerful approach to enhance cybersecurity problem-solving by simulating team-based collaboration. Specialized agents, each with distinct objectives, work together to tackle different aspects of complex security tasks. This mirrors real-world cybersecurity operations, where coordinated efforts and diverse skills are essential for addressing evolving threats and vulnerabilities.

% CTF challenges cover a wide range of domains, including cryptography, reverse engineering, forensics, and web exploitation. Multi-agent systems can distribute the workload by assigning agents to handle specific tasks. This enables parallel problem-solving and emulates the collaborative nature of human teams. For example, one agent may specialize in guiding the fellow agents to what needs to be done, while another executes the instructions, ensuring that tasks are addressed without losing the context, and implementing reasoning from multiple LLMs. This division of labor boosts efficiency and enables problem-solving from multiple perspectives.
% This division of labor enhances efficiency and allows the system to approach problems from multiple perspectives, reflecting the interdisciplinary approach often used in cybersecurity teams.

% Guo et al.~\cite{guo2024largelanguagemodelbased} highlight the strengths of multi-agent LLMs in complex, multi-step tasks where different agents handle specific roles The framework HackSynth~\cite{muzsai2024hacksynth} is a multi-agent penetration testing framework in which agents operate collaboratively to address vulnerabilities in staged environments. Their work emphasizes that when agents work as a cohesive team, they outperform single-agent approaches. This is particularly true when facing layered, iterative challenges. 
% This team-based model of problem-solving aligns closely with how cybersecurity professionals approach real-world security incidents and penetration testing exercises.

% Multi-agent LLM systems have shown effectiveness in various other applications. For instance,  Audit-LLM~\cite{song2024audit} presents a multi-agent framework for insider threat detection using log analysis. It employs agents to decompose tasks, build tools, and use collaborative reasoning to enhance detection accuracy. Liu~\cite{liu2024multi} explores the application of LLM-based multi-agent systems to enhance incident response (IR) in cybersecurity. Utilizing the ``Backdoors \& Breaches" tabletop game as a simulation environment, the study examines centralized, decentralized, and hybrid team structures to assess how LLM agents can improve decision-making, adaptability, and coordination during cyberattack scenarios. AutoSafeCoder~\cite{nunez2024autosafecoder} is a multi-agent system designed to enhance the security of code generated by LLMs. The framework comprises three agents: a Coding Agent responsible for code generation, a Static Analyzer Agent that identifies vulnerabilities through static analysis, and a Fuzzing Agent that performs dynamic testing using mutation-based fuzzing to detect runtime errors. By integrating both static and dynamic testing in an iterative process, AutoSafeCoder aims to produce secure, functionally correct code. 

% To enhance CTF-solving by promoting team-based specialization, we employ a multi-agent CTF solving agent. Within this framework, agents tackle tasks aligned with their strengths. Tasks are executed in parallel, improving efficiency and accelerating progress. Agents share insights, adapt refining strategies based on feedback, and overcome obstacles collectively. This collaborative approach boosts scalability, adaptability, and and resilience, and improves performance in complex challenges.

% This paper presents a comprehensive comparison of D-CIPHER with existing LLM-based CTF-solving literature, as shown in Table~\ref{tab:related_work_comparison}.
% This paper documents the results of  our comprehensive comparison of D-CIPHER with existing LLM-based CTF-solving literature. These results are presented in Table~\ref{tab:related_work_comparison}.




\begin{figure}[b!]
    \centering
    \begin{subfigure}{0.8\textwidth}
        \includegraphics[width=\textwidth, trim=0 0 0 0, clip]{chapter-5/figs/pval-emp.pdf}
        \vspace{-0.6cm}
        \caption{Empirical \pval\ distribution without any correction.}
        \label{chap5/fig:pvalue}
    \end{subfigure}
    \\[0.5cm]
    \centering
    \begin{subfigure}{1.0\textwidth}
        \includegraphics[height=1.8in, trim=0.2cm 0 0.3cm 0, clip]{chapter-5/figs/fpr_zscore.pdf}
        \includegraphics[height=1.8in, trim=0.6cm 0 0.3cm 0, clip]{chapter-5/figs/fpr.pdf}
        \includegraphics[height=1.8in, trim=0.6cm 0 0.3cm 0, clip]{chapter-5/figs/fpr_rect.pdf}
        \caption{
            Empirical FPRs against theoretical ones.
            (\emph{Left}) using $Z$-tests;
            (\emph{Middle}) using new statistical tests presented in~\ref{chap5/sec:new-stats};
            (\emph{Right}) using the new statistical tests with the rectified scoring strategy of~\ref{chap5/sec:rect}.
        }\label{chap5/fig:fpr}
    \end{subfigure} \\
    \caption{
        Empirical checks of \pval\ and false positive rates for different watermarks and values of the context width $k$.
        Results are computed over $10$ secret keys $\times$ 100k sequences of $256$ tokens sampled from Wikipedia.
        Theoretical values do not hold in practice for $Z$-tests even for high values of $k$: \pval s are not uniformly distributed and empirical FPRs do not match theoretical ones.
        This is solved by basing detection on grounded statistical tests and analytic \pval, as well as by revising the scoring strategy with deduplication.
    }
\end{figure}


\section{Ensuring reliable \pval\ and FPR}\label{chap5/sec:stats}

In this section, large-scale evaluations of the FPR show a gap between theory and practice. 
It is closed with new statistical tests and by rectifying the scoring method.


\subsection{Empirical validation of FPR with Z-scores}\label{chap5/sec:zscore}

So far, the FPR has been checked on only around $500$ negative samples~\citep{kirchenbauer2023watermark,kirchenbauer2023reliability, zhao2023provable}.
We scale this further and select $100$k texts from multilingual Wikipedia to cover the distribution of natural text.
We tokenize with Llama's tokenizer, and take $T=256$ tokens/text.
We run detection tests with varying window length $k$ when seeding the RNG. 
We repeat this with $10$ different secret keys, which makes $1$M detection results under $\H_0$ for each method and $k$ value.
For the detection of the greenlist watermark, we use $\gamma=0.25$.

\autoref{chap5/fig:pvalue} first shows that the distribution of \pval s is not uniform.
his should be the case for a $Z$-test under $\H_0$, which calls into question whether this type of test can be used here.
To further emphasize this point, \autoref{chap5/fig:fpr} compares empirical and theoretical FPRs.
Theoretical guarantees do not hold in practice and empirical FPRs are much higher than the theoretical ones.
Besides, the larger the watermarking context window $k$, the closer we get to theoretical guarantees. 
In pratice, one would need $k>>8$ to get reliable \pval, but this makes the watermarking method less robust to attacks on generated text because it hurts synchronization.


\subsection{New non-asymptotical statistical tests}\label{chap5/sec:new-stats}

The Gaussian assumption of $Z$-tests breaks down for short or repetitive texts.
Here are non-asymptotical tests for both methods that reduce the gap between empirical and theoretical FPR, especially at low FPR values as shown in Fig.~\ref{chap5/fig:fpr}.

\paragraph*{\texorpdfstring{\citep{kirchenbauer2023watermark}}{}} 
Under $\H_0$, we assume that the event $x^{(t)}\in\G^{(t)}$ occurs with probability $\gamma$, and that these events are i.i.d.
Therefore, $S_T$~\eqref{chap5/eq:score-kirchenbauer} is distributed as a binomial of parameters $T$ and $\gamma$. Consider a text under scrutiny whose score equals $s$.
The \pval\ is defined as the probability of obtaining a score higher than $s$ under $\H_0$: %
\begin{equation}
    \text{p-value}(s) = \Prob(S_T \geq s \mid \H_0) = I_{\gamma}(s,T-s+1).
\end{equation}
This comes from the fact that $S\sim\mathcal{B}(T,\gamma)$ whose c.d.f. is expressed by $I_x(a,b)$ the regularized incomplete Beta function.

\paragraph*{\texorpdfstring{\citep{aaronson2023watermarking}}{}} 
Under $\H_0$, we assume that the text under scrutiny and the secret vector are independent, so that $\vec{r}_{x^{(t)}} \overset{i.i.d.}{\sim} \mathcal{U}(0,1)$. 
Therefore, $S_T$~\eqref{chap5/eq:score-aaronson} follows a $\Gamma(T,1)$ distribution.
The \pval\ associated to a score $s$ reads:
\begin{equation}
    \text{p-value}(s) = \Prob(S_T \geq s \mid \H_0) = \frac{\Gamma(T,s)}{\Gamma(T)},
\end{equation}
where $\Gamma$ is the upper incomplete gamma function.
Under $\H_1$, the score is expected to be higher as proven in App.~\ref{chap5/app:aaronson_score}, so the \pval\ is likely to be small.




    
    


\subsection{Rectifying the detection scores}\label{chap5/sec:rect}


\begin{figure}[b!]
    \centering
    \begin{tcolorbox}[width=0.7\linewidth, colframe=metablue, colback=white]
        \includegraphics[width=0.99\linewidth, trim=0 110 0 18, clip]{chapter-5/figs/low-pval-text.pdf}
    \end{tcolorbox}
    \centering
    \caption{
    Typical example of a non-watermarked text with low \pval\ because of repeated tokens (we only show half of the text).
    Here \pval$=10^{-21}$ on $256$ tokens.
    }
    \label{chap5/fig:low-pval-text}
\end{figure}

Even with grounded statistical tests, empirical FPRs are still higher than theoretical ones.
In fact, \citet{kirchenbauer2023watermark} mention that random variables are only pseudo-random since repeated windows generate the same secret. 
This can happen even in a short text and especially in formatted data.
For instance in a bullet list, the sequence of tokens \texttt{$\backslash$n$\backslash$n*\_} repeats a lot as shown in Fig.~\ref{chap5/fig:low-pval-text}.
In this text of $256$ tokens, the \pval\ is $10^{-21}$ for the scheme of~\citet{kirchenbauer2023watermark} and with $\gamma=0.25$ and $k=2$.
Repetition pulls down the assumption of independence necessary for computing the \pval.

We experiment with two simple heuristics mitigating this issue.
The first one takes into account a token only if the watermark context window has not already been seen during the detection.
The second scores the tokens for which the $k+1$-tuple formed by \{watermark context + current token\} has not already been seen.
Note, that the latter is present in~\citep{kirchenbauer2023watermark}, although without ablation and without being used in further experiments.
Of the two, the second one is better since it counts more ngrams, and thus has better TPR. 
It can also deal with the specific case of $k=0$.
\autoref{chap5/fig:fpr} reports empirical and theoretical FPRs when choosing not to score already seen $k+1$-tuples.
They now match perfectly, except for $k=0$ where the FPR is still slightly underestimated.
\emph{In short, we guarantee FPR thanks to new statistical tests and by scoring only tokens for which \{watermark context + current token\} has not been scored.}


\section{Main experiments}
\label{chap4/sec:exps}



\subsection{Audio/speech quality}
\label{chap4/sec:quality}

We first evaluate the quality of the watermarked audio using:
Scale Invariant Signal to Noise Ratio (SI-SNR): 
$\textrm{SI-SNR}(s, s_w) = 10 \log_{10} \left( \| \alpha s \|_2^2 / \| \alpha s - s_w \|_2^2 \right)$,
with $s$ the clean audio and $s_w$ the watermarked one,
where $\alpha = \langle s, s_w \rangle / \| s \|_2^2$;
as well as Perceptual Evaluation of Speech Quality (PESQ)~\citep{rix2001perceptual}, 
Virtual Speech Quality Objective Listener (ViSQOL)~\citep{hines2012visqol} and
Short Term Objective Intelligibility (STOI)~\citep{taal2010short} which are objective perceptual metrics measuring the quality of speech signals.

\autoref{chap4/tab:audio_quality} report these metrics.
AudioSeal behaves differently than watermarking methods like WavMark~\citep{chen2023wavmark} that try to minimize the SI-SNR.
In practice, high SI-SNR is indeed not necessarily correlated with good perceptual quality.
AudioSeal is not optimized for SI-SNR but rather for perceptual quality of speech (similarly as DeAR~\citep{DEAR_Liu0FMZY23} which reports 25.96 as SI-SNR).
This is better captured by the other metrics (PESQ, STOI, ViSQOL), where AudioSeal consistently achieves better performance.
Put differently, our goal is to hide as much watermark power as possible while keeping it perceptually indistinguishable from the original.
\autoref{chap4/fig:loc_quali} also visualizes how the watermark signal follows the shape of the speech waveform.

The metric used for our subjective evaluations is the MUSHRA score~\citep{mushra}. 
It is a crowdsourced test in which participants rate the quality of various samples on a scale of 0 to 100. 
The ground truth is provided for reference. 
We utilized 100 speech samples, each lasting 10 seconds. 
Each sample was evaluated by at least 20 participants.
As part of the study, we included a low anchor, which is a very lossy compression at 1.5kbps, encoded using EnCodec. 
Participants who failed to assign the lowest score to the low anchor for at least 80\% of their assignments were excluded from the study.
In this study our samples got superior ratings than WavMark, with an average score of 77.07, 5 points higher than WavMark.
For comparison, the ground truth samples received an average score of 80.49, while the low anchor's average score was 53.21.


\begin{table}[t!]
    \centering
    \caption{
        Audio quality metrics. 
        Compared to traditional watermarking methods that minimize the SNR like WavMark, AudioSeal achieves same or better perceptual quality.
    }\label{chap4/tab:audio_quality}
    \footnotesize
        \begin{tabular}{lccccc}
            \toprule
            \textbf{Methods} & \textbf{SI-SNR} & \textbf{PESQ} & \textbf{STOI} & \textbf{ViSQOL} & \textbf{MUSHRA}  \\
            \midrule
            WavMark & \textbf{38.25} & 4.302 & 0.997 & 4.730 & 71.52 $\pm$ 7.18\\
            AudioSeal & 26.00 & \textbf{4.470} & 0.997 & \textbf{4.829} &  \textbf{77.07} $\pm$ 6.35 \\
            \bottomrule
        \end{tabular}
\end{table}















\subsection{Comparison with passive classifier}\label{chap4/sec:active-passive}


\begin{table}[t!]
    \centering
    \caption{
        Comparison with Voicebox binary classifier. 
        Percentage refers to the fraction of masked input frames.
        Detection with AudioSeal is perfect for all samples, while Voicebox classifier fails on re-synthesized audio.
    }
    \label{chap4/tab:voicebox}
    \footnotesize
        \begin{tabular}{r *{3}{c}  *{3}{c}  *{3}{c} }
            \toprule
            & \multicolumn{3}{c}{\textbf{AudioSeal (Ours)}} & \multicolumn{3}{c}{\textbf{Voicebox Classif.}} \\
            \cmidrule(rr){2-4} \cmidrule(rr){5-7}
            \textbf{\% Mask} & Acc. & TPR & FPR & Acc. & TPR & FPR \\
            \midrule
            \multicolumn{7}{l}{\emph{Original audio vs AI-generated audio}} \\
            30\% & 1.0 & 1.0 & 0.0 & 1.0 & 1.0 & 0.0 \\
            50\% & 1.0 & 1.0 & 0.0 & 1.0 & 1.0 & 0.0 \\
            90\% & 1.0 & 1.0 & 0.0 & 1.0 & 1.0 & 0.0 \\
            \midrule
            \multicolumn{7}{l}{\emph{Re-synthesized audio vs AI-generated audio}} \\
            30\% & \textbf{1.0} & \textbf{1.0} & \textbf{0.0} & 0.704 & 0.680 & 0.194 \\
            50\% & \textbf{1.0} & \textbf{1.0} & \textbf{0.0} & 0.809 & 0.831 & 0.170 \\
            90\% & \textbf{1.0} & \textbf{1.0} & \textbf{0.0} & 0.907 & 0.942 & 0.112 \\
            \bottomrule
        \end{tabular}
\end{table}



We first compare detection results on samples generated with Voicebox~\citep{le2023voicebox}.
We compare to the passive setup where a classifier is trained to discriminate between Voicebox-generated and real audios (this is done with a detector that shares the same architecture as AudioSeal's, trained until reaching 100\% accuracy on Voicebox samples).
Following the approach in the Voicebox study, we evaluate 2,000 approximately 5-second samples from LibriSpeech, these samples have masked frames (90\%, 50\%, and 30\% of the phonemes) pre-Voicebox generation.
We evaluate on the same tasks, \ie, distinguishing between original and generated, or between original and re-synthesized (created by extracting the Mel spectrogram from original audio and then vocoding it with the HiFi-GAN vocoder).

We use the True Positive Rate (\Gls*{TPR}) and the False Positive Rate (\Gls*{FPR}) as key metrics.
TPR measures correct identification of AI-generated samples, while FPR indicates the rate of genuine audio clips falsely flagged.
In practical scenarios, minimizing FPR is crucial. 
For example, on a platform processing 1 billion samples daily, an FPR of $10^{-3}$ and a TPR of $0.5$ means that 1 million samples require manual review each day, yet only half of the watermarked samples are detected.
The \Gls*{ROC} AUC (Area Under the Curve of the Receiver Operating Characteristics) gives a global measure of performance over all threshold levels, and captures the TPR/FPR trade-off.

Both active and passive setups achieve perfect classification in the case when trained to distinguish between natural and Voicebox.
Conversely, the second part of Tab.~\ref{chap4/tab:voicebox} highlights a significant drop in performance when the classifier is trained to differentiate between Voicebox and re-synthesized.
It suggests that the classifier is detecting vocoder artifacts, since the re-synthesized samples are sometimes wrongly flagged.
The classification performance quickly decreases as the quality of the AI-generated sample increases (when the input is less masked).
On the other hand, our proactive detection does not rely on model-specific artifacts but on the watermark presence. %
This allows for perfect detection over all the audio clips. %




\subsection{Robustness and comparison with watermarking}

\begin{table}[t!]
    \centering
    \caption{
        Detection results for different edits applied before detection. 
        Acc. ({\aux{TPR/FPR}}) is the accuracy (and TPR/FPR) obtained for the threshold that gives best accuracy on a balanced set of augmented samples.
        AUC is the area under the ROC curve.
    }
    \label{chap4/tab:wm_robustness}
    \footnotesize
        \begin{tabular}{l *{2}{l}  *{2}{l}}
        \toprule
        & \multicolumn{2}{l}{\textbf{AudioSeal (Ours)}} & \multicolumn{2}{l}{\textbf{WavMark}} \\
        \cmidrule(rr){2-3} \cmidrule(rr){4-5}
        \multicolumn{1}{c}{Edit} & Acc. \aux{TPR/FPR} & AUC & Acc. \aux{TPR/FPR}  & AUC \\
        \cmidrule(rr){1-1} \cmidrule(rr){2-3} \cmidrule(rr){4-5}
        None & 1.00 \aux{1.00/0.00} & 1.00 & 1.00 \aux{1.00/0.00} & 1.00 \\
        Bandpass & 1.00 \aux{1.00/0.00} & 1.00 & 1.00 \aux{1.00/0.00} & 1.00 \\
        Highpass  &  0.61 \aux{0.82/0.60} & 0.61 & \bf 1.00 \aux{1.00/0.00} & \bf 1.00 \\
        Lowpass & \bf 0.99 \aux{0.99/0.00} & \bf 0.99 & 0.50 \aux{1.00/1.00} & 0.50 \\
        Boost & 1.00 \aux{1.00/0.00} & 1.00 & 1.00 \aux{1.00/0.00} & 1.00 \\
        Duck & 1.00 \aux{1.00/0.00} &  1.00 & 1.00 \aux{1.00/0.00} & 1.00 \\
        Echo & \bf 1.00 \aux{1.00/0.00} & \bf 1.00 & 0.93 \aux{0.89/0.03} & 0.98 \\
        Pink & \bf 1.00 \aux{1.00/0.00} & \bf 1.00 & 0.88 \aux{0.81/0.05} & 0.93 \\
        White & \bf 0.91 \aux{0.86/0.04} & \bf 0.95 & 0.50 \aux{0.54/0.54} & 0.50 \\
        Fast (1.25x) & \bf 0.99 \aux{0.99/0.00} & \bf 1.00 & 0.50 \aux{0.01/0.00} & 0.15 \\
        Smooth & \bf 0.99 \aux{0.99/0.00} &  1.00   & 0.94 \aux{0.93/0.04} & 0.98 \\
        Resample & 1.00 \aux{1.00/0.00} &  1.00 & 1.00 \aux{1.00/0.00} & 1.00 \\
        AAC & 1.00 \aux{1.00/0.00} &  1.00 & 1.00 \aux{1.00/0.00} & 1.00 \\
        MP3 & \bf 1.00 \aux{1.00/0.00} & \bf 1.00 & 1.00 \aux{0.99/0.00} & 0.99 \\
        EnCodec & \bf  0.98 \aux{0.98/0.01} & \bf 1.00 & 0.51 \aux{0.52/0.50} & 0.50 \\
        \midrule
        Average & \bf 0.96 \aux{0.98/0.04} & \bf 0.97 & 0.85 \aux{0.85/0.14} & 0.84 \\
        \bottomrule
        \end{tabular}
\end{table}

\paragraph*{Audio editing.}
We evaluate the robustness of the detection on a wide range of audio editing operations: 
time modification (faster, resample), 
filtering (bandpass, highpass, lowpass), 
audio effects (echo, boost audio, duck audio), 
noise (pink noise, random noise),
and compression (MP3, AAC, EnCodec).
In order to show generalization, we chose stronger parameter to the attacks than those used during training (see Sec.~\ref{chap4/sec:training-details}).

\paragraph*{Robustness of the detection.}
Detection is done on 10k ten-seconds audios from our VoxPopuli validation set.
For each edit, we first build a balanced dataset made of the 10k watermarked/ 10k non-watermarked edited audio clips.
We quantify the performance by adjusting the threshold of the detection score, selecting the value that maximizes accuracy (we provide corresponding TPR and FPR at this threshold).
To adapt data-hiding methods (\eg, WavMark) for proactive detection, we embed a binary message (chosen randomly beforehand) in the generated speech before release. The detection score is then computed as the Hamming distance between the original message and the one extracted from the scrutinized audio. 

We observe in Tab.~\ref{chap4/tab:wm_robustness} that AudioSeal is overall more robust, with an average AUC of 0.97 vs. 0.84 for WavMark.
The performance for lowpass and highpass filters indicates that AudioSeal embeds watermarks neither in the low nor in the high frequencies (WavMark focuses on high frequencies).
We give results on more augmentations in Sec.~\ref{chap4/app:robustness}.



\subsection{Localization}


\begin{figure}[b!]
    \centering
    \includegraphics[width=0.48\linewidth, clip, trim={0 1.8in 0 0}, valign=t]{chapter-4/figs/loc_quantitative.pdf}\hfill
    \includegraphics[width=0.48\linewidth, clip, trim={0 0 0 1.3in}, valign=t]{chapter-4/figs/loc_quantitative.pdf}
    \caption{\textbf{Localization results} across different durations of watermarked audio signals in terms of Sample-Level Accuracy and Intersection Over Union (IoU) metrics ($\uparrow$ is better).}
    \label{chap4/fig:loc_quantitative}
\end{figure}


We evaluate localization with the sample-level detection accuracy, \ie, the proportion of correctly labeled samples, and the Intersection over Union (IoU).
The latter is defined as the intersection between the predicted and the ground truth detection masks (1 when watermarked, 0 otherwise), divided by their union.
IoU is a more relevant evaluation of the localization of short watermarks in a longer audio.

This evaluation is carried out on the same audio clips as for detection.
For each one of them, we watermark a randomly placed segment of varying length.
Localization with WavMark is a brute-force detection: a window of 1s slides over the 10s of speech with the default shift value of 0.05s.
The Hammning distance between the 16 pattern bits is used as the detection score.
Whenever a window triggers a positive, we label its 16k samples as watermarked in the detection mask in $\{0,1\}^t$.

\autoref{chap4/fig:loc_quantitative} plots the sample-level accuracy and IoU for different proportions of watermarked speech in the audio clip.
AudioSeal achieves an IoU of 0.99 when just one second of speech is AI-manipulated, compared to WavMark's 0.35.
Moreover, AudioSeal allows for precise detection of minor audio alterations: it can pinpoint AI-generated segments in audio down to the sample level (usually 1/16k sec), while the concurrent WavMark only provides one-second resolution.
This is the reason why it lags behind in terms of IoU more than accuracy.
It is especially relevant for speech samples, where a simple word modification may greatly change meaning. 





\subsection{Attribution}

\begin{table}[t!]
    \centering
    \caption{
        Attribution results.
        We report the accuracy of the attribution (Acc.) and false attribution rate (FAR). 
        Detection is done at FPR=$10^{-3}$ and attribution matches the decoded message to one of $N$ versions.
        We report averaged results over the edits of Tab.~\ref{chap4/tab:wm_robustness}.
    }\label{chap4/tab:attribution}
    \footnotesize
        \begin{tabular}{cr *{5}{c}}
            \toprule
            & N & $1$ & $10$ & $10^2$ & $10^3$ & $10^4$ \\ \midrule
    \multirow{2}{*}{FAR (\%) $\downarrow$} & WavMark      & 0.0 & \textbf{0.20} & \textbf{0.98} & \textbf{1.87} & \textbf{4.02} \\
            & AudioSeal   & 0.0 & 2.52 & 6.83 & 8.96 & 11.84 \\ \midrule
    \multirow{2}{*}{\shortstack{Acc. (\%) $\uparrow$}} & WavMark      & 58.4 & 58.2 & 57.4 & 56.6 & 54.4 \\
            & AudioSeal  & \textbf{68.2} & \textbf{65.4} & \textbf{61.4} & \textbf{59.3} & \textbf{56.4} \\ 
            \bottomrule
        \end{tabular}
\end{table}

Given an audio clip, the objective is now to find if any of $N$ versions of our model generated it (detection), and if so, which one (identification). 
For evaluation, we create $N'=100$ random 16-bits messages and use them to watermark 1k audio clips, each consisting of 5 seconds of speech (not 10s to reduce compute needs). 
This results in a total of 100k audios. 
For WavMark, the first 16 bits (/32) are fixed and the detection score is the number of well decoded pattern bits, while the second half of the payload hides the model version.
An audio clip is flagged if the average output of the detector exceeds a threshold, corresponding to FPR=$10^{-3}$.
Next, we calculate the Hamming distance between the decoded watermark and all $N$ original messages. 
The message with the smallest Hamming distance is selected.
It is worth noting that we can simulate $N>N'$ models by adding extra messages. 
This may represent versions that have not generated any sample.

False Attribution Rate (FAR) is the fraction of wrong attribution \emph{among the detected audios} while the attribution accuracy is the proportion of detections followed by a correct attributions \emph{over all audios}. 
AudioSeal has a higher FAR but overall gives a better accuracy, which is what ultimately matters.
First, we observe that the false attribution rate -- which we define as the proportion of audios that are wrongly attributed among the detected ones -- is higher in our case.
On the other hand \autoref{chap4/tab:attribution} highlights that AudioSeal gives better attribution accuracy.
It is defined as the proportion of watermarked samples that are both flagged and correctly attributed and is what ultimately matters.
In summary, decoupling detection and attribution achieves better detection rate and makes the global accuracy better, at the cost of occasional false attributions.



\subsection{Efficiency analysis}
\label{chap4/sec:speed}


\begin{figure}[b!]
    \centering
    \includegraphics[width=0.65\linewidth, clip, trim={0 0 0 0}]{chapter-4/figs/speed.pdf}
    \caption{Mean runtime ($\downarrow$ is better) of AudioSeal versus WavMark. 
    AudioSeal is one order of magnitude faster for watermark generation and two orders of magnitude faster for watermark detection for the same audio input.
    }
    \label{chap4/fig:efficiency}
\end{figure}



To highlight the efficiency of AudioSeal, we conduct a performance analysis and compare it with WavMark. 
We apply the watermark generator and detector of both models on a dataset of 500 audio segments ranging in length from 1 to 10 seconds, using a single Nvidia Quadro GP100 GPU. 
The results are displayed in Fig.~\ref{chap4/fig:efficiency} and Tab.~\ref{chap4/tab:speed}.
In terms of generation, AudioSeal is 14$\times$ faster than WavMark. 
For detection, AudioSeal outperforms WavMark with two orders of magnitude faster performance on average, notably 485$\times$ faster in scenarios where there is no watermark (Tab.~\ref{chap4/tab:speed}). 
The speed difference in the case of WavMark between watermarked and non-watermarked audios is explained by the fact that whenever the detector flags a 1-second span as watermarked, it will directly skip to the next 1-second span, while for non-watermarked audios, it will slide the window by 0.05s.
AudioSeal's speed is due to the model's localized watermark design, which bypasses the need for watermark synchronization (recall that WavMark relies on 20 pass forwards for a one-second snippet).
AudioSeal's detector provides detection logits for each input sample directly with only one pass to the detector, significantly enhancing the detection's computational efficiency.
This makes our system highly suitable for real-time and large-scale applications.

\begin{table*}[t!]
    \centering
    \caption{
        Average runtime (ms) per sample of AudioSeal model against WavMark~\citep{chen2023wavmark} method. 
        Our experiments were conducted on a dataset of audio segments spanning 1 second to 10 seconds, using a single Nvidia Quadro GP100 GPU. 
        The results demonstrate significant speed improvements for both watermark generation and detection with and without the presence of a watermark. 
        Notably, for watermark detection, AudioSeal is 485$\times$ faster than WavMark when there is no watermark, because the latter relies on more forward passes when trying to synchronize the watermark. 
    }
    \footnotesize
    \label{chap4/tab:speed}
    \begin{tabular}{llll}
    \toprule
               Model & Watermarked &     \textbf{Detection ms (speedup)} &   \textbf{Generation ms (speedup)} \\
    \midrule
             WavMark &          \multirow{2}{*}{\xmarkg}      & 1710.70 $\pm$ 1314.02 &    -- \\
             AudioSeal (ours) &           &       \textbf{3.25 $\pm$ 1.99} \;\; (\textbf{485$\times$}) &    -- \\
    \midrule
             WavMark &         \multirow{2}{*}{\cmarkg} &    106.21 $\pm$ 66.95 & 104.58 $\pm$ 65.66 \\
    AudioSeal (ours) &          &       \textbf{3.30} $\pm$ \textbf{2.03} \;\; (\textbf{35$\times$}) &    \textbf{7.41} $\pm$ \textbf{4.52} \;\; (\textbf{14} $\times$) \\
    
    \bottomrule 
    \end{tabular}
\end{table*}

























\section{Advanced detection schemes}
This section introduces improvements to the detection schemes of Sec.~\ref{chap5/sec:stats}.
Namely, it develops a statistical test when access to the LLM is granted, as well as multi-bit decoding.


\subsection{Neyman-Pearson and simplified score function} 
The following is specific for the scheme of~\citet{aaronson2023watermarking} -- a similar work may be conducted with the one of~\citet{kirchenbauer2023reliability}.
Under $\H_0$, we have $\vec{r}_v\sim\mathcal{U}_{[0,1]}$, whereas $\vec{r}_v\sim Beta(1/p_v,1)$ under $\H_1$ (see Corollary~\eqref{chap5/eq:Coro} in App.~\ref{chap5/app:aaronson_prob}). 
The optimal Neyman-Pearson score function is thus:
\begin{equation*}
    s_T = \sum_{t=1}^{T} \ln\frac{f_{\H_1}(\vec{r}_{x^{(t)}})}{f_{\H_0}(\vec{r}_{x^{(t)}})} = \sum_{t=1}^T \left(\frac{1}{\vec{p}_{x^{(t)}}}-1\right)\ln(\vec{r}_{x^{(t)}})+A
\end{equation*}
where $A$ is a constant that does not depend on $\vec{r}$ and can thus be discarded. 
There are two drawbacks: (1) detection needs the LLM to compute $\vec{p}_{x^{(t)}}$, (2) there is no close-form formula for the p-value.  

This last point may be fixed by resorting to a Chernoff bound, yet without guarantee on its tightness:
$\text{p-value}(s) \leq e^{\sum_t \ln\frac{\lambda_t}{\lambda_t + c} -cs}$,
with $c$ solution of $\sum_t (c+\lambda_t)^{-1}=-s$ and $\lambda_t = p_{x^{(t)}} / (1-p_{x^{(t)}})$.
Experiments show that this detection yields extremely low \pval\ for watermarked text, but they are fragile: any attack increases them to the level of the original detection scheme~\eqref{chap5/eq:score-aaronson}, or even higher because generated logits are sensitive to the overall LLM context. 
An alternative is to remove weighting:
\begin{equation}
 s_T = \sum_{t=1}^T \ln\left(\vec{r}_{x^{(t)}}\right),
 \label{chap5/eq:Detection2}
\end{equation}
whose \pval\ is given by: $\text{p-value}(s) = \frac{\gamma(T,-s)}{\Gamma(T)}$.
In our experiments, this score function does not match the original detection presented in~\citep{aaronson2023watermarking}.


\subsection{Multi-bit watermarking}

       
    


\paragraph*{Theory.} It is rather easy to turn a zero-bit watermarking scheme into multi-bit watermarking, by associating a secret key per message. 
The decoding runs detection with every key and the decoded message is the one associated to the key giving the lowest \pval\ $p$. 
The global \pval\ becomes $1-(1-p)^M$, where $M$ is the number of possible messages.

Running detection for $M$ keys is costly, since it requires $M$ generations of the secret vector.
This is solved by imposing that the secret vectors of the messages $m\in\{0,\ldots,M-1\}$ are crafted as circular shifts of $m$ indices of $\vec{r}=\vec{r}(0)$:
\begin{align*}
\vec{r}(m) &= \mathsf{CyclicShift}(\vec{r},m) \\
    &= \left( \vec{r}_m, \vec{r}_{m+1}, ..,\vec{r}_{d}, \vec{r}_{0}, ..,  \vec{r}_{m-1}  \right).
\end{align*}
Generating $\vec{r}$ as a $d$-dimensional vector, with $d\geq|\V|$, we are able to embed $M\leq d$ different messages, by keeping only the first $|\V|$ dimensions of each circularly-shifted vector. 
Thus, the number of messages may exceed the size of the token vocabulary $|\V|$.
One way is to choose $d =  \textrm{max}(M, |\V|)$.



Besides, the scoring functions~\eqref{chap5/eq:score-kirchenbauer}~\eqref{chap5/eq:score-aaronson}
may be rewritten as:
\begin{equation}
s_T(m) = \sum_{t=1}^T f\left(\vec{r}^{(t)}(m)\right)_{x^{(t)}}  ,
\end{equation}
where $f: \R^d \mapsto \R^d$ is a component-wise function, and $x^{(t)}$ is the selected token during detection. 
This represents the selection of $f\left(\vec{r}^{(t)}(m)\right)$ at position $x^{(t)}$.
From another point of view, if we shift $f\left(\vec{r}^{(t)}\right)$ by $x^{(t)}$, the score for $m=0$ would be its first component, $m=1$ its second one, etc.
We may also write:
\begin{equation}
\vec{S}_T = \sum_{t=1}^T \mathsf{CyclicShift}\left( f\left(\vec{r}^{(t)}\right), x^{(t)} \right) ,
\label{chap5/eq:DetectionMultibit}
\end{equation}
and the first $M$ components of $\vec{S}_T$ are the scores for each $m$.
As a side note, this is a particular case of the parallel computations introduced by~\citet{JAWS}.
The simplified algorithms are given in Alg.~\ref{chap5/alg:multi-bit-gen} and Alg.~\ref{chap5/alg:multi-bit-dec}. 

\noindent
\begin{minipage}{0.48\textwidth}
    \begin{algorithm}[H] \small
    \caption{Generation (one step)}
    \label{chap5/alg:multi-bit-gen}
    \begin{algorithmic}
        \State\hspace*{-0.3cm} \textbf{Requires}: {LLM, dimension $d$, watermark window $k$, message $m\in\{0,\ldots,M-1\}$} \\
        \State logits $\vec{\boldsymbol\ell} \gets \text{LLM} \left( x^{(-C)},\dots, x^{(-1)} \right)$
        \State seed $\gets \mathsf{Hash}(x^{(-k)},\dots, x^{(-1)})$
        \State $\vec{r} \gets \mathsf{RNG_{seed}}(d)$
        \State $\vec{r}(m) \gets \mathsf{CyclicShift}(\vec{r},m)$
        \State $x^{(0)} \gets \mathsf{Sample}(\vec{\boldsymbol\ell},\vec{r}(m)_{1,\dots,|\V|})$
    \end{algorithmic}
\end{algorithm}
\end{minipage}\hfill
\begin{minipage}{0.48\textwidth}
    \begin{algorithm}[H] \small
    \caption{Decoding/identification}
    \label{chap5/alg:multi-bit-dec}
    \begin{algorithmic}
        \State $\vec{S} \gets \vec{0}_d$
        \State{\textbf{for} $t \in \{ k+1, \dots, T\}$:}
            \State \quad seed $\gets \mathsf{Hash}(x^{(t-k)},\dots, x^{(t-1)})$
            \State \quad $\vec{r}^{(t)} \gets \mathsf{RNG_{seed}}(d)$
            \State \quad $\vec{S} \gets \vec{S} +  \mathsf{CyclicShift}(f(\vec{r}^{(t)}),x^{(t)})$
        \State $\vec{p} \gets \textrm{p-value}(\vec{S}_{1,\dots,M})$
        \State $m \gets \textrm{argmin}({\vec{p}}) $
        \State $p \gets 1 - (1 - \vec{p}_m)^M$
    \end{algorithmic}
    \end{algorithm}
\end{minipage}


\paragraph*{Experiments.} 
In a tracing scenario the message is the identifier of a user or a version of the model.
The goal is to decide if any user or model generated a given text (detection) and if so, which one (identification).
There are 3 types of error: \emph{false positive}: flag a vanilla text; \emph{false negative}: miss a watermarked text; \emph{false accusation}: flag a watermarked text but select the wrong identifier.


\begin{table}[t]
    \caption{Identification accuracy for tracing users by watermarking. 
    Sequences are between $4$ and $252$ tokens long, and $149$ on average.
    }
    \label{chap5/tab:identification}
    \centering
    \footnotesize
    \begin{tabular}{cl cccc}
        \toprule
        & & \multicolumn{4}{c}{Number of users $M$} \\
        \cmidrule{3-6}
        & Method & $10$ & $10^2$ & $10^3$ & $10^4$ \\ \midrule
        \multirow{2}{*}{FPR$=10^{-3}$} & [AK] \citep{aaronson2023watermarking}      & 0.80 & 0.72 & 0.67 & 0.62 \\
        & [KGW] \citep{kirchenbauer2023watermark}  & 0.84 & 0.77 & 0.73 & 0.68 \\ \midrule
        \multirow{2}{*}{FPR$=10^{-6}$} & [AK] \citep{aaronson2023watermarking}	      & 0.61 & 0.56 & 0.51 & 0.46 \\
        & [KGW] \citep{kirchenbauer2023watermark} 	                              & 0.69 & 0.64 & 0.59 & 0.55 \\
        \bottomrule
    \end{tabular}
\end{table}


We simulate $M'$=$1000$ users that generate $100$ watermarked texts each, using the Guanaco-7B model. 
Accuracy can then be extrapolated beyond the $M'$ identifiers by adding identifiers with no associated text, for a total of $M>M'$ users.
Text generation uses nucleus sampling with top-p at $0.95$.
For~\citep{kirchenbauer2023watermark}, we use $\delta=3.0$, $\gamma=1/4$ with temperature $\theta$ at $0.8$.
For~\citep{aaronson2023watermarking}, we use $\theta = 1.0$.
For both, the context width is $k=4$.
A text is deemed watermarked if the score is above a threshold set for a given \emph{global} FPR (see~\ref{chap5/sec:stats}).
Then, the source is identified as the user with the lowest p-value.

\autoref{chap5/tab:identification} shows that watermarking performance for identification is dissuasive enough. 
For example, among $10^5$ users, we successfully identify the source of a watermarked text 50\% of the time while maintaining an FPR of $10^{-6}$ (as long as the text is not attacked).
At this scale, the false accusation rate is zero (no wrong identification once we flag a generated text) because the threshold is set high to avoid FPs, making false accusations unlikely. 
The identification accuracy decreases when $M$ increases, because the threshold required to avoid FPs gets higher.
In a nutshell, by giving the possibility to encode several messages, we trade some accuracy of detection against the ability to identify users.

\section{Conclusion}
\label{sec:Conclusion}
This work evaluates proprietary and open-weight models in agentic frameworks for handling ambiguity in software engineering. In code generation, to effectively integrate new information into the solution, an agent must detect ambiguity and ask targeted questions. Our key findings are:
\begin{itemize}[itemsep=0pt, topsep=0pt]
    \item Given an underspecified input, Claude Sonnet 3.5 and Claude Haiku 3.5 with interaction can achieve 80\% of their performance with a well-specified input. In contrast, open-weight models struggle: Deepseek relies on navigational cues to locate relevant files, while Llama 3.1 70B extracts limited information from the user.
    \item LLMs do not interact unless explicitly prompted, and their ambiguity detection is highly sensitive to prompt variations. Only Claude Sonnet 3.5 achieves a higher accuracy of 84\% in distinguishing between well-specified and underspecified input.

    \item Claude Sonnet 3.5, Haiku 3.5, and Deepseek effectively extract new, detailed user information, whereas Llama 3.1 struggles to ask the right questions.
    
\end{itemize}
Despite these advances, a gap remains between resolve rates for underspecified vs. fully specified issues. Open-weight models need better interaction strategies to improve resolution, while proprietary models, particularly Claude Haiku 3.5, require stronger prompting to engage interactively. This work establishes the current state-of-the-art in handling ambiguity through interaction, breaking the resolution process into multiple steps.



