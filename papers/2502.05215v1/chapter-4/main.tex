
\chapter{Proactive Detection of Voice Cloning with Localized Watermarking}\label{chapter:audioseal}

This chapter is based on the paper \fullcite{san2024proactive}.

In the rapidly evolving field of speech generative models, there is a pressing need to ensure audio authenticity against the risks of voice cloning. 
This chapter presents AudioSeal, an audio watermarking technique designed specifically for localized detection of AI-generated speech. 
It is the counterpart to the image watermarking method presented in Chap.~\ref{chapter:stable-signature} in two respects: it is a post-hoc watermarking method (as opposed to watermarking during generation), and it is designed for audio signals (which present different challenges).
AudioSeal employs a generator / detector architecture trained jointly with a localization loss to enable localized watermark detection up to the sample level, and a novel perceptual loss inspired by auditory masking, that enables better imperceptibility. 
AudioSeal achieves state-of-the-art performance in terms of robustness to real life audio manipulations and imperceptibility based on automatic and human evaluation metrics. 
Additionally, AudioSeal is designed with a fast, single-pass detector, that significantly surpasses existing models in speed, achieving detection up to two orders of magnitude faster, making it ideal for large-scale and real-time applications.
Code is available at \url{github.com/facebookresearch/audioseal}.

\newpage
\section{Introduction}
\label{sec:intro}
Treatment non-adherence is a pervasive and persistent challenge in healthcare. Researchers estimate that poor medication adherence leads to 125,000 preventable deaths annually in the U.S. and contributes to \$100-\$300 billion in avoidable healthcare costs \citep{benjamin2012medication}. This issue is particularly prevalent among patients with chronic conditions such as hypertension, with 40-50\% failing to take their medications as prescribed \citep{kleinsinger2018unmet, algabbani2020treatment}. While researchers have extensively documented this problem through surveys and interviews \citep{boratas2018evaluation, fernandez2019adherence, algabbani2020treatment, najjuma2020adherence, schober2021high}, the studies---and ultimately understanding of treatment non-adherence---remain limited by small sample sizes and self-reporting bias \citep{adams1999evidence, stirratt2015self}. Physical solutions to monitor and encourage adherence such as electronic pill caps have shown promise in controlled settings but remain impractical for large-scale deployment due to high costs and implementation challenges~\citep{parker2007adherence,mauro2019effect}.
 
These measurement challenges take on new urgency as healthcare systems increasingly rely on machine learning (ML) models trained on electronic health records (EHRs) to guide treatment decisions~\citep{komorowski2018artificial, brugnara2020multimodal, zheng2021personalized, mroz2024predicting, yi2024development,shen2024data,chen2022clustering}. These machine learning models learn from historical patient data, which assume that prescribed treatments were actually taken. However, this introduces an implicit bias---models trained on non-adherent patients learn patterns that misrepresent true treatment effects. This implicit bias may degrade model performance and disproportionately impact underserved populations, who often face greater barriers to treatment adherence~\citep{bosworth2006racial, schober2021high}.

Recent advances in large language models (LLMs) have shown that LLMs can advance medical understanding by accurately extracting information from EHRs \citep{agrawal2022large, goel2023llmsaccelerateannotationmedical}. Instead of relying on self-reported treatment adherence from questionnaires and interviews, LLMs could serve as a powerful tool for identifying treatment non-adherence directly from EHRs. By analyzing rich but unstructured clinical notes, LLMs can detect documented instances of missed medications, unfilled prescriptions, and patient-reported barriers to adherence, enabling systematic assessment of treatment non-adherence across large patient populations.

In this study, we examine hypertension treatment non-adherence using EHR data from a large academic hospital by leveraging an LLM to analyze clinical notes, and further investigate its impact on causal inference and ML model performance (Figure~\ref{fig:diagram}). With a cohort of 3,623 patients, we identify 786 (21.7\%) cases of non-adherence and extract demographic and clinical factors that are statistically significant. Additionally, we apply topic modeling to clinical notes revealing underlying reasons for non-adherence.

To assess the effect of treatment non-adherence bias on downstream model performance, we perform causal inference and build predictive models using EHR with treatment records.  Our results show that ignoring treatment non-adherence bias could lead to reversed conclusions in treatment effect estimation, significantly degrade the performance of predictive models up to 5\%, and lead to unfair predictions. Furthermore, we highlight the importance of addressing treatment non-adherence bias by showing simply removing patient records with non-adherence, though reducing the size of the training dataset, could improve model performance and lead to fairer predictions.
\\
\newline
The contributions of this work include:
\begin{enumerate}
\item Conducting a large-scale study on treatment non-adherence in hypertension and identifying statistically significant factors associated with non-adherence.

\item Comparing LLM identification against physician annotations, LLMs perform well with 92\% accuracy, precision and recall.


\item Identifying patient-reported reasons for treatment non-adherence including side effects, forgetfulness, difficulties obtaining refills, etc.

\item Demonstrating the harmful impact of ignoring treatment non-adherence bias on causal inference and predictive modeling, leading to poorer performance and exacerbating racial disparities.

\end{enumerate}


\section{Related work}

In this section we give an overview of the detection and watermarking methods for audio data. 
A complementary description of prior works can be found in Chap.~\ref{chapter:related-work}.

\paragraph{Synthetic speech detection.}
The approach of most audio generation and text-to-speech (\Gls*{TTS}) papers~\citep{Borsos2022AudioLMAL, Kharitonov2023SpeakRA, borsos2023soundstorm, le2023voicebox} is to train end-to-end deep learning classifiers on what their models generate, similarly as \citet{zhang2017investigation}.
Most of the time, these approaches rely on faint artifacts left by the vocoder that generate the final audio. 
Therefore, accuracy when comparing synthetic to real is usually good, although not performing well on out of distribution audios (compressed, noised, slowed, etc.).
For the same reason, the detection can be fooled by using a different vocoder (to generate synthetic audio that are not flagged) or by using the same vocoder on non-synthetic audios (to generate audios that spoof the detection).
This is highlighted in Sec.~\ref{chap4/sec:active-passive}, where we show that the detection fails on re-synthesized audio.

\paragraph{Invisible watermarking.} 
Traditional audio watermarking methods embed watermarks in the time or frequency domains, often using domain-specific features. 
Deep learning methods focus on multi-bit watermarking and follow a generator/decoder framework, as explained in Sec.~\ref{chap0/sec:watermarking}.
Few works have explored zero-bit watermarking~\citep{wu2023adversarial, juvela2023collaborative}, which is better adapted for detection of AI-generated content.
Our rationale is that robustness increases as the message payload is reduced to the bare minimum~\citep{furon2007constructive}.

In this study, we compare our work with the state-of-the-art watermarking method, WavMark~\citep{chen2023wavmark}, which outperforms previous ones. 
It uses invertible networks to hide 32 bits in 1-second audio segments.
Detection is done by sliding along the audio in 0.05s steps and decoding the message for each window.
If the 10 first decoded bits match a synchronization pattern the rest of the payload is saved (22 bits), and the window can directly slide 1s (instead of the 0.05).
This brute force detection algorithm is prohibitively slow especially when the watermark is absent, since the algorithm will have to attempt and fail to decode a watermark for each sliding window in the input audio (due to the absence of watermark).

\section{Method}

Our literature search focused on AI-assisted creative research tools in contrast to AI-assisted writing tools, creative research tools assist with the co-creation of concepts and ideas in the research process rather than merely improving stylistic or rhetorical choices in written research. In total, we surveyed 11 systems papers published in top HCI venues (i.e., CHI, CSCW, UIST, and ToCHI) over the last three years (2022-2024)\footnote{We include a relevant CHI 2025 paper made available on arXiv.}; details can be found in Table~\ref{tab:tool_classification}. 

As LLMs became widely used in 2022 with the release of ChatGPT \cite{openai2022chatgpt}, this timeframe was chosen to reflect the period of significant growth in LLM popularity and adoption, allowing us to capture the most relevant and impactful developments in GenAI-driven research tools. Of the systems surveyed, six of the systems integrate LLM-based functionalities, while the other five represent a more traditional AI approach and employ machine learning techniques (e.g., Seq2Seq, BERT, RNN). By examining both GenAI and traditional AI approaches, we aim to understand to what extent GenAI tools represent a fundamental shift in capabilities and design considerations compared to more established AI approaches.

The thematic dimensions presented in Section~\ref{sec:design-space} resulted from an iterative process among the authors. We engaged in extensive internal discussions and consulted with an external expert specializing in knowledge spaces and the role of AI in fostering creativity and sensemaking. This collaborative and iterative approach resulted in the four dimensions presented in the next section.






























\section{Better watermarking evaluations}

\autoref{chap5/sec:robustness-analysis} evaluates the detection with the revised statistical tests and smaller FPRs than the previous literature. 
\autoref{chap5/sec:free-form} shows the impact on NLP benchmarks.

\subsection{Robustness analysis}
\label{chap5/sec:robustness-analysis}

\begin{table}[t!]
    \centering
    \caption{
    Robustness analysis of the watermarks, with rectified statistical tests.
    We report the TPR@FPR=$10^{-5}$ and the S-BERT scores over $10\times 1$k completions, for different hyperparameters controlling the strength of the watermark 
    ($\delta$ in \citep{kirchenbauer2023watermark} and $\temperature$ in \citep{aaronson2023watermarking} - see Sec.~\ref{chap5/sec:background}).
    The `TPR aug.' is the TPR when texts are attacked before detection by randomly replacing tokens with probability 0.3.
    }
    \label{chap5/tab:robustness}
    \footnotesize
    \begin{tabular}{rl *{4}{p{1cm}} @{\hspace{0.5cm}} *{4}{p{1cm}}}
        \toprule
        & & \multicolumn{4}{c}{\citep{aaronson2023watermarking}} &  \multicolumn{4}{c}{\citep{kirchenbauer2023watermark}}  \\
        $k$ & Metric & $\temperature:$ 0.8 & 0.9 & 1.0 & 1.1 & $\delta:$ 1.0 & 2.0 & 3.0 & 4.0 \\
        \cmidrule(rr){3-6} \cmidrule(rr){7-10}
        \multirow{3}{*}{$0$} 
            & S-BERT    & 0.60 & 0.56 & 0.52 & 0.44 & 0.63 & 0.61 & 0.57 & 0.50 \\
            & TPR       & 0.20 & 0.31 & 0.42 & 0.51 & 0.00 & 0.16 & 0.58 & 0.70 \\
            & TPR aug.  & 0.04 & 0.06 & 0.09 & 0.10 & 0.00 & 0.02 & 0.20 & 0.39 \\[4pt]
        \multirow{3}{*}{$1$} 
            & S-BERT    & 0.62 & 0.61 & 0.59 & 0.55 & 0.63 & 0.62 & 0.60 & 0.56 \\
            & TPR       & 0.35 & 0.51 & 0.66 & 0.77 & 0.02 & 0.41 & 0.77 & 0.88 \\
            & TPR aug.  & 0.04 & 0.10 & 0.20 & 0.36 & 0.00 & 0.05 & 0.30 & 0.58 \\[4pt]
        \multirow{3}{*}{$4$} 
            & S-BERT    & 0.62 & 0.62 & 0.61 & 0.59 & 0.62 & 0.62 & 0.60 & 0.57 \\
            & TPR       & 0.43 & 0.59 & 0.71 & 0.80 & 0.02 & 0.44 & 0.76 & 0.88 \\
            & TPR aug.  & 0.01 & 0.02 & 0.06 & 0.18 & 0.00 & 0.00 & 0.03 & 0.14 \\
        \bottomrule
    \end{tabular}
\end{table}   





We now compare watermarking methods by analyzing the TPR when detecting watermarked texts.
For detection, we employ the previous statistical tests and scoring strategy.
They enable precise control over the FPR and therefore to operate at operating-points not yet seen in the literature.
More specifically we flag a text as watermarked if its \pval\ is lower than $10^{-5}$ ensuring an FPR=$10^{-5}$.
We prompt Guanaco-7-B~\citep{dettmers2023qlora}, an instruction fine-tuned version of Llama, with the first $1$k prompts from the Alpaca dataset~\citep{alpaca}.
For generation, we use top-$p$ sampling with $p=0.95$, and in the case of \citep{kirchenbauer2023watermark} a temperature $\theta =0.8$ and $\gamma=1/4$.
We simulate synonym attacks by randomly replacing tokens with probability $0.3$ (other attacks are studied in related work~\citep{kirchenbauer2023reliability}).

\autoref{chap5/tab:robustness} reports the TPR for different strength of the watermark (see Sec.~\ref{chap5/sec:background}), and the S-BERT~\citep{reimers2019sentence} similarity score between the generated texts with and without watermarking to measure the semantic distortion induced by the watermark. 
Results reveals different behaviors.
For instance, \citep{kirchenbauer2023watermark} has a finer control over the trade-off between watermark strength and quality.
Its TPR values ranges from 0.0 to 0.9, while \citep{aaronson2023watermarking} is more consistent but fails to achieve TPR higher than 0.8 even when the S-BERT score is degraded a lot.

The watermark context width also has a big influence. 
When $k$ is low, we observe that repetitions happen more often because the generation is easily biased towards certain repetitions of tokens.
It leads to average S-BERT scores below 0.5 and unusable completions.
On the other hand, low $k$ also makes the watermark more robust, especially for \citep{kirchenbauer2023watermark}.
It is also important to note that $k$ has an influence on the number of analyzed tokens since we only score tokens for which the $k+1$-tuple has not been seen before (see Sec.~\ref{chap5/sec:rect}).
If $k$ is high, almost all these tuples are new, while if $k$ is low, the chance of repeated tuples increases.
For instance in our case, the average number of scored tokens is around 100 for $k=0$, and 150 for $k=1$ and $k=4$.


\subsection{Impact on free-form generation tasks}
\label{chap5/sec:free-form}
Previous studies measure the impact on quality using distortion metrics such as perplexity or similarity score as done in Tab.~\ref{chap5/tab:robustness}.
However, such metrics are not informative of the utility of the model for downstream tasks~\citep{holtzman2019curious}, where the real interest of LLMs lies. 
Indeed, watermarking LLMs could be harmful for tasks that require very precise answers, like code or maths.
This section rather quantifies the impact on typical NLP benchmarks, in order to assess the practicality of watermarking.

LLMs are typically evaluated by comparing samples of plain generation to target references (free-form generation) or by comparing the likelihood of predefined options in a multiple choice question fashion. 
The latter makes little sense in the case of watermarking, which only affects sampling.
We therefore limit our evaluations to free-form generation tasks.
We use the evaluation setup of Llama:
1) Closed-book Question Answering (Natural Questions~\citep{kwiatkowski2019natural}, TriviaQA~\citep{joshi2017triviaqa}): we report the $5$-shot exact match performance;
2) Mathematical reasoning (MathQA~\citep{hendrycks2021measuring}, GSM8k~\citep{cobbe2021training}), we report exact match performance without majority voting;
3) Code generation (HumanEval~\citep{chen2021Evaluating}, MBPP~\citep{austin2021program}), we report the pass@1 scores.
For \citep{kirchenbauer2023watermark}, we shift logits with $\delta=1.0$ before greedy decoding.
For \citep{aaronson2023watermarking}, we use $\theta = 0.8$, apply top-p at $0.95$ to the probability vector, then apply the watermarked sampling.

\autoref{chap5/tab:bench-full} reports the performance of Llama models on the aforementioned benchmarks, with and without the watermark and for different window size $k$. 
The performance of the LLM is not significantly affected by watermarking. 
The approach of \cite{kirchenbauer2023watermark} is slightly more harmful than the one of \cite{aaronson2023watermarking}, but the difference w.r.t. the vanilla model is small.
Interestingly, this difference decreases as the size of the model increases: models with higher generation capabilities are less affected by watermarking. A possible explanation is that the global distribution of the larger models is better and thus more robust to small perturbations.
Overall, evaluating on downstream tasks points out that watermarking may introduce factual errors that are not well captured by perplexity or similarity scores.











\begin{table}[H]
    \centering
    \caption{ 
        Performances on free-form generation benchmarks when completion is done with watermarking.
        $k$ is the watermark context width. 
        We report results for methods: [AK]~\citep{aaronson2023watermarking} / [KGW]~\citep{kirchenbauer2023watermark}.
        ``-'' means no watermarking. 
    }
    \label{chap5/tab:bench-full}
    \resizebox{0.95\textwidth}{!}{
    \begin{tabular}{lllrrrrrrr}
    \toprule
     &  &  & GSM8K & Human Eval & MathQA & MBPP & NQ & TQA & Average \\
    Model & Method & $k$ &  &  &  &  &  &  &  \\
    \midrule
     \multirow[t]{15}{*}{7-B} 
     & None & - & 10.31 & 12.80 & 2.96 & 18.00 & 21.72 & 56.89 & 20.45 \\
     \cmidrule{2-10} 
     & [AK] & 0 & 10.54 & 12.80 & 3.00 & 18.00 & 21.77 & 56.88 & 20.50 \\
      &  & 1 & 10.31 & 12.80 & 2.88 & 18.20 & 21.75 & 56.87 & 20.47 \\
      &  & 2 & 10.31 & 12.80 & 2.94 & 18.00 & 21.75 & 56.86 & 20.44 \\
      &  & 4 & 10.39 & 12.80 & 2.98 & 17.80 & 21.80 & 56.88 & 20.44 \\
      &  & 8 & 10.46 & 12.80 & 2.90 & 18.20 & 21.75 & 56.85 & 20.49 \\
      \cmidrule{2-10} 
      & [KGW] & 0 & 9.63 & 12.80 & 2.20 & 16.20 & 20.06 & 55.09 & 19.33 \\
      &  & 1 & 11.14 & 9.76 & 2.82 & 16.00 & 19.50 & 55.30 & 19.09 \\
      &  & 2 & 11.07 & 6.71 & 2.62 & 16.00 & 20.44 & 55.07 & 18.65 \\
      &  & 4 & 10.77 & 9.15 & 2.76 & 16.40 & 20.17 & 55.14 & 19.06 \\
      &  & 8 & 11.37 & 11.59 & 2.90 & 16.40 & 20.66 & 55.36 & 19.71 \\
    \midrule
    \multirow[t]{15}{*}{13-B} 
    & None & - & 17.21 & 15.24 & 4.30 & 23.00 & 28.17 & 63.60 & 25.25 \\
    \cmidrule{2-10} 
    & [AK] & 0 & 17.29 & 15.24 & 4.24 & 22.80 & 28.17 & 63.60 & 25.22 \\
     &  & 1 & 17.21 & 15.24 & 4.30 & 22.80 & 28.20 & 63.61 & 25.23 \\
     &  & 2 & 17.51 & 15.24 & 4.20 & 22.80 & 28.20 & 63.59 & 25.26 \\
     &  & 4 & 17.21 & 15.24 & 4.20 & 22.60 & 28.20 & 63.63 & 25.18 \\
     &  & 8 & 17.21 & 15.24 & 4.22 & 22.80 & 28.20 & 63.62 & 25.22 \\
     \cmidrule{2-10} 
     & [KGW] & 0 & 14.33 & 14.02 & 3.04 & 20.80 & 24.32 & 62.13 & 23.11 \\
     &  & 1 & 17.29 & 14.63 & 3.62 & 21.20 & 25.12 & 62.23 & 24.02 \\
     &  & 2 & 16.45 & 11.59 & 3.54 & 20.60 & 25.54 & 62.44 & 23.36 \\
     &  & 4 & 16.76 & 15.85 & 4.08 & 21.20 & 24.49 & 62.24 & 24.10 \\
     &  & 8 & 17.29 & 14.63 & 3.68 & 21.00 & 25.46 & 62.17 & 24.04 \\
     \midrule
    \multirow[t]{14}{*}{30-B} 
     & None & - & 35.10 & 20.12 & 6.80 & 29.80 & 33.55 & 70.00 & 32.56 \\
     \cmidrule{2-10} 
     & [AK] & 0 & 35.48 & 20.12 & 6.88 & 29.80 & 33.52 & 69.98 & 32.63 \\
     & & 1 & 35.33 & 20.73 & 6.88 & 29.60 & 33.52 & 70.03 & 32.68 \\
     & & 2 & 35.33 & 20.73 & 6.94 & 30.00 & 33.49 & 70.00 & 32.75 \\
     & & 4 & 35.10 & 20.12 & 6.90 & 29.80 & 33.49 & 70.01 & 32.57 \\
     & & 8 & 35.33 & 20.73 & 6.94 & 30.00 & 33.52 & 70.01 & 32.75 \\
     \cmidrule{2-10} 
     & [KGW] & 0 & 31.84 & 21.95 & 6.88 & 28.40 & 31.66 & 69.03 & 31.63 \\
     &  & 1 & 35.56 & 20.73 & 7.54 & 28.80 & 31.58 & 68.98 & 32.20 \\
     &  & 2 & 33.21 & 17.07 & 6.48 & 27.40 & 31.83 & 69.44 & 30.91 \\
     &  & 4 & 34.12 & 22.56 & 6.96 & 28.80 & 31.55 & 68.74 & 32.12 \\
     &  & 8 & 34.95 & 20.12 & 7.42 & 27.20 & 32.08 & 69.31 & 31.85 \\
    \bottomrule
    \end{tabular}
    }
\end{table}







\section{Adversarial watermark removal}
\label{chap4/sec:attacks}


So far, we considered robustness against regular editing that may happen naturally.
We now examine more damaging deliberate attacks, where attackers might either ``forge'' the watermark by adding it to authentic samples (to overwhelm detection systems) or ``remove'' it to avoid detection. 
Our findings suggest that in order to maintain the effectiveness of watermarking against such adversaries, the code for training watermarking models and the awareness that published audios are watermarked can be made public. 
However, the detector's weights should be kept confidential.

We focus on watermark-removal attacks and consider three types of attacks depending on the adversary's knowledge:
\begin{itemize}
    \item \textit{White-box}: 
    the adversary has access to the detector (\eg, because of a leak), and performs a gradient-based adversarial attack against it.
    The optimization objective is to minimize the detector's output.
    \item \textit{Semi black-box}: 
    the adversary does not have access to any weights, but is able to re-train generator/detector pairs with the same architectures on the same dataset.
    They perform the same gradient-based attack as before, but using the new detector as proxy for the original one.
    \item \textit{Black-box}: 
    the adversary does not have any knowledge on the watermarking algorithm being used, but has access to an API that produces watermarked samples, and to negative speech samples from any public dataset.
    They first collect samples and train a classifier to discriminate between watermarked and not-watermarked.
    They attack this classifier as if it were the true detector.
\end{itemize}


\begin{figure}[b!]
    \centering
    \includegraphics[width=0.55\linewidth, clip, trim={0 0.3cm 0 0}]{chapter-4/figs/attack.pdf}
    \caption{
    \textbf{Watermark-removal attacks.} 
    PESQ is measured between attacked audios and genuine ones (PESQ $<4$ strongly degrades the audio quality).
    The Gaussian noise is used as a reference, but better ``no-box'' attacks are possible (\eg, perceptual autoencoders).
    The more knowledge the attacker has over the watermarking algorithm, the better the attack is.
    }
    \label{chap4/fig:attacks}
\end{figure}

\paragraph{Adversarial attack against a detector.}
Given a watermarked sample $x$ and a detector $D$, we want to find $x' \sim x$ such that $D(x') = 0$.
To that end, we use a gradient-based attack.
It starts by initializing a distortion $\delta_{adv}$ with random gaussian noise.
The algorithm iteratively updates the distortion for a number of steps $n$. 
For each step, the distortion is added to the original audio via $ x' = x + \alpha . \mathrm{tanh} (\delta_{adv})$, passed through the model to get predictions. 
A cross-entropy loss is computed with label 0 (for removal, 1 would be for forging which is not explored here) and back-propagated through the detector to update the distortion, using the Adam optimizer.
At the end of the process, the adversarial audio is $x + \alpha . \mathrm{tanh} (\delta_{adv})$.
In our attack, we use a scaling factor $\alpha=10^{-3}$, a number of steps $n=100$, and a learning rate of $10^{-1}$. 
The $\mathrm{tanh}$ function is used to ensure that the distortion remains small, and gives an upper bound on the SNR of the adversarial audio\footnote{
    This approach is similar to the image optimizations of chapters~\ref{chapter:ssl-watermarking} and~\ref{chapter:active-indexing}; and to the adversarial attacks of Chap.~\ref{chapter:stable-signature}.
}.

\paragraph{Training of the malicious detector.}
For the black-box attack, we want to train and attack a surrogate detector that can distinguish between watermarked and non-watermarked samples, when access to many samples of both types is available.
To train the classifier, we use a dataset made of more than 80k samples of 8 seconds speech from Voicebox~\citep{le2023voicebox} watermarked using our proposed method and a similar amount of genuine (un-watermarked) speech samples. 
The classifier shares the same architecture as AudioSeal's detector. 
The classifier is trained for 200k updates with batches of 64 one-second samples. 
It achieves perfect classification of the samples. 
This is coherent with the findings of Voicebox~\citep{le2023voicebox}.



\paragraph*{Results.}
For every scenario, we watermark 1k samples of 5 seconds, then attack them.
\autoref{chap4/fig:attacks} contrasts various attacks at different intensities, using Gaussian noise as a reference.
The white-box attack is by far the most effective one, increasing the detection error by around 80\%, while maintaining high audio quality (PESQ $>4$).
Other attacks are less effective, requiring significant audio quality degradation to achieve $50\%$ increase the detection error, though they are still more effective than random noise addition.
In summary, the more is disclosed about the watermarking algorithm, the more vulnerable it is. 
The effectiveness of these attacks is limited as long as the detector remains confidential.




\section{Ablation studies and additional results}\label{chap4/sec:ablations}






\subsection{False positive rates for WavMark}\label{chap4/app:fpr}

\paragraph{Theoretical study.}

When doing detection with multi-bit watermarking, previous works usually extract the message $m'(x)$ from the content $x$ and compare it to the original binary signature $m\in \{ 0,1 \}^{k}$ embedded in the speech sample, as done in Chap.~\ref{chapter:stable-signature}, Sec.~\ref{chap3/subsec:statistical-test}.
The detection test relies on the number of matching bits $M(m,m')$:
\begin{equation} 
    \text{if } M\left(m,m'\right) \geq \tau \,\,\textrm{ where }\,\, \tau\in 
\{0,\ldots,k\},
\end{equation}
then the audio is flagged.
This provides theoretical guarantees over the false positive rates.

Formally, the null hypothesis $\H_0$ is: ``The audio signal is not watermarked'', against the alternative $\H_1$: ``The audio signal is watermarked''.
Under $\H_0$ (\ie, for unmarked audio), if the bits $m'_1, \ldots, m'_k$ are independent and identically distributed Bernoulli random variables with parameter $0.5$, then  $M(m, m')$ follows a binomial distribution with parameters ($k$, $0.5$).
The False Positive Rate (FPR) is defined as the probability that $M(m, m')$ exceeds a given threshold $\tau$. 
A closed-form expression can be given using the regularized incomplete beta function $I_x(a;b)$ (linked to the c.d.f. of the binomial distribution):
\begin{align}\label{chap4/eq:p-value}
    \text{FPR}(\tau) & = \mathbb{P}\left(M \geq \tau \mid \H_0\right) = I_{1/2}(\tau, k - \tau +1).
\end{align}


\begin{figure}[b!]
    \centering
    \includegraphics[width=0.7\linewidth, clip, trim={0.1in 0 0.1in 0}]{chapter-4/figs/appendix/fpr_wavmark.pdf}
    \caption{
        (Left) Histogram of scores output by WavMark's extractor on 10k genuine samples. 
        (Right) Empirical and theoretical FPR when the chosen hidden message is all 0.
    }
    \label{chap4/fig:app_fpr_wavmark}
\end{figure}

\paragraph{Empirical study.}
We empirically study the FPR of WavMark-based detection on our validation dataset.
We use the same parameters as in the original paper, \ie, $k=32$-bits are extracted from 1s speech samples.
We first extract the soft bits (before thresholding) from 10k genuine samples and plot the histogram of the scores in Fig.~\ref{chap4/fig:app_fpr_wavmark} (left).
We should observe a Gaussian distribution with mean $0.5$, while empirically the scores are centered around $0.38$. 
This makes the decision heavily biased towards bit 0 on genuine samples.
It is therefore impossible to theoretically set the FPR since this would largely underestimate the actual one.
For instance, \autoref{chap4/fig:app_fpr_wavmark} (right) shows the theoretical and empirical FPR for different values of $\tau$ when the chosen hidden message is full 0.
Put differently, the argument that says that hiding bits allows for theoretical guarantees over the detection rates is not valid in practice.\footnote{
    In Chap.~\ref{chapter:ssl-watermarking} and~\ref{chapter:stable-signature}, we overcome this issue by whitening the outputs of the watermark extractor (see Fig.~\ref{chap1/fig:whitening}).
    This is not possible in this case, since the watermark embedder and extractor operate jointly. 
    It would therefore require at least to retrain or regularize the model to avoid this bias.
}








\subsection{Another architecture}
\label{chap4/app:other-arch}

Our architecture relies on the SOTA compression method EnCodec. 
However, to further validate our approach, we conduct an ablation study using a different architecture DPRNN~\citep{luo2020dual}. 
The results are presented in Tab.~\ref{chap4/tab:dprnn}.
They show that the performance of AudioSeal is consistent across different architectures. 
This indicates that model capacity is not a limiting factor for AudioSeal.

\begin{table}[t!]
    \centering
    \caption{
        Results for different architectures of the generator and detector.
        The IoU is computed for 1s of watermark in 10s audios (corresponding to the leftmost point in Fig.~\ref{chap4/fig:loc_quantitative}).
    }\label{chap4/tab:dprnn}
    \footnotesize
    \begin{tabular}{lccccc}
        \toprule
        Method  & SI-SNR & STOI & PESQ & Acc. & IoU \\
        \midrule
        EnCodec & 26.00 & 0.997 & 4.470 & 1.00 & 0.802 \\
        DPRNN   & 26.7 & 0.996 & 4.421 & 1.00 & 0.796 \\
        \bottomrule
    \end{tabular}
    \end{table}






\subsection{Audio mixing}

We hereby evaluate the scenario where two signals (e.g., vocal and instrumental) are mixed together. 
We use a non-vocal music dataset for the instrumental part, and we normalize and sum the loudness of the watermarked speech and the music segments. 
\autoref{chap4/tab:mixed_signals} shows that the watermark is still detectable in the mixed signal, even when a non-watermarked background music is present, with a slight decrease in performance.

\begin{table}[t!]
    \centering
    \caption{
        Detection results for watermarked speech and music mixed signals.
        \cmarkg\ and \xmarkg\ indicate the presence or absence of the watermark.
    }
    \label{chap4/tab:mixed_signals}
    \footnotesize
    \begin{tabular}{cclll}
    \toprule
    Speech & BG Music & Acc. & \aux{FPR / TPR} & AUC \\
    \midrule
    \cmarkg & \cmarkg & 1.000 & \aux{$3\times 10^{-4}$ / 1.000} & 1.000 \\
    \cmarkg & \xmarkg & 0.979 & \aux{$3.1\times 10^{-2}$ / 0.988} & 0.996 \\
    \bottomrule
    \end{tabular}
\end{table}




\subsection{Out of domain evaluations}\label{chap4/app:ood}

\paragraph{Synthesized speech and audio.} 
\label{chap4/sec:generalization}
We first evaluate how AudioSeal generalizes on AI-generated speech/audio of various domains and languages. 
Specifically, we use the datasets ASVspoof~\citep{liu2023asvspoof} and FakeAVCeleb \citep{khalid2021fakeavceleb}. 
Additionally, we translate speech samples from a subset of the Expresso dataset~\citep{nguyen2023expresso} (studio-quality recordings) using the SeamlessExpressive translation model~\citep{seamless2023}.
We select four target languages: Mandarin Chinese (CMN), French (FR), Italian (IT), and Spanish (SP). 
We also evaluate on non-speech AI-generated audios: music from MusicGen~\citep{copet2023simple} and environmental sounds from AudioGen~\citep{kreuk2023audiogen}. 


\begin{table*}[t]
    \caption{
    Evaluation of AudioSeal Generalization across domains and languages. Namely, translations of speech samples from the Expresso dataset~\citep{nguyen2023expresso} to four target languages: Mandarin Chinese (CMN), French (FR), Italian (IT), and Spanish (SP), using the SeamlessExpressive model~\citep{seamless2023}. Music from MusicGen~\citep{copet2023simple} and environmental sounds from AudioGen~\citep{kreuk2023audiogen}. 
    }
    \label{chap4/tab:ood_data}
    \centering
    \footnotesize
        \begin{tabular}{l| *{6}{p{1.0cm}} *{2}{p{1.0cm}} }
        \toprule
        Aug & \rotatebox[origin=c]{45}{Seamless (Cmn)} & \rotatebox[origin=c]{45}{Seamless (Spa)} & \rotatebox[origin=c]{45}{Seamless (Fra)} & \rotatebox[origin=c]{45}{Seamless(Ita)} & \rotatebox[origin=c]{45}{Seamless (Deu)} & \rotatebox[origin=c]{45}{Voicebox (Eng)} & \rotatebox[origin=c]{45}{AudioGen} & \rotatebox[origin=c]{45}{MusicGen}  \\
        \midrule
        None         & 1.00           & 1.00            & 1.00            & 1.00           & 1.00            &   1.00   &   1.00   &   1.00\\
        \midrule
        Bandpass   & 1.00 &   1.00  &   1.00 & 1.00 &   1.00  &  1.00 &   1.00   &   1.00   \\
            Highpass  & 0.71 &  0.68  &   0.70  & 0.70 &  0.70  &  0.64   &  0.52 &  0.52     \\
        Lowpass    & 1.00 &  0.99 &  1.00 & 1.00 & 1.00 &  1.00  &  1.00  &  1.00  \\
            Boost     & 1.00  &  1.00  &  1.00 & 1.00  &  1.00 &  1.00 &     1.00     &      1.00   \\
            Duck      & 1.00  &  1.00  &  1.00 &1.00 &  1.00 &   1.00  &   1.00   &     1.00    \\
            Echo   & 1.00  &  1.00 &  1.00 &1.00 & 1.00 &  1.00  &      1.00    &     1.00     \\
            Pink   & 0.99 &   1.00  & 0.99 & 1.00 &   0.99 &  1.00 &      1.00    &    1.00  \\
            White  & 1.00 &    1.00  & 1.00 & 1.00 &  1.00 & 1.00   &  1.00   &  1.00   \\
        Fast (x1.25)  & 0.97 & 0.98 & 0.99  & 0.98 & 0.99 & 0.98 & 0.87 & 0.87 \\
            Smooth    &  0.96  &  0.99  &   0.99  &    0.99    &      0.99          & 0.99 &  0.98  &    0.98   \\
            Resample  & 1.00 &  1.00 &  1.00 & 1.00 &    1.00  & 1.00 &   1.00    &   1.00   \\
                AAC & 0.99 &  0.99  &  0.99  & 0.99 &  0.99  &   0.97  &  0.99   &     0.98    \\
                MP3 & 0.99 &  0.99   &  0.99 & 0.99 & 0.99  &    0.97  &  0.99    & 1.00  \\
            Encodec   & 0.97 &  0.98   &  0.99 & 0.99 & 0.98 &   0.96     &  0.95    & 0.95   \\
            \midrule
            Average   &  0.97 &  0.97  & 0.98  & 0.98 & 0.98 & 0.97 & 0.95 & 0.95  \\
            \bottomrule
        \end{tabular}
\end{table*}


\begin{table}[t!]
    \centering
    \caption{
        Audio quality and intelligibility evaluations on AI-generated speech from various models and languages.
    }
    \label{chap4/tab:ood_metrics}
    \footnotesize
    \renewcommand{\arraystretch}{1.2}
    \begin{tabular}{cccccc}
        \toprule
        Model & Dataset & SI-SNR & PESQ & STOI & ViSQOL \\
        \midrule
        \multirow{3}{*}{\scriptsize \rotatebox[origin=c]{90}{AudioSeal}} & Seam. (Deu)       & 23.35 & 4.244 & 0.999 & 4.688 \\
        & Seam. (Fr)        & 24.02 & 4.199 & 0.998 & 4.669 \\
        & Voicebox             & 25.23 & 4.449 & 0.998 & 4.800 \\
        \midrule
        \multirow{3}{*}{\scriptsize \rotatebox[origin=c]{90}{WavMark}} & Seam. (Deu)    & 38.93 & 3.982 & 0.999 & 4.515 \\
        & Seam. (Fr)     & 39.06 & 3.959 & 0.999 & 4.506 \\
        & Voicebox          & 39.63 & 4.211 & 0.998 & 4.695 \\
        \bottomrule
    \end{tabular}
\end{table}

We employ the same set of augmentations and observe very similar detection results, as demonstrated in Tab.~\ref{chap4/tab:ood_data}.
Interestingly, even though we did not train our model on AI-generated speech, we sometimes notice a slight improvement in performance compared to our test data. 
No sample is misclassified among the 10k samples that comprise each of our out-of-distribution (OOD) datasets.
We also provide perceptual metrics results on some OOD data in Tab.~\ref{chap4/tab:ood_metrics}.
We observe that AudioSeal performs similarly on these datasets, with a slight decrease in performance compared to our original dataset.
We explain the decrease by the \ploss\ which makes the watermark hidden in the same frequency bands as English speech, which might not be the case for other languages or audio types.

\paragraph{Fake vs. real datasets.}
We also evaluate AudioSeal on three additional datasets containing real human speech: AudioSet~\citep{gemmeke2017audio}, ASVspoof~\citep{liu2023asvspoof}, and FakeAVCeleb~\citep{khalid2021fakeavceleb}, and observe similar performance, see Tab.~\ref{chap4/tab:other_datasets}.


\begin{table}[t!]
    \centering
    \caption{Evaluation of the detection performances on different datasets. AudioSet is an environmental sounds dataset while ASVspoof~\citep{liu2023asvspoof} and FakeAVCeleb~\citep{khalid2021fakeavceleb} are deep-fake detection datasets.}
    \label{chap4/tab:other_datasets}
    \footnotesize
    \begin{tabular}{l *{2}{l}}
        \toprule
        Dataset & Acc. \aux{TPR/FPR} & AUC \\
        \midrule
        Audioset & 0.9992 \aux{0.9996/0.0011} & 1.0 \\
        ASVspoof & 1.0 \aux{1.0/0.0} & 1.0 \\
        FakeAVCeleb & 1.0 \aux{1.0/0.0} & 1.0 \\
        \bottomrule
    \end{tabular}
\end{table}











\begin{figure}[b!]
    \centering
    \includegraphics[width=1.0\textwidth]{chapter-4/figs/appendix/augmentation_curves.pdf}
    \caption{
        Accuracy of the detector on augmented samples with respect to the strength of the augmentation.
    }
    \label{chap4/fig:app_augmentation_curves}
\end{figure}


\subsection{More robustness results}\label{chap4/app:robustness}

We plot the detection accuracy against the strength of multiple augmentations in Fig.~\ref{chap4/fig:app_augmentation_curves}. 
AudioSeal outperforms WavMark for most augmentations at the same strength.
However, for highpass filters above our training range (500Hz) WavMark has a much better detection accuracy.
Our system's TF-loudness loss embeds the watermark where human speech carries the most energy, typically lower frequencies, due to auditory masking. 
This contrasts with WavMark, which places the watermark in higher frequency bands.
Embedding the watermark in lower frequencies is advantageous. 
For example, speech remains audible with a lowpass filter at 1500 Hz, but not with a highpass filter at the same frequency. 
This difference is measurable with PESQ in relation to the original audio, making it more beneficial to be robust against a lowpass filter at a 1500 Hz cut-off than a highpass filter at the same cut-off:

\begin{center}
    \footnotesize
    \begin{tabular}{cccc}
        Filter Type & PESQ & AudioSeal & WavMark \\
        \midrule
        Highpass 1500Hz & 1.85 \xmarkg & 0.7 & 1.0 \\
        Lowpass 1500Hz & 2.93 \cmarkg & 1.0 & 0.7 \\
    \end{tabular}
\end{center}


\section{Conclusion}\label{chap3/sec:conclusion}

By a quick fine-tuning of the decoder of Latent Diffusion Models, we can embed watermarks in all the images they generate.
This does not alter the diffusion process, making it compatible with most of LDM-based generative models.
These watermarks are robust, invisible to the human eye and can be employed to \emph{detect} generated images and \emph{identify} the user that generated it, with very high performance.
The public release of image generative models has an important societal impact.
With this work, we put to light the usefulness of using watermarking instead of relying on passive detection methods.
We hope it encourages researchers and practitioners to employ similar approaches before making their models publicly available.

The biggest drawback from the method is its intrusive nature.
It requires substantial effort in re-evaluating model image quality and increases implementation cost (because it needs to modify multiple production models instead of maintaining a single encoding model as production use-cases grow).
Embedding the watermark task within the generation process is also not conducive to industry standardization and may limit certain ``data sharing'' opportunities.
Therefore, in cases where the model is not expected to be publicly released, or where the model is expected to be used in a wide range of applications, the watermarking method may not be suitable. 
A post-hoc watermarking method may be preferred, as it is more flexible and can be applied to any model without any modification.
This is the approach we take in Chap.~\ref{chapter:audioseal} for audio watermarking.




