
\chapter{Proactive Detection of Voice Cloning with Localized Watermarking}\label{chapter:audioseal}

This chapter is based on the paper \fullcite{san2024proactive}.

In the rapidly evolving field of speech generative models, there is a pressing need to ensure audio authenticity against the risks of voice cloning. 
This chapter presents AudioSeal, an audio watermarking technique designed specifically for localized detection of AI-generated speech. 
It is the counterpart to the image watermarking method presented in Chap.~\ref{chapter:stable-signature} in two respects: it is a post-hoc watermarking method (as opposed to watermarking during generation), and it is designed for audio signals (which present different challenges).
AudioSeal employs a generator / detector architecture trained jointly with a localization loss to enable localized watermark detection up to the sample level, and a novel perceptual loss inspired by auditory masking, that enables better imperceptibility. 
AudioSeal achieves state-of-the-art performance in terms of robustness to real life audio manipulations and imperceptibility based on automatic and human evaluation metrics. 
Additionally, AudioSeal is designed with a fast, single-pass detector, that significantly surpasses existing models in speed, achieving detection up to two orders of magnitude faster, making it ideal for large-scale and real-time applications.
Code is available at \url{github.com/facebookresearch/audioseal}.

\newpage
\section{Introduction}
% \textcolor{red}{Di: 1) Again, you should not put multiple places in bold, 2) you may need to add five more recently published related papers in the introduction}

Nonlinear dynamical systems are omnipresent across scientific disciplines, and understanding causal relationships in these systems is crucial for unveiling the underlying mechanisms that drive system behaviors. Classic causal inference methods, such as Granger Causality (GC)~\citep{granger1969investigating} and other functional causal models (FCMs), including the Additive Noise Model (ANM)~\citep{hoyer2008nonlinear, liu2024causal} and the Post Nonlinear Model (PNL)~\citep{zhang2015estimation, keropyan2023rank}, struggle with these systems due to their assumption of a predictive relationship from cause to effect, which does not hold in the presence of complex dynamics like coupling and chaos.

% Convergent Cross Mapping (CCM)~\citep{sugihara2012detecting} was proposed as a model-free approach for bivariate causal inference in nonlinear dynamical systems. CCM addresses these limitations by leveraging state-space manifold reconstructions and cross mapping between reconstructed embeddings. Since its introduction, CCM has inspired further developments, including Partial Cross Mapping (PCM)~\citep{leng2020partial}, which aims to distinguish indirect from direct causalities in three-variable systems. However, PCM is limited to mapping operations between univariate delay embeddings, which can be less efficient when dealing with higher-dimensional systems with multiple interconnected variables. \textcolor{red}{Di: are there any related works try to tackle this problem?}

Convergent Cross Mapping (CCM)~\citep{sugihara2012detecting, barraquand2021inferring} was proposed as a model-free approach for bivariate causal inference in coupled dynamical systems. CCM addresses these limitations by leveraging state-space manifold reconstructions and cross mapping between reconstructed embeddings. Since its introduction, CCM has inspired further developments, including Partial Cross Mapping (PCM)~\citep{leng2020partial}, which aims to distinguish indirect from direct causalities in three-variable systems. However, PCM is limited to mapping operations between univariate delay embeddings, which can be less effective or even fail when dealing with complex systems with multiple interconnected variables~\citep{chen2022causation}.

To overcome this limitation, we propose \textbf{multiPCM}, an extension of PCM to the multivariate setting that allows for more effective causal inference by utilizing cross mapping via multivariate embeddings. We further integrate multiPCM with bivariate CCM into a two-phase framework named \textbf{MXMap} (Multivariate Cross Mapping for Causal Discovery). The proposed framework is designed for multivariate causal discovery, and is not only confined to assumptions of directed acyclic graphs (DAGs) but can also handle cycles. In the first phase, bivariate CCM generates an initial, potentially dense causal graph; In the second phase, multiPCM prunes indirect connections, refining the graph to isolate direct causal relationships. We systematically evaluate multiPCM and MXMap on benchmark datasets, including both simulated ecosystems and real-world meteorological data.

The contributions of this work are summarized as follows:

\begin{itemize}
    \item \textbf{Extension of PCM to multivariate settings}: We introduce multiPCM, which extends PCM to utilize multivariate delay embeddings for more robust causal inference in high-dimensional dynamical systems.
    \item \textbf{Two-phase causal discovery framework}: We propose MXMap, combining bivariate CCM with multiPCM to generate and refine causal graphs in nonlinear dynamical systems, which can also detect cycles.
    \item \textbf{Comprehensive evaluation on nonlinear dynamical systems}: We validate multiPCM and MXMap on simulated and real-world datasets. MXMap is compared against multiple baseline methods — including tsFCI, VAR-LiNGAM, PCMCI, Granger Causality, DYNOTEARS, SLARAC — demonstrating advantages in accuracy and refinement capabilities.
\end{itemize}

\section{Related work}

In this section we give an overview of the detection and watermarking methods for audio data. 
A complementary description of prior works can be found in Chap.~\ref{chapter:related-work}.

\paragraph{Synthetic speech detection.}
The approach of most audio generation and text-to-speech (\Gls*{TTS}) papers~\citep{Borsos2022AudioLMAL, Kharitonov2023SpeakRA, borsos2023soundstorm, le2023voicebox} is to train end-to-end deep learning classifiers on what their models generate, similarly as \citet{zhang2017investigation}.
Most of the time, these approaches rely on faint artifacts left by the vocoder that generate the final audio. 
Therefore, accuracy when comparing synthetic to real is usually good, although not performing well on out of distribution audios (compressed, noised, slowed, etc.).
For the same reason, the detection can be fooled by using a different vocoder (to generate synthetic audio that are not flagged) or by using the same vocoder on non-synthetic audios (to generate audios that spoof the detection).
This is highlighted in Sec.~\ref{chap4/sec:active-passive}, where we show that the detection fails on re-synthesized audio.

\paragraph{Invisible watermarking.} 
Traditional audio watermarking methods embed watermarks in the time or frequency domains, often using domain-specific features. 
Deep learning methods focus on multi-bit watermarking and follow a generator/decoder framework, as explained in Sec.~\ref{chap0/sec:watermarking}.
Few works have explored zero-bit watermarking~\citep{wu2023adversarial, juvela2023collaborative}, which is better adapted for detection of AI-generated content.
Our rationale is that robustness increases as the message payload is reduced to the bare minimum~\citep{furon2007constructive}.

In this study, we compare our work with the state-of-the-art watermarking method, WavMark~\citep{chen2023wavmark}, which outperforms previous ones. 
It uses invertible networks to hide 32 bits in 1-second audio segments.
Detection is done by sliding along the audio in 0.05s steps and decoding the message for each window.
If the 10 first decoded bits match a synchronization pattern the rest of the payload is saved (22 bits), and the window can directly slide 1s (instead of the 0.05).
This brute force detection algorithm is prohibitively slow especially when the watermark is absent, since the algorithm will have to attempt and fail to decode a watermark for each sliding window in the input audio (due to the absence of watermark).

\section{Notations}

We consider an auto-regressive language model $M$ with parameters $\theta$. We use $p_\theta(\cdot \vert x)$ to denote $M$'s distribution over the next token given the provided context $x$. 
Given a question $q$ (e.g., \nl{Jane had 4 apples and ate half of her apples. How many apples she has now?}), we denote the model's response as $(\textbf{r}, \textbf{a})$,
where $\textbf{a}$ is the answer (e.g., \nl{2}) and $\textbf{r}$ is a \emph{reasoning path} (or chain-of-thought),  a sequence of logical steps supposedly leading up to this answer (e.g., \nl{If Jane ate half her apples, this means she ate 2 apples. 4 minus 2 is 2.}).

\section{Confidence-Informed Self-Consistency}
\label{sec:cisc}

In this section we present \textit{Confidence-Informed Self-Consistency} (CISC). 
When designing CISC, we hypothesized that it is possible to reduce self-consistency's computational costs by generating a \emph{confidence score} for each reasoning path, and performing a weighted majority vote.

As an intuitive example, consider a hypothetical setting where there exist only two possible answers, one correct and one incorrect. For a model that responds with the correct answer $60\%$ of the time, standard majority voting will require \emph{40 samples} to reach $90\%$ accuracy\footnote{Calculated using the binomial distribution. All the technical details are included in Appendix \ref{appendix:example}}. However, a weighted majority vote that weights correct answers twice as much as incorrect ones, will achieve 90\% accuracy with less than \emph{10 samples}. 

With this motivation in mind, we build on recent findings suggesting that LLMs are capable of judging the correctness of their own outputs \cite{kadavath2022language, tian2023just, zhang2024small}, and incorporate the model’s self-assessment of its reasoning paths into the final answer selection:

\begin{definition}[Confidence-Informed Self-Consistency]
\label{def:cisc}
Given a question $q$ and responses $\{(\textbf{r}_1, \textbf{a}_1), \dots, (\textbf{r}_m, \textbf{a}_m) \}$, CISC involves:

\begin{itemize}
    \item \textbf{Confidence Extraction}: A self-assessed confidence score $c_i\in\R$ is derived for each $(\textbf{r}_i, \textbf{a}_i)$.
    \item \textbf{Confidence Normalization}: The confidence scores are normalized
    using Softmax: $\tilde{c}_i = \frac{\exp\!\bigl(\frac{c_i}{T}\bigr)}{\sum_{j=1}^m \exp\!\bigl(\tfrac{c_j}{T}\bigr)}$, where $T$ is a tunable temperature hyper-parameter (see discussion below).
    \item \textbf{Aggregation}:  The final answer is selected using a confidence-weighted majority vote: $\hat{a}_{CISC} = \arg\max_a\sum_{i=1}^m \textbf{1}[\textbf{a}_i = a]\cdot \tilde{c}_i$. 
\end{itemize}
\end{definition}

The temperature parameter $T$ controls the relative importance of the answer frequency versus the confidence scores. Namely, as $T\to \infty$, the distribution of normalized confidence scores approaches the uniform distribution, and CISC collapses to vanilla self-consistency. Conversely, as $T\to 0$,  the softmax normalization approaches the hard maximum function, prioritizing the single response with the highest confidence and disregarding the overall frequency of answers. This may lead CISC to select a different answer than self-consistency (see Figure \ref{fig:high-level}). 

\input{chapter-4/sections/3-quality}

\section{Main experiments}
\label{chap4/sec:exps}



\subsection{Audio/speech quality}
\label{chap4/sec:quality}

We first evaluate the quality of the watermarked audio using:
Scale Invariant Signal to Noise Ratio (SI-SNR): 
$\textrm{SI-SNR}(s, s_w) = 10 \log_{10} \left( \| \alpha s \|_2^2 / \| \alpha s - s_w \|_2^2 \right)$,
with $s$ the clean audio and $s_w$ the watermarked one,
where $\alpha = \langle s, s_w \rangle / \| s \|_2^2$;
as well as Perceptual Evaluation of Speech Quality (PESQ)~\citep{rix2001perceptual}, 
Virtual Speech Quality Objective Listener (ViSQOL)~\citep{hines2012visqol} and
Short Term Objective Intelligibility (STOI)~\citep{taal2010short} which are objective perceptual metrics measuring the quality of speech signals.

\autoref{chap4/tab:audio_quality} report these metrics.
AudioSeal behaves differently than watermarking methods like WavMark~\citep{chen2023wavmark} that try to minimize the SI-SNR.
In practice, high SI-SNR is indeed not necessarily correlated with good perceptual quality.
AudioSeal is not optimized for SI-SNR but rather for perceptual quality of speech (similarly as DeAR~\citep{DEAR_Liu0FMZY23} which reports 25.96 as SI-SNR).
This is better captured by the other metrics (PESQ, STOI, ViSQOL), where AudioSeal consistently achieves better performance.
Put differently, our goal is to hide as much watermark power as possible while keeping it perceptually indistinguishable from the original.
\autoref{chap4/fig:loc_quali} also visualizes how the watermark signal follows the shape of the speech waveform.

The metric used for our subjective evaluations is the MUSHRA score~\citep{mushra}. 
It is a crowdsourced test in which participants rate the quality of various samples on a scale of 0 to 100. 
The ground truth is provided for reference. 
We utilized 100 speech samples, each lasting 10 seconds. 
Each sample was evaluated by at least 20 participants.
As part of the study, we included a low anchor, which is a very lossy compression at 1.5kbps, encoded using EnCodec. 
Participants who failed to assign the lowest score to the low anchor for at least 80\% of their assignments were excluded from the study.
In this study our samples got superior ratings than WavMark, with an average score of 77.07, 5 points higher than WavMark.
For comparison, the ground truth samples received an average score of 80.49, while the low anchor's average score was 53.21.


\begin{table}[t!]
    \centering
    \caption{
        Audio quality metrics. 
        Compared to traditional watermarking methods that minimize the SNR like WavMark, AudioSeal achieves same or better perceptual quality.
    }\label{chap4/tab:audio_quality}
    \footnotesize
        \begin{tabular}{lccccc}
            \toprule
            \textbf{Methods} & \textbf{SI-SNR} & \textbf{PESQ} & \textbf{STOI} & \textbf{ViSQOL} & \textbf{MUSHRA}  \\
            \midrule
            WavMark & \textbf{38.25} & 4.302 & 0.997 & 4.730 & 71.52 $\pm$ 7.18\\
            AudioSeal & 26.00 & \textbf{4.470} & 0.997 & \textbf{4.829} &  \textbf{77.07} $\pm$ 6.35 \\
            \bottomrule
        \end{tabular}
\end{table}















\subsection{Comparison with passive classifier}\label{chap4/sec:active-passive}


\begin{table}[t!]
    \centering
    \caption{
        Comparison with Voicebox binary classifier. 
        Percentage refers to the fraction of masked input frames.
        Detection with AudioSeal is perfect for all samples, while Voicebox classifier fails on re-synthesized audio.
    }
    \label{chap4/tab:voicebox}
    \footnotesize
        \begin{tabular}{r *{3}{c}  *{3}{c}  *{3}{c} }
            \toprule
            & \multicolumn{3}{c}{\textbf{AudioSeal (Ours)}} & \multicolumn{3}{c}{\textbf{Voicebox Classif.}} \\
            \cmidrule(rr){2-4} \cmidrule(rr){5-7}
            \textbf{\% Mask} & Acc. & TPR & FPR & Acc. & TPR & FPR \\
            \midrule
            \multicolumn{7}{l}{\emph{Original audio vs AI-generated audio}} \\
            30\% & 1.0 & 1.0 & 0.0 & 1.0 & 1.0 & 0.0 \\
            50\% & 1.0 & 1.0 & 0.0 & 1.0 & 1.0 & 0.0 \\
            90\% & 1.0 & 1.0 & 0.0 & 1.0 & 1.0 & 0.0 \\
            \midrule
            \multicolumn{7}{l}{\emph{Re-synthesized audio vs AI-generated audio}} \\
            30\% & \textbf{1.0} & \textbf{1.0} & \textbf{0.0} & 0.704 & 0.680 & 0.194 \\
            50\% & \textbf{1.0} & \textbf{1.0} & \textbf{0.0} & 0.809 & 0.831 & 0.170 \\
            90\% & \textbf{1.0} & \textbf{1.0} & \textbf{0.0} & 0.907 & 0.942 & 0.112 \\
            \bottomrule
        \end{tabular}
\end{table}



We first compare detection results on samples generated with Voicebox~\citep{le2023voicebox}.
We compare to the passive setup where a classifier is trained to discriminate between Voicebox-generated and real audios (this is done with a detector that shares the same architecture as AudioSeal's, trained until reaching 100\% accuracy on Voicebox samples).
Following the approach in the Voicebox study, we evaluate 2,000 approximately 5-second samples from LibriSpeech, these samples have masked frames (90\%, 50\%, and 30\% of the phonemes) pre-Voicebox generation.
We evaluate on the same tasks, \ie, distinguishing between original and generated, or between original and re-synthesized (created by extracting the Mel spectrogram from original audio and then vocoding it with the HiFi-GAN vocoder).

We use the True Positive Rate (\Gls*{TPR}) and the False Positive Rate (\Gls*{FPR}) as key metrics.
TPR measures correct identification of AI-generated samples, while FPR indicates the rate of genuine audio clips falsely flagged.
In practical scenarios, minimizing FPR is crucial. 
For example, on a platform processing 1 billion samples daily, an FPR of $10^{-3}$ and a TPR of $0.5$ means that 1 million samples require manual review each day, yet only half of the watermarked samples are detected.
The \Gls*{ROC} AUC (Area Under the Curve of the Receiver Operating Characteristics) gives a global measure of performance over all threshold levels, and captures the TPR/FPR trade-off.

Both active and passive setups achieve perfect classification in the case when trained to distinguish between natural and Voicebox.
Conversely, the second part of Tab.~\ref{chap4/tab:voicebox} highlights a significant drop in performance when the classifier is trained to differentiate between Voicebox and re-synthesized.
It suggests that the classifier is detecting vocoder artifacts, since the re-synthesized samples are sometimes wrongly flagged.
The classification performance quickly decreases as the quality of the AI-generated sample increases (when the input is less masked).
On the other hand, our proactive detection does not rely on model-specific artifacts but on the watermark presence. %
This allows for perfect detection over all the audio clips. %




\subsection{Robustness and comparison with watermarking}

\begin{table}[t!]
    \centering
    \caption{
        Detection results for different edits applied before detection. 
        Acc. ({\aux{TPR/FPR}}) is the accuracy (and TPR/FPR) obtained for the threshold that gives best accuracy on a balanced set of augmented samples.
        AUC is the area under the ROC curve.
    }
    \label{chap4/tab:wm_robustness}
    \footnotesize
        \begin{tabular}{l *{2}{l}  *{2}{l}}
        \toprule
        & \multicolumn{2}{l}{\textbf{AudioSeal (Ours)}} & \multicolumn{2}{l}{\textbf{WavMark}} \\
        \cmidrule(rr){2-3} \cmidrule(rr){4-5}
        \multicolumn{1}{c}{Edit} & Acc. \aux{TPR/FPR} & AUC & Acc. \aux{TPR/FPR}  & AUC \\
        \cmidrule(rr){1-1} \cmidrule(rr){2-3} \cmidrule(rr){4-5}
        None & 1.00 \aux{1.00/0.00} & 1.00 & 1.00 \aux{1.00/0.00} & 1.00 \\
        Bandpass & 1.00 \aux{1.00/0.00} & 1.00 & 1.00 \aux{1.00/0.00} & 1.00 \\
        Highpass  &  0.61 \aux{0.82/0.60} & 0.61 & \bf 1.00 \aux{1.00/0.00} & \bf 1.00 \\
        Lowpass & \bf 0.99 \aux{0.99/0.00} & \bf 0.99 & 0.50 \aux{1.00/1.00} & 0.50 \\
        Boost & 1.00 \aux{1.00/0.00} & 1.00 & 1.00 \aux{1.00/0.00} & 1.00 \\
        Duck & 1.00 \aux{1.00/0.00} &  1.00 & 1.00 \aux{1.00/0.00} & 1.00 \\
        Echo & \bf 1.00 \aux{1.00/0.00} & \bf 1.00 & 0.93 \aux{0.89/0.03} & 0.98 \\
        Pink & \bf 1.00 \aux{1.00/0.00} & \bf 1.00 & 0.88 \aux{0.81/0.05} & 0.93 \\
        White & \bf 0.91 \aux{0.86/0.04} & \bf 0.95 & 0.50 \aux{0.54/0.54} & 0.50 \\
        Fast (1.25x) & \bf 0.99 \aux{0.99/0.00} & \bf 1.00 & 0.50 \aux{0.01/0.00} & 0.15 \\
        Smooth & \bf 0.99 \aux{0.99/0.00} &  1.00   & 0.94 \aux{0.93/0.04} & 0.98 \\
        Resample & 1.00 \aux{1.00/0.00} &  1.00 & 1.00 \aux{1.00/0.00} & 1.00 \\
        AAC & 1.00 \aux{1.00/0.00} &  1.00 & 1.00 \aux{1.00/0.00} & 1.00 \\
        MP3 & \bf 1.00 \aux{1.00/0.00} & \bf 1.00 & 1.00 \aux{0.99/0.00} & 0.99 \\
        EnCodec & \bf  0.98 \aux{0.98/0.01} & \bf 1.00 & 0.51 \aux{0.52/0.50} & 0.50 \\
        \midrule
        Average & \bf 0.96 \aux{0.98/0.04} & \bf 0.97 & 0.85 \aux{0.85/0.14} & 0.84 \\
        \bottomrule
        \end{tabular}
\end{table}

\paragraph*{Audio editing.}
We evaluate the robustness of the detection on a wide range of audio editing operations: 
time modification (faster, resample), 
filtering (bandpass, highpass, lowpass), 
audio effects (echo, boost audio, duck audio), 
noise (pink noise, random noise),
and compression (MP3, AAC, EnCodec).
In order to show generalization, we chose stronger parameter to the attacks than those used during training (see Sec.~\ref{chap4/sec:training-details}).

\paragraph*{Robustness of the detection.}
Detection is done on 10k ten-seconds audios from our VoxPopuli validation set.
For each edit, we first build a balanced dataset made of the 10k watermarked/ 10k non-watermarked edited audio clips.
We quantify the performance by adjusting the threshold of the detection score, selecting the value that maximizes accuracy (we provide corresponding TPR and FPR at this threshold).
To adapt data-hiding methods (\eg, WavMark) for proactive detection, we embed a binary message (chosen randomly beforehand) in the generated speech before release. The detection score is then computed as the Hamming distance between the original message and the one extracted from the scrutinized audio. 

We observe in Tab.~\ref{chap4/tab:wm_robustness} that AudioSeal is overall more robust, with an average AUC of 0.97 vs. 0.84 for WavMark.
The performance for lowpass and highpass filters indicates that AudioSeal embeds watermarks neither in the low nor in the high frequencies (WavMark focuses on high frequencies).
We give results on more augmentations in Sec.~\ref{chap4/app:robustness}.



\subsection{Localization}


\begin{figure}[b!]
    \centering
    \includegraphics[width=0.48\linewidth, clip, trim={0 1.8in 0 0}, valign=t]{chapter-4/figs/loc_quantitative.pdf}\hfill
    \includegraphics[width=0.48\linewidth, clip, trim={0 0 0 1.3in}, valign=t]{chapter-4/figs/loc_quantitative.pdf}
    \caption{\textbf{Localization results} across different durations of watermarked audio signals in terms of Sample-Level Accuracy and Intersection Over Union (IoU) metrics ($\uparrow$ is better).}
    \label{chap4/fig:loc_quantitative}
\end{figure}


We evaluate localization with the sample-level detection accuracy, \ie, the proportion of correctly labeled samples, and the Intersection over Union (IoU).
The latter is defined as the intersection between the predicted and the ground truth detection masks (1 when watermarked, 0 otherwise), divided by their union.
IoU is a more relevant evaluation of the localization of short watermarks in a longer audio.

This evaluation is carried out on the same audio clips as for detection.
For each one of them, we watermark a randomly placed segment of varying length.
Localization with WavMark is a brute-force detection: a window of 1s slides over the 10s of speech with the default shift value of 0.05s.
The Hammning distance between the 16 pattern bits is used as the detection score.
Whenever a window triggers a positive, we label its 16k samples as watermarked in the detection mask in $\{0,1\}^t$.

\autoref{chap4/fig:loc_quantitative} plots the sample-level accuracy and IoU for different proportions of watermarked speech in the audio clip.
AudioSeal achieves an IoU of 0.99 when just one second of speech is AI-manipulated, compared to WavMark's 0.35.
Moreover, AudioSeal allows for precise detection of minor audio alterations: it can pinpoint AI-generated segments in audio down to the sample level (usually 1/16k sec), while the concurrent WavMark only provides one-second resolution.
This is the reason why it lags behind in terms of IoU more than accuracy.
It is especially relevant for speech samples, where a simple word modification may greatly change meaning. 





\subsection{Attribution}

\begin{table}[t!]
    \centering
    \caption{
        Attribution results.
        We report the accuracy of the attribution (Acc.) and false attribution rate (FAR). 
        Detection is done at FPR=$10^{-3}$ and attribution matches the decoded message to one of $N$ versions.
        We report averaged results over the edits of Tab.~\ref{chap4/tab:wm_robustness}.
    }\label{chap4/tab:attribution}
    \footnotesize
        \begin{tabular}{cr *{5}{c}}
            \toprule
            & N & $1$ & $10$ & $10^2$ & $10^3$ & $10^4$ \\ \midrule
    \multirow{2}{*}{FAR (\%) $\downarrow$} & WavMark      & 0.0 & \textbf{0.20} & \textbf{0.98} & \textbf{1.87} & \textbf{4.02} \\
            & AudioSeal   & 0.0 & 2.52 & 6.83 & 8.96 & 11.84 \\ \midrule
    \multirow{2}{*}{\shortstack{Acc. (\%) $\uparrow$}} & WavMark      & 58.4 & 58.2 & 57.4 & 56.6 & 54.4 \\
            & AudioSeal  & \textbf{68.2} & \textbf{65.4} & \textbf{61.4} & \textbf{59.3} & \textbf{56.4} \\ 
            \bottomrule
        \end{tabular}
\end{table}

Given an audio clip, the objective is now to find if any of $N$ versions of our model generated it (detection), and if so, which one (identification). 
For evaluation, we create $N'=100$ random 16-bits messages and use them to watermark 1k audio clips, each consisting of 5 seconds of speech (not 10s to reduce compute needs). 
This results in a total of 100k audios. 
For WavMark, the first 16 bits (/32) are fixed and the detection score is the number of well decoded pattern bits, while the second half of the payload hides the model version.
An audio clip is flagged if the average output of the detector exceeds a threshold, corresponding to FPR=$10^{-3}$.
Next, we calculate the Hamming distance between the decoded watermark and all $N$ original messages. 
The message with the smallest Hamming distance is selected.
It is worth noting that we can simulate $N>N'$ models by adding extra messages. 
This may represent versions that have not generated any sample.

False Attribution Rate (FAR) is the fraction of wrong attribution \emph{among the detected audios} while the attribution accuracy is the proportion of detections followed by a correct attributions \emph{over all audios}. 
AudioSeal has a higher FAR but overall gives a better accuracy, which is what ultimately matters.
First, we observe that the false attribution rate -- which we define as the proportion of audios that are wrongly attributed among the detected ones -- is higher in our case.
On the other hand \autoref{chap4/tab:attribution} highlights that AudioSeal gives better attribution accuracy.
It is defined as the proportion of watermarked samples that are both flagged and correctly attributed and is what ultimately matters.
In summary, decoupling detection and attribution achieves better detection rate and makes the global accuracy better, at the cost of occasional false attributions.



\subsection{Efficiency analysis}
\label{chap4/sec:speed}


\begin{figure}[b!]
    \centering
    \includegraphics[width=0.65\linewidth, clip, trim={0 0 0 0}]{chapter-4/figs/speed.pdf}
    \caption{Mean runtime ($\downarrow$ is better) of AudioSeal versus WavMark. 
    AudioSeal is one order of magnitude faster for watermark generation and two orders of magnitude faster for watermark detection for the same audio input.
    }
    \label{chap4/fig:efficiency}
\end{figure}



To highlight the efficiency of AudioSeal, we conduct a performance analysis and compare it with WavMark. 
We apply the watermark generator and detector of both models on a dataset of 500 audio segments ranging in length from 1 to 10 seconds, using a single Nvidia Quadro GP100 GPU. 
The results are displayed in Fig.~\ref{chap4/fig:efficiency} and Tab.~\ref{chap4/tab:speed}.
In terms of generation, AudioSeal is 14$\times$ faster than WavMark. 
For detection, AudioSeal outperforms WavMark with two orders of magnitude faster performance on average, notably 485$\times$ faster in scenarios where there is no watermark (Tab.~\ref{chap4/tab:speed}). 
The speed difference in the case of WavMark between watermarked and non-watermarked audios is explained by the fact that whenever the detector flags a 1-second span as watermarked, it will directly skip to the next 1-second span, while for non-watermarked audios, it will slide the window by 0.05s.
AudioSeal's speed is due to the model's localized watermark design, which bypasses the need for watermark synchronization (recall that WavMark relies on 20 pass forwards for a one-second snippet).
AudioSeal's detector provides detection logits for each input sample directly with only one pass to the detector, significantly enhancing the detection's computational efficiency.
This makes our system highly suitable for real-time and large-scale applications.

\begin{table*}[t!]
    \centering
    \caption{
        Average runtime (ms) per sample of AudioSeal model against WavMark~\citep{chen2023wavmark} method. 
        Our experiments were conducted on a dataset of audio segments spanning 1 second to 10 seconds, using a single Nvidia Quadro GP100 GPU. 
        The results demonstrate significant speed improvements for both watermark generation and detection with and without the presence of a watermark. 
        Notably, for watermark detection, AudioSeal is 485$\times$ faster than WavMark when there is no watermark, because the latter relies on more forward passes when trying to synchronize the watermark. 
    }
    \footnotesize
    \label{chap4/tab:speed}
    \begin{tabular}{llll}
    \toprule
               Model & Watermarked &     \textbf{Detection ms (speedup)} &   \textbf{Generation ms (speedup)} \\
    \midrule
             WavMark &          \multirow{2}{*}{\xmarkg}      & 1710.70 $\pm$ 1314.02 &    -- \\
             AudioSeal (ours) &           &       \textbf{3.25 $\pm$ 1.99} \;\; (\textbf{485$\times$}) &    -- \\
    \midrule
             WavMark &         \multirow{2}{*}{\cmarkg} &    106.21 $\pm$ 66.95 & 104.58 $\pm$ 65.66 \\
    AudioSeal (ours) &          &       \textbf{3.30} $\pm$ \textbf{2.03} \;\; (\textbf{35$\times$}) &    \textbf{7.41} $\pm$ \textbf{4.52} \;\; (\textbf{14} $\times$) \\
    
    \bottomrule 
    \end{tabular}
\end{table*}























\section{Adversarial watermark removal}
\label{chap4/sec:attacks}


So far, we considered robustness against regular editing that may happen naturally.
We now examine more damaging deliberate attacks, where attackers might either ``forge'' the watermark by adding it to authentic samples (to overwhelm detection systems) or ``remove'' it to avoid detection. 
Our findings suggest that in order to maintain the effectiveness of watermarking against such adversaries, the code for training watermarking models and the awareness that published audios are watermarked can be made public. 
However, the detector's weights should be kept confidential.

We focus on watermark-removal attacks and consider three types of attacks depending on the adversary's knowledge:
\begin{itemize}
    \item \textit{White-box}: 
    the adversary has access to the detector (\eg, because of a leak), and performs a gradient-based adversarial attack against it.
    The optimization objective is to minimize the detector's output.
    \item \textit{Semi black-box}: 
    the adversary does not have access to any weights, but is able to re-train generator/detector pairs with the same architectures on the same dataset.
    They perform the same gradient-based attack as before, but using the new detector as proxy for the original one.
    \item \textit{Black-box}: 
    the adversary does not have any knowledge on the watermarking algorithm being used, but has access to an API that produces watermarked samples, and to negative speech samples from any public dataset.
    They first collect samples and train a classifier to discriminate between watermarked and not-watermarked.
    They attack this classifier as if it were the true detector.
\end{itemize}


\begin{figure}[b!]
    \centering
    \includegraphics[width=0.55\linewidth, clip, trim={0 0.3cm 0 0}]{chapter-4/figs/attack.pdf}
    \caption{
    \textbf{Watermark-removal attacks.} 
    PESQ is measured between attacked audios and genuine ones (PESQ $<4$ strongly degrades the audio quality).
    The Gaussian noise is used as a reference, but better ``no-box'' attacks are possible (\eg, perceptual autoencoders).
    The more knowledge the attacker has over the watermarking algorithm, the better the attack is.
    }
    \label{chap4/fig:attacks}
\end{figure}

\paragraph{Adversarial attack against a detector.}
Given a watermarked sample $x$ and a detector $D$, we want to find $x' \sim x$ such that $D(x') = 0$.
To that end, we use a gradient-based attack.
It starts by initializing a distortion $\delta_{adv}$ with random gaussian noise.
The algorithm iteratively updates the distortion for a number of steps $n$. 
For each step, the distortion is added to the original audio via $ x' = x + \alpha . \mathrm{tanh} (\delta_{adv})$, passed through the model to get predictions. 
A cross-entropy loss is computed with label 0 (for removal, 1 would be for forging which is not explored here) and back-propagated through the detector to update the distortion, using the Adam optimizer.
At the end of the process, the adversarial audio is $x + \alpha . \mathrm{tanh} (\delta_{adv})$.
In our attack, we use a scaling factor $\alpha=10^{-3}$, a number of steps $n=100$, and a learning rate of $10^{-1}$. 
The $\mathrm{tanh}$ function is used to ensure that the distortion remains small, and gives an upper bound on the SNR of the adversarial audio\footnote{
    This approach is similar to the image optimizations of chapters~\ref{chapter:ssl-watermarking} and~\ref{chapter:active-indexing}; and to the adversarial attacks of Chap.~\ref{chapter:stable-signature}.
}.

\paragraph{Training of the malicious detector.}
For the black-box attack, we want to train and attack a surrogate detector that can distinguish between watermarked and non-watermarked samples, when access to many samples of both types is available.
To train the classifier, we use a dataset made of more than 80k samples of 8 seconds speech from Voicebox~\citep{le2023voicebox} watermarked using our proposed method and a similar amount of genuine (un-watermarked) speech samples. 
The classifier shares the same architecture as AudioSeal's detector. 
The classifier is trained for 200k updates with batches of 64 one-second samples. 
It achieves perfect classification of the samples. 
This is coherent with the findings of Voicebox~\citep{le2023voicebox}.



\paragraph*{Results.}
For every scenario, we watermark 1k samples of 5 seconds, then attack them.
\autoref{chap4/fig:attacks} contrasts various attacks at different intensities, using Gaussian noise as a reference.
The white-box attack is by far the most effective one, increasing the detection error by around 80\%, while maintaining high audio quality (PESQ $>4$).
Other attacks are less effective, requiring significant audio quality degradation to achieve $50\%$ increase the detection error, though they are still more effective than random noise addition.
In summary, the more is disclosed about the watermarking algorithm, the more vulnerable it is. 
The effectiveness of these attacks is limited as long as the detector remains confidential.




\section{Ablation studies and additional results}\label{chap4/sec:ablations}






\subsection{False positive rates for WavMark}\label{chap4/app:fpr}

\paragraph{Theoretical study.}

When doing detection with multi-bit watermarking, previous works usually extract the message $m'(x)$ from the content $x$ and compare it to the original binary signature $m\in \{ 0,1 \}^{k}$ embedded in the speech sample, as done in Chap.~\ref{chapter:stable-signature}, Sec.~\ref{chap3/subsec:statistical-test}.
The detection test relies on the number of matching bits $M(m,m')$:
\begin{equation} 
    \text{if } M\left(m,m'\right) \geq \tau \,\,\textrm{ where }\,\, \tau\in 
\{0,\ldots,k\},
\end{equation}
then the audio is flagged.
This provides theoretical guarantees over the false positive rates.

Formally, the null hypothesis $\H_0$ is: ``The audio signal is not watermarked'', against the alternative $\H_1$: ``The audio signal is watermarked''.
Under $\H_0$ (\ie, for unmarked audio), if the bits $m'_1, \ldots, m'_k$ are independent and identically distributed Bernoulli random variables with parameter $0.5$, then  $M(m, m')$ follows a binomial distribution with parameters ($k$, $0.5$).
The False Positive Rate (FPR) is defined as the probability that $M(m, m')$ exceeds a given threshold $\tau$. 
A closed-form expression can be given using the regularized incomplete beta function $I_x(a;b)$ (linked to the c.d.f. of the binomial distribution):
\begin{align}\label{chap4/eq:p-value}
    \text{FPR}(\tau) & = \mathbb{P}\left(M \geq \tau \mid \H_0\right) = I_{1/2}(\tau, k - \tau +1).
\end{align}


\begin{figure}[b!]
    \centering
    \includegraphics[width=0.7\linewidth, clip, trim={0.1in 0 0.1in 0}]{chapter-4/figs/appendix/fpr_wavmark.pdf}
    \caption{
        (Left) Histogram of scores output by WavMark's extractor on 10k genuine samples. 
        (Right) Empirical and theoretical FPR when the chosen hidden message is all 0.
    }
    \label{chap4/fig:app_fpr_wavmark}
\end{figure}

\paragraph{Empirical study.}
We empirically study the FPR of WavMark-based detection on our validation dataset.
We use the same parameters as in the original paper, \ie, $k=32$-bits are extracted from 1s speech samples.
We first extract the soft bits (before thresholding) from 10k genuine samples and plot the histogram of the scores in Fig.~\ref{chap4/fig:app_fpr_wavmark} (left).
We should observe a Gaussian distribution with mean $0.5$, while empirically the scores are centered around $0.38$. 
This makes the decision heavily biased towards bit 0 on genuine samples.
It is therefore impossible to theoretically set the FPR since this would largely underestimate the actual one.
For instance, \autoref{chap4/fig:app_fpr_wavmark} (right) shows the theoretical and empirical FPR for different values of $\tau$ when the chosen hidden message is full 0.
Put differently, the argument that says that hiding bits allows for theoretical guarantees over the detection rates is not valid in practice.\footnote{
    In Chap.~\ref{chapter:ssl-watermarking} and~\ref{chapter:stable-signature}, we overcome this issue by whitening the outputs of the watermark extractor (see Fig.~\ref{chap1/fig:whitening}).
    This is not possible in this case, since the watermark embedder and extractor operate jointly. 
    It would therefore require at least to retrain or regularize the model to avoid this bias.
}








\subsection{Another architecture}
\label{chap4/app:other-arch}

Our architecture relies on the SOTA compression method EnCodec. 
However, to further validate our approach, we conduct an ablation study using a different architecture DPRNN~\citep{luo2020dual}. 
The results are presented in Tab.~\ref{chap4/tab:dprnn}.
They show that the performance of AudioSeal is consistent across different architectures. 
This indicates that model capacity is not a limiting factor for AudioSeal.

\begin{table}[t!]
    \centering
    \caption{
        Results for different architectures of the generator and detector.
        The IoU is computed for 1s of watermark in 10s audios (corresponding to the leftmost point in Fig.~\ref{chap4/fig:loc_quantitative}).
    }\label{chap4/tab:dprnn}
    \footnotesize
    \begin{tabular}{lccccc}
        \toprule
        Method  & SI-SNR & STOI & PESQ & Acc. & IoU \\
        \midrule
        EnCodec & 26.00 & 0.997 & 4.470 & 1.00 & 0.802 \\
        DPRNN   & 26.7 & 0.996 & 4.421 & 1.00 & 0.796 \\
        \bottomrule
    \end{tabular}
    \end{table}






\subsection{Audio mixing}

We hereby evaluate the scenario where two signals (e.g., vocal and instrumental) are mixed together. 
We use a non-vocal music dataset for the instrumental part, and we normalize and sum the loudness of the watermarked speech and the music segments. 
\autoref{chap4/tab:mixed_signals} shows that the watermark is still detectable in the mixed signal, even when a non-watermarked background music is present, with a slight decrease in performance.

\begin{table}[t!]
    \centering
    \caption{
        Detection results for watermarked speech and music mixed signals.
        \cmarkg\ and \xmarkg\ indicate the presence or absence of the watermark.
    }
    \label{chap4/tab:mixed_signals}
    \footnotesize
    \begin{tabular}{cclll}
    \toprule
    Speech & BG Music & Acc. & \aux{FPR / TPR} & AUC \\
    \midrule
    \cmarkg & \cmarkg & 1.000 & \aux{$3\times 10^{-4}$ / 1.000} & 1.000 \\
    \cmarkg & \xmarkg & 0.979 & \aux{$3.1\times 10^{-2}$ / 0.988} & 0.996 \\
    \bottomrule
    \end{tabular}
\end{table}




\subsection{Out of domain evaluations}\label{chap4/app:ood}

\paragraph{Synthesized speech and audio.} 
\label{chap4/sec:generalization}
We first evaluate how AudioSeal generalizes on AI-generated speech/audio of various domains and languages. 
Specifically, we use the datasets ASVspoof~\citep{liu2023asvspoof} and FakeAVCeleb \citep{khalid2021fakeavceleb}. 
Additionally, we translate speech samples from a subset of the Expresso dataset~\citep{nguyen2023expresso} (studio-quality recordings) using the SeamlessExpressive translation model~\citep{seamless2023}.
We select four target languages: Mandarin Chinese (CMN), French (FR), Italian (IT), and Spanish (SP). 
We also evaluate on non-speech AI-generated audios: music from MusicGen~\citep{copet2023simple} and environmental sounds from AudioGen~\citep{kreuk2023audiogen}. 


\begin{table*}[t]
    \caption{
    Evaluation of AudioSeal Generalization across domains and languages. Namely, translations of speech samples from the Expresso dataset~\citep{nguyen2023expresso} to four target languages: Mandarin Chinese (CMN), French (FR), Italian (IT), and Spanish (SP), using the SeamlessExpressive model~\citep{seamless2023}. Music from MusicGen~\citep{copet2023simple} and environmental sounds from AudioGen~\citep{kreuk2023audiogen}. 
    }
    \label{chap4/tab:ood_data}
    \centering
    \footnotesize
        \begin{tabular}{l| *{6}{p{1.0cm}} *{2}{p{1.0cm}} }
        \toprule
        Aug & \rotatebox[origin=c]{45}{Seamless (Cmn)} & \rotatebox[origin=c]{45}{Seamless (Spa)} & \rotatebox[origin=c]{45}{Seamless (Fra)} & \rotatebox[origin=c]{45}{Seamless(Ita)} & \rotatebox[origin=c]{45}{Seamless (Deu)} & \rotatebox[origin=c]{45}{Voicebox (Eng)} & \rotatebox[origin=c]{45}{AudioGen} & \rotatebox[origin=c]{45}{MusicGen}  \\
        \midrule
        None         & 1.00           & 1.00            & 1.00            & 1.00           & 1.00            &   1.00   &   1.00   &   1.00\\
        \midrule
        Bandpass   & 1.00 &   1.00  &   1.00 & 1.00 &   1.00  &  1.00 &   1.00   &   1.00   \\
            Highpass  & 0.71 &  0.68  &   0.70  & 0.70 &  0.70  &  0.64   &  0.52 &  0.52     \\
        Lowpass    & 1.00 &  0.99 &  1.00 & 1.00 & 1.00 &  1.00  &  1.00  &  1.00  \\
            Boost     & 1.00  &  1.00  &  1.00 & 1.00  &  1.00 &  1.00 &     1.00     &      1.00   \\
            Duck      & 1.00  &  1.00  &  1.00 &1.00 &  1.00 &   1.00  &   1.00   &     1.00    \\
            Echo   & 1.00  &  1.00 &  1.00 &1.00 & 1.00 &  1.00  &      1.00    &     1.00     \\
            Pink   & 0.99 &   1.00  & 0.99 & 1.00 &   0.99 &  1.00 &      1.00    &    1.00  \\
            White  & 1.00 &    1.00  & 1.00 & 1.00 &  1.00 & 1.00   &  1.00   &  1.00   \\
        Fast (x1.25)  & 0.97 & 0.98 & 0.99  & 0.98 & 0.99 & 0.98 & 0.87 & 0.87 \\
            Smooth    &  0.96  &  0.99  &   0.99  &    0.99    &      0.99          & 0.99 &  0.98  &    0.98   \\
            Resample  & 1.00 &  1.00 &  1.00 & 1.00 &    1.00  & 1.00 &   1.00    &   1.00   \\
                AAC & 0.99 &  0.99  &  0.99  & 0.99 &  0.99  &   0.97  &  0.99   &     0.98    \\
                MP3 & 0.99 &  0.99   &  0.99 & 0.99 & 0.99  &    0.97  &  0.99    & 1.00  \\
            Encodec   & 0.97 &  0.98   &  0.99 & 0.99 & 0.98 &   0.96     &  0.95    & 0.95   \\
            \midrule
            Average   &  0.97 &  0.97  & 0.98  & 0.98 & 0.98 & 0.97 & 0.95 & 0.95  \\
            \bottomrule
        \end{tabular}
\end{table*}


\begin{table}[t!]
    \centering
    \caption{
        Audio quality and intelligibility evaluations on AI-generated speech from various models and languages.
    }
    \label{chap4/tab:ood_metrics}
    \footnotesize
    \renewcommand{\arraystretch}{1.2}
    \begin{tabular}{cccccc}
        \toprule
        Model & Dataset & SI-SNR & PESQ & STOI & ViSQOL \\
        \midrule
        \multirow{3}{*}{\scriptsize \rotatebox[origin=c]{90}{AudioSeal}} & Seam. (Deu)       & 23.35 & 4.244 & 0.999 & 4.688 \\
        & Seam. (Fr)        & 24.02 & 4.199 & 0.998 & 4.669 \\
        & Voicebox             & 25.23 & 4.449 & 0.998 & 4.800 \\
        \midrule
        \multirow{3}{*}{\scriptsize \rotatebox[origin=c]{90}{WavMark}} & Seam. (Deu)    & 38.93 & 3.982 & 0.999 & 4.515 \\
        & Seam. (Fr)     & 39.06 & 3.959 & 0.999 & 4.506 \\
        & Voicebox          & 39.63 & 4.211 & 0.998 & 4.695 \\
        \bottomrule
    \end{tabular}
\end{table}

We employ the same set of augmentations and observe very similar detection results, as demonstrated in Tab.~\ref{chap4/tab:ood_data}.
Interestingly, even though we did not train our model on AI-generated speech, we sometimes notice a slight improvement in performance compared to our test data. 
No sample is misclassified among the 10k samples that comprise each of our out-of-distribution (OOD) datasets.
We also provide perceptual metrics results on some OOD data in Tab.~\ref{chap4/tab:ood_metrics}.
We observe that AudioSeal performs similarly on these datasets, with a slight decrease in performance compared to our original dataset.
We explain the decrease by the \ploss\ which makes the watermark hidden in the same frequency bands as English speech, which might not be the case for other languages or audio types.

\paragraph{Fake vs. real datasets.}
We also evaluate AudioSeal on three additional datasets containing real human speech: AudioSet~\citep{gemmeke2017audio}, ASVspoof~\citep{liu2023asvspoof}, and FakeAVCeleb~\citep{khalid2021fakeavceleb}, and observe similar performance, see Tab.~\ref{chap4/tab:other_datasets}.


\begin{table}[t!]
    \centering
    \caption{Evaluation of the detection performances on different datasets. AudioSet is an environmental sounds dataset while ASVspoof~\citep{liu2023asvspoof} and FakeAVCeleb~\citep{khalid2021fakeavceleb} are deep-fake detection datasets.}
    \label{chap4/tab:other_datasets}
    \footnotesize
    \begin{tabular}{l *{2}{l}}
        \toprule
        Dataset & Acc. \aux{TPR/FPR} & AUC \\
        \midrule
        Audioset & 0.9992 \aux{0.9996/0.0011} & 1.0 \\
        ASVspoof & 1.0 \aux{1.0/0.0} & 1.0 \\
        FakeAVCeleb & 1.0 \aux{1.0/0.0} & 1.0 \\
        \bottomrule
    \end{tabular}
\end{table}











\begin{figure}[b!]
    \centering
    \includegraphics[width=1.0\textwidth]{chapter-4/figs/appendix/augmentation_curves.pdf}
    \caption{
        Accuracy of the detector on augmented samples with respect to the strength of the augmentation.
    }
    \label{chap4/fig:app_augmentation_curves}
\end{figure}


\subsection{More robustness results}\label{chap4/app:robustness}

We plot the detection accuracy against the strength of multiple augmentations in Fig.~\ref{chap4/fig:app_augmentation_curves}. 
AudioSeal outperforms WavMark for most augmentations at the same strength.
However, for highpass filters above our training range (500Hz) WavMark has a much better detection accuracy.
Our system's TF-loudness loss embeds the watermark where human speech carries the most energy, typically lower frequencies, due to auditory masking. 
This contrasts with WavMark, which places the watermark in higher frequency bands.
Embedding the watermark in lower frequencies is advantageous. 
For example, speech remains audible with a lowpass filter at 1500 Hz, but not with a highpass filter at the same frequency. 
This difference is measurable with PESQ in relation to the original audio, making it more beneficial to be robust against a lowpass filter at a 1500 Hz cut-off than a highpass filter at the same cut-off:

\begin{center}
    \footnotesize
    \begin{tabular}{cccc}
        Filter Type & PESQ & AudioSeal & WavMark \\
        \midrule
        Highpass 1500Hz & 1.85 \xmarkg & 0.7 & 1.0 \\
        Lowpass 1500Hz & 2.93 \cmarkg & 1.0 & 0.7 \\
    \end{tabular}
\end{center}

\section{Conclusion}

In this chapter we introduce a proactive method for the detection, localization, and attribution of AI-generated speech. 
AudioSeal redesigns audio watermarking to be specific to localized detection rather than data hiding. 
This removes the dependency on slow brute force algorithms, traditionally used to encode and decode audio watermarks.

A key advantage of AudioSeal is its practicality. 
Compared to the previous~\autoref{chapter:stable-signature}, it is post-hoc, meaning it can be applied to any existing audio content without the need to retrain the generator.
It stands as a ready-to-deploy solution for watermarking in voice synthesis APIs (or to authenticate real speech).
This is important for large-scale content provenance on social media and for detecting and eliminating incidents, enabling swift action on instances like the US voters' deepfake case~\citep{murphy2024biden}, long before they spread.
Note that \citet{san2024latent} introduce the in-model counterpart of this chapter.
The authors show how to use AudioSeal to watermark the weights of an audio autoregressive model~\citep{copet2023simple} through a specific watermarking of its training data, although it is less practical and less robust than post-hoc watermarking.

Two key limitations are worth mentioning.
First, the \pval\ of the watermark detection is not mathematically grounded, which makes it hard to scale to extremely low FPRs -- as done in previous and next chapters -- and to provide a theoretical guarantee of the watermark's presence.
Second, the watermark is still brittle to adversarial attacks (\eg, using new neural audio compression algorithms).
Therefore, AudioSeal should be seen as a filtering tool to flag suspicious content, rather than a definitive proof of fakeness  or authenticity.



