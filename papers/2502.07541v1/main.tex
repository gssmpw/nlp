%%
%% This is file `sample-manuscript.tex',
%% generated with the docstrip utility.
%%
%% The original source files were:
%%
%% samples.dtx  (with options: `all,proceedings,bibtex,manuscript')
%% 
%% IMPORTANT NOTICE:
%% 
%% For the copyright see the source file.
%% 
%% Any modified versions of this file must be renamed
%% with new filenames distinct from sample-manuscript.tex.
%% 
%% For distribution of the original source see the terms
%% for copying and modification in the file samples.dtx.
%% 
%% This generated file may be distributed as long as the
%% original source files, as listed above, are part of the
%% same distribution. (The sources need not necessarily be
%% in the same archive or directory.)
%%
%%
%% Commands for TeXCount
%TC:macro \cite [option:text,text]
%TC:macro \citep [option:text,text]
%TC:macro \citet [option:text,text]
%TC:envir table 0 1
%TC:envir table* 0 1
%TC:envir tabular [ignore] word
%TC:envir displaymath 0 word
%TC:envir math 0 word
%TC:envir comment 0 0
%%
%%
%% The first command in your LaTeX source must be the \documentclass
%% command.
%%
%% For submission and review of your manuscript please change the
%% command to \documentclass[manuscript, screen, review]{acmart}.
%%
%% When submitting camera ready or to TAPS, please change the command
%% to \documentclass[sigconf]{acmart} or whichever template is required
%% for your publication.
%%
%%
\documentclass[manuscript,screen]{acmart}

%%
%% \BibTeX command to typeset BibTeX logo in the docs
\AtBeginDocument{%
  \providecommand\BibTeX{{%
    Bib\TeX}}}

%% Rights management information.  This information is sent to you
%% when you complete the rights form.  These commands have SAMPLE
%% values in them; it is your responsibility as an author to replace
%% the commands and values with those provided to you when you
%% complete the rights form.
\setcopyright{acmlicensed}
\copyrightyear{2025}
\acmYear{2025}
\acmDOI{XXXXXXX.XXXXXXX}

%% These commands are for a PROCEEDINGS abstract or paper.
\acmConference[Conference acronym 'XX]{Make sure to enter the correct
  conference title from your rights confirmation emai}{June 03--05,
  2018}{Woodstock, NY}
%%
%%  Uncomment \acmBooktitle if the title of the proceedings is different
%%  from ``Proceedings of ...''!
%%
%%\acmBooktitle{Woodstock '18: ACM Symposium on Neural Gaze Detection,
%%  June 03--05, 2018, Woodstock, NY}
\acmISBN{978-1-4503-XXXX-X/18/06}


%%
%% Submission ID.
%% Use this when submitting an article to a sponsored event. You'll
%% receive a unique submission ID from the organizers
%% of the event, and this ID should be used as the parameter to this command.
%%\acmSubmissionID{123-A56-BU3}

%%
%% For managing citations, it is recommended to use bibliography
%% files in BibTeX format.
%%
%% You can then either use BibTeX with the ACM-Reference-Format style,
%% or BibLaTeX with the acmnumeric or acmauthoryear sytles, that include
%% support for advanced citation of software artefact from the
%% biblatex-software package, also separately available on CTAN.
%%
%% Look at the sample-*-biblatex.tex files for templates showcasing
%% the biblatex styles.
%%

%%
%% The majority of ACM publications use numbered citations and
%% references.  The command \citestyle{authoryear} switches to the
%% "author year" style.
%%
%% If you are preparing content for an event
%% sponsored by ACM SIGGRAPH, you must use the "author year" style of
%% citations and references.
%% Uncommenting
%% the next command will enable that style.

%\citestyle{acmauthoryear}


%%
%% end of the preamble, start of the body of the document source.

\usepackage[utf8]{inputenc}
\usepackage[inline]{enumitem}
% This is not strictly necessary, and may be commented out.
% However, it will improve the layout of the manuscript,
% and will typically save some space.
\usepackage{microtype}


\definecolor{blue_comment}{rgb}{0.2, 0.4, 0.7}
\newcommand{\oana}[1]{\textcolor{blue_comment}{\textbf{Oana:} #1}}
\definecolor{green_comment}{rgb}{0.3, 0.5, 0.3}
\newcommand{\tom}[1]{\textcolor{green_comment}{\textbf{Tom:} #1}}
\definecolor{red_comment}{rgb}{0.7, 0.3, 0.4}
\newcommand{\fms}[1]{\textcolor{red_comment}{\textbf{Fabian:} #1}}
\newcommand{\ignore}[1]{}
%hide comments
\renewcommand{\oana}[1]{}
\renewcommand{\tom}[1]{}
\renewcommand{\fms}[1]{}

% This is also not strictly necessary, and may be commented out.
% However, it will improve the aesthetics of text in
% the typewriter font.
%\usepackage{inconsolata}
%\usepackage{amsmath}

\usepackage{graphicx}

\usepackage{subcaption}
\usepackage{float}
\usepackage{booktabs}

\usepackage{amsthm}
%\usepackage{amssymb} % Load the amssymb package to use the \checkmark command
\usepackage{pifont}
\newcommand{\cmark}{{\color{green_comment}\ding{51}}}%
\newcommand{\gcmark}{{\color{gray}\ding{51}}}
\newcommand{\xmark}{{\color{red_comment}\ding{55}}}%

\newcommand{\task}[1]{\paragraph{Task Description}  #1}
\newcommand{\zeroshot}[1]{\paragraph{Unsupervised Solutions}  #1}
\newcommand{\datasets}[1]{\paragraph{Annotated Datasets}  #1}
\newcommand{\solutions}[1]{\paragraph{Supervised Solutions}  #1}

\usepackage{longtable}

\newtheorem{definition}{Definition}

\usepackage{amsthm}

\newtheorem*{remark}{Task}

\usepackage{tikz}
\usepackage{subcaption}
\usetikzlibrary{trees, positioning, arrows.meta}
\usepackage{multicol}
\usepackage{multirow}

\setlength{\tabcolsep}{4pt}
\usepackage{adjustbox}

\usepackage{wrapfig}
\usepackage{longtable}

\usepackage{array}
\newcolumntype{P}[1]{>{\centering\arraybackslash}p{#1}}



\begin{document}

%%
%% The "title" command has an optional parameter,
%% allowing the author to define a "short title" to be used in page headers.

\title{Corporate Greenwashing Detection in Text - a Survey}
% Fabian: reads like a transition from "Green Claim" to "Greenwashing Detection". Could we do simpler "Greenwashing detection: a survey"?


%%
%% The "author" command and its associated commands are used to define
%% the authors and their affiliations.
%% Of note is the shared affiliation of the first two authors, and the
%% "authornote" and "authornotemark" commands
%% used to denote shared contribution to the research.
\author{Tom Calamai}
%\authornote{Note}
\email{tom.calamai@amundi.com}
\orcid{todo}
\affiliation{%
  \institution{Amundi, Télécom Paris, Inria, Institut Polytechnique de Paris}
  %\city{Palaiseau}
 \country{France}
}

\author{Oana Balalau}
\email{oana.balalau@inria.fr}
\orcid{0000-0003-1469-3664}
\affiliation{%
  \institution{Inria, Institut Polytechnique de Paris}
  %\city{Palaiseau}
  \country{France}
}

\author{Théo Le Guenedal}
%\authornote{Note}
\email{theo.leguenedal-ext@amundi.com}
\orcid{todo}
\affiliation{%
  \institution{Amundi}
  %\city{Palaiseau}
  \country{France}
}
\author{Fabian M. Suchanek}
%\authornote{Note}
\email{fabian.suchanek@telecom-paris.fr}
\orcid{0000-0001-7189-2796}

\affiliation{%
  \institution{Télécom Paris, Institut Polytechnique de Paris}
  %\city{Palaiseau}
  \country{France}
}

%%
%% By default, the full list of authors will be used in the page
%% headers. Often, this list is too long, and will overlap
%% other information printed in the page headers. This command allows
%% the author to define a more concise list
%% of authors' names for this purpose.
\renewcommand{\shortauthors}{Calamai et al.}

%%
%% The abstract is a short summary of the work to be presented in the
%% article.
\begin{abstract}
%Climate change is greatly affecting our planet and our livelihoods and its impact is expected only to increase. In this survey, we focus on an % Fabian: was "major". I thin this would be an overselling... % Fabian: on a second thought, it seems cheap to hijack the climate change theme here. We should not sell ourselves as fighters against climate change...
Greenwashing is an effort to mislead the public about the carbon transition of an entity, such as a state or company. We provide a comprehensive survey of the scientific literature addressing natural language processing methods to identify potentially misleading climate-related corporate communications, indicative of greenwashing. We break the detection of greenwashing into intermediate tasks, and review the state-of-the-art approaches for each of them. We discuss datasets, methods, and results, as well as limitations and open challenges. We also provide an overview of how far the field has come as a whole, and point out future research directions.
\end{abstract}

%%
%% The code below is generated by the tool at http://dl.acm.org/ccs.cfm.
%% Please copy and paste the code instead of the example below.
%%

\begin{CCSXML}
<ccs2012>
   <concept>
       <concept_id>10010147.10010178.10010179</concept_id>
       <concept_desc>Computing methodologies~Natural language processing</concept_desc>
       <concept_significance>500</concept_significance>
       </concept>
 </ccs2012>
\end{CCSXML}

\ccsdesc[500]{Computing methodologies~Natural language processing}

%%
%% Keywords. The author(s) should pick words that accurately describe
%% the work being presented. Separate the keywords with commas.
\keywords{natural language processing; greenwashing detection;  corporate climate-related greenwashing; climate pretrained models; climate topic detection; green claim detection; climate QA; climate-related deceptive techniques; enviromental performance prediction}

\received{20 February 2007}
\received[revised]{12 March 2009}
\received[accepted]{5 June 2009}

%%
%% This command processes the author and affiliation and title
%% information and builds the first part of the formatted document.
\maketitle

%\tom{TOC temporary: should be removed before publishing}

%\clearpage
%\tableofcontents
%\clearpage

\section{Introduction}

Despite the remarkable capabilities of large language models (LLMs)~\cite{DBLP:conf/emnlp/QinZ0CYY23,DBLP:journals/corr/abs-2307-09288}, they often inevitably exhibit hallucinations due to incorrect or outdated knowledge embedded in their parameters~\cite{DBLP:journals/corr/abs-2309-01219, DBLP:journals/corr/abs-2302-12813, DBLP:journals/csur/JiLFYSXIBMF23}.
Given the significant time and expense required to retrain LLMs, there has been growing interest in \emph{model editing} (a.k.a., \emph{knowledge editing})~\cite{DBLP:conf/iclr/SinitsinPPPB20, DBLP:journals/corr/abs-2012-00363, DBLP:conf/acl/DaiDHSCW22, DBLP:conf/icml/MitchellLBMF22, DBLP:conf/nips/MengBAB22, DBLP:conf/iclr/MengSABB23, DBLP:conf/emnlp/YaoWT0LDC023, DBLP:conf/emnlp/ZhongWMPC23, DBLP:conf/icml/MaL0G24, DBLP:journals/corr/abs-2401-04700}, 
which aims to update the knowledge of LLMs cost-effectively.
Some existing methods of model editing achieve this by modifying model parameters, which can be generally divided into two categories~\cite{DBLP:journals/corr/abs-2308-07269, DBLP:conf/emnlp/YaoWT0LDC023}.
Specifically, one type is based on \emph{Meta-Learning}~\cite{DBLP:conf/emnlp/CaoAT21, DBLP:conf/acl/DaiDHSCW22}, while the other is based on \emph{Locate-then-Edit}~\cite{DBLP:conf/acl/DaiDHSCW22, DBLP:conf/nips/MengBAB22, DBLP:conf/iclr/MengSABB23}. This paper primarily focuses on the latter.

\begin{figure}[t]
  \centering
  \includegraphics[width=0.48\textwidth]{figures/demonstration.pdf}
  \vspace{-4mm}
  \caption{(a) Comparison of regular model editing and EAC. EAC compresses the editing information into the dimensions where the editing anchors are located. Here, we utilize the gradients generated during training and the magnitude of the updated knowledge vector to identify anchors. (b) Comparison of general downstream task performance before editing, after regular editing, and after constrained editing by EAC.}
  \vspace{-3mm}
  \label{demo}
\end{figure}

\emph{Sequential} model editing~\cite{DBLP:conf/emnlp/YaoWT0LDC023} can expedite the continual learning of LLMs where a series of consecutive edits are conducted.
This is very important in real-world scenarios because new knowledge continually appears, requiring the model to retain previous knowledge while conducting new edits. 
Some studies have experimentally revealed that in sequential editing, existing methods lead to a decrease in the general abilities of the model across downstream tasks~\cite{DBLP:journals/corr/abs-2401-04700, DBLP:conf/acl/GuptaRA24, DBLP:conf/acl/Yang0MLYC24, DBLP:conf/acl/HuC00024}. 
Besides, \citet{ma2024perturbation} have performed a theoretical analysis to elucidate the bottleneck of the general abilities during sequential editing.
However, previous work has not introduced an effective method that maintains editing performance while preserving general abilities in sequential editing.
This impacts model scalability and presents major challenges for continuous learning in LLMs.

In this paper, a statistical analysis is first conducted to help understand how the model is affected during sequential editing using two popular editing methods, including ROME~\cite{DBLP:conf/nips/MengBAB22} and MEMIT~\cite{DBLP:conf/iclr/MengSABB23}.
Matrix norms, particularly the L1 norm, have been shown to be effective indicators of matrix properties such as sparsity, stability, and conditioning, as evidenced by several theoretical works~\cite{kahan2013tutorial}. In our analysis of matrix norms, we observe significant deviations in the parameter matrix after sequential editing.
Besides, the semantic differences between the facts before and after editing are also visualized, and we find that the differences become larger as the deviation of the parameter matrix after editing increases.
Therefore, we assume that each edit during sequential editing not only updates the editing fact as expected but also unintentionally introduces non-trivial noise that can cause the edited model to deviate from its original semantics space.
Furthermore, the accumulation of non-trivial noise can amplify the negative impact on the general abilities of LLMs.

Inspired by these findings, a framework termed \textbf{E}diting \textbf{A}nchor \textbf{C}ompression (EAC) is proposed to constrain the deviation of the parameter matrix during sequential editing by reducing the norm of the update matrix at each step. 
As shown in Figure~\ref{demo}, EAC first selects a subset of dimension with a high product of gradient and magnitude values, namely editing anchors, that are considered crucial for encoding the new relation through a weighted gradient saliency map.
Retraining is then performed on the dimensions where these important editing anchors are located, effectively compressing the editing information.
By compressing information only in certain dimensions and leaving other dimensions unmodified, the deviation of the parameter matrix after editing is constrained. 
To further regulate changes in the L1 norm of the edited matrix to constrain the deviation, we incorporate a scored elastic net ~\cite{zou2005regularization} into the retraining process, optimizing the previously selected editing anchors.

To validate the effectiveness of the proposed EAC, experiments of applying EAC to \textbf{two popular editing methods} including ROME and MEMIT are conducted.
In addition, \textbf{three LLMs of varying sizes} including GPT2-XL~\cite{radford2019language}, LLaMA-3 (8B)~\cite{llama3} and LLaMA-2 (13B)~\cite{DBLP:journals/corr/abs-2307-09288} and \textbf{four representative tasks} including 
natural language inference~\cite{DBLP:conf/mlcw/DaganGM05}, 
summarization~\cite{gliwa-etal-2019-samsum},
open-domain question-answering~\cite{DBLP:journals/tacl/KwiatkowskiPRCP19},  
and sentiment analysis~\cite{DBLP:conf/emnlp/SocherPWCMNP13} are selected to extensively demonstrate the impact of model editing on the general abilities of LLMs. 
Experimental results demonstrate that in sequential editing, EAC can effectively preserve over 70\% of the general abilities of the model across downstream tasks and better retain the edited knowledge.

In summary, our contributions to this paper are three-fold:
(1) This paper statistically elucidates how deviations in the parameter matrix after editing are responsible for the decreased general abilities of the model across downstream tasks after sequential editing.
(2) A framework termed EAC is proposed, which ultimately aims to constrain the deviation of the parameter matrix after editing by compressing the editing information into editing anchors. 
(3) It is discovered that on models like GPT2-XL and LLaMA-3 (8B), EAC significantly preserves over 70\% of the general abilities across downstream tasks and retains the edited knowledge better.
\section{Related Work}

\subsection{Personalization and Role-Playing}
Recent works have introduced benchmark datasets for personalizing LLM outputs in tasks like email, abstract, and news writing, focusing on shorter outputs (e.g., 300 tokens for product reviews \citep{kumar2024longlamp} and 850 for news writing \citep{shashidhar-etal-2024-unsupervised}). These methods infer user traits from history for task-specific personalization \citep{sun-etal-2024-revealing, sun-etal-2025-persona, pal2024beyond, li2023teach, salemi2025reasoning}. In contrast, we tackle the more subjective problem of long-form story writing, with author stories averaging 1500 tokens. Unlike prior role-playing approaches that use predefined personas (e.g., Tony Stark, Confucius) \citep{wang-etal-2024-rolellm, sadeq-etal-2024-mitigating, tu2023characterchat, xu2023expertprompting}, we propose a novel method to infer story-writing personas from an author’s history to guide role-playing.


\subsection{Story Understanding and Generation}  
Prior work on persona-aware story generation \citep{yunusov-etal-2024-mirrorstories, bae-kim-2024-collective, zhang-etal-2022-persona, chandu-etal-2019-way} defines personas using discrete attributes like personality traits, demographics, or hobbies. Similarly, \citep{zhu-etal-2023-storytrans} explore story style transfer across pre-defined domains (e.g., fairy tales, martial arts, Shakespearean plays). In contrast, we mimic an individual author's writing style based on their history. Our approach differs by (1) inferring long-form author personas—descriptions of an author’s style from their past works, rather than relying on demographics, and (2) handling long-form story generation, averaging 1500 tokens per output, exceeding typical story lengths in prior research.
\section{Preliminaries} % Fabian: again, I propose to upgrade to a section to be in line with what readers would expect
\label{sec:definitions}

\subsection{Legal Context}
In this section, we provide the legal context for the definition of greenwashing. 
We first present international and national laws, followed by how they are applied in the corporate context. 


\paragraph{International and Nation Level Legislation and Guidelines on Climate Change Mitigation}
The first international summit on the effect of humans on the environment was the United Nations Conference on the Human Environment, held in 1972 in Sweden~\cite{timelineclimate}. The \textbf{Intergovernmental Panel on Climate Change (IPCC)} was created in 1988 and released its first report in 1990 ~\cite{change1990ipcc}. It was only in 1992 that the first treaty between countries was signed, the United Nations Framework Convention on Climate Change (UNFCCC)~\cite{timelineclimate}. The treaty only encouraged countries to reduce their emissions, and in 1997, the Kyoto Protocol set commitments for the countries to follow~\cite{kyotoprotocol}. The 36 countries that participated in the Protocol reduced their emissions in 2008–2012 by a large margin in respect to the levels in 1990; however, the global emissions increased by $32\%$ in the same period. 
Since the creation of UNFCCC, the nations, i.e., the parties, have met yearly at the ``Conference of the Parties'' (\textbf{COP}). 

The three main directions of \textbf{climate change policy} are reducing greenhouse emissions, promoting renewable energy, and improving energy efficiency. One major milestone was the introduction in 2005 of the European Union Emissions Trading System, the world's first large-scale emissions trading scheme. The trading system limits the amount of CO2 emitted by European industries and covers $46\%$ of the EU's CO2 emissions. It is estimated that the trading system reduced CO2 emissions in the EU by $3.8\%$ between 2008 and 2016~\cite{bayer2020european}. At COP21, in 2015, 194 nations plus the European Union signed the Paris Agreement, a treaty by which the nations commit to keep the rise in global surface temperature below 2 °C (3.6 °F) above pre-industrial levels. To achieve this goal, each country sends a national climate action plan every five years, and the parties assess the collective progress made towards achieving the climate goals. 
The first such assessment took place at COP28, where it was established that an important direction to achieve the long-term goals of countries was to transition away from fossil fuels to renewable energy. In 2021, the European Climate Law was adopted: the EU commits to reducing its emissions by at least $55\%$ by 2030 with respect to the 1990 levels and becoming climate neutral by 2050~\cite{euclimatelaw}. 
Similarly, countries such as Canada~\cite{canadaclimatelaw}, Taiwan~\cite{taiwanclimatelaw}, South Korea~\cite{koreaclimatelaw}, and Australia~\cite{australiaclimatelaw}, among others, aim to achieve carbon neutrality by 2050 and have passed laws setting this goal.

\paragraph{Corporate Level Laws and Guidelines} Laws and regulations at international and national levels have repercussions on companies that must comply or risk fines. However, assessing if a company has taken the necessary steps is not trivial, given that some laws refer to how a company will operate in the future, for example, by polluting less. Even without laws, investors can be concerned about environmental or social issues; hence, companies have used voluntary disclosures for a long time.  

One of the first organizations to provide standards for reporting on climate change or social aspects is the  Global Reporting Initiative (GRI), with its first guidelines published in 2000~\cite{grihistory}.
In 2015, the Financial Stability Board and the Group of 20 created the \textbf{Task Force for Climate-related Financial Disclosure (TCFD)} guidelines on disclosure in response to shortcomings of COP21, in particular the lack of standards climate-related disclosure~\cite{enwiki:1257716178}.
In 2021, the International Sustainability Standards Board (ISSB) was created to establish standards for climate-related disclosure, and starting in 2024, the standard released by this board will be applied worldwide~\cite{ifrs}. ISSB standards were aligned with GRI disclosure standards to make them complementary and interoperable~\cite{gri}, while the ISSB standards are expected to take over TCFD~\cite{ifrs}.
Differently from the voluntary reporting standards of GRI, TCFD, and ISSB, the Corporate sustainability reporting law in the EU required the creation of the European Sustainability Reporting Standards (ESRS), which are mandatory for companies subject to EU law~\cite{esrs}. ESRS has high interoperability with GRI and ISSB. GRI, TCFD, ESRS and ISSB fall under the umbrella term of environmental, social, and governance (ESG) guidelines for company disclosure. 
The European Union created its first law obliging companies to provide non-financial disclosure reporting in 2014, the Non-Financial Reporting Directive, which focused disclosure on environmental and social aspects. In 2023, the EU expanded this legislation via the Corporate Sustainability Reporting Law. Some European countries anticipated this legislation with their own, for example, the 2001 New Economic Regulations Act in France. Switzerland, which is not part of the EU, has imposed a mandatory TCFD disclosure for large public companies, banks, and insurance companies starting from January 2024~\cite{disclosureswiss}, and similar laws exist in New Zealand~\cite{disclosurezealand}. While not compulsory, initiatives like the Carbon Disclosure Project (CDP) have played a pivotal role by standardizing responses related to the climate disclosure of a company via structured questionnaires, thereby facilitating more systematic and comparable reporting.  


%Science-based targets (SBTi) \fms{what is this? an organization? created by whom? Governemnt?} was created in 2015 and it does not only rely on voluntary disclosure but also works with companies to set targets aligned with the Paris Agreement. SBTi provides a list of companies that set a target\footnote{\url{https://sciencebasedtargets.org/companies-taking-action}}, while also detailing sector-specific documentation on the methodology to set the target. Oana: this can go away, it is indeed one initiative of many, and it is not related to a legal framework as it is voluntary

\subsection{Definition of Greenwashing}

A widely cited and comprehensive definition, synthesizing those commonly found in the literature, is provided by the Oxford English Dictionary~\cite{GreenwashMeaningsEtymology2023}.

\begin{definition}\textbf{Greenwashing:}
    \label{def:greenwashing}
    To mislead the public (or to counter public or media concerns) by falsely representing a person, company, product, etc., as environmentally responsible.
\end{definition}


While individuals, companies, or countries can all engage in greenwashing, we will focus on climate-related greenwashing by companies in this survey, with the following definition: 
\begin{definition}
\label{def:greenwashing2}
   \textbf{Corporate climate-related greenwashing:} To mislead the public into falsely representing the effort made by a company to achieve its carbon transition.
\end{definition}

As climate-related disclosures face increasing regulation and greenwashing poses significant risks to a company's reputation, some companies adopt a strategy of silence, avoiding discussions about their environmental impact. This deliberate lack of communication is known as \textit{greenhushing}~\cite{Letzing}. These definitions imply that greenwashing is a deliberate act; however, in many cases, it results from an error or miscommunication by companies genuinely trying to showcase their sustainability efforts, and that are trying to best follow disclosure standards.

As shown by Definitions \ref{def:greenwashing} and \ref{def:greenwashing2}, greenwashing is not defined by easily identifiable properties but as a general concept. Since the concept is so unspecific, researchers focused on components indicative of potentially misleading communications but easier to define.  
Because of this, in this review we are mentioning ``greenwashing'' explicitly, but also paraphrasing it as ``misleading communications'', ``misrepresentation of the company's environmental impact/stance/performance'', or mentioning only components of it such as ``cheap talk'', ``selective disclosure/transparency'', ``deceptive techniques'', ``biased narrative''. They should all be understood in the context of climate-related misleading communications, as components associated or indicative of potential greenwashing even if they are not synonymous.

% Climate-related disclosure is becoming increasingly regulated, and guidelines and recommendations are numerous, which might help distinguish greenwashing from facts and information. However, the obligations and recommendations are not accessible to those without expertise. They are numerous and spread across multiple documents in different organizations, which may be national or supranational. Therefore, NLP techniques are increasingly employed to analyze and understand large quantities of reports and company communications.

%\fms{The following section seems out of place here, too. In particular, it is not clear why the structure of the text does not match the structure of the sections. Let me try to add an introductory sentence to the next section, so that the following subsection can go away (also in the interest of space)} Oana: I will incorporate part of the following in the introduction, when we describe the sections. 

\begin{comment}
    
%This definition has been rendered more concrete by the Green Claims Directive proposed in 2023 by the European Union~\cite{eugreenclaims}. It defines as greenwashing the practices of ``making an environmental claim related to future environmental performance without clear, objective and verifiable commitments and targets and an independent monitoring system'', ``displaying a sustainability label which is not based on a certification scheme or not established by public authorities'', ``making an environmental claim about the entire product when it concerns only a certain aspect of the product'', ``making a generic environmental claim for which the trader is not able to demonstrate recognized excellent environmental performance relevant to the claim'', or ``presenting requirements imposed by law on all products in the relevant product category on the Union market as a distinctive feature of the trader’s offer''.  Oana: I finnaly removed this as we already mention it in the introduction and it is not actually a definition that it is used in any of the papers we looked at, as it is a new law. - it allows us also to gain space.

%This the definition we will use for the rest of this survey.
%We provide a comprehensive survey of the scientific literature addressing the automated detection of greenwashing in textual data. 
%Of particular importance is \textbf{corporate greenwashing, with a focus on climate-related greenwashing}, i.e., greenwashing that misleads the public about the effort made by a company to achieve its carbon transition. 
%Given the urgency of climate change deadlines, we will emphasize more approaches that can deal with climate-related greenwashing. 

\subsection{Tasks addressed in survey}

Understanding and analyzing climate-related corporate communication has become an essential task in the context of increasing environmental scrutiny. 
However, a relatively recent concept, greenwashing complicates this landscape by introducing the risk of misleading or exaggerating claims about environmental commitments. Identifying greenwashing is inherently complex, as it requires addressing multiple dimensions of communication. Consequently, many works focus on more straightforward, more specific tasks as stepping stones toward the broader goal of detecting greenwashing. This review includes intermediary tasks that contribute to identifying potentially misleading climate-related communication. 

\paragraph{Identifying company content on climate-change:}
Detecting communication that is explicitly or implicitly related to climate topics. This task encompasses identifying climate-related statements, analyzing topics aligned with frameworks like the Task Force on Climate-related Financial Disclosures (TCFD), Environmental, Social, and Governance (ESG) factors, and other sustainability-related themes , and assessing the presence of green claims (see Sections: \ref{sec:climate-related topic}, \ref{sec:sub-topics}, \ref{sec:green claim}).

%\paragraph{Identifying subtopics in company content on climate-change:}
%Understanding the content of corporate communication requires determining the focus of the discussion. This includes analyzing topics aligned with frameworks like the Task Force on Climate-related Financial Disclosures (TCFD), Environmental, Social, and Governance (ESG) factors, and other sustainability-related themes (see Section: \ref{sec: tcfd}, \ref{sec:esg}, \ref{sec:sub-topics}).

\paragraph{Analyzing how companies communicate about climate change :}
Beyond identifying content, it is critical to evaluate the style and intent of communication. This involves examining sentiment, argumentation quality, deceptive practices, and stance to discern the underlying tone and authenticity of the message (see Sections: \ref{sec: climate risk}, \ref{sec: claim characteristics}, \ref{sec:stance detection}, \ref{sec:qa}, \ref{sec:deceptive}).

Finally, we review all approaches proposed to identify greenwashing, integrating insights from these intermediary tasks to assess their contributions to detecting and mitigating misleading corporate communication in the climate domain. This structured approach enables a deeper understanding of the challenges and potential solutions to effectively address greenwashing.

\end{comment}
\section{Intermediary tasks}
\label{sec:intermediary tasks}

%Although some works do not explicitly focus on greenwashing, their analysis of company communications can still reveal potentially misleading information, which may suggest greenwashing. Consequently, many of these works serve as valuable stepping stones toward the broader goal of detecting greenwashing. We will survey them in this section.

\subsection{Pretraining Models on Climate-Related Text}
\label{sec:domain specific model}

The first step in applying a language model for a given task is typically the pretraining process, which involves training the model on relevant corpora. Although general-purpose models such as BERT~\cite{devlin-etal-2019-bert} and LLaMA~\cite{touvron2023llamaopenefficientfoundation} have demonstrated strong performance in various tasks, they are inherently limited by the knowledge and vocabulary present in their training corpora. This limitation can lead to suboptimal results when these models are applied to highly specialized domains with unique terminology and concepts. Hence, domain-specific models are usually trained on targeted datasets to improve their performance in niche areas.


\task A language model is pretrained on climate-related text, usually on the classical tasks of masked token prediction and next sentence prediction. The goal is to produce a domain-specific pre-trained language model that will be subsequently fine-tuned to specific tasks. There are two approaches: \begin{enumerate*} \item training a model with a domain-specific corpus from scratch, or \item further-training\footnote{Further-training refers to the process of training a pretrained model on its original task (e.g., next-token prediction or masked-token prediction) using additional domain-specific data to specialize its knowledge for that domain.} a generalist pretrained model on a domain-specific corpus.
\end{enumerate*}


\datasets There have been several efforts to create a climate-related corpus. \citet{nicolas_webersinke_climatebert_2021} gathered climate-related news articles, research abstracts, and corporate reports. 
\citet{vaghefi2022deep} introduced Deep Climate Change, a dataset composed of abstracts of articles from climate scientists, and built a corpus specific to climate research texts. 
\citet{schimanski_bridging_2023} introduced a dataset that focuses on text related to Environment Social and Governance (ESG). Similarly, \citet{Mehra_2022} built a dataset using text from the Knowledge Hub of Accounting for Sustainability for an ESG domain-specific corpus.
More recently, \citet{thulke2024climategpt} proposed a dataset comprising news, publications (abstracts and articles), books, patents, the English Wikipedia, policy and finance-related texts, Environmental Protection Agency documents, and ESG, and IPCC reports. \citet{mullappilly-etal-2023-arabic} proposed a climate-specific multilingual dataset. \citet{yu_climatebug_2024} published a pretraining dataset constructed with annual and sustainable reports from EU banks. 
Unfortunately, to the best of our knowledge, only~\citet{yu_climatebug_2024}'s ClimateBUG-data dataset is publicly available. 

\solutions Based on the above datasets, the models ClimateBERT, ClimateGPT-2, climateBUG-LM and EnvironmentalBERT, ESGBERT were proposed~\cite{nicolas_webersinke_climatebert_2021, vaghefi2022deep, yu_climatebug_2024, schimanski_bridging_2023, Mehra_2022}. 
More recently, generative domain-specific models such as Llama-2 for ClimateGPT~\cite{thulke2024climategpt} or Vicuna for Arabic Mini-ClimateGPT~\cite{mullappilly-etal-2023-arabic} have also been proposed. 
%This new class of generative models is proficient at zero-shot and RAG and will be detailed in Section~\ref{sec:qa}.

\paragraph{Performance of Models} The domain-specific models drastically improve the performance on domain-specific masked-language modeling~\cite{nicolas_webersinke_climatebert_2021, yu_climatebug_2024} and next token prediction~\cite{vaghefi2022deep}. They were also evaluated on domain-specific downstream tasks. These tasks are either based on pre-existing datasets such as ClimateFEVER~\cite{diggelmann_climate-fever_2020} or introduced by the authors, as in ClimateBERT's climate detection~\cite{nicolas_webersinke_climatebert_2021}. At this stage, we evaluate whether fine-tuning enhances downstream performance, with a detailed analysis of the tasks presented in subsequent sections.

\begin{table}[ht]
    \begin{subtable}[t]{0.46\textwidth}
        \centering
        \resizebox{\textwidth}{!}{
        \begin{tabular}{lccc}
            \toprule
            & \textbf{Cl.BERT} & \textbf{DisRoBERTa} & \textbf{NB} \\
            \midrule
            Climatext \cite{spokoyny2023answering, varini_climatext_2020} & 85.14 & \textbf{86.06} & 83.39 \\
            Climate Detection \cite{nicolas_webersinke_climatebert_2021} & \textbf{99.1}$\pm1$ & 98.6$\pm0.8$ & \\
            %Climate Detection \cite{bingler2023cheaptalkspecificitysentiment} & $97$ & & $87$ \\
            Sentiment \cite{nicolas_webersinke_climatebert_2021} & \textbf{83.8}$\pm3.6$ & 82.5$\pm4.6$ & \\ 
            % Sentiment \cite{bingler_cheap_2021} & 80 & & 72 (+5) \\
            Net-zero/Reduction \cite{tobias_schimanski_climatebert-netzero_2023} & \textbf{96.2}$\pm0.4$ & 94.4$\pm0.6$ \\ 
            \bottomrule
            \toprule
             & \textbf{Cl.BERT} & \textbf{DisRoBERTa} & \textbf{TF-IDF} \\
            \midrule
            TCFD classification \cite{sampson_tcfd-nlp_nodate} & 0.852 & 0.819 & \textbf{0.867} \\
            \bottomrule
            \toprule
             & \textbf{Cl.BERT} & \textbf{DisRoBERTa} & \textbf{SVM} \\
            \midrule
            SciDCC \cite{spokoyny2023answering, mishra2021neuralnere} & \textbf{52.97} & 51.13 & 48.02 \\
            ClimaTOPIC \cite{spokoyny2023answering} & \textbf{64.24} & 63.61 & 58.34 \\
            Cl.FEVER (claim) \cite{xiang_dare_2023, diggelmann_climate-fever_2020} & \textbf{76.8} & 72 & \\
            Cl.FEVER (claim) \cite{nicolas_webersinke_climatebert_2021, diggelmann_climate-fever_2020} & \textbf{75.7}$\pm4.4$ & 74.8$\pm3.6$ & \\
            Cl.FEVER (evid.) \cite{spokoyny2023answering, diggelmann_climate-fever_2020} & \textbf{61.54} & \textbf{61.54} & \\
            Cl.Stance \cite{spokoyny2023answering, vaid-etal-2022-towards} & \textbf{52.84} & 52.51 & 42.92 \\
            ClimateEng \cite{spokoyny2023answering, vaid-etal-2022-towards} & 71.83 & \textbf{72.33} & 51.81 \\
            ClimaINS \cite{spokoyny2023answering} & \textbf{84.80} & 84.38 & 86.00 \\
            ClimaBENCH \cite{spokoyny2023answering} & \textbf{69.44} & 69.27 & \\
            Nature \cite{Schimanski2024nature} & \textbf{94.11} & 94.03 & \\
            \bottomrule
            \toprule
            & \textbf{Cl.BERT} & \textbf{SVM+ELMo} & \textbf{SVM+BoW} \\
            \midrule
            Commitment\&Actions \cite{bingler2023cheaptalkspecificitysentiment} & \textbf{81} & 79 & 76 \\
            Specificity \cite{bingler2023cheaptalkspecificitysentiment} & \textbf{77} & 76 & 75 \\
            \bottomrule
            \toprule
             & \textbf{Cl.BERT} & \textbf{DisBERT} & \textbf{SVM} \\
            \midrule
            Env. Claim \cite{stammbach_environmental_2023} & \textbf{83.8} & 83.7 & 70.9 \\
            \bottomrule
            \toprule
             & \textbf{Cl.BERT} & \textbf{DisBERT} & \textbf{LSTM} \\ %\textbf{POS-Bi-LSTM-Attention} \\
            \midrule
            DARE sentiment \cite{xiang_dare_2023} & 87.4 & \textbf{89.9} & 88.2 \\
            \bottomrule
        \end{tabular}
        }
    \caption{F1-scores of climateBERT when compared to a similarly sized models. ROC-AUC for TCFD classification.}
    \label{tab:comparison climatebert}
    \end{subtable}  
    \begin{subtable}[t]{0.53\textwidth}
    
        \vspace{-46mm}

    \resizebox{\textwidth}{!}{
        \begin{tabular}{lcccc}
        \toprule
        & \textbf{RoBERTa} & \textbf{EnvRoBERTa} & \textbf{DisRoBERTa} & \textbf{EnvDisRoB.} \\
        \midrule
        Environment \cite{schimanski_bridging_2023} & 92.35$\pm2.29$ & \textbf{93.19}$\pm1.65$ & 90.97$\pm2$ & $92.35\pm1.65$ \\
        Social \cite{schimanski_bridging_2023} & 89.87$\pm1.35$ & \textbf{91.90}$\pm1.79$ & 90.59$\pm1.03$ & $91.24\pm1.86$ \\
        Governance \cite{schimanski_bridging_2023} & 77.03$\pm1.82$ & 78.48$\pm2.62$ & 76.65$\pm2.39$ & \textbf{78.86}$\pm1.59$ \\
    \bottomrule
        \toprule
         & \textbf{EnvDisRoB.} & \textbf{ClimateBERT} & \textbf{RoBERTa} & \textbf{DisRoBERTa} \\
        \midrule
        Water \cite{Schimanski2024nature} & 94.47$ \pm 1.37$ & \textbf{95.10}$ \pm 1.13$ & 94.55$ \pm 0.86$ & $94.98 \pm 1.16$ \\
        Forest \cite{Schimanski2024nature}   & \textbf{95.37}$ \pm 0.92$ & 95.34$ \pm 0.94$ & 94.78$ \pm 0.48$ & 95.29$ \pm 0.65$ \\
        Biodiversity \cite{Schimanski2024nature}   & \textbf{92.76}$ \pm 1.01$ & 92.49$ \pm 1.03$ & 92.46$ \pm 1.54$ & 92.29$ \pm 1.23$ \\
        Nature \cite{Schimanski2024nature}  & \textbf{94.19}$ \pm 0.81$ & 93.50$ \pm 0.64$ & 93.97$ \pm 0.26$ & 93.55$ \pm 0.72$ \\
        \bottomrule
        \end{tabular}
        }
    \caption{F1-scores of multiple models compared to EnvironmentalBERT (based on RoBERTa and distilRoBERTa).}
    \label{tab:comparison env}
    

    \resizebox{\textwidth}{!}{
        \begin{tabular}{lcc}
        \midrule
         & \textbf{GPT-2} & \textbf{climateGPT-2} \\
        \midrule
        ClimateFEVER \cite{vaghefi2022deep} & 62 & \textbf{72} \\
        \midrule
        \midrule
         & \textbf{LLama-2-Chat (7B)} & \textbf{ClimateGPT (7B)} \\
        \midrule
        Multiple Benchmarks \cite{thulke2024climategpt} & 71.4 & \textbf{77.1} \\
        \midrule
        \end{tabular}
        }
    \caption{F1-scores of ClimateGPT-2 and custom performances (average Accuracy and F1-score over multiple benchmarks) of ClimateGPT compared to similarly sized models.}
    \label{tab:climateGPT}
    %     \centering
    %     \resizebox{\textwidth}{!}{
    %     \begin{tabular}{p{3.3cm}cc}
    %     \toprule
    %     & \textbf{CLBERT} & \textbf{Longformer} \\ \midrule
    %     climatext \cite{spokoyny2023answering, varini_climatext_2020} & 85.14  &  \textbf{87.80} \\
    %     SciDCC \cite{spokoyny2023answering} & 52.97 & \textbf{54.79} \\
    %     LobbyMap (page) \cite{lai_using_2023} & 74.4 & \textbf{76.5} \\
    %     LobbyMap (query) \cite{lai_using_2023} & 48.9 & \textbf{55} \\
    %     LobbyMap (stance) \cite{lai_using_2023} & 39 & \textbf{44.1} \\     \bottomrule
    %     \toprule
    %     & \textbf{CLBERT} & \textbf{BERT} \\ \midrule
    %     Climatext \cite{garridomerchán2023finetuning, varini_climatext_2020} & \textbf{93} & 91 \\
    %     DARE's Sentiment \cite{xiang_dare_2023} & 87.5 & \textbf{93.1} \\
    %     CLFEVER (claim) \cite{diggelmann_climate-fever_2020, xiang_dare_2023} & 76.8 & \textbf{80.7} \\     \bottomrule
    %     \toprule
    %     & \textbf{CLBERT} & \textbf{FinBERT} \\ \midrule
    %     CLBUG-data \cite{yu_climatebug_2024} & \textbf{91.07} & 90.82 \\
    %     Climate detection \cite{bjarne_brie_mandatory_2022} & \textbf{98.59} & 96.67 \\ \bottomrule
    %     \toprule
    %     & \textbf{CLBERT} & \textbf{RoBERTa} \\ \midrule
    %     Ledger classif. \cite{jain_supply_2023}  & 85.20 & \textbf{87.19} \\
    %     ClimaTOPIC \cite{spokoyny2023answering} & 64.24 & \textbf{65.22} \\
    %     ClimateEng \cite{spokoyny2023answering, vaid-etal-2022-towards} & 69.60 & \textbf{73.50} \\
    %     Env. Claims \cite{stammbach_environmental_2023} & 83.7 & \textbf{84.9} \\
    %     Net-zero/Reduction \cite{tobias_schimanski_climatebert-netzero_2023} & \textbf{96.2} & 95.8 \\
    %     CLStance \cite{spokoyny2023answering, vaid-etal-2022-towards} & 52.84 & \textbf{59.69} \\
    %     Stance \cite{lai_using_2023} & \textbf{90} & 89 \\
    %     ClimaINS \cite{spokoyny2023answering} & 84.80 & \textbf{85.35} \\
    %     ClimaBench \cite{spokoyny2023answering} & 69.44 & \textbf{71.14} \\     \bottomrule
    %     \toprule        
    %     & \textbf{CLBERT} & \textbf{SciBERT} \\ \midrule
    %     CLFEVER (evid.) \cite{diggelmann_climate-fever_2020, spokoyny2023answering} & 61.54 & \textbf{62.68} \\
    %     \bottomrule
    %     \toprule
    %     & \textbf{CLBUG-LM} & \textbf{FinBERT} \\ \midrule
    %     ClimateBUG-data \cite{yu_climatebug_2024} & \textbf{91.36} & 90.82 \\     \bottomrule
    %     \end{tabular}
    %     }
    %     \caption{Comparison of ClimateBERT with larger models (both fine-tuned with an identical experimental settings). The metrics vary depending on the datasets (F1, Accuracy, custom metrics, etc).}
    %     \label{tab:climateBERT vs larger}
    \end{subtable}
    

    \caption{Performance of domain-specific models on domain-specific tasks. The figure displayed in the tables are the values reported by authors in the corresponding studies; with the following abbreviations: \textit{CL} for Climate, \textit{Dis} for Distil, \textit{EnvDisRob} for EnvDistilRoBERTa, \textit{Evid.} for evidences. Each row reports the performances of models fine-tuned in the same experimental setting.
    Detailed performances are reported in Appendix \ref{app:perf}.}
\end{table}



% \begin{tabular}{llc}
        % \toprule
        % \textbf{Dataset}        & \textbf{Model}       & \textbf{Metric} \\ \midrule
        % Climatext \cite{spokoyny2023answering, varini_climatext_2020}         & Longformer           & 87.80                \\
        %                         & ClimateBERT          & 85.14                \\ \midrule
        % Climatext \cite{garridomerchán2023finetuning, varini_climatext_2020}         & BERT           & 91                \\
        %                         & ClimateBERT          & 93                \\ \midrule
        % ClimateBUG-data \cite{yu_climatebug_2024}         & climateBUG-LM        & 91.36                 \\
        %                         & ClimateBERT          & 91.07   \\   
        %                         & FinBERT          & 90.82  \\ \midrule
        % Climate detection \cite{bjarne_brie_mandatory_2022}             & ClimateBERT          & 98.59                 \\
        %                         & FinBERT              & 96.67                 \\ \midrule
        % SciDCC \cite{spokoyny2023answering}         & Longformer           & 54.79                 \\
        %                         & ClimateBERT          & 52.97                 \\ \midrule
        % Ledger classification \cite{jain_supply_2023}  & RoBERTa              & 87.19                 \\
        %                         & ClimateBERT          & 85.20                 \\ \midrule
        % CLIMA-TOPIC \cite{spokoyny2023answering}            & RoBERTa              & 65.22                 \\
        %                         & ClimateBERT          & 64.24                 \\ \midrule
        % ClimateEng \cite{spokoyny2023answering, vaid-etal-2022-towards}             & RoBERTa-Large        & 73.50                 \\
        %                         & ClimateBERT          & 69.60                 \\ \midrule
        % DARE's Sentiment  & BERT-base        & 93.1                 \\
        % Analysis \cite{xiang_dare_2023}       & ClimateBERT          & 87.5                 \\ \midrule
        % Environmental              & RoBERTa-Large        & 84.9             \\
        % Claims \cite{stammbach_environmental_2023}                        & ClimateBERT          & 83.7                 \\ \midrule      
        % Net-zero/Reduction \cite{tobias_schimanski_climatebert-netzero_2023}             & RoBERTa-base        & 95.8             \\
        %                         & ClimateBERT          & 96.2                 \\ \midrule     
        % ClimateFEVER             & SciBERT        & 62.68             \\
        %  (evidence) \cite{diggelmann_climate-fever_2020, spokoyny2023answering}                        & ClimateBERT          & 61.54                 \\ \midrule  
        % ClimateFEVER              & BERT        & 80.7             \\
        %   (claim) \cite{diggelmann_climate-fever_2020, xiang_dare_2023}                      & ClimateBERT          & 76.8                 \\ \midrule
        %  ClimateStance \cite{spokoyny2023answering, vaid-etal-2022-towards}             & RoBERTa        & 59.69                 \\
        %                         & ClimateBERT          & 52.84                 \\ \midrule
        % Stance \cite{lai_using_2023}             & RoBERTa        & 89                 \\
        %                         & ClimateBERT          & 90                 \\ \midrule
        % LobbyMap (page) \cite{lai_using_2023}             & longformer-large        & 76.5                 \\
        %                         & ClimateBERT          & 74.4                 \\ \midrule
        % LobbyMap (query) \cite{lai_using_2023}             & longformer-large        & 55                 \\
        %                         & ClimateBERT          & 48.9                 \\ \midrule
        % LobbyMap (stance) \cite{lai_using_2023}             & longformer-large        & 44.1                 \\
        %                         & ClimateBERT          & 39                \\ \midrule
        % ClimaINS \cite{spokoyny2023answering}             & RoBERTa        & 85.35                 \\
        %                         & ClimateBERT          & 84.80                 \\ \midrule
        % ClimaBench \cite{spokoyny2023answering}             & RoBERTa        & 71.14                 \\
        %                         & ClimateBERT          & 69.44                 \\ \bottomrule
        % \end{tabular}



As shown in Table~\ref{tab:climateGPT}, both ClimateGPT-2 and ClimateGPT significantly outperform the baseline model in their designated tasks. These models benefit from additional training on climate-related data, highlighting the advantages of domain adaptation. However, when trained from scratch, \citet{thulke2024climategpt} report substantially lower performance, underscoring the importance of large-scale pretraining data.
For smaller models (climateBERT, climateBUG-LM and EnvironmentalBERT), the domain-adaptation also enhanced the performance on domain-specific tasks, though the significance of this improvement is less pronounced.
As detailed in Table~\ref{tab:comparison climatebert}, out of the 15 evaluations of fine-tuned domain-specific climateBERT~\cite{nicolas_webersinke_climatebert_2021} on climate-related tasks compared to the fine-tuned base model (distilRoBERTa), 4 reported an improvement compared to the base model (>1\%), 8 reported a marginal improvement (<1\%) and 3 indicate no measurable difference (<=0\%). %In the last two tasks from Table~\ref{tab:comparison climatebert}, climateBERT is compared to another small model, DistilBERT; in one case, there is marginal improvement, and in the other, climateBERT is underperforming. 
When compared to DistilBERT in the last two tasks from Table~\ref{tab:comparison climatebert}, climateBERT exhibits a slight improvement in one case, while performing comparably in the other.
Additionally, climateBERT achieves performance similar to the simple baselines in certain instances (Table~\ref{tab:comparison climatebert}). 
For example, in Specificity Classification~\cite{bingler2023cheaptalkspecificitysentiment}, climateBERT outperforms an SVM with bag-of-word by 2\%, and on the TCFD classification~\cite{sampson_tcfd-nlp_nodate} climateBERT performs slightly below the TF-IDF baseline model. Likewise, climateBUG-LM~\cite{yu_climatebug_2024} demonstrates a slight improvement of less than 1\% over BERT and FinBERT on a climate-related detection task. %Similarly, climateBUG-LM~\cite{yu_climatebug_2024} improved by less than 1\%. 
As shown in Table~\ref{tab:comparison env}, the domain-adapted models shows a small improvement in the performance on the Environmental task, by 1.38\% for distilRoBERTa and 0.84\% for RoBERTa. 


\paragraph{Insights} Domain-specific language models, particularly those developed for environmental and climate-related texts, offer an interesting avenue for improving model performance through enhanced knowledge within specialized domains. Theoretically, such models should outperform general models by leveraging their tailored vocabulary and contextual understanding. However, our findings suggest that the improvements provided by the domain-specific models on the proposed domain-specific downstream tasks are limited. Indeed, we found that fine-tuning general models on domain-specific datasets consistently yields competitive performance. However, these findings do not undermine the significance of existing works that have been fundamental in demonstrating the potential of domain-specific models, particularly in generating domain-specific text (MLM and next token prediction). Moreover, ClimateGPT~\cite{thulke2024climategpt}
demonstrated significant improvement for the domain-adapted Llama-2, showing that domain-adaptation is highly relevant. Future research could focus on designing tasks where a lack of domain understanding—whether in terms of vocabulary or knowledge—significantly hinders performance.

%Further research is needed to fully understand the circumstances under which domain-specific models add measurable value. A key direction of future work lies in building more robust and/or specialized further-training datasets, such as the efforts seen with ClimateGPT, which could more effectively bridge the gap between general and specialized models. Improving the quality and coverage of these datasets may unlock the full potential of domain-specific models in complex, high-stakes fields like climate science.

% There are multiple domain-specific approaches, which can improve performance. The theoretical argument for domain-specific models is that these models have a larger  vocabulary that includes rare domain-specific words, while also the models improve the embedding of infrequent words that might be over-represented in a domain-specific corpus. %The hypothesis is that domain-specific models improve performances on domain-specific task. 



% However, using larger models seems to frequently improve performances. While we question the relevance of using domain-specific models, we also acknowledge that the explanation might come from the task that we define as domain-specific. Climate-related vocabulary is relatively popular, which might make the task not specific enough to require a dedicated model and vocabulary. This is even more true with larger models, that already  have larger vocabularies. Moreover, the relatively high performances of TF-IDF and keyword-based models on topic classification tasks highlights that they rely heavily on vocabulary cues. This might also show that tasks do not require domain-specific understanding, but only syntax and vocabulary understanding.
% maybe a contextual understanding

 
% {\color{gray}
% On another hand, \citet{thulke2024climategpt} evaluated domain-specific Llama compared to Llama-Chat and other LLMs. They found that finetuning LLama on domain-specific instruction dataset improve the accuracy of more than 8\%. \tom{I don't know exactly how to use that conclusion i am looking into it}
% \cite{thulke2024climategpt} also not the best performing models are climate related mistral 7B > climateGPT 70B, and only 2\% from climateGPT 7B.
% }\oana{for the 70B they say they did not have the time to fine tune the parameters. Here there are two things that they do which we do not know which has brought more advantage: the continuous pretraining using climate based data and the instruction fine tuning on climate related questions. we can highlight this - I am not sure if they put the results without the instruction fine tuning?}

% \fms{Any experiments or insights to share here?}

\subsection{Climate-Related Topic Detection}
\label{sec:climate-related topic}

The first step toward detecting climate-related greenwashing is identifying text addressing climate-related topics.


\task Given an input sentence or a paragraph, output a binary label,  ``\textit{climate-related}'' or ``\textit{not climate-related}''.

% \input{latex/tikz_figure/climate_graph}

\begin{table}[ht]
\centering
\begin{tabular}{p{3cm}p{4cm}p{1.3cm}p{5cm}}
\toprule
\textbf{Dataset} & \textbf{Input} & \textbf{Labels} & \textbf{Positive Label Details}  \\ \midrule
ClimateBug-data \cite{yu_climatebug_2024} & sentences from Banks' reports  & \textit{relevant/} \textit{irrelevant} & Climate change and sustainability (including ESG, SDGs related to the environment, recycling and more) \\ \midrule
ClimateBERT's climate detection \cite{bingler2023cheaptalkspecificitysentiment} & paragraphs from reports & \textit{1/0} & Climate policy, climate change or an environmental topic \\ \midrule
Climatext (Wikipedia, 10-K, claims) \cite{varini_climatext_2020} & sentences from Wikipedia, 10-Ks or web scraping & \textit{1/0} & Directly related to climate-change \\ \midrule
Climatext (wiki-doc) \cite{varini_climatext_2020} & sentences from a Wikipedia page & \textit{1/0} & Extracted from a Wikipedia page related to climate-change \\ \midrule
Sustainable signals's reviews \cite{linSUSTAINABLESIGNALSijcai2023} & online product reviews (user comments) & \textit{relevant/} \textit{irrelevant} & Contains terms related to sustainability  \\ \bottomrule
\end{tabular}
\caption{Label definitions of the datasets related to climate change and sustainability topic detection.}
\label{tab:guidelines climate-related}
\end{table}

%\fms{does this mean there is no training data? or that there is no gold standard (then how do they evaluate?) or that there is no dataset?}\tom{done} 
\zeroshot The earliest work on identifying corporate climate-related text was conducted by~\citet{doran_risk_disclosure}, where the authors collected a large corpus of 10-K filings from 1995 to 2008, and filtered climate-related reports using hand-selected keywords. While~\citet{doran_risk_disclosure} focused on identifying climate-related text within company reports to assess how businesses addressed climate change, other early efforts took a different approach by analyzing the external impacts of media attention toward climate change on companies. For instance,~\citet{Engle_hedging_climatechange} developed the WSJ Climate Change News Index based on identifying climate-related news articles in media sources using climate-change vocabulary frequency. This index served as a tool to measure the influence of climate-change attention in the media on companies. \citet{csr_report_greenwashing} proposed measuring the environmental content of CSR (corporate social responsibility) reports using a lexicon-based approach, with the goal of comparing CSR reports of environmental violators to companies with clean records. 
%These early approaches laid the groundwork for identifying climate-related content, either from a corporate disclosure perspective or by assessing the impact of climate-change attention on companies.


\datasets More recent studies have introduced several annotated datasets aimed at climate-related corporate text classification. One of the most prominent is ClimaText~\citep{varini_climatext_2020}, a large dataset from diverse sources such as Wikipedia, U.S. Security and Exchange Commission 10-K filings, and web-scraped content. The dataset is designed to cover a broad range of topics and document types to help assess climate-related discussions across different domains. It is divided into multiple subsets: wiki-doc, Wikipedia, 10-k, claims. Climatext (Wiki-doc) is annotated automatically, while Climatext (Wikipedia, 10-k and claims) is annotated by humans.
In contrast, ClimateBERT's authors~\citep{nicolas_webersinke_climatebert_2021} provide a smaller, more focused dataset consisting of paragraphs from companies' annual and sustainability reports. This dataset was created as a downstream task for evaluating the ClimateBERT model, focusing on corporate disclosures. The dataset was later extended by~\citet{bingler2023cheaptalkspecificitysentiment}, increasing its size and refining the annotation to capture both specificity and sentiment related to corporate climate discourse (see Sections~\ref{sec: climate risk},~\ref{sec: claim characteristics}).
Another large dataset, ClimateBUG-data~\citep{yu_climatebug_2024}, also focuses on corporate communications but in EU banks.
%It aims to classify whether corporate statements are climate-related, providing insights into how European financial institutions communicate about climate change.
Beyond corporate disclosures, Sustainable Signals~\citep{linSUSTAINABLESIGNALSijcai2023} introduces a dataset of product reviews, each classified based on its relevance to sustainability. In addition to focusing on consumer perspectives through reviews, this dataset also examines sustainability aspects in product descriptions, providing a broader understanding of how sustainability is communicated in the context of consumer products.

\paragraph{Labels and Guidelines} As shown in Table \ref{tab:guidelines climate-related}, although these datasets address similar tasks, they differ significantly in their label definitions. The scope of the labels varies, ranging from a narrow focus on climate change to broader topics such as sustainability and environmental impact. 
While the task of the Climatext (wiki-doc) dataset is to detect whether a sentence is related to climate change or not, the weak label actually means that the sentence originates from a Wikipedia page related to climate change. Which is not the same task, as it focuses on the source of the statement rather than the actual content or its relevance to climate-related topics. However, the weakly labeled dataset serves as a useful filtering mechanism, helping to identify sentences that are potentially interesting. It provides an initial pool of data that can be refined with human annotations. Human annotations ensure that the labels align with the task’s goals, something weak labels alone cannot achieve. 
%\fms{This is an understatement!! Discuss here that it makes no sense to detect whether a text comes from Wikipedia!! Same for "contains terms related to"; give examples!}
%Hence, there is a challenge of isolating climate-related corporate discourse, as it is often deeply embedded within broader environmental strategies and discussions. Oana: said in conclusion

\solutions The solutions proposed to tackle that task range from keyword-based models~\cite{varini_climatext_2020, bingler2023cheaptalkspecificitysentiment} to fine-tuning BERT-like models~\cite{varini_climatext_2020, nicolas_webersinke_climatebert_2021, garridomerchán2023finetuning, bingler2023cheaptalkspecificitysentiment, yu_climatebug_2024} (see Table~\ref{tab:models} in the appendix for details). The performance of the best models is above 90\%~\cite{garridomerchán2023finetuning, bingler2023cheaptalkspecificitysentiment, yu_climatebug_2024} (see Table~\ref{tab:reported perf climate} in the appendix for details). 

\paragraph{Insights} %\fms{right direction, but toothless. Tie into discussion about labels above.}\tom{I changed the labels and guidelines + Added one sentence here. It's better ?} 
The task of determining whether a statement is climate-related has been extensively studied, with numerous datasets available to support research in this area. State-of-the-art fine-tuned models now achieve near-perfect performance on these datasets, indicating that, under current conditions, the task is effectively solved.
However, the labels used across the studies vary in scope. 
%Some embrace broader concepts such as sustainability or general environmental themes as part of the positive label, while others focus exclusively on climate-related content. Oana: already said in labels
This highlights the inherent challenge of isolating climate-related discourse, given its close association with broader environmental and sustainability topics.
%These distinctions carry significant practical implications, especially when assessing how companies address climate issues. 
For example, whether to include only explicit references to climate change or incorporate broader environmental and sustainability discussions, can profoundly influence the results and interpretations of such analyses.
Moreover, relying only on weak labels might result in tasks that 
%are both misaligned and 
fail to capture the complexities of the subject.

%Future research could reproduce those works but focusing solely on climate-related content, or by distinguishing between climate-specific and other related topics (environment, ESG, ...). 

% Future research: 
% - Defining properly climate-related
% - Study the distinction between climate, environment, sustainability, ...

\subsection{Thematic Analysis}
\label{sec:sub-topics}

%citations
%direct: \citet{hyewon_kang_analyzing_2022},  \cite{yu_climatebug_2024}, \citet{spokoyny2023answering}, \citet{Schimanski2024nature}, \citet{jain_supply_2023}, \citet{vaid-etal-2022-towards}, \citet{mishra2021neuralnere}

% \citet{garridomerchán2023finetuning} performed multiple runs (25 with =/= seeds)

Once a text is known to be related to climate, one aims to know the exact topic of the text.
Companies may prioritize specific categories while underreporting others, potentially signaling selective transparency practices such as greenwashing or greenhushing.

\task Given an input sentence or a paragraph, output a subtopic related to climate change. This is a multiclass classification task that can be either supervised or unsupervised, depending on the availability of labeled data. Alternatively, it can be framed as a clustering task with the goal of discovering latent subtopic structures.
%The task is to classify a sentence or paragraph into subtopics related to climate change: a multiclass classification task that can be either supervised or unsupervised, depending on the availability of labeled data. Alternatively, it can be framed as a clustering task with the goal of discovering latent subtopic structures. Oana: I proposed a version for this, the tasks definitions will also be part of the experimental paper, otherwise that paper would miss a problem statement.

\begin{table}[ht]
\centering
\begin{tabular}{p{3cm}p{3cm}p{3cm}p{5cm}}
\toprule
\textbf{Dataset} & \textbf{Input} & \textbf{Labels}                                                                                  & \textbf{Label details}                                          \\
\midrule
TCFD rec. \cite{bingler_cheap_2021} & paragraphs from corporate annual reports & \multicolumn{2}{p{8cm}}{TCFD 4 main categories: \textit{Metrics and Targets, Risk Management, Strategy, Governance and General} (see appendix \ref{app:tcfd} for details)}             \\
TCFD \cite{sampson_tcfd-nlp_nodate} & paragraphs from  regulatory and discretionary reports & \multicolumn{2}{p{8cm}}{TCFD 11 recommendations: \textit{Metrics and Targets (board’s oversight,  management’s role), Risk Management (identified risk, impact, resilience), ...} (see Appendix \ref{app:tcfd} for details)}             \\ 
\midrule
FineBERT's ESG \cite{huangFinBERTLargeLanguage2020} & Sentences from 10-K  & \textit{Environmental, Social, Governance, General} & Environmental - e.g., climate
change, natural capital, pollution and waste, and environmental opportunities \\
ESGBERT's ESG \cite{schimanski_bridging_2023} & Sentences from reports and corporate news & \textit{Environment, Social, Governance and None} & Environmental criteria comprise a
company’s energy use, waste management, pollution, [...]
as well as compliance with governmental regulations.
Special areas of interest are climate
change and environmental sustainability.  \\
Multilingual ESG classification \cite{LEE2023119726} & Sentences from Korean reports (English and Korean)  & \textit{Environment, Social,
Governance, Neutral, and Irrelevant} & Environmental factors include the reduction of hazardous substances, eco-friendly management, climate change, carbon emission, natural resources, [...] \\ 
\midrule
SciDCC \cite{mishra2021neuralnere} & News articles (Title, Summary, Body)  & \textit{Environment, Geology, Animals, Ozone Layer, Climate, etc.} & Category in which the article was published (Automatic Label)   \\
Transaction Ledger \cite{jain_supply_2023} & Transaction ledger entry (description of transaction) & \textit{Accounting Adjustments, Administration, Advertising, Benefits \& Insurance, etc.} & Standardized commodity classes  \\
ClimateEng \cite{vaid-etal-2022-towards}  & Tweets posted during COP25 filtered by keywords (relevant to climate-change)  & \textit{Ocean/Water, Politics, Disaster, Agriculture/Forestry, General} & Sub-categories of climate-change\\
ESGBERT's Nature \cite{Schimanski2024nature} & Paragraphs from reports  & \textit{General, Nature, Biodiversity, Forest, Water} & Multi-label Nature-related topics                   \\
ClimaTOPIC \cite{spokoyny2023answering} & CDP responses (short texts) & \textit{Adaptation, Buildings, Climate Hazards, Emissions, Water, etc.} & Category of the question (Automatic Label)          \\
\bottomrule
\end{tabular}
\caption{Summary of datasets, labels, scopes, and sources for climate-related text classification tasks.}
\label{tab:datasets_subtopic}
\end{table}


\subsubsection{Classifying Climate-Related Financial Disclosure (TCFD)}
\label{sec: tcfd}

The Task Force on Climate-Related Financial Disclosures (TCFD) proposes four main categories of climate-related disclosure (Governance, Strategy, Risk Management, and Metrics and Targets) and 11 recommendations for climate-related disclosures (see Appendix \ref{app:tcfd} for details). Finding texts associated to these categories in company communication can help identify
gaps and inconsistencies in reporting.

\zeroshot \citet{dingCarbonEmissionsTCFD2023} introduced a score to quantify the extent of climate-related content in corporate disclosures, as well as 4 TCFD-category-based similarity scores. The score is based on sentence similarity with sentences from TCFD referential documents. They identified discrepancies that could signal greenwashing by comparing these scores with actual carbon emission data. \citet{auzepy_evaluating_2023} examined reports from banks endorsing the TCFD recommendations. They proposed a fine-grained analysis of TCFD categories using a zero-shot entailment classification method on sentences.

\datasets Although unsupervised methods can be employed for thematic analysis, recent studies have demonstrated the value of curated datasets for categorizing corporate disclosures. For instance, \citet{bingler_cheap_2021} leveraged report headings to annotate paragraphs into the four main TCFD categories. Similarly, \citet{sampson_tcfd-nlp_nodate} assessed the quality of climate-related disclosures using a dataset of 162k sentences, where each sentence was labeled according to the TCFD recommendations by experts through a question answering (QA) process.

\subsubsection{Classifying Environmental, Social, and Governance Disclosure (ESG)}

Beyond climate-specific reporting, Environmental, Social, and Governance (ESG) disclosures represent a broader set of non-financial metrics that evaluate a company's sustainability performance. ESG reporting has become a global standard, encompassing climate-related initiatives and other dimensions of corporate responsibility. The standardized nature of ESG metrics makes them valuable for computational models that aim to detect inconsistencies in corporate claims, such as overstating environmental impacts through greenwashing. However, as noted in \citet{berg2022aggregate}'s aggregated analysis of ESG ratings, discrepancies persist across different evaluation frameworks. These inconsistencies raise the question of whether artificial intelligence can be harnessed to improve the consistency and reliability of ESG assessments.

\zeroshot \citet{rouenEvolutionESGReports2023} proposed an ML algorithm to describe the content of ESG reports. They used industry-topic dictionaries to compute the topic frequency using a TF-IDF-based algorithm. They used that classification to identify selective disclosure. \citet{Mehra_2022} experimented with sentence similarity to extract sentences that were the most relevant to environmental factors, and \cite{bronzini_glitter_2023} selected sentences related to ESG using INSTRUCTOR-xl~\cite{su2023embeddertaskinstructionfinetunedtext}. They both subsequently used the extracted sentences to predict ESG scores.

\datasets Multiple works \cite{huangFinBERTLargeLanguage2020, schimanski_bridging_2023, LEE2023119726} proposed datasets for topic classification with the labels: Environment, Social, Governance and General.
\citet{huangFinBERTLargeLanguage2020} focused on reports while \citet{schimanski_bridging_2023} added corporate news. \citet{schimanski_bridging_2023} introduced a larger dataset for pre-training domain-specific models and an evaluation dataset of 2,000 sentences. While most ESG-related reports are typically available in English, some local companies provide them only in their native language. To partially address this, \citet{LEE2023119726} proposed a multilingual dataset constructed from Korean corporate reports.

\subsubsection{Other Topics}
\label{sec:other-sub-topics}

Although TCFD and ESG are established frameworks for analyzing corporate communications, many other topics may be of interest. These topics can follow other frameworks such as CDP, but the thematic analysis can also be unsupervised, effectively discovering the topics mentioned.

\zeroshot \citet{hyewon_kang_analyzing_2022} analyzed the content of sustainable reports with thematic analysis. It was conducted using sentence similarity between the reports and content from the SDGs website. \citet{yu_climatebug_2024} examined the themes covered in the climate-related sections of bank reports by employing a dual approach: using hand-selected keywords to target anticipated topics and applying clustering techniques to group climate-related sentences and uncover underlying themes.  \citet{bjarne_brie_mandatory_2022} leveraged ClimateBERT to extract climate-related paragraphs from corporate documents. They then applied a Structural Topic Model (STM) \citep{STM} to identify thematic clusters within the data. This STM-based approach was also used by \citet{fortes2020tracking} to identify influential topics for EUR/USD exchange rate.
Their findings highlighted a significant discrepancy between the European Central Bank (ECB) and the Federal Reserve (FED), with the ECB attributing greater importance to climate change in the context of financial stability than the FED. Such thematic analysis can be used to identify selective disclosure.

\datasets \citet{spokoyny2023answering} introduced ClimaTOPIC, a topic classification dataset derived from responses to the CDP questionnaires, organized into 12 categories that align with the thematic structure of the CDP questions. This can be used to identify sections relevant to a particular CDP topic (e.g., Emissions, Adaptation, or Energy), enabling the extraction of structured information from unstructured documents. 
\citet{Schimanski2024nature} presents a dataset focused on nature-related topics (such as Water, Forest, and Biodiversity) extracted from annual reports, sustainability reports, and earnings call transcripts (ESGBERT Nature). While less explicitly linked to climate change, these topics often reflect the consequences of environmental degradation associated with climate issues. 
%A more thorough analysis of climate-related content, categorized into subtopics (e.g., emissions, consequences, adaptations, etc.), enables a more nuanced understanding of how companies address different aspects of climate change in their communications, revealing the depth, scope, and alignment of their narratives with their stated commitments. Oana: you are not talking about the previous work, right? it is more of a future work? I removed it
While NLP is often used to analyze corporate report texts, it can also be used to analyze tables. For example, \citet{jain_supply_2023} proposes estimating Scope 3 emissions \footnote{Scope 3 emissions refer to all indirect greenhouse gas (GHG) emissions that occur in a company’s value chain, excluding emissions from the company’s own operations (Scope 1) and its purchased electricity (Scope 2).} based on corporate transactions (see Section~\ref{sec:env prediction} for details). The first step involves grouping expenses into standardized commodity classes, for which they introduced a dataset of 8K examples of corporate expenses categorized accordingly. 
%to map expenditures to CO2 emissions per dollar spent.
As previously discussed, social media and news articles offer a broader perspective for cross-referencing corporate communications. For example, \citet{vaid-etal-2022-towards} introduced ClimateEng, a climate-related topic classification dataset composed of tweets grouped into five categories: Disaster, Ocean/Water, Agriculture/Forestry, Politics, and General. 
This dataset can be used to detect greenwashing by comparing the public’s concerns with the narrative presented in corporate sustainability reports. Similarly, \citet{mishra2021neuralnere} published the Science Daily Climate Change dataset (SciDCC), consisting of 11,000 news articles grouped into 20 climate-related categories. This dataset offers another avenue for identifying greenwashing, as companies’ communications can be cross-referenced with external news coverage, potentially highlighting contradictions or omissions in corporate reports.
%It is built by first filtering nature-related sentences using keywords and GPT-3.5, then by giving a final human annotation. 

% \fms{all of these datasets are not actually climate related, no? They rather talk about nature, no? If so, we should reframe (and relabel) this task...}

\subsubsection{Models and Conclusion}

\solutions The most common solution proposed for these tasks is fine-tuning a Transformer model~\cite{huangFinBERTLargeLanguage2020, schimanski_bridging_2023, LEE2023119726, bingler_cheap_2021, sampson_tcfd-nlp_nodate, vaid-etal-2022-towards, spokoyny2023answering, Schimanski2024nature, jain_supply_2023}. To contextualize the performance of Transformer models, some studies also report performances of classical baselines~\cite{huangFinBERTLargeLanguage2020, spokoyny2023answering, Schimanski2024nature, jain_supply_2023, bingler_cheap_2021, sampson_tcfd-nlp_nodate}. \citet{bingler_cheap_2021} proposed a custom approach combining logistic regression with features from a fine-tuned language model, while \citet{sampson_tcfd-nlp_nodate} also evaluated clustering techniques and stacked models. The performance of the best-performing model is systematically high (above 80\%), except on SciDCC and ClimaTOPIC. %\fms{Be clearer here: classical models usually 20\% worse than transformers, except on dataset 1 and dataset 2. The reason ...}\tom{TO BE DISCUSSED} 
Classical approaches (SVM, NB) also reach good performances, under-performing by less than 20\% on all datasets (except on \citet{bingler_cheap_2021}'s dataset). The TF-IDF baseline even outperformed fine-tuned Transformers on \citet{sampson_tcfd-nlp_nodate}'s dataset. This shows that those topics have distinguishable vocabularies. 

\paragraph{Automatic Labels} The performance gap observed in \citet{bingler_cheap_2021} may be attributed to using paragraph-level inputs rather than sentence-level inputs aggregated for classification, as fine-tuned models also achieve around 20\% precision at the paragraph level.
The lower performances on SciDCC\cite{mishra2021neuralnere} and ClimaTOPIC\cite{spokoyny2023answering} are likely caused by the automation of the labeling process. The labels are therefore not designed to be labels. In SciDCC there are labels that are highly similar (e.g. \textit{Endangered Animals} and \textit{Extinctions}), or labels that include other labels (e.g. \textit{Environment}, \textit{Climate}, \textit{Pollution}). As they are categories from a journal, the categories changed in time. For ClimaTOPIC, the labels are question categories, which are designed to group questions not to precisely identify them. Therefore a question about "the emission of a building" might fit in either \textit{Emissions} or \textit{Building}, yet it is assigned only one label. 

\paragraph{Insights} Topic classification is a well-established area of research in natural language processing (NLP) and has been extensively applied to climate-related topics. Fine-tuned Transformer models proposed in the literature consistently demonstrate near-perfect performance on human-annotated datasets, indicating that the task is largely solved. When reported, classical keyword baselines also perform relatively well, achieving significantly better than random performance and approximately 80\%  of the performance of fine-tuned models, suggesting the topics have distinct vocabularies. Future research could investigate topics that are lexically similar but distinct in their subject or focus.
Topic detection serves as a valuable tool for structuring documents and has been utilized to analyze the extent to which companies address specific topics, enabling the detection of selective disclosure \cite{bingler2023cheaptalkspecificitysentiment, bingler_cheap_2021}. However, it is critical to ensure that labels are well-defined and data is accurately annotated. 
Lower performance observed on automatically generated labels may reflect challenges in predictability, potentially due to difficulties caused by the automatic annotation. These works might be revisited to conduct human annotations. 

\subsection{In-depth Disclosure: Climate Risk Classification}
\label{sec: climate risk}

%citations
%direct : \citet{liCorporateClimateRisk2020}, \citet{kheradmand2021a}, \citet{chou_ESG}, \citet{SAUTNER_cliamte_change_exp},\cite{marco_polignano_nlp_2022, hyewon_kang_analyzing_2022},  \citet{kolbel_ask_2021} , \citet{Friederich_climate_risk_disclosure}, \citet{bingler2023cheaptalkspecificitysentiment},  \citet{xiang_dare_2023} \cite{nicolas_webersinke_climatebert_2021} 

%Greenwashing can take many forms, but two common examples include presenting overly positive information about a company’s environmental impact and downplaying or omitting negative aspects of its environmental performance. 
Climate change can bring both risks and opportunities for companies. Potential risks are, e.g., reputational risks (e.g. environmental controversies), regulatory risks (e.g. new regulations on emissions), and physical risks (e.g. droughts impacting production).  Opportunities are, e.g., financial opportunities (e.g. benefiting from grants that aim to support less polluting industries), market opportunities (e.g. electric cars becoming more popular with environmentally conscious clients), etc. If a company systematically avoids discussing climate-related risks or disproportionately emphasizes opportunities, it creates a biased narrative that can serve as an indicator of greenwashing.

\task Given an input sentence or a paragraph, output ``opportunity'' or ``risk'' label. Some works focus only on risks, classifying them into types of risks (e.g., physical risk, reputational risk, regulatory risk, or transition risk)

\begin{table}[ht]
\centering
\begin{tabular}{p{2cm}p{4cm}p{3cm}p{4cm}}
\toprule
\textbf{Dataset} & \textbf{Input}   &  \textbf{Labels}                                                                                  &                                       \\
\midrule
Ask BERT's Climate Risk \cite{kolbel_ask_2021} & Sentences from TCFD's example reports and non-climate-related sentences &  \multicolumn{2}{p{8cm}}{Risk type: \textit{Transition risk, physical risk, and general risk (no guidelines)}}             \\ \midrule
Climate Risk \cite{Friederich_climate_risk_disclosure} & Paragraphs from European companies annual reports & \multicolumn{2}{p{8cm}}{Risk type:\textit{Acute, Chronic, Policy \& legal, Tech \& Market, Reputational, and Negative} (no guidelines)}       \\ \midrule
ClimateBERT's Sentiment \cite{bingler2023cheaptalkspecificitysentiment}  & Paragraphs from companies' annual reports            & \multicolumn{2}{p{8cm}}{\textit{Risk} or threat that negatively impacts an entity of interest (negative sentiment); or \textit{Opportunity} arising due to climate change (positive sentiment); \textit{Neutral} otherwise.} \\ \midrule
Sentiment Analysis \cite{xiang_dare_2023} & paragraphs from academic texts on climate change and health published between 2013 and 2020  & \multicolumn{2}{p{8cm}}{\textit{Risk} (negative) if it discusses climate change causing public health issues, serious consequences, or worsening trends, greenwashing. \textit{Opportunity} (positive): highlights potential benefits, positive actions, or research addressing gaps. \textit{Neutral} otherwise.} \\
\bottomrule
\end{tabular}
\caption{Summary of datasets, labels, scopes, and sources for climate-related Risk classification.}
\label{tab:datasets_risk}
\end{table}


% no dataset
\zeroshot Similarly to climate-related detection, climate risk classification has been tackled with keyword-based approaches. \citet{liCorporateClimateRisk2020}, \citet{kheradmand2021a} and \citet{chou_ESG} proposed using dictionaries of words related to climate risk to identify paragraphs dealing with climate risk. \citet{liCorporateClimateRisk2020} constructed risk measures based on the frequency of the terms in the risk dictionaries. \citet{chou_ESG} analyzed the topics mentioned in conjunction with physical and transition risks. \citet{SAUTNER_cliamte_change_exp} proposed using one dictionary for risk and one for opportunity classification.

\datasets Several studies have introduced specialized annotated datasets for risk/opportunity classification. For instance, \citet{kolbel_ask_2021} created a dataset for climate-related risk classification with three categories: physical risk, transition risk, and general risk, using active learning for annotation. Similarly, \citet{Friederich_climate_risk_disclosure} developed an annotated dataset for risk classification with five labels, covering acute physical risk, chronic physical risk, policy and legal risks, technology and market risks, and reputational transition risks. \citet{bingler2023cheaptalkspecificitysentiment} released a dataset as part of the ClimateBERT downstream tasks, focusing on classifying paragraphs from corporate reports into three categories: opportunity, neutral, or risk. Extending beyond corporate disclosures, \citet{xiang_dare_2023} compiled a climate-related risk/opportunity classification dataset of academic texts from the Web of Science and Scopus.

\solutions The solutions proposed are similar to the other tasks: fine-tuned Transformer models \cite{hyewon_kang_analyzing_2022, kolbel_ask_2021, Friederich_climate_risk_disclosure, bingler2023cheaptalkspecificitysentiment, nicolas_webersinke_climatebert_2021, xiang_dare_2023} and keyword-based-features with simple models \cite{kolbel_ask_2021, Friederich_climate_risk_disclosure, bingler2023cheaptalkspecificitysentiment} (see Table \ref{tab:models} in the appendix for details). The notable exception is \citet{xiang_dare_2023}, who evaluated LSTM-based solutions alongside the Transformers models. Most studies~\cite{hyewon_kang_analyzing_2022, kolbel_ask_2021, bingler2023cheaptalkspecificitysentiment, nicolas_webersinke_climatebert_2021, xiang_dare_2023} reported high performances, above 80\%. \citet{kolbel_ask_2021} and \citet{bingler2023cheaptalkspecificitysentiment} also reported good performances for the keyword-based baselines (72\% and 84\%). The performance reported by \citet{Friederich_climate_risk_disclosure} on climate risk tasks highlights distinct challenges across subtasks. When tasked with identifying the specific type of risk in sentences already known to be about climate risks, word-based models outperformed fine-tuned Transformers. This suggests that each risk type had a distinct vocabulary. In contrast, when working with a dataset that reflects real-world proportions of risk and non-risk examples—where the data is heavily imbalanced—classical models struggled due to the overlap in vocabulary between general and climate risk-related text. 

\paragraph{Difference between sentiment and risk} It is important to distinguish risk/opportunity classification from sentiment analysis. While risk/opportunity classification is inspired by sentiment analysis \cite{bingler2023cheaptalkspecificitysentiment}, they play a complementary role. A statement such as \textit{``With proactive climate risk management, we are ready to tackle extreme weather disruptions, ensuring resilience''} is classified as positive using a sentiment analysis model \cite{perez2021pysentimiento}, and as mentioning a risk using a risk/opportunity classification model \cite{bingler2023cheaptalkspecificitysentiment}. Sentiment analysis focuses on the form of the statement, while risk is about the content. As the literature on sentiment analysis \cite{Wankhade2022} is broad, the analysis of the tone of the statement can be done using existing models \cite{marco_polignano_nlp_2022, hyewon_kang_analyzing_2022} complementing approaches on risk classification. 

\paragraph{Insights} The performance of fine-tuned models might indicate that the task of identifying texts talking about risks is solved. However, the work by \citet{Friederich_climate_risk_disclosure} shows limitations when experimenting with heavily imbalanced datasets, which actually correspond to real-world settings. Future research could focus on such skewed settings.



\subsection{Green Claim Detection}
\label{sec:green claim}

%citations
% direct:  \cite{panchendrarajan2024claim}, \cite{vinicius_woloszyn_towards_2021, stammbach_environmental_2023, linSUSTAINABLESIGNALSijcai2023}, \citet{linSUSTAINABLESIGNALSijcai2023},  \cite{Arslan_Hassan_Li_Tremayne_2020}

Identifying greenwashing involves more than just finding text related to climate change, as a company might simply state factual information about the climate without making any commitments on it. Hence, we now focus on detecting \textit{green claims}, i.e., claims that a product, service, or corporate practice either contributes positively to environmental sustainability or is less harmful to the environment than alternatives \cite{stammbach_environmental_2023,vinicius_woloszyn_towards_2021}. The European Commission calls these claims ``environmental claims'':

\begin{table}[ht]
\centering
\begin{tabular}{p{2.5cm}p{3cm}p{8.5cm}}
\toprule
\textbf{Dataset}  & \textbf{Input}  & \textbf{Positive Label description}\\ \midrule
Green Claims \cite{vinicius_woloszyn_towards_2021}  & Marketing Tweets  & Environmental (or green) advertisements refer to all appeals that include ecological, environmental sustainability, or nature-friendly messages that target the needs and desires of environmentally concerned stakeholders. \\ \midrule
Environmental Claims \cite{stammbach_environmental_2023} & Paragraph from reports  & Environmental claims refer to the practice of suggesting or otherwise creating the impression [...] that a product or a service is environmentally friendly (i.e., it has a positive impact on the environment) or is less damaging to the environment than competing goods or services [...]
In our case, claims relate to products, services, or specific corporate environmental performance.   \\ \bottomrule
\end{tabular}
\caption{Label definitions of the datasets on green claims detection.}
\label{tab:guidelines characteristics}
\end{table}



\task  Given an input sentence or a paragraph, output a binary label, ``\textit{green claim}'' or ``\textit{not green claim}''. 


\datasets The most notable approaches aiming at identifying climate-related claims introduced annotated datasets \cite{vinicius_woloszyn_towards_2021, stammbach_environmental_2023}.
\citet{vinicius_woloszyn_towards_2021} proposed to focus on social media marketing through the detection of green claims in tweets. \citet{stammbach_environmental_2023} proposed a dataset for environmental claim detection in paragraphs from corporate reports.

\solutions Both \citet{stammbach_environmental_2023} and \citet{vinicius_woloszyn_towards_2021} evaluated fine-tuned Transformer models on the datasets. \citet{stammbach_environmental_2023} also evaluated classical approaches and experimented with a model fine-tuned on general claim detection \cite{Arslan_Hassan_Li_Tremayne_2020} applied to their domain-specific task (see Table~\ref{tab:models} in the Appendix for details). 
\citet{vinicius_woloszyn_towards_2021} and \citet{stammbach_environmental_2023} demonstrated good performances with fine-tuned Transformer (above 84\%), yet they have very different Inter-Annotator Agreement (IAA), with Krippendorff's $\alpha=0.8223$ for \citet{vinicius_woloszyn_towards_2021} and $\alpha=0.47$ for \citet{stammbach_environmental_2023}. This might be explained by the difference in context: marketing tweets might be easier to understand for humans and more self-sufficient compared to paragraphs extracted from reports.
Additionally, \citet{vinicius_woloszyn_towards_2021} experimented with adversarial attack, showing the sensibility to character-swap and word-swap. This shows that the models tend to rely heavily on particular words, showing only a superficial understanding. While it is difficult to generalize that conclusion given the small size of the dataset, it still indicates that BERT-like architecture can over-fit or rely on superficial cues instead of building an accurate representation of the paragraph. 
It is essential to assess model performance in challenging scenarios, where the presence of nonsensical or noisy inputs can reveal the fragility of model comprehension.

\paragraph{Insights}  
Green claims have been analyzed in company reporting \citep{stammbach_environmental_2023} and social media communication \citep{vinicius_woloszyn_towards_2021}. Fine-tuned models can solve the task rather well, even if challenges remain:
\begin{itemize}
\item \textit{Inter annotator agreement:} The annotator agreement remains low for green claim detection in reports~\cite{stammbach_environmental_2023}.
    \item \textit{Integration with existing claim detection literature:} While claim detection is a well-established field with extensive literature, studies on environmental claims do not fully connect with existing research on claim detection. In particular, they do not distinguish between Claims, Verifiable Claims (claims that can be checked), and Check-worthy Claims (claims that are interesting to verify)~\cite{panchendrarajan2024claim}.
    \item \textit{Evaluation Sensitivity and Real-World Robustness:} The literature shows that fine-tuned models are sensitive to adversarial attacks, meaning small perturbations of the text influence greatly the performance of the classifier. As for all tasks, it is also important evaluate models robustness in real-world settings. Poor data quality might induce perturbations in the texts, reducing the performance of models.
\end{itemize}

\subsection{Green Claim Characteristics}
\label{sec: claim characteristics}

Once we have established that a sentence is climate-related (Section~\ref{sec:climate-related topic}) and that it is a claim about the company (Section~\ref{sec:green claim}), we can endeavor to further classify the claim into fine-grained categories. Table~\ref{tab:guidelines characteristics} shows various characteristics of claims that have been studied.

\begin{table}[ht]
\centering
\begin{tabular}{p{2.5cm}p{3.5cm}p{8cm}}
\toprule
\textbf{Dataset}  & \textbf{Input}  & \textbf{Labels}\\ \midrule
Implicit/Explicit Green Claims \cite{vinicius_woloszyn_towards_2021}  & Marketing Tweets  & \textit{Implicit green claims} raise the same ecological and environmental concerns as \textit{explicit green claims} (see definition in Section \ref{sec:green claim}), but without showing any commitment from the company. If the tweet does not contain a green claim then \textit{No Claim}. \\ \midrule
Specificity \cite{bingler2023cheaptalkspecificitysentiment}  & Paragraph from reports  & A paragraph is \textit{Specific} if it includes clear, tangible, and firm-specific details about events, goals, actions, or explanations that directly impact or clarify the firm's operations, strategy, or objectives. \textit{Non-specific} otherwise. \\ \midrule
Commitments \& Actions\cite{bingler2023cheaptalkspecificitysentiment} &  Paragraph from reports  & A paragraph is a commitment or an action if it contains targets for the future or actions already taken in the past. \\ \midrule
Net Zero/Reduction \cite{tobias_schimanski_climatebert-netzero_2023} & Paragraph from Net Zero Tracker~\cite{netzerotracker2023} & The paragraph 
 contains either a \textit{Net-Zero} target, a \textit{Reduction} target or no target (\textit{None})  \\ \bottomrule
\end{tabular}
\caption{Label definitions the datasets related to characterization of green claims.}
\label{tab:guidelines characteristics}
\end{table}

\task Given an input sentence or a paragraph labeled as a green claim, output a more fine-grained characterization of the claim. This is a multi-label classification task; the labels can be about the form (e.g. specificity) or the substance (e.g. action, targets, facts). 

\datasets Existing works characterized both the content and the form of claims. On the content dimension, there are datasets for identifying if the claim is about an action or a commitment~\cite{bingler2023cheaptalkspecificitysentiment, tobias_schimanski_climatebert-netzero_2023}. \citet{bingler2023cheaptalkspecificitysentiment} introduce a dataset for the identification of commitments and actions (Yes/No), and \citet{tobias_schimanski_climatebert-netzero_2023} released one for the identification of reduction targets (net zero, reduction, general). On the form dimension, \citet{vinicius_woloszyn_towards_2021} characterized claims as implicit or explicit, and \citet{bingler2023cheaptalkspecificitysentiment} annotated their dataset on the specificity of claims (Specific/Not specific). \citet{bingler2023cheaptalkspecificitysentiment} ultimately used the specificity and commitment/action characteristics to identify cheap talks related to climate disclosure.  

\solutions \citet{vinicius_woloszyn_towards_2021, tobias_schimanski_climatebert-netzero_2023} and \citet{bingler2023cheaptalkspecificitysentiment} experimented with fine-tuned Transformers. \citet{bingler2023cheaptalkspecificitysentiment} also evaluated classical baselines (SVM, NB), and \citet{tobias_schimanski_climatebert-netzero_2023} experimented with GPT-3.5-turbo (see Table \ref{tab:models} in the Appendix for details). All approaches reached good performances, in particular fine-tuned Transformers reaching performances above 80\%. The exception remains the Specificity classification with a F1-score of 77\% (close to the baselines). This low performance  might be intrinsic to the task. Indeed, humans have disagreement when distinguishing specific and unspecific claims: 
\citet{bingler2023cheaptalkspecificitysentiment} measured a low IAA on Specificity (Krippendorff's $\alpha=0.1703$). This might indicate that the task is not well defined in the first place. 

\paragraph{Insights} 
The high performances of fine-tuned models indicate that this type of task is solved. Furthermore, the demonstrated ability of GPT-3.5 as a zero-shot classifier \cite{tobias_schimanski_climatebert-netzero_2023} highlights the potential of large language models to classify these characteristics without requiring extensive annotated datasets. 
These characteristics are particularly useful for identifying statements that may indicate greenwashing. For example, \citet{bingler2023cheaptalkspecificitysentiment} utilized the proportion of non-specific commitments as a \textit{Cheap Talk Index} demonstrating their practical applications.
However, characteristics such as specificity are inherently ambiguous and subjective, as evidenced by the low inter-annotator agreement (IAA). Therefore, future research could focus on disambiguating what constitutes a specific statement from a non-specific one more objectively.  

\subsection{Green Stance Detection}
\label{sec:stance detection}

Beyond making claims about the environmental impact of their products and processes, companies contribute to environmental discussions and can have an impact on regulatory frameworks through their communications and industry presence. It is thus helpful to understand the stance of an organization on the existence and gravity of climate change, on climate mitigation and adaptation efforts, and on climate-related regulations. 

\begin{table}[ht]
\centering
\begin{tabular}{p{3cm}p{4cm}p{3cm}p{4cm}}
\toprule
\textbf{Dataset} & \textbf{Input} & \textbf{Labels}                                                                                  & \textbf{Label details}                                         \\
\midrule
ClimateFEVER (evidence) ~\cite{diggelmann_climate-fever_2020} 
& A claim and an evidence sentence from Wikipedia  & \textit{Support, Refutes, Not Enough Information} & Determines the relation between a claim and a single evidence sentence \\
\midrule
LobbyMap (Stance)~\cite{morio2023an} & Page from a company communications (report, press release, ...) & \textit{Strongly supporting, Supporting, No or mixed position, Not supporting, Opposing} & Given the policy and the page, classifies the stance \\
\midrule
Global Warming Stance Detection (GWSD)~\cite{luo_detecting_2020}  & Sentences from news about global warming & \multicolumn{2}{p{7cm}}{Stance of the evidence (\textit{Agree, Disagree, Neutral}) toward the claim: Climate-Change is a serious concern.} \\
\midrule
ClimateStance~\cite{vaid-etal-2022-towards} & Tweets posted during COP25 filtered by keywords (relevant to climate-change) & \multicolumn{2}{p{7cm}}{Stance towards climate change prevention: \textit{Favor, Against, Ambiguous}. (Stance used as a broad notion including sentiment, evaluation, appraisal, ...)}  \\
\midrule
Stance on Remediation Effort~\cite{lai_using_2023} & Texts extracted from the TCFD sections of the financial reports  & \multicolumn{2}{p{7cm}}{The text indicates \textit{support} for climate change remediation efforts or \textit{refutation} of such efforts. (No guidelines)} \\
\bottomrule
\multicolumn{4}{c}{\textbf{Related subtask}} \\
\toprule
ClimateFEVER (claim) ~\cite{diggelmann_climate-fever_2020}  & A claim and multiple evidence sentences from Wikipedia & \textit{Support, Refutes, Debated, Not Enough Information} & Determines if a claim is supported by a set of retrieved evidence sentences  \\
\midrule
LobbyMap (Page)~\cite{morio2023an} & \multirow{2}{4cm}[-0.25cm]{Page from a company communications (report, press release, ...)} & \textit{1/0} & Contains a stance on a remediation policy  \\
LobbyMap (Query)~\cite{morio2023an} & & \textit{GHG emission regulation, Renewable energy, Carbon tax, ...} & Classifies the remediation policy \\
\bottomrule
\end{tabular}
\caption{Summary of datasets, labels, scopes, and sources for climate-related stance detection.}
\label{tab:datasets_stance}
\end{table}

\task Given two input sentences or paragraphs, one labeled as the claim and one as the evidence, predict the stance between the two: supports, refutes or neutral. Some studies fix the claim and only vary the evidence (e.g. the claim is always \textit{Climate change poses a severe threat}), training a model to predict the stance of the evidence in respect to the fixed claim. 
Other studies train a model to predict the relation between varying claims/evidences.

\paragraph{Related subtask} The first subtask is collecting the evidences when the claims are already available (e.g. if they were collected manually). In order to build these datasets, research can rely on simple heuristics such as downloading all tweets published during the COP25 filtered with keywords \cite{vaid-etal-2022-towards}. However, other researchers performed a more elaborate procedure. \citet{diggelmann_climate-fever_2020} proposed a pipeline for collecting evidence. To retrieve relevant evidence from Wikipedia for a given claim, the pipeline involves three steps: document-level retrieval using entity-linking and BM25 to identify top articles, sentence-level retrieval using sentence embeddings trained on the FEVER dataset to extract relevant sentences, and sentence re-ranking using a pretrained ALBERT model to classify and rank evidence based on relevance. 
The second one encompasses the broader process: identifying claims and finding evidence, before predicting the relation. \citet{Wang2021EvidenceBA} used a generic claim detection model trained on ClaimBuster \cite{Arslan_Hassan_Li_Tremayne_2020} for the claims and Google Search API to collect evidence. 
Finally, the last subtask is to train multiple models on each step: identifying evidences, identifying the claim and classifying the stance. This is the approach used by \cite{morio2023an}.

\datasets As described previously, researchers built datasets of evidence related to one claim while focusing stance in tweets on seriousness of climate change such as Global Warming Stance Detection (GWSD) \cite{luo_detecting_2020}, ClimateStance \cite{vaid-etal-2022-towards} stance on climate change of tweets posted during the COP25, or \citet{lai_using_2023}'s dataset which focuses on the stance toward climate change remediation efforts.
Another research initiative built a dataset of claim-evidence pairs, ClimateFEVER, and then trained a model on the stance classification \cite{diggelmann_climate-fever_2020, Wang2021EvidenceBA}.
Finally,~\citet{morio2023an} proposed a dataset to assess corporate policy engagement built upon LobbyMap, which tackles the 3 steps: finding pages with evidence, identifying the claims targeted by that evidence, and classifying the stance.

\solutions The solution proposed for to tackle stance classification are fine-tuned Transformer models \cite{vaid-etal-2022-towards, vaghefi2022deep, xiang_dare_2023, morio2023an, nicolas_webersinke_climatebert_2021, Wang2021EvidenceBA, spokoyny2023answering, lai_using_2023, luo_detecting_2020} and classical approaches \cite{spokoyny2023answering, morio2023an, luo_detecting_2020}. \citet{vaid-etal-2022-towards} also experimented with FastText and \citet{xiang_dare_2023} with LSTM-based models (See Table \ref{tab:models} in the Appendix for details). The performances are quite heterogeneous, \citet{lai_using_2023} reaching F1-scores around 90\% on classification of stance on remediation efforts, \citet{luo_detecting_2020} around 72\% on GWSD, while performance on ClimateStance could not exceed 60\%. Performances on ClimateFEVER are also quite low, however, when selecting only non-ambiguous examples, \citet{xiang_dare_2023} could reach performances around 80\%. Finally, performances on LobbyMap~\cite{morio2023an} are quite low (between 31\% and 57.3\% depending on the strictness of the metric). Overall, the performances show that the datasets are challenging (see Table~\ref{tab:reported perf stance} in the Appendix for details).

\paragraph{Exhaustivity} %\fms{misplaced here. } 
Datasets such as ClimateFEVER~\cite{diggelmann_climate-fever_2020} and LobbyMap~\cite{morio2023an} rely on automated construction methods that may not ensure exhaustivity of evidence coverage. ClimateFEVER utilizes BM25 and Wikipedia for evidence retrieval, which could result in missing relevant information, particularly when compared to more advanced retrieval methods. LobbyMap relies on the \url{LobbyMap.org} website, which was not designed to be exhaustive. These limitations should be further investigated.

\paragraph{Insights} If an organization presents itself as environmentally friendly while simultaneously promoting climate-skeptic narratives or opposing climate remediation efforts, it probably is creating a misleading portrayal of its environmental stance. On the contrary, if they are aligned, it supports the authenticity of the organization's efforts.
Fortunately, stance detection has been extensively studied, yielding promising results; however, current performance levels leave room for improvement. Future research should focus on enhancing both methods and datasets to address these gaps.
While existing datasets provide strong foundations, they may suffer from a lack of exhaustivity, particularly in evidence retrieval and coverage, which requires further investigation. %Addressing these limitations will be critical for advancing the reliability of fact-checking systems in combating greenwashing.

\subsection{Question Answering}
\label{sec:qa}

Question answering (QA) is a known task:

\task Given an input question and a set of resources (paragraphs or documents), produce an answer to the question. 

\begin{table}[ht]
\centering
\begin{tabular}{p{2cm}p{5cm}p{3cm}p{4cm}}
\toprule
\textbf{Dataset} & \textbf{Input} & \textbf{Labels}                                                                                  &                                          \\
\midrule
ClimaQA ~\cite{spokoyny2023answering} 
& The text from a \textit{response} to one of the CDP questions; and one of the \textit{questions} from the CDP questionnaire  & \multicolumn{2}{p{7cm}}{\textit{1}: the response answers this question

\textit{0}: The response does not answer this question, but another one} \\
\midrule
ClimateQA~\cite{luccioni_analyzing_2020} & \textit{Sentence} from reports and a \textit{question} based on the TCFD recommendations & \multicolumn{2}{p{7cm}}{\textit{1}: the sentence answers the question

\textit{0}: The sentences does not answer the question} \\
\midrule
ClimaINS ~\cite{spokoyny2023answering} 
& The text from a \textit{response} to one of the questions from the NAIC questionnaire & \textit{MANAGE}, \textit{RISK PLAN}, \textit{MITIGATE}, \textit{ENGAGE}, \textit{ASSESS}, \textit{RISKS} & The labels correspond the 8 questions asked in the NAIC questionnaires \\
\bottomrule
\end{tabular}
\caption{Summary of datasets, labels, scopes, and sources for climate-related QA datasets.}
\label{tab:qa input}
\end{table}

QA can be used for climate-specific applications such as structuring the information related to climate change from documents~\cite{luccioni_analyzing_2020, tobias_schimanski_climatebert-netzero_2023}, building chatbots with climate-related knowledge to make it more accessible to non-experts~\cite{s_vaghefi_chatclimate_2023, cliamtebot_2022}, or helping identify potentially misleading information~\cite{jingwei_ni_paradigm_2023}. 

\paragraph{Solution} The QA process can be divided into two main steps~\cite{krausEnhancingLargeLanguage2023}: the retrieval step and the answer generation step. The retrieval step involves locating the relevant information or answer to the user's query in external documents. The answer generation step involves formulating a response to the user's query from (1) the information retrieved from documents in the first step or (2) the pre-trained internal knowledge of a model. 

\paragraph{Retriever} Retriever systems can be generic, using techniques such as sentence similarity, BM25, or document tags to filter documents and retrieve specific passages. \citet{schimanski-etal-2024-climretrieve} published ClimRetrieve, a benchmark for climate-related information retrieval in corporate reports. They found that simple embedding approaches are limited.
Therefore, several works have specifically focused on improving the identification of answers within climate-related documents~\cite{luccioni_analyzing_2020, spokoyny2023answering}. \citet{luccioni_analyzing_2020} reformulated TCFD recommendations as questions and annotated reports to identify answers to the questions. The dataset includes questions paired with potential answers, each labeled to indicate whether the answer addresses the question or not. Based on the QA model trained on this dataset, they also proposed a tool called ClimateQA. \citet{spokoyny2023answering} focused on questions from the CDP questionnaire and the NAIC Climate Risk Disclosure survey. As the responses to those questionnaires are publicly available, they constructed two datasets, ClimaINS and ClimaQA. ClimaQA is similar to ClimateQA (question/potential answer pairs). On the contrary, ClimaINS is a QA dataset framed as a classification task: it contains the responses from the NAIC survey, and each response is labeled as answering one of the 8 questions of the survey.  Once a model has been trained on these datasets, it can be used to search for answers to the questions in other documents (effectively retrieving the information from an unstructured document). 
For ClimaINS, the authors experimented with multiple fine-tuned Transformers. As ClimaQA is framed as a retrieval task, the authors used BM25, sentence-BERT, and ClimateBERT to rank possible answers. Using the methodology evaluated on ClimaQA, they proposed a system to extract responses to CDP questions directly from unstructured reports and automatically fill the questionnaire~\cite{spokoyny2023answering}. \citet{luccioni_analyzing_2020} trained and evaluated a RoBERTa-based model called ClimateQA on finding answers to TCFD-based questions. They all reached good performance but with room for improvement (see Table~\ref{tab:reported perf question asnwering (qa)} in the Appendix for details), demonstrating the feasibility of the task but also the need for further research.

\paragraph{Answer generation} After the retrieval, the second step of question answering is to generate an answer for the question (given input resources or not). The generation can be simple such as finding a particular information in a paragraph. For example, \citet{tobias_schimanski_climatebert-netzero_2023} used a QA model (RoBERTa SQuaD v2~\cite{rajpurkar-etal-2016-squad}) to extract the target year, the percentage of reduction, and the baseline year from the reduction target of a company from an input text that contains an emission reduction target. While it is possible to rely on generalist models, as they are proficient few-shot learners~\cite{lm_few_shot_learner, thulke2024climategpt}, multiple studies proposed using domain-specific models.  
Climate Bot~\cite{cliamtebot_2022} built a dataset of climate-specific QA. Given a question and a document (scientific/news), the model should find the span answering the question.  \citet{mullappilly-etal-2023-arabic} proposed a dataset of question-answer pairs based on ClimaBench~\cite{spokoyny2023answering} and CCMRC~\cite{cliamtebot_2022} to train a model to generate answers. \citet{thulke2024climategpt} introduced a climate-specific corpus of prompt/completion pairs for Instruction Fine-Tuning created by experts and non-experts. 
\cite{cliamtebot_2022} finetuned an ALBERT model on their climate-specific QA dataset. \citet{mullappilly-etal-2023-arabic} trained a Vicuna-7B on their climate-specific answer generation dataset. And \citet{thulke2024climategpt} trained LLama-2-based models (ClimateGPT) on their domain-specific prompt/completion dataset. They also evaluated their ClimateGPT model alongside multiple LLMs on climate-related benchmarks. They found that the domain-specific models outperformed generalist ones (see Table \ref{tab:reported perf question asnwering (qa)} in the Appendix for details).

\paragraph{Retrieval Augmented Generation} By combining the last two components (a retriever and a QA system), one can develop Retrieval-Augmented Generation (RAG) systems. \citet{cliamtebot_2022} provide the first example of a RAG system in climate-related tasks based on sentence-BERT to retrieve documents and AlBERT to identify the answer. More recently, \citet{jingwei_ni_paradigm_2023}~proposed ChatReport, a methodology based on ChatGPT for analyzing corporate reports through the lens of TCFD questions using RAG.  The reports, the answers to the TCFD questions, and a TCFD conformity assessment are chunked and stored in a vector store for easy retrieval. Each chunk and/or answer can be retrieved and injected into the prompt. They added in the prompt the notions of greenwashing and cheap talk to invite the model to provide a critical analysis of the retrieved answers. Similarly, \citet{s_vaghefi_chatclimate_2023} introduced ChatClimate, a RAG-based pipeline to augment GPT-4 with knowledge about IPCC reports. While the previous LLM-based approaches rely on closed-source models, \citet{mullappilly-etal-2023-arabic} and \citet{thulke2024climategpt} proposed RAG-based pipelines relying on open-source models.
\citet{cliamtebot_2022} is the only study that evaluated their RAG pipeline using classical metrics (F1-score, BLEU, METEOR). Unfortunately, those metrics are not well aligned with human judgment for text generation \cite{chen-etal-2019-evaluating}. Therefore \citet{s_vaghefi_chatclimate_2023} and \citet{mullappilly-etal-2023-arabic} relied on human or ChatGPT evaluations. They conclude that their approaches improve on non-domain-specific RAG systems.

\paragraph{Insights} The main application of Question-Answering (QA) systems is to analyze complex documents, such as corporate climate reports, that can be particularly useful for greenwashing detection. Modern LLMs achieve strong results in QA due to instruction-following fine-tuning, which improves adherence to specific instructions, and Retrieval-Augmented Generation (RAG), which bases its answers on retrieved documents. However, retrieval systems  remain a performance bottleneck~\cite{maekawa-etal-2024-retrieval}. Furthermore, answer generation is challenging to evaluate~\cite{survey_nlg_eval}, often requiring human feedback or advanced model-based assessments.


\subsection{Classification of Deceptive Techniques}
\label{sec:deceptive}

In analyzing climate-related discourse, it can be useful to identify rhetorical strategies that might obscure or misrepresent an entity’s stance. 
For example, expressing support for climate remediation policies on one side \cite{morio2023an}, and promoting arguments downplaying the urgency of climate change on the other \cite{coanComputerassistedClassificationContrarian2021} could signal a lack of authenticity. Another example could be promoting a product through misleading rhetoric, by, for example, claiming that a product is better because it is natural, which would be an ``appeal to nature'' -- a fallacious argument \cite{vaid-etal-2022-towards, jain_supply_2023}.
Detecting these deceptive techniques could help identify misleading communications.

\begin{table}[ht]
\centering
\begin{tabular}{p{3cm}p{3cm}p{4cm}p{4cm}}
\toprule
\textbf{Dataset}  & \textbf{Input}  & \textbf{Labels}                                                                                  & \textbf{(Labels details)}                                        \\
\midrule
LogicClimate~\cite{jin-etal-2022-logical}  & texts from climatefeedback.org  & \textit{Faulty Generalization, Ad Hominem, Ad Populum, False Causality, ...} & Classifies fallacies (Multi-label)\\
\midrule
\raggedright Neutralization Techniques~\cite{bhatia_automatic_2021-1} & paragraphs from other previous works on climate-change & \textit{Denial of Responsibility, Denial of Injury, Denial of Victim, Condemnation of the Condemner, ...} & Classifies neutralization techniques  \\
\midrule
Contrarian Claims~\cite{coanComputerassistedClassificationContrarian2021} & paragraphs from conservative think tank & \textit{No Claim, Global Warming is not happening, Climate Solutions won't work, Climate impacts are not bad, ...} & Classifies arguments into  super/sub-categories of climate science denier's arguments  \\
\bottomrule
\end{tabular}
\caption{Summary of datasets, labels, scopes, and sources for tasks related to deceptive techniques in climate-related context.}
\label{tab:datasets_deceptive}
\end{table}

\task The goal is to classify statements into argumentative categories: fallacies, types of arguments, or rhetorical techniques.

\zeroshot \citet{divinus_oppong-tawiah_corporate_2023} frame greenwashing as fake news. They propose to tackle this identification through the form perspective. 
They developed a profile-deviation-based method to detect greenwashing in corporate tweets by comparing linguistic cues (e.g., quantity, specificity, complexity, diversity, hedging, affect, and vividness) to theoretically ideal profiles of truthful and deceptive communication. They compute a greenwashing score as Euclidean distances.

\datasets An organization seeking to justify limited action might adopt a rhetoric that downplays the urgency of global warming or dismisses the impact of negligent behavior. These kinds of climate contrarian arguments mostly fall into a finite set of categories (e.g. ``Solutions will not work'', or ``Human influence is not demonstrated''). Therefore, \citet{coanComputerassistedClassificationContrarian2021} constructed a taxonomy of such contrarian claims and published a large dataset annotated with such claims. 
Similarly, \citet{bhatia_automatic_2021-1} proposed a dataset to classify neutralization techniques, i.e., rationalizations that individuals use to justify deviant or unethical behavior (e.g. Denial of the victim). Finally, \citet{jin-etal-2022-logical} studied fallacious arguments and how they apply to climate change more broadly. They constructed LogicClimate, a climate-specific dataset of sentences annotated with fallacies, using articles from the \href{https://science.feedback.org/reviews/?_topic=climate}{Climate Feedback website}.

\solutions Fine-tuned Transformer architectures and classical approaches have been evaluated on each dataset~\cite{jin-etal-2022-logical, bhatia_automatic_2021-1, coanComputerassistedClassificationContrarian2021}. \citet{jin-etal-2022-logical} experimented with an ELECTRA model, fine-tuned to use the structure of the argument. The performance of fine-tuned approaches ranges from 58.77\% to 79\%. The models struggle with fallacy detection in LogicClimate, but perform well with generalist fallacy detection, contrarian claims, and neutralization techniques. \citet{thulke2024climategpt} experimented with multiple zero-shot approaches using multiple LLM on the binary classification of contrarian claims using \citet{coanComputerassistedClassificationContrarian2021}'s dataset. Their climateGPT-70B and LLama-2-Chat-70B models both reached an F1-score of 72.5\% (see Table \ref{app:table deceptive techniques} in the Appendix for details). 

\paragraph{Insights} The low performance of models on tasks such as fallacy detection and neutralization classification indicates the inherent complexity of these tasks. Fallacy detection, for one, is known to be inherently subjective~\cite{helwe-etal-2024-mafalda}. %This difficulty arises from the subtlety and diversity of deceptive techniques and the limited size of available annotated datasets. While existing datasets, such as 
The detection of neutralization techniques, too, appears to be subjective, as even human annotators achieve only a moderate performance level of 70\% F1-score.
Furthermore, both LogicClimate~\cite{jin-etal-2022-logical} and the neutralization dataset~\cite{bhatia_automatic_2021-1} are very small in size.
To address these limitations, future research could explore several directions, including increasing the size of the datasets, defining  the labels more precisely, %trying to improve IAA, 
or permitting multiple correct annotations~\cite{helwe-etal-2024-mafalda} to acknowledge viable disagreement. Additionally, the analyses on rhetorical techniques could be combined with other analyses (such as stance detection) to search for communication patterns. 

\subsection{Environmental Performance Prediction}
\label{sec:env prediction}

Greenwashing can be interpreted as a misrepresentation of the company or product's environmental performance. The environmental performance is often summarized by a quantitative metric such as the ESG score, the Finch score\footnote{The Finch Score is a sustainability rating system designed to help consumers make eco-friendly choices by evaluating products on a scale from 1 to 10. See \url{https://www.choosefinch.com/}}, or CO2 emissions. These scores are usually not directly mentioned in the company reports, but have to be inferred based on company communications:
%represent the underlying performance of the company and can not directly be extracted from reports or product descriptions; they have to be inferred.  All the relevant element to infer those score should be avaible in company communication in the product description for the Finch score, and in the company reports for ESG scores.

%\input{latex/tikz_figure/predict_graph} Oana:we still need to cut a lot, so I removed this image

\task Given an input company report, output an environment-related quantitative value (such as the ESG score or the amount of Carbon Emission), even if that value is not mentioned in the report. In variants of this task, the goal is to predict not a value, but a range for that value, out of a set of possible ranges. 
%The task then becomes a classification problem.
%This is a prediction task of a continuous value. However, these scores can also be discretized, allowing for interpretation within a classification framework.

% ESG scoring is extensively studied, usually through traditional regression methods \citet{chowdhuryEnvironmentalSocialGovernance2023}. However, several works have proposed using language models to include communications and news events as features for predicting the score.

\zeroshot \citet{jain_supply_2023} proposed to predict the Scope 3 emissions %, not directly, but via % Fabian: what would ``directly'' mean?
from the list of financial transactions of the company. They first performed classification of all transaction descriptions to map them to their emission factors (emission per \$ spent), and then compute the emissions of each transaction. For ESG score prediction, \citet{bronzini_glitter_2023} proposes to use LLMs as few-shot learners to extract triples of ESG category, action, and company from sustainable reports, and to construct a graph representation with few-shot learning. They demonstrated the triplet generation using Alpaca, WizardLM, Flan-T5, and ChatGPT on a few examples. They could analyze disclosure choices and company similarities using the constructed graph. More importantly, they also used the graph to interpret ESG scores through the interpretability analysis of OLS predictions, effectively predicting the ESG score. 

\datasets \citet{Mehra_2022} proposed to predict the changes in the ESG score instead of the actual value. They constructed their dataset by extracting the three sentences most relevant to the environment from financial reports and associated them with the E score\footnote{ESG scores are usually aggregated scores along multiple dimensions. In this study, they are focusing on the "Environment" part of the ESG score.}'s change and direction of change. Instead of relying on scores, \citet{clarkson_nlp_us_csr} proposed to focus on good/bad CSR (corporate social responsibility) performers. Their approach evaluates CSR performance based on linguistic style rather than content, aiming to identify whether language patterns alone influence the perception of CSR quality. \citet{Greenscreen} introduced a multi-modal dataset of the tweets of companies to predict a company's ESG unmanaged risk. 
Finally, focusing on products and not companies, \citet{linSUSTAINABLESIGNALSijcai2023} introduced a dataset of online product descriptions and reviews used to predict the Finch Score. All these scores can be used to identify companies and products that are actually environmentally friendly, helping users distinguish actual sustainability from greenwashing. However, they can also be used to find discrepancies between the communicated green-ness of a product or company (e.g. the percentage of climate-related texts in reports) and the actual green-ness (e.g. ESG score or quantity of emissions), which might indicate cases of greenwashing.  

\solutions \citet{Mehra_2022} evaluated fine-tuned Transformers on ESG change prediction. \citet{clarkson_nlp_us_csr} experimented with hand-chosen features plugged into random forest and SVM on CSR performers prediction. \citet{linSUSTAINABLESIGNALSijcai2023} also experimented with traditional approaches (e.g. gradient boosting) and custom architectures based on Transformer models on Finch score prediction. \citet{Greenscreen} introduced baseline models, which are simple image and text embedding models (e.g., CLIP or sentence-BERT) with a classification head on ESG score prediction. A detailed list of models is reported in Table~\ref{tab:models} in the Appendix. \citet{bronzini_glitter_2023, clarkson_nlp_us_csr, Greenscreen} did not report baselines, which makes the study difficult to analyze. \cite{Mehra_2022} reported good accuracy, demonstrating that a few sentences hold a significant amount of information to predict ESG score changes. However, there is definitely room for improvement, indicating that the information selected is not sufficient. \citet{lin-etal-2023-linear} reported the performance of the baseline (average score) reaching already mean squared error (MSE) of 11.7\% already quite low. Their Transformer-based approach reached MSE of 7.4\% improving slightly on classical approaches (Gradient Boosting reaching MSE of 8.2\%). Detailed performances can be found in Table \ref{tab:appendix env pred} in the Appendix.

\paragraph{Insights} We expected the prediction of ESG and CSR metrics to be difficult and require an extensive understanding of a company. However, \citet{bronzini_glitter_2023, linSUSTAINABLESIGNALSijcai2023, Mehra_2022} and \citet{clarkson_nlp_us_csr} showed that it is possible to build strong predictors relying only on textual elements. This can be explained because analysts reward transparency~\cite{bronzini_glitter_2023}, so the quantity and complexity of disclosure have strong predictive power~\cite{clarkson_nlp_us_csr}. Although existing studies have examined specific dimensions, future research should investigate the interaction between form~\cite{clarkson_nlp_us_csr} and content~\cite{bronzini_glitter_2023}, as this could help uncover inconsistencies that may indicate greenwashing practices.


\section{Greenwashing Detection}
\label{sec: greenwashing signals}

After having discussed different subtasks of greenwashing detection, we now come to the final, all-englobing task:

\task Greenwashing detection is the task of predicting if a text contains greenwashing. This is a binary classification task. Greenwashing detection can also be done at an aggregated scale (such as by a yearly indicator).

\paragraph{Greenhushing and Selective Disclosure}  There is a large body of work on disclosure (as described in Section~\ref{sec: tcfd}). Quantifying climate-related disclosure can highlight greenhushing and selective disclosure. \citet{bingler_cheap_2021} identify a lack of disclosure in the Strategy and Metrics\&Targets categories, which they describe as \textit{cherry-picking}. They also highlight that merely announcing support for the TCFD does not lead to an increase in disclosure -- on the contrary, it is a practice called \textit{cheap talk}. Similarly, \citet{auzepy_evaluating_2023} conducted a more fine-grained analysis of TCFD-related disclosure in the banking industry. They also highlighted large differences across TCFD categories. In particular, they found a low disclosure rate on the fossil-fuel industry-related topic despite large investments in that sector.

%(omission type greenwashing \citet{defreitasnettoConceptsFormsGreenwashing2020a}).

\paragraph{Climate Communication as an Image-Building Strategy} \citet{dingCarbonEmissionsTCFD2023} and \citet{chou_ESG} found a correlation between climate-related disclosure and carbon emission: companies that emit more tend to disclose more climate-related information. \citet{bingler2023cheaptalkspecificitysentiment} proposed a Cheap talk index based on claims specificity. They reached the same conclusion that larger emitters tend to disclose more, but also that negative news coverage is correlated with cheap talk. \citet{marco_polignano_nlp_2022} showed that while disclosing more, reports mostly focus on positive disclosure.  \citet{hyewon_kang_analyzing_2022} proposed a sentiment ratio metric that highlights overly positive corporate reports. They identified periods of overly positive communications that followed negative environmental controversies; in other words, companies that tried to rebuild their image after a controversy. 
\citet{csr_report_greenwashing} analyzed CSR reports of environmental violators and companies with clean records. They studied an ensemble of variables: quantification of environmental content, readability score, and sentiment analysis. They found that violators publish longer, more positive, and less readable CSR reports with more environment-related content.
This suggests that companies use climate disclosure as a tool to mitigate controversies and/or to improve their image. \citet{kdir23} propose to use the discrepancies between internal disclosure and social media perception of a company to identify potential greenwashing.

\paragraph{The Role of Disclosure Style in Perceived Commitment} \citet{clarkson_nlp_us_csr} found that companies that use a more complex language in their CSR disclosure are associated with a higher CSR rating. \citet{schimanski_bridging_2023} concluded that a higher quantity of ESG communication is associated with higher ESG rating. \citet{rouenEvolutionESGReports2023} also highlighted the relationship between disclosure quantity and complexity with the ESG score. This might indicate that the linguistic style is a good predictor of the substance of the discourse, or that analysts are rewarding companies that communicate on ESG-related issues.

\paragraph{ESG Score in Greenwashing Detection} \citet{LEE_greenwashing} proposed a greenwashing index based on the difference between the ESG score and the ESG score, weighted by the quantity of communication on each topic (E, S and G). 
%The quantity of communication is computed using dictionaries. 
\citet{Greenscreen} define two types of risks: managed risk (the company is addressing it), and unmanaged risk (the company is not currently addressing it but it could). Based on these, they propose an ESG unmanaged risk score as a measure for environmental performance, which can then be used as a signal for identifying greenwashing.
%\citet{rouenEvolutionESGReports2023} study concluded an increase in disclosure of material information over immaterial information, but also an uniformisation of the language.
% and while this does not highlight greenwashing, this shows that analysts might be influenced by the type of language and not only by the content. 
% linked to the GRI topics.

% {\color{gray}
% % shorter version : 
% While other give signals that might be interpreted as greenwashing, without mentioning greenwashing: \citet{bingler_cheap_2021, auzepy_evaluating_2023} highlights the low disclosure rate on the fossil fuel industry-related topic, \citet{bingler_cheap_2021, bingler2023cheaptalkspecificitysentiment} also talk about cheap talks and cherry picking but not explicitly about greenwashing, \citet{dingCarbonEmissionsTCFD2023, chou_ESG} found a correlation between the climate-related information disclosure and carbon emission (higher emitter disclose more). Similarly \citet{clarkson_nlp_us_csr} found that better CSR performer are associated with more advanced CSR disclosure but poor performer disclose more negative sentiment.
% }

%to study the link between cheap talk and various effects such as the introduction of the TCFD recommendation.
% They studied 14,584 annual reports from 2010 to 2020 and arrived at the same conclusion as their previous approach \citettbingler_cheap_2021}.

% give the list of time the papers talks about greenwashing

% While the previous work can be interpreted as greenwashing signals, they do not specifically focus on greenwashing. The following works explicitly mention greenwashing.

 % sentiment based
\paragraph{Contrasting Stances: A Method to Identify Greenwashing in Climate Communications} \citet{morio2023an} proposed to evaluate a company's stance %polarity 
towards climate change mitigation policies. They hypothesize that companies with a mixed stance might be engaging in greenwashing. The hypothesis was not tested and remains unconfirmed; still, such methodology helps understand the narratives of companies around climate change. \citet{coanComputerassistedClassificationContrarian2021} were able to analyze the type of contrarian claims used by conservative think-tank (CTTs) websites and contrarian blogs. They showed a shift over time from climate change denying arguments to opposing climate change mitigation policies. Despite this shift, they found that the primary donors of these think tanks continue to support organizations promoting a climate-denier narrative aimed at discrediting scientific evidence and scientists. Discrepancies between the stance extracted from official communications~\cite{morio2023an} on the one hand, and the stance of third-party media organizations such as think tanks founded by the company \cite{coanComputerassistedClassificationContrarian2021} on the other, might indicate greenwashing.
%DistilBERT fine-tuned on SST-2.
%(computed with a DistilBERT fine-tuned on SST-2)
%DistilBERT base uncased fine-tuned SST-2

%\paragraph{Using Media Perception as ground truth to identify Greenwashing} \citet{kdir23} propose to use the discrepancies between internal disclosure and social media perception of a company to identify potential greenwashing. They applied FinBERT-ESG-9-Categories for ESG classification of internal documents and TextBlob for sentiment analysis of media perception. Oana: moved part of this
 
%They built a dataset of social media communications labeled using the ESG unmanaged risk score. 

% stance based
%such as "Energy transition & zero carbon technologies
% 
\paragraph{Defining the linguistic features of greenwashing} \citet{divinus_oppong-tawiah_corporate_2023} we propose to identify greenwashing using the linguistic profile of tweets. They estimate the deceptiveness of the text using a keyword-based approach and linguistic indicators (e.g. word quantity, sentence quantity). They found a correlation between greenwashing and lower financial performances. A scoring system based solely on linguistic elements has significant limitations. For example, the following tweet gets a  high greenwashing score: ``Read about \#[company’s] commitment to a \#lowcarbon future http://[company website]''. However, it is merely an invitation to read a Web page. 
%This raises the question of whether such statements constitute greenwashing. While the vagueness of the language may contribute to perceptions of misleading intent, the statement merely conveys a commitment, which, in itself, does not provide sufficient evidence of deceptive practices.

% \citet{bhatia_automatic_2021-1} 

%However, has this metric rely on public opinion, it might only indicate known cases of greenwashing and might also be bias by sector, as the study focus only on the pharmaceutical sector.


%focused specifically on greenwashing detection on Twitter. They framed greenwashing as 'Sustainability Fake News'. [...] 

% \datasets While most of the work previously described could help produce a weakly annotated dataset, only \citet{avalon_vinella_leveraging_2023} constructed such a dataset. They used a linear regression fitted on a really small sample of 10 examples, to then asign weak labels on a larger dataset. 

% \solutions \citet{avalon_vinella_leveraging_2023} they finetuned climateBERT on their weakly labeled dataset. 

% \citet{avalon_vinella_leveraging_2023} models could perform quite well, on a seemingly difficult task. However, due to the methodology to construct the dataset, the model essentially learns to predict simultaneously all 4 characteristics. 


% {\color{gray}

% % \citet{schimanski_bridging_2023} concluded that the quantity of ESG communication is associated with higher ESG rating (from Bloomberg, Refinitiv Asset4, and RobecoSAM). This shows that ESG analysts seem to discourage green-hushing.  Once again they highlight the relationship between the disclosure quantity and complexity with the ESG Score similarly to \citet{schimanski_bridging_2023}. 
% \citet{bhatia_automatic_2021-1} define greenwashing as fakenews = deception vocabulary

% }

% These measures of greenwashing demonstrated their potential by identifying known cases of controversies (such as Toyota in 2011 \cite{hyewon_kang_analyzing_2022}). However they have strong limitations such as relying on weak labels \cite{avalon_vinella_leveraging_2023, LEE_greenwashing, Greenscreen} or lagging behind public perception \cite{kdir23}. Moreover, they rely mostly on superficial cues (sentiment \cite{hyewon_kang_analyzing_2022, kdir23} or linguistic \cite{divinus_oppong-tawiah_corporate_2023}). 

% As discussed in section \ref{sec:intro} and \ref{sec:definitions}, there are multiple definitions for greenwashing. While using definition \ref{def:poor_perf}, the approaches using ESG scores can actually identify greenwashing, however using definition \ref{def:greenwashing}, using actual actions from companies might be necessary. 

\paragraph{Insights} Several indicators for greenwashing have been proposed: 
%, yet, as highlighted by \citet{measuring_greenwashing}, these approaches are largely constructed from a theoretical standpoint rather than being informed by empirical examples. These theoretical definitions focus on several characteristics, including 
overly positive sentiment \cite{hyewon_kang_analyzing_2022}, the use of specific linguistic cues~\cite{divinus_oppong-tawiah_corporate_2023}, discrepancies between ESG scores and corporate disclosures~\cite{LEE_greenwashing, Greenscreen}, ambiguous stances~\cite{morio2023an}, and inconsistencies between social media perception and official disclosures~\cite{kdir23}.
While these approaches laid the theoretical groundwork for understanding indicators of greenwashing, they have a significant limitation: they are not empirically evaluated~\cite{measuring_greenwashing}. %Most existing metrics have not been rigorously tested against real-world examples. 
Only a few studies, such as that by \citet{hyewon_kang_analyzing_2022}, have made attempts to validate their signals against actual cases, but even these efforts have not resulted in a comprehensive dataset of greenwashing examples.
This gap between theory and practice highlights the necessity of developing datasets containing real-world instances of greenwashing. Without empirical evaluation, the prediction that a company engages in greenwashing is unfounded at best, and misleading or even defamatory at worst. %A robust dataset is crucial for challenging, refining, and practically assessing the detection methods built upon theoretical definitions. 
However, building such a dataset comes with many challenges. First, one would have to identify suitable sources of documents that are openly accessible and likely to contain greenwashing. Second, one would have to overcome the inconspicuous and subjective nature of greenwashing itself. Finally, the publication of any such dataset exposes the authors to charges of defamation by the companies they accuse of greenwashing.

\section{Discussion}
The development of foundation models has increasingly relied on accessible data support to address complex tasks~\cite{zhang2024data}. Yet major challenges remain in collecting scalable clinical data in the healthcare system, such as data silos and privacy concerns. To overcome these challenges, MedForge integrates multi-center clinical knowledge sources into a cohesive medical foundation model via a collaborative scheme. MedForge offers a collaborative path to asynchronously integrate multi-center knowledge while maintaining strong flexibility for individual contributors.
This key design allows a cost-effective collaboration among clinical centers to build comprehensive medical models, enhancing private resource utilization across healthcare systems.

Inspired by collaborative open-source software development~\cite{raffel2023building, github}, our study allows individual clinical institutions to independently develop branch modules with their data locally. These branch modules are asynchronously integrated into a comprehensive model without the need to share original data, avoiding potential patient raw data leakage. Conceptually similar to the open-source collaborative system, iterative module merging development ensures the aggregation of model knowledge over time while incorporating diverse data insights from distributed institutions. In particular, this asynchronous scheme alleviates the demand for all users to synchronize module updates as required by conventional methods (e.g., LoRAHub~\cite{huang2023lorahub}).


MedForge's framework addresses multiple data challenges in the cycle of medical foundation model development, including data storage, transmission, and leakage. As the data collection process requires a large amount of distributed data, we show that dataset distillation contributes greatly to reducing data storage capacity. In MedForge, individual contributors can simply upload a lightweight version of the dataset to the central model developer. As a result, the remarkable reduction in data volume (e.g., 175 times less in LC25000) alleviates the burden of data transfer among multiple medical centers. For example, we distilled a 10,500 image training set into 60 representative distilled data while maintaining a strong model performance. We choose to use a lightweight dataset as a transformed representation of raw data to avoid the leakage of sensitive raw information.
Second, the asynchronous collaboration mode in MedForge allows flexible model merging, particularly for users from various local medical centers to participate in model knowledge integration. 
Third, MedForge reformulates the conventional top-down workflow of building foundational models by adopting a bottom-up approach. Instead of solely relying on upstream builders to predefine model functionalities, MedForge allows medical centers to actively contribute to model knowledge integration by providing plugin modules (i.e., LoRA) and distilled datasets. This approach supports flexible knowledge integration and allows models to be applicable to wide-ranging clinical tasks, addressing the key limitation of fixed functionalities in traditional workflows.

We demonstrate the strong capacity of MedForge via the asynchronous merging of three image classification tasks. MedForge offered an incremental merging strategy that is highly flexible compared to plain parameter average~\cite{wortsman2022model} and LoRAHub~\cite{huang2023lorahub}. Specifically, plain parameter averaging merges module parameters directly and ignores the contribution differences of each module. Although LoRAHub allows for flexible distribution of coefficients among modules, it lacks the ability to continuously update, limiting its capacity to incorporate new knowledge during the merging process. In contrast, MedForge shows its strong flexibility of continuous updates while considering the contribution differences among center modules. The robustness of MedForge has been demonstrated by shuffling merging order (Tab~\ref{tab:order}), which shows that merging new-coming modules will not hurt the model ability of previous tasks in various orders, mitigating the model catastrophic forgetting. 
MedForge also reveals a strong generality on various choices of component modules. Our experiments on dataset distillation settings (such as DC and without DSA technique) and PEFT techniques (such as DoRA) emphasize the extensible ability of MedForge's module settings. 

To fully exploit multi-scale clinical data, it will be necessary to include broader data modalities (e.g., electronic health records and radiological images). Managing these diverse data formats and standards among numerous contributors can be challenging due to the potential conflict between collaborators. 
Moreover, since MedForge integrates multiple clinical tasks that involve varying numbers of classification categories, conventional classifier heads with fixed class sizes are not applicable. However, the projection head of the CLIP model, designed to calculate similarities between image and text, is well-suited for this scenario. It allows MedForge to flexibly handle medical datasets with different category numbers, thus overcoming the challenge of multi-task classification. That said, this design choice also limits the variety of model architectures that can be utilized, as it depends specifically on the CLIP framework. Future investigations will explore extensive solutions to make the overall architecture more flexible. Additionally, incorporating more sophisticated data anonymization, such as synthetic data generation~\cite{ding2023large}, and encryption methods can also be considerable. To improve data privacy protection, test-time adaptation technique~\cite{wang2020tent, liang2024comprehensive} without substantial training data can be considered to alleviate the burden of data sharing in the healthcare system.



             

\section{Conclusion}
We reveal a tradeoff in robust watermarks: Improved redundancy of watermark information enhances robustness, but increased redundancy raises the risk of watermark leakage. We propose DAPAO attack, a framework that requires only one image for watermark extraction, effectively achieving both watermark removal and spoofing attacks against cutting-edge robust watermarking methods. Our attack reaches an average success rate of 87\% in detection evasion (about 60\% higher than existing evasion attacks) and an average success rate of 85\% in forgery (approximately 51\% higher than current forgery studies).    


\bibliography{custom}
\bibliographystyle{ACM-Reference-Format}

\clearpage
% Appendix Title Page
\begin{titlepage}
    \centering
    \vspace*{\fill} % Center vertically
    {\LARGE Appendix \\[1.5em]} % Adjust font size
    % {\Large Title of Appendix} % Subtitle if needed
    \vspace*{\fill}
\end{titlepage}

\appendix
% To be removed if we don't need it
\fancyfoot[R]{Supplemental Online-only Material}

\subsection{Lloyd-Max Algorithm}
\label{subsec:Lloyd-Max}
For a given quantization bitwidth $B$ and an operand $\bm{X}$, the Lloyd-Max algorithm finds $2^B$ quantization levels $\{\hat{x}_i\}_{i=1}^{2^B}$ such that quantizing $\bm{X}$ by rounding each scalar in $\bm{X}$ to the nearest quantization level minimizes the quantization MSE. 

The algorithm starts with an initial guess of quantization levels and then iteratively computes quantization thresholds $\{\tau_i\}_{i=1}^{2^B-1}$ and updates quantization levels $\{\hat{x}_i\}_{i=1}^{2^B}$. Specifically, at iteration $n$, thresholds are set to the midpoints of the previous iteration's levels:
\begin{align*}
    \tau_i^{(n)}=\frac{\hat{x}_i^{(n-1)}+\hat{x}_{i+1}^{(n-1)}}2 \text{ for } i=1\ldots 2^B-1
\end{align*}
Subsequently, the quantization levels are re-computed as conditional means of the data regions defined by the new thresholds:
\begin{align*}
    \hat{x}_i^{(n)}=\mathbb{E}\left[ \bm{X} \big| \bm{X}\in [\tau_{i-1}^{(n)},\tau_i^{(n)}] \right] \text{ for } i=1\ldots 2^B
\end{align*}
where to satisfy boundary conditions we have $\tau_0=-\infty$ and $\tau_{2^B}=\infty$. The algorithm iterates the above steps until convergence.

Figure \ref{fig:lm_quant} compares the quantization levels of a $7$-bit floating point (E3M3) quantizer (left) to a $7$-bit Lloyd-Max quantizer (right) when quantizing a layer of weights from the GPT3-126M model at a per-tensor granularity. As shown, the Lloyd-Max quantizer achieves substantially lower quantization MSE. Further, Table \ref{tab:FP7_vs_LM7} shows the superior perplexity achieved by Lloyd-Max quantizers for bitwidths of $7$, $6$ and $5$. The difference between the quantizers is clear at 5 bits, where per-tensor FP quantization incurs a drastic and unacceptable increase in perplexity, while Lloyd-Max quantization incurs a much smaller increase. Nevertheless, we note that even the optimal Lloyd-Max quantizer incurs a notable ($\sim 1.5$) increase in perplexity due to the coarse granularity of quantization. 

\begin{figure}[h]
  \centering
  \includegraphics[width=0.7\linewidth]{sections/figures/LM7_FP7.pdf}
  \caption{\small Quantization levels and the corresponding quantization MSE of Floating Point (left) vs Lloyd-Max (right) Quantizers for a layer of weights in the GPT3-126M model.}
  \label{fig:lm_quant}
\end{figure}

\begin{table}[h]\scriptsize
\begin{center}
\caption{\label{tab:FP7_vs_LM7} \small Comparing perplexity (lower is better) achieved by floating point quantizers and Lloyd-Max quantizers on a GPT3-126M model for the Wikitext-103 dataset.}
\begin{tabular}{c|cc|c}
\hline
 \multirow{2}{*}{\textbf{Bitwidth}} & \multicolumn{2}{|c|}{\textbf{Floating-Point Quantizer}} & \textbf{Lloyd-Max Quantizer} \\
 & Best Format & Wikitext-103 Perplexity & Wikitext-103 Perplexity \\
\hline
7 & E3M3 & 18.32 & 18.27 \\
6 & E3M2 & 19.07 & 18.51 \\
5 & E4M0 & 43.89 & 19.71 \\
\hline
\end{tabular}
\end{center}
\end{table}

\subsection{Proof of Local Optimality of LO-BCQ}
\label{subsec:lobcq_opt_proof}
For a given block $\bm{b}_j$, the quantization MSE during LO-BCQ can be empirically evaluated as $\frac{1}{L_b}\lVert \bm{b}_j- \bm{\hat{b}}_j\rVert^2_2$ where $\bm{\hat{b}}_j$ is computed from equation (\ref{eq:clustered_quantization_definition}) as $C_{f(\bm{b}_j)}(\bm{b}_j)$. Further, for a given block cluster $\mathcal{B}_i$, we compute the quantization MSE as $\frac{1}{|\mathcal{B}_{i}|}\sum_{\bm{b} \in \mathcal{B}_{i}} \frac{1}{L_b}\lVert \bm{b}- C_i^{(n)}(\bm{b})\rVert^2_2$. Therefore, at the end of iteration $n$, we evaluate the overall quantization MSE $J^{(n)}$ for a given operand $\bm{X}$ composed of $N_c$ block clusters as:
\begin{align*}
    \label{eq:mse_iter_n}
    J^{(n)} = \frac{1}{N_c} \sum_{i=1}^{N_c} \frac{1}{|\mathcal{B}_{i}^{(n)}|}\sum_{\bm{v} \in \mathcal{B}_{i}^{(n)}} \frac{1}{L_b}\lVert \bm{b}- B_i^{(n)}(\bm{b})\rVert^2_2
\end{align*}

At the end of iteration $n$, the codebooks are updated from $\mathcal{C}^{(n-1)}$ to $\mathcal{C}^{(n)}$. However, the mapping of a given vector $\bm{b}_j$ to quantizers $\mathcal{C}^{(n)}$ remains as  $f^{(n)}(\bm{b}_j)$. At the next iteration, during the vector clustering step, $f^{(n+1)}(\bm{b}_j)$ finds new mapping of $\bm{b}_j$ to updated codebooks $\mathcal{C}^{(n)}$ such that the quantization MSE over the candidate codebooks is minimized. Therefore, we obtain the following result for $\bm{b}_j$:
\begin{align*}
\frac{1}{L_b}\lVert \bm{b}_j - C_{f^{(n+1)}(\bm{b}_j)}^{(n)}(\bm{b}_j)\rVert^2_2 \le \frac{1}{L_b}\lVert \bm{b}_j - C_{f^{(n)}(\bm{b}_j)}^{(n)}(\bm{b}_j)\rVert^2_2
\end{align*}

That is, quantizing $\bm{b}_j$ at the end of the block clustering step of iteration $n+1$ results in lower quantization MSE compared to quantizing at the end of iteration $n$. Since this is true for all $\bm{b} \in \bm{X}$, we assert the following:
\begin{equation}
\begin{split}
\label{eq:mse_ineq_1}
    \tilde{J}^{(n+1)} &= \frac{1}{N_c} \sum_{i=1}^{N_c} \frac{1}{|\mathcal{B}_{i}^{(n+1)}|}\sum_{\bm{b} \in \mathcal{B}_{i}^{(n+1)}} \frac{1}{L_b}\lVert \bm{b} - C_i^{(n)}(b)\rVert^2_2 \le J^{(n)}
\end{split}
\end{equation}
where $\tilde{J}^{(n+1)}$ is the the quantization MSE after the vector clustering step at iteration $n+1$.

Next, during the codebook update step (\ref{eq:quantizers_update}) at iteration $n+1$, the per-cluster codebooks $\mathcal{C}^{(n)}$ are updated to $\mathcal{C}^{(n+1)}$ by invoking the Lloyd-Max algorithm \citep{Lloyd}. We know that for any given value distribution, the Lloyd-Max algorithm minimizes the quantization MSE. Therefore, for a given vector cluster $\mathcal{B}_i$ we obtain the following result:

\begin{equation}
    \frac{1}{|\mathcal{B}_{i}^{(n+1)}|}\sum_{\bm{b} \in \mathcal{B}_{i}^{(n+1)}} \frac{1}{L_b}\lVert \bm{b}- C_i^{(n+1)}(\bm{b})\rVert^2_2 \le \frac{1}{|\mathcal{B}_{i}^{(n+1)}|}\sum_{\bm{b} \in \mathcal{B}_{i}^{(n+1)}} \frac{1}{L_b}\lVert \bm{b}- C_i^{(n)}(\bm{b})\rVert^2_2
\end{equation}

The above equation states that quantizing the given block cluster $\mathcal{B}_i$ after updating the associated codebook from $C_i^{(n)}$ to $C_i^{(n+1)}$ results in lower quantization MSE. Since this is true for all the block clusters, we derive the following result: 
\begin{equation}
\begin{split}
\label{eq:mse_ineq_2}
     J^{(n+1)} &= \frac{1}{N_c} \sum_{i=1}^{N_c} \frac{1}{|\mathcal{B}_{i}^{(n+1)}|}\sum_{\bm{b} \in \mathcal{B}_{i}^{(n+1)}} \frac{1}{L_b}\lVert \bm{b}- C_i^{(n+1)}(\bm{b})\rVert^2_2  \le \tilde{J}^{(n+1)}   
\end{split}
\end{equation}

Following (\ref{eq:mse_ineq_1}) and (\ref{eq:mse_ineq_2}), we find that the quantization MSE is non-increasing for each iteration, that is, $J^{(1)} \ge J^{(2)} \ge J^{(3)} \ge \ldots \ge J^{(M)}$ where $M$ is the maximum number of iterations. 
%Therefore, we can say that if the algorithm converges, then it must be that it has converged to a local minimum. 
\hfill $\blacksquare$


\begin{figure}
    \begin{center}
    \includegraphics[width=0.5\textwidth]{sections//figures/mse_vs_iter.pdf}
    \end{center}
    \caption{\small NMSE vs iterations during LO-BCQ compared to other block quantization proposals}
    \label{fig:nmse_vs_iter}
\end{figure}

Figure \ref{fig:nmse_vs_iter} shows the empirical convergence of LO-BCQ across several block lengths and number of codebooks. Also, the MSE achieved by LO-BCQ is compared to baselines such as MXFP and VSQ. As shown, LO-BCQ converges to a lower MSE than the baselines. Further, we achieve better convergence for larger number of codebooks ($N_c$) and for a smaller block length ($L_b$), both of which increase the bitwidth of BCQ (see Eq \ref{eq:bitwidth_bcq}).


\subsection{Additional Accuracy Results}
%Table \ref{tab:lobcq_config} lists the various LOBCQ configurations and their corresponding bitwidths.
\begin{table}
\setlength{\tabcolsep}{4.75pt}
\begin{center}
\caption{\label{tab:lobcq_config} Various LO-BCQ configurations and their bitwidths.}
\begin{tabular}{|c||c|c|c|c||c|c||c|} 
\hline
 & \multicolumn{4}{|c||}{$L_b=8$} & \multicolumn{2}{|c||}{$L_b=4$} & $L_b=2$ \\
 \hline
 \backslashbox{$L_A$\kern-1em}{\kern-1em$N_c$} & 2 & 4 & 8 & 16 & 2 & 4 & 2 \\
 \hline
 64 & 4.25 & 4.375 & 4.5 & 4.625 & 4.375 & 4.625 & 4.625\\
 \hline
 32 & 4.375 & 4.5 & 4.625& 4.75 & 4.5 & 4.75 & 4.75 \\
 \hline
 16 & 4.625 & 4.75& 4.875 & 5 & 4.75 & 5 & 5 \\
 \hline
\end{tabular}
\end{center}
\end{table}

%\subsection{Perplexity achieved by various LO-BCQ configurations on Wikitext-103 dataset}

\begin{table} \centering
\begin{tabular}{|c||c|c|c|c||c|c||c|} 
\hline
 $L_b \rightarrow$& \multicolumn{4}{c||}{8} & \multicolumn{2}{c||}{4} & 2\\
 \hline
 \backslashbox{$L_A$\kern-1em}{\kern-1em$N_c$} & 2 & 4 & 8 & 16 & 2 & 4 & 2  \\
 %$N_c \rightarrow$ & 2 & 4 & 8 & 16 & 2 & 4 & 2 \\
 \hline
 \hline
 \multicolumn{8}{c}{GPT3-1.3B (FP32 PPL = 9.98)} \\ 
 \hline
 \hline
 64 & 10.40 & 10.23 & 10.17 & 10.15 &  10.28 & 10.18 & 10.19 \\
 \hline
 32 & 10.25 & 10.20 & 10.15 & 10.12 &  10.23 & 10.17 & 10.17 \\
 \hline
 16 & 10.22 & 10.16 & 10.10 & 10.09 &  10.21 & 10.14 & 10.16 \\
 \hline
  \hline
 \multicolumn{8}{c}{GPT3-8B (FP32 PPL = 7.38)} \\ 
 \hline
 \hline
 64 & 7.61 & 7.52 & 7.48 &  7.47 &  7.55 &  7.49 & 7.50 \\
 \hline
 32 & 7.52 & 7.50 & 7.46 &  7.45 &  7.52 &  7.48 & 7.48  \\
 \hline
 16 & 7.51 & 7.48 & 7.44 &  7.44 &  7.51 &  7.49 & 7.47  \\
 \hline
\end{tabular}
\caption{\label{tab:ppl_gpt3_abalation} Wikitext-103 perplexity across GPT3-1.3B and 8B models.}
\end{table}

\begin{table} \centering
\begin{tabular}{|c||c|c|c|c||} 
\hline
 $L_b \rightarrow$& \multicolumn{4}{c||}{8}\\
 \hline
 \backslashbox{$L_A$\kern-1em}{\kern-1em$N_c$} & 2 & 4 & 8 & 16 \\
 %$N_c \rightarrow$ & 2 & 4 & 8 & 16 & 2 & 4 & 2 \\
 \hline
 \hline
 \multicolumn{5}{|c|}{Llama2-7B (FP32 PPL = 5.06)} \\ 
 \hline
 \hline
 64 & 5.31 & 5.26 & 5.19 & 5.18  \\
 \hline
 32 & 5.23 & 5.25 & 5.18 & 5.15  \\
 \hline
 16 & 5.23 & 5.19 & 5.16 & 5.14  \\
 \hline
 \multicolumn{5}{|c|}{Nemotron4-15B (FP32 PPL = 5.87)} \\ 
 \hline
 \hline
 64  & 6.3 & 6.20 & 6.13 & 6.08  \\
 \hline
 32  & 6.24 & 6.12 & 6.07 & 6.03  \\
 \hline
 16  & 6.12 & 6.14 & 6.04 & 6.02  \\
 \hline
 \multicolumn{5}{|c|}{Nemotron4-340B (FP32 PPL = 3.48)} \\ 
 \hline
 \hline
 64 & 3.67 & 3.62 & 3.60 & 3.59 \\
 \hline
 32 & 3.63 & 3.61 & 3.59 & 3.56 \\
 \hline
 16 & 3.61 & 3.58 & 3.57 & 3.55 \\
 \hline
\end{tabular}
\caption{\label{tab:ppl_llama7B_nemo15B} Wikitext-103 perplexity compared to FP32 baseline in Llama2-7B and Nemotron4-15B, 340B models}
\end{table}

%\subsection{Perplexity achieved by various LO-BCQ configurations on MMLU dataset}


\begin{table} \centering
\begin{tabular}{|c||c|c|c|c||c|c|c|c|} 
\hline
 $L_b \rightarrow$& \multicolumn{4}{c||}{8} & \multicolumn{4}{c||}{8}\\
 \hline
 \backslashbox{$L_A$\kern-1em}{\kern-1em$N_c$} & 2 & 4 & 8 & 16 & 2 & 4 & 8 & 16  \\
 %$N_c \rightarrow$ & 2 & 4 & 8 & 16 & 2 & 4 & 2 \\
 \hline
 \hline
 \multicolumn{5}{|c|}{Llama2-7B (FP32 Accuracy = 45.8\%)} & \multicolumn{4}{|c|}{Llama2-70B (FP32 Accuracy = 69.12\%)} \\ 
 \hline
 \hline
 64 & 43.9 & 43.4 & 43.9 & 44.9 & 68.07 & 68.27 & 68.17 & 68.75 \\
 \hline
 32 & 44.5 & 43.8 & 44.9 & 44.5 & 68.37 & 68.51 & 68.35 & 68.27  \\
 \hline
 16 & 43.9 & 42.7 & 44.9 & 45 & 68.12 & 68.77 & 68.31 & 68.59  \\
 \hline
 \hline
 \multicolumn{5}{|c|}{GPT3-22B (FP32 Accuracy = 38.75\%)} & \multicolumn{4}{|c|}{Nemotron4-15B (FP32 Accuracy = 64.3\%)} \\ 
 \hline
 \hline
 64 & 36.71 & 38.85 & 38.13 & 38.92 & 63.17 & 62.36 & 63.72 & 64.09 \\
 \hline
 32 & 37.95 & 38.69 & 39.45 & 38.34 & 64.05 & 62.30 & 63.8 & 64.33  \\
 \hline
 16 & 38.88 & 38.80 & 38.31 & 38.92 & 63.22 & 63.51 & 63.93 & 64.43  \\
 \hline
\end{tabular}
\caption{\label{tab:mmlu_abalation} Accuracy on MMLU dataset across GPT3-22B, Llama2-7B, 70B and Nemotron4-15B models.}
\end{table}


%\subsection{Perplexity achieved by various LO-BCQ configurations on LM evaluation harness}

\begin{table} \centering
\begin{tabular}{|c||c|c|c|c||c|c|c|c|} 
\hline
 $L_b \rightarrow$& \multicolumn{4}{c||}{8} & \multicolumn{4}{c||}{8}\\
 \hline
 \backslashbox{$L_A$\kern-1em}{\kern-1em$N_c$} & 2 & 4 & 8 & 16 & 2 & 4 & 8 & 16  \\
 %$N_c \rightarrow$ & 2 & 4 & 8 & 16 & 2 & 4 & 2 \\
 \hline
 \hline
 \multicolumn{5}{|c|}{Race (FP32 Accuracy = 37.51\%)} & \multicolumn{4}{|c|}{Boolq (FP32 Accuracy = 64.62\%)} \\ 
 \hline
 \hline
 64 & 36.94 & 37.13 & 36.27 & 37.13 & 63.73 & 62.26 & 63.49 & 63.36 \\
 \hline
 32 & 37.03 & 36.36 & 36.08 & 37.03 & 62.54 & 63.51 & 63.49 & 63.55  \\
 \hline
 16 & 37.03 & 37.03 & 36.46 & 37.03 & 61.1 & 63.79 & 63.58 & 63.33  \\
 \hline
 \hline
 \multicolumn{5}{|c|}{Winogrande (FP32 Accuracy = 58.01\%)} & \multicolumn{4}{|c|}{Piqa (FP32 Accuracy = 74.21\%)} \\ 
 \hline
 \hline
 64 & 58.17 & 57.22 & 57.85 & 58.33 & 73.01 & 73.07 & 73.07 & 72.80 \\
 \hline
 32 & 59.12 & 58.09 & 57.85 & 58.41 & 73.01 & 73.94 & 72.74 & 73.18  \\
 \hline
 16 & 57.93 & 58.88 & 57.93 & 58.56 & 73.94 & 72.80 & 73.01 & 73.94  \\
 \hline
\end{tabular}
\caption{\label{tab:mmlu_abalation} Accuracy on LM evaluation harness tasks on GPT3-1.3B model.}
\end{table}

\begin{table} \centering
\begin{tabular}{|c||c|c|c|c||c|c|c|c|} 
\hline
 $L_b \rightarrow$& \multicolumn{4}{c||}{8} & \multicolumn{4}{c||}{8}\\
 \hline
 \backslashbox{$L_A$\kern-1em}{\kern-1em$N_c$} & 2 & 4 & 8 & 16 & 2 & 4 & 8 & 16  \\
 %$N_c \rightarrow$ & 2 & 4 & 8 & 16 & 2 & 4 & 2 \\
 \hline
 \hline
 \multicolumn{5}{|c|}{Race (FP32 Accuracy = 41.34\%)} & \multicolumn{4}{|c|}{Boolq (FP32 Accuracy = 68.32\%)} \\ 
 \hline
 \hline
 64 & 40.48 & 40.10 & 39.43 & 39.90 & 69.20 & 68.41 & 69.45 & 68.56 \\
 \hline
 32 & 39.52 & 39.52 & 40.77 & 39.62 & 68.32 & 67.43 & 68.17 & 69.30  \\
 \hline
 16 & 39.81 & 39.71 & 39.90 & 40.38 & 68.10 & 66.33 & 69.51 & 69.42  \\
 \hline
 \hline
 \multicolumn{5}{|c|}{Winogrande (FP32 Accuracy = 67.88\%)} & \multicolumn{4}{|c|}{Piqa (FP32 Accuracy = 78.78\%)} \\ 
 \hline
 \hline
 64 & 66.85 & 66.61 & 67.72 & 67.88 & 77.31 & 77.42 & 77.75 & 77.64 \\
 \hline
 32 & 67.25 & 67.72 & 67.72 & 67.00 & 77.31 & 77.04 & 77.80 & 77.37  \\
 \hline
 16 & 68.11 & 68.90 & 67.88 & 67.48 & 77.37 & 78.13 & 78.13 & 77.69  \\
 \hline
\end{tabular}
\caption{\label{tab:mmlu_abalation} Accuracy on LM evaluation harness tasks on GPT3-8B model.}
\end{table}

\begin{table} \centering
\begin{tabular}{|c||c|c|c|c||c|c|c|c|} 
\hline
 $L_b \rightarrow$& \multicolumn{4}{c||}{8} & \multicolumn{4}{c||}{8}\\
 \hline
 \backslashbox{$L_A$\kern-1em}{\kern-1em$N_c$} & 2 & 4 & 8 & 16 & 2 & 4 & 8 & 16  \\
 %$N_c \rightarrow$ & 2 & 4 & 8 & 16 & 2 & 4 & 2 \\
 \hline
 \hline
 \multicolumn{5}{|c|}{Race (FP32 Accuracy = 40.67\%)} & \multicolumn{4}{|c|}{Boolq (FP32 Accuracy = 76.54\%)} \\ 
 \hline
 \hline
 64 & 40.48 & 40.10 & 39.43 & 39.90 & 75.41 & 75.11 & 77.09 & 75.66 \\
 \hline
 32 & 39.52 & 39.52 & 40.77 & 39.62 & 76.02 & 76.02 & 75.96 & 75.35  \\
 \hline
 16 & 39.81 & 39.71 & 39.90 & 40.38 & 75.05 & 73.82 & 75.72 & 76.09  \\
 \hline
 \hline
 \multicolumn{5}{|c|}{Winogrande (FP32 Accuracy = 70.64\%)} & \multicolumn{4}{|c|}{Piqa (FP32 Accuracy = 79.16\%)} \\ 
 \hline
 \hline
 64 & 69.14 & 70.17 & 70.17 & 70.56 & 78.24 & 79.00 & 78.62 & 78.73 \\
 \hline
 32 & 70.96 & 69.69 & 71.27 & 69.30 & 78.56 & 79.49 & 79.16 & 78.89  \\
 \hline
 16 & 71.03 & 69.53 & 69.69 & 70.40 & 78.13 & 79.16 & 79.00 & 79.00  \\
 \hline
\end{tabular}
\caption{\label{tab:mmlu_abalation} Accuracy on LM evaluation harness tasks on GPT3-22B model.}
\end{table}

\begin{table} \centering
\begin{tabular}{|c||c|c|c|c||c|c|c|c|} 
\hline
 $L_b \rightarrow$& \multicolumn{4}{c||}{8} & \multicolumn{4}{c||}{8}\\
 \hline
 \backslashbox{$L_A$\kern-1em}{\kern-1em$N_c$} & 2 & 4 & 8 & 16 & 2 & 4 & 8 & 16  \\
 %$N_c \rightarrow$ & 2 & 4 & 8 & 16 & 2 & 4 & 2 \\
 \hline
 \hline
 \multicolumn{5}{|c|}{Race (FP32 Accuracy = 44.4\%)} & \multicolumn{4}{|c|}{Boolq (FP32 Accuracy = 79.29\%)} \\ 
 \hline
 \hline
 64 & 42.49 & 42.51 & 42.58 & 43.45 & 77.58 & 77.37 & 77.43 & 78.1 \\
 \hline
 32 & 43.35 & 42.49 & 43.64 & 43.73 & 77.86 & 75.32 & 77.28 & 77.86  \\
 \hline
 16 & 44.21 & 44.21 & 43.64 & 42.97 & 78.65 & 77 & 76.94 & 77.98  \\
 \hline
 \hline
 \multicolumn{5}{|c|}{Winogrande (FP32 Accuracy = 69.38\%)} & \multicolumn{4}{|c|}{Piqa (FP32 Accuracy = 78.07\%)} \\ 
 \hline
 \hline
 64 & 68.9 & 68.43 & 69.77 & 68.19 & 77.09 & 76.82 & 77.09 & 77.86 \\
 \hline
 32 & 69.38 & 68.51 & 68.82 & 68.90 & 78.07 & 76.71 & 78.07 & 77.86  \\
 \hline
 16 & 69.53 & 67.09 & 69.38 & 68.90 & 77.37 & 77.8 & 77.91 & 77.69  \\
 \hline
\end{tabular}
\caption{\label{tab:mmlu_abalation} Accuracy on LM evaluation harness tasks on Llama2-7B model.}
\end{table}

\begin{table} \centering
\begin{tabular}{|c||c|c|c|c||c|c|c|c|} 
\hline
 $L_b \rightarrow$& \multicolumn{4}{c||}{8} & \multicolumn{4}{c||}{8}\\
 \hline
 \backslashbox{$L_A$\kern-1em}{\kern-1em$N_c$} & 2 & 4 & 8 & 16 & 2 & 4 & 8 & 16  \\
 %$N_c \rightarrow$ & 2 & 4 & 8 & 16 & 2 & 4 & 2 \\
 \hline
 \hline
 \multicolumn{5}{|c|}{Race (FP32 Accuracy = 48.8\%)} & \multicolumn{4}{|c|}{Boolq (FP32 Accuracy = 85.23\%)} \\ 
 \hline
 \hline
 64 & 49.00 & 49.00 & 49.28 & 48.71 & 82.82 & 84.28 & 84.03 & 84.25 \\
 \hline
 32 & 49.57 & 48.52 & 48.33 & 49.28 & 83.85 & 84.46 & 84.31 & 84.93  \\
 \hline
 16 & 49.85 & 49.09 & 49.28 & 48.99 & 85.11 & 84.46 & 84.61 & 83.94  \\
 \hline
 \hline
 \multicolumn{5}{|c|}{Winogrande (FP32 Accuracy = 79.95\%)} & \multicolumn{4}{|c|}{Piqa (FP32 Accuracy = 81.56\%)} \\ 
 \hline
 \hline
 64 & 78.77 & 78.45 & 78.37 & 79.16 & 81.45 & 80.69 & 81.45 & 81.5 \\
 \hline
 32 & 78.45 & 79.01 & 78.69 & 80.66 & 81.56 & 80.58 & 81.18 & 81.34  \\
 \hline
 16 & 79.95 & 79.56 & 79.79 & 79.72 & 81.28 & 81.66 & 81.28 & 80.96  \\
 \hline
\end{tabular}
\caption{\label{tab:mmlu_abalation} Accuracy on LM evaluation harness tasks on Llama2-70B model.}
\end{table}

%\section{MSE Studies}
%\textcolor{red}{TODO}


\subsection{Number Formats and Quantization Method}
\label{subsec:numFormats_quantMethod}
\subsubsection{Integer Format}
An $n$-bit signed integer (INT) is typically represented with a 2s-complement format \citep{yao2022zeroquant,xiao2023smoothquant,dai2021vsq}, where the most significant bit denotes the sign.

\subsubsection{Floating Point Format}
An $n$-bit signed floating point (FP) number $x$ comprises of a 1-bit sign ($x_{\mathrm{sign}}$), $B_m$-bit mantissa ($x_{\mathrm{mant}}$) and $B_e$-bit exponent ($x_{\mathrm{exp}}$) such that $B_m+B_e=n-1$. The associated constant exponent bias ($E_{\mathrm{bias}}$) is computed as $(2^{{B_e}-1}-1)$. We denote this format as $E_{B_e}M_{B_m}$.  

\subsubsection{Quantization Scheme}
\label{subsec:quant_method}
A quantization scheme dictates how a given unquantized tensor is converted to its quantized representation. We consider FP formats for the purpose of illustration. Given an unquantized tensor $\bm{X}$ and an FP format $E_{B_e}M_{B_m}$, we first, we compute the quantization scale factor $s_X$ that maps the maximum absolute value of $\bm{X}$ to the maximum quantization level of the $E_{B_e}M_{B_m}$ format as follows:
\begin{align}
\label{eq:sf}
    s_X = \frac{\mathrm{max}(|\bm{X}|)}{\mathrm{max}(E_{B_e}M_{B_m})}
\end{align}
In the above equation, $|\cdot|$ denotes the absolute value function.

Next, we scale $\bm{X}$ by $s_X$ and quantize it to $\hat{\bm{X}}$ by rounding it to the nearest quantization level of $E_{B_e}M_{B_m}$ as:

\begin{align}
\label{eq:tensor_quant}
    \hat{\bm{X}} = \text{round-to-nearest}\left(\frac{\bm{X}}{s_X}, E_{B_e}M_{B_m}\right)
\end{align}

We perform dynamic max-scaled quantization \citep{wu2020integer}, where the scale factor $s$ for activations is dynamically computed during runtime.

\subsection{Vector Scaled Quantization}
\begin{wrapfigure}{r}{0.35\linewidth}
  \centering
  \includegraphics[width=\linewidth]{sections/figures/vsquant.jpg}
  \caption{\small Vectorwise decomposition for per-vector scaled quantization (VSQ \citep{dai2021vsq}).}
  \label{fig:vsquant}
\end{wrapfigure}
During VSQ \citep{dai2021vsq}, the operand tensors are decomposed into 1D vectors in a hardware friendly manner as shown in Figure \ref{fig:vsquant}. Since the decomposed tensors are used as operands in matrix multiplications during inference, it is beneficial to perform this decomposition along the reduction dimension of the multiplication. The vectorwise quantization is performed similar to tensorwise quantization described in Equations \ref{eq:sf} and \ref{eq:tensor_quant}, where a scale factor $s_v$ is required for each vector $\bm{v}$ that maps the maximum absolute value of that vector to the maximum quantization level. While smaller vector lengths can lead to larger accuracy gains, the associated memory and computational overheads due to the per-vector scale factors increases. To alleviate these overheads, VSQ \citep{dai2021vsq} proposed a second level quantization of the per-vector scale factors to unsigned integers, while MX \citep{rouhani2023shared} quantizes them to integer powers of 2 (denoted as $2^{INT}$).

\subsubsection{MX Format}
The MX format proposed in \citep{rouhani2023microscaling} introduces the concept of sub-block shifting. For every two scalar elements of $b$-bits each, there is a shared exponent bit. The value of this exponent bit is determined through an empirical analysis that targets minimizing quantization MSE. We note that the FP format $E_{1}M_{b}$ is strictly better than MX from an accuracy perspective since it allocates a dedicated exponent bit to each scalar as opposed to sharing it across two scalars. Therefore, we conservatively bound the accuracy of a $b+2$-bit signed MX format with that of a $E_{1}M_{b}$ format in our comparisons. For instance, we use E1M2 format as a proxy for MX4.

\begin{figure}
    \centering
    \includegraphics[width=1\linewidth]{sections//figures/BlockFormats.pdf}
    \caption{\small Comparing LO-BCQ to MX format.}
    \label{fig:block_formats}
\end{figure}

Figure \ref{fig:block_formats} compares our $4$-bit LO-BCQ block format to MX \citep{rouhani2023microscaling}. As shown, both LO-BCQ and MX decompose a given operand tensor into block arrays and each block array into blocks. Similar to MX, we find that per-block quantization ($L_b < L_A$) leads to better accuracy due to increased flexibility. While MX achieves this through per-block $1$-bit micro-scales, we associate a dedicated codebook to each block through a per-block codebook selector. Further, MX quantizes the per-block array scale-factor to E8M0 format without per-tensor scaling. In contrast during LO-BCQ, we find that per-tensor scaling combined with quantization of per-block array scale-factor to E4M3 format results in superior inference accuracy across models. 

\end{document}
\endinput
%%
%% End of file `sample-manuscript.tex'.
