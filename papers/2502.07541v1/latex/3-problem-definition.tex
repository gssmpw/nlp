\section{Preliminaries} % Fabian: again, I propose to upgrade to a section to be in line with what readers would expect
\label{sec:definitions}

\subsection{Legal Context}
In this section, we provide the legal context for the definition of greenwashing. 
We first present international and national laws, followed by how they are applied in the corporate context. 


\paragraph{International and Nation Level Legislation and Guidelines on Climate Change Mitigation}
The first international summit on the effect of humans on the environment was the United Nations Conference on the Human Environment, held in 1972 in Sweden~\cite{timelineclimate}. The \textbf{Intergovernmental Panel on Climate Change (IPCC)} was created in 1988 and released its first report in 1990 ~\cite{change1990ipcc}. It was only in 1992 that the first treaty between countries was signed, the United Nations Framework Convention on Climate Change (UNFCCC)~\cite{timelineclimate}. The treaty only encouraged countries to reduce their emissions, and in 1997, the Kyoto Protocol set commitments for the countries to follow~\cite{kyotoprotocol}. The 36 countries that participated in the Protocol reduced their emissions in 2008–2012 by a large margin in respect to the levels in 1990; however, the global emissions increased by $32\%$ in the same period. 
Since the creation of UNFCCC, the nations, i.e., the parties, have met yearly at the ``Conference of the Parties'' (\textbf{COP}). 

The three main directions of \textbf{climate change policy} are reducing greenhouse emissions, promoting renewable energy, and improving energy efficiency. One major milestone was the introduction in 2005 of the European Union Emissions Trading System, the world's first large-scale emissions trading scheme. The trading system limits the amount of CO2 emitted by European industries and covers $46\%$ of the EU's CO2 emissions. It is estimated that the trading system reduced CO2 emissions in the EU by $3.8\%$ between 2008 and 2016~\cite{bayer2020european}. At COP21, in 2015, 194 nations plus the European Union signed the Paris Agreement, a treaty by which the nations commit to keep the rise in global surface temperature below 2 °C (3.6 °F) above pre-industrial levels. To achieve this goal, each country sends a national climate action plan every five years, and the parties assess the collective progress made towards achieving the climate goals. 
The first such assessment took place at COP28, where it was established that an important direction to achieve the long-term goals of countries was to transition away from fossil fuels to renewable energy. In 2021, the European Climate Law was adopted: the EU commits to reducing its emissions by at least $55\%$ by 2030 with respect to the 1990 levels and becoming climate neutral by 2050~\cite{euclimatelaw}. 
Similarly, countries such as Canada~\cite{canadaclimatelaw}, Taiwan~\cite{taiwanclimatelaw}, South Korea~\cite{koreaclimatelaw}, and Australia~\cite{australiaclimatelaw}, among others, aim to achieve carbon neutrality by 2050 and have passed laws setting this goal.

\paragraph{Corporate Level Laws and Guidelines} Laws and regulations at international and national levels have repercussions on companies that must comply or risk fines. However, assessing if a company has taken the necessary steps is not trivial, given that some laws refer to how a company will operate in the future, for example, by polluting less. Even without laws, investors can be concerned about environmental or social issues; hence, companies have used voluntary disclosures for a long time.  

One of the first organizations to provide standards for reporting on climate change or social aspects is the  Global Reporting Initiative (GRI), with its first guidelines published in 2000~\cite{grihistory}.
In 2015, the Financial Stability Board and the Group of 20 created the \textbf{Task Force for Climate-related Financial Disclosure (TCFD)} guidelines on disclosure in response to shortcomings of COP21, in particular the lack of standards climate-related disclosure~\cite{enwiki:1257716178}.
In 2021, the International Sustainability Standards Board (ISSB) was created to establish standards for climate-related disclosure, and starting in 2024, the standard released by this board will be applied worldwide~\cite{ifrs}. ISSB standards were aligned with GRI disclosure standards to make them complementary and interoperable~\cite{gri}, while the ISSB standards are expected to take over TCFD~\cite{ifrs}.
Differently from the voluntary reporting standards of GRI, TCFD, and ISSB, the Corporate sustainability reporting law in the EU required the creation of the European Sustainability Reporting Standards (ESRS), which are mandatory for companies subject to EU law~\cite{esrs}. ESRS has high interoperability with GRI and ISSB. GRI, TCFD, ESRS and ISSB fall under the umbrella term of environmental, social, and governance (ESG) guidelines for company disclosure. 
The European Union created its first law obliging companies to provide non-financial disclosure reporting in 2014, the Non-Financial Reporting Directive, which focused disclosure on environmental and social aspects. In 2023, the EU expanded this legislation via the Corporate Sustainability Reporting Law. Some European countries anticipated this legislation with their own, for example, the 2001 New Economic Regulations Act in France. Switzerland, which is not part of the EU, has imposed a mandatory TCFD disclosure for large public companies, banks, and insurance companies starting from January 2024~\cite{disclosureswiss}, and similar laws exist in New Zealand~\cite{disclosurezealand}. While not compulsory, initiatives like the Carbon Disclosure Project (CDP) have played a pivotal role by standardizing responses related to the climate disclosure of a company via structured questionnaires, thereby facilitating more systematic and comparable reporting.  


%Science-based targets (SBTi) \fms{what is this? an organization? created by whom? Governemnt?} was created in 2015 and it does not only rely on voluntary disclosure but also works with companies to set targets aligned with the Paris Agreement. SBTi provides a list of companies that set a target\footnote{\url{https://sciencebasedtargets.org/companies-taking-action}}, while also detailing sector-specific documentation on the methodology to set the target. Oana: this can go away, it is indeed one initiative of many, and it is not related to a legal framework as it is voluntary

\subsection{Definition of Greenwashing}

A widely cited and comprehensive definition, synthesizing those commonly found in the literature, is provided by the Oxford English Dictionary~\cite{GreenwashMeaningsEtymology2023}.

\begin{definition}\textbf{Greenwashing:}
    \label{def:greenwashing}
    To mislead the public (or to counter public or media concerns) by falsely representing a person, company, product, etc., as environmentally responsible.
\end{definition}


While individuals, companies, or countries can all engage in greenwashing, we will focus on climate-related greenwashing by companies in this survey, with the following definition: 
\begin{definition}
\label{def:greenwashing2}
   \textbf{Corporate climate-related greenwashing:} To mislead the public into falsely representing the effort made by a company to achieve its carbon transition.
\end{definition}

As climate-related disclosures face increasing regulation and greenwashing poses significant risks to a company's reputation, some companies adopt a strategy of silence, avoiding discussions about their environmental impact. This deliberate lack of communication is known as \textit{greenhushing}~\cite{Letzing}. These definitions imply that greenwashing is a deliberate act; however, in many cases, it results from an error or miscommunication by companies genuinely trying to showcase their sustainability efforts, and that are trying to best follow disclosure standards.

As shown by Definitions \ref{def:greenwashing} and \ref{def:greenwashing2}, greenwashing is not defined by easily identifiable properties but as a general concept. Since the concept is so unspecific, researchers focused on components indicative of potentially misleading communications but easier to define.  
Because of this, in this review we are mentioning ``greenwashing'' explicitly, but also paraphrasing it as ``misleading communications'', ``misrepresentation of the company's environmental impact/stance/performance'', or mentioning only components of it such as ``cheap talk'', ``selective disclosure/transparency'', ``deceptive techniques'', ``biased narrative''. They should all be understood in the context of climate-related misleading communications, as components associated or indicative of potential greenwashing even if they are not synonymous.

% Climate-related disclosure is becoming increasingly regulated, and guidelines and recommendations are numerous, which might help distinguish greenwashing from facts and information. However, the obligations and recommendations are not accessible to those without expertise. They are numerous and spread across multiple documents in different organizations, which may be national or supranational. Therefore, NLP techniques are increasingly employed to analyze and understand large quantities of reports and company communications.

%\fms{The following section seems out of place here, too. In particular, it is not clear why the structure of the text does not match the structure of the sections. Let me try to add an introductory sentence to the next section, so that the following subsection can go away (also in the interest of space)} Oana: I will incorporate part of the following in the introduction, when we describe the sections. 

\begin{comment}
    
%This definition has been rendered more concrete by the Green Claims Directive proposed in 2023 by the European Union~\cite{eugreenclaims}. It defines as greenwashing the practices of ``making an environmental claim related to future environmental performance without clear, objective and verifiable commitments and targets and an independent monitoring system'', ``displaying a sustainability label which is not based on a certification scheme or not established by public authorities'', ``making an environmental claim about the entire product when it concerns only a certain aspect of the product'', ``making a generic environmental claim for which the trader is not able to demonstrate recognized excellent environmental performance relevant to the claim'', or ``presenting requirements imposed by law on all products in the relevant product category on the Union market as a distinctive feature of the trader’s offer''.  Oana: I finnaly removed this as we already mention it in the introduction and it is not actually a definition that it is used in any of the papers we looked at, as it is a new law. - it allows us also to gain space.

%This the definition we will use for the rest of this survey.
%We provide a comprehensive survey of the scientific literature addressing the automated detection of greenwashing in textual data. 
%Of particular importance is \textbf{corporate greenwashing, with a focus on climate-related greenwashing}, i.e., greenwashing that misleads the public about the effort made by a company to achieve its carbon transition. 
%Given the urgency of climate change deadlines, we will emphasize more approaches that can deal with climate-related greenwashing. 

\subsection{Tasks addressed in survey}

Understanding and analyzing climate-related corporate communication has become an essential task in the context of increasing environmental scrutiny. 
However, a relatively recent concept, greenwashing complicates this landscape by introducing the risk of misleading or exaggerating claims about environmental commitments. Identifying greenwashing is inherently complex, as it requires addressing multiple dimensions of communication. Consequently, many works focus on more straightforward, more specific tasks as stepping stones toward the broader goal of detecting greenwashing. This review includes intermediary tasks that contribute to identifying potentially misleading climate-related communication. 

\paragraph{Identifying company content on climate-change:}
Detecting communication that is explicitly or implicitly related to climate topics. This task encompasses identifying climate-related statements, analyzing topics aligned with frameworks like the Task Force on Climate-related Financial Disclosures (TCFD), Environmental, Social, and Governance (ESG) factors, and other sustainability-related themes , and assessing the presence of green claims (see Sections: \ref{sec:climate-related topic}, \ref{sec:sub-topics}, \ref{sec:green claim}).

%\paragraph{Identifying subtopics in company content on climate-change:}
%Understanding the content of corporate communication requires determining the focus of the discussion. This includes analyzing topics aligned with frameworks like the Task Force on Climate-related Financial Disclosures (TCFD), Environmental, Social, and Governance (ESG) factors, and other sustainability-related themes (see Section: \ref{sec: tcfd}, \ref{sec:esg}, \ref{sec:sub-topics}).

\paragraph{Analyzing how companies communicate about climate change :}
Beyond identifying content, it is critical to evaluate the style and intent of communication. This involves examining sentiment, argumentation quality, deceptive practices, and stance to discern the underlying tone and authenticity of the message (see Sections: \ref{sec: climate risk}, \ref{sec: claim characteristics}, \ref{sec:stance detection}, \ref{sec:qa}, \ref{sec:deceptive}).

Finally, we review all approaches proposed to identify greenwashing, integrating insights from these intermediary tasks to assess their contributions to detecting and mitigating misleading corporate communication in the climate domain. This structured approach enables a deeper understanding of the challenges and potential solutions to effectively address greenwashing.

\end{comment}