\section{Related Work} 
\label{sec:related}
% Fabian: I propose to upgrade to a section -- this is more in line with what is usually done, although admittedly it's a very short section...

% Very few surveys have studied automated greenwashing detection. 
\citet{moodaley_greenwashing_2023} analyze the literature on greenwashing, sustainability reporting, and the intersection with research in Artificial Intelligence (AI) and Machine Learning (ML). The authors provide an overview of how the literature has changed over time by examining the number of publications and the different keywords covered by such publications. Unlike our work, this survey does not describe the actual approaches of the publications.
\citet{measuring_greenwashing} review the definitions of greenwashing in business, management, and accounting. They conclude that there is no agreed-upon definition of greenwashing, except that it is defined as a gap between the company's actions and its disclosure. In contrast to our study, the review provides only a brief overview of the techniques for detecting greenwashing.
While our review focuses on the semantic and linguistic aspects of greenwashing identification, other works focus on quantitative approaches that can serve the same objective~\cite{dao2024introduction}.

Other works are less directly related to our survey. For example, a growing community is focused on using AI to tackle climate change\footnote{See for example \url{https://www.climatechange.ai/}}. \citet{Rolnick_tackling_climate} review how machine learning has been used and could be used to tackle climate change. They explore different sectors of the economy, such as electricity, transportation, industry, and agriculture, among others. The survey discusses climate investments and the effects of climate change on finance, but not greenwashing. 
\citet{alonsoMachineLearningMethods2023} provide a survey on ML approaches for climate finance, also known in the literature as green finance or carbon finance. Climate finance is a subfield in economics dedicated to providing tools from financial economics to tackle climate change. Finally, several works study the impact of AI on climate change~\cite{hershcovich_towards_2022,kaack_aligning_2022,rohde_broadening_2023,verdecchia_systematic_2023}. Our work, in contrast, surveys approaches that use AI to detect greenwashing.

%\fms{right?}.
%Oana, I rephrased, they do some statistics on which methods are used, but it is very shallow: From a methodological perspective, GW has been investigated mainly by means of quantitative approaches such as regressions or experimental design (96 cases), including mixed methods that take a quantitative and qualitative approach. Nevertheless, there are also qualitative approaches, such as case studies, field studies, text analysis, and descriptive statistics (31 cases).

%This definition is the preferred one in empirical studies as it can be operationalized more easily than other greenwashing techniques such as misleading claims in text or implicit claims in advertising. The authors hypothesise that the lack of a clear definition could be the result of a lack of a large dataset of real world examples of greenwashing.  \fms{it would be great to end each paragraph with a brief discussion of the shortcomings of that related work, and how we are different}

% In \cite{t_cojoianu_greenwatch-shing_2020}, the authors present an opinion paper on using AI for detecting greenwashing. In particular, they mention an ongoing work on introducing a tool for the detection of greenwashing. Oana: Actually, this is not really a survey, do not include

% \cite{thulke2024climategpt} propose a multi purpose LLM model that is trained on a variety of tasks and datasets, such as classification, QA, summarization on climate related topics. Oana: this does not belong here, currently also in section 5.4.2

% \cite{systematicsurveyESG} ESG is extensive topic - here they collect 55K papers and use NLP to analyse them, I would not add this survey, not really meaningfull


%In \cite{hershcovich_towards_2022}, the authors study how NLP papers published at ACL venues discuss the climate impact of training large language models. 
%The authors compile a list of questions that authors can fill in, to create more transparency around this issue. 
%A holistic perspective of assessing the impact of AI on climate is proposed in~\cite{kaack_aligning_2022}, by considering not just the impact of training the AI models, but also the impact of the applications they will be used on.
%, for example a positive effect when used to mitigate climate change though a more efficient use of transportation, energy or land. Sometimes however, improving efficiency could lead to an overall negative effect: for example while ML can make applications more efficient, it also requires a lot of infrastructure to be deployed, hence the overall climate impact might be negative, in particular in data intensive applications.
%A survey on the social, economic, and environmental impacts of AI and also a work that proposes indicators for measuring sustainable AI, \cite{rohde_broadening_2023} is aligned with the recommendations of~\cite{kaack_aligning_2022}, of going beyond the actual ML model and looking at how it is integrated in society. 
%\cite{verdecchia_systematic_2023} is a survey on Green AI, that highlights trends in the research area, for example works that focus on the trade-off between energy consumption and energy, i.e. large deep neural networks for which we can reduce the size without significantly impacting the quality of the results.






%Green claim detection (conceptual) \cite{moodaley_conceptual_2023} not relevant




