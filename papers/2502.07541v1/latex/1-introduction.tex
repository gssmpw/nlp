\section{Introduction}
\label{sec:intro}



%\oana{Something that we need to clarify is if we can really only talk about climate change or we have to discuss also about environmental pollution? for example because they are too interlinked in the tasks we study. This has to be clear everywhere, from abstract to conclusions.} % Fabian: I agree! Tom?
%\tom{So we don't talk only about climate. The greenwashing section is about greenwashing detection only. Then all the intermediary task are more about climate, but there are a lot of related papers that do not talk about climate. I think it's: Literature review on greenwashing detection, from the paper that mention greenwashing, we talk about the intermediary task that they mention, but also we try to focus on climate only, but it is not necessarly the case for the papers. So it's not clear to me either. But overall, it's more climate-related greenwashing, but in the literature there is not that much on that anyway. I don't know the score that we shoudl mention here.
%Also what i always had in mind the climate-related greenwashing, but i included anything Greenwashing+NLP. But for the task, we have for example Climate detection. And i did not deep dive into Environment more broadly such as plastic, pollution, ...
%I would say that the scope is:
%- coporate greenwashing 
%- the topic that got the more attention is climate, so that's skewed toward corporate climate greenwashing, but indeed it's larger than this.
%Can we say that we wanted to focus on Climate Corporate Greenwashing, but in the end it's linked to overall greenwashing ? Because if a work is broad, it also include climate-greenwashing. If a work was including climate, we included it, but if did not include climate e.g. plastic only then we did not ?}


%%%%%%%%%%%%% Climate change is important

Since the first report of the Intergovernmental Panel on Climate Change in 1990 ~\cite{change1990ipcc},
world leaders and citizens have increasingly recognized the significant adverse impacts of human activities on the environment, particularly on climate change~\cite{climatebriefhistory}. This increased awareness has translated into guidelines, laws, and investments, such as the European Green Deal~\cite{europeangreendeal} or the Inflation Reduction Act in the US~\cite{inflationact}. Many companies have used the financial incentives offered by states, and the guidelines and legislation to make significant steps towards sustainability~\cite{greencompanies}. At the same time, this growing attention also generated an advertising opportunity for companies that aim to promote themselves as environmentally aware and responsible. 
%%%%%%%%%%%%%% Greenwashing is important
Indeed, %although this race to sustainability is beneficial to the shift toward a greener future, in some cases, 
some companies have been found to deliberately manipulate their data and statistics to appear more environment-friendly. The Diesel Scandal around the Volkswagen car company is a prominent example~\cite{volkswagen}. However, such cases are not the norm. More commonly, companies avoid outright data manipulation but present themselves in a misleadingly positive light regarding their environmental impact  -- a practice called \emph{greenwashing}. 

In addition to their broader communication strategies, most companies disseminate information about their environmental initiatives through dedicated sustainability 10-K and annual reports. In addition, corporations also leverage various media channels and traditional communication platforms to convey their environmental credentials. As a result, the corpus of documents used in textual analysis is diverse and heterogeneous.

In such documents,  greenwashing can appear in multiple forms. The Green Claims Directive of the European Union~\cite{eugreenclaims} identifies four types of greenwashing:
\begin{enumerate}
\item using unreliable sustainability labels (e.g., H\&M and Decathlon have used labels such as \textit{eco design} and \textit{conscious} and were prompted by the Netherlands Authority for Consumers and Markets to stop the practice in 2022 as the retailers did not clearly describe why the products received those labels~\cite{decathonhm})
\item presenting legal requirements for the product as its distinctive features (e.g., McDonald's advertises its reduction of plastic waste by using reusable cutlery~\cite{mcdonalds}, while this is a legal obligation since January 2023 in France~\cite{plasticfrance})
\item making green claims about the entire product when the claim applies only to a part/aspect of the product (e.g., Kohl's and Walmart were fined by the U.S. Federal Trade Commission for the misleading claim that their products were made of bamboo, a sustainable material, when they were made out of rayon, which is derived from bamboo via a highly toxic process that removes any environmental benefits of bamboo~\cite{bambooeco}), and finally
\item making environmental claims for which the company cannot provide evidence (e.g., Keurig was fined by Canada's Competition Bureau for misleading claims about the recycling process of its single-use coffee pods~\cite{keurigcoffee}).
\end{enumerate} 
Greenwashing is a widely observed practice rather than a marginal occurrence: the Australian Competition \& Consumer Commission performed a sweep analysis on 247 businesses in October 2022 and found that $57\%$ of them made misleading claims about their environmental performance~\cite{sweepaustralia}. The International Consumer Protection Enforcement Network, too, found in November 2020 that $40\%$ of around 500 websites of businesses covering different sectors contain misleading green claims~\cite{globalsweep}.

Addressing greenwashing is not only in the broader public interest but also in the company's best interest, as the practice is often subject to legal sanctions prohibiting misleading or deceptive conduct, with businesses facing fines or lawsuits as a consequence~\cite{litigatorsgreenwashing}. In addition, political awareness and public awareness of the issue via traditional and social media~\cite{bonpote} can harm the reputation of a company involved in the practice. These factors make greenwashing a risk factor that can affect investments and that should be monitored at the company and portfolio levels.
%, which raises several concerns, in particular, the measurability of the concept. 

The question thus arises to what extent greenwashing can be detected automatically.
Such automated detection could help investors, journalists, prosecutors, and activists spot potentially illicit behavior, but it could also help companies avoid unintentionally misleading communication. In this paper, we provide \textbf{a comprehensive survey of natural language processing (NLP) methods to identify potentially misleading climate-related corporate communication that can be indicative of greenwashing}. We have identified \textbf{61 works that propose ways to check one or more aspects of greenwashing in text}.
% In this paper, we provide a \textbf{comprehensive survey of the scientific literature addressing the automated detection of greenwashing in textual data}, with a particular focus on
% %. Due to the large impact of companies and the short timeline in which we can act on climate change, we attach particular importance to 
% \textbf{corporate climate-related greenwashing}.
We cover the intermediary steps involved in a natural language processing (NLP) analysis of the environmental disclosure of a company, ranging from detecting environmental or climate-related text to greenwashing detection. We structure these works in a unified categorization and discuss tasks, datasets, methods, and results for each of them. Unlike other surveys on the topic~\cite{moodaley_greenwashing_2023, measuring_greenwashing}, which compile and discuss findings reported in the scientific literature, our approach dives into the actual methodologies of these works.

Our main contributions and findings are: 
\begin{itemize}
    \item We provide the first survey on NLP-based  detection of greenwashing that describes and systemizes the methods used in the literature. Our survey focuses on tasks and datasets but also contextualizes the performance of the models and provides future research directions. 
    %that deep dive into the content of the study surveyed.
    \item We find no datasets of positive and negative examples of greenwashing. Hence, it is not possible to train or evaluate models on the task of identifying greenwashing.
    \item While there are no datasets explicitly labeled for greenwashing, many works have studied %individual characteristics of greenwashing. These works have analyzed 
    intermediary tasks, such as detecting misleading claims, vague language, or environmental reporting inconsistencies. Researchers have used these characteristics to develop theoretical definitions of greenwashing and proposed measurable indicators to identify potential cases. Although not a direct substitute for labeled greenwashing datasets, these indicators provide tools for analyzing corporate disclosures and assessing the likelihood of greenwashing.
    \item Open challenges remain: While many models have been proposed for greenwashing-related tasks, it is difficult to compare approaches and solutions as many studies do not report measures of uncertainty (e.g., confidence interval, multiple runs, hyper-parameters tuning), nor do they report simple baselines that help assess the difficulty of datasets such as random baseline or TF-IDF baselines. Even the intermediary tasks thus remain active areas of research.
\end{itemize}

Our survey is structured as follows. We start with a discussion of the related work on greenwashing surveys (Section~\ref{sec:related}), followed by the legal context around climate actions and the definition of greenwashing (Section~\ref{sec:definitions}). We then survey intermediate tasks that have been studied in the literature (Section~\ref{sec:intermediary tasks}): %approaches that identify communication explicitly or implicitly related to climate topics. This task encompasses 
identifying climate-related statements; identifying topics aligned with frameworks such as the Task Force on Climate-related Financial Disclosures (TCFD) or Environmental, Social, and Governance (ESG) factors; assessing the presence of green claims;
%(see Sections: \ref{sec:climate-related topic}, \ref{sec:sub-topics}, \ref{sec:green claim}). Third, we focus on works analyzing how companies communicate about climate change. Beyond identifying content, it is critical to evaluate the style and intent of communication, such as
and examining sentiment, argumentation quality, deceptive practices, and stance to discern the tone and authenticity of a message. % Fabian: the sections were not in numerical order. I think we can stay abstract here.
%(see Sections: \ref{sec: climate risk}, \ref{sec: claim characteristics}, \ref{sec:stance detection}, \ref{sec:qa}, \ref{sec:deceptive}). 
We then review approaches that aim to identify greenwashing as a whole %, integrating insights from the intermediary tasks to assess their contributions to detecting and mitigating misleading corporate communication in the climate domain (see 
(Section~\ref{sec: greenwashing signals}). We conclude by discussing challenges and opportunities in greenwashing detection (Sections~\ref{sec:challenges} and \ref{sec:conclusion}). We note that we have an Appendix section that will be part of the supplemental online-only material.
%This structured approach enables a deeper understanding of the challenges and potential solutions to effectively address greenwashing. 

%We have retrieved all open access datasets mentioned in the papers, we have manually inspected the quality of the datasets, we have performed reproducibility studies, and we have implemented baselines to estimate the complexity of the tasks. This allows us to give a much deeper analysis of the progress in this field of research, and to point out challenges with misleading task descriptions, mislabelled datasets, and missing baselines. One of our main finding is that the name of a task, its definition, and its actual ground truth often do not easily correspond with each other. A major the consequence of this mismatch is that there are datasets where a pretrained BERT model shows very good performance (because it does what the training dataset suggests), but a zero-shot large language model performs very badly (because it does what the task description suggests).
%\tom{TODO: gives some numbers (datasets, papers)} % We conclude with the important directions for future work. In Section TODO explain what each section contains}
