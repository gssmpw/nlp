\section{Greenwashing Detection}
\label{sec: greenwashing signals}

After having discussed different subtasks of greenwashing detection, we now come to the final, all-englobing task:

\task Greenwashing detection is the task of predicting if a text contains greenwashing. This is a binary classification task. Greenwashing detection can also be done at an aggregated scale (such as by a yearly indicator).

\paragraph{Greenhushing and Selective Disclosure}  There is a large body of work on disclosure (as described in Section~\ref{sec: tcfd}). Quantifying climate-related disclosure can highlight greenhushing and selective disclosure. \citet{bingler_cheap_2021} identify a lack of disclosure in the Strategy and Metrics\&Targets categories, which they describe as \textit{cherry-picking}. They also highlight that merely announcing support for the TCFD does not lead to an increase in disclosure -- on the contrary, it is a practice called \textit{cheap talk}. Similarly, \citet{auzepy_evaluating_2023} conducted a more fine-grained analysis of TCFD-related disclosure in the banking industry. They also highlighted large differences across TCFD categories. In particular, they found a low disclosure rate on the fossil-fuel industry-related topic despite large investments in that sector.

%(omission type greenwashing \citet{defreitasnettoConceptsFormsGreenwashing2020a}).

\paragraph{Climate Communication as an Image-Building Strategy} \citet{dingCarbonEmissionsTCFD2023} and \citet{chou_ESG} found a correlation between climate-related disclosure and carbon emission: companies that emit more tend to disclose more climate-related information. \citet{bingler2023cheaptalkspecificitysentiment} proposed a Cheap talk index based on claims specificity. They reached the same conclusion that larger emitters tend to disclose more, but also that negative news coverage is correlated with cheap talk. \citet{marco_polignano_nlp_2022} showed that while disclosing more, reports mostly focus on positive disclosure.  \citet{hyewon_kang_analyzing_2022} proposed a sentiment ratio metric that highlights overly positive corporate reports. They identified periods of overly positive communications that followed negative environmental controversies; in other words, companies that tried to rebuild their image after a controversy. 
\citet{csr_report_greenwashing} analyzed CSR reports of environmental violators and companies with clean records. They studied an ensemble of variables: quantification of environmental content, readability score, and sentiment analysis. They found that violators publish longer, more positive, and less readable CSR reports with more environment-related content.
This suggests that companies use climate disclosure as a tool to mitigate controversies and/or to improve their image. \citet{kdir23} propose to use the discrepancies between internal disclosure and social media perception of a company to identify potential greenwashing.

\paragraph{The Role of Disclosure Style in Perceived Commitment} \citet{clarkson_nlp_us_csr} found that companies that use a more complex language in their CSR disclosure are associated with a higher CSR rating. \citet{schimanski_bridging_2023} concluded that a higher quantity of ESG communication is associated with higher ESG rating. \citet{rouenEvolutionESGReports2023} also highlighted the relationship between disclosure quantity and complexity with the ESG score. This might indicate that the linguistic style is a good predictor of the substance of the discourse, or that analysts are rewarding companies that communicate on ESG-related issues.

\paragraph{ESG Score in Greenwashing Detection} \citet{LEE_greenwashing} proposed a greenwashing index based on the difference between the ESG score and the ESG score, weighted by the quantity of communication on each topic (E, S and G). 
%The quantity of communication is computed using dictionaries. 
\citet{Greenscreen} define two types of risks: managed risk (the company is addressing it), and unmanaged risk (the company is not currently addressing it but it could). Based on these, they propose an ESG unmanaged risk score as a measure for environmental performance, which can then be used as a signal for identifying greenwashing.
%\citet{rouenEvolutionESGReports2023} study concluded an increase in disclosure of material information over immaterial information, but also an uniformisation of the language.
% and while this does not highlight greenwashing, this shows that analysts might be influenced by the type of language and not only by the content. 
% linked to the GRI topics.

% {\color{gray}
% % shorter version : 
% While other give signals that might be interpreted as greenwashing, without mentioning greenwashing: \citet{bingler_cheap_2021, auzepy_evaluating_2023} highlights the low disclosure rate on the fossil fuel industry-related topic, \citet{bingler_cheap_2021, bingler2023cheaptalkspecificitysentiment} also talk about cheap talks and cherry picking but not explicitly about greenwashing, \citet{dingCarbonEmissionsTCFD2023, chou_ESG} found a correlation between the climate-related information disclosure and carbon emission (higher emitter disclose more). Similarly \citet{clarkson_nlp_us_csr} found that better CSR performer are associated with more advanced CSR disclosure but poor performer disclose more negative sentiment.
% }

%to study the link between cheap talk and various effects such as the introduction of the TCFD recommendation.
% They studied 14,584 annual reports from 2010 to 2020 and arrived at the same conclusion as their previous approach \citettbingler_cheap_2021}.

% give the list of time the papers talks about greenwashing

% While the previous work can be interpreted as greenwashing signals, they do not specifically focus on greenwashing. The following works explicitly mention greenwashing.

 % sentiment based
\paragraph{Contrasting Stances: A Method to Identify Greenwashing in Climate Communications} \citet{morio2023an} proposed to evaluate a company's stance %polarity 
towards climate change mitigation policies. They hypothesize that companies with a mixed stance might be engaging in greenwashing. The hypothesis was not tested and remains unconfirmed; still, such methodology helps understand the narratives of companies around climate change. \citet{coanComputerassistedClassificationContrarian2021} were able to analyze the type of contrarian claims used by conservative think-tank (CTTs) websites and contrarian blogs. They showed a shift over time from climate change denying arguments to opposing climate change mitigation policies. Despite this shift, they found that the primary donors of these think tanks continue to support organizations promoting a climate-denier narrative aimed at discrediting scientific evidence and scientists. Discrepancies between the stance extracted from official communications~\cite{morio2023an} on the one hand, and the stance of third-party media organizations such as think tanks founded by the company \cite{coanComputerassistedClassificationContrarian2021} on the other, might indicate greenwashing.
%DistilBERT fine-tuned on SST-2.
%(computed with a DistilBERT fine-tuned on SST-2)
%DistilBERT base uncased fine-tuned SST-2

%\paragraph{Using Media Perception as ground truth to identify Greenwashing} \citet{kdir23} propose to use the discrepancies between internal disclosure and social media perception of a company to identify potential greenwashing. They applied FinBERT-ESG-9-Categories for ESG classification of internal documents and TextBlob for sentiment analysis of media perception. Oana: moved part of this
 
%They built a dataset of social media communications labeled using the ESG unmanaged risk score. 

% stance based
%such as "Energy transition & zero carbon technologies
% 
\paragraph{Defining the linguistic features of greenwashing} \citet{divinus_oppong-tawiah_corporate_2023} we propose to identify greenwashing using the linguistic profile of tweets. They estimate the deceptiveness of the text using a keyword-based approach and linguistic indicators (e.g. word quantity, sentence quantity). They found a correlation between greenwashing and lower financial performances. A scoring system based solely on linguistic elements has significant limitations. For example, the following tweet gets a  high greenwashing score: ``Read about \#[company’s] commitment to a \#lowcarbon future http://[company website]''. However, it is merely an invitation to read a Web page. 
%This raises the question of whether such statements constitute greenwashing. While the vagueness of the language may contribute to perceptions of misleading intent, the statement merely conveys a commitment, which, in itself, does not provide sufficient evidence of deceptive practices.

% \citet{bhatia_automatic_2021-1} 

%However, has this metric rely on public opinion, it might only indicate known cases of greenwashing and might also be bias by sector, as the study focus only on the pharmaceutical sector.


%focused specifically on greenwashing detection on Twitter. They framed greenwashing as 'Sustainability Fake News'. [...] 

% \datasets While most of the work previously described could help produce a weakly annotated dataset, only \citet{avalon_vinella_leveraging_2023} constructed such a dataset. They used a linear regression fitted on a really small sample of 10 examples, to then asign weak labels on a larger dataset. 

% \solutions \citet{avalon_vinella_leveraging_2023} they finetuned climateBERT on their weakly labeled dataset. 

% \citet{avalon_vinella_leveraging_2023} models could perform quite well, on a seemingly difficult task. However, due to the methodology to construct the dataset, the model essentially learns to predict simultaneously all 4 characteristics. 


% {\color{gray}

% % \citet{schimanski_bridging_2023} concluded that the quantity of ESG communication is associated with higher ESG rating (from Bloomberg, Refinitiv Asset4, and RobecoSAM). This shows that ESG analysts seem to discourage green-hushing.  Once again they highlight the relationship between the disclosure quantity and complexity with the ESG Score similarly to \citet{schimanski_bridging_2023}. 
% \citet{bhatia_automatic_2021-1} define greenwashing as fakenews = deception vocabulary

% }

% These measures of greenwashing demonstrated their potential by identifying known cases of controversies (such as Toyota in 2011 \cite{hyewon_kang_analyzing_2022}). However they have strong limitations such as relying on weak labels \cite{avalon_vinella_leveraging_2023, LEE_greenwashing, Greenscreen} or lagging behind public perception \cite{kdir23}. Moreover, they rely mostly on superficial cues (sentiment \cite{hyewon_kang_analyzing_2022, kdir23} or linguistic \cite{divinus_oppong-tawiah_corporate_2023}). 

% As discussed in section \ref{sec:intro} and \ref{sec:definitions}, there are multiple definitions for greenwashing. While using definition \ref{def:poor_perf}, the approaches using ESG scores can actually identify greenwashing, however using definition \ref{def:greenwashing}, using actual actions from companies might be necessary. 

\paragraph{Insights} Several indicators for greenwashing have been proposed: 
%, yet, as highlighted by \citet{measuring_greenwashing}, these approaches are largely constructed from a theoretical standpoint rather than being informed by empirical examples. These theoretical definitions focus on several characteristics, including 
overly positive sentiment \cite{hyewon_kang_analyzing_2022}, the use of specific linguistic cues~\cite{divinus_oppong-tawiah_corporate_2023}, discrepancies between ESG scores and corporate disclosures~\cite{LEE_greenwashing, Greenscreen}, ambiguous stances~\cite{morio2023an}, and inconsistencies between social media perception and official disclosures~\cite{kdir23}.
While these approaches laid the theoretical groundwork for understanding indicators of greenwashing, they have a significant limitation: they are not empirically evaluated~\cite{measuring_greenwashing}. %Most existing metrics have not been rigorously tested against real-world examples. 
Only a few studies, such as that by \citet{hyewon_kang_analyzing_2022}, have made attempts to validate their signals against actual cases, but even these efforts have not resulted in a comprehensive dataset of greenwashing examples.
This gap between theory and practice highlights the necessity of developing datasets containing real-world instances of greenwashing. Without empirical evaluation, the prediction that a company engages in greenwashing is unfounded at best, and misleading or even defamatory at worst. %A robust dataset is crucial for challenging, refining, and practically assessing the detection methods built upon theoretical definitions. 
However, building such a dataset comes with many challenges. First, one would have to identify suitable sources of documents that are openly accessible and likely to contain greenwashing. Second, one would have to overcome the inconspicuous and subjective nature of greenwashing itself. Finally, the publication of any such dataset exposes the authors to charges of defamation by the companies they accuse of greenwashing.
