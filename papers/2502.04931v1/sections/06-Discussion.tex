\subsection{General Discussion}
Our
study provided empirical evidence that the intervention improved some aspects of users' general media literacy, their behavioral intentions to use different strategies for checking misinformation and their performance in discriminating between fake and real news. However, it did not substantially enhance players’ prosuming skills, despite the mechanics of the game being tailored to support content generation. One possible explanation of this result would be that pre-survey data indicated that participants %already 
entered the study with high levels of confidence in their prosuming abilities. This "ceiling effect" likely constrained measurable gains, as participants had or perceived less room for improvement.
Another explanation is that high-level skills of critical prosuming require collaborative efforts and collective intelligence between participants \cite{lin2013understanding}, which was not possible to achieve in the PvP model of the game. Finally, the game only provided four rounds of content creation for each person, which could be sufficient to apply a critical perspective on the content, but not enough to
train creation effectiveness.

Another important finding is that, similarly to the previous studies of another game-based intervention \cite{leder2024feedback} we did not find significant effects from the intervention on the person's self-confidence in tackling misinformation.
Interestingly, game log data indicated that players' proficiency improved during gameplay; by the third and fourth rounds, they typically produced longer and more comprehensive messages compared to the initial rounds. Social Cognitive Theory \cite{bandura1997self}. may explain this discrepancy. According to it, enactive mastery experiences — ie., successful task completion— are the most influential sources of self-efficacy\cite{bandura1997self}. Positive experiences bolster self-efficacy, while repeated failures undermine it. In our game, success was determined by evaluations from LLM-simulated public opinion. The qualitative results revealed that the players found that no single strategy was effective for all characters, requiring frequent adjustments based on the unique characteristics of each character. This unpredictability made it difficult to achieve consistent success. While players developed greater proficiency during the game, the difficulty in achieving consistent success may have limited their perceived self-efficacy. However, this outcom can also be considered through the lens of the educational effects of the game. Previous studies about differential challenges of misinformation showed that young adults %might lack the critical thinking skills needed to assess information effectively and 
often overestimate their ability to assess information effectively\cite{papapicco2022adolescents,porat2018measuring}; in this context, the reaction of our participants, most of whom were young adults could be a positive signal that they became aware of the complexity of misinformation and the absence of one-size-fit-all solutions. 
%I deleted this explanation
%Furthermore, as character evaluations responded to the actions of both players, scores fluctuated frequently, with the gains of one player often neutralized by the moves of their opponent. This dynamic likely hindered the confidence of the players in their ability to discern misinformation successfully, even though their actual performance improved, as evidenced by significant gains in discriminative test results.Thus, while players developed greater proficiency during the game, the lack of sustained clear success may have limited their perceived self-efficacy. 

%For example, in our qualitative findings, participants reported scepticism toward information that purely relied on authoritative sources because they found their opponents used fabricated evidence from these sources to gain LLM-simulated characters’ trust and increase scores. This gameplay experience reminded them of real-life situations, where misinformation often exploits trust by citing credible authorities. As a result, players learned to examine the intent behind messages instead of automatically trusting authoritative sources.
%\textcolor{blue}{This evolution in how participants evaluated information aligns with the Elaboration Likelihood Model (ELM)\cite{petty1984source}, which describes how individuals process persuasive messages via central(critical evaluation of content) or peripheral routes(reliance on heuristic cues). In the gameplay, participants appeared to shift from peripheral processing (trusting authority as a shortcut) to central processing (evaluating the intent and content of the message) when exposed to the misuse of authority in gameplay. Notably, many participants suggested cross-checking sources as a practical solution for verifying information, demonstrating their enhanced critical thinking skills}.

Lastly, In our qualitative findings, participants reported skepticism toward information that purely relied on authoritative sources because they found their opponents used fabricated evidence from these sources to gain the trust of the LLM-simulated characters, with a view to increasing their game score. This gameplay experience reminded them of real-life situations, where misinformation often exploits trust by citing credible authorities. As a result, players learned to examine the intent behind messages instead of automatically trusting authoritative sources. This evolution in how participants evaluated information aligns with the Elaboration Likelihood Model \cite{petty1984source}, which describes how individuals process persuasive messages via central (critical evaluation of content) or peripheral routes (reliance on heuristic cues). In the gameplay, participants appeared to shift from peripheral processing (trusting authority as a shortcut) to central processing (evaluating the intent and content of the message) when exposed to the misuse of authority in gameplay.
Furthermore, our findings align with studies showing that gamified inoculation techniques for pre-bunking misinformation can trigger skepticism related to both false and real news\cite{hameleers2023intended,modirrousta2023gamified}. While this induced skepticism might seem limiting —potentially thus undermining trust in high-credibility sources— we believe it supports the goal of fostering critical media literacy. By encouraging players to evaluate the trustfulness of content, source and intent,  the game develops essential skills for navigating today's complex information landscape. Rather than promoting cynicism, this skepticism cultivates constructive inquiry that helps individuals better discern reliable information. Notably, many participants pointed to the need to cross-check sources as a practical solution for verifying information, demonstrating their enhanced critical thinking skills.  
%Additionally, recent game interventions have explored ways to improve individuals' ability to assess both real and false information. Strategies include teaching players to identify credible versus low-credibility media \cite{micallef2021fakey}, rewarding accurate judgments \cite{barzilai2023misinformation}, and incorporating feedback tests after gameplay \cite{leder2024feedback}. }

\subsection{Outcomes of the Game Mechanics}

\subsubsection{PvP model for Media-Literacy Game}
Unlike prior misinformation education game which broadly employ single-player mechanics \cite{roozenbeek2019fake,camCambridgeGame,harmonysquare,jeon2021chamberbreaker}; "Breaking the News" applies PvP mechanics. While in general, previous studies in serious games showed that PvP games are more engaging and motivating \cite{cagiltay2015effect}; at the same time in competitive environments, the motivation to “win” may overshadow educational goals.  In addition, in recent studies of another misinformation game it showed that some students may not find %gamification or
competition enjoyable or motivating\cite{axelsson2024bad}. 
%our results also indicate players learnd from the opponents. 
Our results demonstrated that, in general, our participants were highly motivated to play one against another. Moreover, they indicated that they learned from each other's strategies, and it helped them better understand the dynamic of misinformation. Therefore, we can conclude that, in our case, this approach was beneficial to fulfilling the intended purpose of the game. 
However, our study was conducted on East Asian participants, who are part of a collectivist-oriented rather than competitive culture \cite{chung1999social}. Previous studies have shown that cultural factors play an important role in the degree of competitiveness in gamified interventions \cite{oyibo2017investigation}. Therefore, it is also possible that in other cultural settings, the game's incentives can trigger more intense competition, which can negatively affect educational results. 


%These games frame the narrative as either misinformation creator or misinformation debunker and asks the player to adopt one of the roles, tasked by discerning misinformation using choice-based interaction. 


%\subsubsection{Free-form Input Generation}
%\textcolor{blue}{One notable feature of Breaking the News is its open world narrative and free-form response format, which contrasts with the linear choice-based formats commonly used in prior game intervention [REF]. 
%Previous works in the field of role-playing games showed that this approach makes game interaction more natural and increases user engagement \cite{csepregi2021effect,ashby2023personalized}.


%This format positively influences players’ learning outcomes by enhancing engagement and replayability. 

%In choice-based formats,interactions are typically brief, requiring players to passively select predetermined options. This structure can leads to "guesswork," where learners may choose correct answers without fully understanding the underlying concepts. In contrast, free-form responses force players to actively reason and articulate original ideas, fostering deeper engagement and critical thinking. 

%Additionally, by allowing players to develop their own narratives and gameplay experiences, each playthrough feels unique, encouraging them to return to the game.

%This variability not only boosts intrinsic motivation but also contributes to the game's educational value. When players repeatedly engage with core mechanics through varied narratives, they continue practicing essential skills, thereby reinforcing and enhancing their learning outcomes\cite{kucklich2004play}}.

%\textcolor{blue}{An additional strength of this format lies in the autonomy it provides. Players can craft personalized narratives that align with their individual learning styles, a factor proven to result in better learning outcomes in serious game design\cite{hwang2012development}. Furthermore, the freedom to explore and shape their own gameplay experiences not only deepens engagement but also excites players, acting as a powerful motivator for repeated play\cite{ravyse2017success}. By promoting intrinsic motivation and repeated practice, the free-form and open-world design demonstrate their potential to enrich both player experience and educational effectiveness. Given that inoculation interventions often show diminishing effectiveness over time, developing an engaging, replayable game that can consistently reinforce players' resistance to misinformation.}

%Free form rewriting: 
\subsubsection{Free-form Input Generation}
One notable feature of Breaking the News is its free-form response format, which contrasts with the linear choice-based formats commonly used in prior game intervention\cite{roozenbeek2019fake,camCambridgeGame,harmonysquare,jeon2021chamberbreaker,micallef2021fakey}. 
Previous works in the field of role-playing games showed that this approach makes game interaction more natural and increases user engagement \cite{csepregi2021effect,ashby2023personalized}.
Based on our data, we can say that this format positively influences players’ learning outcomes by enhancing engagement and replayability. 
By allowing players to develop narratives themselves, each playthrough feels unique, encouraging players to return to the game. This variability not only boosts intrinsic motivation but also contributes to the game's educational value. When players repeatedly engage with core mechanics through varied narratives, they continue practicing essential skills, thereby reinforcing and enhancing their learning outcomes\cite{kucklich2004play}. 

In an educational setting, previous works showed that in choice-based formats, interactions are typically brief, requiring students to select predetermined options. This structure could lead to "guesswork," where learners may choose correct answers without fully understanding the underlying concepts. In contrast, free-form responses force students to actively reason and articulate original ideas, fostering deeper engagement and critical thinking \cite{bryfczynski2012besocratic}. This autonomy allows players to craft responses based on their understanding of their roles, making this another motivating factor for returning to the game\cite{ravyse2017success}.
 
%\textcolor{blue}{An additional strength of this format lies in the autonomy it provides, allowing them to craft responses based on their own understanding of their roles. For example, during gameplay,  we observed that one player in the journalist-debunker role chose to mimic social media comments rather than rely on scientific strategies, crafting responses that resembled typical user comments. The freedom to explore and shape their own gameplay experiences not only increase engagement but also motivates players to return\cite{ravyse2017success}. Given the memory-strengthening effects of repeated learning, requiring people to recall what they have learned helps them relearn and reinforce these lessons\cite{nader2009single}.Therefore, developing an engaging, replayable game can consistently reinforce players' resistance to misinformation.}
%Players can craft personalized narratives that align with their individual learning styles, a factor proven to result in better learning outcomes in serious game design\cite{hwang2012development}. Furthermore, 


\subsubsection{LLM-Powered Feedback}
%\textcolor{blue}{Another innovation of Breaking the News lies in its interaction and feedback mechanisms. The game incorporates LLM-based role-play as evaluators, creating an interactive feedback for players. LLMs demonstrate the ability to simulate human behavior and reactions, consistent with findings from prior research[REF]. To the best of our knowledge, this is the first attempt to integrate LLM role play into a misinformation education game, opening new possibilities for interactive learning experiences in media literacy game interventions.} 

%\textcolor{blue}{In the game, the LLMs mimic human reactions and behaviors by leveraging demographic, personality, behavioral, and psychological features towards misinformation. This approach provides a personalized gameplay experience, fostering greater engagement and enhanced learning outcomes compared to binary feedback mechanisms (e.g., true or false responses). Players interact with LLM personas by crafting strategies based on their perceived characteristics. When players’ actions get the expected results(eg, persona's reaction towards on the side), they experience a sense of achievement. Conversely, when the outcomes deviate from expectations, players adapt by employing alternative persuasive strategies tailored to the persona. This dynamic interaction encourages active learning, problem solving, and critical thinking. Notably, players exhibit higher levels of engagement with personas they found relatable due to personal experiences, focusing more attentively on their feedback. This level of engagement is difficult to achieve with traditional binary feedback mechanisms, which provide limited insights beyond correctness.}

%\textcolor{blue}{Feedback plays an important role in shaping game-based learning outcomes(REF). In our case, users observed that personas react more obviously to emotional elements. This is because persona’s feedback begin with emotional feelings (such as panic), followed by behaviors (like sharing news with friends or taking no action). These emotional responses also reflect their trust level scores. To get better game performance, players used various tactics to trigger emotions or address these emotions in gameplay. As a results, most participants reported that they learned to detect emotional manipulation techniques. To further enhance learning outcomes, future games could adjust LLM outputs to teach diverse strategies more effectively. }
%Another innovation of Breaking the News is the interaction and feedback component. Compared to choice-based format where player present predetermined options and get binary and concise feedback(eg, true or false)Players tasks actively craft their responses and then receive evaluation by five different LLM-stimulated personas. These personas are based on their characteristics, offering unique feedback toward the player’s output. By engaging with these nuanced evaluations, players gain insights into why certain arguments might appear trustworthy or untrustworthy, encouraging them to refine their strategies dynamically. Furthermore, Since the game is last multiple rounds, players will get to know personas and adjust their misinformation narratives or debunking approaches based on the personas’ reactions. Thus, they learn and practice various tactics to identify and debunk misinformation.
%innovation in LLM as evaluator in game and other settings. 
%feedback in serious game's advantage and disadvantage

Another innovation of Breaking the News lies in incorporating LLM-based "characters" as evaluators, creating interactive feedback for players. Although there are other attempts to incorporate LLMs into helping users
learn about misinformation \cite{danry2023don, hsu2024enhancing} (including gamified attempts\cite{tang2024mystery}), in these approaches, AI was mostly the source of correcting information. While this approach has benefits, it is can also be criticised due to  the possibility of LLMs creating incorrect by plausible text \cite{kim2024can,agarwal2024faithfulness}. In this case, it is possible that the intervention will disinform people to an even greater extent. In contrast, in our study, we incorporate AI not as a source of information but as non-playable characters with their own opinions. Therefore, the educational part of the game becomes more robust to resist the negative effects of erroneous generations, as they can only affect the opinion of the "character" but not the main narrative of the game. In general, LLMs demonstrate the ability to simulate human behavior and reactions, consistent with findings from prior research\cite{1park2023generative}; we also find, that the system is capable of emulating the opinion of 5 different characters at the same time and still keeping them consistently different during the game.
 
One advantage of dynamic LLM feedback is that it achieves greater engagement compared to binary feedback (e.g., true or false). Based on our observation, players adapted by employing alternative persuasive strategies tailored to the character and concentrated on the character's feedback. Notably, players exhibit higher levels of engagement with characters they found relatable due to personal experiences, focusing more attentively on their feedback. This level of engagement is difficult to achieve with traditional binary feedback mechanisms, which provide limited insights beyond checking their correctness.%\textcolor{blue}{Feedback plays an important role in shaping game-based learning outcomes(REF). In our case, users observed that characters react more obviously to emotional elements. This is because character’s feedback begin with emotional feelings (such as panic), followed by behaviors (like sharing news with friends or taking no action). These emotional responses also reflect their trust level scores. To get better game performance, players used various tactics to trigger emotions or address these emotions in gameplay. As a results, most participants reported that they learned to detect emotional manipulation techniques.} %To further enhance learning outcomes, future games could adjust LLM outputs to teach diverse strategies more effectively.

%In the game, the LLMs mimic human reactions and behaviours by leveraging demographic, personality, behavioural, and psychological features towards misinformation. Based on our observation, players were able to catch these different clues and crafted strategies based on these characteristics; players adapted by employing alternative persuasive strategies tailored to the character and concentrated on the character's feedback. This level of engagement is difficult to achieve with traditional binary feedback mechanisms, which provide limited insights beyond correctness.

\subsection{Design Implications for the Further Development of Serious Games with LLM-components}
Based on our experience designing and testing the game, we summarize the following recommendations to practitioners working on similar projects.

\subsubsection {Balance Between Freedom and Guidance}
In our game, we sought to challenge the existing choice-based approach in misinformation education games by providing users with free-form input. We found this approach triggers reflection, which helps to build hands-on experience and make the game more enjoyable. Yet, we also found that it relies on players’ existing knowledge of misinformation. For example, players might incorporate unverified information they’ve encountered on social media into the game, which is specifically problematic for the debunker role. While we provided the players with comprehensible instructions to guide their role's actions (how to act as a debunker or an evil influencer), it would be better to incorporate more context-specific tips in each stage of the game to help users explore different ways of winning the game and deepening their learning. We suggest using the approach used in the free-input educational interventions (e.g. \cite{bryfczynski2012besocratic}) to build clear, understandable criteria for free-form answers. These will not stop creativity but help people tailor their answers to the context of the game. We also suggest adding a preliminary stage to the game in the form of a simple choice-based questionnaire, helping people understand the role and the guidelines for the role. For example, we can introduce to the individual playing the debunker role the Debunking Ethics Standards  \cite{edmo2024,afp2024,eeas2024}, and ask them a series of questions about understanding the content of the guidelines before starting the main game. In addition, we can incorporate an additional AI-based mechanic to check the quality of answers, not only in terms of its influentiality, but also of its coherence with debunking practices guidelines. Similarly, we can demonstrate a "Score of Manipulativeness" to the influencer role. This better shows us how the ideas players implement into the message can be judged, from the point of view of dealing with misinformation.
\subsubsection {Replayability and Feedback}
%LLM 
One of the critical challenges in serious games is maintaining replayability, as this is important to facilitate the learning process \cite{adetunji2024unlocking} and making interventions more sustainable \cite{silveira2016open}. Moreover, a lack of replayability in educational games can limit both educational and behavioral change \cite{epstein2021tabletop}. To address this, we incorporated elements designed to ensure the game can be played multiple times, such as offering two distinct player roles and providing a free-form input mechanism that allows participants to explore a broader decision space. However, we recommend that future interventions consider additional methods to further enhance replayability.

One such approach involves leveraging opportunities to introduce various characters to represent public opinion. In our game, we observed that players finds characters' feedback more engaging when this resonates with them personally. By introducing more characters, or by allowing players to customize characters to better reflect their own experience, the game could encourage players to return and interact with new characters. The result would be a more engaging experience. For instance, research has shown that debunking misinformation often occurs within families and can sometimes lead to conflict \cite{scott2023figured}. In such a scenario, players could customize a character based on their previous experiences with family members, thereby practicing their own debunking strategies in a risk-free environment.
%Character feedback can be designed to elicit deeper cognitive engagement and reflective thinking. For example, when a player attempts to spread rumors intended to evoke fear and anxiety, the character might respond with questions such as: “What evidence supports this claim?” “Why might it be spreading so rapidly?” or “What would happen if I believe this information?” Studies have shown that AI-framed questioning can stimulate user reasoning and enable independent information evaluation(REF).This approach can also motivate players to return to the game equipped with new strategies. Over multiple playthroughs, as players become more skilled in critical reasoning, they gain access to increasingly sophisticated dialogues and insights from the characters, further enhancing replayability.

While in-game feedback guides player behavior during play, post-game debriefing sessions help consolidate learning and improve future performance. Research has found that debriefing is a crucial opportunity for players to process and integrate their learning experiences\cite{crookall2014engaging,leder2024feedback,barzilai2024learning}. After gameplay, we suggest arranging debriefing sessions that allow players to review their strategies, assess their effectiveness, and receive constructive feedback, potentially improving learning outcomes. For instance, after a session focused on combating misinformation, a post-game review might present an ideal debunking response or a well-supported counterargument. Such structured reflection enables players to internalize lessons and increases the likelihood that they will re-enter the game with newly gained insights, thereby reinforcing both learning and replayability.
%For example, some characters aligned closely with players’ prior experiences, prompting players to focus on that particular feedback. In other cases, a character proved difficult to persuade, sparking the player’s interest in challenging that character and crafting more tailored responses, sometimes at the expense of others.

\subsection{Limitations and Future Work}
We acknowledge the following limitations. 
% I still think it's need cultural context in limiation
Our participants were mainly from an East Asian country, which can limit the generalizability of our findings. Previous studies have shown that cultural contexts influence individuals' perceptions of misinformation and interventions \cite{noman2024designing}. For example, study revealed that Mexican and Spanish users were more likely to trust -related misinformation compared to users from Ireland, the UK and the USA \cite{roozenbeek2020susceptibility}. At the same time, non-western participants demonstrated a significantly higher willingness to challenge misinformation. The effect of proposed misinformation design interventions was significantly greater than that seen in analysis of the behaviors and attitudes in the UK population\cite{noman2024designing}. 
Previous studies also suggest that certain populations may face greater challenges in being able to critically evaluate information. For example, a large-scale study observed that Asian individuals encounter more difficulties in assessing health information from social media compared to other populations \cite{chandrasekaran2024racial}. 
%This increased difficulty could be attributed to various factors, such as differing educational approaches[REF], access to resources[REF], or cultural norms surrounding information consumption. 
In addition, individuals from an Asian background are more likely to incorporate social media information into their health-related decisions, potentially increasing their susceptibility to misinformation \cite{chandrasekaran2024racial}. Therefore, future work should explore the cross-cultural applicability of our findings in different cultural contexts. However, as the cultural background of our participants can potentially make them more vulnerable to misinformation than other populations, we believe that our intervention is valuable as providing insight into the outlook, activity and the relevance of certain interventions for these populations.
%Serious games have the potential to address this challenge by geographic and cultural boundaries. By tailoring narrative elements and characters to resonate with specific cultural contexts, these games can enhance both engagement and learning outcomes. In our game, we incorporated traditional medicine into the narrative(News in the game), knowing that it is a topic that resonates strongly with East Asian participants. For example, 48.4\% of Hong Kong residents reported using traditional medicine before the COVID-19 pandemic\cite{lam2021public}. Similarly, a national survey in South Korea found a 74.8\% prevalence of traditional medicine use overall\cite{ock2009use}.In China, traditional medicine is formally integrated into the healthcare system as one of the most widely practiced modalities\cite{chung2023implementation}.This cultural relevance provided a familiar backdrop for the game's content, allowing players to engage more intuitively and craft responses based on their lived experiences. 

%Since the game's news scenarios are easily modifiable, future adaptations could incorporate culturally specific settings to enhance relatability and impact for other audiences. By customizing content to align with different cultural contexts, the game can potentially improve its effectiveness in media literacy and reducing susceptibility to misinformation across diverse populations.}

%In a large-scale study of US populations, Chandrasekaran et al. \cite{chandrasekaran2024racial} found that people of Asian origin are more likely to incorporate social media information into their health-related decisions, potentially increasing their susceptibility to misinformation\cite{chandrasekaran2024racial}. At the same time, previous work showed that non-western participants demonstrated a significantly higher willingness to challenge misinformation and that the effect of proposed misinformation design interventions was significantly greater compared to the UK population.
%Taking together, the further studies required to determine if the intervention will be equally efficient on other ethnical groups, however, the current results showed that the game can be used to help the populations potentially more vulnerable to misinformation.
 
Our sample was relatively homogeneous in age. A recent meta-analysis of articles about different intervention approaches showed that neither age nor gender significantly impacts the effectiveness of media literacy interventions \cite{lu2024can}. However, previous work has suggested that media literacy interventions designed for certain age groups (e.g., older adults and adolescents) achieved greater effects\cite{moore2022digital,hartwig2024adolescents}. Future work should determine if our approach is efficient in other age groups of users and, if necessary, tailor scenarios to suit the various needs of different age groups.

Thirdly, our study provided only a one-time intervention and observed immediate learning effects; 
%we are unsure how long these effects will persist.
Previous work showed that even a one-time interaction with an educational game can provide long-term improvement in misinformation recognition. For instance, Maertens et al. tested the game "Bad News" and found that inoculation effects lasted for at least 13 weeks. This suggests the potential for the long-term effectiveness of active inoculation interventions with regular assessment\cite{maertens2021long}. Still, future research should include multiple time points to assess the long-term effectiveness of our game intervention. There should also be comparisons between one-time and multiple play sessions, with explorations of the impacts of players assuming different roles within the games.
%Suppose we had incorporated a control group or compared different types of training. In that case, we might have identified elements that could be better addressed through lectures, potentially reducing gameplay time and preventing participant exhaustion.
%two players learning diff things: spread vs control, diff play flow so how can you assume they played the same game?
In this study, each player was limited to a single role, either a misinformation creator or a debunker.  This resulted in different learning experiences depending on their assigned role. %While participants reported learning from their opponents, the educational outcomes between roles may have varied. 
The primary reason for not having role-switching in our study was the length of the game and its cognitive demands, which we feared would lead to player exhaustion if roles were switched mid-game. In future iterations, we aim to improve the design by allowing players to save their progress and switch roles during subsequent sessions. This could offer a more immersive experience, as players would gain perspectives from both the misinformation creator and debunker roles. Additionally, we aim to introduce new modes, such as a family mode, where players interact with two LLM-simulated characters. This option would reduce cognitive load while maintaining engagement. 
%how is this representing range of possible misinformation? survey is not comprehensive, should be finding that they learned about one subtopic but not another that was not addressed.

The current game is also limited in its sole focus on a pandemic scenario. In reality, misinformation spans multiple domains, with health-related misinformation able to influence political events such as elections. Our game only addressed text-based misinformation, while visual and video-based misinformation pose even greater challenges and are harder to detect. Future work could include multimedia content, such as images and videos, to more accurately simulate the diverse forms of misinformation that exist in the real world.

The current game approach may unintentionally foster skepticism toward both true and false news, a common issue in misinformation pre-bunking interventions\cite{hameleers2023intended,modirrousta2023gamified}. %Recent efforts aim to improve individuals' ability to assess information credibility. These include teaching players to distinguish between credible and low-credibility media \cite{micallef2021fakey}, rewarding accurate judgments \cite{barzilai2023misinformation}, and using feedback tests after gameplay \cite{leder2024feedback}. 
While we believe that the benefits of promoting critical thinking towards sources are very important in prebunking interventions,
%discuss feedback mechanisms in the design implications, 
we further recommend incorporating features that clearly differentiate high- and low-credibility sources during gameplay.

While LLMs like ChatGPT-4o were used to simulate human reactions in the game, these models do not fully replicate the complexities of human behavior. Human reactions are often nuanced and influenced by multiple factors, including culture, history, and personal experience. LLM-generated characters may oversimplify human emotions and fail to grasp the full context of certain situations. Despite us using an advanced ChatGPT-4o model, it may have missed these subtleties, leading to interactions that feel artificial or incomplete. For example, in the gameplay, players employ strong emotional manipulation strategies to provoke specific responses. However, the LLM-generated characters do not react as expected, where participants reported feeling frustrated when their emotional manipulation strategies did not yield the anticipated reactions. This can limit the diversity and depth of the simulated interactions and detract from the realism and fairness of the experience for certain audiences.

LLMs are trained on large datasets that may contain intrinsic biases, which can manifest themselves in unintended ways during gameplay\cite{kasneci2023chatgpt}. Studies have shown that LLMs are prone to so-called "hallucinations" and can reflect stereotypes or skewed perspectives, which could affect how certain characters respond in the game\cite{xie2024can}. For instance, if biased training data influence a simulated character’s reaction, it may inadvertently reinforce player stereotypes about certain groups of people. 

%Finally, the game was primarily designed in English, which may have affected immersion and comprehension for participants who were not fluent in the language.

%how is this representing range of possible misinformation? survey is not comprehensive, should be finding that they learned about one subtopic but not another that was not addressed.

%LLM do not truly represent the public opinion, biased, misrepresented.

%two players learning diff things: spread vs control, diff play flow so how can you assume they played the same game?


%\subsection{Future Work}
%There are several directions for future work. One potential expansion is to incorporate more diverse topics of misinformation beyond pandemics and health, such as political misinformation or climate change, to enhance replayability. Although misinformation tactics share common features across different domains, each topic presents unique characteristics and challenges. We have already demonstrated that games are an effective medium for confronting misinformation, and the mechanics of \textit{Breaking the News} are designed to simulate the complexity of misinformation in the real world. By changing the in-game news topics, we could tailor the educational impact for different audiences relatively easily.

%Another direction is to allow players to switch roles, experiencing both the misinformation creator and debunker perspectives. This could provide a fuller, more immersive experience. To further enhance the educational value and replayability, we also plan to incorporate more tools to help players investigate evidence. For instance, as in MathE, which designed artificial search engines for its game, we aim to introduce investigative techniques like lateral reading and source evaluation to support both player roles.

%Additionally, future work could explore different formats of misinformation within the game, including multimedia content like images and videos. This would address the limitation of focusing mainly on text-based misinformation and would reflect the growing challenges posed by visual misinformation in real life. Allowing players to create or debunk multimedia misinformation would make the game more relevant to the evolving nature of misinformation.