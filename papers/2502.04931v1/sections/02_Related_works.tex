\subsection{Characteristics of Misinformation}
\label{Characteristics of Misinformation}

The term "misinformation" is often used to refer to false information, including so-called "fake news", falsehoods, malicious rumours, and conspiracy theories. Some scholars distinguish between misinformation and disinformation, with the latter referring to information deliberately crafted and spread with the intent to deceive or cause harm\cite{guess2020misinformation,swire2020public}. However, as we believe it is often difficult to prove intent (a view inspired by recent publications in the field) this paper uses "misinformation" as an umbrella term for all forms of false information\cite{ecker2022psychological,southwell2019misinformation,wu2019misinformation,mitgenai2023,hartwig2024adolescents}. 

Several studies have sought to identify the key characteristics that distinguish misinformation from well-sourced, authentic information. This analysis aims to help% news consumers 
people to better make this distinction. For example, Horne et al. analyzed the writing styles of fake information versus real news. They found that texts which can be characterized as fake news typically features longer headlines, simpler word choices, and greater use of proper nouns and verb phrases\cite{horne2017just}.
Research has shown that misinformation and authentic information differ in terms of how the former can be created with the intention of triggering emotions such as fear, anxiety, or sympathy\cite{choudhary2021linguistic, zhang2020overview}. In order to produce these responses, misinformation creators %frequently 
employ personal, negative, and opinionated styles \cite{bessi2015trend,porat2019content}.
Source credibility is another notable characteristic of distinguishing between authentic information and misinformation. Authentic information is usually shared by highly credible sources: reputable websites, mainstream media outlets, professional news organizations, and official state publications \cite{molina2021fake,zhang2020overview,shin2018diffusion}.  In contrast, fake news often originates from lower-quality sources designed to generate revenue through advertisements.  To attract clicks, these stories frequently use unverified quotes, inflammatory narratives, and misleading images. %thus resulting in low-quality content
\cite{molina2021fake,shu2017fake}. However, not all less reliable sources are perceived as equally untrustworthy. People often gravitate toward partisan sources that align with their political ideology, leading to significant variations in news consumption across the political spectrum \cite{faris2017partisanship}. This selective consumption reinforces trust in these sources, even when they may be objectively classified as biased or unreliable \cite{mummolo2016news, pennycook2019fighting}.
Finally, compared to authentic news, misinformation tends to develop in line with a broad dynamic pattern. Shin et al. traced the life-cycle of 17 high-profile political rumors on Twitter over a 13 month period and found that false rumors tend to reemerge, become more intense and extreme over time\cite{shin2018diffusion}. In combination with the "illusion of truth effect" (i.e., the more often people see information, the more likely they are to trust it), it makes people more likely to believe increasingly extreme misinformation, thus making it harder to debunk\cite{silva2017informative,yousif2019illusion}.

%Several works investigated the origin and drives of fake news. They found that fake news stories often generate revenue through advertisements. To attract clicks and attention, the content often feature unverified quotes, opinionated and inflammatory narratives, misleading images\cite{molina2021fake,shu2017fake}in other words, they tend to be low-quality content.} \textcolor{red}{\sout{In addition, several studies have pointed out that misinformation posts more often contain spelling and grammatical errors, misleading images, and unverified quotes; in other words, they tend to be low-quality content.}} \textcolor{blue}{ As a result, false stories often circulate widely on social media and misinformation outlets, bypassing mainstream news organizations\cite{hanley2023golden}. Therefore, source credibility is another notable characteristic to distinguishes authentic information from misinformation. Authentic information is often shared or published by highly credible sources, including reputable websites, mainstream media outlets, professional news organizations, and official government publications\cite{molina2021fake,zhang2020overview,shin2018diffusion}. In contrast, information from lower-quality sources is generally perceived as less credible\cite{pennycook2019fighting}. However, not all less reliable sources are perceived as equally untrustworthy. For example, news consumption patterns exhibit substantial variation across the political spectrum\cite{faris2017partisanship}. Specifically, individuals often gravitate toward partisan sources that align with their political ideology\cite{pennycook2019fighting}. As a result, this selective consumption reinforces trust in these sources, even when they may be biased or unreliable\cite{mummolo2016news}.} 



% need to add source's perception 
 %Because these sites intentionally publish deceptive content for financial gain, they rely heavily on social media for distribution\cite{shu2017fake}. As a result, false stories often circulate widely on social media and misinformation outlets, bypassing mainstream news organizations\cite{hanley2023golden}.

 %Meanwhile, much scholarly attention has been given to the dynamic lifecycle of misinformation on social media.

\subsection{Media Literacy}

Media literacy is commonly defined as the ability to "access, analyze, and produce information for specific outcomes" (\cite{aufderheide2018media}, p. 6). Most modern frameworks on media literacy interpret the term "information" as being produced by media (including social media), while noting that media outlets can generate adversarial tensions in consumers. There is understanding of a need in democracies for citizens to learn a set of skills that can help protect societies against these effects \cite{potter2010state}. 
There are several models outlining the skills that are required to be learned to achieve media literacy. For example, Tandoc et al. proposed a four-component model, which includes technical competency (instrumental knowledge about how to use social media platforms), privacy protection (knowledge about how to protect privacy and manage account security), social literacy (knowledge about how to manage social relations on various platforms), and information awareness (the ability to discern between truthful and false information on social media, identify reputable sources, and cross-check information)\cite{tandoc2021developing}. Chen et al. proposed dividing the skills into critical and functional domains, as well as consuming and producing (presuming) content. In this model, the functional skills are related to the basic competencies required to use and produce media content (e.g., accessing a platform and using integrated tools to write a post). This contrasts with critical skills which involve reflecting on the content, recognizing the motives behind publications, and creating content that includes the author's perspective.\cite{chenwu2011unpacking}. There are also related concepts of "news literacy" \cite{malik2013challenges} and "information literacy"\cite{jones2021does}, which to some extent can be considered part of the media literacy domain specifically aimed at assessing the quality of information and detecting fake news.
As mentioned by Hobbs and Jensen \cite{hobbs2009past}, at the start of the 21st century, Media Literacy Education experienced a paradigm shift from a protectionist position (shielding vulnerable individuals from the negative influence of media) to an empowering paradigm, where people were encouraged to critically engage with media and develop skills to interpret both its negative and positive effects through critical thinking. In this paradigm, interaction with misinformation can also be considered as having educational power if it teaches the person to understand its effects. This has resulted in several recent game-based interventions that use inoculation theory to create media literacy education strategies. Inoculation theory suggests that exposure to a weaker version of misinformation can help to develop stronger protection against future exposure \cite{grace2023examining}.

\subsection{Serious Games in Media Literacy Domain}
In recent years, games have gained significant attention as educational tools, driven by their potential to enhance learning experiences \cite{de2018games}. Game-based learning and gamification, which integrate gaming elements into education, have been studied for their ability to engage learners and facilitate active, experiential learning. These approaches provide immediate feedback and contextual problem-solving and are supported by theories of effective learning \cite{plass2015foundations, tobias2014game}. The concept of serious games was proposed by Michael et. \cite{michael2005serious}; the most common definition of serious games or applied games is that games that do not primarily aim at entertainment, enjoyment, or fun but explicitly emphasize the added educational value of engagement and competition. Troyer et al. \cite{de2017linking} suggest serious games combine serious educational content with the fascinating mechanics, storytelling and narrative of video games, thus making them an effective tool for conveying educational messages in an interactive and engaging way. Serious games have a wide range of applications across various educational settings \cite{connolly2012systematic, connolly2008development}. 
There are multiple attempts to gamify misinformation education aiming to improve their media and information literacy \cite{roozenbeek2019fake}.
These games feature players that can be categorized into two main types: creators and debunkers. The misinformation creator's objective is to create and spread misinformation. For example, in Bad News \cite{roozenbeek2019fake}, Harmony Square \cite{harmonysquare}, Cat Park \cite{Gusmanson.nl_2022} and ChamberBreaker \cite{jeon2021chamberbreaker}, players are tasked with spreading fake news in a social media environment to gain likes or followers while maintaining credibility. The second one focuses on fact-checking, where players adopt the role of a misinformation debunker. In games like MathE \cite{katsaounidou2019mathe} and Escape the Fake \cite{escapefake}, players work to identify fake news using verification tools such as reverse image search. These games support the development of practical skills in identifying misinformation and offer a more direct method of engaging with fake news detection.

These games are typically designed as single-player games, allowing players to experience the game narrative at their own pace. One of the most popular examples is Bad News \cite{roozenbeek2019fake}- In Bad News, players actively learn the strategies used to create and spread fake news within the game's narrative, such as the use of emotionally charged content and the manipulation of social media platforms. Roozenbeek and van der Linden \cite{roozenbeek2019fake} conducted a large-scale evaluation of this game, providing evidence that through this serious game mechanic, players become more aware of the psychological techniques behind misinformation, thus improving their ability to critically evaluate real-world information. Another notable example is Fakey \cite{micallef2021fakey} in which the goal of the player is to support a healthy social media experience by promoting information from reliable sources rather than from low-credibility sources. 
While the multiplayer approach is less popular, there are several attempts to create PvP/team settings for misinformation games. For example, FakeYou! \cite{clever2020fakeyou} offers a simple mobile environment where players can create fake news headlines and test their ability to spread misinformation by challenging another player's ability to recognize misinformation. DoomScroll \cite{wells2024sus} proposes a team-based playing mode, where several players together try to tackle misinformation-related challenges.

\subsection{AI-Driven Solutions in Combat Misinformation and Educational Games} 
AI-based technologies are widely used in combating misinformation. These include automated fact-checking \cite{choi2024fact}, AI-based credibility indicators \cite{lu2022effects}, AI and LLM-based explanations of content veracity \cite{horne2019rating,mitgenai2023}, and personalized AI fact-checking systems \cite{jahanbakhsh2023exploring}. Most of these efforts focus on a debunking approach, where false information is identified and corrected after dissemination. 
In the context of prebunking and media education, there was an attempt to develop a role-playing game where players acted as fact-checkers, reviewing storylines generated by LLMs. The results showed that this approach could empower players to critically engage with misinformation through investigative role-play \cite{tang2024mystery}. Gallotta et al. analyzed the LLM roles used in game and identified LLM-based agent under specific roles are well-suited for real-time, context-based interaction\cite{gallotta2024large}, which can provide players personalized game experiences. Generative LLM-based agents can simulate human behavior based on past events and reflection \cite{1park2023generative}. 
Wang et al. introduced a system where LLM-based narrative planning and character simulation create dynamic stories\cite{wang2024storyverse}. Similarly, Peng et al. explored a text-adventure game where, under a fixed narrative premise, players can freely interact with NPCs generated by GPT-4, leading to emergent gameplay behaviors\cite{peng2024player}. 

Challenges such as so called "hallucinations" remain when LLM produce false information, as shown in research by Park et al. \cite{1park2023generative}.
Previous studies also noted memory limitations in these models. While solutions have been discussed, long interactions and excessive prompts come with the risks of inconsistent outputs \cite{zhong2024memorybank}. Also, variations in writing style and spelling in the input text can significantly impact the outputs, sometimes to the extent of producing incoherent outputs \cite{SolidGold}.