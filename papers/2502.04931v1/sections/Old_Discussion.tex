%\subsection{Educational Effects of the Breaking the News Game}
%In recent years, games have proven to be effective tools for improving media literacy and building resilience against misinformation. Several games, such as Bad News, Fakey, and FakeYou!, have shown promising results in this area\cite{roozenbeek2019fake,micallef2021fakey,clever2020fakeyou}. Breaking the News contributes to this growing body of work by enhancing players' ability to discern unreliable and misleading information. Our findings indicate that Breaking the News is effective in improving participants' media literacy, especially in evaluating and analyzing media content. The recent study DroomScroll game and other similar games can help explain the results and discuss how to balance the entertainment and educational effects of serious game design for combating misinformation.

%Many games employ inoculation theory by exposing participants to weakened forms of misinformation to build resilience\cite{roozenbeek2019fake,jeon2021chamberbreaker,harmonysquare}. Breaking the News follows this approach but differs by introducing more interactive elements, increasing the educational impact. Unlike single-player, choice-based games—where players make decisions with little pressure—Breaking the News uses a competitive PvP role-play format. This dynamic interaction encourages players to continuously craft strategies, whether to spread or debunk misinformation, fostering deeper learning and critical thinking. The competitive nature of the game enhances engagement and motivates players to consider both sides of the misinformation process, further strengthening their media literacy skills. Our findings confirm earlier observations from Doomscroll that multiplayer games need to carefully design player roles to ensure balanced participation\cite{wells2024doomscroll}. %When engagement is uneven, learning outcomes may suffer, especially if some players rely on others. 

%Explicit instructional support also plays a critical role in enhancing learning. As noted by Doomscroll’s authors, providing players with clear guidance is crucial for educational impact\cite{wells2024doomscroll}. In Breaking the News, we included text-based instructions to guide users in identifying misinformation, and the opportunity to practice these skills in-game further reinforced the learning outcomes. Future versions of similar games could benefit from more interactive instructional tools, such as allowing players to investigate sources or use search engines within the game, to further enhance engagement and learning.

%Another key design feature of Breaking the News is the inclusion of real-time feedback from five simulated personas powered by LLMs. This dynamic feedback loop adds a layer of complexity and engagement, helping participants reflect on the broader impact of misinformation. By adapting their strategies based on the personas' reactions, players are able to continuously refine their approach, whether creating or debunking misinformation. This iterative process reinforces the educational value of the game by helping players practice misinformation tactics in a more realistic context.

%The extended gameplay time, which unfolds over multiple rounds, also contributes to the learning outcomes. Unlike other misinformation games that provide users little context, Breaking the News introduces new misinformation in each round, gradually increasing the complexity of the game while maintaining a simplified narrative. Participants recognized the game’s news content as reflective of real-world misinformation, enhancing their learning experience. Considering the complexity of misinformation and the evolving nature of evidence, Future Game can also offer dynamic, context-rich scenarios which simulate the complexities of real-world misinformation. While extended gameplay can deepen understanding, it must be carefully balanced, as participants reported feeling fatigued by the final rounds. This was partly due to the freeform nature of the responses, which required significant time and effort, as well as the game's reading load and delays when waiting for opponents. The next steps in the game's development should aim to balance the depth of learning with the time investment to optimize both engagement and educational outcomes.

%While Breaking the News successfully raised awareness about misinformation and improved players' critical analysis of news content, it did not significantly improve their functional and critical prosuming skills. This suggests that while players became more adept at identifying misinformation, their ability to create media content was not equally enhanced. One likely reason for this is the limited content creation options in the game, which were primarily text-based. The absence of multimedia elements such as images, videos, and hyperlinks likely constrained the development of broader prosuming skills. Additionally, pre-survey data indicated that participants already had high confidence in these skills, which may have limited the potential for significant gains and results in no changes in perceived self-efficacy.

%\subsection{Applying LLMs to Serious Game Design}
%LLM agents have been increasingly leveraged to simulate human interactions in various fields, such as surveys\cite{argyle2023out}, HCI studies\cite{hamalainen2023evaluating} and psychological research\cite{dillion2023can}. There is growing evidence that LLMs can produce consistent behaviours that align with predefined personalities\cite{callison2022dungeons,chuang2023evaluating,park2023generative,shin2024understanding}. In our study, we observed how LLMs can act as personas within a game, interacting with players and simulating human-like responses. This application contributes to the growing body of work exploring how LLMs can simulate diverse individuals in dynamic, real-time environments, offering varied feedback based on player actions.

%In Breaking the News, LLMs simulate different citizen personas and respond to both misinformation and anti-misinformation messages from human players. This allows the game to provide real-time feedback that mirrors the complexity of real-world misinformation scenarios, where audience reactions are unpredictable and diverse. By generating context-aware responses, LLMs add depth to the learning experience, helping players understand how misinformation spreads and how different audiences perceive it. This dynamic interaction creates a more engaging gameplay experience, making the simulation more reflective of real-world challenges in misinformation.

%The use of LLMs in serious games highlights the potential of digital environments as ideal spaces for AI agents to thrive, given their complexity, diversity, controllability, safety, and reproducibility\cite{hu2024survey}. Our game represents a combination of generative AI with misinformation education, offering players more thoughtful, dynamic, and enjoyable interactions compared to traditional static or scripted systems.
%\subsubsection{Challenges of Using LLMs in Serious Games}
%Despite the advantages of using LLMs to simulate human behaviour, there are several inherent challenges. One significant issue is the potential for bias and inaccuracies in the model’s responses\cite{gallegos2024bias}. LLMs are trained on large datasets that may contain intrinsic biases, which can manifest in unintended ways during gameplay\cite{kasneci2023chatgpt}. Studies have shown that LLMs are prone to hallucinations and can reflect stereotypes or skewed perspectives, which could affect how certain personas respond in the game\cite{xie2024can}. For instance, if biased training data influence a simulated persona’s reaction, it may inadvertently reinforce player stereotypes about certain groups of people. Although we attempted to mitigate this by only providing essential demographic information for each persona and using well-grounded literature to guide their reactions, it is impossible to control every interaction or player input. 
%Discussion of the opportunities and ethical and societal risks of generative agents in interactive systems. responsible and ethical usage to mitigate potential risks.
%Additionally, LLM-generated personas may oversimplify human emotions and fail to grasp the full context of certain situations. Human reactions are often complex, nuanced, and influenced by multiple factors, including culture, history, and personal experience. Despite we use advanced GPT-4 model, it may miss these subtleties, leading to interactions that feel artificial or incomplete. For example, in the gameplay, players employ strong emotional manipulation strategies to provoke specific responses, but the LLM-generated personas do not react as expected, where participants reported feeling frustrated when their emotional manipulation strategies did not yield the anticipated reactions. This can limit the diversity and depth of the simulated interactions and detract from the realism and fairness of the experience for certain audiences.

\subsection{General Discussion}
 
\textcolor{blue}{Our mixed-methods study has yielded rich results regarding the empirical effectiveness of the Breaking the News game. The quantitative results suggest that the game intervention effectively enhances players' skills in understanding, analyzing and evaluating media content. Players exhibited significant improvements in discernment ability. However, it did not substantially enhance players’ ability to create media content and their self-efficacy towards misinformation. To better understand these results, we compare the game’s mechanisms with prior interventions. }

\textcolor{blue}{Our results revealed the benefit of the PvP format game. Prior misinformation education game interventions broadly employ single-player, choice-based mechanics\cite{roozenbeek2019fake,camCambridgeGame,harmonysquare,jeon2021chamberbreaker}. These games frame the narrative as either misinformation creator or misinformation debunker and asks the player to adopt one of the roles, tasked by discerning misinformation using choice-based interaction. This approach is effective but may limit engagement due to its static nature. In contrast, Breaking the News combines the role of misinformation creator and debunker in the game and takes a PvP format. Players adopt one of the roles, creating and countering arguments in real-time and defeating each other. This format fosters a more interactive and competitive experience, requiring players to engage deeply with the content. PvP has been found as an effective approach in improving player’s engagement, motivation, and learning experience\cite{cagiltay2015effect,foster2013competition}. Our findings provide evidence that misinformation education context can also benefit from PvP format and not only be limited to single-player fashion. However, multi-player formats are not without challenges. For instance, DoomScroll demonstrates that multi-player(team) based games may dilute individual engagement and contributions, diminishing educational outcomes\cite{wells2024doomscroll}. Thus, designing a multi-player mode needs to take caution, ensuring active participation and skill development for all players. 1-on-1 competition may be a good format for increasing engagement while achieving learning outcomes.}

\textcolor{blue}{Another innovation of Breaking the News is the interaction and feedback component. Compared to the choice-based format, where players passively choose predetermined options and get binary and concise feedback(e.g., true or false), Breaking the News lies in employs an open-world, free-form response system, providing players with creative control. Players tasks actively craft their responses and then receive evaluation by LLM-stimulated personas. These personas are based on their characteristics, offering unique feedback toward the player’s output. By engaging with these nuanced evaluations, players gain insights into why certain arguments might appear trustworthy or untrustworthy, encouraging them to refine their strategies dynamically. Additionally, Since the game is multiple rounds, players will get to know personas and adjust their misinformation narratives or debunking approaches based on the personas’ reactions. Thus, they learn and practice various tactics to identify and debunk misinformation.}

\textcolor{blue}{Despite its strengths, Breaking the News did not significantly improve players’ functional and critical prosuming skills. One possible explanation is that the game's primarily text-based format limits opportunities for players to create diverse forms of media. Unlike real-world misinformation, which often incorporates multimedia elements such as images, videos, and hyperlinks, the game’s design focuses solely on textual interactions. Another misinformation game provides a set of tools (e.g., reverse image search, image verification assistant) to help the players achieve tasks, which may also be a useful mechanism in the future game creation\cite{katsaounidou2019mathe}, for example, by combining reverse search and integrating multimedia content creation tasks, allowing players to practice crafting and analyzing more complex misinformation scenarios. Another possible explanation is that pre-survey data indicated that participants entered the study with high confidence in their prosuming abilities. This "ceiling effect" likely constrained measurable gains, as participants perceived less room for improvement.}

\textcolor{blue}{Participants did not show significant improvement in their self-efficacy regarding their ability to identify or produce news-like information in quantitative analysis. Self-efficacy, defined as one’s perceived ability to complete specific tasks or achieve objectives\cite{bandura1988organisational}. However, game log data indicated that players' proficiency improved during gameplay; by the third and fourth rounds, they consistently produced longer and more comprehensive messages compared to the initial rounds. Social Cognitive Theory may explain this discrepancy. According to the theory, enactive mastery experiences—successful task completion—are the most influential sources of self-efficacy\cite{bandura1997self}. Positive experiences bolster self-efficacy, while repeated failures undermine it. In our game, success was determined by evaluations from LLM-simulated personas. Qualitative results revealed that players found no single strategy effective for all personas, requiring frequent adjustments based on each persona’s unique characteristics. This unpredictability made it difficult to achieve consistent success. Furthermore, as persona evaluations responded to both players’ actions, scores fluctuated frequently, with one player’s gains often neutralized by their opponent’s moves. This dynamic likely hindered players’ confidence in their ability to discern misinformation successfully, even though their actual performance improved, as evidenced by significant gains in discriminative test results. Thus, while players developed greater proficiency during the game, the lack of sustained, clear success may have limited their perceived self-efficacy.
}

\textcolor{blue}{Lastly, In our qualitative findings, participants reported skepticism toward information that purely relied on authoritative sources (high credibility sources). However, these sources usually share and publish authentic content instead of misinformation. Looking back to the gameplay, participants raised this skepticism because they found their opponents used fabricated evidence from authoritative to gain LLM-simulated personas’s trust and increase scores. This gameplay experience reminded them of real-life situations, where misinformation often exploits trust by citing credible authorities. As a result, players learned to examine the intent behind messages instead of automatically trusting authoritative sources. This observation can be explained through the lens of the Elaboration Likelihood Model (ELM)\cite{petty1984source}, which describes how individuals process persuasive messages via central(critical evaluation of content) or peripheral routes(reliance on heuristic cues). In the gameplay, participants appeared to shift from peripheral processing (trusting authority as a shortcut) to central processing (evaluating the intent and content of the message) when exposed to the misuse of authority in gameplay. Notably, many participants suggested cross-checking sources as a practical solution for verifying information, demonstrating their enhanced critical thinking skills}.

\textcolor{blue}{Our findings align with studies on gamified misinformation interventions that gamified inoculation techniques that pre‐bunk misinformation can trigger skepticism related to both false and real news\cite{hameleers2023intended,modirrousta2023gamified}. While this induced skepticism might seem limiting—potentially undermining trust in high-credibility sources—we believe it supports the goal of fostering critical media literacy. By encouraging players to evaluate the trustfulness of content, source and intent,  the game develops essential skills for navigating today's complex information landscape. Rather than promoting cynicism, this skepticism cultivates constructive inquiry that helps individuals better discern reliable information. Additionally, recent game interventions also find how to develop individual's ability in assessing both real and false information. These strategies include teaching players to distinguish actual mainstream and low-credibility media outlets\cite{micallef2021fakey}, rewarding accurate judgment\cite{barzilai2023misinformation}, and adding feedback tests after gameplay\cite{leder2024feedback}. For future iterations of games like Breaking the News that feature narrative and open-form interaction, we suggest implementing these improvements. For instance, the game's narrative could clearly present high-credibility and low-credibility sources before and during gameplay. After gameplay, debriefing sessions could allow players to review their strategies, analyze their effectiveness, and receive constructive feedback, potentially improving their discriminative ability.}

\textcolor{red}{Pre-emptive intervention games have been shown to effectively engage players in recognizing, combating, and navigating misinformation, making them a promising approach to promoting media literacy, fostering critical thinking, and raising awareness of the dangers of false information\cite{roozenbeek2019fake,micallef2021fakey,clever2020fakeyou,katsaounidou2019mathe,escapefake,tang2024mystery}. Our study builds on these insights by exploring how PvP, free-form interactive narrative game, can improve players' ability to identify misinformation. By incorporating dynamic, competitive elements, we encourage active engagement with misinformation. Our findings suggest that \textit{Breaking the News} significantly improves participants' media literacy, particularly in evaluating and analyzing media content. Below, we discuss the game elements that support learning and offer considerations for future designs.}

\textcolor{red}{\subsection{Role of Misinformation Examples in Learning}
One strength of misinformation education games is their ability to present real-world scenarios that expose players to various forms of misinformation. In \textit{Breaking the News}, the in-game news can serve as effective misinformation examples, teaching players to identify key characteristics. This aligns with similar games where players assess the veracity of articles, headlines, or social media posts\cite{roozenbeek2019fake,katsaounidou2019mathe,micallef2021fakey}. By simulating real-world content, players learn to recognize strategies like emotional manipulation, sensationalism, and cultural or national biases. This practical exposure equips them with skills to resist misinformation in their everyday media consumption, reinforcing media literacy.
As with similar games, \textit{Breaking the News} uses %simplified 
short narratives to convey misinformation. This raises a critical design challenge: balancing realism and playability. While simplification is necessary for effective gameplay, it may limit the depth of the learning experience. Participants in our study noted that real-life misinformation is often more complex and subtle. While we attempted to address this by progressively introducing complexity through storyline-connected narratives, it seems that the approach did not fully mimic the realistic flow of misinformation. However, as the game demonstrated good educational outcomes, it appears that the level of similarity was sufficient to ensure the desired effect. }

%We only focused on pandemic scenarios.

%Since various types of misinformation have distinct characteristics, future designs should explore how to include dynamic, context-rich scenarios—such as deepfakes or AI-generated content—to simulate the evolving nature of real-world misinformation and enhance the educational experience

\textcolor{red}{\subsection{Balancing Entertainment and Learning}
One of the core challenges of designing misinformation education games is balancing entertainment with educational impact. Engaging gameplay is necessary for maintaining player interest, but educational goals should not be overshadowed by purely entertaining elements. Many misinformation games are choice-based and single-player mode\cite{roozenbeek2019fake,camCambridgeGame,harmonysquare,jeon2021chamberbreaker}. These games present predetermined options, allowing for fast decision-making and immediate feedback on the consequences of choices. This structured approach minimizes cognitive overload and keeps players focused on specific misinformation tactics, making it good for short, targeted educational experiences. However, the limited interactivity and predefined choices may restrict players’ ability to explore misinformation more deeply, potentially reducing critical thinking opportunities.
In contrast, \textit{Breaking the News} uses an open-world, free-form response system that gives players significant creative control. By allowing players to write their own responses and receive real-time feedback from LLM-powered personas, the game encourages more complex decision-making. This approach more closely mirrors real-world misinformation dynamics, as players must adapt their strategies independently. However, the unstructured nature of free-form games can be mentally taxing, with some participants reporting fatigue in later rounds due to the effort required. 
While free-form games may enhance immersion and retention, designers must carefully balance these elements to maintain engagement without overwhelming players. Providing clear guidance is also needed to ensure educational impact, as players may feel lost or lack the necessary knowledge to progress, potentially undermining learning outcomes\cite{wells2024doomscroll}.
Future designs can choose between game formats based on target audiences and desired learning outcomes. 
Alternatively, hybrid models could guide players through predetermined choices initially, transitioning to free-form gameplay as their skills and confidence develop. This balance between low-cognitive-load choice-based games and immersive free-form environments could optimize both engagement and learning.}

\subsection{Role of Competition and Collaboration}
The competitive nature of \textit{Breaking the News} adds an engaging element to the learning process, but further investigation is needed to understand how competition impacts educational outcomes. For instance, adding mechanisms like information dissemination speed and time limits may encourage players to react more quickly to the opponents' actions. Other games have shown that enhanced competition encourages players to make real-time decisions, gain a better understanding of game mechanics and rules, and develop creative strategies to improve their skills. 

In competitive environments, the motivation to “win” may overshadow educational goals. Previous research on games like Bad News in school settings indicates that some students may not find %gamification or
competition enjoyable or motivating\cite{axelsson2024bad}. Exploring alternative game modes, such as collaborative play, could provide additional avenues for learning. For example, players could collaborate to gather evidence or have one player write fact-checking messages while another creates multimodal content to maximize influence. This cooperative approach may foster a deeper understanding of media literacy and promote social learning.

In terms of social interaction, PvP gameplay often involves strong social dynamics. One of interesting findings in our experiment was that players who were more familiar with each other tended to exhibit greater competitiveness, which in turn enhanced their in-game strategies and overall experience. Investigating how these dynamics affect misinformation detection skills could offer valuable insights into how social environments shape educational outcomes.

\subsection{LLM-Powered Feedback and Debunking Practice}
A unique feature of \textit{Breaking the News} is its use of LLM-powered personas that provide dynamic, real-time feedback. This feedback loop encourages players to refine their strategies continuously. Most games focus primarily on detecting misinformation\cite{katsaounidou2019mathe,tang2024mystery,clever2020fakeyou}, leaving debunking tasks to professionals such as journalists and fact-checkers, as these roles require specialized skills. However, debunking is not limited to professionals; everyday people frequently engage in debunking actions in their daily lives. For example, research shows that many social media users actively seek out comments to gauge others' opinions and assess the credibility of information\cite{geeng2020fake}. Another study found that debunking misinformation often occurs within families, sometimes leading to conflicts\cite{scott2023figured}. Major social media platforms like X introduced "Community Notes,"\cite{xcommunitynotes2024} a crowd source fact-checking program that allows regular users to contribute to debunking efforts, and reports suggest it has been effective in reducing misinformation\cite{chuai2024community}. These examples demonstrate that ordinary people can benefit from learning debunking skills, which can contribute significantly to reducing misinformation within both personal networks and broader communities.

Debunking in daily life can be challenging, with many individuals reluctant to engage publicly due to fear of conflict or judgment\cite{tang2024knows}. Games provide a risk-free environment where players can practice debunking strategies without real-world consequences. In \textit{Breaking the News}, the LLM-powered personas react to various misinformation and anti-misinformation messages, allowing players to experiment with tailored strategies. As a result, players reported increased confidence in their ability to debunk misinformation after successfully persuading the personas within the game. This practice may carry over into real-life scenarios, enabling more effective misinformation countermeasures.

However, caution is needed when relying on LLM-powered feedback, as inherent biases in the AI could skew results toward certain demographics\cite{xie2024can}. Additionally, while in-game personas provide explicit and understandable reactions, real-world responses are often more complex. Future iterations of the game could feature more detailed personas or customizable scenarios, such as a “family mode” where players practice debunking misinformation with personas modeled after family members. This personalized approach could further enhance the game’s educational value.


\subsection{Limitation and Future Work}
%background - age
We acknowledge the following limitations. \textcolor{blue}{Our participants were mainly from an East Asian country, which may limit the generalizability of our findings. Previous studies showed that cultural contexts influence individuals’ perception of misinformation and interventions\cite{noman2024designing}. For instance, studies show that political beliefs, such as conservatism, increase susceptibility to fake news\cite{gupta2023fake,uscinski2020people}. Furthermore, individuals with strong political affiliations are generally less likely to accept fact-checking information\cite{jarman2016influence,walter2020fact}, making interventions less effective for combating misinformation in highly polarized societies. In contrast, our sample primarily comes from East Asian, where political polarization is typically lower. They may have higher willingess to accept misinformation correction message, which could result in more positive outcomes for misinformation interventions, as observed in our study.} 

\textcolor{blue}{Despite Asian individuals are more willing to accept fact-checking message, they are facing significantly higher odds of encountering difficulties in assessing health information from social media compared to non-Hispanic White individuals and other groups, as observed in a large-scale study\cite{chandrasekaran2024racial}. Additionally, this study found that Asian individuals are more likely to incorporate social media information into their health-related decisions, potentially increasing their susceptibility to misinformation\cite{chandrasekaran2024racial}. Therefore, we believed the game intervention can assist them verify online misinformation for some extent and alleviate the problem.}

\textcolor{blue}{Although demographic factors like age and gender have shown mixed results in relation to misinformation susceptibility, a recent meta-analysis indicates that neither age nor gender significantly impacts the effectiveness of media literacy interventions. This suggests that such interventions can be broadly effective across different demographics\cite{lu2024can}. Additionally, studies show that even adults(often considered less susceptible to misinformation) face challenges in verifying online health information\cite{chandrasekaran2024racial,reuters2023dnr}. These findings highlight that adults of all ages can benefit from enhanced media literacy. Nevertheless, research also suggests that media literacy interventions specific for certain age groups (e.g., older adults and teenagers) achieved greater intervention effects\cite{moore2022digital,hartwig2024adolescents}. This indicates that while media literacy interventions work broadly, customized approaches for specific groups may yield stronger outcomes. Future interventions could be tailored based on the characteristics of the target audience. For example, older adults often experience limitations in vision, motor skills, and memory, which can hinder their engagement with digital tools and lead to frustration\cite{wildenbos2018aging}. Research also shows that they are more likely to seek support from younger family members and value clear indicators of progress\cite{vaswani2023novel}. Given their higher voting rates and susceptibility to disinformation\cite{aarp2018oldervoters}, media literacy initiatives designed with these needs in mind could have a meaningful impact. Young adolescents, on the other hand, while adept with digital technology, may lack the critical thinking skills needed to assess information effectively and often overestimate their ability to do so\cite{papapicco2022adolescents,porat2018measuring}. Future interventions could be customized to address these specific vulnerability groups, maximizing the impact of media literacy programs across demographics.}


\textcolor{red}{\sout{Our sample mainly included young people with higher levels of education. This limits our understanding of the game’s effects across various age groups and educations. Specifically, we did not have people with a declared medical background, who probably have more first-hand experience working with health-misinformation debunking and therefore have the practical knowledge of how to debunk these myths; future studies should address it and specifically test the misinformation educational games with experts (journalists and medics) to receive deeper insights.} }
%Additionally, the difference in skill between the two players could have influenced each player’s experience and learning outcomes. While our participants had relatively similar educational backgrounds, we did not observe significant differences between the players’ experiences. However, this similarity may have led to slightly more optimistic results due to potential sample bias.

%background politics - analyze, hypothesize what happens if you it with white people.
%Second, our participants were mainly from East Asian countries, where political affiliations are not as polarized as in other regions. This may limit the generalizability of our findings, particularly concerning the impact of partisanship. Prior research shows that individuals with strong political affiliations are less likely to change their opinions, and since our sample did not include participants from this group, our results may have skewed more positively.
%no control group (between before and after, you don't what can happen)
Additionally, future studies should include comparing different types of media literacy training. Comparing the game with other forms of training—such as lectures could provide deeper insights into its strengths and weaknesses. For example, a similar media literacy education program that incorporated a fact-checking game, lectures, and a control group found that while the game-based approach improved cognitive and civic reasoning skills, a combination of game and lecture led to better attitudinal outcomes, such as self-perceived media literacy and information discernment efficacy\cite{yang2024designing}. This suggests that while games work in delivering experiential and practical skills, lectures may play an important role in fostering deeper understanding and confidence. %Suppose we had incorporated a control group or compared different types of training. In that case, we might have identified elements that could be better addressed through lectures, potentially reducing gameplay time and preventing participant exhaustion.
%two players learning diff things: spread vs control, diff play flow so how can you assume they played the same game?
In this study, each player was limited to a single role—either misinformation creator or debunker—resulting in different learning experiences depending on their assigned role. %While participants reported learning from their opponents, the educational outcomes between roles may have varied. 
The primary reason for not allowing role-switching was the length of the game and its cognitive demands, which we feared would lead to player exhaustion if roles were switched mid-game. In future iterations, we aim to improve the design by allowing players to save their progress and switch roles during subsequent sessions. This could offer a more immersive experience, as players would gain perspectives from both the misinformation creator and debunker roles. Future versions could alsoincorporate more investigative tools to help players assess evidence. For instance, lateral reading and source evaluation techniques, similar to the artificial search engines used in the MathE game, could support both player roles in gathering and evaluating information.
%how is this representing range of possible misinformation? survey is not comprehensive, should be finding that they learned about one subtopic but not another that was not addressed.

The current game is also limited in its focus solemnly on a pandemic scenario. In reality, misinformation spans multiple domains, and health-related misinformation  can influence political events such as elections. Our game only addressed text-based misinformation, while visual and video-based misinformation pose even greater challenges and are harder to detect. Future work could include  multimedia content, such as images and videos, to more accurately simulate the diverse forms of misinformation that exist in the real world.

While LLMs like ChatGPT-4o were used to simulate human reactions in the game, these models do not fully replicate the complexities of human behavior. Human reactions are often nuanced and influenced by multiple factors, including culture, history, and personal experience. LLM-generated personas may oversimplify human emotions and fail to grasp the full context of certain situations. Despite we use advanced ChatGPT-4o model, it may miss these subtleties, leading to interactions that feel artificial or incomplete. For example, in the gameplay, players employ strong emotional manipulation strategies to provoke specific responses, but the LLM-generated personas do not react as expected, where participants reported feeling frustrated when their emotional manipulation strategies did not yield the anticipated reactions. This can limit the diversity and depth of the simulated interactions and detract from the realism and fairness of the experience for certain audiences.

LLMs are trained on large datasets that may contain intrinsic biases, which can manifest in unintended ways during gameplay\cite{kasneci2023chatgpt}. Studies have shown that LLMs are prone to hallucinations and can reflect stereotypes or skewed perspectives, which could affect how certain personas respond in the game\cite{xie2024can}. For instance, if biased training data influence a simulated persona’s reaction, it may inadvertently reinforce player stereotypes about certain groups of people. 

%Finally, the game was primarily designed in English, which may have affected immersion and comprehension for participants who were not fluent in the language.

%how is this representing range of possible misinformation? survey is not comprehensive, should be finding that they learned about one subtopic but not another that was not addressed.

%LLM do not truly represent the public opinion, biased, misrepresented.

%two players learning diff things: spread vs control, diff play flow so how can you assume they played the same game?




%\subsection{Future Work}
%In terms of game design, we only created five personas to represent public opinion within the fictional “Country.” Although we tried to cover a broad spectrum of age, education, profession, and personality, this limited number of personas cannot fully capture the diversity of real-world public opinion. Additionally, the inherent biases within LLMs could have introduced stereotypes in the personas and their reactions, potentially reinforcing existing biases.
%There are several directions for future work. One potential expansion is to incorporate more diverse topics of misinformation beyond pandemics and health, such as political misinformation or climate change, to enhance replayability. Although misinformation tactics share common features across different domains, each topic presents unique characteristics and challenges. We have already demonstrated that games are an effective medium for confronting misinformation, and the mechanics of \textit{Breaking the News} are designed to simulate the complexity of misinformation in the real world. By changing the in-game news topics, we could tailor the educational impact for different audiences relatively easily.

%Another direction is to allow players to switch roles, experiencing both the misinformation creator and debunker perspectives. This could provide a fuller, more immersive experience. To further enhance the educational value and replayability, we also plan to incorporate more tools to help players investigate evidence. For instance, as in MathE, which designed artificial search engines for its game, we aim to introduce investigative techniques like lateral reading and source evaluation to support both player roles.

%Additionally, future work could explore different formats of misinformation within the game, including multimedia content like images and videos. This would address the limitation of focusing mainly on text-based misinformation and would reflect the growing challenges posed by visual misinformation in real life. Allowing players to create or debunk multimedia misinformation would make the game more relevant to the evolving nature of misinformation.