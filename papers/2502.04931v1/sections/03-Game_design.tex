\subsection{Overview of Game Design}
The game we designed focused on the challenges of managing information in a health crisis. One player assumes the role of an influencer (Player 1) hired by a company to promote a remedy based on traditional medicine philosophies but lacking extensive scientific support. This player can create and disseminate misinformation about the remedy. The second player takes on the role of a journalist-debunker (Player 2) advocating for a newly developed medicine supported by current scientific research. This player’s goal is to debunk the misinformation spread by the influencer. The game features a system that simulates public opinion, whereby an LLM models the reactions of five characters who read players' messages. Previous studies showed the robustness of the LLM in evaluating the quality of an argument \cite{mirzakhmedova2024large} (See details in Section \ref{Prompt Engineering}). The objective is to sway the simulated public opinion in favor of one's position.


%The hints was generated by ChatGPT-4o model using the same materials that were utilized to create the in-game news. The model provides suggestions on how each player might respond to the news from their perspective. When authors used to review the quality of hints and how they can affect the news to ensure balanced gameplay.




%\subsubsection{Players}
\begin{figure}[H]
    \centering
    \includegraphics[width=0.9\linewidth]{figs/game_flow_RR.jpg}
    \caption{Game Flow. (A) Both players read the current round's news and instructions. (B) Player 1 starts first as an influencer to generate misinformation by choosing whether to buy the hints. (C) Player 1 inputs text and sends a request to the GPT-4o API. (D) GPT-4o API returns a response of public opinion. (E) Player 2 then starts to counter Player 1's misinformation by choosing whether to buy the hints. (F) Player 2 inputs text and sends a request to the GPT-4o API. (G) GPT-4o API returns an updated public opinion.}
    \label{fig:flow}
\end{figure}
\subsection{Gameplay}
\subsubsection{Game Flow}
Participants were randomly assignedto one of two roles: influencer (Player 1) or journalist-debunker (Player 2) or Player 2. Both players are provided with instructions, including the setting, the fundamental reality regarding the effectiveness of the two medicines, their roles and tasks, demographic information about the public personas of the people with whom they would interact, and the rules of the game (\autoref{fig:flow}A).

The game unfolds over four rounds, each featuring a new set of updated news. In each round, the influencer (Player 1) begins by reviewing the news and any instructions. They can also decide to buy hints provided in the game using in-game currency (\autoref{fig:flow}B). Once ready, the influencer types their information and it is published (\autoref{fig:flow}C). The LLM reacts to this information by simulating public opinion, with this impact of the information on public opinion visible to both players (\autoref{fig:flow}D).

Next, the journalist-debunker (Player 2) takes their turn. Player 2 reviews the current public opinion and where appropriate counters any misinformation by typing their debunking response, which they then publish (\autoref{fig:flow}F). If needed, they can purchase customized hints (\autoref{fig:flow}E) or read the instructions. After publishing their response and receiving feedback from the LLM, the round ends, and the game progresses to the next round (\autoref{fig:flow}G).

At the end of each round, both players can view the results, which reflect the journalist-debunker’s impact on misinformation. The process for the remaining three rounds is the same as in the first round.

%The game consists of four rounds, each introducing players to a new narrative context that reflects the evolving public health crisis in Southland. In each round, Player 1 first generates misinformation, using the instructions and hints provided within the game. After completing the misinformation, Player 1 sends it to the OpenAI API. The GPT-4 model then simulates public opinion, producing a response that is displayed on both players' screens. Subsequently, it is Player 2's turn to generate their information, following the same process as Player 1. Once Player 2 receives the public opinion response, the round is completed, and the game proceeds to the next round.

%In the game, the public opinion is composed of feedback from five distinct personas(See details in Section XX). The message returned by the API primarily consists of three components: the basic information of each persona, their reactions, and their trust level scores regarding the misinformation.


\subsubsection{Narratives}
The game is set in a fictional small country with called Southland. Historically, Southland has been known for producing renowned medical doctors and pharmacists, and its people take pride in their traditional medical practices. However, there are ongoing debates in this country about the comparative merits of modern healthcare methods and traditional medicine. %due to the lack of data on the safety, efficacy, and quality of most medicinal plants. 
The sudden outbreak of the "Zinc Virus" further amplifies these debates. As the healthcare system becomes overwhelmed and the scientific community unable to provide an effective treatment because of limited knowledge about this novel virus, residents turn to traditional medicine in search of hope.

We set the game in a health crisis scenario because, in real life, situations marked by scientific uncertainty – where authorities can be unable to provide confident full explanations or advice – often  fuels rumors and speculation about treatments\cite{wang2019systematic}. These dynamics were observed during the Ebola\cite{fung2016social}, Zika\cite{wood2018propagating}, and COVID-19 pandemics\cite{suarez2021prevalence}. In such scenarios, traditional medicine frequently promoted to prevent or treat viruses\cite{lam2021public,mutombo2023experiences}. Additionally, we incorporated the traditional medicine controversy into our narrative as these debates are well-known to our participants, who were of an East Asian background. Research showed that 48.4\% of Hong Kong residents reported using traditional medicine before the COVID-19 pandemic\cite{lam2021public}. Similarly, a national survey in South Korea found a 74.8\% prevalence of traditional medicine use overall\cite{ock2009use}. In China, traditional medicine is formally integrated into the healthcare system as a widely practised modality\cite{chung2023implementation}. This cultural relevance provided a familiar backdrop for the game's content, allowing players to engage intuitively and to craft responses based on their lived experiences.

To avoid biases or stereotypes associated with traditional medicine, we use "Product R" to refer to traditional medicine. Before gameplay, we explained to players the nature of Product R, particularly stating that it has not undergone rigorous testing and lacks scientific consensus on its safety. We also introduce another medicine called ”Max” to represent science-based medicine, and we also explain the nature of Max before the game begins. %Max has undergone clinical trials, and its results were published in a prestigious medical journal, but its effectiveness against the Zinc virus has been mixed. 
Players are informed that Max has been subject to rigorous clinical trials, with the results published in a peer-reviewed medical journal, however, these studies have demonstrated inconsistent effectiveness against the Zinc virus.
We deliberately avoided making vaccines a topic of this game to prevent players' pre-existing attitudes toward vaccines from influencing their behaviour in the game.
%Considering the game's objective to enhance players' ability to distinguish misinformation in real-life scenarios. We created a semi-realistic game setting inspired by real-world events. 

%The game is set in the Southland, with a profound history of 6,000 years, the Southland has been a place of multiculturalism since ancient times. Southland promotes individual liberty, freedom of expression and independence of the media. Historically, Southland has produced many renowned medical doctors and pharmacists, and its people remain proud of their traditional medicine practices. However, the role of these traditional therapies in modern healthcare has become a contentious issue, leading to ongoing debates. The sudden outbreak of the Zinc Virus shattered the tranquility of Southland. The healthcare system was quickly overwhelmed, hospitals became overcrowded, and medical personnel were overworked. Due to limited knowledge about the virus, the scientific community could not provide effective treatment, and hospitals could only offer standard supportive care. Panic spread throughout the city, with residents seeking ways to protect themselves, turning to traditional medicine as a source of hope. Following the virus outbreak, the streets of Southland became deserted as people minimized outings, and public places shut down. Social media became the primary channel for information dissemination but was also rife with rumors and misleading information. Scientists worked tirelessly to study the virus, hoping to find effective treatments. Meanwhile, traditional medicine gained rapid popularity among the public, despite the lack of scientific consensus on its efficacy. There’s a new medicine, Max, available, but it's expensive, and its effectiveness is uncertain. Only a few hospitals use Max, while most stick to standard treatments. Without clear medical solutions, some people turn to traditional medicine, like R, which has a long history in the country. Subsequently, R became the focus of attention. However, as time went on, controversy over the effectiveness of this traditional medicine and the vested interests behind it surfaced, and the Southland society fell into a fierce debate about science, tradition and trust.


To create the "News" pieces in the game, we first researched examples of misinformation by reviewing relevant literature. We selected key features of misinformation and incorporated them into the game's events (See details in \ref{Characteristics of Misinformation}). To ensure the misinformation was portrayed realistically, we investigated real-world examples from fact-checking websites\footnote{\url{https://www.snopes.com/}}, reputable news outlets\footnote{\url{https://www.wsj.com/}}, and medical websites\footnote{\url{https://www.webmd.com/}}\footnote{\url{https://sciencebasedmedicine.org/}}\footnote{\url{https://healthfeedback.org/}}. For instance, we represented "less credible sources" using personal stories, viral videos, and newspapers that can objectively be classified as being biased. In representing biased newspapers, we are further inspired by the political orientation that may influence preferences for medical care. For example, research has shown that voters who tend to support anti-corruption parties are less likely to seek services from mainstream healthcare providers, and are more inclined to use alternative treatments\cite{valtonen2023political}. Similarly, the "creating fake news for profit" feature was illustrated through scenarios where investors leveraged expert endorsements to boost sales. Next, we crafted "News" pieces for the game based on these findings. However, any similarities to real-world sites or companies are purely coincidental; all names and events was invented solely for the purpose of this study. Table 1 summarizes the key features of misinformation identified in the literature, the associated cited papers, and how these are reflected in the in-game news. The full version of the "News" is in the Appendix. Lastly, To ensure balanced gameplay, we used a ChatGPT-4o model to review the narrative and provide an opinion on the difficulty to players of dealing with each piece of news in the game context. Taking this opinion into account, we made further revisions and corrections when we conducted two pilot tests with four people. The goal was to ensure the game was balanced and gave opportunities to win the game for both players.
%We also incorporated different sources of information, including personal experiences, social media influencers, quackery and biased newspapers, into the narrative to educate players about the complexity of information sources. 

%However, any similarities to real-world sites or companies are purely coincidental; all content was invented solely for the purpose of this study. Table 1 summarizes the key features of misinformation identified in the literature, the associated cited papers, and how these are reflected in the in-game news (The full version of the news is in the appendix).

\begin{table*}[htbp]
\renewcommand\arraystretch{2}
\centering
\caption{Key misinformation features and corresponding "News" in the Game}
\begin{tabular*}{\textwidth}{@{}p{5cm} p{10cm}@{}}
\toprule
\textbf{Misinformation Characteristics} & \textbf{Representation in the Game's News} \\ 
\midrule
False information is often shared by lower-quality media. However, political ideology is positively associated with individuals' perceptions of media trustworthiness. As a result, people may place greater trust in biased sources that align with an individual's political partisanship.\cite{zhang2020overview,molina2021fake,hanley2023golden}
%  
& 
A newspaper reported that a renowned medical expert has advocated Product R, claiming its herbal ingredients could potentially treat the Zinc Virus. This newspaper is known for its anti-corruption stance. (Round 1) \\ \hline
Personal, negative, and opinionated tones predominate in misinformation narratives which frequently provoke dread, anxiety, and mistrust of institutions\cite{bessi2015trend,porat2019content}.  & 
A widow shared her husband's experience. She suspects that Max was ineffective and believes it may have caused renal impairment, eventually leading to her husband’s death. She claims, “He was given a medication we demanded he NOT receive, and his health quickly went downhill,” ultimately resulting in him being “on a ventilator working most of the time at 100\%.” (Round 2) \\ \hline
Fake news, some of which is purposely fabricated to cause harm, generate financial returns, or spread confusion\cite{mustafaraj2017fake,hanley2023golden,shu2017fake} & 
A journalist discovered that the institution of traditional medicine where the famous medical expert works received significant funding from billionaire Jack. Additionally, Jack's ex-wife owns a company that produces and promotes traditional medicine products like R. (Round 3) \\ \hline
False rumors will create feedback loops and evolve into more intense and extreme versions over time\cite{shin2018diffusion} & A popular short video claims that a doctor who practiced alternative medicine and R was murdered to protect the profits of “Big Pharma”. More people are attracted to believe in the validity of traditional medicines and advocate for their use while opposing new drugs. Growth in sentiment that resistance to traditional medicine amounts to being an attack on their cultural heritage. (Round 4) This round's advocate for R becomes more intense than in Round 1, with the focus being less on effectiveness but rather patriotic sentiment.\\ 
\bottomrule
\end{tabular*}
% End resizing
\end{table*}

\subsubsection{Instructions and Hints}
To support players, the game includes instructional content that features definitions\cite{molina2021fake,wu2019misinformation}, examples\cite{bbcCoronavirusBill}, and strategies for both creating and debunking misinformation. This content draws on insight gained into misinformation from research literature and practice. For the misinformation creator role, we applied the Elaboration Likelihood Model and used simple examples to teach players to craft persuasive misinformation\cite{petty1984source,moran2016makes}. For the misinformation debunker role, we used an Agence France-Presse fact-checking style-guide and an guide published by the EU on communicating with proponents of conspiracy theories \cite{afp2024,eeas2024}. These were the inspiration for a user-friendly guide we developed to assist game players to identify misinformation and equip them with effective debunking strategies. 
Additionally, the game offers two hints during each round to inspire players. The detailed hint is crafted by the authors using the same materials used to create the in-game news. The simple hint is generated by ChatGPT-4o model. When authors used it to review the news and ensure balanced gameplay, it provided concise suggestions on how each player might respond from their perspective. (Full instructions and hints can be found in the supplementary materials)

\subsubsection{LLM Basis}
We implemented the LLMs to play the role of "public opinion" in the game for three reasons.  Firstly, LLMs perform well when processing dynamic natural language\cite{hu2024survey}. Secondly, LLMs demonstrate memory capability, such as with working memory being applied to the context of a conversation, and long-term memory allowing past conversational information to be taken into account\cite{hu2024survey}. 
 Research also demonstrates that generative agents powered by LLMs build a high degree of capability for responding to the context of a conversation\cite{1park2023generative,yin2024lies}. Thirdly, LLMs excel at role-playing tasks in the game. Research indicated\cite{wang2023humanoid}\cite{hu2024survey} that directly inserting natural language descriptions of a role’s identity enable LLMs to make better quality evaluations in conversational tasks. These capabilities allow LLMs to effectively serve as "evaluators" in the game, generating continuous, context-aware dialogue and feedback. This approach proves more effective than traditional prebuilt game mechanics, such as trigger keywords for assessment. 

%\textcolor{blue}{
%In our game that incorporates inoculation interventions for misinformation, we provide players with a free-form, natural language game environment to generate misinformation, while their opponent debunks it. To let players know whether their information is effective or not, we designed five game personas that role-play and represent public opinion on their information. To assess these continuous dialogues and determine whether the players' inputs meet specific criteria and how they might influence public opinion to some extent, the LLM proves more effective for this task than traditional prebuilt game mechanics. Traditional approaches, such as relying on trigger keywords to determine whether players’ in-game statement is effective or not are limited.
%}
\subsubsection{Game Mechanics}
The player's success depends on the reaction of LLM-simulated public opinion. Inspired by Jaffe et al.'s research on game balancing systems\cite{jaffe2012evaluating}, we use a metric to determine which player's information is more effective in term of it being more compelling and convincing to the public.

In each round, each LLM character evaluates the messages provided by the players and provides an opinion and trust level score on a 10-point scale:
\begin{itemize}
    \item 10: The character fully trusts the misinformation published by the Influencer (Player 1)
    \item 0: The character completely trusts the debunking response published by the Journalist-debunker (Player 2)
\end{itemize}

%: with higher scores favoring Player 1 and lower scores favoring Player 2. 
The average score across all five personas determines the outcome of the round. 
The final winning condition is determined by the trust level score generated in the last round. Each round's score is influenced by previous rounds, meaning the score achieved after the final round reflects cumulative performance rather than just performance in the final round:
\begin{itemize}
    \item If the final score is above 5, Influencer (Player 1) wins.
    \item If the final score is below 5, Journalist-debunker (Player 2) wins.
    \item If the score equals 5 (indicating neutral public opinion), the player with more remaining in-game currency wins.
\end{itemize}

The game includes a reward system based on the trust score scale, and this serves to maintain player engagement. Players earn in-game currency based on the average public trust score, and can use this to purchase game hints.

%\textcolor{blue}{
%For the game’s winning conditions, inspired by Jaffe et al.’s research on game balancing systems\cite{jaffe2012evaluating}, we employed the basic scale score system, refined through iterative playtesting. We employ a metric to determine which player’s information is more effective, in other words, more compelling and convincing to the public. To achieve this, we prompted the LLM to evaluate and generate a public opinion trust level score, representing the public's trust in the information(see details in Section 3.3). The score is based on a 10-point scale:
%\begin{itemize}
    %\item 10: Public opinion strongly trusts the misinformation published by Influencer (Player 1).
    %\item 0: Public opinion least trusts the misinformation, which means they strongly trust the debunking information published by Journalist-debunker (Player 2).
%\end{itemize}
%The game uses five personas to represent public opinion, and we calculate the average trust level score across all personas to produce a final score for each turn. A higher average score (close to 10) indicates that public opinion agrees more with Influencer (Player 1), while a lower average (close to 0) indicates stronger trust in Journalist-debunker (Player 2). The final winning condition is determined by the trust level score in the last round. Each round’s score is influenced by the scores of all previous rounds, meaning that the final round’s score reflects cumulative performance rather than being solely dependent on the last round:
%\begin{itemize}
    %\item If the final score is above 5, Influencer (Player 1) wins.
    %\item If the final score is below 5, Journalist-debunker (Player 2) wins.
    %\item For rare cases that the score equals 5 (indicating public opinion is neutral), the player with more remaining in-game currency wins.
%\end{itemize}
%The game features a reward system tied to a trust score scale. This scale determines how much in-game currency players earn based on how effectively they influence public opinion.
%Trust scores are categorized into five intervals: 0–2, 2–4, 4–6, 6–8, and 8–10. Each interval corresponds to a different level of in-game currency awarded at the end of a round. For Player 1, the highest reward (1,000 in-game currency) is given for scores in the 8-10 range. The reward decreases as the score falls into lower intervals. For Player 2, the highest reward (also 1,000 in-game currency) is given for scores in the 0-2 range, with the reward decreasing as the score rises into higher intervals.In addtion, player can also using the in-game currency to buy hints.  
%}
%To enhance the players engagement and easier to understand the game, we provide free in-game instructions and optional hints that can be purchased. The game’s currency system is a basic reward system based on the trust score's level intervals, divided into five intervals: 0-2, 2-4, 4-6, 6-8, and 8-10. For Player 1, the highest reward (1,000 in-game currency) is given for scores in the 8-10 range. The reward decreases as the score falls into lower intervals. For Player 2, the highest reward (also 1,000 in-game currency) is given for scores in the 0-2 range, with the reward decreasing as the score rises into higher intervals.
\begin{figure}[htbp]
    \centering
    \includegraphics[width=1\linewidth]{figs/game_interface_RR.jpg}
    \caption{Game Interface. (A) Players can view the current round's news and all previous information published by both players. (B) Players can view LLM-simulated public opinion information. (C) Players can edit their information in the text editing area, view instructions and buy hints. (D) Players can view their own holdings of in-game currency.}
    \label{fig:interface}
\end{figure}

\subsubsection{Game Interface}
%The game's interface is designed to be intuitive and supportive of the player roles, ensuring that the focus remains on information generation. 

The game’s interface is text-based, thus making it easy for players to generate information, with the assistance of references and instructions. The interface is divided into sections:
\begin{itemize}
    \item Information Viewing Section (\autoref{fig:interface}A): This section includes news and all previously published information from both players. It is designed to allow players to quickly review existing information and adjust their in-game strategies accordingly.
    \item Public Opinion Section (\autoref{fig:interface}B): This section displays the LLM-simulated public opinion, represented visually by five circles and an area labeled \textit{Average Public Opinion Score}, which shows the overall level of trust for the information. The data is retrieved from the OpenAI API. Each circle is color-codedby trust score, ranging from blue (0), indicating complete distrust, to red (10), indicating complete trust in Player 1's information. The average public opinion score is calculated from the trust scores of the five characters. Players can click on each circle to view detailed reactions from individual characters.
    \item Text Editing Section (\autoref{fig:interface}C): This section provides tools for text editing, including instructions and purchasable hints that help players in generating their information. These hints vary each round, both in content and cost, and are designed to help players generate more convincing information.
    \item In-Game Currency Panel (\autoref{fig:interface}D): This panel displays the current player’s available in-game currency, helping them to decide whether or not to buy hints.
\end{itemize}

\begin{figure}[htbp]
    \centering
    \includegraphics[width=1\linewidth]{figs/game_prompt_RR.jpg}
    \caption{Prompt and Workflow.(A) The prompt in the game is structured into five sections: Story Context, Role Descriptions and Rules (Criteria), Personas, Response Format and Additional Notes. These five sections form the System message (B). Once the system message is completed, it is applied into the game (C). During gameplay, the LLM generates and simulates public opinions based on the previously established system message. After four rounds, the game ends.}
    \label{fig:prompt}
\end{figure}
\subsection{Prompt Engineering }
\label{Prompt Engineering}
%The configuration of prompt engineering is to prompt the GPT-4o model, helping it understand its tasks based on our game’s context, mechanics and flow. This is critical to more accurately simulate public opinion based on the persona provided, including trust level scores and reactions, upon receiving player input. 

We employed the GPT-4o model to be the LLM-simulated role in the game. The prompt in the game is structured into five sections (\autoref{fig:prompt}A): Story Context, Role Descriptions and Rules (Criteria), Personas, Response Format and Additional Notes. These five sections form the System message (\autoref{fig:prompt}B).
\subsubsection{Story Context}
This section (\autoref{fig:prompt}A-1) serves as the basis of the game. It provides an LLM generated model of the fundamental setting for the "world". This includes background information and each round’s news. The model has to evaluate every player's actions based on this predefined "world" setting. The full example prompt appears below:
\begin{quote}
    \textit{"The Southland, with a 6,000-year history, boasts rich natural resources, diverse ecosystems, and a culture that values liberty, free expression, and media independence. Historically, Southland has been known for its renowned doctors and pharmacists, and its people are proud of their traditional medicine ..."}
\end{quote}

%\textcolor{red}{This section (as shown in Section 1.1 and 1.3 of \autoref{fig:prompt}A) provides the game’s world setting and each round’s background events/news, which also forms the basis for the players' inputs during the game.} %The model's better understanding of the game and the players' input content ensures the generation of relevant and contextually accurate responses.

\subsubsection{Role Description}
The "Role Descriptions and Rules (Criteria)" section (\autoref{fig:prompt}A-2) serves to ensure the model understands its role in the game, i.e., how the game runs. A example prompt is below:
\begin{quote}
    "\textit{Your role is to simulate how the five personas react to each piece of information they receive ...}"
\end{quote}
Additionally, when evaluating and generating public opinions, we provide specific rules and criteria  for the model to follow (\autoref{fig:prompt}A-2.2, 2.3, 2.4, 2.5, 2.6). Specifically, we explain to the model of how to evaluate each player’s inputs separately for Player 1 (\autoref{fig:prompt}A-2.2) and Player 2 (\autoref{fig:prompt}A-2.3), as well as how to evaluate each persona’s reactions to the misinformation (\autoref{fig:prompt}A-2.4) and anti-misinformation (\autoref{fig:prompt}A-2.5) input by each player. Lastly, we provide criteria by which the model provides trust level scores (\autoref{fig:prompt}A-2.6).
%is also critical for generating accurate and relevant responses. 
%We provide two sets of rules, each to evaluate the content generated by Player 1 and Player 2. 

%These rules ensure that the model executes its tasks according to established standards, thereby maintaining consistency and accuracy in its responses.

\subsubsection{Personas}
\label{Personas}

%\textcolor{red}{This section (\autoref{fig:prompt}B) includes three components: detailed information about the personas, the rules for simulating their reactions when confronted with misinformation, and the rules for their reactions when exposed to debunking information.} %These three components form the basis for simulating public opinion within the game. %We leveraged the capabilities of GPT-4o to prompt the model on all personas’ characteristics, such as personality traits, thinking styles, and other relevant factors. This allows the model to fully understand each persona and provide public reactions and trust level scores to the information.
We designed five distinct fictional personas, each with unique demographic characteristics (\autoref{fig:prompt}A-3). These personas were developed based on insights from the literature, incorporating four key group factors: demographics, psychological traits, personality, and behavioral features. These factors guided how each persona responded to misinformation and anti-misinformation messages \cite{nan2022people,liu2023checking,lee2024misinformation,shin2024understanding}. (See detailed prompt in Supplementary materials)
\begin{quote}
    "\textit{Alex Smith. Age: 36. Gender: Male. Project Manager. Education Level: Undergraduate. Political party affiliation: Strongly support Liberal. Enjoys social activities but also values alone time for work and personal projects. Generally trusting and cooperative, but can be skeptical of new information. Balances responsibilities but can be hasty in decision-making. Generally calm and composed, rarely anxious or stressed. Open to new experiences but sometimes prefers convenience over exploration. Busy with work responsibilities and managing projects, often skims through news during short breaks. Follows news via quick-read apps and social media. Often shares articles based on headlines without reading fully. Some critical thinking skills but lacks depth in media literacy.}"
\end{quote}
%Building upon previous work\cite{shin2024understanding}, we utilized GPT-4o to prompt simulations of responses, using the personas' characteristics to generate realistic reactions. This allows the model to fully understand each persona and provide public reactions and trust level scores to the information.

\subsubsection{Response Format}
This section defines the standard response data format to ensure the game can process the data easily after they are received (\autoref{fig:prompt}A-4). When the data are returned to the game, a predefined framework extracts them and assigns them to the relevant game elements for display. This standardized formatting ensures the integration of the API responses into the game, allowing for consistent and accurate gameplay experiences.
\begin{quote}
    "\textit{Persona 1 Basic Details (e.g., age, gender, occupation), Persona Reaction, Trust Level Score, Persona 2 [...], Persona 3 [...], Persona 4 [...], Persona 5 [...]}"
\end{quote}
%\begin{quote}
    %Persona 1 Basic Details (e.g., age, gender, occupation)
    %Persona Reaction
    %Trust Level Score
%\end{quote}
Every time the API receives new information from players, it will return a response in this format, consisting of five personas' public opinions. Each persona's public opinion is composed of the demographic information, a trust level score, and reaction. In the game, these three aspects of data are extracted and displayed (\autoref{fig:response}).

\subsubsection{Additional Notes}
This section is designed for supplementary notes such as edge cases (\autoref{fig:prompt}A-5).
\begin{figure}[H]
    \centering
    \includegraphics[width=0.75\linewidth]{figs/game_response_RR.jpg}
    \caption{Extraction of OpenAI Response Data and Displayed in the Game.}
    \label{fig:response}
\end{figure}

\subsection{Implementation}
\subsubsection{Multiplayer Setup}
To support the multiplayer functionality, we used Photon Unity Networking (PUN). Photon enables real-time multiplayer interactions by providing the network server connections to players, thus creating a shared game state that is synchronized across all clients. The game begins by establishing an exclusive online Photon room (\autoref{fig:system}), where only participants can join and interact. % Photon’s cloud-based architecture ensures low latency, reliable connections, %even when players are geographically away .
In this configuration, critical game variables and data (such as the player’s actions, messages, and game state) are synchronized across both players' screens using Photon’s Remote Procedure Calls (RPCs). This synchronization ensures that any action taken by one player is immediately reflected and displayed on the other player’s screen.%, such as updating the game’s UI elements like input fields, buttons.
\begin{figure}
    \centering
    \includegraphics[width=1\linewidth]{figs/game_system_RR.jpg}
    \caption{Interaction System. Player 1 creates the online game room as the host and Player 2 joins. After Player 1 sends a request to GPT-4o API}, Player 2 can view it. Both players in the online room receive the API response, which is processed and updated on their own game screens. Then Player 2 takes their turn to input information, following the same process as Player 1. At the end of the round, all in-game events (including player inputs, API responses, time, etc.) are logged locally.
    \label{fig:system}
\end{figure}
\subsubsection{Data Storage}
The game employs a logging system to store and manage game data locally on the player's device (\autoref{fig:system}). The log files record various in-game events, including player inputs, API responses, and game state changes. This data is used for analyzing players’ behavior, such as how players interact with the game. To make the data easy to access and ensure compatibility on cross-platform such as Windows and macOS, the log files are saved in a created folder named GameLogs under the players’ commonly used directory. This directory structure is automatically created when the game starts.% ensuring that all log files are organized and easily retrievable.