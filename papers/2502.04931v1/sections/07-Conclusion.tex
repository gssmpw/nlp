Game-based approaches have shown great promise as tools for inoculating individuals against the tactics commonly used to spread misinformation. Most existing games in this domain are single-player games which offer players limited, predefined choices. While this design reduces cognitive load, it often results in interactions which feel less natural and engaging. In response, we designed a two-player, PvP game that pits a misinformation creator against a misinformation stopper. By integrating LLM-powered personas to evaluate player outputs and provide real-time feedback, we created a more open-ended and immersive experience.
We found that the game we developed effectively improved players’ media literacy. Participants demonstrated an enhanced ability to evaluate and analyze media content, identify unreliable or misleading information, and employ effective counter-misinformation strategies. Moreover, the game's engaging mechanics, combined with the competitive element, motivated players to learn from both their own strategies and those of their opponents.
These findings suggest that integrating dynamic feedback systems and competitive gameplay elements into misinformation education games offers a compelling method to deepen users' engagement, while also improving their critical media skills. Future research can build on these insights to explore other forms of interactive learning environments, focusing on diverse player experiences and varying misinformation challenges.