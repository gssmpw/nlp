\section{Real Data}\label{sec:real_data}

In this section, we present two representative examples, the \acrfull{MIMIC-III}\footnote{https://www.kaggle.com/datasets/asjad99/mimiciii} and MovieLens\footnote{https://grouplens.org/datasets/movielens/1m/} dataset, to illustrate the tasks and paradigms discussed in the previous sections. These examples provide a clear and concrete demonstration of how the entire \acrshort{CDM} process can be applied in contexts such as clinical trials, recommendation systems, and beyond. %A detailed analysis of the data is provided in Section \ref{sec:real_data}. %Here, we provide a brief overview of the tasks and paradigms before delving into the methodologies in the subsequent sections.


%It includes 47 features, such as demographics, vital signs, and lab test results, for a cohort of sepsis patients who meet the Sepsis-3 definition criteria. As researchers, it is crucial for us to understand the personalized effects of specific drugs for sepsis, in order to determine the optimal volume of fluids to administer, thereby decreasing mortality rates. In this context, the variable \textit{IV\_input} can be considered as the action $A$, the mortality status \textit{Died\_within\_48H} as the outcome or reward $R$ of interest, and all other variables, without knowing additional information or expert knowledge, can be generally treated as contextual information.

% The first task is CSL, which aims to uncover the causal relationships between variables. This step is crucial for identifying the topological order of variables in a causal diagram, allowing us to pinpoint state variables and mediators that may influence IV input and mortality status. In this context, the SOFA score (Sepsis-related Organ Failure Assessment score), which describes organ dysfunction or failure, serves as a mediator affected by IV input and shown to causally influence patient mortality.

% The second task is CEL. Building on the causal relationships identified in CSL, CEL quantifies the strength of causal links within the diagram. For example, it estimates the average treatment effect to determine how IV input impacts mortality rates across the general patient population and the heterogeneous treatment effect for individual patients. When the SOFA score acts as a mediator, CEL enables us to decompose the total effect into the direct effect (from IV input to mortality) and the indirect effect (through the SOFA score), which is essential for understanding the underlying causal mechanisms.

% The final task is CPL, which seeks to identify the optimal \textit{IV\_input} administration policy to minimize patient mortality rates. Each patient visit is treated as a stage, and depending on the MIMIC-III data structure, various CPL algorithms can be applied to the appropriate paradigm to learn the optimal policy from observational data.



%MIMIC-III is a large, open-access, anonymized, single-center database containing comprehensive clinical data from 61,532 critical care admissions between 2001 and 2012, collected at a Boston teaching hospital. The dataset includes 47 features (such as demographics, vitals, and lab test results) for a cohort of sepsis patients who meet the Sepsis-3 definition criteria. In this example, we will walk through three main tasks: CSL, CEL, and CPL. Our objectives are to:


%First, in CEL, we aim to learn the heterogeneous effect of an action $a$ given any user information $S = s$, represented by $\mathbb{E}[R(a) | S = s]$. CPL then comes into play by identifying the optimal personalized action $a^* = \pi^*(s)$ that maximizes $\mathbb{E}[R(a) | S = s]$. Since each rating is recorded with a precise timestamp, this study can also be framed as an online recommendation system where both exploitation and exploration are considered at each stage of decision-making. As we demonstrate in Section~\ref{sec:real_data}, the offline counterfactual estimation results obtained in CEL provide an informative warm-up, which proves highly beneficial for fostering bandit learning in online testing.


%As researchers, we aim to analyze users' personalized preferences across various genres, enabling us to recommend movies that best match their individual tastes. In this framework, user information (such as age, gender and occupation) is treated as the state information, denoted by $S$. Each movie genre represents an action $a \in \mathcal{A}$ within the action space, and the reward we seek to maximize is the user rating, denoted by $R$.




\subsection{\acrshort{MIMIC-III}}
The \acrshort{MIMIC-III} dataset \citep{johnson2016mimic} is a large, de-identified, and publicly available collection of medical records that contains comprehensive clinical data from patients admitted to the \acrfull{ICU} at a large tertiary care hospital. Of particular interest are the records of patients with sepsis, a life-threatening response to infection caused by harmful microorganisms, which significantly contributes to clinical research and practice. This disease accounts for over 200,000 annual deaths in the U.S. alone, showing the urgent need for effective and timely interventions. Given its treatable nature, sepsis demands prompt emergency care and robust decision-making frameworks to improve patient outcomes and reduce mortality.

However, the \acrfull{EHR} collected from patients with sepsis in ICUs present significant challenges to the application of existing decision-making methods. Specifically, records from hundreds of thousands of sepsis patients treated at the Beth Israel Deaconess Medical Center between 2001 and 2012 include numerous covariates such as demographics, vital signs, medical interventions, lab test results, and post-treatment outcomes. For researchers, it is essential to disentangle the complex and intricate causal relationships among these variables, understand the impacts of specific sepsis treatments, and select a sufficient and necessary set of variables to analyze the disease. This comprehensive causal understanding is critical to optimizing treatments and ultimately reducing the mortality associated with sepsis. 

To overcome these challenges, we utilize the three main tasks aforementioned and demonstrate the causal decision-making process using \acrshort{MIMIC-III} as an example. The first step \acrshort{CSL}, which aims to uncover the causal relationships between variables, allows us to pinpoint the right and informative set of state variables, the treatments, and mediators that may influence the mortality due to sepsis. In this context, we identify the IV-Input as an actionable treatment with the \acrfull{SOFA} score serves as an important mediator, which is known to describe organ dysfunction or failure. %, serves as an important mediator. % affected by IV input, and is shown to causally influence patient mortality.
The second step \acrshort{CEL}, building upon the causal relationships identified in \acrshort{CSL}, quantifies the strength of causal links and thus measures the most effective treatments and informative mediators. For example, it estimates the average treatment effect to determine how intravenous fluid input (IV-Input) impacts mortality rates across the general patient population and the heterogeneous treatment effect for individual patients. %When the SOFA score acts as a mediator, CEL enables us to decompose the total effect into the direct effect (from IV input to mortality) and the indirect effect (through the SOFA score), which is essential for understanding the underlying causal mechanisms.
The final step \acrshort{CPL} seeks to determine the optimal administration policy to minimize patient mortality rates. Each patient visit is treated as a stage, and depending on the \acrshort{MIMIC-III} data structure, various \acrshort{CPL} algorithms can be applied to the appropriate paradigm to learn the optimal policy from observational data.




%\acrshort{MIMIC-III} is a large open-access anonymized single-center database which consists of comprehensive clinical data of $61,532$ critical care admissions from 2001–2012 collected at a Boston teaching hospital. 
Due to privacy concerns, we utilized a subset of the original data that is publicly available on Kaggle. For illustration purposes, we selected several representative features for the following analysis, as listed in Table \ref{table:MIMIC3}.
\begin{table}[ht]
\centering
\begin{tabular}{l|p{12cm}}
\hline
\textbf{Variable} & \textbf{Description} \\
\hline
Glucose & Glucose values of patients \\
%PaO2 & The partial pressure of oxygen \\
PaO2\_FiO2 & The partial pressure of oxygen (PaO2) divided by the fraction of oxygen delivered (FIO2) \\
\acrshort{SOFA} & Sepsis-related Organ Failure Assessment score to describe organ dysfunction/failure \\
IV-Input & The volume of fluids that have been administered to the patient \\
Died\_within\_48H & The mortality status of the patient 48 hours after being administered \\
\hline
\end{tabular}
\caption{Description of variables used in \acrshort{MIMIC-III} data analysis}\label{table:MIMIC3}
\end{table}

In the next sections, we will start from \acrshort{CSL} to learn a significant causal diagram from the data, and then quantify the effect of treatment (IV-input) on the outcome (mortality status, denoted by Died\_within\_48H variable in the data) through \acrshort{CEL}. Finally, we use \acrshort{CPL} to find the best-individualized treatment rules under different settings.

\subsubsection{\acrshort{CSL} on \acrshort{MIMIC-III}}

For our analysis of the \acrshort{MIMIC-III} dataset, we employ the score-based method in \acrshort{CSL} to estimate underlying causal relationships among several key clinical features. The \acrshort{MIMIC-III} dataset comprises a comprehensive range of clinical data from a large cohort of \acrshort{ICU} patients. For this analysis, we selected a subset of variables including Glucose levels, PaO2/FiO2 ratio, \acrshort{SOFA} scores, IV-Input, and mortality within 48 hours. The details of selected features are pivotal in understanding patient outcomes in intensive care, and their descriptions are provided in Table \ref{table:MIMIC3}. Specifically, we utilize the NOTEARS algorithm \citep{zheng2018dags}, which is designed to learn a \acrshort{DAG} without cycle constraints from continuous data, given the complexity of the observed data. %This algorithm optimizes a smooth loss function with a continuous constraint that the learned graph must be acyclic, effectively combining gradient descent with a feasibility check for acyclicity.

\begin{figure}
    \centering
    \includegraphics[width=0.5\linewidth]{Figure/MIMIC3.png}
    \caption{Estimated directed acyclic graph for the selected \acrshort{MIMIC-III} data analysis.}
    \label{fig:DAG_mimic}
\end{figure}

The resulting \acrshort{DAG} from the NOTEARS algorithm is presented in Figure \ref{fig:DAG_mimic}, which reveals a plausible causal structure among the variables. In particular, PaO2/FiO2, a measure of lung efficiency, is identified as an exogenous variable that influences other downstream variables but is not influenced by any of the variables selected in this analysis. Glucose levels and IV-Input appear causally prior to other variables, suggesting their role in early medical intervention. \acrshort{SOFA} score, a critical measure of organ failure, is influenced by both glucose levels and IV-Input, and it further influences the mortality outcome. 
The mortality within 48 hours variable is positioned as an end-point in the causal chain, influenced by the \acrshort{SOFA} score.

The learned causal graph highlights several clinically relevant pathways. Notably, the direct influence of IV-Input on \acrshort{SOFA} score and mortality shows the impact of fluid management in critical care settings. Additionally, the model suggests a pathway from metabolic control (Glucose) through organ function (\acrshort{SOFA} score) to mortality outcomes. These findings could inform targeted interventions aiming to optimize patient care and improve survival outcomes in the \acrshort{ICU}.

\subsubsection{\acrshort{CEL} on \acrshort{MIMIC-III}}
After establishing the causal diagram between the relevant variables, the next step is to quantify the causal effect of IV-Input on reducing the mortality rate of patients. For simplicity, we categorize the treatment, specifically IV-Input, into a binary action space: $A_i=1$ represents the ``High-IV-Input'' group with IV-Input volume greater than $1$, and $A_i=0$ corresponds to the ``Low-IV-Input'' group with IV-Input less than or equal to $1$. Our goal is to apply \acrshort{CEL} methods to estimate the ATE, quantifying the impact as $\mathbb{E}\{R(1) - R(0)\}$.

In the \acrshort{CSL} analysis, we examined the causal relationships between the variables listed in Table \ref{table:MIMIC3} and identified a mediator, the \acrshort{SOFA} score, which is influenced by the treatment (IV-Input) and subsequently impacts the mortality status of patients within the next 48 hours. Utilizing the direct estimator proposed by \citet{robins1992identifiability}, the IPW estimator from \citet{hong2010ratio}, and the robust estimator introduced by \citet{tchetgen2012semiparametric}, we evaluated the natural direct and indirect effects of the treatment regime based on observational data. The results are summarized in Table \ref{table:mediation_MIMIC3} below.

\begin{table}[ht]
\centering
\begin{tabular}{l|ccc}
\hline
\textbf{Methods}        & \textbf{DE}  & \textbf{IE}  & \textbf{TE}    \\ \hline
{Direct Estimator} & -0.2133 & 0.0030  & -0.2104 \\ \hline
{Inverse Probability Weighting}     & -0.2332 & 0       & -0.2332 \\ \hline
{Doubly Robust}  & -0.2276 & -0.0164 & -0.2440 \\ \hline
\end{tabular}
\caption{Comparison of DE, IE, and TE across different estimation methods.}
\label{table:mediation_MIMIC3}
\end{table}
Specifically, compared to the lower IV-Input group, constantly administering a high volume of IV-Input shows a negative impact on survival rates, with an estimated $20\%-25\%$ increase in mortality. This effect is largely driven by the direct influence of the treatment on the final outcome. While this result may seem counterintuitive at first, it could be partly attributed to the general overuse of IV-Input in medical care. 

Therefore, developing a personalized treatment regime to administer the optimal volume of IV-Input tailored to individual patient characteristics is crucial to avoid overuse and meet specific treatment needs. This challenge motivates our exploration of offline policy learning in the next section.

%A closer look at the individualized treatment effects, as analyzed by the HTE learners in Section \ref{sec:CEL_HTE_p1}, reveals that most patients—who are not severely affected by sepsis—tend to experience a slight decrease in mortality after 48 hours when given sufficient IV fluids. However, we observe a significant increase in mortality among a few severe patients, who genuinely benefit from IV administration. This highlights the ongoing challenge of 


\subsubsection{Offline \acrshort{CPL} on \acrshort{MIMIC-III}}
In this section, we demonstrate the results when applying classic offline \acrshort{CPL} methods to this dataset. 
The simplest modeling usually starts from paradigm 1, where we aggregated the observations for each patient and used the averaged observations as the dataset for the analysis.

As an example, we use the Q-learning algorithm to evaluate two simple treatment rules based on the observed data. 
We specify the following linear model as our Q-function: 
$$
\begin{aligned}
Q(s, a, \beta) &= \beta_{00} + \beta_{01} \cdot \text{Glucose} + \beta_{02} \cdot \text{PaO2\_FiO2} \\ \qquad\qquad\qquad
&+ I(a_1 = 1) \cdot \left(\beta_{10} + \beta_{11} \cdot \text{Glucose} + \beta_{12} \cdot \text{PaO2\_FiO2}\right)
\end{aligned}
$$
We evaluate two target policies. The first is a fixed treatment regime that does not apply treatment, for which Q-learning has an estimated value of $.99$. Another is a fixed treatment regime that applies treatment all the time, with an estimated value of $.76$. Therefore, the result implies that a high dose of IV-Input naively always is worse and increases the mortality rate, aligned with the \acrshort{CEL} results. 

We take one step further to find an optimal policy maximizing the expected value. We use the Q-learning algorithm again to do policy optimization. 
Using the regression model we specified above, the estimated optimal policy is to recommend $A = 0 \ (\text{IV-Input} = 0) \text{ if } -0.0003 \cdot \text{Glucose} + 0.0012 \cdot \text{PaO2\_FiO2} < 0.5633$ and $A = 1 \ (\text{IV-Input} = 1)$ otherwise. 
When applying the estimated optimal treatment regime to individuals in the observed data, IV-Input would be administered to 6 out of the 57 patients.


Based on the domain knowledge, it is usually more plausible to believe the outcome of a patient depends only on his treatment and condition in the past stage. Therefore, we can apply a 3-stage Q-learning in Paradigm 3 to learn the policy. The Q-function is specified as a linear one that considers all the previous stages' states and actions. The learned optimal policy is as follows. 
\begin{itemize}
    \item \textbf{Stage 1:} recommend \( A = 0 \) (IV-Input = 0) if 
        $ 0.0001 \cdot \text{Glucose}_1 + 0.0012 \cdot \text{PaO2\_FiO2}_1 > 0.0551$ and \( A = 1 (\text{IV-Input} = 1)\)  otherwise.     
    \item \textbf{Stage 2:} recommend \( A = 0 \) (IV-Input = 0) if 
        $ 0.0002 \cdot \text{Glucose}_2 - 0.00001 \cdot \text{PaO2\_FiO2}_2 + 0.0070 \cdot \text{\acrshort{SOFA}}_1 < 0.0721 $ and \( A = 1 \) (IV-Input = 1)  otherwise.        
    \item \textbf{Stage 3:} recommend \( A = 0 \) (IV-Input = 0) if 
        $ -0.0005 \cdot \text{Glucose}_2 + 0.0008 \cdot \text{PaO2\_FiO2}_2 - 0.0114 \cdot \text{\acrshort{SOFA}}_2 < 0.2068$ and recommend \( A = 1 \) (IV-Input = 1)  otherwise.    
\end{itemize}

Applying the estimated optimal regime to individuals in the observed data, we can have personalized treatment plans. 
For example, 23 patients will receive IV-Input in the first two stages and no inputs in the last one; while 10 other patients will receive IV-Input only in the first phase. 



% summarize the regime pattern for each patients in the following table:

% The estimated value of the estimated optimal policy is .9999.







\subsection{MovieLens}

Recommender systems play a crucial role in personalizing user experiences across various industries. A common example is movie recommendation, where understanding user preferences across different movie genres is essential. However, this task is challenging due to the inherent difficulty of estimating counterfactuals, i.e. how users would have responded if presented with different options. To illustrate how recommender systems can be optimized through causal decision-making, we use movie recommendations as an example, starting with the well-known Movielens dataset.


The MovieLens 1M dataset, derived from an online movie recommendation experiment, is a widely used benchmark for generating data in online bandit simulation studies. User information in this dataset is categorized by age, gender, and occupation. For simplicity, we focus on the top five movie genres in the dataset (Comedy $a=0$, Drama $a=1$, Action $a=2$, Thriller $a=3$, Sci-Fi $a=4$), and analyze users from the top five occupations. The realized reward $R$ is a numerical variable ranging from $\{1,2,3,4,5\}$, with $1$ indicating the lowest level of satisfaction and $5$ representing the highest. As the causal structure of this problem has been well defined, with movie genre as the action and users' rating as the reward, our objectives are two-fold, mainly focusing on CEL and CPL.

First, we begin the process with \acrshort{CEL}, where scientists analyze the logged data to identify general patterns of user preferences. Specifically, the \acrshort{ATE} of one movie genre relative to another is calculated to reveal overarching trends across the user population, and \acrshort{HTE}s are estimated to capture variations in preferences across different user segments, providing a more granular understanding of how different groups respond to various types of content. These insights lay the groundwork for \acrshort{CPL}.

Second, given the dynamic nature of user preferences and frequent interactions between users and the system, movie recommendation is typically approached as an online \acrshort{CPL} problem. The primary challenge is balancing the exploitation of existing knowledge about user preferences with the need to explore new data to improve counterfactual estimations. We will detail in this section that, offline counterfactual estimates obtained through \acrshort{CEL} offer valuable guidance for managing the exploration-exploitation trade-off in the early stages of online policy learning by informing data collection strategies. By further employing diverse online \acrshort{CPL} methodologies, the recommender system can dynamically adapt and refine its recommendations, leading to a more personalized and optimal user experience.

To simulate a real-world recommender system, we randomly sampled 1\% of the dataset to serve as the logged data currently available for offline analysis, while using the entire dataset to estimate the true reward distribution, which will be used to simulate the observed rewards during online interactions.

%Our objective is to first apply \acrshort{CEL} to learn the reward distribution across different movie genres, followed by the use of online \acrshort{CPL} (bandits) to recommend the optimal genres to maximize cumulative user satisfaction. 


\subsubsection{\acrshort{CEL} on MovieLens}
In \acrshort{CEL}, we aim to estimate the potential outcomes (i.e. the expected ratings) of users across different movie genres. %To do so, we sample $1\%$ of the data associated with the top five movie genres. 
Using the T-learner approach, we fit a separate regression model for each genre (arm) to estimate the expected rating for each individual user. Table \ref{tab:CEL_movielens} below provides a summary of the expected ratings for two subgroups, Female and Male, across these different movie genres.
\begin{table}[h!]
\centering
\begin{tabular}{c|c|c}
\hline
\textbf{Genre} & {Expected Rating (Female)} & {Expected Rating (Male)} \\ \hline
        Comedy  & 3.580 & 3.445 \\
        Drama   & 3.403 & 3.424 \\
        Action  & 3.282 & 3.073 \\
        Thriller & 3.512 & 3.236 \\
        Sci-Fi  & 3.082 & 2.958 \\ \hline
\end{tabular}
\caption{Expected ratings of movie genres for different gender group}\label{tab:CEL_movielens}
\end{table}

In table \ref{tab:CEL_movielens}, we can see that except for Drama, females tend to provide higher expected ratings compared to males across the various movie genres. Depending on the researcher's objectives, \acrshort{CEL} approaches, including both ATE and HTE estimators provided in Section \ref{sec:CEL_p1}, offer multiple avenues for understanding individual preferences in movie genres.


\subsubsection{\acrshort{CPL} on MovieLens}
\begin{figure}
    \centering
    \includegraphics[width=.8\linewidth]{Figure/MovieLens.pdf}
    \caption{Simulation results for movie recommendation. Shaded areas indicate the 95\% confidence interval of the averages over 50 replicates.}
    \label{Fig:MovieLens_result}
\end{figure}


Movie recommendation has been extensively studied as an online bandit problem (paradigm 5), with its continuous feasibility of data collection. In this section, we simulate a real-world movie recommendation system using the full MovieLens dataset to demonstrate the necessity of online learning in decision-making and to further illustrate how insights from \acrshort{CEL} can be applied to \acrshort{CPL}. As an example, we implement the Linear Thompson Sampling (LinTS) algorithm to learn the optimal policy online. Specifically, we assume that for each arm, $R_t(a)\sim\mathcal{N}(s_a^T\gamma, \sigma^2)$, where $s_a$ is a vector contains feature information for the movie genre $a$, $\gamma$ is a vector of parameters, and $\sigma^2$ is the variance of the random noise. Using the 1\% logged data and the estimates from the \acrshort{CEL} step, we first estimated $\sigma$ and $\gamma$. These estimates were then used to construct an informative LinTS, with $\mathcal{N}(\hat{\boldsymbol{\gamma}}, 0.05\boldsymbol{I})$ as the prior distribution for $\gamma$, where $\boldsymbol{I}$ is an identity matrix and 0.05 is the prior variance, selected to reflect a reasonable confidence in the estimated $\hat{\boldsymbol{\gamma}}$ from the logged data while acknowledging the remaining uncertainty that requires further exploration. 

To highlight the advantages of incorporating information from the \acrshort{CEL} step, we also implement an uninformative LinTS using $\mathcal{N}(\boldsymbol{0},1000\boldsymbol{I})$ as the prior for $\gamma$. Additionally, in \acrshort{CEL}, we observed that both females and males, on average, prefer thriller movies with the highest expected ratings. Based on this insight, a simple greedy algorithm recommending thrillers to all users is considered as a baseline. We also implement a personalized greedy algorithm that generates tailored recommendations derived from more granular, individual-level estimates produced by \acrshort{CEL}. The simulation results are presented in Figure \ref{Fig:MovieLens_result}.

Overall, as expected, the naive greedy algorithm, which ignores personalized preferences, performs the worst. While the personalized greedy algorithm outperforms the LinTS algorithms in the early stages due to less random exploration, both LinTS algorithms continue to gather new information from the environment and eventually surpass the performance of the personalized greedy algorithm. Furthermore, when comparing the uninformative LinTS to the informative LinTS, it is evident that the latter performs better—especially in the early stages—thanks to the prior knowledge acquired from the logged data and the \acrshort{CEL} step.

