%\section{\acrshort{CSL}/\acrshort{CEL} Assists \acrshort{CPL} for Other Desiderata}
\section{Conclusion and Discussion}\label{sec:conclusion}

Causality seeks to explain \textit{how actions lead to effects}, while decision-making focuses on \textit{how to take actions that yield the greatest effects}. In this paper, we present a comprehensive framework for decision-making through a causal lens. We decompose \acrshort{CDM} into three key tasks (\acrshort{CSL}, \acrshort{CEL}, \acrshort{CPL}) and six paradigms (distinguished by differences in data structure and offline/online learning settings), with each accompanied by a detailed review of state-of-the-art methods. We take an affirmative step in highlighting the widespread use of causality in decision-making by integrating all three tasks into a unified framework (see Section \ref{Sec:CSL}-\ref{sec:Online CPL}), with an extra emphasis on the assumption violated scenarios (see Section \ref{sec:assump_violated}). %We hope this review helps to bridge the gap between causal inference and classical RL's perspective in decision making. 
%In this paper, we have explored multiple dimensions illustrating why causality and decision-making are inseparable: the general need for structure learning through causal discovery, the role of causal effect estimation in providing foundational insights for decision-making, as well as the overarching causal perspective that has been widely presented in both offline and online policy learning.
To provide a hands-on tutorial, we developed a \href{https://causaldm.github.io/Causal-Decision-Making}{GitHub notebook} with a Python package that summarizes popular methods for each task (\acrshort{CSL}, \acrshort{CEL}, \acrshort{CPL}), which are widely used in real-world applications. Combined with the real-data applications discussed in Section \ref{sec:real_data}, we believe that this paper offers a comprehensive tutorial for practitioners interested in the intersection of causality and decision-making. 


Several intriguing extensions to classical decision making methods have been, and continue to be, actively explored. These include scenarios where we extend beyond the objective of reward maximization. For instance, recent studies have increasingly investigated how \acrshort{CSL} and \acrshort{CEL} techniques can enhance decision-making by incorporating additional objectives such as \textbf{fairness} and \textbf{explainability}. %, paving the way for more ethical and interpretable frameworks.
%Beyond facilitating more efficient learning and addressing assumption violations, recent studies have explored how \acrshort{CSL} and \acrshort{CEL} techniques can enhance decision-making by considering objectives beyond reward optimization, such as fairness and explainability, paving the way for more ethical and interpretable frameworks. 
Motivated by the raising awareness of potential discrimination issues, which is essential in building a trustworthy recommendation system, \acrshort{SCM} is widely utilized to help understand the \textbf{fairness} issue. % in decision making. 
\citet{zhang2018fairness} decomposes the effect of a natural variation of a feature, and adopts the \acrshort{SCM} to infer and distinguish different types of natural discriminations; \citet{huang2022achieving} evaluates the counterfactual effect of sensitive attributes on the reward and limits the action space to arms satisfying the counterfactual fairness constraints; and \citet{balakrishnan2022scales} defines a \acrfull{PCE} to quantify the causal effect of protected attribute on reward through a specific path and formulates the fairness-aware recommendation problem as a constrained \acrshort{MDP} problem. Causal knowledge is also useful in enhancing the \textbf{explanability} of decision making. \citet{madumal2020explainable} introduced an action influence model that captures the causal relationships between variables using structural equations. By continuously learning the \acrshort{SCM}, they provide insights into the behavior of \acrshort{RL} agents by generating explanations for ``why A" and ``why not A" questions through counterfactual reasoning based on the learned \acrshort{SCM}. Instead of focusing on explaining a single action choice, \citet{tsirtsis2021counterfactual} aims to explain an observed sequence of multiple, interdependent actions. In scenarios involving multiple agents or more complex environments, through counterfactual reasoning using \acrshort{SCM}, \citet{triantafyllou2022actual} investigated multi-agent \acrshort{RL} to disentangle the contributions of individual agents, while \citet{mesnard2020counterfactual} differentiated the effect of an action from that of external factors on future rewards. These interconnected topics not only highlight their synergy within causal decision-making, but also pave the way for exciting future research directions.



%Causality seeks to explain \textit{how actions lead to effects}, while decision-making focuses on \textit{how to take actions that yield the greatest effects}. As such, causality serves as an indispensable guide for decision-making, providing deeper insights into the underlying relationships from a causal perspective. In this paper, we have explored multiple dimensions illustrating why causality and decision-making are inseparable: the general need for structure learning through causal discovery, the role of causal effect estimation in providing foundational insights for decision-making, as well as the overarching causal perspective that has been widely presented in both offline and online policy learning. These interconnected concepts not only highlight their synergy but also pave the way for exciting future research directions in \acrshort{CDM}.  %We hope this review helps to bridge the gap between causal inference and classical RL's perspective in decision making. 
















%For example, motivated by the raising awareness of potential discrimination issues, which is essensial in building a trustworthy recommendation system, \acrshort{SCM} is widely utilized to help understand the \textbf{fairness} in decision making. \citet{zhang2018fairness} decomposes the effect of a natural variation of a feature on the outcome into couterfatual direct, indirect and spurious effects, and adopts the \acrshort{SCM} to quatitatively evaluate each to infer and distinguish three different types of natural discriminations. To further establish a fairness-aware recommendation system, \citet{huang2022achieving} incorporates causal inference into bandits by evaluating the counterfactual effect of sensitive attributes on the reward and limiting the action space to arms satisfying the couterfactual fariness constriant. Similarly, \citet{balakrishnan2022scales} defines a \acrfull{PCE} to quantify the causal effect of protected attribute on reward through a specific path and incorporates all \acrshort{PCE} of interest into the objective function as conatriants with corresponding threshold, and hence formulate the fairness-aware recommendation pproblem as a constraind \acrshort{MDP} problem. 

%Causal knowledge is also useful in enhancing the \textbf{explanability} of decision making. Assuming a known \acrshort{DAG} among state variables and actions of model-free \acrshort{RL} agents, \citet{madumal2020explainable} introduced an action influence model that captures the causal relationships between variables using structural equations. By continuously learning the \acrshort{SCM} during the reinforcement learning process, they provide insights into the behavior of \acrshort{RL} agents by generating explanations for ``why A" and ``why not A" questions through counterfactual reasoning based on the learned \acrshort{SCM}. Instead of focusing on explaining a single action choice, \citet{tsirtsis2021counterfactual} aim to explain an observed sequence of multiple, interdependent actions. Specifically, they provide causal explanations by identifying an alternative action sequence that deviates from the observed one by at most a specified number of steps and could potentially lead to a better outcome. Building upon the \acrshort{MDP} and the Gumbel-Max \acrshort{SCM} \citep{oberst2019counterfactual}, they proposed a polynomial time dynamic programming algorithm to find such an optimal conterfactual sequence. In scenarios involving multiple agents or more complex environments, it is crucial to understand how each action contributes to the outcome of interest, addressing what is known as the credit assignment problem. Through counterfactual reasoning using \acrshort{SCM}, \citet{triantafyllou2022actual} investigated multi-agent \acrshort{RL} to disentangle the contributions of individual agents, while \citet{mesnard2020counterfactual} differentiated the effect of an action from that of external factors on future rewards.


 