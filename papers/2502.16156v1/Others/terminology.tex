The Relative Strength Index (RSI) is a technical analysis indicator that measures the magnitude of recent price changes to assess overbought or oversold conditions in a security or other asset. It is commonly used to identify potential reversal points in the price of an asset and to help traders make buy or sell decisions.

The RSI is calculated using the average of the gains and losses of an asset over a given period of time. Specifically, the RSI is calculated as the ratio of the average gain to the average loss over the specified time period, multiplied by 100. A high RSI value (above 70) indicates that an asset may be overbought, while a low RSI value (below 30) indicates that it may be oversold.

The RSI is often used in conjunction with other technical analysis indicators, such as moving averages, to provide a more comprehensive view of an asset's price trends and potential entry or exit points for trades. It is also widely used in algorithmic trading systems to help automate the decision-making process.

