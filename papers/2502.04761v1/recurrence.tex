\section{Implementing $\tip$ for Linear Integer Arithmetic}
\label{sec:rec}

We now explain how to compute transitive projections for quantifier-free linear integer arithmetic via recurrence analysis.
%
As in SMT-LIB \cite{smtlib}, in our setting linear integer arithmetic also features (in)divisibility predicates of the form $e|t$ (or $e\!{\not|}t$) where $e \in \NN_+$ and $t$ is an integer-valued term.
%
Then we have $\sigma \models e|t$ iff $\sigma(t)$ is a multiple of $e$, and $\sigma \models e\!{\not|}t$, otherwise.

The technique that we use is inspired by the recurrence analysis from \cite{kincaid15}.
%
However, there are some important differences.
%
The approach from \cite{kincaid15} computes convex hulls to over-approximate disjunctions by conjunctions, and it relies on polyhedral projections.
%
In our setting, we always have a suitable model at hand, so that we can use $\mbip$ instead.
%
Hence, our recurrence analysis can be implemented more efficiently\footnote{The \emph{double description method}, which is popular for computing polyhedral projections and convex hulls, and other state-of-the-art approaches have exponential complexity \cite{dd-exp,fmplex}.
  %
  See \cite{convex-hull} for an easily accessible discussion of the complexity of the double description method.
  %
  In contrast, combining the model based projection from \cite{spacer} with syntactic implicant projection \cite{adcl} yields a polynomial time algorithm for $\mbip$.}.
%
Additionally, our recurrence analysis can handle divisibility predicates, which are not covered in \cite{kincaid15}.

On the other hand, \cite{kincaid15} yields an over-approximation of the transitive closure of the given relation, whereas our approach performs an implicit case analysis (via $\mbip$) and only yields an over-approximation of the transitive closure of one out of finitely many cases.

Moreover, the recurrence analysis from \cite{kincaid15} also discovers non-linear relations, and then uses linearization techniques to eliminate them.
%
For simplicity, our recurrence analysis only derives linear relations so far.
%
However, just like \cite{kincaid15}, we could also derive non-linear relations and linearize them afterwards.
%
Apart from these differences, our technique is analogous to \cite{kincaid15}.

In the sequel, let $\tau$ and $\sigma \models \tau$ be fixed.
%
Our implementation of $\tip(\tau,\sigma)$ first searches for \emph{recurrent literals}, i.e., literals of the form\footnote{W.l.o.g., we assume that literals are never negated, as we can negate the corresponding (in)equalities or divisibility predicates directly instead.
  %
  Furthermore, in our implementation, we replace disequalities $s \not\doteq t$ with $s >
  t \lor s < t$ and eliminate the resulting disjunction via $\mbip$ to obtain a transition
  without disequalities.}
\[
  t \bowtie 0 \text{ or } e|t \quad \text{where} \quad t = \sum_{x \in \vec{x}} c_x
  \cdot (x'-x) + c, \; {\bowtie} \in \{\leq,\geq,<,>,\doteq\},
  \text{ and } c_x,c \in \ZZ.
\]
Hence, these literals provide information about the change of values of variables.
%
To find such literals, we introduce a fresh variable $x_\delta$ for each $x \in \vec{x}$, and we conjoin $x_\delta \doteq x' - x$ to $\tau$, i.e., we compute
\[
  \tau_{\land \delta} \Def \tau \land \bigwedge_{x \in \vec{x}} x_\delta \doteq x' - x.
\]
%
So the value of $x_\delta$ corresponds to the change of $x$ when applying $\tau$.
%
Next, we use $\mbip$ to eliminate all variables but $\{x_\delta \mid x \in \vec{x}\}$ from $\tau_{\land\delta}$, resulting in $\tau_\delta$.
%
More precisely, we have
\[
  \tau_\delta \Def \mbp(\tau_{\land \delta}, \quad \sigma \uplus [x_\delta/\sigma(x'-x) \mid x \in \vec{x}], \quad \{x_\delta \mid x \in \vec{x}\}).
\]
Finally, to obtain a formula where all literals are recurrent, we replace each $x_\delta$ by its definition, i.e., we compute
\[
  \tau_\rec \Def \tau_\delta[x_\delta / x' - x \mid x \in \vec{x}].
\]
%
\begin{example}[Finding Recurrent Literals]
  \label{ex:finding-rec}
  Consider the transition $\tau_\dec$.
  %
  We first construct the formula
  \[
    \tau_{\land \delta} \Def \tau_\dec \land w_\delta \doteq w' - w \land x_\delta \doteq x' - x \land y_\delta \doteq y' - y.
  \]
  Then for any model $\sigma \models \tau_\dec$, we get\footnote{In the case of $\tau_\dec$, we obtain the same formula $\tau_\delta$ for \emph{every} model $\sigma \models \tau_\dec$, as variables can simply be eliminated by propagating equalities.}
  \begin{align*}
    \tau_\delta \Def {} & \mbp(\tau_{\land \delta}, \quad \sigma \uplus [w_\delta/0, \; x_\delta/{-}1, \; y_\delta/{-}1], \quad \{w_\delta, x_\delta, y_\delta\}) \\
                {} = {} & w_\delta \doteq 0 \land x_\delta \doteq -1 \land y_\delta \doteq -1.
  \end{align*}
  Next, replacing $w_\delta,x_\delta$, and $y_\delta$ with their definition results in
  \begin{align*}
    \tau_\rec \Def {} & w' - w \doteq 0 \land x' - x \doteq -1 \land y' - y \doteq -1        \\
    \equiv {}         & w' - w \doteq 0 \land x' - x + 1 \doteq 0 \land y' - y + 1 \doteq 0.
  \end{align*}
\end{example}
%
Then the construction of $\tip(\tau,\sigma)$ proceeds as follows:
%
\begin{itemize}
  \item $\tip(\tau,\sigma)$ contains the literal $n > 0$, where $n \in \VV$ is a fresh extra variable
  \item for each literal $\sum_{x \in \vec{x}} c_x \cdot (x'-x) + c \bowtie 0$ of $\tau_\rec$, $\tip(\tau,\sigma)$ contains the literal $\sum_{x \in \vec{x}} c_x \cdot (x'-x) + n \cdot c \bowtie 0$
  \item for each literal $e|\sum_{x \in \vec{x}} c_x \cdot (x'-x) + c$ of $\tau_\rec$, $\tip(\tau,\sigma)$ contains the literal $e|\sum_{x \in \vec{x}} c_x \cdot (x'-x) + n \cdot c$
\end{itemize}
%
Intuitively, the extra variable $n$ can be thought of as a ``loop counter'', i.e., when
$n$ is instantiated with some constant $k$, then the literals above approximate the change
of variables when $\to_\tau$ is applied $k$ times.

\begin{example}[Computing $\tip$ (1)]
  \label{ex:tip1}
  Continuing \Cref{ex:finding-rec}, $\tip(\tau_\dec,\sigma)$ contains the literals
  \begin{align*}
    {}           & n > 0 \land w' - w + n \cdot 0 \doteq 0 \land x' - x + n \cdot 1 \doteq 0 \land y' - y + n \cdot 1 \doteq 0 \\
    {} \equiv {} & n > 0 \land w' \doteq w \land x' \doteq x - n \land y' \doteq y - n
  \end{align*}
  for any model $\sigma \models \tau_\dec$.
  %
  Note that in our example, this formula precisely characterizes the change of the variables after $n$ iterations of $\to_{\tau_\dec}$.
  %
  To simplify the formula above, we can propagate the equality $n = x - x'$, resulting in:
  \begin{align}
    & x - x' > 0 \land w' \doteq w \land y' \doteq y - x + x' \notag \\
    {} \equiv {} & w' \doteq w \land x' < x \land x' - x \doteq y' - y \label{eq:rel}
  \end{align}
\end{example}
%
Compared to $\tau_\dec^+$, \eqref{eq:rel} lacks the literal $w \doteq 1$.
%
To incorporate information about the pre- and post-variables (but not about their relation) we conjoin $\mbp(\tau,\sigma,\vec{x})$ and $\mbp(\tau,\sigma,\vec{x}')$ to $\tip(\tau,\sigma)$.

\begin{example}[Computing $\tip$ (2)]
  We finish \Cref{ex:tip1} by conjoining
  \[
    \mbp(\tau_\dec,\sigma,\{w,x,y\}) = w \doteq 1 \qquad \text{and} \qquad \mbp(\tau_\dec,\sigma,\{w',x',y'\}) = w' \doteq 1
  \]
  to \eqref{eq:rel}, resulting in:
  \[
    w' \doteq w \land x' < x \land x' - x \doteq y' - y \land w \doteq 1 \land w' \doteq 1 \quad \equiv \quad \tau_\dec^+
  \]
\end{example}

\begin{example}[Divisibility]
  To see how $\tip$ can handle divisibility constraints, consider the transition
  \[
    \tau \Def 2|x \land 3|x' - x + 1.
  \]
  Then our approach identifies the recurrent literal $\tau_\rec = 3|x' - x + 1$, so that $\tip(\tau,\sigma)$ contains the literal $3|x' - x + n$.
  %
  To see why we conjoin this literal to $\tip(\tau,\sigma)$, note that $3|x'-x+1$ and
  $3|x'-x+n$ are equivalent to $x'-x + 1 \equiv_3 0$ and $x'-x + n \equiv_3 0$, respectively, where ``$\equiv_3$'' denotes congruence modulo $3$.
  %
  So $e \to^n_{\tau} e'$ implies $e'-e+n \equiv_3 0$, just like $e \to^n_{\pi} e'$ implies
  $e'-e+n=0$ for $\pi \Def x'-x+1 \doteq 0$.
  Moreover, we have $\mbp(\tau,\sigma,\vec{x}) = 2|x$, and thus
  \[
    \tip(\tau,\sigma) = n>0 \land 3|x'-x+n \land 2|x.
  \]
\end{example}

\newcounter{tip}
\setcounter{tip}{\value{theorem}}
\begin{theorem}[Correctness of $\tip$]
  \label{thm:tip}
  The function $\tip$ as defined above is a transitive projection.
\end{theorem}
\makeproof*{thm:tip}{
  \setcounter{theorem}{\thetip}
  \begin{theorem}[Correctness of $\tip$]
    \label{thm:Correctness_tip}
    The function $\tip$ as defined in \Cref{sec:rec} is a transitive projection.
  \end{theorem}
  \begin{proof}
    We have to prove that
    \begin{itemize}
      \item[(a)] $\tip(\tau,\sigma)$ is consistent with $\sigma$,
      \item[(b)] $\{\tip(\tau,\theta) \mid \theta \models \tau\}$ is finite, and
      \item[(c)] $\to_{\tip(\tau,\sigma)}$ is transitive
    \end{itemize}
    for all transitions $\tau$ and all $\sigma \models \tau$.

    For item (a), we first prove $\sigma \models \tau_\rec$.
    %
    We have:
    \begin{align*}
                             & \sigma \models \tau                                                                                                          \\
      {} \curvearrowright {} & \sigma \uplus [x_\delta / \sigma(x'-x) \mid x \in \vec{x}] \models \tau_{\land\delta} \tag{by def.\ of $\tau_{\land\delta}$} \\
      {} \curvearrowright {} & [x_\delta / \sigma(x'-x) \mid x \in \vec{x}] \models \tau_\delta \tag{by def.\ of $\mbp$ and $\tau_{\delta}$}                \\
      {} \curvearrowright {} & [x/\sigma(x),x'/\sigma(x') \mid x \in \vec{x}] \models \tau_\rec \tag{by def.\ of $\tau_{\rec}$}                             \\
      {} \curvearrowright {} & \sigma \models \tau_\rec
    \end{align*}
    %
    Now we prove
    $\sigma' \Def \sigma \uplus [n/1] \models \tip(\tau,\sigma)$.
    %
    To this end, we consider the literals that are added to $\tip(\tau,\sigma)$ by the procedure described in \Cref{sec:rec} independently:
    %
    \begin{itemize}
      \item $\sigma' \models n > 0$ is trivial
      \item if $\sigma \models (\sum_{x \in \vec{x}} c_x \cdot (x'-x) + c) \bowtie 0$, then $\sigma' \models (\sum_{x \in \vec{x}} c_x \cdot (x'-x) + n \cdot c) \bowtie 0$
      \item if $\sigma \models e|(\sum_{x \in \vec{x}} c_x \cdot (x'-x) + c)$, then $\sigma' \models e|(\sum_{x \in \vec{x}} c_x \cdot (x'-x) + n \cdot c)$
    \end{itemize}
    Moreover, we have $\sigma \models \mbp(\tau,\sigma,\vec{x})$ and $\sigma \models \mbp(\tau,\sigma,\vec{x}')$ by definition of $\mbp$, and hence we also have $\sigma' \models \mbp(\tau,\sigma,\vec{x})$ and $\sigma' \models \mbp(\tau,\sigma,\vec{x}')$.
    %
    Therefore, $\sigma'$ is a model of $\tip(\tau,\sigma)$, so $\sigma$ is consistent with $\tip(\tau,\sigma)$.

    For item (b), note that $\tip$ only uses the provided model for computing conjunctive variable projections.
    %
    As $\mbp$ has a finite image, the claim follows.

    For item (c), note that the conjuncts $\mbp(\tau,\sigma,\vec{x})$ and $\mbp(\tau,\sigma,\vec{x}')$ of $\tip(\tau,\sigma)$ are irrelevant for transitivity, as they do not relate $\vec{x}$ and $\vec{x}'$.
    %
    Let $\tau' \Def \tip(\tau,\sigma)$.
    %
    It suffices to prove
    \[
      {\to^2_{\tau'}} \subseteq {\to_{\tau'}}.
    \]
    Then the claim follows from a straightforward induction.
    %
    Assume $\theta \models \tau'$, $\theta' \models \tau'$, and
    \[
      \theta(\vec{x}) \to_{\tau'} \theta(\vec{x}') = \theta'(\vec{x}) \to \theta'(\vec{x}').
    \]
    We prove $\hat{\theta} \models \tau'$ where
    \[
      \hat{\theta} \Def [\vec{x} / \theta(\vec{x})] \uplus [\vec{x}' / \theta'(\vec{x}')] \uplus [n / \theta(n) + \theta'(n)].
    \]
    Then the claim follows.
    %
    Again, we consider all literals independently:
    \begin{itemize}
      \item If $\theta \models n > 0$ and $\theta' \models n > 0$, then $\hat{\theta} \models n > 0$.
      \item Consider a literal of the form $\iota \Def t \bowtie 0$ where $t = \sum_{x \in \vec{x}} c_x \cdot (x'-x) + n \cdot c$.
            %
            We have
            %
            \begin{align*}
              \theta(t) = {} & \sum_{x \in \vec{x}} c_x \cdot (\theta(x')-\theta(x)) + \theta(n) \cdot c                                                                               \\
              {} = {}        & \sum_{x \in \vec{x}} c_x \cdot \theta(x') - \sum_{x \in \vec{x}} c_x \cdot \theta(x) + \theta(n) \cdot c                                                \\
              {} = {}        & \sum_{x \in \vec{x}} c_x \cdot \theta'(x) - \sum_{x \in \vec{x}} c_x \cdot \theta(x) + \theta(n) \cdot c \tag{as $\theta(\vec{x}') = \theta'(\vec{x})$}
            \end{align*}
            and
            \begin{align*}
              \theta'(t) = {} & \sum_{x \in \vec{x}} c_x \cdot (\theta'(x')-\theta'(x)) + \theta'(n) \cdot c                                 \\
              {} = {}         & \sum_{x \in \vec{x}} c_x \cdot \theta'(x') - \sum_{x \in \vec{x}} c_x \cdot \theta'(x) + \theta'(n) \cdot c.
            \end{align*}
            Thus, we have:
            \begin{align*}
              \theta(t) + \theta'(t) = {} & -\sum_{x \in \vec{x}} c_x \cdot \theta(x) + \theta(n) \cdot c + \sum_{x \in \vec{x}} c_x \cdot \theta'(x')+ \theta'(n) \cdot c \\
              {} = {}                     & \sum_{x \in \vec{x}} c_x \cdot \theta'(x') - \sum_{x \in \vec{x}} c_x \cdot \theta(x) + (\theta(n) + \theta'(n)) \cdot c       \\
              {} = {}                     & \sum_{x \in \vec{x}} c_x \cdot \hat{\theta}(x') - \sum_{x \in \vec{x}} c_x \cdot \hat{\theta}(x) + \hat{\theta}(n) \cdot c     \\
              {} = {}                     & \hat{\theta}(t)
            \end{align*}
            Therefore, $\theta \models \iota$ and $\theta' \models \iota$ implies $\hat{\theta} \models \iota$ for all ${\bowtie} \in \{\leq,\geq,<,>,=\}$.
      \item Consider a literal of the form $\iota \Def e|t$ where $t = \sum_{x \in \vec{x}} c_x \cdot (x'-x) + n \cdot c$.
            %
            Then we again obtain
            \[
              \theta(t) + \theta'(t) = \hat{\theta}(t)
            \]
            as above.
            %
            Thus, $\theta \models \iota$ and $\theta' \models \iota$ imply $\hat{\theta} \models \iota$.
            %
            \qed
    \end{itemize}
  \end{proof}
}


