\documentclass[runningheads,orivec,envcountsame]{llncs}

\overfullrule=1mm

\usepackage[usenames,dvipsnames]{xcolor}
\usepackage{centernot}
\usepackage{amssymb}
\usepackage[linesnumbered,noend,boxruled]{algorithm2e}
\usepackage{stmaryrd}
\usepackage{bm}
\usepackage{tikz}
\usetikzlibrary{shapes,calc,arrows,automata,arrows.meta}
\usepackage{multirow}
\usepackage{array}
\usepackage{mathtools}
\usepackage{enumitem}
\usepackage[bookmarks,unicode,colorlinks=true]{hyperref}
\usepackage{proof}
\usepackage{varioref}
\usepackage[capitalize,nameinlink]{cleveref}
\usepackage[location=appendix,manual]{moveproofs}
\usepackage{thm-restate}
\usepackage{subfig}
\usepackage{hhline}
\usepackage{graphicx}
\usepackage{microtype}
\usepackage{cite}
\usepackage{pgfplots}
\usepackage{listings}
\pgfplotsset{compat=1.18}
\usepackage[edges]{forest}
\usepackage{empheq}
\usepackage[font=small,skip=0pt]{caption}
\usepackage{wrapfig}
\usepackage{marvosym}

\makeatletter
\newcommand{\algorithmstyle}[1]{\renewcommand{\algocf@style}{#1}}
\newcommand{\nosemic}{\renewcommand{\@endalgocfline}{\relax}}% Drop semi-colon ;
\newcommand{\dosemic}{\renewcommand{\@endalgocfline}{\algocf@endline}}% Reinstate semi-colon ;
\newcommand{\pushline}{\Indp}% Indent
\newcommand{\popline}{\Indm\dosemic}% Undent
\let\oldnl\nl% Store \nl in \oldnl
\newcommand{\nonl}{\renewcommand{\nl}{\let\nl\oldnl}}% Remove line number for one line
\makeatother

\newcommand{\report}[1]{#1}
\newcommand{\paper}[1]{}

\paper{
  % avoid lines at end of paragraphs with few words
  \everypar{\looseness=-1}
  % allow page breaks in formulas
  \allowdisplaybreaks[4]
  % allow page breaks right before formulas
  \predisplaypenalty=0
  % less space before and after algorithms
  \setlength{\intextsep}{3pt}
  \setlength{\textfloatsep}{1pt}
  % \setlist{nosep}
  \setlist{itemsep=1pt, topsep=3pt}
  \AtBeginDocument{
    \addtolength\abovedisplayskip{-0.3\baselineskip}
    \addtolength\belowdisplayskip{-0.3\baselineskip}
    \addtolength\abovedisplayshortskip{-0.3\baselineskip}
    \addtolength\belowdisplayshortskip{-0.3\baselineskip}
  }
}

\SetKwIF{If}{ElseIf}{Else}{if}{do}{else if}{else}{end if}%
\SetKwFor{While}{while}{}{end}%
% \SetKwFor{For}{for}{:}{end}%

% \renewcommand{\baselinestretch}{0.97}

\newcommand\mycommfont[1]{\scriptsize\ttfamily\textcolor{blue}{#1}}
\SetCommentSty{mycommfont}

\DeclareRobustCommand*{\modeledby}{%
  \Relbar\joinrel\mathrel{|}%
}

\DontPrintSemicolon

\hypersetup{%
  pdftitle={Infinite State Model Checking by Learning Transitive Relations},
  colorlinks=true,
  linkcolor=blue,
  citecolor=olive,
  filecolor=magenta,
  urlcolor=cyan
}



% ORCID
\makeatletter
\RequirePackage[bookmarks,unicode,colorlinks=true]{hyperref}%
   \def\@citecolor{blue}%
   \def\@urlcolor{blue}%
   \def\@linkcolor{blue}%
\def\UrlFont{\rmfamily}
\def\orcidID#1{\href{http://orcid.org/#1}{\protect\raisebox{-1.25pt}{\protect\includegraphics{orcid_color.eps}}}}
\makeatother


% Since we use the hyperref package, Springer asks us to include the following line
% to display URLs in blue roman font according to Springer's eBook style:
\renewcommand\UrlFont{\color{blue}\rmfamily}

\newenvironment{proofsketch}{%
  \renewcommand{\proofname}{Proof (Sketch)}\proof}{\endproof}

\renewcommand{\epsilon}{\varepsilon}
\let\oldphi\phi
\let\oldvarphi\varphi
\renewcommand{\varphi}{\oldphi}
\renewcommand{\phi}{\oldvarphi}

\newcommand{\pl}[1]{\textsf{#1}}
\newcommand{\tool}[1]{\textsf{#1}}
\def\increment{\hspace{-.05em}\raisebox{.4ex}{\tiny\bf ++}}
\def\decrement{\hspace{+.05em}\raisebox{.4ex}{\tiny\bf {-}{-}}}
\def\CXX{{\pl{C}\nolinebreak[4]\increment}}

\renewcommand{\vec}[1]{\bm{\mathrm{#1}}}

\newcommand{\mgu}{\mathsf{mgu}}
\newcommand{\rec}{\mathsf{rec}}
\newcommand{\push}{\mathsf{push}}
\newcommand{\pop}{\mathsf{pop}}
\newcommand{\add}{\mathsf{add}}
\newcommand{\tis}{\textsc{Tips}}
\newcommand{\ti}{\mathsf{ti}}
\newcommand{\accel}{\mathsf{accel}}
\newcommand{\underapprox}{\mathsf{ua}}
\newcommand{\tail}{\mathsf{tail}}
\newcommand{\lem}{\mathsf{lem}}
\newcommand{\Loop}{\mathsf{loop}}
\newcommand{\inc}{\mathsf{inc}}
\newcommand{\dec}{\mathsf{dec}}
\newcommand{\blocked}{\textsc{blocked}}
\newcommand{\blockingclause}{\mathsf{blocking\_clause}}
\newcommand{\backtrack}{\mathsf{backtrack}}
\newcommand{\checksat}{\mathsf{check}\_\mathsf{sat}}
\newcommand{\getmodel}{\mathsf{model}}
\newcommand{\mbp}{\mathsf{cvp}}
\newcommand{\mbip}{\mathsf{cvp}}
\newcommand{\sip}{\mathsf{sip}}
\newcommand{\unsat}{\mathsf{unsat}}
\newcommand{\unsafe}{\mathsf{unsafe}}
\newcommand{\unknown}{\mathsf{unknown}}
\newcommand{\encode}{{\sf encode}}
\newcommand{\sat}{\mathsf{sat}}
\newcommand{\safe}{\mathsf{safe}}
\newcommand{\tip}{\mathsf{tp}}
\renewcommand{\AA}{\mathcal{A}}
\newcommand{\BB}{\mathcal{B}}
\newcommand{\CC}{\mathcal{C}}
\newcommand{\LL}{\mathcal{L}}
\newcommand{\VV}{\mathcal{V}}
\newcommand{\MM}{\mathcal{M}}
\newcommand{\len}{\mathsf{len}}
\newcommand{\trace}{\mathsf{trace}}
\newcommand{\compose}{{\mathsf{compose}}}
\newcommand{\concat}{\mathrel{::}}
\newcommand{\init}{\mathsf{init}}
\newcommand{\err}{\mathsf{err}}
\newcommand{\QF}{\mathsf{QF}}
\renewcommand{\partial}{\rightharpoonup}

\def\mystack#1\over#2_#3{%
   \mathrel {%
      \setbox0=\hbox{$\scriptscriptstyle #1$}%
      \setbox1=\hbox{$#2$}%
      \ifdim\wd1>\wd0 \kern .5\wd1 \else \kern .5\wd0 \fi
      \vbox{
         \offinterlineskip
         \moveleft.5\wd0 \box0
         \kern.3ex
         \moveleft.5\wd1 \hbox{$#2_#3$}
}}}

\newcommand{\ind}[3][]{
  \ifthenelse{\equal{#1}{}}{\overset{\scriptscriptstyle(#3)}{#2}}{{\mystack (#3) \over #2_#1}}
}
\newcommand{\twodots}{%
  \mathinner{{\ldotp}{\ldotp}}%
}

% integers, rationals, reals, ...
\newcommand{\ZZ}{\mathbb{Z}}
\newcommand{\QQ}{\mathbb{Q}}
\newcommand{\NN}{\mathbb{N}}

% sets of transitions / programs etc.
\newcommand{\PP}{\mathcal{P}}
\newcommand{\TT}{\mathcal{T}}
\newcommand{\XX}{\mathcal{X}}
\newcommand{\RR}{\mathcal{R}}
\newcommand{\DG}{\mathcal{DG}}
\newcommand{\GG}{\mathcal{G}}
\renewcommand{\SS}{\mathcal{S}}

% big-oh
\newcommand{\OO}{\mathcal{O}}
% characteristic function
\newcommand{\charfun}[1]{I_{#1}}
% defining equation
\newcommand{\Def}{\mathrel{\mathop:}=}
% scalable version of \mid
\newcommand{\relmiddle}[1]{\mathrel{}\middle#1\mathrel{}}
% shortcut for space-saving matrices
\newcommand{\mat}[1]{\left(\begin{smallmatrix} #1 \end{smallmatrix}\right)}

\newcommand{\arity}{\mathsf{arity}}
\newcommand{\sem}[1]{\llbracket #1 \rrbracket}

\renewcommand{\emptyset}{\varnothing}

% comments
\def\me{JG}
\usepackage{ifthen}
\newcommand{\comment}[2][ALL]{%
  \ifthenelse{\equal{ALL}{#1}}%
  {\footnote{!!! #2}}%
  {%
    \ifthenelse{\equal{\me}{#1}}%
    {\footnote{!!! #2}}%
    {}%
  }%
}
%\renewcommand{\comment}[1]{}

\DeclareMathOperator{\dom}{dom}
\DeclareMathOperator{\img}{img}
\newcommand{\id}{\mathsf{id}}

\crefname{algorithm}{alg.}{algorithms}%
\crefname{equation}{eq.}{equations}%
\crefname{chapter}{chapter}{chapters}%
\crefname{section}{sect.}{sections}%
\crefname{appendix}{app.}{appendices}%
\crefname{enumi}{item}{items}%
\crefname{footnote}{footnote}{footnotes}%
\crefname{figure}{fig.}{figures}%
\crefname{table}{table}{tables}%
\crefname{theorem}{thm.}{theorems}%
\crefname{lemma}{lemma}{lemmas}%
\crefname{corollary}{cor.}{corollaries}%
\crefname{proposition}{proposition}{propositions}%
\crefname{definition}{def.}{definitions}%
\crefname{result}{result}{results}%
\crefname{example}{ex.}{examples}%
\crefname{remark}{remark}{remarks}%
\crefname{note}{note}{notes}%
\crefname{lstlisting}{listing}{listings}%
\crefname{requirement}{req.}{requirements}%

\title{Infinite State Model Checking by Learning Transitive Relations}
\author{Florian Frohn$^{(\href{mailto:florian.frohn@informatik.rwth-aachen.de}{\mbox{\Letter}})}$\orcidID{0000-0003-0902-1994} and Jürgen Giesl$^{(\href{mailto:giesl@informatik.rwth-aachen.de}{\mbox{\Letter}})}$\orcidID{0000-0003-0283-8520}}
\paper{\institute{RWTH Aachen University,  Aachen, Germany}}
\report{\institute{RWTH Aachen University,  Aachen, Germany\\
\email{\{florian.frohn,giesl\}@informatik.rwth-aachen.de}}}
\authorrunning{F.\ Frohn, J.\ Giesl}
  


\begin{document}

\renewcommand{\thelstlisting}{\arabic{lstlisting}}

\maketitle

\begin{abstract}  
Test time scaling is currently one of the most active research areas that shows promise after training time scaling has reached its limits.
Deep-thinking (DT) models are a class of recurrent models that can perform easy-to-hard generalization by assigning more compute to harder test samples.
However, due to their inability to determine the complexity of a test sample, DT models have to use a large amount of computation for both easy and hard test samples.
Excessive test time computation is wasteful and can cause the ``overthinking'' problem where more test time computation leads to worse results.
In this paper, we introduce a test time training method for determining the optimal amount of computation needed for each sample during test time.
We also propose Conv-LiGRU, a novel recurrent architecture for efficient and robust visual reasoning. 
Extensive experiments demonstrate that Conv-LiGRU is more stable than DT, effectively mitigates the ``overthinking'' phenomenon, and achieves superior accuracy.
\end{abstract}  
\section{Introduction}
\label{sec:introduction}
The business processes of organizations are experiencing ever-increasing complexity due to the large amount of data, high number of users, and high-tech devices involved \cite{martin2021pmopportunitieschallenges, beerepoot2023biggestbpmproblems}. This complexity may cause business processes to deviate from normal control flow due to unforeseen and disruptive anomalies \cite{adams2023proceddsriftdetection}. These control-flow anomalies manifest as unknown, skipped, and wrongly-ordered activities in the traces of event logs monitored from the execution of business processes \cite{ko2023adsystematicreview}. For the sake of clarity, let us consider an illustrative example of such anomalies. Figure \ref{FP_ANOMALIES} shows a so-called event log footprint, which captures the control flow relations of four activities of a hypothetical event log. In particular, this footprint captures the control-flow relations between activities \texttt{a}, \texttt{b}, \texttt{c} and \texttt{d}. These are the causal ($\rightarrow$) relation, concurrent ($\parallel$) relation, and other ($\#$) relations such as exclusivity or non-local dependency \cite{aalst2022pmhandbook}. In addition, on the right are six traces, of which five exhibit skipped, wrongly-ordered and unknown control-flow anomalies. For example, $\langle$\texttt{a b d}$\rangle$ has a skipped activity, which is \texttt{c}. Because of this skipped activity, the control-flow relation \texttt{b}$\,\#\,$\texttt{d} is violated, since \texttt{d} directly follows \texttt{b} in the anomalous trace.
\begin{figure}[!t]
\centering
\includegraphics[width=0.9\columnwidth]{images/FP_ANOMALIES.png}
\caption{An example event log footprint with six traces, of which five exhibit control-flow anomalies.}
\label{FP_ANOMALIES}
\end{figure}

\subsection{Control-flow anomaly detection}
Control-flow anomaly detection techniques aim to characterize the normal control flow from event logs and verify whether these deviations occur in new event logs \cite{ko2023adsystematicreview}. To develop control-flow anomaly detection techniques, \revision{process mining} has seen widespread adoption owing to process discovery and \revision{conformance checking}. On the one hand, process discovery is a set of algorithms that encode control-flow relations as a set of model elements and constraints according to a given modeling formalism \cite{aalst2022pmhandbook}; hereafter, we refer to the Petri net, a widespread modeling formalism. On the other hand, \revision{conformance checking} is an explainable set of algorithms that allows linking any deviations with the reference Petri net and providing the fitness measure, namely a measure of how much the Petri net fits the new event log \cite{aalst2022pmhandbook}. Many control-flow anomaly detection techniques based on \revision{conformance checking} (hereafter, \revision{conformance checking}-based techniques) use the fitness measure to determine whether an event log is anomalous \cite{bezerra2009pmad, bezerra2013adlogspais, myers2018icsadpm, pecchia2020applicationfailuresanalysispm}. 

The scientific literature also includes many \revision{conformance checking}-independent techniques for control-flow anomaly detection that combine specific types of trace encodings with machine/deep learning \cite{ko2023adsystematicreview, tavares2023pmtraceencoding}. Whereas these techniques are very effective, their explainability is challenging due to both the type of trace encoding employed and the machine/deep learning model used \cite{rawal2022trustworthyaiadvances,li2023explainablead}. Hence, in the following, we focus on the shortcomings of \revision{conformance checking}-based techniques to investigate whether it is possible to support the development of competitive control-flow anomaly detection techniques while maintaining the explainable nature of \revision{conformance checking}.
\begin{figure}[!t]
\centering
\includegraphics[width=\columnwidth]{images/HIGH_LEVEL_VIEW.png}
\caption{A high-level view of the proposed framework for combining \revision{process mining}-based feature extraction with dimensionality reduction for control-flow anomaly detection.}
\label{HIGH_LEVEL_VIEW}
\end{figure}

\subsection{Shortcomings of \revision{conformance checking}-based techniques}
Unfortunately, the detection effectiveness of \revision{conformance checking}-based techniques is affected by noisy data and low-quality Petri nets, which may be due to human errors in the modeling process or representational bias of process discovery algorithms \cite{bezerra2013adlogspais, pecchia2020applicationfailuresanalysispm, aalst2016pm}. Specifically, on the one hand, noisy data may introduce infrequent and deceptive control-flow relations that may result in inconsistent fitness measures, whereas, on the other hand, checking event logs against a low-quality Petri net could lead to an unreliable distribution of fitness measures. Nonetheless, such Petri nets can still be used as references to obtain insightful information for \revision{process mining}-based feature extraction, supporting the development of competitive and explainable \revision{conformance checking}-based techniques for control-flow anomaly detection despite the problems above. For example, a few works outline that token-based \revision{conformance checking} can be used for \revision{process mining}-based feature extraction to build tabular data and develop effective \revision{conformance checking}-based techniques for control-flow anomaly detection \cite{singh2022lapmsh, debenedictis2023dtadiiot}. However, to the best of our knowledge, the scientific literature lacks a structured proposal for \revision{process mining}-based feature extraction using the state-of-the-art \revision{conformance checking} variant, namely alignment-based \revision{conformance checking}.

\subsection{Contributions}
We propose a novel \revision{process mining}-based feature extraction approach with alignment-based \revision{conformance checking}. This variant aligns the deviating control flow with a reference Petri net; the resulting alignment can be inspected to extract additional statistics such as the number of times a given activity caused mismatches \cite{aalst2022pmhandbook}. We integrate this approach into a flexible and explainable framework for developing techniques for control-flow anomaly detection. The framework combines \revision{process mining}-based feature extraction and dimensionality reduction to handle high-dimensional feature sets, achieve detection effectiveness, and support explainability. Notably, in addition to our proposed \revision{process mining}-based feature extraction approach, the framework allows employing other approaches, enabling a fair comparison of multiple \revision{conformance checking}-based and \revision{conformance checking}-independent techniques for control-flow anomaly detection. Figure \ref{HIGH_LEVEL_VIEW} shows a high-level view of the framework. Business processes are monitored, and event logs obtained from the database of information systems. Subsequently, \revision{process mining}-based feature extraction is applied to these event logs and tabular data input to dimensionality reduction to identify control-flow anomalies. We apply several \revision{conformance checking}-based and \revision{conformance checking}-independent framework techniques to publicly available datasets, simulated data of a case study from railways, and real-world data of a case study from healthcare. We show that the framework techniques implementing our approach outperform the baseline \revision{conformance checking}-based techniques while maintaining the explainable nature of \revision{conformance checking}.

In summary, the contributions of this paper are as follows.
\begin{itemize}
    \item{
        A novel \revision{process mining}-based feature extraction approach to support the development of competitive and explainable \revision{conformance checking}-based techniques for control-flow anomaly detection.
    }
    \item{
        A flexible and explainable framework for developing techniques for control-flow anomaly detection using \revision{process mining}-based feature extraction and dimensionality reduction.
    }
    \item{
        Application to synthetic and real-world datasets of several \revision{conformance checking}-based and \revision{conformance checking}-independent framework techniques, evaluating their detection effectiveness and explainability.
    }
\end{itemize}

The rest of the paper is organized as follows.
\begin{itemize}
    \item Section \ref{sec:related_work} reviews the existing techniques for control-flow anomaly detection, categorizing them into \revision{conformance checking}-based and \revision{conformance checking}-independent techniques.
    \item Section \ref{sec:abccfe} provides the preliminaries of \revision{process mining} to establish the notation used throughout the paper, and delves into the details of the proposed \revision{process mining}-based feature extraction approach with alignment-based \revision{conformance checking}.
    \item Section \ref{sec:framework} describes the framework for developing \revision{conformance checking}-based and \revision{conformance checking}-independent techniques for control-flow anomaly detection that combine \revision{process mining}-based feature extraction and dimensionality reduction.
    \item Section \ref{sec:evaluation} presents the experiments conducted with multiple framework and baseline techniques using data from publicly available datasets and case studies.
    \item Section \ref{sec:conclusions} draws the conclusions and presents future work.
\end{itemize}
\section{Overview}

\revision{In this section, we first explain the foundational concept of Hausdorff distance-based penetration depth algorithms, which are essential for understanding our method (Sec.~\ref{sec:preliminary}).
We then provide a brief overview of our proposed RT-based penetration depth algorithm (Sec.~\ref{subsec:algo_overview}).}



\section{Preliminaries }
\label{sec:Preliminaries}

% Before we introduce our method, we first overview the important basics of 3D dynamic human modeling with Gaussian splatting. Then, we discuss the diffusion-based 3d generation techniques, and how they can be applied to human modeling.
% \ZY{I stopp here. TBC.}
% \subsection{Dynamic human modeling with Gaussian splatting}
\subsection{3D Gaussian Splatting}
3D Gaussian splatting~\cite{kerbl3Dgaussians} is an explicit scene representation that allows high-quality real-time rendering. The given scene is represented by a set of static 3D Gaussians, which are parameterized as follows: Gaussian center $x\in {\mathbb{R}^3}$, color $c\in {\mathbb{R}^3}$, opacity $\alpha\in {\mathbb{R}}$, spatial rotation in the form of quaternion $q\in {\mathbb{R}^4}$, and scaling factor $s\in {\mathbb{R}^3}$. Given these properties, the rendering process is represented as:
\begin{equation}
  I = Splatting(x, c, s, \alpha, q, r),
  \label{eq:splattingGA}
\end{equation}
where $I$ is the rendered image, $r$ is a set of query rays crossing the scene, and $Splatting(\cdot)$ is a differentiable rendering process. We refer readers to Kerbl et al.'s paper~\cite{kerbl3Dgaussians} for the details of Gaussian splatting. 



% \ZY{I would suggest move this part to the method part.}
% GaissianAvatar is a dynamic human generation model based on Gaussian splitting. Given a sequence of RGB images, this method utilizes fitted SMPLs and sampled points on its surface to obtain a pose-dependent feature map by a pose encoder. The pose-dependent features and a geometry feature are fed in a Gaussian decoder, which is employed to establish a functional mapping from the underlying geometry of the human form to diverse attributes of 3D Gaussians on the canonical surfaces. The parameter prediction process is articulated as follows:
% \begin{equation}
%   (\Delta x,c,s)=G_{\theta}(S+P),
%   \label{eq:gaussiandecoder}
% \end{equation}
%  where $G_{\theta}$ represents the Gaussian decoder, and $(S+P)$ is the multiplication of geometry feature S and pose feature P. Instead of optimizing all attributes of Gaussian, this decoder predicts 3D positional offset $\Delta{x} \in {\mathbb{R}^3}$, color $c\in\mathbb{R}^3$, and 3D scaling factor $ s\in\mathbb{R}^3$. To enhance geometry reconstruction accuracy, the opacity $\alpha$ and 3D rotation $q$ are set to fixed values of $1$ and $(1,0,0,0)$ respectively.
 
%  To render the canonical avatar in observation space, we seamlessly combine the Linear Blend Skinning function with the Gaussian Splatting~\cite{kerbl3Dgaussians} rendering process: 
% \begin{equation}
%   I_{\theta}=Splatting(x_o,Q,d),
%   \label{eq:splatting}
% \end{equation}
% \begin{equation}
%   x_o = T_{lbs}(x_c,p,w),
%   \label{eq:LBS}
% \end{equation}
% where $I_{\theta}$ represents the final rendered image, and the canonical Gaussian position $x_c$ is the sum of the initial position $x$ and the predicted offset $\Delta x$. The LBS function $T_{lbs}$ applies the SMPL skeleton pose $p$ and blending weights $w$ to deform $x_c$ into observation space as $x_o$. $Q$ denotes the remaining attributes of the Gaussians. With the rendering process, they can now reposition these canonical 3D Gaussians into the observation space.



\subsection{Score Distillation Sampling}
Score Distillation Sampling (SDS)~\cite{poole2022dreamfusion} builds a bridge between diffusion models and 3D representations. In SDS, the noised input is denoised in one time-step, and the difference between added noise and predicted noise is considered SDS loss, expressed as:

% \begin{equation}
%   \mathcal{L}_{SDS}(I_{\Phi}) \triangleq E_{t,\epsilon}[w(t)(\epsilon_{\phi}(z_t,y,t)-\epsilon)\frac{\partial I_{\Phi}}{\partial\Phi}],
%   \label{eq:SDSObserv}
% \end{equation}
\begin{equation}
    \mathcal{L}_{\text{SDS}}(I_{\Phi}) \triangleq \mathbb{E}_{t,\epsilon} \left[ w(t) \left( \epsilon_{\phi}(z_t, y, t) - \epsilon \right) \frac{\partial I_{\Phi}}{\partial \Phi} \right],
  \label{eq:SDSObservGA}
\end{equation}
where the input $I_{\Phi}$ represents a rendered image from a 3D representation, such as 3D Gaussians, with optimizable parameters $\Phi$. $\epsilon_{\phi}$ corresponds to the predicted noise of diffusion networks, which is produced by incorporating the noise image $z_t$ as input and conditioning it with a text or image $y$ at timestep $t$. The noise image $z_t$ is derived by introducing noise $\epsilon$ into $I_{\Phi}$ at timestep $t$. The loss is weighted by the diffusion scheduler $w(t)$. 
% \vspace{-3mm}

\subsection{Overview of the RTPD Algorithm}\label{subsec:algo_overview}
Fig.~\ref{fig:Overview} presents an overview of our RTPD algorithm.
It is grounded in the Hausdorff distance-based penetration depth calculation method (Sec.~\ref{sec:preliminary}).
%, similar to that of Tang et al.~\shortcite{SIG09HIST}.
The process consists of two primary phases: penetration surface extraction and Hausdorff distance calculation.
We leverage the RTX platform's capabilities to accelerate both of these steps.

\begin{figure*}[t]
    \centering
    \includegraphics[width=0.8\textwidth]{Image/overview.pdf}
    \caption{The overview of RT-based penetration depth calculation algorithm overview}
    \label{fig:Overview}
\end{figure*}

The penetration surface extraction phase focuses on identifying the overlapped region between two objects.
\revision{The penetration surface is defined as a set of polygons from one object, where at least one of its vertices lies within the other object. 
Note that in our work, we focus on triangles rather than general polygons, as they are processed most efficiently on the RTX platform.}
To facilitate this extraction, we introduce a ray-tracing-based \revision{Point-in-Polyhedron} test (RT-PIP), significantly accelerated through the use of RT cores (Sec.~\ref{sec:RT-PIP}).
This test capitalizes on the ray-surface intersection capabilities of the RTX platform.
%
Initially, a Geometry Acceleration Structure (GAS) is generated for each object, as required by the RTX platform.
The RT-PIP module takes the GAS of one object (e.g., $GAS_{A}$) and the point set of the other object (e.g., $P_{B}$).
It outputs a set of points (e.g., $P_{\partial B}$) representing the penetration region, indicating their location inside the opposing object.
Subsequently, a penetration surface (e.g., $\partial B$) is constructed using this point set (e.g., $P_{\partial B}$) (Sec.~\ref{subsec:surfaceGen}).
%
The generated penetration surfaces (e.g., $\partial A$ and $\partial B$) are then forwarded to the next step. 

The Hausdorff distance calculation phase utilizes the ray-surface intersection test of the RTX platform (Sec.~\ref{sec:RT-Hausdorff}) to compute the Hausdorff distance between two objects.
We introduce a novel Ray-Tracing-based Hausdorff DISTance algorithm, RT-HDIST.
It begins by generating GAS for the two penetration surfaces, $P_{\partial A}$ and $P_{\partial B}$, derived from the preceding step.
RT-HDIST processes the GAS of a penetration surface (e.g., $GAS_{\partial A}$) alongside the point set of the other penetration surface (e.g., $P_{\partial B}$) to compute the penetration depth between them.
The algorithm operates bidirectionally, considering both directions ($\partial A \to \partial B$ and $\partial B \to \partial A$).
The final penetration depth between the two objects, A and B, is determined by selecting the larger value from these two directional computations.

%In the Hausdorff distance calculation step, we compute the Hausdorff distance between given two objects using a ray-surface-intersection test. (Sec.~\ref{sec:RT-Hausdorff}) Initially, we construct the GAS for both $\partial A$ and $\partial B$ to utilize the RT-core effectively. The RT-based Hausdorff distance algorithms then determine the Hausdorff distance by processing the GAS of one object (e.g. $GAS_{\partial A}$) and set of the vertices of the other (e.g. $P_{\partial B}$). Following the Hausdorff distance definition (Eq.~\ref{equation:hausdorff_definition}), we compute the Hausdorff distance to both directions ($\partial A \to \partial B$) and ($\partial B \to \partial A$). As a result, the bigger one is the final Hausdorff distance, and also it is the penetration depth between input object $A$ and $B$.


%the proposed RT-based penetration depth calculation pipeline.
%Our proposed methods adopt Tang's Hausdorff-based penetration depth methods~\cite{SIG09HIST}. The pipeline is divided into the penetration surface extraction step and the Hausdorff distance calculation between the penetration surface steps. However, since Tang's approach is not suitable for the RT platform in detail, we modified and applied it with appropriate methods.

%The penetration surface extraction step is extracting overlapped surfaces on other objects. To utilize the RT core, we use the ray-intersection-based PIP(Point-In-Polygon) algorithms instead of collision detection between two objects which Tang et al.~\cite{SIG09HIST} used. (Sec.~\ref{sec:RT-PIP})
%RT core-based PIP test uses a ray-surface intersection test. For purpose this, we generate the GAS(Geometry Acceleration Structure) for each object. RT core-based PIP test takes the GAS of one object (e.g. $GAS_{A}$) and a set of vertex of another one (e.g. $P_{B}$). Then this computes the penetrated vertex set of another one (e.g. $P_{\partial B}$). To calculate the Hausdorff distance, these vertex sets change to objects constructed by penetrated surface (e.g. $\partial B$). Finally, the two generated overlapped surface objects $\partial A$ and $\partial B$ are used in the Hausdorff distance calculation step.
% !TEX root =  ../main.tex
\section{Background on causality and abstraction}\label{sec:preliminaries}

This section provides the notation and key concepts related to causal modeling and abstraction theory.

\spara{Notation.} The set of integers from $1$ to $n$ is $[n]$.
The vectors of zeros and ones of size $n$ are $\zeros_n$ and $\ones_n$.
The identity matrix of size $n \times n$ is $\identity_n$. The Frobenius norm is $\frob{\mathbf{A}}$.
The set of positive definite matrices over $\reall^{n\times n}$ is $\pd^n$. The Hadamard product is $\odot$.
Function composition is $\circ$.
The domain of a function is $\dom{\cdot}$ and its kernel $\ker$.
Let $\mathcal{M}(\mathcal{X}^n)$ be the set of Borel measures over $\mathcal{X}^n \subseteq \reall^n$. Given a measure $\mu^n \in \mathcal{M}(\mathcal{X}^n)$ and a measurable map $\varphi^{\V}$, $\mathcal{X}^n \ni \mathbf{x} \overset{\varphi^{\V}}{\longmapsto} \V^\top \mathbf{x} \in \mathcal{X}^m$, we denote by $\varphi^{\V}_{\#}(\mu^n) \coloneqq \mu^n(\varphi^{\V^{-1}}(\mathbf{x}))$ the pushforward measure $\mu^m \in \mathcal{M}(\mathcal{X}^m)$. 


We now present the standard definition of SCM.

\begin{definition}[SCM, \citealp{pearl2009causality}]\label{def:SCM}
A (Markovian) structural causal model (SCM) $\scm^n$ is a tuple $\langle \myendogenous, \myexogenous, \myfunctional, \zeta^\myexogenous \rangle$, where \emph{(i)} $\myendogenous = \{X_1, \ldots, X_n\}$ is a set of $n$ endogenous random variables; \emph{(ii)} $\myexogenous =\{Z_1,\ldots,Z_n\}$ is a set of $n$ exogenous variables; \emph{(iii)} $\myfunctional$ is a set of $n$ functional assignments such that $X_i=f_i(\parents_i, Z_i)$, $\forall \; i \in [n]$, with $ \parents_i \subseteq \myendogenous \setminus \{ X_i\}$; \emph{(iv)} $\zeta^\myexogenous$ is a product probability measure over independent exogenous variables $\zeta^\myexogenous=\prod_{i \in [n]} \zeta^i$, where $\zeta^i=P(Z_i)$. 
\end{definition}
A Markovian SCM induces a directed acyclic graph (DAG) $\mathcal{G}_{\scm^n}$ where the nodes represent the variables $\myendogenous$ and the edges are determined by the structural functions $\myfunctional$; $ \parents_i$ constitutes then the parent set for $X_i$. Furthermore, we can recursively rewrite the set of structural function $\myfunctional$ as a set of mixing functions $\mymixing$ dependent only on the exogenous variables (cf. \cref{app:CA}). A key feature for studying causality is the possibility of defining interventions on the model:
\begin{definition}[Hard intervention, \citealp{pearl2009causality}]\label{def:intervention}
Given SCM $\scm^n = \langle \myendogenous, \myexogenous, \myfunctional, \zeta^\myexogenous \rangle$, a (hard) intervention $\iota = \operatorname{do}(\myendogenous^{\iota} = \mathbf{x}^{\iota})$, $\myendogenous^{\iota}\subseteq \myendogenous$,
is an operator that generates a new post-intervention SCM $\scm^n_\iota = \langle \myendogenous, \myexogenous, \myfunctional_\iota, \zeta^\myexogenous \rangle$ by replacing each function $f_i$ for $X_i\in\myendogenous^{\iota}$ with the constant $x_i^\iota\in \mathbf{x}^\iota$. 
Graphically, an intervention mutilates $\mathcal{G}_{\mathsf{M}^n}$ by removing all the incoming edges of the variables in $\myendogenous^{\iota}$.
\end{definition}

Given multiple SCMs describing the same system at different levels of granularity, CA provides the definition of an $\alpha$-abstraction map to relate these SCMs:
\begin{definition}[$\abst$-abstraction, \citealp{rischel2020category}]\label{def:abstraction}
Given low-level $\mathsf{M}^\ell$ and high-level $\mathsf{M}^h$ SCMs, an $\abst$-abstraction is a triple $\abst = \langle \Rset, \amap, \alphamap{} \rangle$, where \emph{(i)} $\Rset \subseteq \datalow$ is a subset of relevant variables in $\mathsf{M}^\ell$; \emph{(ii)} $\amap: \Rset \rightarrow \datahigh$ is a surjective function between the relevant variables of $\mathsf{M}^\ell$ and the endogenous variables of $\mathsf{M}^h$; \emph{(iii)} $\alphamap{}: \dom{\Rset} \rightarrow \dom{\datahigh}$ is a modular function $\alphamap{} = \bigotimes_{i\in[n]} \alphamap{X^h_i}$ made up by surjective functions $\alphamap{X^h_i}: \dom{\amap^{-1}(X^h_i)} \rightarrow \dom{X^h_i}$ from the outcome of low-level variables $\amap^{-1}(X^h_i) \in \datalow$ onto outcomes of the high-level variables $X^h_i \in \datahigh$.
\end{definition}
Notice that an $\abst$-abstraction simultaneously maps variables via the function $\amap$ and values through the function $\alphamap{}$. The definition itself does not place any constraint on these functions, although a common requirement in the literature is for the abstraction to satisfy \emph{interventional consistency} \cite{rubenstein2017causal,rischel2020category,beckers2019abstracting}. An important class of such well-behaved abstractions is \emph{constructive linear abstraction}, for which the following properties hold. By constructivity, \emph{(i)} $\abst$ is interventionally consistent; \emph{(ii)} all low-level variables are relevant $\Rset=\datalow$; \emph{(iii)} in addition to the map $\alphamap{}$ between endogenous variables, there exists a map ${\alphamap{}}_U$ between exogenous variables satisfying interventional consistency \cite{beckers2019abstracting,schooltink2024aligning}. By linearity, $\alphamap{} = \V^\top \in \reall^{h \times \ell}$ \cite{massidda2024learningcausalabstractionslinear}. \cref{app:CA} provides formal definitions for interventional consistency, linear and constructive abstraction.
\section{Transitive Relation Learning}
\label{sec:til}
%
In this section, we present our novel model checking algorithm \emph{Transitive Relation
Learning} (TRL) in detail, see \Cref{alg}.
%
Here and in the following, for all
$i,j \in \NN_+ = \NN \setminus \{0\}$
%JG added definition of N_+
we define $\mu_{i,j}(x') \Def \ind{x}{i+j}$ if $x' \in \vec{x}'$ and $\mu_{i,j}(x) \Def \ind{x}{i}$, otherwise.
%
So in particular, we have $\mu_{i,j}(\vec{x}) = \ind{\vec{x}}{i}$ and $\mu_{i,j}(\vec{x}') = \ind{\vec{x}}{i+j}$, where we assume that $\ind{\vec{x}}{1},\ind{\vec{x}}{2}, \ldots \in \VV^d$ are disjoint vectors of pairwise different fresh variables.
%
Intuitively, the variables $\ind{\vec{x}}{i}$ represent the $i^{th}$ state in a run, and applying $\mu_{i,j}$ to a relational formula yields a formula that relates the $i^{th}$ and the $(i+j)^{th}$ state of a run.
%
For convenience, we define $\mu_{i} \Def \mu_{i,1}$ for all $i \in \NN$, i.e., $\mu_i(\vec{x}) = \ind{\vec{x}}{i}$ and $\mu_i(\vec{x}') = \ind{\vec{x}}{i+1}$.
%
As in SMT-based BMC, TRL uses an incremental SMT solver to unroll the transition relation step by step (\Cref{alg:unroll}), but in contrast to BMC, TRL infers \emph{learned relations} on the fly (\Cref{alg:learn2}).
%
The \emph{input formula} $\tau$ as well as all learned relations are stored in $\vec{\pi}$.
%
Before each unrolling, we set a backtracking point with the command $\push$ and add a suitably variable-renamed version of the description of the error states to the SMT problem in \Cref{alg:err1}.
%
Then the command $\checksat$ checks for reachability of error states, and the command $\pop$ removes all formulas from the SMT problem that have been added since the last invocation of $\push$ (\Cref{alg:err2}), i.e., it removes the encoding of the error states (unless the check succeeds, so that TRL fails).
%
For each unrolling, suitably variable renamed variants of $\vec{\pi}$'s elements are added to the underlying SMT problem with the command $\add$ in \Cref{alg:unroll}.
%
If no error state is reachable after $b-1$ steps, but the transition relation cannot be
unrolled $b$ times (i.e., the SMT problem that corresponds to the $b$-fold unrolling is
unsatisfiable), then the diameter of the analyzed system (including learned relations) has
been reached, and hence safety has been proven (\Cref{alg:safe}).

The remainder of this section is structured as follows:
%
First, \Cref{sec:basics} introduces \emph{conjunctive variable projections} that are used to
compute the \emph{trace} (\Cref{alg:trace} of \Cref{alg}).
%
Next, \Cref{sec:loops} defines \emph{loops} and
discusses how to find \emph{non-redundant loops} that are suitable for learning new relations (\Cref{alg:loop,alg:model,alg:redundant,alg:learn1}).
%
Then, \Cref{sec:transitiveProjections} introduces \emph{transitive projections}
that are used to learn relations (\Cref{alg:trans,alg:learn2}).
% FF that sounds weird, I think
%, and discuss their transitivity (see \Cref{alg:trans}).
%JG ok
\report{Afterwards}\paper{Finally}, \Cref{sec:block} presents
\emph{blocking clauses}, which ensure that learned re\-la\-tions are preferred over other (sequences of) transitions
% needed for \Cref{alg:block1}, \ref{alg:safe}, and
(\Cref{alg:block1,alg:pick,alg:block2,alg:backtrack}).
\report{Finally, we illustrate \Cref{alg} with
a complete example in \Cref{sec:example}.}

\begin{algorithm}[t]
  $b \gets 0; \quad \vec{\pi} \gets [\tau]; \quad \blocked \gets \emptyset$\; \label{alg:init}
  $\add(\mu_{1}(\psi_\init))$ \tcp*{encode the initial states}
  \While(\tcp*[f]{main loop}){$\top$}{
    $b\increment; \quad \push(); \quad \add(\mu_{b}(\psi_\err))$ \tcp*{encode the error states} \label{alg:err1}
    \leIf{$\checksat()$}{
      \Return{$\unknown$} \label{alg:err2}
    }{
      $\pop()$ \tcp*[f]{check their reachability\hspace{-.9em}}
    }
    $\push()$ \tcp*{add backtracking point}
    \lIf(\tcp*[f]{encode transitivity}){$b>1$}{$\add(\ind[\id]{x}{b} \doteq 1 \lor
      \ind[\id]{x}{b} \not\doteq \ind[\id]{x}{b-1})$} \label{alg:trans}
    $\add(\mu_{b}(\bigvee_{n=1}^{|\vec{\pi}|} (\pi_n \land x_\id \doteq n)))$ \tcp*{encode
      $\to_\tau$ and learned relations} \label{alg:unroll}
    $\add(\bigwedge_{(b,\pi) \in \blocked} \pi)$ \tcp*{add blocking clauses for this $b$} \label{alg:block1}
    \lIf{$\neg\checksat()$}{
      \Return{$\safe$} \label{alg:safe} \tcp*[f]{check if the search space is exhausted}}
    $\sigma \gets \getmodel(); \quad \vec{\tau} \gets \trace_b(\sigma,\vec{\pi})$ \label{alg:trace} \tcp*{build trace from current model}
    \If(\tcp*[f]{search loop}){$[\tau_s,\ldots,\tau_{s+\ell-1}]$ is a loop \label{alg:loop}}{
      $\sigma_{\Loop} \gets [x / \sigma(\mu_{s,\ell}(x)) \mid x \in \vec{x} \cup \vec{x}']$ \label{alg:model}\tcp*{build the valuation for the loop}
      \If(\tcp*[f]{redundancy check}){no $\pi \in \tail(\vec{\pi})$ is consistent with $\sigma_{\Loop}$ \label{alg:redundant}}{
        $\tau_{\Loop} \gets \mu_{s,\ell}^{-1}(\bigwedge_{i=s}^{s+\ell-1} \mu_{i}(\tau_i))$ \label{alg:learn1} \tcp*{build the loop}
        $\vec{\pi} \gets \vec{\pi}\concat\tip(\tau_{\Loop}, \sigma \circ \mu_{s,\ell})$ \label{alg:learn2} \tcp*[f]{learn relation}
      }
      $\text{let } \pi \in \tail(\vec{\pi}) \text{ and } \overline{\sigma} \supseteq
      \sigma_{\Loop}$ s.t.\ $\overline{\sigma} \models
\pi$ \tcp*{pick suitable learned rel.} \label{alg:pick}
      $\blocked.\add(s+\ell-1,\blockingclause(s,\ell,\pi,\overline{\sigma}))$ \tcp*{block the loop} \label{alg:block2}
      \lWhile{$b > s$}{$\{ \, \pop(); \ b\decrement \,\}$ \tcp*[f]{backtrack to the start of the loop}\label{alg:backtrack}}
    }
  }
  \caption{TRL -- Input: a safety problem $\TT = (\psi_\init,\tau,\psi_\err)$}
  \label{alg}
\end{algorithm}

\subsection{Conjunctive Variable Projections and Traces}
\label{sec:basics}

To decide when to learn a new relation, TRL inspects the \emph{trace} (Lines \ref{alg:trace} and \ref{alg:loop}).
%
The trace is a sequence of transitions induced by the formulas that have been added to the SMT problem while unrolling the transition relation, and by the current model (\Cref{alg:trace}).
%
To compute them,  we use \emph{conjunctive variable projections}, which are like
\emph{model based projections} \cite{spacer}, but always 
yield\paper{ \pagebreak[3]} conjunctions.%
%
\begin{definition}[Conjunctive Variable Projection]
  \label{def:projections}
  A function $\mbip$ is called a \emph{conjunctive variable projection} if

  \vspace{-0.5em}
  \noindent
  \begin{minipage}[t]{0.49\textwidth}
    \begin{enumerate}
    \item $\sigma \models \mbip(\tau,\sigma,X)$,
    \item $\mbip(\tau,\sigma,X) \models \tau$,
    \item $\{\mbip(\tau,\theta,X) \mid \theta \models \tau\}$ is finite,
    \end{enumerate}
  \end{minipage}
  \begin{minipage}[t]{0.49\textwidth}
    \begin{enumerate}
      \setcounter{enumi}{3}
    \item $\VV(\mbip(\tau,\sigma,X)) \subseteq X \cap \VV(\tau)$, and
    \item $\mbip(\tau,\sigma,X) \in \QF_\land(\Sigma)$
    \end{enumerate}
  \end{minipage}
  \medskip

  \noindent
  for all $\tau \in \QF(\Sigma)$, $X \subseteq \VV$, and $\sigma \models
  \tau$.
  %
  We abbreviate $\mbip(\tau,\sigma,\vec{x} \cup \vec{x}')$ by $\mbip(\tau,\sigma)$.
\end{definition}
%
So like model based projection, $\mbip$ under-approximates quantifier elimination by projecting to the variables $X$ (by (2) and (4)).
%
To do so, it implicitly performs a finite case analysis (by (3)), which is driven by
the model $\sigma$ (by (1)).
%
In contrast to model based projections, $\mbip$ always yields conjunctions (by (5)).
%
% FF I think the following remark does not make much sense with the new version of (4):
% "Note that by (1) and (4), $\mbip(\tau,\sigma,X)$ only contains variables from $\dom(\sigma) \cap X\cap \VV(\tau)$."
% The reason is that we explicitly say $\VV(\mbip(\tau,\sigma,X)) \subseteq X \cap \VV(\tau)$, and we have $\VV(\tau) \subseteq \dom(\sigma)$ by definition of $\models$.
% Hence, this remark is equivalent to (4).
%JG The purpose of this remark was to remind the reader that (1) implies
%$\VV(\mbip(\tau,\sigma,X)) \subseteq \dom(\sigma)$. This is due to our definition of
%$\models$ which is not the standard one in logic. Thus, I added this part of the remark again.
Note that by (1),
$\mbip(\tau,\sigma,X)$ may only contain variables from $\dom(\sigma)$.


\begin{remark}[$\mbip$ and $\mathsf{mbp}$]
  Conjunctive variable projections are obtained by combining a model based projection $\mathsf{mbp}$ (which satisfies \Cref{def:projections} (1--4)) with \emph{syntactic implicant projection} $\sip$ \cite{adcl}, where $\sip(\tau,\sigma)$ is the conjunction of all literals of $\tau$'s negation normal form that are satisfied by $\sigma$.
  %
  Then $\mbip(\tau,\sigma) \Def \sip(\mathsf{mbp}(\tau,\sigma),\sigma)$.
\end{remark}
%
\begin{remark}[$\mbip$ and Quantifier Elimination]
  \Cref{def:projections} (1--3) imply
  \begin{align*}
        \exists \vec{y}.\ \tau & {} \equiv \bigvee \{\mbp(\tau,\sigma) \mid \sigma \models \tau\} \label{mbp-property} \qquad \text{where $\vec{y}$ are $\tau$'s extra variables.}
\end{align*}
So $\mbp$ yields a quantifier elimination procedure $\mathsf{qe}$ which maps $\exists
\vec{y}.\ \tau$ to $\mathit{res}$:
%JG added argument of qe. This shows that this is the same argument that is used for qe
%four lines later.

\report{\vspace{-1.6em}}


\algorithmstyle{plain}
\begin{algorithm}[h!]
\nonl $\mathit{res} \gets \bot;\hspace{.75em}$ \lWhile{$\tau$ has a model $\sigma$}{$\{\mathit{res} \gets \mathit{res} \lor \mbp(\tau,\sigma);\hspace{.75em} \tau \gets \tau \land \neg \mbp(\tau,\sigma)\}$}
\end{algorithm}

\paper{\vspace{-.2em}}
\report{\vspace{-2em}}

\noindent
But for a single model $\sigma$, $\mbp(\tau,\sigma)$ under-ap\-prox\-i\-mates quantifier elimination.
\end{remark}
%
The details of implementing $\mbip$ are beyond the scope of this paper.
%
A good intuition\paper{ \pagebreak[3]} is that $\mbip(\tau,\sigma)$ just computes one disjunct of $\mathsf{qe}(\exists \vec{y}.\ \tau)$ which is satisfied by $\sigma$.
%
However, like model based projection, $\mbip$ can be implemented efficiently for many theories with effective, but very expensive quantifier elimination procedures.%
%
\begin{example}[$\mbip$]
  \label{ex:projections}
  Consider the following formula $\ind{\tau}{1 \twodots 3}$:
  \small
  \[
    \begin{array}{rcl}
      (w \doteq 0 \land \ind{x}{2} \doteq x + 1 \land \ind{y}{2} \doteq y + 1) & \lor & (\ind{w}{2} \doteq w \land w \doteq 1 \land \ind{x}{2} \doteq x - 1 \land \ind{y}{2} \doteq y - 1) \land {} \\
      (\ind{w}{2} \doteq 0 \land x' \doteq \ind{x}{2} + 1 \land y' \doteq \ind{y}{2} + 1) & \lor & (w' \doteq \ind{w}{2} \land \ind{w}{2} \doteq 1 \land x' \doteq \ind{x}{2} - 1 \land y' \doteq \ind{y}{2} - 1)
    \end{array}
  \]
  \normalsize It encodes two steps with \Cref{ex:ex1},
  where $\ind{\vec{x}}{2} = [\ind{w}{2},\ind{x}{2},\ind{y}{2}]$ represents the values after one
  step.
  In \Cref{alg:trace}, \Cref{alg} might find a run like $\sigma(\ind{\vec{x}}{1}) \to_\tau \sigma(\ind{\vec{x}}{2}) \to_\tau \sigma(\ind{\vec{x}}{3})$ for
  \[
  \begin{array}{rcl@{\;\;}l@{\;\;}l}
  \sigma & \Def  & [\ind{w}{1}/\ind{x}{1}/\ind{y}{1}/0, & \ind{w}{2}/\ind{x}{2}/\ind{y}{2}/1, & \ind{w}{3}/1,\ind{x}{3}/\ind{y}{3}/0].
  \end{array}
  \]
 Here, $[w/x/y/c, \ldots]$ abbreviates $[w/c, x/c, y/c, \ldots]$.
  %
  Then the variable renaming $\mu_{1,2}$ allows us to
instantiate the pre- and post-variables
by the first and last state,
resulting in the following model of
$\ind{\tau}{1 \twodots 3}$:
  %
   \[
    \sigma' \Def \sigma \circ \mu_{1,2} = \sigma \cup [w/x/y/0,\;\; w'/1,x'/y'/0] \qquad \text{where } (\sigma \circ \mu_{1,2})(x) = \sigma(\mu_{1,2}(x))
  \]
  %
  To get rid of $\ind{w}{2},\ind{x}{2},\ind{y}{2}$, one could compute $\mathsf{qe}(\exists \ind{w}{2},\ind{x}{2},\ind{y}{2}.\ \ind{\tau}{1 \twodots 3})$, resulting in:
  \begin{align}
    & w \doteq 0 \land w' \doteq 0 \land x' \doteq x+2 \land y' \doteq y+2 \tag{\ensuremath{\inc}} \\
    {} \lor {} & w \doteq 0 \land w' \doteq 1 \land x' \doteq x \land y' \doteq y \label{eq:mbp-ex} \tag{\ensuremath{\mathsf{eq}}}\\
    {} \lor {} &w \doteq 1 \land w' \doteq 1 \land x' \doteq x-2 \land y' \doteq y-2. \tag{\ensuremath{\dec}}
  \end{align}
  Instead, we may have $\mbp(\ind{\tau}{1 \twodots 3},\sigma') = \eqref{eq:mbp-ex}$, as
  $\sigma' \models \eqref{eq:mbp-ex}$.
  %JG Changed  $\sigma \models
   % \eqref{eq:mbp-ex}$ to $\sigma' \models
   % \eqref{eq:mbp-ex}$.
\end{example}
%
Intuitively, a relational formula $\tau$ describes how states can change, so it is composed of many different cases.
%
These cases may be given explicitly (by disjunctions) or implicitly (by
extra variables, which express non-determinism).
%
Given a model $\sigma$ of $\tau$ that describes a \emph{concrete} change of state, $\mbip$ computes a description of the corresponding case.
%
Computing \emph{all} cases amounts to eliminating all extra variables and converting the result to DNF, which is impractical.

When unrolling the transition relation in \Cref{alg:unroll} of \Cref{alg}, we identify
each relational formula $\pi_n$ with its index $n$ in the sequence $\vec{\pi}$.
%
To this end, we use a fresh variable $x_\id$, and our SMT encoding forces
$\ind[\id]{x}{i}$ to be the identifier of the relation that is used for the $i^{th}$ step.
%
Similarly to \cite{abmc}, the \emph{trace} is the sequence of transitions that results from applying $\mbip$ to the unrolling
of the transition relation that is constructed by \Cref{alg} in \Cref{alg:unroll}.
%
So a trace is a sequence of transitions that can be applied subsequently, starting in an initial state.
%
\begin{definition}[Trace]
  \label{def:trace}
  Let $\vec{\pi}$ be a sequence of relational formulas, let
  \begin{align}
    \label{eq:trace}
    \paper{\textstyle}
    \sigma \models \bigwedge_{i=1}^{b} \mu_{i}\left(\bigvee_{n=1}^{|\vec{\pi}|} (\pi_n \land x_\id \doteq n)\right) \qquad \text{where $b \in \NN_+$},
  \end{align}
  and let $\id(i) \Def \sigma(\ind[\id]{x}{i})$.
  %
  Then the \emph{trace induced by $\sigma$} is
  \[
    \trace_b(\sigma,\vec{\pi}) \Def [\mbip(\pi_{\id(i)}, \sigma \circ \mu_{i})]_{i=1}^{b}.
  \]
\end{definition}

Recall that $\mu_i$ renames $\vec{x}$ and $\vec{x}'$ into $\ind{\vec{x}}{i}$ and
$\ind{\vec{x}}{i+1}$, and $\id(i) = \sigma(\ind[\id]{x}{i})$ is the index of the
relation from $\vec{\pi}$ that is used for the $i^{th}$ step.\comment[NONE]{FF If
  we have this remark here, then we should remove the similar remark ``i.e.,
  $\mu_i(\vec{x}) = \ind{\vec{x}}{i}$
  and $\mu_i(\vec{x}') =\ind{\vec{x}}{i+1}$.\\
  JG Ok, but only if it at least saves a line. Otherwise, we can just as well keep the
  remark there as well.}
%
So each model $\sigma$ of \eqref{eq:trace} corresponds to a run $\sigma(\mu_1(\vec{x}))
\to_{\pi_{\id(1)}} \ldots \to_{\pi_{\id(b)}} \sigma(\mu_{b}(\vec{x}'))$,\paper{ \pagebreak[3]} and the trace
induced by $\sigma$ contains the transitions that were used in this run.

%
\begin{example}[Trace]
  \label{ex:trace}
  Consider the extension of $\sigma$ from \Cref{ex:projections} with $[\ind[\id]{x}{1}/1,\;
    \ind[\id]{x}{2}/1]$:
  \[
  \sigma \Def [\ind{w}{1}/\ind{x}{1}/\ind{y}{1}/0,\ind[\id]{x}{1}/1, \quad
    \ind{w}{2}/\ind{x}{2}/\ind{y}{2}/\ind[\id]{x}{2}/1, \quad \ind{w}{3}/1,\ind{x}{3}/\ind{y}{3}/0],
  \]
  Thus, $\id(1) = \sigma(\ind[\id]{x}{1}) = 1$, $\id(2) =  \sigma(\ind[\id]{x}{2}) = 1$, and
  $\pi_{\id(1)} = \pi_{\id(2)} =
  \pi_1 = \tau$.
  %
  Then
  \begin{align*}
    & \trace_2(\sigma, [\tau,\tau]) = [\mbip(\tau,\sigma \circ \mu_{1}), \mbip(\tau,\sigma \circ \mu_{2})]           \\
    {} = {} & [\mbip(\tau,[w/x/y/0, \; w'/x'/y'/1]), \mbip(\tau,[w/x/y/1, \; w'/1,x'/y'/0])] = [\tau_\inc, \tau_\dec].
  \end{align*}
\end{example}

\subsection{Loops}
\label{sec:loops}

As $\vec{\pi}$ only gives rise to finitely many transitions, the trace is bound to contain \emph{loops}, eventually (unless \Cref{alg} terminates beforehand).
%
\begin{definition}[Loop]
  A sequence of transitions $\tau_1,\ldots,\tau_k$ is called a \emph{loop} if there are $\vec{v}_0,\ldots,\vec{v}_{k+1} \in \CC^d$ such that $\vec{v}_0 \to_{\tau_1} \ldots \to_{\tau_k} \vec{v}_{k} \to_{\tau_1} \vec{v}_{k + 1}$.
\end{definition}
%
Intuitively, these loops are the reason why BMC may diverge.
%
To prevent divergence, TRL learns a new relation when a loop is detected (\Cref{alg:loop}).
%
\begin{remark}[Finding Loops]
  Loops can be detected by SMT solving.
  %JG I think that it is not correct to say that something is detected by "SMT". I think
  %it should be "SMT solving".
%
A cheaper way is to look for duplicates, but then loops are found ``later'', as
%
a trace $[\ldots, \pi, \pi, \ldots]$\linebreak is needed to detect a loop $\pi$, but one occurrence of $\pi$ is insufficient.
%
As a trade-off between precision and efficiency, our im\-ple\-men\-ta\-tion uses \emph{dependency graphs} \cite{abmc}.
\end{remark}

\begin{remark}[Disregarding ``Learned'' Loops]
  \label{remark:loops}
    One should disregard ``loops'' consisting of a single \emph{learned transition}, i.e., a transition that results from applying $\mbip$ to some $\pi \in \tail(\vec{\pi})$.
    %
    Here, $\tail(\tau\concat\vec{\pi}') \Def \vec{\pi}'$ contains all learned relations, as the first element of $\vec{\pi}$ is the input formula $\tau$.
    %
    The reason is that our goal is to deduce transitive relations, but learned relations are already transitive.
    %
    In the sequel, we assume that the check in \Cref{alg:loop} fails for such loops.
\end{remark}
%
If there are several choices for $s$ and $\ell$ in \Cref{alg:loop}, then our implementation only considers loops of minimal length and, among those, it minimizes $s$.
%
\begin{example}[Detecting Loops]
  \label{ex:loops}
  Consider the model
  \[
    \sigma \Def [\ind{w}{1}/\ind{x}{1}/\ind{y}{1}/0, \ind[\id]{x}{1}/1, \quad \ind{w}{2}/0,\ind{x}{2}/\ind{y}{2}/1]
  \]
  %
for $\tau$ from our running example (\Cref{ex:ex1}).
   Then $\trace_1(\sigma, [\tau]) = [\tau_\inc]$.
  %
  As $\tau_\inc$ is a loop, TRL learns a relation like $\tau^+_{\inc}$ at this point.
\end{example}
%
TRL only learns relations from loops that are \emph{non-redundant} w.r.t.\ all relations that have been learned before \cite{adcl}.
%
\begin{definition}[Redundancy]
  \label{def:redundancy}
  If ${\to_\tau} \subseteq {\to_{\tau'}}$, then $\tau$ is \emph{redundant} w.r.t.\ $\tau'$.
\end{definition}
% FF I think it's fine to omit the title of examples if they are directly preceded by the
% corresponding definition
% JG ok
\begin{example}
  The relation $\tau_\inc$ is redundant w.r.t.\ $\tau^+_\inc$, but $\tau_\dec$ is not.
\end{example}
%
\Cref{alg:redundant} uses a sufficient criterion for non-redundancy:
%
If all learned relations are falsified by the values before and after the loop, then
$\tau_s, \ldots, \tau_{s + \ell -1}$ cannot be simulated by a previously learned relation, so it is non-redundant and we learn a new relation.
%
The values before and after the loop are obtained from the current model $\sigma$ by
setting $\vec{x}$ to $\sigma(\ind{\vec{x}}{s})$ and $\vec{x}'$ to
$\sigma(\ind{\vec{x}}{s+\ell})$, i.e., we use $\sigma \circ \mu_{s,\ell}$ in
\Cref{alg:model}.

To learn a new relation,
we first compute the  relation\paper{ \pagebreak[3]} 
\begin{equation}
  \label{eq:loop-formula}
  \paper{\textstyle}
  \tau_\Loop \Def \mu_{s,\ell}^{-1}(\phi_\Loop) \qquad \text{where} \qquad \phi_\Loop \Def
  \bigwedge_{i=s}^{s+\ell-1} \mu_{i}(\tau_i) 
\end{equation}
of the loop in \Cref{alg:learn1}, where $\mu^{-1}_{s,\ell}$ is the inverse of $\mu_{s,\ell}$.
%
So in \Cref{ex:loops}, we have $\sigma \circ \mu_{1,1} \supseteq [w/x/y/0,\,
  w'/0,x'/y'/1]$ and $\tau_\Loop \Def \mu^{-1}_{1,1}(\mu_1(\tau_\inc)) = \tau_\inc$ as $s = \ell = 1$.
%
So $\sigma \circ \mu_{s,\ell}$ indeed corresponds to one evaluation of the loop, as $\sigma \circ \mu_{s,\ell} \models \tau_\Loop$.

To see that $\tau_\Loop$ is also the desired relation in general, note that $\phi_\Loop$ is the conjunction of the transitions that constitute the loop, where all variables are renamed as in \Cref{alg:unroll} of \Cref{alg}, i.e., in such a way that the post-variables of the $i^{th}$ step are equal to the pre-variables of the $(i+1)^{th}$ step.
%
So we have $\sigma \models \phi_\Loop$ and thus $\sigma \circ \mu_{s,\ell} \models \tau_\Loop$.
%
Hence,
we can use $\tau_\Loop$ and $\sigma \circ \mu_{s,\ell}$ to learn a new relation via
so-called \emph{transitive projections}
in \Cref{alg:learn2}.


\subsection{Transitive Projections}
\label{sec:transitiveProjections}



We now define \emph{transitive projections} that approximate transitive closures of loops.
%
As explained in \Cref{sec:overview}, we do not restrict ourselves to under- or over-approximations, but we allow ``mixtures'' of both.
%
Analogously to $\mbip$, transitive projections perform a finite case analysis that is
driven by the provided model $\sigma$.

\begin{definition}[Transitive Projection]
  \label{def:ti}
  A function $\tip$ is called a \emph{transitive projection}
  if the following holds for all transitions $\tau \in \QF(\Sigma)$ and all $\sigma \models \tau$:
  \begin{minipage}[t]{0.49\textwidth}
    \begin{enumerate}
    \item $\tip(\tau,\sigma)$ is consistent with $\sigma$
    \item $\{\tip(\tau,\theta) \mid \theta \models \tau\}$ is finite
    \end{enumerate}
  \end{minipage}
  \begin{minipage}[t]{0.49\textwidth}
    \begin{enumerate}
      \setcounter{enumi}{2}
    \item $\to_{\tip(\tau,\sigma)}$ is transitive
    \end{enumerate}
  \end{minipage}
\end{definition}

\begin{example}
  \label{ex:Transition Invariants}
  For \Cref{ex:ex1}, $\tau_\ti \Def x' - x \doteq y' - y$ over-approximates the transitive
  closure $\to^+_\tau$.
  %
  Such over-approximations are also called \emph{transition invariants} \cite{transition_invariants}.
  %
  With $\tau_\ti$, one can prove safety for any $\psi_\init$ with $\psi_\init \models x \doteq y$, as then $\psi_\init \land \tau_\ti \models x' \doteq y'$, which shows that no error state with $w \doteq 1 \land x \leq 0 \land y > 0$ is reachable.

  By using $\mbip$, TRL instead considers $\tau_\inc$ and $\tau_\dec$ separately and learns
  \begin{align*}
    \tip(\tau_\inc,\sigma_\inc) & {} \Def w \doteq 0 \land x' > x \land x' - x \doteq y' - y                   \tag{$\tau^+_{\inc}$} \\
    \tip(\tau_\dec,\sigma_\dec) & {} \Def w' \doteq w \land w \doteq 1 \land x' < x \land x' - x \doteq y' - y \tag{$\tau^+_{\dec}$}
  \end{align*}
  if $\sigma_\inc \models \tau_\inc$ and $\sigma_\dec \models \tau_\dec$.
  %
  In this way, \Cref{alg} can learn disjunctive relations like $\tau^+_{\inc} \lor \tau^+_{\dec}$, even if $\tip$ only yields conjunctive relational formulas (which is true for our current implementation of $\tip$ -- see \Cref{sec:rec} -- but not enforced by \Cref{def:ti}).
\end{example}
%
In contrast to conjunctive variable projections, $\tip(\tau,\sigma)$ may contain extra variables that do not occur in $\tau$ (which will be exploited in \Cref{sec:rec}).
%
Hence, instead of $\sigma \models \tip(\tau,\sigma)$ we require consistency with $\sigma$, i.e., $\sigma(\tip(\tau,\sigma))$ must be satisfiable.

\begin{remark}[Properties of $\tip$]
  \label{remark:properties-tip}
 Due to \Cref{def:ti} (1), our definition of $\tip$ implies
 \[
  \paper{\textstyle}
  \tau \models \exists \vec{y}. \, \bigvee_{\sigma \models \tau} \tip(\tau,\sigma), \quad \text{and thus,}
  \quad
  {\to_{\tau}} \subseteq \bigcup_{\sigma \models \tau}
  {\to_{\tip(\tau,\sigma)}},
\]
where $\vec{y}$ are the extra variables of $\bigvee_{\sigma \models \tau} \tip(\tau,\sigma)$.
%
However, \Cref{def:ti} does \emph{not} ensure
${\to^+_\tau} \subseteq \bigcup_{\sigma \models \tau}
  {\to_{\tip(\tau,\sigma)}}$.
  %
  So there is no guarantee that $\tip$ covers $\to^+_\tau$ entirely, i.e., $\tip$ cannot be used to compute transition invariants, in general.
  % 
  \Cref{def:ti} does not ensure ${\to^+_\tau} \supseteq \bigcup_{\sigma \models \tau}
  {\to_{\tip(\tau,\sigma)}}$ either, as $\tip(\tau,\sigma)$ does not imply $\sigma(\vec{x}) \to^+_\tau \sigma(\vec{x}')$.
\end{remark}

\begin{example}
  \label{Counterex-tip}
  To see that $\tip$ computes no over- or under-approximations, let
  \[
    \tau \Def x' \doteq x + 1 \land y' \doteq y + x.\paper{\pagebreak[3]}
  \]
  Then for all $\sigma \models \tau$, we might have:
  \[
    \tip(\tau,\sigma) =
    \begin{cases}
      x \geq 0 \land x' > x \land y' \geq y, & \text{if } \sigma(x) \geq 0 \\
      x < 0 \land x' > x \land y' < y,   & \text{if } \sigma(x) < 0
    \end{cases}
  \]
  However, $(x \geq 0 \land x' > x \land y' \geq y) \lor (x < 0 \land x' > x \land y' < y)$ is not an over-approximation of $\to^+_\tau$ (i.e., ${\to^+_\tau} \not\subseteq \bigcup_{\sigma \models \tau}
    {\to_{\tip(\tau,\sigma)}}$), as we have, e.g.,
  \[
    (-1,0) \to_\tau (0,-1) \to_\tau (1,-1) \to_\tau (2,0), \qquad \text{but} \qquad (-1,0) \not\to_{\tip(\tau,\sigma)} (2,0)
  \]
  for all $\sigma \models \tau$.
  % 
  Moreover, we also have ${\to^+_\tau} \not\supseteq \bigcup_{\sigma \models \tau} {\to_{\tip(\tau,\sigma)}}$, since
  \[
    (-1,0) \to_{\tip(\tau,\sigma)} (10,-20), \qquad \text{but} \qquad (-1,0) \not\to^+_\tau (10,-20)
  \]
  if $\sigma(x) < 0$.
  %
  In contrast to $\tip(\tau, \sigma)$, linear over-approximations for $\to^+_\tau$ like $x' > x$
  cannot distinguish whether $y$ increases or decreases.
\end{example}

As TRL proves safety via \emph{blocking clauses} (\Cref{sec:block}) that only block steps that are cov\-er\-ed by learned relations, the fact that $\tip$ does not yield over-ap\-prox\-i\-ma\-tions does not affect soundness.
%
However, it may cause divergence (\Cref{remark:termination}).

Recall that our SMT encoding forces $\ind[\id]{x}{i}$ to be the identifier of the relation that\linebreak is used for the $i^{th}$ step (\Cref{alg:unroll}).
%
To exploit transitivity of $\tip$, we add the constraint
$\ind[\id]{x}{b} \doteq 1 \lor \ind[\id]{x}{b} \not\doteq \ind[\id]{x}{b-1}$ in
\Cref{alg:trans}, so that learned relations (with an index $>1$) are not used several times in a row, since this is unnecessary for transitive relations.

Clearly, the specifics of $\tip$ depend on the underlying theory.
%
Our implementation for quantifier-free linear integer arithmetic will be explained in \Cref{sec:rec}.


\subsection{Blocking Clauses}
\label{sec:block}

In \Cref{alg:pick}, we are guaranteed to find a learned relation $\pi$ which is consistent
with $\sigma_\Loop$: If our sufficient criterion for non-redundancy in
\Cref{alg:redundant} failed, then the existence of $\pi$ is guaranteed.
%
Otherwise, we learned a new relation $\pi$ in \Cref{alg:learn2} which is consistent with $\sigma_\Loop \subseteq \sigma \circ \mu_{s,\ell}$ by definition of $\tip$.
%
Thus, we can use $\pi$ and a model $\overline{\sigma} \supseteq \sigma_\Loop$ of $\pi$ to record a \emph{blocking clause} in \Cref{alg:block2}.
%
\begin{definition}[Blocking Clauses]
  \label{def:blocking}
  We define:
  \[
    \blockingclause(s,\ell,\pi,\overline{\sigma}) \Def
    \begin{cases}
      \mu_{s,\ell}(\neg \mbip(\pi, \overline{\sigma})) \lor \ind[\id]{x}{s} > 1, & \text{if } \ell = 1 \\
      \mu_{s,\ell}(\neg \mbip(\pi, \overline{\sigma})),                         & \text{if } \ell > 1
    \end{cases}
  \]
\end{definition}
%
Here, $s$ and $\ell$ are natural numbers such that $[\tau_i]_{i=s}^{s+\ell-1}$ is a (possibly) redundant loop on the trace.
%
Blocking clauses exclude models that correspond to runs
\begin{equation}
  \label{blockedRun}
  \vec{v}_1 \to_{\tau_1} \ldots \to_{\tau_{s-1}} \vec{v}_s \to_{\tau_{s}} \ldots \to_{\tau_{s+\ell-1}} \vec{v}_{s+\ell}
\end{equation}
where $\vec{v}_{s} \to_{\pi} \vec{v}_{s+\ell}$. Intuitively, if $\ell = 1$ then
$\blockingclause(s,\ell,\pi,\overline{\sigma})$ states that one may still evaluate 
$\vec{v}_s$ to
$\vec{v}_{s+\ell}$, but one has to use a learned transition. If $\ell > 1$, then
$\blockingclause(s,\ell,\pi,\overline{\sigma})$
states that one may still  evaluate 
$\vec{v}_s$ to
$\vec{v}_{s+\ell}$, but not in $\ell$ steps.
More precisely, 
blocking clauses take into account that%

\vspace{-0.8em}
\noindent
\begin{minipage}{0.44\textwidth}
  \begin{equation}
    \label{prefix}
    \vec{v}_1 \to_{\tau_1} \ldots \to_{\tau_{s+\ell-2}} \vec{v}_{s+\ell-1}
  \end{equation}
\end{minipage}
\begin{minipage}{0.1\textwidth}
  \begin{equation*}
    \text{and}
  \end{equation*}
\end{minipage}
\begin{minipage}{0.44\textwidth}
  \begin{equation}
    \label{unblockedRun}
    \vec{v}_1 \to_{\tau_1} \ldots \to_{\tau_{s-1}} \vec{v}_s \to_{\pi} \vec{v}_{s+\ell}
  \end{equation}
\end{minipage}

\medskip
\noindent
must not be blocked to ensure that $\vec{v}_2,\ldots,\vec{v}_{s+\ell}$ remain reachable.
%
For the former, note that blocking clauses affect the suffix $\vec{v}_{s} \to_{\tau_s} \ldots \to_{\tau_{s+\ell-1}} \vec{v}_{s+\ell}$ of \eqref{blockedRun} (as they contain $\mu_{s,\ell}(\neg \mbip(\pi, \overline{\sigma}))$), but not \eqref{prefix}, so $\vec{v}_{2},\ldots,\vec{v}_{s+\ell-1}$ remain reachable.

Regarding \eqref{unblockedRun},\paper{ \pagebreak[3]} first consider the case $\ell > 1$.
%
Then \eqref{unblockedRun} is not affected by the blocking clause, as it requires less than $s+\ell$ steps.
%
If $\ell = 1$, then the loop that needs to be blocked is a single \emph{original transition} (i.e., a transition that results from applying $\mbip$ to $\tau$) due to \Cref{remark:loops}.
%
So $\ind[\id]{x}{s} > 1$ is falsified by \eqref{blockedRun}, as $\tau_s$ is an original transition, i.e., using it for the $s^{th}$ step implies $\ind[\id]{x}{s} \doteq 1$.
%
However, $\ind[\id]{x}{s} > 1$ is satisfied by \eqref{unblockedRun}, as $\pi$ is a learned
transition, so using it implies
$\ind[\id]{x}{s} > 1$.

\begin{remark}[Extra Variables and Negation]
In \Cref{def:blocking}, $\mbip$ is used to project $\pi$ according to the model $\overline{\sigma}$.
%
In this way, negation has the intended effect, i.e., 
\[
  [\vec{x}/\vec{v},\vec{x}'/\vec{v}'] \models \neg\mbip(\ldots) \qquad \text{iff} \qquad \vec{v} \not\to_{\mbip(\ldots)} \vec{v}',
\]
as $\mbip(\ldots)$ has no extra variables.
%
To see why this is important here, consider the relation $\tau \Def n > 0 \land x' \doteq x + n$, where $n$ is an extra variable.
%
Then $0 \to_\tau 1$, but
\[
  \neg\tau[x/0,x'/1] = (n \leq 0 \lor x' \not\doteq x + n)[x/0,x'/1] = n \leq 0 \lor 1 \not\doteq n
\]
is satisfiable, so $\neg\tau$ is not a suitable characterization of $\not\to_\tau$.
%
The reason is that $n$ is implicitly existentially quantified in $\tau$.
%
So to characterize $\not\to_\tau$, we have to negate $\exists n.\ \tau$ instead of $\tau$, resulting in $\forall n.\ n \leq 0 \lor x' \not\doteq x + n$.
%
Then, as desired,
\[
  (\forall n.\ n \leq 0 \lor x' \not\doteq x + n)[x/0,x'/1] = \forall n.\ n \leq 0 \lor 1 \not\doteq n
\]
is invalid.
%
To avoid quantifiers, we eliminate extra variables via $\mbip$ instead.
\end{remark}

In \Cref{alg:block2}, a pair consisting of $s + \ell - 1$ and the blocking clause is added to $\blocked$.
%
The first component means that the blocking clause has to be added to the SMT encoding when the transition relation is unrolled for the $(s+\ell-1)^{th}$ time, i.e., when $b = s + \ell - 1$.
%
So blocking clauses are added to the SMT encoding ``on demand'' (in \Cref{alg:block1}) to block
loops that have been found on the trace at some point.
%
Afterwards, TRL backtracks to the last step before the loop in \Cref{alg:backtrack}.

\begin{remark}[Adding Blocking Clauses]
To see why blocking clauses must only be added to the SMT encoding
in the $(s+\ell-1)^{th}$ unrolling, assume $\pi \equiv \top$.
%
Then, e.g., $\blockingclause(1,2,\pi,\overline{\sigma}) \equiv \bot$.
%
This means that unrolling the transition relation twice is superfluous, as every state can be reached in a single step with $\pi$, so the diameter is $1$.
%
But after learning $\pi$ when $b=2$ and backtracking to $b=0$, adding such a blocking clause too early (e.g., before the first unrolling of the transition relation) would \emph{immediately} result in an unsatisfiable SMT problem.
\end{remark}


\begin{example}[Blocking Redundant Loops]
  \label{ex:redundant}
  Consider the model
  \[
    \sigma \Def [\ind{w}{1}/\ind{x}{1}/\ind{y}{1}/0,\ind[\id]{x}{1}/2, \quad
      \ind{w}{2}/0,\ind{x}{2}/\ind{y}{2}/2,\ind[\id]{x}{2}/1, \quad
      \ind{x}{3}/\ind{y}{3}/3]
  \]
  and assume that TRL has already learned the relation $\tau^+_\inc$ (i.e., $\vec{\pi} = [\tau,\tau^+_\inc]$).
  %
  Moreover, assume that the trace is $[\tau^+_\inc,\tau_\inc]$, so that TRL detects the loop $\tau_\inc$.
  %
  To check for non-redundancy, we instantiate the pre- and post-variables in $\tau^+_\inc$ according to $\sigma$, taking the renaming $\mu_{2}$ into account (note that here $s = 2$, $\ell = 1$, and $\mu_{s,\ell} = \mu_{2,1} = \mu_2$):
  \[
    \sigma(\mu_{2}(\tau^+_\inc)) = \tau^+_\inc[w/0,x/y/2,x'/y'/3] \equiv \top.
  \]
  So our sufficient criterion for non-redundancy fails, as $\tau_\inc$ is indeed redundant w.r.t.\ $\tau^+_\inc$.
  %
  Thus, TRL records that the following blocking clause has to be added for the second unrolling (i.e., when $b = s + \ell -1 = 2$).
  \paper{\begin{align*}
            & \mu_{s,\ell}(\neg\mbip(\tau^+_\inc,\overline{\sigma})) \lor \ind[\id]{x}{s}
      > 1 \hspace{.4em} = \hspace{.4em} \mu_{s,\ell}(\neg\tau^+_\inc) \lor \ind[\id]{x}{s}
      > 1 \tag{as $\tau^+_\inc$ is a transition} \end{align*}
    \paper{ \vspace*{-.3cm}\pagebreak[3]}
   \begin{align*}  
   {} = {} & \mu_{2}(\neg (w \doteq 0 \land x' > x \land x' - x \doteq y' - y)) \lor \ind[\id]{x}{2} > 1\\ 
    {} = {} & (w \not\doteq 0 \lor x' \leq x \lor x' - x \not\doteq y' - y)[w/\ind{w}{2}, x/\ind{x}{2},y/\ind{y}{2}, x'/\ind{x}{3},y'/\ind{y}{3}] \lor \ind[\id]{x}{2} > 1 \\
    {} = {} & \ind{w}{2} \not\doteq 0 \lor \ind{x}{3} \leq \ind{x}{2} \lor \ind{x}{3} - \ind{x}{2} \not\doteq \ind{y}{3} - \ind{y}{2} \lor \ind[\id]{x}{2} >1
  \end{align*}}
  \report{\begin{align*}
            & \mu_{s,\ell}(\neg\mbip(\tau^+_\inc,\overline{\sigma})) \lor \ind[\id]{x}{s} > 1 \hspace{.4em} = \hspace{.4em} \mu_{s,\ell}(\neg\tau^+_\inc) \lor \ind[\id]{x}{s} > 1 \tag{as $\tau^+_\inc$ is a transition} \\
    {} = {} & \mu_{2}(\neg (w \doteq 0 \land x' > x \land x' - x \doteq y' - y)) \lor
    \ind[\id]{x}{2} > 1\\
    {} = {} & (w \not\doteq 0 \lor x' \leq x \lor x' - x \not\doteq y' - y)[w/\ind{w}{2}, x/\ind{x}{2},y/\ind{y}{2}, x'/\ind{x}{3},y'/\ind{y}{3}] \lor \ind[\id]{x}{2} > 1 \\
    {} = {} & \ind{w}{2} \not\doteq 0 \lor \ind{x}{3} \leq \ind{x}{2} \lor \ind{x}{3} - \ind{x}{2} \not\doteq \ind{y}{3} - \ind{y}{2} \lor \ind[\id]{x}{2} >1
  \end{align*}}  
  As this blocking clause is falsified by $\sigma$, it prevents TRL from finding the same model again after backtracking in \Cref{alg:backtrack}, so that TRL makes progress.
\end{example}
%
The following theorem states that our approach is sound.
%
%\paper{See \cite{arxiv} for all proofs.}%
%JG I mentioned that in the intro now.
%
\begin{restatable}
  {theorem}{soundness}
  \label{thm:soundness}
  If $\text{TRL}(\TT)$ returns $\safe$, then $\TT$ is safe.
\end{restatable}
\makeproof*{thm:soundness}{
  \soundness*
  \begin{proof}
    Consider the SMT problem that is checked in \Cref{alg:safe}.
    %
    In the $m^{th}$ iteration (starting with $m=1$), this problem is of the form $\varphi(m) \Def$
    \begin{align*}
       & \overbrace{\mu_{1}(\psi_\init)}^{\substack{\text{initial states}                                             \\
      \text{\Cref{alg:init}}}} \land                                                                                                 \\
                      & \overbrace{\bigwedge_{j=1}^{b(m)}}^{\substack{\text{one conjunct}                                          \\
        \text{per step}}}
      \left( \overbrace{(\ind[\id]{x}{j} \doteq 1 \lor \ind[\id]{x}{j} \not\doteq \ind[\id]{x}{j-1})}^{\substack{\text{transitivity} \\
          \text{\Cref{alg:trans}}}}
      \land \overbrace{\bigvee_{i=1}^{\ell'(m)} \mu_{j}(\pi_i \land x_{\id} \doteq i)}^{\substack{\text{transition relation}       \\
          \text{\Cref{alg:unroll}}}} \land \overbrace{\bigwedge_{(j,\pi) \in \blocked(m)}
      \pi}^{\mathclap{\substack{\text{blocking clauses}                                                                              \\
            \text{\Cref{alg:block1}}}}} \right)
    \end{align*}
    where $b(m)$, $\ell'(m)$, and $\blocked(m)$ are the value of $b$, the length of $\vec{\pi}$, and the values of $\blocked$ in \Cref{alg:safe} in the $m^{th}$ iteration, respectively.
    %
    For simplicity, here we use an additional variable $\ind[\id]{x}{0}$ which only occurs in the first transitivity constraint (which is not generated by \Cref{alg}):
    \[
      \ind[\id]{x}{1} \doteq 1 \lor \ind[\id]{x}{1} \not\doteq \ind[\id]{x}{0}
    \]
    Therefore, this constraint is trivially satisfiable (e.g., by setting $\ind[\id]{x}{0}$ to $1$).

    Assume that $\TT$ is unsafe, but \Cref{alg} returns $\safe$.
    %
    Then there is some $k \in \NN$ such that $\varphi(k)$ is unsatisfiable, and $\varphi(k')$ is satisfiable for all $1 \leq k' < k$ (otherwise, \Cref{alg} would have returned $\safe$ in an earlier iteration).

    As \Cref{alg} backtracks in \Cref{alg:backtrack}, we consider the sequence of natural numbers $1 \leq i_1 < \ldots < i_{b(k)} = k$ such that for all $1 \leq c \leq b(k)$, $i_c$ is the last iteration where $b=c$ in \Cref{alg:safe}.
    %
    Then for all $1 \leq B \leq b(k)$, we have $\varphi(i_B) = \phi(B)$ where
    \begin{align*}
      \phi(B) \Def & \mu_{1}(\psi_\init) \land \\
                   & \bigwedge_{j=1}^{B}
      \left( (\ind[\id]{x}{j} \doteq 1 \lor \ind[\id]{x}{j} \not\doteq \ind[\id]{x}{j-1}) \land \bigvee_{i=1}^{\ell(B)}\mu_{j}(\pi_i \land x_{\id} \doteq i) \land \bigwedge_{(j,\pi) \in \blocked(k)} \pi \right).
    \end{align*}
    Here, we have $\ell(B) \Def \ell'(i_B)$.
    %
    A blocking clause $\pi$ is only added to the SMT encoding if $(j,\pi) \in \blocked$ and $b=j$ (see \Cref{alg:block1}) and \Cref{alg} backtracks until $b \leq j$ whenever such an element is added to $\blocked$ (see \Cref{alg:backtrack}).
    %
    So when only considering the iterations $i_1, \ldots, i_{b(k)}$, then $b(i_B) = B$ and all blocking clauses of the form $(j,\pi) \in \blocked(k)$ where $j < B$ are already present when unrolling the transition relation for the $B^{th}$ time, i.e., they are already contained in $\blocked(i_B)$.

    %
    In other words, we have
    \[
      \{(j,\pi) \mid (j,\pi) \in \blocked(k) \mid j < B\} \subseteq \blocked(i_B).
    \]
    Thus, we may use $\blocked(k)$ instead of $\blocked(i_B)$ in the definition of
    $\phi$.

    Let $c \in \NN$ and $\vec{v}_0,\ldots\vec{v}_c \in \CC^d$ be arbitrary but fixed where $[\vec{x}/\vec{v}_0] \models \psi_\init$ and
    %
    \[
      \vec{v}_0 \to_{\tau} \ldots \to_{\tau} \vec{v}_c.
    \]
    %
    We use induction on $c$ to show that\footnote{While $\phi(B)$ only corresponds to a formula that is checked by \Cref{alg} if $B > 0$, $\phi(0) \equiv \mu_1(\psi_\init)$ is well defined, too.}
    \begin{multline}
      \label{eq:goal}
      \forall 0 \leq i \leq c.\ \exists B(i) < b(k), h(0,i) < \ldots < h(B(i),i).\ h(0,i) = 0 \land h(B(i),i) = i\\
      {} \land \phi(B(i)) \text{ is consistent with } [\mu_{j+1}(\vec{x})/\vec{v}_{h(j,i)} \mid 0 \leq j \leq B(i)].
    \end{multline}
    Intuitively, $B(i)$ is the number of steps that are needed to reach $\vec{v}_i$ when also using learned relations, and $\vec{v}_{h(j,i)}$ is the $j^{th}$ state in the resulting run that leads to $\vec{v}_i$.
    %
    Thus, \eqref{eq:goal} shows that for all (arbitrary long) runs that start in a state satifying $\psi_\init$, all states that are reachable with $\to_\tau$ in arbitrarily many steps can also be reached in less than $b(k)$ steps (i.e., in constantly many steps) if one may also use the (transitive) learned relations.
    %
    Of course, this only holds provided that \Cref{alg} returns $\safe$ in the $k^{th}$ iteration.

    Once we have shown \eqref{eq:goal}, we can prove the theorem:
    %
    We had assumed that $\TT$ is unsafe, i.e., that there is a reachable error state $\vec{v}_c$ and that $\varphi(k) = \varphi(i_{b(k)}) = \phi(b(k))$ is unsatisfiable.
    %
    We use \eqref{eq:goal} for $i = c$:
    %
    By \eqref{eq:goal}, there is some $B(c) < b(k)$ such that
    \[
      \phi(B(c)) \text{ is consistent with } [\mu_{j+1}(\vec{x})/\vec{v}_{h(j,c)} \mid 0 \leq j \leq B(c)].
    \]
    Moreover, as $\vec{v}_c$ is an error state, $\psi_\err$ is consistent with $[\vec{x}/\vec{v}_{c}]$ and hence,
    \[
      \mu_{B(c)+1}(\psi_\err) \text{ is consistent with } [\mu_{B(c)+1}(\vec{x})/\vec{v}_{c}] = [\mu_{B(c)+1}(\vec{x})/\vec{v}_{h(B(c),c)}].
    \]
    Thus,
    \[
      \phi(B(c)) \land \mu_{B(c)+1}(\psi_\err) \text{ is consistent with } [\mu_{j+1}(\vec{x})/\vec{v}_{h(j,c)} \mid 0 \leq j \leq B(c)].
    \]
    Hence,
    \begin{equation}
      \label{eq:contradiction}
      \text{\Cref{alg} returns $\unknown$ in \Cref{alg:err2} in iteration $i_{B(c)+1}$}
    \end{equation}
    or earlier.
    %
    The reason is that we have $b = B(c)+1$ in iteration $i_{B(c)+1}$, so in this iteration \Cref{alg} checks satisfiability of the formula $\phi(B(c)) \land \mu_{B(c)+1}(\psi_\err)$ in \Cref{alg:err2}.
    %
    As we have $b(k) > B(c)$, we get $k = i_{b(k)} \geq i_{B(c)+1}$.
    %
    Hence, \eqref{eq:contradiction} contradicts the assumption that \Cref{alg} returns $\safe$ in \Cref{alg:safe} in iteration $k$. 

    We now prove \eqref{eq:goal}. 
    In the induction base, we have
    \begin{align*}
                             & [\vec{x}/\vec{v}_0] \models \psi_\init                                                     \\
      {} \curvearrowright {} & [\mu_{1}(\vec{x})/\vec{v}_0] \models \mu_{1}(\psi_\init)                                   \\
      {} \curvearrowright {} & [\mu_{1}(\vec{x})/\vec{v}_0] \models \phi(0) \tag{as $\phi(0) \equiv \mu_{1}(\psi_\init)$}
    \end{align*}
    Hence, the claim follows for
    \[
      B(0) = 0 = h(0,0).
    \]
    In the induction step, the induction hypothesis implies
    \begin{multline}
      \label{eq:IH}
      \forall 0 \leq i < c.\ \exists B(i) < b(k), h(0,i) < \ldots < h(B(i),i).\ h(0,i) = 0 \land h(B(i),i) = i\\
      {} \land \phi(B(i)) \text{ is consistent with } [\mu_{j+1}(\vec{x})/\vec{v}_{h(j,i)} \mid 0 \leq j \leq B(i)].
    \end{multline}
    Let:
    \begin{align*}
      I & {} \Def \min \{i \mid 0 \leq i < c, 1 \leq j \leq \ell(B(i)+1), \vec{v}_i \to_{\pi_j} \vec{v}_c\} \\
      J & {} \Def \max \{j \mid 1 \leq j \leq \ell(B(I)+1), \vec{v}_I \to_{\pi_j} \vec{v}_c\}
    \end{align*}
    So $I$ is the minimal index such that we can make a step from $\vec{v}_I$ to $\vec{v}_c$, and among the relational formulas that can be used for this step, $\pi_J$ is the one that was learned last.
    %
    Note that $I$ (and hence also $J$) exists, as we have $\vec{v}_{c-1} \to_{\tau} \vec{v}_c$ and $\tau = \pi_1$.
    %
    By \eqref{eq:IH},
    \[
      \phi(B(I)) \text{ is consistent with } [\mu_{j+1}(\vec{x})/\vec{v}_{h(j,I)} \mid 0 \leq j \leq B(I)].
    \]
    Let $\theta_I$ be an extension of $[\mu_{j+1}(\vec{x})/\vec{v}_{h(j,I)} \mid 0 \leq j \leq B(I)]$ such that $\theta_I \models \phi(B(I))$, and let $\theta$ be an extension of
    \begin{equation}
      \label{eq:theta}
      \theta_I \uplus [\mu_{B(I)+2}(\vec{x})/\vec{v}_c] \uplus [\mu_{B(I)+1}(x_{\id}) / J]
    \end{equation}
    such that $\theta \models \mu_{B(I)+1}(\pi_J)$, which exists as we have:
    \begin{align*}
      & \theta(\mu_{B(I)+1}(\vec{x})) \\
      {} = {} & \theta_I(\mu_{B(I)+1}(\vec{x})) \tag{by \eqref{eq:theta}} \\
      {} = {} & \vec{v}_{h(B(I),I)} \tag{def.\ of $\theta_I$} \\
      {} = {} & \vec{v}_I  \tag{as $h(B(I),I) = I$} \\
      {} \to_{\pi_J} {} & \vec{v}_c \tag{def.\ of $J$}\\
      {} = {} & \mu_{B(I)+1}(\vec{x}')[\mu_{B(I)+1}(\vec{x}')/\vec{v}_c] \\
      {} = {} & \mu_{B(I)+1}(\vec{x}')[\mu_{B(I)+2}(\vec{x})/\vec{v}_c] \tag{as $\mu_{B(I)+2}(\vec{x}) = \mu_{B(I)+1}(\vec{x}')$} \\
      {} = {} & \theta(\mu_{B(I)+1}(\vec{x}')) \tag{by \eqref{eq:theta}}
    \end{align*}
    %
    So we have
    \[
      \theta(\ind{\vec{x}}{B(I)+1}) = \theta_I(\ind{\vec{x}}{B(I)+1}) = \vec{v}_{h(B(I),I)} = \vec{v}_I
    \]
    and
    \[
      \theta(\ind{\vec{x}}{B(I)+2}) = \vec{v}_c.
    \]

    We now show $\theta \models \phi(B(I)+1)$.
    %
    Then we get $B(c) = B(I) + 1$, $h(j,c) = h(j,I)$ for all $j \leq B(I)$, and $h(B(c),c) = c$, which finishes the proof of \eqref{eq:goal}.

    Since $\theta$ is an extension of $\theta_I$, we have $\theta \models \phi(B(I))$, so we only need to show
    \[
      \theta \models (\ind[\id]{x}{B(I)+1} \doteq 1 \lor \ind[\id]{x}{B(I)+1} \not\doteq \ind[\id]{x}{B(I)}) \land \bigvee_{i=1}^{\ell(B(I)+1)}\mu_{B(I)+1}(\pi_i \land x_{\id} \doteq i) \land \bigwedge_{\mathclap{(B(I)+1,\pi) \in \blocked(k)}} \pi,
    \]
    i.e., we only need to show that $\theta$ is a model of the last conjunct of $\phi(B(I)+1)$.
    %
    For the disjunction
    \begin{equation}
      \label{eq:disjunction}
      \bigvee_{i=1}^{\ell(B(I)+1)}\mu_{B(I)+1}(\pi_i \land x_{\id} \doteq i),
    \end{equation}
    note that $J \leq \ell(B(I)+1)$.
    %
    Thus, to show $\theta \models \eqref{eq:disjunction}$, it suffices if
    \[
      \theta \models \mu_{B(I)+1}(\pi_J \land x_{\id} \doteq J),
    \]
    which holds by construction of $\theta$.
    %
    Thus, it remains to show
    \[
      \theta \models (\ind[\id]{x}{B(I)+1} \doteq 1 \lor \ind[\id]{x}{B(I)+1} \not\doteq \ind[\id]{x}{B(I)}) \land \bigwedge_{\mathclap{(B(I)+1,\pi) \in \blocked(k)}} \pi.
    \]

    We first consider the disjunction $\ind[\id]{x}{B(I)+1} \doteq 1 \lor \ind[\id]{x}{B(I)+1} \not\doteq \ind[\id]{x}{B(I)}$.
    %
    We show that we always have $J=1$ or $\theta(\ind[\id]{x}{B(I)}) \neq J$.
    %
    Then this disjunction is clearly satisfied by $\theta$ since $\theta(\ind[\id]{x}{B(I)+1}) = J$.
    %
    To see why $J=1$ or $\theta(\ind[\id]{x}{B(I)}) \neq J$ holds, assume that $\theta(\ind[\id]{x}{B(I)}) = J > 1$.
    %
    Then:
    \begin{align*}
                             & \theta \models \phi(B(I)) \tag {as $\theta$ is an extension of $\theta_I$} \\
      {} \curvearrowright {} & \theta \models \mu_{B(I)}(\pi_J \land x_{\id} \doteq J) \tag{as $\theta(\ind[\id]{x}{B(I)}) = J$ by assumption}                                                                                 \\
      {} \curvearrowright {} & \mu_{B(I)}(\pi_J) \text{ is consistent with } [\mu_{j+1}(\vec{x})/\vec{v}_{h(j,I)} \mid 0 \leq j \leq B(I)] \tag{as $[\mu_{j+1}(\vec{x})/\vec{v}_{h(j,I)} \mid 0 \leq j \leq B(I)] \subseteq \theta$} \\
      {} \curvearrowright {} & \mu_{B(I)}(\pi_J) \text{ is consistent with } [\mu_{B(I)}(\vec{x})/\vec{v}_{h(B(I)-1,I)},\mu_{B(I)+1}(\vec{x})/\vec{v}_{h(B(I),I)}] \tag{by instantiating $j$ with $B(I)-1$ and $B(I)$}           \\
      {} \curvearrowright {} & \pi_J \text{ is consistent with } [\vec{x}/\vec{v}_{h(B(I)-1,I)},\vec{x}'/\vec{v}_{h(B(I),I)}]                                                                                                      \\
      {} \curvearrowright {} & \pi_J \text{ is consistent with } [\vec{x}/\vec{v}_{h(B(I)-1,I)},\vec{x}'/\vec{v}_{I}] \tag{as $h(B(I),I) = I$}                                                                                     \\
      {} \curvearrowright {} & \vec{v}_{h(B(I)-1,I)} \to_{\pi_J} \vec{v}_{I}.
    \end{align*}
    By the definition of $I$ and $J$, we also have $\vec{v}_{I} \to_{\pi_J} \vec{v}_{c}$.
    %
    So by transitivity of $\to_{\pi_J}$ (which holds since $J > 1$), we get $\vec{v}_{h(B(I)-1,I)} \to_{\pi_J} \vec{v}_{c}$.
    %
    As we have $h(B(I)-1,I) < I$ by definition of $h$, this contradicts minimality of $I$.
    %
    Note that here is the only point where we need the transitivity of learned relations.

    Therefore, it remains to show
    \[
      \theta \models \bigwedge_{\mathclap{(B(I)+1,\pi) \in \blocked(k)}} \pi.
    \]
    The elements of $\blocked(k)$ have the form $(s + \ell - 1, \blockingclause(s,\ell,\pi_j,\overline{\sigma}))$ where $[\tau_i]_{i=s}^{s+\ell-1}$ is a loop on the trace, $\pi_j \in \tail(\vec{\pi})$ is a learned relation, and $\overline{\sigma} \models \pi_j$.
    %
    Moreover, we have $s + \ell - 1 = B(I) + 1$.
    %
    We now perform a case analysis for the two cases where $\ell = 1$ and $\ell > 1$.

    In Case 1 (where $\ell = 1$), the blocking clause has the form
    $\blockingclause(s,1,\linebreak \pi_j,\overline{\sigma})$ for $s = B(I)+1$.
    %
    Thus, we show that $\theta$ cannot violate a blocking clause of the form
    \[
      \blockingclause(B(I),1,\pi_j,\overline{\sigma}) = \mu_{B(I)+1}(\neg\mbip(\pi_j,\overline{\sigma})) \lor \ind[\id]{x}{B(I)+1} > 1.
    \]
    To see this, we first consider the case where the negation of the first disjunct holds and prove that then the second disjunct is true.
    %
    The reason is that we have:
    \begin{align*}
                             & \theta \models \mu_{B(I)+1}(\mbip(\pi_j, \overline{\sigma}))                                                                                     \\
      {} \curvearrowright {} & \theta \circ \mu_{B(I)+1} \text{ is consistent with } \pi_j \tag{since $\mbip(\pi_j,\overline{\sigma}) \models \pi_j$ by \Cref{def:projections}} \\
      {} \curvearrowright {} & \theta(\mu_{B(I)+1}(\vec{x})) \to_{\pi_j} \theta(\mu_{B(I)+1}(\vec{x}'))                                                                \\
      {} \curvearrowright {} & \theta(\mu_{B(I)+1}(\vec{x})) \to_{\pi_j} \theta(\mu_{B(I)+2}(\vec{x}))                                                                \\
      {} \curvearrowright {} & \vec{v}_{h(B(I),I)} \to_{\pi_j} \vec{v}_{c} \tag{def.\ of $\theta$}                                                                 \\
      {} \curvearrowright {} & \vec{v}_{I} \to_{\pi_j} \vec{v}_{c} \tag{as $h(B(I),I) = I$}
    \end{align*}
    So the step fom $\vec{v}_{I}$ to $\vec{v}_{c}$ can be done by a learned relation $\pi_j$.
    %
    As $\pi_j$ is a learned relation, we have $j > 1$.
    %
    This implies $\theta(\ind[\id]{x}{B(I)+1}) = J > 1$, as $J$ is maximal and hence $J \geq j > 1$.

    Now we consider the case where the negation of the first disjunct does not hold and prove that then the first disjunct is true.
    %
    The reason is that due to the completeness of $\AA$, $\theta \centernot\models \mu_{B(I)+1}(\mbip(\pi_j, \overline{\sigma}))$ implies $\theta \models \mu_{B(I)+1}(\neg\mbip(\pi_j, \overline{\sigma}))$.
    %
    So in this case the blocking clause is satisfied as well.

    In Case 2 (where $\ell > 1$), we show that $\theta$ cannot violate a blocking clause of the form
    \[
      \blockingclause(s,\ell,\pi_j,\overline{\sigma}) = \mu_{s,\ell}(\neg\mbip(\pi_j, \overline{\sigma}))
    \]
    where $s + \ell - 1 = B(I)+1$, i.e., $s + \ell = B(I) + 2$.
    %
    The reason is that we have
    \begin{align*}
                             & \theta \centernot\models \mu_{s,\ell}(\neg\mbip(\pi_j, \overline{\sigma}))                                                                       \\
      {} \curvearrowright {} & \theta \models \mu_{s,\ell}(\mbip(\pi_j, \overline{\sigma})) \tag{as $\AA$ is complete} \\
      {} \curvearrowright {} & \theta \circ \mu_{s,\ell} \text{ is consistent with } \pi_j \tag{since $\mbip(\pi_j,\overline{\sigma}) \models \pi_j$ by \Cref{def:projections}} \\
      {} \curvearrowright {} & \theta(\mu_{s,\ell}(\vec{x})) \to_{\pi_j} \theta(\mu_{s,\ell}(\vec{x}'))                                                              \\
      {} \curvearrowright {} & \vec{v}_{h(s-1,I)} \to_{\pi_j} \vec{v}_{c} \tag{def.\ of $\theta$, as $s+\ell = B(I)+2$}
    \end{align*}
    For the last step, note that
    \begin{align*}
      \theta(\mu_{s,\ell}(\vec{x})) & {} = \theta(\mu_{s}(\vec{x})) = \theta_I(\mu_{s}(\vec{x})) = \vec{v}_{h(s-1,I)} & \text{and} \\
      \theta(\mu_{s,\ell}(\vec{x}')) & {} = \theta(\ind{\vec{x}}{s + \ell}) = \theta(\ind{\vec{x}}{B(I)+2}) = \vec{v}_{c}.
    \end{align*}
    We have $s - 1 < B(I)$ and thus $h(s-1,I) < h(B(I),I) = I$, which contradicts minimality of $I$.
    %
    This finishes the proof of \eqref{eq:goal}.
    \qed
  \end{proof}
}

\begin{remark}[Termination]
  \label{remark:termination}
In general, \Cref{alg} does not terminate, since
 $\tip$ decomposes the relation into finitely many cases and
approximates their transitive closures independently, but
${\to^+_{\tau_\Loop}} \subseteq \bigcup_{\sigma \models \tau_{\Loop}}
{\to_{\tip(\tau_{\Loop},\sigma)}}$ is not guaranteed (\Cref{remark:properties-tip}).
%
To see why this may prevent termination, consider a loop $\tau_{\Loop}$ and assume that there are reachable states
$\vec{v},\vec{v}'$ with $\vec{v} \to_{\tau_\Loop}^+ \vec{v}'$, but $\vec{v}
\not\to_{\tip(\tau_{\Loop},\sigma)} \vec{v}'$ for all models $\sigma$ of
$\tau_\Loop$.
%
Then TRL may find a model that corresponds to a run from $\vec{v}$ to $\vec{v}'$.
%
Unless $\vec{v}$ can be evaluated to $\vec{v}'$ with another learned transition $\pi \notin \{\tip(\tau_{\Loop},\sigma) \mid \sigma \models \tau_{\Loop}\}$ by coincidence, this loop cannot be blocked and TRL learns a new relation.
%
Thus, TRL may keep learning new relations as long as there are loops whose transitive closure is not yet covered by learned relations.

As the elements of $\{\tip(\tau,\sigma) \mid \sigma \models \tau\}$ are independent of each other, a more ``global'' view may help to enforce convergence.
%
We leave that to future work.
\end{remark}

\paper{
  \begin{example}[\Cref{ex:ex1} Finished]
    After learning $\tau^+_\dec$ and $\tau^+_\inc$, the underlying SMT problem becomes unsatisfiable when $b=3$ after adding appropriate blocking clauses, so that $\tau^+_\dec$ and $\tau^+_\inc$ are preferred over $\tau_\dec$ and $\tau_\inc$.
    %
    The reason is that $\tau^+_\dec$ and $\tau^+_\inc$ must not be used twice in a row due to \Cref{alg:trans} of \Cref{alg}, and $\tau^+_\inc$ cannot be used after $\tau^+_\dec$, as it requires $w \doteq 0$, but $\tau^+_\dec$ sets $w$ to $1$.
    %
    Thus, \Cref{alg} returns $\safe$.
    %
    See \cite{arxiv} for a detailed run of
\Cref{alg} 
on \Cref{ex:ex1}.
  \end{example}
}

\report{
\subsection{A Complete Example}
\label{sec:example}

For a complete run of TRL on our example, assume that we obtain the following traces (where the detected loops are underlined):
\begin{enumerate}
  \item \label{it:a}
        $[\underline{\tau_\inc}]$, resulting in the learned relation $\tau^+_\inc$ and the blocking clause $\mu_{1}(\neg\tau^+_\inc \lor x_\id > 1)$ which ensures that if $b = 1$, then we cannot use $\tau_\inc$ but would have to use $\tau^+_\inc$.
  \item \label{it:b}
        $[\underline{\tau_\dec}]$, resulting in the learned relation $\tau^+_\dec$ and the blocking clause $\mu_{1}(\neg\tau^+_\dec \lor x_\id > 1)$ which ensures that if $b = 1$, then we cannot use $\tau_\dec$ but would have to use $\tau^+_\dec$.
  \item \label{it:c}
        $[\tau^+_\inc,\underline{\tau_\inc}]$, resulting in the blocking clause $\mu_{2}(\neg\tau^+_\inc \lor x_\id > 1)$ which ensures that if $b = 2$, then we cannot use $\tau_\inc$.
        %
        Using $\tau^+_\inc$ twice after each other is also not possible due to transitivity (\Cref{alg:trans}).
  \item \label{it:d}
        $[\tau^+_\dec,\underline{\tau_\dec}]$, resulting in the blocking clause $\mu_{2}(\neg\tau^+_\dec \lor x_\id > 1)$ which ensures that if $b = 2$, then we cannot use $\tau_\dec$.
        %
        Using $\tau^+_\dec$ twice after each other is also not possible due to transitivity (\Cref{alg:trans}).
  \item \label{it:e}
        $[\tau^+_\inc, \tau^+_\dec, \underline{\tau_\dec}]$, resulting in the blocking clause $\mu_{3}(\neg\tau^+_\dec \lor x_\id > 1)$ which ensures that if $b = 3$, then we cannot use $\tau_\dec$.
\end{enumerate}
Now we are in the following situation:
\begin{itemize}
  \item The first element of the trace cannot be $\tau_\inc$ or $\tau_\dec$ due to \eqref{it:a} and \eqref{it:b}.
  \item If the first element of the trace is $\tau^+_\inc$, then:
        \begin{itemize}
          \item The second element of the trace cannot be $\tau_\inc$ or $\tau_\dec$ due to \eqref{it:c} and \eqref{it:d}.
          \item The second element of the trace cannot be $\tau^+_\inc$ due to \Cref{alg:trans}.
          \item If the second element of the trace is $\tau^+_\dec$, then:
                \begin{itemize}
                  \item The third element of the trace cannot be $\tau_\inc$ or $\tau^+_\inc$, as $\tau^+_\dec$ sets $w$ to $1$, but $\tau_\inc$ and $\tau^+_\inc$ require $w = 0$.
                  \item The third element of the trace cannot be $\tau_\dec$ due to \eqref{it:e}.
                  \item The third element of the trace cannot be $\tau^+_\dec$ due to \Cref{alg:trans}.
                \end{itemize}
        \end{itemize}
        So this case becomes infeasible with the $3^{rd}$ unrolling of the transition relation.
  \item If the first element of the trace is $\tau^+_\dec$, then:
        \begin{itemize}
          \item The second element of the trace cannot be $\tau_\inc$ or $\tau^+_\inc$, as $\tau^+_\dec$ sets $w$ to $1$, but $\tau_\inc$ and $\tau^+_\inc$ require $w = 0$.
          \item The second element of the trace cannot be $\tau_\dec$ due to \eqref{it:d}.
          \item The second element of the trace cannot be $\tau^+_\dec$ due to \Cref{alg:trans}.
        \end{itemize}
        So this case becomes infeasible with the $2^{nd}$ unrolling of the transition relation.
\end{itemize}
Thus, the underlying SMT problem becomes unsatisfiable when $b=3$, such that $\safe$ is returned in \Cref{alg:safe}.
}

\report{
  \section{Implementing $\tip$ for Linear Integer Arithmetic}
\label{sec:rec}

We now explain how to compute transitive projections for quantifier-free linear integer arithmetic via recurrence analysis.
%
As in SMT-LIB \cite{smtlib}, in our setting linear integer arithmetic also features (in)divisibility predicates of the form $e|t$ (or $e\!{\not|}t$) where $e \in \NN_+$ and $t$ is an integer-valued term.
%
Then we have $\sigma \models e|t$ iff $\sigma(t)$ is a multiple of $e$, and $\sigma \models e\!{\not|}t$, otherwise.

The technique that we use is inspired by the recurrence analysis from \cite{kincaid15}.
%
However, there are some important differences.
%
The approach from \cite{kincaid15} computes convex hulls to over-approximate disjunctions by conjunctions, and it relies on polyhedral projections.
%
In our setting, we always have a suitable model at hand, so that we can use $\mbip$ instead.
%
Hence, our recurrence analysis can be implemented more efficiently\footnote{The \emph{double description method}, which is popular for computing polyhedral projections and convex hulls, and other state-of-the-art approaches have exponential complexity \cite{dd-exp,fmplex}.
  %
  See \cite{convex-hull} for an easily accessible discussion of the complexity of the double description method.
  %
  In contrast, combining the model based projection from \cite{spacer} with syntactic implicant projection \cite{adcl} yields a polynomial time algorithm for $\mbip$.}.
%
Additionally, our recurrence analysis can handle divisibility predicates, which are not covered in \cite{kincaid15}.

On the other hand, \cite{kincaid15} yields an over-approximation of the transitive closure of the given relation, whereas our approach performs an implicit case analysis (via $\mbip$) and only yields an over-approximation of the transitive closure of one out of finitely many cases.

Moreover, the recurrence analysis from \cite{kincaid15} also discovers non-linear relations, and then uses linearization techniques to eliminate them.
%
For simplicity, our recurrence analysis only derives linear relations so far.
%
However, just like \cite{kincaid15}, we could also derive non-linear relations and linearize them afterwards.
%
Apart from these differences, our technique is analogous to \cite{kincaid15}.

In the sequel, let $\tau$ and $\sigma \models \tau$ be fixed.
%
Our implementation of $\tip(\tau,\sigma)$ first searches for \emph{recurrent literals}, i.e., literals of the form\footnote{W.l.o.g., we assume that literals are never negated, as we can negate the corresponding (in)equalities or divisibility predicates directly instead.
  %
  Furthermore, in our implementation, we replace disequalities $s \not\doteq t$ with $s >
  t \lor s < t$ and eliminate the resulting disjunction via $\mbip$ to obtain a transition
  without disequalities.}
\[
  t \bowtie 0 \text{ or } e|t \quad \text{where} \quad t = \sum_{x \in \vec{x}} c_x
  \cdot (x'-x) + c, \; {\bowtie} \in \{\leq,\geq,<,>,\doteq\},
  \text{ and } c_x,c \in \ZZ.
\]
Hence, these literals provide information about the change of values of variables.
%
To find such literals, we introduce a fresh variable $x_\delta$ for each $x \in \vec{x}$, and we conjoin $x_\delta \doteq x' - x$ to $\tau$, i.e., we compute
\[
  \tau_{\land \delta} \Def \tau \land \bigwedge_{x \in \vec{x}} x_\delta \doteq x' - x.
\]
%
So the value of $x_\delta$ corresponds to the change of $x$ when applying $\tau$.
%
Next, we use $\mbip$ to eliminate all variables but $\{x_\delta \mid x \in \vec{x}\}$ from $\tau_{\land\delta}$, resulting in $\tau_\delta$.
%
More precisely, we have
\[
  \tau_\delta \Def \mbp(\tau_{\land \delta}, \quad \sigma \uplus [x_\delta/\sigma(x'-x) \mid x \in \vec{x}], \quad \{x_\delta \mid x \in \vec{x}\}).
\]
Finally, to obtain a formula where all literals are recurrent, we replace each $x_\delta$ by its definition, i.e., we compute
\[
  \tau_\rec \Def \tau_\delta[x_\delta / x' - x \mid x \in \vec{x}].
\]
%
\begin{example}[Finding Recurrent Literals]
  \label{ex:finding-rec}
  Consider the transition $\tau_\dec$.
  %
  We first construct the formula
  \[
    \tau_{\land \delta} \Def \tau_\dec \land w_\delta \doteq w' - w \land x_\delta \doteq x' - x \land y_\delta \doteq y' - y.
  \]
  Then for any model $\sigma \models \tau_\dec$, we get\footnote{In the case of $\tau_\dec$, we obtain the same formula $\tau_\delta$ for \emph{every} model $\sigma \models \tau_\dec$, as variables can simply be eliminated by propagating equalities.}
  \begin{align*}
    \tau_\delta \Def {} & \mbp(\tau_{\land \delta}, \quad \sigma \uplus [w_\delta/0, \; x_\delta/{-}1, \; y_\delta/{-}1], \quad \{w_\delta, x_\delta, y_\delta\}) \\
                {} = {} & w_\delta \doteq 0 \land x_\delta \doteq -1 \land y_\delta \doteq -1.
  \end{align*}
  Next, replacing $w_\delta,x_\delta$, and $y_\delta$ with their definition results in
  \begin{align*}
    \tau_\rec \Def {} & w' - w \doteq 0 \land x' - x \doteq -1 \land y' - y \doteq -1        \\
    \equiv {}         & w' - w \doteq 0 \land x' - x + 1 \doteq 0 \land y' - y + 1 \doteq 0.
  \end{align*}
\end{example}
%
Then the construction of $\tip(\tau,\sigma)$ proceeds as follows:
%
\begin{itemize}
  \item $\tip(\tau,\sigma)$ contains the literal $n > 0$, where $n \in \VV$ is a fresh extra variable
  \item for each literal $\sum_{x \in \vec{x}} c_x \cdot (x'-x) + c \bowtie 0$ of $\tau_\rec$, $\tip(\tau,\sigma)$ contains the literal $\sum_{x \in \vec{x}} c_x \cdot (x'-x) + n \cdot c \bowtie 0$
  \item for each literal $e|\sum_{x \in \vec{x}} c_x \cdot (x'-x) + c$ of $\tau_\rec$, $\tip(\tau,\sigma)$ contains the literal $e|\sum_{x \in \vec{x}} c_x \cdot (x'-x) + n \cdot c$
\end{itemize}
%
Intuitively, the extra variable $n$ can be thought of as a ``loop counter'', i.e., when
$n$ is instantiated with some constant $k$, then the literals above approximate the change
of variables when $\to_\tau$ is applied $k$ times.

\begin{example}[Computing $\tip$ (1)]
  \label{ex:tip1}
  Continuing \Cref{ex:finding-rec}, $\tip(\tau_\dec,\sigma)$ contains the literals
  \begin{align*}
    {}           & n > 0 \land w' - w + n \cdot 0 \doteq 0 \land x' - x + n \cdot 1 \doteq 0 \land y' - y + n \cdot 1 \doteq 0 \\
    {} \equiv {} & n > 0 \land w' \doteq w \land x' \doteq x - n \land y' \doteq y - n
  \end{align*}
  for any model $\sigma \models \tau_\dec$.
  %
  Note that in our example, this formula precisely characterizes the change of the variables after $n$ iterations of $\to_{\tau_\dec}$.
  %
  To simplify the formula above, we can propagate the equality $n = x - x'$, resulting in:
  \begin{align}
    & x - x' > 0 \land w' \doteq w \land y' \doteq y - x + x' \notag \\
    {} \equiv {} & w' \doteq w \land x' < x \land x' - x \doteq y' - y \label{eq:rel}
  \end{align}
\end{example}
%
Compared to $\tau_\dec^+$, \eqref{eq:rel} lacks the literal $w \doteq 1$.
%
To incorporate information about the pre- and post-variables (but not about their relation) we conjoin $\mbp(\tau,\sigma,\vec{x})$ and $\mbp(\tau,\sigma,\vec{x}')$ to $\tip(\tau,\sigma)$.

\begin{example}[Computing $\tip$ (2)]
  We finish \Cref{ex:tip1} by conjoining
  \[
    \mbp(\tau_\dec,\sigma,\{w,x,y\}) = w \doteq 1 \qquad \text{and} \qquad \mbp(\tau_\dec,\sigma,\{w',x',y'\}) = w' \doteq 1
  \]
  to \eqref{eq:rel}, resulting in:
  \[
    w' \doteq w \land x' < x \land x' - x \doteq y' - y \land w \doteq 1 \land w' \doteq 1 \quad \equiv \quad \tau_\dec^+
  \]
\end{example}

\begin{example}[Divisibility]
  To see how $\tip$ can handle divisibility constraints, consider the transition
  \[
    \tau \Def 2|x \land 3|x' - x + 1.
  \]
  Then our approach identifies the recurrent literal $\tau_\rec = 3|x' - x + 1$, so that $\tip(\tau,\sigma)$ contains the literal $3|x' - x + n$.
  %
  To see why we conjoin this literal to $\tip(\tau,\sigma)$, note that $3|x'-x+1$ and
  $3|x'-x+n$ are equivalent to $x'-x + 1 \equiv_3 0$ and $x'-x + n \equiv_3 0$, respectively, where ``$\equiv_3$'' denotes congruence modulo $3$.
  %
  So $e \to^n_{\tau} e'$ implies $e'-e+n \equiv_3 0$, just like $e \to^n_{\pi} e'$ implies
  $e'-e+n=0$ for $\pi \Def x'-x+1 \doteq 0$.
  Moreover, we have $\mbp(\tau,\sigma,\vec{x}) = 2|x$, and thus
  \[
    \tip(\tau,\sigma) = n>0 \land 3|x'-x+n \land 2|x.
  \]
\end{example}

\newcounter{tip}
\setcounter{tip}{\value{theorem}}
\begin{theorem}[Correctness of $\tip$]
  \label{thm:tip}
  The function $\tip$ as defined above is a transitive projection.
\end{theorem}
\makeproof*{thm:tip}{
  \setcounter{theorem}{\thetip}
  \begin{theorem}[Correctness of $\tip$]
    \label{thm:Correctness_tip}
    The function $\tip$ as defined in \Cref{sec:rec} is a transitive projection.
  \end{theorem}
  \begin{proof}
    We have to prove that
    \begin{itemize}
      \item[(a)] $\tip(\tau,\sigma)$ is consistent with $\sigma$,
      \item[(b)] $\{\tip(\tau,\theta) \mid \theta \models \tau\}$ is finite, and
      \item[(c)] $\to_{\tip(\tau,\sigma)}$ is transitive
    \end{itemize}
    for all transitions $\tau$ and all $\sigma \models \tau$.

    For item (a), we first prove $\sigma \models \tau_\rec$.
    %
    We have:
    \begin{align*}
                             & \sigma \models \tau                                                                                                          \\
      {} \curvearrowright {} & \sigma \uplus [x_\delta / \sigma(x'-x) \mid x \in \vec{x}] \models \tau_{\land\delta} \tag{by def.\ of $\tau_{\land\delta}$} \\
      {} \curvearrowright {} & [x_\delta / \sigma(x'-x) \mid x \in \vec{x}] \models \tau_\delta \tag{by def.\ of $\mbp$ and $\tau_{\delta}$}                \\
      {} \curvearrowright {} & [x/\sigma(x),x'/\sigma(x') \mid x \in \vec{x}] \models \tau_\rec \tag{by def.\ of $\tau_{\rec}$}                             \\
      {} \curvearrowright {} & \sigma \models \tau_\rec
    \end{align*}
    %
    Now we prove
    $\sigma' \Def \sigma \uplus [n/1] \models \tip(\tau,\sigma)$.
    %
    To this end, we consider the literals that are added to $\tip(\tau,\sigma)$ by the procedure described in \Cref{sec:rec} independently:
    %
    \begin{itemize}
      \item $\sigma' \models n > 0$ is trivial
      \item if $\sigma \models (\sum_{x \in \vec{x}} c_x \cdot (x'-x) + c) \bowtie 0$, then $\sigma' \models (\sum_{x \in \vec{x}} c_x \cdot (x'-x) + n \cdot c) \bowtie 0$
      \item if $\sigma \models e|(\sum_{x \in \vec{x}} c_x \cdot (x'-x) + c)$, then $\sigma' \models e|(\sum_{x \in \vec{x}} c_x \cdot (x'-x) + n \cdot c)$
    \end{itemize}
    Moreover, we have $\sigma \models \mbp(\tau,\sigma,\vec{x})$ and $\sigma \models \mbp(\tau,\sigma,\vec{x}')$ by definition of $\mbp$, and hence we also have $\sigma' \models \mbp(\tau,\sigma,\vec{x})$ and $\sigma' \models \mbp(\tau,\sigma,\vec{x}')$.
    %
    Therefore, $\sigma'$ is a model of $\tip(\tau,\sigma)$, so $\sigma$ is consistent with $\tip(\tau,\sigma)$.

    For item (b), note that $\tip$ only uses the provided model for computing conjunctive variable projections.
    %
    As $\mbp$ has a finite image, the claim follows.

    For item (c), note that the conjuncts $\mbp(\tau,\sigma,\vec{x})$ and $\mbp(\tau,\sigma,\vec{x}')$ of $\tip(\tau,\sigma)$ are irrelevant for transitivity, as they do not relate $\vec{x}$ and $\vec{x}'$.
    %
    Let $\tau' \Def \tip(\tau,\sigma)$.
    %
    It suffices to prove
    \[
      {\to^2_{\tau'}} \subseteq {\to_{\tau'}}.
    \]
    Then the claim follows from a straightforward induction.
    %
    Assume $\theta \models \tau'$, $\theta' \models \tau'$, and
    \[
      \theta(\vec{x}) \to_{\tau'} \theta(\vec{x}') = \theta'(\vec{x}) \to \theta'(\vec{x}').
    \]
    We prove $\hat{\theta} \models \tau'$ where
    \[
      \hat{\theta} \Def [\vec{x} / \theta(\vec{x})] \uplus [\vec{x}' / \theta'(\vec{x}')] \uplus [n / \theta(n) + \theta'(n)].
    \]
    Then the claim follows.
    %
    Again, we consider all literals independently:
    \begin{itemize}
      \item If $\theta \models n > 0$ and $\theta' \models n > 0$, then $\hat{\theta} \models n > 0$.
      \item Consider a literal of the form $\iota \Def t \bowtie 0$ where $t = \sum_{x \in \vec{x}} c_x \cdot (x'-x) + n \cdot c$.
            %
            We have
            %
            \begin{align*}
              \theta(t) = {} & \sum_{x \in \vec{x}} c_x \cdot (\theta(x')-\theta(x)) + \theta(n) \cdot c                                                                               \\
              {} = {}        & \sum_{x \in \vec{x}} c_x \cdot \theta(x') - \sum_{x \in \vec{x}} c_x \cdot \theta(x) + \theta(n) \cdot c                                                \\
              {} = {}        & \sum_{x \in \vec{x}} c_x \cdot \theta'(x) - \sum_{x \in \vec{x}} c_x \cdot \theta(x) + \theta(n) \cdot c \tag{as $\theta(\vec{x}') = \theta'(\vec{x})$}
            \end{align*}
            and
            \begin{align*}
              \theta'(t) = {} & \sum_{x \in \vec{x}} c_x \cdot (\theta'(x')-\theta'(x)) + \theta'(n) \cdot c                                 \\
              {} = {}         & \sum_{x \in \vec{x}} c_x \cdot \theta'(x') - \sum_{x \in \vec{x}} c_x \cdot \theta'(x) + \theta'(n) \cdot c.
            \end{align*}
            Thus, we have:
            \begin{align*}
              \theta(t) + \theta'(t) = {} & -\sum_{x \in \vec{x}} c_x \cdot \theta(x) + \theta(n) \cdot c + \sum_{x \in \vec{x}} c_x \cdot \theta'(x')+ \theta'(n) \cdot c \\
              {} = {}                     & \sum_{x \in \vec{x}} c_x \cdot \theta'(x') - \sum_{x \in \vec{x}} c_x \cdot \theta(x) + (\theta(n) + \theta'(n)) \cdot c       \\
              {} = {}                     & \sum_{x \in \vec{x}} c_x \cdot \hat{\theta}(x') - \sum_{x \in \vec{x}} c_x \cdot \hat{\theta}(x) + \hat{\theta}(n) \cdot c     \\
              {} = {}                     & \hat{\theta}(t)
            \end{align*}
            Therefore, $\theta \models \iota$ and $\theta' \models \iota$ implies $\hat{\theta} \models \iota$ for all ${\bowtie} \in \{\leq,\geq,<,>,=\}$.
      \item Consider a literal of the form $\iota \Def e|t$ where $t = \sum_{x \in \vec{x}} c_x \cdot (x'-x) + n \cdot c$.
            %
            Then we again obtain
            \[
              \theta(t) + \theta'(t) = \hat{\theta}(t)
            \]
            as above.
            %
            Thus, $\theta \models \iota$ and $\theta' \models \iota$ imply $\hat{\theta} \models \iota$.
            %
            \qed
    \end{itemize}
  \end{proof}
}



  \section{Proving Unsafety}\label{sec:Unsafety}

We now explain how to adapt \Cref{alg} for also proving unsafety.
%
Assume that the satisfiability check in \Cref{alg:err2} is successful.
%
Then an error state is reachable from an initial state via the current trace.
%
However, the trace may contain learned transitions, so this does not imply unsafety of $\TT$.
%
The idea for proving unsafety is to replace learned transitions with \emph{accelerated transitions} that result from applying \emph{acceleration techniques}.
%
\begin{definition}[Acceleration]
  A function $\accel: \QF(\Sigma) \to \QF(\Sigma)$ is called an \emph{acceleration technique} if ${\to_{\accel(\pi)}} \subseteq {\to^+_{\pi}}$ for all relational formulas $\pi$.
\end{definition}
%
So in contrast to TRL's learned transitions, accelerated transitions under-ap\-prox\-i\-mate transitive closures, and hence they are suitable for proving unsafety.
%
For arithmetical theories, acceleration techniques are well studied \cite{kroening13,bozga10,acceleration-calculus} (our implementation uses the technique from \cite{acceleration-calculus}).

For any vector of transition formulas $\vec{\rho} = [\rho_1,\ldots,\rho_k]$, let $\compose(\vec{\rho})$ be a transition formula such that ${\to_{\compose(\vec{\rho})}}$ is the composition of the relations ${\to_{\rho_1}}, \ldots, {\to_{\rho_k}}$.
%
Moreover, for a learned transition $\pi$, we say that $\vec{\tau}_\Loop$ \emph{induced} $\pi$ if we had $\vec{\tau}_\Loop = [\tau_s,\ldots,\tau_{s+\ell-1}]$ in \Cref{alg:loop} when $\pi$ was learned.
%
Let $\succ$ be the total ordering on the elements of the vector $\vec{\pi}$ from \Cref{alg} with $\pi_i \succ \pi_j$ iff $i > j$, i.e., the input formula $\tau$ is minimal w.r.t.\ $\succ$, and a learned relation is smaller than all relations that were learned later.
%
Then for vectors of transitions from $\{\mbip(\pi,\sigma) \mid \pi \in \vec{\pi}, \sigma \models \pi\}$, we define the function $\underapprox$ (which yields an \underline{u}nder-\underline{a}pproximation) as follows:
\begin{align*}
  &\underapprox([\eta_1,\ldots,\eta_k]) \Def \compose([\underapprox(\eta_1),\ldots,\underapprox(\eta_k)])\\
  &\underapprox(\eta) \Def
  \begin{cases}
    \eta & \text{if } \eta \models \tau \\
    \accel(\underapprox(\vec{\tau}_\Loop)) & \text{if } \eta \not\models \tau \text{ and}\\
& \phantom{\text{if }} \text{the loop } \vec{\tau}_\Loop \text{ induced }  \underset{\succ}{\min}\{\pi \in \vec{\pi} \mid \eta \models \pi\}
  \end{cases}
\end{align*}
%
Note that $\underapprox$ is well defined,
 as the following holds for all $\eta' \in \vec{\tau}_\Loop$ in the case $\eta \not\models
 \tau$:
\[
  \min_\succ\{\pi \in \vec{\pi} \mid \eta \models \pi\} \succ \min_\succ\{\pi \in \vec{\pi} \mid \eta' \models \pi\}
\]
%
The reason is that
 $\min_\succ\{\pi \in \vec{\pi} \mid \eta \models \pi\}$
is induced by $\vec{\tau}_\Loop$, and thus the elements of $\vec{\tau}_\Loop$ are
conjunctive variable projections of $\tau$, or of relations that were learned before
$\min_\succ\{\pi \in \vec{\pi} \mid \eta \models \pi\}$.
Hence, when defining $\underapprox(\eta)$, the ``recursive call''
$\underapprox(\vec{\tau}_\Loop)$ only refers to formulas with smaller index in $\vec{\pi}$, i.e., the
recursion ``terminates''.



So $\underapprox$ leaves original transitions unchanged.
%
For learned transitions
$\eta$, it first
computes an under-approximation
of the loop $\vec{\tau}_\Loop$ that induced $\eta$, and then it applies acceleration,
resulting in an under-approximation of the transitive closure.
%
The reason for applying $\underapprox$ before acceleration is that $\vec{\tau}_\Loop$ may again contain learned transitions, which have to be under-approximated first.

To improve \Cref{alg},
instead of returning $\unknown$ in \Cref{alg:err2}, now we first obtain a model $\sigma$ from the SMT solver.
%
Then we compute the current trace $\vec{\tau} = \trace_{b-1}(\sigma,\vec{\pi}) = [\tau_1,\ldots,\tau_{b-1}]$ (note that the trace has length $b-1$, as $b$ was incremented in \Cref{alg:err1}).
%
Next, computing $\pi_\underapprox \Def \underapprox(\vec{\tau})$ yields an under-approximation of the states that are reachable with the current trace, but more importantly, $\pi_\underapprox$ also under-approximates $\to^+_\tau$.
%
The reason is that $\pi_\underapprox$ is constructed from original transitions
by applying $\compose$ (which is exact) and $\accel$ (which yields under-approximations).
%
Hence, we return $\unsafe$ if there is an initial state $\vec{v}$ and an error state $\vec{v}'$ with $\vec{v} \to_{\pi_\underapprox} \vec{v}'$, i.e., if $\psi_\init \land \pi_\underapprox \land \psi_\err[\vec{x}/\vec{x}']$ is satisfiable.

\begin{example}[Proving Unsafety]
  Consider the relation
  \[
    \underbrace{y > 0 \land x' \doteq x + 1 \land y' \doteq y - 1 \land z' \doteq z}_{\tau_>} {} \lor {} \underbrace{y \doteq 0 \land x' \doteq x \land y' \doteq z \land z' \doteq z}_{\tau_=} \tag{$\tau$}
  \]
  with the initial states given by $\psi_\init \Def x \leq 0$ and the error states given
  by
  $\psi_\err \Def x \geq 1000$.
  %
  Assume that TRL obtains the trace $[\tau_>]$ and learns the relation
  \begin{equation}
    \label{eq:unsafe-learned1}
    n > 0 \land y > 0 \land x' \doteq x + n \land y' \doteq y - n \land z' \doteq z \land y' \geq 0. \tag{$\tau_>^+$}
  \end{equation}
  Next, assume that TRL obtains the trace $[\tau_=,\tau_>^+]$ and learns the relation
  \begin{equation}
    \label{eq:unsafe-learned1}
    y \doteq 0 \land z > 0 \land x' > x \land z > y' \land y' \geq 0 \land z' \doteq z. \tag{$\hat{\tau}^+$}
  \end{equation}
  Note that the transitive closure of the loop $[\tau_=,\tau_>^+]$ cannot be expressed precisely with linear arithmetic, and hence we only obtain the inequations above for $x'$ and $y'$.
  %
  Then an error state is reachable with the trace $[\hat{\tau}^+]$ and we\report{
    \pagebreak[3]} get:
  \begin{align*}
    & \underapprox([\hat{\tau}^+]) \\
    {} = {} & \accel(\underapprox([\tau_=,\tau_>^+])) \tag{as $[\tau_=,\tau_>^+]$ induced $\hat{\tau}^+$} \\
    {} = {} & \accel(\compose(\underapprox(\tau_=),\underapprox(\tau_>^+))) \\
    {} = {} & \accel(\compose(\tau_=,\accel(\underapprox([\tau_>])))) \tag{as $\tau_= \models \tau$ and $[\tau_>]$ induced $\tau_>^+$} \\
    {} = {} & \accel(\compose(\tau_=,\accel(\tau_>))) \tag{as $\tau_> \models \tau$}\\
    {} = {} & \accel(\compose(\tau_=,\tau^+_>)) \tag{as ${\to_{\tau_>}^+} = {\to_{\tau^+_>}}$} \\
    {} = {} & \accel(y \doteq 0 \land n > 0 \land z > 0 \land x' \doteq x + n \land y' \doteq z - n \land z' \doteq z \land y' \geq 0) \\
    {} = {} & y \doteq 0 \land m > 0 \land z > 0 \land x' \doteq x + m \cdot z \land y' \doteq 0 \land z' \doteq z \tag{$\check{\tau}^+$}
  \end{align*}
  For the last step, note that $\check{\tau}^+$ precisely characterizes the transitive closure of $\compose(\tau_=,\tau^+_>)$ for the case $n \doteq z$, and hence it is a valid under-approximation.
  %
  Then a model like $[x/y/0, z/1, m/1000, \; 
 x'/1000,  y'/0,z'/1]$ satisfies $\psi_\init \land \check{\tau}^+ \land \psi_\err[\vec{x}/\vec{x}'] = x \leq 0 \land \check{\tau}^+ \land x' \geq 1000$, which proves unsafety.
\end{example}

  \putsec{related}{Related Work}

\noindent \textbf{Efficient Radiance Field Rendering.}
%
The introduction of Neural Radiance Fields (NeRF)~\cite{mil:sri20} has
generated significant interest in efficient 3D scene representation and
rendering for radiance fields.
%
Over the past years, there has been a large amount of research aimed at
accelerating NeRFs through algorithmic or software
optimizations~\cite{mul:eva22,fri:yu22,che:fun23,sun:sun22}, and the
development of hardware
accelerators~\cite{lee:cho23,li:li23,son:wen23,mub:kan23,fen:liu24}.
%
The state-of-the-art method, 3D Gaussian splatting~\cite{ker:kop23}, has
further fueled interest in accelerating radiance field
rendering~\cite{rad:ste24,lee:lee24,nie:stu24,lee:rho24,ham:mel24} as it
employs rasterization primitives that can be rendered much faster than NeRFs.
%
However, previous research focused on software graphics rendering on
programmable cores or building dedicated hardware accelerators. In contrast,
\name{} investigates the potential of efficient radiance field rendering while
utilizing fixed-function units in graphics hardware.
%
To our knowledge, this is the first work that assesses the performance
implications of rendering Gaussian-based radiance fields on the hardware
graphics pipeline with software and hardware optimizations.

%%%%%%%%%%%%%%%%%%%%%%%%%%%%%%%%%%%%%%%%%%%%%%%%%%%%%%%%%%%%%%%%%%%%%%%%%%
\myparagraph{Enhancing Graphics Rendering Hardware.}
%
The performance advantage of executing graphics rendering on either
programmable shader cores or fixed-function units varies depending on the
rendering methods and hardware designs.
%
Previous studies have explored the performance implication of graphics hardware
design by developing simulation infrastructures for graphics
workloads~\cite{bar:gon06,gub:aam19,tin:sax23,arn:par13}.
%
Additionally, several studies have aimed to improve the performance of
special-purpose hardware such as ray tracing units in graphics
hardware~\cite{cho:now23,liu:cha21} and proposed hardware accelerators for
graphics applications~\cite{lu:hua17,ram:gri09}.
%
In contrast to these works, which primarily evaluate traditional graphics
workloads, our work focuses on improving the performance of volume rendering
workloads, such as Gaussian splatting, which require blending a huge number of
fragments per pixel.

%%%%%%%%%%%%%%%%%%%%%%%%%%%%%%%%%%%%%%%%%%%%%%%%%%%%%%%%%%%%%%%%%%%%%%%%%%
%
In the context of multi-sample anti-aliasing, prior work proposed reducing the
amount of redundant shading by merging fragments from adjacent triangles in a
mesh at the quad granularity~\cite{fat:bou10}.
%
While both our work and quad-fragment merging (QFM)~\cite{fat:bou10} aim to
reduce operations by merging quads, our proposed technique differs from QFM in
many aspects.
%
Our method aims to blend \emph{overlapping primitives} along the depth
direction and applies to quads from any primitive. In contrast, QFM merges quad
fragments from small (e.g., pixel-sized) triangles that \emph{share} an edge
(i.e., \emph{connected}, \emph{non-overlapping} triangles).
%
As such, QFM is not applicable to the scenes consisting of a number of
unconnected transparent triangles, such as those in 3D Gaussian splatting.
%
In addition, our method computes the \emph{exact} color for each pixel by
offloading blending operations from ROPs to shader units, whereas QFM
\emph{approximates} pixel colors by using the color from one triangle when
multiple triangles are merged into a single quad.


  \section{Experiments}
\label{sec:experiments}
The experiments are designed to address two key research questions.
First, \textbf{RQ1} evaluates whether the average $L_2$-norm of the counterfactual perturbation vectors ($\overline{||\perturb||}$) decreases as the model overfits the data, thereby providing further empirical validation for our hypothesis.
Second, \textbf{RQ2} evaluates the ability of the proposed counterfactual regularized loss, as defined in (\ref{eq:regularized_loss2}), to mitigate overfitting when compared to existing regularization techniques.

% The experiments are designed to address three key research questions. First, \textbf{RQ1} investigates whether the mean perturbation vector norm decreases as the model overfits the data, aiming to further validate our intuition. Second, \textbf{RQ2} explores whether the mean perturbation vector norm can be effectively leveraged as a regularization term during training, offering insights into its potential role in mitigating overfitting. Finally, \textbf{RQ3} examines whether our counterfactual regularizer enables the model to achieve superior performance compared to existing regularization methods, thus highlighting its practical advantage.

\subsection{Experimental Setup}
\textbf{\textit{Datasets, Models, and Tasks.}}
The experiments are conducted on three datasets: \textit{Water Potability}~\cite{kadiwal2020waterpotability}, \textit{Phomene}~\cite{phomene}, and \textit{CIFAR-10}~\cite{krizhevsky2009learning}. For \textit{Water Potability} and \textit{Phomene}, we randomly select $80\%$ of the samples for the training set, and the remaining $20\%$ for the test set, \textit{CIFAR-10} comes already split. Furthermore, we consider the following models: Logistic Regression, Multi-Layer Perceptron (MLP) with 100 and 30 neurons on each hidden layer, and PreactResNet-18~\cite{he2016cvecvv} as a Convolutional Neural Network (CNN) architecture.
We focus on binary classification tasks and leave the extension to multiclass scenarios for future work. However, for datasets that are inherently multiclass, we transform the problem into a binary classification task by selecting two classes, aligning with our assumption.

\smallskip
\noindent\textbf{\textit{Evaluation Measures.}} To characterize the degree of overfitting, we use the test loss, as it serves as a reliable indicator of the model's generalization capability to unseen data. Additionally, we evaluate the predictive performance of each model using the test accuracy.

\smallskip
\noindent\textbf{\textit{Baselines.}} We compare CF-Reg with the following regularization techniques: L1 (``Lasso''), L2 (``Ridge''), and Dropout.

\smallskip
\noindent\textbf{\textit{Configurations.}}
For each model, we adopt specific configurations as follows.
\begin{itemize}
\item \textit{Logistic Regression:} To induce overfitting in the model, we artificially increase the dimensionality of the data beyond the number of training samples by applying a polynomial feature expansion. This approach ensures that the model has enough capacity to overfit the training data, allowing us to analyze the impact of our counterfactual regularizer. The degree of the polynomial is chosen as the smallest degree that makes the number of features greater than the number of data.
\item \textit{Neural Networks (MLP and CNN):} To take advantage of the closed-form solution for computing the optimal perturbation vector as defined in (\ref{eq:opt-delta}), we use a local linear approximation of the neural network models. Hence, given an instance $\inst_i$, we consider the (optimal) counterfactual not with respect to $\model$ but with respect to:
\begin{equation}
\label{eq:taylor}
    \model^{lin}(\inst) = \model(\inst_i) + \nabla_{\inst}\model(\inst_i)(\inst - \inst_i),
\end{equation}
where $\model^{lin}$ represents the first-order Taylor approximation of $\model$ at $\inst_i$.
Note that this step is unnecessary for Logistic Regression, as it is inherently a linear model.
\end{itemize}

\smallskip
\noindent \textbf{\textit{Implementation Details.}} We run all experiments on a machine equipped with an AMD Ryzen 9 7900 12-Core Processor and an NVIDIA GeForce RTX 4090 GPU. Our implementation is based on the PyTorch Lightning framework. We use stochastic gradient descent as the optimizer with a learning rate of $\eta = 0.001$ and no weight decay. We use a batch size of $128$. The training and test steps are conducted for $6000$ epochs on the \textit{Water Potability} and \textit{Phoneme} datasets, while for the \textit{CIFAR-10} dataset, they are performed for $200$ epochs.
Finally, the contribution $w_i^{\varepsilon}$ of each training point $\inst_i$ is uniformly set as $w_i^{\varepsilon} = 1~\forall i\in \{1,\ldots,m\}$.

The source code implementation for our experiments is available at the following GitHub repository: \url{https://anonymous.4open.science/r/COCE-80B4/README.md} 

\subsection{RQ1: Counterfactual Perturbation vs. Overfitting}
To address \textbf{RQ1}, we analyze the relationship between the test loss and the average $L_2$-norm of the counterfactual perturbation vectors ($\overline{||\perturb||}$) over training epochs.

In particular, Figure~\ref{fig:delta_loss_epochs} depicts the evolution of $\overline{||\perturb||}$ alongside the test loss for an MLP trained \textit{without} regularization on the \textit{Water Potability} dataset. 
\begin{figure}[ht]
    \centering
    \includegraphics[width=0.85\linewidth]{img/delta_loss_epochs.png}
    \caption{The average counterfactual perturbation vector $\overline{||\perturb||}$ (left $y$-axis) and the cross-entropy test loss (right $y$-axis) over training epochs ($x$-axis) for an MLP trained on the \textit{Water Potability} dataset \textit{without} regularization.}
    \label{fig:delta_loss_epochs}
\end{figure}

The plot shows a clear trend as the model starts to overfit the data (evidenced by an increase in test loss). 
Notably, $\overline{||\perturb||}$ begins to decrease, which aligns with the hypothesis that the average distance to the optimal counterfactual example gets smaller as the model's decision boundary becomes increasingly adherent to the training data.

It is worth noting that this trend is heavily influenced by the choice of the counterfactual generator model. In particular, the relationship between $\overline{||\perturb||}$ and the degree of overfitting may become even more pronounced when leveraging more accurate counterfactual generators. However, these models often come at the cost of higher computational complexity, and their exploration is left to future work.

Nonetheless, we expect that $\overline{||\perturb||}$ will eventually stabilize at a plateau, as the average $L_2$-norm of the optimal counterfactual perturbations cannot vanish to zero.

% Additionally, the choice of employing the score-based counterfactual explanation framework to generate counterfactuals was driven to promote computational efficiency.

% Future enhancements to the framework may involve adopting models capable of generating more precise counterfactuals. While such approaches may yield to performance improvements, they are likely to come at the cost of increased computational complexity.


\subsection{RQ2: Counterfactual Regularization Performance}
To answer \textbf{RQ2}, we evaluate the effectiveness of the proposed counterfactual regularization (CF-Reg) by comparing its performance against existing baselines: unregularized training loss (No-Reg), L1 regularization (L1-Reg), L2 regularization (L2-Reg), and Dropout.
Specifically, for each model and dataset combination, Table~\ref{tab:regularization_comparison} presents the mean value and standard deviation of test accuracy achieved by each method across 5 random initialization. 

The table illustrates that our regularization technique consistently delivers better results than existing methods across all evaluated scenarios, except for one case -- i.e., Logistic Regression on the \textit{Phomene} dataset. 
However, this setting exhibits an unusual pattern, as the highest model accuracy is achieved without any regularization. Even in this case, CF-Reg still surpasses other regularization baselines.

From the results above, we derive the following key insights. First, CF-Reg proves to be effective across various model types, ranging from simple linear models (Logistic Regression) to deep architectures like MLPs and CNNs, and across diverse datasets, including both tabular and image data. 
Second, CF-Reg's strong performance on the \textit{Water} dataset with Logistic Regression suggests that its benefits may be more pronounced when applied to simpler models. However, the unexpected outcome on the \textit{Phoneme} dataset calls for further investigation into this phenomenon.


\begin{table*}[h!]
    \centering
    \caption{Mean value and standard deviation of test accuracy across 5 random initializations for different model, dataset, and regularization method. The best results are highlighted in \textbf{bold}.}
    \label{tab:regularization_comparison}
    \begin{tabular}{|c|c|c|c|c|c|c|}
        \hline
        \textbf{Model} & \textbf{Dataset} & \textbf{No-Reg} & \textbf{L1-Reg} & \textbf{L2-Reg} & \textbf{Dropout} & \textbf{CF-Reg (ours)} \\ \hline
        Logistic Regression   & \textit{Water}   & $0.6595 \pm 0.0038$   & $0.6729 \pm 0.0056$   & $0.6756 \pm 0.0046$  & N/A    & $\mathbf{0.6918 \pm 0.0036}$                     \\ \hline
        MLP   & \textit{Water}   & $0.6756 \pm 0.0042$   & $0.6790 \pm 0.0058$   & $0.6790 \pm 0.0023$  & $0.6750 \pm 0.0036$    & $\mathbf{0.6802 \pm 0.0046}$                    \\ \hline
%        MLP   & \textit{Adult}   & $0.8404 \pm 0.0010$   & $\mathbf{0.8495 \pm 0.0007}$   & $0.8489 \pm 0.0014$  & $\mathbf{0.8495 \pm 0.0016}$     & $0.8449 \pm 0.0019$                    \\ \hline
        Logistic Regression   & \textit{Phomene}   & $\mathbf{0.8148 \pm 0.0020}$   & $0.8041 \pm 0.0028$   & $0.7835 \pm 0.0176$  & N/A    & $0.8098 \pm 0.0055$                     \\ \hline
        MLP   & \textit{Phomene}   & $0.8677 \pm 0.0033$   & $0.8374 \pm 0.0080$   & $0.8673 \pm 0.0045$  & $0.8672 \pm 0.0042$     & $\mathbf{0.8718 \pm 0.0040}$                    \\ \hline
        CNN   & \textit{CIFAR-10} & $0.6670 \pm 0.0233$   & $0.6229 \pm 0.0850$   & $0.7348 \pm 0.0365$   & N/A    & $\mathbf{0.7427 \pm 0.0571}$                     \\ \hline
    \end{tabular}
\end{table*}

\begin{table*}[htb!]
    \centering
    \caption{Hyperparameter configurations utilized for the generation of Table \ref{tab:regularization_comparison}. For our regularization the hyperparameters are reported as $\mathbf{\alpha/\beta}$.}
    \label{tab:performance_parameters}
    \begin{tabular}{|c|c|c|c|c|c|c|}
        \hline
        \textbf{Model} & \textbf{Dataset} & \textbf{No-Reg} & \textbf{L1-Reg} & \textbf{L2-Reg} & \textbf{Dropout} & \textbf{CF-Reg (ours)} \\ \hline
        Logistic Regression   & \textit{Water}   & N/A   & $0.0093$   & $0.6927$  & N/A    & $0.3791/1.0355$                     \\ \hline
        MLP   & \textit{Water}   & N/A   & $0.0007$   & $0.0022$  & $0.0002$    & $0.2567/1.9775$                    \\ \hline
        Logistic Regression   &
        \textit{Phomene}   & N/A   & $0.0097$   & $0.7979$  & N/A    & $0.0571/1.8516$                     \\ \hline
        MLP   & \textit{Phomene}   & N/A   & $0.0007$   & $4.24\cdot10^{-5}$  & $0.0015$    & $0.0516/2.2700$                    \\ \hline
       % MLP   & \textit{Adult}   & N/A   & $0.0018$   & $0.0018$  & $0.0601$     & $0.0764/2.2068$                    \\ \hline
        CNN   & \textit{CIFAR-10} & N/A   & $0.0050$   & $0.0864$ & N/A    & $0.3018/
        2.1502$                     \\ \hline
    \end{tabular}
\end{table*}

\begin{table*}[htb!]
    \centering
    \caption{Mean value and standard deviation of training time across 5 different runs. The reported time (in seconds) corresponds to the generation of each entry in Table \ref{tab:regularization_comparison}. Times are }
    \label{tab:times}
    \begin{tabular}{|c|c|c|c|c|c|c|}
        \hline
        \textbf{Model} & \textbf{Dataset} & \textbf{No-Reg} & \textbf{L1-Reg} & \textbf{L2-Reg} & \textbf{Dropout} & \textbf{CF-Reg (ours)} \\ \hline
        Logistic Regression   & \textit{Water}   & $222.98 \pm 1.07$   & $239.94 \pm 2.59$   & $241.60 \pm 1.88$  & N/A    & $251.50 \pm 1.93$                     \\ \hline
        MLP   & \textit{Water}   & $225.71 \pm 3.85$   & $250.13 \pm 4.44$   & $255.78 \pm 2.38$  & $237.83 \pm 3.45$    & $266.48 \pm 3.46$                    \\ \hline
        Logistic Regression   & \textit{Phomene}   & $266.39 \pm 0.82$ & $367.52 \pm 6.85$   & $361.69 \pm 4.04$  & N/A   & $310.48 \pm 0.76$                    \\ \hline
        MLP   &
        \textit{Phomene} & $335.62 \pm 1.77$   & $390.86 \pm 2.11$   & $393.96 \pm 1.95$ & $363.51 \pm 5.07$    & $403.14 \pm 1.92$                     \\ \hline
       % MLP   & \textit{Adult}   & N/A   & $0.0018$   & $0.0018$  & $0.0601$     & $0.0764/2.2068$                    \\ \hline
        CNN   & \textit{CIFAR-10} & $370.09 \pm 0.18$   & $395.71 \pm 0.55$   & $401.38 \pm 0.16$ & N/A    & $1287.8 \pm 0.26$                     \\ \hline
    \end{tabular}
\end{table*}

\subsection{Feasibility of our Method}
A crucial requirement for any regularization technique is that it should impose minimal impact on the overall training process.
In this respect, CF-Reg introduces an overhead that depends on the time required to find the optimal counterfactual example for each training instance. 
As such, the more sophisticated the counterfactual generator model probed during training the higher would be the time required. However, a more advanced counterfactual generator might provide a more effective regularization. We discuss this trade-off in more details in Section~\ref{sec:discussion}.

Table~\ref{tab:times} presents the average training time ($\pm$ standard deviation) for each model and dataset combination listed in Table~\ref{tab:regularization_comparison}.
We can observe that the higher accuracy achieved by CF-Reg using the score-based counterfactual generator comes with only minimal overhead. However, when applied to deep neural networks with many hidden layers, such as \textit{PreactResNet-18}, the forward derivative computation required for the linearization of the network introduces a more noticeable computational cost, explaining the longer training times in the table.

\subsection{Hyperparameter Sensitivity Analysis}
The proposed counterfactual regularization technique relies on two key hyperparameters: $\alpha$ and $\beta$. The former is intrinsic to the loss formulation defined in (\ref{eq:cf-train}), while the latter is closely tied to the choice of the score-based counterfactual explanation method used.

Figure~\ref{fig:test_alpha_beta} illustrates how the test accuracy of an MLP trained on the \textit{Water Potability} dataset changes for different combinations of $\alpha$ and $\beta$.

\begin{figure}[ht]
    \centering
    \includegraphics[width=0.85\linewidth]{img/test_acc_alpha_beta.png}
    \caption{The test accuracy of an MLP trained on the \textit{Water Potability} dataset, evaluated while varying the weight of our counterfactual regularizer ($\alpha$) for different values of $\beta$.}
    \label{fig:test_alpha_beta}
\end{figure}

We observe that, for a fixed $\beta$, increasing the weight of our counterfactual regularizer ($\alpha$) can slightly improve test accuracy until a sudden drop is noticed for $\alpha > 0.1$.
This behavior was expected, as the impact of our penalty, like any regularization term, can be disruptive if not properly controlled.

Moreover, this finding further demonstrates that our regularization method, CF-Reg, is inherently data-driven. Therefore, it requires specific fine-tuning based on the combination of the model and dataset at hand.
}
\paper{
  \input{recurrence_short}
  \input{unsafety_short}
  \input{related_short}
  \input{experiments_short}
}

\bibliographystyle{splncs04}
\paper{
  \bibliography{refs,crossrefs,strings}
}
\report{
  % This must be in the first 5 lines to tell arXiv to use pdfLaTeX, which is strongly recommended.
\pdfoutput=1
% In particular, the hyperref package requires pdfLaTeX in order to break URLs across lines.

\documentclass[11pt]{article}

% Change "review" to "final" to generate the final (sometimes called camera-ready) version.
% Change to "preprint" to generate a non-anonymous version with page numbers.
\usepackage{acl}

% Standard package includes
\usepackage{times}
\usepackage{latexsym}

% Draw tables
\usepackage{booktabs}
\usepackage{multirow}
\usepackage{xcolor}
\usepackage{colortbl}
\usepackage{array} 
\usepackage{amsmath}

\newcolumntype{C}{>{\centering\arraybackslash}p{0.07\textwidth}}
% For proper rendering and hyphenation of words containing Latin characters (including in bib files)
\usepackage[T1]{fontenc}
% For Vietnamese characters
% \usepackage[T5]{fontenc}
% See https://www.latex-project.org/help/documentation/encguide.pdf for other character sets
% This assumes your files are encoded as UTF8
\usepackage[utf8]{inputenc}

% This is not strictly necessary, and may be commented out,
% but it will improve the layout of the manuscript,
% and will typically save some space.
\usepackage{microtype}
\DeclareMathOperator*{\argmax}{arg\,max}
% This is also not strictly necessary, and may be commented out.
% However, it will improve the aesthetics of text in
% the typewriter font.
\usepackage{inconsolata}

%Including images in your LaTeX document requires adding
%additional package(s)
\usepackage{graphicx}
% If the title and author information does not fit in the area allocated, uncomment the following
%
%\setlength\titlebox{<dim>}
%
% and set <dim> to something 5cm or larger.

\title{Wi-Chat: Large Language Model Powered Wi-Fi Sensing}

% Author information can be set in various styles:
% For several authors from the same institution:
% \author{Author 1 \and ... \and Author n \\
%         Address line \\ ... \\ Address line}
% if the names do not fit well on one line use
%         Author 1 \\ {\bf Author 2} \\ ... \\ {\bf Author n} \\
% For authors from different institutions:
% \author{Author 1 \\ Address line \\  ... \\ Address line
%         \And  ... \And
%         Author n \\ Address line \\ ... \\ Address line}
% To start a separate ``row'' of authors use \AND, as in
% \author{Author 1 \\ Address line \\  ... \\ Address line
%         \AND
%         Author 2 \\ Address line \\ ... \\ Address line \And
%         Author 3 \\ Address line \\ ... \\ Address line}

% \author{First Author \\
%   Affiliation / Address line 1 \\
%   Affiliation / Address line 2 \\
%   Affiliation / Address line 3 \\
%   \texttt{email@domain} \\\And
%   Second Author \\
%   Affiliation / Address line 1 \\
%   Affiliation / Address line 2 \\
%   Affiliation / Address line 3 \\
%   \texttt{email@domain} \\}
% \author{Haohan Yuan \qquad Haopeng Zhang\thanks{corresponding author} \\ 
%   ALOHA Lab, University of Hawaii at Manoa \\
%   % Affiliation / Address line 2 \\
%   % Affiliation / Address line 3 \\
%   \texttt{\{haohany,haopengz\}@hawaii.edu}}
  
\author{
{Haopeng Zhang$\dag$\thanks{These authors contributed equally to this work.}, Yili Ren$\ddagger$\footnotemark[1], Haohan Yuan$\dag$, Jingzhe Zhang$\ddagger$, Yitong Shen$\ddagger$} \\
ALOHA Lab, University of Hawaii at Manoa$\dag$, University of South Florida$\ddagger$ \\
\{haopengz, haohany\}@hawaii.edu\\
\{yiliren, jingzhe, shen202\}@usf.edu\\}



  
%\author{
%  \textbf{First Author\textsuperscript{1}},
%  \textbf{Second Author\textsuperscript{1,2}},
%  \textbf{Third T. Author\textsuperscript{1}},
%  \textbf{Fourth Author\textsuperscript{1}},
%\\
%  \textbf{Fifth Author\textsuperscript{1,2}},
%  \textbf{Sixth Author\textsuperscript{1}},
%  \textbf{Seventh Author\textsuperscript{1}},
%  \textbf{Eighth Author \textsuperscript{1,2,3,4}},
%\\
%  \textbf{Ninth Author\textsuperscript{1}},
%  \textbf{Tenth Author\textsuperscript{1}},
%  \textbf{Eleventh E. Author\textsuperscript{1,2,3,4,5}},
%  \textbf{Twelfth Author\textsuperscript{1}},
%\\
%  \textbf{Thirteenth Author\textsuperscript{3}},
%  \textbf{Fourteenth F. Author\textsuperscript{2,4}},
%  \textbf{Fifteenth Author\textsuperscript{1}},
%  \textbf{Sixteenth Author\textsuperscript{1}},
%\\
%  \textbf{Seventeenth S. Author\textsuperscript{4,5}},
%  \textbf{Eighteenth Author\textsuperscript{3,4}},
%  \textbf{Nineteenth N. Author\textsuperscript{2,5}},
%  \textbf{Twentieth Author\textsuperscript{1}}
%\\
%\\
%  \textsuperscript{1}Affiliation 1,
%  \textsuperscript{2}Affiliation 2,
%  \textsuperscript{3}Affiliation 3,
%  \textsuperscript{4}Affiliation 4,
%  \textsuperscript{5}Affiliation 5
%\\
%  \small{
%    \textbf{Correspondence:} \href{mailto:email@domain}{email@domain}
%  }
%}

\begin{document}
\maketitle
\begin{abstract}
Recent advancements in Large Language Models (LLMs) have demonstrated remarkable capabilities across diverse tasks. However, their potential to integrate physical model knowledge for real-world signal interpretation remains largely unexplored. In this work, we introduce Wi-Chat, the first LLM-powered Wi-Fi-based human activity recognition system. We demonstrate that LLMs can process raw Wi-Fi signals and infer human activities by incorporating Wi-Fi sensing principles into prompts. Our approach leverages physical model insights to guide LLMs in interpreting Channel State Information (CSI) data without traditional signal processing techniques. Through experiments on real-world Wi-Fi datasets, we show that LLMs exhibit strong reasoning capabilities, achieving zero-shot activity recognition. These findings highlight a new paradigm for Wi-Fi sensing, expanding LLM applications beyond conventional language tasks and enhancing the accessibility of wireless sensing for real-world deployments.
\end{abstract}

\section{Introduction}

In today’s rapidly evolving digital landscape, the transformative power of web technologies has redefined not only how services are delivered but also how complex tasks are approached. Web-based systems have become increasingly prevalent in risk control across various domains. This widespread adoption is due their accessibility, scalability, and ability to remotely connect various types of users. For example, these systems are used for process safety management in industry~\cite{kannan2016web}, safety risk early warning in urban construction~\cite{ding2013development}, and safe monitoring of infrastructural systems~\cite{repetto2018web}. Within these web-based risk management systems, the source search problem presents a huge challenge. Source search refers to the task of identifying the origin of a risky event, such as a gas leak and the emission point of toxic substances. This source search capability is crucial for effective risk management and decision-making.

Traditional approaches to implementing source search capabilities into the web systems often rely on solely algorithmic solutions~\cite{ristic2016study}. These methods, while relatively straightforward to implement, often struggle to achieve acceptable performances due to algorithmic local optima and complex unknown environments~\cite{zhao2020searching}. More recently, web crowdsourcing has emerged as a promising alternative for tackling the source search problem by incorporating human efforts in these web systems on-the-fly~\cite{zhao2024user}. This approach outsources the task of addressing issues encountered during the source search process to human workers, leveraging their capabilities to enhance system performance.

These solutions often employ a human-AI collaborative way~\cite{zhao2023leveraging} where algorithms handle exploration-exploitation and report the encountered problems while human workers resolve complex decision-making bottlenecks to help the algorithms getting rid of local deadlocks~\cite{zhao2022crowd}. Although effective, this paradigm suffers from two inherent limitations: increased operational costs from continuous human intervention, and slow response times of human workers due to sequential decision-making. These challenges motivate our investigation into developing autonomous systems that preserve human-like reasoning capabilities while reducing dependency on massive crowdsourced labor.

Furthermore, recent advancements in large language models (LLMs)~\cite{chang2024survey} and multi-modal LLMs (MLLMs)~\cite{huang2023chatgpt} have unveiled promising avenues for addressing these challenges. One clear opportunity involves the seamless integration of visual understanding and linguistic reasoning for robust decision-making in search tasks. However, whether large models-assisted source search is really effective and efficient for improving the current source search algorithms~\cite{ji2022source} remains unknown. \textit{To address the research gap, we are particularly interested in answering the following two research questions in this work:}

\textbf{\textit{RQ1: }}How can source search capabilities be integrated into web-based systems to support decision-making in time-sensitive risk management scenarios? 
% \sq{I mention ``time-sensitive'' here because I feel like we shall say something about the response time -- LLM has to be faster than humans}

\textbf{\textit{RQ2: }}How can MLLMs and LLMs enhance the effectiveness and efficiency of existing source search algorithms? 

% \textit{\textbf{RQ2:}} To what extent does the performance of large models-assisted search align with or approach the effectiveness of human-AI collaborative search? 

To answer the research questions, we propose a novel framework called Auto-\
S$^2$earch (\textbf{Auto}nomous \textbf{S}ource \textbf{Search}) and implement a prototype system that leverages advanced web technologies to simulate real-world conditions for zero-shot source search. Unlike traditional methods that rely on pre-defined heuristics or extensive human intervention, AutoS$^2$earch employs a carefully designed prompt that encapsulates human rationales, thereby guiding the MLLM to generate coherent and accurate scene descriptions from visual inputs about four directional choices. Based on these language-based descriptions, the LLM is enabled to determine the optimal directional choice through chain-of-thought (CoT) reasoning. Comprehensive empirical validation demonstrates that AutoS$^2$-\ 
earch achieves a success rate of 95–98\%, closely approaching the performance of human-AI collaborative search across 20 benchmark scenarios~\cite{zhao2023leveraging}. 

Our work indicates that the role of humans in future web crowdsourcing tasks may evolve from executors to validators or supervisors. Furthermore, incorporating explanations of LLM decisions into web-based system interfaces has the potential to help humans enhance task performance in risk control.






\section{Related Work}
\label{sec:relatedworks}

% \begin{table*}[t]
% \centering 
% \renewcommand\arraystretch{0.98}
% \fontsize{8}{10}\selectfont \setlength{\tabcolsep}{0.4em}
% \begin{tabular}{@{}lc|cc|cc|cc@{}}
% \toprule
% \textbf{Methods}           & \begin{tabular}[c]{@{}c@{}}\textbf{Training}\\ \textbf{Paradigm}\end{tabular} & \begin{tabular}[c]{@{}c@{}}\textbf{$\#$ PT Data}\\ \textbf{(Tokens)}\end{tabular} & \begin{tabular}[c]{@{}c@{}}\textbf{$\#$ IFT Data}\\ \textbf{(Samples)}\end{tabular} & \textbf{Code}  & \begin{tabular}[c]{@{}c@{}}\textbf{Natural}\\ \textbf{Language}\end{tabular} & \begin{tabular}[c]{@{}c@{}}\textbf{Action}\\ \textbf{Trajectories}\end{tabular} & \begin{tabular}[c]{@{}c@{}}\textbf{API}\\ \textbf{Documentation}\end{tabular}\\ \midrule 
% NexusRaven~\citep{srinivasan2023nexusraven} & IFT & - & - & \textcolor{green}{\CheckmarkBold} & \textcolor{green}{\CheckmarkBold} &\textcolor{red}{\XSolidBrush}&\textcolor{red}{\XSolidBrush}\\
% AgentInstruct~\citep{zeng2023agenttuning} & IFT & - & 2k & \textcolor{green}{\CheckmarkBold} & \textcolor{green}{\CheckmarkBold} &\textcolor{red}{\XSolidBrush}&\textcolor{red}{\XSolidBrush} \\
% AgentEvol~\citep{xi2024agentgym} & IFT & - & 14.5k & \textcolor{green}{\CheckmarkBold} & \textcolor{green}{\CheckmarkBold} &\textcolor{green}{\CheckmarkBold}&\textcolor{red}{\XSolidBrush} \\
% Gorilla~\citep{patil2023gorilla}& IFT & - & 16k & \textcolor{green}{\CheckmarkBold} & \textcolor{green}{\CheckmarkBold} &\textcolor{red}{\XSolidBrush}&\textcolor{green}{\CheckmarkBold}\\
% OpenFunctions-v2~\citep{patil2023gorilla} & IFT & - & 65k & \textcolor{green}{\CheckmarkBold} & \textcolor{green}{\CheckmarkBold} &\textcolor{red}{\XSolidBrush}&\textcolor{green}{\CheckmarkBold}\\
% LAM~\citep{zhang2024agentohana} & IFT & - & 42.6k & \textcolor{green}{\CheckmarkBold} & \textcolor{green}{\CheckmarkBold} &\textcolor{green}{\CheckmarkBold}&\textcolor{red}{\XSolidBrush} \\
% xLAM~\citep{liu2024apigen} & IFT & - & 60k & \textcolor{green}{\CheckmarkBold} & \textcolor{green}{\CheckmarkBold} &\textcolor{green}{\CheckmarkBold}&\textcolor{red}{\XSolidBrush} \\\midrule
% LEMUR~\citep{xu2024lemur} & PT & 90B & 300k & \textcolor{green}{\CheckmarkBold} & \textcolor{green}{\CheckmarkBold} &\textcolor{green}{\CheckmarkBold}&\textcolor{red}{\XSolidBrush}\\
% \rowcolor{teal!12} \method & PT & 103B & 95k & \textcolor{green}{\CheckmarkBold} & \textcolor{green}{\CheckmarkBold} & \textcolor{green}{\CheckmarkBold} & \textcolor{green}{\CheckmarkBold} \\
% \bottomrule
% \end{tabular}
% \caption{Summary of existing tuning- and pretraining-based LLM agents with their training sample sizes. "PT" and "IFT" denote "Pre-Training" and "Instruction Fine-Tuning", respectively. }
% \label{tab:related}
% \end{table*}

\begin{table*}[ht]
\begin{threeparttable}
\centering 
\renewcommand\arraystretch{0.98}
\fontsize{7}{9}\selectfont \setlength{\tabcolsep}{0.2em}
\begin{tabular}{@{}l|c|c|ccc|cc|cc|cccc@{}}
\toprule
\textbf{Methods} & \textbf{Datasets}           & \begin{tabular}[c]{@{}c@{}}\textbf{Training}\\ \textbf{Paradigm}\end{tabular} & \begin{tabular}[c]{@{}c@{}}\textbf{\# PT Data}\\ \textbf{(Tokens)}\end{tabular} & \begin{tabular}[c]{@{}c@{}}\textbf{\# IFT Data}\\ \textbf{(Samples)}\end{tabular} & \textbf{\# APIs} & \textbf{Code}  & \begin{tabular}[c]{@{}c@{}}\textbf{Nat.}\\ \textbf{Lang.}\end{tabular} & \begin{tabular}[c]{@{}c@{}}\textbf{Action}\\ \textbf{Traj.}\end{tabular} & \begin{tabular}[c]{@{}c@{}}\textbf{API}\\ \textbf{Doc.}\end{tabular} & \begin{tabular}[c]{@{}c@{}}\textbf{Func.}\\ \textbf{Call}\end{tabular} & \begin{tabular}[c]{@{}c@{}}\textbf{Multi.}\\ \textbf{Step}\end{tabular}  & \begin{tabular}[c]{@{}c@{}}\textbf{Plan}\\ \textbf{Refine}\end{tabular}  & \begin{tabular}[c]{@{}c@{}}\textbf{Multi.}\\ \textbf{Turn}\end{tabular}\\ \midrule 
\multicolumn{13}{l}{\emph{Instruction Finetuning-based LLM Agents for Intrinsic Reasoning}}  \\ \midrule
FireAct~\cite{chen2023fireact} & FireAct & IFT & - & 2.1K & 10 & \textcolor{red}{\XSolidBrush} &\textcolor{green}{\CheckmarkBold} &\textcolor{green}{\CheckmarkBold}  & \textcolor{red}{\XSolidBrush} &\textcolor{green}{\CheckmarkBold} & \textcolor{red}{\XSolidBrush} &\textcolor{green}{\CheckmarkBold} & \textcolor{red}{\XSolidBrush} \\
ToolAlpaca~\cite{tang2023toolalpaca} & ToolAlpaca & IFT & - & 4.0K & 400 & \textcolor{red}{\XSolidBrush} &\textcolor{green}{\CheckmarkBold} &\textcolor{green}{\CheckmarkBold} & \textcolor{red}{\XSolidBrush} &\textcolor{green}{\CheckmarkBold} & \textcolor{red}{\XSolidBrush}  &\textcolor{green}{\CheckmarkBold} & \textcolor{red}{\XSolidBrush}  \\
ToolLLaMA~\cite{qin2023toolllm} & ToolBench & IFT & - & 12.7K & 16,464 & \textcolor{red}{\XSolidBrush} &\textcolor{green}{\CheckmarkBold} &\textcolor{green}{\CheckmarkBold} &\textcolor{red}{\XSolidBrush} &\textcolor{green}{\CheckmarkBold}&\textcolor{green}{\CheckmarkBold}&\textcolor{green}{\CheckmarkBold} &\textcolor{green}{\CheckmarkBold}\\
AgentEvol~\citep{xi2024agentgym} & AgentTraj-L & IFT & - & 14.5K & 24 &\textcolor{red}{\XSolidBrush} & \textcolor{green}{\CheckmarkBold} &\textcolor{green}{\CheckmarkBold}&\textcolor{red}{\XSolidBrush} &\textcolor{green}{\CheckmarkBold}&\textcolor{red}{\XSolidBrush} &\textcolor{red}{\XSolidBrush} &\textcolor{green}{\CheckmarkBold}\\
Lumos~\cite{yin2024agent} & Lumos & IFT  & - & 20.0K & 16 &\textcolor{red}{\XSolidBrush} & \textcolor{green}{\CheckmarkBold} & \textcolor{green}{\CheckmarkBold} &\textcolor{red}{\XSolidBrush} & \textcolor{green}{\CheckmarkBold} & \textcolor{green}{\CheckmarkBold} &\textcolor{red}{\XSolidBrush} & \textcolor{green}{\CheckmarkBold}\\
Agent-FLAN~\cite{chen2024agent} & Agent-FLAN & IFT & - & 24.7K & 20 &\textcolor{red}{\XSolidBrush} & \textcolor{green}{\CheckmarkBold} & \textcolor{green}{\CheckmarkBold} &\textcolor{red}{\XSolidBrush} & \textcolor{green}{\CheckmarkBold}& \textcolor{green}{\CheckmarkBold}&\textcolor{red}{\XSolidBrush} & \textcolor{green}{\CheckmarkBold}\\
AgentTuning~\citep{zeng2023agenttuning} & AgentInstruct & IFT & - & 35.0K & - &\textcolor{red}{\XSolidBrush} & \textcolor{green}{\CheckmarkBold} & \textcolor{green}{\CheckmarkBold} &\textcolor{red}{\XSolidBrush} & \textcolor{green}{\CheckmarkBold} &\textcolor{red}{\XSolidBrush} &\textcolor{red}{\XSolidBrush} & \textcolor{green}{\CheckmarkBold}\\\midrule
\multicolumn{13}{l}{\emph{Instruction Finetuning-based LLM Agents for Function Calling}} \\\midrule
NexusRaven~\citep{srinivasan2023nexusraven} & NexusRaven & IFT & - & - & 116 & \textcolor{green}{\CheckmarkBold} & \textcolor{green}{\CheckmarkBold}  & \textcolor{green}{\CheckmarkBold} &\textcolor{red}{\XSolidBrush} & \textcolor{green}{\CheckmarkBold} &\textcolor{red}{\XSolidBrush} &\textcolor{red}{\XSolidBrush}&\textcolor{red}{\XSolidBrush}\\
Gorilla~\citep{patil2023gorilla} & Gorilla & IFT & - & 16.0K & 1,645 & \textcolor{green}{\CheckmarkBold} &\textcolor{red}{\XSolidBrush} &\textcolor{red}{\XSolidBrush}&\textcolor{green}{\CheckmarkBold} &\textcolor{green}{\CheckmarkBold} &\textcolor{red}{\XSolidBrush} &\textcolor{red}{\XSolidBrush} &\textcolor{red}{\XSolidBrush}\\
OpenFunctions-v2~\citep{patil2023gorilla} & OpenFunctions-v2 & IFT & - & 65.0K & - & \textcolor{green}{\CheckmarkBold} & \textcolor{green}{\CheckmarkBold} &\textcolor{red}{\XSolidBrush} &\textcolor{green}{\CheckmarkBold} &\textcolor{green}{\CheckmarkBold} &\textcolor{red}{\XSolidBrush} &\textcolor{red}{\XSolidBrush} &\textcolor{red}{\XSolidBrush}\\
API Pack~\cite{guo2024api} & API Pack & IFT & - & 1.1M & 11,213 &\textcolor{green}{\CheckmarkBold} &\textcolor{red}{\XSolidBrush} &\textcolor{green}{\CheckmarkBold} &\textcolor{red}{\XSolidBrush} &\textcolor{green}{\CheckmarkBold} &\textcolor{red}{\XSolidBrush}&\textcolor{red}{\XSolidBrush}&\textcolor{red}{\XSolidBrush}\\ 
LAM~\citep{zhang2024agentohana} & AgentOhana & IFT & - & 42.6K & - & \textcolor{green}{\CheckmarkBold} & \textcolor{green}{\CheckmarkBold} &\textcolor{green}{\CheckmarkBold}&\textcolor{red}{\XSolidBrush} &\textcolor{green}{\CheckmarkBold}&\textcolor{red}{\XSolidBrush}&\textcolor{green}{\CheckmarkBold}&\textcolor{green}{\CheckmarkBold}\\
xLAM~\citep{liu2024apigen} & APIGen & IFT & - & 60.0K & 3,673 & \textcolor{green}{\CheckmarkBold} & \textcolor{green}{\CheckmarkBold} &\textcolor{green}{\CheckmarkBold}&\textcolor{red}{\XSolidBrush} &\textcolor{green}{\CheckmarkBold}&\textcolor{red}{\XSolidBrush}&\textcolor{green}{\CheckmarkBold}&\textcolor{green}{\CheckmarkBold}\\\midrule
\multicolumn{13}{l}{\emph{Pretraining-based LLM Agents}}  \\\midrule
% LEMUR~\citep{xu2024lemur} & PT & 90B & 300.0K & - & \textcolor{green}{\CheckmarkBold} & \textcolor{green}{\CheckmarkBold} &\textcolor{green}{\CheckmarkBold}&\textcolor{red}{\XSolidBrush} & \textcolor{red}{\XSolidBrush} &\textcolor{green}{\CheckmarkBold} &\textcolor{red}{\XSolidBrush}&\textcolor{red}{\XSolidBrush}\\
\rowcolor{teal!12} \method & \dataset & PT & 103B & 95.0K  & 76,537  & \textcolor{green}{\CheckmarkBold} & \textcolor{green}{\CheckmarkBold} & \textcolor{green}{\CheckmarkBold} & \textcolor{green}{\CheckmarkBold} & \textcolor{green}{\CheckmarkBold} & \textcolor{green}{\CheckmarkBold} & \textcolor{green}{\CheckmarkBold} & \textcolor{green}{\CheckmarkBold}\\
\bottomrule
\end{tabular}
% \begin{tablenotes}
%     \item $^*$ In addition, the StarCoder-API can offer 4.77M more APIs.
% \end{tablenotes}
\caption{Summary of existing instruction finetuning-based LLM agents for intrinsic reasoning and function calling, along with their training resources and sample sizes. "PT" and "IFT" denote "Pre-Training" and "Instruction Fine-Tuning", respectively.}
\vspace{-2ex}
\label{tab:related}
\end{threeparttable}
\end{table*}

\noindent \textbf{Prompting-based LLM Agents.} Due to the lack of agent-specific pre-training corpus, existing LLM agents rely on either prompt engineering~\cite{hsieh2023tool,lu2024chameleon,yao2022react,wang2023voyager} or instruction fine-tuning~\cite{chen2023fireact,zeng2023agenttuning} to understand human instructions, decompose high-level tasks, generate grounded plans, and execute multi-step actions. 
However, prompting-based methods mainly depend on the capabilities of backbone LLMs (usually commercial LLMs), failing to introduce new knowledge and struggling to generalize to unseen tasks~\cite{sun2024adaplanner,zhuang2023toolchain}. 

\noindent \textbf{Instruction Finetuning-based LLM Agents.} Considering the extensive diversity of APIs and the complexity of multi-tool instructions, tool learning inherently presents greater challenges than natural language tasks, such as text generation~\cite{qin2023toolllm}.
Post-training techniques focus more on instruction following and aligning output with specific formats~\cite{patil2023gorilla,hao2024toolkengpt,qin2023toolllm,schick2024toolformer}, rather than fundamentally improving model knowledge or capabilities. 
Moreover, heavy fine-tuning can hinder generalization or even degrade performance in non-agent use cases, potentially suppressing the original base model capabilities~\cite{ghosh2024a}.

\noindent \textbf{Pretraining-based LLM Agents.} While pre-training serves as an essential alternative, prior works~\cite{nijkamp2023codegen,roziere2023code,xu2024lemur,patil2023gorilla} have primarily focused on improving task-specific capabilities (\eg, code generation) instead of general-domain LLM agents, due to single-source, uni-type, small-scale, and poor-quality pre-training data. 
Existing tool documentation data for agent training either lacks diverse real-world APIs~\cite{patil2023gorilla, tang2023toolalpaca} or is constrained to single-tool or single-round tool execution. 
Furthermore, trajectory data mostly imitate expert behavior or follow function-calling rules with inferior planning and reasoning, failing to fully elicit LLMs' capabilities and handle complex instructions~\cite{qin2023toolllm}. 
Given a wide range of candidate API functions, each comprising various function names and parameters available at every planning step, identifying globally optimal solutions and generalizing across tasks remains highly challenging.



\section{Preliminaries}
\label{Preliminaries}
\begin{figure*}[t]
    \centering
    \includegraphics[width=0.95\linewidth]{fig/HealthGPT_Framework.png}
    \caption{The \ourmethod{} architecture integrates hierarchical visual perception and H-LoRA, employing a task-specific hard router to select visual features and H-LoRA plugins, ultimately generating outputs with an autoregressive manner.}
    \label{fig:architecture}
\end{figure*}
\noindent\textbf{Large Vision-Language Models.} 
The input to a LVLM typically consists of an image $x^{\text{img}}$ and a discrete text sequence $x^{\text{txt}}$. The visual encoder $\mathcal{E}^{\text{img}}$ converts the input image $x^{\text{img}}$ into a sequence of visual tokens $\mathcal{V} = [v_i]_{i=1}^{N_v}$, while the text sequence $x^{\text{txt}}$ is mapped into a sequence of text tokens $\mathcal{T} = [t_i]_{i=1}^{N_t}$ using an embedding function $\mathcal{E}^{\text{txt}}$. The LLM $\mathcal{M_\text{LLM}}(\cdot|\theta)$ models the joint probability of the token sequence $\mathcal{U} = \{\mathcal{V},\mathcal{T}\}$, which is expressed as:
\begin{equation}
    P_\theta(R | \mathcal{U}) = \prod_{i=1}^{N_r} P_\theta(r_i | \{\mathcal{U}, r_{<i}\}),
\end{equation}
where $R = [r_i]_{i=1}^{N_r}$ is the text response sequence. The LVLM iteratively generates the next token $r_i$ based on $r_{<i}$. The optimization objective is to minimize the cross-entropy loss of the response $\mathcal{R}$.
% \begin{equation}
%     \mathcal{L}_{\text{VLM}} = \mathbb{E}_{R|\mathcal{U}}\left[-\log P_\theta(R | \mathcal{U})\right]
% \end{equation}
It is worth noting that most LVLMs adopt a design paradigm based on ViT, alignment adapters, and pre-trained LLMs\cite{liu2023llava,liu2024improved}, enabling quick adaptation to downstream tasks.


\noindent\textbf{VQGAN.}
VQGAN~\cite{esser2021taming} employs latent space compression and indexing mechanisms to effectively learn a complete discrete representation of images. VQGAN first maps the input image $x^{\text{img}}$ to a latent representation $z = \mathcal{E}(x)$ through a encoder $\mathcal{E}$. Then, the latent representation is quantized using a codebook $\mathcal{Z} = \{z_k\}_{k=1}^K$, generating a discrete index sequence $\mathcal{I} = [i_m]_{m=1}^N$, where $i_m \in \mathcal{Z}$ represents the quantized code index:
\begin{equation}
    \mathcal{I} = \text{Quantize}(z|\mathcal{Z}) = \arg\min_{z_k \in \mathcal{Z}} \| z - z_k \|_2.
\end{equation}
In our approach, the discrete index sequence $\mathcal{I}$ serves as a supervisory signal for the generation task, enabling the model to predict the index sequence $\hat{\mathcal{I}}$ from input conditions such as text or other modality signals.  
Finally, the predicted index sequence $\hat{\mathcal{I}}$ is upsampled by the VQGAN decoder $G$, generating the high-quality image $\hat{x}^\text{img} = G(\hat{\mathcal{I}})$.



\noindent\textbf{Low Rank Adaptation.} 
LoRA\cite{hu2021lora} effectively captures the characteristics of downstream tasks by introducing low-rank adapters. The core idea is to decompose the bypass weight matrix $\Delta W\in\mathbb{R}^{d^{\text{in}} \times d^{\text{out}}}$ into two low-rank matrices $ \{A \in \mathbb{R}^{d^{\text{in}} \times r}, B \in \mathbb{R}^{r \times d^{\text{out}}} \}$, where $ r \ll \min\{d^{\text{in}}, d^{\text{out}}\} $, significantly reducing learnable parameters. The output with the LoRA adapter for the input $x$ is then given by:
\begin{equation}
    h = x W_0 + \alpha x \Delta W/r = x W_0 + \alpha xAB/r,
\end{equation}
where matrix $ A $ is initialized with a Gaussian distribution, while the matrix $ B $ is initialized as a zero matrix. The scaling factor $ \alpha/r $ controls the impact of $ \Delta W $ on the model.

\section{HealthGPT}
\label{Method}


\subsection{Unified Autoregressive Generation.}  
% As shown in Figure~\ref{fig:architecture}, 
\ourmethod{} (Figure~\ref{fig:architecture}) utilizes a discrete token representation that covers both text and visual outputs, unifying visual comprehension and generation as an autoregressive task. 
For comprehension, $\mathcal{M}_\text{llm}$ receives the input joint sequence $\mathcal{U}$ and outputs a series of text token $\mathcal{R} = [r_1, r_2, \dots, r_{N_r}]$, where $r_i \in \mathcal{V}_{\text{txt}}$, and $\mathcal{V}_{\text{txt}}$ represents the LLM's vocabulary:
\begin{equation}
    P_\theta(\mathcal{R} \mid \mathcal{U}) = \prod_{i=1}^{N_r} P_\theta(r_i \mid \mathcal{U}, r_{<i}).
\end{equation}
For generation, $\mathcal{M}_\text{llm}$ first receives a special start token $\langle \text{START\_IMG} \rangle$, then generates a series of tokens corresponding to the VQGAN indices $\mathcal{I} = [i_1, i_2, \dots, i_{N_i}]$, where $i_j \in \mathcal{V}_{\text{vq}}$, and $\mathcal{V}_{\text{vq}}$ represents the index range of VQGAN. Upon completion of generation, the LLM outputs an end token $\langle \text{END\_IMG} \rangle$:
\begin{equation}
    P_\theta(\mathcal{I} \mid \mathcal{U}) = \prod_{j=1}^{N_i} P_\theta(i_j \mid \mathcal{U}, i_{<j}).
\end{equation}
Finally, the generated index sequence $\mathcal{I}$ is fed into the decoder $G$, which reconstructs the target image $\hat{x}^{\text{img}} = G(\mathcal{I})$.

\subsection{Hierarchical Visual Perception}  
Given the differences in visual perception between comprehension and generation tasks—where the former focuses on abstract semantics and the latter emphasizes complete semantics—we employ ViT to compress the image into discrete visual tokens at multiple hierarchical levels.
Specifically, the image is converted into a series of features $\{f_1, f_2, \dots, f_L\}$ as it passes through $L$ ViT blocks.

To address the needs of various tasks, the hidden states are divided into two types: (i) \textit{Concrete-grained features} $\mathcal{F}^{\text{Con}} = \{f_1, f_2, \dots, f_k\}, k < L$, derived from the shallower layers of ViT, containing sufficient global features, suitable for generation tasks; 
(ii) \textit{Abstract-grained features} $\mathcal{F}^{\text{Abs}} = \{f_{k+1}, f_{k+2}, \dots, f_L\}$, derived from the deeper layers of ViT, which contain abstract semantic information closer to the text space, suitable for comprehension tasks.

The task type $T$ (comprehension or generation) determines which set of features is selected as the input for the downstream large language model:
\begin{equation}
    \mathcal{F}^{\text{img}}_T =
    \begin{cases}
        \mathcal{F}^{\text{Con}}, & \text{if } T = \text{generation task} \\
        \mathcal{F}^{\text{Abs}}, & \text{if } T = \text{comprehension task}
    \end{cases}
\end{equation}
We integrate the image features $\mathcal{F}^{\text{img}}_T$ and text features $\mathcal{T}$ into a joint sequence through simple concatenation, which is then fed into the LLM $\mathcal{M}_{\text{llm}}$ for autoregressive generation.
% :
% \begin{equation}
%     \mathcal{R} = \mathcal{M}_{\text{llm}}(\mathcal{U}|\theta), \quad \mathcal{U} = [\mathcal{F}^{\text{img}}_T; \mathcal{T}]
% \end{equation}
\subsection{Heterogeneous Knowledge Adaptation}
We devise H-LoRA, which stores heterogeneous knowledge from comprehension and generation tasks in separate modules and dynamically routes to extract task-relevant knowledge from these modules. 
At the task level, for each task type $ T $, we dynamically assign a dedicated H-LoRA submodule $ \theta^T $, which is expressed as:
\begin{equation}
    \mathcal{R} = \mathcal{M}_\text{LLM}(\mathcal{U}|\theta, \theta^T), \quad \theta^T = \{A^T, B^T, \mathcal{R}^T_\text{outer}\}.
\end{equation}
At the feature level for a single task, H-LoRA integrates the idea of Mixture of Experts (MoE)~\cite{masoudnia2014mixture} and designs an efficient matrix merging and routing weight allocation mechanism, thus avoiding the significant computational delay introduced by matrix splitting in existing MoELoRA~\cite{luo2024moelora}. Specifically, we first merge the low-rank matrices (rank = r) of $ k $ LoRA experts into a unified matrix:
\begin{equation}
    \mathbf{A}^{\text{merged}}, \mathbf{B}^{\text{merged}} = \text{Concat}(\{A_i\}_1^k), \text{Concat}(\{B_i\}_1^k),
\end{equation}
where $ \mathbf{A}^{\text{merged}} \in \mathbb{R}^{d^\text{in} \times rk} $ and $ \mathbf{B}^{\text{merged}} \in \mathbb{R}^{rk \times d^\text{out}} $. The $k$-dimension routing layer generates expert weights $ \mathcal{W} \in \mathbb{R}^{\text{token\_num} \times k} $ based on the input hidden state $ x $, and these are expanded to $ \mathbb{R}^{\text{token\_num} \times rk} $ as follows:
\begin{equation}
    \mathcal{W}^\text{expanded} = \alpha k \mathcal{W} / r \otimes \mathbf{1}_r,
\end{equation}
where $ \otimes $ denotes the replication operation.
The overall output of H-LoRA is computed as:
\begin{equation}
    \mathcal{O}^\text{H-LoRA} = (x \mathbf{A}^{\text{merged}} \odot \mathcal{W}^\text{expanded}) \mathbf{B}^{\text{merged}},
\end{equation}
where $ \odot $ represents element-wise multiplication. Finally, the output of H-LoRA is added to the frozen pre-trained weights to produce the final output:
\begin{equation}
    \mathcal{O} = x W_0 + \mathcal{O}^\text{H-LoRA}.
\end{equation}
% In summary, H-LoRA is a task-based dynamic PEFT method that achieves high efficiency in single-task fine-tuning.

\subsection{Training Pipeline}

\begin{figure}[t]
    \centering
    \hspace{-4mm}
    \includegraphics[width=0.94\linewidth]{fig/data.pdf}
    \caption{Data statistics of \texttt{VL-Health}. }
    \label{fig:data}
\end{figure}
\noindent \textbf{1st Stage: Multi-modal Alignment.} 
In the first stage, we design separate visual adapters and H-LoRA submodules for medical unified tasks. For the medical comprehension task, we train abstract-grained visual adapters using high-quality image-text pairs to align visual embeddings with textual embeddings, thereby enabling the model to accurately describe medical visual content. During this process, the pre-trained LLM and its corresponding H-LoRA submodules remain frozen. In contrast, the medical generation task requires training concrete-grained adapters and H-LoRA submodules while keeping the LLM frozen. Meanwhile, we extend the textual vocabulary to include multimodal tokens, enabling the support of additional VQGAN vector quantization indices. The model trains on image-VQ pairs, endowing the pre-trained LLM with the capability for image reconstruction. This design ensures pixel-level consistency of pre- and post-LVLM. The processes establish the initial alignment between the LLM’s outputs and the visual inputs.

\noindent \textbf{2nd Stage: Heterogeneous H-LoRA Plugin Adaptation.}  
The submodules of H-LoRA share the word embedding layer and output head but may encounter issues such as bias and scale inconsistencies during training across different tasks. To ensure that the multiple H-LoRA plugins seamlessly interface with the LLMs and form a unified base, we fine-tune the word embedding layer and output head using a small amount of mixed data to maintain consistency in the model weights. Specifically, during this stage, all H-LoRA submodules for different tasks are kept frozen, with only the word embedding layer and output head being optimized. Through this stage, the model accumulates foundational knowledge for unified tasks by adapting H-LoRA plugins.

\begin{table*}[!t]
\centering
\caption{Comparison of \ourmethod{} with other LVLMs and unified multi-modal models on medical visual comprehension tasks. \textbf{Bold} and \underline{underlined} text indicates the best performance and second-best performance, respectively.}
\resizebox{\textwidth}{!}{
\begin{tabular}{c|lcc|cccccccc|c}
\toprule
\rowcolor[HTML]{E9F3FE} &  &  &  & \multicolumn{2}{c}{\textbf{VQA-RAD \textuparrow}} & \multicolumn{2}{c}{\textbf{SLAKE \textuparrow}} & \multicolumn{2}{c}{\textbf{PathVQA \textuparrow}} &  &  &  \\ 
\cline{5-10}
\rowcolor[HTML]{E9F3FE}\multirow{-2}{*}{\textbf{Type}} & \multirow{-2}{*}{\textbf{Model}} & \multirow{-2}{*}{\textbf{\# Params}} & \multirow{-2}{*}{\makecell{\textbf{Medical} \\ \textbf{LVLM}}} & \textbf{close} & \textbf{all} & \textbf{close} & \textbf{all} & \textbf{close} & \textbf{all} & \multirow{-2}{*}{\makecell{\textbf{MMMU} \\ \textbf{-Med}}\textuparrow} & \multirow{-2}{*}{\textbf{OMVQA}\textuparrow} & \multirow{-2}{*}{\textbf{Avg. \textuparrow}} \\ 
\midrule \midrule
\multirow{9}{*}{\textbf{Comp. Only}} 
& Med-Flamingo & 8.3B & \Large \ding{51} & 58.6 & 43.0 & 47.0 & 25.5 & 61.9 & 31.3 & 28.7 & 34.9 & 41.4 \\
& LLaVA-Med & 7B & \Large \ding{51} & 60.2 & 48.1 & 58.4 & 44.8 & 62.3 & 35.7 & 30.0 & 41.3 & 47.6 \\
& HuatuoGPT-Vision & 7B & \Large \ding{51} & 66.9 & 53.0 & 59.8 & 49.1 & 52.9 & 32.0 & 42.0 & 50.0 & 50.7 \\
& BLIP-2 & 6.7B & \Large \ding{55} & 43.4 & 36.8 & 41.6 & 35.3 & 48.5 & 28.8 & 27.3 & 26.9 & 36.1 \\
& LLaVA-v1.5 & 7B & \Large \ding{55} & 51.8 & 42.8 & 37.1 & 37.7 & 53.5 & 31.4 & 32.7 & 44.7 & 41.5 \\
& InstructBLIP & 7B & \Large \ding{55} & 61.0 & 44.8 & 66.8 & 43.3 & 56.0 & 32.3 & 25.3 & 29.0 & 44.8 \\
& Yi-VL & 6B & \Large \ding{55} & 52.6 & 42.1 & 52.4 & 38.4 & 54.9 & 30.9 & 38.0 & 50.2 & 44.9 \\
& InternVL2 & 8B & \Large \ding{55} & 64.9 & 49.0 & 66.6 & 50.1 & 60.0 & 31.9 & \underline{43.3} & 54.5 & 52.5\\
& Llama-3.2 & 11B & \Large \ding{55} & 68.9 & 45.5 & 72.4 & 52.1 & 62.8 & 33.6 & 39.3 & 63.2 & 54.7 \\
\midrule
\multirow{5}{*}{\textbf{Comp. \& Gen.}} 
& Show-o & 1.3B & \Large \ding{55} & 50.6 & 33.9 & 31.5 & 17.9 & 52.9 & 28.2 & 22.7 & 45.7 & 42.6 \\
& Unified-IO 2 & 7B & \Large \ding{55} & 46.2 & 32.6 & 35.9 & 21.9 & 52.5 & 27.0 & 25.3 & 33.0 & 33.8 \\
& Janus & 1.3B & \Large \ding{55} & 70.9 & 52.8 & 34.7 & 26.9 & 51.9 & 27.9 & 30.0 & 26.8 & 33.5 \\
& \cellcolor[HTML]{DAE0FB}HealthGPT-M3 & \cellcolor[HTML]{DAE0FB}3.8B & \cellcolor[HTML]{DAE0FB}\Large \ding{51} & \cellcolor[HTML]{DAE0FB}\underline{73.7} & \cellcolor[HTML]{DAE0FB}\underline{55.9} & \cellcolor[HTML]{DAE0FB}\underline{74.6} & \cellcolor[HTML]{DAE0FB}\underline{56.4} & \cellcolor[HTML]{DAE0FB}\underline{78.7} & \cellcolor[HTML]{DAE0FB}\underline{39.7} & \cellcolor[HTML]{DAE0FB}\underline{43.3} & \cellcolor[HTML]{DAE0FB}\underline{68.5} & \cellcolor[HTML]{DAE0FB}\underline{61.3} \\
& \cellcolor[HTML]{DAE0FB}HealthGPT-L14 & \cellcolor[HTML]{DAE0FB}14B & \cellcolor[HTML]{DAE0FB}\Large \ding{51} & \cellcolor[HTML]{DAE0FB}\textbf{77.7} & \cellcolor[HTML]{DAE0FB}\textbf{58.3} & \cellcolor[HTML]{DAE0FB}\textbf{76.4} & \cellcolor[HTML]{DAE0FB}\textbf{64.5} & \cellcolor[HTML]{DAE0FB}\textbf{85.9} & \cellcolor[HTML]{DAE0FB}\textbf{44.4} & \cellcolor[HTML]{DAE0FB}\textbf{49.2} & \cellcolor[HTML]{DAE0FB}\textbf{74.4} & \cellcolor[HTML]{DAE0FB}\textbf{66.4} \\
\bottomrule
\end{tabular}
}
\label{tab:results}
\end{table*}
\begin{table*}[ht]
    \centering
    \caption{The experimental results for the four modality conversion tasks.}
    \resizebox{\textwidth}{!}{
    \begin{tabular}{l|ccc|ccc|ccc|ccc}
        \toprule
        \rowcolor[HTML]{E9F3FE} & \multicolumn{3}{c}{\textbf{CT to MRI (Brain)}} & \multicolumn{3}{c}{\textbf{CT to MRI (Pelvis)}} & \multicolumn{3}{c}{\textbf{MRI to CT (Brain)}} & \multicolumn{3}{c}{\textbf{MRI to CT (Pelvis)}} \\
        \cline{2-13}
        \rowcolor[HTML]{E9F3FE}\multirow{-2}{*}{\textbf{Model}}& \textbf{SSIM $\uparrow$} & \textbf{PSNR $\uparrow$} & \textbf{MSE $\downarrow$} & \textbf{SSIM $\uparrow$} & \textbf{PSNR $\uparrow$} & \textbf{MSE $\downarrow$} & \textbf{SSIM $\uparrow$} & \textbf{PSNR $\uparrow$} & \textbf{MSE $\downarrow$} & \textbf{SSIM $\uparrow$} & \textbf{PSNR $\uparrow$} & \textbf{MSE $\downarrow$} \\
        \midrule \midrule
        pix2pix & 71.09 & 32.65 & 36.85 & 59.17 & 31.02 & 51.91 & 78.79 & 33.85 & 28.33 & 72.31 & 32.98 & 36.19 \\
        CycleGAN & 54.76 & 32.23 & 40.56 & 54.54 & 30.77 & 55.00 & 63.75 & 31.02 & 52.78 & 50.54 & 29.89 & 67.78 \\
        BBDM & {71.69} & {32.91} & {34.44} & 57.37 & 31.37 & 48.06 & \textbf{86.40} & 34.12 & 26.61 & {79.26} & 33.15 & 33.60 \\
        Vmanba & 69.54 & 32.67 & 36.42 & {63.01} & {31.47} & {46.99} & 79.63 & 34.12 & 26.49 & 77.45 & 33.53 & 31.85 \\
        DiffMa & 71.47 & 32.74 & 35.77 & 62.56 & 31.43 & 47.38 & 79.00 & {34.13} & {26.45} & 78.53 & {33.68} & {30.51} \\
        \rowcolor[HTML]{DAE0FB}HealthGPT-M3 & \underline{79.38} & \underline{33.03} & \underline{33.48} & \underline{71.81} & \underline{31.83} & \underline{43.45} & {85.06} & \textbf{34.40} & \textbf{25.49} & \underline{84.23} & \textbf{34.29} & \textbf{27.99} \\
        \rowcolor[HTML]{DAE0FB}HealthGPT-L14 & \textbf{79.73} & \textbf{33.10} & \textbf{32.96} & \textbf{71.92} & \textbf{31.87} & \textbf{43.09} & \underline{85.31} & \underline{34.29} & \underline{26.20} & \textbf{84.96} & \underline{34.14} & \underline{28.13} \\
        \bottomrule
    \end{tabular}
    }
    \label{tab:conversion}
\end{table*}

\noindent \textbf{3rd Stage: Visual Instruction Fine-Tuning.}  
In the third stage, we introduce additional task-specific data to further optimize the model and enhance its adaptability to downstream tasks such as medical visual comprehension (e.g., medical QA, medical dialogues, and report generation) or generation tasks (e.g., super-resolution, denoising, and modality conversion). Notably, by this stage, the word embedding layer and output head have been fine-tuned, only the H-LoRA modules and adapter modules need to be trained. This strategy significantly improves the model's adaptability and flexibility across different tasks.


\section{Experiment}
\label{s:experiment}

\subsection{Data Description}
We evaluate our method on FI~\cite{you2016building}, Twitter\_LDL~\cite{yang2017learning} and Artphoto~\cite{machajdik2010affective}.
FI is a public dataset built from Flickr and Instagram, with 23,308 images and eight emotion categories, namely \textit{amusement}, \textit{anger}, \textit{awe},  \textit{contentment}, \textit{disgust}, \textit{excitement},  \textit{fear}, and \textit{sadness}. 
% Since images in FI are all copyrighted by law, some images are corrupted now, so we remove these samples and retain 21,828 images.
% T4SA contains images from Twitter, which are classified into three categories: \textit{positive}, \textit{neutral}, and \textit{negative}. In this paper, we adopt the base version of B-T4SA, which contains 470,586 images and provides text descriptions of the corresponding tweets.
Twitter\_LDL contains 10,045 images from Twitter, with the same eight categories as the FI dataset.
% 。
For these two datasets, they are randomly split into 80\%
training and 20\% testing set.
Artphoto contains 806 artistic photos from the DeviantArt website, which we use to further evaluate the zero-shot capability of our model.
% on the small-scale dataset.
% We construct and publicly release the first image sentiment analysis dataset containing metadata.
% 。

% Based on these datasets, we are the first to construct and publicly release metadata-enhanced image sentiment analysis datasets. These datasets include scenes, tags, descriptions, and corresponding confidence scores, and are available at this link for future research purposes.


% 
\begin{table}[t]
\centering
% \begin{center}
\caption{Overall performance of different models on FI and Twitter\_LDL datasets.}
\label{tab:cap1}
% \resizebox{\linewidth}{!}
{
\begin{tabular}{l|c|c|c|c}
\hline
\multirow{2}{*}{\textbf{Model}} & \multicolumn{2}{c|}{\textbf{FI}}  & \multicolumn{2}{c}{\textbf{Twitter\_LDL}} \\ \cline{2-5} 
  & \textbf{Accuracy} & \textbf{F1} & \textbf{Accuracy} & \textbf{F1}  \\ \hline
% (\rownumber)~AlexNet~\cite{krizhevsky2017imagenet}  & 58.13\% & 56.35\%  & 56.24\%& 55.02\%  \\ 
% (\rownumber)~VGG16~\cite{simonyan2014very}  & 63.75\%& 63.08\%  & 59.34\%& 59.02\%  \\ 
(\rownumber)~ResNet101~\cite{he2016deep} & 66.16\%& 65.56\%  & 62.02\% & 61.34\%  \\ 
(\rownumber)~CDA~\cite{han2023boosting} & 66.71\%& 65.37\%  & 64.14\% & 62.85\%  \\ 
(\rownumber)~CECCN~\cite{ruan2024color} & 67.96\%& 66.74\%  & 64.59\%& 64.72\% \\ 
(\rownumber)~EmoVIT~\cite{xie2024emovit} & 68.09\%& 67.45\%  & 63.12\% & 61.97\%  \\ 
(\rownumber)~ComLDL~\cite{zhang2022compound} & 68.83\%& 67.28\%  & 65.29\% & 63.12\%  \\ 
(\rownumber)~WSDEN~\cite{li2023weakly} & 69.78\%& 69.61\%  & 67.04\% & 65.49\% \\ 
(\rownumber)~ECWA~\cite{deng2021emotion} & 70.87\%& 69.08\%  & 67.81\% & 66.87\%  \\ 
(\rownumber)~EECon~\cite{yang2023exploiting} & 71.13\%& 68.34\%  & 64.27\%& 63.16\%  \\ 
(\rownumber)~MAM~\cite{zhang2024affective} & 71.44\%  & 70.83\% & 67.18\%  & 65.01\%\\ 
(\rownumber)~TGCA-PVT~\cite{chen2024tgca}   & 73.05\%  & 71.46\% & 69.87\%  & 68.32\% \\ 
(\rownumber)~OEAN~\cite{zhang2024object}   & 73.40\%  & 72.63\% & 70.52\%  & 69.47\% \\ \hline
(\rownumber)~\shortname  & \textbf{79.48\%} & \textbf{79.22\%} & \textbf{74.12\%} & \textbf{73.09\%} \\ \hline
\end{tabular}
}
\vspace{-6mm}
% \end{center}
\end{table}
% 

\subsection{Experiment Setting}
% \subsubsection{Model Setting.}
% 
\textbf{Model Setting:}
For feature representation, we set $k=10$ to select object tags, and adopt clip-vit-base-patch32 as the pre-trained model for unified feature representation.
Moreover, we empirically set $(d_e, d_h, d_k, d_s) = (512, 128, 16, 64)$, and set the classification class $L$ to 8.

% 

\textbf{Training Setting:}
To initialize the model, we set all weights such as $\boldsymbol{W}$ following the truncated normal distribution, and use AdamW optimizer with the learning rate of $1 \times 10^{-4}$.
% warmup scheduler of cosine, warmup steps of 2000.
Furthermore, we set the batch size to 32 and the epoch of the training process to 200.
During the implementation, we utilize \textit{PyTorch} to build our entire model.
% , and our project codes are publicly available at https://github.com/zzmyrep/MESN.
% Our project codes as well as data are all publicly available on GitHub\footnote{https://github.com/zzmyrep/KBCEN}.
% Code is available at \href{https://github.com/zzmyrep/KBCEN}{https://github.com/zzmyrep/KBCEN}.

\textbf{Evaluation Metrics:}
Following~\cite{zhang2024affective, chen2024tgca, zhang2024object}, we adopt \textit{accuracy} and \textit{F1} as our evaluation metrics to measure the performance of different methods for image sentiment analysis. 



\subsection{Experiment Result}
% We compare our model against the following baselines: AlexNet~\cite{krizhevsky2017imagenet}, VGG16~\cite{simonyan2014very}, ResNet101~\cite{he2016deep}, CECCN~\cite{ruan2024color}, EmoVIT~\cite{xie2024emovit}, WSCNet~\cite{yang2018weakly}, ECWA~\cite{deng2021emotion}, EECon~\cite{yang2023exploiting}, MAM~\cite{zhang2024affective} and TGCA-PVT~\cite{chen2024tgca}, and the overall results are summarized in Table~\ref{tab:cap1}.
We compare our model against several baselines, and the overall results are summarized in Table~\ref{tab:cap1}.
We observe that our model achieves the best performance in both accuracy and F1 metrics, significantly outperforming the previous models. 
This superior performance is mainly attributed to our effective utilization of metadata to enhance image sentiment analysis, as well as the exceptional capability of the unified sentiment transformer framework we developed. These results strongly demonstrate that our proposed method can bring encouraging performance for image sentiment analysis.

\setcounter{magicrownumbers}{0} 
\begin{table}[t]
\begin{center}
\caption{Ablation study of~\shortname~on FI dataset.} 
% \vspace{1mm}
\label{tab:cap2}
\resizebox{.9\linewidth}{!}
{
\begin{tabular}{lcc}
  \hline
  \textbf{Model} & \textbf{Accuracy} & \textbf{F1} \\
  \hline
  (\rownumber)~Ours (w/o vision) & 65.72\% & 64.54\% \\
  (\rownumber)~Ours (w/o text description) & 74.05\% & 72.58\% \\
  (\rownumber)~Ours (w/o object tag) & 77.45\% & 76.84\% \\
  (\rownumber)~Ours (w/o scene tag) & 78.47\% & 78.21\% \\
  \hline
  (\rownumber)~Ours (w/o unified embedding) & 76.41\% & 76.23\% \\
  (\rownumber)~Ours (w/o adaptive learning) & 76.83\% & 76.56\% \\
  (\rownumber)~Ours (w/o cross-modal fusion) & 76.85\% & 76.49\% \\
  \hline
  (\rownumber)~Ours  & \textbf{79.48\%} & \textbf{79.22\%} \\
  \hline
\end{tabular}
}
\end{center}
\vspace{-5mm}
\end{table}


\begin{figure}[t]
\centering
% \vspace{-2mm}
\includegraphics[width=0.42\textwidth]{fig/2dvisual-linux4-paper2.pdf}
\caption{Visualization of feature distribution on eight categories before (left) and after (right) model processing.}
% 
\label{fig:visualization}
\vspace{-5mm}
\end{figure}

\subsection{Ablation Performance}
In this subsection, we conduct an ablation study to examine which component is really important for performance improvement. The results are reported in Table~\ref{tab:cap2}.

For information utilization, we observe a significant decline in model performance when visual features are removed. Additionally, the performance of \shortname~decreases when different metadata are removed separately, which means that text description, object tag, and scene tag are all critical for image sentiment analysis.
Recalling the model architecture, we separately remove transformer layers of the unified representation module, the adaptive learning module, and the cross-modal fusion module, replacing them with MLPs of the same parameter scale.
In this way, we can observe varying degrees of decline in model performance, indicating that these modules are indispensable for our model to achieve better performance.

\subsection{Visualization}
% 


% % 开始使用minipage进行左右排列
% \begin{minipage}[t]{0.45\textwidth}  % 子图1宽度为45%
%     \centering
%     \includegraphics[width=\textwidth]{2dvisual.pdf}  % 插入图片
%     \captionof{figure}{Visualization of feature distribution.}  % 使用captionof添加图片标题
%     \label{fig:visualization}
% \end{minipage}


% \begin{figure}[t]
% \centering
% \vspace{-2mm}
% \includegraphics[width=0.45\textwidth]{fig/2dvisual.pdf}
% \caption{Visualization of feature distribution.}
% \label{fig:visualization}
% % \vspace{-4mm}
% \end{figure}

% \begin{figure}[t]
% \centering
% \vspace{-2mm}
% \includegraphics[width=0.45\textwidth]{fig/2dvisual-linux3-paper.pdf}
% \caption{Visualization of feature distribution.}
% \label{fig:visualization}
% % \vspace{-4mm}
% \end{figure}



\begin{figure}[tbp]   
\vspace{-4mm}
  \centering            
  \subfloat[Depth of adaptive learning layers]   
  {
    \label{fig:subfig1}\includegraphics[width=0.22\textwidth]{fig/fig_sensitivity-a5}
  }
  \subfloat[Depth of fusion layers]
  {
    % \label{fig:subfig2}\includegraphics[width=0.22\textwidth]{fig/fig_sensitivity-b2}
    \label{fig:subfig2}\includegraphics[width=0.22\textwidth]{fig/fig_sensitivity-b2-num.pdf}
  }
  \caption{Sensitivity study of \shortname~on different depth. }   
  \label{fig:fig_sensitivity}  
\vspace{-2mm}
\end{figure}

% \begin{figure}[htbp]
% \centerline{\includegraphics{2dvisual.pdf}}
% \caption{Visualization of feature distribution.}
% \label{fig:visualization}
% \end{figure}

% In Fig.~\ref{fig:visualization}, we use t-SNE~\cite{van2008visualizing} to reduce the dimension of data features for visualization, Figure in left represents the metadata features before model processing, the features are obtained by embedding through the CLIP model, and figure in right shows the features of the data after model processing, it can be observed that after the model processing, the data with different label categories fall in different regions in the space, therefore, we can conclude that the Therefore, we can conclude that the model can effectively utilize the information contained in the metadata and use it to guide the model for classification.

In Fig.~\ref{fig:visualization}, we use t-SNE~\cite{van2008visualizing} to reduce the dimension of data features for visualization.
The left figure shows metadata features before being processed by our model (\textit{i.e.}, embedded by CLIP), while the right shows the distribution of features after being processed by our model.
We can observe that after the model processing, data with the same label are closer to each other, while others are farther away.
Therefore, it shows that the model can effectively utilize the information contained in the metadata and use it to guide the classification process.

\subsection{Sensitivity Analysis}
% 
In this subsection, we conduct a sensitivity analysis to figure out the effect of different depth settings of adaptive learning layers and fusion layers. 
% In this subsection, we conduct a sensitivity analysis to figure out the effect of different depth settings on the model. 
% Fig.~\ref{fig:fig_sensitivity} presents the effect of different depth settings of adaptive learning layers and fusion layers. 
Taking Fig.~\ref{fig:fig_sensitivity} (a) as an example, the model performance improves with increasing depth, reaching the best performance at a depth of 4.
% Taking Fig.~\ref{fig:fig_sensitivity} (a) as an example, the performance of \shortname~improves with the increase of depth at first, reaching the best performance at a depth of 4.
When the depth continues to increase, the accuracy decreases to varying degrees.
Similar results can be observed in Fig.~\ref{fig:fig_sensitivity} (b).
Therefore, we set their depths to 4 and 6 respectively to achieve the best results.

% Through our experiments, we can observe that the effect of modifying these hyperparameters on the results of the experiments is very weak, and the surface model is not sensitive to the hyperparameters.


\subsection{Zero-shot Capability}
% 

% (1)~GCH~\cite{2010Analyzing} & 21.78\% & (5)~RA-DLNet~\cite{2020A} & 34.01\% \\ \hline
% (2)~WSCNet~\cite{2019WSCNet}  & 30.25\% & (6)~CECCN~\cite{ruan2024color} & 43.83\% \\ \hline
% (3)~PCNN~\cite{2015Robust} & 31.68\%  & (7)~EmoVIT~\cite{xie2024emovit} & 44.90\% \\ \hline
% (4)~AR~\cite{2018Visual} & 32.67\% & (8)~Ours (Zero-shot) & 47.83\% \\ \hline


\begin{table}[t]
\centering
\caption{Zero-shot capability of \shortname.}
\label{tab:cap3}
\resizebox{1\linewidth}{!}
{
\begin{tabular}{lc|lc}
\hline
\textbf{Model} & \textbf{Accuracy} & \textbf{Model} & \textbf{Accuracy} \\ \hline
(1)~WSCNet~\cite{2019WSCNet}  & 30.25\% & (5)~MAM~\cite{zhang2024affective} & 39.56\%  \\ \hline
(2)~AR~\cite{2018Visual} & 32.67\% & (6)~CECCN~\cite{ruan2024color} & 43.83\% \\ \hline
(3)~RA-DLNet~\cite{2020A} & 34.01\%  & (7)~EmoVIT~\cite{xie2024emovit} & 44.90\% \\ \hline
(4)~CDA~\cite{han2023boosting} & 38.64\% & (8)~Ours (Zero-shot) & 47.83\% \\ \hline
\end{tabular}
}
\vspace{-5mm}
\end{table}

% We use the model trained on the FI dataset to test on the artphoto dataset to verify the model's generalization ability as well as robustness to other distributed datasets.
% We can observe that the MESN model shows strong competitiveness in terms of accuracy when compared to other trained models, which suggests that the model has a good generalization ability in the OOD task.

To validate the model's generalization ability and robustness to other distributed datasets, we directly test the model trained on the FI dataset, without training on Artphoto. 
% As observed in Table 3, compared to other models trained on Artphoto, we achieve highly competitive zero-shot performance, indicating that the model has good generalization ability in out-of-distribution tasks.
From Table~\ref{tab:cap3}, we can observe that compared with other models trained on Artphoto, we achieve competitive zero-shot performance, which shows that the model has good generalization ability in out-of-distribution tasks.


%%%%%%%%%%%%
%  E2E     %
%%%%%%%%%%%%


\section{Conclusion}
In this paper, we introduced Wi-Chat, the first LLM-powered Wi-Fi-based human activity recognition system that integrates the reasoning capabilities of large language models with the sensing potential of wireless signals. Our experimental results on a self-collected Wi-Fi CSI dataset demonstrate the promising potential of LLMs in enabling zero-shot Wi-Fi sensing. These findings suggest a new paradigm for human activity recognition that does not rely on extensive labeled data. We hope future research will build upon this direction, further exploring the applications of LLMs in signal processing domains such as IoT, mobile sensing, and radar-based systems.

\section*{Limitations}
While our work represents the first attempt to leverage LLMs for processing Wi-Fi signals, it is a preliminary study focused on a relatively simple task: Wi-Fi-based human activity recognition. This choice allows us to explore the feasibility of LLMs in wireless sensing but also comes with certain limitations.

Our approach primarily evaluates zero-shot performance, which, while promising, may still lag behind traditional supervised learning methods in highly complex or fine-grained recognition tasks. Besides, our study is limited to a controlled environment with a self-collected dataset, and the generalizability of LLMs to diverse real-world scenarios with varying Wi-Fi conditions, environmental interference, and device heterogeneity remains an open question.

Additionally, we have yet to explore the full potential of LLMs in more advanced Wi-Fi sensing applications, such as fine-grained gesture recognition, occupancy detection, and passive health monitoring. Future work should investigate the scalability of LLM-based approaches, their robustness to domain shifts, and their integration with multimodal sensing techniques in broader IoT applications.


% Bibliography entries for the entire Anthology, followed by custom entries
%\bibliography{anthology,custom}
% Custom bibliography entries only
\bibliography{main}
\newpage
\appendix

\section{Experiment prompts}
\label{sec:prompt}
The prompts used in the LLM experiments are shown in the following Table~\ref{tab:prompts}.

\definecolor{titlecolor}{rgb}{0.9, 0.5, 0.1}
\definecolor{anscolor}{rgb}{0.2, 0.5, 0.8}
\definecolor{labelcolor}{HTML}{48a07e}
\begin{table*}[h]
	\centering
	
 % \vspace{-0.2cm}
	
	\begin{center}
		\begin{tikzpicture}[
				chatbox_inner/.style={rectangle, rounded corners, opacity=0, text opacity=1, font=\sffamily\scriptsize, text width=5in, text height=9pt, inner xsep=6pt, inner ysep=6pt},
				chatbox_prompt_inner/.style={chatbox_inner, align=flush left, xshift=0pt, text height=11pt},
				chatbox_user_inner/.style={chatbox_inner, align=flush left, xshift=0pt},
				chatbox_gpt_inner/.style={chatbox_inner, align=flush left, xshift=0pt},
				chatbox/.style={chatbox_inner, draw=black!25, fill=gray!7, opacity=1, text opacity=0},
				chatbox_prompt/.style={chatbox, align=flush left, fill=gray!1.5, draw=black!30, text height=10pt},
				chatbox_user/.style={chatbox, align=flush left},
				chatbox_gpt/.style={chatbox, align=flush left},
				chatbox2/.style={chatbox_gpt, fill=green!25},
				chatbox3/.style={chatbox_gpt, fill=red!20, draw=black!20},
				chatbox4/.style={chatbox_gpt, fill=yellow!30},
				labelbox/.style={rectangle, rounded corners, draw=black!50, font=\sffamily\scriptsize\bfseries, fill=gray!5, inner sep=3pt},
			]
											
			\node[chatbox_user] (q1) {
				\textbf{System prompt}
				\newline
				\newline
				You are a helpful and precise assistant for segmenting and labeling sentences. We would like to request your help on curating a dataset for entity-level hallucination detection.
				\newline \newline
                We will give you a machine generated biography and a list of checked facts about the biography. Each fact consists of a sentence and a label (True/False). Please do the following process. First, breaking down the biography into words. Second, by referring to the provided list of facts, merging some broken down words in the previous step to form meaningful entities. For example, ``strategic thinking'' should be one entity instead of two. Third, according to the labels in the list of facts, labeling each entity as True or False. Specifically, for facts that share a similar sentence structure (\eg, \textit{``He was born on Mach 9, 1941.''} (\texttt{True}) and \textit{``He was born in Ramos Mejia.''} (\texttt{False})), please first assign labels to entities that differ across atomic facts. For example, first labeling ``Mach 9, 1941'' (\texttt{True}) and ``Ramos Mejia'' (\texttt{False}) in the above case. For those entities that are the same across atomic facts (\eg, ``was born'') or are neutral (\eg, ``he,'' ``in,'' and ``on''), please label them as \texttt{True}. For the cases that there is no atomic fact that shares the same sentence structure, please identify the most informative entities in the sentence and label them with the same label as the atomic fact while treating the rest of the entities as \texttt{True}. In the end, output the entities and labels in the following format:
                \begin{itemize}[nosep]
                    \item Entity 1 (Label 1)
                    \item Entity 2 (Label 2)
                    \item ...
                    \item Entity N (Label N)
                \end{itemize}
                % \newline \newline
                Here are two examples:
                \newline\newline
                \textbf{[Example 1]}
                \newline
                [The start of the biography]
                \newline
                \textcolor{titlecolor}{Marianne McAndrew is an American actress and singer, born on November 21, 1942, in Cleveland, Ohio. She began her acting career in the late 1960s, appearing in various television shows and films.}
                \newline
                [The end of the biography]
                \newline \newline
                [The start of the list of checked facts]
                \newline
                \textcolor{anscolor}{[Marianne McAndrew is an American. (False); Marianne McAndrew is an actress. (True); Marianne McAndrew is a singer. (False); Marianne McAndrew was born on November 21, 1942. (False); Marianne McAndrew was born in Cleveland, Ohio. (False); She began her acting career in the late 1960s. (True); She has appeared in various television shows. (True); She has appeared in various films. (True)]}
                \newline
                [The end of the list of checked facts]
                \newline \newline
                [The start of the ideal output]
                \newline
                \textcolor{labelcolor}{[Marianne McAndrew (True); is (True); an (True); American (False); actress (True); and (True); singer (False); , (True); born (True); on (True); November 21, 1942 (False); , (True); in (True); Cleveland, Ohio (False); . (True); She (True); began (True); her (True); acting career (True); in (True); the late 1960s (True); , (True); appearing (True); in (True); various (True); television shows (True); and (True); films (True); . (True)]}
                \newline
                [The end of the ideal output]
				\newline \newline
                \textbf{[Example 2]}
                \newline
                [The start of the biography]
                \newline
                \textcolor{titlecolor}{Doug Sheehan is an American actor who was born on April 27, 1949, in Santa Monica, California. He is best known for his roles in soap operas, including his portrayal of Joe Kelly on ``General Hospital'' and Ben Gibson on ``Knots Landing.''}
                \newline
                [The end of the biography]
                \newline \newline
                [The start of the list of checked facts]
                \newline
                \textcolor{anscolor}{[Doug Sheehan is an American. (True); Doug Sheehan is an actor. (True); Doug Sheehan was born on April 27, 1949. (True); Doug Sheehan was born in Santa Monica, California. (False); He is best known for his roles in soap operas. (True); He portrayed Joe Kelly. (True); Joe Kelly was in General Hospital. (True); General Hospital is a soap opera. (True); He portrayed Ben Gibson. (True); Ben Gibson was in Knots Landing. (True); Knots Landing is a soap opera. (True)]}
                \newline
                [The end of the list of checked facts]
                \newline \newline
                [The start of the ideal output]
                \newline
                \textcolor{labelcolor}{[Doug Sheehan (True); is (True); an (True); American (True); actor (True); who (True); was born (True); on (True); April 27, 1949 (True); in (True); Santa Monica, California (False); . (True); He (True); is (True); best known (True); for (True); his roles in soap operas (True); , (True); including (True); in (True); his portrayal (True); of (True); Joe Kelly (True); on (True); ``General Hospital'' (True); and (True); Ben Gibson (True); on (True); ``Knots Landing.'' (True)]}
                \newline
                [The end of the ideal output]
				\newline \newline
				\textbf{User prompt}
				\newline
				\newline
				[The start of the biography]
				\newline
				\textcolor{magenta}{\texttt{\{BIOGRAPHY\}}}
				\newline
				[The ebd of the biography]
				\newline \newline
				[The start of the list of checked facts]
				\newline
				\textcolor{magenta}{\texttt{\{LIST OF CHECKED FACTS\}}}
				\newline
				[The end of the list of checked facts]
			};
			\node[chatbox_user_inner] (q1_text) at (q1) {
				\textbf{System prompt}
				\newline
				\newline
				You are a helpful and precise assistant for segmenting and labeling sentences. We would like to request your help on curating a dataset for entity-level hallucination detection.
				\newline \newline
                We will give you a machine generated biography and a list of checked facts about the biography. Each fact consists of a sentence and a label (True/False). Please do the following process. First, breaking down the biography into words. Second, by referring to the provided list of facts, merging some broken down words in the previous step to form meaningful entities. For example, ``strategic thinking'' should be one entity instead of two. Third, according to the labels in the list of facts, labeling each entity as True or False. Specifically, for facts that share a similar sentence structure (\eg, \textit{``He was born on Mach 9, 1941.''} (\texttt{True}) and \textit{``He was born in Ramos Mejia.''} (\texttt{False})), please first assign labels to entities that differ across atomic facts. For example, first labeling ``Mach 9, 1941'' (\texttt{True}) and ``Ramos Mejia'' (\texttt{False}) in the above case. For those entities that are the same across atomic facts (\eg, ``was born'') or are neutral (\eg, ``he,'' ``in,'' and ``on''), please label them as \texttt{True}. For the cases that there is no atomic fact that shares the same sentence structure, please identify the most informative entities in the sentence and label them with the same label as the atomic fact while treating the rest of the entities as \texttt{True}. In the end, output the entities and labels in the following format:
                \begin{itemize}[nosep]
                    \item Entity 1 (Label 1)
                    \item Entity 2 (Label 2)
                    \item ...
                    \item Entity N (Label N)
                \end{itemize}
                % \newline \newline
                Here are two examples:
                \newline\newline
                \textbf{[Example 1]}
                \newline
                [The start of the biography]
                \newline
                \textcolor{titlecolor}{Marianne McAndrew is an American actress and singer, born on November 21, 1942, in Cleveland, Ohio. She began her acting career in the late 1960s, appearing in various television shows and films.}
                \newline
                [The end of the biography]
                \newline \newline
                [The start of the list of checked facts]
                \newline
                \textcolor{anscolor}{[Marianne McAndrew is an American. (False); Marianne McAndrew is an actress. (True); Marianne McAndrew is a singer. (False); Marianne McAndrew was born on November 21, 1942. (False); Marianne McAndrew was born in Cleveland, Ohio. (False); She began her acting career in the late 1960s. (True); She has appeared in various television shows. (True); She has appeared in various films. (True)]}
                \newline
                [The end of the list of checked facts]
                \newline \newline
                [The start of the ideal output]
                \newline
                \textcolor{labelcolor}{[Marianne McAndrew (True); is (True); an (True); American (False); actress (True); and (True); singer (False); , (True); born (True); on (True); November 21, 1942 (False); , (True); in (True); Cleveland, Ohio (False); . (True); She (True); began (True); her (True); acting career (True); in (True); the late 1960s (True); , (True); appearing (True); in (True); various (True); television shows (True); and (True); films (True); . (True)]}
                \newline
                [The end of the ideal output]
				\newline \newline
                \textbf{[Example 2]}
                \newline
                [The start of the biography]
                \newline
                \textcolor{titlecolor}{Doug Sheehan is an American actor who was born on April 27, 1949, in Santa Monica, California. He is best known for his roles in soap operas, including his portrayal of Joe Kelly on ``General Hospital'' and Ben Gibson on ``Knots Landing.''}
                \newline
                [The end of the biography]
                \newline \newline
                [The start of the list of checked facts]
                \newline
                \textcolor{anscolor}{[Doug Sheehan is an American. (True); Doug Sheehan is an actor. (True); Doug Sheehan was born on April 27, 1949. (True); Doug Sheehan was born in Santa Monica, California. (False); He is best known for his roles in soap operas. (True); He portrayed Joe Kelly. (True); Joe Kelly was in General Hospital. (True); General Hospital is a soap opera. (True); He portrayed Ben Gibson. (True); Ben Gibson was in Knots Landing. (True); Knots Landing is a soap opera. (True)]}
                \newline
                [The end of the list of checked facts]
                \newline \newline
                [The start of the ideal output]
                \newline
                \textcolor{labelcolor}{[Doug Sheehan (True); is (True); an (True); American (True); actor (True); who (True); was born (True); on (True); April 27, 1949 (True); in (True); Santa Monica, California (False); . (True); He (True); is (True); best known (True); for (True); his roles in soap operas (True); , (True); including (True); in (True); his portrayal (True); of (True); Joe Kelly (True); on (True); ``General Hospital'' (True); and (True); Ben Gibson (True); on (True); ``Knots Landing.'' (True)]}
                \newline
                [The end of the ideal output]
				\newline \newline
				\textbf{User prompt}
				\newline
				\newline
				[The start of the biography]
				\newline
				\textcolor{magenta}{\texttt{\{BIOGRAPHY\}}}
				\newline
				[The ebd of the biography]
				\newline \newline
				[The start of the list of checked facts]
				\newline
				\textcolor{magenta}{\texttt{\{LIST OF CHECKED FACTS\}}}
				\newline
				[The end of the list of checked facts]
			};
		\end{tikzpicture}
        \caption{GPT-4o prompt for labeling hallucinated entities.}\label{tb:gpt-4-prompt}
	\end{center}
\vspace{-0cm}
\end{table*}
% \section{Full Experiment Results}
% \begin{table*}[th]
    \centering
    \small
    \caption{Classification Results}
    \begin{tabular}{lcccc}
        \toprule
        \textbf{Method} & \textbf{Accuracy} & \textbf{Precision} & \textbf{Recall} & \textbf{F1-score} \\
        \midrule
        \multicolumn{5}{c}{\textbf{Zero Shot}} \\
                Zero-shot E-eyes & 0.26 & 0.26 & 0.27 & 0.26 \\
        Zero-shot CARM & 0.24 & 0.24 & 0.24 & 0.24 \\
                Zero-shot SVM & 0.27 & 0.28 & 0.28 & 0.27 \\
        Zero-shot CNN & 0.23 & 0.24 & 0.23 & 0.23 \\
        Zero-shot RNN & 0.26 & 0.26 & 0.26 & 0.26 \\
DeepSeek-0shot & 0.54 & 0.61 & 0.54 & 0.52 \\
DeepSeek-0shot-COT & 0.33 & 0.24 & 0.33 & 0.23 \\
DeepSeek-0shot-Knowledge & 0.45 & 0.46 & 0.45 & 0.44 \\
Gemma2-0shot & 0.35 & 0.22 & 0.38 & 0.27 \\
Gemma2-0shot-COT & 0.36 & 0.22 & 0.36 & 0.27 \\
Gemma2-0shot-Knowledge & 0.32 & 0.18 & 0.34 & 0.20 \\
GPT-4o-mini-0shot & 0.48 & 0.53 & 0.48 & 0.41 \\
GPT-4o-mini-0shot-COT & 0.33 & 0.50 & 0.33 & 0.38 \\
GPT-4o-mini-0shot-Knowledge & 0.49 & 0.31 & 0.49 & 0.36 \\
GPT-4o-0shot & 0.62 & 0.62 & 0.47 & 0.42 \\
GPT-4o-0shot-COT & 0.29 & 0.45 & 0.29 & 0.21 \\
GPT-4o-0shot-Knowledge & 0.44 & 0.52 & 0.44 & 0.39 \\
LLaMA-0shot & 0.32 & 0.25 & 0.32 & 0.24 \\
LLaMA-0shot-COT & 0.12 & 0.25 & 0.12 & 0.09 \\
LLaMA-0shot-Knowledge & 0.32 & 0.25 & 0.32 & 0.28 \\
Mistral-0shot & 0.19 & 0.23 & 0.19 & 0.10 \\
Mistral-0shot-Knowledge & 0.21 & 0.40 & 0.21 & 0.11 \\
        \midrule
        \multicolumn{5}{c}{\textbf{4 Shot}} \\
GPT-4o-mini-4shot & 0.58 & 0.59 & 0.58 & 0.53 \\
GPT-4o-mini-4shot-COT & 0.57 & 0.53 & 0.57 & 0.50 \\
GPT-4o-mini-4shot-Knowledge & 0.56 & 0.51 & 0.56 & 0.47 \\
GPT-4o-4shot & 0.77 & 0.84 & 0.77 & 0.73 \\
GPT-4o-4shot-COT & 0.63 & 0.76 & 0.63 & 0.53 \\
GPT-4o-4shot-Knowledge & 0.72 & 0.82 & 0.71 & 0.66 \\
LLaMA-4shot & 0.29 & 0.24 & 0.29 & 0.21 \\
LLaMA-4shot-COT & 0.20 & 0.30 & 0.20 & 0.13 \\
LLaMA-4shot-Knowledge & 0.15 & 0.23 & 0.13 & 0.13 \\
Mistral-4shot & 0.02 & 0.02 & 0.02 & 0.02 \\
Mistral-4shot-Knowledge & 0.21 & 0.27 & 0.21 & 0.20 \\
        \midrule
        
        \multicolumn{5}{c}{\textbf{Suprevised}} \\
        SVM & 0.94 & 0.92 & 0.91 & 0.91 \\
        CNN & 0.98 & 0.98 & 0.97 & 0.97 \\
        RNN & 0.99 & 0.99 & 0.99 & 0.99 \\
        % \midrule
        % \multicolumn{5}{c}{\textbf{Conventional Wi-Fi-based Human Activity Recognition Systems}} \\
        E-eyes & 1.00 & 1.00 & 1.00 & 1.00 \\
        CARM & 0.98 & 0.98 & 0.98 & 0.98 \\
\midrule
 \multicolumn{5}{c}{\textbf{Vision Models}} \\
           Zero-shot SVM & 0.26 & 0.25 & 0.25 & 0.25 \\
        Zero-shot CNN & 0.26 & 0.25 & 0.26 & 0.26 \\
        Zero-shot RNN & 0.28 & 0.28 & 0.29 & 0.28 \\
        SVM & 0.99 & 0.99 & 0.99 & 0.99 \\
        CNN & 0.98 & 0.99 & 0.98 & 0.98 \\
        RNN & 0.98 & 0.99 & 0.98 & 0.98 \\
GPT-4o-mini-Vision & 0.84 & 0.85 & 0.84 & 0.84 \\
GPT-4o-mini-Vision-COT & 0.90 & 0.91 & 0.90 & 0.90 \\
GPT-4o-Vision & 0.74 & 0.82 & 0.74 & 0.73 \\
GPT-4o-Vision-COT & 0.70 & 0.83 & 0.70 & 0.68 \\
LLaMA-Vision & 0.20 & 0.23 & 0.20 & 0.09 \\
LLaMA-Vision-Knowledge & 0.22 & 0.05 & 0.22 & 0.08 \\

        \bottomrule
    \end{tabular}
    \label{full}
\end{table*}




\end{document}

  \clearpage 
  \section{Proofs}
\label{sec:appendix}


}

\end{document}
