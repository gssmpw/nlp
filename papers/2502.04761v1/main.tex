\documentclass[runningheads,orivec,envcountsame]{llncs}

\overfullrule=1mm

\usepackage[usenames,dvipsnames]{xcolor}
\usepackage{centernot}
\usepackage{amssymb}
\usepackage[linesnumbered,noend,boxruled]{algorithm2e}
\usepackage{stmaryrd}
\usepackage{bm}
\usepackage{tikz}
\usetikzlibrary{shapes,calc,arrows,automata,arrows.meta}
\usepackage{multirow}
\usepackage{array}
\usepackage{mathtools}
\usepackage{enumitem}
\usepackage[bookmarks,unicode,colorlinks=true]{hyperref}
\usepackage{proof}
\usepackage{varioref}
\usepackage[capitalize,nameinlink]{cleveref}
\usepackage[location=appendix,manual]{moveproofs}
\usepackage{thm-restate}
\usepackage{subfig}
\usepackage{hhline}
\usepackage{graphicx}
\usepackage{microtype}
\usepackage{cite}
\usepackage{pgfplots}
\usepackage{listings}
\pgfplotsset{compat=1.18}
\usepackage[edges]{forest}
\usepackage{empheq}
\usepackage[font=small,skip=0pt]{caption}
\usepackage{wrapfig}
\usepackage{marvosym}

\makeatletter
\newcommand{\algorithmstyle}[1]{\renewcommand{\algocf@style}{#1}}
\newcommand{\nosemic}{\renewcommand{\@endalgocfline}{\relax}}% Drop semi-colon ;
\newcommand{\dosemic}{\renewcommand{\@endalgocfline}{\algocf@endline}}% Reinstate semi-colon ;
\newcommand{\pushline}{\Indp}% Indent
\newcommand{\popline}{\Indm\dosemic}% Undent
\let\oldnl\nl% Store \nl in \oldnl
\newcommand{\nonl}{\renewcommand{\nl}{\let\nl\oldnl}}% Remove line number for one line
\makeatother

\newcommand{\report}[1]{#1}
\newcommand{\paper}[1]{}

\paper{
  % avoid lines at end of paragraphs with few words
  \everypar{\looseness=-1}
  % allow page breaks in formulas
  \allowdisplaybreaks[4]
  % allow page breaks right before formulas
  \predisplaypenalty=0
  % less space before and after algorithms
  \setlength{\intextsep}{3pt}
  \setlength{\textfloatsep}{1pt}
  % \setlist{nosep}
  \setlist{itemsep=1pt, topsep=3pt}
  \AtBeginDocument{
    \addtolength\abovedisplayskip{-0.3\baselineskip}
    \addtolength\belowdisplayskip{-0.3\baselineskip}
    \addtolength\abovedisplayshortskip{-0.3\baselineskip}
    \addtolength\belowdisplayshortskip{-0.3\baselineskip}
  }
}

\SetKwIF{If}{ElseIf}{Else}{if}{do}{else if}{else}{end if}%
\SetKwFor{While}{while}{}{end}%
% \SetKwFor{For}{for}{:}{end}%

% \renewcommand{\baselinestretch}{0.97}

\newcommand\mycommfont[1]{\scriptsize\ttfamily\textcolor{blue}{#1}}
\SetCommentSty{mycommfont}

\DeclareRobustCommand*{\modeledby}{%
  \Relbar\joinrel\mathrel{|}%
}

\DontPrintSemicolon

\hypersetup{%
  pdftitle={Infinite State Model Checking by Learning Transitive Relations},
  colorlinks=true,
  linkcolor=blue,
  citecolor=olive,
  filecolor=magenta,
  urlcolor=cyan
}



% ORCID
\makeatletter
\RequirePackage[bookmarks,unicode,colorlinks=true]{hyperref}%
   \def\@citecolor{blue}%
   \def\@urlcolor{blue}%
   \def\@linkcolor{blue}%
\def\UrlFont{\rmfamily}
\def\orcidID#1{\href{http://orcid.org/#1}{\protect\raisebox{-1.25pt}{\protect\includegraphics{orcid_color.eps}}}}
\makeatother


% Since we use the hyperref package, Springer asks us to include the following line
% to display URLs in blue roman font according to Springer's eBook style:
\renewcommand\UrlFont{\color{blue}\rmfamily}

\newenvironment{proofsketch}{%
  \renewcommand{\proofname}{Proof (Sketch)}\proof}{\endproof}

\renewcommand{\epsilon}{\varepsilon}
\let\oldphi\phi
\let\oldvarphi\varphi
\renewcommand{\varphi}{\oldphi}
\renewcommand{\phi}{\oldvarphi}

\newcommand{\pl}[1]{\textsf{#1}}
\newcommand{\tool}[1]{\textsf{#1}}
\def\increment{\hspace{-.05em}\raisebox{.4ex}{\tiny\bf ++}}
\def\decrement{\hspace{+.05em}\raisebox{.4ex}{\tiny\bf {-}{-}}}
\def\CXX{{\pl{C}\nolinebreak[4]\increment}}

\renewcommand{\vec}[1]{\bm{\mathrm{#1}}}

\newcommand{\mgu}{\mathsf{mgu}}
\newcommand{\rec}{\mathsf{rec}}
\newcommand{\push}{\mathsf{push}}
\newcommand{\pop}{\mathsf{pop}}
\newcommand{\add}{\mathsf{add}}
\newcommand{\tis}{\textsc{Tips}}
\newcommand{\ti}{\mathsf{ti}}
\newcommand{\accel}{\mathsf{accel}}
\newcommand{\underapprox}{\mathsf{ua}}
\newcommand{\tail}{\mathsf{tail}}
\newcommand{\lem}{\mathsf{lem}}
\newcommand{\Loop}{\mathsf{loop}}
\newcommand{\inc}{\mathsf{inc}}
\newcommand{\dec}{\mathsf{dec}}
\newcommand{\blocked}{\textsc{blocked}}
\newcommand{\blockingclause}{\mathsf{blocking\_clause}}
\newcommand{\backtrack}{\mathsf{backtrack}}
\newcommand{\checksat}{\mathsf{check}\_\mathsf{sat}}
\newcommand{\getmodel}{\mathsf{model}}
\newcommand{\mbp}{\mathsf{cvp}}
\newcommand{\mbip}{\mathsf{cvp}}
\newcommand{\sip}{\mathsf{sip}}
\newcommand{\unsat}{\mathsf{unsat}}
\newcommand{\unsafe}{\mathsf{unsafe}}
\newcommand{\unknown}{\mathsf{unknown}}
\newcommand{\encode}{{\sf encode}}
\newcommand{\sat}{\mathsf{sat}}
\newcommand{\safe}{\mathsf{safe}}
\newcommand{\tip}{\mathsf{tp}}
\renewcommand{\AA}{\mathcal{A}}
\newcommand{\BB}{\mathcal{B}}
\newcommand{\CC}{\mathcal{C}}
\newcommand{\LL}{\mathcal{L}}
\newcommand{\VV}{\mathcal{V}}
\newcommand{\MM}{\mathcal{M}}
\newcommand{\len}{\mathsf{len}}
\newcommand{\trace}{\mathsf{trace}}
\newcommand{\compose}{{\mathsf{compose}}}
\newcommand{\concat}{\mathrel{::}}
\newcommand{\init}{\mathsf{init}}
\newcommand{\err}{\mathsf{err}}
\newcommand{\QF}{\mathsf{QF}}
\renewcommand{\partial}{\rightharpoonup}

\def\mystack#1\over#2_#3{%
   \mathrel {%
      \setbox0=\hbox{$\scriptscriptstyle #1$}%
      \setbox1=\hbox{$#2$}%
      \ifdim\wd1>\wd0 \kern .5\wd1 \else \kern .5\wd0 \fi
      \vbox{
         \offinterlineskip
         \moveleft.5\wd0 \box0
         \kern.3ex
         \moveleft.5\wd1 \hbox{$#2_#3$}
}}}

\newcommand{\ind}[3][]{
  \ifthenelse{\equal{#1}{}}{\overset{\scriptscriptstyle(#3)}{#2}}{{\mystack (#3) \over #2_#1}}
}
\newcommand{\twodots}{%
  \mathinner{{\ldotp}{\ldotp}}%
}

% integers, rationals, reals, ...
\newcommand{\ZZ}{\mathbb{Z}}
\newcommand{\QQ}{\mathbb{Q}}
\newcommand{\NN}{\mathbb{N}}

% sets of transitions / programs etc.
\newcommand{\PP}{\mathcal{P}}
\newcommand{\TT}{\mathcal{T}}
\newcommand{\XX}{\mathcal{X}}
\newcommand{\RR}{\mathcal{R}}
\newcommand{\DG}{\mathcal{DG}}
\newcommand{\GG}{\mathcal{G}}
\renewcommand{\SS}{\mathcal{S}}

% big-oh
\newcommand{\OO}{\mathcal{O}}
% characteristic function
\newcommand{\charfun}[1]{I_{#1}}
% defining equation
\newcommand{\Def}{\mathrel{\mathop:}=}
% scalable version of \mid
\newcommand{\relmiddle}[1]{\mathrel{}\middle#1\mathrel{}}
% shortcut for space-saving matrices
\newcommand{\mat}[1]{\left(\begin{smallmatrix} #1 \end{smallmatrix}\right)}

\newcommand{\arity}{\mathsf{arity}}
\newcommand{\sem}[1]{\llbracket #1 \rrbracket}

\renewcommand{\emptyset}{\varnothing}

% comments
\def\me{JG}
\usepackage{ifthen}
\newcommand{\comment}[2][ALL]{%
  \ifthenelse{\equal{ALL}{#1}}%
  {\footnote{!!! #2}}%
  {%
    \ifthenelse{\equal{\me}{#1}}%
    {\footnote{!!! #2}}%
    {}%
  }%
}
%\renewcommand{\comment}[1]{}

\DeclareMathOperator{\dom}{dom}
\DeclareMathOperator{\img}{img}
\newcommand{\id}{\mathsf{id}}

\crefname{algorithm}{alg.}{algorithms}%
\crefname{equation}{eq.}{equations}%
\crefname{chapter}{chapter}{chapters}%
\crefname{section}{sect.}{sections}%
\crefname{appendix}{app.}{appendices}%
\crefname{enumi}{item}{items}%
\crefname{footnote}{footnote}{footnotes}%
\crefname{figure}{fig.}{figures}%
\crefname{table}{table}{tables}%
\crefname{theorem}{thm.}{theorems}%
\crefname{lemma}{lemma}{lemmas}%
\crefname{corollary}{cor.}{corollaries}%
\crefname{proposition}{proposition}{propositions}%
\crefname{definition}{def.}{definitions}%
\crefname{result}{result}{results}%
\crefname{example}{ex.}{examples}%
\crefname{remark}{remark}{remarks}%
\crefname{note}{note}{notes}%
\crefname{lstlisting}{listing}{listings}%
\crefname{requirement}{req.}{requirements}%

\title{Infinite State Model Checking by Learning Transitive Relations}
\author{Florian Frohn$^{(\href{mailto:florian.frohn@informatik.rwth-aachen.de}{\mbox{\Letter}})}$\orcidID{0000-0003-0902-1994} and Jürgen Giesl$^{(\href{mailto:giesl@informatik.rwth-aachen.de}{\mbox{\Letter}})}$\orcidID{0000-0003-0283-8520}}
\paper{\institute{RWTH Aachen University,  Aachen, Germany}}
\report{\institute{RWTH Aachen University,  Aachen, Germany\\
\email{\{florian.frohn,giesl\}@informatik.rwth-aachen.de}}}
\authorrunning{F.\ Frohn, J.\ Giesl}
  


\begin{document}

\renewcommand{\thelstlisting}{\arabic{lstlisting}}

\maketitle

\begin{abstract}
Retrieval-Augmented Generation (RAG) is often used with Large Language Models (LLMs) to infuse domain knowledge or user-specific information. In RAG, given a user query, a retriever extracts chunks of relevant text from a knowledge base. These chunks are sent to an LLM as part of the input prompt. Typically, any given chunk is repeatedly retrieved across user questions. However, currently, for every question, attention-layers in LLMs fully compute the key values (KVs) repeatedly for the input chunks, as state-of-the-art methods cannot reuse KV-caches when chunks appear at arbitrary locations with arbitrary contexts. Naive reuse leads to output quality degradation.  This leads to potentially redundant computations on expensive GPUs and increases latency. In this work, we propose \sys, a system for managing and reusing precomputed KVs corresponding to the text chunks (we call \textit{chunk-caches}) in RAG-based systems. We present how to identify \hl{\textit{chunk-caches} that are reusable}, how to efficiently perform a small fraction of recomputation to \textit{fix} the cache to maintain output quality, and how to efficiently store and evict \textit{chunk-caches} in the hardware for maximizing reuse while masking any overheads. With real production workloads as well as synthetic datasets, we show that \sys reduces redundant computation by \textbf{51\%} over SOTA prefix-caching and \textbf{75\%} over full recomputation.
\hl{Additionally, with continuous batching on a real production workload, we get a \textbf{1.6$\times$} speedup in throughput and a \textbf{2$\times$} reduction in end-to-end response latency over prefix-caching while maintaining quality, for both the \llama-3-8B and \llama-3-70B models. 
}
\end{abstract}





\documentclass[../main.tex]{subfiles}
\graphicspath{{../images/}}
\makeatletter
\def\input@path{{../images/}}
\makeatother
\begin{document}
\section{Introduction}
\begin{figure}
\centering
\begin{tikzpicture}
\node[inner sep=0pt] (ws) at (0, 0) {
\includegraphics[height=.4\textwidth, trim={10cm 0 10cm 0},clip]{world_space.png}};
\node[inner sep=0pt] (cs) at (6,0) {\includegraphics[height=.4\textwidth, trim={10cm 1cm 10cm 4cm},clip]{conf_space.png}};
\end{tikzpicture}
\vspace{-5pt}
\label{fig:pbrm_intro}
\caption{\textbf{Left}: Shows world space obstacles as grey spheres. Robots start and goal configuration is colored red and green, respectively. Configurations along the computed path are colored transparent blue. \textbf{Right:} Mapped world space scenario to configuration space. Obstacle region is the grey mesh. Red spheres are collision-free regions computed by the neural SCDF. The optimized shortest path in the convex corridor is the blue curve.}
\vspace{-25pt}
\end{figure}
Motion planning is the problem of finding a collision-free trajectory that connects a given start and goal configuration. The planning takes place in the configuration space of the robot. For single body robots, like mobile robots or drones, the configuration space and the world space are usually the same. This simplifies the planning, since explicit obstacle representations are available which enables geometrical tools like separating hyperplanes, smallest distance to obstacles etc., to be used when designing motion planning algorithms. For multi-body robots like manipulators, the situation is completely different. The world space obstacles are usually mapped to non-convex regions, and to make the problem even harder, the mapping is usually not known. Forming explicit representations of the obstacle region in the configuration space is usually too expensive or intractable. Despite all of this, sampling based planners are used with great success, which mainly is due to their use of implicit representations of the obstacle region. The basic idea is to construct a graph in the configuration space that covers and connects the collision-free region. From this graph, a path can be extracted that connects a given start and goal configuration. The approach is computationally expensive, since the graph is constructed with the smallest geometrical building block available, points, which represents a collision-check. Furthermore, the extracted paths from the graph are non-smooth and jagged due to the stochastic nature of the approach. This adds an additional post-processing step to the process, where the paths are shortcutted and smoothened, before the path can be used for tracking. Clearly a lot of time is invested to form this graph and produce smooth paths. Thus, if the obstacles start to move, then all of this work is done in no use, since all points that make up this graph need to be re-verified, which is simply too time consuming to be done in real time.
\\\\
In this work, we want to address the existing drawbacks of the sampling based planners. Our main contribution is an improved motion planner where each vertex in the graph covers a collision-free region in the form of a sphere instead of a point and where the edges are formed with neighboring intersecting spheres. This representation has the advantage of instead of returning piecewise linear paths, returning a sequence of overlapping spheres, i.e. a convex corridor, that connects a given start and goal configuration, illustrated in Figure \ref{fig:pbrm_intro}. This convex corridor allows us to use convex optimization to produce smooth trajectories, instead of computationally expensive post-processing methods. The representation further allows us to estimate the coverage of the collision-free space, which gives us awareness and feedback in the offline roadmap construction phase. Finally, our representation is simple to adapt to moving obstacles, simply requery for the new radii and recheck for intersections. 
\\\\
The spherical collision-free regions are formed using a signed distance function (SDF), which is a function that returns the smallest distance from an arbitrary point to the boundary of an obstacle. As the name implies, the distance is signed, thus if the point is inside the obstacle it is negative otherwise positive. If the distance is positive, a sphere with radius equal to the distance is guaranteed to cover a collision-free region. Using an SDF in motion planning is not new, but what is novel about our approach is that we express the distance in the configuration space instead of the world space and by doing so allows us to form these convex collision-free regions. We refer to the resulting SDF as a signed configuration distance function (SCDF). Computing an SCDF analytically is non-trivial, our approach is therefore to parameterize the SCDF with a deep neural network and learn the mapping by supervised learning. Our resulting neural SCDF can compute distances for different parameter values of obstacle shapes and we also show how multiple distances can be combined, thus making our approach flexible.
\section{Related work}
Motion planning algorithms can roughly be divided into three families, grid-based, sampling based and optimization based methods. Grid-based methods (GBM) discretize the planning space from which a graph is then compiled. A standard search method is A$^\star$ \citep{a_star}, which is classified as an \textit{informed} search method, since it employs a heuristic function to speed up the search. A$^\star$ guarantees to return an optimal path at the level of discretization used. GBMs usually discretize the planning space by a regular lattice and this limits the GBMs to problems with low dimensionality due to the curse of dimensionality. Thus, GBMs are usually limited to single-body robots where the degrees of freedom (DOF) are low. To overcome the inherent scaling problem with the GBMs, stochastic methods are usually used for multi-body robots. These methods are termed as sampling-based methods (SBM) and core members within this family are the rapidly-exploring random trees (RRT) \citep{rrt} and the probabilistic roadmap (PRM) \citep{prm}. RRT grows a tree from the start configuration and explores the collision-free region in a rapid way until it is able to connect to the goal region. RRT is usually improved by bi-directional planning \citep{rrt_connect}, i.e. an additional tree is grown from the goal configuration and the trees are tested for connection after any tree has been expanded. RRT is a single-query method, thus it searches for a path from scratch each time it is queried. Contrary to this, PRM is a multi-query method, which solves for multiple queries without starting from scratch. PRM does this by creating a roadmap (graph) that covers the collision-free space as an offline step. The graph is then used to solve for multiple queries. PRMs are used in cases where the environment does not change since the extra offline step is too computationally costly and needs to be re-done if the environment is changed. In our work, we address this inherent issue by using a different roadmap representation. Our vertices in the graph cover a collision-free region in the form of spheres and we form the edges by checking for intersecting spheres. If something in the environment changes, we recompute the spheres radii and recheck the intersections, without relying on collision detection. We use a trained neural network to compute the sphere radius, therefore querying for the radius can be done fast, hence our representation enables the PRM for dynamic environments.
\\\\
In the recent decades, optimization based methods (OBM) \citep{chomp, schulman, itomp, stomp} have been introduced as an alternative to SBM for multi-body robots. Like the SBM, the OBMs scale well to higher dimensional problems and produce smoother motion. It is common to use a SDF in the optimization since it is a smooth function, thus enabling gradient-based methods. However, the standard way of expressing the SDF is in world space. The distance therefore needs to be mapped to the configuration space by the forward kinematics. This mapping makes the optimization problem a non-linear program (NLP), which is computationally expensive to solve. Recently, a different approach has been proposed. In \cite{mp_gcs} motion planning is formulated as a convex optimization problem by using the graph of convex sets framework \citep{gcs}. The underlying idea is to decompose the collision-free space into intersecting convex sets from which a convex optimization problem is formulated. In cases where an explicit representation of the obstacles in the configuration space exists, like for single-body robots, creating collision-free convex regions can be done fast \citep{iris}. For multi-body robots, this is non-trivial. Existing work does this successfully \citep{iris_nlp, iris_c} by an optimization based approach, but the methods are still too time consuming to be used in the presence of moving obstacles. Our approach is instead to use deep learning to learn an SDF expressed in the configuration space. With this, we can query for shortest distances to the collision boundary, which allows us to expand spherical regions which are collision-free. Our approach is fast and therefore enables our suggested roadmap planner to be used in dynamic environments.
\\\\
Recent research has focused on learning collision detection \citep{fk_kernel_distance, diffco, graphdistnet} by predicting the signed distance between the robot links and the surrounding obstacles in the world space. The learned SDF is used in trajectory optimization but since the distance is expressed in the world space, the problem becomes an NLP and therefore takes a long time to solve. We take a novel approach and suggest to instead express the signed distance in the configuration space. This allows us to improve the PRM at the same time as it enables convex optimization for trajectory optimization, which runs faster and is more reliable than NLP solvers. In \cite{cspf} a learned signed distance function in the configuration space is proposed similar to our approach. However, their approach is restricted to point cloud representations, while we propose to represent the obstacles as parameterized geometric shapes, e.g. spheres. Furthermore, we also show how to use our learned SCDF to improve an existing roadmap planner.
\section{Problem formulation}
A robot is located in the world space, $\W \subset \R^3 $. The unique location of the robot is given by its configuration $\q \in \C$, where $\C$ is the configuration space. The set of points covered by the robots bodies at a certain configuration is expressed as $\B(\q) \subset \W$. The robot is surrounded by $\NrObst$ obstacles $\O = \bigcup_{i=1}^{\NrObst} \O_i$, where  $\O_i \subset \W$. The representation of the obstacle in the configuration space is the set $\C\O_i = \{\q \in \C \: |\: \B(\q) \cap \O_i \neq \emptyset \}$. The obstacle space is formed as $\Co = \bigcup_{i=1}^{\NrObst} \C \O_i$. The complement is referred to as the free space, $\Cf = \C \setminus \Co$. The path planning problem is a tuple, ($\Cf$, $\qStart$, $\qGoal$), where we want to connect a query pair, consisting of a start, $\qStart$, and goal configuration, $\qGoal$, with a geometric path, $\q(s): [0, 1] \mapsto \Cf$, such that $\q(0)=\qStart$ and $\q(1)=\qGoal$, or report correctly when such a path does not exist.
\end{document}

\begin{figure*}[t]
\begin{center}
\includegraphics[width=.85\linewidth]{fig_overview_v3.pdf}
\end{center}
\caption{
FastAtlas Overview: In each frame, we compute charts spanning fully or partially visible triangles (a), determine texture space bounding boxes for the visible portions of the view-space projections of each chart, and tightly pack these boxes into atlases (b, here $2K \times 2K$). We simultaneously bijectively parameterize and shade the charts into their atlas boxes, obtaining high quality texture space shading (c), and use this shading to render the shaded frames (d).}
\label{fig:overview}
\label{fig:alg_overview}
\end{figure*}

\section{Overview}
\label{sec:overview}
Our work has two core contributions: a real-time, GPU-based algorithm for tight packing of general parameterized charts into compact atlases; and a real-time TSS method that
utilizes this packing.  

\paragraph*{FastAtlas Packing.}
FastAtlas runs entirely on the GPU as a series of compute shaders. It takes the bounding boxes of parameterized charts as input, and packs them into an atlas (Fig~\ref{fig:overview}b, Sec.~\ref{sec:pack}). As such, the only input it requires are the dimensions of the bounding boxes.
Its outputs are deterministic; identical input charts are packed into identical atlases. This is critical for TSS and similar applications, as it ensures that consecutive frames taken from the same camera view have the same shading. Even minute shading differences across such frames can cause sampling jitter, leading to undesirable flicker \cite{baker2012rock}. 
While prior methods such as \cite{mueller2018shading,hladky2019tessellated,hladky2021snakebinning,Neff2022MSA} cap the dimensions of the charts that can be packed as-is for a given atlas size, and scale down all charts that exceed these dimensions, we scale all charts by the same factor, and do so only when strictly necessary to achieve packing success (Figs~\ref{fig:atlas},~\ref{fig:sas_issues}). 

\paragraph*{TSS using FastAtlas.}
Our end-to-end TSS atlas generation method combines the packing method above with a novel approach for computing seamless per-frame charts. 
We define our charts as the connected components of the visible surfaces in each frame (Fig.~\ref{fig:overview}a), and efficiently compute them using a parallel union-find algorithm (Sec.~\ref{sec:visible}). Since the boundaries of these charts coincide with the contours of the rendered surface, they are {\em invisible} to the viewer. This approach 
eliminates the artifacts caused by shading discontinuities along visible seams (Fig.~\ref{fig:seams}). 

\begin{parWithWrapFigure}
\begin{wrapfigure}{l}{.27\columnwidth}%
\includegraphics[width=\linewidth]{fig_inset_view_plane.pdf}%
\end{wrapfigure}
We bijectively parametrize the {\em visible portions} of our charts by projecting them to view space (inset). This maps a constant number of texels to each pixel in the final rendered output, evenly distributing residual undersampling error across all image pixels. While conceptually straightforward, efficiently parameterizing charts containing partially visible triangles using viewspace projection is non-trivial, as the visible portions may no longer be triangular (e.g. green triangle in the inset); applying naive projection to triangles with vertices behind the camera may produce ill-posed results. Clipping triangles before projection is both computationally expensive and significantly complicates downstream operations. We avoid explicit clipping by observing that all that is required for atlas packing is the dimensions of, potentially conservative, bounding boxes of these projected visible portions. We compute such bounding boxes without explicit chart clipping by adapting a conservative screen coverage estimator \shortcite{Blinn:CalculatingScreenCoverage} (Sec.~\ref{sec:box}). We then pack the computed boxes using FastAtlas. 
\end{parWithWrapFigure}

Finally, we shade the visible portion of each chart into its corresponding atlas bounding box (Fig~\ref{fig:overview}c). 
The resulting texture is then used during rasterization as a standard texture map (Fig. ~\ref{fig:overview}d). 
Our framework is compatible with all existing approaches for texture space shading, including forward shading, raytraced illumination, or deferred shading in texture space \cite{baker:2016}. In the examples shown, we use the standard forward shading based rendering pipeline included in the G3D Innovation Engine \cite{G3D17}, a commercial grade renderer.


\section{Preliminaries}\label{sec:preliminaries}



%We denote by $(\Ac(x_\Ac),\Bc(x_\Bc))(z)$ a random execution of $\pi$ with private inputs $(x_\Ac,y_\Ac)$, and common input $z$.

%\Jnote{Move to DP}
% At the end of such an execution, the protocol outputs a public transcript denoted by the random variable $\trans_\pi(x_\Ac,x_\Ac,z)$ we denotes the common as $\out(\trans_\pi(x_\Ac,x_\Ac,z)$, and each party $\Pc \in \set{\Ac,\Bc}$ obtains his view denoted $\view^\Pc_\pi(x_\Ac,x_\Bc,z)$, which may also contain a ``local output'' \Jnote{Local} $\out^\Pc(x_\Ac,x_\Bc,z)$ (if the protocol specifies such an output). \Jnote{Common output, and parties output}


\subsection{Distributions and Random Variables}\label{sec:prelim:dist}
The support of a distribution $P$ over a finite set $\cS$ is defined by $\Supp(P) \eqdef \set{x\in \cS: P(x)>0}$. For a distribution or a random variable $D$, let $d\from D$ denote that $d$ was sampled according to $D$. Similarly,  for a set $\cS$, let $x \from \cS$ denote that $x$ is drawn uniformly from $\cS$, and denote by $\cU_{\cS}$ the uniform distribution over $\cS$. For a finite set $\cX$ and a distribution $C_X$ over $\cX$, we use the capital letter $X$ to denote the random variable that takes values in $\cX$ and is sampled according to $C_X$. The {\sf statistical distance} (\aka {\sf~variation distance}) of two distributions $P$ and $Q$ over a discrete domain $\cX$ is defined by $\sdist{P}{Q} \eqdef \max_{\cS\subseteq \cX} \size{P(\cS)-Q(\cS)} = \frac{1}{2} \sum_{x \in \cS}\size{P(x)-Q(x)}$. 
For a vector $x = (x_1,\ldots,x_n)$ and index $i\in [n]$, we let $x_{-i} = (x_1,\ldots,x_{i-1},x_{i+1},\ldots,x_n)$ and $x^{(i)} = (x_1,\ldots,x_{i-1}, -x_i, x_{i+1},\ldots,x_n)$, for a set $\cS \subseteq [n]$ we let $x_{\cS} = (x_i)_{i \in \cS}$ and $x_{-\cS} = (x_i)_{i \in [n]\setminus \cS}$, and for a vector $r \in \zo^n$ we let $x_r = (x_i)_{\set{i \colon r_i = 1}}$ and $x_{-r} = (x_i)_{\set{i \colon r_i = 0}}$.

%For $n \in \N$ we let $U_n$ be the uniform distribution over $\oo^n$, and let $S_n$ be the distribution induces by the sum of $n$ i.i.d.\ random variables, each is distributed according to $U_1$. Let $\cN(0,1)$ be the standard normal distribution.
%For a distribution $\cD$ and a function $f$, we define by $f(\cD)$ the distribution that is induced by the output of $f(x)$ for $x \from \cD$. 





% \begin{theorem}[\cite{McGregorMPRTV10}]\label{thm:sv-extracotr}
% 	\Enote{Remove if not needed}
% 	There is a constant $c$ to make the following holds. Let $X$ be an $\alpha$-SV source on $\{0,1\}^n$, let $Y$ be a source on $\{0,1\}^n$ with min-entropy at least $\beta n$ (independent from $X$), and let $Z=\ip{X,Y}\mbox{mod m}$ for some $m\in\mathbb{N}$. Then for every $\delta\in[0,1]$, the random variable $(Y,Z)$ is $\delta$-close to $(Y,U)$ where $U$ is uniform on $\mathbb{Z}_m$ and independent of $Y$, provided that
% 	$$
% 	n\geq c\cdot\frac{m^2}{\alpha\beta}\cdot\log(\frac{m}{\beta})\cdot\log(\frac{m}{\delta}).
% 	$$
% \end{theorem}



\Enote{I removed the definition of DP since it already appears in the intro}
\remove{
\subsection{Differential Privacy}\label{sec:prelim:DP}
We use the following standard definition of (information theoretic) differential privacy, due to \citet{DMNS06}. For notational convenience, we focus on databases over $\oo$.
\begin{definition}[Differentially private mechanisms]\label{def:mech}
	A randomized function $f\colon\oo^n\mapsto \zs$ is an {\sf $n$-size, $(\eps,\delta)$-differentially private mechanism} (denoted $(\eps,\delta)$-\DP) if for every neighboring $w,w'\in \oo^n$ and every function $g\colon \zs\mapsto \zo$, it holds that 
	$$
	\pr{g(f(w))=1}\leq e^{\eps}\cdot \pr{g(f(w'))=1} +\delta.
	$$ 	
	If $\delta=0$, we omit it from the notation.
\end{definition}
}


\subsubsection{Computational Differential Privacy}
There are several ways for defining computational differential privacy (see \cref{sec:related-works}). We use the most relaxed version due to \cite{BNO08}. For notational convenience, we focus on databases over $\oo$.
\begin{definition}[Computational differentially private mechanisms]\label{def:ComMech}
	A randomized function ensemble $f=\set{f_\pk\colon\oo^{n(\pk)}\mapsto \zs}$ is an {\sf $n$-size, $(\eps,\delta)$-computationally differentially private} (denoted $(\eps,\delta)$-$\CDP$) if for every poly-size circuit family $\set{\Ac_\pk}_{\pk\in \N}$, the following holds for every large enough $\pk$ and every neighboring $w,w'\in\oo^{n(\pk)}$:
	$$
	\pr{\Ac_\pk(f_\pk(w))=1}\leq e^{\eps(\pk)}\cdot \pr{\Ac_\pk(f_\pk(w'))=1} +\delta(\pk).
	$$ 
	If $\delta(\pk) = \negl(\pk)$, we omit it from the notation. 
\end{definition}



\subsubsection{Two-Party Differential Privacy}\label{sec:DP}
In this section we formally define distributed differential privacy mechanism (\ie protocols). %For the ease of notation, we consider protocol with no common input.

\begin{definition}\label{def:DP}%\Nnote{fix security parameter}
	A two-party protocol $\Pi=(\Ac,\Bc)$ is {\sf $(\eps,\delta)$-differentially private}, denoted $(\eps,\delta)$-$\DP$, if the following holds for every algorithm $\Dc$: let $\V^\Pc(x,y)(\pk)$ be the view of party $\Pc$ in a random execution of $\Pi(x,y)(1^\pk)$. Then for every $\pk,n \in \N$, $x\in \oo^n$ and neighboring $y,y'\in\oo^n$:
	\begin{align*}
	\pr{\Dc(V^\Ac(x,y)(\pk))=1}\le e^{\eps(\pk)}\cdot \pr{\Dc(V^\Ac (x,y')(\pk))=1}+\delta(\pk),
	\end{align*} 
	and for every $y\in \oo^n$ and neighboring $x,x'\in\oo^{n}$:
	\begin{align*}
	\pr{\Dc(V^\Bc(x,y)(\pk))=1}\le e^{\eps(\pk)}\cdot \pr{\Dc(V^\Bc (x',y)(\pk))=1}+\delta(\pk).
	\end{align*} 	
	Protocol $\Pi$ is {\sf $(\eps,\delta)$-computational differentially private}, denoted $(\eps,\delta)$-$\CDP$, if the above inequalities only hold for a non-uniform \ppt $\Dc$ and large enough $\pk$. We omit $\delta = \negl(\pk)$ from the notation. \footnote{Note that define we give for two-party differentially private protocols is a semi-honest definition, in which we ask for the security to hold when the parties interact in an honest execution of the protocol. Since we are proving a lower bound, starting from this weaker guarantee (as opposed to security against malicious players), yields a stronger result.}
\end{definition}
%We omit $\delta$ from the notation if $\delta$ is a negligible function of $n$.

%\Enote{simulation-based}
\begin{remark}[The definition for computational differential privacy we use]\label{rem:comDPChannel} 
	An alternative, stronger definition of computational differential privacy, known as simulation-based computational differential privacy, requires that the distribution of each party’s view be computationally indistinguishable from a distribution that ensures privacy in an information-theoretic sense. \cref{def:DP} is a weaker notion in comparison. Consequently, establishing a lower bound for a protocol that satisfies this weaker guarantee (as we do in this work) yields a stronger result.%Actually, our lower bound only requires the privacy to hold against \emph{uniform} external observer.
	%\Nnote{Maybe add: When only interesting in \Dp against external observer, the two definitions can be achieve using key-agreement and (single-party) \Dp mechanism. }
\end{remark}




\subsection{Useful Claims}
\remove{
In this section, we state generic lemmas and propositions that we will use later in our proofs.

The following lemma which we prove in \cref{sec:missing-proofs:distance-I}, measures the distance between two uniform stings conditioned one a random index $i$ either being fixed to $0$ or to $1$.

\def\distanceILemma{
    Let $R \la \zo^n$. For any (randomized) function $f:\{0,1\}^n\rightarrow \{0,1\}$ and $\alpha > 0$, it holds that
    \begin{align}\label{eq:f-alpha}
        \ppr{i \la [n]}{\size{\:\ex{f(R) \mid R_i = 0}-\ex{f(R) \mid R_i = 1}\:}\geq \alpha} \leq \frac{2}{n \alpha^2},
    \end{align}
    where the expectations are taken over $R$ and the randomness of $f$.
}

\begin{lemma}\label{lem:distance-I}
    \distanceILemma
\end{lemma}
}

The following two propositions state that given the output of a differentially private function, it is not possible to predict well even a random index (even if all other indexes are leaked). The first proposition handles the information-theoretic case and the second handles the computation case. Both propositions are proven in \cref{sec:missing-proofs:hard-to-guess}. 

\def\propHardToGuessInf{
    Let $f\colon \oo^n \rightarrow \cY$ be an $(\eps,\delta)$-\DP function, let $g \colon [n] \times \oo^{n-1} \times \cY \rightarrow \set{-1,1,\bot}$ be a (randomized) function, and let $X = (X_1,\ldots,X_n) \la \oo^n$. Then the following holds for every $i \in [n]$ where $X_i^* = g(i,X_{-i},f(X_1,\ldots,X_n))$:
    \begin{align*}
        \pr{X_i^* = X_i} \leq e^{\eps}\cdot \pr{X_i^* = -X_i} + \delta.
    \end{align*}
}

\begin{proposition}\label{prop:hard-to-guess-inf}
    \propHardToGuessInf
\end{proposition}


\def\propHardToGuessComp{
    Let $f = \set{f_{\pk} \colon \oo^{n(\pk)} \rightarrow \zo^{m(\pk)}}_{\pk \in \bbN}$ be an $(\eps,\delta)$-\CDP function ensemble, and let $\set{g_{\pk}}_{\pk \in \bbN}$ be a poly-size circuit family. Then, for large enough $\pk$ and $X = (X_1,\ldots,X_{n(\pk)}) \la \oo^{n(\pk)}$, the following holds for every $i \in [n(\pk)]$ where $X_i^* = g_{\pk}(i,X_{-i},f_{\pk}(X_1,\ldots,X_n))$:
    \begin{align*}
        \pr{X_i^* = X_i} \leq e^{\eps(\pk)}\cdot \pr{X_i^* = -X_i} + \delta(\pk).
    \end{align*}
}

\begin{proposition}\label{prop:hard-to-guess-comp}
    \propHardToGuessComp
\end{proposition}





\remove{
\Enote{Chao's old statement:}
\begin{lemma}\label{lem:distance-I-old}
        Let $R \la \zo^n$. 
	For any function $f:\{0,1\}^n\rightarrow \{0,1\}$ and $\alpha<0.01$, it holds that
	$$
	\Pr_{i\la[n]}\left[\: \size{\:\mathbb{E}[f(R) \mid R_i = 0]-\mathbb{E}[f(R) \mid R_i = 1]\:}\geq \alpha\right]\leq \frac{2+2\log(\frac{1}{\alpha})}{n\alpha^2}.
	$$
\end{lemma}
\begin{proof}
	Define $S_1=\{r \in \zo^n \colon f(r)=1\}$. Then for any $i\in[n]$, we have
	$$
	\begin{array}{rl}
		\size{\mathbb{E}[f(R) \mid R_i = 0]-\mathbb{E}[f(R) \mid R_i = 1]}
		&=\size{\Pr[R\in S_1|R_i=0]-\Pr[R\in S_1|R_i=1]}\\
		&=\size{\frac{\Pr[R_i=0|R\in S_1]\cdot\Pr[R\in S_1]}{\Pr[R_i=0]}-\frac{\Pr[R_i=1|R\in S_1]\cdot\Pr[R\in S_1]}{\Pr[R_i=1]}}\\
		&=\frac{2\size{S_1}}{2^n}\size{\Pr[R_i=0|R\in S_1]-\Pr[R_i=1|R\in S_1]}
	\end{array}
	$$
	When $|S_1|\leq \alpha\cdot 2^{n-1}$, we have $\size{\mathbb{E}[f(R) \mid R_i = 0]-\mathbb{E}[f(R) \mid R_i = 1]}\leq\frac{2\size{S_1}}{2^n}\leq \alpha$ for any $i\in[n]$. Hence, in the following, we assume $|S_1|> \alpha\cdot 2^{n-1}$.

	%Define $I_{bad}=\{i|\size{\Pr[R_i=0|R\in S_1]-\Pr[R_i=1|R\in S_1]}>2\alpha\}$ and $k=\size{I_{bad}}$, then for any $i\notin I_{bad}$, we have 
    %$$
    %\begin{array}{rl}
    %    2\alpha&\geq \size{\Pr[R_i=0|R\in S_1]-\Pr[R_i=1|R\in S_1]}\\
    %    &=\size{\frac{\Pr[R\in S_1|R_i=0]\cdot\Pr[R_i=0]}{\Pr[R\in S_1]}-\frac{\Pr[R\in S_1|R_i=1]\cdot\Pr[R_i=1]}{\Pr[R\in S_1]}}\\
    %    &=\size{\Pr[R\in S_1|R_i=0]-\Pr[R\in S_1|R_i=1]}\cdot\frac{1}{2\Pr[R\in S_1]}\\
    %    &\geq \size{\mathbb{E}[f(R) \mid R_i = 0]-\mathbb{E}[f(R) \mid R_i = 1]}\cdot \frac{1}{2},
    %\end{array}
    %$$ 
    %where the last inequality is because $\Pr[R\in S_1]\leq 1$. So that $\size{\mathbb{E}}[f(R) \mid R_i = 0]-\mathbb{E}[f(R) \mid R_i = 1]\leq %4\alpha$.
    Define $I_{bad}=\{i \colon \size{\Pr[R_i=0|R\in S_1]-\Pr[R_i=1|R\in S_1]} \geq 2\alpha\}$ and $k=\size{I_{bad}}$, and denote $I_{bad}=\{i_1,\dots,i_k\}$. Define $(X_{i_1}, \ldots X_{i_k}) = (R_{i_1},\dots,R_{i_k})\mid_{R \in S_1}$. 
    Consider the min-entropy
	$$
	\begin{array}{rl}
		H_{min}(X_{i_1},\dots,X_{i_k})&\leq H(X_{i_1},\dots,X_{i_k})\\
		&\leq \sum_{j=1}^k H(X_{i_j})\\
		&\leq k\cdot \left(-(\frac{1}{2}+2\alpha)\cdot\log(\frac{1}{2}+2\alpha)-(\frac{1}{2}-2\alpha)\cdot\log(\frac{1}{2}-2\alpha)\right)\\
            &=k\cdot \left(-(\frac{1}{2}+2\alpha)\cdot(\log(1+4\alpha)-1)-(\frac{1}{2}-2\alpha)\cdot(\log(1-4\alpha)-1)\right)\\
            &=k\cdot \left(1-(\frac{1}{2}+2\alpha)\cdot\log(1+4\alpha)-(\frac{1}{2}-2\alpha)\cdot\log(1-4\alpha)\right),
		
	\end{array}
	$$
	where $H_{min}(Y)$ is the minimum entropy of $Y$ and $H(Y)$ is the Shannon entropy of $Y$.\Enote{add to preliminaries.}
        The third inequality holds since by the definition of $I_{bad}$, for every $j \in [k]$ it holds that $\size{\pr{X_{i_j} = 1}-\pr{X_{i_j} = 0}} > 2\alpha$, and therefore $H(X_{i_j}) \leq H(1/2 + 2\alpha)$\Enote{define}.
	
	Therefore, there exists $b_1,\dots,b_k\in\{0,1\}$, such that 
	
	\begin{align}\label{eq:min-entropy-result}
		\Pr\left[(R_{i_1},\ldots,R_{i_k}) = (b_1,\ldots,b_k) \mid R\in S_1\right]
		&= \pr{(X_{i_1},\ldots,X_{i_k}) = (b_1,\ldots,b_k)}\\
		&= 2^{-H_{min}(X_{i_1},\dots,X_{i_k})}\nonumber\\
		&\geq 2^{k\cdot \left(-1+(\frac{1}{2}+2\alpha)\cdot\log(1+4\alpha)+(\frac{1}{2}-2\alpha)\cdot\log(1-4\alpha)\right)}.\nonumber
	\end{align}
	
	Let $S_{bad}=\{r \in \zo^n  \colon \set{(r_{i_1},\ldots,r_{i_k}) = (b_1,\ldots,b_k)} \land \set{r\in S_1}\}$.
	It holds that
	\begin{align*}
		|S_{bad}|
		&= \size{S_1} \cdot \Pr\left[(R_{i_1},\ldots,R_{i_k}) = (b_1,\ldots,b_k) \mid R\in S_1\right]\\
		&\geq \alpha\cdot 2^{n-1}\cdot2^{k\cdot \left(-1+(\frac{1}{2}+2\alpha)\cdot\log(1+4\alpha)+(\frac{1}{2}-2\alpha)\cdot\log(1-4\alpha)\right)},
	\end{align*} 
	where the inequality holds by \cref{eq:min-entropy-result} and since $\size{S_1} \geq \alpha\cdot 2^{n-1}$.
	Notice that any string in $S_{bad}$ depends on at most $n-k$ bits. It implies that $|S_{bad}|\leq 2^{n-k}$. Therefore, we have
	$$
	\begin{array}{rl}
		&2^{n-k}\geq \alpha\cdot 2^{n-1}\cdot2^{k\cdot \left(-1+(\frac{1}{2}+2\alpha)\cdot\log(1+4\alpha)+(\frac{1}{2}-2\alpha)\cdot\log(1-4\alpha)\right)} \\
		\Rightarrow& n-k \geq \log \alpha+n-1+k\cdot \left(-1+(\frac{1}{2}+2\alpha)\cdot\log(1+4\alpha)+(\frac{1}{2}-2\alpha)\cdot\log(1-4\alpha)\right)\\
		\Rightarrow& 1-\log \alpha \geq k\cdot((\frac{1}{2}+2\alpha)\cdot\log(1+4\alpha)+(\frac{1}{2}-2\alpha)\cdot\log(1-4\alpha))\\
		\Rightarrow& 1-\log \alpha \geq k\cdot(4\alpha\cdot\log(1+4\alpha)+(\frac{1}{2}-2\alpha)\cdot\log(1-16\alpha^2))\\
        \Rightarrow& 1-\log\alpha \geq k\cdot(15.9\alpha^2-8\alpha^2+32\alpha^3)=k\cdot(7.9\alpha^2+32\alpha^3)>0.5k\alpha^2\\
		\Rightarrow& k\leq \frac{2-2\log \alpha}{\alpha^2} = \frac{2+2\log (1/\alpha)}{\alpha^2},
	\end{array}
	$$
	Where the third transition holds since 
	\begin{align*}
		\lefteqn{(\frac{1}{2}+2\alpha)\cdot\log(1+4\alpha)+(\frac{1}{2}-2\alpha)\cdot\log(1-4\alpha)}\\
		&= 4\alpha\cdot\log(1+4\alpha) + (\frac{1}{2}-2\alpha)\paren{\log(1+4\alpha)+\log(1-4\alpha)}\\
		&= 4\alpha\cdot\log(1+4\alpha)+(\frac{1}{2}-2\alpha)\cdot\log(1-16\alpha^2),
	\end{align*}
	and the forth transition holds since $4\alpha\cdot\log(1+4\alpha)+(\frac{1}{2}-2\alpha)\cdot\log(1-16\alpha^2) > 15.9\alpha^2-8\alpha^2+32\alpha^3$ for $\alpha < 0.01$.
	Thus, we conclude that 
	$$
	\Pr_{i\la[n]}\left[\size{\mathbb{E}[f(R) \mid R_i=0]-\mathbb{E}[f(R) \mid R_i = 1]}\geq \alpha\right]\leq \frac{k}{n}\leq \frac{2+2\log (1/\alpha)}{n\alpha^2}.
	$$
\end{proof}
}


\subsection{Channels and Two-Party Protocols}\label{sec:protocol}

\paragraph{Channels.}A channel is simply a distribution of a pair of tuples defined as follows. 
\begin{definition}[Channels]\label{def:channel} A {\sf channel} $C_{(X,U)(Y,V)}$ of size $\isize$ over alphabet $\Sigma$ is a probability distribution over $(\Sigma^\isize \times\zo^\ast) \times(\Sigma^\isize \times\zo^\ast)$. The ensemble $C_{(X,U)(Y,V)}= \set{C_{(X_\pk,U_\pk)(Y_\pk,V_\pk)}}_{\pk\in \N}$ is an $\isize$-size channel ensemble, if for every $\pk\in \N$, $C_{(X_\pk,U_\pk)(Y_\pk,V_\pk)}$ is an $\isize(\pk)$-size channel. %We denote a channel of size one by a \emph{single-bit} channel. 
We refer to $X$ and $Y$ as the {\sf local outputs}, and to $U$ and $V$ as the {\sf views}.	
\end{definition}

We view a  channel as the experiment in which there are two parties $\Ac$ and $\Bc$.  Party $\Ac$ receives ``output'' $X$ and ``view'' $U$, and party $\Bc$ receives ``output'' $Y$ and ``view'' $V$. Unless stated otherwise, the channels we consider are over the alphabet $\Sigma = \oo$. We naturally identify channels with the distribution that characterizes their output.








\subsubsection{Two-Party Protocols}

A two-party protocol $\Pi=(\Ac,\Bc)$ is \ppt if the running time of both parties is polynomial in their input length. We let $\Pi(x,y)(z)$ or $(\Ac(x),\Bc(y))(z)$ denote a random execution of $\Pi$ on a common input $z$, and private inputs $x,y$.%We assume \wlg that a protocol has a common output (part of its transcript).\Jnote{This is not really the case we consider in this paper..}

\begin{definition}[Oracle-aided protocols]\label{def:ChannelAidedProtocol}
	In a two-party protocol $\Pi$ with oracle access to a {\sf protocol} $\Psi$, denoted $\Pi^\Psi$, the parties make use of the \textit{next-message function} of $\Psi$.\footnote{The function that on a partial view of one of the parties, returns its next message.} In a two-party protocol $\Pi$ with oracle access to a {\sf channel} $C_{Z W}$, denoted $\Pi^C$, the parties can jointly invoke $C$ for several times. In each call, an independent pair $(z,w)$ is sampled according to $C_{Z W}$, one party gets $z$, the other gets $w$.
\end{definition}


\begin{definition}[The channel of a protocol]\label{def:ChannlOfProtocol}
	For a no-input two-party protocol $\Pi= (\Ac,\Bc)$, we associate the channel $C_\Pi$, defined by $\C_\Pi= C_{(X, U),(Y, V)}$, where $X$ and $Y$ are the local outputs of $\Ac$ and $\Bc$ (respectively) and
	$U$ and $V$ are the local views of $\Ac$ and $\Bc$ (respectively).
    
	For a two-party protocol $\Pi$ that gets a security parameter $1^\pk$ as its (only, common) input, we associate the channel ensemble $ \set{C_{\Pi(1^\pk)}}_{\pk\in \N}$. 
\end{definition}

\begin{definition}[$(\alpha,\gamma)$-Accurate channel]\label{def:accurate-func}
	A channel $C = C_{(X, U),(Y, V)}$ is {\sf $(\alpha,\gamma)$-accurate for the function $f$}, if $\ppr{C}{\size{\out(V)-f(X,Y)}\leq \alpha}\ge \gamma$, where $\out(V)$ is the designated output.
    A channel ensemble $C_{(X, U),(Y, V)}= \set{C_{(X_\pk, U_\pk),(Y_\pk, V_\pk)}}_{\pk\in \N}$ is  $(\alpha,\gamma)$-accurate for  $f$ if $C_{(X_\pk, U_\pk),(Y_\pk, V_\pk)}$ is $(\alpha(\pk),\gamma(\pk))$-accurate for $f$, for every $\pk \in \N$.
\end{definition}

\subsubsection{Differentially Private Channels}\label{sec:DPChannel}
Differentially private channels are naturally defined as follows:
\begin{definition}[Differentially private channels]\label{def:DPChannel}
	An $n$-size channel $C = C_{(X, U),(Y, V)}$ with $X, Y$ over $\oo^n$ 
	is {\sf$(\eps,\delta)$-differentially private} (denoted $(\eps,\delta)$-$\DP$) if for every $x \in \Supp(X)$ there exists an $n$-size $(\eps,\delta)$-$\DP$ mechanisms $\Mc_x$ such that $(X,Y,U) \equiv (X,Y,\Mc_X(Y))$, and for every $y \in \Supp(Y)$ there exists an $n$-size $(\eps,\delta)$-$\DP$ mechanisms $\Mc_y'$ such that $(X,Y,V) \equiv (X,Y,\Mc_Y'(X))$. In addition, we say that the channel is \emph{uniform} if $X$ and $Y$ are independent random variables uniformly distributed in $\oo^n$. 
\end{definition}

\begin{definition}[Computational differentially private channels]\label{def:CDPChannel}
	An $n$-size channel ensemble $C = \set{C_{(X_\pk, U_\pk),(Y_\pk, V_\pk)}}_{\pk\in\N}$ with $X_\pk, Y_\pk$ over $\oo^n$ 
	is {\sf$(\eps,\delta)$-computationally differentially private} (denoted $(\eps,\delta)$-$\CDP$) if for every ensemble $\set{x_\pk \in \Supp(X_\pk)}_{\pk\in\N}$ there exists an $n$-size $(\eps,\delta)$-\CDP mechanisms ensemble $\set{\Mc_{x_\pk}}_{\pk\in\N}$ such that $(X_\pk,Y_\pk,U_\pk) \equiv (X_\pk,Y_\pk,\Mc_{X_\pk}(Y_\pk))$, for every $\pk\in\N$, and for every ensemble $\set{y_\pk \in \Supp(Y_\pk)}_{\pk\in\N}$ there exists an $n$-size $(\eps,\delta)$-$\CDP$ mechanisms ensemble $\set{\Mc'_{y_\pk}}_{\pk\in\N}$ such that $(X_\pk,Y_\pk,V_\pk) \equiv (X_\pk,Y_\pk,\Mc_{Y_\pk}'(X_\pk))$ for every $\pk\in \N$. In addition, we say that the channel is \emph{uniform} if $X_\pk$ and $Y_\pk$ are independent random variables uniformly distributed in $\{\pm 1\}^n$ for all $\pk\in\N$.
\end{definition}




% \begin{lemma}~\label{lem:dp-sv-source}
% 	Let $P$ be an $\varepsilon$-DP randomized protocol. Let $X$ and $Y$ be independent random variables uniformly distributed in $\{\pm 1\}^n$ and let random variable $\Pi(X,Y)$ denote the transcript of running $P(X,y)$. Then for every $\pi\in Supp(\Pi)$, the random variables corresponding to the inputs conditioned on transcript $\pi$, $X_\pi$ and $Y_\pi$, are independent $e^{-\varepsilon}$-strong SV source.
% \end{lemma}





\subsubsection{Weak Erasure Channel (\WEC)}

\begin{definition}[\WEC]\label{def:WEC}
	A channel $((O_A,V_A), (O_B,V_B))$ with $O_A \in \set{0,1}$ and $O_B \in \set{0,1,\bot}$ is a {\sf weak erasure channel}, denoted $(\alpha,p,q)$-$\WEC$, if:
	\begin{itemize}
		%\item $O_A\in \set{-1,1}$ and $O_B\in \set{-1,1,\bot}$.
		\item Random erasure: $\pr{O_B = \perp} = 1/2$.
		
		\item Agreement: $\pr{O_A\ne O_B\mid O_B\ne \bot}\le \alpha$.
		
		\item Secrecy:
		
		\begin{enumerate}
			\item For every algorithm $\Dc$ it holds that\label{WEC:item:A}
			\begin{align*}
				%\size{\pr{\Ac(O_A,V_A) = 1 \mid O_B \neq \perp} - \pr{\Ac(O_A,V_A) = 1 \mid O_B = \perp}} \le p
				\size{\pr{\Dc(V_A) = 1 \mid O_B \neq \perp} - \pr{\Dc(V_A) = 1 \mid O_B = \perp}} \le p
			\end{align*}
			(Alice doesn't know if $O_B = \perp$.)
			
			\item For every algorithm $\Dc$ it holds that\label{WEC:item:B}
			\begin{align*}
				\pr{\Dc(V_B) = O_A \mid O_B=\bot} \leq \frac{1+q}{2}.
			\end{align*}
			(i.e., if $O_B=\bot$, Bob don't know what is the value of $O_A$).
			
			%\item $SD((O_A U|O_B=\bot),(O_A U|O_B\ne \bot))\le p$ (The sender don't know if $O_B=\bot$).
			
			%\item $SD(V O_A|O_B=\bot,V(-O_A)|O_B=\bot)\le q$ (If $O_B=\bot$, Bob don't know what the value of $O_A$).
		\end{enumerate}
	\end{itemize}
   We say that a channel ensemble $C=\set{C_\pk}_{\pk\in N}$ is a {\sf computational weak erasure channel}, denoted $(\alpha,p,q)$-\CompWEC, if for every \ppt algorithm $\Dc$ and every sufficiently large $\pk\in\N$, $C_\pk$ satisfies the properties stated in the items above, where the secrecy property holds with respect to a \ppt algorithm $\Dc$. A protocol $\Lambda$ is said to be $(\alpha,p,q)$-$\CompWEC$, if the ensemble induces by the protocol (that is, $C=\set{C_{\Lambda(\pk)}}_{\pk\in\N}$) is $(\alpha,p,q)$-$\CompWEC$.  
\end{definition}



\subsubsection{Approximate Weak Erasure Channel (\AWEC)}\label{sec:AWEC}

\begin{definition}[\AWEC]\label{def:AWEC}
	A channel $C = ((O_A,V_A), (O_B,V_B))$ over $([-n,n] \times \zo^*) \times (([-n,n] \cup \bot)  \times \zo^*)$ is an {\sf approximate weak erasure channel}, denoted $(\ell,\alpha,p,q)$-\AWEC if:
	\begin{itemize}
		
		\item Random erasure: $\pr{O_B = \perp} = 1/2$.
		
		\item Accuracy: $\pr{\size{O_A - O_B} > \ell \mid O_B \ne \bot}\le \alpha$.
		
		\item Secrecy:
		
		\begin{enumerate}
			\item For every algorithm $\Dc$ it holds that\label{AWEC:item:A}
			\begin{align*}
				%\size{\pr{\Ac(O_A,V_A) = 1 \mid O_B \neq \perp} - \pr{\Ac(O_A,V_A) = 1 \mid O_B = \perp}} \le p
				\size{\pr{\Dc(V_A) = 1 \mid O_B \neq \perp} - \pr{\Dc(V_A) = 1 \mid O_B = \perp}} \le p
			\end{align*}
			(Alice doesn't know if $O_B=\bot$).
			
			\item For every algorithm $\Dc$ it holds that\label{AWEC:item:B}
			\begin{align*}
				\pr{\size{\Dc(V_B) - O_A} \leq 1000 \ell \mid O_B=\bot} \leq q.
			\end{align*}
			(i.e., if $O_B=\bot$, Bob can't estimate the value of $O_A$ with error $\leq 1000 \ell$).
		\end{enumerate}
	\end{itemize}
     We say that a channel ensemble $C=\set{C_\pk}_{\pk\in N}$ is a {\sf computational approximate weak erasure channel}, denoted $(\ell,\alpha,p,q)$-\CompAWEC, if for every \ppt algorithm $\Dc$ and every sufficiently large $\pk\in\N$, $C_\pk$ satisfies the properties stated in the items above. A protocol $\Gamma$ is said to be $(\ell,\alpha,p,q)$-$\CompAWEC$, if the ensemble induced by the protocol (that is, $C=\set{C_{\Gamma(\pk)}}_{\pk\in\N}$) is $(\ell,\alpha,p,q)$-$\CompAWEC$.  
\end{definition}

We will make use of the following lemma, which shows that for some choices of the parameters, \AWEC implies \WEC. The lemma is proven in \cref{sec:AWEC-to-WEC}.

\begin{lemma}\label{lemma:AWEC-to-WEC}
	For every $\ell> 0$, there exists a \ppt protocol $\Lambda = (\Pc_1,\Pc_2)$ such that given an oracle access to an $(\ell,\alpha,p,q)$-\AWEC $C$, the channel $\tilde{C}$ induced by $\Lambda^C$ is $(\alpha'=\alpha+0.001,\: p' = p ,\:  q' = 1/2 + 2(q+0.01))$-\WEC.
	Furthermore, the proof is constructive in a black-box manner:
	\begin{enumerate}
		\item There exists an oracle-aided \ppt algorithm $\Ec_1$ such that for every channel $C = ((\OA,\VA), (\OB,\VB))$ and algorithm $\Dc$ violating the \WEC secrecy property~\ref{WEC:item:A} of $\tilde{C}$, algorithm $\Ec_1^{\Dc}$ violates the \AWEC secrecy property~\ref{AWEC:item:A} of $C$.
		
		\item There exists an oracle-aided \ppt algorithm $\Ec_2$ such that for every channel $C = ((\OA,\VA), (\OB,\VB))$ and algorithm $\Dc$ violating the \WEC secrecy property~\ref{WEC:item:B} of $\tilde{C}$, algorithm $\Ec_2^{\Dc}$ violates the \AWEC secrecy property~\ref{AWEC:item:B} of $C$.
	\end{enumerate}
\end{lemma}

Since \cref{lemma:AWEC-to-WEC} is constructive, the following is an immediate corollary.
\begin{corollary}\label{cor:CompAWEC to CompWEC}
There exists an oracle aided \ppt protocol $\Lambda$, such that given a protocol $\Gamma$ that induces $(\ell,\alpha,p,q)$-\CompAWEC, it holds that $\Lambda^\Gamma$ is $(\alpha'=\alpha+0.001,\: p' = p ,\:  q' = 1/2 + 2(q+0.01))$-\CompWEC.  
\end{corollary}
\begin{proof}[Proof of \ref{cor:CompAWEC to CompWEC}]
Let $\Lambda$ be the \ppt algorithm guaranteed  by Lemma \ref{lemma:AWEC-to-WEC}. Given an $(\ell,\alpha,p,q)$-\CompAWEC protocol $\Gamma$, we define $\Lambda(\pk)={\Lambda^{\Gamma(\pk)}(\pk)}$. Assume towards a contradiction that $\Lambda$ is not a $(\alpha',p',q')$-\CompWEC. It follows that there exists a \ppt $\Dc$ that for infinity many $\pk\in\N$ contradicts one of the \WEC secrecy properties of channel ensemble $\set{C_{\Lambda(\pk)}}_{\pk\in\N}$. Fix $\pk\in\N$ for which this holds. By Lemma \ref{lemma:AWEC-to-WEC}, there exists a \ppt $\Ec^\Dc$ that for every such $\pk$  contradicts one of the secrecy properties of the channel $C_{\Gamma(\pk)}$. This implies that for infinity many $\pk\in\N$, $\Ec^\Dc$  contradict the secrecy of the channel ensemble $\set{C_{\Gamma(\pk)}}_{\pk\in\N}$, which is a contradiction since this would means that $\Gamma$ is not a $(\ell,\alpha,p,q)$-\CompAWEC.       
\end{proof}



\subsection{Oblivious Transfer (\OT)}

\paragraph{Secure Computation.}
We use the standard notion of securely computing a functionality, \cf  \cite{Goldreich04}.
\begin{definition}[Secure computation]\label{def:SFE}
	A two-party protocol {\sf securely computes a functionality $f$}, if it does so according to the real/ideal paradigm.   We add the term perfectly/statistically/computationally/non-uniform computationally, if the simulator's output is  perfect/statistical/computationally indistinguishable/  non-uniformly indistinguishable from  the real distribution.  The protocol have the above notions of security {\sf against semi-honest  adversaries}, if its security only  guaranteed to holds against an adversary that follows the prescribed protocol.   Finally, for the case of perfectly secure computation, we naturally apply the above notion also to the non-asymptotic case: the protocol with no security parameter perfectly  compute a functionality $f$.
	
	A two-party protocol {\sf securely computes a functionality ensemble $f$ with oracle to a channel $C$}, if it does so according to the above definition when the parties have access to a trusted party computing $C$. All the above adjectives naturally extend to this setting.
\end{definition}

\paragraph{Oblivious Transfer.}
The (one-out-of-two) oblivious transfer functionality is defined as follows.
\begin{definition}[oblivious transfer functionality $f_{\OT}$]\label{def:OTfunc}
	The oblivious transfer functionality over $\zo \times (\zs)^2$ is defined by  $f_{\OT} (i,(\sigma_0,\sigma_1)) = (\perp,\sigma_i)$.
\end{definition}
A protocol is $\ast$ secure OT,   for \\$\ast\in \set{\text{semi-honest statistically/computationally/computationally non-uniform}}$, if it  compute the $f_{\OT}$  functionality with $\ast$ security.





% \begin{definition}[Computational oblivious transfer, semi-honest model]
% A protocol $\Pi=(\Ac,\Bc)$ is a semi-honest 1-out-of-2 computational oblivious transfer (comp-OT) protocol if the following holds. Given a common input $1^{\pk}$, the parties $\Ac$ and $\Bc$ run the protocol $\Pi(1^\pk)$ (in an honest manner) and    
% $\Ac$ outputs $X=(m_1,m_2)\in \zo\times\zo$ and has a view $U$ and $\Bc$ outputs $Y=(i,\hat{m})\in\zo\times\zo$ and has a view $V$, and the following properties are satisfied:
% \begin{enumerate}
%     \item \textbf{Correctness:} 
%     $\pr{\hat{m}\neq m_i}<\negl(\pk).$ 
    
%     \item \textbf{A's Privacy:} For every \ppt $\Dc$ and every sufficiently large $\pk$:
%     $\pr{\Dc(V)=m_{i-1}}<(1+\negl(\pk))/2$
    
%     \item \textbf{B's Privacy:} For every \ppt $\Dc$ and every sufficiently large $\pk$:
%     $\pr{\Dc(U)=i}<(1+\negl(\pk))/2$  
% \end{enumerate}
% \end{definition}

We make use of the following useful results by Wullschleger on oblivious transfer amplification from weak channels.
\begin{theorem}[\cite{Wullschleger09}, from \WEC to statistically secure \OT]\label{thm:WEC TO OT IT}
    There exists an oracle aided protocol $\Pi$ such that the following holds: Given a $(\alpha,p,q)$-\WEC $C$, if $44(\alpha+p)\le 1-q$ then $\Pi^{C}(1^\pk)$ is a semi-honest statistically secure \OT.
\end{theorem}

The following computational version of \cref{thm:WEC TO OT IT} is implicit in \cite{Wullschleger09} and is based on the computational proof explicitly stated in \cite{Wul07} (see Section 6 in \cite{Wullschleger09} for discussion).   

\begin{theorem}[\cite{Wullschleger09,   Wul07}, from \CompWEC to computinally secure \OT]\label{thm:WEC TO OT Comp}
    There exists an oracle aided protocol $\Pi$ such that the following holds: Given a $(\alpha,p,q)$-\CompWEC protocol $\Lambda$, if $44(\alpha+p)\le 1-q$ then $\Pi^{\Lambda}$ is a semi-honest computational secure \OT.
\end{theorem}



% \begin{definition}[Computational 1-out-of-2 Oblivious Transfer, semi-honest model]
% A protocol $\Pi=(\Ac,\Bc)$ is a semi-honest 1-out-of-2 $(\eps,\alpha,\beta)$-oblivious transfer (OT) protocol if the following holds. 

% The parties $\Ac$ and $\Bc$ run the protocol (in an honest manner) and    
% $\Ac$ outputs $X=(m_1,m_2)\in \zo\times\zo$ and has a view $U$ and $\Bc$ outputs $Y=(i,\hat{m})\in\zo\times\zo$ and has a view $V$, and following properties are satisfied:
% \begin{enumerate}
%     \item \textbf{Correctness:} 
%     $\pr{\hat{m}\neq m_i}<\eps.$ 
    
%     \item \textbf{A's Privacy:} For every adversary $\Dc$:
%     $\pr{\Dc(V)=m_{i-1}}<(1+\alpha)/2$
    
%     \item \textbf{B's Privacy:} For every adversary $\Dc$: $\pr{\Dc(U)=i}<(1+\beta)/2$  
% \end{enumerate}
% \end{definition}
\section{Transitive Relation Learning}
\label{sec:til}
%
In this section, we present our novel model checking algorithm \emph{Transitive Relation
Learning} (TRL) in detail, see \Cref{alg}.
%
Here and in the following, for all
$i,j \in \NN_+ = \NN \setminus \{0\}$
%JG added definition of N_+
we define $\mu_{i,j}(x') \Def \ind{x}{i+j}$ if $x' \in \vec{x}'$ and $\mu_{i,j}(x) \Def \ind{x}{i}$, otherwise.
%
So in particular, we have $\mu_{i,j}(\vec{x}) = \ind{\vec{x}}{i}$ and $\mu_{i,j}(\vec{x}') = \ind{\vec{x}}{i+j}$, where we assume that $\ind{\vec{x}}{1},\ind{\vec{x}}{2}, \ldots \in \VV^d$ are disjoint vectors of pairwise different fresh variables.
%
Intuitively, the variables $\ind{\vec{x}}{i}$ represent the $i^{th}$ state in a run, and applying $\mu_{i,j}$ to a relational formula yields a formula that relates the $i^{th}$ and the $(i+j)^{th}$ state of a run.
%
For convenience, we define $\mu_{i} \Def \mu_{i,1}$ for all $i \in \NN$, i.e., $\mu_i(\vec{x}) = \ind{\vec{x}}{i}$ and $\mu_i(\vec{x}') = \ind{\vec{x}}{i+1}$.
%
As in SMT-based BMC, TRL uses an incremental SMT solver to unroll the transition relation step by step (\Cref{alg:unroll}), but in contrast to BMC, TRL infers \emph{learned relations} on the fly (\Cref{alg:learn2}).
%
The \emph{input formula} $\tau$ as well as all learned relations are stored in $\vec{\pi}$.
%
Before each unrolling, we set a backtracking point with the command $\push$ and add a suitably variable-renamed version of the description of the error states to the SMT problem in \Cref{alg:err1}.
%
Then the command $\checksat$ checks for reachability of error states, and the command $\pop$ removes all formulas from the SMT problem that have been added since the last invocation of $\push$ (\Cref{alg:err2}), i.e., it removes the encoding of the error states (unless the check succeeds, so that TRL fails).
%
For each unrolling, suitably variable renamed variants of $\vec{\pi}$'s elements are added to the underlying SMT problem with the command $\add$ in \Cref{alg:unroll}.
%
If no error state is reachable after $b-1$ steps, but the transition relation cannot be
unrolled $b$ times (i.e., the SMT problem that corresponds to the $b$-fold unrolling is
unsatisfiable), then the diameter of the analyzed system (including learned relations) has
been reached, and hence safety has been proven (\Cref{alg:safe}).

The remainder of this section is structured as follows:
%
First, \Cref{sec:basics} introduces \emph{conjunctive variable projections} that are used to
compute the \emph{trace} (\Cref{alg:trace} of \Cref{alg}).
%
Next, \Cref{sec:loops} defines \emph{loops} and
discusses how to find \emph{non-redundant loops} that are suitable for learning new relations (\Cref{alg:loop,alg:model,alg:redundant,alg:learn1}).
%
Then, \Cref{sec:transitiveProjections} introduces \emph{transitive projections}
that are used to learn relations (\Cref{alg:trans,alg:learn2}).
% FF that sounds weird, I think
%, and discuss their transitivity (see \Cref{alg:trans}).
%JG ok
\report{Afterwards}\paper{Finally}, \Cref{sec:block} presents
\emph{blocking clauses}, which ensure that learned re\-la\-tions are preferred over other (sequences of) transitions
% needed for \Cref{alg:block1}, \ref{alg:safe}, and
(\Cref{alg:block1,alg:pick,alg:block2,alg:backtrack}).
\report{Finally, we illustrate \Cref{alg} with
a complete example in \Cref{sec:example}.}

\begin{algorithm}[t]
  $b \gets 0; \quad \vec{\pi} \gets [\tau]; \quad \blocked \gets \emptyset$\; \label{alg:init}
  $\add(\mu_{1}(\psi_\init))$ \tcp*{encode the initial states}
  \While(\tcp*[f]{main loop}){$\top$}{
    $b\increment; \quad \push(); \quad \add(\mu_{b}(\psi_\err))$ \tcp*{encode the error states} \label{alg:err1}
    \leIf{$\checksat()$}{
      \Return{$\unknown$} \label{alg:err2}
    }{
      $\pop()$ \tcp*[f]{check their reachability\hspace{-.9em}}
    }
    $\push()$ \tcp*{add backtracking point}
    \lIf(\tcp*[f]{encode transitivity}){$b>1$}{$\add(\ind[\id]{x}{b} \doteq 1 \lor
      \ind[\id]{x}{b} \not\doteq \ind[\id]{x}{b-1})$} \label{alg:trans}
    $\add(\mu_{b}(\bigvee_{n=1}^{|\vec{\pi}|} (\pi_n \land x_\id \doteq n)))$ \tcp*{encode
      $\to_\tau$ and learned relations} \label{alg:unroll}
    $\add(\bigwedge_{(b,\pi) \in \blocked} \pi)$ \tcp*{add blocking clauses for this $b$} \label{alg:block1}
    \lIf{$\neg\checksat()$}{
      \Return{$\safe$} \label{alg:safe} \tcp*[f]{check if the search space is exhausted}}
    $\sigma \gets \getmodel(); \quad \vec{\tau} \gets \trace_b(\sigma,\vec{\pi})$ \label{alg:trace} \tcp*{build trace from current model}
    \If(\tcp*[f]{search loop}){$[\tau_s,\ldots,\tau_{s+\ell-1}]$ is a loop \label{alg:loop}}{
      $\sigma_{\Loop} \gets [x / \sigma(\mu_{s,\ell}(x)) \mid x \in \vec{x} \cup \vec{x}']$ \label{alg:model}\tcp*{build the valuation for the loop}
      \If(\tcp*[f]{redundancy check}){no $\pi \in \tail(\vec{\pi})$ is consistent with $\sigma_{\Loop}$ \label{alg:redundant}}{
        $\tau_{\Loop} \gets \mu_{s,\ell}^{-1}(\bigwedge_{i=s}^{s+\ell-1} \mu_{i}(\tau_i))$ \label{alg:learn1} \tcp*{build the loop}
        $\vec{\pi} \gets \vec{\pi}\concat\tip(\tau_{\Loop}, \sigma \circ \mu_{s,\ell})$ \label{alg:learn2} \tcp*[f]{learn relation}
      }
      $\text{let } \pi \in \tail(\vec{\pi}) \text{ and } \overline{\sigma} \supseteq
      \sigma_{\Loop}$ s.t.\ $\overline{\sigma} \models
\pi$ \tcp*{pick suitable learned rel.} \label{alg:pick}
      $\blocked.\add(s+\ell-1,\blockingclause(s,\ell,\pi,\overline{\sigma}))$ \tcp*{block the loop} \label{alg:block2}
      \lWhile{$b > s$}{$\{ \, \pop(); \ b\decrement \,\}$ \tcp*[f]{backtrack to the start of the loop}\label{alg:backtrack}}
    }
  }
  \caption{TRL -- Input: a safety problem $\TT = (\psi_\init,\tau,\psi_\err)$}
  \label{alg}
\end{algorithm}

\subsection{Conjunctive Variable Projections and Traces}
\label{sec:basics}

To decide when to learn a new relation, TRL inspects the \emph{trace} (Lines \ref{alg:trace} and \ref{alg:loop}).
%
The trace is a sequence of transitions induced by the formulas that have been added to the SMT problem while unrolling the transition relation, and by the current model (\Cref{alg:trace}).
%
To compute them,  we use \emph{conjunctive variable projections}, which are like
\emph{model based projections} \cite{spacer}, but always 
yield\paper{ \pagebreak[3]} conjunctions.%
%
\begin{definition}[Conjunctive Variable Projection]
  \label{def:projections}
  A function $\mbip$ is called a \emph{conjunctive variable projection} if

  \vspace{-0.5em}
  \noindent
  \begin{minipage}[t]{0.49\textwidth}
    \begin{enumerate}
    \item $\sigma \models \mbip(\tau,\sigma,X)$,
    \item $\mbip(\tau,\sigma,X) \models \tau$,
    \item $\{\mbip(\tau,\theta,X) \mid \theta \models \tau\}$ is finite,
    \end{enumerate}
  \end{minipage}
  \begin{minipage}[t]{0.49\textwidth}
    \begin{enumerate}
      \setcounter{enumi}{3}
    \item $\VV(\mbip(\tau,\sigma,X)) \subseteq X \cap \VV(\tau)$, and
    \item $\mbip(\tau,\sigma,X) \in \QF_\land(\Sigma)$
    \end{enumerate}
  \end{minipage}
  \medskip

  \noindent
  for all $\tau \in \QF(\Sigma)$, $X \subseteq \VV$, and $\sigma \models
  \tau$.
  %
  We abbreviate $\mbip(\tau,\sigma,\vec{x} \cup \vec{x}')$ by $\mbip(\tau,\sigma)$.
\end{definition}
%
So like model based projection, $\mbip$ under-approximates quantifier elimination by projecting to the variables $X$ (by (2) and (4)).
%
To do so, it implicitly performs a finite case analysis (by (3)), which is driven by
the model $\sigma$ (by (1)).
%
In contrast to model based projections, $\mbip$ always yields conjunctions (by (5)).
%
% FF I think the following remark does not make much sense with the new version of (4):
% "Note that by (1) and (4), $\mbip(\tau,\sigma,X)$ only contains variables from $\dom(\sigma) \cap X\cap \VV(\tau)$."
% The reason is that we explicitly say $\VV(\mbip(\tau,\sigma,X)) \subseteq X \cap \VV(\tau)$, and we have $\VV(\tau) \subseteq \dom(\sigma)$ by definition of $\models$.
% Hence, this remark is equivalent to (4).
%JG The purpose of this remark was to remind the reader that (1) implies
%$\VV(\mbip(\tau,\sigma,X)) \subseteq \dom(\sigma)$. This is due to our definition of
%$\models$ which is not the standard one in logic. Thus, I added this part of the remark again.
Note that by (1),
$\mbip(\tau,\sigma,X)$ may only contain variables from $\dom(\sigma)$.


\begin{remark}[$\mbip$ and $\mathsf{mbp}$]
  Conjunctive variable projections are obtained by combining a model based projection $\mathsf{mbp}$ (which satisfies \Cref{def:projections} (1--4)) with \emph{syntactic implicant projection} $\sip$ \cite{adcl}, where $\sip(\tau,\sigma)$ is the conjunction of all literals of $\tau$'s negation normal form that are satisfied by $\sigma$.
  %
  Then $\mbip(\tau,\sigma) \Def \sip(\mathsf{mbp}(\tau,\sigma),\sigma)$.
\end{remark}
%
\begin{remark}[$\mbip$ and Quantifier Elimination]
  \Cref{def:projections} (1--3) imply
  \begin{align*}
        \exists \vec{y}.\ \tau & {} \equiv \bigvee \{\mbp(\tau,\sigma) \mid \sigma \models \tau\} \label{mbp-property} \qquad \text{where $\vec{y}$ are $\tau$'s extra variables.}
\end{align*}
So $\mbp$ yields a quantifier elimination procedure $\mathsf{qe}$ which maps $\exists
\vec{y}.\ \tau$ to $\mathit{res}$:
%JG added argument of qe. This shows that this is the same argument that is used for qe
%four lines later.

\report{\vspace{-1.6em}}


\algorithmstyle{plain}
\begin{algorithm}[h!]
\nonl $\mathit{res} \gets \bot;\hspace{.75em}$ \lWhile{$\tau$ has a model $\sigma$}{$\{\mathit{res} \gets \mathit{res} \lor \mbp(\tau,\sigma);\hspace{.75em} \tau \gets \tau \land \neg \mbp(\tau,\sigma)\}$}
\end{algorithm}

\paper{\vspace{-.2em}}
\report{\vspace{-2em}}

\noindent
But for a single model $\sigma$, $\mbp(\tau,\sigma)$ under-ap\-prox\-i\-mates quantifier elimination.
\end{remark}
%
The details of implementing $\mbip$ are beyond the scope of this paper.
%
A good intuition\paper{ \pagebreak[3]} is that $\mbip(\tau,\sigma)$ just computes one disjunct of $\mathsf{qe}(\exists \vec{y}.\ \tau)$ which is satisfied by $\sigma$.
%
However, like model based projection, $\mbip$ can be implemented efficiently for many theories with effective, but very expensive quantifier elimination procedures.%
%
\begin{example}[$\mbip$]
  \label{ex:projections}
  Consider the following formula $\ind{\tau}{1 \twodots 3}$:
  \small
  \[
    \begin{array}{rcl}
      (w \doteq 0 \land \ind{x}{2} \doteq x + 1 \land \ind{y}{2} \doteq y + 1) & \lor & (\ind{w}{2} \doteq w \land w \doteq 1 \land \ind{x}{2} \doteq x - 1 \land \ind{y}{2} \doteq y - 1) \land {} \\
      (\ind{w}{2} \doteq 0 \land x' \doteq \ind{x}{2} + 1 \land y' \doteq \ind{y}{2} + 1) & \lor & (w' \doteq \ind{w}{2} \land \ind{w}{2} \doteq 1 \land x' \doteq \ind{x}{2} - 1 \land y' \doteq \ind{y}{2} - 1)
    \end{array}
  \]
  \normalsize It encodes two steps with \Cref{ex:ex1},
  where $\ind{\vec{x}}{2} = [\ind{w}{2},\ind{x}{2},\ind{y}{2}]$ represents the values after one
  step.
  In \Cref{alg:trace}, \Cref{alg} might find a run like $\sigma(\ind{\vec{x}}{1}) \to_\tau \sigma(\ind{\vec{x}}{2}) \to_\tau \sigma(\ind{\vec{x}}{3})$ for
  \[
  \begin{array}{rcl@{\;\;}l@{\;\;}l}
  \sigma & \Def  & [\ind{w}{1}/\ind{x}{1}/\ind{y}{1}/0, & \ind{w}{2}/\ind{x}{2}/\ind{y}{2}/1, & \ind{w}{3}/1,\ind{x}{3}/\ind{y}{3}/0].
  \end{array}
  \]
 Here, $[w/x/y/c, \ldots]$ abbreviates $[w/c, x/c, y/c, \ldots]$.
  %
  Then the variable renaming $\mu_{1,2}$ allows us to
instantiate the pre- and post-variables
by the first and last state,
resulting in the following model of
$\ind{\tau}{1 \twodots 3}$:
  %
   \[
    \sigma' \Def \sigma \circ \mu_{1,2} = \sigma \cup [w/x/y/0,\;\; w'/1,x'/y'/0] \qquad \text{where } (\sigma \circ \mu_{1,2})(x) = \sigma(\mu_{1,2}(x))
  \]
  %
  To get rid of $\ind{w}{2},\ind{x}{2},\ind{y}{2}$, one could compute $\mathsf{qe}(\exists \ind{w}{2},\ind{x}{2},\ind{y}{2}.\ \ind{\tau}{1 \twodots 3})$, resulting in:
  \begin{align}
    & w \doteq 0 \land w' \doteq 0 \land x' \doteq x+2 \land y' \doteq y+2 \tag{\ensuremath{\inc}} \\
    {} \lor {} & w \doteq 0 \land w' \doteq 1 \land x' \doteq x \land y' \doteq y \label{eq:mbp-ex} \tag{\ensuremath{\mathsf{eq}}}\\
    {} \lor {} &w \doteq 1 \land w' \doteq 1 \land x' \doteq x-2 \land y' \doteq y-2. \tag{\ensuremath{\dec}}
  \end{align}
  Instead, we may have $\mbp(\ind{\tau}{1 \twodots 3},\sigma') = \eqref{eq:mbp-ex}$, as
  $\sigma' \models \eqref{eq:mbp-ex}$.
  %JG Changed  $\sigma \models
   % \eqref{eq:mbp-ex}$ to $\sigma' \models
   % \eqref{eq:mbp-ex}$.
\end{example}
%
Intuitively, a relational formula $\tau$ describes how states can change, so it is composed of many different cases.
%
These cases may be given explicitly (by disjunctions) or implicitly (by
extra variables, which express non-determinism).
%
Given a model $\sigma$ of $\tau$ that describes a \emph{concrete} change of state, $\mbip$ computes a description of the corresponding case.
%
Computing \emph{all} cases amounts to eliminating all extra variables and converting the result to DNF, which is impractical.

When unrolling the transition relation in \Cref{alg:unroll} of \Cref{alg}, we identify
each relational formula $\pi_n$ with its index $n$ in the sequence $\vec{\pi}$.
%
To this end, we use a fresh variable $x_\id$, and our SMT encoding forces
$\ind[\id]{x}{i}$ to be the identifier of the relation that is used for the $i^{th}$ step.
%
Similarly to \cite{abmc}, the \emph{trace} is the sequence of transitions that results from applying $\mbip$ to the unrolling
of the transition relation that is constructed by \Cref{alg} in \Cref{alg:unroll}.
%
So a trace is a sequence of transitions that can be applied subsequently, starting in an initial state.
%
\begin{definition}[Trace]
  \label{def:trace}
  Let $\vec{\pi}$ be a sequence of relational formulas, let
  \begin{align}
    \label{eq:trace}
    \paper{\textstyle}
    \sigma \models \bigwedge_{i=1}^{b} \mu_{i}\left(\bigvee_{n=1}^{|\vec{\pi}|} (\pi_n \land x_\id \doteq n)\right) \qquad \text{where $b \in \NN_+$},
  \end{align}
  and let $\id(i) \Def \sigma(\ind[\id]{x}{i})$.
  %
  Then the \emph{trace induced by $\sigma$} is
  \[
    \trace_b(\sigma,\vec{\pi}) \Def [\mbip(\pi_{\id(i)}, \sigma \circ \mu_{i})]_{i=1}^{b}.
  \]
\end{definition}

Recall that $\mu_i$ renames $\vec{x}$ and $\vec{x}'$ into $\ind{\vec{x}}{i}$ and
$\ind{\vec{x}}{i+1}$, and $\id(i) = \sigma(\ind[\id]{x}{i})$ is the index of the
relation from $\vec{\pi}$ that is used for the $i^{th}$ step.\comment[NONE]{FF If
  we have this remark here, then we should remove the similar remark ``i.e.,
  $\mu_i(\vec{x}) = \ind{\vec{x}}{i}$
  and $\mu_i(\vec{x}') =\ind{\vec{x}}{i+1}$.\\
  JG Ok, but only if it at least saves a line. Otherwise, we can just as well keep the
  remark there as well.}
%
So each model $\sigma$ of \eqref{eq:trace} corresponds to a run $\sigma(\mu_1(\vec{x}))
\to_{\pi_{\id(1)}} \ldots \to_{\pi_{\id(b)}} \sigma(\mu_{b}(\vec{x}'))$,\paper{ \pagebreak[3]} and the trace
induced by $\sigma$ contains the transitions that were used in this run.

%
\begin{example}[Trace]
  \label{ex:trace}
  Consider the extension of $\sigma$ from \Cref{ex:projections} with $[\ind[\id]{x}{1}/1,\;
    \ind[\id]{x}{2}/1]$:
  \[
  \sigma \Def [\ind{w}{1}/\ind{x}{1}/\ind{y}{1}/0,\ind[\id]{x}{1}/1, \quad
    \ind{w}{2}/\ind{x}{2}/\ind{y}{2}/\ind[\id]{x}{2}/1, \quad \ind{w}{3}/1,\ind{x}{3}/\ind{y}{3}/0],
  \]
  Thus, $\id(1) = \sigma(\ind[\id]{x}{1}) = 1$, $\id(2) =  \sigma(\ind[\id]{x}{2}) = 1$, and
  $\pi_{\id(1)} = \pi_{\id(2)} =
  \pi_1 = \tau$.
  %
  Then
  \begin{align*}
    & \trace_2(\sigma, [\tau,\tau]) = [\mbip(\tau,\sigma \circ \mu_{1}), \mbip(\tau,\sigma \circ \mu_{2})]           \\
    {} = {} & [\mbip(\tau,[w/x/y/0, \; w'/x'/y'/1]), \mbip(\tau,[w/x/y/1, \; w'/1,x'/y'/0])] = [\tau_\inc, \tau_\dec].
  \end{align*}
\end{example}

\subsection{Loops}
\label{sec:loops}

As $\vec{\pi}$ only gives rise to finitely many transitions, the trace is bound to contain \emph{loops}, eventually (unless \Cref{alg} terminates beforehand).
%
\begin{definition}[Loop]
  A sequence of transitions $\tau_1,\ldots,\tau_k$ is called a \emph{loop} if there are $\vec{v}_0,\ldots,\vec{v}_{k+1} \in \CC^d$ such that $\vec{v}_0 \to_{\tau_1} \ldots \to_{\tau_k} \vec{v}_{k} \to_{\tau_1} \vec{v}_{k + 1}$.
\end{definition}
%
Intuitively, these loops are the reason why BMC may diverge.
%
To prevent divergence, TRL learns a new relation when a loop is detected (\Cref{alg:loop}).
%
\begin{remark}[Finding Loops]
  Loops can be detected by SMT solving.
  %JG I think that it is not correct to say that something is detected by "SMT". I think
  %it should be "SMT solving".
%
A cheaper way is to look for duplicates, but then loops are found ``later'', as
%
a trace $[\ldots, \pi, \pi, \ldots]$\linebreak is needed to detect a loop $\pi$, but one occurrence of $\pi$ is insufficient.
%
As a trade-off between precision and efficiency, our im\-ple\-men\-ta\-tion uses \emph{dependency graphs} \cite{abmc}.
\end{remark}

\begin{remark}[Disregarding ``Learned'' Loops]
  \label{remark:loops}
    One should disregard ``loops'' consisting of a single \emph{learned transition}, i.e., a transition that results from applying $\mbip$ to some $\pi \in \tail(\vec{\pi})$.
    %
    Here, $\tail(\tau\concat\vec{\pi}') \Def \vec{\pi}'$ contains all learned relations, as the first element of $\vec{\pi}$ is the input formula $\tau$.
    %
    The reason is that our goal is to deduce transitive relations, but learned relations are already transitive.
    %
    In the sequel, we assume that the check in \Cref{alg:loop} fails for such loops.
\end{remark}
%
If there are several choices for $s$ and $\ell$ in \Cref{alg:loop}, then our implementation only considers loops of minimal length and, among those, it minimizes $s$.
%
\begin{example}[Detecting Loops]
  \label{ex:loops}
  Consider the model
  \[
    \sigma \Def [\ind{w}{1}/\ind{x}{1}/\ind{y}{1}/0, \ind[\id]{x}{1}/1, \quad \ind{w}{2}/0,\ind{x}{2}/\ind{y}{2}/1]
  \]
  %
for $\tau$ from our running example (\Cref{ex:ex1}).
   Then $\trace_1(\sigma, [\tau]) = [\tau_\inc]$.
  %
  As $\tau_\inc$ is a loop, TRL learns a relation like $\tau^+_{\inc}$ at this point.
\end{example}
%
TRL only learns relations from loops that are \emph{non-redundant} w.r.t.\ all relations that have been learned before \cite{adcl}.
%
\begin{definition}[Redundancy]
  \label{def:redundancy}
  If ${\to_\tau} \subseteq {\to_{\tau'}}$, then $\tau$ is \emph{redundant} w.r.t.\ $\tau'$.
\end{definition}
% FF I think it's fine to omit the title of examples if they are directly preceded by the
% corresponding definition
% JG ok
\begin{example}
  The relation $\tau_\inc$ is redundant w.r.t.\ $\tau^+_\inc$, but $\tau_\dec$ is not.
\end{example}
%
\Cref{alg:redundant} uses a sufficient criterion for non-redundancy:
%
If all learned relations are falsified by the values before and after the loop, then
$\tau_s, \ldots, \tau_{s + \ell -1}$ cannot be simulated by a previously learned relation, so it is non-redundant and we learn a new relation.
%
The values before and after the loop are obtained from the current model $\sigma$ by
setting $\vec{x}$ to $\sigma(\ind{\vec{x}}{s})$ and $\vec{x}'$ to
$\sigma(\ind{\vec{x}}{s+\ell})$, i.e., we use $\sigma \circ \mu_{s,\ell}$ in
\Cref{alg:model}.

To learn a new relation,
we first compute the  relation\paper{ \pagebreak[3]} 
\begin{equation}
  \label{eq:loop-formula}
  \paper{\textstyle}
  \tau_\Loop \Def \mu_{s,\ell}^{-1}(\phi_\Loop) \qquad \text{where} \qquad \phi_\Loop \Def
  \bigwedge_{i=s}^{s+\ell-1} \mu_{i}(\tau_i) 
\end{equation}
of the loop in \Cref{alg:learn1}, where $\mu^{-1}_{s,\ell}$ is the inverse of $\mu_{s,\ell}$.
%
So in \Cref{ex:loops}, we have $\sigma \circ \mu_{1,1} \supseteq [w/x/y/0,\,
  w'/0,x'/y'/1]$ and $\tau_\Loop \Def \mu^{-1}_{1,1}(\mu_1(\tau_\inc)) = \tau_\inc$ as $s = \ell = 1$.
%
So $\sigma \circ \mu_{s,\ell}$ indeed corresponds to one evaluation of the loop, as $\sigma \circ \mu_{s,\ell} \models \tau_\Loop$.

To see that $\tau_\Loop$ is also the desired relation in general, note that $\phi_\Loop$ is the conjunction of the transitions that constitute the loop, where all variables are renamed as in \Cref{alg:unroll} of \Cref{alg}, i.e., in such a way that the post-variables of the $i^{th}$ step are equal to the pre-variables of the $(i+1)^{th}$ step.
%
So we have $\sigma \models \phi_\Loop$ and thus $\sigma \circ \mu_{s,\ell} \models \tau_\Loop$.
%
Hence,
we can use $\tau_\Loop$ and $\sigma \circ \mu_{s,\ell}$ to learn a new relation via
so-called \emph{transitive projections}
in \Cref{alg:learn2}.


\subsection{Transitive Projections}
\label{sec:transitiveProjections}



We now define \emph{transitive projections} that approximate transitive closures of loops.
%
As explained in \Cref{sec:overview}, we do not restrict ourselves to under- or over-approximations, but we allow ``mixtures'' of both.
%
Analogously to $\mbip$, transitive projections perform a finite case analysis that is
driven by the provided model $\sigma$.

\begin{definition}[Transitive Projection]
  \label{def:ti}
  A function $\tip$ is called a \emph{transitive projection}
  if the following holds for all transitions $\tau \in \QF(\Sigma)$ and all $\sigma \models \tau$:
  \begin{minipage}[t]{0.49\textwidth}
    \begin{enumerate}
    \item $\tip(\tau,\sigma)$ is consistent with $\sigma$
    \item $\{\tip(\tau,\theta) \mid \theta \models \tau\}$ is finite
    \end{enumerate}
  \end{minipage}
  \begin{minipage}[t]{0.49\textwidth}
    \begin{enumerate}
      \setcounter{enumi}{2}
    \item $\to_{\tip(\tau,\sigma)}$ is transitive
    \end{enumerate}
  \end{minipage}
\end{definition}

\begin{example}
  \label{ex:Transition Invariants}
  For \Cref{ex:ex1}, $\tau_\ti \Def x' - x \doteq y' - y$ over-approximates the transitive
  closure $\to^+_\tau$.
  %
  Such over-approximations are also called \emph{transition invariants} \cite{transition_invariants}.
  %
  With $\tau_\ti$, one can prove safety for any $\psi_\init$ with $\psi_\init \models x \doteq y$, as then $\psi_\init \land \tau_\ti \models x' \doteq y'$, which shows that no error state with $w \doteq 1 \land x \leq 0 \land y > 0$ is reachable.

  By using $\mbip$, TRL instead considers $\tau_\inc$ and $\tau_\dec$ separately and learns
  \begin{align*}
    \tip(\tau_\inc,\sigma_\inc) & {} \Def w \doteq 0 \land x' > x \land x' - x \doteq y' - y                   \tag{$\tau^+_{\inc}$} \\
    \tip(\tau_\dec,\sigma_\dec) & {} \Def w' \doteq w \land w \doteq 1 \land x' < x \land x' - x \doteq y' - y \tag{$\tau^+_{\dec}$}
  \end{align*}
  if $\sigma_\inc \models \tau_\inc$ and $\sigma_\dec \models \tau_\dec$.
  %
  In this way, \Cref{alg} can learn disjunctive relations like $\tau^+_{\inc} \lor \tau^+_{\dec}$, even if $\tip$ only yields conjunctive relational formulas (which is true for our current implementation of $\tip$ -- see \Cref{sec:rec} -- but not enforced by \Cref{def:ti}).
\end{example}
%
In contrast to conjunctive variable projections, $\tip(\tau,\sigma)$ may contain extra variables that do not occur in $\tau$ (which will be exploited in \Cref{sec:rec}).
%
Hence, instead of $\sigma \models \tip(\tau,\sigma)$ we require consistency with $\sigma$, i.e., $\sigma(\tip(\tau,\sigma))$ must be satisfiable.

\begin{remark}[Properties of $\tip$]
  \label{remark:properties-tip}
 Due to \Cref{def:ti} (1), our definition of $\tip$ implies
 \[
  \paper{\textstyle}
  \tau \models \exists \vec{y}. \, \bigvee_{\sigma \models \tau} \tip(\tau,\sigma), \quad \text{and thus,}
  \quad
  {\to_{\tau}} \subseteq \bigcup_{\sigma \models \tau}
  {\to_{\tip(\tau,\sigma)}},
\]
where $\vec{y}$ are the extra variables of $\bigvee_{\sigma \models \tau} \tip(\tau,\sigma)$.
%
However, \Cref{def:ti} does \emph{not} ensure
${\to^+_\tau} \subseteq \bigcup_{\sigma \models \tau}
  {\to_{\tip(\tau,\sigma)}}$.
  %
  So there is no guarantee that $\tip$ covers $\to^+_\tau$ entirely, i.e., $\tip$ cannot be used to compute transition invariants, in general.
  % 
  \Cref{def:ti} does not ensure ${\to^+_\tau} \supseteq \bigcup_{\sigma \models \tau}
  {\to_{\tip(\tau,\sigma)}}$ either, as $\tip(\tau,\sigma)$ does not imply $\sigma(\vec{x}) \to^+_\tau \sigma(\vec{x}')$.
\end{remark}

\begin{example}
  \label{Counterex-tip}
  To see that $\tip$ computes no over- or under-approximations, let
  \[
    \tau \Def x' \doteq x + 1 \land y' \doteq y + x.\paper{\pagebreak[3]}
  \]
  Then for all $\sigma \models \tau$, we might have:
  \[
    \tip(\tau,\sigma) =
    \begin{cases}
      x \geq 0 \land x' > x \land y' \geq y, & \text{if } \sigma(x) \geq 0 \\
      x < 0 \land x' > x \land y' < y,   & \text{if } \sigma(x) < 0
    \end{cases}
  \]
  However, $(x \geq 0 \land x' > x \land y' \geq y) \lor (x < 0 \land x' > x \land y' < y)$ is not an over-approximation of $\to^+_\tau$ (i.e., ${\to^+_\tau} \not\subseteq \bigcup_{\sigma \models \tau}
    {\to_{\tip(\tau,\sigma)}}$), as we have, e.g.,
  \[
    (-1,0) \to_\tau (0,-1) \to_\tau (1,-1) \to_\tau (2,0), \qquad \text{but} \qquad (-1,0) \not\to_{\tip(\tau,\sigma)} (2,0)
  \]
  for all $\sigma \models \tau$.
  % 
  Moreover, we also have ${\to^+_\tau} \not\supseteq \bigcup_{\sigma \models \tau} {\to_{\tip(\tau,\sigma)}}$, since
  \[
    (-1,0) \to_{\tip(\tau,\sigma)} (10,-20), \qquad \text{but} \qquad (-1,0) \not\to^+_\tau (10,-20)
  \]
  if $\sigma(x) < 0$.
  %
  In contrast to $\tip(\tau, \sigma)$, linear over-approximations for $\to^+_\tau$ like $x' > x$
  cannot distinguish whether $y$ increases or decreases.
\end{example}

As TRL proves safety via \emph{blocking clauses} (\Cref{sec:block}) that only block steps that are cov\-er\-ed by learned relations, the fact that $\tip$ does not yield over-ap\-prox\-i\-ma\-tions does not affect soundness.
%
However, it may cause divergence (\Cref{remark:termination}).

Recall that our SMT encoding forces $\ind[\id]{x}{i}$ to be the identifier of the relation that\linebreak is used for the $i^{th}$ step (\Cref{alg:unroll}).
%
To exploit transitivity of $\tip$, we add the constraint
$\ind[\id]{x}{b} \doteq 1 \lor \ind[\id]{x}{b} \not\doteq \ind[\id]{x}{b-1}$ in
\Cref{alg:trans}, so that learned relations (with an index $>1$) are not used several times in a row, since this is unnecessary for transitive relations.

Clearly, the specifics of $\tip$ depend on the underlying theory.
%
Our implementation for quantifier-free linear integer arithmetic will be explained in \Cref{sec:rec}.


\subsection{Blocking Clauses}
\label{sec:block}

In \Cref{alg:pick}, we are guaranteed to find a learned relation $\pi$ which is consistent
with $\sigma_\Loop$: If our sufficient criterion for non-redundancy in
\Cref{alg:redundant} failed, then the existence of $\pi$ is guaranteed.
%
Otherwise, we learned a new relation $\pi$ in \Cref{alg:learn2} which is consistent with $\sigma_\Loop \subseteq \sigma \circ \mu_{s,\ell}$ by definition of $\tip$.
%
Thus, we can use $\pi$ and a model $\overline{\sigma} \supseteq \sigma_\Loop$ of $\pi$ to record a \emph{blocking clause} in \Cref{alg:block2}.
%
\begin{definition}[Blocking Clauses]
  \label{def:blocking}
  We define:
  \[
    \blockingclause(s,\ell,\pi,\overline{\sigma}) \Def
    \begin{cases}
      \mu_{s,\ell}(\neg \mbip(\pi, \overline{\sigma})) \lor \ind[\id]{x}{s} > 1, & \text{if } \ell = 1 \\
      \mu_{s,\ell}(\neg \mbip(\pi, \overline{\sigma})),                         & \text{if } \ell > 1
    \end{cases}
  \]
\end{definition}
%
Here, $s$ and $\ell$ are natural numbers such that $[\tau_i]_{i=s}^{s+\ell-1}$ is a (possibly) redundant loop on the trace.
%
Blocking clauses exclude models that correspond to runs
\begin{equation}
  \label{blockedRun}
  \vec{v}_1 \to_{\tau_1} \ldots \to_{\tau_{s-1}} \vec{v}_s \to_{\tau_{s}} \ldots \to_{\tau_{s+\ell-1}} \vec{v}_{s+\ell}
\end{equation}
where $\vec{v}_{s} \to_{\pi} \vec{v}_{s+\ell}$. Intuitively, if $\ell = 1$ then
$\blockingclause(s,\ell,\pi,\overline{\sigma})$ states that one may still evaluate 
$\vec{v}_s$ to
$\vec{v}_{s+\ell}$, but one has to use a learned transition. If $\ell > 1$, then
$\blockingclause(s,\ell,\pi,\overline{\sigma})$
states that one may still  evaluate 
$\vec{v}_s$ to
$\vec{v}_{s+\ell}$, but not in $\ell$ steps.
More precisely, 
blocking clauses take into account that%

\vspace{-0.8em}
\noindent
\begin{minipage}{0.44\textwidth}
  \begin{equation}
    \label{prefix}
    \vec{v}_1 \to_{\tau_1} \ldots \to_{\tau_{s+\ell-2}} \vec{v}_{s+\ell-1}
  \end{equation}
\end{minipage}
\begin{minipage}{0.1\textwidth}
  \begin{equation*}
    \text{and}
  \end{equation*}
\end{minipage}
\begin{minipage}{0.44\textwidth}
  \begin{equation}
    \label{unblockedRun}
    \vec{v}_1 \to_{\tau_1} \ldots \to_{\tau_{s-1}} \vec{v}_s \to_{\pi} \vec{v}_{s+\ell}
  \end{equation}
\end{minipage}

\medskip
\noindent
must not be blocked to ensure that $\vec{v}_2,\ldots,\vec{v}_{s+\ell}$ remain reachable.
%
For the former, note that blocking clauses affect the suffix $\vec{v}_{s} \to_{\tau_s} \ldots \to_{\tau_{s+\ell-1}} \vec{v}_{s+\ell}$ of \eqref{blockedRun} (as they contain $\mu_{s,\ell}(\neg \mbip(\pi, \overline{\sigma}))$), but not \eqref{prefix}, so $\vec{v}_{2},\ldots,\vec{v}_{s+\ell-1}$ remain reachable.

Regarding \eqref{unblockedRun},\paper{ \pagebreak[3]} first consider the case $\ell > 1$.
%
Then \eqref{unblockedRun} is not affected by the blocking clause, as it requires less than $s+\ell$ steps.
%
If $\ell = 1$, then the loop that needs to be blocked is a single \emph{original transition} (i.e., a transition that results from applying $\mbip$ to $\tau$) due to \Cref{remark:loops}.
%
So $\ind[\id]{x}{s} > 1$ is falsified by \eqref{blockedRun}, as $\tau_s$ is an original transition, i.e., using it for the $s^{th}$ step implies $\ind[\id]{x}{s} \doteq 1$.
%
However, $\ind[\id]{x}{s} > 1$ is satisfied by \eqref{unblockedRun}, as $\pi$ is a learned
transition, so using it implies
$\ind[\id]{x}{s} > 1$.

\begin{remark}[Extra Variables and Negation]
In \Cref{def:blocking}, $\mbip$ is used to project $\pi$ according to the model $\overline{\sigma}$.
%
In this way, negation has the intended effect, i.e., 
\[
  [\vec{x}/\vec{v},\vec{x}'/\vec{v}'] \models \neg\mbip(\ldots) \qquad \text{iff} \qquad \vec{v} \not\to_{\mbip(\ldots)} \vec{v}',
\]
as $\mbip(\ldots)$ has no extra variables.
%
To see why this is important here, consider the relation $\tau \Def n > 0 \land x' \doteq x + n$, where $n$ is an extra variable.
%
Then $0 \to_\tau 1$, but
\[
  \neg\tau[x/0,x'/1] = (n \leq 0 \lor x' \not\doteq x + n)[x/0,x'/1] = n \leq 0 \lor 1 \not\doteq n
\]
is satisfiable, so $\neg\tau$ is not a suitable characterization of $\not\to_\tau$.
%
The reason is that $n$ is implicitly existentially quantified in $\tau$.
%
So to characterize $\not\to_\tau$, we have to negate $\exists n.\ \tau$ instead of $\tau$, resulting in $\forall n.\ n \leq 0 \lor x' \not\doteq x + n$.
%
Then, as desired,
\[
  (\forall n.\ n \leq 0 \lor x' \not\doteq x + n)[x/0,x'/1] = \forall n.\ n \leq 0 \lor 1 \not\doteq n
\]
is invalid.
%
To avoid quantifiers, we eliminate extra variables via $\mbip$ instead.
\end{remark}

In \Cref{alg:block2}, a pair consisting of $s + \ell - 1$ and the blocking clause is added to $\blocked$.
%
The first component means that the blocking clause has to be added to the SMT encoding when the transition relation is unrolled for the $(s+\ell-1)^{th}$ time, i.e., when $b = s + \ell - 1$.
%
So blocking clauses are added to the SMT encoding ``on demand'' (in \Cref{alg:block1}) to block
loops that have been found on the trace at some point.
%
Afterwards, TRL backtracks to the last step before the loop in \Cref{alg:backtrack}.

\begin{remark}[Adding Blocking Clauses]
To see why blocking clauses must only be added to the SMT encoding
in the $(s+\ell-1)^{th}$ unrolling, assume $\pi \equiv \top$.
%
Then, e.g., $\blockingclause(1,2,\pi,\overline{\sigma}) \equiv \bot$.
%
This means that unrolling the transition relation twice is superfluous, as every state can be reached in a single step with $\pi$, so the diameter is $1$.
%
But after learning $\pi$ when $b=2$ and backtracking to $b=0$, adding such a blocking clause too early (e.g., before the first unrolling of the transition relation) would \emph{immediately} result in an unsatisfiable SMT problem.
\end{remark}


\begin{example}[Blocking Redundant Loops]
  \label{ex:redundant}
  Consider the model
  \[
    \sigma \Def [\ind{w}{1}/\ind{x}{1}/\ind{y}{1}/0,\ind[\id]{x}{1}/2, \quad
      \ind{w}{2}/0,\ind{x}{2}/\ind{y}{2}/2,\ind[\id]{x}{2}/1, \quad
      \ind{x}{3}/\ind{y}{3}/3]
  \]
  and assume that TRL has already learned the relation $\tau^+_\inc$ (i.e., $\vec{\pi} = [\tau,\tau^+_\inc]$).
  %
  Moreover, assume that the trace is $[\tau^+_\inc,\tau_\inc]$, so that TRL detects the loop $\tau_\inc$.
  %
  To check for non-redundancy, we instantiate the pre- and post-variables in $\tau^+_\inc$ according to $\sigma$, taking the renaming $\mu_{2}$ into account (note that here $s = 2$, $\ell = 1$, and $\mu_{s,\ell} = \mu_{2,1} = \mu_2$):
  \[
    \sigma(\mu_{2}(\tau^+_\inc)) = \tau^+_\inc[w/0,x/y/2,x'/y'/3] \equiv \top.
  \]
  So our sufficient criterion for non-redundancy fails, as $\tau_\inc$ is indeed redundant w.r.t.\ $\tau^+_\inc$.
  %
  Thus, TRL records that the following blocking clause has to be added for the second unrolling (i.e., when $b = s + \ell -1 = 2$).
  \paper{\begin{align*}
            & \mu_{s,\ell}(\neg\mbip(\tau^+_\inc,\overline{\sigma})) \lor \ind[\id]{x}{s}
      > 1 \hspace{.4em} = \hspace{.4em} \mu_{s,\ell}(\neg\tau^+_\inc) \lor \ind[\id]{x}{s}
      > 1 \tag{as $\tau^+_\inc$ is a transition} \end{align*}
    \paper{ \vspace*{-.3cm}\pagebreak[3]}
   \begin{align*}  
   {} = {} & \mu_{2}(\neg (w \doteq 0 \land x' > x \land x' - x \doteq y' - y)) \lor \ind[\id]{x}{2} > 1\\ 
    {} = {} & (w \not\doteq 0 \lor x' \leq x \lor x' - x \not\doteq y' - y)[w/\ind{w}{2}, x/\ind{x}{2},y/\ind{y}{2}, x'/\ind{x}{3},y'/\ind{y}{3}] \lor \ind[\id]{x}{2} > 1 \\
    {} = {} & \ind{w}{2} \not\doteq 0 \lor \ind{x}{3} \leq \ind{x}{2} \lor \ind{x}{3} - \ind{x}{2} \not\doteq \ind{y}{3} - \ind{y}{2} \lor \ind[\id]{x}{2} >1
  \end{align*}}
  \report{\begin{align*}
            & \mu_{s,\ell}(\neg\mbip(\tau^+_\inc,\overline{\sigma})) \lor \ind[\id]{x}{s} > 1 \hspace{.4em} = \hspace{.4em} \mu_{s,\ell}(\neg\tau^+_\inc) \lor \ind[\id]{x}{s} > 1 \tag{as $\tau^+_\inc$ is a transition} \\
    {} = {} & \mu_{2}(\neg (w \doteq 0 \land x' > x \land x' - x \doteq y' - y)) \lor
    \ind[\id]{x}{2} > 1\\
    {} = {} & (w \not\doteq 0 \lor x' \leq x \lor x' - x \not\doteq y' - y)[w/\ind{w}{2}, x/\ind{x}{2},y/\ind{y}{2}, x'/\ind{x}{3},y'/\ind{y}{3}] \lor \ind[\id]{x}{2} > 1 \\
    {} = {} & \ind{w}{2} \not\doteq 0 \lor \ind{x}{3} \leq \ind{x}{2} \lor \ind{x}{3} - \ind{x}{2} \not\doteq \ind{y}{3} - \ind{y}{2} \lor \ind[\id]{x}{2} >1
  \end{align*}}  
  As this blocking clause is falsified by $\sigma$, it prevents TRL from finding the same model again after backtracking in \Cref{alg:backtrack}, so that TRL makes progress.
\end{example}
%
The following theorem states that our approach is sound.
%
%\paper{See \cite{arxiv} for all proofs.}%
%JG I mentioned that in the intro now.
%
\begin{restatable}
  {theorem}{soundness}
  \label{thm:soundness}
  If $\text{TRL}(\TT)$ returns $\safe$, then $\TT$ is safe.
\end{restatable}
\makeproof*{thm:soundness}{
  \soundness*
  \begin{proof}
    Consider the SMT problem that is checked in \Cref{alg:safe}.
    %
    In the $m^{th}$ iteration (starting with $m=1$), this problem is of the form $\varphi(m) \Def$
    \begin{align*}
       & \overbrace{\mu_{1}(\psi_\init)}^{\substack{\text{initial states}                                             \\
      \text{\Cref{alg:init}}}} \land                                                                                                 \\
                      & \overbrace{\bigwedge_{j=1}^{b(m)}}^{\substack{\text{one conjunct}                                          \\
        \text{per step}}}
      \left( \overbrace{(\ind[\id]{x}{j} \doteq 1 \lor \ind[\id]{x}{j} \not\doteq \ind[\id]{x}{j-1})}^{\substack{\text{transitivity} \\
          \text{\Cref{alg:trans}}}}
      \land \overbrace{\bigvee_{i=1}^{\ell'(m)} \mu_{j}(\pi_i \land x_{\id} \doteq i)}^{\substack{\text{transition relation}       \\
          \text{\Cref{alg:unroll}}}} \land \overbrace{\bigwedge_{(j,\pi) \in \blocked(m)}
      \pi}^{\mathclap{\substack{\text{blocking clauses}                                                                              \\
            \text{\Cref{alg:block1}}}}} \right)
    \end{align*}
    where $b(m)$, $\ell'(m)$, and $\blocked(m)$ are the value of $b$, the length of $\vec{\pi}$, and the values of $\blocked$ in \Cref{alg:safe} in the $m^{th}$ iteration, respectively.
    %
    For simplicity, here we use an additional variable $\ind[\id]{x}{0}$ which only occurs in the first transitivity constraint (which is not generated by \Cref{alg}):
    \[
      \ind[\id]{x}{1} \doteq 1 \lor \ind[\id]{x}{1} \not\doteq \ind[\id]{x}{0}
    \]
    Therefore, this constraint is trivially satisfiable (e.g., by setting $\ind[\id]{x}{0}$ to $1$).

    Assume that $\TT$ is unsafe, but \Cref{alg} returns $\safe$.
    %
    Then there is some $k \in \NN$ such that $\varphi(k)$ is unsatisfiable, and $\varphi(k')$ is satisfiable for all $1 \leq k' < k$ (otherwise, \Cref{alg} would have returned $\safe$ in an earlier iteration).

    As \Cref{alg} backtracks in \Cref{alg:backtrack}, we consider the sequence of natural numbers $1 \leq i_1 < \ldots < i_{b(k)} = k$ such that for all $1 \leq c \leq b(k)$, $i_c$ is the last iteration where $b=c$ in \Cref{alg:safe}.
    %
    Then for all $1 \leq B \leq b(k)$, we have $\varphi(i_B) = \phi(B)$ where
    \begin{align*}
      \phi(B) \Def & \mu_{1}(\psi_\init) \land \\
                   & \bigwedge_{j=1}^{B}
      \left( (\ind[\id]{x}{j} \doteq 1 \lor \ind[\id]{x}{j} \not\doteq \ind[\id]{x}{j-1}) \land \bigvee_{i=1}^{\ell(B)}\mu_{j}(\pi_i \land x_{\id} \doteq i) \land \bigwedge_{(j,\pi) \in \blocked(k)} \pi \right).
    \end{align*}
    Here, we have $\ell(B) \Def \ell'(i_B)$.
    %
    A blocking clause $\pi$ is only added to the SMT encoding if $(j,\pi) \in \blocked$ and $b=j$ (see \Cref{alg:block1}) and \Cref{alg} backtracks until $b \leq j$ whenever such an element is added to $\blocked$ (see \Cref{alg:backtrack}).
    %
    So when only considering the iterations $i_1, \ldots, i_{b(k)}$, then $b(i_B) = B$ and all blocking clauses of the form $(j,\pi) \in \blocked(k)$ where $j < B$ are already present when unrolling the transition relation for the $B^{th}$ time, i.e., they are already contained in $\blocked(i_B)$.

    %
    In other words, we have
    \[
      \{(j,\pi) \mid (j,\pi) \in \blocked(k) \mid j < B\} \subseteq \blocked(i_B).
    \]
    Thus, we may use $\blocked(k)$ instead of $\blocked(i_B)$ in the definition of
    $\phi$.

    Let $c \in \NN$ and $\vec{v}_0,\ldots\vec{v}_c \in \CC^d$ be arbitrary but fixed where $[\vec{x}/\vec{v}_0] \models \psi_\init$ and
    %
    \[
      \vec{v}_0 \to_{\tau} \ldots \to_{\tau} \vec{v}_c.
    \]
    %
    We use induction on $c$ to show that\footnote{While $\phi(B)$ only corresponds to a formula that is checked by \Cref{alg} if $B > 0$, $\phi(0) \equiv \mu_1(\psi_\init)$ is well defined, too.}
    \begin{multline}
      \label{eq:goal}
      \forall 0 \leq i \leq c.\ \exists B(i) < b(k), h(0,i) < \ldots < h(B(i),i).\ h(0,i) = 0 \land h(B(i),i) = i\\
      {} \land \phi(B(i)) \text{ is consistent with } [\mu_{j+1}(\vec{x})/\vec{v}_{h(j,i)} \mid 0 \leq j \leq B(i)].
    \end{multline}
    Intuitively, $B(i)$ is the number of steps that are needed to reach $\vec{v}_i$ when also using learned relations, and $\vec{v}_{h(j,i)}$ is the $j^{th}$ state in the resulting run that leads to $\vec{v}_i$.
    %
    Thus, \eqref{eq:goal} shows that for all (arbitrary long) runs that start in a state satifying $\psi_\init$, all states that are reachable with $\to_\tau$ in arbitrarily many steps can also be reached in less than $b(k)$ steps (i.e., in constantly many steps) if one may also use the (transitive) learned relations.
    %
    Of course, this only holds provided that \Cref{alg} returns $\safe$ in the $k^{th}$ iteration.

    Once we have shown \eqref{eq:goal}, we can prove the theorem:
    %
    We had assumed that $\TT$ is unsafe, i.e., that there is a reachable error state $\vec{v}_c$ and that $\varphi(k) = \varphi(i_{b(k)}) = \phi(b(k))$ is unsatisfiable.
    %
    We use \eqref{eq:goal} for $i = c$:
    %
    By \eqref{eq:goal}, there is some $B(c) < b(k)$ such that
    \[
      \phi(B(c)) \text{ is consistent with } [\mu_{j+1}(\vec{x})/\vec{v}_{h(j,c)} \mid 0 \leq j \leq B(c)].
    \]
    Moreover, as $\vec{v}_c$ is an error state, $\psi_\err$ is consistent with $[\vec{x}/\vec{v}_{c}]$ and hence,
    \[
      \mu_{B(c)+1}(\psi_\err) \text{ is consistent with } [\mu_{B(c)+1}(\vec{x})/\vec{v}_{c}] = [\mu_{B(c)+1}(\vec{x})/\vec{v}_{h(B(c),c)}].
    \]
    Thus,
    \[
      \phi(B(c)) \land \mu_{B(c)+1}(\psi_\err) \text{ is consistent with } [\mu_{j+1}(\vec{x})/\vec{v}_{h(j,c)} \mid 0 \leq j \leq B(c)].
    \]
    Hence,
    \begin{equation}
      \label{eq:contradiction}
      \text{\Cref{alg} returns $\unknown$ in \Cref{alg:err2} in iteration $i_{B(c)+1}$}
    \end{equation}
    or earlier.
    %
    The reason is that we have $b = B(c)+1$ in iteration $i_{B(c)+1}$, so in this iteration \Cref{alg} checks satisfiability of the formula $\phi(B(c)) \land \mu_{B(c)+1}(\psi_\err)$ in \Cref{alg:err2}.
    %
    As we have $b(k) > B(c)$, we get $k = i_{b(k)} \geq i_{B(c)+1}$.
    %
    Hence, \eqref{eq:contradiction} contradicts the assumption that \Cref{alg} returns $\safe$ in \Cref{alg:safe} in iteration $k$. 

    We now prove \eqref{eq:goal}. 
    In the induction base, we have
    \begin{align*}
                             & [\vec{x}/\vec{v}_0] \models \psi_\init                                                     \\
      {} \curvearrowright {} & [\mu_{1}(\vec{x})/\vec{v}_0] \models \mu_{1}(\psi_\init)                                   \\
      {} \curvearrowright {} & [\mu_{1}(\vec{x})/\vec{v}_0] \models \phi(0) \tag{as $\phi(0) \equiv \mu_{1}(\psi_\init)$}
    \end{align*}
    Hence, the claim follows for
    \[
      B(0) = 0 = h(0,0).
    \]
    In the induction step, the induction hypothesis implies
    \begin{multline}
      \label{eq:IH}
      \forall 0 \leq i < c.\ \exists B(i) < b(k), h(0,i) < \ldots < h(B(i),i).\ h(0,i) = 0 \land h(B(i),i) = i\\
      {} \land \phi(B(i)) \text{ is consistent with } [\mu_{j+1}(\vec{x})/\vec{v}_{h(j,i)} \mid 0 \leq j \leq B(i)].
    \end{multline}
    Let:
    \begin{align*}
      I & {} \Def \min \{i \mid 0 \leq i < c, 1 \leq j \leq \ell(B(i)+1), \vec{v}_i \to_{\pi_j} \vec{v}_c\} \\
      J & {} \Def \max \{j \mid 1 \leq j \leq \ell(B(I)+1), \vec{v}_I \to_{\pi_j} \vec{v}_c\}
    \end{align*}
    So $I$ is the minimal index such that we can make a step from $\vec{v}_I$ to $\vec{v}_c$, and among the relational formulas that can be used for this step, $\pi_J$ is the one that was learned last.
    %
    Note that $I$ (and hence also $J$) exists, as we have $\vec{v}_{c-1} \to_{\tau} \vec{v}_c$ and $\tau = \pi_1$.
    %
    By \eqref{eq:IH},
    \[
      \phi(B(I)) \text{ is consistent with } [\mu_{j+1}(\vec{x})/\vec{v}_{h(j,I)} \mid 0 \leq j \leq B(I)].
    \]
    Let $\theta_I$ be an extension of $[\mu_{j+1}(\vec{x})/\vec{v}_{h(j,I)} \mid 0 \leq j \leq B(I)]$ such that $\theta_I \models \phi(B(I))$, and let $\theta$ be an extension of
    \begin{equation}
      \label{eq:theta}
      \theta_I \uplus [\mu_{B(I)+2}(\vec{x})/\vec{v}_c] \uplus [\mu_{B(I)+1}(x_{\id}) / J]
    \end{equation}
    such that $\theta \models \mu_{B(I)+1}(\pi_J)$, which exists as we have:
    \begin{align*}
      & \theta(\mu_{B(I)+1}(\vec{x})) \\
      {} = {} & \theta_I(\mu_{B(I)+1}(\vec{x})) \tag{by \eqref{eq:theta}} \\
      {} = {} & \vec{v}_{h(B(I),I)} \tag{def.\ of $\theta_I$} \\
      {} = {} & \vec{v}_I  \tag{as $h(B(I),I) = I$} \\
      {} \to_{\pi_J} {} & \vec{v}_c \tag{def.\ of $J$}\\
      {} = {} & \mu_{B(I)+1}(\vec{x}')[\mu_{B(I)+1}(\vec{x}')/\vec{v}_c] \\
      {} = {} & \mu_{B(I)+1}(\vec{x}')[\mu_{B(I)+2}(\vec{x})/\vec{v}_c] \tag{as $\mu_{B(I)+2}(\vec{x}) = \mu_{B(I)+1}(\vec{x}')$} \\
      {} = {} & \theta(\mu_{B(I)+1}(\vec{x}')) \tag{by \eqref{eq:theta}}
    \end{align*}
    %
    So we have
    \[
      \theta(\ind{\vec{x}}{B(I)+1}) = \theta_I(\ind{\vec{x}}{B(I)+1}) = \vec{v}_{h(B(I),I)} = \vec{v}_I
    \]
    and
    \[
      \theta(\ind{\vec{x}}{B(I)+2}) = \vec{v}_c.
    \]

    We now show $\theta \models \phi(B(I)+1)$.
    %
    Then we get $B(c) = B(I) + 1$, $h(j,c) = h(j,I)$ for all $j \leq B(I)$, and $h(B(c),c) = c$, which finishes the proof of \eqref{eq:goal}.

    Since $\theta$ is an extension of $\theta_I$, we have $\theta \models \phi(B(I))$, so we only need to show
    \[
      \theta \models (\ind[\id]{x}{B(I)+1} \doteq 1 \lor \ind[\id]{x}{B(I)+1} \not\doteq \ind[\id]{x}{B(I)}) \land \bigvee_{i=1}^{\ell(B(I)+1)}\mu_{B(I)+1}(\pi_i \land x_{\id} \doteq i) \land \bigwedge_{\mathclap{(B(I)+1,\pi) \in \blocked(k)}} \pi,
    \]
    i.e., we only need to show that $\theta$ is a model of the last conjunct of $\phi(B(I)+1)$.
    %
    For the disjunction
    \begin{equation}
      \label{eq:disjunction}
      \bigvee_{i=1}^{\ell(B(I)+1)}\mu_{B(I)+1}(\pi_i \land x_{\id} \doteq i),
    \end{equation}
    note that $J \leq \ell(B(I)+1)$.
    %
    Thus, to show $\theta \models \eqref{eq:disjunction}$, it suffices if
    \[
      \theta \models \mu_{B(I)+1}(\pi_J \land x_{\id} \doteq J),
    \]
    which holds by construction of $\theta$.
    %
    Thus, it remains to show
    \[
      \theta \models (\ind[\id]{x}{B(I)+1} \doteq 1 \lor \ind[\id]{x}{B(I)+1} \not\doteq \ind[\id]{x}{B(I)}) \land \bigwedge_{\mathclap{(B(I)+1,\pi) \in \blocked(k)}} \pi.
    \]

    We first consider the disjunction $\ind[\id]{x}{B(I)+1} \doteq 1 \lor \ind[\id]{x}{B(I)+1} \not\doteq \ind[\id]{x}{B(I)}$.
    %
    We show that we always have $J=1$ or $\theta(\ind[\id]{x}{B(I)}) \neq J$.
    %
    Then this disjunction is clearly satisfied by $\theta$ since $\theta(\ind[\id]{x}{B(I)+1}) = J$.
    %
    To see why $J=1$ or $\theta(\ind[\id]{x}{B(I)}) \neq J$ holds, assume that $\theta(\ind[\id]{x}{B(I)}) = J > 1$.
    %
    Then:
    \begin{align*}
                             & \theta \models \phi(B(I)) \tag {as $\theta$ is an extension of $\theta_I$} \\
      {} \curvearrowright {} & \theta \models \mu_{B(I)}(\pi_J \land x_{\id} \doteq J) \tag{as $\theta(\ind[\id]{x}{B(I)}) = J$ by assumption}                                                                                 \\
      {} \curvearrowright {} & \mu_{B(I)}(\pi_J) \text{ is consistent with } [\mu_{j+1}(\vec{x})/\vec{v}_{h(j,I)} \mid 0 \leq j \leq B(I)] \tag{as $[\mu_{j+1}(\vec{x})/\vec{v}_{h(j,I)} \mid 0 \leq j \leq B(I)] \subseteq \theta$} \\
      {} \curvearrowright {} & \mu_{B(I)}(\pi_J) \text{ is consistent with } [\mu_{B(I)}(\vec{x})/\vec{v}_{h(B(I)-1,I)},\mu_{B(I)+1}(\vec{x})/\vec{v}_{h(B(I),I)}] \tag{by instantiating $j$ with $B(I)-1$ and $B(I)$}           \\
      {} \curvearrowright {} & \pi_J \text{ is consistent with } [\vec{x}/\vec{v}_{h(B(I)-1,I)},\vec{x}'/\vec{v}_{h(B(I),I)}]                                                                                                      \\
      {} \curvearrowright {} & \pi_J \text{ is consistent with } [\vec{x}/\vec{v}_{h(B(I)-1,I)},\vec{x}'/\vec{v}_{I}] \tag{as $h(B(I),I) = I$}                                                                                     \\
      {} \curvearrowright {} & \vec{v}_{h(B(I)-1,I)} \to_{\pi_J} \vec{v}_{I}.
    \end{align*}
    By the definition of $I$ and $J$, we also have $\vec{v}_{I} \to_{\pi_J} \vec{v}_{c}$.
    %
    So by transitivity of $\to_{\pi_J}$ (which holds since $J > 1$), we get $\vec{v}_{h(B(I)-1,I)} \to_{\pi_J} \vec{v}_{c}$.
    %
    As we have $h(B(I)-1,I) < I$ by definition of $h$, this contradicts minimality of $I$.
    %
    Note that here is the only point where we need the transitivity of learned relations.

    Therefore, it remains to show
    \[
      \theta \models \bigwedge_{\mathclap{(B(I)+1,\pi) \in \blocked(k)}} \pi.
    \]
    The elements of $\blocked(k)$ have the form $(s + \ell - 1, \blockingclause(s,\ell,\pi_j,\overline{\sigma}))$ where $[\tau_i]_{i=s}^{s+\ell-1}$ is a loop on the trace, $\pi_j \in \tail(\vec{\pi})$ is a learned relation, and $\overline{\sigma} \models \pi_j$.
    %
    Moreover, we have $s + \ell - 1 = B(I) + 1$.
    %
    We now perform a case analysis for the two cases where $\ell = 1$ and $\ell > 1$.

    In Case 1 (where $\ell = 1$), the blocking clause has the form
    $\blockingclause(s,1,\linebreak \pi_j,\overline{\sigma})$ for $s = B(I)+1$.
    %
    Thus, we show that $\theta$ cannot violate a blocking clause of the form
    \[
      \blockingclause(B(I),1,\pi_j,\overline{\sigma}) = \mu_{B(I)+1}(\neg\mbip(\pi_j,\overline{\sigma})) \lor \ind[\id]{x}{B(I)+1} > 1.
    \]
    To see this, we first consider the case where the negation of the first disjunct holds and prove that then the second disjunct is true.
    %
    The reason is that we have:
    \begin{align*}
                             & \theta \models \mu_{B(I)+1}(\mbip(\pi_j, \overline{\sigma}))                                                                                     \\
      {} \curvearrowright {} & \theta \circ \mu_{B(I)+1} \text{ is consistent with } \pi_j \tag{since $\mbip(\pi_j,\overline{\sigma}) \models \pi_j$ by \Cref{def:projections}} \\
      {} \curvearrowright {} & \theta(\mu_{B(I)+1}(\vec{x})) \to_{\pi_j} \theta(\mu_{B(I)+1}(\vec{x}'))                                                                \\
      {} \curvearrowright {} & \theta(\mu_{B(I)+1}(\vec{x})) \to_{\pi_j} \theta(\mu_{B(I)+2}(\vec{x}))                                                                \\
      {} \curvearrowright {} & \vec{v}_{h(B(I),I)} \to_{\pi_j} \vec{v}_{c} \tag{def.\ of $\theta$}                                                                 \\
      {} \curvearrowright {} & \vec{v}_{I} \to_{\pi_j} \vec{v}_{c} \tag{as $h(B(I),I) = I$}
    \end{align*}
    So the step fom $\vec{v}_{I}$ to $\vec{v}_{c}$ can be done by a learned relation $\pi_j$.
    %
    As $\pi_j$ is a learned relation, we have $j > 1$.
    %
    This implies $\theta(\ind[\id]{x}{B(I)+1}) = J > 1$, as $J$ is maximal and hence $J \geq j > 1$.

    Now we consider the case where the negation of the first disjunct does not hold and prove that then the first disjunct is true.
    %
    The reason is that due to the completeness of $\AA$, $\theta \centernot\models \mu_{B(I)+1}(\mbip(\pi_j, \overline{\sigma}))$ implies $\theta \models \mu_{B(I)+1}(\neg\mbip(\pi_j, \overline{\sigma}))$.
    %
    So in this case the blocking clause is satisfied as well.

    In Case 2 (where $\ell > 1$), we show that $\theta$ cannot violate a blocking clause of the form
    \[
      \blockingclause(s,\ell,\pi_j,\overline{\sigma}) = \mu_{s,\ell}(\neg\mbip(\pi_j, \overline{\sigma}))
    \]
    where $s + \ell - 1 = B(I)+1$, i.e., $s + \ell = B(I) + 2$.
    %
    The reason is that we have
    \begin{align*}
                             & \theta \centernot\models \mu_{s,\ell}(\neg\mbip(\pi_j, \overline{\sigma}))                                                                       \\
      {} \curvearrowright {} & \theta \models \mu_{s,\ell}(\mbip(\pi_j, \overline{\sigma})) \tag{as $\AA$ is complete} \\
      {} \curvearrowright {} & \theta \circ \mu_{s,\ell} \text{ is consistent with } \pi_j \tag{since $\mbip(\pi_j,\overline{\sigma}) \models \pi_j$ by \Cref{def:projections}} \\
      {} \curvearrowright {} & \theta(\mu_{s,\ell}(\vec{x})) \to_{\pi_j} \theta(\mu_{s,\ell}(\vec{x}'))                                                              \\
      {} \curvearrowright {} & \vec{v}_{h(s-1,I)} \to_{\pi_j} \vec{v}_{c} \tag{def.\ of $\theta$, as $s+\ell = B(I)+2$}
    \end{align*}
    For the last step, note that
    \begin{align*}
      \theta(\mu_{s,\ell}(\vec{x})) & {} = \theta(\mu_{s}(\vec{x})) = \theta_I(\mu_{s}(\vec{x})) = \vec{v}_{h(s-1,I)} & \text{and} \\
      \theta(\mu_{s,\ell}(\vec{x}')) & {} = \theta(\ind{\vec{x}}{s + \ell}) = \theta(\ind{\vec{x}}{B(I)+2}) = \vec{v}_{c}.
    \end{align*}
    We have $s - 1 < B(I)$ and thus $h(s-1,I) < h(B(I),I) = I$, which contradicts minimality of $I$.
    %
    This finishes the proof of \eqref{eq:goal}.
    \qed
  \end{proof}
}

\begin{remark}[Termination]
  \label{remark:termination}
In general, \Cref{alg} does not terminate, since
 $\tip$ decomposes the relation into finitely many cases and
approximates their transitive closures independently, but
${\to^+_{\tau_\Loop}} \subseteq \bigcup_{\sigma \models \tau_{\Loop}}
{\to_{\tip(\tau_{\Loop},\sigma)}}$ is not guaranteed (\Cref{remark:properties-tip}).
%
To see why this may prevent termination, consider a loop $\tau_{\Loop}$ and assume that there are reachable states
$\vec{v},\vec{v}'$ with $\vec{v} \to_{\tau_\Loop}^+ \vec{v}'$, but $\vec{v}
\not\to_{\tip(\tau_{\Loop},\sigma)} \vec{v}'$ for all models $\sigma$ of
$\tau_\Loop$.
%
Then TRL may find a model that corresponds to a run from $\vec{v}$ to $\vec{v}'$.
%
Unless $\vec{v}$ can be evaluated to $\vec{v}'$ with another learned transition $\pi \notin \{\tip(\tau_{\Loop},\sigma) \mid \sigma \models \tau_{\Loop}\}$ by coincidence, this loop cannot be blocked and TRL learns a new relation.
%
Thus, TRL may keep learning new relations as long as there are loops whose transitive closure is not yet covered by learned relations.

As the elements of $\{\tip(\tau,\sigma) \mid \sigma \models \tau\}$ are independent of each other, a more ``global'' view may help to enforce convergence.
%
We leave that to future work.
\end{remark}

\paper{
  \begin{example}[\Cref{ex:ex1} Finished]
    After learning $\tau^+_\dec$ and $\tau^+_\inc$, the underlying SMT problem becomes unsatisfiable when $b=3$ after adding appropriate blocking clauses, so that $\tau^+_\dec$ and $\tau^+_\inc$ are preferred over $\tau_\dec$ and $\tau_\inc$.
    %
    The reason is that $\tau^+_\dec$ and $\tau^+_\inc$ must not be used twice in a row due to \Cref{alg:trans} of \Cref{alg}, and $\tau^+_\inc$ cannot be used after $\tau^+_\dec$, as it requires $w \doteq 0$, but $\tau^+_\dec$ sets $w$ to $1$.
    %
    Thus, \Cref{alg} returns $\safe$.
    %
    See \cite{arxiv} for a detailed run of
\Cref{alg} 
on \Cref{ex:ex1}.
  \end{example}
}

\report{
\subsection{A Complete Example}
\label{sec:example}

For a complete run of TRL on our example, assume that we obtain the following traces (where the detected loops are underlined):
\begin{enumerate}
  \item \label{it:a}
        $[\underline{\tau_\inc}]$, resulting in the learned relation $\tau^+_\inc$ and the blocking clause $\mu_{1}(\neg\tau^+_\inc \lor x_\id > 1)$ which ensures that if $b = 1$, then we cannot use $\tau_\inc$ but would have to use $\tau^+_\inc$.
  \item \label{it:b}
        $[\underline{\tau_\dec}]$, resulting in the learned relation $\tau^+_\dec$ and the blocking clause $\mu_{1}(\neg\tau^+_\dec \lor x_\id > 1)$ which ensures that if $b = 1$, then we cannot use $\tau_\dec$ but would have to use $\tau^+_\dec$.
  \item \label{it:c}
        $[\tau^+_\inc,\underline{\tau_\inc}]$, resulting in the blocking clause $\mu_{2}(\neg\tau^+_\inc \lor x_\id > 1)$ which ensures that if $b = 2$, then we cannot use $\tau_\inc$.
        %
        Using $\tau^+_\inc$ twice after each other is also not possible due to transitivity (\Cref{alg:trans}).
  \item \label{it:d}
        $[\tau^+_\dec,\underline{\tau_\dec}]$, resulting in the blocking clause $\mu_{2}(\neg\tau^+_\dec \lor x_\id > 1)$ which ensures that if $b = 2$, then we cannot use $\tau_\dec$.
        %
        Using $\tau^+_\dec$ twice after each other is also not possible due to transitivity (\Cref{alg:trans}).
  \item \label{it:e}
        $[\tau^+_\inc, \tau^+_\dec, \underline{\tau_\dec}]$, resulting in the blocking clause $\mu_{3}(\neg\tau^+_\dec \lor x_\id > 1)$ which ensures that if $b = 3$, then we cannot use $\tau_\dec$.
\end{enumerate}
Now we are in the following situation:
\begin{itemize}
  \item The first element of the trace cannot be $\tau_\inc$ or $\tau_\dec$ due to \eqref{it:a} and \eqref{it:b}.
  \item If the first element of the trace is $\tau^+_\inc$, then:
        \begin{itemize}
          \item The second element of the trace cannot be $\tau_\inc$ or $\tau_\dec$ due to \eqref{it:c} and \eqref{it:d}.
          \item The second element of the trace cannot be $\tau^+_\inc$ due to \Cref{alg:trans}.
          \item If the second element of the trace is $\tau^+_\dec$, then:
                \begin{itemize}
                  \item The third element of the trace cannot be $\tau_\inc$ or $\tau^+_\inc$, as $\tau^+_\dec$ sets $w$ to $1$, but $\tau_\inc$ and $\tau^+_\inc$ require $w = 0$.
                  \item The third element of the trace cannot be $\tau_\dec$ due to \eqref{it:e}.
                  \item The third element of the trace cannot be $\tau^+_\dec$ due to \Cref{alg:trans}.
                \end{itemize}
        \end{itemize}
        So this case becomes infeasible with the $3^{rd}$ unrolling of the transition relation.
  \item If the first element of the trace is $\tau^+_\dec$, then:
        \begin{itemize}
          \item The second element of the trace cannot be $\tau_\inc$ or $\tau^+_\inc$, as $\tau^+_\dec$ sets $w$ to $1$, but $\tau_\inc$ and $\tau^+_\inc$ require $w = 0$.
          \item The second element of the trace cannot be $\tau_\dec$ due to \eqref{it:d}.
          \item The second element of the trace cannot be $\tau^+_\dec$ due to \Cref{alg:trans}.
        \end{itemize}
        So this case becomes infeasible with the $2^{nd}$ unrolling of the transition relation.
\end{itemize}
Thus, the underlying SMT problem becomes unsatisfiable when $b=3$, such that $\safe$ is returned in \Cref{alg:safe}.
}

\report{
  \section{Implementing $\tip$ for Linear Integer Arithmetic}
\label{sec:rec}

We now explain how to compute transitive projections for quantifier-free linear integer arithmetic via recurrence analysis.
%
As in SMT-LIB \cite{smtlib}, in our setting linear integer arithmetic also features (in)divisibility predicates of the form $e|t$ (or $e\!{\not|}t$) where $e \in \NN_+$ and $t$ is an integer-valued term.
%
Then we have $\sigma \models e|t$ iff $\sigma(t)$ is a multiple of $e$, and $\sigma \models e\!{\not|}t$, otherwise.

The technique that we use is inspired by the recurrence analysis from \cite{kincaid15}.
%
However, there are some important differences.
%
The approach from \cite{kincaid15} computes convex hulls to over-approximate disjunctions by conjunctions, and it relies on polyhedral projections.
%
In our setting, we always have a suitable model at hand, so that we can use $\mbip$ instead.
%
Hence, our recurrence analysis can be implemented more efficiently\footnote{The \emph{double description method}, which is popular for computing polyhedral projections and convex hulls, and other state-of-the-art approaches have exponential complexity \cite{dd-exp,fmplex}.
  %
  See \cite{convex-hull} for an easily accessible discussion of the complexity of the double description method.
  %
  In contrast, combining the model based projection from \cite{spacer} with syntactic implicant projection \cite{adcl} yields a polynomial time algorithm for $\mbip$.}.
%
Additionally, our recurrence analysis can handle divisibility predicates, which are not covered in \cite{kincaid15}.

On the other hand, \cite{kincaid15} yields an over-approximation of the transitive closure of the given relation, whereas our approach performs an implicit case analysis (via $\mbip$) and only yields an over-approximation of the transitive closure of one out of finitely many cases.

Moreover, the recurrence analysis from \cite{kincaid15} also discovers non-linear relations, and then uses linearization techniques to eliminate them.
%
For simplicity, our recurrence analysis only derives linear relations so far.
%
However, just like \cite{kincaid15}, we could also derive non-linear relations and linearize them afterwards.
%
Apart from these differences, our technique is analogous to \cite{kincaid15}.

In the sequel, let $\tau$ and $\sigma \models \tau$ be fixed.
%
Our implementation of $\tip(\tau,\sigma)$ first searches for \emph{recurrent literals}, i.e., literals of the form\footnote{W.l.o.g., we assume that literals are never negated, as we can negate the corresponding (in)equalities or divisibility predicates directly instead.
  %
  Furthermore, in our implementation, we replace disequalities $s \not\doteq t$ with $s >
  t \lor s < t$ and eliminate the resulting disjunction via $\mbip$ to obtain a transition
  without disequalities.}
\[
  t \bowtie 0 \text{ or } e|t \quad \text{where} \quad t = \sum_{x \in \vec{x}} c_x
  \cdot (x'-x) + c, \; {\bowtie} \in \{\leq,\geq,<,>,\doteq\},
  \text{ and } c_x,c \in \ZZ.
\]
Hence, these literals provide information about the change of values of variables.
%
To find such literals, we introduce a fresh variable $x_\delta$ for each $x \in \vec{x}$, and we conjoin $x_\delta \doteq x' - x$ to $\tau$, i.e., we compute
\[
  \tau_{\land \delta} \Def \tau \land \bigwedge_{x \in \vec{x}} x_\delta \doteq x' - x.
\]
%
So the value of $x_\delta$ corresponds to the change of $x$ when applying $\tau$.
%
Next, we use $\mbip$ to eliminate all variables but $\{x_\delta \mid x \in \vec{x}\}$ from $\tau_{\land\delta}$, resulting in $\tau_\delta$.
%
More precisely, we have
\[
  \tau_\delta \Def \mbp(\tau_{\land \delta}, \quad \sigma \uplus [x_\delta/\sigma(x'-x) \mid x \in \vec{x}], \quad \{x_\delta \mid x \in \vec{x}\}).
\]
Finally, to obtain a formula where all literals are recurrent, we replace each $x_\delta$ by its definition, i.e., we compute
\[
  \tau_\rec \Def \tau_\delta[x_\delta / x' - x \mid x \in \vec{x}].
\]
%
\begin{example}[Finding Recurrent Literals]
  \label{ex:finding-rec}
  Consider the transition $\tau_\dec$.
  %
  We first construct the formula
  \[
    \tau_{\land \delta} \Def \tau_\dec \land w_\delta \doteq w' - w \land x_\delta \doteq x' - x \land y_\delta \doteq y' - y.
  \]
  Then for any model $\sigma \models \tau_\dec$, we get\footnote{In the case of $\tau_\dec$, we obtain the same formula $\tau_\delta$ for \emph{every} model $\sigma \models \tau_\dec$, as variables can simply be eliminated by propagating equalities.}
  \begin{align*}
    \tau_\delta \Def {} & \mbp(\tau_{\land \delta}, \quad \sigma \uplus [w_\delta/0, \; x_\delta/{-}1, \; y_\delta/{-}1], \quad \{w_\delta, x_\delta, y_\delta\}) \\
                {} = {} & w_\delta \doteq 0 \land x_\delta \doteq -1 \land y_\delta \doteq -1.
  \end{align*}
  Next, replacing $w_\delta,x_\delta$, and $y_\delta$ with their definition results in
  \begin{align*}
    \tau_\rec \Def {} & w' - w \doteq 0 \land x' - x \doteq -1 \land y' - y \doteq -1        \\
    \equiv {}         & w' - w \doteq 0 \land x' - x + 1 \doteq 0 \land y' - y + 1 \doteq 0.
  \end{align*}
\end{example}
%
Then the construction of $\tip(\tau,\sigma)$ proceeds as follows:
%
\begin{itemize}
  \item $\tip(\tau,\sigma)$ contains the literal $n > 0$, where $n \in \VV$ is a fresh extra variable
  \item for each literal $\sum_{x \in \vec{x}} c_x \cdot (x'-x) + c \bowtie 0$ of $\tau_\rec$, $\tip(\tau,\sigma)$ contains the literal $\sum_{x \in \vec{x}} c_x \cdot (x'-x) + n \cdot c \bowtie 0$
  \item for each literal $e|\sum_{x \in \vec{x}} c_x \cdot (x'-x) + c$ of $\tau_\rec$, $\tip(\tau,\sigma)$ contains the literal $e|\sum_{x \in \vec{x}} c_x \cdot (x'-x) + n \cdot c$
\end{itemize}
%
Intuitively, the extra variable $n$ can be thought of as a ``loop counter'', i.e., when
$n$ is instantiated with some constant $k$, then the literals above approximate the change
of variables when $\to_\tau$ is applied $k$ times.

\begin{example}[Computing $\tip$ (1)]
  \label{ex:tip1}
  Continuing \Cref{ex:finding-rec}, $\tip(\tau_\dec,\sigma)$ contains the literals
  \begin{align*}
    {}           & n > 0 \land w' - w + n \cdot 0 \doteq 0 \land x' - x + n \cdot 1 \doteq 0 \land y' - y + n \cdot 1 \doteq 0 \\
    {} \equiv {} & n > 0 \land w' \doteq w \land x' \doteq x - n \land y' \doteq y - n
  \end{align*}
  for any model $\sigma \models \tau_\dec$.
  %
  Note that in our example, this formula precisely characterizes the change of the variables after $n$ iterations of $\to_{\tau_\dec}$.
  %
  To simplify the formula above, we can propagate the equality $n = x - x'$, resulting in:
  \begin{align}
    & x - x' > 0 \land w' \doteq w \land y' \doteq y - x + x' \notag \\
    {} \equiv {} & w' \doteq w \land x' < x \land x' - x \doteq y' - y \label{eq:rel}
  \end{align}
\end{example}
%
Compared to $\tau_\dec^+$, \eqref{eq:rel} lacks the literal $w \doteq 1$.
%
To incorporate information about the pre- and post-variables (but not about their relation) we conjoin $\mbp(\tau,\sigma,\vec{x})$ and $\mbp(\tau,\sigma,\vec{x}')$ to $\tip(\tau,\sigma)$.

\begin{example}[Computing $\tip$ (2)]
  We finish \Cref{ex:tip1} by conjoining
  \[
    \mbp(\tau_\dec,\sigma,\{w,x,y\}) = w \doteq 1 \qquad \text{and} \qquad \mbp(\tau_\dec,\sigma,\{w',x',y'\}) = w' \doteq 1
  \]
  to \eqref{eq:rel}, resulting in:
  \[
    w' \doteq w \land x' < x \land x' - x \doteq y' - y \land w \doteq 1 \land w' \doteq 1 \quad \equiv \quad \tau_\dec^+
  \]
\end{example}

\begin{example}[Divisibility]
  To see how $\tip$ can handle divisibility constraints, consider the transition
  \[
    \tau \Def 2|x \land 3|x' - x + 1.
  \]
  Then our approach identifies the recurrent literal $\tau_\rec = 3|x' - x + 1$, so that $\tip(\tau,\sigma)$ contains the literal $3|x' - x + n$.
  %
  To see why we conjoin this literal to $\tip(\tau,\sigma)$, note that $3|x'-x+1$ and
  $3|x'-x+n$ are equivalent to $x'-x + 1 \equiv_3 0$ and $x'-x + n \equiv_3 0$, respectively, where ``$\equiv_3$'' denotes congruence modulo $3$.
  %
  So $e \to^n_{\tau} e'$ implies $e'-e+n \equiv_3 0$, just like $e \to^n_{\pi} e'$ implies
  $e'-e+n=0$ for $\pi \Def x'-x+1 \doteq 0$.
  Moreover, we have $\mbp(\tau,\sigma,\vec{x}) = 2|x$, and thus
  \[
    \tip(\tau,\sigma) = n>0 \land 3|x'-x+n \land 2|x.
  \]
\end{example}

\newcounter{tip}
\setcounter{tip}{\value{theorem}}
\begin{theorem}[Correctness of $\tip$]
  \label{thm:tip}
  The function $\tip$ as defined above is a transitive projection.
\end{theorem}
\makeproof*{thm:tip}{
  \setcounter{theorem}{\thetip}
  \begin{theorem}[Correctness of $\tip$]
    \label{thm:Correctness_tip}
    The function $\tip$ as defined in \Cref{sec:rec} is a transitive projection.
  \end{theorem}
  \begin{proof}
    We have to prove that
    \begin{itemize}
      \item[(a)] $\tip(\tau,\sigma)$ is consistent with $\sigma$,
      \item[(b)] $\{\tip(\tau,\theta) \mid \theta \models \tau\}$ is finite, and
      \item[(c)] $\to_{\tip(\tau,\sigma)}$ is transitive
    \end{itemize}
    for all transitions $\tau$ and all $\sigma \models \tau$.

    For item (a), we first prove $\sigma \models \tau_\rec$.
    %
    We have:
    \begin{align*}
                             & \sigma \models \tau                                                                                                          \\
      {} \curvearrowright {} & \sigma \uplus [x_\delta / \sigma(x'-x) \mid x \in \vec{x}] \models \tau_{\land\delta} \tag{by def.\ of $\tau_{\land\delta}$} \\
      {} \curvearrowright {} & [x_\delta / \sigma(x'-x) \mid x \in \vec{x}] \models \tau_\delta \tag{by def.\ of $\mbp$ and $\tau_{\delta}$}                \\
      {} \curvearrowright {} & [x/\sigma(x),x'/\sigma(x') \mid x \in \vec{x}] \models \tau_\rec \tag{by def.\ of $\tau_{\rec}$}                             \\
      {} \curvearrowright {} & \sigma \models \tau_\rec
    \end{align*}
    %
    Now we prove
    $\sigma' \Def \sigma \uplus [n/1] \models \tip(\tau,\sigma)$.
    %
    To this end, we consider the literals that are added to $\tip(\tau,\sigma)$ by the procedure described in \Cref{sec:rec} independently:
    %
    \begin{itemize}
      \item $\sigma' \models n > 0$ is trivial
      \item if $\sigma \models (\sum_{x \in \vec{x}} c_x \cdot (x'-x) + c) \bowtie 0$, then $\sigma' \models (\sum_{x \in \vec{x}} c_x \cdot (x'-x) + n \cdot c) \bowtie 0$
      \item if $\sigma \models e|(\sum_{x \in \vec{x}} c_x \cdot (x'-x) + c)$, then $\sigma' \models e|(\sum_{x \in \vec{x}} c_x \cdot (x'-x) + n \cdot c)$
    \end{itemize}
    Moreover, we have $\sigma \models \mbp(\tau,\sigma,\vec{x})$ and $\sigma \models \mbp(\tau,\sigma,\vec{x}')$ by definition of $\mbp$, and hence we also have $\sigma' \models \mbp(\tau,\sigma,\vec{x})$ and $\sigma' \models \mbp(\tau,\sigma,\vec{x}')$.
    %
    Therefore, $\sigma'$ is a model of $\tip(\tau,\sigma)$, so $\sigma$ is consistent with $\tip(\tau,\sigma)$.

    For item (b), note that $\tip$ only uses the provided model for computing conjunctive variable projections.
    %
    As $\mbp$ has a finite image, the claim follows.

    For item (c), note that the conjuncts $\mbp(\tau,\sigma,\vec{x})$ and $\mbp(\tau,\sigma,\vec{x}')$ of $\tip(\tau,\sigma)$ are irrelevant for transitivity, as they do not relate $\vec{x}$ and $\vec{x}'$.
    %
    Let $\tau' \Def \tip(\tau,\sigma)$.
    %
    It suffices to prove
    \[
      {\to^2_{\tau'}} \subseteq {\to_{\tau'}}.
    \]
    Then the claim follows from a straightforward induction.
    %
    Assume $\theta \models \tau'$, $\theta' \models \tau'$, and
    \[
      \theta(\vec{x}) \to_{\tau'} \theta(\vec{x}') = \theta'(\vec{x}) \to \theta'(\vec{x}').
    \]
    We prove $\hat{\theta} \models \tau'$ where
    \[
      \hat{\theta} \Def [\vec{x} / \theta(\vec{x})] \uplus [\vec{x}' / \theta'(\vec{x}')] \uplus [n / \theta(n) + \theta'(n)].
    \]
    Then the claim follows.
    %
    Again, we consider all literals independently:
    \begin{itemize}
      \item If $\theta \models n > 0$ and $\theta' \models n > 0$, then $\hat{\theta} \models n > 0$.
      \item Consider a literal of the form $\iota \Def t \bowtie 0$ where $t = \sum_{x \in \vec{x}} c_x \cdot (x'-x) + n \cdot c$.
            %
            We have
            %
            \begin{align*}
              \theta(t) = {} & \sum_{x \in \vec{x}} c_x \cdot (\theta(x')-\theta(x)) + \theta(n) \cdot c                                                                               \\
              {} = {}        & \sum_{x \in \vec{x}} c_x \cdot \theta(x') - \sum_{x \in \vec{x}} c_x \cdot \theta(x) + \theta(n) \cdot c                                                \\
              {} = {}        & \sum_{x \in \vec{x}} c_x \cdot \theta'(x) - \sum_{x \in \vec{x}} c_x \cdot \theta(x) + \theta(n) \cdot c \tag{as $\theta(\vec{x}') = \theta'(\vec{x})$}
            \end{align*}
            and
            \begin{align*}
              \theta'(t) = {} & \sum_{x \in \vec{x}} c_x \cdot (\theta'(x')-\theta'(x)) + \theta'(n) \cdot c                                 \\
              {} = {}         & \sum_{x \in \vec{x}} c_x \cdot \theta'(x') - \sum_{x \in \vec{x}} c_x \cdot \theta'(x) + \theta'(n) \cdot c.
            \end{align*}
            Thus, we have:
            \begin{align*}
              \theta(t) + \theta'(t) = {} & -\sum_{x \in \vec{x}} c_x \cdot \theta(x) + \theta(n) \cdot c + \sum_{x \in \vec{x}} c_x \cdot \theta'(x')+ \theta'(n) \cdot c \\
              {} = {}                     & \sum_{x \in \vec{x}} c_x \cdot \theta'(x') - \sum_{x \in \vec{x}} c_x \cdot \theta(x) + (\theta(n) + \theta'(n)) \cdot c       \\
              {} = {}                     & \sum_{x \in \vec{x}} c_x \cdot \hat{\theta}(x') - \sum_{x \in \vec{x}} c_x \cdot \hat{\theta}(x) + \hat{\theta}(n) \cdot c     \\
              {} = {}                     & \hat{\theta}(t)
            \end{align*}
            Therefore, $\theta \models \iota$ and $\theta' \models \iota$ implies $\hat{\theta} \models \iota$ for all ${\bowtie} \in \{\leq,\geq,<,>,=\}$.
      \item Consider a literal of the form $\iota \Def e|t$ where $t = \sum_{x \in \vec{x}} c_x \cdot (x'-x) + n \cdot c$.
            %
            Then we again obtain
            \[
              \theta(t) + \theta'(t) = \hat{\theta}(t)
            \]
            as above.
            %
            Thus, $\theta \models \iota$ and $\theta' \models \iota$ imply $\hat{\theta} \models \iota$.
            %
            \qed
    \end{itemize}
  \end{proof}
}



  \section{Proving Unsafety}\label{sec:Unsafety}

We now explain how to adapt \Cref{alg} for also proving unsafety.
%
Assume that the satisfiability check in \Cref{alg:err2} is successful.
%
Then an error state is reachable from an initial state via the current trace.
%
However, the trace may contain learned transitions, so this does not imply unsafety of $\TT$.
%
The idea for proving unsafety is to replace learned transitions with \emph{accelerated transitions} that result from applying \emph{acceleration techniques}.
%
\begin{definition}[Acceleration]
  A function $\accel: \QF(\Sigma) \to \QF(\Sigma)$ is called an \emph{acceleration technique} if ${\to_{\accel(\pi)}} \subseteq {\to^+_{\pi}}$ for all relational formulas $\pi$.
\end{definition}
%
So in contrast to TRL's learned transitions, accelerated transitions under-ap\-prox\-i\-mate transitive closures, and hence they are suitable for proving unsafety.
%
For arithmetical theories, acceleration techniques are well studied \cite{kroening13,bozga10,acceleration-calculus} (our implementation uses the technique from \cite{acceleration-calculus}).

For any vector of transition formulas $\vec{\rho} = [\rho_1,\ldots,\rho_k]$, let $\compose(\vec{\rho})$ be a transition formula such that ${\to_{\compose(\vec{\rho})}}$ is the composition of the relations ${\to_{\rho_1}}, \ldots, {\to_{\rho_k}}$.
%
Moreover, for a learned transition $\pi$, we say that $\vec{\tau}_\Loop$ \emph{induced} $\pi$ if we had $\vec{\tau}_\Loop = [\tau_s,\ldots,\tau_{s+\ell-1}]$ in \Cref{alg:loop} when $\pi$ was learned.
%
Let $\succ$ be the total ordering on the elements of the vector $\vec{\pi}$ from \Cref{alg} with $\pi_i \succ \pi_j$ iff $i > j$, i.e., the input formula $\tau$ is minimal w.r.t.\ $\succ$, and a learned relation is smaller than all relations that were learned later.
%
Then for vectors of transitions from $\{\mbip(\pi,\sigma) \mid \pi \in \vec{\pi}, \sigma \models \pi\}$, we define the function $\underapprox$ (which yields an \underline{u}nder-\underline{a}pproximation) as follows:
\begin{align*}
  &\underapprox([\eta_1,\ldots,\eta_k]) \Def \compose([\underapprox(\eta_1),\ldots,\underapprox(\eta_k)])\\
  &\underapprox(\eta) \Def
  \begin{cases}
    \eta & \text{if } \eta \models \tau \\
    \accel(\underapprox(\vec{\tau}_\Loop)) & \text{if } \eta \not\models \tau \text{ and}\\
& \phantom{\text{if }} \text{the loop } \vec{\tau}_\Loop \text{ induced }  \underset{\succ}{\min}\{\pi \in \vec{\pi} \mid \eta \models \pi\}
  \end{cases}
\end{align*}
%
Note that $\underapprox$ is well defined,
 as the following holds for all $\eta' \in \vec{\tau}_\Loop$ in the case $\eta \not\models
 \tau$:
\[
  \min_\succ\{\pi \in \vec{\pi} \mid \eta \models \pi\} \succ \min_\succ\{\pi \in \vec{\pi} \mid \eta' \models \pi\}
\]
%
The reason is that
 $\min_\succ\{\pi \in \vec{\pi} \mid \eta \models \pi\}$
is induced by $\vec{\tau}_\Loop$, and thus the elements of $\vec{\tau}_\Loop$ are
conjunctive variable projections of $\tau$, or of relations that were learned before
$\min_\succ\{\pi \in \vec{\pi} \mid \eta \models \pi\}$.
Hence, when defining $\underapprox(\eta)$, the ``recursive call''
$\underapprox(\vec{\tau}_\Loop)$ only refers to formulas with smaller index in $\vec{\pi}$, i.e., the
recursion ``terminates''.



So $\underapprox$ leaves original transitions unchanged.
%
For learned transitions
$\eta$, it first
computes an under-approximation
of the loop $\vec{\tau}_\Loop$ that induced $\eta$, and then it applies acceleration,
resulting in an under-approximation of the transitive closure.
%
The reason for applying $\underapprox$ before acceleration is that $\vec{\tau}_\Loop$ may again contain learned transitions, which have to be under-approximated first.

To improve \Cref{alg},
instead of returning $\unknown$ in \Cref{alg:err2}, now we first obtain a model $\sigma$ from the SMT solver.
%
Then we compute the current trace $\vec{\tau} = \trace_{b-1}(\sigma,\vec{\pi}) = [\tau_1,\ldots,\tau_{b-1}]$ (note that the trace has length $b-1$, as $b$ was incremented in \Cref{alg:err1}).
%
Next, computing $\pi_\underapprox \Def \underapprox(\vec{\tau})$ yields an under-approximation of the states that are reachable with the current trace, but more importantly, $\pi_\underapprox$ also under-approximates $\to^+_\tau$.
%
The reason is that $\pi_\underapprox$ is constructed from original transitions
by applying $\compose$ (which is exact) and $\accel$ (which yields under-approximations).
%
Hence, we return $\unsafe$ if there is an initial state $\vec{v}$ and an error state $\vec{v}'$ with $\vec{v} \to_{\pi_\underapprox} \vec{v}'$, i.e., if $\psi_\init \land \pi_\underapprox \land \psi_\err[\vec{x}/\vec{x}']$ is satisfiable.

\begin{example}[Proving Unsafety]
  Consider the relation
  \[
    \underbrace{y > 0 \land x' \doteq x + 1 \land y' \doteq y - 1 \land z' \doteq z}_{\tau_>} {} \lor {} \underbrace{y \doteq 0 \land x' \doteq x \land y' \doteq z \land z' \doteq z}_{\tau_=} \tag{$\tau$}
  \]
  with the initial states given by $\psi_\init \Def x \leq 0$ and the error states given
  by
  $\psi_\err \Def x \geq 1000$.
  %
  Assume that TRL obtains the trace $[\tau_>]$ and learns the relation
  \begin{equation}
    \label{eq:unsafe-learned1}
    n > 0 \land y > 0 \land x' \doteq x + n \land y' \doteq y - n \land z' \doteq z \land y' \geq 0. \tag{$\tau_>^+$}
  \end{equation}
  Next, assume that TRL obtains the trace $[\tau_=,\tau_>^+]$ and learns the relation
  \begin{equation}
    \label{eq:unsafe-learned1}
    y \doteq 0 \land z > 0 \land x' > x \land z > y' \land y' \geq 0 \land z' \doteq z. \tag{$\hat{\tau}^+$}
  \end{equation}
  Note that the transitive closure of the loop $[\tau_=,\tau_>^+]$ cannot be expressed precisely with linear arithmetic, and hence we only obtain the inequations above for $x'$ and $y'$.
  %
  Then an error state is reachable with the trace $[\hat{\tau}^+]$ and we\report{
    \pagebreak[3]} get:
  \begin{align*}
    & \underapprox([\hat{\tau}^+]) \\
    {} = {} & \accel(\underapprox([\tau_=,\tau_>^+])) \tag{as $[\tau_=,\tau_>^+]$ induced $\hat{\tau}^+$} \\
    {} = {} & \accel(\compose(\underapprox(\tau_=),\underapprox(\tau_>^+))) \\
    {} = {} & \accel(\compose(\tau_=,\accel(\underapprox([\tau_>])))) \tag{as $\tau_= \models \tau$ and $[\tau_>]$ induced $\tau_>^+$} \\
    {} = {} & \accel(\compose(\tau_=,\accel(\tau_>))) \tag{as $\tau_> \models \tau$}\\
    {} = {} & \accel(\compose(\tau_=,\tau^+_>)) \tag{as ${\to_{\tau_>}^+} = {\to_{\tau^+_>}}$} \\
    {} = {} & \accel(y \doteq 0 \land n > 0 \land z > 0 \land x' \doteq x + n \land y' \doteq z - n \land z' \doteq z \land y' \geq 0) \\
    {} = {} & y \doteq 0 \land m > 0 \land z > 0 \land x' \doteq x + m \cdot z \land y' \doteq 0 \land z' \doteq z \tag{$\check{\tau}^+$}
  \end{align*}
  For the last step, note that $\check{\tau}^+$ precisely characterizes the transitive closure of $\compose(\tau_=,\tau^+_>)$ for the case $n \doteq z$, and hence it is a valid under-approximation.
  %
  Then a model like $[x/y/0, z/1, m/1000, \; 
 x'/1000,  y'/0,z'/1]$ satisfies $\psi_\init \land \check{\tau}^+ \land \psi_\err[\vec{x}/\vec{x}'] = x \leq 0 \land \check{\tau}^+ \land x' \geq 1000$, which proves unsafety.
\end{example}

  
\section{Related Work} \label{sec:related}

% \textbf{Adversarial Attack}
\textbf{Attacks on SLAM.} 
%With the rise of machine learning, 
The robustness of computer vision systems is being actively investigated. With the emergence of adversarial images in the digital domain by adding optimized noise directly to images~\cite{szegedy2013intriguing,carlini2017towards}, researchers find that such attacks also exist physically in the real world \cite{eykholt2018robust,song2018physical,zhao2019seeing}. To fill the gap between attacks in the digital and physical worlds, recent studies have demonstrated that attacks on real-world computer vision systems are practical \cite{eykholt2018robust,li2019adversarial,man2020ghostimage,sharif2016accessorize,zhao2019seeing,zhou2018invisible}. However, attacks on traditional computer vision methods such as SLAM are relatively less explored. \cite{yoshida2022adversarial} proposes an attack against the scan matching algorithm in LiDAR-based SLAM, while most SLAMs in AR/VR devices rely on different sensors like RGB/depth cameras and IMUs. \cite{ikram2022perceptual} and \cite{chen2024adversary} mislead visual SLAM by poisoning the images with special patterns, and \cite{wang2021can} causes the camera to fail using infrared light. In our work, we demonstrate attacks on Visual-Inertial SLAM (VI-SLAM) by perturbing the IMU readings, rather than cameras, and showing its impact on XR user experience. 

\textbf{Acoustic Injection Attacks.} Among various physical attacks, acoustic injection attacks are attractive due to their low cost. Son~\etal~\cite{son2015rocking} were the first to introduce acoustic attacks on MEMS gyroscopes, demonstrating how these attacks could lead to sensor denial-of-service and result in drone crashes. WALNUT~\cite{trippel2017walnut} expanded on this by developing output biasing and control attacks that enable precise manipulation of MEMS accelerometer outputs using modulated sound waves. Wang et al.~\cite{wang2017sonic} demonstrated a sonic gun, showcasing the vulnerability of various smart devices (\eg drones and self-balancing vehicles) to acoustic attacks. Tu et al. \cite{tu2018injected} designed side-swing and switching attacks to alter the outputs of MEMS gyroscopes and accelerometers. Furthermore, Ji et al. \cite{ji2021poltergeist} fool the object detectors by applying acoustic attack to the image stabilizers commonly used in modern cameras. However, none of the existing works study the relationship between the acoustic injections and SLAM outputs on recent XR devices. 

% \zijian{Do we need one session about security in AR/VR?}
% \yicheng{TODO}
%\jiasi{cite the AIVR paper (UMass Amherst?) paper is we have not already. They add IMU perturbation but w/o SLAM, iirc} \yicheng{Cited}

\textbf{XR Security and Privacy.} 
%Security and privacy concerns in XR systems have gained significant attention. 
For single-user XR systems, researchers have demonstrated various side-channel attacks to extract sensitive information (\eg keystrokes) through video feeds~\cite{ling2019know}, head movements~\cite{nair2023unique, slocum2023going}, architectural hints~\cite{zhang2023its,shang2020arspy}, power usage~\cite{li2024dangers}, and EM side-channel leakages~\cite{al2021vr}. In multi-user XR systems, Su et al.~\cite{su2024remote} use avatar motion data to infer keystrokes in shared VR environments. Slocum et al.~\cite{slocum2024doesn} reveal vulnerabilities in the shared state frameworks of multi-user AR. Similarly, Lebeck et al.~\cite{lebeck2017securing} highlight risks like deceptive virtual objects and emphasize access control for managing shared physical and virtual spaces. Ruth et al.~\cite{ruth2019secure} further propose a secure multi-user AR framework focusing on content sharing and permissions.
Chandio et al.~\cite{chandio2024stealthy} %introduced a multi-modal spatiotemporal attack that 
simultaneously manipulated visual and inertial sensors to disrupt XR pose estimation. However, their study evaluated the attack using offline datasets and assumed the attacker's capability to manipulate IMU data streams through acoustic means, without real experiments. Ours is the first to demonstrate acoustic injection attacks on recent XR devices, like the Hololens 2, in the real world.
 


  \section{Experiments: Planning outperforms Heuristics}
\label{sec:experiment}

We begin our empirical demonstrations by showcasing the effectiveness of our planning framework on both synthetic and real datasets. We focus on the simplest planning algorithm, 1-step lookaheads (Algorithm~\ref{alg:complete}), and show that even basic planning can hold great promise. 
We illustrate our framework using two uncertainty quantification modules---GPs and 
\ensembles/ \ensembleplus. 

Throughout this section, we focus on evaluating the mean squared error of 
a regression model $\model$,  and develop adaptive policies that minimize uncertainty on $g(f)$ defined in~\eqref{eqn:l2-g-f}.
When GPs provide a valid model of uncertainty, 
our experiments show that our planning framework significantly outperforms other baselines. 
We further demonstrate that our conceptual framework extends to deep learning-based uncertainty quantification methods such as  \ensembleplus while highlighting computational challenges that need to be resolved in order to scale our ideas. 
For simplicity, we assume a naive predictor, i.e., $\psi(\cdot) \equiv 0$. However, we emphasize that this problem is just as complex as if we were using a sophisticated model $\psi(.)$. The performance gap between the algorithms 
primarily depends
on the level  of uncertainty in our prior beliefs.

To evaluate the performance of our algorithm, we benchmark it against several baselines. 
%Active learning baselines use an acquisition function $\ac$ to select points that have the highest   function value: $X\opt_t \in \argmax_{X \in \xpoolj{t}} \ac({X})$ at every step $t$. These methods may also need an UQ module, which we simply use the same UQ module as in our algorithm, and it  outputs $V(X)$ that measures the the uncertainty of each point $X \in \xpoolj{t}$.
Our first set of baselines are from active learning~\citep{AggarwalKoGuHaPh14}:
\\ % \noindent\textbf{Active Learning Heuristics:} 
\textbf{(1)} 
\textsf{Uncertainty Sampling (Static):}  In this approach, we query the samples for which the model is least certain about. Specifically, we estimate the variance of the latent output $f(X)$ for each $X \in \xpool$ using the UQ module and select the top-$K$ points with the highest uncertainty. \\
\textbf{(2)} \textsf{Uncertainty Sampling (Sequential):} This is a greedy heuristic that sequentially selects the points with the highest uncertainty within a batch, while updating the posterior beliefs using pseudo labels from the current posterior state. Unlike \textsf{Uncertainty Sampling (Static)}, this method takes into account the information gained from each point within batch, and hence tries to diversify the selected points within a batch. 

 
We also compare our approach to the  \textbf{(3)} \textsf{Random Sampling}, which selects each batch uniformly at random from the pool. Additionally, we compare solving the planning problem using  \textsf{REINFORCE}-based policy gradients with   $\mathsf{Smoothed\text{-}Autodiff}$ policy gradients.\footnote{Our code repository is available at
  \url{https://github.com/namkoong-lab/adaptive-labeling}.}
%Detailed experimental setups are provided in Section \ref{sec:details-experiments}.

%We repeat all experiments with 10 random seeds.




\begin{figure}[t]
\centering
\begin{minipage}[b]{0.49\textwidth}
\centering
\includegraphics[width=\textwidth, height=5cm]{figures/original_scale/Var_of_l_2_loss.pdf}
\caption{(Synthetic data) Variance of mean squared loss evaluated through the posterior belief $\mu_t$ at each horizon $t$. This is the objective that policy gradient methods like \textsf{REINFORCE} and $\ouralgo$ optimizes. 1-step lookaheads are surprisingly effective even in long horizons.}
\label{fig:var-l2-sim}
\end{minipage}
\hfill
\begin{minipage}[b]{0.49\textwidth}
\centering \includegraphics[width=\textwidth, height=5cm]{figures/original_scale/Error_of_estimated_model_l_2_loss.pdf}
\caption{(Synthetic data) Error between MSE calculated based on collected data $\mc{D}^{0:T}$ vs. population oracle MSE over $\mc{D}_{\rm eval} \sim P_X$. Reducing uncertainty over posteriors directly leads to better OOD evaluations. 1-step lookaheads significantly outperform active learning heuristics in small horizons.}
\label{fig:mean-l2-sim}
\end{minipage}
%\caption{Simulated data for GPs}
%\label{fig:both_plots}
\end{figure}

\subsection{Planning with Gaussian processes}
\label{sec:experiment-plan-GP}
We now briefly describe the data generation process for the GP experiments,  deferring a more detailed discussion of the dataset generation to Section~\ref{sec:details-experiments}. 
We use both the synthetic data and the real data to test our methodology.
For the \emph{simulated data},  we construct a setting where the general population is distributed across \emph{51 non-overlapping clusters} while the initial labeled data $\dtrain$ just comes from one cluster. In contrast, both $\dpool \defeq (\xpool,\ypool),\deval \defeq (\xeval,\yeval)$ are generated   from all the clusters. 
We begin with a low-dimensional scenario, generating a one-dimensional regression setting using a GP. %Gaussian Process (GP).
Although the data-generating process is not known to the algorithms,  we assume that the GP hyperparameters are known to all the algorithms
to ensure fair comparisons. This can be viewed as a setting where our prior is well-specified, allowing us to isolate the effects
of different policy optimization approaches
 without any concerns about the misspecified priors. We select $10$ batches, each of size $K=5$ across $T = 10$ time horizons.

To examine the robustness of our method against the distributional assumptions made  in the simulated case, we then move to a real dataset where the correct prior is not known. We simulate selection bias from the eICU dataset~\citep{PollardJoRaCeMaBa18}, which contains real-world patient data with in-hospital mortality outcomes. 
We conduct a $k$-means clustering to generate 51 clusters and then select data from those clusters. We view this to be a credible replication of practice, as severe distribution shifts are common due to selection bias in clinical labels.  To convert the binary mortality labels into a regression setting, we train a  random forest classifier and fit a GP on predicted scores, which serves as the UQ module for all the algorithms. As before, the task is to select 10 batches, each consisting of 5 samples, across 10 time horizons.

 In Figures~\ref{fig:var-l2-sim} and~\ref{fig:mean-l2-sim}, we present results for the simulated data. 
Figure~\ref{fig:var-l2-sim} shows the variance of $\ell_2$ loss, and Figure~\ref{fig:mean-l2-sim} presents the error in the estimated $\ell_2$ loss using $\mu_t$ (relative to true $\ell_2$ loss, that is unknown to the algorithm). 
As we can see from these plots, our method one-step lookahead  gives substantial improvements  over active learning baselines and random sampling. In addition,
compared to the one-step lookahead planning approach using \textsf{REINFORCE}-based policy gradients, 
we observe that $\mathsf{Smoothed\text{-}Autodiff}$-based policy gradients provide significantly more robust performance over all horizons.

In Figures~\ref{fig:var-l2-real}~and~\ref{fig:mean-l2-real}, we observe similar findings on the eICU data. We see that planning policies (\textsf{REINFORCE} and $\mathsf{Smoothed\text{-}Autodiff}$) consistently outperform other heuristics by a large margin.  Active learning baselines perform poorly in these small-horizon batched problems and can sometimes be even worse than the random search baselines.  Overall, our results show the importance of careful planning in adaptive labeling for reliable model evaluation. 

We offer some intuition as to why one-step lookahead planning may outperform other heuristic algorithms. 
 First,  \textsf{Uncertainty sampling (Static)} while myopically selects the
 top-$K$ inputs with the highest uncertainty, it fails to consider 
the overlap in information content among the ``best” instances; see \citep{AggarwalKoGuHaPh14} for more details. 
In other words,  it might acquire points from the same region with high uncertainty while failing to induce diversity among the batch.
Although \textsf{Uncertainty Sampling (Sequential)} somewhat addresses the issue of information overlap, a significant drawback of 
this algorithm
is the disconnect between the objective we aim to optimize and the algorithm. For example, it might sample from a region with high uncertainty but very low density. 

\begin{figure}[t]
\centering
\begin{minipage}[b]{0.48\textwidth}
\centering
\includegraphics[width=\textwidth, height=5cm]{figures/original_scale/Var_of_l_2_loss_real.pdf}
\caption{(Real-world eICU data) Variance of mean squared loss evaluated through the posterior belief $\mu_t$ at each horizon $t$. Even 1-step lookaheads are extremely effective planners, and auto-differentiation-based pathwise policy gradients provide a reliable optimization algorithm based on low-variance gradient estimates.}
\label{fig:var-l2-real}
\end{minipage}
\hfill
\begin{minipage}[b]{0.48\textwidth}
\centering \includegraphics[width=\textwidth, height=5cm]{figures/original_scale/Error_of_estimated_model_l_2_loss_real.pdf}
\caption{(Real-world eICU data) Error between MSE calculated based on collected data $\mc{D}^{0:T}$ vs. population oracle MSE over $\mc{D}_{\rm eval} \sim P_X$. Reducing uncertainty over posteriors directly leads to better OOD evaluations. Our method significantly outperforms active learning-based heuristics, and random sampling.}
\label{fig:mean-l2-real}
\end{minipage}
%\caption{Real data for GPs}
\end{figure}
 
%\vspace{-1.5cm}
% \begin{wrapfigure}{r}{.32\columnwidth}
%   \vspace{-.5cm} 
%   \centering
% \includegraphics[scale=.29]{figures/Var of l2l_2 loss.pdf}
%   \vspace{-0.2cm}
%   \caption{Results of GP}
% \label{fig:var-l2-gp}
%   \vspace{-0.1cm}
% \end{wrapfigure}


% Attempts have been made  in the past to address these  drawbacks heuristically  (see \citep{AggarwalKoGuHaPh14}). We give a unified computational framework while approaching the problem in a more principled manner and solving it more optimally.




\subsection{Planning with  neural network-based uncertainty quantification methods ($\ensembleplus$)}


We now provide a proof-of-concept that shows the generalizability of our conceptual framework  to the deep learning-based UQ modules, specifically focusing on $\ensembleplus$ due to their previously observed superior performance~\citep{OsbandWenAsDwIbLuRo23}. Recall that implementing our framework with deep learning-based UQ modules  requires us to retrain the model across multiple possible random actions $\bm{a}(\theta)$ sampled from the current policy $\pi_\theta$.
This requires significant computational resources, in sharp contrast to the GPs where the posteriors are in closed form and can be readily updated and differentiated. 

Due to the computational constraints, we test $\ensembleplus$ on a toy setting to demonstrate the generalizability of our framework. We consider a setting where the general population consists of four clusters, while the initial labeled data only comes from one cluster. Again we generate data using GPs.  The task is to select a batch of 2 points in one horizon. We detail the $\ensembleplus$ architecture in Section \ref{sec:details-experiments}, and we assume prior uncertainty to be large (depends on the scaling of the prior generating functions). 
The results are summarized in the Table~\ref{tab:UQ_ensemble}.

% \begin{table}[H]
% \vspace{-10pt}
% \caption{Performance under \ensembleplus as UQ module}
%     \centering
%     \begin{tabular}{|m{3cm}|m{2.5cm}|m{2cm}|} 
%     \hline
%       Algorithm   & Variance of $\loss_2$ loss estimate & Error of $\loss_2$ loss estimate  \\ \hline Random Sampling 
%          & $1710.9 \pm 1352.1$ & $8.67\pm6.62$ 
%       \\ \hline \ouralgo & $1.30 \pm 0.68$ & $0.91\pm0.25$ \\ \hline
%     \end{tabular}
%     \label{tab:UQ_ensemble}
%     %\vspace{-10pt}
% \end{table}




\begin{table}[h]
\vspace{-10pt}
\caption{Performance under \ensembleplus as the UQ module}
\centering
\begin{tabular}{|l|l|l|}
\hline
Algorithm   & Variance of $\loss_2$ loss estimate & Error of $\loss_2$ loss estimate  \\
\hline
\textsf{Random sampling} & 7129.8 $\pm$ 1027.0 & 136.2 $\pm$ 8.28 \\ \hline
\textsf{Uncertainty sampling (Static)} & 10852 $\pm$ 0.0 & 162.156 $\pm$ 0.0 \\ \hline
\textsf{Uncertainty sampling (Sequential)} & 8585.5 $\pm$ 898.9 & 144 $\pm$ 6.93 \\ \hline
\textsf{REINFORCE} & 1697.1 $\pm$ 0.0 & 45.27 $\pm$ 0.0 \\ \hline
\ouralgo & 1697.1 $\pm$ 0.0 & 45.27 $\pm$ 0.0 \\ \hline
\end{tabular}
%\caption{Comparison of different algorithms based on variance   and   error in $\ell_2$ loss estimation with Ensemble $+$ as the UQ module. Our results demonstrate that {\ouralgo} and REINFORCE outperformthe other active learning based heuristics, confirming the benefits of our MDP formulation for the adaptive labeling problem, as also demonstrated in Section 4.\\
%\footnotesize{Experimental details: We use Gaussian Processes as our data generating process, GP parameters are the same as in Section D.3.  The task is to select a batch of 2 points along one horizon.The marginal distribution $p_X$ has 4 \textit{non-overlapping} clusters. Initial data comes from one cluster, while pool and evaluation points comes from all the clusters. We have $20$ initial labeled data points, $10$ pool points, and $252$ evaluation points.  Training procedures are similar to the one in Section D.3.} }
\label{tab:UQ_ensemble}
\end{table}



% We faced  issues in scaling up these experiments which will be our focus in the future. 





% \begin{itemize}
%     \item Posteriors should be consistent. Two dimensions: even with less training,  
%     \item the inference should be  fast enough
% \end{itemize}


% Potential research directions for uncertainty quantification

% In this section we consider a simple setting We consider a simpler setting and 


% For synthetic dataset generation, we use ...... For real datasets, we use ...... We compare our methodolgy to several baselines ()    This Section is structured as follows:
% \begin{itemize}
%     \item \textbf{GPs, square loss objective} (Section \ref{}): 
%     %the broad aim of the experiments  in this section is to isolate the performance of our methodology without any concerns for the inefficiencies induced due to a mis-specified prior or imperfect posterior inference. To accomplish this we generate synthetic datasets using GPs (detailed later). We use the well specified prior (GPs - with same hyperparameter setting) as our UQ module.   
%      As GPs provide differentaible posterior inference - any errors induced due to imperfect posterior updates are also isolated. We note that under this setting
%      \item In Section\ref{} we demonstrate why our methodology performs better than other baselines - by devising various synthetic experiments ()
%     \item  \textbf{UQ Benchmarking }(Section \ref{}): Before diving into the experiments using $\ensembleplus$ and ENNs,  we showcase our benchmarking experiments in Section \ref{}. We use real datasets We observe that ENNs perform better
%      \item \textbf{Ensemble $+$}, objective: recall, accuracy
%     \item \textbf{ENN}, objective: recall, accuracy
% \end{itemize}




% In Section {}, we test 
% \subsection{Experimental details}

% \begin{itemize}
%     \item UQ methodologies - GPs, ENNs
%     \item Objectives - Recall,  ATE
%     \item Datasets - ATE-synthetic datasets, Recall-synthetic, real datasets
%     \item Baselines - 
%     \begin{itemize}
%         \item Random sampling
%         \item Active learning - Uncertainty based sampling - In regression setting almost all of the 
%         \item Myopic greedy - Greedy Batch based sampling
%         \item Policy Gradient
%     \end{itemize}
    
% \end{itemize}

% \subsection{Experiments}
%     \begin{itemize}
%     \item GPs with square loss
%     \item Benchmarking ENN
%         \item ENNs with ATE
%         \item ENNs with Recall
%     \end{itemize}

% \subsection{Benefits over other algorithms - intuition and experiments}

%Active learning - Myopic greedy / Don't rely on the objective rather some entropy version.


%%% Local Variables:
%%% mode: latex
%%% TeX-master: "main"
%%% End:

}
\paper{
  \input{recurrence_short}
  \input{unsafety_short}
  \input{related_short}
  \input{experiments_short}
}

\bibliographystyle{splncs04}
\paper{
  \bibliography{refs,crossrefs,strings}
}
\report{
  \documentclass{MITstyle}

%\usepackage[table]{xcolor}
\usepackage{chngcntr}
\usepackage{hyperref}
\usepackage{microtype}

\title{A Lightweight and Extensible Cell Segmentation and Classification Model for Whole Slide Images}

\author{Nikita Shvetsov~$^{1, }$\footnote{Correspondence e-mail: nikita.shvetsov@uit.no}, Thomas K. Kilvaer~$^{2, 3}$, Masoud Tafavvoghi~$^{4}$, Anders Sildnes~$^{1}$, \\ Kajsa Møllersen~$^{4}$, Lill-Tove Rasmussen Busund~$^{5, 6}$, Lars Ailo Bongo~$^{1}$ \\
%
\vspace{1em} % Space between authors and afilliations
%
\normalfont{\small $^{1}$Department of Computer Science, UiT The Arctic University of Norway}\\
\normalfont{\small $^{2}$Department of Oncology, University Hospital of North Norway}\\
\normalfont{\small $^{3}$Department of Clinical Medicine, UiT The Arctic University of Norway}\\
\normalfont{\small $^{4}$Department of Community Medicine, UiT The Arctic University of Norway}\\
\normalfont{\small $^{5}$Department of Medical Biology, UiT The Arctic University of Norway} \\
\normalfont{\small $^{6}$Department of Clinical Pathology, University Hospital of North Norway} %\vspace{2em}
}

\begin{document}
\maketitle

\section*{Abstract}

% \begin{abstract}
% Developing clinically useful cell-level analysis tools in digital pathology remains challenging due to limitations in dataset granularity, inconsistent annotations, computational demands of advanced models, and difficulties in integrating new technologies into clinical workflows. To address these challenges, we propose a multi-faceted solution that enhances data quality, model performance, and usability to create a lightweight and extensible cell segmentation and classification model.

% First, we update data labels by employing a cross-relabeling process that refines the labels of two existing datasets, PanNuke and MoNuSAC, to create a new unified dataset with enhanced granularity, encompassing seven distinct cell types. Second, we leverage the H-Optimus foundation model as a fixed encoder to improve feature representation for simultaneous cell segmentation and classification tasks. Third, to address the computational demands of foundation models, we employ knowledge distillation to reduce model size and complexity while maintaining comparable performance. Finally, to facilitate integration into clinical workflows, we integrate the distilled model into the QuPath software, a widely used open-source platform in digital pathology.

% Our results demonstrate improvements in cell segmentation and classification performance using the H‑Optimus-based model compared to a CNN-based model. Specifically, the average $R^2$ improved from 0.575 to 0.871, and the average $PQ$ score improved from 0.450 to 0.492, indicating better alignment with actual cell counts and enhanced segmentation and classification quality. Furthermore, the distilled student model maintains performance comparable to the larger foundation model while reducing the parameter count by a factor of 48.
% Overall, by reducing computational complexity and integrating it into existing workflows, the proposed approach may significantly impact diagnostic processes, reduce the workload of pathologists, and contribute to improved patient outcomes. Though our approach shows potential enhancements in efficiency and usability of cell segmentation and classification models in digital pathology, extensive validation is needed to deploy these models in clinical practice.
% \end{abstract}

%%% shortened abstract
\begin{abstract}
Developing clinically useful cell-level analysis tools in digital pathology remains challenging due to limitations in dataset granularity, inconsistent annotations, high computational demands, and difficulties integrating new technologies into workflows. To address these issues, we propose a solution that enhances data quality, model performance, and usability by creating a lightweight, extensible cell segmentation and classification model. 

First, we update data labels through cross-relabeling to refine annotations of PanNuke and MoNuSAC, producing a unified dataset with seven distinct cell types. Second, we leverage the H-Optimus foundation model as a fixed encoder to improve feature representation for simultaneous segmentation and classification tasks. Third, to address foundation models' computational demands, we distill knowledge to reduce model size and complexity while maintaining comparable performance. Finally, we integrate the distilled model into QuPath, a widely used open-source digital pathology platform. 

Results demonstrate improved segmentation and classification performance using the H-Optimus-based model compared to a CNN-based model. Specifically, average $R^2$ improved from 0.575 to 0.871, and average $PQ$ score improved from 0.450 to 0.492, indicating better alignment with actual cell counts and enhanced segmentation quality. The distilled model maintains comparable performance while reducing parameter count by a factor of 48. By reducing computational complexity and integrating into workflows, this approach may significantly impact diagnostics, reduce pathologist workload, and improve outcomes. Although the method shows promise, extensive validation is necessary prior to clinical deployment.
\end{abstract}
\clearpage

\section{Introduction}
In digital pathology, accurate segmentation and classification of cells are crucial for many diagnostic, prognostic, and predictive analyses \cite{Jaber_Beziaeva_etal._2019,Lin_Pan_etal._2022,Park_Ock_etal._2022,Shen_Choi_etal._2024}. Nowadays, developments in computational pathology offer multiple solutions \cite{H._Qu_P._Wu_etal._2020,Javed_Mahmood_etal._2020} to utilize cell-level datasets to train machine learning models that solve these problems. The quality and specificity of training datasets are critical for robust and accurate models. Adhering to the principle of "garbage in, garbage out", it is essential to ensure that these datasets are extensively and accurately labeled with distinct classes that reflect the diverse biological characteristics of different cell types. Unfortunately, the number of open-source datasets comprising such high-quality annotations is limited. Existing cell segmentation datasets \cite{Gamper_Koohbanani_etal._2019,Graham_Vu_etal._2019,Verma_Kumar_etal._2021} may offer extensive annotations for certain cell types while providing more general labels for others. For example, in PanNuke, which is one of the largest open-source datasets comprising labeled cells, various types of morphologically and functionally different inflammatory cells like macrophages and lymphocytes are clustered in a broad "inflammatory" class. Consequently, these classes are frequently omitted from analyses or aggregated into broader meta-classes \cite{Gamper_Koohbanani_etal._2020} and likely interfere with other cell classes included in the dataset. This and similar inconsistencies in annotation granularity limit the ability of machine learning models to learn the comprehensive and nuanced features necessary for accurate cell segmentation and classification. To address these challenges, methods for refining and standardizing dataset annotations are essential to enhance the quality of training data.

A complementary approach to mitigate the absence of high-quality training data is the use of foundation models. Foundation models as encoders are defined as large-scale, versatile networks pre-trained on vast, diverse datasets using self-supervised learning, contrasting with convolutional neural network (CNN) pre-trained encoders that rely on supervised learning with labeled data. In practice, foundation models leverage enormous amounts of weakly or unlabeled data from millions of whole slide images (WSIs) and employ self-attention mechanisms to capture long-range dependencies and global context \cite{Chen_Ding_etal._2024,Saillard_Jenatton_etal._2024,Vorontsov_Bozkurt_etal._2024,Xu_Usuyama_etal._2024}. As a consequence, foundation models are able to produce transferable feature representations across different cell types and tissue environments. The feature representations can be leveraged by decoder networks to produce segmentation masks and pixel-level classifications. Because foundation models have comprehensive feature representations, they can be effectively fine-tuned using much smaller amounts of cell-level data compared to the large datasets needed to train models from scratch. Furthermore, foundation models incorporate adversarial training elements or contrastive learning \cite{Chen_Ding_etal._2024,Xu_Usuyama_etal._2024}, enhancing their resilience and adaptability by exposing them to challenging and varied scenarios during training. This may result in more generalizable models, often making them well-suited for diverse and complex tasks in digital pathology.

Despite the inherent advantages of foundation models, their deployment for practical use faces its own obstacles. In particular, they require substantial computational power, financial investments and rigorous testing to ensure reliability and efficacy for a given task \cite{Akkus_Dangott_etal._2022,Dragomir_Cocuz_etal._2022,Go_2022,Jafri_Farooqui_etal._2024}. Moreover, while foundation models enhance feature representation and performance, they depend on the quality of available annotations for decoder fine-tuning and, like any other model, cannot resolve existing inconsistencies or ambiguities in data labels. Therefore, there remains a critical need for solutions that address both data quality and practical deployment considerations.
Further, integrating new technologies into existing clinical workflows often encounters resistance, as it necessitates adjustments to established diagnostic processes. So, there is a need to develop solutions that could be integrated into current practices, minimizing the burden on medical professionals to adopt new tools \cite{King_Williams_etal._2023}.

Existing solutions \cite{Goldsborough_Philps_etal._2024,Hörst_Rempe_etal._2024}, while addressing some aspects of these challenges, fall short in providing a comprehensive approach. To address the data quality and clinical deployment issues, we propose a multi-faceted solution that encompasses data refinement, model optimization, and integration with existing pathology tools (\hyperref[fig:fig1]{Figure 1}). The outcome is a lightweight cell segmentation and classification model that can be integrated into digital pathology workflows for practical clinical use.

\begin{figure}[h!]
    \centering
    \includegraphics[width=\textwidth, height=0.82\textheight, keepaspectratio]{images/Figure_1.pdf}
    \caption{Overview of the proposed solution, including 1) Data refinement using cross-relabeling, 2) Teacher model development and fine tuning, 3) Student model optimization with knowledge distillation and 4) Student model and QuPath integration}
    \label{fig:fig1}
\end{figure}
\clearpage

Our approach begins with preparing the data for the fine-tuning and training of the machine learning models. We create a refined dataset, acquired via cross-relabeling two cell-level datasets, enhancing annotation specificity and consistency of the labeled data. Subsequently, we create a cell segmentation and classification model based on the foundation model. We leverage the foundation model as a fixed encoder and fine-tune a decoder using the refined dataset to improve generalization across diverse tissue- and cell types.
To ensure that the model remains lightweight and deployable in a possibly resource-constrained environment, we employ knowledge distillation to approximate the functionality of the foundation model. Finally, to facilitate the practical application of our model in digital pathology workflows, we integrate it with the QuPath \cite{Bankhead_Loughrey_etal._2017} application. Each methodological component contributes to the overarching goal of enhancing model performance, generalizability, and usability in clinical settings.

The primary contributions of this paper are:
\begin{enumerate}
    \item \textit{Data labels refinement through cross-relabeling:}
    
    We propose a new method for refining labels of cell-level datasets through cross-relabeling. This method employs classification models to re-label broad and ambiguous instances, resulting in a more diverse dataset. Our evaluation demonstrates that these classification models achieve high accuracy on test subsets, indicating the reliability of the method for label refinement.

    \item \textit{Enhanced model performance via foundation models:}
    
    We employ a foundation model as a feature extractor for the cell segmentation and classification task. In comparison with training a CNN model from scratch, the foundation model backbone only needs fine-tuning, which significantly reduces training time, computational resources and data requirements. We show that using a foundation model encoder leads to better performance in cell segmentation and classification networks than using a CNN-based encoder. This improvement may enable the model to generalize more effectively across various tissue types and imaging methods.
    
    \item \textit{Model optimization through knowledge distillation:}
    
    We show that a smaller student model trained using knowledge distillation on the refined dataset obtained via our cross-relabeling approach from a foundation model achieves comparable performance in cell segmentation and quantification tasks. As a result, this model is more suitable for deployment in environments without high-performance computing resources.
    
    \item \textit{Integration with QuPath:}
    
    We integrate the distilled cell segmentation and classification model into QuPath, a widely used open-source digital pathology platform, to accelerate clinical adaptation by enabling pathologists to more easily incorporate advanced computational tools into their existing workflows.
\end{enumerate}

Through these methodological steps, we aim to bridge the gap between advanced machine learning techniques and practical clinical applications, making accurate and efficient digital pathology accessible in a broader range of healthcare settings.

\section{Refining Existing Datasets Using Cross-Relabeling}
To address the limitations of sparse and ambiguous labeling of cell-level datasets, we propose a generalizable cross-relabeling strategy that can be applied to any dataset containing broadly categorized or imprecisely labeled cell types. This approach involves training and subsequently leveraging classification models to refine broad categories into more specific or biologically relevant classes.
When applied to cell-level data, the methodology includes extracting individual cell images from the dataset patches, preprocessing these images to standardize the size and accommodate partial cells, and then training deep learning classifiers capable of distinguishing between the finer cell subtypes within the coarser categories. 
To illustrate our approach, we focus on the PanNuke \cite{Gamper_Koohbanani_etal._2020, Gamper_Koohbanani_etal._2019} and MoNuSAC \cite{Verma_Kumar_etal._2021} datasets that we have used to train models for cell quantification in our previous works \cite{Shvetsov_Grønnesby_etal._2022,Shvetsov_Sildnes_etal._2024}. We find that for better cell differentiation we have to introduce more granular labels. PanNuke includes a broad classification of "inflammatory" cells, encompassing lymphocytes, macrophages, and neutrophils. Each cell type differs significantly in structure, function, and clinical relevance. Conversely, MoNuSAC uses the label "epithelial" for a class that comprises both benign epithelial cells and malignant neoplastic cells. This practice makes it challenging to differentiate between benign and malignant epithelial cells in the dataset, which is a critical distinction when identifying tumor areas within tissue samples. To address these issues, we implement a cross-relabeling strategy as shown in \hyperref[fig:fig2]{Figure 2}. The key components are two classification models: one is trained on singular cell images from PanNuke data to classify the epithelial meta-class into epithelial and neoplastic classes. The other is trained on MoNuSAC to refine the inflammatory class into lymphocytes, neutrophils, and macrophages.

\begin{figure}[h!]
    \centering
    \includegraphics[width=\textwidth]{images/Figure_2.pdf}
    \caption{Refined dataset generation via cross relabeling}
    \label{fig:fig2}
\end{figure}

The refining approach consists of three consecutive steps. The first is the preprocessing step, in which we extract individual cells from both datasets (\hyperref[fig:fig3]{Figure 3}). The specifics of PanNuke and MoNuSAC patch preparation before cell preprocessing are provided in \hyperref[chap:S1]{Appendix S1}.

\begin{figure}[h!]
    \centering
    \includegraphics[width=\textwidth]{images/Figure_3.pdf}
    \caption{Cell instances preprocessing including (1) cell map extraction, (2) bounding box delineation, (3) adjusting cell boxes and (4) cropping and resizing of cell images}
    \label{fig:fig3}
\end{figure}

During preprocessing, we extract cell type maps from the ground truth label mask and calculate bounding boxes around each cell instance. To accommodate partial cells at patch borders, a common issue in cropped patch images, we employ mirror padding and extend the field of view of the cell label by 15 pixels to capture adjacent cells. We then crop and resize the identified regions to $64 \times 64$ pixels using bicubic interpolation.

The preprocessed PanNuke dataset comprises 68,031 neoplastic and 23,207 epithelial cell images, while MoNuSAC comprises  33,104 lymphocytes, 1,252 neutrophils, and 1,695 macrophages, which we subsequently use in training cell classification models and classifying the cell image data \hyperref[fig:S2]{Appendix Figure S2 (1)}. 

The next step is to train two distinct ResNet50-based classifiers tailored to address the specific labeling challenges inherent in each dataset. We use ResNet50 for classification models due to its proven effectiveness for image classification tasks in histopathology \cite{pan2022reviewmachinelearningapproaches}, and its compatibility with small images. For the PanNuke dataset, we design the classifier, trained on MoNuSAC data, to disaggregate the heterogeneous "inflammatory" cell category into distinct subtypes: lymphocytes, macrophages, and neutrophils. Similarly, for the MoNuSAC dataset, the classifier is trained on PanNuke data and distinguishes between benign and malignant epithelial cells within the overarching "epithelial" label. By applying these targeted classifiers to their respective datasets, we assign more specific labels to individual cell instances, thus enabling us to create a unified dataset.
To ensure a balanced representation of classes, we train both models on datasets that had been equalized to match the size of the least represented class. Thus, we obtain datasets comprising 23,207 samples per class for PanNuke and 1,252 samples per class for MoNuSAC data. Next, we partition both of them into training (70\%), validation (20\%), and testing (10\%) subsets. To mitigate the risk of overfitting, we use a single dropout layer with a rate of p=0.5 in both models and data augmentation using randomized color perturbations, rotation, and horizontal and vertical flipping. We employ AdamW optimizer and the cross-entropy loss function for the training criterion.

To evaluate the two trained models, we measure the classification accuracy on the respective test subsets. The accuracies on the test subset for both classifiers are presented in \hyperref[tab:1]{Table 1}. The PanNuke model achieves an average accuracy of 93.57\%, with higher accuracy for neoplastic cells (96.06\%) compared to epithelial cells (86.26\%). The confusion matrix in Figure A3.1 shows that the model predominantly distinguishes accurately between epithelial and neoplastic tissues, with a substantial number of correct classifications and relatively few misclassifications. The MoNuSAC model demonstrates an average accuracy of 98.92\%, excelling in classifying lymphocytes (99.67\%) and macrophages (94.12\%), with lower performance for neutrophils (85.71\%). The confusion matrix in Figure A3.2 shows that the model identifies lymphocytes and performs reasonably well with macrophages and neutrophils.

\begin{table}[h!]
\renewcommand{\arraystretch}{1.5}
  \centering
  \caption{Cell classification results for PanNuke and MoNuSAC trained models (CI 95\%).}
  \label{tab:1}
  \begin{tabular}{|l|c|c|}
   \hline
   %\rowcolor{gray!30}
    Accuracy               & PanNuke model              & MoNuSAC model              \\
    \hline
    Average      & 0.936 (0.931--0.941)         & 0.989 (0.986--0.993)        \\
    \hline
    Neoplastic   & 0.961 (0.956--0.965)         & -                          \\
    \hline
    Epithelial   & 0.863 (0.849--0.877)         & -                          \\
    \hline
    Lymphocytes  & -                          & 0.997 (0.995--0.999)        \\
    \hline
    Neutrophils  & -                          & 0.857 (0.796--0.918)        \\
    \hline
    Macrophages  & -                          & 0.941 (0.906--0.976)        \\
    \hline
  \end{tabular}
\end{table}

Finally, during the last step, we use the model trained on PanNuke data for epithelial cells in MoNuSAC and the model trained on MoNuSAC for the inflammatory cells class in PanNuke. Specifically, we use classifier models to relabel epithelial cells in MoNuSAC and inflammatory cells in PanNuke data. Then we combine cells with refined labels and the rest of the cells in both datasets to create a refined dataset (\hyperref[fig:S2]{Appendix Figure S2 (2)}). The process of relabeling cells and visualizing them on a patch is shown in \hyperref[fig:fig4]{Figure 4}. The cell counts in the refined dataset are provided in \hyperref[tab:S4]{Appendix Table S4}.

\begin{figure}[h!]
    \centering
    \includegraphics[width=\textwidth, height=0.42\textheight, keepaspectratio]{images/Figure_4.pdf}
    \caption{Cell relabeling procedure for epithelial and inflammatory cell classes}
    \label{fig:fig4}
\end{figure}

%\hfill

Relabeling and combining datasets have been explored in a prior study \cite{Parulekar_Kanwat_etal._2023}, where consecutive fine-tuning on multiple datasets was employed to account for hierarchical class label structures. While the method presented in \cite{Parulekar_Kanwat_etal._2023} is intuitive, it often lacks consistency and requires multiple fine-tuning runs, which can be cumbersome and time-consuming. 
In contrast, cross-relabeling simplifies this process by using specialized classification models tailored to each dataset's specific labeling challenges. This approach provides better transparency and produces a unified dataset encompassing seven distinct cell types across multiple tissue samples, enhancing data diversity for further model training or fine-tuning.

Despite these improvements, cross-relabeling does not entirely resolve issues related to poor labeling quality or the amount of labeled data. Specifically, our results show lower accuracies persist for underrepresented classes, such as macrophages, which may stem from a limited sample availability and intrinsic challenges in distinguishing these cells based solely on H\&E staining. Furthermore, while our method enhances label specificity, it relies on the initial quality of the broad labels; thus, any fundamental inaccuracies in the original annotations can propagate through the relabeling process. Addressing the overall problem of limited data labels may require integrating additional data sources or utilizing complementary immunohistochemical staining methods.
Although the reported performance metrics are obtained from evaluations on the native test sets of each dataset, it is important to note that the primary application of these classifiers is to perform cross-relabeling, where a model trained on one dataset (e.g., PanNuke) is applied to another (e.g., MoNuSAC) and vice versa. We acknowledge that a more systematic evaluation of cross-dataset generalization is needed and could be performed in future work.

Overall, the refined dataset produced by our approach can enhance the supervised training or fine-tuning of cell segmentation and classification models, especially those that utilize pre-trained foundation models to improve feature extraction robustness. In addition, these models can detect nuanced classes that enable researchers to conduct more detailed analyses of biological processes in computational pathology.

\section{Foundation models for robust cell segmentation and classification}

Accurate cell segmentation and classification in digital pathology are hindered by limited labeled data and the fact that conventional CNNs are unable to capture global contextual information due to their local receptive field constraints \cite{Gheflati_Rivaz_2022,Yang_Marcus_etal.}. Traditional approaches in cell quantification have predominantly relied on CNN encoders, such as ResNet50, given their proven effectiveness in semantic segmentation tasks \cite{Deshmane_2023,Graham_Vu_etal._2019,Mukasheva_Koishiyeva_etal._2024,Stringer_Wang_etal._2021}. However, approaches that include fine-tuning of pretrained CNNs, data augmentation, and stain normalization to partially increase data variability and address staining differences often fail to achieve the necessary generalization and robustness across diverse tissue types and staining conditions \cite{G._Wang_W._Li_etal._2018,Gao_Bagci_etal._2018,Karim_El_Khoury_Martin_Fockedey_etal._2021}.

To overcome these challenges, we leverage an encoder-decoder network that uses a foundation model as the encoder and a CNN upsampling decoder (\hyperref[fig:fig5]{Figure 5}) for simultaneous cell segmentation and classification in 2D patches extracted from WSIs. Foundation models with transformer-based architectures are viable alternatives to CNN-based encoders \cite{Shamshad_Khan_etal._2023,Sourget_2023}. They enable the creation of more advanced architectures that can decode or transform learned features more effectively \cite{Chen_Duan_etal._2023,Cheng_Misra_etal._2022,Xie_Wang_etal._2021}.

\begin{figure}[h!]
    \centering
    \includegraphics[width=\textwidth]{images/Figure_5.pdf}
    \caption{UNETR-like model with foundational model as backbone}
    \label{fig:fig5}
\end{figure}

By utilizing a transformer-based encoder, we incorporate global contextual information into the feature extraction process, which is a key advantage of such architectures \cite{Chen_Lu_etal._2021}. This foundation model integration facilitates accurate pixel-wise segmentation and classification without the need for extensive encoder training, thereby potentially improving generalization across varied cellular structures and tissue types.
In our implementation, we employ a modified UNETR \cite{Hatamizadeh_Tang_etal._2021} architecture that combines a vision transformer (ViT) \cite{Dosovitskiy_Beyer_etal._2021} encoder with a CNN-based decoder. The encoder utilizes the pretrained H-Optimus foundation model, which contains 1.1 billion parameters and is trained on over 500,000 H\&E stained WSIs \cite{Saillard_Jenatton_etal._2024}. We extract outputs from four evenly spaced transformer blocks $Z_i$, where $i \in [1, 14, 26, 38]$, to serve as residual connections for the CNN decoder. We select these blocks based on our observation that features from non-adjacent levels of the encoder lead to better overall performance on the test subset.

The CNN decoder upsamples the feature representations, acquired from the transformer blocks, to generate an intermediate vector that is handled by two task-specific layers that generate cell segmentation and classification masks. The first task-specific layer is the ‘Cellpose head’,  which is used to delineate cell instances. The layer generates horizontal and vertical gradient maps to form vector fields that are refined through gradient tracking in a post-processing step using the Cellpose algorithm \cite{Stringer_Wang_etal._2021}, known for its efficacy in cell segmentation tasks and generalizability across multiple domains \cite{Pachitariu_Stringer_2022,Stringer_Pachitariu_2024}. The second task-specific layer is the "Cell type head", which assigns labels to individual pixels. In the post-processing step, we determine the output classification label of each segmented cell instance by majority voting over the labeled pixels that comprise the cell in the segmentation map.

To evaluate model performance and measure the impact of adding a foundation model as backbone, we compare it to a ResNet50-based model. ResNet50 is a widely used solution for encoders in segmentation architectures in the medical domain \cite{Deshmane_2023,Graham_Vu_etal._2019,Mukasheva_Koishiyeva_etal._2024,Stringer_Wang_etal._2021}. For the H-Optimus-based model, we utilize frozen weights for the encoder and only fine-tune the decoder to take advantage of the extensive pre-training of the foundation model. For the ResNet50-based model we start with ImageNet \cite{Deng_Dong_etal.} weights and train both encoder and decoder parts. Hyperparameters for the training step are set to be identical, where possible, for comparable evaluation. 
For this evaluation, we deliberately use the PanNuke dataset to provide a standardized and controlled comparison between the H‑Optimus and ResNet50-based models (\hyperref[fig:S2]{Appendix Figure S2 (3)}). Specifically, we use two of the default PanNuke dataset splits (66\%) for training and validation, and reserve the third split (33\%) for testing.

To address the challenge of cell class imbalance in the PanNuke dataset, which is a common characteristic in most cell-level H\&E patch datasets, both models’ training processes employ a weighted loss function comprising cross-entropy and focal loss \cite{Lin_Goyal_etal._2018}. The focal loss component is adjusted with coefficients derived from each cell class' instance frequency, emphasizing learning from underrepresented classes and enhancing the model's sensitivity to rare but significant cellular patterns. The cross-entropy loss is augmented with spectral decoupling regularization \cite{Pezeshki_Kaba_etal._2021,Pohjonen_Stürenberg_etal._2022} and spatially varying label smoothing \cite{Islam_Glocker_2021}, which potentially stabilizes training and improves generalization in case of complex tissue morphologies. For optimization, we employ the \textit{AdamW} \cite{Loshchilov_Hutter_2019} to counter unbalanced class scenarios, with cosine annealing learning rate scheduler.

We utilize the scikit-learn library \cite{Van_der_Walt_Schönberger_etal._2014} and HoVer-Net \cite{Graham_Vu_etal._2019} implementations of $R^2$ (the coefficient of determination) and $PQ$ (panoptic quality) to evaluate our experiments. Complete mathematical formulations and detailed explanations of these metrics are provided in \hyperref[chap:S5]{Appendix S5}. To compute confidence intervals, we use nonparametric bootstrapping, where after calculating the metric on the full sample, we generated 1000 bootstrap replicates by resampling with replacement and then determined the 95\% confidence intervals as the 2.5th and 97.5th percentiles of the resulting empirical distribution.

%\hfill

The model comparisons are summarized in \hyperref[tab:2]{Table 2}. The H‑Optimus-based model achieves higher $R^2$ across all cell classes compared to the ResNet50-based model, which means that its predictions are more closely aligned with the PanNuke cell counts, indicating a stronger correlation with the observed data. Notably, the improvement of $R^2_{dead}$ may be an indicator of better global contextual representations provided by the foundation model backbone. In terms of segmentation and classification quality combined, measured by the PQ score, the H‑Optimus-based model demonstrates notable improvements across most cell classes. Overall, the average $R^2$ improved from 0.575 to 0.871, while the average $PQ$ score improved from 0.450 to 0.492, demonstrating better performance of the H-Optimus-based model.

\begin{table}[h!]
\renewcommand{\arraystretch}{1.5}
  \centering
  \caption{Cell quantification metrics for baseline and proposed models (CI 95\%).}
  \label{tab:2}
  \begin{tabular}{|l|c|c|}
    \hline
    %\rowcolor{gray!30}
    Metric             & Resnet50-based            & H-optimus-based              \\
    \hline
    $R^2_{neoplastic}$    & 0.681 (0.576--0.769)       & \textbf{0.941 (0.917--0.960)} \\
    \hline
    $R^2_{inflammatory}$  & 0.863 (0.778--0.903)       & \textbf{0.949 (0.918--0.966)} \\
    \hline
    $R^2_{connective}$    & 0.600 (0.488--0.698)       & 0.609 (0.436--0.772)          \\
    \hline
    $R^2_{dead}$          & 0.097 (-11.389--0.669)     & 0.925 (0.404--0.982)          \\
    \hline
    $R^2_{epithelial}$    & 0.635 (0.490--0.747)       & \textbf{0.930 (0.886--0.964)} \\
    \hline
    $PQ_{neoplastic}$       & 0.517 (0.499--0.535)       & \textbf{0.589 (0.575--0.604)} \\
    \hline
    $PQ_{inflammatory}$     & 0.455 (0.429--0.482)       & \textbf{0.528 (0.507--0.549)} \\
    \hline
    $PQ_{connective}$       & 0.416 (0.400--0.431)       & \textbf{0.451 (0.436--0.465)} \\
    \hline
    $PQ_{dead}$             & 0.374 (0.342--0.408)       & 0.292 (0.209--0.365)          \\
    \hline
    $PQ_{epithelial}$       & 0.488 (0.460--0.519)       & \textbf{0.599 (0.579--0.618)} \\
    \hline
  \end{tabular}
\end{table}

Our results  show that integrating the H‑Optimus foundation model within the UNETR architecture enhances the model's ability to segment and classify cells across diverse tissues from PanNuke data. The pretrained transformer encoder provides robust feature representations, resulting in higher average $R^2$ and $PQ$ scores compared to the CNN-based model. This leads to more reliable cell quantification and more accurate downstream analysis. Additionally, the streamlined fine-tuning process reduces computational overhead and training time, making the model more adaptable for new data.

Despite these advancements, the foundation model-based approach does not fully resolve all challenges related to cell segmentation and classification. We observe lower metric scores for underrepresented classes in the training data. Furthermore, foundation models typically encompass billions of parameters, resulting in substantial computational and memory requirements. It therefore poses challenges for deployment in resource-constrained environments, limiting their practical applicability in certain clinical settings.

\section{Model optimization via Knowledge Distillation}

To address the limitations posed by the extensive size of foundation models, we implement knowledge distillation — a model compression technique that leverages the teacher-student paradigm \cite{Hinton_Vinyals_etal._2015}. By training a smaller, more efficient student model to replicate the output of a larger, pre-trained teacher model, we retain performance while significantly reducing the model's complexity and resource requirements (\hyperref[fig:fig6]{Figure 6}).

\begin{figure}[h!]
    \centering
    \includegraphics[width=\textwidth, height=0.45\textheight, keepaspectratio]{images/Figure_6.pdf}
    \caption{Knowledge distillation framework for training a student model using a pre-trained teacher}
    \label{fig:fig6}
\end{figure}

We employ knowledge distillation to compress the H‑Optimus-based teacher model into a more efficient student model. The teacher model is the modified UNETR architecture with the H‑Optimus foundation model described in the previous chapter. The student model is based on a UNet architecture augmented with residual connections and incorporates a smaller ViT encoder with 9 million parameters \cite{Steiner_Kolesnikov_etal._2022,Wightman_2019}. 

First, we fine-tune the teacher model using the refined dataset from the cross-relabeling procedure (Section 2). Initially we train the decoder of the teacher model while keeping the encoder weights frozen. We split the refined dataset into train (70\%), validation (20\%) and test (10\%) subsets (\hyperref[fig:S2]{Appendix Figure S2 (4)}). During fine-tuning, we use the train and validation subsets, while leaving the test subset for model evaluation. We set the training procedure and model hyperparameters to be identical to those that were used to demonstrate the utility of foundation models for the simultaneous cell segmentation and classification task.

Next, we perform knowledge distillation from teacher to student using the refined dataset used to fine-tune the teacher model. The student model is trained to replicate the teacher model's outputs. We utilize a specialized loss function that aligns the student's predicted probability distribution with the teacher's, incorporating the teacher's class probability distribution derived from the output. Following the methodology of Hinton et al. \cite{Hinton_Vinyals_etal._2015}, we experiment with various hyperparameter settings for the temperature ($T$) and the balancing coefficients ($\alpha$ and $\beta$) in the loss function. We vary $T$ from 1 to 20 and adjust $\alpha$ and $\beta$ to balance the distillation and student losses. Through iterative tuning and evaluation, we identify that setting $T=14$, $\alpha=0.3$, and $\beta=0.7$ yields a configuration that converges and closely approximates the teacher model's performance during training.

Finally, we assess the performance of both models using the $R^2$ and $PQ$ (defined in \hyperref[chap:S5]{Appendix S5}) on the test set of the refined dataset (\hyperref[tab:3]{Table 3}). We observe that the 95\% confidence intervals overlap for most cell types, so we cannot claim statistically significant performance differences between the teacher and student models. One exception appears in the neoplastic class. The teacher model produces an $R^2$ of 0.919, while the student model shows an $R^2$ of 0.852. In addition, the student model achieves higher $PQ$ values for the neoplastic and connective classes, though the confidence intervals show overlap.

\begin{table}[h!]
\renewcommand{\arraystretch}{1.5}
  \centering
  \caption{Cell quantification metrics for teacher and distilled student models (CI 95\%).}
  \label{tab:3}
  \begin{tabular}{|l|c|c|}
    \hline
    %\rowcolor{gray!30}
    Metric & Teacher & Student \\
    \hline
    $R^2_{neoplastic}$    & \textbf{0.919} (0.898--0.939) & 0.852 (0.800--0.891) \\
    \hline
    $R^2_{lymphocyte}$    & 0.969 (0.956--0.977)         & 0.969 (0.956--0.978) \\
    \hline
    $R^2_{connective}$    & 0.694 (0.548--0.809)         & 0.618 (0.469--0.741) \\
    \hline
    $R^2_{dead}$          & 0.755 (0.400--0.908)         & 0.424 (0.100--0.731) \\
    \hline
    $R^2_{epithelial}$    & 0.922 (0.870--0.958)         & 0.843 (0.738--0.917) \\
    \hline
    $R^2_{macrophage}$    & 0.384 (-0.369--0.724)        & 0.704 (0.352--0.859) \\
    \hline
    $R^2_{neutrofil}$     & 0.854 (0.578--0.929)         & 0.833 (0.502--0.925) \\
    \hline
    $PQ_{neoplastic}$       & 0.581 (0.569--0.593)         & 0.601 (0.588--0.613) \\
    \hline
    $PQ_{lymphocyte}$       & 0.536 (0.520--0.553)         & 0.563 (0.544--0.579) \\
    \hline
    $PQ_{connective}$       & 0.436 (0.421--0.451)         & 0.457 (0.441--0.474) \\
    \hline
    $PQ_{dead}$             & 0.272 (0.235--0.315)         & 0.279 (0.201--0.369) \\
    \hline
    $PQ_{epithelial}$       & 0.522 (0.500--0.545)         & 0.530 (0.506--0.555) \\
    \hline
    $PQ_{macrophage}$       & 0.524 (0.459--0.588)         & 0.474 (0.405--0.543) \\
    \hline
    $PQ_{neutrofil}$        & 0.541 (0.490--0.592)         & 0.565 (0.522--0.607) \\
    \hline
  \end{tabular}
\end{table}


We further decompose the $PQ$ metric into its $SQ$ and $DQ$ components (\hyperref[tab:S6]{Appendix Table S6}). Both models produce nearly identical $SQ$ values, which indicates that they predict instance boundaries with similar precision. Although the student model shows some improvement in $DQ$ scores for certain classes, the confidence intervals overlap and do not confirm a statistically significant difference.

We observe that the student and teacher models yield comparable detection performance despite the student model using a much smaller and simpler architecture. A model with fewer parameters reduces the risk of overfitting when training data are scarce relative to the model’s complexity \cite{Farias_Ludermir_etal._2022}. The knowledge distillation process also encourages the student model to focus on the most generalizable detection features learned from the teacher. These factors enable the student model to achieve similar detection performance across different cell types.

Additionally, considering the model sizes reported in \hyperref[tab:4]{Table 4}, the distilled model achieves a significant reduction compared to the teacher model, with a 48-fold decrease in parameter count and a 5.5-fold reduction in on-disk size. In inference mode, the teacher model requires 16 GB of VRAM for a batch size of 32, while the distilled model only needs 3 GB of VRAM for the same batch size. These reductions make the distilled model significantly more practical for fine-tuning and deployment in resource-constrained environments.

\begin{table}[h!]
\renewcommand{\arraystretch}{1.5}
  \centering
  \caption{Parameter counts and size of teacher and distilled model}
  \label{tab:4}
  \adjustbox{max width=\textwidth}{%
  \begin{tabular}{|l|c|c|c|}
    \hline
    %\rowcolor{gray!30}
    Metric & H-optimus-based (Teacher) & mobileViT-based (Student) & Magnitude of difference \\
    \hline
    Parameters count       & 1,158,917,906   & \textbf{24,093,393}   & \textbf{48x}  \\
    \hline
    Estimated Total Size (MB) & 87,912       & \textbf{15,935}    & \textbf{5.5x} \\
    \hline
  \end{tabular}%
}
\end{table}

%\hfill

With recent advancements in complex network architectures and the use of pretrained encoders to achieve state-of-the-art performance \cite{Baumann_Dislich_etal._2024,Hörst_Rempe_etal._2024} in cell segmentation and classification tasks, model size, computational complexity, and processing times have increased. This limits the scalability and accessibility of these models. As we demonstrate, this may be mitigated using knowledge distillation. Studies in the field of natural language processing have demonstrated the efficacy of knowledge distillation in retaining the capabilities of the teacher model while achieving significant reductions in size and complexity \cite{Huangpu_Gao_2024,Sun_Yu_etal.}. 

We demonstrate the feasibility of knowledge distillation in digital pathology, specifically for cell segmentation and classification tasks. Moreover, we achieve this performance while also significantly reducing the parameter count. In addressing the challenge of knowledge transfer, we found that distillation from a transformer-based model to a smaller transformer is more straightforward than attempting to map transformer features to CNN blocks. In our experiments, using a CNN-based network as a student results in worse cell quantification performance due to the structural constraints of CNN feature space dimensions. 

Although our primary approach relies on a transformer-based student model that performs well, it can be further optimized to incorporate advantages from CNN architectures. For example, employing alternative techniques such as using ViT adapters \cite{Chen_Duan_etal._2023} or $1 \times 1$ convolutions to adjust feature map sizes may be beneficial for harnessing CNN advantages like enhanced local feature extraction. Moreover, if additional performance improvements are desired, the process can be further enhanced by applying supplementary knowledge distillation techniques, such as self-distillation \cite{Zhang_Song_etal._2019} or online distillation \cite{Houyon_Cioppa_etal._2023}.

Despite these promising results, further validation on independent datasets is necessary to fully understand the model's limitations. Underrepresented classes may pose challenges when addressing complex cases. Pathologists need to validate these models to adopt them in clinical settings. While the distilled models are smaller and more deployable, a technological gap persists because pathologists traditionally rely on established methods for inspecting WSIs and diagnosing diseases. Addressing the complexities involved in deploying models for inference and supporting pathologists in adopting new tools is essential for integrating these models into clinical workflows.

\section{Model integration with QuPath}
Digital pathology tools with graphical user interfaces are essential for visualizing and analyzing WSIs. To make our student model useful in clinical pathology workflows, it needs to be integrated into a tool that enables inspecting regions, creating annotations, and providing quantitative analyses of biomarkers. Therefore, we integrate the trained student model from the previous chapter into the QuPath open‑source platform \cite{Bankhead_Loughrey_etal._2017}. QuPath provides the required annotation, visualization, and analysis tools to interpret complex histological data, including workflows for cell segmentation, classification, and quantification (\hyperref[fig:fig7]{Figure 7}). 

\begin{figure}[h!]
    \centering
    \includegraphics[width=\textwidth]{images/Figure_7.pdf}
    \caption{Visualization of model-generated cell quantification annotations (left) and the corresponding unannotated slide (right) in QuPath}
    \label{fig:fig7}
\end{figure}

To identify the regions in a WSI critical for prognosticating tumor development, such as specific tumor areas or border regions without overlapping healthy tissue, the pathologist uses QuPath to outline these regions. Then, the pathologist initiates a cell segmentation and classification script through the QuPath interface for the selected regions. The resulting annotations and quantified cell information are then directly overlaid onto the WSI in the QuPath interface. Additional design and implementation details are in \hyperref[chap:S7]{Appendix S7}. 

Two common approaches for integrating deep learning models into QuPath are Java‑based native QuPath extensions \cite{Goldsborough_Philps_etal._2024} and the execution of RESTful API requests to a model server coupled with handling the response via an extension, as demonstrated in the application of cell segmentation models applied to immunofluorescence images \cite{Sugawara_2023}. While the community is actively working on these integration strategies, there is currently no universal solution that fully addresses all integration and performance requirements.

Extensions may offer better integration with QuPath, allowing slightly improved performance and more widespread usage of the built-in QuPath models, but they lack the flexibility to customize models and modify their behavior. For example, the newest version of QuPath includes models such as StarDist \cite{Weigert_Schmidt} and InstanSeg \cite{Goldsborough_Philps_etal._2024} that can perform cell segmentation. Both models pose limitations when applied to simultaneous cell segmentation and classification. StarDist performs well only on convex, round shapes by design, whereas some neoplastic, inflammatory, and connective cells exhibit complex and non-convex shapes. InstanSeg provides only semantic segmentation without assigning classes to the segmented cells.

%\hfill

In contrast, our approach offers an alternative integration strategy. It utilizes the paquo library to directly interact with QuPath’s internal application programming interface from within Python. This enables data exchange and processing without the need for intermediate conversion steps and provides greater control over model customization, retraining, and the incorporation of custom processing steps.

The integration of our custom model with QuPath underscores its potential to significantly enhance the diagnostic process by reducing the time burden on pathologists and enabling them to focus on more complex interpretative tasks using familiar software. Leveraging a tool that is already well-established among pathologists increases the likelihood of its adoption into daily clinical workflows. The quantitative data generated through the automated workflow is critical for both clinical decision-making and research, facilitating more accurate biomarker analysis, enabling robust statistical evaluations, and supporting hypothesis generation and testing. Additionally, by streamlining cell segmentation and classification, the tool enhances the scalability and reproducibility of pathological assessments, ultimately contributing to improved diagnostic accuracy and patient outcomes.

\section{Conclusion and future work}

In this study, we address critical challenges in digital pathology and tackle the usability and deployment issues of the developed models in standard computing environments without the need for high-performance computing systems. Our multi-faceted approach encompasses data refinement through cross-relabeling, leveraging foundation models for robust cell segmentation and classification, optimizing model performance via knowledge distillation, and integrating the optimized model into the QuPath software for practical application. This approach is used to construct a capable, versatile, and adjustable model for cell segmentation and classification, with enhanced performance and usability.

\begin{sloppypar}
While our approach shows potential in the field of computational pathology, certain limitations persist. 
For example, our implementation currently exhibits lower performance in detecting macrophages. 
This serves as an instance of the broader challenge of accurately identifying complex cell types. In order to address this issue, extending our approach to incorporate additional data sources, exploring alternative modeling approaches, and integrating other imaging modalities such as immunohistochemical staining may help improve detection accuracy. Moreover, although the distilled model reduces computational demands, integrating advanced deep learning models into clinical practice requires addressing technological gaps and potential resistance to adopting new tools within established diagnostic processes.
\end{sloppypar}

Future work could focus on several key areas to refine the proposed approach and facilitate its adoption in clinical environments. Enhancing the cell-relabeling process with additional datasets \cite{Graham_Jahanifar_etal._2021} could improve the representation of underrepresented cell types and enhance overall model performance. Also, incorporating additional data sources, such as multi-modal imaging or complementary staining methods, may address limitations related to cell type differentiation and class imbalance. Exploring other foundation models \cite{Vorontsov_Bozkurt_etal._2024,Zimmermann_Vorontsov_etal._2024} or introducing additional modalities \cite{Ding_Wagner_etal._2024,Vaidya_Zhang_etal._2025} may provide alternative architectures better suited to specific tasks or offer improved efficiency. Implementing more complex knowledge distillation techniques \cite{Houyon_Cioppa_etal._2023,Zhang_Song_etal._2019} could further optimize the model's performance and adaptability. Additionally, deeper integration with QuPath or other digital pathology software could provide pathologists more control over cell quantification analysis directly within the QuPath interface, thereby increasing accessibility and usability. Such enhancements would not only refine model performance but also ensure greater adaptability and scalability within various clinical environments. Finally, extensive validation of the model by pathologists and benchmarking against independent datasets are essential steps toward establishing the model's reliability and fostering confidence in its clinical utility.

\section*{Acknowledgments} 
This work was funded in part by the Research Council of Norway grant no. 309439 SFI Visual Intelligence, and the North Norwegian Health Authority grant no. HNF1521-20.

\bibliographystyle{IEEEtran}
\begin{sloppypar}
\begin{thebibliography}{99}

\bibitem{chaplot2020neural} Chaplot, Devendra Singh, et al. "Neural topological slam for visual navigation." Proceedings of the IEEE/CVF conference on computer vision and pattern recognition. 2020.

\bibitem{maksymets2021thda} Maksymets, Oleksandr, et al. "Thda: Treasure hunt data augmentation for semantic navigation." Proceedings of the IEEE/CVF International Conference on Computer Vision. 2021.

\bibitem{mezghan2022memory} Mezghan, Lina, et al. "Memory-augmented reinforcement learning for image-goal navigation." 2022 IEEE/RSJ International Conference on Intelligent Robots and Systems (IROS). IEEE, 2022.

\bibitem{al2022zero} Al-Halah, Ziad, Santhosh Kumar Ramakrishnan, and Kristen Grauman. "Zero experience required: Plug \& play modular transfer learning for semantic visual navigation." Proceedings of the IEEE/CVF Conference on Computer Vision and Pattern Recognition. 2022.

\bibitem{ye2021auxiliary} Ye, Joel, et al. "Auxiliary tasks and exploration enable objectgoal navigation." Proceedings of the IEEE/CVF international conference on computer vision. 2021.

\bibitem{chaplot2020object} Chaplot, Devendra Singh, et al. "Object goal navigation using goal-oriented semantic exploration." Advances in Neural Information Processing Systems 33 (2020)

\bibitem{ramakrishnan2022poni} Ramakrishnan, Santhosh Kumar, et al. "Poni: Potential functions for objectgoal navigation with interaction-free learning." Proceedings of the IEEE/CVF Conference on Computer Vision and Pattern Recognition. 2022.

\bibitem{ramrakhya2022habitat} Ramrakhya, Ram, et al. "Habitat-web: Learning embodied object-search strategies from human demonstrations at scale." Proceedings of the IEEE/CVF Conference on Computer Vision and Pattern Recognition. 2022.

\bibitem{mousavian2019visual} Mousavian, Arsalan, et al. "Visual representations for semantic target driven navigation." 2019 International Conference on Robotics and Automation (ICRA). IEEE, 2019.

\bibitem{dhariwal2021diffusion} Dhariwal, Prafulla, and Alexander Nichol. "Diffusion models beat gans on image synthesis." Advances in neural information processing systems 34 (2021)

\bibitem{ho2022classifier} Ho, Jonathan, and Tim Salimans. "Classifier-free diffusion guidance." arXiv preprint arXiv:2207.12598 (2022).

\bibitem{nichol2021glide} Nichol, Alex, et al. "Glide: Towards photorealistic image generation and editing with text-guided diffusion models." arXiv preprint arXiv:2112.10741 (2021)

\bibitem{brooks2023instructpix2pix} Brooks, Tim, Aleksander Holynski, and Alexei A. Efros. "Instructpix2pix: Learning to follow image editing instructions." Proceedings of the IEEE/CVF Conference on Computer Vision and Pattern Recognition. 2023.

\bibitem{fu2023guiding} Fu, Tsu-Jui, et al. "Guiding instruction-based image editing via multimodal large language models." arXiv preprint arXiv:2309.17102 (2023).

\bibitem{geng2024instructdiffusion} Geng, Zigang, et al. "Instructdiffusion: A generalist modeling interface for vision tasks." Proceedings of the IEEE/CVF Conference on Computer Vision and Pattern Recognition. 2024.

\bibitem{zhou2024minedreamer} Zhou, Enshen, et al. "Minedreamer: Learning to follow instructions via chain-of-imagination for simulated-world control." arXiv preprint arXiv:2403.12037 (2024).

\bibitem{zhou2023esc} Zhou, Kaiwen, et al. "Esc: Exploration with soft commonsense constraints for zero-shot object navigation." International Conference on Machine Learning. PMLR, 2023.

\bibitem{yu2023l3mvn} Yu, Bangguo, Hamidreza Kasaei, and Ming Cao. "L3mvn: Leveraging large language models for visual target navigation." 2023 IEEE/RSJ International Conference on Intelligent Robots and Systems (IROS). IEEE, 2023.

\bibitem{gadre2023cows} Gadre, Samir Yitzhak, et al. "Cows on pasture: Baselines and benchmarks for language-driven zero-shot object navigation." Proceedings of the IEEE/CVF Conference on Computer Vision and Pattern Recognition. 2023.

\bibitem{shah2023navigation} Shah, Dhruv, et al. "Navigation with large language models: Semantic guesswork as a heuristic for planning." Conference on Robot Learning. PMLR, 2023.

\bibitem{cai2024bridging} Cai, Wenzhe, et al. "Bridging zero-shot object navigation and foundation models through pixel-guided navigation skill." 2024 IEEE International Conference on Robotics and Automation (ICRA). IEEE, 2024.

\bibitem{yu2023co} Yu, Bangguo, Hamidreza Kasaei, and Ming Cao. "Co-NavGPT: Multi-robot cooperative visual semantic navigation using large language models." arXiv preprint arXiv:2310.07937 (2023).

\bibitem{wu2024voronav} Wu, Pengying, et al. "Voronav: Voronoi-based zero-shot object navigation with large language model." arXiv preprint arXiv:2401.02695 (2024).

\bibitem{qin2023mp5} Qin, Yiran, et al. "Mp5: A multi-modal open-ended embodied system in minecraft via active perception." arXiv preprint arXiv:2312.07472 (2023).

\bibitem{du2024learning} Du, Yilun, et al. "Learning universal policies via text-guided video generation." Advances in Neural Information Processing Systems 36 (2024).

\bibitem{ajay2024compositional} Ajay, Anurag, et al. "Compositional foundation models for hierarchical planning." Advances in Neural Information Processing Systems 36 (2024).

\bibitem{liang2024skilldiffuser} Liang, Zhixuan, et al. "Skilldiffuser: Interpretable hierarchical planning via skill abstractions in diffusion-based task execution." Proceedings of the IEEE/CVF Conference on Computer Vision and Pattern Recognition. 2024.

\bibitem{heusel2017gans} Heusel, Martin, et al. "Gans trained by a two time-scale update rule converge to a local nash equilibrium." Advances in neural information processing systems 30 (2017).

\bibitem{zhang2018unreasonable} Zhang, Richard, et al. "The unreasonable effectiveness of deep features as a perceptual metric." Proceedings of the IEEE conference on computer vision and pattern recognition. 2018.

\bibitem{brown2020language} Brown, Tom B. "Language models are few-shot learners." arXiv preprint arXiv:2005.14165 (2020).

\bibitem{podell2023sdxl} Podell, Dustin, et al. "Sdxl: Improving latent diffusion models for high-resolution image synthesis." arXiv preprint arXiv:2307.01952 (2023).

\bibitem{brohan2022rt} Brohan, Anthony, et al. "Rt-1: Robotics transformer for real-world control at scale." arXiv preprint arXiv:2212.06817 (2022).

\bibitem{brohan2023rt} Brohan, Anthony, et al. "Rt-2: Vision-language-action models transfer web knowledge to robotic control." arXiv preprint arXiv:2307.15818 (2023).

\bibitem{li2024manipllm} Li, Xiaoqi, et al. "Manipllm: Embodied multimodal large language model for object-centric robotic manipulation." Proceedings of the IEEE/CVF Conference on Computer Vision and Pattern Recognition. 2024.

\bibitem{shah2023vint} Shah, Dhruv, et al. "ViNT: A foundation model for visual navigation." arXiv preprint arXiv:2306.14846 (2023).

\bibitem{liu2024visual} Liu, Haotian, et al. "Visual instruction tuning." Advances in neural information processing systems 36 (2024).

\bibitem{hu2021lora} Hu, Edward J., et al. "Lora: Low-rank adaptation of large language models." arXiv preprint arXiv:2106.09685 (2021).

\bibitem{qin2023supfusion} Qin, Yiran, et al. "SupFusion: Supervised LiDAR-camera fusion for 3D object detection." Proceedings of the IEEE/CVF International Conference on Computer Vision. 2023.

\bibitem{qin2024worldsimbench} Qin, Yiran, et al. "Worldsimbench: Towards video generation models as world simulators." arXiv preprint arXiv:2410.18072 (2024).

\bibitem{yu2025gamefactory} Yu, Jiwen, et al. "GameFactory: Creating New Games with Generative Interactive Videos." arXiv preprint arXiv:2501.08325 (2025).

\bibitem{zhou2024code} Zhou, Enshen, et al. "Code-as-Monitor: Constraint-aware Visual Programming for Reactive and Proactive Robotic Failure Detection." arXiv preprint arXiv:2412.04455 (2024).

\bibitem{zhang2024ad} Zhang, Zaibin, et al. "AD-H: Autonomous Driving with Hierarchical Agents." arXiv preprint arXiv:2406.03474 (2024).

\bibitem{wang2024toward} Wang, Chaoqun, et al. "Toward Accurate Camera-based 3D Object Detection via Cascade Depth Estimation and Calibration." arXiv preprint arXiv:2402.04883 (2024).

\bibitem{huang2024story3d} Huang, Yuzhou, et al. "Story3d-agent: Exploring 3d storytelling visualization with large language models." arXiv preprint arXiv:2408.11801 (2024).

\bibitem{savinov2018semi} Savinov, Nikolay, Alexey Dosovitskiy, and Vladlen Koltun. "Semi-parametric topological memory for navigation." arXiv preprint arXiv:1803.00653 (2018).

\bibitem{majumdar2022zson} Majumdar, Arjun, et al. "Zson: Zero-shot object-goal navigation using multimodal goal embeddings." Advances in Neural Information Processing Systems 35 (2022): 32340-32352.

\bibitem{yadav2023offline} Yadav, Karmesh, et al. "Offline visual representation learning for embodied navigation." Workshop on Reincarnating Reinforcement Learning at ICLR 2023. 2023.

\bibitem{yadav2023ovrl} Yadav, Karmesh, et al. "Ovrl-v2: A simple state-of-art baseline for imagenav and objectnav." arXiv preprint arXiv:2303.07798 (2023).

\bibitem{sun2024fgprompt} Sun, Xinyu, et al. "FGPrompt: fine-grained goal prompting for image-goal navigation." Advances in Neural Information Processing Systems 36 (2024).

\bibitem{zhu2017target} Zhu, Yuke, et al. "Target-driven visual navigation in indoor scenes using deep reinforcement learning." 2017 IEEE international conference on robotics and automation (ICRA). IEEE, 2017.

\bibitem{koh2024generating} Koh, Jing Yu, Daniel Fried, and Russ R. Salakhutdinov. "Generating images with multimodal language models." Advances in Neural Information Processing Systems 36 (2024).

\bibitem{krantz2022instance} Krantz, Jacob, et al. "Instance-specific image goal navigation: Training embodied agents to find object instances." arXiv preprint arXiv:2211.15876 (2022).

\bibitem{schulman2017proximal} Schulman, John, et al. "Proximal policy optimization algorithms." arXiv preprint arXiv:1707.06347 (2017).

\bibitem{anderson2018evaluation} Anderson, Peter, et al. "On evaluation of embodied navigation agents." arXiv preprint arXiv:1807.06757 (2018).

\bibitem{lin2024navcot} Lin, Bingqian, et al. "NavCoT: Boosting LLM-Based Vision-and-Language Navigation via Learning Disentangled Reasoning." arXiv preprint arXiv:2403.07376 (2024).

\bibitem{NavGPT} Zhou, Gengze, Yicong Hong, and Qi Wu. "Navgpt: Explicit reasoning in vision-and-language navigation with large language models." Proceedings of the AAAI Conference on Artificial Intelligence.

\bibitem{hahn2021no} Hahn, Meera, et al. "No rl, no simulation: Learning to navigate without navigating." Advances in Neural Information Processing Systems 34 (2021): 26661-26673.

\bibitem{li2025t2isafety} Li, Lijun, et al. "T2ISafety: Benchmark for Assessing Fairness, Toxicity, and Privacy in Image Generation." arXiv preprint arXiv:2501.12612 (2025).

\bibitem{an2024agfsync} An, Jingkun, et al. "AGFSync: Leveraging AI-Generated Feedback for Preference Optimization in Text-to-Image Generation." arXiv preprint arXiv:2403.13352 (2024).


\end{thebibliography}
\end{sloppypar}

\clearpage
\beginsupplement
\section*{Appendix}
\renewcommand{\thesubsection}{S\arabic{subsection}}

\subsection{\label{chap:S1}PanNuke and MoNuSAC preprocessing}
The PanNuke dataset comprises a set of 7,901 RGB patches, each with dimensions of $256 \times 256$ pixels, which we set as the standard patch size for our analysis. In contrast, the MoNuSAC dataset encompasses 294 images of heterogeneous dimensions. To standardize the MoNuSAC images with our experiments, we implement a standardization protocol. Specifically, for images exceeding the dimensions of $256 \times 256$ pixels, we segment them into equal-sized patches and apply mirror padding to the remaining portions to avoid information loss at the peripherals. Patches with dimensions less than $128 \times 128$ pixels are excluded from the dataset due to the insufficient resolution to capture relevant cellular details. For patches where either dimension falls between 128 and 256 pixels, we employ upsampling to achieve the standard patch size. As a result, we obtain a total of 2,823 RGB patches derived from the MoNuSAC dataset for subsequent analysis. For additional details on the MoNuSAC data preparation process, refer to the source code \cite{Shvetsov_2025a}.
\clearpage

\subsection{\label{chap:S2}Data usage for the methodology}

\counterwithin{figure}{subsection}
\renewcommand{\thefigure}{S\arabic{subsection}}

\begin{figure}[h!]
    \centering
    \includegraphics[width=\textwidth, height=0.85\textheight, keepaspectratio]{images/A2.pdf}
    \caption{Overview of the methodology for cross-labeling, dataset refinement, and model comparison. (1) Cross-relabeling - training and testing cell classification models, (2) Cross-relabeling - using cell classification models to create refined dataset, (3) Fine-tuning and training models for comparison, (4) Student knowledge distillation with refined dataset}
    \label{fig:S2}
\end{figure}
\clearpage

\subsection{\label{chap:S3}Confusion matrices for classification models}
\counterwithin{figure}{subsection}
\renewcommand{\thefigure}{S\arabic{subsection}.\arabic{figure}}

\begin{figure}[h!]
    \centering
    \includegraphics[width=\textwidth, height=0.4\textheight, keepaspectratio]{images/A3_1.pdf}
    \caption{Confusion matrix for PanNuke trained model}
    \label{fig:S3.1}
\end{figure}

\begin{figure}[h!]
    \centering
    \includegraphics[width=\textwidth, height=0.4\textheight, keepaspectratio]{images/A3_2.pdf}
    \caption{Confusion matrix for MoNuSAC trained model}
    \label{fig:S3.2}
\end{figure}

\clearpage

\subsection{\label{chap:S4}Datasets cell counts}

\counterwithin{table}{subsection}
\renewcommand{\thetable}{S\arabic{subsection}}

\begin{table}[h!]
\renewcommand{\arraystretch}{2.0}
\centering
\caption{\label{tab:S4}Cell counts for PanNuke, MoNuSAC and refined datasets. Numbers in parentheses indicate preprocessed cell counts for cell classifier models training and testing.}
%\adjustbox{max width=\textwidth}{%
\begin{tabular}{|l|c|c|c|}
\hline
%\rowcolor{gray!30}
Cell type & PanNuke & MoNuSAC & Refined \\
\hline
Neoplastic & 77,403 (68,031) & - & 105,451 \\
\hline
Epithelial & 26,572 (23,207) & - & 29,926 \\
\hline
Epithelial (benign and malignant) & - & 31,402 & - \\
\hline
Inflammatory & 32,276 & - & - \\
\hline
Lymphocytes & - & 37,045 (33,104) & 65,275 \\
\hline
Neutrophils & - & 1,355 (1,252) & 3,833 \\
\hline
Macrophage & - & 1,842 (1,695) & 3,410 \\
\hline
Dead & 2,908 & - & 2,908 \\
\hline
Connective & 50,585 & - & 50,585 \\
\hline
\end{tabular}
%
%}
\end{table}



\clearpage

\subsection{\label{chap:S5}Definition of validation metrics}
\counterwithin{equation}{subsection}
\renewcommand{\theequation}{\arabic{equation}}

\subsubsection{\label{chap:S5.1}R\textsuperscript{2}}
The coefficient of determination, denoted as $R^2$, is a statistical measure that represents the proportion of variance in the dependent variable that is predictable from the independent variables. In the context of cell quantification in pathology, $R^2$ is used to assess how well the predicted quantities of different cell types in a patch align with the actual quantities observed in the ground truth data, with higher values representing more accurate quantification. $R^2$ is defined as
\begin{equation*}
R^2 = 1 - \frac{\sum_{i=1}^n (y_i - \hat{y}_i)^2}{\sum_{i=1}^n (y_i - \bar{y})^2},
\end{equation*}
where $y_i$ represents the actual number of cells of a specific type in the $i$-th image, $\hat{y}_i$ represents the predicted number of cells of that type in the $i$-th image, $\bar{y}$ is the mean of the actual numbers across all images, and $n$ is the total number of images in the dataset.

The $R^2$ metric has a range of $(-\infty, 1]$. An $R^2$ of 1 indicates perfect prediction, where all predicted values exactly match the actual values. An $R^2$ of 0 suggests that the model explains none of the variability of the response data around its mean. If $R^2$ is negative, it indicates that the model performs worse than a model that simply predicts the mean of the actual values for all observations.

\subsubsection{\label{chap:S5.2}PQ}
Panoptic Quality ($PQ$) is a comprehensive metric used to evaluate the performance of segmentation models in tasks that require both instance segmentation and classification. $PQ$ provides a single score that encapsulates both the detection accuracy (i.e., how many objects were correctly identified) and the segmentation quality (i.e., how accurately the objects' boundaries were delineated). This metric is particularly useful in multiclass scenarios where each pixel is classified into distinct categories, such as different cell types in pathology images.

$PQ$ is calculated as the product of two terms: Detection Quality ($DQ$) and Segmentation Quality ($SQ$). It can be expressed as
\begin{equation*}
PQ = DQ \cdot SQ,
\end{equation*}
where
\begin{equation*}
DQ = \frac{TP}{TP + 0.5\, FP + 0.5\, FN},
\end{equation*}
\begin{equation*}
SQ = \frac{\sum_{(p, g) \in \mathcal{M}} IoU(p, g)}{TP}.
\end{equation*}
In these formulas, $TP$ denotes the number of correctly matched instances between ground truth and prediction, $FP$ denotes the predicted instances that have no corresponding ground truth, $FN$ denotes the ground truth instances that were not detected, $IoU(p, g)$ is the Intersection over Union for a pair of matched instances $p$ (prediction) and $g$ (ground truth), and $\mathcal{M}$ is the set of matched pairs.

The $PQ$ metric is calculated for each class and is averaged across classes to provide a global performance measure.

The $PQ$ score has a range of $[0, 1.0]$, where a higher score indicates better performance in both detecting and segmenting the instances correctly. A $PQ$ of 1 signifies perfect identification and segmentation of all instances, whereas a $PQ$ of 0 indicates that no instances were correctly identified and segmented.

\clearpage

\subsection{\label{chap:S6}Segmentation and Detection quality metrics for teacher and student models}

\begin{table}[h!]
\renewcommand{\arraystretch}{2.0}
\centering
\caption{Segmentation and detection quality for student and teacher models (CI 95\%)}
\label{tab:S6}
%\adjustbox{max width=\textwidth}{%
\begin{tabular}{|l|c|c|}
\hline
%\rowcolor{gray!30}
Metric & Teacher & Student \\
\hline
$SQ_{neoplastic}$ & 0.819 (0.815--0.823) & 0.824 (0.819--0.828) \\
\hline
$SQ_{lymphocyte}$ & 0.795 (0.788--0.802) & 0.790 (0.783--0.796) \\
\hline
$SQ_{connective}$ & 0.770 (0.762--0.776) & 0.780 (0.772--0.786) \\
\hline
$SQ_{dead}$ & 0.659 (0.623--0.688) & 0.657 (0.624--0.695) \\
\hline
$SQ_{epithelial}$ & 0.780 (0.770--0.790) & 0.788 (0.779--0.797) \\
\hline
$SQ_{macrophage}$ & 0.788 (0.760--0.810) & 0.757 (0.730--0.783) \\
\hline
$SQ_{neutrofil}$ & 0.782 (0.761--0.801) & 0.775 (0.759--0.792) \\
\hline
$DQ_{neoplastic}$ & 0.706 (0.692--0.719) & 0.727 (0.712--0.741) \\
\hline
$DQ_{lymphocyte}$ & 0.675 (0.656--0.698) & 0.713 (0.691--0.734) \\
\hline
$DQ_{connective}$ & 0.566 (0.546--0.584) & 0.583 (0.565--0.602) \\
\hline
$DQ_{dead}$ & 0.410 (0.361--0.465) & 0.435 (0.306--0.561) \\
\hline
$DQ_{epithelial}$ & 0.668 (0.639--0.694) & 0.673 (0.644--0.702) \\
\hline
$DQ_{macrophage}$ & 0.657 (0.583--0.727) & 0.615 (0.531--0.703) \\
\hline
$DQ_{neutrofil}$ & 0.691 (0.625--0.753) & 0.729 (0.679--0.778) \\
\hline
\end{tabular}
%
%}
\end{table}

\clearpage

\subsection{\label{chap:S7}QuPath integration method}
We adopt an integration strategy leveraging the paquo \cite{Bayer_AG} library, a Python package that enables direct interaction with QuPath’s internal API, thereby facilitating seamless data exchange without intermediate conversion steps. The data processing pipeline (\hyperref[fig:S7]{Appendix Figure S7}) begins with the acquisition of WSIs and their associated annotations from QuPath, which are represented as Shapely \cite{Gillies_Wel_etal._2024} polygons. Utilizing paquo, we directly read, create, and modify these annotations and detections within a QuPath project in the Python environment. Images are then cropped using these polygons and processed by cell segmentation and classification models employing standard vision processing toolkits such as OpenCV, pyvips, and PyTorch. Additionally, QuPath employs Groovy scripts to initiate a Python process that starts the entire pipeline from QuPath graphical interface: fetching polygons, extracting images from them, and running deep learning model inference on the cropped images. 
The results are returned to QuPath, leveraging paquo's Python bindings to manipulate QuPath data while minimizing the computational overhead typically associated with cross-environment communication.

\counterwithin{figure}{subsection}
\renewcommand{\thefigure}{S\arabic{subsection}}

\begin{figure}[h!]
    \centering
    \includegraphics[width=\textwidth]{images/A7.pdf}
    \caption{QuPath integration workflow using Python environment}
    \label{fig:S7}
\end{figure}

Compared to traditional workflows that involve exporting annotations as GeoJSON, classifying them in Python, and reimporting them into QuPath, our approach offers several advantages. We eliminate the need to switch between programming languages, providing a cohesive and streamlined development process entirely within QuPath software and removing the necessity to use other tools. Meanwhile, we avoid storing annotations as intermediate JSON files unless required for external use or archiving. By conducting the entire inference and post-processing workflow within the Python environment, we leverage the power and flexibility of Python libraries for image processing and machine learning. This approach also enables adjustments to any set of labels and models, thereby improving its applicability.

%\hfill

The distilled model and QuPath integration code are packaged into a Docker container, enabling streamlined execution with the Docker engine. Detailed integration code and deployment instructions can be found in the GitHub repository \cite{Shvetsov_2025b}.

Despite these benefits, we acknowledge that the paquo library is a proof‑of‑concept project in its early development stage and has not been tested across all versions of QuPath.

\clearpage

\subsection{\label{chap:S8}Data and code availability statement}
All datasets, models, and code used in this study are publicly available and can be obtained from the repositories listed below. 
The PanNuke \cite{Gamper_Koohbanani_etal._2019} and MoNuSAC \cite{Verma_Kumar_etal._2021} datasets are publicly accessible, and download information along with detailed descriptions can be found in their respective articles. Preprocessing scripts for PanNuke and MoNuSAC data, as well as individual cell extraction scripts, are available on GitHub \cite{Shvetsov_2025a}. The H-Optimus foundation model used in our experiments can be downloaded from the HuggingFace repository \cite{hoptimus2024}, and model information is available on GitHub \cite{Saillard_Jenatton_etal._2024}. In addition, the integration code for QuPath and the distilled model packaged in a Docker container are provided in the repository \cite{Shvetsov_2025b}, and paquo Python library is available from the authors GitHub repository \cite{Bayer_AG}.
\clearpage

\end{document}

  \clearpage 
  \section{Proofs for Deterministic Safety Algorithms}\label{sec:proofs-det}

In this section we prove the correctness of our algorithm in \Cref{sec:warmup}.
We first prove that the \textsc{Round} procedure in \Cref{alg:aac-byz} satisfies the properties below, and then prove that \Cref{alg:skeleton} solves consensus under Byzantine faults.
\begin{description}
    \item[Strong Validity] If all correct processes propose the same value $v$ and a correct process returns a pair $\langle \textsc{Grade}, v' \rangle$, then $\textsc{Grade} = \textsc{Commit}$ and $v' = v$.
    \item[Consistency] If any correct process returns  $\langle \textsc{Commit}, v \rangle$, then no correct process returns $(\cdot, v' \ne v)$.
    \item[Termination] If all correct processes propose, then every correct process eventually returns.
\end{description}

In our proofs we rely on the following properties of Byzantine Reliable Broadcast (BRB)~\cite{book}:
\begin{description}
    \item[BRB-Validity] If a correct process $p$ broadcasts a message $m$, then every correct process eventually delivers $m$.
    \item[BRB-No-duplication] Every correct process delivers at most one message.
    \item[BRB-Integrity] If some correct process delivers a message $m$ with sender $p$ and process $p$ is correct, then $m$ was previously broadcast by $p$.
    \item[BRB-Consistency] If some correct process delivers a message $m$ and another correct process delivers a message $m$, then $m = m$.
    \item[BRB-Totality] If some message is delivered by any correct process, every correct process eventually delivers a message.
\end{description}

\begin{lemma}\label{lem:byz-round-validity}
    With Byzantine faults and $n=3f+1$, \Cref{alg:aac-byz} satisfies strong validity.
\end{lemma}
\begin{proof}
    If all correct processes propose the same value $v$, then at least $2f+1$ processes BRB-broadcast an \textsc{Init} message for $v$, and therefore at most $f$ processes BRB-broadcast an \textsc{Init} message for $1-v$. Thus $v$ will be the majority value among all \textsc{Init} messages delivered in phase 1, at all correct processes. Thus all correct processes will BRB-broadcast an \textsc{Echo} message for $v$. Furthermore, no Byzantine process can produce a valid \textsc{Echo} message for $1-v$, since to do so would require a set of $2f+1$ \textsc{Init} message with a majority value of $1-v$. This is impossible due to the properties of BRB and the fact that at most $f$ processes have BRB-broadcast an \textsc{Init} message for $1-v$.
    So, all valid \textsc{Echo} messages received by correct processes will be for $v$, so all correct processes will commit $v$ at line~\ref{line:fac-byz-commit}.
\end{proof}

\begin{lemma}
    With Byzantine faults and $n=3f+1$, \Cref{alg:aac-byz} satisfies consistency.
\end{lemma}
\begin{proof}
    If a correct process $p_1$ commits $v$ at line~\ref{line:fac-byz-commit}, then it must have delivered a set $S_1$ of $2f+1$ \textsc{Echo} messages for $v$ at line~\ref{line:fac-byz-wait-echo}. Take now another process $p_2$ and consider the set $S_2$ of $2f+1$ \textsc{Echo} messages it delivers at line~\ref{line:fac-byz-wait-echo}. By quorum intersection, $S_1$ and $S_2$ must intersect in at least $f+1$ messages. By the BRB-Consistency property, these $f+1$ messages must be identical at $p_1$ and $p_2$. Thus $p_2$ delivers at least $f+1$ \textsc{Echo} messages for $v$, which constitutes a majority of the $2f+1$ \textsc{Echo} messages it delivers overall. So if $p_2$ commits a value at line~\ref{line:fac-byz-commit}, then it must commit $v$, and if $p_2$ adopts a value at line~\ref{line:fac-byz-adopt}, then it must adopt $v$.
\end{proof}

\begin{lemma}
    With Byzantine faults and $n=3f+1$, \Cref{alg:aac-byz} satisfies termination.
\end{lemma}
\begin{proof}
    Follows immediately from the algorithm and from the properties of Byzantine Reliable Broadcast. Processes perform two phases; the only blocking step of each phase is waiting for $n-f$ messages (lines~\ref{line:fac-byz-wait-init} and~\ref{line:fac-byz-wait-echo}). This waiting eventually terminates, by the BRB-Validity property and the fact that there are at least $n-f$ correct processes.
\end{proof}

\begin{theorem}\label{thm:validity-byz}
    With Byzantine faults and $n=3f+1$, \Cref{alg:skeleton} satisfies strong validity.
\end{theorem}
\begin{proof}
    This follows from the strong validity property of the \textsc{Round} procedure (\Cref{lem:byz-round-validity}): if all correct processes propose $v$ to consensus, then all correct processes propose $v$ to \textsc{Round} in the first round, where by \Cref{lem:byz-round-validity}, all correct processes commit $v$, and thus all correct processes decide $v$ at line~\ref{line:skeleton-decide}.
\end{proof}

\begin{theorem}\label{thm:agreement-byz}
    With Byzantine faults and $n=3f+1$, \Cref{alg:skeleton} satisfies agreement.
\end{theorem}
\begin{proof}
    Let $r$ be the earliest round at which some process decides and let $p$ be a process that decides $v$ at round $r$. We will show that any other process $p'$ that decides, must decide $v$. 
    
    For $p$ to decide $v$ at round $r$, \textsc{Round} must output $(\textsc{Commit}, v)$ in that round. Thus, by the consistency property of \textsc{Round}, $\textsc{Round}(r,\cdot)$ must output $(\cdot, v)$ at all correct processes. If $\textsc{Round}(r,\cdot)$ outputs $(\textsc{Commit}, v)$ for $p'$, then $p'$ decides $v$ at round $r$ (line~\ref{line:skeleton-decide}). Otherwise, all correct processes input $v$ to $\textsc{Round}(r+1,\cdot)$, and by the strong validity property, all processes (including $p'$) will output $(\textsc{Commit}, v)$ and decide $v$ at round $r+1$.
\end{proof}

\begin{theorem}\label{thm:termination-byz}
    With Byzantine faults and $n=3f+1$, \Cref{alg:skeleton} satisfies termination.
\end{theorem}
\begin{proof}
     We can describe the execution of the protocol as a Markov chain with states $0,\ldots,n-f=2f+1$; the system is at state $i$ if $i$ correct processes have estimate ($est_i$ variable) equal to $0$ before invoking $\textsc{Round}$. Due to the strong validity property of the $\textsc{Round}$ procedure, states $0$ and $2f+1$ are absorbing states. There is a non-zero transition probability from each state (including $0$ and $2f+1$), to state $0$ or $2f+1$, or both (we show this below). Therefore, with probability $1$, the system will eventually reach one of the two absorbing states and remain there. Once this happens (i.e., once all processes have the same $est_i$ variable), the strong validity property of $\textsc{Round}$ ensures that all processes (who have not decided yet) will decide within a round.
    
    It only remains to show that there is a non-zero transition probability from each state to at least one of the absorbing states $0$ and $2f+1$. Consider a state $i \notin \{0,2f+1\}$; there is a schedule $S$ with non-zero probability which leads the system from $i$ to $0$ or $2f+1$ in one invocation of \textsc{Round}. We consider two cases:
    \begin{itemize}
        \item $i < f+1$: in this case $0$ is the minority value among correct processes. In schedule $S$, the $n-f$ \textsc{Init} messages delivered by correct process at line~\ref{line:fac-byz-wait-init} are all from correct processes. Thus, every correct process sees $i$ $0$s and $2f+1-i$ $1$; $1$ is the majority value, so all correct processes adopt it for phase 2. In phase 2, $S$ again ensures that the $n-f$ \textsc{Echo} messages delivered by correct process at line~\ref{line:fac-byz-wait-echo} are all from correct processes. Thus, all correct processes see $2f+1$ \textsc{Echo} messages for $1$ and commit $1$, bringing the system to state $0$.
        \item $i \geq f+1$: in this case $0$ is the majority value among correct processes. This case is symmetrical with respect to the previous one: the only difference is that all correct processes adopt $0$ (the majority value) at the end of phase 1, and all correct processes deliver $2f+1$ \textsc{Echo} messages for $0$, thus committing $0$ and bringing the system to state $2f+1$.
    \end{itemize}
\end{proof}
}

\end{document}
