\section{Preliminaries}
\label{sec:preliminaries}

We assume familiarity with basics of first-order logic\cite{enderton}.
%
$\VV$ is a countably infinite set of variables and $\AA$ is a first-order theory with signature $\Sigma$ and carrier $\CC$.
%
For each entity $e$, $\VV(e)$ is the set of variables that occur in $e$.
%
$\QF(\Sigma)$ denotes the set of all quantifier-free first-order formulas over $\Sigma$, and $\QF_\land(\Sigma)$ only contains conjunctions of $\Sigma$-literals.
%
$\top$ and $\bot$ stand for ``true'' and ``false'', respectively.

Given $\psi \in \QF(\Sigma)$ with $\VV(\psi) = \vec{y}$, we say that $\psi$ is $\AA$-\emph{valid} (written $\models_\AA \psi$) if every model of $\AA$ satisfies the universal closure $\forall \vec{y}.\ \psi$ of $\psi$.
%
A partial function $\sigma: \VV \partial \CC$ is called a \emph{valuation}.
%
If $\VV(\psi) \subseteq \dom(\sigma)$ and $\models_\AA \sigma(\psi)$, then $\sigma$ is an $\AA$-\emph{model} of $\psi$ (written $\sigma \models_\AA \psi$).
%
Here, $\sigma(\psi)$ results from $\psi$ by instantiating all variables according to $\sigma$.
%
If $\psi$ has an $\AA$-model, then $\psi$ is $\AA$-\emph{satisfiable}.
%
If $\sigma(\psi)$ is $\AA$-satisfiable (but not necessarily $\VV(\psi) \subseteq \dom(\sigma)$), then we say that $\psi$ is $\AA$-\emph{consistent} with $\sigma$.
%
We write $\psi \models_\AA \psi'$ for $\models_\AA \psi \implies \psi'$, and $\psi \equiv_\AA \psi'$ means $\models_\AA \psi \iff \psi'$.
%
In the sequel, we omit the subscript $\AA$, and we just say ``valid'', ``model'', ``satisfiable'', and ``consistent''.
%
We assume that $\AA$ is complete (i.e., $\models \psi$ or $\models \neg \psi$ holds for every closed formula over $\Sigma$) and that $\AA$ has an effective quantifier elimination procedure (i.e., quantifier elimination is computable).

We write $\vec{x}$ for sequences and $x_i$ is the $i^{th}$ element of $\vec{x}$, where $x_1$ denotes the first element.
%
We use ``$\concat$'' for concatenation of sequences, where we identify sequences of length $1$ with their elements, so, e.g., $x\concat\vec{x} = [x]\concat\vec{x}$.

Let $d \in \NN$ be fixed, and let $\vec{x},\vec{x}' \in \VV^d$ be disjoint vectors of pairwise different variables, called the \emph{pre-} and \emph{post-variables}.
%
All other variables are \emph{extra variables}.
%
Each $\tau \in \QF(\Sigma)$ induces a \emph{transition relation} $\to_\tau$ on \emph{states}, i.e., elements of $\CC^d$, where $\vec{v} \to_\tau \vec{v}'$ iff $\tau[\vec{x}/\vec{v},\vec{x}'/\vec{v}']$ is satisfiable.
%
Here, $[\vec{x}/\vec{v},\vec{x}'/\vec{v}']$ maps
$x^{(\prime)}_i$ to $v^{(\prime)}_i$.

We call $\tau \in \QF(\Sigma)$ a \emph{relational formula} if we are interested in $\tau$'s induced transition relation.
%
\emph{Transitions} are conjunctive relational formulas without extra variables (i.e., conjunctions of literals over pre- and post-variables).
%
We sometimes identify $\tau$ with $\to_\tau$, so we may call $\tau$ a relation.

A \emph{$\tau$-run} is a sequence $\vec{v}_1 \to_\tau \ldots \to_\tau \vec{v}_k$.
%
A \emph{safety problem} $\TT$ is a triple $(\psi_{\init}, \tau,\psi_{\err}) \in \QF(\Sigma) \times \QF(\Sigma) \times \QF(\Sigma)$ where $\VV(\psi_{\init}) \cup \VV(\psi_{\err}) \subseteq \vec{x}$.
%
 It is \emph{unsafe} if there\paper{ \vspace*{-.2cm}\pagebreak} are $\vec{v},\vec{v}' \in \CC^d$ such that  $[\vec{x} / \vec{v}] \models \psi_\init$, $\vec{v} \to^*_\tau \vec{v}'$, and $[\vec{x} / \vec{v}'] \models \psi_\err$.

Throughout the paper, we use $c,d,e,k,\ell,s$ for integer constants (where $d$ always
denotes the size of $\vec{x}$, and $s$ and $\ell$ always denote the start and length of a
loop as in \Cref{alg1:loop} of \Cref{alg:overview}), $\vec{v}$ for states, $w,x,y$ for
variables, $\tau,\pi$ for relational formulas, $\sigma,\theta$ for valuations, and $\mu$
for variable renamings.
 
