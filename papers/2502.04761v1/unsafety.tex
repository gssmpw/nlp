\section{Proving Unsafety}\label{sec:Unsafety}

We now explain how to adapt \Cref{alg} for also proving unsafety.
%
Assume that the satisfiability check in \Cref{alg:err2} is successful.
%
Then an error state is reachable from an initial state via the current trace.
%
However, the trace may contain learned transitions, so this does not imply unsafety of $\TT$.
%
The idea for proving unsafety is to replace learned transitions with \emph{accelerated transitions} that result from applying \emph{acceleration techniques}.
%
\begin{definition}[Acceleration]
  A function $\accel: \QF(\Sigma) \to \QF(\Sigma)$ is called an \emph{acceleration technique} if ${\to_{\accel(\pi)}} \subseteq {\to^+_{\pi}}$ for all relational formulas $\pi$.
\end{definition}
%
So in contrast to TRL's learned transitions, accelerated transitions under-ap\-prox\-i\-mate transitive closures, and hence they are suitable for proving unsafety.
%
For arithmetical theories, acceleration techniques are well studied \cite{kroening13,bozga10,acceleration-calculus} (our implementation uses the technique from \cite{acceleration-calculus}).

For any vector of transition formulas $\vec{\rho} = [\rho_1,\ldots,\rho_k]$, let $\compose(\vec{\rho})$ be a transition formula such that ${\to_{\compose(\vec{\rho})}}$ is the composition of the relations ${\to_{\rho_1}}, \ldots, {\to_{\rho_k}}$.
%
Moreover, for a learned transition $\pi$, we say that $\vec{\tau}_\Loop$ \emph{induced} $\pi$ if we had $\vec{\tau}_\Loop = [\tau_s,\ldots,\tau_{s+\ell-1}]$ in \Cref{alg:loop} when $\pi$ was learned.
%
Let $\succ$ be the total ordering on the elements of the vector $\vec{\pi}$ from \Cref{alg} with $\pi_i \succ \pi_j$ iff $i > j$, i.e., the input formula $\tau$ is minimal w.r.t.\ $\succ$, and a learned relation is smaller than all relations that were learned later.
%
Then for vectors of transitions from $\{\mbip(\pi,\sigma) \mid \pi \in \vec{\pi}, \sigma \models \pi\}$, we define the function $\underapprox$ (which yields an \underline{u}nder-\underline{a}pproximation) as follows:
\begin{align*}
  &\underapprox([\eta_1,\ldots,\eta_k]) \Def \compose([\underapprox(\eta_1),\ldots,\underapprox(\eta_k)])\\
  &\underapprox(\eta) \Def
  \begin{cases}
    \eta & \text{if } \eta \models \tau \\
    \accel(\underapprox(\vec{\tau}_\Loop)) & \text{if } \eta \not\models \tau \text{ and}\\
& \phantom{\text{if }} \text{the loop } \vec{\tau}_\Loop \text{ induced }  \underset{\succ}{\min}\{\pi \in \vec{\pi} \mid \eta \models \pi\}
  \end{cases}
\end{align*}
%
Note that $\underapprox$ is well defined,
 as the following holds for all $\eta' \in \vec{\tau}_\Loop$ in the case $\eta \not\models
 \tau$:
\[
  \min_\succ\{\pi \in \vec{\pi} \mid \eta \models \pi\} \succ \min_\succ\{\pi \in \vec{\pi} \mid \eta' \models \pi\}
\]
%
The reason is that
 $\min_\succ\{\pi \in \vec{\pi} \mid \eta \models \pi\}$
is induced by $\vec{\tau}_\Loop$, and thus the elements of $\vec{\tau}_\Loop$ are
conjunctive variable projections of $\tau$, or of relations that were learned before
$\min_\succ\{\pi \in \vec{\pi} \mid \eta \models \pi\}$.
Hence, when defining $\underapprox(\eta)$, the ``recursive call''
$\underapprox(\vec{\tau}_\Loop)$ only refers to formulas with smaller index in $\vec{\pi}$, i.e., the
recursion ``terminates''.



So $\underapprox$ leaves original transitions unchanged.
%
For learned transitions
$\eta$, it first
computes an under-approximation
of the loop $\vec{\tau}_\Loop$ that induced $\eta$, and then it applies acceleration,
resulting in an under-approximation of the transitive closure.
%
The reason for applying $\underapprox$ before acceleration is that $\vec{\tau}_\Loop$ may again contain learned transitions, which have to be under-approximated first.

To improve \Cref{alg},
instead of returning $\unknown$ in \Cref{alg:err2}, now we first obtain a model $\sigma$ from the SMT solver.
%
Then we compute the current trace $\vec{\tau} = \trace_{b-1}(\sigma,\vec{\pi}) = [\tau_1,\ldots,\tau_{b-1}]$ (note that the trace has length $b-1$, as $b$ was incremented in \Cref{alg:err1}).
%
Next, computing $\pi_\underapprox \Def \underapprox(\vec{\tau})$ yields an under-approximation of the states that are reachable with the current trace, but more importantly, $\pi_\underapprox$ also under-approximates $\to^+_\tau$.
%
The reason is that $\pi_\underapprox$ is constructed from original transitions
by applying $\compose$ (which is exact) and $\accel$ (which yields under-approximations).
%
Hence, we return $\unsafe$ if there is an initial state $\vec{v}$ and an error state $\vec{v}'$ with $\vec{v} \to_{\pi_\underapprox} \vec{v}'$, i.e., if $\psi_\init \land \pi_\underapprox \land \psi_\err[\vec{x}/\vec{x}']$ is satisfiable.

\begin{example}[Proving Unsafety]
  Consider the relation
  \[
    \underbrace{y > 0 \land x' \doteq x + 1 \land y' \doteq y - 1 \land z' \doteq z}_{\tau_>} {} \lor {} \underbrace{y \doteq 0 \land x' \doteq x \land y' \doteq z \land z' \doteq z}_{\tau_=} \tag{$\tau$}
  \]
  with the initial states given by $\psi_\init \Def x \leq 0$ and the error states given
  by
  $\psi_\err \Def x \geq 1000$.
  %
  Assume that TRL obtains the trace $[\tau_>]$ and learns the relation
  \begin{equation}
    \label{eq:unsafe-learned1}
    n > 0 \land y > 0 \land x' \doteq x + n \land y' \doteq y - n \land z' \doteq z \land y' \geq 0. \tag{$\tau_>^+$}
  \end{equation}
  Next, assume that TRL obtains the trace $[\tau_=,\tau_>^+]$ and learns the relation
  \begin{equation}
    \label{eq:unsafe-learned1}
    y \doteq 0 \land z > 0 \land x' > x \land z > y' \land y' \geq 0 \land z' \doteq z. \tag{$\hat{\tau}^+$}
  \end{equation}
  Note that the transitive closure of the loop $[\tau_=,\tau_>^+]$ cannot be expressed precisely with linear arithmetic, and hence we only obtain the inequations above for $x'$ and $y'$.
  %
  Then an error state is reachable with the trace $[\hat{\tau}^+]$ and we\report{
    \pagebreak[3]} get:
  \begin{align*}
    & \underapprox([\hat{\tau}^+]) \\
    {} = {} & \accel(\underapprox([\tau_=,\tau_>^+])) \tag{as $[\tau_=,\tau_>^+]$ induced $\hat{\tau}^+$} \\
    {} = {} & \accel(\compose(\underapprox(\tau_=),\underapprox(\tau_>^+))) \\
    {} = {} & \accel(\compose(\tau_=,\accel(\underapprox([\tau_>])))) \tag{as $\tau_= \models \tau$ and $[\tau_>]$ induced $\tau_>^+$} \\
    {} = {} & \accel(\compose(\tau_=,\accel(\tau_>))) \tag{as $\tau_> \models \tau$}\\
    {} = {} & \accel(\compose(\tau_=,\tau^+_>)) \tag{as ${\to_{\tau_>}^+} = {\to_{\tau^+_>}}$} \\
    {} = {} & \accel(y \doteq 0 \land n > 0 \land z > 0 \land x' \doteq x + n \land y' \doteq z - n \land z' \doteq z \land y' \geq 0) \\
    {} = {} & y \doteq 0 \land m > 0 \land z > 0 \land x' \doteq x + m \cdot z \land y' \doteq 0 \land z' \doteq z \tag{$\check{\tau}^+$}
  \end{align*}
  For the last step, note that $\check{\tau}^+$ precisely characterizes the transitive closure of $\compose(\tau_=,\tau^+_>)$ for the case $n \doteq z$, and hence it is a valid under-approximation.
  %
  Then a model like $[x/y/0, z/1, m/1000, \; 
 x'/1000,  y'/0,z'/1]$ satisfies $\psi_\init \land \check{\tau}^+ \land \psi_\err[\vec{x}/\vec{x}'] = x \leq 0 \land \check{\tau}^+ \land x' \geq 1000$, which proves unsafety.
\end{example}
