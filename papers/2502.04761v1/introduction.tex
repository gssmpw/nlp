\section{Introduction}
\label{sec:intro}
\label{sec:overview}

We consider relations defined by SMT formulas over two disjoint vectors of \emph{pre-} and \emph{post-variables} $\vec{x}$ and $\vec{x}'$.
%
Such \emph{relational formulas} can easily represent \emph{transition systems}
(TSs), linear \emph{Constrained Horn Clauses} (CHCs), and \emph{control-flow automata}
(CFAs).\footnote{To this end, it suffices to introduce one additional variable that represents the control-flow location (for TSs and CFAs) or the predicate (for linear CHCs).}
Thus, they subsume many popular intermediate representations used for verification of systems specified in more expressive languages.

In contrast to, e.g., source code, relational formulas are unstructured.
%
However, source code may be unstructured, too (e.g., due to {\tt goto}s), so being independent from the structure of the input makes our approach broadly applicable.

\begin{example}[Running Example]
  \label{ex:ex1}
  Let $\tau \Def \tau_{\inc} \lor \tau_{\dec}$ with:
  \begin{align*}
    w \doteq 0 \land x' \doteq x + 1 \land y' \doteq y + 1 \tag{$\tau_\inc$} \\
    w' \doteq w \land w \doteq 1 \land x' \doteq x - 1 \land y' \doteq y - 1 \tag{$\tau_\dec$}
  \end{align*}
  We use ``$\doteq$'' for equality in relational formulas.
  %
  The formula $\tau$ defines a relation $\to_{\tau}$ on $\ZZ \times \ZZ \times \ZZ$ by relating the unprimed pre-variables with the primed post-variables.
  %
  So for all $v_w,v_x,v_y, v'_w,v'_x, v'_y \in \ZZ$, we have $(v_w,v_x,v_y) \to_{\tau} (v'_w,v'_x, v'_y)$ iff $[w/v_w,x/v_x,y/v_y,w'/v'_w,x'/v'_x,y'/v'_y]$ is a model of $\tau$.
  %
  Let the set of \emph{error states} be given by $\psi_{\err} \Def w \doteq 1 \land x \leq 0 \land y > 0$.
\end{example}
%
With the \emph{initial states} $\psi_{\init} \Def x \doteq 0 \land y \doteq 0$ this example\footnote{\href{https://github.com/chc-comp/extra-small-lia/blob/master/bouncy_symmetry.smt2}{\tt extra-small-lia/bouncy\_symmetry} from the \href{https://github.com/orgs/chc-comp/repositories}{CHC competition}}
is challenging for existing model checkers:
%
Neither the default configuration of \tool{Z3/Spacer} \cite{z3,spacer}, nor \tool{Golem}'s \cite{golem} implementation of Spacer, LAWI \cite{lawi}, IMC \cite{imc}, TPA \cite{tpa-multiloop},
PDKIND \cite{pdkind}, or predicate abstraction \cite{predicate_abstraction} can prove its safety.
%
In contrast, all of these techniques\paper{\pagebreak[3]} can prove safety with the more general initial states $\psi_{\init} \Def x \doteq y$.
%
As all of them are based on \emph{interpolation}, the reason might be that the inductive invariant $x \doteq y$ is now a subterm of $\psi_{\init}$, so it is likely to occur in interpolants.
%
However, this explanation is insufficient, as all techniques fail again for $\psi_{\init} \Def x \doteq y \land y \doteq 0$.

This illustrates a well-known issue of interpolation-based verification techniques:
%
They are highly sensitive to minor changes of the input or the underlying interpolating SMT solver (e.g., \cite[p.~102]{gspacer}).
%
So while they can often solve difficult problems quickly, they sometimes fail for easy examples like the one above.

In another line of research, \emph{recurrence analysis} has been used for software verification \cite{kincaid15,kincaid17}.
%
Here, the idea is to extract recurrence equations from loops and solve them to summarize the effect of arbitrarily many iterations.
%
While recurrence analysis often yields very precise summaries for loops without branching, these summaries are conjunctive.
%
However, for loops with branching, disjunctive summaries are often important to distinguish the branches.

In this work, we embed recurrence analysis into \emph{bounded model checking}
(BMC) \cite{bmc2}, resulting in a robust, competitive model checking algorithm.
%
To find disjunctive summaries, we exploit the structure of the relational
formula to partition the\linebreak  state
space on the fly via \emph{model based projection} \cite{spacer} (which
approximates quanti\-fier elimination), and a variation of recurrence analysis
called \emph{transitive projection}.\linebreak
\indent
Our approach is inspired by ABMC \cite{abmc}, which combines BMC with \emph{acceleration} \cite{acceleration-calculus,octagonsP,bozga10}.
%
ABMC uses \emph{blocking clauses} to speed up the search for counterexamples, but they turned out to be of little use for this purpose.
%
Instead, they enable ABMC to prove safety of certain challenging benchmarks. This
moti\-vated the development of our novel dedicated algorithm for proving safety via
BMC and blocking clauses.
%
See \Cref{sec:related} for a detailed comparison with ABMC.

% \subsubsection{Outline}

% We start with a high level overview of our approach in \Cref{sec:overview}.
% %
% Then, after introducing preliminaries in \Cref{sec:preliminaries}, we present our new algorithm TRL in \Cref{sec:til}.
% %
% As TRL builds upon \emph{transitive projections}, we show how to implement such a
% projection for quantifier-free linear integer arithmetic in \Cref{sec:rec}.
% In \Cref{sec:Unsafety}, we explain how to adapt TRL in order to prove unsafety.
% %
% Finally, we discuss related work in \Cref{sec:related}, and we evaluate our approach empirically in \Cref{sec:experiments}.
% %
% All proofs can be found in \Cref{sec:proofs}.


