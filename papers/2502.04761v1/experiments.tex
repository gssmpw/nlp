\section{Experiments and Conclusion}
\label{sec:experiments}

We implemented TRL in our tool \tool{LoAT} \cite{loat}.
%
Our implementation uses the SMT solvers \tool{Yices} \cite{yices} and \tool{Z3} \cite{z3}, and currently, it is restricted to linear integer arithmetic.
%
We evaluated our approach on the examples from the category LIA-Lin (linear CHCs with linear integer arithmetic) from the CHC competitions~2023 and 2024 \cite{CHC-COMP} (excluding duplicates), resulting in a collection of 625 problems from numerous applications like verification of \href{https://github.com/chc-comp/hcai-bench}{\tool{C}}, \href{https://github.com/chc-comp/rust-horn}{\tool{Rust}}, \href{https://github.com/chc-comp/jayhorn-benchmarks}{\tool{Java}}, and \href{https://github.com/chc-comp/hopv}{higher-order} programs, and \href{https://github.com/mattulbrich/llreve}{regression verification of \tool{LLVM} programs},
%
see \cite{chc-comp23,chc-comp24} for details.
%
By using CHCs as input format, our approach can be used by any CHC-based tool like \tool{Korn} \cite{korn} and \tool{SeaHorn} \cite{seahorn} for \pl{C} and \CXX{} programs, \tool{JayHorn} for \pl{Java} programs \cite{jayhorn}, \tool{HornDroid} for \pl{Android} \cite{horndroid}, \tool{RustHorn} for \pl{Rust} programs \cite{rusthorn}, and \tool{SmartACE} \cite{smartACE} and \tool{SolCMC} \cite{solcmc} for \pl{Solidity}.

We compared TRL with the techniques of leading CHC solvers.
%
More precisely, we evaluated the following configurations:
\begin{description}
  \item[\tool{LoAT}] We used \tool{LoAT}'s implementations of TRL (\tool{LoAT TRL}), ABMC \cite{abmc}
    (\tool{LoAT ABMC}), and \emph{$k$-induction} \cite{kind} (\tool{LoAT KIND})
  \item[\tool{Z3} \cite{z3}] We used \tool{Z3} 4.13.3, where we evaluated its implementations of the Spacer \cite{spacer} (\tool{Z3 Spacer}) and GSpacer \cite{gspacer} (\tool{Z3 GSpacer}) algorithms, and of BMC (\tool{Z3 BMC}).
  \item[\tool{Golem} \cite{golem}] We used \tool{Golem} 0.6.2, where we evaluated its implementations of \emph{transition power abstraction}~\cite{tpa-multiloop} (\tool{Golem TPA}), \emph{interpolation based model checking} \cite{imc} (\tool{Golem IMC}), \emph{lazy abstraction with interpolants} \cite{lawi} (\tool{Golem LAWI}), \emph{predicate abstraction} \cite{predicate_abstraction} with \emph{CEGAR} \cite{cegar} (\tool{Golem PA}), \emph{property directed $k$-induction} \cite{pdkind} (\tool{Golem PDKIND}), Spacer (\tool{Golem Spacer}), and BMC (\tool{Golem BMC}).
        %
        Note that \tool{Golem}'s implementation of $k$-induction only applies to linear CHC problems with a specific structure (just one fact, one recursive rule, and one query), so we excluded it from our evaluation.
  \item[\tool{Eldarica} \cite{eldarica}] We used the default configuration of \tool{Eldarica} 2.1, which is based on predicate abstraction and CEGAR.
\end{description}
%
We ran our experiments on \href{https://help.itc.rwth-aachen.de/service/rhr4fjjutttf/article/fbd107191cf14c4b8307f44f545cf68a/}{CLAIX-2023-HPC nodes} of the \paper{\href{https://help.itc.rwth-aachen.de/service/rhr4fjjutttf/}{RWTH Uni\-ver\-si\-ty High Performance Computing Cluster}}\report{RWTH Uni\-ver\-si\-ty High Performance Computing Cluster\footnote{\url{https://help.itc.rwth-aachen.de/service/rhr4fjjutttf}}} with a memory limit of 10560 MiB ($\approx$~11GB) and a timeout of 60~s per example.

\begin{figure}[th!]
  \begin{minipage}{0.53\textwidth}
    \begin{tabular}{|c||c||c|c||c|c|}
      \hline                       & \multirow{2}{*}{\checkmark} & \multicolumn{2}{c||}{$\safe$} & \multicolumn{2}{c|}{$\unsafe$} \\
      \hhline{~~----}              &                 & \checkmark                    & {\bf !}   & \checkmark      & {\bf !} \\
      \hline\hline \tool{LoAT TRL} & \underline{372} & 259                           & 18(23)    & 113             & 0(10)   \\
      \hline \tool{Z3 GSpacer}     & 359             & \underline{274}               & 20(20)    & 85              & 2(2)    \\
      \hline \tool{Z3 Spacer}      & 317             & 217                           & --        & 100             & --      \\
      \hline \tool{Golem Spacer}   & 312             & 209                           & --        & 103             & --      \\
      \hline \tool{Golem PDKIND}   & 292             & 207                           & 1(1)      & 85              & 0(0)    \\
      \hline \tool{Golem IMC}      & 289             & 195                           & 3(3)      & 94              & 0(0)    \\
      \hline \tool{Eldarica}       & 275             & 199                           & 5(5)      & 76              & 0(0)    \\
      \hline \tool{Golem TPA}      & 273             & 163                           & 0(1)      & 110             & 4(5)    \\
      \hline \tool{Golem LAWI}     & 271             & 167                           & 0(0)      & 104             & 0(0)    \\
      \hline \tool{LoAT ABMC}      & 264             & 140                           & 1(--)     & \underline{124} & 1(--)   \\
      \hline \tool{Golem PA}       & 239             & 152                           & --        & 87              & --      \\
      \hline \tool{LoAT KIND}      & 239             & 132                           & 0(0)      & 107             & 1(1)    \\
      \hline \tool{Golem BMC}      & 141             & 34                            & 0(0)      & 107             & 0(0)    \\
      \hline \tool{Z3 BMC}         & 138             & 32                            & --        & 106             & --      \\
      \hline
    \end{tabular}
    \label{tab}
  \end{minipage}
  %
  \hspace{-1em}
  \begin{minipage}{0.48\textwidth}
    \begin{tikzpicture}[scale=0.775]
      \begin{axis}[
        legend pos=south east,
        ylabel=solved instances,
        y label style={at={(axis description cs:-0.04,.5)},anchor=south},
        y tick label style={rotate=90},
        xticklabel={$\pgfmathprintnumber{\tick}$s},
        ymin=120]
        \addplot[color=black,solid,thick] table[col sep=comma,header=false,x index=0,y index=1] {til.csv}; \addlegendentry{\tool{\scriptsize LoAT TRL}}
        \addplot[color=violet,densely dashed,thick] table[col sep=comma,header=false,x index=0,y index=1] {gspacer.csv}; \addlegendentry{\tool{\scriptsize Z3 GSpacer}}
        \addplot[color=blue,densely dotted,thick] table[col sep=comma,header=false,x index=0,y index=1] {z3spacer.csv}; \addlegendentry{\tool{\scriptsize Z3 Spacer}}
        \addplot[color=red,dashed,thick] table[col sep=comma,header=false,x index=0,y index=1] {spacer.csv}; \addlegendentry{\tool{\scriptsize Golem Spacer}}
        \addplot[color=lightgray,dotted,thick] table[col sep=comma,header=false,x index=0,y index=1] {pdkind.csv}; \addlegendentry{\tool{\scriptsize Golem PDKIND}}
        \addplot[color=green,loosely dashed,thick] table[col sep=comma,header=false,x index=0,y index=1] {imc.csv}; \addlegendentry{\tool{\scriptsize Golem IMC}}
        \addplot[color=purple,loosely dotted,thick] table[col sep=comma,header=false,x index=0,y index=1] {eld.csv}; \addlegendentry{\tool{\scriptsize Eldarica}}
      \end{axis}
    \end{tikzpicture}
  \end{minipage}
\end{figure}

\noindent
The results can be seen in the table \vpageref{tab}.
%
The columns with {\bf !} show the num\-ber of unique proofs, i.e., the number of examples that could only be solved by the corresponding configuration.
%
Such a comparison only makes sense if just one implementation of each algorithm is considered.
%
For instance, GSpacer is an improved version of Spacer, so \tool{Z3 GSpacer}, \tool{Z3 Spacer}, and \tool{Golem Spacer} work well on the same class of examples.
%
Hence, if all of them were considered, they would find very few unique proofs.
%
The same is true for \tool{Eldarica} and \tool{Golem PA} as well as \tool{Golem BMC} and \tool{Z3 BMC}.
%
Thus, for {\bf !} we disregarded \tool{Z3 Spacer}, \tool{Golem Spacer}, \tool{Golem PA}, and \tool{Z3 BMC}.

The numbers in parentheses in the columns marked with {\bf !} show the number of unique proofs when also disregarding \tool{LoAT ABMC}.
%
We included this number since TRL was inspired by ABMC.
%
So analogously to GSpacer and Spacer, one could consider TRL to be an improved version of ABMC for proving safety.

The table shows that TRL is highly competitive:
%
Overall, it solves the most instances.
%
Even more importantly, it finds many unique proofs, i.e., it is orthogonal to existing model checking algorithms, and hence it improves the state of the art.

Regarding safe instances, TRL is only outperformed by GSpacer.
%
Regarding unsafe instances, TRL is only outperformed by ABMC, which has specifically been
developed for finding (deep) counterexamples, whereas TRL's main purpose is proving
safety.\comment[NONE]{JG As mentioned in \Cref{ABMCComparison}, it does not become clear what the conceptual
  difference between the ABMC algorithm and the TRL algorithm from \Cref{sec:Unsafety} is, which
  enables ABMC to specifically find deeper counterexamples than TRL.

  FF It does not find ``deeper'' counterexamples, it just finds more counterexamples.
  %
  Anyway, I extended the comparison in \Cref{sec:related}.
JG ok}

The plot on the right shows how many instances were solved within 60~s, where we only include the seven best configurations for readability.
%
It shows that TRL is highly competitive, not only in terms of solved examples, but also in terms of runtime.

Our implementation is open-source and available on Github.
%
For the sources, a pre-compiled binary, and more information on our evaluation, we refer to \cite{website}.

\subsubsection*{Conclusion}

We presented \emph{Transitive Relation Learning (TRL)}, a powerful model checking algorithm for infinite state systems.
%
Instead of searching for invariants, TRL adds transitive relations to the analyzed system until its \emph{diameter} (the number of steps that is required to cover all reachable states) becomes finite, which facilitates a safety proof.
%
As it does not search for invariants, TRL does not need interpolation, which is in contrast to most other state-of-the-art techniques.
%
Nevertheless, our evaluation shows that TRL is highly competitive with interpolation-based approaches.
%
Moreover, not using interpolation allows us to avoid the well-known fragility of interpolation-based approaches.
%
Finally, integrating \emph{acceleration techniques} into TRL also yields a competitive technique for proving \emph{un}safety.

In future work, we plan to support other theories like reals, bitvectors, and arrays, and
we will investigate an extension to temporal verification.\comment[NONE]{JG Do you really mean the extension
  to CHCs with several negative literals? Why is that a problem currently, since we work
  on arbitrary relational formulas? Or do you mean an extension in order to compute $\tip$ for
  non-linear arithmetic?

  FF Yes, I mean CHCs with several negative literals.
  %
  Our relational formulas do not contain uninterpreted predicates at all.
  %
  That's equivalent to linear CHCs, as in this case, the (only) predicate can be encoded with an additional variable.
  %
  With non-linear CHCs, we no longer have (linear) traces, but trees, and that makes everything a lot more complicated.

  FF I now removed ``non-linear CHCs'' and wrote something about temporal verification instead, hoping that this will cause less confusion.}

