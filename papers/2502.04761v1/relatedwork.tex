\section{Related Work}
\label{sec:related}
%\label{sec:experiments}
%JG removed "experiments" from the title, since they are only presented in the next
%section in the report

Most state-of-the-art infinite state model checking algorithms rely on \emph{interpolation} \cite{spacer,gspacer,tpa-multiloop,imc,lawi,eldarica,pdkind,chatterjee24}.
%
As discussed in \Cref{sec:intro}, these techniques are very powerful, but also fragile, as interpolation depends on the implementation details of the underlying SMT solver.
%
In contrast, robustness of TRL depends on the underlying transitive projection.
%
Our implementation for linear integer arithmetic is a variation of the recurrence analysis from \cite{kincaid15}, which is very robust in our experience.
%
In particular, our recurrence analysis does not require an SMT solver.

Some state-of-the-art solvers like \textsf{Golem} \cite{golem} also implement \emph{$k$-induction} \cite{kind}, which is more robust than interpolation-based techniques.
%
However, $k$-induction can only prove $k$-inductive properties, and our experiments (see \Cref{sec:experiments}) show that it is not competitive with other state-of-the-art approaches.

Currently, the most powerful model checking algorithm is \emph{Spacer with global guidance} (GSpacer).
%
In GSpacer, interpolation is optional, and thus it is more robust than Spacer (without global guidance) and other interpolation-based techniques.
%
GSpacer is part of the IC3/PDR family of model checking algorithms \cite{ic3}, which differ fundamentally from BMC-based approaches like TRL:
%
While the latter unroll the transition relation step by step, IC3/PDR proves global properties by combining local reasoning about a single step with induction, interpolation, or other mechanisms like global guidance.
%
This fundamental difference can also be seen in our evaluation (\Cref{sec:experiments}), where both GSpacer and TRL find many unique safety proofs.
%
A main advantage of IC3/PDR over BMC-based techniques is scalability, as unrolling the transition relation can quickly become a bottleneck for large systems.
%
However, TRL mitigates this drawback, as it always backtracks when the trace contains a loop.
%
Hence, TRL rarely needs to consider long unrollings.

For program verification, there are several techniques that use \emph{transition invariants} or closely related concepts like \emph{loop summarization} or \emph{acceleration} \cite{kincaid15,kroening15,abmc,adcl,silverman19,kincaid24,kroening13,bozga10,FlatFramework}.

Acceleration-based techniques \cite{kroening15,abmc,adcl,bozga10,FlatFramework} try to compute transitive closures of loops exactly, or to under-ap\-prox\-i\-mate them.
%
However, even for simple loops like {\tt while(...) x = x + y}, meaningful under-ap\-prox\-i\-ma\-tions cannot be expressed in a decidable logic.
%
For example, for the loop above, the non-linear formula $\exists n.\ n > 0 \land x' \doteq x + n \cdot y$ would be a precise characterization of the transitive closure.
%
Thus, these approaches either only use acceleration for loops where decidable logics are sufficient, or they resort to undecidable logics, which is challenging for the underlying SMT solver.
%
TRL is not restricted to under-ap\-prox\-i\-ma\-tions, and hence it can ap\-prox\-i\-mate arbitrary loops without resorting to undecidable logics.

As mentioned in \Cref{sec:intro}, TRL has been inspired by our work on accelerated bounded
model checking (ABMC) \cite{abmc}.
%
For the reasons explained above, TRL is more powerful than ABMC for proving safety (see also \Cref{sec:experiments}), but it does not subsume ABMC for that purpose.
%
The reason is that ABMC's acceleration sometimes yields non-linear relations, whereas TRL only learns linear relations.
%
On the other hand, ABMC is more powerful than TRL for proving unsafety, as TRL only tries to construct a counterexample from a single trace, whereas ABMC explores many different traces.
%
Hence, both approaches are orthogonal in practice.
%
Apart from that, a major difference is that ABMC uses blocking clauses only sparingly, which is disadvantageous for proving safety.
%
The reason is that blocking clauses reduce the search space for counterexamples, and to prove safety, this search space has to be exhausted, eventually.
%
However, using them whenever possible is disadvantageous as well, as it blows up the underlying SMT problem unnecessarily.
%
Hence, TRL uses them on demand instead.
%
Other important conceptual differences include TRL's use of: model based projection, which
cannot be used by ABMC, as acceleration yields formulas in theories without effective
quantifier elimination; backtracking, which is useful for proving safety as it keeps the
underlying SMT problem small, but it would slow down the discovery of counterexamples in
ABMC.
  

In contrast to acceleration-based techniques, recurrence analysis and loop summarization \cite{kincaid15,silverman19,kincaid24,kroening13} over-ap\-prox\-i\-mate transitive closures of loops.
%
TRL is not restricted to over-ap\-prox\-i\-ma\-tions, which allows for increased precision, as our transitive projections can perform a case analysis based on the provided model.
%
Note that this cannot be done in a framework that requires over-ap\-prox\-i\-ma\-tions, as $\tip$ cannot be used to compute over-approximations, in general (see \Cref{remark:properties-tip}).

The techniques from \cite{silverman19,kincaid24} express over-ap\-prox\-i\-ma\-tions using (variations of) \emph{vector addition systems with resets (VASRs)}, which are less expressive than our transitive projections, as the latter may yield arbitrary relational formulas.
%
However, \cite{silverman19,kincaid24} ensure that the computed ap\-prox\-i\-ma\-tions are the best VASR-ap\-prox\-i\-ma\-tions.
%
Our approach does not provide any such guarantees, so both approaches are orthogonal.

Apart from these techniques, transition invariants have mainly been used in the context of termination analysis and temporal verification \cite{tsitovich11,compositional_transition_invariants,heizmann10,podelski11,zuleger18}.
%
We expect that our approach can be adapted to these problems, but it is unclear whether it would be competitive.